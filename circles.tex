\chapter{Classification of the Circles Data Set on \acrshort{bss2}}
\label{circles}
In the first of two deep learning experiments which are implemented on the \gls{bss2} platform, a spiking feedforward network is emulated as an \gls{ann} using rate coding. As derived in \cref{ratecoding} the \gls{snn} can be then trained with conventional backpropagation. In addition the experiment focuses on a fully on-chip implementation. %, i.e. the training of the network is executed using only on-chip available resources, such as the dedicated \acrlong{ppu}.
Only for monitoring purposes of the training progress the \gls{ppu}'s memory is continuously read out through the \gls{fpga} interface.

\begin{table}[htb!]\centering\ra{1.3}
	\begin{tabular}{@{}rccc@{}}\toprule
		& Input A								& Input B 			& Output A $\veebar$ B	\\ \midrule
		& 0	& 	0	&	0	\\
		& 0	& 	1 	&	1	\\
		& 1 &	0	& 	1\\
		& 1&	1	& 	0\\
		\bottomrule
	\end{tabular}
	\caption[XOR classification.]{The XOR classification follows the prediction given in the table: if the binary inputs are not equal the XOR operator returns true and false otherwise.} 
	\label{xortruthtable}
\end{table}

In deep learning solving the XOR or ``exclusive or" problem (c.f. \cref{xortruthtable}) has become a very first benchmark to test the functionality of novel network designs and algorithms, since it requires a multi-layer network structure with a non-linear activation function to be solved (\citealp{Goodfellow-et-al-2016}). As part of a preliminary work, an on-chip implementation of \glsfirst{sgd} on the prototype \gls{dls} has been successfully tested with the XOR problem.

In a next step, the difficulty of the task has been increased with the classification of the \emph{Circles} data set. The data set consists of two classes representing an inner and outer circle as shown in \cref{circlestasksketch}. A deep network then tries to place a decision boundary in between the two circles. Before discussing the details of the training process, the rate coding of the task and the implementation of the network's training method \gls{sgd} are presented in the following sections. 

%At the beginnings of the machine learing era, \cite{perceptron} have shown that a simplified single-layer neuronal network, the perceptron, cannot solve the XOR problem or ``exclusive or". The XOR operator represents a class of problems, that are non-linear and thus require a multi-layer network structure with a non-linear activation function to be solved (\citealp{Goodfellow-et-al-2016}). In modern machine learning solving the XOR problem has therefore become a very first benchmark to test the functionality of novel network designs and algorithms.


%The classification of the \emph{Circles} data set is the first of two deep learning experiments implemented on the \gls{bss2} platform and focuses on a rate coding approach which has been motivated in \cref{ratecoding}. After successfully testing an on-chip \acrlong{sgd} implementation on the prototype \gls{dls} with an XOR problem, the difficulty of the task has been increased with the Circles data set. The task is implemented again as an on-chip experiment, i.e. the training of the network is executed using only on-chip available resources, such as the dedicated \acrlong{ppu}. Only for monitoring purposes of the training progress the chip's memory is continuously read out through the FPGA interface.
%A detailed description of the \textit{on-chip} implementation is presented in the next paragraphs.

%- single hidden layer network with 11 neurons (2 synapse lines necessary for each neuron -> 22)
%- 2 noise sources (exc/inh)
%- 2x2 input sources (exc/inh)
%(A single hidden layer network is trained using rate coding and feedback alignment.) maybe restructure the intro somehow...


\section{Circles Task}
\label{circlestask}
\begin{wrapfigure}{R}{0.33\textwidth}
	\centering
	\vspace{-0.6cm}
	%% Creator: Matplotlib, PGF backend
%%
%% To include the figure in your LaTeX document, write
%%   \input{<filename>.pgf}
%%
%% Make sure the required packages are loaded in your preamble
%%   \usepackage{pgf}
%%
%% Figures using additional raster images can only be included by \input if
%% they are in the same directory as the main LaTeX file. For loading figures
%% from other directories you can use the `import` package
%%   \usepackage{import}
%% and then include the figures with
%%   \import{<path to file>}{<filename>.pgf}
%%
%% Matplotlib used the following preamble
%%   \usepackage{amsmath} \usepackage{pifont} \usepackage{xcolor} \definecolor{green}{HTML}{467821} \definecolor{red}{HTML}{CF4457} \usepackage[detect-all]{siunitx}
%%   \usepackage{fontspec}
%%
\begingroup%
\makeatletter%
\begin{pgfpicture}%
\pgfpathrectangle{\pgfpointorigin}{\pgfqpoint{2.154255in}{1.710348in}}%
\pgfusepath{use as bounding box, clip}%
\begin{pgfscope}%
\pgfsetbuttcap%
\pgfsetmiterjoin%
\pgfsetlinewidth{0.000000pt}%
\definecolor{currentstroke}{rgb}{0.000000,0.000000,0.000000}%
\pgfsetstrokecolor{currentstroke}%
\pgfsetstrokeopacity{0.000000}%
\pgfsetdash{}{0pt}%
\pgfpathmoveto{\pgfqpoint{0.000000in}{0.000000in}}%
\pgfpathlineto{\pgfqpoint{2.154255in}{0.000000in}}%
\pgfpathlineto{\pgfqpoint{2.154255in}{1.710348in}}%
\pgfpathlineto{\pgfqpoint{0.000000in}{1.710348in}}%
\pgfpathclose%
\pgfusepath{}%
\end{pgfscope}%
\begin{pgfscope}%
\pgfsetbuttcap%
\pgfsetmiterjoin%
\pgfsetlinewidth{0.000000pt}%
\definecolor{currentstroke}{rgb}{0.000000,0.000000,0.000000}%
\pgfsetstrokecolor{currentstroke}%
\pgfsetstrokeopacity{0.000000}%
\pgfsetdash{}{0pt}%
\pgfpathmoveto{\pgfqpoint{0.403630in}{0.251348in}}%
\pgfpathlineto{\pgfqpoint{1.798630in}{0.251348in}}%
\pgfpathlineto{\pgfqpoint{1.798630in}{1.610348in}}%
\pgfpathlineto{\pgfqpoint{0.403630in}{1.610348in}}%
\pgfpathclose%
\pgfusepath{}%
\end{pgfscope}%
\begin{pgfscope}%
\pgfsetbuttcap%
\pgfsetroundjoin%
\definecolor{currentfill}{rgb}{0.317647,0.317647,0.317647}%
\pgfsetfillcolor{currentfill}%
\pgfsetlinewidth{0.501875pt}%
\definecolor{currentstroke}{rgb}{0.317647,0.317647,0.317647}%
\pgfsetstrokecolor{currentstroke}%
\pgfsetdash{}{0pt}%
\pgfsys@defobject{currentmarker}{\pgfqpoint{0.000000in}{-0.020833in}}{\pgfqpoint{0.000000in}{0.000000in}}{%
\pgfpathmoveto{\pgfqpoint{0.000000in}{0.000000in}}%
\pgfpathlineto{\pgfqpoint{0.000000in}{-0.020833in}}%
\pgfusepath{stroke,fill}%
}%
\begin{pgfscope}%
\pgfsys@transformshift{0.556807in}{0.251348in}%
\pgfsys@useobject{currentmarker}{}%
\end{pgfscope}%
\end{pgfscope}%
\begin{pgfscope}%
\definecolor{textcolor}{rgb}{0.317647,0.317647,0.317647}%
\pgfsetstrokecolor{textcolor}%
\pgfsetfillcolor{textcolor}%
\pgftext[x=0.556807in,y=0.202737in,,top]{\color{textcolor}\rmfamily\fontsize{8.330000}{9.996000}\selectfont \(\displaystyle -100\)}%
\end{pgfscope}%
\begin{pgfscope}%
\pgfsetbuttcap%
\pgfsetroundjoin%
\definecolor{currentfill}{rgb}{0.317647,0.317647,0.317647}%
\pgfsetfillcolor{currentfill}%
\pgfsetlinewidth{0.501875pt}%
\definecolor{currentstroke}{rgb}{0.317647,0.317647,0.317647}%
\pgfsetstrokecolor{currentstroke}%
\pgfsetdash{}{0pt}%
\pgfsys@defobject{currentmarker}{\pgfqpoint{0.000000in}{-0.020833in}}{\pgfqpoint{0.000000in}{0.000000in}}{%
\pgfpathmoveto{\pgfqpoint{0.000000in}{0.000000in}}%
\pgfpathlineto{\pgfqpoint{0.000000in}{-0.020833in}}%
\pgfusepath{stroke,fill}%
}%
\begin{pgfscope}%
\pgfsys@transformshift{1.103866in}{0.251348in}%
\pgfsys@useobject{currentmarker}{}%
\end{pgfscope}%
\end{pgfscope}%
\begin{pgfscope}%
\definecolor{textcolor}{rgb}{0.317647,0.317647,0.317647}%
\pgfsetstrokecolor{textcolor}%
\pgfsetfillcolor{textcolor}%
\pgftext[x=1.103866in,y=0.202737in,,top]{\color{textcolor}\rmfamily\fontsize{8.330000}{9.996000}\selectfont \(\displaystyle 0\)}%
\end{pgfscope}%
\begin{pgfscope}%
\pgfsetbuttcap%
\pgfsetroundjoin%
\definecolor{currentfill}{rgb}{0.317647,0.317647,0.317647}%
\pgfsetfillcolor{currentfill}%
\pgfsetlinewidth{0.501875pt}%
\definecolor{currentstroke}{rgb}{0.317647,0.317647,0.317647}%
\pgfsetstrokecolor{currentstroke}%
\pgfsetdash{}{0pt}%
\pgfsys@defobject{currentmarker}{\pgfqpoint{0.000000in}{-0.020833in}}{\pgfqpoint{0.000000in}{0.000000in}}{%
\pgfpathmoveto{\pgfqpoint{0.000000in}{0.000000in}}%
\pgfpathlineto{\pgfqpoint{0.000000in}{-0.020833in}}%
\pgfusepath{stroke,fill}%
}%
\begin{pgfscope}%
\pgfsys@transformshift{1.650925in}{0.251348in}%
\pgfsys@useobject{currentmarker}{}%
\end{pgfscope}%
\end{pgfscope}%
\begin{pgfscope}%
\definecolor{textcolor}{rgb}{0.317647,0.317647,0.317647}%
\pgfsetstrokecolor{textcolor}%
\pgfsetfillcolor{textcolor}%
\pgftext[x=1.650925in,y=0.202737in,,top]{\color{textcolor}\rmfamily\fontsize{8.330000}{9.996000}\selectfont \(\displaystyle 100\)}%
\end{pgfscope}%
\begin{pgfscope}%
\pgfsetbuttcap%
\pgfsetroundjoin%
\definecolor{currentfill}{rgb}{0.317647,0.317647,0.317647}%
\pgfsetfillcolor{currentfill}%
\pgfsetlinewidth{0.501875pt}%
\definecolor{currentstroke}{rgb}{0.317647,0.317647,0.317647}%
\pgfsetstrokecolor{currentstroke}%
\pgfsetdash{}{0pt}%
\pgfsys@defobject{currentmarker}{\pgfqpoint{-0.020833in}{0.000000in}}{\pgfqpoint{0.000000in}{0.000000in}}{%
\pgfpathmoveto{\pgfqpoint{0.000000in}{0.000000in}}%
\pgfpathlineto{\pgfqpoint{-0.020833in}{0.000000in}}%
\pgfusepath{stroke,fill}%
}%
\begin{pgfscope}%
\pgfsys@transformshift{0.403630in}{0.400571in}%
\pgfsys@useobject{currentmarker}{}%
\end{pgfscope}%
\end{pgfscope}%
\begin{pgfscope}%
\definecolor{textcolor}{rgb}{0.317647,0.317647,0.317647}%
\pgfsetstrokecolor{textcolor}%
\pgfsetfillcolor{textcolor}%
\pgftext[x=0.100000in,y=0.360425in,left,base]{\color{textcolor}\rmfamily\fontsize{8.330000}{9.996000}\selectfont \(\displaystyle -100\)}%
\end{pgfscope}%
\begin{pgfscope}%
\pgfsetbuttcap%
\pgfsetroundjoin%
\definecolor{currentfill}{rgb}{0.317647,0.317647,0.317647}%
\pgfsetfillcolor{currentfill}%
\pgfsetlinewidth{0.501875pt}%
\definecolor{currentstroke}{rgb}{0.317647,0.317647,0.317647}%
\pgfsetstrokecolor{currentstroke}%
\pgfsetdash{}{0pt}%
\pgfsys@defobject{currentmarker}{\pgfqpoint{-0.020833in}{0.000000in}}{\pgfqpoint{0.000000in}{0.000000in}}{%
\pgfpathmoveto{\pgfqpoint{0.000000in}{0.000000in}}%
\pgfpathlineto{\pgfqpoint{-0.020833in}{0.000000in}}%
\pgfusepath{stroke,fill}%
}%
\begin{pgfscope}%
\pgfsys@transformshift{0.403630in}{0.933512in}%
\pgfsys@useobject{currentmarker}{}%
\end{pgfscope}%
\end{pgfscope}%
\begin{pgfscope}%
\definecolor{textcolor}{rgb}{0.317647,0.317647,0.317647}%
\pgfsetstrokecolor{textcolor}%
\pgfsetfillcolor{textcolor}%
\pgftext[x=0.309880in,y=0.893366in,left,base]{\color{textcolor}\rmfamily\fontsize{8.330000}{9.996000}\selectfont \(\displaystyle 0\)}%
\end{pgfscope}%
\begin{pgfscope}%
\pgfsetbuttcap%
\pgfsetroundjoin%
\definecolor{currentfill}{rgb}{0.317647,0.317647,0.317647}%
\pgfsetfillcolor{currentfill}%
\pgfsetlinewidth{0.501875pt}%
\definecolor{currentstroke}{rgb}{0.317647,0.317647,0.317647}%
\pgfsetstrokecolor{currentstroke}%
\pgfsetdash{}{0pt}%
\pgfsys@defobject{currentmarker}{\pgfqpoint{-0.020833in}{0.000000in}}{\pgfqpoint{0.000000in}{0.000000in}}{%
\pgfpathmoveto{\pgfqpoint{0.000000in}{0.000000in}}%
\pgfpathlineto{\pgfqpoint{-0.020833in}{0.000000in}}%
\pgfusepath{stroke,fill}%
}%
\begin{pgfscope}%
\pgfsys@transformshift{0.403630in}{1.466454in}%
\pgfsys@useobject{currentmarker}{}%
\end{pgfscope}%
\end{pgfscope}%
\begin{pgfscope}%
\definecolor{textcolor}{rgb}{0.317647,0.317647,0.317647}%
\pgfsetstrokecolor{textcolor}%
\pgfsetfillcolor{textcolor}%
\pgftext[x=0.191822in,y=1.426308in,left,base]{\color{textcolor}\rmfamily\fontsize{8.330000}{9.996000}\selectfont \(\displaystyle 100\)}%
\end{pgfscope}%
\begin{pgfscope}%
\pgfpathrectangle{\pgfqpoint{0.403630in}{0.251348in}}{\pgfqpoint{1.395000in}{1.359000in}}%
\pgfusepath{clip}%
\pgfsetbuttcap%
\pgfsetroundjoin%
\definecolor{currentfill}{rgb}{0.333333,0.333333,0.333333}%
\pgfsetfillcolor{currentfill}%
\pgfsetlinewidth{0.501875pt}%
\definecolor{currentstroke}{rgb}{0.333333,0.333333,0.333333}%
\pgfsetstrokecolor{currentstroke}%
\pgfsetdash{}{0pt}%
\pgfpathmoveto{\pgfqpoint{0.885042in}{0.870044in}}%
\pgfpathcurveto{\pgfqpoint{0.890567in}{0.870044in}}{\pgfqpoint{0.895867in}{0.872239in}}{\pgfqpoint{0.899774in}{0.876146in}}%
\pgfpathcurveto{\pgfqpoint{0.903680in}{0.880052in}}{\pgfqpoint{0.905876in}{0.885352in}}{\pgfqpoint{0.905876in}{0.890877in}}%
\pgfpathcurveto{\pgfqpoint{0.905876in}{0.896402in}}{\pgfqpoint{0.903680in}{0.901702in}}{\pgfqpoint{0.899774in}{0.905608in}}%
\pgfpathcurveto{\pgfqpoint{0.895867in}{0.909515in}}{\pgfqpoint{0.890567in}{0.911710in}}{\pgfqpoint{0.885042in}{0.911710in}}%
\pgfpathcurveto{\pgfqpoint{0.879517in}{0.911710in}}{\pgfqpoint{0.874218in}{0.909515in}}{\pgfqpoint{0.870311in}{0.905608in}}%
\pgfpathcurveto{\pgfqpoint{0.866404in}{0.901702in}}{\pgfqpoint{0.864209in}{0.896402in}}{\pgfqpoint{0.864209in}{0.890877in}}%
\pgfpathcurveto{\pgfqpoint{0.864209in}{0.885352in}}{\pgfqpoint{0.866404in}{0.880052in}}{\pgfqpoint{0.870311in}{0.876146in}}%
\pgfpathcurveto{\pgfqpoint{0.874218in}{0.872239in}}{\pgfqpoint{0.879517in}{0.870044in}}{\pgfqpoint{0.885042in}{0.870044in}}%
\pgfpathclose%
\pgfusepath{stroke,fill}%
\end{pgfscope}%
\begin{pgfscope}%
\pgfpathrectangle{\pgfqpoint{0.403630in}{0.251348in}}{\pgfqpoint{1.395000in}{1.359000in}}%
\pgfusepath{clip}%
\pgfsetbuttcap%
\pgfsetroundjoin%
\definecolor{currentfill}{rgb}{0.333333,0.333333,0.333333}%
\pgfsetfillcolor{currentfill}%
\pgfsetlinewidth{0.501875pt}%
\definecolor{currentstroke}{rgb}{0.333333,0.333333,0.333333}%
\pgfsetstrokecolor{currentstroke}%
\pgfsetdash{}{0pt}%
\pgfpathmoveto{\pgfqpoint{1.306277in}{0.838067in}}%
\pgfpathcurveto{\pgfqpoint{1.311803in}{0.838067in}}{\pgfqpoint{1.317102in}{0.840262in}}{\pgfqpoint{1.321009in}{0.844169in}}%
\pgfpathcurveto{\pgfqpoint{1.324916in}{0.848076in}}{\pgfqpoint{1.327111in}{0.853376in}}{\pgfqpoint{1.327111in}{0.858901in}}%
\pgfpathcurveto{\pgfqpoint{1.327111in}{0.864426in}}{\pgfqpoint{1.324916in}{0.869725in}}{\pgfqpoint{1.321009in}{0.873632in}}%
\pgfpathcurveto{\pgfqpoint{1.317102in}{0.877539in}}{\pgfqpoint{1.311803in}{0.879734in}}{\pgfqpoint{1.306277in}{0.879734in}}%
\pgfpathcurveto{\pgfqpoint{1.300752in}{0.879734in}}{\pgfqpoint{1.295453in}{0.877539in}}{\pgfqpoint{1.291546in}{0.873632in}}%
\pgfpathcurveto{\pgfqpoint{1.287639in}{0.869725in}}{\pgfqpoint{1.285444in}{0.864426in}}{\pgfqpoint{1.285444in}{0.858901in}}%
\pgfpathcurveto{\pgfqpoint{1.285444in}{0.853376in}}{\pgfqpoint{1.287639in}{0.848076in}}{\pgfqpoint{1.291546in}{0.844169in}}%
\pgfpathcurveto{\pgfqpoint{1.295453in}{0.840262in}}{\pgfqpoint{1.300752in}{0.838067in}}{\pgfqpoint{1.306277in}{0.838067in}}%
\pgfpathclose%
\pgfusepath{stroke,fill}%
\end{pgfscope}%
\begin{pgfscope}%
\pgfpathrectangle{\pgfqpoint{0.403630in}{0.251348in}}{\pgfqpoint{1.395000in}{1.359000in}}%
\pgfusepath{clip}%
\pgfsetbuttcap%
\pgfsetroundjoin%
\definecolor{currentfill}{rgb}{0.333333,0.333333,0.333333}%
\pgfsetfillcolor{currentfill}%
\pgfsetlinewidth{0.501875pt}%
\definecolor{currentstroke}{rgb}{0.333333,0.333333,0.333333}%
\pgfsetstrokecolor{currentstroke}%
\pgfsetdash{}{0pt}%
\pgfpathmoveto{\pgfqpoint{1.005395in}{0.779444in}}%
\pgfpathcurveto{\pgfqpoint{1.010920in}{0.779444in}}{\pgfqpoint{1.016220in}{0.781639in}}{\pgfqpoint{1.020127in}{0.785546in}}%
\pgfpathcurveto{\pgfqpoint{1.024033in}{0.789452in}}{\pgfqpoint{1.026228in}{0.794752in}}{\pgfqpoint{1.026228in}{0.800277in}}%
\pgfpathcurveto{\pgfqpoint{1.026228in}{0.805802in}}{\pgfqpoint{1.024033in}{0.811102in}}{\pgfqpoint{1.020127in}{0.815008in}}%
\pgfpathcurveto{\pgfqpoint{1.016220in}{0.818915in}}{\pgfqpoint{1.010920in}{0.821110in}}{\pgfqpoint{1.005395in}{0.821110in}}%
\pgfpathcurveto{\pgfqpoint{0.999870in}{0.821110in}}{\pgfqpoint{0.994571in}{0.818915in}}{\pgfqpoint{0.990664in}{0.815008in}}%
\pgfpathcurveto{\pgfqpoint{0.986757in}{0.811102in}}{\pgfqpoint{0.984562in}{0.805802in}}{\pgfqpoint{0.984562in}{0.800277in}}%
\pgfpathcurveto{\pgfqpoint{0.984562in}{0.794752in}}{\pgfqpoint{0.986757in}{0.789452in}}{\pgfqpoint{0.990664in}{0.785546in}}%
\pgfpathcurveto{\pgfqpoint{0.994571in}{0.781639in}}{\pgfqpoint{0.999870in}{0.779444in}}{\pgfqpoint{1.005395in}{0.779444in}}%
\pgfpathclose%
\pgfusepath{stroke,fill}%
\end{pgfscope}%
\begin{pgfscope}%
\pgfpathrectangle{\pgfqpoint{0.403630in}{0.251348in}}{\pgfqpoint{1.395000in}{1.359000in}}%
\pgfusepath{clip}%
\pgfsetbuttcap%
\pgfsetroundjoin%
\definecolor{currentfill}{rgb}{0.333333,0.333333,0.333333}%
\pgfsetfillcolor{currentfill}%
\pgfsetlinewidth{0.501875pt}%
\definecolor{currentstroke}{rgb}{0.333333,0.333333,0.333333}%
\pgfsetstrokecolor{currentstroke}%
\pgfsetdash{}{0pt}%
\pgfpathmoveto{\pgfqpoint{1.273454in}{1.061903in}}%
\pgfpathcurveto{\pgfqpoint{1.278979in}{1.061903in}}{\pgfqpoint{1.284279in}{1.064098in}}{\pgfqpoint{1.288185in}{1.068004in}}%
\pgfpathcurveto{\pgfqpoint{1.292092in}{1.071911in}}{\pgfqpoint{1.294287in}{1.077211in}}{\pgfqpoint{1.294287in}{1.082736in}}%
\pgfpathcurveto{\pgfqpoint{1.294287in}{1.088261in}}{\pgfqpoint{1.292092in}{1.093560in}}{\pgfqpoint{1.288185in}{1.097467in}}%
\pgfpathcurveto{\pgfqpoint{1.284279in}{1.101374in}}{\pgfqpoint{1.278979in}{1.103569in}}{\pgfqpoint{1.273454in}{1.103569in}}%
\pgfpathcurveto{\pgfqpoint{1.267929in}{1.103569in}}{\pgfqpoint{1.262629in}{1.101374in}}{\pgfqpoint{1.258723in}{1.097467in}}%
\pgfpathcurveto{\pgfqpoint{1.254816in}{1.093560in}}{\pgfqpoint{1.252621in}{1.088261in}}{\pgfqpoint{1.252621in}{1.082736in}}%
\pgfpathcurveto{\pgfqpoint{1.252621in}{1.077211in}}{\pgfqpoint{1.254816in}{1.071911in}}{\pgfqpoint{1.258723in}{1.068004in}}%
\pgfpathcurveto{\pgfqpoint{1.262629in}{1.064098in}}{\pgfqpoint{1.267929in}{1.061903in}}{\pgfqpoint{1.273454in}{1.061903in}}%
\pgfpathclose%
\pgfusepath{stroke,fill}%
\end{pgfscope}%
\begin{pgfscope}%
\pgfpathrectangle{\pgfqpoint{0.403630in}{0.251348in}}{\pgfqpoint{1.395000in}{1.359000in}}%
\pgfusepath{clip}%
\pgfsetbuttcap%
\pgfsetroundjoin%
\definecolor{currentfill}{rgb}{0.333333,0.333333,0.333333}%
\pgfsetfillcolor{currentfill}%
\pgfsetlinewidth{0.501875pt}%
\definecolor{currentstroke}{rgb}{0.333333,0.333333,0.333333}%
\pgfsetstrokecolor{currentstroke}%
\pgfsetdash{}{0pt}%
\pgfpathmoveto{\pgfqpoint{1.218748in}{0.752797in}}%
\pgfpathcurveto{\pgfqpoint{1.224273in}{0.752797in}}{\pgfqpoint{1.229573in}{0.754992in}}{\pgfqpoint{1.233479in}{0.758899in}}%
\pgfpathcurveto{\pgfqpoint{1.237386in}{0.762805in}}{\pgfqpoint{1.239581in}{0.768105in}}{\pgfqpoint{1.239581in}{0.773630in}}%
\pgfpathcurveto{\pgfqpoint{1.239581in}{0.779155in}}{\pgfqpoint{1.237386in}{0.784455in}}{\pgfqpoint{1.233479in}{0.788361in}}%
\pgfpathcurveto{\pgfqpoint{1.229573in}{0.792268in}}{\pgfqpoint{1.224273in}{0.794463in}}{\pgfqpoint{1.218748in}{0.794463in}}%
\pgfpathcurveto{\pgfqpoint{1.213223in}{0.794463in}}{\pgfqpoint{1.207923in}{0.792268in}}{\pgfqpoint{1.204017in}{0.788361in}}%
\pgfpathcurveto{\pgfqpoint{1.200110in}{0.784455in}}{\pgfqpoint{1.197915in}{0.779155in}}{\pgfqpoint{1.197915in}{0.773630in}}%
\pgfpathcurveto{\pgfqpoint{1.197915in}{0.768105in}}{\pgfqpoint{1.200110in}{0.762805in}}{\pgfqpoint{1.204017in}{0.758899in}}%
\pgfpathcurveto{\pgfqpoint{1.207923in}{0.754992in}}{\pgfqpoint{1.213223in}{0.752797in}}{\pgfqpoint{1.218748in}{0.752797in}}%
\pgfpathclose%
\pgfusepath{stroke,fill}%
\end{pgfscope}%
\begin{pgfscope}%
\pgfpathrectangle{\pgfqpoint{0.403630in}{0.251348in}}{\pgfqpoint{1.395000in}{1.359000in}}%
\pgfusepath{clip}%
\pgfsetbuttcap%
\pgfsetroundjoin%
\definecolor{currentfill}{rgb}{0.333333,0.333333,0.333333}%
\pgfsetfillcolor{currentfill}%
\pgfsetlinewidth{0.501875pt}%
\definecolor{currentstroke}{rgb}{0.333333,0.333333,0.333333}%
\pgfsetstrokecolor{currentstroke}%
\pgfsetdash{}{0pt}%
\pgfpathmoveto{\pgfqpoint{1.103866in}{1.125855in}}%
\pgfpathcurveto{\pgfqpoint{1.109391in}{1.125855in}}{\pgfqpoint{1.114690in}{1.128051in}}{\pgfqpoint{1.118597in}{1.131957in}}%
\pgfpathcurveto{\pgfqpoint{1.122504in}{1.135864in}}{\pgfqpoint{1.124699in}{1.141164in}}{\pgfqpoint{1.124699in}{1.146689in}}%
\pgfpathcurveto{\pgfqpoint{1.124699in}{1.152214in}}{\pgfqpoint{1.122504in}{1.157513in}}{\pgfqpoint{1.118597in}{1.161420in}}%
\pgfpathcurveto{\pgfqpoint{1.114690in}{1.165327in}}{\pgfqpoint{1.109391in}{1.167522in}}{\pgfqpoint{1.103866in}{1.167522in}}%
\pgfpathcurveto{\pgfqpoint{1.098341in}{1.167522in}}{\pgfqpoint{1.093041in}{1.165327in}}{\pgfqpoint{1.089134in}{1.161420in}}%
\pgfpathcurveto{\pgfqpoint{1.085228in}{1.157513in}}{\pgfqpoint{1.083032in}{1.152214in}}{\pgfqpoint{1.083032in}{1.146689in}}%
\pgfpathcurveto{\pgfqpoint{1.083032in}{1.141164in}}{\pgfqpoint{1.085228in}{1.135864in}}{\pgfqpoint{1.089134in}{1.131957in}}%
\pgfpathcurveto{\pgfqpoint{1.093041in}{1.128051in}}{\pgfqpoint{1.098341in}{1.125855in}}{\pgfqpoint{1.103866in}{1.125855in}}%
\pgfpathclose%
\pgfusepath{stroke,fill}%
\end{pgfscope}%
\begin{pgfscope}%
\pgfpathrectangle{\pgfqpoint{0.403630in}{0.251348in}}{\pgfqpoint{1.395000in}{1.359000in}}%
\pgfusepath{clip}%
\pgfsetbuttcap%
\pgfsetroundjoin%
\definecolor{currentfill}{rgb}{0.333333,0.333333,0.333333}%
\pgfsetfillcolor{currentfill}%
\pgfsetlinewidth{0.501875pt}%
\definecolor{currentstroke}{rgb}{0.333333,0.333333,0.333333}%
\pgfsetstrokecolor{currentstroke}%
\pgfsetdash{}{0pt}%
\pgfpathmoveto{\pgfqpoint{1.021807in}{1.109867in}}%
\pgfpathcurveto{\pgfqpoint{1.027332in}{1.109867in}}{\pgfqpoint{1.032631in}{1.112062in}}{\pgfqpoint{1.036538in}{1.115969in}}%
\pgfpathcurveto{\pgfqpoint{1.040445in}{1.119876in}}{\pgfqpoint{1.042640in}{1.125176in}}{\pgfqpoint{1.042640in}{1.130701in}}%
\pgfpathcurveto{\pgfqpoint{1.042640in}{1.136226in}}{\pgfqpoint{1.040445in}{1.141525in}}{\pgfqpoint{1.036538in}{1.145432in}}%
\pgfpathcurveto{\pgfqpoint{1.032631in}{1.149339in}}{\pgfqpoint{1.027332in}{1.151534in}}{\pgfqpoint{1.021807in}{1.151534in}}%
\pgfpathcurveto{\pgfqpoint{1.016282in}{1.151534in}}{\pgfqpoint{1.010982in}{1.149339in}}{\pgfqpoint{1.007075in}{1.145432in}}%
\pgfpathcurveto{\pgfqpoint{1.003169in}{1.141525in}}{\pgfqpoint{1.000974in}{1.136226in}}{\pgfqpoint{1.000974in}{1.130701in}}%
\pgfpathcurveto{\pgfqpoint{1.000974in}{1.125176in}}{\pgfqpoint{1.003169in}{1.119876in}}{\pgfqpoint{1.007075in}{1.115969in}}%
\pgfpathcurveto{\pgfqpoint{1.010982in}{1.112062in}}{\pgfqpoint{1.016282in}{1.109867in}}{\pgfqpoint{1.021807in}{1.109867in}}%
\pgfpathclose%
\pgfusepath{stroke,fill}%
\end{pgfscope}%
\begin{pgfscope}%
\pgfpathrectangle{\pgfqpoint{0.403630in}{0.251348in}}{\pgfqpoint{1.395000in}{1.359000in}}%
\pgfusepath{clip}%
\pgfsetbuttcap%
\pgfsetroundjoin%
\definecolor{currentfill}{rgb}{0.333333,0.333333,0.333333}%
\pgfsetfillcolor{currentfill}%
\pgfsetlinewidth{0.501875pt}%
\definecolor{currentstroke}{rgb}{0.333333,0.333333,0.333333}%
\pgfsetstrokecolor{currentstroke}%
\pgfsetdash{}{0pt}%
\pgfpathmoveto{\pgfqpoint{0.917866in}{0.758126in}}%
\pgfpathcurveto{\pgfqpoint{0.923391in}{0.758126in}}{\pgfqpoint{0.928690in}{0.760321in}}{\pgfqpoint{0.932597in}{0.764228in}}%
\pgfpathcurveto{\pgfqpoint{0.936504in}{0.768135in}}{\pgfqpoint{0.938699in}{0.773434in}}{\pgfqpoint{0.938699in}{0.778959in}}%
\pgfpathcurveto{\pgfqpoint{0.938699in}{0.784484in}}{\pgfqpoint{0.936504in}{0.789784in}}{\pgfqpoint{0.932597in}{0.793691in}}%
\pgfpathcurveto{\pgfqpoint{0.928690in}{0.797598in}}{\pgfqpoint{0.923391in}{0.799793in}}{\pgfqpoint{0.917866in}{0.799793in}}%
\pgfpathcurveto{\pgfqpoint{0.912341in}{0.799793in}}{\pgfqpoint{0.907041in}{0.797598in}}{\pgfqpoint{0.903134in}{0.793691in}}%
\pgfpathcurveto{\pgfqpoint{0.899228in}{0.789784in}}{\pgfqpoint{0.897032in}{0.784484in}}{\pgfqpoint{0.897032in}{0.778959in}}%
\pgfpathcurveto{\pgfqpoint{0.897032in}{0.773434in}}{\pgfqpoint{0.899228in}{0.768135in}}{\pgfqpoint{0.903134in}{0.764228in}}%
\pgfpathcurveto{\pgfqpoint{0.907041in}{0.760321in}}{\pgfqpoint{0.912341in}{0.758126in}}{\pgfqpoint{0.917866in}{0.758126in}}%
\pgfpathclose%
\pgfusepath{stroke,fill}%
\end{pgfscope}%
\begin{pgfscope}%
\pgfpathrectangle{\pgfqpoint{0.403630in}{0.251348in}}{\pgfqpoint{1.395000in}{1.359000in}}%
\pgfusepath{clip}%
\pgfsetbuttcap%
\pgfsetroundjoin%
\definecolor{currentfill}{rgb}{0.333333,0.333333,0.333333}%
\pgfsetfillcolor{currentfill}%
\pgfsetlinewidth{0.501875pt}%
\definecolor{currentstroke}{rgb}{0.333333,0.333333,0.333333}%
\pgfsetstrokecolor{currentstroke}%
\pgfsetdash{}{0pt}%
\pgfpathmoveto{\pgfqpoint{0.874101in}{0.854055in}}%
\pgfpathcurveto{\pgfqpoint{0.879626in}{0.854055in}}{\pgfqpoint{0.884926in}{0.856251in}}{\pgfqpoint{0.888832in}{0.860157in}}%
\pgfpathcurveto{\pgfqpoint{0.892739in}{0.864064in}}{\pgfqpoint{0.894934in}{0.869364in}}{\pgfqpoint{0.894934in}{0.874889in}}%
\pgfpathcurveto{\pgfqpoint{0.894934in}{0.880414in}}{\pgfqpoint{0.892739in}{0.885713in}}{\pgfqpoint{0.888832in}{0.889620in}}%
\pgfpathcurveto{\pgfqpoint{0.884926in}{0.893527in}}{\pgfqpoint{0.879626in}{0.895722in}}{\pgfqpoint{0.874101in}{0.895722in}}%
\pgfpathcurveto{\pgfqpoint{0.868576in}{0.895722in}}{\pgfqpoint{0.863276in}{0.893527in}}{\pgfqpoint{0.859370in}{0.889620in}}%
\pgfpathcurveto{\pgfqpoint{0.855463in}{0.885713in}}{\pgfqpoint{0.853268in}{0.880414in}}{\pgfqpoint{0.853268in}{0.874889in}}%
\pgfpathcurveto{\pgfqpoint{0.853268in}{0.869364in}}{\pgfqpoint{0.855463in}{0.864064in}}{\pgfqpoint{0.859370in}{0.860157in}}%
\pgfpathcurveto{\pgfqpoint{0.863276in}{0.856251in}}{\pgfqpoint{0.868576in}{0.854055in}}{\pgfqpoint{0.874101in}{0.854055in}}%
\pgfpathclose%
\pgfusepath{stroke,fill}%
\end{pgfscope}%
\begin{pgfscope}%
\pgfpathrectangle{\pgfqpoint{0.403630in}{0.251348in}}{\pgfqpoint{1.395000in}{1.359000in}}%
\pgfusepath{clip}%
\pgfsetbuttcap%
\pgfsetroundjoin%
\definecolor{currentfill}{rgb}{0.333333,0.333333,0.333333}%
\pgfsetfillcolor{currentfill}%
\pgfsetlinewidth{0.501875pt}%
\definecolor{currentstroke}{rgb}{0.333333,0.333333,0.333333}%
\pgfsetstrokecolor{currentstroke}%
\pgfsetdash{}{0pt}%
\pgfpathmoveto{\pgfqpoint{1.273454in}{1.040585in}}%
\pgfpathcurveto{\pgfqpoint{1.278979in}{1.040585in}}{\pgfqpoint{1.284279in}{1.042780in}}{\pgfqpoint{1.288185in}{1.046687in}}%
\pgfpathcurveto{\pgfqpoint{1.292092in}{1.050594in}}{\pgfqpoint{1.294287in}{1.055893in}}{\pgfqpoint{1.294287in}{1.061418in}}%
\pgfpathcurveto{\pgfqpoint{1.294287in}{1.066943in}}{\pgfqpoint{1.292092in}{1.072243in}}{\pgfqpoint{1.288185in}{1.076150in}}%
\pgfpathcurveto{\pgfqpoint{1.284279in}{1.080056in}}{\pgfqpoint{1.278979in}{1.082252in}}{\pgfqpoint{1.273454in}{1.082252in}}%
\pgfpathcurveto{\pgfqpoint{1.267929in}{1.082252in}}{\pgfqpoint{1.262629in}{1.080056in}}{\pgfqpoint{1.258723in}{1.076150in}}%
\pgfpathcurveto{\pgfqpoint{1.254816in}{1.072243in}}{\pgfqpoint{1.252621in}{1.066943in}}{\pgfqpoint{1.252621in}{1.061418in}}%
\pgfpathcurveto{\pgfqpoint{1.252621in}{1.055893in}}{\pgfqpoint{1.254816in}{1.050594in}}{\pgfqpoint{1.258723in}{1.046687in}}%
\pgfpathcurveto{\pgfqpoint{1.262629in}{1.042780in}}{\pgfqpoint{1.267929in}{1.040585in}}{\pgfqpoint{1.273454in}{1.040585in}}%
\pgfpathclose%
\pgfusepath{stroke,fill}%
\end{pgfscope}%
\begin{pgfscope}%
\pgfpathrectangle{\pgfqpoint{0.403630in}{0.251348in}}{\pgfqpoint{1.395000in}{1.359000in}}%
\pgfusepath{clip}%
\pgfsetbuttcap%
\pgfsetroundjoin%
\definecolor{currentfill}{rgb}{0.333333,0.333333,0.333333}%
\pgfsetfillcolor{currentfill}%
\pgfsetlinewidth{0.501875pt}%
\definecolor{currentstroke}{rgb}{0.333333,0.333333,0.333333}%
\pgfsetstrokecolor{currentstroke}%
\pgfsetdash{}{0pt}%
\pgfpathmoveto{\pgfqpoint{0.901454in}{0.800761in}}%
\pgfpathcurveto{\pgfqpoint{0.906979in}{0.800761in}}{\pgfqpoint{0.912279in}{0.802956in}}{\pgfqpoint{0.916185in}{0.806863in}}%
\pgfpathcurveto{\pgfqpoint{0.920092in}{0.810770in}}{\pgfqpoint{0.922287in}{0.816070in}}{\pgfqpoint{0.922287in}{0.821595in}}%
\pgfpathcurveto{\pgfqpoint{0.922287in}{0.827120in}}{\pgfqpoint{0.920092in}{0.832419in}}{\pgfqpoint{0.916185in}{0.836326in}}%
\pgfpathcurveto{\pgfqpoint{0.912279in}{0.840233in}}{\pgfqpoint{0.906979in}{0.842428in}}{\pgfqpoint{0.901454in}{0.842428in}}%
\pgfpathcurveto{\pgfqpoint{0.895929in}{0.842428in}}{\pgfqpoint{0.890629in}{0.840233in}}{\pgfqpoint{0.886723in}{0.836326in}}%
\pgfpathcurveto{\pgfqpoint{0.882816in}{0.832419in}}{\pgfqpoint{0.880621in}{0.827120in}}{\pgfqpoint{0.880621in}{0.821595in}}%
\pgfpathcurveto{\pgfqpoint{0.880621in}{0.816070in}}{\pgfqpoint{0.882816in}{0.810770in}}{\pgfqpoint{0.886723in}{0.806863in}}%
\pgfpathcurveto{\pgfqpoint{0.890629in}{0.802956in}}{\pgfqpoint{0.895929in}{0.800761in}}{\pgfqpoint{0.901454in}{0.800761in}}%
\pgfpathclose%
\pgfusepath{stroke,fill}%
\end{pgfscope}%
\begin{pgfscope}%
\pgfpathrectangle{\pgfqpoint{0.403630in}{0.251348in}}{\pgfqpoint{1.395000in}{1.359000in}}%
\pgfusepath{clip}%
\pgfsetbuttcap%
\pgfsetroundjoin%
\definecolor{currentfill}{rgb}{0.333333,0.333333,0.333333}%
\pgfsetfillcolor{currentfill}%
\pgfsetlinewidth{0.501875pt}%
\definecolor{currentstroke}{rgb}{0.333333,0.333333,0.333333}%
\pgfsetstrokecolor{currentstroke}%
\pgfsetdash{}{0pt}%
\pgfpathmoveto{\pgfqpoint{0.912395in}{1.019267in}}%
\pgfpathcurveto{\pgfqpoint{0.917920in}{1.019267in}}{\pgfqpoint{0.923220in}{1.021462in}}{\pgfqpoint{0.927127in}{1.025369in}}%
\pgfpathcurveto{\pgfqpoint{0.931033in}{1.029276in}}{\pgfqpoint{0.933228in}{1.034576in}}{\pgfqpoint{0.933228in}{1.040101in}}%
\pgfpathcurveto{\pgfqpoint{0.933228in}{1.045626in}}{\pgfqpoint{0.931033in}{1.050925in}}{\pgfqpoint{0.927127in}{1.054832in}}%
\pgfpathcurveto{\pgfqpoint{0.923220in}{1.058739in}}{\pgfqpoint{0.917920in}{1.060934in}}{\pgfqpoint{0.912395in}{1.060934in}}%
\pgfpathcurveto{\pgfqpoint{0.906870in}{1.060934in}}{\pgfqpoint{0.901571in}{1.058739in}}{\pgfqpoint{0.897664in}{1.054832in}}%
\pgfpathcurveto{\pgfqpoint{0.893757in}{1.050925in}}{\pgfqpoint{0.891562in}{1.045626in}}{\pgfqpoint{0.891562in}{1.040101in}}%
\pgfpathcurveto{\pgfqpoint{0.891562in}{1.034576in}}{\pgfqpoint{0.893757in}{1.029276in}}{\pgfqpoint{0.897664in}{1.025369in}}%
\pgfpathcurveto{\pgfqpoint{0.901571in}{1.021462in}}{\pgfqpoint{0.906870in}{1.019267in}}{\pgfqpoint{0.912395in}{1.019267in}}%
\pgfpathclose%
\pgfusepath{stroke,fill}%
\end{pgfscope}%
\begin{pgfscope}%
\pgfpathrectangle{\pgfqpoint{0.403630in}{0.251348in}}{\pgfqpoint{1.395000in}{1.359000in}}%
\pgfusepath{clip}%
\pgfsetbuttcap%
\pgfsetroundjoin%
\definecolor{currentfill}{rgb}{0.333333,0.333333,0.333333}%
\pgfsetfillcolor{currentfill}%
\pgfsetlinewidth{0.501875pt}%
\definecolor{currentstroke}{rgb}{0.333333,0.333333,0.333333}%
\pgfsetstrokecolor{currentstroke}%
\pgfsetdash{}{0pt}%
\pgfpathmoveto{\pgfqpoint{0.890513in}{0.800761in}}%
\pgfpathcurveto{\pgfqpoint{0.896038in}{0.800761in}}{\pgfqpoint{0.901337in}{0.802956in}}{\pgfqpoint{0.905244in}{0.806863in}}%
\pgfpathcurveto{\pgfqpoint{0.909151in}{0.810770in}}{\pgfqpoint{0.911346in}{0.816070in}}{\pgfqpoint{0.911346in}{0.821595in}}%
\pgfpathcurveto{\pgfqpoint{0.911346in}{0.827120in}}{\pgfqpoint{0.909151in}{0.832419in}}{\pgfqpoint{0.905244in}{0.836326in}}%
\pgfpathcurveto{\pgfqpoint{0.901337in}{0.840233in}}{\pgfqpoint{0.896038in}{0.842428in}}{\pgfqpoint{0.890513in}{0.842428in}}%
\pgfpathcurveto{\pgfqpoint{0.884988in}{0.842428in}}{\pgfqpoint{0.879688in}{0.840233in}}{\pgfqpoint{0.875781in}{0.836326in}}%
\pgfpathcurveto{\pgfqpoint{0.871875in}{0.832419in}}{\pgfqpoint{0.869679in}{0.827120in}}{\pgfqpoint{0.869679in}{0.821595in}}%
\pgfpathcurveto{\pgfqpoint{0.869679in}{0.816070in}}{\pgfqpoint{0.871875in}{0.810770in}}{\pgfqpoint{0.875781in}{0.806863in}}%
\pgfpathcurveto{\pgfqpoint{0.879688in}{0.802956in}}{\pgfqpoint{0.884988in}{0.800761in}}{\pgfqpoint{0.890513in}{0.800761in}}%
\pgfpathclose%
\pgfusepath{stroke,fill}%
\end{pgfscope}%
\begin{pgfscope}%
\pgfpathrectangle{\pgfqpoint{0.403630in}{0.251348in}}{\pgfqpoint{1.395000in}{1.359000in}}%
\pgfusepath{clip}%
\pgfsetbuttcap%
\pgfsetroundjoin%
\definecolor{currentfill}{rgb}{0.333333,0.333333,0.333333}%
\pgfsetfillcolor{currentfill}%
\pgfsetlinewidth{0.501875pt}%
\definecolor{currentstroke}{rgb}{0.333333,0.333333,0.333333}%
\pgfsetstrokecolor{currentstroke}%
\pgfsetdash{}{0pt}%
\pgfpathmoveto{\pgfqpoint{0.923336in}{1.072561in}}%
\pgfpathcurveto{\pgfqpoint{0.928861in}{1.072561in}}{\pgfqpoint{0.934161in}{1.074756in}}{\pgfqpoint{0.938068in}{1.078663in}}%
\pgfpathcurveto{\pgfqpoint{0.941975in}{1.082570in}}{\pgfqpoint{0.944170in}{1.087870in}}{\pgfqpoint{0.944170in}{1.093395in}}%
\pgfpathcurveto{\pgfqpoint{0.944170in}{1.098920in}}{\pgfqpoint{0.941975in}{1.104219in}}{\pgfqpoint{0.938068in}{1.108126in}}%
\pgfpathcurveto{\pgfqpoint{0.934161in}{1.112033in}}{\pgfqpoint{0.928861in}{1.114228in}}{\pgfqpoint{0.923336in}{1.114228in}}%
\pgfpathcurveto{\pgfqpoint{0.917811in}{1.114228in}}{\pgfqpoint{0.912512in}{1.112033in}}{\pgfqpoint{0.908605in}{1.108126in}}%
\pgfpathcurveto{\pgfqpoint{0.904698in}{1.104219in}}{\pgfqpoint{0.902503in}{1.098920in}}{\pgfqpoint{0.902503in}{1.093395in}}%
\pgfpathcurveto{\pgfqpoint{0.902503in}{1.087870in}}{\pgfqpoint{0.904698in}{1.082570in}}{\pgfqpoint{0.908605in}{1.078663in}}%
\pgfpathcurveto{\pgfqpoint{0.912512in}{1.074756in}}{\pgfqpoint{0.917811in}{1.072561in}}{\pgfqpoint{0.923336in}{1.072561in}}%
\pgfpathclose%
\pgfusepath{stroke,fill}%
\end{pgfscope}%
\begin{pgfscope}%
\pgfpathrectangle{\pgfqpoint{0.403630in}{0.251348in}}{\pgfqpoint{1.395000in}{1.359000in}}%
\pgfusepath{clip}%
\pgfsetbuttcap%
\pgfsetroundjoin%
\definecolor{currentfill}{rgb}{0.333333,0.333333,0.333333}%
\pgfsetfillcolor{currentfill}%
\pgfsetlinewidth{0.501875pt}%
\definecolor{currentstroke}{rgb}{0.333333,0.333333,0.333333}%
\pgfsetstrokecolor{currentstroke}%
\pgfsetdash{}{0pt}%
\pgfpathmoveto{\pgfqpoint{1.071042in}{1.109867in}}%
\pgfpathcurveto{\pgfqpoint{1.076567in}{1.109867in}}{\pgfqpoint{1.081867in}{1.112062in}}{\pgfqpoint{1.085774in}{1.115969in}}%
\pgfpathcurveto{\pgfqpoint{1.089680in}{1.119876in}}{\pgfqpoint{1.091876in}{1.125176in}}{\pgfqpoint{1.091876in}{1.130701in}}%
\pgfpathcurveto{\pgfqpoint{1.091876in}{1.136226in}}{\pgfqpoint{1.089680in}{1.141525in}}{\pgfqpoint{1.085774in}{1.145432in}}%
\pgfpathcurveto{\pgfqpoint{1.081867in}{1.149339in}}{\pgfqpoint{1.076567in}{1.151534in}}{\pgfqpoint{1.071042in}{1.151534in}}%
\pgfpathcurveto{\pgfqpoint{1.065517in}{1.151534in}}{\pgfqpoint{1.060218in}{1.149339in}}{\pgfqpoint{1.056311in}{1.145432in}}%
\pgfpathcurveto{\pgfqpoint{1.052404in}{1.141525in}}{\pgfqpoint{1.050209in}{1.136226in}}{\pgfqpoint{1.050209in}{1.130701in}}%
\pgfpathcurveto{\pgfqpoint{1.050209in}{1.125176in}}{\pgfqpoint{1.052404in}{1.119876in}}{\pgfqpoint{1.056311in}{1.115969in}}%
\pgfpathcurveto{\pgfqpoint{1.060218in}{1.112062in}}{\pgfqpoint{1.065517in}{1.109867in}}{\pgfqpoint{1.071042in}{1.109867in}}%
\pgfpathclose%
\pgfusepath{stroke,fill}%
\end{pgfscope}%
\begin{pgfscope}%
\pgfpathrectangle{\pgfqpoint{0.403630in}{0.251348in}}{\pgfqpoint{1.395000in}{1.359000in}}%
\pgfusepath{clip}%
\pgfsetbuttcap%
\pgfsetroundjoin%
\definecolor{currentfill}{rgb}{0.333333,0.333333,0.333333}%
\pgfsetfillcolor{currentfill}%
\pgfsetlinewidth{0.501875pt}%
\definecolor{currentstroke}{rgb}{0.333333,0.333333,0.333333}%
\pgfsetstrokecolor{currentstroke}%
\pgfsetdash{}{0pt}%
\pgfpathmoveto{\pgfqpoint{0.885042in}{0.875373in}}%
\pgfpathcurveto{\pgfqpoint{0.890567in}{0.875373in}}{\pgfqpoint{0.895867in}{0.877568in}}{\pgfqpoint{0.899774in}{0.881475in}}%
\pgfpathcurveto{\pgfqpoint{0.903680in}{0.885382in}}{\pgfqpoint{0.905876in}{0.890681in}}{\pgfqpoint{0.905876in}{0.896206in}}%
\pgfpathcurveto{\pgfqpoint{0.905876in}{0.901732in}}{\pgfqpoint{0.903680in}{0.907031in}}{\pgfqpoint{0.899774in}{0.910938in}}%
\pgfpathcurveto{\pgfqpoint{0.895867in}{0.914845in}}{\pgfqpoint{0.890567in}{0.917040in}}{\pgfqpoint{0.885042in}{0.917040in}}%
\pgfpathcurveto{\pgfqpoint{0.879517in}{0.917040in}}{\pgfqpoint{0.874218in}{0.914845in}}{\pgfqpoint{0.870311in}{0.910938in}}%
\pgfpathcurveto{\pgfqpoint{0.866404in}{0.907031in}}{\pgfqpoint{0.864209in}{0.901732in}}{\pgfqpoint{0.864209in}{0.896206in}}%
\pgfpathcurveto{\pgfqpoint{0.864209in}{0.890681in}}{\pgfqpoint{0.866404in}{0.885382in}}{\pgfqpoint{0.870311in}{0.881475in}}%
\pgfpathcurveto{\pgfqpoint{0.874218in}{0.877568in}}{\pgfqpoint{0.879517in}{0.875373in}}{\pgfqpoint{0.885042in}{0.875373in}}%
\pgfpathclose%
\pgfusepath{stroke,fill}%
\end{pgfscope}%
\begin{pgfscope}%
\pgfpathrectangle{\pgfqpoint{0.403630in}{0.251348in}}{\pgfqpoint{1.395000in}{1.359000in}}%
\pgfusepath{clip}%
\pgfsetbuttcap%
\pgfsetroundjoin%
\definecolor{currentfill}{rgb}{0.333333,0.333333,0.333333}%
\pgfsetfillcolor{currentfill}%
\pgfsetlinewidth{0.501875pt}%
\definecolor{currentstroke}{rgb}{0.333333,0.333333,0.333333}%
\pgfsetstrokecolor{currentstroke}%
\pgfsetdash{}{0pt}%
\pgfpathmoveto{\pgfqpoint{1.251572in}{0.832738in}}%
\pgfpathcurveto{\pgfqpoint{1.257097in}{0.832738in}}{\pgfqpoint{1.262396in}{0.834933in}}{\pgfqpoint{1.266303in}{0.838840in}}%
\pgfpathcurveto{\pgfqpoint{1.270210in}{0.842747in}}{\pgfqpoint{1.272405in}{0.848046in}}{\pgfqpoint{1.272405in}{0.853571in}}%
\pgfpathcurveto{\pgfqpoint{1.272405in}{0.859096in}}{\pgfqpoint{1.270210in}{0.864396in}}{\pgfqpoint{1.266303in}{0.868303in}}%
\pgfpathcurveto{\pgfqpoint{1.262396in}{0.872209in}}{\pgfqpoint{1.257097in}{0.874405in}}{\pgfqpoint{1.251572in}{0.874405in}}%
\pgfpathcurveto{\pgfqpoint{1.246047in}{0.874405in}}{\pgfqpoint{1.240747in}{0.872209in}}{\pgfqpoint{1.236840in}{0.868303in}}%
\pgfpathcurveto{\pgfqpoint{1.232933in}{0.864396in}}{\pgfqpoint{1.230738in}{0.859096in}}{\pgfqpoint{1.230738in}{0.853571in}}%
\pgfpathcurveto{\pgfqpoint{1.230738in}{0.848046in}}{\pgfqpoint{1.232933in}{0.842747in}}{\pgfqpoint{1.236840in}{0.838840in}}%
\pgfpathcurveto{\pgfqpoint{1.240747in}{0.834933in}}{\pgfqpoint{1.246047in}{0.832738in}}{\pgfqpoint{1.251572in}{0.832738in}}%
\pgfpathclose%
\pgfusepath{stroke,fill}%
\end{pgfscope}%
\begin{pgfscope}%
\pgfpathrectangle{\pgfqpoint{0.403630in}{0.251348in}}{\pgfqpoint{1.395000in}{1.359000in}}%
\pgfusepath{clip}%
\pgfsetbuttcap%
\pgfsetroundjoin%
\definecolor{currentfill}{rgb}{0.333333,0.333333,0.333333}%
\pgfsetfillcolor{currentfill}%
\pgfsetlinewidth{0.501875pt}%
\definecolor{currentstroke}{rgb}{0.333333,0.333333,0.333333}%
\pgfsetstrokecolor{currentstroke}%
\pgfsetdash{}{0pt}%
\pgfpathmoveto{\pgfqpoint{1.169513in}{0.736808in}}%
\pgfpathcurveto{\pgfqpoint{1.175038in}{0.736808in}}{\pgfqpoint{1.180337in}{0.739004in}}{\pgfqpoint{1.184244in}{0.742910in}}%
\pgfpathcurveto{\pgfqpoint{1.188151in}{0.746817in}}{\pgfqpoint{1.190346in}{0.752117in}}{\pgfqpoint{1.190346in}{0.757642in}}%
\pgfpathcurveto{\pgfqpoint{1.190346in}{0.763167in}}{\pgfqpoint{1.188151in}{0.768466in}}{\pgfqpoint{1.184244in}{0.772373in}}%
\pgfpathcurveto{\pgfqpoint{1.180337in}{0.776280in}}{\pgfqpoint{1.175038in}{0.778475in}}{\pgfqpoint{1.169513in}{0.778475in}}%
\pgfpathcurveto{\pgfqpoint{1.163988in}{0.778475in}}{\pgfqpoint{1.158688in}{0.776280in}}{\pgfqpoint{1.154781in}{0.772373in}}%
\pgfpathcurveto{\pgfqpoint{1.150875in}{0.768466in}}{\pgfqpoint{1.148679in}{0.763167in}}{\pgfqpoint{1.148679in}{0.757642in}}%
\pgfpathcurveto{\pgfqpoint{1.148679in}{0.752117in}}{\pgfqpoint{1.150875in}{0.746817in}}{\pgfqpoint{1.154781in}{0.742910in}}%
\pgfpathcurveto{\pgfqpoint{1.158688in}{0.739004in}}{\pgfqpoint{1.163988in}{0.736808in}}{\pgfqpoint{1.169513in}{0.736808in}}%
\pgfpathclose%
\pgfusepath{stroke,fill}%
\end{pgfscope}%
\begin{pgfscope}%
\pgfpathrectangle{\pgfqpoint{0.403630in}{0.251348in}}{\pgfqpoint{1.395000in}{1.359000in}}%
\pgfusepath{clip}%
\pgfsetbuttcap%
\pgfsetroundjoin%
\definecolor{currentfill}{rgb}{0.333333,0.333333,0.333333}%
\pgfsetfillcolor{currentfill}%
\pgfsetlinewidth{0.501875pt}%
\definecolor{currentstroke}{rgb}{0.333333,0.333333,0.333333}%
\pgfsetstrokecolor{currentstroke}%
\pgfsetdash{}{0pt}%
\pgfpathmoveto{\pgfqpoint{0.923336in}{1.045914in}}%
\pgfpathcurveto{\pgfqpoint{0.928861in}{1.045914in}}{\pgfqpoint{0.934161in}{1.048109in}}{\pgfqpoint{0.938068in}{1.052016in}}%
\pgfpathcurveto{\pgfqpoint{0.941975in}{1.055923in}}{\pgfqpoint{0.944170in}{1.061223in}}{\pgfqpoint{0.944170in}{1.066748in}}%
\pgfpathcurveto{\pgfqpoint{0.944170in}{1.072273in}}{\pgfqpoint{0.941975in}{1.077572in}}{\pgfqpoint{0.938068in}{1.081479in}}%
\pgfpathcurveto{\pgfqpoint{0.934161in}{1.085386in}}{\pgfqpoint{0.928861in}{1.087581in}}{\pgfqpoint{0.923336in}{1.087581in}}%
\pgfpathcurveto{\pgfqpoint{0.917811in}{1.087581in}}{\pgfqpoint{0.912512in}{1.085386in}}{\pgfqpoint{0.908605in}{1.081479in}}%
\pgfpathcurveto{\pgfqpoint{0.904698in}{1.077572in}}{\pgfqpoint{0.902503in}{1.072273in}}{\pgfqpoint{0.902503in}{1.066748in}}%
\pgfpathcurveto{\pgfqpoint{0.902503in}{1.061223in}}{\pgfqpoint{0.904698in}{1.055923in}}{\pgfqpoint{0.908605in}{1.052016in}}%
\pgfpathcurveto{\pgfqpoint{0.912512in}{1.048109in}}{\pgfqpoint{0.917811in}{1.045914in}}{\pgfqpoint{0.923336in}{1.045914in}}%
\pgfpathclose%
\pgfusepath{stroke,fill}%
\end{pgfscope}%
\begin{pgfscope}%
\pgfpathrectangle{\pgfqpoint{0.403630in}{0.251348in}}{\pgfqpoint{1.395000in}{1.359000in}}%
\pgfusepath{clip}%
\pgfsetbuttcap%
\pgfsetroundjoin%
\definecolor{currentfill}{rgb}{0.333333,0.333333,0.333333}%
\pgfsetfillcolor{currentfill}%
\pgfsetlinewidth{0.501875pt}%
\definecolor{currentstroke}{rgb}{0.333333,0.333333,0.333333}%
\pgfsetstrokecolor{currentstroke}%
\pgfsetdash{}{0pt}%
\pgfpathmoveto{\pgfqpoint{1.289866in}{0.965973in}}%
\pgfpathcurveto{\pgfqpoint{1.295391in}{0.965973in}}{\pgfqpoint{1.300690in}{0.968168in}}{\pgfqpoint{1.304597in}{0.972075in}}%
\pgfpathcurveto{\pgfqpoint{1.308504in}{0.975982in}}{\pgfqpoint{1.310699in}{0.981281in}}{\pgfqpoint{1.310699in}{0.986806in}}%
\pgfpathcurveto{\pgfqpoint{1.310699in}{0.992332in}}{\pgfqpoint{1.308504in}{0.997631in}}{\pgfqpoint{1.304597in}{1.001538in}}%
\pgfpathcurveto{\pgfqpoint{1.300690in}{1.005445in}}{\pgfqpoint{1.295391in}{1.007640in}}{\pgfqpoint{1.289866in}{1.007640in}}%
\pgfpathcurveto{\pgfqpoint{1.284341in}{1.007640in}}{\pgfqpoint{1.279041in}{1.005445in}}{\pgfqpoint{1.275134in}{1.001538in}}%
\pgfpathcurveto{\pgfqpoint{1.271228in}{0.997631in}}{\pgfqpoint{1.269032in}{0.992332in}}{\pgfqpoint{1.269032in}{0.986806in}}%
\pgfpathcurveto{\pgfqpoint{1.269032in}{0.981281in}}{\pgfqpoint{1.271228in}{0.975982in}}{\pgfqpoint{1.275134in}{0.972075in}}%
\pgfpathcurveto{\pgfqpoint{1.279041in}{0.968168in}}{\pgfqpoint{1.284341in}{0.965973in}}{\pgfqpoint{1.289866in}{0.965973in}}%
\pgfpathclose%
\pgfusepath{stroke,fill}%
\end{pgfscope}%
\begin{pgfscope}%
\pgfpathrectangle{\pgfqpoint{0.403630in}{0.251348in}}{\pgfqpoint{1.395000in}{1.359000in}}%
\pgfusepath{clip}%
\pgfsetbuttcap%
\pgfsetroundjoin%
\definecolor{currentfill}{rgb}{0.333333,0.333333,0.333333}%
\pgfsetfillcolor{currentfill}%
\pgfsetlinewidth{0.501875pt}%
\definecolor{currentstroke}{rgb}{0.333333,0.333333,0.333333}%
\pgfsetstrokecolor{currentstroke}%
\pgfsetdash{}{0pt}%
\pgfpathmoveto{\pgfqpoint{1.060101in}{1.109867in}}%
\pgfpathcurveto{\pgfqpoint{1.065626in}{1.109867in}}{\pgfqpoint{1.070926in}{1.112062in}}{\pgfqpoint{1.074832in}{1.115969in}}%
\pgfpathcurveto{\pgfqpoint{1.078739in}{1.119876in}}{\pgfqpoint{1.080934in}{1.125176in}}{\pgfqpoint{1.080934in}{1.130701in}}%
\pgfpathcurveto{\pgfqpoint{1.080934in}{1.136226in}}{\pgfqpoint{1.078739in}{1.141525in}}{\pgfqpoint{1.074832in}{1.145432in}}%
\pgfpathcurveto{\pgfqpoint{1.070926in}{1.149339in}}{\pgfqpoint{1.065626in}{1.151534in}}{\pgfqpoint{1.060101in}{1.151534in}}%
\pgfpathcurveto{\pgfqpoint{1.054576in}{1.151534in}}{\pgfqpoint{1.049276in}{1.149339in}}{\pgfqpoint{1.045370in}{1.145432in}}%
\pgfpathcurveto{\pgfqpoint{1.041463in}{1.141525in}}{\pgfqpoint{1.039268in}{1.136226in}}{\pgfqpoint{1.039268in}{1.130701in}}%
\pgfpathcurveto{\pgfqpoint{1.039268in}{1.125176in}}{\pgfqpoint{1.041463in}{1.119876in}}{\pgfqpoint{1.045370in}{1.115969in}}%
\pgfpathcurveto{\pgfqpoint{1.049276in}{1.112062in}}{\pgfqpoint{1.054576in}{1.109867in}}{\pgfqpoint{1.060101in}{1.109867in}}%
\pgfpathclose%
\pgfusepath{stroke,fill}%
\end{pgfscope}%
\begin{pgfscope}%
\pgfpathrectangle{\pgfqpoint{0.403630in}{0.251348in}}{\pgfqpoint{1.395000in}{1.359000in}}%
\pgfusepath{clip}%
\pgfsetbuttcap%
\pgfsetroundjoin%
\definecolor{currentfill}{rgb}{0.333333,0.333333,0.333333}%
\pgfsetfillcolor{currentfill}%
\pgfsetlinewidth{0.501875pt}%
\definecolor{currentstroke}{rgb}{0.333333,0.333333,0.333333}%
\pgfsetstrokecolor{currentstroke}%
\pgfsetdash{}{0pt}%
\pgfpathmoveto{\pgfqpoint{0.956160in}{1.003279in}}%
\pgfpathcurveto{\pgfqpoint{0.961685in}{1.003279in}}{\pgfqpoint{0.966984in}{1.005474in}}{\pgfqpoint{0.970891in}{1.009381in}}%
\pgfpathcurveto{\pgfqpoint{0.974798in}{1.013288in}}{\pgfqpoint{0.976993in}{1.018587in}}{\pgfqpoint{0.976993in}{1.024112in}}%
\pgfpathcurveto{\pgfqpoint{0.976993in}{1.029637in}}{\pgfqpoint{0.974798in}{1.034937in}}{\pgfqpoint{0.970891in}{1.038844in}}%
\pgfpathcurveto{\pgfqpoint{0.966984in}{1.042751in}}{\pgfqpoint{0.961685in}{1.044946in}}{\pgfqpoint{0.956160in}{1.044946in}}%
\pgfpathcurveto{\pgfqpoint{0.950635in}{1.044946in}}{\pgfqpoint{0.945335in}{1.042751in}}{\pgfqpoint{0.941428in}{1.038844in}}%
\pgfpathcurveto{\pgfqpoint{0.937522in}{1.034937in}}{\pgfqpoint{0.935326in}{1.029637in}}{\pgfqpoint{0.935326in}{1.024112in}}%
\pgfpathcurveto{\pgfqpoint{0.935326in}{1.018587in}}{\pgfqpoint{0.937522in}{1.013288in}}{\pgfqpoint{0.941428in}{1.009381in}}%
\pgfpathcurveto{\pgfqpoint{0.945335in}{1.005474in}}{\pgfqpoint{0.950635in}{1.003279in}}{\pgfqpoint{0.956160in}{1.003279in}}%
\pgfpathclose%
\pgfusepath{stroke,fill}%
\end{pgfscope}%
\begin{pgfscope}%
\pgfpathrectangle{\pgfqpoint{0.403630in}{0.251348in}}{\pgfqpoint{1.395000in}{1.359000in}}%
\pgfusepath{clip}%
\pgfsetbuttcap%
\pgfsetroundjoin%
\definecolor{currentfill}{rgb}{0.333333,0.333333,0.333333}%
\pgfsetfillcolor{currentfill}%
\pgfsetlinewidth{0.501875pt}%
\definecolor{currentstroke}{rgb}{0.333333,0.333333,0.333333}%
\pgfsetstrokecolor{currentstroke}%
\pgfsetdash{}{0pt}%
\pgfpathmoveto{\pgfqpoint{1.131219in}{1.072561in}}%
\pgfpathcurveto{\pgfqpoint{1.136744in}{1.072561in}}{\pgfqpoint{1.142043in}{1.074756in}}{\pgfqpoint{1.145950in}{1.078663in}}%
\pgfpathcurveto{\pgfqpoint{1.149857in}{1.082570in}}{\pgfqpoint{1.152052in}{1.087870in}}{\pgfqpoint{1.152052in}{1.093395in}}%
\pgfpathcurveto{\pgfqpoint{1.152052in}{1.098920in}}{\pgfqpoint{1.149857in}{1.104219in}}{\pgfqpoint{1.145950in}{1.108126in}}%
\pgfpathcurveto{\pgfqpoint{1.142043in}{1.112033in}}{\pgfqpoint{1.136744in}{1.114228in}}{\pgfqpoint{1.131219in}{1.114228in}}%
\pgfpathcurveto{\pgfqpoint{1.125694in}{1.114228in}}{\pgfqpoint{1.120394in}{1.112033in}}{\pgfqpoint{1.116487in}{1.108126in}}%
\pgfpathcurveto{\pgfqpoint{1.112580in}{1.104219in}}{\pgfqpoint{1.110385in}{1.098920in}}{\pgfqpoint{1.110385in}{1.093395in}}%
\pgfpathcurveto{\pgfqpoint{1.110385in}{1.087870in}}{\pgfqpoint{1.112580in}{1.082570in}}{\pgfqpoint{1.116487in}{1.078663in}}%
\pgfpathcurveto{\pgfqpoint{1.120394in}{1.074756in}}{\pgfqpoint{1.125694in}{1.072561in}}{\pgfqpoint{1.131219in}{1.072561in}}%
\pgfpathclose%
\pgfusepath{stroke,fill}%
\end{pgfscope}%
\begin{pgfscope}%
\pgfpathrectangle{\pgfqpoint{0.403630in}{0.251348in}}{\pgfqpoint{1.395000in}{1.359000in}}%
\pgfusepath{clip}%
\pgfsetbuttcap%
\pgfsetroundjoin%
\definecolor{currentfill}{rgb}{0.333333,0.333333,0.333333}%
\pgfsetfillcolor{currentfill}%
\pgfsetlinewidth{0.501875pt}%
\definecolor{currentstroke}{rgb}{0.333333,0.333333,0.333333}%
\pgfsetstrokecolor{currentstroke}%
\pgfsetdash{}{0pt}%
\pgfpathmoveto{\pgfqpoint{1.289866in}{0.848726in}}%
\pgfpathcurveto{\pgfqpoint{1.295391in}{0.848726in}}{\pgfqpoint{1.300690in}{0.850921in}}{\pgfqpoint{1.304597in}{0.854828in}}%
\pgfpathcurveto{\pgfqpoint{1.308504in}{0.858735in}}{\pgfqpoint{1.310699in}{0.864034in}}{\pgfqpoint{1.310699in}{0.869559in}}%
\pgfpathcurveto{\pgfqpoint{1.310699in}{0.875084in}}{\pgfqpoint{1.308504in}{0.880384in}}{\pgfqpoint{1.304597in}{0.884291in}}%
\pgfpathcurveto{\pgfqpoint{1.300690in}{0.888198in}}{\pgfqpoint{1.295391in}{0.890393in}}{\pgfqpoint{1.289866in}{0.890393in}}%
\pgfpathcurveto{\pgfqpoint{1.284341in}{0.890393in}}{\pgfqpoint{1.279041in}{0.888198in}}{\pgfqpoint{1.275134in}{0.884291in}}%
\pgfpathcurveto{\pgfqpoint{1.271228in}{0.880384in}}{\pgfqpoint{1.269032in}{0.875084in}}{\pgfqpoint{1.269032in}{0.869559in}}%
\pgfpathcurveto{\pgfqpoint{1.269032in}{0.864034in}}{\pgfqpoint{1.271228in}{0.858735in}}{\pgfqpoint{1.275134in}{0.854828in}}%
\pgfpathcurveto{\pgfqpoint{1.279041in}{0.850921in}}{\pgfqpoint{1.284341in}{0.848726in}}{\pgfqpoint{1.289866in}{0.848726in}}%
\pgfpathclose%
\pgfusepath{stroke,fill}%
\end{pgfscope}%
\begin{pgfscope}%
\pgfpathrectangle{\pgfqpoint{0.403630in}{0.251348in}}{\pgfqpoint{1.395000in}{1.359000in}}%
\pgfusepath{clip}%
\pgfsetbuttcap%
\pgfsetroundjoin%
\definecolor{currentfill}{rgb}{0.333333,0.333333,0.333333}%
\pgfsetfillcolor{currentfill}%
\pgfsetlinewidth{0.501875pt}%
\definecolor{currentstroke}{rgb}{0.333333,0.333333,0.333333}%
\pgfsetstrokecolor{currentstroke}%
\pgfsetdash{}{0pt}%
\pgfpathmoveto{\pgfqpoint{1.224219in}{0.763455in}}%
\pgfpathcurveto{\pgfqpoint{1.229744in}{0.763455in}}{\pgfqpoint{1.235043in}{0.765651in}}{\pgfqpoint{1.238950in}{0.769557in}}%
\pgfpathcurveto{\pgfqpoint{1.242857in}{0.773464in}}{\pgfqpoint{1.245052in}{0.778764in}}{\pgfqpoint{1.245052in}{0.784289in}}%
\pgfpathcurveto{\pgfqpoint{1.245052in}{0.789814in}}{\pgfqpoint{1.242857in}{0.795113in}}{\pgfqpoint{1.238950in}{0.799020in}}%
\pgfpathcurveto{\pgfqpoint{1.235043in}{0.802927in}}{\pgfqpoint{1.229744in}{0.805122in}}{\pgfqpoint{1.224219in}{0.805122in}}%
\pgfpathcurveto{\pgfqpoint{1.218694in}{0.805122in}}{\pgfqpoint{1.213394in}{0.802927in}}{\pgfqpoint{1.209487in}{0.799020in}}%
\pgfpathcurveto{\pgfqpoint{1.205580in}{0.795113in}}{\pgfqpoint{1.203385in}{0.789814in}}{\pgfqpoint{1.203385in}{0.784289in}}%
\pgfpathcurveto{\pgfqpoint{1.203385in}{0.778764in}}{\pgfqpoint{1.205580in}{0.773464in}}{\pgfqpoint{1.209487in}{0.769557in}}%
\pgfpathcurveto{\pgfqpoint{1.213394in}{0.765651in}}{\pgfqpoint{1.218694in}{0.763455in}}{\pgfqpoint{1.224219in}{0.763455in}}%
\pgfpathclose%
\pgfusepath{stroke,fill}%
\end{pgfscope}%
\begin{pgfscope}%
\pgfpathrectangle{\pgfqpoint{0.403630in}{0.251348in}}{\pgfqpoint{1.395000in}{1.359000in}}%
\pgfusepath{clip}%
\pgfsetbuttcap%
\pgfsetroundjoin%
\definecolor{currentfill}{rgb}{0.333333,0.333333,0.333333}%
\pgfsetfillcolor{currentfill}%
\pgfsetlinewidth{0.501875pt}%
\definecolor{currentstroke}{rgb}{0.333333,0.333333,0.333333}%
\pgfsetstrokecolor{currentstroke}%
\pgfsetdash{}{0pt}%
\pgfpathmoveto{\pgfqpoint{0.895983in}{0.832738in}}%
\pgfpathcurveto{\pgfqpoint{0.901508in}{0.832738in}}{\pgfqpoint{0.906808in}{0.834933in}}{\pgfqpoint{0.910715in}{0.838840in}}%
\pgfpathcurveto{\pgfqpoint{0.914622in}{0.842747in}}{\pgfqpoint{0.916817in}{0.848046in}}{\pgfqpoint{0.916817in}{0.853571in}}%
\pgfpathcurveto{\pgfqpoint{0.916817in}{0.859096in}}{\pgfqpoint{0.914622in}{0.864396in}}{\pgfqpoint{0.910715in}{0.868303in}}%
\pgfpathcurveto{\pgfqpoint{0.906808in}{0.872209in}}{\pgfqpoint{0.901508in}{0.874405in}}{\pgfqpoint{0.895983in}{0.874405in}}%
\pgfpathcurveto{\pgfqpoint{0.890458in}{0.874405in}}{\pgfqpoint{0.885159in}{0.872209in}}{\pgfqpoint{0.881252in}{0.868303in}}%
\pgfpathcurveto{\pgfqpoint{0.877345in}{0.864396in}}{\pgfqpoint{0.875150in}{0.859096in}}{\pgfqpoint{0.875150in}{0.853571in}}%
\pgfpathcurveto{\pgfqpoint{0.875150in}{0.848046in}}{\pgfqpoint{0.877345in}{0.842747in}}{\pgfqpoint{0.881252in}{0.838840in}}%
\pgfpathcurveto{\pgfqpoint{0.885159in}{0.834933in}}{\pgfqpoint{0.890458in}{0.832738in}}{\pgfqpoint{0.895983in}{0.832738in}}%
\pgfpathclose%
\pgfusepath{stroke,fill}%
\end{pgfscope}%
\begin{pgfscope}%
\pgfpathrectangle{\pgfqpoint{0.403630in}{0.251348in}}{\pgfqpoint{1.395000in}{1.359000in}}%
\pgfusepath{clip}%
\pgfsetbuttcap%
\pgfsetroundjoin%
\definecolor{currentfill}{rgb}{0.333333,0.333333,0.333333}%
\pgfsetfillcolor{currentfill}%
\pgfsetlinewidth{0.501875pt}%
\definecolor{currentstroke}{rgb}{0.333333,0.333333,0.333333}%
\pgfsetstrokecolor{currentstroke}%
\pgfsetdash{}{0pt}%
\pgfpathmoveto{\pgfqpoint{0.939748in}{0.976632in}}%
\pgfpathcurveto{\pgfqpoint{0.945273in}{0.976632in}}{\pgfqpoint{0.950573in}{0.978827in}}{\pgfqpoint{0.954479in}{0.982734in}}%
\pgfpathcurveto{\pgfqpoint{0.958386in}{0.986641in}}{\pgfqpoint{0.960581in}{0.991940in}}{\pgfqpoint{0.960581in}{0.997465in}}%
\pgfpathcurveto{\pgfqpoint{0.960581in}{1.002990in}}{\pgfqpoint{0.958386in}{1.008290in}}{\pgfqpoint{0.954479in}{1.012197in}}%
\pgfpathcurveto{\pgfqpoint{0.950573in}{1.016103in}}{\pgfqpoint{0.945273in}{1.018299in}}{\pgfqpoint{0.939748in}{1.018299in}}%
\pgfpathcurveto{\pgfqpoint{0.934223in}{1.018299in}}{\pgfqpoint{0.928923in}{1.016103in}}{\pgfqpoint{0.925017in}{1.012197in}}%
\pgfpathcurveto{\pgfqpoint{0.921110in}{1.008290in}}{\pgfqpoint{0.918915in}{1.002990in}}{\pgfqpoint{0.918915in}{0.997465in}}%
\pgfpathcurveto{\pgfqpoint{0.918915in}{0.991940in}}{\pgfqpoint{0.921110in}{0.986641in}}{\pgfqpoint{0.925017in}{0.982734in}}%
\pgfpathcurveto{\pgfqpoint{0.928923in}{0.978827in}}{\pgfqpoint{0.934223in}{0.976632in}}{\pgfqpoint{0.939748in}{0.976632in}}%
\pgfpathclose%
\pgfusepath{stroke,fill}%
\end{pgfscope}%
\begin{pgfscope}%
\pgfpathrectangle{\pgfqpoint{0.403630in}{0.251348in}}{\pgfqpoint{1.395000in}{1.359000in}}%
\pgfusepath{clip}%
\pgfsetbuttcap%
\pgfsetroundjoin%
\definecolor{currentfill}{rgb}{0.333333,0.333333,0.333333}%
\pgfsetfillcolor{currentfill}%
\pgfsetlinewidth{0.501875pt}%
\definecolor{currentstroke}{rgb}{0.333333,0.333333,0.333333}%
\pgfsetstrokecolor{currentstroke}%
\pgfsetdash{}{0pt}%
\pgfpathmoveto{\pgfqpoint{1.322689in}{0.944655in}}%
\pgfpathcurveto{\pgfqpoint{1.328214in}{0.944655in}}{\pgfqpoint{1.333514in}{0.946851in}}{\pgfqpoint{1.337421in}{0.950757in}}%
\pgfpathcurveto{\pgfqpoint{1.341327in}{0.954664in}}{\pgfqpoint{1.343523in}{0.959964in}}{\pgfqpoint{1.343523in}{0.965489in}}%
\pgfpathcurveto{\pgfqpoint{1.343523in}{0.971014in}}{\pgfqpoint{1.341327in}{0.976313in}}{\pgfqpoint{1.337421in}{0.980220in}}%
\pgfpathcurveto{\pgfqpoint{1.333514in}{0.984127in}}{\pgfqpoint{1.328214in}{0.986322in}}{\pgfqpoint{1.322689in}{0.986322in}}%
\pgfpathcurveto{\pgfqpoint{1.317164in}{0.986322in}}{\pgfqpoint{1.311865in}{0.984127in}}{\pgfqpoint{1.307958in}{0.980220in}}%
\pgfpathcurveto{\pgfqpoint{1.304051in}{0.976313in}}{\pgfqpoint{1.301856in}{0.971014in}}{\pgfqpoint{1.301856in}{0.965489in}}%
\pgfpathcurveto{\pgfqpoint{1.301856in}{0.959964in}}{\pgfqpoint{1.304051in}{0.954664in}}{\pgfqpoint{1.307958in}{0.950757in}}%
\pgfpathcurveto{\pgfqpoint{1.311865in}{0.946851in}}{\pgfqpoint{1.317164in}{0.944655in}}{\pgfqpoint{1.322689in}{0.944655in}}%
\pgfpathclose%
\pgfusepath{stroke,fill}%
\end{pgfscope}%
\begin{pgfscope}%
\pgfpathrectangle{\pgfqpoint{0.403630in}{0.251348in}}{\pgfqpoint{1.395000in}{1.359000in}}%
\pgfusepath{clip}%
\pgfsetbuttcap%
\pgfsetroundjoin%
\definecolor{currentfill}{rgb}{0.333333,0.333333,0.333333}%
\pgfsetfillcolor{currentfill}%
\pgfsetlinewidth{0.501875pt}%
\definecolor{currentstroke}{rgb}{0.333333,0.333333,0.333333}%
\pgfsetstrokecolor{currentstroke}%
\pgfsetdash{}{0pt}%
\pgfpathmoveto{\pgfqpoint{1.043689in}{1.083220in}}%
\pgfpathcurveto{\pgfqpoint{1.049214in}{1.083220in}}{\pgfqpoint{1.054514in}{1.085415in}}{\pgfqpoint{1.058421in}{1.089322in}}%
\pgfpathcurveto{\pgfqpoint{1.062327in}{1.093229in}}{\pgfqpoint{1.064523in}{1.098528in}}{\pgfqpoint{1.064523in}{1.104054in}}%
\pgfpathcurveto{\pgfqpoint{1.064523in}{1.109579in}}{\pgfqpoint{1.062327in}{1.114878in}}{\pgfqpoint{1.058421in}{1.118785in}}%
\pgfpathcurveto{\pgfqpoint{1.054514in}{1.122692in}}{\pgfqpoint{1.049214in}{1.124887in}}{\pgfqpoint{1.043689in}{1.124887in}}%
\pgfpathcurveto{\pgfqpoint{1.038164in}{1.124887in}}{\pgfqpoint{1.032865in}{1.122692in}}{\pgfqpoint{1.028958in}{1.118785in}}%
\pgfpathcurveto{\pgfqpoint{1.025051in}{1.114878in}}{\pgfqpoint{1.022856in}{1.109579in}}{\pgfqpoint{1.022856in}{1.104054in}}%
\pgfpathcurveto{\pgfqpoint{1.022856in}{1.098528in}}{\pgfqpoint{1.025051in}{1.093229in}}{\pgfqpoint{1.028958in}{1.089322in}}%
\pgfpathcurveto{\pgfqpoint{1.032865in}{1.085415in}}{\pgfqpoint{1.038164in}{1.083220in}}{\pgfqpoint{1.043689in}{1.083220in}}%
\pgfpathclose%
\pgfusepath{stroke,fill}%
\end{pgfscope}%
\begin{pgfscope}%
\pgfpathrectangle{\pgfqpoint{0.403630in}{0.251348in}}{\pgfqpoint{1.395000in}{1.359000in}}%
\pgfusepath{clip}%
\pgfsetbuttcap%
\pgfsetroundjoin%
\definecolor{currentfill}{rgb}{0.333333,0.333333,0.333333}%
\pgfsetfillcolor{currentfill}%
\pgfsetlinewidth{0.501875pt}%
\definecolor{currentstroke}{rgb}{0.333333,0.333333,0.333333}%
\pgfsetstrokecolor{currentstroke}%
\pgfsetdash{}{0pt}%
\pgfpathmoveto{\pgfqpoint{1.262513in}{0.752797in}}%
\pgfpathcurveto{\pgfqpoint{1.268038in}{0.752797in}}{\pgfqpoint{1.273337in}{0.754992in}}{\pgfqpoint{1.277244in}{0.758899in}}%
\pgfpathcurveto{\pgfqpoint{1.281151in}{0.762805in}}{\pgfqpoint{1.283346in}{0.768105in}}{\pgfqpoint{1.283346in}{0.773630in}}%
\pgfpathcurveto{\pgfqpoint{1.283346in}{0.779155in}}{\pgfqpoint{1.281151in}{0.784455in}}{\pgfqpoint{1.277244in}{0.788361in}}%
\pgfpathcurveto{\pgfqpoint{1.273337in}{0.792268in}}{\pgfqpoint{1.268038in}{0.794463in}}{\pgfqpoint{1.262513in}{0.794463in}}%
\pgfpathcurveto{\pgfqpoint{1.256988in}{0.794463in}}{\pgfqpoint{1.251688in}{0.792268in}}{\pgfqpoint{1.247781in}{0.788361in}}%
\pgfpathcurveto{\pgfqpoint{1.243875in}{0.784455in}}{\pgfqpoint{1.241679in}{0.779155in}}{\pgfqpoint{1.241679in}{0.773630in}}%
\pgfpathcurveto{\pgfqpoint{1.241679in}{0.768105in}}{\pgfqpoint{1.243875in}{0.762805in}}{\pgfqpoint{1.247781in}{0.758899in}}%
\pgfpathcurveto{\pgfqpoint{1.251688in}{0.754992in}}{\pgfqpoint{1.256988in}{0.752797in}}{\pgfqpoint{1.262513in}{0.752797in}}%
\pgfpathclose%
\pgfusepath{stroke,fill}%
\end{pgfscope}%
\begin{pgfscope}%
\pgfpathrectangle{\pgfqpoint{0.403630in}{0.251348in}}{\pgfqpoint{1.395000in}{1.359000in}}%
\pgfusepath{clip}%
\pgfsetbuttcap%
\pgfsetroundjoin%
\definecolor{currentfill}{rgb}{0.333333,0.333333,0.333333}%
\pgfsetfillcolor{currentfill}%
\pgfsetlinewidth{0.501875pt}%
\definecolor{currentstroke}{rgb}{0.333333,0.333333,0.333333}%
\pgfsetstrokecolor{currentstroke}%
\pgfsetdash{}{0pt}%
\pgfpathmoveto{\pgfqpoint{1.267983in}{1.051244in}}%
\pgfpathcurveto{\pgfqpoint{1.273508in}{1.051244in}}{\pgfqpoint{1.278808in}{1.053439in}}{\pgfqpoint{1.282715in}{1.057346in}}%
\pgfpathcurveto{\pgfqpoint{1.286622in}{1.061252in}}{\pgfqpoint{1.288817in}{1.066552in}}{\pgfqpoint{1.288817in}{1.072077in}}%
\pgfpathcurveto{\pgfqpoint{1.288817in}{1.077602in}}{\pgfqpoint{1.286622in}{1.082902in}}{\pgfqpoint{1.282715in}{1.086808in}}%
\pgfpathcurveto{\pgfqpoint{1.278808in}{1.090715in}}{\pgfqpoint{1.273508in}{1.092910in}}{\pgfqpoint{1.267983in}{1.092910in}}%
\pgfpathcurveto{\pgfqpoint{1.262458in}{1.092910in}}{\pgfqpoint{1.257159in}{1.090715in}}{\pgfqpoint{1.253252in}{1.086808in}}%
\pgfpathcurveto{\pgfqpoint{1.249345in}{1.082902in}}{\pgfqpoint{1.247150in}{1.077602in}}{\pgfqpoint{1.247150in}{1.072077in}}%
\pgfpathcurveto{\pgfqpoint{1.247150in}{1.066552in}}{\pgfqpoint{1.249345in}{1.061252in}}{\pgfqpoint{1.253252in}{1.057346in}}%
\pgfpathcurveto{\pgfqpoint{1.257159in}{1.053439in}}{\pgfqpoint{1.262458in}{1.051244in}}{\pgfqpoint{1.267983in}{1.051244in}}%
\pgfpathclose%
\pgfusepath{stroke,fill}%
\end{pgfscope}%
\begin{pgfscope}%
\pgfpathrectangle{\pgfqpoint{0.403630in}{0.251348in}}{\pgfqpoint{1.395000in}{1.359000in}}%
\pgfusepath{clip}%
\pgfsetbuttcap%
\pgfsetroundjoin%
\definecolor{currentfill}{rgb}{0.333333,0.333333,0.333333}%
\pgfsetfillcolor{currentfill}%
\pgfsetlinewidth{0.501875pt}%
\definecolor{currentstroke}{rgb}{0.333333,0.333333,0.333333}%
\pgfsetstrokecolor{currentstroke}%
\pgfsetdash{}{0pt}%
\pgfpathmoveto{\pgfqpoint{1.092925in}{0.683514in}}%
\pgfpathcurveto{\pgfqpoint{1.098450in}{0.683514in}}{\pgfqpoint{1.103749in}{0.685709in}}{\pgfqpoint{1.107656in}{0.689616in}}%
\pgfpathcurveto{\pgfqpoint{1.111563in}{0.693523in}}{\pgfqpoint{1.113758in}{0.698823in}}{\pgfqpoint{1.113758in}{0.704348in}}%
\pgfpathcurveto{\pgfqpoint{1.113758in}{0.709873in}}{\pgfqpoint{1.111563in}{0.715172in}}{\pgfqpoint{1.107656in}{0.719079in}}%
\pgfpathcurveto{\pgfqpoint{1.103749in}{0.722986in}}{\pgfqpoint{1.098450in}{0.725181in}}{\pgfqpoint{1.092925in}{0.725181in}}%
\pgfpathcurveto{\pgfqpoint{1.087399in}{0.725181in}}{\pgfqpoint{1.082100in}{0.722986in}}{\pgfqpoint{1.078193in}{0.719079in}}%
\pgfpathcurveto{\pgfqpoint{1.074286in}{0.715172in}}{\pgfqpoint{1.072091in}{0.709873in}}{\pgfqpoint{1.072091in}{0.704348in}}%
\pgfpathcurveto{\pgfqpoint{1.072091in}{0.698823in}}{\pgfqpoint{1.074286in}{0.693523in}}{\pgfqpoint{1.078193in}{0.689616in}}%
\pgfpathcurveto{\pgfqpoint{1.082100in}{0.685709in}}{\pgfqpoint{1.087399in}{0.683514in}}{\pgfqpoint{1.092925in}{0.683514in}}%
\pgfpathclose%
\pgfusepath{stroke,fill}%
\end{pgfscope}%
\begin{pgfscope}%
\pgfpathrectangle{\pgfqpoint{0.403630in}{0.251348in}}{\pgfqpoint{1.395000in}{1.359000in}}%
\pgfusepath{clip}%
\pgfsetbuttcap%
\pgfsetroundjoin%
\definecolor{currentfill}{rgb}{0.333333,0.333333,0.333333}%
\pgfsetfillcolor{currentfill}%
\pgfsetlinewidth{0.501875pt}%
\definecolor{currentstroke}{rgb}{0.333333,0.333333,0.333333}%
\pgfsetstrokecolor{currentstroke}%
\pgfsetdash{}{0pt}%
\pgfpathmoveto{\pgfqpoint{1.311748in}{1.013938in}}%
\pgfpathcurveto{\pgfqpoint{1.317273in}{1.013938in}}{\pgfqpoint{1.322573in}{1.016133in}}{\pgfqpoint{1.326479in}{1.020040in}}%
\pgfpathcurveto{\pgfqpoint{1.330386in}{1.023947in}}{\pgfqpoint{1.332581in}{1.029246in}}{\pgfqpoint{1.332581in}{1.034771in}}%
\pgfpathcurveto{\pgfqpoint{1.332581in}{1.040296in}}{\pgfqpoint{1.330386in}{1.045596in}}{\pgfqpoint{1.326479in}{1.049503in}}%
\pgfpathcurveto{\pgfqpoint{1.322573in}{1.053409in}}{\pgfqpoint{1.317273in}{1.055605in}}{\pgfqpoint{1.311748in}{1.055605in}}%
\pgfpathcurveto{\pgfqpoint{1.306223in}{1.055605in}}{\pgfqpoint{1.300923in}{1.053409in}}{\pgfqpoint{1.297017in}{1.049503in}}%
\pgfpathcurveto{\pgfqpoint{1.293110in}{1.045596in}}{\pgfqpoint{1.290915in}{1.040296in}}{\pgfqpoint{1.290915in}{1.034771in}}%
\pgfpathcurveto{\pgfqpoint{1.290915in}{1.029246in}}{\pgfqpoint{1.293110in}{1.023947in}}{\pgfqpoint{1.297017in}{1.020040in}}%
\pgfpathcurveto{\pgfqpoint{1.300923in}{1.016133in}}{\pgfqpoint{1.306223in}{1.013938in}}{\pgfqpoint{1.311748in}{1.013938in}}%
\pgfpathclose%
\pgfusepath{stroke,fill}%
\end{pgfscope}%
\begin{pgfscope}%
\pgfpathrectangle{\pgfqpoint{0.403630in}{0.251348in}}{\pgfqpoint{1.395000in}{1.359000in}}%
\pgfusepath{clip}%
\pgfsetbuttcap%
\pgfsetroundjoin%
\definecolor{currentfill}{rgb}{0.333333,0.333333,0.333333}%
\pgfsetfillcolor{currentfill}%
\pgfsetlinewidth{0.501875pt}%
\definecolor{currentstroke}{rgb}{0.333333,0.333333,0.333333}%
\pgfsetstrokecolor{currentstroke}%
\pgfsetdash{}{0pt}%
\pgfpathmoveto{\pgfqpoint{0.939748in}{0.763455in}}%
\pgfpathcurveto{\pgfqpoint{0.945273in}{0.763455in}}{\pgfqpoint{0.950573in}{0.765651in}}{\pgfqpoint{0.954479in}{0.769557in}}%
\pgfpathcurveto{\pgfqpoint{0.958386in}{0.773464in}}{\pgfqpoint{0.960581in}{0.778764in}}{\pgfqpoint{0.960581in}{0.784289in}}%
\pgfpathcurveto{\pgfqpoint{0.960581in}{0.789814in}}{\pgfqpoint{0.958386in}{0.795113in}}{\pgfqpoint{0.954479in}{0.799020in}}%
\pgfpathcurveto{\pgfqpoint{0.950573in}{0.802927in}}{\pgfqpoint{0.945273in}{0.805122in}}{\pgfqpoint{0.939748in}{0.805122in}}%
\pgfpathcurveto{\pgfqpoint{0.934223in}{0.805122in}}{\pgfqpoint{0.928923in}{0.802927in}}{\pgfqpoint{0.925017in}{0.799020in}}%
\pgfpathcurveto{\pgfqpoint{0.921110in}{0.795113in}}{\pgfqpoint{0.918915in}{0.789814in}}{\pgfqpoint{0.918915in}{0.784289in}}%
\pgfpathcurveto{\pgfqpoint{0.918915in}{0.778764in}}{\pgfqpoint{0.921110in}{0.773464in}}{\pgfqpoint{0.925017in}{0.769557in}}%
\pgfpathcurveto{\pgfqpoint{0.928923in}{0.765651in}}{\pgfqpoint{0.934223in}{0.763455in}}{\pgfqpoint{0.939748in}{0.763455in}}%
\pgfpathclose%
\pgfusepath{stroke,fill}%
\end{pgfscope}%
\begin{pgfscope}%
\pgfpathrectangle{\pgfqpoint{0.403630in}{0.251348in}}{\pgfqpoint{1.395000in}{1.359000in}}%
\pgfusepath{clip}%
\pgfsetbuttcap%
\pgfsetroundjoin%
\definecolor{currentfill}{rgb}{0.333333,0.333333,0.333333}%
\pgfsetfillcolor{currentfill}%
\pgfsetlinewidth{0.501875pt}%
\definecolor{currentstroke}{rgb}{0.333333,0.333333,0.333333}%
\pgfsetstrokecolor{currentstroke}%
\pgfsetdash{}{0pt}%
\pgfpathmoveto{\pgfqpoint{1.273454in}{1.035255in}}%
\pgfpathcurveto{\pgfqpoint{1.278979in}{1.035255in}}{\pgfqpoint{1.284279in}{1.037451in}}{\pgfqpoint{1.288185in}{1.041357in}}%
\pgfpathcurveto{\pgfqpoint{1.292092in}{1.045264in}}{\pgfqpoint{1.294287in}{1.050564in}}{\pgfqpoint{1.294287in}{1.056089in}}%
\pgfpathcurveto{\pgfqpoint{1.294287in}{1.061614in}}{\pgfqpoint{1.292092in}{1.066913in}}{\pgfqpoint{1.288185in}{1.070820in}}%
\pgfpathcurveto{\pgfqpoint{1.284279in}{1.074727in}}{\pgfqpoint{1.278979in}{1.076922in}}{\pgfqpoint{1.273454in}{1.076922in}}%
\pgfpathcurveto{\pgfqpoint{1.267929in}{1.076922in}}{\pgfqpoint{1.262629in}{1.074727in}}{\pgfqpoint{1.258723in}{1.070820in}}%
\pgfpathcurveto{\pgfqpoint{1.254816in}{1.066913in}}{\pgfqpoint{1.252621in}{1.061614in}}{\pgfqpoint{1.252621in}{1.056089in}}%
\pgfpathcurveto{\pgfqpoint{1.252621in}{1.050564in}}{\pgfqpoint{1.254816in}{1.045264in}}{\pgfqpoint{1.258723in}{1.041357in}}%
\pgfpathcurveto{\pgfqpoint{1.262629in}{1.037451in}}{\pgfqpoint{1.267929in}{1.035255in}}{\pgfqpoint{1.273454in}{1.035255in}}%
\pgfpathclose%
\pgfusepath{stroke,fill}%
\end{pgfscope}%
\begin{pgfscope}%
\pgfpathrectangle{\pgfqpoint{0.403630in}{0.251348in}}{\pgfqpoint{1.395000in}{1.359000in}}%
\pgfusepath{clip}%
\pgfsetbuttcap%
\pgfsetroundjoin%
\definecolor{currentfill}{rgb}{0.333333,0.333333,0.333333}%
\pgfsetfillcolor{currentfill}%
\pgfsetlinewidth{0.501875pt}%
\definecolor{currentstroke}{rgb}{0.333333,0.333333,0.333333}%
\pgfsetstrokecolor{currentstroke}%
\pgfsetdash{}{0pt}%
\pgfpathmoveto{\pgfqpoint{1.289866in}{0.880703in}}%
\pgfpathcurveto{\pgfqpoint{1.295391in}{0.880703in}}{\pgfqpoint{1.300690in}{0.882898in}}{\pgfqpoint{1.304597in}{0.886804in}}%
\pgfpathcurveto{\pgfqpoint{1.308504in}{0.890711in}}{\pgfqpoint{1.310699in}{0.896011in}}{\pgfqpoint{1.310699in}{0.901536in}}%
\pgfpathcurveto{\pgfqpoint{1.310699in}{0.907061in}}{\pgfqpoint{1.308504in}{0.912360in}}{\pgfqpoint{1.304597in}{0.916267in}}%
\pgfpathcurveto{\pgfqpoint{1.300690in}{0.920174in}}{\pgfqpoint{1.295391in}{0.922369in}}{\pgfqpoint{1.289866in}{0.922369in}}%
\pgfpathcurveto{\pgfqpoint{1.284341in}{0.922369in}}{\pgfqpoint{1.279041in}{0.920174in}}{\pgfqpoint{1.275134in}{0.916267in}}%
\pgfpathcurveto{\pgfqpoint{1.271228in}{0.912360in}}{\pgfqpoint{1.269032in}{0.907061in}}{\pgfqpoint{1.269032in}{0.901536in}}%
\pgfpathcurveto{\pgfqpoint{1.269032in}{0.896011in}}{\pgfqpoint{1.271228in}{0.890711in}}{\pgfqpoint{1.275134in}{0.886804in}}%
\pgfpathcurveto{\pgfqpoint{1.279041in}{0.882898in}}{\pgfqpoint{1.284341in}{0.880703in}}{\pgfqpoint{1.289866in}{0.880703in}}%
\pgfpathclose%
\pgfusepath{stroke,fill}%
\end{pgfscope}%
\begin{pgfscope}%
\pgfpathrectangle{\pgfqpoint{0.403630in}{0.251348in}}{\pgfqpoint{1.395000in}{1.359000in}}%
\pgfusepath{clip}%
\pgfsetbuttcap%
\pgfsetroundjoin%
\definecolor{currentfill}{rgb}{0.333333,0.333333,0.333333}%
\pgfsetfillcolor{currentfill}%
\pgfsetlinewidth{0.501875pt}%
\definecolor{currentstroke}{rgb}{0.333333,0.333333,0.333333}%
\pgfsetstrokecolor{currentstroke}%
\pgfsetdash{}{0pt}%
\pgfpathmoveto{\pgfqpoint{0.939748in}{0.875373in}}%
\pgfpathcurveto{\pgfqpoint{0.945273in}{0.875373in}}{\pgfqpoint{0.950573in}{0.877568in}}{\pgfqpoint{0.954479in}{0.881475in}}%
\pgfpathcurveto{\pgfqpoint{0.958386in}{0.885382in}}{\pgfqpoint{0.960581in}{0.890681in}}{\pgfqpoint{0.960581in}{0.896206in}}%
\pgfpathcurveto{\pgfqpoint{0.960581in}{0.901732in}}{\pgfqpoint{0.958386in}{0.907031in}}{\pgfqpoint{0.954479in}{0.910938in}}%
\pgfpathcurveto{\pgfqpoint{0.950573in}{0.914845in}}{\pgfqpoint{0.945273in}{0.917040in}}{\pgfqpoint{0.939748in}{0.917040in}}%
\pgfpathcurveto{\pgfqpoint{0.934223in}{0.917040in}}{\pgfqpoint{0.928923in}{0.914845in}}{\pgfqpoint{0.925017in}{0.910938in}}%
\pgfpathcurveto{\pgfqpoint{0.921110in}{0.907031in}}{\pgfqpoint{0.918915in}{0.901732in}}{\pgfqpoint{0.918915in}{0.896206in}}%
\pgfpathcurveto{\pgfqpoint{0.918915in}{0.890681in}}{\pgfqpoint{0.921110in}{0.885382in}}{\pgfqpoint{0.925017in}{0.881475in}}%
\pgfpathcurveto{\pgfqpoint{0.928923in}{0.877568in}}{\pgfqpoint{0.934223in}{0.875373in}}{\pgfqpoint{0.939748in}{0.875373in}}%
\pgfpathclose%
\pgfusepath{stroke,fill}%
\end{pgfscope}%
\begin{pgfscope}%
\pgfpathrectangle{\pgfqpoint{0.403630in}{0.251348in}}{\pgfqpoint{1.395000in}{1.359000in}}%
\pgfusepath{clip}%
\pgfsetbuttcap%
\pgfsetroundjoin%
\definecolor{currentfill}{rgb}{0.333333,0.333333,0.333333}%
\pgfsetfillcolor{currentfill}%
\pgfsetlinewidth{0.501875pt}%
\definecolor{currentstroke}{rgb}{0.333333,0.333333,0.333333}%
\pgfsetstrokecolor{currentstroke}%
\pgfsetdash{}{0pt}%
\pgfpathmoveto{\pgfqpoint{1.109336in}{1.099208in}}%
\pgfpathcurveto{\pgfqpoint{1.114861in}{1.099208in}}{\pgfqpoint{1.120161in}{1.101404in}}{\pgfqpoint{1.124068in}{1.105310in}}%
\pgfpathcurveto{\pgfqpoint{1.127975in}{1.109217in}}{\pgfqpoint{1.130170in}{1.114517in}}{\pgfqpoint{1.130170in}{1.120042in}}%
\pgfpathcurveto{\pgfqpoint{1.130170in}{1.125567in}}{\pgfqpoint{1.127975in}{1.130866in}}{\pgfqpoint{1.124068in}{1.134773in}}%
\pgfpathcurveto{\pgfqpoint{1.120161in}{1.138680in}}{\pgfqpoint{1.114861in}{1.140875in}}{\pgfqpoint{1.109336in}{1.140875in}}%
\pgfpathcurveto{\pgfqpoint{1.103811in}{1.140875in}}{\pgfqpoint{1.098512in}{1.138680in}}{\pgfqpoint{1.094605in}{1.134773in}}%
\pgfpathcurveto{\pgfqpoint{1.090698in}{1.130866in}}{\pgfqpoint{1.088503in}{1.125567in}}{\pgfqpoint{1.088503in}{1.120042in}}%
\pgfpathcurveto{\pgfqpoint{1.088503in}{1.114517in}}{\pgfqpoint{1.090698in}{1.109217in}}{\pgfqpoint{1.094605in}{1.105310in}}%
\pgfpathcurveto{\pgfqpoint{1.098512in}{1.101404in}}{\pgfqpoint{1.103811in}{1.099208in}}{\pgfqpoint{1.109336in}{1.099208in}}%
\pgfpathclose%
\pgfusepath{stroke,fill}%
\end{pgfscope}%
\begin{pgfscope}%
\pgfpathrectangle{\pgfqpoint{0.403630in}{0.251348in}}{\pgfqpoint{1.395000in}{1.359000in}}%
\pgfusepath{clip}%
\pgfsetbuttcap%
\pgfsetroundjoin%
\definecolor{currentfill}{rgb}{0.333333,0.333333,0.333333}%
\pgfsetfillcolor{currentfill}%
\pgfsetlinewidth{0.501875pt}%
\definecolor{currentstroke}{rgb}{0.333333,0.333333,0.333333}%
\pgfsetstrokecolor{currentstroke}%
\pgfsetdash{}{0pt}%
\pgfpathmoveto{\pgfqpoint{1.191395in}{0.747467in}}%
\pgfpathcurveto{\pgfqpoint{1.196920in}{0.747467in}}{\pgfqpoint{1.202220in}{0.749662in}}{\pgfqpoint{1.206127in}{0.753569in}}%
\pgfpathcurveto{\pgfqpoint{1.210033in}{0.757476in}}{\pgfqpoint{1.212228in}{0.762776in}}{\pgfqpoint{1.212228in}{0.768301in}}%
\pgfpathcurveto{\pgfqpoint{1.212228in}{0.773826in}}{\pgfqpoint{1.210033in}{0.779125in}}{\pgfqpoint{1.206127in}{0.783032in}}%
\pgfpathcurveto{\pgfqpoint{1.202220in}{0.786939in}}{\pgfqpoint{1.196920in}{0.789134in}}{\pgfqpoint{1.191395in}{0.789134in}}%
\pgfpathcurveto{\pgfqpoint{1.185870in}{0.789134in}}{\pgfqpoint{1.180571in}{0.786939in}}{\pgfqpoint{1.176664in}{0.783032in}}%
\pgfpathcurveto{\pgfqpoint{1.172757in}{0.779125in}}{\pgfqpoint{1.170562in}{0.773826in}}{\pgfqpoint{1.170562in}{0.768301in}}%
\pgfpathcurveto{\pgfqpoint{1.170562in}{0.762776in}}{\pgfqpoint{1.172757in}{0.757476in}}{\pgfqpoint{1.176664in}{0.753569in}}%
\pgfpathcurveto{\pgfqpoint{1.180571in}{0.749662in}}{\pgfqpoint{1.185870in}{0.747467in}}{\pgfqpoint{1.191395in}{0.747467in}}%
\pgfpathclose%
\pgfusepath{stroke,fill}%
\end{pgfscope}%
\begin{pgfscope}%
\pgfpathrectangle{\pgfqpoint{0.403630in}{0.251348in}}{\pgfqpoint{1.395000in}{1.359000in}}%
\pgfusepath{clip}%
\pgfsetbuttcap%
\pgfsetroundjoin%
\definecolor{currentfill}{rgb}{0.333333,0.333333,0.333333}%
\pgfsetfillcolor{currentfill}%
\pgfsetlinewidth{0.501875pt}%
\definecolor{currentstroke}{rgb}{0.333333,0.333333,0.333333}%
\pgfsetstrokecolor{currentstroke}%
\pgfsetdash{}{0pt}%
\pgfpathmoveto{\pgfqpoint{1.120277in}{0.683514in}}%
\pgfpathcurveto{\pgfqpoint{1.125803in}{0.683514in}}{\pgfqpoint{1.131102in}{0.685709in}}{\pgfqpoint{1.135009in}{0.689616in}}%
\pgfpathcurveto{\pgfqpoint{1.138916in}{0.693523in}}{\pgfqpoint{1.141111in}{0.698823in}}{\pgfqpoint{1.141111in}{0.704348in}}%
\pgfpathcurveto{\pgfqpoint{1.141111in}{0.709873in}}{\pgfqpoint{1.138916in}{0.715172in}}{\pgfqpoint{1.135009in}{0.719079in}}%
\pgfpathcurveto{\pgfqpoint{1.131102in}{0.722986in}}{\pgfqpoint{1.125803in}{0.725181in}}{\pgfqpoint{1.120277in}{0.725181in}}%
\pgfpathcurveto{\pgfqpoint{1.114752in}{0.725181in}}{\pgfqpoint{1.109453in}{0.722986in}}{\pgfqpoint{1.105546in}{0.719079in}}%
\pgfpathcurveto{\pgfqpoint{1.101639in}{0.715172in}}{\pgfqpoint{1.099444in}{0.709873in}}{\pgfqpoint{1.099444in}{0.704348in}}%
\pgfpathcurveto{\pgfqpoint{1.099444in}{0.698823in}}{\pgfqpoint{1.101639in}{0.693523in}}{\pgfqpoint{1.105546in}{0.689616in}}%
\pgfpathcurveto{\pgfqpoint{1.109453in}{0.685709in}}{\pgfqpoint{1.114752in}{0.683514in}}{\pgfqpoint{1.120277in}{0.683514in}}%
\pgfpathclose%
\pgfusepath{stroke,fill}%
\end{pgfscope}%
\begin{pgfscope}%
\pgfpathrectangle{\pgfqpoint{0.403630in}{0.251348in}}{\pgfqpoint{1.395000in}{1.359000in}}%
\pgfusepath{clip}%
\pgfsetbuttcap%
\pgfsetroundjoin%
\definecolor{currentfill}{rgb}{0.333333,0.333333,0.333333}%
\pgfsetfillcolor{currentfill}%
\pgfsetlinewidth{0.501875pt}%
\definecolor{currentstroke}{rgb}{0.333333,0.333333,0.333333}%
\pgfsetstrokecolor{currentstroke}%
\pgfsetdash{}{0pt}%
\pgfpathmoveto{\pgfqpoint{0.939748in}{1.013938in}}%
\pgfpathcurveto{\pgfqpoint{0.945273in}{1.013938in}}{\pgfqpoint{0.950573in}{1.016133in}}{\pgfqpoint{0.954479in}{1.020040in}}%
\pgfpathcurveto{\pgfqpoint{0.958386in}{1.023947in}}{\pgfqpoint{0.960581in}{1.029246in}}{\pgfqpoint{0.960581in}{1.034771in}}%
\pgfpathcurveto{\pgfqpoint{0.960581in}{1.040296in}}{\pgfqpoint{0.958386in}{1.045596in}}{\pgfqpoint{0.954479in}{1.049503in}}%
\pgfpathcurveto{\pgfqpoint{0.950573in}{1.053409in}}{\pgfqpoint{0.945273in}{1.055605in}}{\pgfqpoint{0.939748in}{1.055605in}}%
\pgfpathcurveto{\pgfqpoint{0.934223in}{1.055605in}}{\pgfqpoint{0.928923in}{1.053409in}}{\pgfqpoint{0.925017in}{1.049503in}}%
\pgfpathcurveto{\pgfqpoint{0.921110in}{1.045596in}}{\pgfqpoint{0.918915in}{1.040296in}}{\pgfqpoint{0.918915in}{1.034771in}}%
\pgfpathcurveto{\pgfqpoint{0.918915in}{1.029246in}}{\pgfqpoint{0.921110in}{1.023947in}}{\pgfqpoint{0.925017in}{1.020040in}}%
\pgfpathcurveto{\pgfqpoint{0.928923in}{1.016133in}}{\pgfqpoint{0.934223in}{1.013938in}}{\pgfqpoint{0.939748in}{1.013938in}}%
\pgfpathclose%
\pgfusepath{stroke,fill}%
\end{pgfscope}%
\begin{pgfscope}%
\pgfpathrectangle{\pgfqpoint{0.403630in}{0.251348in}}{\pgfqpoint{1.395000in}{1.359000in}}%
\pgfusepath{clip}%
\pgfsetbuttcap%
\pgfsetroundjoin%
\definecolor{currentfill}{rgb}{0.333333,0.333333,0.333333}%
\pgfsetfillcolor{currentfill}%
\pgfsetlinewidth{0.501875pt}%
\definecolor{currentstroke}{rgb}{0.333333,0.333333,0.333333}%
\pgfsetstrokecolor{currentstroke}%
\pgfsetdash{}{0pt}%
\pgfpathmoveto{\pgfqpoint{1.218748in}{1.083220in}}%
\pgfpathcurveto{\pgfqpoint{1.224273in}{1.083220in}}{\pgfqpoint{1.229573in}{1.085415in}}{\pgfqpoint{1.233479in}{1.089322in}}%
\pgfpathcurveto{\pgfqpoint{1.237386in}{1.093229in}}{\pgfqpoint{1.239581in}{1.098528in}}{\pgfqpoint{1.239581in}{1.104054in}}%
\pgfpathcurveto{\pgfqpoint{1.239581in}{1.109579in}}{\pgfqpoint{1.237386in}{1.114878in}}{\pgfqpoint{1.233479in}{1.118785in}}%
\pgfpathcurveto{\pgfqpoint{1.229573in}{1.122692in}}{\pgfqpoint{1.224273in}{1.124887in}}{\pgfqpoint{1.218748in}{1.124887in}}%
\pgfpathcurveto{\pgfqpoint{1.213223in}{1.124887in}}{\pgfqpoint{1.207923in}{1.122692in}}{\pgfqpoint{1.204017in}{1.118785in}}%
\pgfpathcurveto{\pgfqpoint{1.200110in}{1.114878in}}{\pgfqpoint{1.197915in}{1.109579in}}{\pgfqpoint{1.197915in}{1.104054in}}%
\pgfpathcurveto{\pgfqpoint{1.197915in}{1.098528in}}{\pgfqpoint{1.200110in}{1.093229in}}{\pgfqpoint{1.204017in}{1.089322in}}%
\pgfpathcurveto{\pgfqpoint{1.207923in}{1.085415in}}{\pgfqpoint{1.213223in}{1.083220in}}{\pgfqpoint{1.218748in}{1.083220in}}%
\pgfpathclose%
\pgfusepath{stroke,fill}%
\end{pgfscope}%
\begin{pgfscope}%
\pgfpathrectangle{\pgfqpoint{0.403630in}{0.251348in}}{\pgfqpoint{1.395000in}{1.359000in}}%
\pgfusepath{clip}%
\pgfsetbuttcap%
\pgfsetroundjoin%
\definecolor{currentfill}{rgb}{0.333333,0.333333,0.333333}%
\pgfsetfillcolor{currentfill}%
\pgfsetlinewidth{0.501875pt}%
\definecolor{currentstroke}{rgb}{0.333333,0.333333,0.333333}%
\pgfsetstrokecolor{currentstroke}%
\pgfsetdash{}{0pt}%
\pgfpathmoveto{\pgfqpoint{1.196866in}{1.125855in}}%
\pgfpathcurveto{\pgfqpoint{1.202391in}{1.125855in}}{\pgfqpoint{1.207690in}{1.128051in}}{\pgfqpoint{1.211597in}{1.131957in}}%
\pgfpathcurveto{\pgfqpoint{1.215504in}{1.135864in}}{\pgfqpoint{1.217699in}{1.141164in}}{\pgfqpoint{1.217699in}{1.146689in}}%
\pgfpathcurveto{\pgfqpoint{1.217699in}{1.152214in}}{\pgfqpoint{1.215504in}{1.157513in}}{\pgfqpoint{1.211597in}{1.161420in}}%
\pgfpathcurveto{\pgfqpoint{1.207690in}{1.165327in}}{\pgfqpoint{1.202391in}{1.167522in}}{\pgfqpoint{1.196866in}{1.167522in}}%
\pgfpathcurveto{\pgfqpoint{1.191341in}{1.167522in}}{\pgfqpoint{1.186041in}{1.165327in}}{\pgfqpoint{1.182134in}{1.161420in}}%
\pgfpathcurveto{\pgfqpoint{1.178228in}{1.157513in}}{\pgfqpoint{1.176032in}{1.152214in}}{\pgfqpoint{1.176032in}{1.146689in}}%
\pgfpathcurveto{\pgfqpoint{1.176032in}{1.141164in}}{\pgfqpoint{1.178228in}{1.135864in}}{\pgfqpoint{1.182134in}{1.131957in}}%
\pgfpathcurveto{\pgfqpoint{1.186041in}{1.128051in}}{\pgfqpoint{1.191341in}{1.125855in}}{\pgfqpoint{1.196866in}{1.125855in}}%
\pgfpathclose%
\pgfusepath{stroke,fill}%
\end{pgfscope}%
\begin{pgfscope}%
\pgfpathrectangle{\pgfqpoint{0.403630in}{0.251348in}}{\pgfqpoint{1.395000in}{1.359000in}}%
\pgfusepath{clip}%
\pgfsetbuttcap%
\pgfsetroundjoin%
\definecolor{currentfill}{rgb}{0.333333,0.333333,0.333333}%
\pgfsetfillcolor{currentfill}%
\pgfsetlinewidth{0.501875pt}%
\definecolor{currentstroke}{rgb}{0.333333,0.333333,0.333333}%
\pgfsetstrokecolor{currentstroke}%
\pgfsetdash{}{0pt}%
\pgfpathmoveto{\pgfqpoint{0.912395in}{1.051244in}}%
\pgfpathcurveto{\pgfqpoint{0.917920in}{1.051244in}}{\pgfqpoint{0.923220in}{1.053439in}}{\pgfqpoint{0.927127in}{1.057346in}}%
\pgfpathcurveto{\pgfqpoint{0.931033in}{1.061252in}}{\pgfqpoint{0.933228in}{1.066552in}}{\pgfqpoint{0.933228in}{1.072077in}}%
\pgfpathcurveto{\pgfqpoint{0.933228in}{1.077602in}}{\pgfqpoint{0.931033in}{1.082902in}}{\pgfqpoint{0.927127in}{1.086808in}}%
\pgfpathcurveto{\pgfqpoint{0.923220in}{1.090715in}}{\pgfqpoint{0.917920in}{1.092910in}}{\pgfqpoint{0.912395in}{1.092910in}}%
\pgfpathcurveto{\pgfqpoint{0.906870in}{1.092910in}}{\pgfqpoint{0.901571in}{1.090715in}}{\pgfqpoint{0.897664in}{1.086808in}}%
\pgfpathcurveto{\pgfqpoint{0.893757in}{1.082902in}}{\pgfqpoint{0.891562in}{1.077602in}}{\pgfqpoint{0.891562in}{1.072077in}}%
\pgfpathcurveto{\pgfqpoint{0.891562in}{1.066552in}}{\pgfqpoint{0.893757in}{1.061252in}}{\pgfqpoint{0.897664in}{1.057346in}}%
\pgfpathcurveto{\pgfqpoint{0.901571in}{1.053439in}}{\pgfqpoint{0.906870in}{1.051244in}}{\pgfqpoint{0.912395in}{1.051244in}}%
\pgfpathclose%
\pgfusepath{stroke,fill}%
\end{pgfscope}%
\begin{pgfscope}%
\pgfpathrectangle{\pgfqpoint{0.403630in}{0.251348in}}{\pgfqpoint{1.395000in}{1.359000in}}%
\pgfusepath{clip}%
\pgfsetbuttcap%
\pgfsetroundjoin%
\definecolor{currentfill}{rgb}{0.333333,0.333333,0.333333}%
\pgfsetfillcolor{currentfill}%
\pgfsetlinewidth{0.501875pt}%
\definecolor{currentstroke}{rgb}{0.333333,0.333333,0.333333}%
\pgfsetstrokecolor{currentstroke}%
\pgfsetdash{}{0pt}%
\pgfpathmoveto{\pgfqpoint{1.081983in}{1.072561in}}%
\pgfpathcurveto{\pgfqpoint{1.087508in}{1.072561in}}{\pgfqpoint{1.092808in}{1.074756in}}{\pgfqpoint{1.096715in}{1.078663in}}%
\pgfpathcurveto{\pgfqpoint{1.100622in}{1.082570in}}{\pgfqpoint{1.102817in}{1.087870in}}{\pgfqpoint{1.102817in}{1.093395in}}%
\pgfpathcurveto{\pgfqpoint{1.102817in}{1.098920in}}{\pgfqpoint{1.100622in}{1.104219in}}{\pgfqpoint{1.096715in}{1.108126in}}%
\pgfpathcurveto{\pgfqpoint{1.092808in}{1.112033in}}{\pgfqpoint{1.087508in}{1.114228in}}{\pgfqpoint{1.081983in}{1.114228in}}%
\pgfpathcurveto{\pgfqpoint{1.076458in}{1.114228in}}{\pgfqpoint{1.071159in}{1.112033in}}{\pgfqpoint{1.067252in}{1.108126in}}%
\pgfpathcurveto{\pgfqpoint{1.063345in}{1.104219in}}{\pgfqpoint{1.061150in}{1.098920in}}{\pgfqpoint{1.061150in}{1.093395in}}%
\pgfpathcurveto{\pgfqpoint{1.061150in}{1.087870in}}{\pgfqpoint{1.063345in}{1.082570in}}{\pgfqpoint{1.067252in}{1.078663in}}%
\pgfpathcurveto{\pgfqpoint{1.071159in}{1.074756in}}{\pgfqpoint{1.076458in}{1.072561in}}{\pgfqpoint{1.081983in}{1.072561in}}%
\pgfpathclose%
\pgfusepath{stroke,fill}%
\end{pgfscope}%
\begin{pgfscope}%
\pgfpathrectangle{\pgfqpoint{0.403630in}{0.251348in}}{\pgfqpoint{1.395000in}{1.359000in}}%
\pgfusepath{clip}%
\pgfsetbuttcap%
\pgfsetroundjoin%
\definecolor{currentfill}{rgb}{0.333333,0.333333,0.333333}%
\pgfsetfillcolor{currentfill}%
\pgfsetlinewidth{0.501875pt}%
\definecolor{currentstroke}{rgb}{0.333333,0.333333,0.333333}%
\pgfsetstrokecolor{currentstroke}%
\pgfsetdash{}{0pt}%
\pgfpathmoveto{\pgfqpoint{1.065572in}{0.704832in}}%
\pgfpathcurveto{\pgfqpoint{1.071097in}{0.704832in}}{\pgfqpoint{1.076396in}{0.707027in}}{\pgfqpoint{1.080303in}{0.710934in}}%
\pgfpathcurveto{\pgfqpoint{1.084210in}{0.714841in}}{\pgfqpoint{1.086405in}{0.720140in}}{\pgfqpoint{1.086405in}{0.725665in}}%
\pgfpathcurveto{\pgfqpoint{1.086405in}{0.731190in}}{\pgfqpoint{1.084210in}{0.736490in}}{\pgfqpoint{1.080303in}{0.740397in}}%
\pgfpathcurveto{\pgfqpoint{1.076396in}{0.744303in}}{\pgfqpoint{1.071097in}{0.746499in}}{\pgfqpoint{1.065572in}{0.746499in}}%
\pgfpathcurveto{\pgfqpoint{1.060047in}{0.746499in}}{\pgfqpoint{1.054747in}{0.744303in}}{\pgfqpoint{1.050840in}{0.740397in}}%
\pgfpathcurveto{\pgfqpoint{1.046933in}{0.736490in}}{\pgfqpoint{1.044738in}{0.731190in}}{\pgfqpoint{1.044738in}{0.725665in}}%
\pgfpathcurveto{\pgfqpoint{1.044738in}{0.720140in}}{\pgfqpoint{1.046933in}{0.714841in}}{\pgfqpoint{1.050840in}{0.710934in}}%
\pgfpathcurveto{\pgfqpoint{1.054747in}{0.707027in}}{\pgfqpoint{1.060047in}{0.704832in}}{\pgfqpoint{1.065572in}{0.704832in}}%
\pgfpathclose%
\pgfusepath{stroke,fill}%
\end{pgfscope}%
\begin{pgfscope}%
\pgfpathrectangle{\pgfqpoint{0.403630in}{0.251348in}}{\pgfqpoint{1.395000in}{1.359000in}}%
\pgfusepath{clip}%
\pgfsetbuttcap%
\pgfsetroundjoin%
\definecolor{currentfill}{rgb}{0.333333,0.333333,0.333333}%
\pgfsetfillcolor{currentfill}%
\pgfsetlinewidth{0.501875pt}%
\definecolor{currentstroke}{rgb}{0.333333,0.333333,0.333333}%
\pgfsetstrokecolor{currentstroke}%
\pgfsetdash{}{0pt}%
\pgfpathmoveto{\pgfqpoint{1.174983in}{0.715491in}}%
\pgfpathcurveto{\pgfqpoint{1.180508in}{0.715491in}}{\pgfqpoint{1.185808in}{0.717686in}}{\pgfqpoint{1.189715in}{0.721593in}}%
\pgfpathcurveto{\pgfqpoint{1.193622in}{0.725500in}}{\pgfqpoint{1.195817in}{0.730799in}}{\pgfqpoint{1.195817in}{0.736324in}}%
\pgfpathcurveto{\pgfqpoint{1.195817in}{0.741849in}}{\pgfqpoint{1.193622in}{0.747149in}}{\pgfqpoint{1.189715in}{0.751056in}}%
\pgfpathcurveto{\pgfqpoint{1.185808in}{0.754962in}}{\pgfqpoint{1.180508in}{0.757157in}}{\pgfqpoint{1.174983in}{0.757157in}}%
\pgfpathcurveto{\pgfqpoint{1.169458in}{0.757157in}}{\pgfqpoint{1.164159in}{0.754962in}}{\pgfqpoint{1.160252in}{0.751056in}}%
\pgfpathcurveto{\pgfqpoint{1.156345in}{0.747149in}}{\pgfqpoint{1.154150in}{0.741849in}}{\pgfqpoint{1.154150in}{0.736324in}}%
\pgfpathcurveto{\pgfqpoint{1.154150in}{0.730799in}}{\pgfqpoint{1.156345in}{0.725500in}}{\pgfqpoint{1.160252in}{0.721593in}}%
\pgfpathcurveto{\pgfqpoint{1.164159in}{0.717686in}}{\pgfqpoint{1.169458in}{0.715491in}}{\pgfqpoint{1.174983in}{0.715491in}}%
\pgfpathclose%
\pgfusepath{stroke,fill}%
\end{pgfscope}%
\begin{pgfscope}%
\pgfpathrectangle{\pgfqpoint{0.403630in}{0.251348in}}{\pgfqpoint{1.395000in}{1.359000in}}%
\pgfusepath{clip}%
\pgfsetbuttcap%
\pgfsetroundjoin%
\definecolor{currentfill}{rgb}{0.333333,0.333333,0.333333}%
\pgfsetfillcolor{currentfill}%
\pgfsetlinewidth{0.501875pt}%
\definecolor{currentstroke}{rgb}{0.333333,0.333333,0.333333}%
\pgfsetstrokecolor{currentstroke}%
\pgfsetdash{}{0pt}%
\pgfpathmoveto{\pgfqpoint{1.174983in}{0.742138in}}%
\pgfpathcurveto{\pgfqpoint{1.180508in}{0.742138in}}{\pgfqpoint{1.185808in}{0.744333in}}{\pgfqpoint{1.189715in}{0.748240in}}%
\pgfpathcurveto{\pgfqpoint{1.193622in}{0.752147in}}{\pgfqpoint{1.195817in}{0.757446in}}{\pgfqpoint{1.195817in}{0.762971in}}%
\pgfpathcurveto{\pgfqpoint{1.195817in}{0.768496in}}{\pgfqpoint{1.193622in}{0.773796in}}{\pgfqpoint{1.189715in}{0.777703in}}%
\pgfpathcurveto{\pgfqpoint{1.185808in}{0.781609in}}{\pgfqpoint{1.180508in}{0.783805in}}{\pgfqpoint{1.174983in}{0.783805in}}%
\pgfpathcurveto{\pgfqpoint{1.169458in}{0.783805in}}{\pgfqpoint{1.164159in}{0.781609in}}{\pgfqpoint{1.160252in}{0.777703in}}%
\pgfpathcurveto{\pgfqpoint{1.156345in}{0.773796in}}{\pgfqpoint{1.154150in}{0.768496in}}{\pgfqpoint{1.154150in}{0.762971in}}%
\pgfpathcurveto{\pgfqpoint{1.154150in}{0.757446in}}{\pgfqpoint{1.156345in}{0.752147in}}{\pgfqpoint{1.160252in}{0.748240in}}%
\pgfpathcurveto{\pgfqpoint{1.164159in}{0.744333in}}{\pgfqpoint{1.169458in}{0.742138in}}{\pgfqpoint{1.174983in}{0.742138in}}%
\pgfpathclose%
\pgfusepath{stroke,fill}%
\end{pgfscope}%
\begin{pgfscope}%
\pgfpathrectangle{\pgfqpoint{0.403630in}{0.251348in}}{\pgfqpoint{1.395000in}{1.359000in}}%
\pgfusepath{clip}%
\pgfsetbuttcap%
\pgfsetroundjoin%
\definecolor{currentfill}{rgb}{0.333333,0.333333,0.333333}%
\pgfsetfillcolor{currentfill}%
\pgfsetlinewidth{0.501875pt}%
\definecolor{currentstroke}{rgb}{0.333333,0.333333,0.333333}%
\pgfsetstrokecolor{currentstroke}%
\pgfsetdash{}{0pt}%
\pgfpathmoveto{\pgfqpoint{0.901454in}{0.784773in}}%
\pgfpathcurveto{\pgfqpoint{0.906979in}{0.784773in}}{\pgfqpoint{0.912279in}{0.786968in}}{\pgfqpoint{0.916185in}{0.790875in}}%
\pgfpathcurveto{\pgfqpoint{0.920092in}{0.794782in}}{\pgfqpoint{0.922287in}{0.800081in}}{\pgfqpoint{0.922287in}{0.805606in}}%
\pgfpathcurveto{\pgfqpoint{0.922287in}{0.811132in}}{\pgfqpoint{0.920092in}{0.816431in}}{\pgfqpoint{0.916185in}{0.820338in}}%
\pgfpathcurveto{\pgfqpoint{0.912279in}{0.824245in}}{\pgfqpoint{0.906979in}{0.826440in}}{\pgfqpoint{0.901454in}{0.826440in}}%
\pgfpathcurveto{\pgfqpoint{0.895929in}{0.826440in}}{\pgfqpoint{0.890629in}{0.824245in}}{\pgfqpoint{0.886723in}{0.820338in}}%
\pgfpathcurveto{\pgfqpoint{0.882816in}{0.816431in}}{\pgfqpoint{0.880621in}{0.811132in}}{\pgfqpoint{0.880621in}{0.805606in}}%
\pgfpathcurveto{\pgfqpoint{0.880621in}{0.800081in}}{\pgfqpoint{0.882816in}{0.794782in}}{\pgfqpoint{0.886723in}{0.790875in}}%
\pgfpathcurveto{\pgfqpoint{0.890629in}{0.786968in}}{\pgfqpoint{0.895929in}{0.784773in}}{\pgfqpoint{0.901454in}{0.784773in}}%
\pgfpathclose%
\pgfusepath{stroke,fill}%
\end{pgfscope}%
\begin{pgfscope}%
\pgfpathrectangle{\pgfqpoint{0.403630in}{0.251348in}}{\pgfqpoint{1.395000in}{1.359000in}}%
\pgfusepath{clip}%
\pgfsetbuttcap%
\pgfsetroundjoin%
\definecolor{currentfill}{rgb}{0.333333,0.333333,0.333333}%
\pgfsetfillcolor{currentfill}%
\pgfsetlinewidth{0.501875pt}%
\definecolor{currentstroke}{rgb}{0.333333,0.333333,0.333333}%
\pgfsetstrokecolor{currentstroke}%
\pgfsetdash{}{0pt}%
\pgfpathmoveto{\pgfqpoint{1.120277in}{1.093879in}}%
\pgfpathcurveto{\pgfqpoint{1.125803in}{1.093879in}}{\pgfqpoint{1.131102in}{1.096074in}}{\pgfqpoint{1.135009in}{1.099981in}}%
\pgfpathcurveto{\pgfqpoint{1.138916in}{1.103888in}}{\pgfqpoint{1.141111in}{1.109187in}}{\pgfqpoint{1.141111in}{1.114712in}}%
\pgfpathcurveto{\pgfqpoint{1.141111in}{1.120237in}}{\pgfqpoint{1.138916in}{1.125537in}}{\pgfqpoint{1.135009in}{1.129444in}}%
\pgfpathcurveto{\pgfqpoint{1.131102in}{1.133351in}}{\pgfqpoint{1.125803in}{1.135546in}}{\pgfqpoint{1.120277in}{1.135546in}}%
\pgfpathcurveto{\pgfqpoint{1.114752in}{1.135546in}}{\pgfqpoint{1.109453in}{1.133351in}}{\pgfqpoint{1.105546in}{1.129444in}}%
\pgfpathcurveto{\pgfqpoint{1.101639in}{1.125537in}}{\pgfqpoint{1.099444in}{1.120237in}}{\pgfqpoint{1.099444in}{1.114712in}}%
\pgfpathcurveto{\pgfqpoint{1.099444in}{1.109187in}}{\pgfqpoint{1.101639in}{1.103888in}}{\pgfqpoint{1.105546in}{1.099981in}}%
\pgfpathcurveto{\pgfqpoint{1.109453in}{1.096074in}}{\pgfqpoint{1.114752in}{1.093879in}}{\pgfqpoint{1.120277in}{1.093879in}}%
\pgfpathclose%
\pgfusepath{stroke,fill}%
\end{pgfscope}%
\begin{pgfscope}%
\pgfpathrectangle{\pgfqpoint{0.403630in}{0.251348in}}{\pgfqpoint{1.395000in}{1.359000in}}%
\pgfusepath{clip}%
\pgfsetbuttcap%
\pgfsetroundjoin%
\definecolor{currentfill}{rgb}{0.333333,0.333333,0.333333}%
\pgfsetfillcolor{currentfill}%
\pgfsetlinewidth{0.501875pt}%
\definecolor{currentstroke}{rgb}{0.333333,0.333333,0.333333}%
\pgfsetstrokecolor{currentstroke}%
\pgfsetdash{}{0pt}%
\pgfpathmoveto{\pgfqpoint{0.879572in}{0.971303in}}%
\pgfpathcurveto{\pgfqpoint{0.885097in}{0.971303in}}{\pgfqpoint{0.890396in}{0.973498in}}{\pgfqpoint{0.894303in}{0.977404in}}%
\pgfpathcurveto{\pgfqpoint{0.898210in}{0.981311in}}{\pgfqpoint{0.900405in}{0.986611in}}{\pgfqpoint{0.900405in}{0.992136in}}%
\pgfpathcurveto{\pgfqpoint{0.900405in}{0.997661in}}{\pgfqpoint{0.898210in}{1.002960in}}{\pgfqpoint{0.894303in}{1.006867in}}%
\pgfpathcurveto{\pgfqpoint{0.890396in}{1.010774in}}{\pgfqpoint{0.885097in}{1.012969in}}{\pgfqpoint{0.879572in}{1.012969in}}%
\pgfpathcurveto{\pgfqpoint{0.874047in}{1.012969in}}{\pgfqpoint{0.868747in}{1.010774in}}{\pgfqpoint{0.864840in}{1.006867in}}%
\pgfpathcurveto{\pgfqpoint{0.860933in}{1.002960in}}{\pgfqpoint{0.858738in}{0.997661in}}{\pgfqpoint{0.858738in}{0.992136in}}%
\pgfpathcurveto{\pgfqpoint{0.858738in}{0.986611in}}{\pgfqpoint{0.860933in}{0.981311in}}{\pgfqpoint{0.864840in}{0.977404in}}%
\pgfpathcurveto{\pgfqpoint{0.868747in}{0.973498in}}{\pgfqpoint{0.874047in}{0.971303in}}{\pgfqpoint{0.879572in}{0.971303in}}%
\pgfpathclose%
\pgfusepath{stroke,fill}%
\end{pgfscope}%
\begin{pgfscope}%
\pgfpathrectangle{\pgfqpoint{0.403630in}{0.251348in}}{\pgfqpoint{1.395000in}{1.359000in}}%
\pgfusepath{clip}%
\pgfsetbuttcap%
\pgfsetroundjoin%
\definecolor{currentfill}{rgb}{0.333333,0.333333,0.333333}%
\pgfsetfillcolor{currentfill}%
\pgfsetlinewidth{0.501875pt}%
\definecolor{currentstroke}{rgb}{0.333333,0.333333,0.333333}%
\pgfsetstrokecolor{currentstroke}%
\pgfsetdash{}{0pt}%
\pgfpathmoveto{\pgfqpoint{0.863160in}{0.944655in}}%
\pgfpathcurveto{\pgfqpoint{0.868685in}{0.944655in}}{\pgfqpoint{0.873984in}{0.946851in}}{\pgfqpoint{0.877891in}{0.950757in}}%
\pgfpathcurveto{\pgfqpoint{0.881798in}{0.954664in}}{\pgfqpoint{0.883993in}{0.959964in}}{\pgfqpoint{0.883993in}{0.965489in}}%
\pgfpathcurveto{\pgfqpoint{0.883993in}{0.971014in}}{\pgfqpoint{0.881798in}{0.976313in}}{\pgfqpoint{0.877891in}{0.980220in}}%
\pgfpathcurveto{\pgfqpoint{0.873984in}{0.984127in}}{\pgfqpoint{0.868685in}{0.986322in}}{\pgfqpoint{0.863160in}{0.986322in}}%
\pgfpathcurveto{\pgfqpoint{0.857635in}{0.986322in}}{\pgfqpoint{0.852335in}{0.984127in}}{\pgfqpoint{0.848428in}{0.980220in}}%
\pgfpathcurveto{\pgfqpoint{0.844522in}{0.976313in}}{\pgfqpoint{0.842326in}{0.971014in}}{\pgfqpoint{0.842326in}{0.965489in}}%
\pgfpathcurveto{\pgfqpoint{0.842326in}{0.959964in}}{\pgfqpoint{0.844522in}{0.954664in}}{\pgfqpoint{0.848428in}{0.950757in}}%
\pgfpathcurveto{\pgfqpoint{0.852335in}{0.946851in}}{\pgfqpoint{0.857635in}{0.944655in}}{\pgfqpoint{0.863160in}{0.944655in}}%
\pgfpathclose%
\pgfusepath{stroke,fill}%
\end{pgfscope}%
\begin{pgfscope}%
\pgfpathrectangle{\pgfqpoint{0.403630in}{0.251348in}}{\pgfqpoint{1.395000in}{1.359000in}}%
\pgfusepath{clip}%
\pgfsetbuttcap%
\pgfsetroundjoin%
\definecolor{currentfill}{rgb}{0.333333,0.333333,0.333333}%
\pgfsetfillcolor{currentfill}%
\pgfsetlinewidth{0.501875pt}%
\definecolor{currentstroke}{rgb}{0.333333,0.333333,0.333333}%
\pgfsetstrokecolor{currentstroke}%
\pgfsetdash{}{0pt}%
\pgfpathmoveto{\pgfqpoint{1.207807in}{0.736808in}}%
\pgfpathcurveto{\pgfqpoint{1.213332in}{0.736808in}}{\pgfqpoint{1.218631in}{0.739004in}}{\pgfqpoint{1.222538in}{0.742910in}}%
\pgfpathcurveto{\pgfqpoint{1.226445in}{0.746817in}}{\pgfqpoint{1.228640in}{0.752117in}}{\pgfqpoint{1.228640in}{0.757642in}}%
\pgfpathcurveto{\pgfqpoint{1.228640in}{0.763167in}}{\pgfqpoint{1.226445in}{0.768466in}}{\pgfqpoint{1.222538in}{0.772373in}}%
\pgfpathcurveto{\pgfqpoint{1.218631in}{0.776280in}}{\pgfqpoint{1.213332in}{0.778475in}}{\pgfqpoint{1.207807in}{0.778475in}}%
\pgfpathcurveto{\pgfqpoint{1.202282in}{0.778475in}}{\pgfqpoint{1.196982in}{0.776280in}}{\pgfqpoint{1.193075in}{0.772373in}}%
\pgfpathcurveto{\pgfqpoint{1.189169in}{0.768466in}}{\pgfqpoint{1.186974in}{0.763167in}}{\pgfqpoint{1.186974in}{0.757642in}}%
\pgfpathcurveto{\pgfqpoint{1.186974in}{0.752117in}}{\pgfqpoint{1.189169in}{0.746817in}}{\pgfqpoint{1.193075in}{0.742910in}}%
\pgfpathcurveto{\pgfqpoint{1.196982in}{0.739004in}}{\pgfqpoint{1.202282in}{0.736808in}}{\pgfqpoint{1.207807in}{0.736808in}}%
\pgfpathclose%
\pgfusepath{stroke,fill}%
\end{pgfscope}%
\begin{pgfscope}%
\pgfpathrectangle{\pgfqpoint{0.403630in}{0.251348in}}{\pgfqpoint{1.395000in}{1.359000in}}%
\pgfusepath{clip}%
\pgfsetbuttcap%
\pgfsetroundjoin%
\definecolor{currentfill}{rgb}{0.333333,0.333333,0.333333}%
\pgfsetfillcolor{currentfill}%
\pgfsetlinewidth{0.501875pt}%
\definecolor{currentstroke}{rgb}{0.333333,0.333333,0.333333}%
\pgfsetstrokecolor{currentstroke}%
\pgfsetdash{}{0pt}%
\pgfpathmoveto{\pgfqpoint{1.174983in}{0.683514in}}%
\pgfpathcurveto{\pgfqpoint{1.180508in}{0.683514in}}{\pgfqpoint{1.185808in}{0.685709in}}{\pgfqpoint{1.189715in}{0.689616in}}%
\pgfpathcurveto{\pgfqpoint{1.193622in}{0.693523in}}{\pgfqpoint{1.195817in}{0.698823in}}{\pgfqpoint{1.195817in}{0.704348in}}%
\pgfpathcurveto{\pgfqpoint{1.195817in}{0.709873in}}{\pgfqpoint{1.193622in}{0.715172in}}{\pgfqpoint{1.189715in}{0.719079in}}%
\pgfpathcurveto{\pgfqpoint{1.185808in}{0.722986in}}{\pgfqpoint{1.180508in}{0.725181in}}{\pgfqpoint{1.174983in}{0.725181in}}%
\pgfpathcurveto{\pgfqpoint{1.169458in}{0.725181in}}{\pgfqpoint{1.164159in}{0.722986in}}{\pgfqpoint{1.160252in}{0.719079in}}%
\pgfpathcurveto{\pgfqpoint{1.156345in}{0.715172in}}{\pgfqpoint{1.154150in}{0.709873in}}{\pgfqpoint{1.154150in}{0.704348in}}%
\pgfpathcurveto{\pgfqpoint{1.154150in}{0.698823in}}{\pgfqpoint{1.156345in}{0.693523in}}{\pgfqpoint{1.160252in}{0.689616in}}%
\pgfpathcurveto{\pgfqpoint{1.164159in}{0.685709in}}{\pgfqpoint{1.169458in}{0.683514in}}{\pgfqpoint{1.174983in}{0.683514in}}%
\pgfpathclose%
\pgfusepath{stroke,fill}%
\end{pgfscope}%
\begin{pgfscope}%
\pgfpathrectangle{\pgfqpoint{0.403630in}{0.251348in}}{\pgfqpoint{1.395000in}{1.359000in}}%
\pgfusepath{clip}%
\pgfsetbuttcap%
\pgfsetroundjoin%
\definecolor{currentfill}{rgb}{0.333333,0.333333,0.333333}%
\pgfsetfillcolor{currentfill}%
\pgfsetlinewidth{0.501875pt}%
\definecolor{currentstroke}{rgb}{0.333333,0.333333,0.333333}%
\pgfsetstrokecolor{currentstroke}%
\pgfsetdash{}{0pt}%
\pgfpathmoveto{\pgfqpoint{1.071042in}{0.720820in}}%
\pgfpathcurveto{\pgfqpoint{1.076567in}{0.720820in}}{\pgfqpoint{1.081867in}{0.723015in}}{\pgfqpoint{1.085774in}{0.726922in}}%
\pgfpathcurveto{\pgfqpoint{1.089680in}{0.730829in}}{\pgfqpoint{1.091876in}{0.736128in}}{\pgfqpoint{1.091876in}{0.741654in}}%
\pgfpathcurveto{\pgfqpoint{1.091876in}{0.747179in}}{\pgfqpoint{1.089680in}{0.752478in}}{\pgfqpoint{1.085774in}{0.756385in}}%
\pgfpathcurveto{\pgfqpoint{1.081867in}{0.760292in}}{\pgfqpoint{1.076567in}{0.762487in}}{\pgfqpoint{1.071042in}{0.762487in}}%
\pgfpathcurveto{\pgfqpoint{1.065517in}{0.762487in}}{\pgfqpoint{1.060218in}{0.760292in}}{\pgfqpoint{1.056311in}{0.756385in}}%
\pgfpathcurveto{\pgfqpoint{1.052404in}{0.752478in}}{\pgfqpoint{1.050209in}{0.747179in}}{\pgfqpoint{1.050209in}{0.741654in}}%
\pgfpathcurveto{\pgfqpoint{1.050209in}{0.736128in}}{\pgfqpoint{1.052404in}{0.730829in}}{\pgfqpoint{1.056311in}{0.726922in}}%
\pgfpathcurveto{\pgfqpoint{1.060218in}{0.723015in}}{\pgfqpoint{1.065517in}{0.720820in}}{\pgfqpoint{1.071042in}{0.720820in}}%
\pgfpathclose%
\pgfusepath{stroke,fill}%
\end{pgfscope}%
\begin{pgfscope}%
\pgfpathrectangle{\pgfqpoint{0.403630in}{0.251348in}}{\pgfqpoint{1.395000in}{1.359000in}}%
\pgfusepath{clip}%
\pgfsetbuttcap%
\pgfsetroundjoin%
\definecolor{currentfill}{rgb}{0.333333,0.333333,0.333333}%
\pgfsetfillcolor{currentfill}%
\pgfsetlinewidth{0.501875pt}%
\definecolor{currentstroke}{rgb}{0.333333,0.333333,0.333333}%
\pgfsetstrokecolor{currentstroke}%
\pgfsetdash{}{0pt}%
\pgfpathmoveto{\pgfqpoint{1.262513in}{1.067232in}}%
\pgfpathcurveto{\pgfqpoint{1.268038in}{1.067232in}}{\pgfqpoint{1.273337in}{1.069427in}}{\pgfqpoint{1.277244in}{1.073334in}}%
\pgfpathcurveto{\pgfqpoint{1.281151in}{1.077241in}}{\pgfqpoint{1.283346in}{1.082540in}}{\pgfqpoint{1.283346in}{1.088065in}}%
\pgfpathcurveto{\pgfqpoint{1.283346in}{1.093590in}}{\pgfqpoint{1.281151in}{1.098890in}}{\pgfqpoint{1.277244in}{1.102797in}}%
\pgfpathcurveto{\pgfqpoint{1.273337in}{1.106703in}}{\pgfqpoint{1.268038in}{1.108899in}}{\pgfqpoint{1.262513in}{1.108899in}}%
\pgfpathcurveto{\pgfqpoint{1.256988in}{1.108899in}}{\pgfqpoint{1.251688in}{1.106703in}}{\pgfqpoint{1.247781in}{1.102797in}}%
\pgfpathcurveto{\pgfqpoint{1.243875in}{1.098890in}}{\pgfqpoint{1.241679in}{1.093590in}}{\pgfqpoint{1.241679in}{1.088065in}}%
\pgfpathcurveto{\pgfqpoint{1.241679in}{1.082540in}}{\pgfqpoint{1.243875in}{1.077241in}}{\pgfqpoint{1.247781in}{1.073334in}}%
\pgfpathcurveto{\pgfqpoint{1.251688in}{1.069427in}}{\pgfqpoint{1.256988in}{1.067232in}}{\pgfqpoint{1.262513in}{1.067232in}}%
\pgfpathclose%
\pgfusepath{stroke,fill}%
\end{pgfscope}%
\begin{pgfscope}%
\pgfpathrectangle{\pgfqpoint{0.403630in}{0.251348in}}{\pgfqpoint{1.395000in}{1.359000in}}%
\pgfusepath{clip}%
\pgfsetbuttcap%
\pgfsetroundjoin%
\definecolor{currentfill}{rgb}{0.333333,0.333333,0.333333}%
\pgfsetfillcolor{currentfill}%
\pgfsetlinewidth{0.501875pt}%
\definecolor{currentstroke}{rgb}{0.333333,0.333333,0.333333}%
\pgfsetstrokecolor{currentstroke}%
\pgfsetdash{}{0pt}%
\pgfpathmoveto{\pgfqpoint{0.912395in}{0.816750in}}%
\pgfpathcurveto{\pgfqpoint{0.917920in}{0.816750in}}{\pgfqpoint{0.923220in}{0.818945in}}{\pgfqpoint{0.927127in}{0.822852in}}%
\pgfpathcurveto{\pgfqpoint{0.931033in}{0.826758in}}{\pgfqpoint{0.933228in}{0.832058in}}{\pgfqpoint{0.933228in}{0.837583in}}%
\pgfpathcurveto{\pgfqpoint{0.933228in}{0.843108in}}{\pgfqpoint{0.931033in}{0.848408in}}{\pgfqpoint{0.927127in}{0.852314in}}%
\pgfpathcurveto{\pgfqpoint{0.923220in}{0.856221in}}{\pgfqpoint{0.917920in}{0.858416in}}{\pgfqpoint{0.912395in}{0.858416in}}%
\pgfpathcurveto{\pgfqpoint{0.906870in}{0.858416in}}{\pgfqpoint{0.901571in}{0.856221in}}{\pgfqpoint{0.897664in}{0.852314in}}%
\pgfpathcurveto{\pgfqpoint{0.893757in}{0.848408in}}{\pgfqpoint{0.891562in}{0.843108in}}{\pgfqpoint{0.891562in}{0.837583in}}%
\pgfpathcurveto{\pgfqpoint{0.891562in}{0.832058in}}{\pgfqpoint{0.893757in}{0.826758in}}{\pgfqpoint{0.897664in}{0.822852in}}%
\pgfpathcurveto{\pgfqpoint{0.901571in}{0.818945in}}{\pgfqpoint{0.906870in}{0.816750in}}{\pgfqpoint{0.912395in}{0.816750in}}%
\pgfpathclose%
\pgfusepath{stroke,fill}%
\end{pgfscope}%
\begin{pgfscope}%
\pgfpathrectangle{\pgfqpoint{0.403630in}{0.251348in}}{\pgfqpoint{1.395000in}{1.359000in}}%
\pgfusepath{clip}%
\pgfsetbuttcap%
\pgfsetroundjoin%
\definecolor{currentfill}{rgb}{0.333333,0.333333,0.333333}%
\pgfsetfillcolor{currentfill}%
\pgfsetlinewidth{0.501875pt}%
\definecolor{currentstroke}{rgb}{0.333333,0.333333,0.333333}%
\pgfsetstrokecolor{currentstroke}%
\pgfsetdash{}{0pt}%
\pgfpathmoveto{\pgfqpoint{1.328160in}{0.816750in}}%
\pgfpathcurveto{\pgfqpoint{1.333685in}{0.816750in}}{\pgfqpoint{1.338984in}{0.818945in}}{\pgfqpoint{1.342891in}{0.822852in}}%
\pgfpathcurveto{\pgfqpoint{1.346798in}{0.826758in}}{\pgfqpoint{1.348993in}{0.832058in}}{\pgfqpoint{1.348993in}{0.837583in}}%
\pgfpathcurveto{\pgfqpoint{1.348993in}{0.843108in}}{\pgfqpoint{1.346798in}{0.848408in}}{\pgfqpoint{1.342891in}{0.852314in}}%
\pgfpathcurveto{\pgfqpoint{1.338984in}{0.856221in}}{\pgfqpoint{1.333685in}{0.858416in}}{\pgfqpoint{1.328160in}{0.858416in}}%
\pgfpathcurveto{\pgfqpoint{1.322635in}{0.858416in}}{\pgfqpoint{1.317335in}{0.856221in}}{\pgfqpoint{1.313428in}{0.852314in}}%
\pgfpathcurveto{\pgfqpoint{1.309522in}{0.848408in}}{\pgfqpoint{1.307326in}{0.843108in}}{\pgfqpoint{1.307326in}{0.837583in}}%
\pgfpathcurveto{\pgfqpoint{1.307326in}{0.832058in}}{\pgfqpoint{1.309522in}{0.826758in}}{\pgfqpoint{1.313428in}{0.822852in}}%
\pgfpathcurveto{\pgfqpoint{1.317335in}{0.818945in}}{\pgfqpoint{1.322635in}{0.816750in}}{\pgfqpoint{1.328160in}{0.816750in}}%
\pgfpathclose%
\pgfusepath{stroke,fill}%
\end{pgfscope}%
\begin{pgfscope}%
\pgfpathrectangle{\pgfqpoint{0.403630in}{0.251348in}}{\pgfqpoint{1.395000in}{1.359000in}}%
\pgfusepath{clip}%
\pgfsetbuttcap%
\pgfsetroundjoin%
\definecolor{currentfill}{rgb}{0.333333,0.333333,0.333333}%
\pgfsetfillcolor{currentfill}%
\pgfsetlinewidth{0.501875pt}%
\definecolor{currentstroke}{rgb}{0.333333,0.333333,0.333333}%
\pgfsetstrokecolor{currentstroke}%
\pgfsetdash{}{0pt}%
\pgfpathmoveto{\pgfqpoint{1.224219in}{0.736808in}}%
\pgfpathcurveto{\pgfqpoint{1.229744in}{0.736808in}}{\pgfqpoint{1.235043in}{0.739004in}}{\pgfqpoint{1.238950in}{0.742910in}}%
\pgfpathcurveto{\pgfqpoint{1.242857in}{0.746817in}}{\pgfqpoint{1.245052in}{0.752117in}}{\pgfqpoint{1.245052in}{0.757642in}}%
\pgfpathcurveto{\pgfqpoint{1.245052in}{0.763167in}}{\pgfqpoint{1.242857in}{0.768466in}}{\pgfqpoint{1.238950in}{0.772373in}}%
\pgfpathcurveto{\pgfqpoint{1.235043in}{0.776280in}}{\pgfqpoint{1.229744in}{0.778475in}}{\pgfqpoint{1.224219in}{0.778475in}}%
\pgfpathcurveto{\pgfqpoint{1.218694in}{0.778475in}}{\pgfqpoint{1.213394in}{0.776280in}}{\pgfqpoint{1.209487in}{0.772373in}}%
\pgfpathcurveto{\pgfqpoint{1.205580in}{0.768466in}}{\pgfqpoint{1.203385in}{0.763167in}}{\pgfqpoint{1.203385in}{0.757642in}}%
\pgfpathcurveto{\pgfqpoint{1.203385in}{0.752117in}}{\pgfqpoint{1.205580in}{0.746817in}}{\pgfqpoint{1.209487in}{0.742910in}}%
\pgfpathcurveto{\pgfqpoint{1.213394in}{0.739004in}}{\pgfqpoint{1.218694in}{0.736808in}}{\pgfqpoint{1.224219in}{0.736808in}}%
\pgfpathclose%
\pgfusepath{stroke,fill}%
\end{pgfscope}%
\begin{pgfscope}%
\pgfpathrectangle{\pgfqpoint{0.403630in}{0.251348in}}{\pgfqpoint{1.395000in}{1.359000in}}%
\pgfusepath{clip}%
\pgfsetbuttcap%
\pgfsetroundjoin%
\definecolor{currentfill}{rgb}{0.333333,0.333333,0.333333}%
\pgfsetfillcolor{currentfill}%
\pgfsetlinewidth{0.501875pt}%
\definecolor{currentstroke}{rgb}{0.333333,0.333333,0.333333}%
\pgfsetstrokecolor{currentstroke}%
\pgfsetdash{}{0pt}%
\pgfpathmoveto{\pgfqpoint{1.317219in}{0.918008in}}%
\pgfpathcurveto{\pgfqpoint{1.322744in}{0.918008in}}{\pgfqpoint{1.328043in}{0.920204in}}{\pgfqpoint{1.331950in}{0.924110in}}%
\pgfpathcurveto{\pgfqpoint{1.335857in}{0.928017in}}{\pgfqpoint{1.338052in}{0.933317in}}{\pgfqpoint{1.338052in}{0.938842in}}%
\pgfpathcurveto{\pgfqpoint{1.338052in}{0.944367in}}{\pgfqpoint{1.335857in}{0.949666in}}{\pgfqpoint{1.331950in}{0.953573in}}%
\pgfpathcurveto{\pgfqpoint{1.328043in}{0.957480in}}{\pgfqpoint{1.322744in}{0.959675in}}{\pgfqpoint{1.317219in}{0.959675in}}%
\pgfpathcurveto{\pgfqpoint{1.311694in}{0.959675in}}{\pgfqpoint{1.306394in}{0.957480in}}{\pgfqpoint{1.302487in}{0.953573in}}%
\pgfpathcurveto{\pgfqpoint{1.298580in}{0.949666in}}{\pgfqpoint{1.296385in}{0.944367in}}{\pgfqpoint{1.296385in}{0.938842in}}%
\pgfpathcurveto{\pgfqpoint{1.296385in}{0.933317in}}{\pgfqpoint{1.298580in}{0.928017in}}{\pgfqpoint{1.302487in}{0.924110in}}%
\pgfpathcurveto{\pgfqpoint{1.306394in}{0.920204in}}{\pgfqpoint{1.311694in}{0.918008in}}{\pgfqpoint{1.317219in}{0.918008in}}%
\pgfpathclose%
\pgfusepath{stroke,fill}%
\end{pgfscope}%
\begin{pgfscope}%
\pgfpathrectangle{\pgfqpoint{0.403630in}{0.251348in}}{\pgfqpoint{1.395000in}{1.359000in}}%
\pgfusepath{clip}%
\pgfsetbuttcap%
\pgfsetroundjoin%
\definecolor{currentfill}{rgb}{0.333333,0.333333,0.333333}%
\pgfsetfillcolor{currentfill}%
\pgfsetlinewidth{0.501875pt}%
\definecolor{currentstroke}{rgb}{0.333333,0.333333,0.333333}%
\pgfsetstrokecolor{currentstroke}%
\pgfsetdash{}{0pt}%
\pgfpathmoveto{\pgfqpoint{1.213277in}{1.051244in}}%
\pgfpathcurveto{\pgfqpoint{1.218803in}{1.051244in}}{\pgfqpoint{1.224102in}{1.053439in}}{\pgfqpoint{1.228009in}{1.057346in}}%
\pgfpathcurveto{\pgfqpoint{1.231916in}{1.061252in}}{\pgfqpoint{1.234111in}{1.066552in}}{\pgfqpoint{1.234111in}{1.072077in}}%
\pgfpathcurveto{\pgfqpoint{1.234111in}{1.077602in}}{\pgfqpoint{1.231916in}{1.082902in}}{\pgfqpoint{1.228009in}{1.086808in}}%
\pgfpathcurveto{\pgfqpoint{1.224102in}{1.090715in}}{\pgfqpoint{1.218803in}{1.092910in}}{\pgfqpoint{1.213277in}{1.092910in}}%
\pgfpathcurveto{\pgfqpoint{1.207752in}{1.092910in}}{\pgfqpoint{1.202453in}{1.090715in}}{\pgfqpoint{1.198546in}{1.086808in}}%
\pgfpathcurveto{\pgfqpoint{1.194639in}{1.082902in}}{\pgfqpoint{1.192444in}{1.077602in}}{\pgfqpoint{1.192444in}{1.072077in}}%
\pgfpathcurveto{\pgfqpoint{1.192444in}{1.066552in}}{\pgfqpoint{1.194639in}{1.061252in}}{\pgfqpoint{1.198546in}{1.057346in}}%
\pgfpathcurveto{\pgfqpoint{1.202453in}{1.053439in}}{\pgfqpoint{1.207752in}{1.051244in}}{\pgfqpoint{1.213277in}{1.051244in}}%
\pgfpathclose%
\pgfusepath{stroke,fill}%
\end{pgfscope}%
\begin{pgfscope}%
\pgfpathrectangle{\pgfqpoint{0.403630in}{0.251348in}}{\pgfqpoint{1.395000in}{1.359000in}}%
\pgfusepath{clip}%
\pgfsetbuttcap%
\pgfsetroundjoin%
\definecolor{currentfill}{rgb}{0.333333,0.333333,0.333333}%
\pgfsetfillcolor{currentfill}%
\pgfsetlinewidth{0.501875pt}%
\definecolor{currentstroke}{rgb}{0.333333,0.333333,0.333333}%
\pgfsetstrokecolor{currentstroke}%
\pgfsetdash{}{0pt}%
\pgfpathmoveto{\pgfqpoint{1.071042in}{0.699503in}}%
\pgfpathcurveto{\pgfqpoint{1.076567in}{0.699503in}}{\pgfqpoint{1.081867in}{0.701698in}}{\pgfqpoint{1.085774in}{0.705604in}}%
\pgfpathcurveto{\pgfqpoint{1.089680in}{0.709511in}}{\pgfqpoint{1.091876in}{0.714811in}}{\pgfqpoint{1.091876in}{0.720336in}}%
\pgfpathcurveto{\pgfqpoint{1.091876in}{0.725861in}}{\pgfqpoint{1.089680in}{0.731160in}}{\pgfqpoint{1.085774in}{0.735067in}}%
\pgfpathcurveto{\pgfqpoint{1.081867in}{0.738974in}}{\pgfqpoint{1.076567in}{0.741169in}}{\pgfqpoint{1.071042in}{0.741169in}}%
\pgfpathcurveto{\pgfqpoint{1.065517in}{0.741169in}}{\pgfqpoint{1.060218in}{0.738974in}}{\pgfqpoint{1.056311in}{0.735067in}}%
\pgfpathcurveto{\pgfqpoint{1.052404in}{0.731160in}}{\pgfqpoint{1.050209in}{0.725861in}}{\pgfqpoint{1.050209in}{0.720336in}}%
\pgfpathcurveto{\pgfqpoint{1.050209in}{0.714811in}}{\pgfqpoint{1.052404in}{0.709511in}}{\pgfqpoint{1.056311in}{0.705604in}}%
\pgfpathcurveto{\pgfqpoint{1.060218in}{0.701698in}}{\pgfqpoint{1.065517in}{0.699503in}}{\pgfqpoint{1.071042in}{0.699503in}}%
\pgfpathclose%
\pgfusepath{stroke,fill}%
\end{pgfscope}%
\begin{pgfscope}%
\pgfpathrectangle{\pgfqpoint{0.403630in}{0.251348in}}{\pgfqpoint{1.395000in}{1.359000in}}%
\pgfusepath{clip}%
\pgfsetbuttcap%
\pgfsetroundjoin%
\definecolor{currentfill}{rgb}{0.333333,0.333333,0.333333}%
\pgfsetfillcolor{currentfill}%
\pgfsetlinewidth{0.501875pt}%
\definecolor{currentstroke}{rgb}{0.333333,0.333333,0.333333}%
\pgfsetstrokecolor{currentstroke}%
\pgfsetdash{}{0pt}%
\pgfpathmoveto{\pgfqpoint{0.906925in}{1.019267in}}%
\pgfpathcurveto{\pgfqpoint{0.912450in}{1.019267in}}{\pgfqpoint{0.917749in}{1.021462in}}{\pgfqpoint{0.921656in}{1.025369in}}%
\pgfpathcurveto{\pgfqpoint{0.925563in}{1.029276in}}{\pgfqpoint{0.927758in}{1.034576in}}{\pgfqpoint{0.927758in}{1.040101in}}%
\pgfpathcurveto{\pgfqpoint{0.927758in}{1.045626in}}{\pgfqpoint{0.925563in}{1.050925in}}{\pgfqpoint{0.921656in}{1.054832in}}%
\pgfpathcurveto{\pgfqpoint{0.917749in}{1.058739in}}{\pgfqpoint{0.912450in}{1.060934in}}{\pgfqpoint{0.906925in}{1.060934in}}%
\pgfpathcurveto{\pgfqpoint{0.901399in}{1.060934in}}{\pgfqpoint{0.896100in}{1.058739in}}{\pgfqpoint{0.892193in}{1.054832in}}%
\pgfpathcurveto{\pgfqpoint{0.888286in}{1.050925in}}{\pgfqpoint{0.886091in}{1.045626in}}{\pgfqpoint{0.886091in}{1.040101in}}%
\pgfpathcurveto{\pgfqpoint{0.886091in}{1.034576in}}{\pgfqpoint{0.888286in}{1.029276in}}{\pgfqpoint{0.892193in}{1.025369in}}%
\pgfpathcurveto{\pgfqpoint{0.896100in}{1.021462in}}{\pgfqpoint{0.901399in}{1.019267in}}{\pgfqpoint{0.906925in}{1.019267in}}%
\pgfpathclose%
\pgfusepath{stroke,fill}%
\end{pgfscope}%
\begin{pgfscope}%
\pgfpathrectangle{\pgfqpoint{0.403630in}{0.251348in}}{\pgfqpoint{1.395000in}{1.359000in}}%
\pgfusepath{clip}%
\pgfsetbuttcap%
\pgfsetroundjoin%
\definecolor{currentfill}{rgb}{0.333333,0.333333,0.333333}%
\pgfsetfillcolor{currentfill}%
\pgfsetlinewidth{0.501875pt}%
\definecolor{currentstroke}{rgb}{0.333333,0.333333,0.333333}%
\pgfsetstrokecolor{currentstroke}%
\pgfsetdash{}{0pt}%
\pgfpathmoveto{\pgfqpoint{0.939748in}{1.035255in}}%
\pgfpathcurveto{\pgfqpoint{0.945273in}{1.035255in}}{\pgfqpoint{0.950573in}{1.037451in}}{\pgfqpoint{0.954479in}{1.041357in}}%
\pgfpathcurveto{\pgfqpoint{0.958386in}{1.045264in}}{\pgfqpoint{0.960581in}{1.050564in}}{\pgfqpoint{0.960581in}{1.056089in}}%
\pgfpathcurveto{\pgfqpoint{0.960581in}{1.061614in}}{\pgfqpoint{0.958386in}{1.066913in}}{\pgfqpoint{0.954479in}{1.070820in}}%
\pgfpathcurveto{\pgfqpoint{0.950573in}{1.074727in}}{\pgfqpoint{0.945273in}{1.076922in}}{\pgfqpoint{0.939748in}{1.076922in}}%
\pgfpathcurveto{\pgfqpoint{0.934223in}{1.076922in}}{\pgfqpoint{0.928923in}{1.074727in}}{\pgfqpoint{0.925017in}{1.070820in}}%
\pgfpathcurveto{\pgfqpoint{0.921110in}{1.066913in}}{\pgfqpoint{0.918915in}{1.061614in}}{\pgfqpoint{0.918915in}{1.056089in}}%
\pgfpathcurveto{\pgfqpoint{0.918915in}{1.050564in}}{\pgfqpoint{0.921110in}{1.045264in}}{\pgfqpoint{0.925017in}{1.041357in}}%
\pgfpathcurveto{\pgfqpoint{0.928923in}{1.037451in}}{\pgfqpoint{0.934223in}{1.035255in}}{\pgfqpoint{0.939748in}{1.035255in}}%
\pgfpathclose%
\pgfusepath{stroke,fill}%
\end{pgfscope}%
\begin{pgfscope}%
\pgfpathrectangle{\pgfqpoint{0.403630in}{0.251348in}}{\pgfqpoint{1.395000in}{1.359000in}}%
\pgfusepath{clip}%
\pgfsetbuttcap%
\pgfsetroundjoin%
\definecolor{currentfill}{rgb}{0.333333,0.333333,0.333333}%
\pgfsetfillcolor{currentfill}%
\pgfsetlinewidth{0.501875pt}%
\definecolor{currentstroke}{rgb}{0.333333,0.333333,0.333333}%
\pgfsetstrokecolor{currentstroke}%
\pgfsetdash{}{0pt}%
\pgfpathmoveto{\pgfqpoint{1.328160in}{0.939326in}}%
\pgfpathcurveto{\pgfqpoint{1.333685in}{0.939326in}}{\pgfqpoint{1.338984in}{0.941521in}}{\pgfqpoint{1.342891in}{0.945428in}}%
\pgfpathcurveto{\pgfqpoint{1.346798in}{0.949335in}}{\pgfqpoint{1.348993in}{0.954634in}}{\pgfqpoint{1.348993in}{0.960159in}}%
\pgfpathcurveto{\pgfqpoint{1.348993in}{0.965684in}}{\pgfqpoint{1.346798in}{0.970984in}}{\pgfqpoint{1.342891in}{0.974891in}}%
\pgfpathcurveto{\pgfqpoint{1.338984in}{0.978798in}}{\pgfqpoint{1.333685in}{0.980993in}}{\pgfqpoint{1.328160in}{0.980993in}}%
\pgfpathcurveto{\pgfqpoint{1.322635in}{0.980993in}}{\pgfqpoint{1.317335in}{0.978798in}}{\pgfqpoint{1.313428in}{0.974891in}}%
\pgfpathcurveto{\pgfqpoint{1.309522in}{0.970984in}}{\pgfqpoint{1.307326in}{0.965684in}}{\pgfqpoint{1.307326in}{0.960159in}}%
\pgfpathcurveto{\pgfqpoint{1.307326in}{0.954634in}}{\pgfqpoint{1.309522in}{0.949335in}}{\pgfqpoint{1.313428in}{0.945428in}}%
\pgfpathcurveto{\pgfqpoint{1.317335in}{0.941521in}}{\pgfqpoint{1.322635in}{0.939326in}}{\pgfqpoint{1.328160in}{0.939326in}}%
\pgfpathclose%
\pgfusepath{stroke,fill}%
\end{pgfscope}%
\begin{pgfscope}%
\pgfpathrectangle{\pgfqpoint{0.403630in}{0.251348in}}{\pgfqpoint{1.395000in}{1.359000in}}%
\pgfusepath{clip}%
\pgfsetbuttcap%
\pgfsetroundjoin%
\definecolor{currentfill}{rgb}{0.333333,0.333333,0.333333}%
\pgfsetfillcolor{currentfill}%
\pgfsetlinewidth{0.501875pt}%
\definecolor{currentstroke}{rgb}{0.333333,0.333333,0.333333}%
\pgfsetstrokecolor{currentstroke}%
\pgfsetdash{}{0pt}%
\pgfpathmoveto{\pgfqpoint{0.895983in}{0.912679in}}%
\pgfpathcurveto{\pgfqpoint{0.901508in}{0.912679in}}{\pgfqpoint{0.906808in}{0.914874in}}{\pgfqpoint{0.910715in}{0.918781in}}%
\pgfpathcurveto{\pgfqpoint{0.914622in}{0.922688in}}{\pgfqpoint{0.916817in}{0.927987in}}{\pgfqpoint{0.916817in}{0.933512in}}%
\pgfpathcurveto{\pgfqpoint{0.916817in}{0.939037in}}{\pgfqpoint{0.914622in}{0.944337in}}{\pgfqpoint{0.910715in}{0.948244in}}%
\pgfpathcurveto{\pgfqpoint{0.906808in}{0.952151in}}{\pgfqpoint{0.901508in}{0.954346in}}{\pgfqpoint{0.895983in}{0.954346in}}%
\pgfpathcurveto{\pgfqpoint{0.890458in}{0.954346in}}{\pgfqpoint{0.885159in}{0.952151in}}{\pgfqpoint{0.881252in}{0.948244in}}%
\pgfpathcurveto{\pgfqpoint{0.877345in}{0.944337in}}{\pgfqpoint{0.875150in}{0.939037in}}{\pgfqpoint{0.875150in}{0.933512in}}%
\pgfpathcurveto{\pgfqpoint{0.875150in}{0.927987in}}{\pgfqpoint{0.877345in}{0.922688in}}{\pgfqpoint{0.881252in}{0.918781in}}%
\pgfpathcurveto{\pgfqpoint{0.885159in}{0.914874in}}{\pgfqpoint{0.890458in}{0.912679in}}{\pgfqpoint{0.895983in}{0.912679in}}%
\pgfpathclose%
\pgfusepath{stroke,fill}%
\end{pgfscope}%
\begin{pgfscope}%
\pgfpathrectangle{\pgfqpoint{0.403630in}{0.251348in}}{\pgfqpoint{1.395000in}{1.359000in}}%
\pgfusepath{clip}%
\pgfsetbuttcap%
\pgfsetroundjoin%
\definecolor{currentfill}{rgb}{0.333333,0.333333,0.333333}%
\pgfsetfillcolor{currentfill}%
\pgfsetlinewidth{0.501875pt}%
\definecolor{currentstroke}{rgb}{0.333333,0.333333,0.333333}%
\pgfsetstrokecolor{currentstroke}%
\pgfsetdash{}{0pt}%
\pgfpathmoveto{\pgfqpoint{1.246101in}{0.806091in}}%
\pgfpathcurveto{\pgfqpoint{1.251626in}{0.806091in}}{\pgfqpoint{1.256926in}{0.808286in}}{\pgfqpoint{1.260832in}{0.812193in}}%
\pgfpathcurveto{\pgfqpoint{1.264739in}{0.816100in}}{\pgfqpoint{1.266934in}{0.821399in}}{\pgfqpoint{1.266934in}{0.826924in}}%
\pgfpathcurveto{\pgfqpoint{1.266934in}{0.832449in}}{\pgfqpoint{1.264739in}{0.837749in}}{\pgfqpoint{1.260832in}{0.841656in}}%
\pgfpathcurveto{\pgfqpoint{1.256926in}{0.845562in}}{\pgfqpoint{1.251626in}{0.847757in}}{\pgfqpoint{1.246101in}{0.847757in}}%
\pgfpathcurveto{\pgfqpoint{1.240576in}{0.847757in}}{\pgfqpoint{1.235276in}{0.845562in}}{\pgfqpoint{1.231370in}{0.841656in}}%
\pgfpathcurveto{\pgfqpoint{1.227463in}{0.837749in}}{\pgfqpoint{1.225268in}{0.832449in}}{\pgfqpoint{1.225268in}{0.826924in}}%
\pgfpathcurveto{\pgfqpoint{1.225268in}{0.821399in}}{\pgfqpoint{1.227463in}{0.816100in}}{\pgfqpoint{1.231370in}{0.812193in}}%
\pgfpathcurveto{\pgfqpoint{1.235276in}{0.808286in}}{\pgfqpoint{1.240576in}{0.806091in}}{\pgfqpoint{1.246101in}{0.806091in}}%
\pgfpathclose%
\pgfusepath{stroke,fill}%
\end{pgfscope}%
\begin{pgfscope}%
\pgfpathrectangle{\pgfqpoint{0.403630in}{0.251348in}}{\pgfqpoint{1.395000in}{1.359000in}}%
\pgfusepath{clip}%
\pgfsetbuttcap%
\pgfsetroundjoin%
\definecolor{currentfill}{rgb}{0.333333,0.333333,0.333333}%
\pgfsetfillcolor{currentfill}%
\pgfsetlinewidth{0.501875pt}%
\definecolor{currentstroke}{rgb}{0.333333,0.333333,0.333333}%
\pgfsetstrokecolor{currentstroke}%
\pgfsetdash{}{0pt}%
\pgfpathmoveto{\pgfqpoint{1.103866in}{0.747467in}}%
\pgfpathcurveto{\pgfqpoint{1.109391in}{0.747467in}}{\pgfqpoint{1.114690in}{0.749662in}}{\pgfqpoint{1.118597in}{0.753569in}}%
\pgfpathcurveto{\pgfqpoint{1.122504in}{0.757476in}}{\pgfqpoint{1.124699in}{0.762776in}}{\pgfqpoint{1.124699in}{0.768301in}}%
\pgfpathcurveto{\pgfqpoint{1.124699in}{0.773826in}}{\pgfqpoint{1.122504in}{0.779125in}}{\pgfqpoint{1.118597in}{0.783032in}}%
\pgfpathcurveto{\pgfqpoint{1.114690in}{0.786939in}}{\pgfqpoint{1.109391in}{0.789134in}}{\pgfqpoint{1.103866in}{0.789134in}}%
\pgfpathcurveto{\pgfqpoint{1.098341in}{0.789134in}}{\pgfqpoint{1.093041in}{0.786939in}}{\pgfqpoint{1.089134in}{0.783032in}}%
\pgfpathcurveto{\pgfqpoint{1.085228in}{0.779125in}}{\pgfqpoint{1.083032in}{0.773826in}}{\pgfqpoint{1.083032in}{0.768301in}}%
\pgfpathcurveto{\pgfqpoint{1.083032in}{0.762776in}}{\pgfqpoint{1.085228in}{0.757476in}}{\pgfqpoint{1.089134in}{0.753569in}}%
\pgfpathcurveto{\pgfqpoint{1.093041in}{0.749662in}}{\pgfqpoint{1.098341in}{0.747467in}}{\pgfqpoint{1.103866in}{0.747467in}}%
\pgfpathclose%
\pgfusepath{stroke,fill}%
\end{pgfscope}%
\begin{pgfscope}%
\pgfpathrectangle{\pgfqpoint{0.403630in}{0.251348in}}{\pgfqpoint{1.395000in}{1.359000in}}%
\pgfusepath{clip}%
\pgfsetbuttcap%
\pgfsetroundjoin%
\definecolor{currentfill}{rgb}{0.333333,0.333333,0.333333}%
\pgfsetfillcolor{currentfill}%
\pgfsetlinewidth{0.501875pt}%
\definecolor{currentstroke}{rgb}{0.333333,0.333333,0.333333}%
\pgfsetstrokecolor{currentstroke}%
\pgfsetdash{}{0pt}%
\pgfpathmoveto{\pgfqpoint{0.939748in}{1.024597in}}%
\pgfpathcurveto{\pgfqpoint{0.945273in}{1.024597in}}{\pgfqpoint{0.950573in}{1.026792in}}{\pgfqpoint{0.954479in}{1.030699in}}%
\pgfpathcurveto{\pgfqpoint{0.958386in}{1.034605in}}{\pgfqpoint{0.960581in}{1.039905in}}{\pgfqpoint{0.960581in}{1.045430in}}%
\pgfpathcurveto{\pgfqpoint{0.960581in}{1.050955in}}{\pgfqpoint{0.958386in}{1.056255in}}{\pgfqpoint{0.954479in}{1.060161in}}%
\pgfpathcurveto{\pgfqpoint{0.950573in}{1.064068in}}{\pgfqpoint{0.945273in}{1.066263in}}{\pgfqpoint{0.939748in}{1.066263in}}%
\pgfpathcurveto{\pgfqpoint{0.934223in}{1.066263in}}{\pgfqpoint{0.928923in}{1.064068in}}{\pgfqpoint{0.925017in}{1.060161in}}%
\pgfpathcurveto{\pgfqpoint{0.921110in}{1.056255in}}{\pgfqpoint{0.918915in}{1.050955in}}{\pgfqpoint{0.918915in}{1.045430in}}%
\pgfpathcurveto{\pgfqpoint{0.918915in}{1.039905in}}{\pgfqpoint{0.921110in}{1.034605in}}{\pgfqpoint{0.925017in}{1.030699in}}%
\pgfpathcurveto{\pgfqpoint{0.928923in}{1.026792in}}{\pgfqpoint{0.934223in}{1.024597in}}{\pgfqpoint{0.939748in}{1.024597in}}%
\pgfpathclose%
\pgfusepath{stroke,fill}%
\end{pgfscope}%
\begin{pgfscope}%
\pgfpathrectangle{\pgfqpoint{0.403630in}{0.251348in}}{\pgfqpoint{1.395000in}{1.359000in}}%
\pgfusepath{clip}%
\pgfsetbuttcap%
\pgfsetroundjoin%
\definecolor{currentfill}{rgb}{0.333333,0.333333,0.333333}%
\pgfsetfillcolor{currentfill}%
\pgfsetlinewidth{0.501875pt}%
\definecolor{currentstroke}{rgb}{0.333333,0.333333,0.333333}%
\pgfsetstrokecolor{currentstroke}%
\pgfsetdash{}{0pt}%
\pgfpathmoveto{\pgfqpoint{1.054630in}{0.715491in}}%
\pgfpathcurveto{\pgfqpoint{1.060155in}{0.715491in}}{\pgfqpoint{1.065455in}{0.717686in}}{\pgfqpoint{1.069362in}{0.721593in}}%
\pgfpathcurveto{\pgfqpoint{1.073269in}{0.725500in}}{\pgfqpoint{1.075464in}{0.730799in}}{\pgfqpoint{1.075464in}{0.736324in}}%
\pgfpathcurveto{\pgfqpoint{1.075464in}{0.741849in}}{\pgfqpoint{1.073269in}{0.747149in}}{\pgfqpoint{1.069362in}{0.751056in}}%
\pgfpathcurveto{\pgfqpoint{1.065455in}{0.754962in}}{\pgfqpoint{1.060155in}{0.757157in}}{\pgfqpoint{1.054630in}{0.757157in}}%
\pgfpathcurveto{\pgfqpoint{1.049105in}{0.757157in}}{\pgfqpoint{1.043806in}{0.754962in}}{\pgfqpoint{1.039899in}{0.751056in}}%
\pgfpathcurveto{\pgfqpoint{1.035992in}{0.747149in}}{\pgfqpoint{1.033797in}{0.741849in}}{\pgfqpoint{1.033797in}{0.736324in}}%
\pgfpathcurveto{\pgfqpoint{1.033797in}{0.730799in}}{\pgfqpoint{1.035992in}{0.725500in}}{\pgfqpoint{1.039899in}{0.721593in}}%
\pgfpathcurveto{\pgfqpoint{1.043806in}{0.717686in}}{\pgfqpoint{1.049105in}{0.715491in}}{\pgfqpoint{1.054630in}{0.715491in}}%
\pgfpathclose%
\pgfusepath{stroke,fill}%
\end{pgfscope}%
\begin{pgfscope}%
\pgfpathrectangle{\pgfqpoint{0.403630in}{0.251348in}}{\pgfqpoint{1.395000in}{1.359000in}}%
\pgfusepath{clip}%
\pgfsetbuttcap%
\pgfsetroundjoin%
\definecolor{currentfill}{rgb}{0.333333,0.333333,0.333333}%
\pgfsetfillcolor{currentfill}%
\pgfsetlinewidth{0.501875pt}%
\definecolor{currentstroke}{rgb}{0.333333,0.333333,0.333333}%
\pgfsetstrokecolor{currentstroke}%
\pgfsetdash{}{0pt}%
\pgfpathmoveto{\pgfqpoint{1.295336in}{0.811420in}}%
\pgfpathcurveto{\pgfqpoint{1.300861in}{0.811420in}}{\pgfqpoint{1.306161in}{0.813615in}}{\pgfqpoint{1.310068in}{0.817522in}}%
\pgfpathcurveto{\pgfqpoint{1.313975in}{0.821429in}}{\pgfqpoint{1.316170in}{0.826728in}}{\pgfqpoint{1.316170in}{0.832254in}}%
\pgfpathcurveto{\pgfqpoint{1.316170in}{0.837779in}}{\pgfqpoint{1.313975in}{0.843078in}}{\pgfqpoint{1.310068in}{0.846985in}}%
\pgfpathcurveto{\pgfqpoint{1.306161in}{0.850892in}}{\pgfqpoint{1.300861in}{0.853087in}}{\pgfqpoint{1.295336in}{0.853087in}}%
\pgfpathcurveto{\pgfqpoint{1.289811in}{0.853087in}}{\pgfqpoint{1.284512in}{0.850892in}}{\pgfqpoint{1.280605in}{0.846985in}}%
\pgfpathcurveto{\pgfqpoint{1.276698in}{0.843078in}}{\pgfqpoint{1.274503in}{0.837779in}}{\pgfqpoint{1.274503in}{0.832254in}}%
\pgfpathcurveto{\pgfqpoint{1.274503in}{0.826728in}}{\pgfqpoint{1.276698in}{0.821429in}}{\pgfqpoint{1.280605in}{0.817522in}}%
\pgfpathcurveto{\pgfqpoint{1.284512in}{0.813615in}}{\pgfqpoint{1.289811in}{0.811420in}}{\pgfqpoint{1.295336in}{0.811420in}}%
\pgfpathclose%
\pgfusepath{stroke,fill}%
\end{pgfscope}%
\begin{pgfscope}%
\pgfpathrectangle{\pgfqpoint{0.403630in}{0.251348in}}{\pgfqpoint{1.395000in}{1.359000in}}%
\pgfusepath{clip}%
\pgfsetbuttcap%
\pgfsetroundjoin%
\definecolor{currentfill}{rgb}{0.333333,0.333333,0.333333}%
\pgfsetfillcolor{currentfill}%
\pgfsetlinewidth{0.501875pt}%
\definecolor{currentstroke}{rgb}{0.333333,0.333333,0.333333}%
\pgfsetstrokecolor{currentstroke}%
\pgfsetdash{}{0pt}%
\pgfpathmoveto{\pgfqpoint{0.912395in}{0.965973in}}%
\pgfpathcurveto{\pgfqpoint{0.917920in}{0.965973in}}{\pgfqpoint{0.923220in}{0.968168in}}{\pgfqpoint{0.927127in}{0.972075in}}%
\pgfpathcurveto{\pgfqpoint{0.931033in}{0.975982in}}{\pgfqpoint{0.933228in}{0.981281in}}{\pgfqpoint{0.933228in}{0.986806in}}%
\pgfpathcurveto{\pgfqpoint{0.933228in}{0.992332in}}{\pgfqpoint{0.931033in}{0.997631in}}{\pgfqpoint{0.927127in}{1.001538in}}%
\pgfpathcurveto{\pgfqpoint{0.923220in}{1.005445in}}{\pgfqpoint{0.917920in}{1.007640in}}{\pgfqpoint{0.912395in}{1.007640in}}%
\pgfpathcurveto{\pgfqpoint{0.906870in}{1.007640in}}{\pgfqpoint{0.901571in}{1.005445in}}{\pgfqpoint{0.897664in}{1.001538in}}%
\pgfpathcurveto{\pgfqpoint{0.893757in}{0.997631in}}{\pgfqpoint{0.891562in}{0.992332in}}{\pgfqpoint{0.891562in}{0.986806in}}%
\pgfpathcurveto{\pgfqpoint{0.891562in}{0.981281in}}{\pgfqpoint{0.893757in}{0.975982in}}{\pgfqpoint{0.897664in}{0.972075in}}%
\pgfpathcurveto{\pgfqpoint{0.901571in}{0.968168in}}{\pgfqpoint{0.906870in}{0.965973in}}{\pgfqpoint{0.912395in}{0.965973in}}%
\pgfpathclose%
\pgfusepath{stroke,fill}%
\end{pgfscope}%
\begin{pgfscope}%
\pgfpathrectangle{\pgfqpoint{0.403630in}{0.251348in}}{\pgfqpoint{1.395000in}{1.359000in}}%
\pgfusepath{clip}%
\pgfsetbuttcap%
\pgfsetroundjoin%
\definecolor{currentfill}{rgb}{0.333333,0.333333,0.333333}%
\pgfsetfillcolor{currentfill}%
\pgfsetlinewidth{0.501875pt}%
\definecolor{currentstroke}{rgb}{0.333333,0.333333,0.333333}%
\pgfsetstrokecolor{currentstroke}%
\pgfsetdash{}{0pt}%
\pgfpathmoveto{\pgfqpoint{1.054630in}{1.147173in}}%
\pgfpathcurveto{\pgfqpoint{1.060155in}{1.147173in}}{\pgfqpoint{1.065455in}{1.149368in}}{\pgfqpoint{1.069362in}{1.153275in}}%
\pgfpathcurveto{\pgfqpoint{1.073269in}{1.157182in}}{\pgfqpoint{1.075464in}{1.162481in}}{\pgfqpoint{1.075464in}{1.168006in}}%
\pgfpathcurveto{\pgfqpoint{1.075464in}{1.173532in}}{\pgfqpoint{1.073269in}{1.178831in}}{\pgfqpoint{1.069362in}{1.182738in}}%
\pgfpathcurveto{\pgfqpoint{1.065455in}{1.186645in}}{\pgfqpoint{1.060155in}{1.188840in}}{\pgfqpoint{1.054630in}{1.188840in}}%
\pgfpathcurveto{\pgfqpoint{1.049105in}{1.188840in}}{\pgfqpoint{1.043806in}{1.186645in}}{\pgfqpoint{1.039899in}{1.182738in}}%
\pgfpathcurveto{\pgfqpoint{1.035992in}{1.178831in}}{\pgfqpoint{1.033797in}{1.173532in}}{\pgfqpoint{1.033797in}{1.168006in}}%
\pgfpathcurveto{\pgfqpoint{1.033797in}{1.162481in}}{\pgfqpoint{1.035992in}{1.157182in}}{\pgfqpoint{1.039899in}{1.153275in}}%
\pgfpathcurveto{\pgfqpoint{1.043806in}{1.149368in}}{\pgfqpoint{1.049105in}{1.147173in}}{\pgfqpoint{1.054630in}{1.147173in}}%
\pgfpathclose%
\pgfusepath{stroke,fill}%
\end{pgfscope}%
\begin{pgfscope}%
\pgfpathrectangle{\pgfqpoint{0.403630in}{0.251348in}}{\pgfqpoint{1.395000in}{1.359000in}}%
\pgfusepath{clip}%
\pgfsetbuttcap%
\pgfsetroundjoin%
\definecolor{currentfill}{rgb}{0.333333,0.333333,0.333333}%
\pgfsetfillcolor{currentfill}%
\pgfsetlinewidth{0.501875pt}%
\definecolor{currentstroke}{rgb}{0.333333,0.333333,0.333333}%
\pgfsetstrokecolor{currentstroke}%
\pgfsetdash{}{0pt}%
\pgfpathmoveto{\pgfqpoint{0.917866in}{0.870044in}}%
\pgfpathcurveto{\pgfqpoint{0.923391in}{0.870044in}}{\pgfqpoint{0.928690in}{0.872239in}}{\pgfqpoint{0.932597in}{0.876146in}}%
\pgfpathcurveto{\pgfqpoint{0.936504in}{0.880052in}}{\pgfqpoint{0.938699in}{0.885352in}}{\pgfqpoint{0.938699in}{0.890877in}}%
\pgfpathcurveto{\pgfqpoint{0.938699in}{0.896402in}}{\pgfqpoint{0.936504in}{0.901702in}}{\pgfqpoint{0.932597in}{0.905608in}}%
\pgfpathcurveto{\pgfqpoint{0.928690in}{0.909515in}}{\pgfqpoint{0.923391in}{0.911710in}}{\pgfqpoint{0.917866in}{0.911710in}}%
\pgfpathcurveto{\pgfqpoint{0.912341in}{0.911710in}}{\pgfqpoint{0.907041in}{0.909515in}}{\pgfqpoint{0.903134in}{0.905608in}}%
\pgfpathcurveto{\pgfqpoint{0.899228in}{0.901702in}}{\pgfqpoint{0.897032in}{0.896402in}}{\pgfqpoint{0.897032in}{0.890877in}}%
\pgfpathcurveto{\pgfqpoint{0.897032in}{0.885352in}}{\pgfqpoint{0.899228in}{0.880052in}}{\pgfqpoint{0.903134in}{0.876146in}}%
\pgfpathcurveto{\pgfqpoint{0.907041in}{0.872239in}}{\pgfqpoint{0.912341in}{0.870044in}}{\pgfqpoint{0.917866in}{0.870044in}}%
\pgfpathclose%
\pgfusepath{stroke,fill}%
\end{pgfscope}%
\begin{pgfscope}%
\pgfpathrectangle{\pgfqpoint{0.403630in}{0.251348in}}{\pgfqpoint{1.395000in}{1.359000in}}%
\pgfusepath{clip}%
\pgfsetbuttcap%
\pgfsetroundjoin%
\definecolor{currentfill}{rgb}{0.333333,0.333333,0.333333}%
\pgfsetfillcolor{currentfill}%
\pgfsetlinewidth{0.501875pt}%
\definecolor{currentstroke}{rgb}{0.333333,0.333333,0.333333}%
\pgfsetstrokecolor{currentstroke}%
\pgfsetdash{}{0pt}%
\pgfpathmoveto{\pgfqpoint{0.874101in}{0.854055in}}%
\pgfpathcurveto{\pgfqpoint{0.879626in}{0.854055in}}{\pgfqpoint{0.884926in}{0.856251in}}{\pgfqpoint{0.888832in}{0.860157in}}%
\pgfpathcurveto{\pgfqpoint{0.892739in}{0.864064in}}{\pgfqpoint{0.894934in}{0.869364in}}{\pgfqpoint{0.894934in}{0.874889in}}%
\pgfpathcurveto{\pgfqpoint{0.894934in}{0.880414in}}{\pgfqpoint{0.892739in}{0.885713in}}{\pgfqpoint{0.888832in}{0.889620in}}%
\pgfpathcurveto{\pgfqpoint{0.884926in}{0.893527in}}{\pgfqpoint{0.879626in}{0.895722in}}{\pgfqpoint{0.874101in}{0.895722in}}%
\pgfpathcurveto{\pgfqpoint{0.868576in}{0.895722in}}{\pgfqpoint{0.863276in}{0.893527in}}{\pgfqpoint{0.859370in}{0.889620in}}%
\pgfpathcurveto{\pgfqpoint{0.855463in}{0.885713in}}{\pgfqpoint{0.853268in}{0.880414in}}{\pgfqpoint{0.853268in}{0.874889in}}%
\pgfpathcurveto{\pgfqpoint{0.853268in}{0.869364in}}{\pgfqpoint{0.855463in}{0.864064in}}{\pgfqpoint{0.859370in}{0.860157in}}%
\pgfpathcurveto{\pgfqpoint{0.863276in}{0.856251in}}{\pgfqpoint{0.868576in}{0.854055in}}{\pgfqpoint{0.874101in}{0.854055in}}%
\pgfpathclose%
\pgfusepath{stroke,fill}%
\end{pgfscope}%
\begin{pgfscope}%
\pgfpathrectangle{\pgfqpoint{0.403630in}{0.251348in}}{\pgfqpoint{1.395000in}{1.359000in}}%
\pgfusepath{clip}%
\pgfsetbuttcap%
\pgfsetroundjoin%
\definecolor{currentfill}{rgb}{0.333333,0.333333,0.333333}%
\pgfsetfillcolor{currentfill}%
\pgfsetlinewidth{0.501875pt}%
\definecolor{currentstroke}{rgb}{0.333333,0.333333,0.333333}%
\pgfsetstrokecolor{currentstroke}%
\pgfsetdash{}{0pt}%
\pgfpathmoveto{\pgfqpoint{1.306277in}{0.912679in}}%
\pgfpathcurveto{\pgfqpoint{1.311803in}{0.912679in}}{\pgfqpoint{1.317102in}{0.914874in}}{\pgfqpoint{1.321009in}{0.918781in}}%
\pgfpathcurveto{\pgfqpoint{1.324916in}{0.922688in}}{\pgfqpoint{1.327111in}{0.927987in}}{\pgfqpoint{1.327111in}{0.933512in}}%
\pgfpathcurveto{\pgfqpoint{1.327111in}{0.939037in}}{\pgfqpoint{1.324916in}{0.944337in}}{\pgfqpoint{1.321009in}{0.948244in}}%
\pgfpathcurveto{\pgfqpoint{1.317102in}{0.952151in}}{\pgfqpoint{1.311803in}{0.954346in}}{\pgfqpoint{1.306277in}{0.954346in}}%
\pgfpathcurveto{\pgfqpoint{1.300752in}{0.954346in}}{\pgfqpoint{1.295453in}{0.952151in}}{\pgfqpoint{1.291546in}{0.948244in}}%
\pgfpathcurveto{\pgfqpoint{1.287639in}{0.944337in}}{\pgfqpoint{1.285444in}{0.939037in}}{\pgfqpoint{1.285444in}{0.933512in}}%
\pgfpathcurveto{\pgfqpoint{1.285444in}{0.927987in}}{\pgfqpoint{1.287639in}{0.922688in}}{\pgfqpoint{1.291546in}{0.918781in}}%
\pgfpathcurveto{\pgfqpoint{1.295453in}{0.914874in}}{\pgfqpoint{1.300752in}{0.912679in}}{\pgfqpoint{1.306277in}{0.912679in}}%
\pgfpathclose%
\pgfusepath{stroke,fill}%
\end{pgfscope}%
\begin{pgfscope}%
\pgfpathrectangle{\pgfqpoint{0.403630in}{0.251348in}}{\pgfqpoint{1.395000in}{1.359000in}}%
\pgfusepath{clip}%
\pgfsetbuttcap%
\pgfsetroundjoin%
\definecolor{currentfill}{rgb}{0.333333,0.333333,0.333333}%
\pgfsetfillcolor{currentfill}%
\pgfsetlinewidth{0.501875pt}%
\definecolor{currentstroke}{rgb}{0.333333,0.333333,0.333333}%
\pgfsetstrokecolor{currentstroke}%
\pgfsetdash{}{0pt}%
\pgfpathmoveto{\pgfqpoint{1.060101in}{1.125855in}}%
\pgfpathcurveto{\pgfqpoint{1.065626in}{1.125855in}}{\pgfqpoint{1.070926in}{1.128051in}}{\pgfqpoint{1.074832in}{1.131957in}}%
\pgfpathcurveto{\pgfqpoint{1.078739in}{1.135864in}}{\pgfqpoint{1.080934in}{1.141164in}}{\pgfqpoint{1.080934in}{1.146689in}}%
\pgfpathcurveto{\pgfqpoint{1.080934in}{1.152214in}}{\pgfqpoint{1.078739in}{1.157513in}}{\pgfqpoint{1.074832in}{1.161420in}}%
\pgfpathcurveto{\pgfqpoint{1.070926in}{1.165327in}}{\pgfqpoint{1.065626in}{1.167522in}}{\pgfqpoint{1.060101in}{1.167522in}}%
\pgfpathcurveto{\pgfqpoint{1.054576in}{1.167522in}}{\pgfqpoint{1.049276in}{1.165327in}}{\pgfqpoint{1.045370in}{1.161420in}}%
\pgfpathcurveto{\pgfqpoint{1.041463in}{1.157513in}}{\pgfqpoint{1.039268in}{1.152214in}}{\pgfqpoint{1.039268in}{1.146689in}}%
\pgfpathcurveto{\pgfqpoint{1.039268in}{1.141164in}}{\pgfqpoint{1.041463in}{1.135864in}}{\pgfqpoint{1.045370in}{1.131957in}}%
\pgfpathcurveto{\pgfqpoint{1.049276in}{1.128051in}}{\pgfqpoint{1.054576in}{1.125855in}}{\pgfqpoint{1.060101in}{1.125855in}}%
\pgfpathclose%
\pgfusepath{stroke,fill}%
\end{pgfscope}%
\begin{pgfscope}%
\pgfpathrectangle{\pgfqpoint{0.403630in}{0.251348in}}{\pgfqpoint{1.395000in}{1.359000in}}%
\pgfusepath{clip}%
\pgfsetbuttcap%
\pgfsetroundjoin%
\definecolor{currentfill}{rgb}{0.333333,0.333333,0.333333}%
\pgfsetfillcolor{currentfill}%
\pgfsetlinewidth{0.501875pt}%
\definecolor{currentstroke}{rgb}{0.333333,0.333333,0.333333}%
\pgfsetstrokecolor{currentstroke}%
\pgfsetdash{}{0pt}%
\pgfpathmoveto{\pgfqpoint{1.043689in}{0.726150in}}%
\pgfpathcurveto{\pgfqpoint{1.049214in}{0.726150in}}{\pgfqpoint{1.054514in}{0.728345in}}{\pgfqpoint{1.058421in}{0.732252in}}%
\pgfpathcurveto{\pgfqpoint{1.062327in}{0.736158in}}{\pgfqpoint{1.064523in}{0.741458in}}{\pgfqpoint{1.064523in}{0.746983in}}%
\pgfpathcurveto{\pgfqpoint{1.064523in}{0.752508in}}{\pgfqpoint{1.062327in}{0.757808in}}{\pgfqpoint{1.058421in}{0.761714in}}%
\pgfpathcurveto{\pgfqpoint{1.054514in}{0.765621in}}{\pgfqpoint{1.049214in}{0.767816in}}{\pgfqpoint{1.043689in}{0.767816in}}%
\pgfpathcurveto{\pgfqpoint{1.038164in}{0.767816in}}{\pgfqpoint{1.032865in}{0.765621in}}{\pgfqpoint{1.028958in}{0.761714in}}%
\pgfpathcurveto{\pgfqpoint{1.025051in}{0.757808in}}{\pgfqpoint{1.022856in}{0.752508in}}{\pgfqpoint{1.022856in}{0.746983in}}%
\pgfpathcurveto{\pgfqpoint{1.022856in}{0.741458in}}{\pgfqpoint{1.025051in}{0.736158in}}{\pgfqpoint{1.028958in}{0.732252in}}%
\pgfpathcurveto{\pgfqpoint{1.032865in}{0.728345in}}{\pgfqpoint{1.038164in}{0.726150in}}{\pgfqpoint{1.043689in}{0.726150in}}%
\pgfpathclose%
\pgfusepath{stroke,fill}%
\end{pgfscope}%
\begin{pgfscope}%
\pgfpathrectangle{\pgfqpoint{0.403630in}{0.251348in}}{\pgfqpoint{1.395000in}{1.359000in}}%
\pgfusepath{clip}%
\pgfsetbuttcap%
\pgfsetroundjoin%
\definecolor{currentfill}{rgb}{0.333333,0.333333,0.333333}%
\pgfsetfillcolor{currentfill}%
\pgfsetlinewidth{0.501875pt}%
\definecolor{currentstroke}{rgb}{0.333333,0.333333,0.333333}%
\pgfsetstrokecolor{currentstroke}%
\pgfsetdash{}{0pt}%
\pgfpathmoveto{\pgfqpoint{1.060101in}{0.720820in}}%
\pgfpathcurveto{\pgfqpoint{1.065626in}{0.720820in}}{\pgfqpoint{1.070926in}{0.723015in}}{\pgfqpoint{1.074832in}{0.726922in}}%
\pgfpathcurveto{\pgfqpoint{1.078739in}{0.730829in}}{\pgfqpoint{1.080934in}{0.736128in}}{\pgfqpoint{1.080934in}{0.741654in}}%
\pgfpathcurveto{\pgfqpoint{1.080934in}{0.747179in}}{\pgfqpoint{1.078739in}{0.752478in}}{\pgfqpoint{1.074832in}{0.756385in}}%
\pgfpathcurveto{\pgfqpoint{1.070926in}{0.760292in}}{\pgfqpoint{1.065626in}{0.762487in}}{\pgfqpoint{1.060101in}{0.762487in}}%
\pgfpathcurveto{\pgfqpoint{1.054576in}{0.762487in}}{\pgfqpoint{1.049276in}{0.760292in}}{\pgfqpoint{1.045370in}{0.756385in}}%
\pgfpathcurveto{\pgfqpoint{1.041463in}{0.752478in}}{\pgfqpoint{1.039268in}{0.747179in}}{\pgfqpoint{1.039268in}{0.741654in}}%
\pgfpathcurveto{\pgfqpoint{1.039268in}{0.736128in}}{\pgfqpoint{1.041463in}{0.730829in}}{\pgfqpoint{1.045370in}{0.726922in}}%
\pgfpathcurveto{\pgfqpoint{1.049276in}{0.723015in}}{\pgfqpoint{1.054576in}{0.720820in}}{\pgfqpoint{1.060101in}{0.720820in}}%
\pgfpathclose%
\pgfusepath{stroke,fill}%
\end{pgfscope}%
\begin{pgfscope}%
\pgfpathrectangle{\pgfqpoint{0.403630in}{0.251348in}}{\pgfqpoint{1.395000in}{1.359000in}}%
\pgfusepath{clip}%
\pgfsetbuttcap%
\pgfsetroundjoin%
\definecolor{currentfill}{rgb}{0.333333,0.333333,0.333333}%
\pgfsetfillcolor{currentfill}%
\pgfsetlinewidth{0.501875pt}%
\definecolor{currentstroke}{rgb}{0.333333,0.333333,0.333333}%
\pgfsetstrokecolor{currentstroke}%
\pgfsetdash{}{0pt}%
\pgfpathmoveto{\pgfqpoint{1.262513in}{0.800761in}}%
\pgfpathcurveto{\pgfqpoint{1.268038in}{0.800761in}}{\pgfqpoint{1.273337in}{0.802956in}}{\pgfqpoint{1.277244in}{0.806863in}}%
\pgfpathcurveto{\pgfqpoint{1.281151in}{0.810770in}}{\pgfqpoint{1.283346in}{0.816070in}}{\pgfqpoint{1.283346in}{0.821595in}}%
\pgfpathcurveto{\pgfqpoint{1.283346in}{0.827120in}}{\pgfqpoint{1.281151in}{0.832419in}}{\pgfqpoint{1.277244in}{0.836326in}}%
\pgfpathcurveto{\pgfqpoint{1.273337in}{0.840233in}}{\pgfqpoint{1.268038in}{0.842428in}}{\pgfqpoint{1.262513in}{0.842428in}}%
\pgfpathcurveto{\pgfqpoint{1.256988in}{0.842428in}}{\pgfqpoint{1.251688in}{0.840233in}}{\pgfqpoint{1.247781in}{0.836326in}}%
\pgfpathcurveto{\pgfqpoint{1.243875in}{0.832419in}}{\pgfqpoint{1.241679in}{0.827120in}}{\pgfqpoint{1.241679in}{0.821595in}}%
\pgfpathcurveto{\pgfqpoint{1.241679in}{0.816070in}}{\pgfqpoint{1.243875in}{0.810770in}}{\pgfqpoint{1.247781in}{0.806863in}}%
\pgfpathcurveto{\pgfqpoint{1.251688in}{0.802956in}}{\pgfqpoint{1.256988in}{0.800761in}}{\pgfqpoint{1.262513in}{0.800761in}}%
\pgfpathclose%
\pgfusepath{stroke,fill}%
\end{pgfscope}%
\begin{pgfscope}%
\pgfpathrectangle{\pgfqpoint{0.403630in}{0.251348in}}{\pgfqpoint{1.395000in}{1.359000in}}%
\pgfusepath{clip}%
\pgfsetbuttcap%
\pgfsetroundjoin%
\definecolor{currentfill}{rgb}{0.333333,0.333333,0.333333}%
\pgfsetfillcolor{currentfill}%
\pgfsetlinewidth{0.501875pt}%
\definecolor{currentstroke}{rgb}{0.333333,0.333333,0.333333}%
\pgfsetstrokecolor{currentstroke}%
\pgfsetdash{}{0pt}%
\pgfpathmoveto{\pgfqpoint{1.218748in}{1.035255in}}%
\pgfpathcurveto{\pgfqpoint{1.224273in}{1.035255in}}{\pgfqpoint{1.229573in}{1.037451in}}{\pgfqpoint{1.233479in}{1.041357in}}%
\pgfpathcurveto{\pgfqpoint{1.237386in}{1.045264in}}{\pgfqpoint{1.239581in}{1.050564in}}{\pgfqpoint{1.239581in}{1.056089in}}%
\pgfpathcurveto{\pgfqpoint{1.239581in}{1.061614in}}{\pgfqpoint{1.237386in}{1.066913in}}{\pgfqpoint{1.233479in}{1.070820in}}%
\pgfpathcurveto{\pgfqpoint{1.229573in}{1.074727in}}{\pgfqpoint{1.224273in}{1.076922in}}{\pgfqpoint{1.218748in}{1.076922in}}%
\pgfpathcurveto{\pgfqpoint{1.213223in}{1.076922in}}{\pgfqpoint{1.207923in}{1.074727in}}{\pgfqpoint{1.204017in}{1.070820in}}%
\pgfpathcurveto{\pgfqpoint{1.200110in}{1.066913in}}{\pgfqpoint{1.197915in}{1.061614in}}{\pgfqpoint{1.197915in}{1.056089in}}%
\pgfpathcurveto{\pgfqpoint{1.197915in}{1.050564in}}{\pgfqpoint{1.200110in}{1.045264in}}{\pgfqpoint{1.204017in}{1.041357in}}%
\pgfpathcurveto{\pgfqpoint{1.207923in}{1.037451in}}{\pgfqpoint{1.213223in}{1.035255in}}{\pgfqpoint{1.218748in}{1.035255in}}%
\pgfpathclose%
\pgfusepath{stroke,fill}%
\end{pgfscope}%
\begin{pgfscope}%
\pgfpathrectangle{\pgfqpoint{0.403630in}{0.251348in}}{\pgfqpoint{1.395000in}{1.359000in}}%
\pgfusepath{clip}%
\pgfsetbuttcap%
\pgfsetroundjoin%
\definecolor{currentfill}{rgb}{0.333333,0.333333,0.333333}%
\pgfsetfillcolor{currentfill}%
\pgfsetlinewidth{0.501875pt}%
\definecolor{currentstroke}{rgb}{0.333333,0.333333,0.333333}%
\pgfsetstrokecolor{currentstroke}%
\pgfsetdash{}{0pt}%
\pgfpathmoveto{\pgfqpoint{1.010866in}{1.045914in}}%
\pgfpathcurveto{\pgfqpoint{1.016391in}{1.045914in}}{\pgfqpoint{1.021690in}{1.048109in}}{\pgfqpoint{1.025597in}{1.052016in}}%
\pgfpathcurveto{\pgfqpoint{1.029504in}{1.055923in}}{\pgfqpoint{1.031699in}{1.061223in}}{\pgfqpoint{1.031699in}{1.066748in}}%
\pgfpathcurveto{\pgfqpoint{1.031699in}{1.072273in}}{\pgfqpoint{1.029504in}{1.077572in}}{\pgfqpoint{1.025597in}{1.081479in}}%
\pgfpathcurveto{\pgfqpoint{1.021690in}{1.085386in}}{\pgfqpoint{1.016391in}{1.087581in}}{\pgfqpoint{1.010866in}{1.087581in}}%
\pgfpathcurveto{\pgfqpoint{1.005341in}{1.087581in}}{\pgfqpoint{1.000041in}{1.085386in}}{\pgfqpoint{0.996134in}{1.081479in}}%
\pgfpathcurveto{\pgfqpoint{0.992228in}{1.077572in}}{\pgfqpoint{0.990032in}{1.072273in}}{\pgfqpoint{0.990032in}{1.066748in}}%
\pgfpathcurveto{\pgfqpoint{0.990032in}{1.061223in}}{\pgfqpoint{0.992228in}{1.055923in}}{\pgfqpoint{0.996134in}{1.052016in}}%
\pgfpathcurveto{\pgfqpoint{1.000041in}{1.048109in}}{\pgfqpoint{1.005341in}{1.045914in}}{\pgfqpoint{1.010866in}{1.045914in}}%
\pgfpathclose%
\pgfusepath{stroke,fill}%
\end{pgfscope}%
\begin{pgfscope}%
\pgfpathrectangle{\pgfqpoint{0.403630in}{0.251348in}}{\pgfqpoint{1.395000in}{1.359000in}}%
\pgfusepath{clip}%
\pgfsetbuttcap%
\pgfsetroundjoin%
\definecolor{currentfill}{rgb}{0.333333,0.333333,0.333333}%
\pgfsetfillcolor{currentfill}%
\pgfsetlinewidth{0.501875pt}%
\definecolor{currentstroke}{rgb}{0.333333,0.333333,0.333333}%
\pgfsetstrokecolor{currentstroke}%
\pgfsetdash{}{0pt}%
\pgfpathmoveto{\pgfqpoint{0.906925in}{0.880703in}}%
\pgfpathcurveto{\pgfqpoint{0.912450in}{0.880703in}}{\pgfqpoint{0.917749in}{0.882898in}}{\pgfqpoint{0.921656in}{0.886804in}}%
\pgfpathcurveto{\pgfqpoint{0.925563in}{0.890711in}}{\pgfqpoint{0.927758in}{0.896011in}}{\pgfqpoint{0.927758in}{0.901536in}}%
\pgfpathcurveto{\pgfqpoint{0.927758in}{0.907061in}}{\pgfqpoint{0.925563in}{0.912360in}}{\pgfqpoint{0.921656in}{0.916267in}}%
\pgfpathcurveto{\pgfqpoint{0.917749in}{0.920174in}}{\pgfqpoint{0.912450in}{0.922369in}}{\pgfqpoint{0.906925in}{0.922369in}}%
\pgfpathcurveto{\pgfqpoint{0.901399in}{0.922369in}}{\pgfqpoint{0.896100in}{0.920174in}}{\pgfqpoint{0.892193in}{0.916267in}}%
\pgfpathcurveto{\pgfqpoint{0.888286in}{0.912360in}}{\pgfqpoint{0.886091in}{0.907061in}}{\pgfqpoint{0.886091in}{0.901536in}}%
\pgfpathcurveto{\pgfqpoint{0.886091in}{0.896011in}}{\pgfqpoint{0.888286in}{0.890711in}}{\pgfqpoint{0.892193in}{0.886804in}}%
\pgfpathcurveto{\pgfqpoint{0.896100in}{0.882898in}}{\pgfqpoint{0.901399in}{0.880703in}}{\pgfqpoint{0.906925in}{0.880703in}}%
\pgfpathclose%
\pgfusepath{stroke,fill}%
\end{pgfscope}%
\begin{pgfscope}%
\pgfpathrectangle{\pgfqpoint{0.403630in}{0.251348in}}{\pgfqpoint{1.395000in}{1.359000in}}%
\pgfusepath{clip}%
\pgfsetbuttcap%
\pgfsetroundjoin%
\definecolor{currentfill}{rgb}{0.333333,0.333333,0.333333}%
\pgfsetfillcolor{currentfill}%
\pgfsetlinewidth{0.501875pt}%
\definecolor{currentstroke}{rgb}{0.333333,0.333333,0.333333}%
\pgfsetstrokecolor{currentstroke}%
\pgfsetdash{}{0pt}%
\pgfpathmoveto{\pgfqpoint{1.300807in}{0.923338in}}%
\pgfpathcurveto{\pgfqpoint{1.306332in}{0.923338in}}{\pgfqpoint{1.311631in}{0.925533in}}{\pgfqpoint{1.315538in}{0.929440in}}%
\pgfpathcurveto{\pgfqpoint{1.319445in}{0.933347in}}{\pgfqpoint{1.321640in}{0.938646in}}{\pgfqpoint{1.321640in}{0.944171in}}%
\pgfpathcurveto{\pgfqpoint{1.321640in}{0.949696in}}{\pgfqpoint{1.319445in}{0.954996in}}{\pgfqpoint{1.315538in}{0.958903in}}%
\pgfpathcurveto{\pgfqpoint{1.311631in}{0.962809in}}{\pgfqpoint{1.306332in}{0.965005in}}{\pgfqpoint{1.300807in}{0.965005in}}%
\pgfpathcurveto{\pgfqpoint{1.295282in}{0.965005in}}{\pgfqpoint{1.289982in}{0.962809in}}{\pgfqpoint{1.286075in}{0.958903in}}%
\pgfpathcurveto{\pgfqpoint{1.282169in}{0.954996in}}{\pgfqpoint{1.279974in}{0.949696in}}{\pgfqpoint{1.279974in}{0.944171in}}%
\pgfpathcurveto{\pgfqpoint{1.279974in}{0.938646in}}{\pgfqpoint{1.282169in}{0.933347in}}{\pgfqpoint{1.286075in}{0.929440in}}%
\pgfpathcurveto{\pgfqpoint{1.289982in}{0.925533in}}{\pgfqpoint{1.295282in}{0.923338in}}{\pgfqpoint{1.300807in}{0.923338in}}%
\pgfpathclose%
\pgfusepath{stroke,fill}%
\end{pgfscope}%
\begin{pgfscope}%
\pgfpathrectangle{\pgfqpoint{0.403630in}{0.251348in}}{\pgfqpoint{1.395000in}{1.359000in}}%
\pgfusepath{clip}%
\pgfsetbuttcap%
\pgfsetroundjoin%
\definecolor{currentfill}{rgb}{0.333333,0.333333,0.333333}%
\pgfsetfillcolor{currentfill}%
\pgfsetlinewidth{0.501875pt}%
\definecolor{currentstroke}{rgb}{0.333333,0.333333,0.333333}%
\pgfsetstrokecolor{currentstroke}%
\pgfsetdash{}{0pt}%
\pgfpathmoveto{\pgfqpoint{0.994454in}{1.051244in}}%
\pgfpathcurveto{\pgfqpoint{0.999979in}{1.051244in}}{\pgfqpoint{1.005279in}{1.053439in}}{\pgfqpoint{1.009185in}{1.057346in}}%
\pgfpathcurveto{\pgfqpoint{1.013092in}{1.061252in}}{\pgfqpoint{1.015287in}{1.066552in}}{\pgfqpoint{1.015287in}{1.072077in}}%
\pgfpathcurveto{\pgfqpoint{1.015287in}{1.077602in}}{\pgfqpoint{1.013092in}{1.082902in}}{\pgfqpoint{1.009185in}{1.086808in}}%
\pgfpathcurveto{\pgfqpoint{1.005279in}{1.090715in}}{\pgfqpoint{0.999979in}{1.092910in}}{\pgfqpoint{0.994454in}{1.092910in}}%
\pgfpathcurveto{\pgfqpoint{0.988929in}{1.092910in}}{\pgfqpoint{0.983629in}{1.090715in}}{\pgfqpoint{0.979723in}{1.086808in}}%
\pgfpathcurveto{\pgfqpoint{0.975816in}{1.082902in}}{\pgfqpoint{0.973621in}{1.077602in}}{\pgfqpoint{0.973621in}{1.072077in}}%
\pgfpathcurveto{\pgfqpoint{0.973621in}{1.066552in}}{\pgfqpoint{0.975816in}{1.061252in}}{\pgfqpoint{0.979723in}{1.057346in}}%
\pgfpathcurveto{\pgfqpoint{0.983629in}{1.053439in}}{\pgfqpoint{0.988929in}{1.051244in}}{\pgfqpoint{0.994454in}{1.051244in}}%
\pgfpathclose%
\pgfusepath{stroke,fill}%
\end{pgfscope}%
\begin{pgfscope}%
\pgfpathrectangle{\pgfqpoint{0.403630in}{0.251348in}}{\pgfqpoint{1.395000in}{1.359000in}}%
\pgfusepath{clip}%
\pgfsetbuttcap%
\pgfsetroundjoin%
\definecolor{currentfill}{rgb}{0.333333,0.333333,0.333333}%
\pgfsetfillcolor{currentfill}%
\pgfsetlinewidth{0.501875pt}%
\definecolor{currentstroke}{rgb}{0.333333,0.333333,0.333333}%
\pgfsetstrokecolor{currentstroke}%
\pgfsetdash{}{0pt}%
\pgfpathmoveto{\pgfqpoint{0.945219in}{1.088550in}}%
\pgfpathcurveto{\pgfqpoint{0.950744in}{1.088550in}}{\pgfqpoint{0.956043in}{1.090745in}}{\pgfqpoint{0.959950in}{1.094652in}}%
\pgfpathcurveto{\pgfqpoint{0.963857in}{1.098558in}}{\pgfqpoint{0.966052in}{1.103858in}}{\pgfqpoint{0.966052in}{1.109383in}}%
\pgfpathcurveto{\pgfqpoint{0.966052in}{1.114908in}}{\pgfqpoint{0.963857in}{1.120208in}}{\pgfqpoint{0.959950in}{1.124114in}}%
\pgfpathcurveto{\pgfqpoint{0.956043in}{1.128021in}}{\pgfqpoint{0.950744in}{1.130216in}}{\pgfqpoint{0.945219in}{1.130216in}}%
\pgfpathcurveto{\pgfqpoint{0.939694in}{1.130216in}}{\pgfqpoint{0.934394in}{1.128021in}}{\pgfqpoint{0.930487in}{1.124114in}}%
\pgfpathcurveto{\pgfqpoint{0.926580in}{1.120208in}}{\pgfqpoint{0.924385in}{1.114908in}}{\pgfqpoint{0.924385in}{1.109383in}}%
\pgfpathcurveto{\pgfqpoint{0.924385in}{1.103858in}}{\pgfqpoint{0.926580in}{1.098558in}}{\pgfqpoint{0.930487in}{1.094652in}}%
\pgfpathcurveto{\pgfqpoint{0.934394in}{1.090745in}}{\pgfqpoint{0.939694in}{1.088550in}}{\pgfqpoint{0.945219in}{1.088550in}}%
\pgfpathclose%
\pgfusepath{stroke,fill}%
\end{pgfscope}%
\begin{pgfscope}%
\pgfpathrectangle{\pgfqpoint{0.403630in}{0.251348in}}{\pgfqpoint{1.395000in}{1.359000in}}%
\pgfusepath{clip}%
\pgfsetbuttcap%
\pgfsetroundjoin%
\definecolor{currentfill}{rgb}{0.333333,0.333333,0.333333}%
\pgfsetfillcolor{currentfill}%
\pgfsetlinewidth{0.501875pt}%
\definecolor{currentstroke}{rgb}{0.333333,0.333333,0.333333}%
\pgfsetstrokecolor{currentstroke}%
\pgfsetdash{}{0pt}%
\pgfpathmoveto{\pgfqpoint{1.136689in}{1.088550in}}%
\pgfpathcurveto{\pgfqpoint{1.142214in}{1.088550in}}{\pgfqpoint{1.147514in}{1.090745in}}{\pgfqpoint{1.151421in}{1.094652in}}%
\pgfpathcurveto{\pgfqpoint{1.155327in}{1.098558in}}{\pgfqpoint{1.157523in}{1.103858in}}{\pgfqpoint{1.157523in}{1.109383in}}%
\pgfpathcurveto{\pgfqpoint{1.157523in}{1.114908in}}{\pgfqpoint{1.155327in}{1.120208in}}{\pgfqpoint{1.151421in}{1.124114in}}%
\pgfpathcurveto{\pgfqpoint{1.147514in}{1.128021in}}{\pgfqpoint{1.142214in}{1.130216in}}{\pgfqpoint{1.136689in}{1.130216in}}%
\pgfpathcurveto{\pgfqpoint{1.131164in}{1.130216in}}{\pgfqpoint{1.125865in}{1.128021in}}{\pgfqpoint{1.121958in}{1.124114in}}%
\pgfpathcurveto{\pgfqpoint{1.118051in}{1.120208in}}{\pgfqpoint{1.115856in}{1.114908in}}{\pgfqpoint{1.115856in}{1.109383in}}%
\pgfpathcurveto{\pgfqpoint{1.115856in}{1.103858in}}{\pgfqpoint{1.118051in}{1.098558in}}{\pgfqpoint{1.121958in}{1.094652in}}%
\pgfpathcurveto{\pgfqpoint{1.125865in}{1.090745in}}{\pgfqpoint{1.131164in}{1.088550in}}{\pgfqpoint{1.136689in}{1.088550in}}%
\pgfpathclose%
\pgfusepath{stroke,fill}%
\end{pgfscope}%
\begin{pgfscope}%
\pgfpathrectangle{\pgfqpoint{0.403630in}{0.251348in}}{\pgfqpoint{1.395000in}{1.359000in}}%
\pgfusepath{clip}%
\pgfsetbuttcap%
\pgfsetroundjoin%
\definecolor{currentfill}{rgb}{0.333333,0.333333,0.333333}%
\pgfsetfillcolor{currentfill}%
\pgfsetlinewidth{0.501875pt}%
\definecolor{currentstroke}{rgb}{0.333333,0.333333,0.333333}%
\pgfsetstrokecolor{currentstroke}%
\pgfsetdash{}{0pt}%
\pgfpathmoveto{\pgfqpoint{0.939748in}{1.045914in}}%
\pgfpathcurveto{\pgfqpoint{0.945273in}{1.045914in}}{\pgfqpoint{0.950573in}{1.048109in}}{\pgfqpoint{0.954479in}{1.052016in}}%
\pgfpathcurveto{\pgfqpoint{0.958386in}{1.055923in}}{\pgfqpoint{0.960581in}{1.061223in}}{\pgfqpoint{0.960581in}{1.066748in}}%
\pgfpathcurveto{\pgfqpoint{0.960581in}{1.072273in}}{\pgfqpoint{0.958386in}{1.077572in}}{\pgfqpoint{0.954479in}{1.081479in}}%
\pgfpathcurveto{\pgfqpoint{0.950573in}{1.085386in}}{\pgfqpoint{0.945273in}{1.087581in}}{\pgfqpoint{0.939748in}{1.087581in}}%
\pgfpathcurveto{\pgfqpoint{0.934223in}{1.087581in}}{\pgfqpoint{0.928923in}{1.085386in}}{\pgfqpoint{0.925017in}{1.081479in}}%
\pgfpathcurveto{\pgfqpoint{0.921110in}{1.077572in}}{\pgfqpoint{0.918915in}{1.072273in}}{\pgfqpoint{0.918915in}{1.066748in}}%
\pgfpathcurveto{\pgfqpoint{0.918915in}{1.061223in}}{\pgfqpoint{0.921110in}{1.055923in}}{\pgfqpoint{0.925017in}{1.052016in}}%
\pgfpathcurveto{\pgfqpoint{0.928923in}{1.048109in}}{\pgfqpoint{0.934223in}{1.045914in}}{\pgfqpoint{0.939748in}{1.045914in}}%
\pgfpathclose%
\pgfusepath{stroke,fill}%
\end{pgfscope}%
\begin{pgfscope}%
\pgfpathrectangle{\pgfqpoint{0.403630in}{0.251348in}}{\pgfqpoint{1.395000in}{1.359000in}}%
\pgfusepath{clip}%
\pgfsetbuttcap%
\pgfsetroundjoin%
\definecolor{currentfill}{rgb}{0.333333,0.333333,0.333333}%
\pgfsetfillcolor{currentfill}%
\pgfsetlinewidth{0.501875pt}%
\definecolor{currentstroke}{rgb}{0.333333,0.333333,0.333333}%
\pgfsetstrokecolor{currentstroke}%
\pgfsetdash{}{0pt}%
\pgfpathmoveto{\pgfqpoint{1.158572in}{0.694173in}}%
\pgfpathcurveto{\pgfqpoint{1.164097in}{0.694173in}}{\pgfqpoint{1.169396in}{0.696368in}}{\pgfqpoint{1.173303in}{0.700275in}}%
\pgfpathcurveto{\pgfqpoint{1.177210in}{0.704182in}}{\pgfqpoint{1.179405in}{0.709481in}}{\pgfqpoint{1.179405in}{0.715006in}}%
\pgfpathcurveto{\pgfqpoint{1.179405in}{0.720532in}}{\pgfqpoint{1.177210in}{0.725831in}}{\pgfqpoint{1.173303in}{0.729738in}}%
\pgfpathcurveto{\pgfqpoint{1.169396in}{0.733645in}}{\pgfqpoint{1.164097in}{0.735840in}}{\pgfqpoint{1.158572in}{0.735840in}}%
\pgfpathcurveto{\pgfqpoint{1.153047in}{0.735840in}}{\pgfqpoint{1.147747in}{0.733645in}}{\pgfqpoint{1.143840in}{0.729738in}}%
\pgfpathcurveto{\pgfqpoint{1.139933in}{0.725831in}}{\pgfqpoint{1.137738in}{0.720532in}}{\pgfqpoint{1.137738in}{0.715006in}}%
\pgfpathcurveto{\pgfqpoint{1.137738in}{0.709481in}}{\pgfqpoint{1.139933in}{0.704182in}}{\pgfqpoint{1.143840in}{0.700275in}}%
\pgfpathcurveto{\pgfqpoint{1.147747in}{0.696368in}}{\pgfqpoint{1.153047in}{0.694173in}}{\pgfqpoint{1.158572in}{0.694173in}}%
\pgfpathclose%
\pgfusepath{stroke,fill}%
\end{pgfscope}%
\begin{pgfscope}%
\pgfpathrectangle{\pgfqpoint{0.403630in}{0.251348in}}{\pgfqpoint{1.395000in}{1.359000in}}%
\pgfusepath{clip}%
\pgfsetbuttcap%
\pgfsetroundjoin%
\definecolor{currentfill}{rgb}{0.333333,0.333333,0.333333}%
\pgfsetfillcolor{currentfill}%
\pgfsetlinewidth{0.501875pt}%
\definecolor{currentstroke}{rgb}{0.333333,0.333333,0.333333}%
\pgfsetstrokecolor{currentstroke}%
\pgfsetdash{}{0pt}%
\pgfpathmoveto{\pgfqpoint{1.076513in}{0.699503in}}%
\pgfpathcurveto{\pgfqpoint{1.082038in}{0.699503in}}{\pgfqpoint{1.087337in}{0.701698in}}{\pgfqpoint{1.091244in}{0.705604in}}%
\pgfpathcurveto{\pgfqpoint{1.095151in}{0.709511in}}{\pgfqpoint{1.097346in}{0.714811in}}{\pgfqpoint{1.097346in}{0.720336in}}%
\pgfpathcurveto{\pgfqpoint{1.097346in}{0.725861in}}{\pgfqpoint{1.095151in}{0.731160in}}{\pgfqpoint{1.091244in}{0.735067in}}%
\pgfpathcurveto{\pgfqpoint{1.087337in}{0.738974in}}{\pgfqpoint{1.082038in}{0.741169in}}{\pgfqpoint{1.076513in}{0.741169in}}%
\pgfpathcurveto{\pgfqpoint{1.070988in}{0.741169in}}{\pgfqpoint{1.065688in}{0.738974in}}{\pgfqpoint{1.061781in}{0.735067in}}%
\pgfpathcurveto{\pgfqpoint{1.057875in}{0.731160in}}{\pgfqpoint{1.055679in}{0.725861in}}{\pgfqpoint{1.055679in}{0.720336in}}%
\pgfpathcurveto{\pgfqpoint{1.055679in}{0.714811in}}{\pgfqpoint{1.057875in}{0.709511in}}{\pgfqpoint{1.061781in}{0.705604in}}%
\pgfpathcurveto{\pgfqpoint{1.065688in}{0.701698in}}{\pgfqpoint{1.070988in}{0.699503in}}{\pgfqpoint{1.076513in}{0.699503in}}%
\pgfpathclose%
\pgfusepath{stroke,fill}%
\end{pgfscope}%
\begin{pgfscope}%
\pgfpathrectangle{\pgfqpoint{0.403630in}{0.251348in}}{\pgfqpoint{1.395000in}{1.359000in}}%
\pgfusepath{clip}%
\pgfsetbuttcap%
\pgfsetroundjoin%
\definecolor{currentfill}{rgb}{0.333333,0.333333,0.333333}%
\pgfsetfillcolor{currentfill}%
\pgfsetlinewidth{0.501875pt}%
\definecolor{currentstroke}{rgb}{0.333333,0.333333,0.333333}%
\pgfsetstrokecolor{currentstroke}%
\pgfsetdash{}{0pt}%
\pgfpathmoveto{\pgfqpoint{1.333630in}{0.848726in}}%
\pgfpathcurveto{\pgfqpoint{1.339155in}{0.848726in}}{\pgfqpoint{1.344455in}{0.850921in}}{\pgfqpoint{1.348362in}{0.854828in}}%
\pgfpathcurveto{\pgfqpoint{1.352269in}{0.858735in}}{\pgfqpoint{1.354464in}{0.864034in}}{\pgfqpoint{1.354464in}{0.869559in}}%
\pgfpathcurveto{\pgfqpoint{1.354464in}{0.875084in}}{\pgfqpoint{1.352269in}{0.880384in}}{\pgfqpoint{1.348362in}{0.884291in}}%
\pgfpathcurveto{\pgfqpoint{1.344455in}{0.888198in}}{\pgfqpoint{1.339155in}{0.890393in}}{\pgfqpoint{1.333630in}{0.890393in}}%
\pgfpathcurveto{\pgfqpoint{1.328105in}{0.890393in}}{\pgfqpoint{1.322806in}{0.888198in}}{\pgfqpoint{1.318899in}{0.884291in}}%
\pgfpathcurveto{\pgfqpoint{1.314992in}{0.880384in}}{\pgfqpoint{1.312797in}{0.875084in}}{\pgfqpoint{1.312797in}{0.869559in}}%
\pgfpathcurveto{\pgfqpoint{1.312797in}{0.864034in}}{\pgfqpoint{1.314992in}{0.858735in}}{\pgfqpoint{1.318899in}{0.854828in}}%
\pgfpathcurveto{\pgfqpoint{1.322806in}{0.850921in}}{\pgfqpoint{1.328105in}{0.848726in}}{\pgfqpoint{1.333630in}{0.848726in}}%
\pgfpathclose%
\pgfusepath{stroke,fill}%
\end{pgfscope}%
\begin{pgfscope}%
\pgfpathrectangle{\pgfqpoint{0.403630in}{0.251348in}}{\pgfqpoint{1.395000in}{1.359000in}}%
\pgfusepath{clip}%
\pgfsetbuttcap%
\pgfsetroundjoin%
\definecolor{currentfill}{rgb}{0.333333,0.333333,0.333333}%
\pgfsetfillcolor{currentfill}%
\pgfsetlinewidth{0.501875pt}%
\definecolor{currentstroke}{rgb}{0.333333,0.333333,0.333333}%
\pgfsetstrokecolor{currentstroke}%
\pgfsetdash{}{0pt}%
\pgfpathmoveto{\pgfqpoint{1.054630in}{0.720820in}}%
\pgfpathcurveto{\pgfqpoint{1.060155in}{0.720820in}}{\pgfqpoint{1.065455in}{0.723015in}}{\pgfqpoint{1.069362in}{0.726922in}}%
\pgfpathcurveto{\pgfqpoint{1.073269in}{0.730829in}}{\pgfqpoint{1.075464in}{0.736128in}}{\pgfqpoint{1.075464in}{0.741654in}}%
\pgfpathcurveto{\pgfqpoint{1.075464in}{0.747179in}}{\pgfqpoint{1.073269in}{0.752478in}}{\pgfqpoint{1.069362in}{0.756385in}}%
\pgfpathcurveto{\pgfqpoint{1.065455in}{0.760292in}}{\pgfqpoint{1.060155in}{0.762487in}}{\pgfqpoint{1.054630in}{0.762487in}}%
\pgfpathcurveto{\pgfqpoint{1.049105in}{0.762487in}}{\pgfqpoint{1.043806in}{0.760292in}}{\pgfqpoint{1.039899in}{0.756385in}}%
\pgfpathcurveto{\pgfqpoint{1.035992in}{0.752478in}}{\pgfqpoint{1.033797in}{0.747179in}}{\pgfqpoint{1.033797in}{0.741654in}}%
\pgfpathcurveto{\pgfqpoint{1.033797in}{0.736128in}}{\pgfqpoint{1.035992in}{0.730829in}}{\pgfqpoint{1.039899in}{0.726922in}}%
\pgfpathcurveto{\pgfqpoint{1.043806in}{0.723015in}}{\pgfqpoint{1.049105in}{0.720820in}}{\pgfqpoint{1.054630in}{0.720820in}}%
\pgfpathclose%
\pgfusepath{stroke,fill}%
\end{pgfscope}%
\begin{pgfscope}%
\pgfpathrectangle{\pgfqpoint{0.403630in}{0.251348in}}{\pgfqpoint{1.395000in}{1.359000in}}%
\pgfusepath{clip}%
\pgfsetbuttcap%
\pgfsetroundjoin%
\definecolor{currentfill}{rgb}{0.333333,0.333333,0.333333}%
\pgfsetfillcolor{currentfill}%
\pgfsetlinewidth{0.501875pt}%
\definecolor{currentstroke}{rgb}{0.333333,0.333333,0.333333}%
\pgfsetstrokecolor{currentstroke}%
\pgfsetdash{}{0pt}%
\pgfpathmoveto{\pgfqpoint{1.278925in}{1.013938in}}%
\pgfpathcurveto{\pgfqpoint{1.284450in}{1.013938in}}{\pgfqpoint{1.289749in}{1.016133in}}{\pgfqpoint{1.293656in}{1.020040in}}%
\pgfpathcurveto{\pgfqpoint{1.297563in}{1.023947in}}{\pgfqpoint{1.299758in}{1.029246in}}{\pgfqpoint{1.299758in}{1.034771in}}%
\pgfpathcurveto{\pgfqpoint{1.299758in}{1.040296in}}{\pgfqpoint{1.297563in}{1.045596in}}{\pgfqpoint{1.293656in}{1.049503in}}%
\pgfpathcurveto{\pgfqpoint{1.289749in}{1.053409in}}{\pgfqpoint{1.284450in}{1.055605in}}{\pgfqpoint{1.278925in}{1.055605in}}%
\pgfpathcurveto{\pgfqpoint{1.273399in}{1.055605in}}{\pgfqpoint{1.268100in}{1.053409in}}{\pgfqpoint{1.264193in}{1.049503in}}%
\pgfpathcurveto{\pgfqpoint{1.260286in}{1.045596in}}{\pgfqpoint{1.258091in}{1.040296in}}{\pgfqpoint{1.258091in}{1.034771in}}%
\pgfpathcurveto{\pgfqpoint{1.258091in}{1.029246in}}{\pgfqpoint{1.260286in}{1.023947in}}{\pgfqpoint{1.264193in}{1.020040in}}%
\pgfpathcurveto{\pgfqpoint{1.268100in}{1.016133in}}{\pgfqpoint{1.273399in}{1.013938in}}{\pgfqpoint{1.278925in}{1.013938in}}%
\pgfpathclose%
\pgfusepath{stroke,fill}%
\end{pgfscope}%
\begin{pgfscope}%
\pgfpathrectangle{\pgfqpoint{0.403630in}{0.251348in}}{\pgfqpoint{1.395000in}{1.359000in}}%
\pgfusepath{clip}%
\pgfsetbuttcap%
\pgfsetroundjoin%
\definecolor{currentfill}{rgb}{0.333333,0.333333,0.333333}%
\pgfsetfillcolor{currentfill}%
\pgfsetlinewidth{0.501875pt}%
\definecolor{currentstroke}{rgb}{0.333333,0.333333,0.333333}%
\pgfsetstrokecolor{currentstroke}%
\pgfsetdash{}{0pt}%
\pgfpathmoveto{\pgfqpoint{0.967101in}{1.008608in}}%
\pgfpathcurveto{\pgfqpoint{0.972626in}{1.008608in}}{\pgfqpoint{0.977926in}{1.010804in}}{\pgfqpoint{0.981832in}{1.014710in}}%
\pgfpathcurveto{\pgfqpoint{0.985739in}{1.018617in}}{\pgfqpoint{0.987934in}{1.023917in}}{\pgfqpoint{0.987934in}{1.029442in}}%
\pgfpathcurveto{\pgfqpoint{0.987934in}{1.034967in}}{\pgfqpoint{0.985739in}{1.040266in}}{\pgfqpoint{0.981832in}{1.044173in}}%
\pgfpathcurveto{\pgfqpoint{0.977926in}{1.048080in}}{\pgfqpoint{0.972626in}{1.050275in}}{\pgfqpoint{0.967101in}{1.050275in}}%
\pgfpathcurveto{\pgfqpoint{0.961576in}{1.050275in}}{\pgfqpoint{0.956276in}{1.048080in}}{\pgfqpoint{0.952370in}{1.044173in}}%
\pgfpathcurveto{\pgfqpoint{0.948463in}{1.040266in}}{\pgfqpoint{0.946268in}{1.034967in}}{\pgfqpoint{0.946268in}{1.029442in}}%
\pgfpathcurveto{\pgfqpoint{0.946268in}{1.023917in}}{\pgfqpoint{0.948463in}{1.018617in}}{\pgfqpoint{0.952370in}{1.014710in}}%
\pgfpathcurveto{\pgfqpoint{0.956276in}{1.010804in}}{\pgfqpoint{0.961576in}{1.008608in}}{\pgfqpoint{0.967101in}{1.008608in}}%
\pgfpathclose%
\pgfusepath{stroke,fill}%
\end{pgfscope}%
\begin{pgfscope}%
\pgfpathrectangle{\pgfqpoint{0.403630in}{0.251348in}}{\pgfqpoint{1.395000in}{1.359000in}}%
\pgfusepath{clip}%
\pgfsetbuttcap%
\pgfsetroundjoin%
\definecolor{currentfill}{rgb}{0.333333,0.333333,0.333333}%
\pgfsetfillcolor{currentfill}%
\pgfsetlinewidth{0.501875pt}%
\definecolor{currentstroke}{rgb}{0.333333,0.333333,0.333333}%
\pgfsetstrokecolor{currentstroke}%
\pgfsetdash{}{0pt}%
\pgfpathmoveto{\pgfqpoint{1.103866in}{0.688844in}}%
\pgfpathcurveto{\pgfqpoint{1.109391in}{0.688844in}}{\pgfqpoint{1.114690in}{0.691039in}}{\pgfqpoint{1.118597in}{0.694946in}}%
\pgfpathcurveto{\pgfqpoint{1.122504in}{0.698852in}}{\pgfqpoint{1.124699in}{0.704152in}}{\pgfqpoint{1.124699in}{0.709677in}}%
\pgfpathcurveto{\pgfqpoint{1.124699in}{0.715202in}}{\pgfqpoint{1.122504in}{0.720502in}}{\pgfqpoint{1.118597in}{0.724408in}}%
\pgfpathcurveto{\pgfqpoint{1.114690in}{0.728315in}}{\pgfqpoint{1.109391in}{0.730510in}}{\pgfqpoint{1.103866in}{0.730510in}}%
\pgfpathcurveto{\pgfqpoint{1.098341in}{0.730510in}}{\pgfqpoint{1.093041in}{0.728315in}}{\pgfqpoint{1.089134in}{0.724408in}}%
\pgfpathcurveto{\pgfqpoint{1.085228in}{0.720502in}}{\pgfqpoint{1.083032in}{0.715202in}}{\pgfqpoint{1.083032in}{0.709677in}}%
\pgfpathcurveto{\pgfqpoint{1.083032in}{0.704152in}}{\pgfqpoint{1.085228in}{0.698852in}}{\pgfqpoint{1.089134in}{0.694946in}}%
\pgfpathcurveto{\pgfqpoint{1.093041in}{0.691039in}}{\pgfqpoint{1.098341in}{0.688844in}}{\pgfqpoint{1.103866in}{0.688844in}}%
\pgfpathclose%
\pgfusepath{stroke,fill}%
\end{pgfscope}%
\begin{pgfscope}%
\pgfpathrectangle{\pgfqpoint{0.403630in}{0.251348in}}{\pgfqpoint{1.395000in}{1.359000in}}%
\pgfusepath{clip}%
\pgfsetbuttcap%
\pgfsetroundjoin%
\definecolor{currentfill}{rgb}{0.333333,0.333333,0.333333}%
\pgfsetfillcolor{currentfill}%
\pgfsetlinewidth{0.501875pt}%
\definecolor{currentstroke}{rgb}{0.333333,0.333333,0.333333}%
\pgfsetstrokecolor{currentstroke}%
\pgfsetdash{}{0pt}%
\pgfpathmoveto{\pgfqpoint{1.021807in}{0.726150in}}%
\pgfpathcurveto{\pgfqpoint{1.027332in}{0.726150in}}{\pgfqpoint{1.032631in}{0.728345in}}{\pgfqpoint{1.036538in}{0.732252in}}%
\pgfpathcurveto{\pgfqpoint{1.040445in}{0.736158in}}{\pgfqpoint{1.042640in}{0.741458in}}{\pgfqpoint{1.042640in}{0.746983in}}%
\pgfpathcurveto{\pgfqpoint{1.042640in}{0.752508in}}{\pgfqpoint{1.040445in}{0.757808in}}{\pgfqpoint{1.036538in}{0.761714in}}%
\pgfpathcurveto{\pgfqpoint{1.032631in}{0.765621in}}{\pgfqpoint{1.027332in}{0.767816in}}{\pgfqpoint{1.021807in}{0.767816in}}%
\pgfpathcurveto{\pgfqpoint{1.016282in}{0.767816in}}{\pgfqpoint{1.010982in}{0.765621in}}{\pgfqpoint{1.007075in}{0.761714in}}%
\pgfpathcurveto{\pgfqpoint{1.003169in}{0.757808in}}{\pgfqpoint{1.000974in}{0.752508in}}{\pgfqpoint{1.000974in}{0.746983in}}%
\pgfpathcurveto{\pgfqpoint{1.000974in}{0.741458in}}{\pgfqpoint{1.003169in}{0.736158in}}{\pgfqpoint{1.007075in}{0.732252in}}%
\pgfpathcurveto{\pgfqpoint{1.010982in}{0.728345in}}{\pgfqpoint{1.016282in}{0.726150in}}{\pgfqpoint{1.021807in}{0.726150in}}%
\pgfpathclose%
\pgfusepath{stroke,fill}%
\end{pgfscope}%
\begin{pgfscope}%
\pgfpathrectangle{\pgfqpoint{0.403630in}{0.251348in}}{\pgfqpoint{1.395000in}{1.359000in}}%
\pgfusepath{clip}%
\pgfsetbuttcap%
\pgfsetroundjoin%
\definecolor{currentfill}{rgb}{0.333333,0.333333,0.333333}%
\pgfsetfillcolor{currentfill}%
\pgfsetlinewidth{0.501875pt}%
\definecolor{currentstroke}{rgb}{0.333333,0.333333,0.333333}%
\pgfsetstrokecolor{currentstroke}%
\pgfsetdash{}{0pt}%
\pgfpathmoveto{\pgfqpoint{1.273454in}{0.944655in}}%
\pgfpathcurveto{\pgfqpoint{1.278979in}{0.944655in}}{\pgfqpoint{1.284279in}{0.946851in}}{\pgfqpoint{1.288185in}{0.950757in}}%
\pgfpathcurveto{\pgfqpoint{1.292092in}{0.954664in}}{\pgfqpoint{1.294287in}{0.959964in}}{\pgfqpoint{1.294287in}{0.965489in}}%
\pgfpathcurveto{\pgfqpoint{1.294287in}{0.971014in}}{\pgfqpoint{1.292092in}{0.976313in}}{\pgfqpoint{1.288185in}{0.980220in}}%
\pgfpathcurveto{\pgfqpoint{1.284279in}{0.984127in}}{\pgfqpoint{1.278979in}{0.986322in}}{\pgfqpoint{1.273454in}{0.986322in}}%
\pgfpathcurveto{\pgfqpoint{1.267929in}{0.986322in}}{\pgfqpoint{1.262629in}{0.984127in}}{\pgfqpoint{1.258723in}{0.980220in}}%
\pgfpathcurveto{\pgfqpoint{1.254816in}{0.976313in}}{\pgfqpoint{1.252621in}{0.971014in}}{\pgfqpoint{1.252621in}{0.965489in}}%
\pgfpathcurveto{\pgfqpoint{1.252621in}{0.959964in}}{\pgfqpoint{1.254816in}{0.954664in}}{\pgfqpoint{1.258723in}{0.950757in}}%
\pgfpathcurveto{\pgfqpoint{1.262629in}{0.946851in}}{\pgfqpoint{1.267929in}{0.944655in}}{\pgfqpoint{1.273454in}{0.944655in}}%
\pgfpathclose%
\pgfusepath{stroke,fill}%
\end{pgfscope}%
\begin{pgfscope}%
\pgfpathrectangle{\pgfqpoint{0.403630in}{0.251348in}}{\pgfqpoint{1.395000in}{1.359000in}}%
\pgfusepath{clip}%
\pgfsetbuttcap%
\pgfsetroundjoin%
\definecolor{currentfill}{rgb}{0.333333,0.333333,0.333333}%
\pgfsetfillcolor{currentfill}%
\pgfsetlinewidth{0.501875pt}%
\definecolor{currentstroke}{rgb}{0.333333,0.333333,0.333333}%
\pgfsetstrokecolor{currentstroke}%
\pgfsetdash{}{0pt}%
\pgfpathmoveto{\pgfqpoint{1.322689in}{0.971303in}}%
\pgfpathcurveto{\pgfqpoint{1.328214in}{0.971303in}}{\pgfqpoint{1.333514in}{0.973498in}}{\pgfqpoint{1.337421in}{0.977404in}}%
\pgfpathcurveto{\pgfqpoint{1.341327in}{0.981311in}}{\pgfqpoint{1.343523in}{0.986611in}}{\pgfqpoint{1.343523in}{0.992136in}}%
\pgfpathcurveto{\pgfqpoint{1.343523in}{0.997661in}}{\pgfqpoint{1.341327in}{1.002960in}}{\pgfqpoint{1.337421in}{1.006867in}}%
\pgfpathcurveto{\pgfqpoint{1.333514in}{1.010774in}}{\pgfqpoint{1.328214in}{1.012969in}}{\pgfqpoint{1.322689in}{1.012969in}}%
\pgfpathcurveto{\pgfqpoint{1.317164in}{1.012969in}}{\pgfqpoint{1.311865in}{1.010774in}}{\pgfqpoint{1.307958in}{1.006867in}}%
\pgfpathcurveto{\pgfqpoint{1.304051in}{1.002960in}}{\pgfqpoint{1.301856in}{0.997661in}}{\pgfqpoint{1.301856in}{0.992136in}}%
\pgfpathcurveto{\pgfqpoint{1.301856in}{0.986611in}}{\pgfqpoint{1.304051in}{0.981311in}}{\pgfqpoint{1.307958in}{0.977404in}}%
\pgfpathcurveto{\pgfqpoint{1.311865in}{0.973498in}}{\pgfqpoint{1.317164in}{0.971303in}}{\pgfqpoint{1.322689in}{0.971303in}}%
\pgfpathclose%
\pgfusepath{stroke,fill}%
\end{pgfscope}%
\begin{pgfscope}%
\pgfpathrectangle{\pgfqpoint{0.403630in}{0.251348in}}{\pgfqpoint{1.395000in}{1.359000in}}%
\pgfusepath{clip}%
\pgfsetbuttcap%
\pgfsetroundjoin%
\definecolor{currentfill}{rgb}{0.333333,0.333333,0.333333}%
\pgfsetfillcolor{currentfill}%
\pgfsetlinewidth{0.501875pt}%
\definecolor{currentstroke}{rgb}{0.333333,0.333333,0.333333}%
\pgfsetstrokecolor{currentstroke}%
\pgfsetdash{}{0pt}%
\pgfpathmoveto{\pgfqpoint{1.071042in}{1.077891in}}%
\pgfpathcurveto{\pgfqpoint{1.076567in}{1.077891in}}{\pgfqpoint{1.081867in}{1.080086in}}{\pgfqpoint{1.085774in}{1.083993in}}%
\pgfpathcurveto{\pgfqpoint{1.089680in}{1.087900in}}{\pgfqpoint{1.091876in}{1.093199in}}{\pgfqpoint{1.091876in}{1.098724in}}%
\pgfpathcurveto{\pgfqpoint{1.091876in}{1.104249in}}{\pgfqpoint{1.089680in}{1.109549in}}{\pgfqpoint{1.085774in}{1.113456in}}%
\pgfpathcurveto{\pgfqpoint{1.081867in}{1.117362in}}{\pgfqpoint{1.076567in}{1.119557in}}{\pgfqpoint{1.071042in}{1.119557in}}%
\pgfpathcurveto{\pgfqpoint{1.065517in}{1.119557in}}{\pgfqpoint{1.060218in}{1.117362in}}{\pgfqpoint{1.056311in}{1.113456in}}%
\pgfpathcurveto{\pgfqpoint{1.052404in}{1.109549in}}{\pgfqpoint{1.050209in}{1.104249in}}{\pgfqpoint{1.050209in}{1.098724in}}%
\pgfpathcurveto{\pgfqpoint{1.050209in}{1.093199in}}{\pgfqpoint{1.052404in}{1.087900in}}{\pgfqpoint{1.056311in}{1.083993in}}%
\pgfpathcurveto{\pgfqpoint{1.060218in}{1.080086in}}{\pgfqpoint{1.065517in}{1.077891in}}{\pgfqpoint{1.071042in}{1.077891in}}%
\pgfpathclose%
\pgfusepath{stroke,fill}%
\end{pgfscope}%
\begin{pgfscope}%
\pgfpathrectangle{\pgfqpoint{0.403630in}{0.251348in}}{\pgfqpoint{1.395000in}{1.359000in}}%
\pgfusepath{clip}%
\pgfsetbuttcap%
\pgfsetroundjoin%
\definecolor{currentfill}{rgb}{0.333333,0.333333,0.333333}%
\pgfsetfillcolor{currentfill}%
\pgfsetlinewidth{0.501875pt}%
\definecolor{currentstroke}{rgb}{0.333333,0.333333,0.333333}%
\pgfsetstrokecolor{currentstroke}%
\pgfsetdash{}{0pt}%
\pgfpathmoveto{\pgfqpoint{1.142160in}{1.093879in}}%
\pgfpathcurveto{\pgfqpoint{1.147685in}{1.093879in}}{\pgfqpoint{1.152984in}{1.096074in}}{\pgfqpoint{1.156891in}{1.099981in}}%
\pgfpathcurveto{\pgfqpoint{1.160798in}{1.103888in}}{\pgfqpoint{1.162993in}{1.109187in}}{\pgfqpoint{1.162993in}{1.114712in}}%
\pgfpathcurveto{\pgfqpoint{1.162993in}{1.120237in}}{\pgfqpoint{1.160798in}{1.125537in}}{\pgfqpoint{1.156891in}{1.129444in}}%
\pgfpathcurveto{\pgfqpoint{1.152984in}{1.133351in}}{\pgfqpoint{1.147685in}{1.135546in}}{\pgfqpoint{1.142160in}{1.135546in}}%
\pgfpathcurveto{\pgfqpoint{1.136635in}{1.135546in}}{\pgfqpoint{1.131335in}{1.133351in}}{\pgfqpoint{1.127428in}{1.129444in}}%
\pgfpathcurveto{\pgfqpoint{1.123522in}{1.125537in}}{\pgfqpoint{1.121326in}{1.120237in}}{\pgfqpoint{1.121326in}{1.114712in}}%
\pgfpathcurveto{\pgfqpoint{1.121326in}{1.109187in}}{\pgfqpoint{1.123522in}{1.103888in}}{\pgfqpoint{1.127428in}{1.099981in}}%
\pgfpathcurveto{\pgfqpoint{1.131335in}{1.096074in}}{\pgfqpoint{1.136635in}{1.093879in}}{\pgfqpoint{1.142160in}{1.093879in}}%
\pgfpathclose%
\pgfusepath{stroke,fill}%
\end{pgfscope}%
\begin{pgfscope}%
\pgfpathrectangle{\pgfqpoint{0.403630in}{0.251348in}}{\pgfqpoint{1.395000in}{1.359000in}}%
\pgfusepath{clip}%
\pgfsetbuttcap%
\pgfsetroundjoin%
\definecolor{currentfill}{rgb}{0.333333,0.333333,0.333333}%
\pgfsetfillcolor{currentfill}%
\pgfsetlinewidth{0.501875pt}%
\definecolor{currentstroke}{rgb}{0.333333,0.333333,0.333333}%
\pgfsetstrokecolor{currentstroke}%
\pgfsetdash{}{0pt}%
\pgfpathmoveto{\pgfqpoint{1.240630in}{0.779444in}}%
\pgfpathcurveto{\pgfqpoint{1.246155in}{0.779444in}}{\pgfqpoint{1.251455in}{0.781639in}}{\pgfqpoint{1.255362in}{0.785546in}}%
\pgfpathcurveto{\pgfqpoint{1.259269in}{0.789452in}}{\pgfqpoint{1.261464in}{0.794752in}}{\pgfqpoint{1.261464in}{0.800277in}}%
\pgfpathcurveto{\pgfqpoint{1.261464in}{0.805802in}}{\pgfqpoint{1.259269in}{0.811102in}}{\pgfqpoint{1.255362in}{0.815008in}}%
\pgfpathcurveto{\pgfqpoint{1.251455in}{0.818915in}}{\pgfqpoint{1.246155in}{0.821110in}}{\pgfqpoint{1.240630in}{0.821110in}}%
\pgfpathcurveto{\pgfqpoint{1.235105in}{0.821110in}}{\pgfqpoint{1.229806in}{0.818915in}}{\pgfqpoint{1.225899in}{0.815008in}}%
\pgfpathcurveto{\pgfqpoint{1.221992in}{0.811102in}}{\pgfqpoint{1.219797in}{0.805802in}}{\pgfqpoint{1.219797in}{0.800277in}}%
\pgfpathcurveto{\pgfqpoint{1.219797in}{0.794752in}}{\pgfqpoint{1.221992in}{0.789452in}}{\pgfqpoint{1.225899in}{0.785546in}}%
\pgfpathcurveto{\pgfqpoint{1.229806in}{0.781639in}}{\pgfqpoint{1.235105in}{0.779444in}}{\pgfqpoint{1.240630in}{0.779444in}}%
\pgfpathclose%
\pgfusepath{stroke,fill}%
\end{pgfscope}%
\begin{pgfscope}%
\pgfpathrectangle{\pgfqpoint{0.403630in}{0.251348in}}{\pgfqpoint{1.395000in}{1.359000in}}%
\pgfusepath{clip}%
\pgfsetbuttcap%
\pgfsetroundjoin%
\definecolor{currentfill}{rgb}{0.333333,0.333333,0.333333}%
\pgfsetfillcolor{currentfill}%
\pgfsetlinewidth{0.501875pt}%
\definecolor{currentstroke}{rgb}{0.333333,0.333333,0.333333}%
\pgfsetstrokecolor{currentstroke}%
\pgfsetdash{}{0pt}%
\pgfpathmoveto{\pgfqpoint{0.923336in}{0.886032in}}%
\pgfpathcurveto{\pgfqpoint{0.928861in}{0.886032in}}{\pgfqpoint{0.934161in}{0.888227in}}{\pgfqpoint{0.938068in}{0.892134in}}%
\pgfpathcurveto{\pgfqpoint{0.941975in}{0.896041in}}{\pgfqpoint{0.944170in}{0.901340in}}{\pgfqpoint{0.944170in}{0.906865in}}%
\pgfpathcurveto{\pgfqpoint{0.944170in}{0.912390in}}{\pgfqpoint{0.941975in}{0.917690in}}{\pgfqpoint{0.938068in}{0.921597in}}%
\pgfpathcurveto{\pgfqpoint{0.934161in}{0.925503in}}{\pgfqpoint{0.928861in}{0.927699in}}{\pgfqpoint{0.923336in}{0.927699in}}%
\pgfpathcurveto{\pgfqpoint{0.917811in}{0.927699in}}{\pgfqpoint{0.912512in}{0.925503in}}{\pgfqpoint{0.908605in}{0.921597in}}%
\pgfpathcurveto{\pgfqpoint{0.904698in}{0.917690in}}{\pgfqpoint{0.902503in}{0.912390in}}{\pgfqpoint{0.902503in}{0.906865in}}%
\pgfpathcurveto{\pgfqpoint{0.902503in}{0.901340in}}{\pgfqpoint{0.904698in}{0.896041in}}{\pgfqpoint{0.908605in}{0.892134in}}%
\pgfpathcurveto{\pgfqpoint{0.912512in}{0.888227in}}{\pgfqpoint{0.917811in}{0.886032in}}{\pgfqpoint{0.923336in}{0.886032in}}%
\pgfpathclose%
\pgfusepath{stroke,fill}%
\end{pgfscope}%
\begin{pgfscope}%
\pgfpathrectangle{\pgfqpoint{0.403630in}{0.251348in}}{\pgfqpoint{1.395000in}{1.359000in}}%
\pgfusepath{clip}%
\pgfsetbuttcap%
\pgfsetroundjoin%
\definecolor{currentfill}{rgb}{0.333333,0.333333,0.333333}%
\pgfsetfillcolor{currentfill}%
\pgfsetlinewidth{0.501875pt}%
\definecolor{currentstroke}{rgb}{0.333333,0.333333,0.333333}%
\pgfsetstrokecolor{currentstroke}%
\pgfsetdash{}{0pt}%
\pgfpathmoveto{\pgfqpoint{0.928807in}{1.045914in}}%
\pgfpathcurveto{\pgfqpoint{0.934332in}{1.045914in}}{\pgfqpoint{0.939631in}{1.048109in}}{\pgfqpoint{0.943538in}{1.052016in}}%
\pgfpathcurveto{\pgfqpoint{0.947445in}{1.055923in}}{\pgfqpoint{0.949640in}{1.061223in}}{\pgfqpoint{0.949640in}{1.066748in}}%
\pgfpathcurveto{\pgfqpoint{0.949640in}{1.072273in}}{\pgfqpoint{0.947445in}{1.077572in}}{\pgfqpoint{0.943538in}{1.081479in}}%
\pgfpathcurveto{\pgfqpoint{0.939631in}{1.085386in}}{\pgfqpoint{0.934332in}{1.087581in}}{\pgfqpoint{0.928807in}{1.087581in}}%
\pgfpathcurveto{\pgfqpoint{0.923282in}{1.087581in}}{\pgfqpoint{0.917982in}{1.085386in}}{\pgfqpoint{0.914075in}{1.081479in}}%
\pgfpathcurveto{\pgfqpoint{0.910169in}{1.077572in}}{\pgfqpoint{0.907974in}{1.072273in}}{\pgfqpoint{0.907974in}{1.066748in}}%
\pgfpathcurveto{\pgfqpoint{0.907974in}{1.061223in}}{\pgfqpoint{0.910169in}{1.055923in}}{\pgfqpoint{0.914075in}{1.052016in}}%
\pgfpathcurveto{\pgfqpoint{0.917982in}{1.048109in}}{\pgfqpoint{0.923282in}{1.045914in}}{\pgfqpoint{0.928807in}{1.045914in}}%
\pgfpathclose%
\pgfusepath{stroke,fill}%
\end{pgfscope}%
\begin{pgfscope}%
\pgfpathrectangle{\pgfqpoint{0.403630in}{0.251348in}}{\pgfqpoint{1.395000in}{1.359000in}}%
\pgfusepath{clip}%
\pgfsetbuttcap%
\pgfsetroundjoin%
\definecolor{currentfill}{rgb}{0.333333,0.333333,0.333333}%
\pgfsetfillcolor{currentfill}%
\pgfsetlinewidth{0.501875pt}%
\definecolor{currentstroke}{rgb}{0.333333,0.333333,0.333333}%
\pgfsetstrokecolor{currentstroke}%
\pgfsetdash{}{0pt}%
\pgfpathmoveto{\pgfqpoint{0.988983in}{0.720820in}}%
\pgfpathcurveto{\pgfqpoint{0.994508in}{0.720820in}}{\pgfqpoint{0.999808in}{0.723015in}}{\pgfqpoint{1.003715in}{0.726922in}}%
\pgfpathcurveto{\pgfqpoint{1.007622in}{0.730829in}}{\pgfqpoint{1.009817in}{0.736128in}}{\pgfqpoint{1.009817in}{0.741654in}}%
\pgfpathcurveto{\pgfqpoint{1.009817in}{0.747179in}}{\pgfqpoint{1.007622in}{0.752478in}}{\pgfqpoint{1.003715in}{0.756385in}}%
\pgfpathcurveto{\pgfqpoint{0.999808in}{0.760292in}}{\pgfqpoint{0.994508in}{0.762487in}}{\pgfqpoint{0.988983in}{0.762487in}}%
\pgfpathcurveto{\pgfqpoint{0.983458in}{0.762487in}}{\pgfqpoint{0.978159in}{0.760292in}}{\pgfqpoint{0.974252in}{0.756385in}}%
\pgfpathcurveto{\pgfqpoint{0.970345in}{0.752478in}}{\pgfqpoint{0.968150in}{0.747179in}}{\pgfqpoint{0.968150in}{0.741654in}}%
\pgfpathcurveto{\pgfqpoint{0.968150in}{0.736128in}}{\pgfqpoint{0.970345in}{0.730829in}}{\pgfqpoint{0.974252in}{0.726922in}}%
\pgfpathcurveto{\pgfqpoint{0.978159in}{0.723015in}}{\pgfqpoint{0.983458in}{0.720820in}}{\pgfqpoint{0.988983in}{0.720820in}}%
\pgfpathclose%
\pgfusepath{stroke,fill}%
\end{pgfscope}%
\begin{pgfscope}%
\pgfpathrectangle{\pgfqpoint{0.403630in}{0.251348in}}{\pgfqpoint{1.395000in}{1.359000in}}%
\pgfusepath{clip}%
\pgfsetbuttcap%
\pgfsetroundjoin%
\definecolor{currentfill}{rgb}{0.333333,0.333333,0.333333}%
\pgfsetfillcolor{currentfill}%
\pgfsetlinewidth{0.501875pt}%
\definecolor{currentstroke}{rgb}{0.333333,0.333333,0.333333}%
\pgfsetstrokecolor{currentstroke}%
\pgfsetdash{}{0pt}%
\pgfpathmoveto{\pgfqpoint{1.032748in}{1.093879in}}%
\pgfpathcurveto{\pgfqpoint{1.038273in}{1.093879in}}{\pgfqpoint{1.043573in}{1.096074in}}{\pgfqpoint{1.047479in}{1.099981in}}%
\pgfpathcurveto{\pgfqpoint{1.051386in}{1.103888in}}{\pgfqpoint{1.053581in}{1.109187in}}{\pgfqpoint{1.053581in}{1.114712in}}%
\pgfpathcurveto{\pgfqpoint{1.053581in}{1.120237in}}{\pgfqpoint{1.051386in}{1.125537in}}{\pgfqpoint{1.047479in}{1.129444in}}%
\pgfpathcurveto{\pgfqpoint{1.043573in}{1.133351in}}{\pgfqpoint{1.038273in}{1.135546in}}{\pgfqpoint{1.032748in}{1.135546in}}%
\pgfpathcurveto{\pgfqpoint{1.027223in}{1.135546in}}{\pgfqpoint{1.021923in}{1.133351in}}{\pgfqpoint{1.018017in}{1.129444in}}%
\pgfpathcurveto{\pgfqpoint{1.014110in}{1.125537in}}{\pgfqpoint{1.011915in}{1.120237in}}{\pgfqpoint{1.011915in}{1.114712in}}%
\pgfpathcurveto{\pgfqpoint{1.011915in}{1.109187in}}{\pgfqpoint{1.014110in}{1.103888in}}{\pgfqpoint{1.018017in}{1.099981in}}%
\pgfpathcurveto{\pgfqpoint{1.021923in}{1.096074in}}{\pgfqpoint{1.027223in}{1.093879in}}{\pgfqpoint{1.032748in}{1.093879in}}%
\pgfpathclose%
\pgfusepath{stroke,fill}%
\end{pgfscope}%
\begin{pgfscope}%
\pgfpathrectangle{\pgfqpoint{0.403630in}{0.251348in}}{\pgfqpoint{1.395000in}{1.359000in}}%
\pgfusepath{clip}%
\pgfsetbuttcap%
\pgfsetroundjoin%
\definecolor{currentfill}{rgb}{0.333333,0.333333,0.333333}%
\pgfsetfillcolor{currentfill}%
\pgfsetlinewidth{0.501875pt}%
\definecolor{currentstroke}{rgb}{0.333333,0.333333,0.333333}%
\pgfsetstrokecolor{currentstroke}%
\pgfsetdash{}{0pt}%
\pgfpathmoveto{\pgfqpoint{0.972572in}{0.811420in}}%
\pgfpathcurveto{\pgfqpoint{0.978097in}{0.811420in}}{\pgfqpoint{0.983396in}{0.813615in}}{\pgfqpoint{0.987303in}{0.817522in}}%
\pgfpathcurveto{\pgfqpoint{0.991210in}{0.821429in}}{\pgfqpoint{0.993405in}{0.826728in}}{\pgfqpoint{0.993405in}{0.832254in}}%
\pgfpathcurveto{\pgfqpoint{0.993405in}{0.837779in}}{\pgfqpoint{0.991210in}{0.843078in}}{\pgfqpoint{0.987303in}{0.846985in}}%
\pgfpathcurveto{\pgfqpoint{0.983396in}{0.850892in}}{\pgfqpoint{0.978097in}{0.853087in}}{\pgfqpoint{0.972572in}{0.853087in}}%
\pgfpathcurveto{\pgfqpoint{0.967047in}{0.853087in}}{\pgfqpoint{0.961747in}{0.850892in}}{\pgfqpoint{0.957840in}{0.846985in}}%
\pgfpathcurveto{\pgfqpoint{0.953933in}{0.843078in}}{\pgfqpoint{0.951738in}{0.837779in}}{\pgfqpoint{0.951738in}{0.832254in}}%
\pgfpathcurveto{\pgfqpoint{0.951738in}{0.826728in}}{\pgfqpoint{0.953933in}{0.821429in}}{\pgfqpoint{0.957840in}{0.817522in}}%
\pgfpathcurveto{\pgfqpoint{0.961747in}{0.813615in}}{\pgfqpoint{0.967047in}{0.811420in}}{\pgfqpoint{0.972572in}{0.811420in}}%
\pgfpathclose%
\pgfusepath{stroke,fill}%
\end{pgfscope}%
\begin{pgfscope}%
\pgfpathrectangle{\pgfqpoint{0.403630in}{0.251348in}}{\pgfqpoint{1.395000in}{1.359000in}}%
\pgfusepath{clip}%
\pgfsetbuttcap%
\pgfsetroundjoin%
\definecolor{currentfill}{rgb}{0.333333,0.333333,0.333333}%
\pgfsetfillcolor{currentfill}%
\pgfsetlinewidth{0.501875pt}%
\definecolor{currentstroke}{rgb}{0.333333,0.333333,0.333333}%
\pgfsetstrokecolor{currentstroke}%
\pgfsetdash{}{0pt}%
\pgfpathmoveto{\pgfqpoint{1.262513in}{0.774114in}}%
\pgfpathcurveto{\pgfqpoint{1.268038in}{0.774114in}}{\pgfqpoint{1.273337in}{0.776309in}}{\pgfqpoint{1.277244in}{0.780216in}}%
\pgfpathcurveto{\pgfqpoint{1.281151in}{0.784123in}}{\pgfqpoint{1.283346in}{0.789423in}}{\pgfqpoint{1.283346in}{0.794948in}}%
\pgfpathcurveto{\pgfqpoint{1.283346in}{0.800473in}}{\pgfqpoint{1.281151in}{0.805772in}}{\pgfqpoint{1.277244in}{0.809679in}}%
\pgfpathcurveto{\pgfqpoint{1.273337in}{0.813586in}}{\pgfqpoint{1.268038in}{0.815781in}}{\pgfqpoint{1.262513in}{0.815781in}}%
\pgfpathcurveto{\pgfqpoint{1.256988in}{0.815781in}}{\pgfqpoint{1.251688in}{0.813586in}}{\pgfqpoint{1.247781in}{0.809679in}}%
\pgfpathcurveto{\pgfqpoint{1.243875in}{0.805772in}}{\pgfqpoint{1.241679in}{0.800473in}}{\pgfqpoint{1.241679in}{0.794948in}}%
\pgfpathcurveto{\pgfqpoint{1.241679in}{0.789423in}}{\pgfqpoint{1.243875in}{0.784123in}}{\pgfqpoint{1.247781in}{0.780216in}}%
\pgfpathcurveto{\pgfqpoint{1.251688in}{0.776309in}}{\pgfqpoint{1.256988in}{0.774114in}}{\pgfqpoint{1.262513in}{0.774114in}}%
\pgfpathclose%
\pgfusepath{stroke,fill}%
\end{pgfscope}%
\begin{pgfscope}%
\pgfpathrectangle{\pgfqpoint{0.403630in}{0.251348in}}{\pgfqpoint{1.395000in}{1.359000in}}%
\pgfusepath{clip}%
\pgfsetbuttcap%
\pgfsetroundjoin%
\definecolor{currentfill}{rgb}{0.686275,0.352941,0.313725}%
\pgfsetfillcolor{currentfill}%
\pgfsetlinewidth{0.501875pt}%
\definecolor{currentstroke}{rgb}{0.686275,0.352941,0.313725}%
\pgfsetstrokecolor{currentstroke}%
\pgfsetdash{}{0pt}%
\pgfpathmoveto{\pgfqpoint{0.753748in}{0.630220in}}%
\pgfpathcurveto{\pgfqpoint{0.759273in}{0.630220in}}{\pgfqpoint{0.764573in}{0.632415in}}{\pgfqpoint{0.768479in}{0.636322in}}%
\pgfpathcurveto{\pgfqpoint{0.772386in}{0.640229in}}{\pgfqpoint{0.774581in}{0.645528in}}{\pgfqpoint{0.774581in}{0.651054in}}%
\pgfpathcurveto{\pgfqpoint{0.774581in}{0.656579in}}{\pgfqpoint{0.772386in}{0.661878in}}{\pgfqpoint{0.768479in}{0.665785in}}%
\pgfpathcurveto{\pgfqpoint{0.764573in}{0.669692in}}{\pgfqpoint{0.759273in}{0.671887in}}{\pgfqpoint{0.753748in}{0.671887in}}%
\pgfpathcurveto{\pgfqpoint{0.748223in}{0.671887in}}{\pgfqpoint{0.742923in}{0.669692in}}{\pgfqpoint{0.739017in}{0.665785in}}%
\pgfpathcurveto{\pgfqpoint{0.735110in}{0.661878in}}{\pgfqpoint{0.732915in}{0.656579in}}{\pgfqpoint{0.732915in}{0.651054in}}%
\pgfpathcurveto{\pgfqpoint{0.732915in}{0.645528in}}{\pgfqpoint{0.735110in}{0.640229in}}{\pgfqpoint{0.739017in}{0.636322in}}%
\pgfpathcurveto{\pgfqpoint{0.742923in}{0.632415in}}{\pgfqpoint{0.748223in}{0.630220in}}{\pgfqpoint{0.753748in}{0.630220in}}%
\pgfpathclose%
\pgfusepath{stroke,fill}%
\end{pgfscope}%
\begin{pgfscope}%
\pgfpathrectangle{\pgfqpoint{0.403630in}{0.251348in}}{\pgfqpoint{1.395000in}{1.359000in}}%
\pgfusepath{clip}%
\pgfsetbuttcap%
\pgfsetroundjoin%
\definecolor{currentfill}{rgb}{0.686275,0.352941,0.313725}%
\pgfsetfillcolor{currentfill}%
\pgfsetlinewidth{0.501875pt}%
\definecolor{currentstroke}{rgb}{0.686275,0.352941,0.313725}%
\pgfsetstrokecolor{currentstroke}%
\pgfsetdash{}{0pt}%
\pgfpathmoveto{\pgfqpoint{0.677160in}{0.731479in}}%
\pgfpathcurveto{\pgfqpoint{0.682685in}{0.731479in}}{\pgfqpoint{0.687984in}{0.733674in}}{\pgfqpoint{0.691891in}{0.737581in}}%
\pgfpathcurveto{\pgfqpoint{0.695798in}{0.741488in}}{\pgfqpoint{0.697993in}{0.746787in}}{\pgfqpoint{0.697993in}{0.752312in}}%
\pgfpathcurveto{\pgfqpoint{0.697993in}{0.757837in}}{\pgfqpoint{0.695798in}{0.763137in}}{\pgfqpoint{0.691891in}{0.767044in}}%
\pgfpathcurveto{\pgfqpoint{0.687984in}{0.770951in}}{\pgfqpoint{0.682685in}{0.773146in}}{\pgfqpoint{0.677160in}{0.773146in}}%
\pgfpathcurveto{\pgfqpoint{0.671635in}{0.773146in}}{\pgfqpoint{0.666335in}{0.770951in}}{\pgfqpoint{0.662428in}{0.767044in}}%
\pgfpathcurveto{\pgfqpoint{0.658522in}{0.763137in}}{\pgfqpoint{0.656326in}{0.757837in}}{\pgfqpoint{0.656326in}{0.752312in}}%
\pgfpathcurveto{\pgfqpoint{0.656326in}{0.746787in}}{\pgfqpoint{0.658522in}{0.741488in}}{\pgfqpoint{0.662428in}{0.737581in}}%
\pgfpathcurveto{\pgfqpoint{0.666335in}{0.733674in}}{\pgfqpoint{0.671635in}{0.731479in}}{\pgfqpoint{0.677160in}{0.731479in}}%
\pgfpathclose%
\pgfusepath{stroke,fill}%
\end{pgfscope}%
\begin{pgfscope}%
\pgfpathrectangle{\pgfqpoint{0.403630in}{0.251348in}}{\pgfqpoint{1.395000in}{1.359000in}}%
\pgfusepath{clip}%
\pgfsetbuttcap%
\pgfsetroundjoin%
\definecolor{currentfill}{rgb}{0.686275,0.352941,0.313725}%
\pgfsetfillcolor{currentfill}%
\pgfsetlinewidth{0.501875pt}%
\definecolor{currentstroke}{rgb}{0.686275,0.352941,0.313725}%
\pgfsetstrokecolor{currentstroke}%
\pgfsetdash{}{0pt}%
\pgfpathmoveto{\pgfqpoint{0.797513in}{0.566267in}}%
\pgfpathcurveto{\pgfqpoint{0.803038in}{0.566267in}}{\pgfqpoint{0.808337in}{0.568462in}}{\pgfqpoint{0.812244in}{0.572369in}}%
\pgfpathcurveto{\pgfqpoint{0.816151in}{0.576276in}}{\pgfqpoint{0.818346in}{0.581576in}}{\pgfqpoint{0.818346in}{0.587101in}}%
\pgfpathcurveto{\pgfqpoint{0.818346in}{0.592626in}}{\pgfqpoint{0.816151in}{0.597925in}}{\pgfqpoint{0.812244in}{0.601832in}}%
\pgfpathcurveto{\pgfqpoint{0.808337in}{0.605739in}}{\pgfqpoint{0.803038in}{0.607934in}}{\pgfqpoint{0.797513in}{0.607934in}}%
\pgfpathcurveto{\pgfqpoint{0.791988in}{0.607934in}}{\pgfqpoint{0.786688in}{0.605739in}}{\pgfqpoint{0.782781in}{0.601832in}}%
\pgfpathcurveto{\pgfqpoint{0.778875in}{0.597925in}}{\pgfqpoint{0.776679in}{0.592626in}}{\pgfqpoint{0.776679in}{0.587101in}}%
\pgfpathcurveto{\pgfqpoint{0.776679in}{0.581576in}}{\pgfqpoint{0.778875in}{0.576276in}}{\pgfqpoint{0.782781in}{0.572369in}}%
\pgfpathcurveto{\pgfqpoint{0.786688in}{0.568462in}}{\pgfqpoint{0.791988in}{0.566267in}}{\pgfqpoint{0.797513in}{0.566267in}}%
\pgfpathclose%
\pgfusepath{stroke,fill}%
\end{pgfscope}%
\begin{pgfscope}%
\pgfpathrectangle{\pgfqpoint{0.403630in}{0.251348in}}{\pgfqpoint{1.395000in}{1.359000in}}%
\pgfusepath{clip}%
\pgfsetbuttcap%
\pgfsetroundjoin%
\definecolor{currentfill}{rgb}{0.686275,0.352941,0.313725}%
\pgfsetfillcolor{currentfill}%
\pgfsetlinewidth{0.501875pt}%
\definecolor{currentstroke}{rgb}{0.686275,0.352941,0.313725}%
\pgfsetstrokecolor{currentstroke}%
\pgfsetdash{}{0pt}%
\pgfpathmoveto{\pgfqpoint{1.054630in}{1.365679in}}%
\pgfpathcurveto{\pgfqpoint{1.060155in}{1.365679in}}{\pgfqpoint{1.065455in}{1.367874in}}{\pgfqpoint{1.069362in}{1.371781in}}%
\pgfpathcurveto{\pgfqpoint{1.073269in}{1.375688in}}{\pgfqpoint{1.075464in}{1.380987in}}{\pgfqpoint{1.075464in}{1.386512in}}%
\pgfpathcurveto{\pgfqpoint{1.075464in}{1.392037in}}{\pgfqpoint{1.073269in}{1.397337in}}{\pgfqpoint{1.069362in}{1.401244in}}%
\pgfpathcurveto{\pgfqpoint{1.065455in}{1.405151in}}{\pgfqpoint{1.060155in}{1.407346in}}{\pgfqpoint{1.054630in}{1.407346in}}%
\pgfpathcurveto{\pgfqpoint{1.049105in}{1.407346in}}{\pgfqpoint{1.043806in}{1.405151in}}{\pgfqpoint{1.039899in}{1.401244in}}%
\pgfpathcurveto{\pgfqpoint{1.035992in}{1.397337in}}{\pgfqpoint{1.033797in}{1.392037in}}{\pgfqpoint{1.033797in}{1.386512in}}%
\pgfpathcurveto{\pgfqpoint{1.033797in}{1.380987in}}{\pgfqpoint{1.035992in}{1.375688in}}{\pgfqpoint{1.039899in}{1.371781in}}%
\pgfpathcurveto{\pgfqpoint{1.043806in}{1.367874in}}{\pgfqpoint{1.049105in}{1.365679in}}{\pgfqpoint{1.054630in}{1.365679in}}%
\pgfpathclose%
\pgfusepath{stroke,fill}%
\end{pgfscope}%
\begin{pgfscope}%
\pgfpathrectangle{\pgfqpoint{0.403630in}{0.251348in}}{\pgfqpoint{1.395000in}{1.359000in}}%
\pgfusepath{clip}%
\pgfsetbuttcap%
\pgfsetroundjoin%
\definecolor{currentfill}{rgb}{0.686275,0.352941,0.313725}%
\pgfsetfillcolor{currentfill}%
\pgfsetlinewidth{0.501875pt}%
\definecolor{currentstroke}{rgb}{0.686275,0.352941,0.313725}%
\pgfsetstrokecolor{currentstroke}%
\pgfsetdash{}{0pt}%
\pgfpathmoveto{\pgfqpoint{1.481336in}{0.624891in}}%
\pgfpathcurveto{\pgfqpoint{1.486861in}{0.624891in}}{\pgfqpoint{1.492161in}{0.627086in}}{\pgfqpoint{1.496068in}{0.630993in}}%
\pgfpathcurveto{\pgfqpoint{1.499975in}{0.634900in}}{\pgfqpoint{1.502170in}{0.640199in}}{\pgfqpoint{1.502170in}{0.645724in}}%
\pgfpathcurveto{\pgfqpoint{1.502170in}{0.651249in}}{\pgfqpoint{1.499975in}{0.656549in}}{\pgfqpoint{1.496068in}{0.660456in}}%
\pgfpathcurveto{\pgfqpoint{1.492161in}{0.664362in}}{\pgfqpoint{1.486861in}{0.666557in}}{\pgfqpoint{1.481336in}{0.666557in}}%
\pgfpathcurveto{\pgfqpoint{1.475811in}{0.666557in}}{\pgfqpoint{1.470512in}{0.664362in}}{\pgfqpoint{1.466605in}{0.660456in}}%
\pgfpathcurveto{\pgfqpoint{1.462698in}{0.656549in}}{\pgfqpoint{1.460503in}{0.651249in}}{\pgfqpoint{1.460503in}{0.645724in}}%
\pgfpathcurveto{\pgfqpoint{1.460503in}{0.640199in}}{\pgfqpoint{1.462698in}{0.634900in}}{\pgfqpoint{1.466605in}{0.630993in}}%
\pgfpathcurveto{\pgfqpoint{1.470512in}{0.627086in}}{\pgfqpoint{1.475811in}{0.624891in}}{\pgfqpoint{1.481336in}{0.624891in}}%
\pgfpathclose%
\pgfusepath{stroke,fill}%
\end{pgfscope}%
\begin{pgfscope}%
\pgfpathrectangle{\pgfqpoint{0.403630in}{0.251348in}}{\pgfqpoint{1.395000in}{1.359000in}}%
\pgfusepath{clip}%
\pgfsetbuttcap%
\pgfsetroundjoin%
\definecolor{currentfill}{rgb}{0.686275,0.352941,0.313725}%
\pgfsetfillcolor{currentfill}%
\pgfsetlinewidth{0.501875pt}%
\definecolor{currentstroke}{rgb}{0.686275,0.352941,0.313725}%
\pgfsetstrokecolor{currentstroke}%
\pgfsetdash{}{0pt}%
\pgfpathmoveto{\pgfqpoint{0.671689in}{0.795432in}}%
\pgfpathcurveto{\pgfqpoint{0.677214in}{0.795432in}}{\pgfqpoint{0.682514in}{0.797627in}}{\pgfqpoint{0.686421in}{0.801534in}}%
\pgfpathcurveto{\pgfqpoint{0.690327in}{0.805441in}}{\pgfqpoint{0.692523in}{0.810740in}}{\pgfqpoint{0.692523in}{0.816265in}}%
\pgfpathcurveto{\pgfqpoint{0.692523in}{0.821790in}}{\pgfqpoint{0.690327in}{0.827090in}}{\pgfqpoint{0.686421in}{0.830997in}}%
\pgfpathcurveto{\pgfqpoint{0.682514in}{0.834903in}}{\pgfqpoint{0.677214in}{0.837099in}}{\pgfqpoint{0.671689in}{0.837099in}}%
\pgfpathcurveto{\pgfqpoint{0.666164in}{0.837099in}}{\pgfqpoint{0.660865in}{0.834903in}}{\pgfqpoint{0.656958in}{0.830997in}}%
\pgfpathcurveto{\pgfqpoint{0.653051in}{0.827090in}}{\pgfqpoint{0.650856in}{0.821790in}}{\pgfqpoint{0.650856in}{0.816265in}}%
\pgfpathcurveto{\pgfqpoint{0.650856in}{0.810740in}}{\pgfqpoint{0.653051in}{0.805441in}}{\pgfqpoint{0.656958in}{0.801534in}}%
\pgfpathcurveto{\pgfqpoint{0.660865in}{0.797627in}}{\pgfqpoint{0.666164in}{0.795432in}}{\pgfqpoint{0.671689in}{0.795432in}}%
\pgfpathclose%
\pgfusepath{stroke,fill}%
\end{pgfscope}%
\begin{pgfscope}%
\pgfpathrectangle{\pgfqpoint{0.403630in}{0.251348in}}{\pgfqpoint{1.395000in}{1.359000in}}%
\pgfusepath{clip}%
\pgfsetbuttcap%
\pgfsetroundjoin%
\definecolor{currentfill}{rgb}{0.686275,0.352941,0.313725}%
\pgfsetfillcolor{currentfill}%
\pgfsetlinewidth{0.501875pt}%
\definecolor{currentstroke}{rgb}{0.686275,0.352941,0.313725}%
\pgfsetstrokecolor{currentstroke}%
\pgfsetdash{}{0pt}%
\pgfpathmoveto{\pgfqpoint{1.541513in}{1.109867in}}%
\pgfpathcurveto{\pgfqpoint{1.547038in}{1.109867in}}{\pgfqpoint{1.552337in}{1.112062in}}{\pgfqpoint{1.556244in}{1.115969in}}%
\pgfpathcurveto{\pgfqpoint{1.560151in}{1.119876in}}{\pgfqpoint{1.562346in}{1.125176in}}{\pgfqpoint{1.562346in}{1.130701in}}%
\pgfpathcurveto{\pgfqpoint{1.562346in}{1.136226in}}{\pgfqpoint{1.560151in}{1.141525in}}{\pgfqpoint{1.556244in}{1.145432in}}%
\pgfpathcurveto{\pgfqpoint{1.552337in}{1.149339in}}{\pgfqpoint{1.547038in}{1.151534in}}{\pgfqpoint{1.541513in}{1.151534in}}%
\pgfpathcurveto{\pgfqpoint{1.535988in}{1.151534in}}{\pgfqpoint{1.530688in}{1.149339in}}{\pgfqpoint{1.526781in}{1.145432in}}%
\pgfpathcurveto{\pgfqpoint{1.522875in}{1.141525in}}{\pgfqpoint{1.520679in}{1.136226in}}{\pgfqpoint{1.520679in}{1.130701in}}%
\pgfpathcurveto{\pgfqpoint{1.520679in}{1.125176in}}{\pgfqpoint{1.522875in}{1.119876in}}{\pgfqpoint{1.526781in}{1.115969in}}%
\pgfpathcurveto{\pgfqpoint{1.530688in}{1.112062in}}{\pgfqpoint{1.535988in}{1.109867in}}{\pgfqpoint{1.541513in}{1.109867in}}%
\pgfpathclose%
\pgfusepath{stroke,fill}%
\end{pgfscope}%
\begin{pgfscope}%
\pgfpathrectangle{\pgfqpoint{0.403630in}{0.251348in}}{\pgfqpoint{1.395000in}{1.359000in}}%
\pgfusepath{clip}%
\pgfsetbuttcap%
\pgfsetroundjoin%
\definecolor{currentfill}{rgb}{0.686275,0.352941,0.313725}%
\pgfsetfillcolor{currentfill}%
\pgfsetlinewidth{0.501875pt}%
\definecolor{currentstroke}{rgb}{0.686275,0.352941,0.313725}%
\pgfsetstrokecolor{currentstroke}%
\pgfsetdash{}{0pt}%
\pgfpathmoveto{\pgfqpoint{1.054630in}{1.365679in}}%
\pgfpathcurveto{\pgfqpoint{1.060155in}{1.365679in}}{\pgfqpoint{1.065455in}{1.367874in}}{\pgfqpoint{1.069362in}{1.371781in}}%
\pgfpathcurveto{\pgfqpoint{1.073269in}{1.375688in}}{\pgfqpoint{1.075464in}{1.380987in}}{\pgfqpoint{1.075464in}{1.386512in}}%
\pgfpathcurveto{\pgfqpoint{1.075464in}{1.392037in}}{\pgfqpoint{1.073269in}{1.397337in}}{\pgfqpoint{1.069362in}{1.401244in}}%
\pgfpathcurveto{\pgfqpoint{1.065455in}{1.405151in}}{\pgfqpoint{1.060155in}{1.407346in}}{\pgfqpoint{1.054630in}{1.407346in}}%
\pgfpathcurveto{\pgfqpoint{1.049105in}{1.407346in}}{\pgfqpoint{1.043806in}{1.405151in}}{\pgfqpoint{1.039899in}{1.401244in}}%
\pgfpathcurveto{\pgfqpoint{1.035992in}{1.397337in}}{\pgfqpoint{1.033797in}{1.392037in}}{\pgfqpoint{1.033797in}{1.386512in}}%
\pgfpathcurveto{\pgfqpoint{1.033797in}{1.380987in}}{\pgfqpoint{1.035992in}{1.375688in}}{\pgfqpoint{1.039899in}{1.371781in}}%
\pgfpathcurveto{\pgfqpoint{1.043806in}{1.367874in}}{\pgfqpoint{1.049105in}{1.365679in}}{\pgfqpoint{1.054630in}{1.365679in}}%
\pgfpathclose%
\pgfusepath{stroke,fill}%
\end{pgfscope}%
\begin{pgfscope}%
\pgfpathrectangle{\pgfqpoint{0.403630in}{0.251348in}}{\pgfqpoint{1.395000in}{1.359000in}}%
\pgfusepath{clip}%
\pgfsetbuttcap%
\pgfsetroundjoin%
\definecolor{currentfill}{rgb}{0.686275,0.352941,0.313725}%
\pgfsetfillcolor{currentfill}%
\pgfsetlinewidth{0.501875pt}%
\definecolor{currentstroke}{rgb}{0.686275,0.352941,0.313725}%
\pgfsetstrokecolor{currentstroke}%
\pgfsetdash{}{0pt}%
\pgfpathmoveto{\pgfqpoint{1.399277in}{1.328373in}}%
\pgfpathcurveto{\pgfqpoint{1.404803in}{1.328373in}}{\pgfqpoint{1.410102in}{1.330568in}}{\pgfqpoint{1.414009in}{1.334475in}}%
\pgfpathcurveto{\pgfqpoint{1.417916in}{1.338382in}}{\pgfqpoint{1.420111in}{1.343681in}}{\pgfqpoint{1.420111in}{1.349206in}}%
\pgfpathcurveto{\pgfqpoint{1.420111in}{1.354732in}}{\pgfqpoint{1.417916in}{1.360031in}}{\pgfqpoint{1.414009in}{1.363938in}}%
\pgfpathcurveto{\pgfqpoint{1.410102in}{1.367845in}}{\pgfqpoint{1.404803in}{1.370040in}}{\pgfqpoint{1.399277in}{1.370040in}}%
\pgfpathcurveto{\pgfqpoint{1.393752in}{1.370040in}}{\pgfqpoint{1.388453in}{1.367845in}}{\pgfqpoint{1.384546in}{1.363938in}}%
\pgfpathcurveto{\pgfqpoint{1.380639in}{1.360031in}}{\pgfqpoint{1.378444in}{1.354732in}}{\pgfqpoint{1.378444in}{1.349206in}}%
\pgfpathcurveto{\pgfqpoint{1.378444in}{1.343681in}}{\pgfqpoint{1.380639in}{1.338382in}}{\pgfqpoint{1.384546in}{1.334475in}}%
\pgfpathcurveto{\pgfqpoint{1.388453in}{1.330568in}}{\pgfqpoint{1.393752in}{1.328373in}}{\pgfqpoint{1.399277in}{1.328373in}}%
\pgfpathclose%
\pgfusepath{stroke,fill}%
\end{pgfscope}%
\begin{pgfscope}%
\pgfpathrectangle{\pgfqpoint{0.403630in}{0.251348in}}{\pgfqpoint{1.395000in}{1.359000in}}%
\pgfusepath{clip}%
\pgfsetbuttcap%
\pgfsetroundjoin%
\definecolor{currentfill}{rgb}{0.686275,0.352941,0.313725}%
\pgfsetfillcolor{currentfill}%
\pgfsetlinewidth{0.501875pt}%
\definecolor{currentstroke}{rgb}{0.686275,0.352941,0.313725}%
\pgfsetstrokecolor{currentstroke}%
\pgfsetdash{}{0pt}%
\pgfpathmoveto{\pgfqpoint{1.536042in}{0.752797in}}%
\pgfpathcurveto{\pgfqpoint{1.541567in}{0.752797in}}{\pgfqpoint{1.546867in}{0.754992in}}{\pgfqpoint{1.550774in}{0.758899in}}%
\pgfpathcurveto{\pgfqpoint{1.554680in}{0.762805in}}{\pgfqpoint{1.556876in}{0.768105in}}{\pgfqpoint{1.556876in}{0.773630in}}%
\pgfpathcurveto{\pgfqpoint{1.556876in}{0.779155in}}{\pgfqpoint{1.554680in}{0.784455in}}{\pgfqpoint{1.550774in}{0.788361in}}%
\pgfpathcurveto{\pgfqpoint{1.546867in}{0.792268in}}{\pgfqpoint{1.541567in}{0.794463in}}{\pgfqpoint{1.536042in}{0.794463in}}%
\pgfpathcurveto{\pgfqpoint{1.530517in}{0.794463in}}{\pgfqpoint{1.525218in}{0.792268in}}{\pgfqpoint{1.521311in}{0.788361in}}%
\pgfpathcurveto{\pgfqpoint{1.517404in}{0.784455in}}{\pgfqpoint{1.515209in}{0.779155in}}{\pgfqpoint{1.515209in}{0.773630in}}%
\pgfpathcurveto{\pgfqpoint{1.515209in}{0.768105in}}{\pgfqpoint{1.517404in}{0.762805in}}{\pgfqpoint{1.521311in}{0.758899in}}%
\pgfpathcurveto{\pgfqpoint{1.525218in}{0.754992in}}{\pgfqpoint{1.530517in}{0.752797in}}{\pgfqpoint{1.536042in}{0.752797in}}%
\pgfpathclose%
\pgfusepath{stroke,fill}%
\end{pgfscope}%
\begin{pgfscope}%
\pgfpathrectangle{\pgfqpoint{0.403630in}{0.251348in}}{\pgfqpoint{1.395000in}{1.359000in}}%
\pgfusepath{clip}%
\pgfsetbuttcap%
\pgfsetroundjoin%
\definecolor{currentfill}{rgb}{0.686275,0.352941,0.313725}%
\pgfsetfillcolor{currentfill}%
\pgfsetlinewidth{0.501875pt}%
\definecolor{currentstroke}{rgb}{0.686275,0.352941,0.313725}%
\pgfsetstrokecolor{currentstroke}%
\pgfsetdash{}{0pt}%
\pgfpathmoveto{\pgfqpoint{1.273454in}{0.443691in}}%
\pgfpathcurveto{\pgfqpoint{1.278979in}{0.443691in}}{\pgfqpoint{1.284279in}{0.445886in}}{\pgfqpoint{1.288185in}{0.449793in}}%
\pgfpathcurveto{\pgfqpoint{1.292092in}{0.453700in}}{\pgfqpoint{1.294287in}{0.458999in}}{\pgfqpoint{1.294287in}{0.464524in}}%
\pgfpathcurveto{\pgfqpoint{1.294287in}{0.470049in}}{\pgfqpoint{1.292092in}{0.475349in}}{\pgfqpoint{1.288185in}{0.479256in}}%
\pgfpathcurveto{\pgfqpoint{1.284279in}{0.483162in}}{\pgfqpoint{1.278979in}{0.485357in}}{\pgfqpoint{1.273454in}{0.485357in}}%
\pgfpathcurveto{\pgfqpoint{1.267929in}{0.485357in}}{\pgfqpoint{1.262629in}{0.483162in}}{\pgfqpoint{1.258723in}{0.479256in}}%
\pgfpathcurveto{\pgfqpoint{1.254816in}{0.475349in}}{\pgfqpoint{1.252621in}{0.470049in}}{\pgfqpoint{1.252621in}{0.464524in}}%
\pgfpathcurveto{\pgfqpoint{1.252621in}{0.458999in}}{\pgfqpoint{1.254816in}{0.453700in}}{\pgfqpoint{1.258723in}{0.449793in}}%
\pgfpathcurveto{\pgfqpoint{1.262629in}{0.445886in}}{\pgfqpoint{1.267929in}{0.443691in}}{\pgfqpoint{1.273454in}{0.443691in}}%
\pgfpathclose%
\pgfusepath{stroke,fill}%
\end{pgfscope}%
\begin{pgfscope}%
\pgfpathrectangle{\pgfqpoint{0.403630in}{0.251348in}}{\pgfqpoint{1.395000in}{1.359000in}}%
\pgfusepath{clip}%
\pgfsetbuttcap%
\pgfsetroundjoin%
\definecolor{currentfill}{rgb}{0.686275,0.352941,0.313725}%
\pgfsetfillcolor{currentfill}%
\pgfsetlinewidth{0.501875pt}%
\definecolor{currentstroke}{rgb}{0.686275,0.352941,0.313725}%
\pgfsetstrokecolor{currentstroke}%
\pgfsetdash{}{0pt}%
\pgfpathmoveto{\pgfqpoint{1.174983in}{0.417044in}}%
\pgfpathcurveto{\pgfqpoint{1.180508in}{0.417044in}}{\pgfqpoint{1.185808in}{0.419239in}}{\pgfqpoint{1.189715in}{0.423146in}}%
\pgfpathcurveto{\pgfqpoint{1.193622in}{0.427052in}}{\pgfqpoint{1.195817in}{0.432352in}}{\pgfqpoint{1.195817in}{0.437877in}}%
\pgfpathcurveto{\pgfqpoint{1.195817in}{0.443402in}}{\pgfqpoint{1.193622in}{0.448702in}}{\pgfqpoint{1.189715in}{0.452608in}}%
\pgfpathcurveto{\pgfqpoint{1.185808in}{0.456515in}}{\pgfqpoint{1.180508in}{0.458710in}}{\pgfqpoint{1.174983in}{0.458710in}}%
\pgfpathcurveto{\pgfqpoint{1.169458in}{0.458710in}}{\pgfqpoint{1.164159in}{0.456515in}}{\pgfqpoint{1.160252in}{0.452608in}}%
\pgfpathcurveto{\pgfqpoint{1.156345in}{0.448702in}}{\pgfqpoint{1.154150in}{0.443402in}}{\pgfqpoint{1.154150in}{0.437877in}}%
\pgfpathcurveto{\pgfqpoint{1.154150in}{0.432352in}}{\pgfqpoint{1.156345in}{0.427052in}}{\pgfqpoint{1.160252in}{0.423146in}}%
\pgfpathcurveto{\pgfqpoint{1.164159in}{0.419239in}}{\pgfqpoint{1.169458in}{0.417044in}}{\pgfqpoint{1.174983in}{0.417044in}}%
\pgfpathclose%
\pgfusepath{stroke,fill}%
\end{pgfscope}%
\begin{pgfscope}%
\pgfpathrectangle{\pgfqpoint{0.403630in}{0.251348in}}{\pgfqpoint{1.395000in}{1.359000in}}%
\pgfusepath{clip}%
\pgfsetbuttcap%
\pgfsetroundjoin%
\definecolor{currentfill}{rgb}{0.686275,0.352941,0.313725}%
\pgfsetfillcolor{currentfill}%
\pgfsetlinewidth{0.501875pt}%
\definecolor{currentstroke}{rgb}{0.686275,0.352941,0.313725}%
\pgfsetstrokecolor{currentstroke}%
\pgfsetdash{}{0pt}%
\pgfpathmoveto{\pgfqpoint{1.546983in}{0.939326in}}%
\pgfpathcurveto{\pgfqpoint{1.552508in}{0.939326in}}{\pgfqpoint{1.557808in}{0.941521in}}{\pgfqpoint{1.561715in}{0.945428in}}%
\pgfpathcurveto{\pgfqpoint{1.565622in}{0.949335in}}{\pgfqpoint{1.567817in}{0.954634in}}{\pgfqpoint{1.567817in}{0.960159in}}%
\pgfpathcurveto{\pgfqpoint{1.567817in}{0.965684in}}{\pgfqpoint{1.565622in}{0.970984in}}{\pgfqpoint{1.561715in}{0.974891in}}%
\pgfpathcurveto{\pgfqpoint{1.557808in}{0.978798in}}{\pgfqpoint{1.552508in}{0.980993in}}{\pgfqpoint{1.546983in}{0.980993in}}%
\pgfpathcurveto{\pgfqpoint{1.541458in}{0.980993in}}{\pgfqpoint{1.536159in}{0.978798in}}{\pgfqpoint{1.532252in}{0.974891in}}%
\pgfpathcurveto{\pgfqpoint{1.528345in}{0.970984in}}{\pgfqpoint{1.526150in}{0.965684in}}{\pgfqpoint{1.526150in}{0.960159in}}%
\pgfpathcurveto{\pgfqpoint{1.526150in}{0.954634in}}{\pgfqpoint{1.528345in}{0.949335in}}{\pgfqpoint{1.532252in}{0.945428in}}%
\pgfpathcurveto{\pgfqpoint{1.536159in}{0.941521in}}{\pgfqpoint{1.541458in}{0.939326in}}{\pgfqpoint{1.546983in}{0.939326in}}%
\pgfpathclose%
\pgfusepath{stroke,fill}%
\end{pgfscope}%
\begin{pgfscope}%
\pgfpathrectangle{\pgfqpoint{0.403630in}{0.251348in}}{\pgfqpoint{1.395000in}{1.359000in}}%
\pgfusepath{clip}%
\pgfsetbuttcap%
\pgfsetroundjoin%
\definecolor{currentfill}{rgb}{0.686275,0.352941,0.313725}%
\pgfsetfillcolor{currentfill}%
\pgfsetlinewidth{0.501875pt}%
\definecolor{currentstroke}{rgb}{0.686275,0.352941,0.313725}%
\pgfsetstrokecolor{currentstroke}%
\pgfsetdash{}{0pt}%
\pgfpathmoveto{\pgfqpoint{1.038219in}{1.381667in}}%
\pgfpathcurveto{\pgfqpoint{1.043744in}{1.381667in}}{\pgfqpoint{1.049043in}{1.383862in}}{\pgfqpoint{1.052950in}{1.387769in}}%
\pgfpathcurveto{\pgfqpoint{1.056857in}{1.391676in}}{\pgfqpoint{1.059052in}{1.396976in}}{\pgfqpoint{1.059052in}{1.402501in}}%
\pgfpathcurveto{\pgfqpoint{1.059052in}{1.408026in}}{\pgfqpoint{1.056857in}{1.413325in}}{\pgfqpoint{1.052950in}{1.417232in}}%
\pgfpathcurveto{\pgfqpoint{1.049043in}{1.421139in}}{\pgfqpoint{1.043744in}{1.423334in}}{\pgfqpoint{1.038219in}{1.423334in}}%
\pgfpathcurveto{\pgfqpoint{1.032694in}{1.423334in}}{\pgfqpoint{1.027394in}{1.421139in}}{\pgfqpoint{1.023487in}{1.417232in}}%
\pgfpathcurveto{\pgfqpoint{1.019580in}{1.413325in}}{\pgfqpoint{1.017385in}{1.408026in}}{\pgfqpoint{1.017385in}{1.402501in}}%
\pgfpathcurveto{\pgfqpoint{1.017385in}{1.396976in}}{\pgfqpoint{1.019580in}{1.391676in}}{\pgfqpoint{1.023487in}{1.387769in}}%
\pgfpathcurveto{\pgfqpoint{1.027394in}{1.383862in}}{\pgfqpoint{1.032694in}{1.381667in}}{\pgfqpoint{1.038219in}{1.381667in}}%
\pgfpathclose%
\pgfusepath{stroke,fill}%
\end{pgfscope}%
\begin{pgfscope}%
\pgfpathrectangle{\pgfqpoint{0.403630in}{0.251348in}}{\pgfqpoint{1.395000in}{1.359000in}}%
\pgfusepath{clip}%
\pgfsetbuttcap%
\pgfsetroundjoin%
\definecolor{currentfill}{rgb}{0.686275,0.352941,0.313725}%
\pgfsetfillcolor{currentfill}%
\pgfsetlinewidth{0.501875pt}%
\definecolor{currentstroke}{rgb}{0.686275,0.352941,0.313725}%
\pgfsetstrokecolor{currentstroke}%
\pgfsetdash{}{0pt}%
\pgfpathmoveto{\pgfqpoint{1.481336in}{0.667526in}}%
\pgfpathcurveto{\pgfqpoint{1.486861in}{0.667526in}}{\pgfqpoint{1.492161in}{0.669721in}}{\pgfqpoint{1.496068in}{0.673628in}}%
\pgfpathcurveto{\pgfqpoint{1.499975in}{0.677535in}}{\pgfqpoint{1.502170in}{0.682834in}}{\pgfqpoint{1.502170in}{0.688359in}}%
\pgfpathcurveto{\pgfqpoint{1.502170in}{0.693884in}}{\pgfqpoint{1.499975in}{0.699184in}}{\pgfqpoint{1.496068in}{0.703091in}}%
\pgfpathcurveto{\pgfqpoint{1.492161in}{0.706998in}}{\pgfqpoint{1.486861in}{0.709193in}}{\pgfqpoint{1.481336in}{0.709193in}}%
\pgfpathcurveto{\pgfqpoint{1.475811in}{0.709193in}}{\pgfqpoint{1.470512in}{0.706998in}}{\pgfqpoint{1.466605in}{0.703091in}}%
\pgfpathcurveto{\pgfqpoint{1.462698in}{0.699184in}}{\pgfqpoint{1.460503in}{0.693884in}}{\pgfqpoint{1.460503in}{0.688359in}}%
\pgfpathcurveto{\pgfqpoint{1.460503in}{0.682834in}}{\pgfqpoint{1.462698in}{0.677535in}}{\pgfqpoint{1.466605in}{0.673628in}}%
\pgfpathcurveto{\pgfqpoint{1.470512in}{0.669721in}}{\pgfqpoint{1.475811in}{0.667526in}}{\pgfqpoint{1.481336in}{0.667526in}}%
\pgfpathclose%
\pgfusepath{stroke,fill}%
\end{pgfscope}%
\begin{pgfscope}%
\pgfpathrectangle{\pgfqpoint{0.403630in}{0.251348in}}{\pgfqpoint{1.395000in}{1.359000in}}%
\pgfusepath{clip}%
\pgfsetbuttcap%
\pgfsetroundjoin%
\definecolor{currentfill}{rgb}{0.686275,0.352941,0.313725}%
\pgfsetfillcolor{currentfill}%
\pgfsetlinewidth{0.501875pt}%
\definecolor{currentstroke}{rgb}{0.686275,0.352941,0.313725}%
\pgfsetstrokecolor{currentstroke}%
\pgfsetdash{}{0pt}%
\pgfpathmoveto{\pgfqpoint{0.709983in}{1.157832in}}%
\pgfpathcurveto{\pgfqpoint{0.715508in}{1.157832in}}{\pgfqpoint{0.720808in}{1.160027in}}{\pgfqpoint{0.724715in}{1.163934in}}%
\pgfpathcurveto{\pgfqpoint{0.728622in}{1.167841in}}{\pgfqpoint{0.730817in}{1.173140in}}{\pgfqpoint{0.730817in}{1.178665in}}%
\pgfpathcurveto{\pgfqpoint{0.730817in}{1.184190in}}{\pgfqpoint{0.728622in}{1.189490in}}{\pgfqpoint{0.724715in}{1.193397in}}%
\pgfpathcurveto{\pgfqpoint{0.720808in}{1.197303in}}{\pgfqpoint{0.715508in}{1.199499in}}{\pgfqpoint{0.709983in}{1.199499in}}%
\pgfpathcurveto{\pgfqpoint{0.704458in}{1.199499in}}{\pgfqpoint{0.699159in}{1.197303in}}{\pgfqpoint{0.695252in}{1.193397in}}%
\pgfpathcurveto{\pgfqpoint{0.691345in}{1.189490in}}{\pgfqpoint{0.689150in}{1.184190in}}{\pgfqpoint{0.689150in}{1.178665in}}%
\pgfpathcurveto{\pgfqpoint{0.689150in}{1.173140in}}{\pgfqpoint{0.691345in}{1.167841in}}{\pgfqpoint{0.695252in}{1.163934in}}%
\pgfpathcurveto{\pgfqpoint{0.699159in}{1.160027in}}{\pgfqpoint{0.704458in}{1.157832in}}{\pgfqpoint{0.709983in}{1.157832in}}%
\pgfpathclose%
\pgfusepath{stroke,fill}%
\end{pgfscope}%
\begin{pgfscope}%
\pgfpathrectangle{\pgfqpoint{0.403630in}{0.251348in}}{\pgfqpoint{1.395000in}{1.359000in}}%
\pgfusepath{clip}%
\pgfsetbuttcap%
\pgfsetroundjoin%
\definecolor{currentfill}{rgb}{0.686275,0.352941,0.313725}%
\pgfsetfillcolor{currentfill}%
\pgfsetlinewidth{0.501875pt}%
\definecolor{currentstroke}{rgb}{0.686275,0.352941,0.313725}%
\pgfsetstrokecolor{currentstroke}%
\pgfsetdash{}{0pt}%
\pgfpathmoveto{\pgfqpoint{1.169513in}{1.402985in}}%
\pgfpathcurveto{\pgfqpoint{1.175038in}{1.402985in}}{\pgfqpoint{1.180337in}{1.405180in}}{\pgfqpoint{1.184244in}{1.409087in}}%
\pgfpathcurveto{\pgfqpoint{1.188151in}{1.412994in}}{\pgfqpoint{1.190346in}{1.418293in}}{\pgfqpoint{1.190346in}{1.423818in}}%
\pgfpathcurveto{\pgfqpoint{1.190346in}{1.429343in}}{\pgfqpoint{1.188151in}{1.434643in}}{\pgfqpoint{1.184244in}{1.438550in}}%
\pgfpathcurveto{\pgfqpoint{1.180337in}{1.442456in}}{\pgfqpoint{1.175038in}{1.444652in}}{\pgfqpoint{1.169513in}{1.444652in}}%
\pgfpathcurveto{\pgfqpoint{1.163988in}{1.444652in}}{\pgfqpoint{1.158688in}{1.442456in}}{\pgfqpoint{1.154781in}{1.438550in}}%
\pgfpathcurveto{\pgfqpoint{1.150875in}{1.434643in}}{\pgfqpoint{1.148679in}{1.429343in}}{\pgfqpoint{1.148679in}{1.423818in}}%
\pgfpathcurveto{\pgfqpoint{1.148679in}{1.418293in}}{\pgfqpoint{1.150875in}{1.412994in}}{\pgfqpoint{1.154781in}{1.409087in}}%
\pgfpathcurveto{\pgfqpoint{1.158688in}{1.405180in}}{\pgfqpoint{1.163988in}{1.402985in}}{\pgfqpoint{1.169513in}{1.402985in}}%
\pgfpathclose%
\pgfusepath{stroke,fill}%
\end{pgfscope}%
\begin{pgfscope}%
\pgfpathrectangle{\pgfqpoint{0.403630in}{0.251348in}}{\pgfqpoint{1.395000in}{1.359000in}}%
\pgfusepath{clip}%
\pgfsetbuttcap%
\pgfsetroundjoin%
\definecolor{currentfill}{rgb}{0.686275,0.352941,0.313725}%
\pgfsetfillcolor{currentfill}%
\pgfsetlinewidth{0.501875pt}%
\definecolor{currentstroke}{rgb}{0.686275,0.352941,0.313725}%
\pgfsetstrokecolor{currentstroke}%
\pgfsetdash{}{0pt}%
\pgfpathmoveto{\pgfqpoint{1.453983in}{1.211126in}}%
\pgfpathcurveto{\pgfqpoint{1.459508in}{1.211126in}}{\pgfqpoint{1.464808in}{1.213321in}}{\pgfqpoint{1.468715in}{1.217228in}}%
\pgfpathcurveto{\pgfqpoint{1.472622in}{1.221135in}}{\pgfqpoint{1.474817in}{1.226434in}}{\pgfqpoint{1.474817in}{1.231959in}}%
\pgfpathcurveto{\pgfqpoint{1.474817in}{1.237484in}}{\pgfqpoint{1.472622in}{1.242784in}}{\pgfqpoint{1.468715in}{1.246691in}}%
\pgfpathcurveto{\pgfqpoint{1.464808in}{1.250598in}}{\pgfqpoint{1.459508in}{1.252793in}}{\pgfqpoint{1.453983in}{1.252793in}}%
\pgfpathcurveto{\pgfqpoint{1.448458in}{1.252793in}}{\pgfqpoint{1.443159in}{1.250598in}}{\pgfqpoint{1.439252in}{1.246691in}}%
\pgfpathcurveto{\pgfqpoint{1.435345in}{1.242784in}}{\pgfqpoint{1.433150in}{1.237484in}}{\pgfqpoint{1.433150in}{1.231959in}}%
\pgfpathcurveto{\pgfqpoint{1.433150in}{1.226434in}}{\pgfqpoint{1.435345in}{1.221135in}}{\pgfqpoint{1.439252in}{1.217228in}}%
\pgfpathcurveto{\pgfqpoint{1.443159in}{1.213321in}}{\pgfqpoint{1.448458in}{1.211126in}}{\pgfqpoint{1.453983in}{1.211126in}}%
\pgfpathclose%
\pgfusepath{stroke,fill}%
\end{pgfscope}%
\begin{pgfscope}%
\pgfpathrectangle{\pgfqpoint{0.403630in}{0.251348in}}{\pgfqpoint{1.395000in}{1.359000in}}%
\pgfusepath{clip}%
\pgfsetbuttcap%
\pgfsetroundjoin%
\definecolor{currentfill}{rgb}{0.686275,0.352941,0.313725}%
\pgfsetfillcolor{currentfill}%
\pgfsetlinewidth{0.501875pt}%
\definecolor{currentstroke}{rgb}{0.686275,0.352941,0.313725}%
\pgfsetstrokecolor{currentstroke}%
\pgfsetdash{}{0pt}%
\pgfpathmoveto{\pgfqpoint{1.289866in}{1.365679in}}%
\pgfpathcurveto{\pgfqpoint{1.295391in}{1.365679in}}{\pgfqpoint{1.300690in}{1.367874in}}{\pgfqpoint{1.304597in}{1.371781in}}%
\pgfpathcurveto{\pgfqpoint{1.308504in}{1.375688in}}{\pgfqpoint{1.310699in}{1.380987in}}{\pgfqpoint{1.310699in}{1.386512in}}%
\pgfpathcurveto{\pgfqpoint{1.310699in}{1.392037in}}{\pgfqpoint{1.308504in}{1.397337in}}{\pgfqpoint{1.304597in}{1.401244in}}%
\pgfpathcurveto{\pgfqpoint{1.300690in}{1.405151in}}{\pgfqpoint{1.295391in}{1.407346in}}{\pgfqpoint{1.289866in}{1.407346in}}%
\pgfpathcurveto{\pgfqpoint{1.284341in}{1.407346in}}{\pgfqpoint{1.279041in}{1.405151in}}{\pgfqpoint{1.275134in}{1.401244in}}%
\pgfpathcurveto{\pgfqpoint{1.271228in}{1.397337in}}{\pgfqpoint{1.269032in}{1.392037in}}{\pgfqpoint{1.269032in}{1.386512in}}%
\pgfpathcurveto{\pgfqpoint{1.269032in}{1.380987in}}{\pgfqpoint{1.271228in}{1.375688in}}{\pgfqpoint{1.275134in}{1.371781in}}%
\pgfpathcurveto{\pgfqpoint{1.279041in}{1.367874in}}{\pgfqpoint{1.284341in}{1.365679in}}{\pgfqpoint{1.289866in}{1.365679in}}%
\pgfpathclose%
\pgfusepath{stroke,fill}%
\end{pgfscope}%
\begin{pgfscope}%
\pgfpathrectangle{\pgfqpoint{0.403630in}{0.251348in}}{\pgfqpoint{1.395000in}{1.359000in}}%
\pgfusepath{clip}%
\pgfsetbuttcap%
\pgfsetroundjoin%
\definecolor{currentfill}{rgb}{0.686275,0.352941,0.313725}%
\pgfsetfillcolor{currentfill}%
\pgfsetlinewidth{0.501875pt}%
\definecolor{currentstroke}{rgb}{0.686275,0.352941,0.313725}%
\pgfsetstrokecolor{currentstroke}%
\pgfsetdash{}{0pt}%
\pgfpathmoveto{\pgfqpoint{0.693572in}{0.683514in}}%
\pgfpathcurveto{\pgfqpoint{0.699097in}{0.683514in}}{\pgfqpoint{0.704396in}{0.685709in}}{\pgfqpoint{0.708303in}{0.689616in}}%
\pgfpathcurveto{\pgfqpoint{0.712210in}{0.693523in}}{\pgfqpoint{0.714405in}{0.698823in}}{\pgfqpoint{0.714405in}{0.704348in}}%
\pgfpathcurveto{\pgfqpoint{0.714405in}{0.709873in}}{\pgfqpoint{0.712210in}{0.715172in}}{\pgfqpoint{0.708303in}{0.719079in}}%
\pgfpathcurveto{\pgfqpoint{0.704396in}{0.722986in}}{\pgfqpoint{0.699097in}{0.725181in}}{\pgfqpoint{0.693572in}{0.725181in}}%
\pgfpathcurveto{\pgfqpoint{0.688047in}{0.725181in}}{\pgfqpoint{0.682747in}{0.722986in}}{\pgfqpoint{0.678840in}{0.719079in}}%
\pgfpathcurveto{\pgfqpoint{0.674933in}{0.715172in}}{\pgfqpoint{0.672738in}{0.709873in}}{\pgfqpoint{0.672738in}{0.704348in}}%
\pgfpathcurveto{\pgfqpoint{0.672738in}{0.698823in}}{\pgfqpoint{0.674933in}{0.693523in}}{\pgfqpoint{0.678840in}{0.689616in}}%
\pgfpathcurveto{\pgfqpoint{0.682747in}{0.685709in}}{\pgfqpoint{0.688047in}{0.683514in}}{\pgfqpoint{0.693572in}{0.683514in}}%
\pgfpathclose%
\pgfusepath{stroke,fill}%
\end{pgfscope}%
\begin{pgfscope}%
\pgfpathrectangle{\pgfqpoint{0.403630in}{0.251348in}}{\pgfqpoint{1.395000in}{1.359000in}}%
\pgfusepath{clip}%
\pgfsetbuttcap%
\pgfsetroundjoin%
\definecolor{currentfill}{rgb}{0.686275,0.352941,0.313725}%
\pgfsetfillcolor{currentfill}%
\pgfsetlinewidth{0.501875pt}%
\definecolor{currentstroke}{rgb}{0.686275,0.352941,0.313725}%
\pgfsetstrokecolor{currentstroke}%
\pgfsetdash{}{0pt}%
\pgfpathmoveto{\pgfqpoint{1.448513in}{0.603573in}}%
\pgfpathcurveto{\pgfqpoint{1.454038in}{0.603573in}}{\pgfqpoint{1.459337in}{0.605768in}}{\pgfqpoint{1.463244in}{0.609675in}}%
\pgfpathcurveto{\pgfqpoint{1.467151in}{0.613582in}}{\pgfqpoint{1.469346in}{0.618881in}}{\pgfqpoint{1.469346in}{0.624406in}}%
\pgfpathcurveto{\pgfqpoint{1.469346in}{0.629932in}}{\pgfqpoint{1.467151in}{0.635231in}}{\pgfqpoint{1.463244in}{0.639138in}}%
\pgfpathcurveto{\pgfqpoint{1.459337in}{0.643045in}}{\pgfqpoint{1.454038in}{0.645240in}}{\pgfqpoint{1.448513in}{0.645240in}}%
\pgfpathcurveto{\pgfqpoint{1.442988in}{0.645240in}}{\pgfqpoint{1.437688in}{0.643045in}}{\pgfqpoint{1.433781in}{0.639138in}}%
\pgfpathcurveto{\pgfqpoint{1.429875in}{0.635231in}}{\pgfqpoint{1.427679in}{0.629932in}}{\pgfqpoint{1.427679in}{0.624406in}}%
\pgfpathcurveto{\pgfqpoint{1.427679in}{0.618881in}}{\pgfqpoint{1.429875in}{0.613582in}}{\pgfqpoint{1.433781in}{0.609675in}}%
\pgfpathcurveto{\pgfqpoint{1.437688in}{0.605768in}}{\pgfqpoint{1.442988in}{0.603573in}}{\pgfqpoint{1.448513in}{0.603573in}}%
\pgfpathclose%
\pgfusepath{stroke,fill}%
\end{pgfscope}%
\begin{pgfscope}%
\pgfpathrectangle{\pgfqpoint{0.403630in}{0.251348in}}{\pgfqpoint{1.395000in}{1.359000in}}%
\pgfusepath{clip}%
\pgfsetbuttcap%
\pgfsetroundjoin%
\definecolor{currentfill}{rgb}{0.686275,0.352941,0.313725}%
\pgfsetfillcolor{currentfill}%
\pgfsetlinewidth{0.501875pt}%
\definecolor{currentstroke}{rgb}{0.686275,0.352941,0.313725}%
\pgfsetstrokecolor{currentstroke}%
\pgfsetdash{}{0pt}%
\pgfpathmoveto{\pgfqpoint{0.622454in}{0.902020in}}%
\pgfpathcurveto{\pgfqpoint{0.627979in}{0.902020in}}{\pgfqpoint{0.633279in}{0.904215in}}{\pgfqpoint{0.637185in}{0.908122in}}%
\pgfpathcurveto{\pgfqpoint{0.641092in}{0.912029in}}{\pgfqpoint{0.643287in}{0.917328in}}{\pgfqpoint{0.643287in}{0.922854in}}%
\pgfpathcurveto{\pgfqpoint{0.643287in}{0.928379in}}{\pgfqpoint{0.641092in}{0.933678in}}{\pgfqpoint{0.637185in}{0.937585in}}%
\pgfpathcurveto{\pgfqpoint{0.633279in}{0.941492in}}{\pgfqpoint{0.627979in}{0.943687in}}{\pgfqpoint{0.622454in}{0.943687in}}%
\pgfpathcurveto{\pgfqpoint{0.616929in}{0.943687in}}{\pgfqpoint{0.611629in}{0.941492in}}{\pgfqpoint{0.607723in}{0.937585in}}%
\pgfpathcurveto{\pgfqpoint{0.603816in}{0.933678in}}{\pgfqpoint{0.601621in}{0.928379in}}{\pgfqpoint{0.601621in}{0.922854in}}%
\pgfpathcurveto{\pgfqpoint{0.601621in}{0.917328in}}{\pgfqpoint{0.603816in}{0.912029in}}{\pgfqpoint{0.607723in}{0.908122in}}%
\pgfpathcurveto{\pgfqpoint{0.611629in}{0.904215in}}{\pgfqpoint{0.616929in}{0.902020in}}{\pgfqpoint{0.622454in}{0.902020in}}%
\pgfpathclose%
\pgfusepath{stroke,fill}%
\end{pgfscope}%
\begin{pgfscope}%
\pgfpathrectangle{\pgfqpoint{0.403630in}{0.251348in}}{\pgfqpoint{1.395000in}{1.359000in}}%
\pgfusepath{clip}%
\pgfsetbuttcap%
\pgfsetroundjoin%
\definecolor{currentfill}{rgb}{0.686275,0.352941,0.313725}%
\pgfsetfillcolor{currentfill}%
\pgfsetlinewidth{0.501875pt}%
\definecolor{currentstroke}{rgb}{0.686275,0.352941,0.313725}%
\pgfsetstrokecolor{currentstroke}%
\pgfsetdash{}{0pt}%
\pgfpathmoveto{\pgfqpoint{1.464925in}{1.173820in}}%
\pgfpathcurveto{\pgfqpoint{1.470450in}{1.173820in}}{\pgfqpoint{1.475749in}{1.176015in}}{\pgfqpoint{1.479656in}{1.179922in}}%
\pgfpathcurveto{\pgfqpoint{1.483563in}{1.183829in}}{\pgfqpoint{1.485758in}{1.189128in}}{\pgfqpoint{1.485758in}{1.194654in}}%
\pgfpathcurveto{\pgfqpoint{1.485758in}{1.200179in}}{\pgfqpoint{1.483563in}{1.205478in}}{\pgfqpoint{1.479656in}{1.209385in}}%
\pgfpathcurveto{\pgfqpoint{1.475749in}{1.213292in}}{\pgfqpoint{1.470450in}{1.215487in}}{\pgfqpoint{1.464925in}{1.215487in}}%
\pgfpathcurveto{\pgfqpoint{1.459399in}{1.215487in}}{\pgfqpoint{1.454100in}{1.213292in}}{\pgfqpoint{1.450193in}{1.209385in}}%
\pgfpathcurveto{\pgfqpoint{1.446286in}{1.205478in}}{\pgfqpoint{1.444091in}{1.200179in}}{\pgfqpoint{1.444091in}{1.194654in}}%
\pgfpathcurveto{\pgfqpoint{1.444091in}{1.189128in}}{\pgfqpoint{1.446286in}{1.183829in}}{\pgfqpoint{1.450193in}{1.179922in}}%
\pgfpathcurveto{\pgfqpoint{1.454100in}{1.176015in}}{\pgfqpoint{1.459399in}{1.173820in}}{\pgfqpoint{1.464925in}{1.173820in}}%
\pgfpathclose%
\pgfusepath{stroke,fill}%
\end{pgfscope}%
\begin{pgfscope}%
\pgfpathrectangle{\pgfqpoint{0.403630in}{0.251348in}}{\pgfqpoint{1.395000in}{1.359000in}}%
\pgfusepath{clip}%
\pgfsetbuttcap%
\pgfsetroundjoin%
\definecolor{currentfill}{rgb}{0.686275,0.352941,0.313725}%
\pgfsetfillcolor{currentfill}%
\pgfsetlinewidth{0.501875pt}%
\definecolor{currentstroke}{rgb}{0.686275,0.352941,0.313725}%
\pgfsetstrokecolor{currentstroke}%
\pgfsetdash{}{0pt}%
\pgfpathmoveto{\pgfqpoint{1.382866in}{0.571597in}}%
\pgfpathcurveto{\pgfqpoint{1.388391in}{0.571597in}}{\pgfqpoint{1.393690in}{0.573792in}}{\pgfqpoint{1.397597in}{0.577699in}}%
\pgfpathcurveto{\pgfqpoint{1.401504in}{0.581605in}}{\pgfqpoint{1.403699in}{0.586905in}}{\pgfqpoint{1.403699in}{0.592430in}}%
\pgfpathcurveto{\pgfqpoint{1.403699in}{0.597955in}}{\pgfqpoint{1.401504in}{0.603255in}}{\pgfqpoint{1.397597in}{0.607161in}}%
\pgfpathcurveto{\pgfqpoint{1.393690in}{0.611068in}}{\pgfqpoint{1.388391in}{0.613263in}}{\pgfqpoint{1.382866in}{0.613263in}}%
\pgfpathcurveto{\pgfqpoint{1.377341in}{0.613263in}}{\pgfqpoint{1.372041in}{0.611068in}}{\pgfqpoint{1.368134in}{0.607161in}}%
\pgfpathcurveto{\pgfqpoint{1.364228in}{0.603255in}}{\pgfqpoint{1.362032in}{0.597955in}}{\pgfqpoint{1.362032in}{0.592430in}}%
\pgfpathcurveto{\pgfqpoint{1.362032in}{0.586905in}}{\pgfqpoint{1.364228in}{0.581605in}}{\pgfqpoint{1.368134in}{0.577699in}}%
\pgfpathcurveto{\pgfqpoint{1.372041in}{0.573792in}}{\pgfqpoint{1.377341in}{0.571597in}}{\pgfqpoint{1.382866in}{0.571597in}}%
\pgfpathclose%
\pgfusepath{stroke,fill}%
\end{pgfscope}%
\begin{pgfscope}%
\pgfpathrectangle{\pgfqpoint{0.403630in}{0.251348in}}{\pgfqpoint{1.395000in}{1.359000in}}%
\pgfusepath{clip}%
\pgfsetbuttcap%
\pgfsetroundjoin%
\definecolor{currentfill}{rgb}{0.686275,0.352941,0.313725}%
\pgfsetfillcolor{currentfill}%
\pgfsetlinewidth{0.501875pt}%
\definecolor{currentstroke}{rgb}{0.686275,0.352941,0.313725}%
\pgfsetstrokecolor{currentstroke}%
\pgfsetdash{}{0pt}%
\pgfpathmoveto{\pgfqpoint{0.731866in}{0.624891in}}%
\pgfpathcurveto{\pgfqpoint{0.737391in}{0.624891in}}{\pgfqpoint{0.742690in}{0.627086in}}{\pgfqpoint{0.746597in}{0.630993in}}%
\pgfpathcurveto{\pgfqpoint{0.750504in}{0.634900in}}{\pgfqpoint{0.752699in}{0.640199in}}{\pgfqpoint{0.752699in}{0.645724in}}%
\pgfpathcurveto{\pgfqpoint{0.752699in}{0.651249in}}{\pgfqpoint{0.750504in}{0.656549in}}{\pgfqpoint{0.746597in}{0.660456in}}%
\pgfpathcurveto{\pgfqpoint{0.742690in}{0.664362in}}{\pgfqpoint{0.737391in}{0.666557in}}{\pgfqpoint{0.731866in}{0.666557in}}%
\pgfpathcurveto{\pgfqpoint{0.726341in}{0.666557in}}{\pgfqpoint{0.721041in}{0.664362in}}{\pgfqpoint{0.717134in}{0.660456in}}%
\pgfpathcurveto{\pgfqpoint{0.713228in}{0.656549in}}{\pgfqpoint{0.711032in}{0.651249in}}{\pgfqpoint{0.711032in}{0.645724in}}%
\pgfpathcurveto{\pgfqpoint{0.711032in}{0.640199in}}{\pgfqpoint{0.713228in}{0.634900in}}{\pgfqpoint{0.717134in}{0.630993in}}%
\pgfpathcurveto{\pgfqpoint{0.721041in}{0.627086in}}{\pgfqpoint{0.726341in}{0.624891in}}{\pgfqpoint{0.731866in}{0.624891in}}%
\pgfpathclose%
\pgfusepath{stroke,fill}%
\end{pgfscope}%
\begin{pgfscope}%
\pgfpathrectangle{\pgfqpoint{0.403630in}{0.251348in}}{\pgfqpoint{1.395000in}{1.359000in}}%
\pgfusepath{clip}%
\pgfsetbuttcap%
\pgfsetroundjoin%
\definecolor{currentfill}{rgb}{0.686275,0.352941,0.313725}%
\pgfsetfillcolor{currentfill}%
\pgfsetlinewidth{0.501875pt}%
\definecolor{currentstroke}{rgb}{0.686275,0.352941,0.313725}%
\pgfsetstrokecolor{currentstroke}%
\pgfsetdash{}{0pt}%
\pgfpathmoveto{\pgfqpoint{1.021807in}{0.465008in}}%
\pgfpathcurveto{\pgfqpoint{1.027332in}{0.465008in}}{\pgfqpoint{1.032631in}{0.467204in}}{\pgfqpoint{1.036538in}{0.471110in}}%
\pgfpathcurveto{\pgfqpoint{1.040445in}{0.475017in}}{\pgfqpoint{1.042640in}{0.480317in}}{\pgfqpoint{1.042640in}{0.485842in}}%
\pgfpathcurveto{\pgfqpoint{1.042640in}{0.491367in}}{\pgfqpoint{1.040445in}{0.496666in}}{\pgfqpoint{1.036538in}{0.500573in}}%
\pgfpathcurveto{\pgfqpoint{1.032631in}{0.504480in}}{\pgfqpoint{1.027332in}{0.506675in}}{\pgfqpoint{1.021807in}{0.506675in}}%
\pgfpathcurveto{\pgfqpoint{1.016282in}{0.506675in}}{\pgfqpoint{1.010982in}{0.504480in}}{\pgfqpoint{1.007075in}{0.500573in}}%
\pgfpathcurveto{\pgfqpoint{1.003169in}{0.496666in}}{\pgfqpoint{1.000974in}{0.491367in}}{\pgfqpoint{1.000974in}{0.485842in}}%
\pgfpathcurveto{\pgfqpoint{1.000974in}{0.480317in}}{\pgfqpoint{1.003169in}{0.475017in}}{\pgfqpoint{1.007075in}{0.471110in}}%
\pgfpathcurveto{\pgfqpoint{1.010982in}{0.467204in}}{\pgfqpoint{1.016282in}{0.465008in}}{\pgfqpoint{1.021807in}{0.465008in}}%
\pgfpathclose%
\pgfusepath{stroke,fill}%
\end{pgfscope}%
\begin{pgfscope}%
\pgfpathrectangle{\pgfqpoint{0.403630in}{0.251348in}}{\pgfqpoint{1.395000in}{1.359000in}}%
\pgfusepath{clip}%
\pgfsetbuttcap%
\pgfsetroundjoin%
\definecolor{currentfill}{rgb}{0.686275,0.352941,0.313725}%
\pgfsetfillcolor{currentfill}%
\pgfsetlinewidth{0.501875pt}%
\definecolor{currentstroke}{rgb}{0.686275,0.352941,0.313725}%
\pgfsetstrokecolor{currentstroke}%
\pgfsetdash{}{0pt}%
\pgfpathmoveto{\pgfqpoint{1.426630in}{1.211126in}}%
\pgfpathcurveto{\pgfqpoint{1.432155in}{1.211126in}}{\pgfqpoint{1.437455in}{1.213321in}}{\pgfqpoint{1.441362in}{1.217228in}}%
\pgfpathcurveto{\pgfqpoint{1.445269in}{1.221135in}}{\pgfqpoint{1.447464in}{1.226434in}}{\pgfqpoint{1.447464in}{1.231959in}}%
\pgfpathcurveto{\pgfqpoint{1.447464in}{1.237484in}}{\pgfqpoint{1.445269in}{1.242784in}}{\pgfqpoint{1.441362in}{1.246691in}}%
\pgfpathcurveto{\pgfqpoint{1.437455in}{1.250598in}}{\pgfqpoint{1.432155in}{1.252793in}}{\pgfqpoint{1.426630in}{1.252793in}}%
\pgfpathcurveto{\pgfqpoint{1.421105in}{1.252793in}}{\pgfqpoint{1.415806in}{1.250598in}}{\pgfqpoint{1.411899in}{1.246691in}}%
\pgfpathcurveto{\pgfqpoint{1.407992in}{1.242784in}}{\pgfqpoint{1.405797in}{1.237484in}}{\pgfqpoint{1.405797in}{1.231959in}}%
\pgfpathcurveto{\pgfqpoint{1.405797in}{1.226434in}}{\pgfqpoint{1.407992in}{1.221135in}}{\pgfqpoint{1.411899in}{1.217228in}}%
\pgfpathcurveto{\pgfqpoint{1.415806in}{1.213321in}}{\pgfqpoint{1.421105in}{1.211126in}}{\pgfqpoint{1.426630in}{1.211126in}}%
\pgfpathclose%
\pgfusepath{stroke,fill}%
\end{pgfscope}%
\begin{pgfscope}%
\pgfpathrectangle{\pgfqpoint{0.403630in}{0.251348in}}{\pgfqpoint{1.395000in}{1.359000in}}%
\pgfusepath{clip}%
\pgfsetbuttcap%
\pgfsetroundjoin%
\definecolor{currentfill}{rgb}{0.686275,0.352941,0.313725}%
\pgfsetfillcolor{currentfill}%
\pgfsetlinewidth{0.501875pt}%
\definecolor{currentstroke}{rgb}{0.686275,0.352941,0.313725}%
\pgfsetstrokecolor{currentstroke}%
\pgfsetdash{}{0pt}%
\pgfpathmoveto{\pgfqpoint{0.775630in}{0.560938in}}%
\pgfpathcurveto{\pgfqpoint{0.781155in}{0.560938in}}{\pgfqpoint{0.786455in}{0.563133in}}{\pgfqpoint{0.790362in}{0.567040in}}%
\pgfpathcurveto{\pgfqpoint{0.794269in}{0.570947in}}{\pgfqpoint{0.796464in}{0.576246in}}{\pgfqpoint{0.796464in}{0.581771in}}%
\pgfpathcurveto{\pgfqpoint{0.796464in}{0.587296in}}{\pgfqpoint{0.794269in}{0.592596in}}{\pgfqpoint{0.790362in}{0.596503in}}%
\pgfpathcurveto{\pgfqpoint{0.786455in}{0.600409in}}{\pgfqpoint{0.781155in}{0.602605in}}{\pgfqpoint{0.775630in}{0.602605in}}%
\pgfpathcurveto{\pgfqpoint{0.770105in}{0.602605in}}{\pgfqpoint{0.764806in}{0.600409in}}{\pgfqpoint{0.760899in}{0.596503in}}%
\pgfpathcurveto{\pgfqpoint{0.756992in}{0.592596in}}{\pgfqpoint{0.754797in}{0.587296in}}{\pgfqpoint{0.754797in}{0.581771in}}%
\pgfpathcurveto{\pgfqpoint{0.754797in}{0.576246in}}{\pgfqpoint{0.756992in}{0.570947in}}{\pgfqpoint{0.760899in}{0.567040in}}%
\pgfpathcurveto{\pgfqpoint{0.764806in}{0.563133in}}{\pgfqpoint{0.770105in}{0.560938in}}{\pgfqpoint{0.775630in}{0.560938in}}%
\pgfpathclose%
\pgfusepath{stroke,fill}%
\end{pgfscope}%
\begin{pgfscope}%
\pgfpathrectangle{\pgfqpoint{0.403630in}{0.251348in}}{\pgfqpoint{1.395000in}{1.359000in}}%
\pgfusepath{clip}%
\pgfsetbuttcap%
\pgfsetroundjoin%
\definecolor{currentfill}{rgb}{0.686275,0.352941,0.313725}%
\pgfsetfillcolor{currentfill}%
\pgfsetlinewidth{0.501875pt}%
\definecolor{currentstroke}{rgb}{0.686275,0.352941,0.313725}%
\pgfsetstrokecolor{currentstroke}%
\pgfsetdash{}{0pt}%
\pgfpathmoveto{\pgfqpoint{1.246101in}{0.502314in}}%
\pgfpathcurveto{\pgfqpoint{1.251626in}{0.502314in}}{\pgfqpoint{1.256926in}{0.504509in}}{\pgfqpoint{1.260832in}{0.508416in}}%
\pgfpathcurveto{\pgfqpoint{1.264739in}{0.512323in}}{\pgfqpoint{1.266934in}{0.517623in}}{\pgfqpoint{1.266934in}{0.523148in}}%
\pgfpathcurveto{\pgfqpoint{1.266934in}{0.528673in}}{\pgfqpoint{1.264739in}{0.533972in}}{\pgfqpoint{1.260832in}{0.537879in}}%
\pgfpathcurveto{\pgfqpoint{1.256926in}{0.541786in}}{\pgfqpoint{1.251626in}{0.543981in}}{\pgfqpoint{1.246101in}{0.543981in}}%
\pgfpathcurveto{\pgfqpoint{1.240576in}{0.543981in}}{\pgfqpoint{1.235276in}{0.541786in}}{\pgfqpoint{1.231370in}{0.537879in}}%
\pgfpathcurveto{\pgfqpoint{1.227463in}{0.533972in}}{\pgfqpoint{1.225268in}{0.528673in}}{\pgfqpoint{1.225268in}{0.523148in}}%
\pgfpathcurveto{\pgfqpoint{1.225268in}{0.517623in}}{\pgfqpoint{1.227463in}{0.512323in}}{\pgfqpoint{1.231370in}{0.508416in}}%
\pgfpathcurveto{\pgfqpoint{1.235276in}{0.504509in}}{\pgfqpoint{1.240576in}{0.502314in}}{\pgfqpoint{1.246101in}{0.502314in}}%
\pgfpathclose%
\pgfusepath{stroke,fill}%
\end{pgfscope}%
\begin{pgfscope}%
\pgfpathrectangle{\pgfqpoint{0.403630in}{0.251348in}}{\pgfqpoint{1.395000in}{1.359000in}}%
\pgfusepath{clip}%
\pgfsetbuttcap%
\pgfsetroundjoin%
\definecolor{currentfill}{rgb}{0.686275,0.352941,0.313725}%
\pgfsetfillcolor{currentfill}%
\pgfsetlinewidth{0.501875pt}%
\definecolor{currentstroke}{rgb}{0.686275,0.352941,0.313725}%
\pgfsetstrokecolor{currentstroke}%
\pgfsetdash{}{0pt}%
\pgfpathmoveto{\pgfqpoint{0.600572in}{0.854055in}}%
\pgfpathcurveto{\pgfqpoint{0.606097in}{0.854055in}}{\pgfqpoint{0.611396in}{0.856251in}}{\pgfqpoint{0.615303in}{0.860157in}}%
\pgfpathcurveto{\pgfqpoint{0.619210in}{0.864064in}}{\pgfqpoint{0.621405in}{0.869364in}}{\pgfqpoint{0.621405in}{0.874889in}}%
\pgfpathcurveto{\pgfqpoint{0.621405in}{0.880414in}}{\pgfqpoint{0.619210in}{0.885713in}}{\pgfqpoint{0.615303in}{0.889620in}}%
\pgfpathcurveto{\pgfqpoint{0.611396in}{0.893527in}}{\pgfqpoint{0.606097in}{0.895722in}}{\pgfqpoint{0.600572in}{0.895722in}}%
\pgfpathcurveto{\pgfqpoint{0.595047in}{0.895722in}}{\pgfqpoint{0.589747in}{0.893527in}}{\pgfqpoint{0.585840in}{0.889620in}}%
\pgfpathcurveto{\pgfqpoint{0.581933in}{0.885713in}}{\pgfqpoint{0.579738in}{0.880414in}}{\pgfqpoint{0.579738in}{0.874889in}}%
\pgfpathcurveto{\pgfqpoint{0.579738in}{0.869364in}}{\pgfqpoint{0.581933in}{0.864064in}}{\pgfqpoint{0.585840in}{0.860157in}}%
\pgfpathcurveto{\pgfqpoint{0.589747in}{0.856251in}}{\pgfqpoint{0.595047in}{0.854055in}}{\pgfqpoint{0.600572in}{0.854055in}}%
\pgfpathclose%
\pgfusepath{stroke,fill}%
\end{pgfscope}%
\begin{pgfscope}%
\pgfpathrectangle{\pgfqpoint{0.403630in}{0.251348in}}{\pgfqpoint{1.395000in}{1.359000in}}%
\pgfusepath{clip}%
\pgfsetbuttcap%
\pgfsetroundjoin%
\definecolor{currentfill}{rgb}{0.686275,0.352941,0.313725}%
\pgfsetfillcolor{currentfill}%
\pgfsetlinewidth{0.501875pt}%
\definecolor{currentstroke}{rgb}{0.686275,0.352941,0.313725}%
\pgfsetstrokecolor{currentstroke}%
\pgfsetdash{}{0pt}%
\pgfpathmoveto{\pgfqpoint{1.027277in}{0.475667in}}%
\pgfpathcurveto{\pgfqpoint{1.032803in}{0.475667in}}{\pgfqpoint{1.038102in}{0.477862in}}{\pgfqpoint{1.042009in}{0.481769in}}%
\pgfpathcurveto{\pgfqpoint{1.045916in}{0.485676in}}{\pgfqpoint{1.048111in}{0.490976in}}{\pgfqpoint{1.048111in}{0.496501in}}%
\pgfpathcurveto{\pgfqpoint{1.048111in}{0.502026in}}{\pgfqpoint{1.045916in}{0.507325in}}{\pgfqpoint{1.042009in}{0.511232in}}%
\pgfpathcurveto{\pgfqpoint{1.038102in}{0.515139in}}{\pgfqpoint{1.032803in}{0.517334in}}{\pgfqpoint{1.027277in}{0.517334in}}%
\pgfpathcurveto{\pgfqpoint{1.021752in}{0.517334in}}{\pgfqpoint{1.016453in}{0.515139in}}{\pgfqpoint{1.012546in}{0.511232in}}%
\pgfpathcurveto{\pgfqpoint{1.008639in}{0.507325in}}{\pgfqpoint{1.006444in}{0.502026in}}{\pgfqpoint{1.006444in}{0.496501in}}%
\pgfpathcurveto{\pgfqpoint{1.006444in}{0.490976in}}{\pgfqpoint{1.008639in}{0.485676in}}{\pgfqpoint{1.012546in}{0.481769in}}%
\pgfpathcurveto{\pgfqpoint{1.016453in}{0.477862in}}{\pgfqpoint{1.021752in}{0.475667in}}{\pgfqpoint{1.027277in}{0.475667in}}%
\pgfpathclose%
\pgfusepath{stroke,fill}%
\end{pgfscope}%
\begin{pgfscope}%
\pgfpathrectangle{\pgfqpoint{0.403630in}{0.251348in}}{\pgfqpoint{1.395000in}{1.359000in}}%
\pgfusepath{clip}%
\pgfsetbuttcap%
\pgfsetroundjoin%
\definecolor{currentfill}{rgb}{0.686275,0.352941,0.313725}%
\pgfsetfillcolor{currentfill}%
\pgfsetlinewidth{0.501875pt}%
\definecolor{currentstroke}{rgb}{0.686275,0.352941,0.313725}%
\pgfsetstrokecolor{currentstroke}%
\pgfsetdash{}{0pt}%
\pgfpathmoveto{\pgfqpoint{0.912395in}{0.507644in}}%
\pgfpathcurveto{\pgfqpoint{0.917920in}{0.507644in}}{\pgfqpoint{0.923220in}{0.509839in}}{\pgfqpoint{0.927127in}{0.513746in}}%
\pgfpathcurveto{\pgfqpoint{0.931033in}{0.517652in}}{\pgfqpoint{0.933228in}{0.522952in}}{\pgfqpoint{0.933228in}{0.528477in}}%
\pgfpathcurveto{\pgfqpoint{0.933228in}{0.534002in}}{\pgfqpoint{0.931033in}{0.539302in}}{\pgfqpoint{0.927127in}{0.543208in}}%
\pgfpathcurveto{\pgfqpoint{0.923220in}{0.547115in}}{\pgfqpoint{0.917920in}{0.549310in}}{\pgfqpoint{0.912395in}{0.549310in}}%
\pgfpathcurveto{\pgfqpoint{0.906870in}{0.549310in}}{\pgfqpoint{0.901571in}{0.547115in}}{\pgfqpoint{0.897664in}{0.543208in}}%
\pgfpathcurveto{\pgfqpoint{0.893757in}{0.539302in}}{\pgfqpoint{0.891562in}{0.534002in}}{\pgfqpoint{0.891562in}{0.528477in}}%
\pgfpathcurveto{\pgfqpoint{0.891562in}{0.522952in}}{\pgfqpoint{0.893757in}{0.517652in}}{\pgfqpoint{0.897664in}{0.513746in}}%
\pgfpathcurveto{\pgfqpoint{0.901571in}{0.509839in}}{\pgfqpoint{0.906870in}{0.507644in}}{\pgfqpoint{0.912395in}{0.507644in}}%
\pgfpathclose%
\pgfusepath{stroke,fill}%
\end{pgfscope}%
\begin{pgfscope}%
\pgfpathrectangle{\pgfqpoint{0.403630in}{0.251348in}}{\pgfqpoint{1.395000in}{1.359000in}}%
\pgfusepath{clip}%
\pgfsetbuttcap%
\pgfsetroundjoin%
\definecolor{currentfill}{rgb}{0.686275,0.352941,0.313725}%
\pgfsetfillcolor{currentfill}%
\pgfsetlinewidth{0.501875pt}%
\definecolor{currentstroke}{rgb}{0.686275,0.352941,0.313725}%
\pgfsetstrokecolor{currentstroke}%
\pgfsetdash{}{0pt}%
\pgfpathmoveto{\pgfqpoint{1.508689in}{1.136514in}}%
\pgfpathcurveto{\pgfqpoint{1.514214in}{1.136514in}}{\pgfqpoint{1.519514in}{1.138709in}}{\pgfqpoint{1.523421in}{1.142616in}}%
\pgfpathcurveto{\pgfqpoint{1.527327in}{1.146523in}}{\pgfqpoint{1.529523in}{1.151823in}}{\pgfqpoint{1.529523in}{1.157348in}}%
\pgfpathcurveto{\pgfqpoint{1.529523in}{1.162873in}}{\pgfqpoint{1.527327in}{1.168172in}}{\pgfqpoint{1.523421in}{1.172079in}}%
\pgfpathcurveto{\pgfqpoint{1.519514in}{1.175986in}}{\pgfqpoint{1.514214in}{1.178181in}}{\pgfqpoint{1.508689in}{1.178181in}}%
\pgfpathcurveto{\pgfqpoint{1.503164in}{1.178181in}}{\pgfqpoint{1.497865in}{1.175986in}}{\pgfqpoint{1.493958in}{1.172079in}}%
\pgfpathcurveto{\pgfqpoint{1.490051in}{1.168172in}}{\pgfqpoint{1.487856in}{1.162873in}}{\pgfqpoint{1.487856in}{1.157348in}}%
\pgfpathcurveto{\pgfqpoint{1.487856in}{1.151823in}}{\pgfqpoint{1.490051in}{1.146523in}}{\pgfqpoint{1.493958in}{1.142616in}}%
\pgfpathcurveto{\pgfqpoint{1.497865in}{1.138709in}}{\pgfqpoint{1.503164in}{1.136514in}}{\pgfqpoint{1.508689in}{1.136514in}}%
\pgfpathclose%
\pgfusepath{stroke,fill}%
\end{pgfscope}%
\begin{pgfscope}%
\pgfpathrectangle{\pgfqpoint{0.403630in}{0.251348in}}{\pgfqpoint{1.395000in}{1.359000in}}%
\pgfusepath{clip}%
\pgfsetbuttcap%
\pgfsetroundjoin%
\definecolor{currentfill}{rgb}{0.686275,0.352941,0.313725}%
\pgfsetfillcolor{currentfill}%
\pgfsetlinewidth{0.501875pt}%
\definecolor{currentstroke}{rgb}{0.686275,0.352941,0.313725}%
\pgfsetstrokecolor{currentstroke}%
\pgfsetdash{}{0pt}%
\pgfpathmoveto{\pgfqpoint{1.142160in}{1.392326in}}%
\pgfpathcurveto{\pgfqpoint{1.147685in}{1.392326in}}{\pgfqpoint{1.152984in}{1.394521in}}{\pgfqpoint{1.156891in}{1.398428in}}%
\pgfpathcurveto{\pgfqpoint{1.160798in}{1.402335in}}{\pgfqpoint{1.162993in}{1.407634in}}{\pgfqpoint{1.162993in}{1.413159in}}%
\pgfpathcurveto{\pgfqpoint{1.162993in}{1.418684in}}{\pgfqpoint{1.160798in}{1.423984in}}{\pgfqpoint{1.156891in}{1.427891in}}%
\pgfpathcurveto{\pgfqpoint{1.152984in}{1.431798in}}{\pgfqpoint{1.147685in}{1.433993in}}{\pgfqpoint{1.142160in}{1.433993in}}%
\pgfpathcurveto{\pgfqpoint{1.136635in}{1.433993in}}{\pgfqpoint{1.131335in}{1.431798in}}{\pgfqpoint{1.127428in}{1.427891in}}%
\pgfpathcurveto{\pgfqpoint{1.123522in}{1.423984in}}{\pgfqpoint{1.121326in}{1.418684in}}{\pgfqpoint{1.121326in}{1.413159in}}%
\pgfpathcurveto{\pgfqpoint{1.121326in}{1.407634in}}{\pgfqpoint{1.123522in}{1.402335in}}{\pgfqpoint{1.127428in}{1.398428in}}%
\pgfpathcurveto{\pgfqpoint{1.131335in}{1.394521in}}{\pgfqpoint{1.136635in}{1.392326in}}{\pgfqpoint{1.142160in}{1.392326in}}%
\pgfpathclose%
\pgfusepath{stroke,fill}%
\end{pgfscope}%
\begin{pgfscope}%
\pgfpathrectangle{\pgfqpoint{0.403630in}{0.251348in}}{\pgfqpoint{1.395000in}{1.359000in}}%
\pgfusepath{clip}%
\pgfsetbuttcap%
\pgfsetroundjoin%
\definecolor{currentfill}{rgb}{0.686275,0.352941,0.313725}%
\pgfsetfillcolor{currentfill}%
\pgfsetlinewidth{0.501875pt}%
\definecolor{currentstroke}{rgb}{0.686275,0.352941,0.313725}%
\pgfsetstrokecolor{currentstroke}%
\pgfsetdash{}{0pt}%
\pgfpathmoveto{\pgfqpoint{1.344572in}{0.486326in}}%
\pgfpathcurveto{\pgfqpoint{1.350097in}{0.486326in}}{\pgfqpoint{1.355396in}{0.488521in}}{\pgfqpoint{1.359303in}{0.492428in}}%
\pgfpathcurveto{\pgfqpoint{1.363210in}{0.496335in}}{\pgfqpoint{1.365405in}{0.501634in}}{\pgfqpoint{1.365405in}{0.507159in}}%
\pgfpathcurveto{\pgfqpoint{1.365405in}{0.512684in}}{\pgfqpoint{1.363210in}{0.517984in}}{\pgfqpoint{1.359303in}{0.521891in}}%
\pgfpathcurveto{\pgfqpoint{1.355396in}{0.525798in}}{\pgfqpoint{1.350097in}{0.527993in}}{\pgfqpoint{1.344572in}{0.527993in}}%
\pgfpathcurveto{\pgfqpoint{1.339047in}{0.527993in}}{\pgfqpoint{1.333747in}{0.525798in}}{\pgfqpoint{1.329840in}{0.521891in}}%
\pgfpathcurveto{\pgfqpoint{1.325933in}{0.517984in}}{\pgfqpoint{1.323738in}{0.512684in}}{\pgfqpoint{1.323738in}{0.507159in}}%
\pgfpathcurveto{\pgfqpoint{1.323738in}{0.501634in}}{\pgfqpoint{1.325933in}{0.496335in}}{\pgfqpoint{1.329840in}{0.492428in}}%
\pgfpathcurveto{\pgfqpoint{1.333747in}{0.488521in}}{\pgfqpoint{1.339047in}{0.486326in}}{\pgfqpoint{1.344572in}{0.486326in}}%
\pgfpathclose%
\pgfusepath{stroke,fill}%
\end{pgfscope}%
\begin{pgfscope}%
\pgfpathrectangle{\pgfqpoint{0.403630in}{0.251348in}}{\pgfqpoint{1.395000in}{1.359000in}}%
\pgfusepath{clip}%
\pgfsetbuttcap%
\pgfsetroundjoin%
\definecolor{currentfill}{rgb}{0.686275,0.352941,0.313725}%
\pgfsetfillcolor{currentfill}%
\pgfsetlinewidth{0.501875pt}%
\definecolor{currentstroke}{rgb}{0.686275,0.352941,0.313725}%
\pgfsetstrokecolor{currentstroke}%
\pgfsetdash{}{0pt}%
\pgfpathmoveto{\pgfqpoint{0.775630in}{0.523632in}}%
\pgfpathcurveto{\pgfqpoint{0.781155in}{0.523632in}}{\pgfqpoint{0.786455in}{0.525827in}}{\pgfqpoint{0.790362in}{0.529734in}}%
\pgfpathcurveto{\pgfqpoint{0.794269in}{0.533641in}}{\pgfqpoint{0.796464in}{0.538940in}}{\pgfqpoint{0.796464in}{0.544465in}}%
\pgfpathcurveto{\pgfqpoint{0.796464in}{0.549990in}}{\pgfqpoint{0.794269in}{0.555290in}}{\pgfqpoint{0.790362in}{0.559197in}}%
\pgfpathcurveto{\pgfqpoint{0.786455in}{0.563103in}}{\pgfqpoint{0.781155in}{0.565299in}}{\pgfqpoint{0.775630in}{0.565299in}}%
\pgfpathcurveto{\pgfqpoint{0.770105in}{0.565299in}}{\pgfqpoint{0.764806in}{0.563103in}}{\pgfqpoint{0.760899in}{0.559197in}}%
\pgfpathcurveto{\pgfqpoint{0.756992in}{0.555290in}}{\pgfqpoint{0.754797in}{0.549990in}}{\pgfqpoint{0.754797in}{0.544465in}}%
\pgfpathcurveto{\pgfqpoint{0.754797in}{0.538940in}}{\pgfqpoint{0.756992in}{0.533641in}}{\pgfqpoint{0.760899in}{0.529734in}}%
\pgfpathcurveto{\pgfqpoint{0.764806in}{0.525827in}}{\pgfqpoint{0.770105in}{0.523632in}}{\pgfqpoint{0.775630in}{0.523632in}}%
\pgfpathclose%
\pgfusepath{stroke,fill}%
\end{pgfscope}%
\begin{pgfscope}%
\pgfpathrectangle{\pgfqpoint{0.403630in}{0.251348in}}{\pgfqpoint{1.395000in}{1.359000in}}%
\pgfusepath{clip}%
\pgfsetbuttcap%
\pgfsetroundjoin%
\definecolor{currentfill}{rgb}{0.686275,0.352941,0.313725}%
\pgfsetfillcolor{currentfill}%
\pgfsetlinewidth{0.501875pt}%
\definecolor{currentstroke}{rgb}{0.686275,0.352941,0.313725}%
\pgfsetstrokecolor{currentstroke}%
\pgfsetdash{}{0pt}%
\pgfpathmoveto{\pgfqpoint{0.956160in}{0.502314in}}%
\pgfpathcurveto{\pgfqpoint{0.961685in}{0.502314in}}{\pgfqpoint{0.966984in}{0.504509in}}{\pgfqpoint{0.970891in}{0.508416in}}%
\pgfpathcurveto{\pgfqpoint{0.974798in}{0.512323in}}{\pgfqpoint{0.976993in}{0.517623in}}{\pgfqpoint{0.976993in}{0.523148in}}%
\pgfpathcurveto{\pgfqpoint{0.976993in}{0.528673in}}{\pgfqpoint{0.974798in}{0.533972in}}{\pgfqpoint{0.970891in}{0.537879in}}%
\pgfpathcurveto{\pgfqpoint{0.966984in}{0.541786in}}{\pgfqpoint{0.961685in}{0.543981in}}{\pgfqpoint{0.956160in}{0.543981in}}%
\pgfpathcurveto{\pgfqpoint{0.950635in}{0.543981in}}{\pgfqpoint{0.945335in}{0.541786in}}{\pgfqpoint{0.941428in}{0.537879in}}%
\pgfpathcurveto{\pgfqpoint{0.937522in}{0.533972in}}{\pgfqpoint{0.935326in}{0.528673in}}{\pgfqpoint{0.935326in}{0.523148in}}%
\pgfpathcurveto{\pgfqpoint{0.935326in}{0.517623in}}{\pgfqpoint{0.937522in}{0.512323in}}{\pgfqpoint{0.941428in}{0.508416in}}%
\pgfpathcurveto{\pgfqpoint{0.945335in}{0.504509in}}{\pgfqpoint{0.950635in}{0.502314in}}{\pgfqpoint{0.956160in}{0.502314in}}%
\pgfpathclose%
\pgfusepath{stroke,fill}%
\end{pgfscope}%
\begin{pgfscope}%
\pgfpathrectangle{\pgfqpoint{0.403630in}{0.251348in}}{\pgfqpoint{1.395000in}{1.359000in}}%
\pgfusepath{clip}%
\pgfsetbuttcap%
\pgfsetroundjoin%
\definecolor{currentfill}{rgb}{0.686275,0.352941,0.313725}%
\pgfsetfillcolor{currentfill}%
\pgfsetlinewidth{0.501875pt}%
\definecolor{currentstroke}{rgb}{0.686275,0.352941,0.313725}%
\pgfsetstrokecolor{currentstroke}%
\pgfsetdash{}{0pt}%
\pgfpathmoveto{\pgfqpoint{0.693572in}{0.651538in}}%
\pgfpathcurveto{\pgfqpoint{0.699097in}{0.651538in}}{\pgfqpoint{0.704396in}{0.653733in}}{\pgfqpoint{0.708303in}{0.657640in}}%
\pgfpathcurveto{\pgfqpoint{0.712210in}{0.661547in}}{\pgfqpoint{0.714405in}{0.666846in}}{\pgfqpoint{0.714405in}{0.672371in}}%
\pgfpathcurveto{\pgfqpoint{0.714405in}{0.677896in}}{\pgfqpoint{0.712210in}{0.683196in}}{\pgfqpoint{0.708303in}{0.687103in}}%
\pgfpathcurveto{\pgfqpoint{0.704396in}{0.691009in}}{\pgfqpoint{0.699097in}{0.693205in}}{\pgfqpoint{0.693572in}{0.693205in}}%
\pgfpathcurveto{\pgfqpoint{0.688047in}{0.693205in}}{\pgfqpoint{0.682747in}{0.691009in}}{\pgfqpoint{0.678840in}{0.687103in}}%
\pgfpathcurveto{\pgfqpoint{0.674933in}{0.683196in}}{\pgfqpoint{0.672738in}{0.677896in}}{\pgfqpoint{0.672738in}{0.672371in}}%
\pgfpathcurveto{\pgfqpoint{0.672738in}{0.666846in}}{\pgfqpoint{0.674933in}{0.661547in}}{\pgfqpoint{0.678840in}{0.657640in}}%
\pgfpathcurveto{\pgfqpoint{0.682747in}{0.653733in}}{\pgfqpoint{0.688047in}{0.651538in}}{\pgfqpoint{0.693572in}{0.651538in}}%
\pgfpathclose%
\pgfusepath{stroke,fill}%
\end{pgfscope}%
\begin{pgfscope}%
\pgfpathrectangle{\pgfqpoint{0.403630in}{0.251348in}}{\pgfqpoint{1.395000in}{1.359000in}}%
\pgfusepath{clip}%
\pgfsetbuttcap%
\pgfsetroundjoin%
\definecolor{currentfill}{rgb}{0.686275,0.352941,0.313725}%
\pgfsetfillcolor{currentfill}%
\pgfsetlinewidth{0.501875pt}%
\definecolor{currentstroke}{rgb}{0.686275,0.352941,0.313725}%
\pgfsetstrokecolor{currentstroke}%
\pgfsetdash{}{0pt}%
\pgfpathmoveto{\pgfqpoint{1.563395in}{0.736808in}}%
\pgfpathcurveto{\pgfqpoint{1.568920in}{0.736808in}}{\pgfqpoint{1.574220in}{0.739004in}}{\pgfqpoint{1.578127in}{0.742910in}}%
\pgfpathcurveto{\pgfqpoint{1.582033in}{0.746817in}}{\pgfqpoint{1.584228in}{0.752117in}}{\pgfqpoint{1.584228in}{0.757642in}}%
\pgfpathcurveto{\pgfqpoint{1.584228in}{0.763167in}}{\pgfqpoint{1.582033in}{0.768466in}}{\pgfqpoint{1.578127in}{0.772373in}}%
\pgfpathcurveto{\pgfqpoint{1.574220in}{0.776280in}}{\pgfqpoint{1.568920in}{0.778475in}}{\pgfqpoint{1.563395in}{0.778475in}}%
\pgfpathcurveto{\pgfqpoint{1.557870in}{0.778475in}}{\pgfqpoint{1.552571in}{0.776280in}}{\pgfqpoint{1.548664in}{0.772373in}}%
\pgfpathcurveto{\pgfqpoint{1.544757in}{0.768466in}}{\pgfqpoint{1.542562in}{0.763167in}}{\pgfqpoint{1.542562in}{0.757642in}}%
\pgfpathcurveto{\pgfqpoint{1.542562in}{0.752117in}}{\pgfqpoint{1.544757in}{0.746817in}}{\pgfqpoint{1.548664in}{0.742910in}}%
\pgfpathcurveto{\pgfqpoint{1.552571in}{0.739004in}}{\pgfqpoint{1.557870in}{0.736808in}}{\pgfqpoint{1.563395in}{0.736808in}}%
\pgfpathclose%
\pgfusepath{stroke,fill}%
\end{pgfscope}%
\begin{pgfscope}%
\pgfpathrectangle{\pgfqpoint{0.403630in}{0.251348in}}{\pgfqpoint{1.395000in}{1.359000in}}%
\pgfusepath{clip}%
\pgfsetbuttcap%
\pgfsetroundjoin%
\definecolor{currentfill}{rgb}{0.686275,0.352941,0.313725}%
\pgfsetfillcolor{currentfill}%
\pgfsetlinewidth{0.501875pt}%
\definecolor{currentstroke}{rgb}{0.686275,0.352941,0.313725}%
\pgfsetstrokecolor{currentstroke}%
\pgfsetdash{}{0pt}%
\pgfpathmoveto{\pgfqpoint{1.514160in}{0.667526in}}%
\pgfpathcurveto{\pgfqpoint{1.519685in}{0.667526in}}{\pgfqpoint{1.524984in}{0.669721in}}{\pgfqpoint{1.528891in}{0.673628in}}%
\pgfpathcurveto{\pgfqpoint{1.532798in}{0.677535in}}{\pgfqpoint{1.534993in}{0.682834in}}{\pgfqpoint{1.534993in}{0.688359in}}%
\pgfpathcurveto{\pgfqpoint{1.534993in}{0.693884in}}{\pgfqpoint{1.532798in}{0.699184in}}{\pgfqpoint{1.528891in}{0.703091in}}%
\pgfpathcurveto{\pgfqpoint{1.524984in}{0.706998in}}{\pgfqpoint{1.519685in}{0.709193in}}{\pgfqpoint{1.514160in}{0.709193in}}%
\pgfpathcurveto{\pgfqpoint{1.508635in}{0.709193in}}{\pgfqpoint{1.503335in}{0.706998in}}{\pgfqpoint{1.499428in}{0.703091in}}%
\pgfpathcurveto{\pgfqpoint{1.495522in}{0.699184in}}{\pgfqpoint{1.493326in}{0.693884in}}{\pgfqpoint{1.493326in}{0.688359in}}%
\pgfpathcurveto{\pgfqpoint{1.493326in}{0.682834in}}{\pgfqpoint{1.495522in}{0.677535in}}{\pgfqpoint{1.499428in}{0.673628in}}%
\pgfpathcurveto{\pgfqpoint{1.503335in}{0.669721in}}{\pgfqpoint{1.508635in}{0.667526in}}{\pgfqpoint{1.514160in}{0.667526in}}%
\pgfpathclose%
\pgfusepath{stroke,fill}%
\end{pgfscope}%
\begin{pgfscope}%
\pgfpathrectangle{\pgfqpoint{0.403630in}{0.251348in}}{\pgfqpoint{1.395000in}{1.359000in}}%
\pgfusepath{clip}%
\pgfsetbuttcap%
\pgfsetroundjoin%
\definecolor{currentfill}{rgb}{0.686275,0.352941,0.313725}%
\pgfsetfillcolor{currentfill}%
\pgfsetlinewidth{0.501875pt}%
\definecolor{currentstroke}{rgb}{0.686275,0.352941,0.313725}%
\pgfsetstrokecolor{currentstroke}%
\pgfsetdash{}{0pt}%
\pgfpathmoveto{\pgfqpoint{0.682630in}{0.688844in}}%
\pgfpathcurveto{\pgfqpoint{0.688155in}{0.688844in}}{\pgfqpoint{0.693455in}{0.691039in}}{\pgfqpoint{0.697362in}{0.694946in}}%
\pgfpathcurveto{\pgfqpoint{0.701269in}{0.698852in}}{\pgfqpoint{0.703464in}{0.704152in}}{\pgfqpoint{0.703464in}{0.709677in}}%
\pgfpathcurveto{\pgfqpoint{0.703464in}{0.715202in}}{\pgfqpoint{0.701269in}{0.720502in}}{\pgfqpoint{0.697362in}{0.724408in}}%
\pgfpathcurveto{\pgfqpoint{0.693455in}{0.728315in}}{\pgfqpoint{0.688155in}{0.730510in}}{\pgfqpoint{0.682630in}{0.730510in}}%
\pgfpathcurveto{\pgfqpoint{0.677105in}{0.730510in}}{\pgfqpoint{0.671806in}{0.728315in}}{\pgfqpoint{0.667899in}{0.724408in}}%
\pgfpathcurveto{\pgfqpoint{0.663992in}{0.720502in}}{\pgfqpoint{0.661797in}{0.715202in}}{\pgfqpoint{0.661797in}{0.709677in}}%
\pgfpathcurveto{\pgfqpoint{0.661797in}{0.704152in}}{\pgfqpoint{0.663992in}{0.698852in}}{\pgfqpoint{0.667899in}{0.694946in}}%
\pgfpathcurveto{\pgfqpoint{0.671806in}{0.691039in}}{\pgfqpoint{0.677105in}{0.688844in}}{\pgfqpoint{0.682630in}{0.688844in}}%
\pgfpathclose%
\pgfusepath{stroke,fill}%
\end{pgfscope}%
\begin{pgfscope}%
\pgfpathrectangle{\pgfqpoint{0.403630in}{0.251348in}}{\pgfqpoint{1.395000in}{1.359000in}}%
\pgfusepath{clip}%
\pgfsetbuttcap%
\pgfsetroundjoin%
\definecolor{currentfill}{rgb}{0.686275,0.352941,0.313725}%
\pgfsetfillcolor{currentfill}%
\pgfsetlinewidth{0.501875pt}%
\definecolor{currentstroke}{rgb}{0.686275,0.352941,0.313725}%
\pgfsetstrokecolor{currentstroke}%
\pgfsetdash{}{0pt}%
\pgfpathmoveto{\pgfqpoint{0.715454in}{1.163161in}}%
\pgfpathcurveto{\pgfqpoint{0.720979in}{1.163161in}}{\pgfqpoint{0.726279in}{1.165356in}}{\pgfqpoint{0.730185in}{1.169263in}}%
\pgfpathcurveto{\pgfqpoint{0.734092in}{1.173170in}}{\pgfqpoint{0.736287in}{1.178470in}}{\pgfqpoint{0.736287in}{1.183995in}}%
\pgfpathcurveto{\pgfqpoint{0.736287in}{1.189520in}}{\pgfqpoint{0.734092in}{1.194819in}}{\pgfqpoint{0.730185in}{1.198726in}}%
\pgfpathcurveto{\pgfqpoint{0.726279in}{1.202633in}}{\pgfqpoint{0.720979in}{1.204828in}}{\pgfqpoint{0.715454in}{1.204828in}}%
\pgfpathcurveto{\pgfqpoint{0.709929in}{1.204828in}}{\pgfqpoint{0.704629in}{1.202633in}}{\pgfqpoint{0.700723in}{1.198726in}}%
\pgfpathcurveto{\pgfqpoint{0.696816in}{1.194819in}}{\pgfqpoint{0.694621in}{1.189520in}}{\pgfqpoint{0.694621in}{1.183995in}}%
\pgfpathcurveto{\pgfqpoint{0.694621in}{1.178470in}}{\pgfqpoint{0.696816in}{1.173170in}}{\pgfqpoint{0.700723in}{1.169263in}}%
\pgfpathcurveto{\pgfqpoint{0.704629in}{1.165356in}}{\pgfqpoint{0.709929in}{1.163161in}}{\pgfqpoint{0.715454in}{1.163161in}}%
\pgfpathclose%
\pgfusepath{stroke,fill}%
\end{pgfscope}%
\begin{pgfscope}%
\pgfpathrectangle{\pgfqpoint{0.403630in}{0.251348in}}{\pgfqpoint{1.395000in}{1.359000in}}%
\pgfusepath{clip}%
\pgfsetbuttcap%
\pgfsetroundjoin%
\definecolor{currentfill}{rgb}{0.686275,0.352941,0.313725}%
\pgfsetfillcolor{currentfill}%
\pgfsetlinewidth{0.501875pt}%
\definecolor{currentstroke}{rgb}{0.686275,0.352941,0.313725}%
\pgfsetstrokecolor{currentstroke}%
\pgfsetdash{}{0pt}%
\pgfpathmoveto{\pgfqpoint{0.638866in}{1.051244in}}%
\pgfpathcurveto{\pgfqpoint{0.644391in}{1.051244in}}{\pgfqpoint{0.649690in}{1.053439in}}{\pgfqpoint{0.653597in}{1.057346in}}%
\pgfpathcurveto{\pgfqpoint{0.657504in}{1.061252in}}{\pgfqpoint{0.659699in}{1.066552in}}{\pgfqpoint{0.659699in}{1.072077in}}%
\pgfpathcurveto{\pgfqpoint{0.659699in}{1.077602in}}{\pgfqpoint{0.657504in}{1.082902in}}{\pgfqpoint{0.653597in}{1.086808in}}%
\pgfpathcurveto{\pgfqpoint{0.649690in}{1.090715in}}{\pgfqpoint{0.644391in}{1.092910in}}{\pgfqpoint{0.638866in}{1.092910in}}%
\pgfpathcurveto{\pgfqpoint{0.633341in}{1.092910in}}{\pgfqpoint{0.628041in}{1.090715in}}{\pgfqpoint{0.624134in}{1.086808in}}%
\pgfpathcurveto{\pgfqpoint{0.620228in}{1.082902in}}{\pgfqpoint{0.618032in}{1.077602in}}{\pgfqpoint{0.618032in}{1.072077in}}%
\pgfpathcurveto{\pgfqpoint{0.618032in}{1.066552in}}{\pgfqpoint{0.620228in}{1.061252in}}{\pgfqpoint{0.624134in}{1.057346in}}%
\pgfpathcurveto{\pgfqpoint{0.628041in}{1.053439in}}{\pgfqpoint{0.633341in}{1.051244in}}{\pgfqpoint{0.638866in}{1.051244in}}%
\pgfpathclose%
\pgfusepath{stroke,fill}%
\end{pgfscope}%
\begin{pgfscope}%
\pgfpathrectangle{\pgfqpoint{0.403630in}{0.251348in}}{\pgfqpoint{1.395000in}{1.359000in}}%
\pgfusepath{clip}%
\pgfsetbuttcap%
\pgfsetroundjoin%
\definecolor{currentfill}{rgb}{0.686275,0.352941,0.313725}%
\pgfsetfillcolor{currentfill}%
\pgfsetlinewidth{0.501875pt}%
\definecolor{currentstroke}{rgb}{0.686275,0.352941,0.313725}%
\pgfsetstrokecolor{currentstroke}%
\pgfsetdash{}{0pt}%
\pgfpathmoveto{\pgfqpoint{1.536042in}{0.694173in}}%
\pgfpathcurveto{\pgfqpoint{1.541567in}{0.694173in}}{\pgfqpoint{1.546867in}{0.696368in}}{\pgfqpoint{1.550774in}{0.700275in}}%
\pgfpathcurveto{\pgfqpoint{1.554680in}{0.704182in}}{\pgfqpoint{1.556876in}{0.709481in}}{\pgfqpoint{1.556876in}{0.715006in}}%
\pgfpathcurveto{\pgfqpoint{1.556876in}{0.720532in}}{\pgfqpoint{1.554680in}{0.725831in}}{\pgfqpoint{1.550774in}{0.729738in}}%
\pgfpathcurveto{\pgfqpoint{1.546867in}{0.733645in}}{\pgfqpoint{1.541567in}{0.735840in}}{\pgfqpoint{1.536042in}{0.735840in}}%
\pgfpathcurveto{\pgfqpoint{1.530517in}{0.735840in}}{\pgfqpoint{1.525218in}{0.733645in}}{\pgfqpoint{1.521311in}{0.729738in}}%
\pgfpathcurveto{\pgfqpoint{1.517404in}{0.725831in}}{\pgfqpoint{1.515209in}{0.720532in}}{\pgfqpoint{1.515209in}{0.715006in}}%
\pgfpathcurveto{\pgfqpoint{1.515209in}{0.709481in}}{\pgfqpoint{1.517404in}{0.704182in}}{\pgfqpoint{1.521311in}{0.700275in}}%
\pgfpathcurveto{\pgfqpoint{1.525218in}{0.696368in}}{\pgfqpoint{1.530517in}{0.694173in}}{\pgfqpoint{1.536042in}{0.694173in}}%
\pgfpathclose%
\pgfusepath{stroke,fill}%
\end{pgfscope}%
\begin{pgfscope}%
\pgfpathrectangle{\pgfqpoint{0.403630in}{0.251348in}}{\pgfqpoint{1.395000in}{1.359000in}}%
\pgfusepath{clip}%
\pgfsetbuttcap%
\pgfsetroundjoin%
\definecolor{currentfill}{rgb}{0.686275,0.352941,0.313725}%
\pgfsetfillcolor{currentfill}%
\pgfsetlinewidth{0.501875pt}%
\definecolor{currentstroke}{rgb}{0.686275,0.352941,0.313725}%
\pgfsetstrokecolor{currentstroke}%
\pgfsetdash{}{0pt}%
\pgfpathmoveto{\pgfqpoint{0.622454in}{1.024597in}}%
\pgfpathcurveto{\pgfqpoint{0.627979in}{1.024597in}}{\pgfqpoint{0.633279in}{1.026792in}}{\pgfqpoint{0.637185in}{1.030699in}}%
\pgfpathcurveto{\pgfqpoint{0.641092in}{1.034605in}}{\pgfqpoint{0.643287in}{1.039905in}}{\pgfqpoint{0.643287in}{1.045430in}}%
\pgfpathcurveto{\pgfqpoint{0.643287in}{1.050955in}}{\pgfqpoint{0.641092in}{1.056255in}}{\pgfqpoint{0.637185in}{1.060161in}}%
\pgfpathcurveto{\pgfqpoint{0.633279in}{1.064068in}}{\pgfqpoint{0.627979in}{1.066263in}}{\pgfqpoint{0.622454in}{1.066263in}}%
\pgfpathcurveto{\pgfqpoint{0.616929in}{1.066263in}}{\pgfqpoint{0.611629in}{1.064068in}}{\pgfqpoint{0.607723in}{1.060161in}}%
\pgfpathcurveto{\pgfqpoint{0.603816in}{1.056255in}}{\pgfqpoint{0.601621in}{1.050955in}}{\pgfqpoint{0.601621in}{1.045430in}}%
\pgfpathcurveto{\pgfqpoint{0.601621in}{1.039905in}}{\pgfqpoint{0.603816in}{1.034605in}}{\pgfqpoint{0.607723in}{1.030699in}}%
\pgfpathcurveto{\pgfqpoint{0.611629in}{1.026792in}}{\pgfqpoint{0.616929in}{1.024597in}}{\pgfqpoint{0.622454in}{1.024597in}}%
\pgfpathclose%
\pgfusepath{stroke,fill}%
\end{pgfscope}%
\begin{pgfscope}%
\pgfpathrectangle{\pgfqpoint{0.403630in}{0.251348in}}{\pgfqpoint{1.395000in}{1.359000in}}%
\pgfusepath{clip}%
\pgfsetbuttcap%
\pgfsetroundjoin%
\definecolor{currentfill}{rgb}{0.686275,0.352941,0.313725}%
\pgfsetfillcolor{currentfill}%
\pgfsetlinewidth{0.501875pt}%
\definecolor{currentstroke}{rgb}{0.686275,0.352941,0.313725}%
\pgfsetstrokecolor{currentstroke}%
\pgfsetdash{}{0pt}%
\pgfpathmoveto{\pgfqpoint{0.611513in}{1.008608in}}%
\pgfpathcurveto{\pgfqpoint{0.617038in}{1.008608in}}{\pgfqpoint{0.622337in}{1.010804in}}{\pgfqpoint{0.626244in}{1.014710in}}%
\pgfpathcurveto{\pgfqpoint{0.630151in}{1.018617in}}{\pgfqpoint{0.632346in}{1.023917in}}{\pgfqpoint{0.632346in}{1.029442in}}%
\pgfpathcurveto{\pgfqpoint{0.632346in}{1.034967in}}{\pgfqpoint{0.630151in}{1.040266in}}{\pgfqpoint{0.626244in}{1.044173in}}%
\pgfpathcurveto{\pgfqpoint{0.622337in}{1.048080in}}{\pgfqpoint{0.617038in}{1.050275in}}{\pgfqpoint{0.611513in}{1.050275in}}%
\pgfpathcurveto{\pgfqpoint{0.605988in}{1.050275in}}{\pgfqpoint{0.600688in}{1.048080in}}{\pgfqpoint{0.596781in}{1.044173in}}%
\pgfpathcurveto{\pgfqpoint{0.592875in}{1.040266in}}{\pgfqpoint{0.590679in}{1.034967in}}{\pgfqpoint{0.590679in}{1.029442in}}%
\pgfpathcurveto{\pgfqpoint{0.590679in}{1.023917in}}{\pgfqpoint{0.592875in}{1.018617in}}{\pgfqpoint{0.596781in}{1.014710in}}%
\pgfpathcurveto{\pgfqpoint{0.600688in}{1.010804in}}{\pgfqpoint{0.605988in}{1.008608in}}{\pgfqpoint{0.611513in}{1.008608in}}%
\pgfpathclose%
\pgfusepath{stroke,fill}%
\end{pgfscope}%
\begin{pgfscope}%
\pgfpathrectangle{\pgfqpoint{0.403630in}{0.251348in}}{\pgfqpoint{1.395000in}{1.359000in}}%
\pgfusepath{clip}%
\pgfsetbuttcap%
\pgfsetroundjoin%
\definecolor{currentfill}{rgb}{0.686275,0.352941,0.313725}%
\pgfsetfillcolor{currentfill}%
\pgfsetlinewidth{0.501875pt}%
\definecolor{currentstroke}{rgb}{0.686275,0.352941,0.313725}%
\pgfsetstrokecolor{currentstroke}%
\pgfsetdash{}{0pt}%
\pgfpathmoveto{\pgfqpoint{1.251572in}{0.496985in}}%
\pgfpathcurveto{\pgfqpoint{1.257097in}{0.496985in}}{\pgfqpoint{1.262396in}{0.499180in}}{\pgfqpoint{1.266303in}{0.503087in}}%
\pgfpathcurveto{\pgfqpoint{1.270210in}{0.506994in}}{\pgfqpoint{1.272405in}{0.512293in}}{\pgfqpoint{1.272405in}{0.517818in}}%
\pgfpathcurveto{\pgfqpoint{1.272405in}{0.523343in}}{\pgfqpoint{1.270210in}{0.528643in}}{\pgfqpoint{1.266303in}{0.532550in}}%
\pgfpathcurveto{\pgfqpoint{1.262396in}{0.536456in}}{\pgfqpoint{1.257097in}{0.538652in}}{\pgfqpoint{1.251572in}{0.538652in}}%
\pgfpathcurveto{\pgfqpoint{1.246047in}{0.538652in}}{\pgfqpoint{1.240747in}{0.536456in}}{\pgfqpoint{1.236840in}{0.532550in}}%
\pgfpathcurveto{\pgfqpoint{1.232933in}{0.528643in}}{\pgfqpoint{1.230738in}{0.523343in}}{\pgfqpoint{1.230738in}{0.517818in}}%
\pgfpathcurveto{\pgfqpoint{1.230738in}{0.512293in}}{\pgfqpoint{1.232933in}{0.506994in}}{\pgfqpoint{1.236840in}{0.503087in}}%
\pgfpathcurveto{\pgfqpoint{1.240747in}{0.499180in}}{\pgfqpoint{1.246047in}{0.496985in}}{\pgfqpoint{1.251572in}{0.496985in}}%
\pgfpathclose%
\pgfusepath{stroke,fill}%
\end{pgfscope}%
\begin{pgfscope}%
\pgfpathrectangle{\pgfqpoint{0.403630in}{0.251348in}}{\pgfqpoint{1.395000in}{1.359000in}}%
\pgfusepath{clip}%
\pgfsetbuttcap%
\pgfsetroundjoin%
\definecolor{currentfill}{rgb}{0.686275,0.352941,0.313725}%
\pgfsetfillcolor{currentfill}%
\pgfsetlinewidth{0.501875pt}%
\definecolor{currentstroke}{rgb}{0.686275,0.352941,0.313725}%
\pgfsetstrokecolor{currentstroke}%
\pgfsetdash{}{0pt}%
\pgfpathmoveto{\pgfqpoint{0.802983in}{1.227114in}}%
\pgfpathcurveto{\pgfqpoint{0.808508in}{1.227114in}}{\pgfqpoint{0.813808in}{1.229309in}}{\pgfqpoint{0.817715in}{1.233216in}}%
\pgfpathcurveto{\pgfqpoint{0.821622in}{1.237123in}}{\pgfqpoint{0.823817in}{1.242423in}}{\pgfqpoint{0.823817in}{1.247948in}}%
\pgfpathcurveto{\pgfqpoint{0.823817in}{1.253473in}}{\pgfqpoint{0.821622in}{1.258772in}}{\pgfqpoint{0.817715in}{1.262679in}}%
\pgfpathcurveto{\pgfqpoint{0.813808in}{1.266586in}}{\pgfqpoint{0.808508in}{1.268781in}}{\pgfqpoint{0.802983in}{1.268781in}}%
\pgfpathcurveto{\pgfqpoint{0.797458in}{1.268781in}}{\pgfqpoint{0.792159in}{1.266586in}}{\pgfqpoint{0.788252in}{1.262679in}}%
\pgfpathcurveto{\pgfqpoint{0.784345in}{1.258772in}}{\pgfqpoint{0.782150in}{1.253473in}}{\pgfqpoint{0.782150in}{1.247948in}}%
\pgfpathcurveto{\pgfqpoint{0.782150in}{1.242423in}}{\pgfqpoint{0.784345in}{1.237123in}}{\pgfqpoint{0.788252in}{1.233216in}}%
\pgfpathcurveto{\pgfqpoint{0.792159in}{1.229309in}}{\pgfqpoint{0.797458in}{1.227114in}}{\pgfqpoint{0.802983in}{1.227114in}}%
\pgfpathclose%
\pgfusepath{stroke,fill}%
\end{pgfscope}%
\begin{pgfscope}%
\pgfpathrectangle{\pgfqpoint{0.403630in}{0.251348in}}{\pgfqpoint{1.395000in}{1.359000in}}%
\pgfusepath{clip}%
\pgfsetbuttcap%
\pgfsetroundjoin%
\definecolor{currentfill}{rgb}{0.686275,0.352941,0.313725}%
\pgfsetfillcolor{currentfill}%
\pgfsetlinewidth{0.501875pt}%
\definecolor{currentstroke}{rgb}{0.686275,0.352941,0.313725}%
\pgfsetstrokecolor{currentstroke}%
\pgfsetdash{}{0pt}%
\pgfpathmoveto{\pgfqpoint{0.660748in}{1.083220in}}%
\pgfpathcurveto{\pgfqpoint{0.666273in}{1.083220in}}{\pgfqpoint{0.671573in}{1.085415in}}{\pgfqpoint{0.675479in}{1.089322in}}%
\pgfpathcurveto{\pgfqpoint{0.679386in}{1.093229in}}{\pgfqpoint{0.681581in}{1.098528in}}{\pgfqpoint{0.681581in}{1.104054in}}%
\pgfpathcurveto{\pgfqpoint{0.681581in}{1.109579in}}{\pgfqpoint{0.679386in}{1.114878in}}{\pgfqpoint{0.675479in}{1.118785in}}%
\pgfpathcurveto{\pgfqpoint{0.671573in}{1.122692in}}{\pgfqpoint{0.666273in}{1.124887in}}{\pgfqpoint{0.660748in}{1.124887in}}%
\pgfpathcurveto{\pgfqpoint{0.655223in}{1.124887in}}{\pgfqpoint{0.649923in}{1.122692in}}{\pgfqpoint{0.646017in}{1.118785in}}%
\pgfpathcurveto{\pgfqpoint{0.642110in}{1.114878in}}{\pgfqpoint{0.639915in}{1.109579in}}{\pgfqpoint{0.639915in}{1.104054in}}%
\pgfpathcurveto{\pgfqpoint{0.639915in}{1.098528in}}{\pgfqpoint{0.642110in}{1.093229in}}{\pgfqpoint{0.646017in}{1.089322in}}%
\pgfpathcurveto{\pgfqpoint{0.649923in}{1.085415in}}{\pgfqpoint{0.655223in}{1.083220in}}{\pgfqpoint{0.660748in}{1.083220in}}%
\pgfpathclose%
\pgfusepath{stroke,fill}%
\end{pgfscope}%
\begin{pgfscope}%
\pgfpathrectangle{\pgfqpoint{0.403630in}{0.251348in}}{\pgfqpoint{1.395000in}{1.359000in}}%
\pgfusepath{clip}%
\pgfsetbuttcap%
\pgfsetroundjoin%
\definecolor{currentfill}{rgb}{0.686275,0.352941,0.313725}%
\pgfsetfillcolor{currentfill}%
\pgfsetlinewidth{0.501875pt}%
\definecolor{currentstroke}{rgb}{0.686275,0.352941,0.313725}%
\pgfsetstrokecolor{currentstroke}%
\pgfsetdash{}{0pt}%
\pgfpathmoveto{\pgfqpoint{1.404748in}{1.301726in}}%
\pgfpathcurveto{\pgfqpoint{1.410273in}{1.301726in}}{\pgfqpoint{1.415573in}{1.303921in}}{\pgfqpoint{1.419479in}{1.307828in}}%
\pgfpathcurveto{\pgfqpoint{1.423386in}{1.311735in}}{\pgfqpoint{1.425581in}{1.317034in}}{\pgfqpoint{1.425581in}{1.322559in}}%
\pgfpathcurveto{\pgfqpoint{1.425581in}{1.328084in}}{\pgfqpoint{1.423386in}{1.333384in}}{\pgfqpoint{1.419479in}{1.337291in}}%
\pgfpathcurveto{\pgfqpoint{1.415573in}{1.341198in}}{\pgfqpoint{1.410273in}{1.343393in}}{\pgfqpoint{1.404748in}{1.343393in}}%
\pgfpathcurveto{\pgfqpoint{1.399223in}{1.343393in}}{\pgfqpoint{1.393923in}{1.341198in}}{\pgfqpoint{1.390017in}{1.337291in}}%
\pgfpathcurveto{\pgfqpoint{1.386110in}{1.333384in}}{\pgfqpoint{1.383915in}{1.328084in}}{\pgfqpoint{1.383915in}{1.322559in}}%
\pgfpathcurveto{\pgfqpoint{1.383915in}{1.317034in}}{\pgfqpoint{1.386110in}{1.311735in}}{\pgfqpoint{1.390017in}{1.307828in}}%
\pgfpathcurveto{\pgfqpoint{1.393923in}{1.303921in}}{\pgfqpoint{1.399223in}{1.301726in}}{\pgfqpoint{1.404748in}{1.301726in}}%
\pgfpathclose%
\pgfusepath{stroke,fill}%
\end{pgfscope}%
\begin{pgfscope}%
\pgfpathrectangle{\pgfqpoint{0.403630in}{0.251348in}}{\pgfqpoint{1.395000in}{1.359000in}}%
\pgfusepath{clip}%
\pgfsetbuttcap%
\pgfsetroundjoin%
\definecolor{currentfill}{rgb}{0.686275,0.352941,0.313725}%
\pgfsetfillcolor{currentfill}%
\pgfsetlinewidth{0.501875pt}%
\definecolor{currentstroke}{rgb}{0.686275,0.352941,0.313725}%
\pgfsetstrokecolor{currentstroke}%
\pgfsetdash{}{0pt}%
\pgfpathmoveto{\pgfqpoint{1.410219in}{0.598244in}}%
\pgfpathcurveto{\pgfqpoint{1.415744in}{0.598244in}}{\pgfqpoint{1.421043in}{0.600439in}}{\pgfqpoint{1.424950in}{0.604346in}}%
\pgfpathcurveto{\pgfqpoint{1.428857in}{0.608252in}}{\pgfqpoint{1.431052in}{0.613552in}}{\pgfqpoint{1.431052in}{0.619077in}}%
\pgfpathcurveto{\pgfqpoint{1.431052in}{0.624602in}}{\pgfqpoint{1.428857in}{0.629902in}}{\pgfqpoint{1.424950in}{0.633808in}}%
\pgfpathcurveto{\pgfqpoint{1.421043in}{0.637715in}}{\pgfqpoint{1.415744in}{0.639910in}}{\pgfqpoint{1.410219in}{0.639910in}}%
\pgfpathcurveto{\pgfqpoint{1.404694in}{0.639910in}}{\pgfqpoint{1.399394in}{0.637715in}}{\pgfqpoint{1.395487in}{0.633808in}}%
\pgfpathcurveto{\pgfqpoint{1.391580in}{0.629902in}}{\pgfqpoint{1.389385in}{0.624602in}}{\pgfqpoint{1.389385in}{0.619077in}}%
\pgfpathcurveto{\pgfqpoint{1.389385in}{0.613552in}}{\pgfqpoint{1.391580in}{0.608252in}}{\pgfqpoint{1.395487in}{0.604346in}}%
\pgfpathcurveto{\pgfqpoint{1.399394in}{0.600439in}}{\pgfqpoint{1.404694in}{0.598244in}}{\pgfqpoint{1.410219in}{0.598244in}}%
\pgfpathclose%
\pgfusepath{stroke,fill}%
\end{pgfscope}%
\begin{pgfscope}%
\pgfpathrectangle{\pgfqpoint{0.403630in}{0.251348in}}{\pgfqpoint{1.395000in}{1.359000in}}%
\pgfusepath{clip}%
\pgfsetbuttcap%
\pgfsetroundjoin%
\definecolor{currentfill}{rgb}{0.686275,0.352941,0.313725}%
\pgfsetfillcolor{currentfill}%
\pgfsetlinewidth{0.501875pt}%
\definecolor{currentstroke}{rgb}{0.686275,0.352941,0.313725}%
\pgfsetstrokecolor{currentstroke}%
\pgfsetdash{}{0pt}%
\pgfpathmoveto{\pgfqpoint{1.514160in}{1.152503in}}%
\pgfpathcurveto{\pgfqpoint{1.519685in}{1.152503in}}{\pgfqpoint{1.524984in}{1.154698in}}{\pgfqpoint{1.528891in}{1.158604in}}%
\pgfpathcurveto{\pgfqpoint{1.532798in}{1.162511in}}{\pgfqpoint{1.534993in}{1.167811in}}{\pgfqpoint{1.534993in}{1.173336in}}%
\pgfpathcurveto{\pgfqpoint{1.534993in}{1.178861in}}{\pgfqpoint{1.532798in}{1.184160in}}{\pgfqpoint{1.528891in}{1.188067in}}%
\pgfpathcurveto{\pgfqpoint{1.524984in}{1.191974in}}{\pgfqpoint{1.519685in}{1.194169in}}{\pgfqpoint{1.514160in}{1.194169in}}%
\pgfpathcurveto{\pgfqpoint{1.508635in}{1.194169in}}{\pgfqpoint{1.503335in}{1.191974in}}{\pgfqpoint{1.499428in}{1.188067in}}%
\pgfpathcurveto{\pgfqpoint{1.495522in}{1.184160in}}{\pgfqpoint{1.493326in}{1.178861in}}{\pgfqpoint{1.493326in}{1.173336in}}%
\pgfpathcurveto{\pgfqpoint{1.493326in}{1.167811in}}{\pgfqpoint{1.495522in}{1.162511in}}{\pgfqpoint{1.499428in}{1.158604in}}%
\pgfpathcurveto{\pgfqpoint{1.503335in}{1.154698in}}{\pgfqpoint{1.508635in}{1.152503in}}{\pgfqpoint{1.514160in}{1.152503in}}%
\pgfpathclose%
\pgfusepath{stroke,fill}%
\end{pgfscope}%
\begin{pgfscope}%
\pgfpathrectangle{\pgfqpoint{0.403630in}{0.251348in}}{\pgfqpoint{1.395000in}{1.359000in}}%
\pgfusepath{clip}%
\pgfsetbuttcap%
\pgfsetroundjoin%
\definecolor{currentfill}{rgb}{0.686275,0.352941,0.313725}%
\pgfsetfillcolor{currentfill}%
\pgfsetlinewidth{0.501875pt}%
\definecolor{currentstroke}{rgb}{0.686275,0.352941,0.313725}%
\pgfsetstrokecolor{currentstroke}%
\pgfsetdash{}{0pt}%
\pgfpathmoveto{\pgfqpoint{1.421160in}{0.603573in}}%
\pgfpathcurveto{\pgfqpoint{1.426685in}{0.603573in}}{\pgfqpoint{1.431984in}{0.605768in}}{\pgfqpoint{1.435891in}{0.609675in}}%
\pgfpathcurveto{\pgfqpoint{1.439798in}{0.613582in}}{\pgfqpoint{1.441993in}{0.618881in}}{\pgfqpoint{1.441993in}{0.624406in}}%
\pgfpathcurveto{\pgfqpoint{1.441993in}{0.629932in}}{\pgfqpoint{1.439798in}{0.635231in}}{\pgfqpoint{1.435891in}{0.639138in}}%
\pgfpathcurveto{\pgfqpoint{1.431984in}{0.643045in}}{\pgfqpoint{1.426685in}{0.645240in}}{\pgfqpoint{1.421160in}{0.645240in}}%
\pgfpathcurveto{\pgfqpoint{1.415635in}{0.645240in}}{\pgfqpoint{1.410335in}{0.643045in}}{\pgfqpoint{1.406428in}{0.639138in}}%
\pgfpathcurveto{\pgfqpoint{1.402522in}{0.635231in}}{\pgfqpoint{1.400326in}{0.629932in}}{\pgfqpoint{1.400326in}{0.624406in}}%
\pgfpathcurveto{\pgfqpoint{1.400326in}{0.618881in}}{\pgfqpoint{1.402522in}{0.613582in}}{\pgfqpoint{1.406428in}{0.609675in}}%
\pgfpathcurveto{\pgfqpoint{1.410335in}{0.605768in}}{\pgfqpoint{1.415635in}{0.603573in}}{\pgfqpoint{1.421160in}{0.603573in}}%
\pgfpathclose%
\pgfusepath{stroke,fill}%
\end{pgfscope}%
\begin{pgfscope}%
\pgfpathrectangle{\pgfqpoint{0.403630in}{0.251348in}}{\pgfqpoint{1.395000in}{1.359000in}}%
\pgfusepath{clip}%
\pgfsetbuttcap%
\pgfsetroundjoin%
\definecolor{currentfill}{rgb}{0.686275,0.352941,0.313725}%
\pgfsetfillcolor{currentfill}%
\pgfsetlinewidth{0.501875pt}%
\definecolor{currentstroke}{rgb}{0.686275,0.352941,0.313725}%
\pgfsetstrokecolor{currentstroke}%
\pgfsetdash{}{0pt}%
\pgfpathmoveto{\pgfqpoint{0.660748in}{0.944655in}}%
\pgfpathcurveto{\pgfqpoint{0.666273in}{0.944655in}}{\pgfqpoint{0.671573in}{0.946851in}}{\pgfqpoint{0.675479in}{0.950757in}}%
\pgfpathcurveto{\pgfqpoint{0.679386in}{0.954664in}}{\pgfqpoint{0.681581in}{0.959964in}}{\pgfqpoint{0.681581in}{0.965489in}}%
\pgfpathcurveto{\pgfqpoint{0.681581in}{0.971014in}}{\pgfqpoint{0.679386in}{0.976313in}}{\pgfqpoint{0.675479in}{0.980220in}}%
\pgfpathcurveto{\pgfqpoint{0.671573in}{0.984127in}}{\pgfqpoint{0.666273in}{0.986322in}}{\pgfqpoint{0.660748in}{0.986322in}}%
\pgfpathcurveto{\pgfqpoint{0.655223in}{0.986322in}}{\pgfqpoint{0.649923in}{0.984127in}}{\pgfqpoint{0.646017in}{0.980220in}}%
\pgfpathcurveto{\pgfqpoint{0.642110in}{0.976313in}}{\pgfqpoint{0.639915in}{0.971014in}}{\pgfqpoint{0.639915in}{0.965489in}}%
\pgfpathcurveto{\pgfqpoint{0.639915in}{0.959964in}}{\pgfqpoint{0.642110in}{0.954664in}}{\pgfqpoint{0.646017in}{0.950757in}}%
\pgfpathcurveto{\pgfqpoint{0.649923in}{0.946851in}}{\pgfqpoint{0.655223in}{0.944655in}}{\pgfqpoint{0.660748in}{0.944655in}}%
\pgfpathclose%
\pgfusepath{stroke,fill}%
\end{pgfscope}%
\begin{pgfscope}%
\pgfpathrectangle{\pgfqpoint{0.403630in}{0.251348in}}{\pgfqpoint{1.395000in}{1.359000in}}%
\pgfusepath{clip}%
\pgfsetbuttcap%
\pgfsetroundjoin%
\definecolor{currentfill}{rgb}{0.686275,0.352941,0.313725}%
\pgfsetfillcolor{currentfill}%
\pgfsetlinewidth{0.501875pt}%
\definecolor{currentstroke}{rgb}{0.686275,0.352941,0.313725}%
\pgfsetstrokecolor{currentstroke}%
\pgfsetdash{}{0pt}%
\pgfpathmoveto{\pgfqpoint{1.185925in}{1.344361in}}%
\pgfpathcurveto{\pgfqpoint{1.191450in}{1.344361in}}{\pgfqpoint{1.196749in}{1.346556in}}{\pgfqpoint{1.200656in}{1.350463in}}%
\pgfpathcurveto{\pgfqpoint{1.204563in}{1.354370in}}{\pgfqpoint{1.206758in}{1.359670in}}{\pgfqpoint{1.206758in}{1.365195in}}%
\pgfpathcurveto{\pgfqpoint{1.206758in}{1.370720in}}{\pgfqpoint{1.204563in}{1.376019in}}{\pgfqpoint{1.200656in}{1.379926in}}%
\pgfpathcurveto{\pgfqpoint{1.196749in}{1.383833in}}{\pgfqpoint{1.191450in}{1.386028in}}{\pgfqpoint{1.185925in}{1.386028in}}%
\pgfpathcurveto{\pgfqpoint{1.180399in}{1.386028in}}{\pgfqpoint{1.175100in}{1.383833in}}{\pgfqpoint{1.171193in}{1.379926in}}%
\pgfpathcurveto{\pgfqpoint{1.167286in}{1.376019in}}{\pgfqpoint{1.165091in}{1.370720in}}{\pgfqpoint{1.165091in}{1.365195in}}%
\pgfpathcurveto{\pgfqpoint{1.165091in}{1.359670in}}{\pgfqpoint{1.167286in}{1.354370in}}{\pgfqpoint{1.171193in}{1.350463in}}%
\pgfpathcurveto{\pgfqpoint{1.175100in}{1.346556in}}{\pgfqpoint{1.180399in}{1.344361in}}{\pgfqpoint{1.185925in}{1.344361in}}%
\pgfpathclose%
\pgfusepath{stroke,fill}%
\end{pgfscope}%
\begin{pgfscope}%
\pgfpathrectangle{\pgfqpoint{0.403630in}{0.251348in}}{\pgfqpoint{1.395000in}{1.359000in}}%
\pgfusepath{clip}%
\pgfsetbuttcap%
\pgfsetroundjoin%
\definecolor{currentfill}{rgb}{0.686275,0.352941,0.313725}%
\pgfsetfillcolor{currentfill}%
\pgfsetlinewidth{0.501875pt}%
\definecolor{currentstroke}{rgb}{0.686275,0.352941,0.313725}%
\pgfsetstrokecolor{currentstroke}%
\pgfsetdash{}{0pt}%
\pgfpathmoveto{\pgfqpoint{1.366454in}{1.285738in}}%
\pgfpathcurveto{\pgfqpoint{1.371979in}{1.285738in}}{\pgfqpoint{1.377279in}{1.287933in}}{\pgfqpoint{1.381185in}{1.291840in}}%
\pgfpathcurveto{\pgfqpoint{1.385092in}{1.295747in}}{\pgfqpoint{1.387287in}{1.301046in}}{\pgfqpoint{1.387287in}{1.306571in}}%
\pgfpathcurveto{\pgfqpoint{1.387287in}{1.312096in}}{\pgfqpoint{1.385092in}{1.317396in}}{\pgfqpoint{1.381185in}{1.321303in}}%
\pgfpathcurveto{\pgfqpoint{1.377279in}{1.325209in}}{\pgfqpoint{1.371979in}{1.327405in}}{\pgfqpoint{1.366454in}{1.327405in}}%
\pgfpathcurveto{\pgfqpoint{1.360929in}{1.327405in}}{\pgfqpoint{1.355629in}{1.325209in}}{\pgfqpoint{1.351723in}{1.321303in}}%
\pgfpathcurveto{\pgfqpoint{1.347816in}{1.317396in}}{\pgfqpoint{1.345621in}{1.312096in}}{\pgfqpoint{1.345621in}{1.306571in}}%
\pgfpathcurveto{\pgfqpoint{1.345621in}{1.301046in}}{\pgfqpoint{1.347816in}{1.295747in}}{\pgfqpoint{1.351723in}{1.291840in}}%
\pgfpathcurveto{\pgfqpoint{1.355629in}{1.287933in}}{\pgfqpoint{1.360929in}{1.285738in}}{\pgfqpoint{1.366454in}{1.285738in}}%
\pgfpathclose%
\pgfusepath{stroke,fill}%
\end{pgfscope}%
\begin{pgfscope}%
\pgfpathrectangle{\pgfqpoint{0.403630in}{0.251348in}}{\pgfqpoint{1.395000in}{1.359000in}}%
\pgfusepath{clip}%
\pgfsetbuttcap%
\pgfsetroundjoin%
\definecolor{currentfill}{rgb}{0.686275,0.352941,0.313725}%
\pgfsetfillcolor{currentfill}%
\pgfsetlinewidth{0.501875pt}%
\definecolor{currentstroke}{rgb}{0.686275,0.352941,0.313725}%
\pgfsetstrokecolor{currentstroke}%
\pgfsetdash{}{0pt}%
\pgfpathmoveto{\pgfqpoint{0.655277in}{0.854055in}}%
\pgfpathcurveto{\pgfqpoint{0.660803in}{0.854055in}}{\pgfqpoint{0.666102in}{0.856251in}}{\pgfqpoint{0.670009in}{0.860157in}}%
\pgfpathcurveto{\pgfqpoint{0.673916in}{0.864064in}}{\pgfqpoint{0.676111in}{0.869364in}}{\pgfqpoint{0.676111in}{0.874889in}}%
\pgfpathcurveto{\pgfqpoint{0.676111in}{0.880414in}}{\pgfqpoint{0.673916in}{0.885713in}}{\pgfqpoint{0.670009in}{0.889620in}}%
\pgfpathcurveto{\pgfqpoint{0.666102in}{0.893527in}}{\pgfqpoint{0.660803in}{0.895722in}}{\pgfqpoint{0.655277in}{0.895722in}}%
\pgfpathcurveto{\pgfqpoint{0.649752in}{0.895722in}}{\pgfqpoint{0.644453in}{0.893527in}}{\pgfqpoint{0.640546in}{0.889620in}}%
\pgfpathcurveto{\pgfqpoint{0.636639in}{0.885713in}}{\pgfqpoint{0.634444in}{0.880414in}}{\pgfqpoint{0.634444in}{0.874889in}}%
\pgfpathcurveto{\pgfqpoint{0.634444in}{0.869364in}}{\pgfqpoint{0.636639in}{0.864064in}}{\pgfqpoint{0.640546in}{0.860157in}}%
\pgfpathcurveto{\pgfqpoint{0.644453in}{0.856251in}}{\pgfqpoint{0.649752in}{0.854055in}}{\pgfqpoint{0.655277in}{0.854055in}}%
\pgfpathclose%
\pgfusepath{stroke,fill}%
\end{pgfscope}%
\begin{pgfscope}%
\pgfpathrectangle{\pgfqpoint{0.403630in}{0.251348in}}{\pgfqpoint{1.395000in}{1.359000in}}%
\pgfusepath{clip}%
\pgfsetbuttcap%
\pgfsetroundjoin%
\definecolor{currentfill}{rgb}{0.686275,0.352941,0.313725}%
\pgfsetfillcolor{currentfill}%
\pgfsetlinewidth{0.501875pt}%
\definecolor{currentstroke}{rgb}{0.686275,0.352941,0.313725}%
\pgfsetstrokecolor{currentstroke}%
\pgfsetdash{}{0pt}%
\pgfpathmoveto{\pgfqpoint{1.010866in}{1.333703in}}%
\pgfpathcurveto{\pgfqpoint{1.016391in}{1.333703in}}{\pgfqpoint{1.021690in}{1.335898in}}{\pgfqpoint{1.025597in}{1.339804in}}%
\pgfpathcurveto{\pgfqpoint{1.029504in}{1.343711in}}{\pgfqpoint{1.031699in}{1.349011in}}{\pgfqpoint{1.031699in}{1.354536in}}%
\pgfpathcurveto{\pgfqpoint{1.031699in}{1.360061in}}{\pgfqpoint{1.029504in}{1.365360in}}{\pgfqpoint{1.025597in}{1.369267in}}%
\pgfpathcurveto{\pgfqpoint{1.021690in}{1.373174in}}{\pgfqpoint{1.016391in}{1.375369in}}{\pgfqpoint{1.010866in}{1.375369in}}%
\pgfpathcurveto{\pgfqpoint{1.005341in}{1.375369in}}{\pgfqpoint{1.000041in}{1.373174in}}{\pgfqpoint{0.996134in}{1.369267in}}%
\pgfpathcurveto{\pgfqpoint{0.992228in}{1.365360in}}{\pgfqpoint{0.990032in}{1.360061in}}{\pgfqpoint{0.990032in}{1.354536in}}%
\pgfpathcurveto{\pgfqpoint{0.990032in}{1.349011in}}{\pgfqpoint{0.992228in}{1.343711in}}{\pgfqpoint{0.996134in}{1.339804in}}%
\pgfpathcurveto{\pgfqpoint{1.000041in}{1.335898in}}{\pgfqpoint{1.005341in}{1.333703in}}{\pgfqpoint{1.010866in}{1.333703in}}%
\pgfpathclose%
\pgfusepath{stroke,fill}%
\end{pgfscope}%
\begin{pgfscope}%
\pgfpathrectangle{\pgfqpoint{0.403630in}{0.251348in}}{\pgfqpoint{1.395000in}{1.359000in}}%
\pgfusepath{clip}%
\pgfsetbuttcap%
\pgfsetroundjoin%
\definecolor{currentfill}{rgb}{0.686275,0.352941,0.313725}%
\pgfsetfillcolor{currentfill}%
\pgfsetlinewidth{0.501875pt}%
\definecolor{currentstroke}{rgb}{0.686275,0.352941,0.313725}%
\pgfsetstrokecolor{currentstroke}%
\pgfsetdash{}{0pt}%
\pgfpathmoveto{\pgfqpoint{1.459454in}{0.608903in}}%
\pgfpathcurveto{\pgfqpoint{1.464979in}{0.608903in}}{\pgfqpoint{1.470279in}{0.611098in}}{\pgfqpoint{1.474185in}{0.615004in}}%
\pgfpathcurveto{\pgfqpoint{1.478092in}{0.618911in}}{\pgfqpoint{1.480287in}{0.624211in}}{\pgfqpoint{1.480287in}{0.629736in}}%
\pgfpathcurveto{\pgfqpoint{1.480287in}{0.635261in}}{\pgfqpoint{1.478092in}{0.640560in}}{\pgfqpoint{1.474185in}{0.644467in}}%
\pgfpathcurveto{\pgfqpoint{1.470279in}{0.648374in}}{\pgfqpoint{1.464979in}{0.650569in}}{\pgfqpoint{1.459454in}{0.650569in}}%
\pgfpathcurveto{\pgfqpoint{1.453929in}{0.650569in}}{\pgfqpoint{1.448629in}{0.648374in}}{\pgfqpoint{1.444723in}{0.644467in}}%
\pgfpathcurveto{\pgfqpoint{1.440816in}{0.640560in}}{\pgfqpoint{1.438621in}{0.635261in}}{\pgfqpoint{1.438621in}{0.629736in}}%
\pgfpathcurveto{\pgfqpoint{1.438621in}{0.624211in}}{\pgfqpoint{1.440816in}{0.618911in}}{\pgfqpoint{1.444723in}{0.615004in}}%
\pgfpathcurveto{\pgfqpoint{1.448629in}{0.611098in}}{\pgfqpoint{1.453929in}{0.608903in}}{\pgfqpoint{1.459454in}{0.608903in}}%
\pgfpathclose%
\pgfusepath{stroke,fill}%
\end{pgfscope}%
\begin{pgfscope}%
\pgfpathrectangle{\pgfqpoint{0.403630in}{0.251348in}}{\pgfqpoint{1.395000in}{1.359000in}}%
\pgfusepath{clip}%
\pgfsetbuttcap%
\pgfsetroundjoin%
\definecolor{currentfill}{rgb}{0.686275,0.352941,0.313725}%
\pgfsetfillcolor{currentfill}%
\pgfsetlinewidth{0.501875pt}%
\definecolor{currentstroke}{rgb}{0.686275,0.352941,0.313725}%
\pgfsetstrokecolor{currentstroke}%
\pgfsetdash{}{0pt}%
\pgfpathmoveto{\pgfqpoint{0.611513in}{0.859385in}}%
\pgfpathcurveto{\pgfqpoint{0.617038in}{0.859385in}}{\pgfqpoint{0.622337in}{0.861580in}}{\pgfqpoint{0.626244in}{0.865487in}}%
\pgfpathcurveto{\pgfqpoint{0.630151in}{0.869394in}}{\pgfqpoint{0.632346in}{0.874693in}}{\pgfqpoint{0.632346in}{0.880218in}}%
\pgfpathcurveto{\pgfqpoint{0.632346in}{0.885743in}}{\pgfqpoint{0.630151in}{0.891043in}}{\pgfqpoint{0.626244in}{0.894950in}}%
\pgfpathcurveto{\pgfqpoint{0.622337in}{0.898856in}}{\pgfqpoint{0.617038in}{0.901052in}}{\pgfqpoint{0.611513in}{0.901052in}}%
\pgfpathcurveto{\pgfqpoint{0.605988in}{0.901052in}}{\pgfqpoint{0.600688in}{0.898856in}}{\pgfqpoint{0.596781in}{0.894950in}}%
\pgfpathcurveto{\pgfqpoint{0.592875in}{0.891043in}}{\pgfqpoint{0.590679in}{0.885743in}}{\pgfqpoint{0.590679in}{0.880218in}}%
\pgfpathcurveto{\pgfqpoint{0.590679in}{0.874693in}}{\pgfqpoint{0.592875in}{0.869394in}}{\pgfqpoint{0.596781in}{0.865487in}}%
\pgfpathcurveto{\pgfqpoint{0.600688in}{0.861580in}}{\pgfqpoint{0.605988in}{0.859385in}}{\pgfqpoint{0.611513in}{0.859385in}}%
\pgfpathclose%
\pgfusepath{stroke,fill}%
\end{pgfscope}%
\begin{pgfscope}%
\pgfpathrectangle{\pgfqpoint{0.403630in}{0.251348in}}{\pgfqpoint{1.395000in}{1.359000in}}%
\pgfusepath{clip}%
\pgfsetbuttcap%
\pgfsetroundjoin%
\definecolor{currentfill}{rgb}{0.686275,0.352941,0.313725}%
\pgfsetfillcolor{currentfill}%
\pgfsetlinewidth{0.501875pt}%
\definecolor{currentstroke}{rgb}{0.686275,0.352941,0.313725}%
\pgfsetstrokecolor{currentstroke}%
\pgfsetdash{}{0pt}%
\pgfpathmoveto{\pgfqpoint{1.311748in}{0.454350in}}%
\pgfpathcurveto{\pgfqpoint{1.317273in}{0.454350in}}{\pgfqpoint{1.322573in}{0.456545in}}{\pgfqpoint{1.326479in}{0.460452in}}%
\pgfpathcurveto{\pgfqpoint{1.330386in}{0.464358in}}{\pgfqpoint{1.332581in}{0.469658in}}{\pgfqpoint{1.332581in}{0.475183in}}%
\pgfpathcurveto{\pgfqpoint{1.332581in}{0.480708in}}{\pgfqpoint{1.330386in}{0.486008in}}{\pgfqpoint{1.326479in}{0.489914in}}%
\pgfpathcurveto{\pgfqpoint{1.322573in}{0.493821in}}{\pgfqpoint{1.317273in}{0.496016in}}{\pgfqpoint{1.311748in}{0.496016in}}%
\pgfpathcurveto{\pgfqpoint{1.306223in}{0.496016in}}{\pgfqpoint{1.300923in}{0.493821in}}{\pgfqpoint{1.297017in}{0.489914in}}%
\pgfpathcurveto{\pgfqpoint{1.293110in}{0.486008in}}{\pgfqpoint{1.290915in}{0.480708in}}{\pgfqpoint{1.290915in}{0.475183in}}%
\pgfpathcurveto{\pgfqpoint{1.290915in}{0.469658in}}{\pgfqpoint{1.293110in}{0.464358in}}{\pgfqpoint{1.297017in}{0.460452in}}%
\pgfpathcurveto{\pgfqpoint{1.300923in}{0.456545in}}{\pgfqpoint{1.306223in}{0.454350in}}{\pgfqpoint{1.311748in}{0.454350in}}%
\pgfpathclose%
\pgfusepath{stroke,fill}%
\end{pgfscope}%
\begin{pgfscope}%
\pgfpathrectangle{\pgfqpoint{0.403630in}{0.251348in}}{\pgfqpoint{1.395000in}{1.359000in}}%
\pgfusepath{clip}%
\pgfsetbuttcap%
\pgfsetroundjoin%
\definecolor{currentfill}{rgb}{0.686275,0.352941,0.313725}%
\pgfsetfillcolor{currentfill}%
\pgfsetlinewidth{0.501875pt}%
\definecolor{currentstroke}{rgb}{0.686275,0.352941,0.313725}%
\pgfsetstrokecolor{currentstroke}%
\pgfsetdash{}{0pt}%
\pgfpathmoveto{\pgfqpoint{0.595101in}{0.886032in}}%
\pgfpathcurveto{\pgfqpoint{0.600626in}{0.886032in}}{\pgfqpoint{0.605926in}{0.888227in}}{\pgfqpoint{0.609832in}{0.892134in}}%
\pgfpathcurveto{\pgfqpoint{0.613739in}{0.896041in}}{\pgfqpoint{0.615934in}{0.901340in}}{\pgfqpoint{0.615934in}{0.906865in}}%
\pgfpathcurveto{\pgfqpoint{0.615934in}{0.912390in}}{\pgfqpoint{0.613739in}{0.917690in}}{\pgfqpoint{0.609832in}{0.921597in}}%
\pgfpathcurveto{\pgfqpoint{0.605926in}{0.925503in}}{\pgfqpoint{0.600626in}{0.927699in}}{\pgfqpoint{0.595101in}{0.927699in}}%
\pgfpathcurveto{\pgfqpoint{0.589576in}{0.927699in}}{\pgfqpoint{0.584276in}{0.925503in}}{\pgfqpoint{0.580370in}{0.921597in}}%
\pgfpathcurveto{\pgfqpoint{0.576463in}{0.917690in}}{\pgfqpoint{0.574268in}{0.912390in}}{\pgfqpoint{0.574268in}{0.906865in}}%
\pgfpathcurveto{\pgfqpoint{0.574268in}{0.901340in}}{\pgfqpoint{0.576463in}{0.896041in}}{\pgfqpoint{0.580370in}{0.892134in}}%
\pgfpathcurveto{\pgfqpoint{0.584276in}{0.888227in}}{\pgfqpoint{0.589576in}{0.886032in}}{\pgfqpoint{0.595101in}{0.886032in}}%
\pgfpathclose%
\pgfusepath{stroke,fill}%
\end{pgfscope}%
\begin{pgfscope}%
\pgfpathrectangle{\pgfqpoint{0.403630in}{0.251348in}}{\pgfqpoint{1.395000in}{1.359000in}}%
\pgfusepath{clip}%
\pgfsetbuttcap%
\pgfsetroundjoin%
\definecolor{currentfill}{rgb}{0.686275,0.352941,0.313725}%
\pgfsetfillcolor{currentfill}%
\pgfsetlinewidth{0.501875pt}%
\definecolor{currentstroke}{rgb}{0.686275,0.352941,0.313725}%
\pgfsetstrokecolor{currentstroke}%
\pgfsetdash{}{0pt}%
\pgfpathmoveto{\pgfqpoint{0.748277in}{0.651538in}}%
\pgfpathcurveto{\pgfqpoint{0.753803in}{0.651538in}}{\pgfqpoint{0.759102in}{0.653733in}}{\pgfqpoint{0.763009in}{0.657640in}}%
\pgfpathcurveto{\pgfqpoint{0.766916in}{0.661547in}}{\pgfqpoint{0.769111in}{0.666846in}}{\pgfqpoint{0.769111in}{0.672371in}}%
\pgfpathcurveto{\pgfqpoint{0.769111in}{0.677896in}}{\pgfqpoint{0.766916in}{0.683196in}}{\pgfqpoint{0.763009in}{0.687103in}}%
\pgfpathcurveto{\pgfqpoint{0.759102in}{0.691009in}}{\pgfqpoint{0.753803in}{0.693205in}}{\pgfqpoint{0.748277in}{0.693205in}}%
\pgfpathcurveto{\pgfqpoint{0.742752in}{0.693205in}}{\pgfqpoint{0.737453in}{0.691009in}}{\pgfqpoint{0.733546in}{0.687103in}}%
\pgfpathcurveto{\pgfqpoint{0.729639in}{0.683196in}}{\pgfqpoint{0.727444in}{0.677896in}}{\pgfqpoint{0.727444in}{0.672371in}}%
\pgfpathcurveto{\pgfqpoint{0.727444in}{0.666846in}}{\pgfqpoint{0.729639in}{0.661547in}}{\pgfqpoint{0.733546in}{0.657640in}}%
\pgfpathcurveto{\pgfqpoint{0.737453in}{0.653733in}}{\pgfqpoint{0.742752in}{0.651538in}}{\pgfqpoint{0.748277in}{0.651538in}}%
\pgfpathclose%
\pgfusepath{stroke,fill}%
\end{pgfscope}%
\begin{pgfscope}%
\pgfpathrectangle{\pgfqpoint{0.403630in}{0.251348in}}{\pgfqpoint{1.395000in}{1.359000in}}%
\pgfusepath{clip}%
\pgfsetbuttcap%
\pgfsetroundjoin%
\definecolor{currentfill}{rgb}{0.686275,0.352941,0.313725}%
\pgfsetfillcolor{currentfill}%
\pgfsetlinewidth{0.501875pt}%
\definecolor{currentstroke}{rgb}{0.686275,0.352941,0.313725}%
\pgfsetstrokecolor{currentstroke}%
\pgfsetdash{}{0pt}%
\pgfpathmoveto{\pgfqpoint{0.819395in}{1.243103in}}%
\pgfpathcurveto{\pgfqpoint{0.824920in}{1.243103in}}{\pgfqpoint{0.830220in}{1.245298in}}{\pgfqpoint{0.834127in}{1.249204in}}%
\pgfpathcurveto{\pgfqpoint{0.838033in}{1.253111in}}{\pgfqpoint{0.840228in}{1.258411in}}{\pgfqpoint{0.840228in}{1.263936in}}%
\pgfpathcurveto{\pgfqpoint{0.840228in}{1.269461in}}{\pgfqpoint{0.838033in}{1.274760in}}{\pgfqpoint{0.834127in}{1.278667in}}%
\pgfpathcurveto{\pgfqpoint{0.830220in}{1.282574in}}{\pgfqpoint{0.824920in}{1.284769in}}{\pgfqpoint{0.819395in}{1.284769in}}%
\pgfpathcurveto{\pgfqpoint{0.813870in}{1.284769in}}{\pgfqpoint{0.808571in}{1.282574in}}{\pgfqpoint{0.804664in}{1.278667in}}%
\pgfpathcurveto{\pgfqpoint{0.800757in}{1.274760in}}{\pgfqpoint{0.798562in}{1.269461in}}{\pgfqpoint{0.798562in}{1.263936in}}%
\pgfpathcurveto{\pgfqpoint{0.798562in}{1.258411in}}{\pgfqpoint{0.800757in}{1.253111in}}{\pgfqpoint{0.804664in}{1.249204in}}%
\pgfpathcurveto{\pgfqpoint{0.808571in}{1.245298in}}{\pgfqpoint{0.813870in}{1.243103in}}{\pgfqpoint{0.819395in}{1.243103in}}%
\pgfpathclose%
\pgfusepath{stroke,fill}%
\end{pgfscope}%
\begin{pgfscope}%
\pgfpathrectangle{\pgfqpoint{0.403630in}{0.251348in}}{\pgfqpoint{1.395000in}{1.359000in}}%
\pgfusepath{clip}%
\pgfsetbuttcap%
\pgfsetroundjoin%
\definecolor{currentfill}{rgb}{0.686275,0.352941,0.313725}%
\pgfsetfillcolor{currentfill}%
\pgfsetlinewidth{0.501875pt}%
\definecolor{currentstroke}{rgb}{0.686275,0.352941,0.313725}%
\pgfsetstrokecolor{currentstroke}%
\pgfsetdash{}{0pt}%
\pgfpathmoveto{\pgfqpoint{1.317219in}{0.486326in}}%
\pgfpathcurveto{\pgfqpoint{1.322744in}{0.486326in}}{\pgfqpoint{1.328043in}{0.488521in}}{\pgfqpoint{1.331950in}{0.492428in}}%
\pgfpathcurveto{\pgfqpoint{1.335857in}{0.496335in}}{\pgfqpoint{1.338052in}{0.501634in}}{\pgfqpoint{1.338052in}{0.507159in}}%
\pgfpathcurveto{\pgfqpoint{1.338052in}{0.512684in}}{\pgfqpoint{1.335857in}{0.517984in}}{\pgfqpoint{1.331950in}{0.521891in}}%
\pgfpathcurveto{\pgfqpoint{1.328043in}{0.525798in}}{\pgfqpoint{1.322744in}{0.527993in}}{\pgfqpoint{1.317219in}{0.527993in}}%
\pgfpathcurveto{\pgfqpoint{1.311694in}{0.527993in}}{\pgfqpoint{1.306394in}{0.525798in}}{\pgfqpoint{1.302487in}{0.521891in}}%
\pgfpathcurveto{\pgfqpoint{1.298580in}{0.517984in}}{\pgfqpoint{1.296385in}{0.512684in}}{\pgfqpoint{1.296385in}{0.507159in}}%
\pgfpathcurveto{\pgfqpoint{1.296385in}{0.501634in}}{\pgfqpoint{1.298580in}{0.496335in}}{\pgfqpoint{1.302487in}{0.492428in}}%
\pgfpathcurveto{\pgfqpoint{1.306394in}{0.488521in}}{\pgfqpoint{1.311694in}{0.486326in}}{\pgfqpoint{1.317219in}{0.486326in}}%
\pgfpathclose%
\pgfusepath{stroke,fill}%
\end{pgfscope}%
\begin{pgfscope}%
\pgfpathrectangle{\pgfqpoint{0.403630in}{0.251348in}}{\pgfqpoint{1.395000in}{1.359000in}}%
\pgfusepath{clip}%
\pgfsetbuttcap%
\pgfsetroundjoin%
\definecolor{currentfill}{rgb}{0.686275,0.352941,0.313725}%
\pgfsetfillcolor{currentfill}%
\pgfsetlinewidth{0.501875pt}%
\definecolor{currentstroke}{rgb}{0.686275,0.352941,0.313725}%
\pgfsetstrokecolor{currentstroke}%
\pgfsetdash{}{0pt}%
\pgfpathmoveto{\pgfqpoint{1.388336in}{1.248432in}}%
\pgfpathcurveto{\pgfqpoint{1.393861in}{1.248432in}}{\pgfqpoint{1.399161in}{1.250627in}}{\pgfqpoint{1.403068in}{1.254534in}}%
\pgfpathcurveto{\pgfqpoint{1.406975in}{1.258441in}}{\pgfqpoint{1.409170in}{1.263740in}}{\pgfqpoint{1.409170in}{1.269265in}}%
\pgfpathcurveto{\pgfqpoint{1.409170in}{1.274790in}}{\pgfqpoint{1.406975in}{1.280090in}}{\pgfqpoint{1.403068in}{1.283997in}}%
\pgfpathcurveto{\pgfqpoint{1.399161in}{1.287903in}}{\pgfqpoint{1.393861in}{1.290099in}}{\pgfqpoint{1.388336in}{1.290099in}}%
\pgfpathcurveto{\pgfqpoint{1.382811in}{1.290099in}}{\pgfqpoint{1.377512in}{1.287903in}}{\pgfqpoint{1.373605in}{1.283997in}}%
\pgfpathcurveto{\pgfqpoint{1.369698in}{1.280090in}}{\pgfqpoint{1.367503in}{1.274790in}}{\pgfqpoint{1.367503in}{1.269265in}}%
\pgfpathcurveto{\pgfqpoint{1.367503in}{1.263740in}}{\pgfqpoint{1.369698in}{1.258441in}}{\pgfqpoint{1.373605in}{1.254534in}}%
\pgfpathcurveto{\pgfqpoint{1.377512in}{1.250627in}}{\pgfqpoint{1.382811in}{1.248432in}}{\pgfqpoint{1.388336in}{1.248432in}}%
\pgfpathclose%
\pgfusepath{stroke,fill}%
\end{pgfscope}%
\begin{pgfscope}%
\pgfpathrectangle{\pgfqpoint{0.403630in}{0.251348in}}{\pgfqpoint{1.395000in}{1.359000in}}%
\pgfusepath{clip}%
\pgfsetbuttcap%
\pgfsetroundjoin%
\definecolor{currentfill}{rgb}{0.686275,0.352941,0.313725}%
\pgfsetfillcolor{currentfill}%
\pgfsetlinewidth{0.501875pt}%
\definecolor{currentstroke}{rgb}{0.686275,0.352941,0.313725}%
\pgfsetstrokecolor{currentstroke}%
\pgfsetdash{}{0pt}%
\pgfpathmoveto{\pgfqpoint{1.557925in}{1.072561in}}%
\pgfpathcurveto{\pgfqpoint{1.563450in}{1.072561in}}{\pgfqpoint{1.568749in}{1.074756in}}{\pgfqpoint{1.572656in}{1.078663in}}%
\pgfpathcurveto{\pgfqpoint{1.576563in}{1.082570in}}{\pgfqpoint{1.578758in}{1.087870in}}{\pgfqpoint{1.578758in}{1.093395in}}%
\pgfpathcurveto{\pgfqpoint{1.578758in}{1.098920in}}{\pgfqpoint{1.576563in}{1.104219in}}{\pgfqpoint{1.572656in}{1.108126in}}%
\pgfpathcurveto{\pgfqpoint{1.568749in}{1.112033in}}{\pgfqpoint{1.563450in}{1.114228in}}{\pgfqpoint{1.557925in}{1.114228in}}%
\pgfpathcurveto{\pgfqpoint{1.552399in}{1.114228in}}{\pgfqpoint{1.547100in}{1.112033in}}{\pgfqpoint{1.543193in}{1.108126in}}%
\pgfpathcurveto{\pgfqpoint{1.539286in}{1.104219in}}{\pgfqpoint{1.537091in}{1.098920in}}{\pgfqpoint{1.537091in}{1.093395in}}%
\pgfpathcurveto{\pgfqpoint{1.537091in}{1.087870in}}{\pgfqpoint{1.539286in}{1.082570in}}{\pgfqpoint{1.543193in}{1.078663in}}%
\pgfpathcurveto{\pgfqpoint{1.547100in}{1.074756in}}{\pgfqpoint{1.552399in}{1.072561in}}{\pgfqpoint{1.557925in}{1.072561in}}%
\pgfpathclose%
\pgfusepath{stroke,fill}%
\end{pgfscope}%
\begin{pgfscope}%
\pgfpathrectangle{\pgfqpoint{0.403630in}{0.251348in}}{\pgfqpoint{1.395000in}{1.359000in}}%
\pgfusepath{clip}%
\pgfsetbuttcap%
\pgfsetroundjoin%
\definecolor{currentfill}{rgb}{0.686275,0.352941,0.313725}%
\pgfsetfillcolor{currentfill}%
\pgfsetlinewidth{0.501875pt}%
\definecolor{currentstroke}{rgb}{0.686275,0.352941,0.313725}%
\pgfsetstrokecolor{currentstroke}%
\pgfsetdash{}{0pt}%
\pgfpathmoveto{\pgfqpoint{0.622454in}{1.040585in}}%
\pgfpathcurveto{\pgfqpoint{0.627979in}{1.040585in}}{\pgfqpoint{0.633279in}{1.042780in}}{\pgfqpoint{0.637185in}{1.046687in}}%
\pgfpathcurveto{\pgfqpoint{0.641092in}{1.050594in}}{\pgfqpoint{0.643287in}{1.055893in}}{\pgfqpoint{0.643287in}{1.061418in}}%
\pgfpathcurveto{\pgfqpoint{0.643287in}{1.066943in}}{\pgfqpoint{0.641092in}{1.072243in}}{\pgfqpoint{0.637185in}{1.076150in}}%
\pgfpathcurveto{\pgfqpoint{0.633279in}{1.080056in}}{\pgfqpoint{0.627979in}{1.082252in}}{\pgfqpoint{0.622454in}{1.082252in}}%
\pgfpathcurveto{\pgfqpoint{0.616929in}{1.082252in}}{\pgfqpoint{0.611629in}{1.080056in}}{\pgfqpoint{0.607723in}{1.076150in}}%
\pgfpathcurveto{\pgfqpoint{0.603816in}{1.072243in}}{\pgfqpoint{0.601621in}{1.066943in}}{\pgfqpoint{0.601621in}{1.061418in}}%
\pgfpathcurveto{\pgfqpoint{0.601621in}{1.055893in}}{\pgfqpoint{0.603816in}{1.050594in}}{\pgfqpoint{0.607723in}{1.046687in}}%
\pgfpathcurveto{\pgfqpoint{0.611629in}{1.042780in}}{\pgfqpoint{0.616929in}{1.040585in}}{\pgfqpoint{0.622454in}{1.040585in}}%
\pgfpathclose%
\pgfusepath{stroke,fill}%
\end{pgfscope}%
\begin{pgfscope}%
\pgfpathrectangle{\pgfqpoint{0.403630in}{0.251348in}}{\pgfqpoint{1.395000in}{1.359000in}}%
\pgfusepath{clip}%
\pgfsetbuttcap%
\pgfsetroundjoin%
\definecolor{currentfill}{rgb}{0.686275,0.352941,0.313725}%
\pgfsetfillcolor{currentfill}%
\pgfsetlinewidth{0.501875pt}%
\definecolor{currentstroke}{rgb}{0.686275,0.352941,0.313725}%
\pgfsetstrokecolor{currentstroke}%
\pgfsetdash{}{0pt}%
\pgfpathmoveto{\pgfqpoint{1.306277in}{1.355020in}}%
\pgfpathcurveto{\pgfqpoint{1.311803in}{1.355020in}}{\pgfqpoint{1.317102in}{1.357215in}}{\pgfqpoint{1.321009in}{1.361122in}}%
\pgfpathcurveto{\pgfqpoint{1.324916in}{1.365029in}}{\pgfqpoint{1.327111in}{1.370328in}}{\pgfqpoint{1.327111in}{1.375854in}}%
\pgfpathcurveto{\pgfqpoint{1.327111in}{1.381379in}}{\pgfqpoint{1.324916in}{1.386678in}}{\pgfqpoint{1.321009in}{1.390585in}}%
\pgfpathcurveto{\pgfqpoint{1.317102in}{1.394492in}}{\pgfqpoint{1.311803in}{1.396687in}}{\pgfqpoint{1.306277in}{1.396687in}}%
\pgfpathcurveto{\pgfqpoint{1.300752in}{1.396687in}}{\pgfqpoint{1.295453in}{1.394492in}}{\pgfqpoint{1.291546in}{1.390585in}}%
\pgfpathcurveto{\pgfqpoint{1.287639in}{1.386678in}}{\pgfqpoint{1.285444in}{1.381379in}}{\pgfqpoint{1.285444in}{1.375854in}}%
\pgfpathcurveto{\pgfqpoint{1.285444in}{1.370328in}}{\pgfqpoint{1.287639in}{1.365029in}}{\pgfqpoint{1.291546in}{1.361122in}}%
\pgfpathcurveto{\pgfqpoint{1.295453in}{1.357215in}}{\pgfqpoint{1.300752in}{1.355020in}}{\pgfqpoint{1.306277in}{1.355020in}}%
\pgfpathclose%
\pgfusepath{stroke,fill}%
\end{pgfscope}%
\begin{pgfscope}%
\pgfpathrectangle{\pgfqpoint{0.403630in}{0.251348in}}{\pgfqpoint{1.395000in}{1.359000in}}%
\pgfusepath{clip}%
\pgfsetbuttcap%
\pgfsetroundjoin%
\definecolor{currentfill}{rgb}{0.686275,0.352941,0.313725}%
\pgfsetfillcolor{currentfill}%
\pgfsetlinewidth{0.501875pt}%
\definecolor{currentstroke}{rgb}{0.686275,0.352941,0.313725}%
\pgfsetstrokecolor{currentstroke}%
\pgfsetdash{}{0pt}%
\pgfpathmoveto{\pgfqpoint{1.530572in}{1.051244in}}%
\pgfpathcurveto{\pgfqpoint{1.536097in}{1.051244in}}{\pgfqpoint{1.541396in}{1.053439in}}{\pgfqpoint{1.545303in}{1.057346in}}%
\pgfpathcurveto{\pgfqpoint{1.549210in}{1.061252in}}{\pgfqpoint{1.551405in}{1.066552in}}{\pgfqpoint{1.551405in}{1.072077in}}%
\pgfpathcurveto{\pgfqpoint{1.551405in}{1.077602in}}{\pgfqpoint{1.549210in}{1.082902in}}{\pgfqpoint{1.545303in}{1.086808in}}%
\pgfpathcurveto{\pgfqpoint{1.541396in}{1.090715in}}{\pgfqpoint{1.536097in}{1.092910in}}{\pgfqpoint{1.530572in}{1.092910in}}%
\pgfpathcurveto{\pgfqpoint{1.525047in}{1.092910in}}{\pgfqpoint{1.519747in}{1.090715in}}{\pgfqpoint{1.515840in}{1.086808in}}%
\pgfpathcurveto{\pgfqpoint{1.511933in}{1.082902in}}{\pgfqpoint{1.509738in}{1.077602in}}{\pgfqpoint{1.509738in}{1.072077in}}%
\pgfpathcurveto{\pgfqpoint{1.509738in}{1.066552in}}{\pgfqpoint{1.511933in}{1.061252in}}{\pgfqpoint{1.515840in}{1.057346in}}%
\pgfpathcurveto{\pgfqpoint{1.519747in}{1.053439in}}{\pgfqpoint{1.525047in}{1.051244in}}{\pgfqpoint{1.530572in}{1.051244in}}%
\pgfpathclose%
\pgfusepath{stroke,fill}%
\end{pgfscope}%
\begin{pgfscope}%
\pgfpathrectangle{\pgfqpoint{0.403630in}{0.251348in}}{\pgfqpoint{1.395000in}{1.359000in}}%
\pgfusepath{clip}%
\pgfsetbuttcap%
\pgfsetroundjoin%
\definecolor{currentfill}{rgb}{0.686275,0.352941,0.313725}%
\pgfsetfillcolor{currentfill}%
\pgfsetlinewidth{0.501875pt}%
\definecolor{currentstroke}{rgb}{0.686275,0.352941,0.313725}%
\pgfsetstrokecolor{currentstroke}%
\pgfsetdash{}{0pt}%
\pgfpathmoveto{\pgfqpoint{0.961630in}{0.465008in}}%
\pgfpathcurveto{\pgfqpoint{0.967155in}{0.465008in}}{\pgfqpoint{0.972455in}{0.467204in}}{\pgfqpoint{0.976362in}{0.471110in}}%
\pgfpathcurveto{\pgfqpoint{0.980269in}{0.475017in}}{\pgfqpoint{0.982464in}{0.480317in}}{\pgfqpoint{0.982464in}{0.485842in}}%
\pgfpathcurveto{\pgfqpoint{0.982464in}{0.491367in}}{\pgfqpoint{0.980269in}{0.496666in}}{\pgfqpoint{0.976362in}{0.500573in}}%
\pgfpathcurveto{\pgfqpoint{0.972455in}{0.504480in}}{\pgfqpoint{0.967155in}{0.506675in}}{\pgfqpoint{0.961630in}{0.506675in}}%
\pgfpathcurveto{\pgfqpoint{0.956105in}{0.506675in}}{\pgfqpoint{0.950806in}{0.504480in}}{\pgfqpoint{0.946899in}{0.500573in}}%
\pgfpathcurveto{\pgfqpoint{0.942992in}{0.496666in}}{\pgfqpoint{0.940797in}{0.491367in}}{\pgfqpoint{0.940797in}{0.485842in}}%
\pgfpathcurveto{\pgfqpoint{0.940797in}{0.480317in}}{\pgfqpoint{0.942992in}{0.475017in}}{\pgfqpoint{0.946899in}{0.471110in}}%
\pgfpathcurveto{\pgfqpoint{0.950806in}{0.467204in}}{\pgfqpoint{0.956105in}{0.465008in}}{\pgfqpoint{0.961630in}{0.465008in}}%
\pgfpathclose%
\pgfusepath{stroke,fill}%
\end{pgfscope}%
\begin{pgfscope}%
\pgfpathrectangle{\pgfqpoint{0.403630in}{0.251348in}}{\pgfqpoint{1.395000in}{1.359000in}}%
\pgfusepath{clip}%
\pgfsetbuttcap%
\pgfsetroundjoin%
\definecolor{currentfill}{rgb}{0.686275,0.352941,0.313725}%
\pgfsetfillcolor{currentfill}%
\pgfsetlinewidth{0.501875pt}%
\definecolor{currentstroke}{rgb}{0.686275,0.352941,0.313725}%
\pgfsetstrokecolor{currentstroke}%
\pgfsetdash{}{0pt}%
\pgfpathmoveto{\pgfqpoint{0.611513in}{0.912679in}}%
\pgfpathcurveto{\pgfqpoint{0.617038in}{0.912679in}}{\pgfqpoint{0.622337in}{0.914874in}}{\pgfqpoint{0.626244in}{0.918781in}}%
\pgfpathcurveto{\pgfqpoint{0.630151in}{0.922688in}}{\pgfqpoint{0.632346in}{0.927987in}}{\pgfqpoint{0.632346in}{0.933512in}}%
\pgfpathcurveto{\pgfqpoint{0.632346in}{0.939037in}}{\pgfqpoint{0.630151in}{0.944337in}}{\pgfqpoint{0.626244in}{0.948244in}}%
\pgfpathcurveto{\pgfqpoint{0.622337in}{0.952151in}}{\pgfqpoint{0.617038in}{0.954346in}}{\pgfqpoint{0.611513in}{0.954346in}}%
\pgfpathcurveto{\pgfqpoint{0.605988in}{0.954346in}}{\pgfqpoint{0.600688in}{0.952151in}}{\pgfqpoint{0.596781in}{0.948244in}}%
\pgfpathcurveto{\pgfqpoint{0.592875in}{0.944337in}}{\pgfqpoint{0.590679in}{0.939037in}}{\pgfqpoint{0.590679in}{0.933512in}}%
\pgfpathcurveto{\pgfqpoint{0.590679in}{0.927987in}}{\pgfqpoint{0.592875in}{0.922688in}}{\pgfqpoint{0.596781in}{0.918781in}}%
\pgfpathcurveto{\pgfqpoint{0.600688in}{0.914874in}}{\pgfqpoint{0.605988in}{0.912679in}}{\pgfqpoint{0.611513in}{0.912679in}}%
\pgfpathclose%
\pgfusepath{stroke,fill}%
\end{pgfscope}%
\begin{pgfscope}%
\pgfpathrectangle{\pgfqpoint{0.403630in}{0.251348in}}{\pgfqpoint{1.395000in}{1.359000in}}%
\pgfusepath{clip}%
\pgfsetbuttcap%
\pgfsetroundjoin%
\definecolor{currentfill}{rgb}{0.686275,0.352941,0.313725}%
\pgfsetfillcolor{currentfill}%
\pgfsetlinewidth{0.501875pt}%
\definecolor{currentstroke}{rgb}{0.686275,0.352941,0.313725}%
\pgfsetstrokecolor{currentstroke}%
\pgfsetdash{}{0pt}%
\pgfpathmoveto{\pgfqpoint{0.660748in}{0.747467in}}%
\pgfpathcurveto{\pgfqpoint{0.666273in}{0.747467in}}{\pgfqpoint{0.671573in}{0.749662in}}{\pgfqpoint{0.675479in}{0.753569in}}%
\pgfpathcurveto{\pgfqpoint{0.679386in}{0.757476in}}{\pgfqpoint{0.681581in}{0.762776in}}{\pgfqpoint{0.681581in}{0.768301in}}%
\pgfpathcurveto{\pgfqpoint{0.681581in}{0.773826in}}{\pgfqpoint{0.679386in}{0.779125in}}{\pgfqpoint{0.675479in}{0.783032in}}%
\pgfpathcurveto{\pgfqpoint{0.671573in}{0.786939in}}{\pgfqpoint{0.666273in}{0.789134in}}{\pgfqpoint{0.660748in}{0.789134in}}%
\pgfpathcurveto{\pgfqpoint{0.655223in}{0.789134in}}{\pgfqpoint{0.649923in}{0.786939in}}{\pgfqpoint{0.646017in}{0.783032in}}%
\pgfpathcurveto{\pgfqpoint{0.642110in}{0.779125in}}{\pgfqpoint{0.639915in}{0.773826in}}{\pgfqpoint{0.639915in}{0.768301in}}%
\pgfpathcurveto{\pgfqpoint{0.639915in}{0.762776in}}{\pgfqpoint{0.642110in}{0.757476in}}{\pgfqpoint{0.646017in}{0.753569in}}%
\pgfpathcurveto{\pgfqpoint{0.649923in}{0.749662in}}{\pgfqpoint{0.655223in}{0.747467in}}{\pgfqpoint{0.660748in}{0.747467in}}%
\pgfpathclose%
\pgfusepath{stroke,fill}%
\end{pgfscope}%
\begin{pgfscope}%
\pgfpathrectangle{\pgfqpoint{0.403630in}{0.251348in}}{\pgfqpoint{1.395000in}{1.359000in}}%
\pgfusepath{clip}%
\pgfsetbuttcap%
\pgfsetroundjoin%
\definecolor{currentfill}{rgb}{0.686275,0.352941,0.313725}%
\pgfsetfillcolor{currentfill}%
\pgfsetlinewidth{0.501875pt}%
\definecolor{currentstroke}{rgb}{0.686275,0.352941,0.313725}%
\pgfsetstrokecolor{currentstroke}%
\pgfsetdash{}{0pt}%
\pgfpathmoveto{\pgfqpoint{0.704513in}{1.168491in}}%
\pgfpathcurveto{\pgfqpoint{0.710038in}{1.168491in}}{\pgfqpoint{0.715337in}{1.170686in}}{\pgfqpoint{0.719244in}{1.174593in}}%
\pgfpathcurveto{\pgfqpoint{0.723151in}{1.178500in}}{\pgfqpoint{0.725346in}{1.183799in}}{\pgfqpoint{0.725346in}{1.189324in}}%
\pgfpathcurveto{\pgfqpoint{0.725346in}{1.194849in}}{\pgfqpoint{0.723151in}{1.200149in}}{\pgfqpoint{0.719244in}{1.204056in}}%
\pgfpathcurveto{\pgfqpoint{0.715337in}{1.207962in}}{\pgfqpoint{0.710038in}{1.210157in}}{\pgfqpoint{0.704513in}{1.210157in}}%
\pgfpathcurveto{\pgfqpoint{0.698988in}{1.210157in}}{\pgfqpoint{0.693688in}{1.207962in}}{\pgfqpoint{0.689781in}{1.204056in}}%
\pgfpathcurveto{\pgfqpoint{0.685875in}{1.200149in}}{\pgfqpoint{0.683679in}{1.194849in}}{\pgfqpoint{0.683679in}{1.189324in}}%
\pgfpathcurveto{\pgfqpoint{0.683679in}{1.183799in}}{\pgfqpoint{0.685875in}{1.178500in}}{\pgfqpoint{0.689781in}{1.174593in}}%
\pgfpathcurveto{\pgfqpoint{0.693688in}{1.170686in}}{\pgfqpoint{0.698988in}{1.168491in}}{\pgfqpoint{0.704513in}{1.168491in}}%
\pgfpathclose%
\pgfusepath{stroke,fill}%
\end{pgfscope}%
\begin{pgfscope}%
\pgfpathrectangle{\pgfqpoint{0.403630in}{0.251348in}}{\pgfqpoint{1.395000in}{1.359000in}}%
\pgfusepath{clip}%
\pgfsetbuttcap%
\pgfsetroundjoin%
\definecolor{currentfill}{rgb}{0.686275,0.352941,0.313725}%
\pgfsetfillcolor{currentfill}%
\pgfsetlinewidth{0.501875pt}%
\definecolor{currentstroke}{rgb}{0.686275,0.352941,0.313725}%
\pgfsetstrokecolor{currentstroke}%
\pgfsetdash{}{0pt}%
\pgfpathmoveto{\pgfqpoint{1.229689in}{0.480997in}}%
\pgfpathcurveto{\pgfqpoint{1.235214in}{0.480997in}}{\pgfqpoint{1.240514in}{0.483192in}}{\pgfqpoint{1.244421in}{0.487099in}}%
\pgfpathcurveto{\pgfqpoint{1.248327in}{0.491005in}}{\pgfqpoint{1.250523in}{0.496305in}}{\pgfqpoint{1.250523in}{0.501830in}}%
\pgfpathcurveto{\pgfqpoint{1.250523in}{0.507355in}}{\pgfqpoint{1.248327in}{0.512655in}}{\pgfqpoint{1.244421in}{0.516561in}}%
\pgfpathcurveto{\pgfqpoint{1.240514in}{0.520468in}}{\pgfqpoint{1.235214in}{0.522663in}}{\pgfqpoint{1.229689in}{0.522663in}}%
\pgfpathcurveto{\pgfqpoint{1.224164in}{0.522663in}}{\pgfqpoint{1.218865in}{0.520468in}}{\pgfqpoint{1.214958in}{0.516561in}}%
\pgfpathcurveto{\pgfqpoint{1.211051in}{0.512655in}}{\pgfqpoint{1.208856in}{0.507355in}}{\pgfqpoint{1.208856in}{0.501830in}}%
\pgfpathcurveto{\pgfqpoint{1.208856in}{0.496305in}}{\pgfqpoint{1.211051in}{0.491005in}}{\pgfqpoint{1.214958in}{0.487099in}}%
\pgfpathcurveto{\pgfqpoint{1.218865in}{0.483192in}}{\pgfqpoint{1.224164in}{0.480997in}}{\pgfqpoint{1.229689in}{0.480997in}}%
\pgfpathclose%
\pgfusepath{stroke,fill}%
\end{pgfscope}%
\begin{pgfscope}%
\pgfpathrectangle{\pgfqpoint{0.403630in}{0.251348in}}{\pgfqpoint{1.395000in}{1.359000in}}%
\pgfusepath{clip}%
\pgfsetbuttcap%
\pgfsetroundjoin%
\definecolor{currentfill}{rgb}{0.686275,0.352941,0.313725}%
\pgfsetfillcolor{currentfill}%
\pgfsetlinewidth{0.501875pt}%
\definecolor{currentstroke}{rgb}{0.686275,0.352941,0.313725}%
\pgfsetstrokecolor{currentstroke}%
\pgfsetdash{}{0pt}%
\pgfpathmoveto{\pgfqpoint{1.508689in}{0.598244in}}%
\pgfpathcurveto{\pgfqpoint{1.514214in}{0.598244in}}{\pgfqpoint{1.519514in}{0.600439in}}{\pgfqpoint{1.523421in}{0.604346in}}%
\pgfpathcurveto{\pgfqpoint{1.527327in}{0.608252in}}{\pgfqpoint{1.529523in}{0.613552in}}{\pgfqpoint{1.529523in}{0.619077in}}%
\pgfpathcurveto{\pgfqpoint{1.529523in}{0.624602in}}{\pgfqpoint{1.527327in}{0.629902in}}{\pgfqpoint{1.523421in}{0.633808in}}%
\pgfpathcurveto{\pgfqpoint{1.519514in}{0.637715in}}{\pgfqpoint{1.514214in}{0.639910in}}{\pgfqpoint{1.508689in}{0.639910in}}%
\pgfpathcurveto{\pgfqpoint{1.503164in}{0.639910in}}{\pgfqpoint{1.497865in}{0.637715in}}{\pgfqpoint{1.493958in}{0.633808in}}%
\pgfpathcurveto{\pgfqpoint{1.490051in}{0.629902in}}{\pgfqpoint{1.487856in}{0.624602in}}{\pgfqpoint{1.487856in}{0.619077in}}%
\pgfpathcurveto{\pgfqpoint{1.487856in}{0.613552in}}{\pgfqpoint{1.490051in}{0.608252in}}{\pgfqpoint{1.493958in}{0.604346in}}%
\pgfpathcurveto{\pgfqpoint{1.497865in}{0.600439in}}{\pgfqpoint{1.503164in}{0.598244in}}{\pgfqpoint{1.508689in}{0.598244in}}%
\pgfpathclose%
\pgfusepath{stroke,fill}%
\end{pgfscope}%
\begin{pgfscope}%
\pgfpathrectangle{\pgfqpoint{0.403630in}{0.251348in}}{\pgfqpoint{1.395000in}{1.359000in}}%
\pgfusepath{clip}%
\pgfsetbuttcap%
\pgfsetroundjoin%
\definecolor{currentfill}{rgb}{0.686275,0.352941,0.313725}%
\pgfsetfillcolor{currentfill}%
\pgfsetlinewidth{0.501875pt}%
\definecolor{currentstroke}{rgb}{0.686275,0.352941,0.313725}%
\pgfsetstrokecolor{currentstroke}%
\pgfsetdash{}{0pt}%
\pgfpathmoveto{\pgfqpoint{0.863160in}{1.291067in}}%
\pgfpathcurveto{\pgfqpoint{0.868685in}{1.291067in}}{\pgfqpoint{0.873984in}{1.293262in}}{\pgfqpoint{0.877891in}{1.297169in}}%
\pgfpathcurveto{\pgfqpoint{0.881798in}{1.301076in}}{\pgfqpoint{0.883993in}{1.306376in}}{\pgfqpoint{0.883993in}{1.311901in}}%
\pgfpathcurveto{\pgfqpoint{0.883993in}{1.317426in}}{\pgfqpoint{0.881798in}{1.322725in}}{\pgfqpoint{0.877891in}{1.326632in}}%
\pgfpathcurveto{\pgfqpoint{0.873984in}{1.330539in}}{\pgfqpoint{0.868685in}{1.332734in}}{\pgfqpoint{0.863160in}{1.332734in}}%
\pgfpathcurveto{\pgfqpoint{0.857635in}{1.332734in}}{\pgfqpoint{0.852335in}{1.330539in}}{\pgfqpoint{0.848428in}{1.326632in}}%
\pgfpathcurveto{\pgfqpoint{0.844522in}{1.322725in}}{\pgfqpoint{0.842326in}{1.317426in}}{\pgfqpoint{0.842326in}{1.311901in}}%
\pgfpathcurveto{\pgfqpoint{0.842326in}{1.306376in}}{\pgfqpoint{0.844522in}{1.301076in}}{\pgfqpoint{0.848428in}{1.297169in}}%
\pgfpathcurveto{\pgfqpoint{0.852335in}{1.293262in}}{\pgfqpoint{0.857635in}{1.291067in}}{\pgfqpoint{0.863160in}{1.291067in}}%
\pgfpathclose%
\pgfusepath{stroke,fill}%
\end{pgfscope}%
\begin{pgfscope}%
\pgfpathrectangle{\pgfqpoint{0.403630in}{0.251348in}}{\pgfqpoint{1.395000in}{1.359000in}}%
\pgfusepath{clip}%
\pgfsetbuttcap%
\pgfsetroundjoin%
\definecolor{currentfill}{rgb}{0.686275,0.352941,0.313725}%
\pgfsetfillcolor{currentfill}%
\pgfsetlinewidth{0.501875pt}%
\definecolor{currentstroke}{rgb}{0.686275,0.352941,0.313725}%
\pgfsetstrokecolor{currentstroke}%
\pgfsetdash{}{0pt}%
\pgfpathmoveto{\pgfqpoint{1.426630in}{1.232444in}}%
\pgfpathcurveto{\pgfqpoint{1.432155in}{1.232444in}}{\pgfqpoint{1.437455in}{1.234639in}}{\pgfqpoint{1.441362in}{1.238546in}}%
\pgfpathcurveto{\pgfqpoint{1.445269in}{1.242452in}}{\pgfqpoint{1.447464in}{1.247752in}}{\pgfqpoint{1.447464in}{1.253277in}}%
\pgfpathcurveto{\pgfqpoint{1.447464in}{1.258802in}}{\pgfqpoint{1.445269in}{1.264102in}}{\pgfqpoint{1.441362in}{1.268008in}}%
\pgfpathcurveto{\pgfqpoint{1.437455in}{1.271915in}}{\pgfqpoint{1.432155in}{1.274110in}}{\pgfqpoint{1.426630in}{1.274110in}}%
\pgfpathcurveto{\pgfqpoint{1.421105in}{1.274110in}}{\pgfqpoint{1.415806in}{1.271915in}}{\pgfqpoint{1.411899in}{1.268008in}}%
\pgfpathcurveto{\pgfqpoint{1.407992in}{1.264102in}}{\pgfqpoint{1.405797in}{1.258802in}}{\pgfqpoint{1.405797in}{1.253277in}}%
\pgfpathcurveto{\pgfqpoint{1.405797in}{1.247752in}}{\pgfqpoint{1.407992in}{1.242452in}}{\pgfqpoint{1.411899in}{1.238546in}}%
\pgfpathcurveto{\pgfqpoint{1.415806in}{1.234639in}}{\pgfqpoint{1.421105in}{1.232444in}}{\pgfqpoint{1.426630in}{1.232444in}}%
\pgfpathclose%
\pgfusepath{stroke,fill}%
\end{pgfscope}%
\begin{pgfscope}%
\pgfpathrectangle{\pgfqpoint{0.403630in}{0.251348in}}{\pgfqpoint{1.395000in}{1.359000in}}%
\pgfusepath{clip}%
\pgfsetbuttcap%
\pgfsetroundjoin%
\definecolor{currentfill}{rgb}{0.686275,0.352941,0.313725}%
\pgfsetfillcolor{currentfill}%
\pgfsetlinewidth{0.501875pt}%
\definecolor{currentstroke}{rgb}{0.686275,0.352941,0.313725}%
\pgfsetstrokecolor{currentstroke}%
\pgfsetdash{}{0pt}%
\pgfpathmoveto{\pgfqpoint{1.191395in}{1.333703in}}%
\pgfpathcurveto{\pgfqpoint{1.196920in}{1.333703in}}{\pgfqpoint{1.202220in}{1.335898in}}{\pgfqpoint{1.206127in}{1.339804in}}%
\pgfpathcurveto{\pgfqpoint{1.210033in}{1.343711in}}{\pgfqpoint{1.212228in}{1.349011in}}{\pgfqpoint{1.212228in}{1.354536in}}%
\pgfpathcurveto{\pgfqpoint{1.212228in}{1.360061in}}{\pgfqpoint{1.210033in}{1.365360in}}{\pgfqpoint{1.206127in}{1.369267in}}%
\pgfpathcurveto{\pgfqpoint{1.202220in}{1.373174in}}{\pgfqpoint{1.196920in}{1.375369in}}{\pgfqpoint{1.191395in}{1.375369in}}%
\pgfpathcurveto{\pgfqpoint{1.185870in}{1.375369in}}{\pgfqpoint{1.180571in}{1.373174in}}{\pgfqpoint{1.176664in}{1.369267in}}%
\pgfpathcurveto{\pgfqpoint{1.172757in}{1.365360in}}{\pgfqpoint{1.170562in}{1.360061in}}{\pgfqpoint{1.170562in}{1.354536in}}%
\pgfpathcurveto{\pgfqpoint{1.170562in}{1.349011in}}{\pgfqpoint{1.172757in}{1.343711in}}{\pgfqpoint{1.176664in}{1.339804in}}%
\pgfpathcurveto{\pgfqpoint{1.180571in}{1.335898in}}{\pgfqpoint{1.185870in}{1.333703in}}{\pgfqpoint{1.191395in}{1.333703in}}%
\pgfpathclose%
\pgfusepath{stroke,fill}%
\end{pgfscope}%
\begin{pgfscope}%
\pgfpathrectangle{\pgfqpoint{0.403630in}{0.251348in}}{\pgfqpoint{1.395000in}{1.359000in}}%
\pgfusepath{clip}%
\pgfsetbuttcap%
\pgfsetroundjoin%
\definecolor{currentfill}{rgb}{0.686275,0.352941,0.313725}%
\pgfsetfillcolor{currentfill}%
\pgfsetlinewidth{0.501875pt}%
\definecolor{currentstroke}{rgb}{0.686275,0.352941,0.313725}%
\pgfsetstrokecolor{currentstroke}%
\pgfsetdash{}{0pt}%
\pgfpathmoveto{\pgfqpoint{0.655277in}{0.859385in}}%
\pgfpathcurveto{\pgfqpoint{0.660803in}{0.859385in}}{\pgfqpoint{0.666102in}{0.861580in}}{\pgfqpoint{0.670009in}{0.865487in}}%
\pgfpathcurveto{\pgfqpoint{0.673916in}{0.869394in}}{\pgfqpoint{0.676111in}{0.874693in}}{\pgfqpoint{0.676111in}{0.880218in}}%
\pgfpathcurveto{\pgfqpoint{0.676111in}{0.885743in}}{\pgfqpoint{0.673916in}{0.891043in}}{\pgfqpoint{0.670009in}{0.894950in}}%
\pgfpathcurveto{\pgfqpoint{0.666102in}{0.898856in}}{\pgfqpoint{0.660803in}{0.901052in}}{\pgfqpoint{0.655277in}{0.901052in}}%
\pgfpathcurveto{\pgfqpoint{0.649752in}{0.901052in}}{\pgfqpoint{0.644453in}{0.898856in}}{\pgfqpoint{0.640546in}{0.894950in}}%
\pgfpathcurveto{\pgfqpoint{0.636639in}{0.891043in}}{\pgfqpoint{0.634444in}{0.885743in}}{\pgfqpoint{0.634444in}{0.880218in}}%
\pgfpathcurveto{\pgfqpoint{0.634444in}{0.874693in}}{\pgfqpoint{0.636639in}{0.869394in}}{\pgfqpoint{0.640546in}{0.865487in}}%
\pgfpathcurveto{\pgfqpoint{0.644453in}{0.861580in}}{\pgfqpoint{0.649752in}{0.859385in}}{\pgfqpoint{0.655277in}{0.859385in}}%
\pgfpathclose%
\pgfusepath{stroke,fill}%
\end{pgfscope}%
\begin{pgfscope}%
\pgfpathrectangle{\pgfqpoint{0.403630in}{0.251348in}}{\pgfqpoint{1.395000in}{1.359000in}}%
\pgfusepath{clip}%
\pgfsetbuttcap%
\pgfsetroundjoin%
\definecolor{currentfill}{rgb}{0.686275,0.352941,0.313725}%
\pgfsetfillcolor{currentfill}%
\pgfsetlinewidth{0.501875pt}%
\definecolor{currentstroke}{rgb}{0.686275,0.352941,0.313725}%
\pgfsetstrokecolor{currentstroke}%
\pgfsetdash{}{0pt}%
\pgfpathmoveto{\pgfqpoint{1.536042in}{1.163161in}}%
\pgfpathcurveto{\pgfqpoint{1.541567in}{1.163161in}}{\pgfqpoint{1.546867in}{1.165356in}}{\pgfqpoint{1.550774in}{1.169263in}}%
\pgfpathcurveto{\pgfqpoint{1.554680in}{1.173170in}}{\pgfqpoint{1.556876in}{1.178470in}}{\pgfqpoint{1.556876in}{1.183995in}}%
\pgfpathcurveto{\pgfqpoint{1.556876in}{1.189520in}}{\pgfqpoint{1.554680in}{1.194819in}}{\pgfqpoint{1.550774in}{1.198726in}}%
\pgfpathcurveto{\pgfqpoint{1.546867in}{1.202633in}}{\pgfqpoint{1.541567in}{1.204828in}}{\pgfqpoint{1.536042in}{1.204828in}}%
\pgfpathcurveto{\pgfqpoint{1.530517in}{1.204828in}}{\pgfqpoint{1.525218in}{1.202633in}}{\pgfqpoint{1.521311in}{1.198726in}}%
\pgfpathcurveto{\pgfqpoint{1.517404in}{1.194819in}}{\pgfqpoint{1.515209in}{1.189520in}}{\pgfqpoint{1.515209in}{1.183995in}}%
\pgfpathcurveto{\pgfqpoint{1.515209in}{1.178470in}}{\pgfqpoint{1.517404in}{1.173170in}}{\pgfqpoint{1.521311in}{1.169263in}}%
\pgfpathcurveto{\pgfqpoint{1.525218in}{1.165356in}}{\pgfqpoint{1.530517in}{1.163161in}}{\pgfqpoint{1.536042in}{1.163161in}}%
\pgfpathclose%
\pgfusepath{stroke,fill}%
\end{pgfscope}%
\begin{pgfscope}%
\pgfpathrectangle{\pgfqpoint{0.403630in}{0.251348in}}{\pgfqpoint{1.395000in}{1.359000in}}%
\pgfusepath{clip}%
\pgfsetbuttcap%
\pgfsetroundjoin%
\definecolor{currentfill}{rgb}{0.686275,0.352941,0.313725}%
\pgfsetfillcolor{currentfill}%
\pgfsetlinewidth{0.501875pt}%
\definecolor{currentstroke}{rgb}{0.686275,0.352941,0.313725}%
\pgfsetstrokecolor{currentstroke}%
\pgfsetdash{}{0pt}%
\pgfpathmoveto{\pgfqpoint{1.251572in}{0.475667in}}%
\pgfpathcurveto{\pgfqpoint{1.257097in}{0.475667in}}{\pgfqpoint{1.262396in}{0.477862in}}{\pgfqpoint{1.266303in}{0.481769in}}%
\pgfpathcurveto{\pgfqpoint{1.270210in}{0.485676in}}{\pgfqpoint{1.272405in}{0.490976in}}{\pgfqpoint{1.272405in}{0.496501in}}%
\pgfpathcurveto{\pgfqpoint{1.272405in}{0.502026in}}{\pgfqpoint{1.270210in}{0.507325in}}{\pgfqpoint{1.266303in}{0.511232in}}%
\pgfpathcurveto{\pgfqpoint{1.262396in}{0.515139in}}{\pgfqpoint{1.257097in}{0.517334in}}{\pgfqpoint{1.251572in}{0.517334in}}%
\pgfpathcurveto{\pgfqpoint{1.246047in}{0.517334in}}{\pgfqpoint{1.240747in}{0.515139in}}{\pgfqpoint{1.236840in}{0.511232in}}%
\pgfpathcurveto{\pgfqpoint{1.232933in}{0.507325in}}{\pgfqpoint{1.230738in}{0.502026in}}{\pgfqpoint{1.230738in}{0.496501in}}%
\pgfpathcurveto{\pgfqpoint{1.230738in}{0.490976in}}{\pgfqpoint{1.232933in}{0.485676in}}{\pgfqpoint{1.236840in}{0.481769in}}%
\pgfpathcurveto{\pgfqpoint{1.240747in}{0.477862in}}{\pgfqpoint{1.246047in}{0.475667in}}{\pgfqpoint{1.251572in}{0.475667in}}%
\pgfpathclose%
\pgfusepath{stroke,fill}%
\end{pgfscope}%
\begin{pgfscope}%
\pgfpathrectangle{\pgfqpoint{0.403630in}{0.251348in}}{\pgfqpoint{1.395000in}{1.359000in}}%
\pgfusepath{clip}%
\pgfsetbuttcap%
\pgfsetroundjoin%
\definecolor{currentfill}{rgb}{0.686275,0.352941,0.313725}%
\pgfsetfillcolor{currentfill}%
\pgfsetlinewidth{0.501875pt}%
\definecolor{currentstroke}{rgb}{0.686275,0.352941,0.313725}%
\pgfsetstrokecolor{currentstroke}%
\pgfsetdash{}{0pt}%
\pgfpathmoveto{\pgfqpoint{0.846748in}{0.555608in}}%
\pgfpathcurveto{\pgfqpoint{0.852273in}{0.555608in}}{\pgfqpoint{0.857573in}{0.557804in}}{\pgfqpoint{0.861479in}{0.561710in}}%
\pgfpathcurveto{\pgfqpoint{0.865386in}{0.565617in}}{\pgfqpoint{0.867581in}{0.570917in}}{\pgfqpoint{0.867581in}{0.576442in}}%
\pgfpathcurveto{\pgfqpoint{0.867581in}{0.581967in}}{\pgfqpoint{0.865386in}{0.587266in}}{\pgfqpoint{0.861479in}{0.591173in}}%
\pgfpathcurveto{\pgfqpoint{0.857573in}{0.595080in}}{\pgfqpoint{0.852273in}{0.597275in}}{\pgfqpoint{0.846748in}{0.597275in}}%
\pgfpathcurveto{\pgfqpoint{0.841223in}{0.597275in}}{\pgfqpoint{0.835923in}{0.595080in}}{\pgfqpoint{0.832017in}{0.591173in}}%
\pgfpathcurveto{\pgfqpoint{0.828110in}{0.587266in}}{\pgfqpoint{0.825915in}{0.581967in}}{\pgfqpoint{0.825915in}{0.576442in}}%
\pgfpathcurveto{\pgfqpoint{0.825915in}{0.570917in}}{\pgfqpoint{0.828110in}{0.565617in}}{\pgfqpoint{0.832017in}{0.561710in}}%
\pgfpathcurveto{\pgfqpoint{0.835923in}{0.557804in}}{\pgfqpoint{0.841223in}{0.555608in}}{\pgfqpoint{0.846748in}{0.555608in}}%
\pgfpathclose%
\pgfusepath{stroke,fill}%
\end{pgfscope}%
\begin{pgfscope}%
\pgfpathrectangle{\pgfqpoint{0.403630in}{0.251348in}}{\pgfqpoint{1.395000in}{1.359000in}}%
\pgfusepath{clip}%
\pgfsetbuttcap%
\pgfsetroundjoin%
\definecolor{currentfill}{rgb}{0.686275,0.352941,0.313725}%
\pgfsetfillcolor{currentfill}%
\pgfsetlinewidth{0.501875pt}%
\definecolor{currentstroke}{rgb}{0.686275,0.352941,0.313725}%
\pgfsetstrokecolor{currentstroke}%
\pgfsetdash{}{0pt}%
\pgfpathmoveto{\pgfqpoint{0.835807in}{1.291067in}}%
\pgfpathcurveto{\pgfqpoint{0.841332in}{1.291067in}}{\pgfqpoint{0.846631in}{1.293262in}}{\pgfqpoint{0.850538in}{1.297169in}}%
\pgfpathcurveto{\pgfqpoint{0.854445in}{1.301076in}}{\pgfqpoint{0.856640in}{1.306376in}}{\pgfqpoint{0.856640in}{1.311901in}}%
\pgfpathcurveto{\pgfqpoint{0.856640in}{1.317426in}}{\pgfqpoint{0.854445in}{1.322725in}}{\pgfqpoint{0.850538in}{1.326632in}}%
\pgfpathcurveto{\pgfqpoint{0.846631in}{1.330539in}}{\pgfqpoint{0.841332in}{1.332734in}}{\pgfqpoint{0.835807in}{1.332734in}}%
\pgfpathcurveto{\pgfqpoint{0.830282in}{1.332734in}}{\pgfqpoint{0.824982in}{1.330539in}}{\pgfqpoint{0.821075in}{1.326632in}}%
\pgfpathcurveto{\pgfqpoint{0.817169in}{1.322725in}}{\pgfqpoint{0.814974in}{1.317426in}}{\pgfqpoint{0.814974in}{1.311901in}}%
\pgfpathcurveto{\pgfqpoint{0.814974in}{1.306376in}}{\pgfqpoint{0.817169in}{1.301076in}}{\pgfqpoint{0.821075in}{1.297169in}}%
\pgfpathcurveto{\pgfqpoint{0.824982in}{1.293262in}}{\pgfqpoint{0.830282in}{1.291067in}}{\pgfqpoint{0.835807in}{1.291067in}}%
\pgfpathclose%
\pgfusepath{stroke,fill}%
\end{pgfscope}%
\begin{pgfscope}%
\pgfpathrectangle{\pgfqpoint{0.403630in}{0.251348in}}{\pgfqpoint{1.395000in}{1.359000in}}%
\pgfusepath{clip}%
\pgfsetbuttcap%
\pgfsetroundjoin%
\definecolor{currentfill}{rgb}{0.686275,0.352941,0.313725}%
\pgfsetfillcolor{currentfill}%
\pgfsetlinewidth{0.501875pt}%
\definecolor{currentstroke}{rgb}{0.686275,0.352941,0.313725}%
\pgfsetstrokecolor{currentstroke}%
\pgfsetdash{}{0pt}%
\pgfpathmoveto{\pgfqpoint{1.590748in}{0.811420in}}%
\pgfpathcurveto{\pgfqpoint{1.596273in}{0.811420in}}{\pgfqpoint{1.601573in}{0.813615in}}{\pgfqpoint{1.605479in}{0.817522in}}%
\pgfpathcurveto{\pgfqpoint{1.609386in}{0.821429in}}{\pgfqpoint{1.611581in}{0.826728in}}{\pgfqpoint{1.611581in}{0.832254in}}%
\pgfpathcurveto{\pgfqpoint{1.611581in}{0.837779in}}{\pgfqpoint{1.609386in}{0.843078in}}{\pgfqpoint{1.605479in}{0.846985in}}%
\pgfpathcurveto{\pgfqpoint{1.601573in}{0.850892in}}{\pgfqpoint{1.596273in}{0.853087in}}{\pgfqpoint{1.590748in}{0.853087in}}%
\pgfpathcurveto{\pgfqpoint{1.585223in}{0.853087in}}{\pgfqpoint{1.579923in}{0.850892in}}{\pgfqpoint{1.576017in}{0.846985in}}%
\pgfpathcurveto{\pgfqpoint{1.572110in}{0.843078in}}{\pgfqpoint{1.569915in}{0.837779in}}{\pgfqpoint{1.569915in}{0.832254in}}%
\pgfpathcurveto{\pgfqpoint{1.569915in}{0.826728in}}{\pgfqpoint{1.572110in}{0.821429in}}{\pgfqpoint{1.576017in}{0.817522in}}%
\pgfpathcurveto{\pgfqpoint{1.579923in}{0.813615in}}{\pgfqpoint{1.585223in}{0.811420in}}{\pgfqpoint{1.590748in}{0.811420in}}%
\pgfpathclose%
\pgfusepath{stroke,fill}%
\end{pgfscope}%
\begin{pgfscope}%
\pgfpathrectangle{\pgfqpoint{0.403630in}{0.251348in}}{\pgfqpoint{1.395000in}{1.359000in}}%
\pgfusepath{clip}%
\pgfsetbuttcap%
\pgfsetroundjoin%
\definecolor{currentfill}{rgb}{0.686275,0.352941,0.313725}%
\pgfsetfillcolor{currentfill}%
\pgfsetlinewidth{0.501875pt}%
\definecolor{currentstroke}{rgb}{0.686275,0.352941,0.313725}%
\pgfsetstrokecolor{currentstroke}%
\pgfsetdash{}{0pt}%
\pgfpathmoveto{\pgfqpoint{0.863160in}{1.269750in}}%
\pgfpathcurveto{\pgfqpoint{0.868685in}{1.269750in}}{\pgfqpoint{0.873984in}{1.271945in}}{\pgfqpoint{0.877891in}{1.275852in}}%
\pgfpathcurveto{\pgfqpoint{0.881798in}{1.279758in}}{\pgfqpoint{0.883993in}{1.285058in}}{\pgfqpoint{0.883993in}{1.290583in}}%
\pgfpathcurveto{\pgfqpoint{0.883993in}{1.296108in}}{\pgfqpoint{0.881798in}{1.301408in}}{\pgfqpoint{0.877891in}{1.305314in}}%
\pgfpathcurveto{\pgfqpoint{0.873984in}{1.309221in}}{\pgfqpoint{0.868685in}{1.311416in}}{\pgfqpoint{0.863160in}{1.311416in}}%
\pgfpathcurveto{\pgfqpoint{0.857635in}{1.311416in}}{\pgfqpoint{0.852335in}{1.309221in}}{\pgfqpoint{0.848428in}{1.305314in}}%
\pgfpathcurveto{\pgfqpoint{0.844522in}{1.301408in}}{\pgfqpoint{0.842326in}{1.296108in}}{\pgfqpoint{0.842326in}{1.290583in}}%
\pgfpathcurveto{\pgfqpoint{0.842326in}{1.285058in}}{\pgfqpoint{0.844522in}{1.279758in}}{\pgfqpoint{0.848428in}{1.275852in}}%
\pgfpathcurveto{\pgfqpoint{0.852335in}{1.271945in}}{\pgfqpoint{0.857635in}{1.269750in}}{\pgfqpoint{0.863160in}{1.269750in}}%
\pgfpathclose%
\pgfusepath{stroke,fill}%
\end{pgfscope}%
\begin{pgfscope}%
\pgfpathrectangle{\pgfqpoint{0.403630in}{0.251348in}}{\pgfqpoint{1.395000in}{1.359000in}}%
\pgfusepath{clip}%
\pgfsetbuttcap%
\pgfsetroundjoin%
\definecolor{currentfill}{rgb}{0.686275,0.352941,0.313725}%
\pgfsetfillcolor{currentfill}%
\pgfsetlinewidth{0.501875pt}%
\definecolor{currentstroke}{rgb}{0.686275,0.352941,0.313725}%
\pgfsetstrokecolor{currentstroke}%
\pgfsetdash{}{0pt}%
\pgfpathmoveto{\pgfqpoint{0.600572in}{0.827408in}}%
\pgfpathcurveto{\pgfqpoint{0.606097in}{0.827408in}}{\pgfqpoint{0.611396in}{0.829604in}}{\pgfqpoint{0.615303in}{0.833510in}}%
\pgfpathcurveto{\pgfqpoint{0.619210in}{0.837417in}}{\pgfqpoint{0.621405in}{0.842717in}}{\pgfqpoint{0.621405in}{0.848242in}}%
\pgfpathcurveto{\pgfqpoint{0.621405in}{0.853767in}}{\pgfqpoint{0.619210in}{0.859066in}}{\pgfqpoint{0.615303in}{0.862973in}}%
\pgfpathcurveto{\pgfqpoint{0.611396in}{0.866880in}}{\pgfqpoint{0.606097in}{0.869075in}}{\pgfqpoint{0.600572in}{0.869075in}}%
\pgfpathcurveto{\pgfqpoint{0.595047in}{0.869075in}}{\pgfqpoint{0.589747in}{0.866880in}}{\pgfqpoint{0.585840in}{0.862973in}}%
\pgfpathcurveto{\pgfqpoint{0.581933in}{0.859066in}}{\pgfqpoint{0.579738in}{0.853767in}}{\pgfqpoint{0.579738in}{0.848242in}}%
\pgfpathcurveto{\pgfqpoint{0.579738in}{0.842717in}}{\pgfqpoint{0.581933in}{0.837417in}}{\pgfqpoint{0.585840in}{0.833510in}}%
\pgfpathcurveto{\pgfqpoint{0.589747in}{0.829604in}}{\pgfqpoint{0.595047in}{0.827408in}}{\pgfqpoint{0.600572in}{0.827408in}}%
\pgfpathclose%
\pgfusepath{stroke,fill}%
\end{pgfscope}%
\begin{pgfscope}%
\pgfpathrectangle{\pgfqpoint{0.403630in}{0.251348in}}{\pgfqpoint{1.395000in}{1.359000in}}%
\pgfusepath{clip}%
\pgfsetbuttcap%
\pgfsetroundjoin%
\definecolor{currentfill}{rgb}{0.686275,0.352941,0.313725}%
\pgfsetfillcolor{currentfill}%
\pgfsetlinewidth{0.501875pt}%
\definecolor{currentstroke}{rgb}{0.686275,0.352941,0.313725}%
\pgfsetstrokecolor{currentstroke}%
\pgfsetdash{}{0pt}%
\pgfpathmoveto{\pgfqpoint{1.546983in}{0.822079in}}%
\pgfpathcurveto{\pgfqpoint{1.552508in}{0.822079in}}{\pgfqpoint{1.557808in}{0.824274in}}{\pgfqpoint{1.561715in}{0.828181in}}%
\pgfpathcurveto{\pgfqpoint{1.565622in}{0.832088in}}{\pgfqpoint{1.567817in}{0.837387in}}{\pgfqpoint{1.567817in}{0.842912in}}%
\pgfpathcurveto{\pgfqpoint{1.567817in}{0.848437in}}{\pgfqpoint{1.565622in}{0.853737in}}{\pgfqpoint{1.561715in}{0.857644in}}%
\pgfpathcurveto{\pgfqpoint{1.557808in}{0.861551in}}{\pgfqpoint{1.552508in}{0.863746in}}{\pgfqpoint{1.546983in}{0.863746in}}%
\pgfpathcurveto{\pgfqpoint{1.541458in}{0.863746in}}{\pgfqpoint{1.536159in}{0.861551in}}{\pgfqpoint{1.532252in}{0.857644in}}%
\pgfpathcurveto{\pgfqpoint{1.528345in}{0.853737in}}{\pgfqpoint{1.526150in}{0.848437in}}{\pgfqpoint{1.526150in}{0.842912in}}%
\pgfpathcurveto{\pgfqpoint{1.526150in}{0.837387in}}{\pgfqpoint{1.528345in}{0.832088in}}{\pgfqpoint{1.532252in}{0.828181in}}%
\pgfpathcurveto{\pgfqpoint{1.536159in}{0.824274in}}{\pgfqpoint{1.541458in}{0.822079in}}{\pgfqpoint{1.546983in}{0.822079in}}%
\pgfpathclose%
\pgfusepath{stroke,fill}%
\end{pgfscope}%
\begin{pgfscope}%
\pgfpathrectangle{\pgfqpoint{0.403630in}{0.251348in}}{\pgfqpoint{1.395000in}{1.359000in}}%
\pgfusepath{clip}%
\pgfsetbuttcap%
\pgfsetroundjoin%
\definecolor{currentfill}{rgb}{0.686275,0.352941,0.313725}%
\pgfsetfillcolor{currentfill}%
\pgfsetlinewidth{0.501875pt}%
\definecolor{currentstroke}{rgb}{0.686275,0.352941,0.313725}%
\pgfsetstrokecolor{currentstroke}%
\pgfsetdash{}{0pt}%
\pgfpathmoveto{\pgfqpoint{1.399277in}{1.259091in}}%
\pgfpathcurveto{\pgfqpoint{1.404803in}{1.259091in}}{\pgfqpoint{1.410102in}{1.261286in}}{\pgfqpoint{1.414009in}{1.265193in}}%
\pgfpathcurveto{\pgfqpoint{1.417916in}{1.269100in}}{\pgfqpoint{1.420111in}{1.274399in}}{\pgfqpoint{1.420111in}{1.279924in}}%
\pgfpathcurveto{\pgfqpoint{1.420111in}{1.285449in}}{\pgfqpoint{1.417916in}{1.290749in}}{\pgfqpoint{1.414009in}{1.294656in}}%
\pgfpathcurveto{\pgfqpoint{1.410102in}{1.298562in}}{\pgfqpoint{1.404803in}{1.300757in}}{\pgfqpoint{1.399277in}{1.300757in}}%
\pgfpathcurveto{\pgfqpoint{1.393752in}{1.300757in}}{\pgfqpoint{1.388453in}{1.298562in}}{\pgfqpoint{1.384546in}{1.294656in}}%
\pgfpathcurveto{\pgfqpoint{1.380639in}{1.290749in}}{\pgfqpoint{1.378444in}{1.285449in}}{\pgfqpoint{1.378444in}{1.279924in}}%
\pgfpathcurveto{\pgfqpoint{1.378444in}{1.274399in}}{\pgfqpoint{1.380639in}{1.269100in}}{\pgfqpoint{1.384546in}{1.265193in}}%
\pgfpathcurveto{\pgfqpoint{1.388453in}{1.261286in}}{\pgfqpoint{1.393752in}{1.259091in}}{\pgfqpoint{1.399277in}{1.259091in}}%
\pgfpathclose%
\pgfusepath{stroke,fill}%
\end{pgfscope}%
\begin{pgfscope}%
\pgfpathrectangle{\pgfqpoint{0.403630in}{0.251348in}}{\pgfqpoint{1.395000in}{1.359000in}}%
\pgfusepath{clip}%
\pgfsetbuttcap%
\pgfsetroundjoin%
\definecolor{currentfill}{rgb}{0.686275,0.352941,0.313725}%
\pgfsetfillcolor{currentfill}%
\pgfsetlinewidth{0.501875pt}%
\definecolor{currentstroke}{rgb}{0.686275,0.352941,0.313725}%
\pgfsetstrokecolor{currentstroke}%
\pgfsetdash{}{0pt}%
\pgfpathmoveto{\pgfqpoint{0.633395in}{0.939326in}}%
\pgfpathcurveto{\pgfqpoint{0.638920in}{0.939326in}}{\pgfqpoint{0.644220in}{0.941521in}}{\pgfqpoint{0.648127in}{0.945428in}}%
\pgfpathcurveto{\pgfqpoint{0.652033in}{0.949335in}}{\pgfqpoint{0.654228in}{0.954634in}}{\pgfqpoint{0.654228in}{0.960159in}}%
\pgfpathcurveto{\pgfqpoint{0.654228in}{0.965684in}}{\pgfqpoint{0.652033in}{0.970984in}}{\pgfqpoint{0.648127in}{0.974891in}}%
\pgfpathcurveto{\pgfqpoint{0.644220in}{0.978798in}}{\pgfqpoint{0.638920in}{0.980993in}}{\pgfqpoint{0.633395in}{0.980993in}}%
\pgfpathcurveto{\pgfqpoint{0.627870in}{0.980993in}}{\pgfqpoint{0.622571in}{0.978798in}}{\pgfqpoint{0.618664in}{0.974891in}}%
\pgfpathcurveto{\pgfqpoint{0.614757in}{0.970984in}}{\pgfqpoint{0.612562in}{0.965684in}}{\pgfqpoint{0.612562in}{0.960159in}}%
\pgfpathcurveto{\pgfqpoint{0.612562in}{0.954634in}}{\pgfqpoint{0.614757in}{0.949335in}}{\pgfqpoint{0.618664in}{0.945428in}}%
\pgfpathcurveto{\pgfqpoint{0.622571in}{0.941521in}}{\pgfqpoint{0.627870in}{0.939326in}}{\pgfqpoint{0.633395in}{0.939326in}}%
\pgfpathclose%
\pgfusepath{stroke,fill}%
\end{pgfscope}%
\begin{pgfscope}%
\pgfpathrectangle{\pgfqpoint{0.403630in}{0.251348in}}{\pgfqpoint{1.395000in}{1.359000in}}%
\pgfusepath{clip}%
\pgfsetbuttcap%
\pgfsetroundjoin%
\definecolor{currentfill}{rgb}{0.686275,0.352941,0.313725}%
\pgfsetfillcolor{currentfill}%
\pgfsetlinewidth{0.501875pt}%
\definecolor{currentstroke}{rgb}{0.686275,0.352941,0.313725}%
\pgfsetstrokecolor{currentstroke}%
\pgfsetdash{}{0pt}%
\pgfpathmoveto{\pgfqpoint{0.595101in}{0.981961in}}%
\pgfpathcurveto{\pgfqpoint{0.600626in}{0.981961in}}{\pgfqpoint{0.605926in}{0.984156in}}{\pgfqpoint{0.609832in}{0.988063in}}%
\pgfpathcurveto{\pgfqpoint{0.613739in}{0.991970in}}{\pgfqpoint{0.615934in}{0.997270in}}{\pgfqpoint{0.615934in}{1.002795in}}%
\pgfpathcurveto{\pgfqpoint{0.615934in}{1.008320in}}{\pgfqpoint{0.613739in}{1.013619in}}{\pgfqpoint{0.609832in}{1.017526in}}%
\pgfpathcurveto{\pgfqpoint{0.605926in}{1.021433in}}{\pgfqpoint{0.600626in}{1.023628in}}{\pgfqpoint{0.595101in}{1.023628in}}%
\pgfpathcurveto{\pgfqpoint{0.589576in}{1.023628in}}{\pgfqpoint{0.584276in}{1.021433in}}{\pgfqpoint{0.580370in}{1.017526in}}%
\pgfpathcurveto{\pgfqpoint{0.576463in}{1.013619in}}{\pgfqpoint{0.574268in}{1.008320in}}{\pgfqpoint{0.574268in}{1.002795in}}%
\pgfpathcurveto{\pgfqpoint{0.574268in}{0.997270in}}{\pgfqpoint{0.576463in}{0.991970in}}{\pgfqpoint{0.580370in}{0.988063in}}%
\pgfpathcurveto{\pgfqpoint{0.584276in}{0.984156in}}{\pgfqpoint{0.589576in}{0.981961in}}{\pgfqpoint{0.595101in}{0.981961in}}%
\pgfpathclose%
\pgfusepath{stroke,fill}%
\end{pgfscope}%
\begin{pgfscope}%
\pgfpathrectangle{\pgfqpoint{0.403630in}{0.251348in}}{\pgfqpoint{1.395000in}{1.359000in}}%
\pgfusepath{clip}%
\pgfsetbuttcap%
\pgfsetroundjoin%
\definecolor{currentfill}{rgb}{0.686275,0.352941,0.313725}%
\pgfsetfillcolor{currentfill}%
\pgfsetlinewidth{0.501875pt}%
\definecolor{currentstroke}{rgb}{0.686275,0.352941,0.313725}%
\pgfsetstrokecolor{currentstroke}%
\pgfsetdash{}{0pt}%
\pgfpathmoveto{\pgfqpoint{1.612630in}{0.864714in}}%
\pgfpathcurveto{\pgfqpoint{1.618155in}{0.864714in}}{\pgfqpoint{1.623455in}{0.866909in}}{\pgfqpoint{1.627362in}{0.870816in}}%
\pgfpathcurveto{\pgfqpoint{1.631269in}{0.874723in}}{\pgfqpoint{1.633464in}{0.880023in}}{\pgfqpoint{1.633464in}{0.885548in}}%
\pgfpathcurveto{\pgfqpoint{1.633464in}{0.891073in}}{\pgfqpoint{1.631269in}{0.896372in}}{\pgfqpoint{1.627362in}{0.900279in}}%
\pgfpathcurveto{\pgfqpoint{1.623455in}{0.904186in}}{\pgfqpoint{1.618155in}{0.906381in}}{\pgfqpoint{1.612630in}{0.906381in}}%
\pgfpathcurveto{\pgfqpoint{1.607105in}{0.906381in}}{\pgfqpoint{1.601806in}{0.904186in}}{\pgfqpoint{1.597899in}{0.900279in}}%
\pgfpathcurveto{\pgfqpoint{1.593992in}{0.896372in}}{\pgfqpoint{1.591797in}{0.891073in}}{\pgfqpoint{1.591797in}{0.885548in}}%
\pgfpathcurveto{\pgfqpoint{1.591797in}{0.880023in}}{\pgfqpoint{1.593992in}{0.874723in}}{\pgfqpoint{1.597899in}{0.870816in}}%
\pgfpathcurveto{\pgfqpoint{1.601806in}{0.866909in}}{\pgfqpoint{1.607105in}{0.864714in}}{\pgfqpoint{1.612630in}{0.864714in}}%
\pgfpathclose%
\pgfusepath{stroke,fill}%
\end{pgfscope}%
\begin{pgfscope}%
\pgfpathrectangle{\pgfqpoint{0.403630in}{0.251348in}}{\pgfqpoint{1.395000in}{1.359000in}}%
\pgfusepath{clip}%
\pgfsetbuttcap%
\pgfsetroundjoin%
\definecolor{currentfill}{rgb}{0.686275,0.352941,0.313725}%
\pgfsetfillcolor{currentfill}%
\pgfsetlinewidth{0.501875pt}%
\definecolor{currentstroke}{rgb}{0.686275,0.352941,0.313725}%
\pgfsetstrokecolor{currentstroke}%
\pgfsetdash{}{0pt}%
\pgfpathmoveto{\pgfqpoint{0.660748in}{0.886032in}}%
\pgfpathcurveto{\pgfqpoint{0.666273in}{0.886032in}}{\pgfqpoint{0.671573in}{0.888227in}}{\pgfqpoint{0.675479in}{0.892134in}}%
\pgfpathcurveto{\pgfqpoint{0.679386in}{0.896041in}}{\pgfqpoint{0.681581in}{0.901340in}}{\pgfqpoint{0.681581in}{0.906865in}}%
\pgfpathcurveto{\pgfqpoint{0.681581in}{0.912390in}}{\pgfqpoint{0.679386in}{0.917690in}}{\pgfqpoint{0.675479in}{0.921597in}}%
\pgfpathcurveto{\pgfqpoint{0.671573in}{0.925503in}}{\pgfqpoint{0.666273in}{0.927699in}}{\pgfqpoint{0.660748in}{0.927699in}}%
\pgfpathcurveto{\pgfqpoint{0.655223in}{0.927699in}}{\pgfqpoint{0.649923in}{0.925503in}}{\pgfqpoint{0.646017in}{0.921597in}}%
\pgfpathcurveto{\pgfqpoint{0.642110in}{0.917690in}}{\pgfqpoint{0.639915in}{0.912390in}}{\pgfqpoint{0.639915in}{0.906865in}}%
\pgfpathcurveto{\pgfqpoint{0.639915in}{0.901340in}}{\pgfqpoint{0.642110in}{0.896041in}}{\pgfqpoint{0.646017in}{0.892134in}}%
\pgfpathcurveto{\pgfqpoint{0.649923in}{0.888227in}}{\pgfqpoint{0.655223in}{0.886032in}}{\pgfqpoint{0.660748in}{0.886032in}}%
\pgfpathclose%
\pgfusepath{stroke,fill}%
\end{pgfscope}%
\begin{pgfscope}%
\pgfpathrectangle{\pgfqpoint{0.403630in}{0.251348in}}{\pgfqpoint{1.395000in}{1.359000in}}%
\pgfusepath{clip}%
\pgfsetbuttcap%
\pgfsetroundjoin%
\definecolor{currentfill}{rgb}{0.686275,0.352941,0.313725}%
\pgfsetfillcolor{currentfill}%
\pgfsetlinewidth{0.501875pt}%
\definecolor{currentstroke}{rgb}{0.686275,0.352941,0.313725}%
\pgfsetstrokecolor{currentstroke}%
\pgfsetdash{}{0pt}%
\pgfpathmoveto{\pgfqpoint{1.426630in}{0.587585in}}%
\pgfpathcurveto{\pgfqpoint{1.432155in}{0.587585in}}{\pgfqpoint{1.437455in}{0.589780in}}{\pgfqpoint{1.441362in}{0.593687in}}%
\pgfpathcurveto{\pgfqpoint{1.445269in}{0.597594in}}{\pgfqpoint{1.447464in}{0.602893in}}{\pgfqpoint{1.447464in}{0.608418in}}%
\pgfpathcurveto{\pgfqpoint{1.447464in}{0.613943in}}{\pgfqpoint{1.445269in}{0.619243in}}{\pgfqpoint{1.441362in}{0.623150in}}%
\pgfpathcurveto{\pgfqpoint{1.437455in}{0.627056in}}{\pgfqpoint{1.432155in}{0.629252in}}{\pgfqpoint{1.426630in}{0.629252in}}%
\pgfpathcurveto{\pgfqpoint{1.421105in}{0.629252in}}{\pgfqpoint{1.415806in}{0.627056in}}{\pgfqpoint{1.411899in}{0.623150in}}%
\pgfpathcurveto{\pgfqpoint{1.407992in}{0.619243in}}{\pgfqpoint{1.405797in}{0.613943in}}{\pgfqpoint{1.405797in}{0.608418in}}%
\pgfpathcurveto{\pgfqpoint{1.405797in}{0.602893in}}{\pgfqpoint{1.407992in}{0.597594in}}{\pgfqpoint{1.411899in}{0.593687in}}%
\pgfpathcurveto{\pgfqpoint{1.415806in}{0.589780in}}{\pgfqpoint{1.421105in}{0.587585in}}{\pgfqpoint{1.426630in}{0.587585in}}%
\pgfpathclose%
\pgfusepath{stroke,fill}%
\end{pgfscope}%
\begin{pgfscope}%
\pgfpathrectangle{\pgfqpoint{0.403630in}{0.251348in}}{\pgfqpoint{1.395000in}{1.359000in}}%
\pgfusepath{clip}%
\pgfsetbuttcap%
\pgfsetroundjoin%
\definecolor{currentfill}{rgb}{0.686275,0.352941,0.313725}%
\pgfsetfillcolor{currentfill}%
\pgfsetlinewidth{0.501875pt}%
\definecolor{currentstroke}{rgb}{0.686275,0.352941,0.313725}%
\pgfsetstrokecolor{currentstroke}%
\pgfsetdash{}{0pt}%
\pgfpathmoveto{\pgfqpoint{1.207807in}{1.349691in}}%
\pgfpathcurveto{\pgfqpoint{1.213332in}{1.349691in}}{\pgfqpoint{1.218631in}{1.351886in}}{\pgfqpoint{1.222538in}{1.355793in}}%
\pgfpathcurveto{\pgfqpoint{1.226445in}{1.359700in}}{\pgfqpoint{1.228640in}{1.364999in}}{\pgfqpoint{1.228640in}{1.370524in}}%
\pgfpathcurveto{\pgfqpoint{1.228640in}{1.376049in}}{\pgfqpoint{1.226445in}{1.381349in}}{\pgfqpoint{1.222538in}{1.385256in}}%
\pgfpathcurveto{\pgfqpoint{1.218631in}{1.389162in}}{\pgfqpoint{1.213332in}{1.391357in}}{\pgfqpoint{1.207807in}{1.391357in}}%
\pgfpathcurveto{\pgfqpoint{1.202282in}{1.391357in}}{\pgfqpoint{1.196982in}{1.389162in}}{\pgfqpoint{1.193075in}{1.385256in}}%
\pgfpathcurveto{\pgfqpoint{1.189169in}{1.381349in}}{\pgfqpoint{1.186974in}{1.376049in}}{\pgfqpoint{1.186974in}{1.370524in}}%
\pgfpathcurveto{\pgfqpoint{1.186974in}{1.364999in}}{\pgfqpoint{1.189169in}{1.359700in}}{\pgfqpoint{1.193075in}{1.355793in}}%
\pgfpathcurveto{\pgfqpoint{1.196982in}{1.351886in}}{\pgfqpoint{1.202282in}{1.349691in}}{\pgfqpoint{1.207807in}{1.349691in}}%
\pgfpathclose%
\pgfusepath{stroke,fill}%
\end{pgfscope}%
\begin{pgfscope}%
\pgfpathrectangle{\pgfqpoint{0.403630in}{0.251348in}}{\pgfqpoint{1.395000in}{1.359000in}}%
\pgfusepath{clip}%
\pgfsetbuttcap%
\pgfsetroundjoin%
\definecolor{currentfill}{rgb}{0.686275,0.352941,0.313725}%
\pgfsetfillcolor{currentfill}%
\pgfsetlinewidth{0.501875pt}%
\definecolor{currentstroke}{rgb}{0.686275,0.352941,0.313725}%
\pgfsetstrokecolor{currentstroke}%
\pgfsetdash{}{0pt}%
\pgfpathmoveto{\pgfqpoint{1.109336in}{0.422373in}}%
\pgfpathcurveto{\pgfqpoint{1.114861in}{0.422373in}}{\pgfqpoint{1.120161in}{0.424568in}}{\pgfqpoint{1.124068in}{0.428475in}}%
\pgfpathcurveto{\pgfqpoint{1.127975in}{0.432382in}}{\pgfqpoint{1.130170in}{0.437681in}}{\pgfqpoint{1.130170in}{0.443206in}}%
\pgfpathcurveto{\pgfqpoint{1.130170in}{0.448732in}}{\pgfqpoint{1.127975in}{0.454031in}}{\pgfqpoint{1.124068in}{0.457938in}}%
\pgfpathcurveto{\pgfqpoint{1.120161in}{0.461845in}}{\pgfqpoint{1.114861in}{0.464040in}}{\pgfqpoint{1.109336in}{0.464040in}}%
\pgfpathcurveto{\pgfqpoint{1.103811in}{0.464040in}}{\pgfqpoint{1.098512in}{0.461845in}}{\pgfqpoint{1.094605in}{0.457938in}}%
\pgfpathcurveto{\pgfqpoint{1.090698in}{0.454031in}}{\pgfqpoint{1.088503in}{0.448732in}}{\pgfqpoint{1.088503in}{0.443206in}}%
\pgfpathcurveto{\pgfqpoint{1.088503in}{0.437681in}}{\pgfqpoint{1.090698in}{0.432382in}}{\pgfqpoint{1.094605in}{0.428475in}}%
\pgfpathcurveto{\pgfqpoint{1.098512in}{0.424568in}}{\pgfqpoint{1.103811in}{0.422373in}}{\pgfqpoint{1.109336in}{0.422373in}}%
\pgfpathclose%
\pgfusepath{stroke,fill}%
\end{pgfscope}%
\begin{pgfscope}%
\pgfpathrectangle{\pgfqpoint{0.403630in}{0.251348in}}{\pgfqpoint{1.395000in}{1.359000in}}%
\pgfusepath{clip}%
\pgfsetbuttcap%
\pgfsetroundjoin%
\definecolor{currentfill}{rgb}{0.686275,0.352941,0.313725}%
\pgfsetfillcolor{currentfill}%
\pgfsetlinewidth{0.501875pt}%
\definecolor{currentstroke}{rgb}{0.686275,0.352941,0.313725}%
\pgfsetstrokecolor{currentstroke}%
\pgfsetdash{}{0pt}%
\pgfpathmoveto{\pgfqpoint{1.448513in}{0.608903in}}%
\pgfpathcurveto{\pgfqpoint{1.454038in}{0.608903in}}{\pgfqpoint{1.459337in}{0.611098in}}{\pgfqpoint{1.463244in}{0.615004in}}%
\pgfpathcurveto{\pgfqpoint{1.467151in}{0.618911in}}{\pgfqpoint{1.469346in}{0.624211in}}{\pgfqpoint{1.469346in}{0.629736in}}%
\pgfpathcurveto{\pgfqpoint{1.469346in}{0.635261in}}{\pgfqpoint{1.467151in}{0.640560in}}{\pgfqpoint{1.463244in}{0.644467in}}%
\pgfpathcurveto{\pgfqpoint{1.459337in}{0.648374in}}{\pgfqpoint{1.454038in}{0.650569in}}{\pgfqpoint{1.448513in}{0.650569in}}%
\pgfpathcurveto{\pgfqpoint{1.442988in}{0.650569in}}{\pgfqpoint{1.437688in}{0.648374in}}{\pgfqpoint{1.433781in}{0.644467in}}%
\pgfpathcurveto{\pgfqpoint{1.429875in}{0.640560in}}{\pgfqpoint{1.427679in}{0.635261in}}{\pgfqpoint{1.427679in}{0.629736in}}%
\pgfpathcurveto{\pgfqpoint{1.427679in}{0.624211in}}{\pgfqpoint{1.429875in}{0.618911in}}{\pgfqpoint{1.433781in}{0.615004in}}%
\pgfpathcurveto{\pgfqpoint{1.437688in}{0.611098in}}{\pgfqpoint{1.442988in}{0.608903in}}{\pgfqpoint{1.448513in}{0.608903in}}%
\pgfpathclose%
\pgfusepath{stroke,fill}%
\end{pgfscope}%
\begin{pgfscope}%
\pgfpathrectangle{\pgfqpoint{0.403630in}{0.251348in}}{\pgfqpoint{1.395000in}{1.359000in}}%
\pgfusepath{clip}%
\pgfsetbuttcap%
\pgfsetroundjoin%
\definecolor{currentfill}{rgb}{0.686275,0.352941,0.313725}%
\pgfsetfillcolor{currentfill}%
\pgfsetlinewidth{0.501875pt}%
\definecolor{currentstroke}{rgb}{0.686275,0.352941,0.313725}%
\pgfsetstrokecolor{currentstroke}%
\pgfsetdash{}{0pt}%
\pgfpathmoveto{\pgfqpoint{1.443042in}{0.587585in}}%
\pgfpathcurveto{\pgfqpoint{1.448567in}{0.587585in}}{\pgfqpoint{1.453867in}{0.589780in}}{\pgfqpoint{1.457774in}{0.593687in}}%
\pgfpathcurveto{\pgfqpoint{1.461680in}{0.597594in}}{\pgfqpoint{1.463876in}{0.602893in}}{\pgfqpoint{1.463876in}{0.608418in}}%
\pgfpathcurveto{\pgfqpoint{1.463876in}{0.613943in}}{\pgfqpoint{1.461680in}{0.619243in}}{\pgfqpoint{1.457774in}{0.623150in}}%
\pgfpathcurveto{\pgfqpoint{1.453867in}{0.627056in}}{\pgfqpoint{1.448567in}{0.629252in}}{\pgfqpoint{1.443042in}{0.629252in}}%
\pgfpathcurveto{\pgfqpoint{1.437517in}{0.629252in}}{\pgfqpoint{1.432218in}{0.627056in}}{\pgfqpoint{1.428311in}{0.623150in}}%
\pgfpathcurveto{\pgfqpoint{1.424404in}{0.619243in}}{\pgfqpoint{1.422209in}{0.613943in}}{\pgfqpoint{1.422209in}{0.608418in}}%
\pgfpathcurveto{\pgfqpoint{1.422209in}{0.602893in}}{\pgfqpoint{1.424404in}{0.597594in}}{\pgfqpoint{1.428311in}{0.593687in}}%
\pgfpathcurveto{\pgfqpoint{1.432218in}{0.589780in}}{\pgfqpoint{1.437517in}{0.587585in}}{\pgfqpoint{1.443042in}{0.587585in}}%
\pgfpathclose%
\pgfusepath{stroke,fill}%
\end{pgfscope}%
\begin{pgfscope}%
\pgfpathrectangle{\pgfqpoint{0.403630in}{0.251348in}}{\pgfqpoint{1.395000in}{1.359000in}}%
\pgfusepath{clip}%
\pgfsetbuttcap%
\pgfsetroundjoin%
\definecolor{currentfill}{rgb}{0.686275,0.352941,0.313725}%
\pgfsetfillcolor{currentfill}%
\pgfsetlinewidth{0.501875pt}%
\definecolor{currentstroke}{rgb}{0.686275,0.352941,0.313725}%
\pgfsetstrokecolor{currentstroke}%
\pgfsetdash{}{0pt}%
\pgfpathmoveto{\pgfqpoint{1.470395in}{0.582255in}}%
\pgfpathcurveto{\pgfqpoint{1.475920in}{0.582255in}}{\pgfqpoint{1.481220in}{0.584451in}}{\pgfqpoint{1.485127in}{0.588357in}}%
\pgfpathcurveto{\pgfqpoint{1.489033in}{0.592264in}}{\pgfqpoint{1.491228in}{0.597564in}}{\pgfqpoint{1.491228in}{0.603089in}}%
\pgfpathcurveto{\pgfqpoint{1.491228in}{0.608614in}}{\pgfqpoint{1.489033in}{0.613913in}}{\pgfqpoint{1.485127in}{0.617820in}}%
\pgfpathcurveto{\pgfqpoint{1.481220in}{0.621727in}}{\pgfqpoint{1.475920in}{0.623922in}}{\pgfqpoint{1.470395in}{0.623922in}}%
\pgfpathcurveto{\pgfqpoint{1.464870in}{0.623922in}}{\pgfqpoint{1.459571in}{0.621727in}}{\pgfqpoint{1.455664in}{0.617820in}}%
\pgfpathcurveto{\pgfqpoint{1.451757in}{0.613913in}}{\pgfqpoint{1.449562in}{0.608614in}}{\pgfqpoint{1.449562in}{0.603089in}}%
\pgfpathcurveto{\pgfqpoint{1.449562in}{0.597564in}}{\pgfqpoint{1.451757in}{0.592264in}}{\pgfqpoint{1.455664in}{0.588357in}}%
\pgfpathcurveto{\pgfqpoint{1.459571in}{0.584451in}}{\pgfqpoint{1.464870in}{0.582255in}}{\pgfqpoint{1.470395in}{0.582255in}}%
\pgfpathclose%
\pgfusepath{stroke,fill}%
\end{pgfscope}%
\begin{pgfscope}%
\pgfpathrectangle{\pgfqpoint{0.403630in}{0.251348in}}{\pgfqpoint{1.395000in}{1.359000in}}%
\pgfusepath{clip}%
\pgfsetbuttcap%
\pgfsetroundjoin%
\definecolor{currentfill}{rgb}{0.686275,0.352941,0.313725}%
\pgfsetfillcolor{currentfill}%
\pgfsetlinewidth{0.501875pt}%
\definecolor{currentstroke}{rgb}{0.686275,0.352941,0.313725}%
\pgfsetstrokecolor{currentstroke}%
\pgfsetdash{}{0pt}%
\pgfpathmoveto{\pgfqpoint{0.682630in}{0.768785in}}%
\pgfpathcurveto{\pgfqpoint{0.688155in}{0.768785in}}{\pgfqpoint{0.693455in}{0.770980in}}{\pgfqpoint{0.697362in}{0.774887in}}%
\pgfpathcurveto{\pgfqpoint{0.701269in}{0.778794in}}{\pgfqpoint{0.703464in}{0.784093in}}{\pgfqpoint{0.703464in}{0.789618in}}%
\pgfpathcurveto{\pgfqpoint{0.703464in}{0.795143in}}{\pgfqpoint{0.701269in}{0.800443in}}{\pgfqpoint{0.697362in}{0.804350in}}%
\pgfpathcurveto{\pgfqpoint{0.693455in}{0.808256in}}{\pgfqpoint{0.688155in}{0.810452in}}{\pgfqpoint{0.682630in}{0.810452in}}%
\pgfpathcurveto{\pgfqpoint{0.677105in}{0.810452in}}{\pgfqpoint{0.671806in}{0.808256in}}{\pgfqpoint{0.667899in}{0.804350in}}%
\pgfpathcurveto{\pgfqpoint{0.663992in}{0.800443in}}{\pgfqpoint{0.661797in}{0.795143in}}{\pgfqpoint{0.661797in}{0.789618in}}%
\pgfpathcurveto{\pgfqpoint{0.661797in}{0.784093in}}{\pgfqpoint{0.663992in}{0.778794in}}{\pgfqpoint{0.667899in}{0.774887in}}%
\pgfpathcurveto{\pgfqpoint{0.671806in}{0.770980in}}{\pgfqpoint{0.677105in}{0.768785in}}{\pgfqpoint{0.682630in}{0.768785in}}%
\pgfpathclose%
\pgfusepath{stroke,fill}%
\end{pgfscope}%
\begin{pgfscope}%
\pgfpathrectangle{\pgfqpoint{0.403630in}{0.251348in}}{\pgfqpoint{1.395000in}{1.359000in}}%
\pgfusepath{clip}%
\pgfsetbuttcap%
\pgfsetroundjoin%
\definecolor{currentfill}{rgb}{0.686275,0.352941,0.313725}%
\pgfsetfillcolor{currentfill}%
\pgfsetlinewidth{0.501875pt}%
\definecolor{currentstroke}{rgb}{0.686275,0.352941,0.313725}%
\pgfsetstrokecolor{currentstroke}%
\pgfsetdash{}{0pt}%
\pgfpathmoveto{\pgfqpoint{1.486807in}{1.131185in}}%
\pgfpathcurveto{\pgfqpoint{1.492332in}{1.131185in}}{\pgfqpoint{1.497631in}{1.133380in}}{\pgfqpoint{1.501538in}{1.137287in}}%
\pgfpathcurveto{\pgfqpoint{1.505445in}{1.141194in}}{\pgfqpoint{1.507640in}{1.146493in}}{\pgfqpoint{1.507640in}{1.152018in}}%
\pgfpathcurveto{\pgfqpoint{1.507640in}{1.157543in}}{\pgfqpoint{1.505445in}{1.162843in}}{\pgfqpoint{1.501538in}{1.166750in}}%
\pgfpathcurveto{\pgfqpoint{1.497631in}{1.170656in}}{\pgfqpoint{1.492332in}{1.172852in}}{\pgfqpoint{1.486807in}{1.172852in}}%
\pgfpathcurveto{\pgfqpoint{1.481282in}{1.172852in}}{\pgfqpoint{1.475982in}{1.170656in}}{\pgfqpoint{1.472075in}{1.166750in}}%
\pgfpathcurveto{\pgfqpoint{1.468169in}{1.162843in}}{\pgfqpoint{1.465974in}{1.157543in}}{\pgfqpoint{1.465974in}{1.152018in}}%
\pgfpathcurveto{\pgfqpoint{1.465974in}{1.146493in}}{\pgfqpoint{1.468169in}{1.141194in}}{\pgfqpoint{1.472075in}{1.137287in}}%
\pgfpathcurveto{\pgfqpoint{1.475982in}{1.133380in}}{\pgfqpoint{1.481282in}{1.131185in}}{\pgfqpoint{1.486807in}{1.131185in}}%
\pgfpathclose%
\pgfusepath{stroke,fill}%
\end{pgfscope}%
\begin{pgfscope}%
\pgfpathrectangle{\pgfqpoint{0.403630in}{0.251348in}}{\pgfqpoint{1.395000in}{1.359000in}}%
\pgfusepath{clip}%
\pgfsetbuttcap%
\pgfsetroundjoin%
\definecolor{currentfill}{rgb}{0.686275,0.352941,0.313725}%
\pgfsetfillcolor{currentfill}%
\pgfsetlinewidth{0.501875pt}%
\definecolor{currentstroke}{rgb}{0.686275,0.352941,0.313725}%
\pgfsetstrokecolor{currentstroke}%
\pgfsetdash{}{0pt}%
\pgfpathmoveto{\pgfqpoint{1.530572in}{1.157832in}}%
\pgfpathcurveto{\pgfqpoint{1.536097in}{1.157832in}}{\pgfqpoint{1.541396in}{1.160027in}}{\pgfqpoint{1.545303in}{1.163934in}}%
\pgfpathcurveto{\pgfqpoint{1.549210in}{1.167841in}}{\pgfqpoint{1.551405in}{1.173140in}}{\pgfqpoint{1.551405in}{1.178665in}}%
\pgfpathcurveto{\pgfqpoint{1.551405in}{1.184190in}}{\pgfqpoint{1.549210in}{1.189490in}}{\pgfqpoint{1.545303in}{1.193397in}}%
\pgfpathcurveto{\pgfqpoint{1.541396in}{1.197303in}}{\pgfqpoint{1.536097in}{1.199499in}}{\pgfqpoint{1.530572in}{1.199499in}}%
\pgfpathcurveto{\pgfqpoint{1.525047in}{1.199499in}}{\pgfqpoint{1.519747in}{1.197303in}}{\pgfqpoint{1.515840in}{1.193397in}}%
\pgfpathcurveto{\pgfqpoint{1.511933in}{1.189490in}}{\pgfqpoint{1.509738in}{1.184190in}}{\pgfqpoint{1.509738in}{1.178665in}}%
\pgfpathcurveto{\pgfqpoint{1.509738in}{1.173140in}}{\pgfqpoint{1.511933in}{1.167841in}}{\pgfqpoint{1.515840in}{1.163934in}}%
\pgfpathcurveto{\pgfqpoint{1.519747in}{1.160027in}}{\pgfqpoint{1.525047in}{1.157832in}}{\pgfqpoint{1.530572in}{1.157832in}}%
\pgfpathclose%
\pgfusepath{stroke,fill}%
\end{pgfscope}%
\begin{pgfscope}%
\pgfpathrectangle{\pgfqpoint{0.403630in}{0.251348in}}{\pgfqpoint{1.395000in}{1.359000in}}%
\pgfusepath{clip}%
\pgfsetbuttcap%
\pgfsetroundjoin%
\definecolor{currentfill}{rgb}{0.686275,0.352941,0.313725}%
\pgfsetfillcolor{currentfill}%
\pgfsetlinewidth{0.501875pt}%
\definecolor{currentstroke}{rgb}{0.686275,0.352941,0.313725}%
\pgfsetstrokecolor{currentstroke}%
\pgfsetdash{}{0pt}%
\pgfpathmoveto{\pgfqpoint{1.071042in}{0.475667in}}%
\pgfpathcurveto{\pgfqpoint{1.076567in}{0.475667in}}{\pgfqpoint{1.081867in}{0.477862in}}{\pgfqpoint{1.085774in}{0.481769in}}%
\pgfpathcurveto{\pgfqpoint{1.089680in}{0.485676in}}{\pgfqpoint{1.091876in}{0.490976in}}{\pgfqpoint{1.091876in}{0.496501in}}%
\pgfpathcurveto{\pgfqpoint{1.091876in}{0.502026in}}{\pgfqpoint{1.089680in}{0.507325in}}{\pgfqpoint{1.085774in}{0.511232in}}%
\pgfpathcurveto{\pgfqpoint{1.081867in}{0.515139in}}{\pgfqpoint{1.076567in}{0.517334in}}{\pgfqpoint{1.071042in}{0.517334in}}%
\pgfpathcurveto{\pgfqpoint{1.065517in}{0.517334in}}{\pgfqpoint{1.060218in}{0.515139in}}{\pgfqpoint{1.056311in}{0.511232in}}%
\pgfpathcurveto{\pgfqpoint{1.052404in}{0.507325in}}{\pgfqpoint{1.050209in}{0.502026in}}{\pgfqpoint{1.050209in}{0.496501in}}%
\pgfpathcurveto{\pgfqpoint{1.050209in}{0.490976in}}{\pgfqpoint{1.052404in}{0.485676in}}{\pgfqpoint{1.056311in}{0.481769in}}%
\pgfpathcurveto{\pgfqpoint{1.060218in}{0.477862in}}{\pgfqpoint{1.065517in}{0.475667in}}{\pgfqpoint{1.071042in}{0.475667in}}%
\pgfpathclose%
\pgfusepath{stroke,fill}%
\end{pgfscope}%
\begin{pgfscope}%
\pgfpathrectangle{\pgfqpoint{0.403630in}{0.251348in}}{\pgfqpoint{1.395000in}{1.359000in}}%
\pgfusepath{clip}%
\pgfsetbuttcap%
\pgfsetroundjoin%
\definecolor{currentfill}{rgb}{0.686275,0.352941,0.313725}%
\pgfsetfillcolor{currentfill}%
\pgfsetlinewidth{0.501875pt}%
\definecolor{currentstroke}{rgb}{0.686275,0.352941,0.313725}%
\pgfsetstrokecolor{currentstroke}%
\pgfsetdash{}{0pt}%
\pgfpathmoveto{\pgfqpoint{0.616983in}{0.747467in}}%
\pgfpathcurveto{\pgfqpoint{0.622508in}{0.747467in}}{\pgfqpoint{0.627808in}{0.749662in}}{\pgfqpoint{0.631715in}{0.753569in}}%
\pgfpathcurveto{\pgfqpoint{0.635622in}{0.757476in}}{\pgfqpoint{0.637817in}{0.762776in}}{\pgfqpoint{0.637817in}{0.768301in}}%
\pgfpathcurveto{\pgfqpoint{0.637817in}{0.773826in}}{\pgfqpoint{0.635622in}{0.779125in}}{\pgfqpoint{0.631715in}{0.783032in}}%
\pgfpathcurveto{\pgfqpoint{0.627808in}{0.786939in}}{\pgfqpoint{0.622508in}{0.789134in}}{\pgfqpoint{0.616983in}{0.789134in}}%
\pgfpathcurveto{\pgfqpoint{0.611458in}{0.789134in}}{\pgfqpoint{0.606159in}{0.786939in}}{\pgfqpoint{0.602252in}{0.783032in}}%
\pgfpathcurveto{\pgfqpoint{0.598345in}{0.779125in}}{\pgfqpoint{0.596150in}{0.773826in}}{\pgfqpoint{0.596150in}{0.768301in}}%
\pgfpathcurveto{\pgfqpoint{0.596150in}{0.762776in}}{\pgfqpoint{0.598345in}{0.757476in}}{\pgfqpoint{0.602252in}{0.753569in}}%
\pgfpathcurveto{\pgfqpoint{0.606159in}{0.749662in}}{\pgfqpoint{0.611458in}{0.747467in}}{\pgfqpoint{0.616983in}{0.747467in}}%
\pgfpathclose%
\pgfusepath{stroke,fill}%
\end{pgfscope}%
\begin{pgfscope}%
\pgfpathrectangle{\pgfqpoint{0.403630in}{0.251348in}}{\pgfqpoint{1.395000in}{1.359000in}}%
\pgfusepath{clip}%
\pgfsetbuttcap%
\pgfsetroundjoin%
\definecolor{currentfill}{rgb}{0.686275,0.352941,0.313725}%
\pgfsetfillcolor{currentfill}%
\pgfsetlinewidth{0.501875pt}%
\definecolor{currentstroke}{rgb}{0.686275,0.352941,0.313725}%
\pgfsetstrokecolor{currentstroke}%
\pgfsetdash{}{0pt}%
\pgfpathmoveto{\pgfqpoint{0.622454in}{0.768785in}}%
\pgfpathcurveto{\pgfqpoint{0.627979in}{0.768785in}}{\pgfqpoint{0.633279in}{0.770980in}}{\pgfqpoint{0.637185in}{0.774887in}}%
\pgfpathcurveto{\pgfqpoint{0.641092in}{0.778794in}}{\pgfqpoint{0.643287in}{0.784093in}}{\pgfqpoint{0.643287in}{0.789618in}}%
\pgfpathcurveto{\pgfqpoint{0.643287in}{0.795143in}}{\pgfqpoint{0.641092in}{0.800443in}}{\pgfqpoint{0.637185in}{0.804350in}}%
\pgfpathcurveto{\pgfqpoint{0.633279in}{0.808256in}}{\pgfqpoint{0.627979in}{0.810452in}}{\pgfqpoint{0.622454in}{0.810452in}}%
\pgfpathcurveto{\pgfqpoint{0.616929in}{0.810452in}}{\pgfqpoint{0.611629in}{0.808256in}}{\pgfqpoint{0.607723in}{0.804350in}}%
\pgfpathcurveto{\pgfqpoint{0.603816in}{0.800443in}}{\pgfqpoint{0.601621in}{0.795143in}}{\pgfqpoint{0.601621in}{0.789618in}}%
\pgfpathcurveto{\pgfqpoint{0.601621in}{0.784093in}}{\pgfqpoint{0.603816in}{0.778794in}}{\pgfqpoint{0.607723in}{0.774887in}}%
\pgfpathcurveto{\pgfqpoint{0.611629in}{0.770980in}}{\pgfqpoint{0.616929in}{0.768785in}}{\pgfqpoint{0.622454in}{0.768785in}}%
\pgfpathclose%
\pgfusepath{stroke,fill}%
\end{pgfscope}%
\begin{pgfscope}%
\pgfpathrectangle{\pgfqpoint{0.403630in}{0.251348in}}{\pgfqpoint{1.395000in}{1.359000in}}%
\pgfusepath{clip}%
\pgfsetbuttcap%
\pgfsetroundjoin%
\definecolor{currentfill}{rgb}{0.686275,0.352941,0.313725}%
\pgfsetfillcolor{currentfill}%
\pgfsetlinewidth{0.501875pt}%
\definecolor{currentstroke}{rgb}{0.686275,0.352941,0.313725}%
\pgfsetstrokecolor{currentstroke}%
\pgfsetdash{}{0pt}%
\pgfpathmoveto{\pgfqpoint{0.699042in}{1.093879in}}%
\pgfpathcurveto{\pgfqpoint{0.704567in}{1.093879in}}{\pgfqpoint{0.709867in}{1.096074in}}{\pgfqpoint{0.713774in}{1.099981in}}%
\pgfpathcurveto{\pgfqpoint{0.717680in}{1.103888in}}{\pgfqpoint{0.719876in}{1.109187in}}{\pgfqpoint{0.719876in}{1.114712in}}%
\pgfpathcurveto{\pgfqpoint{0.719876in}{1.120237in}}{\pgfqpoint{0.717680in}{1.125537in}}{\pgfqpoint{0.713774in}{1.129444in}}%
\pgfpathcurveto{\pgfqpoint{0.709867in}{1.133351in}}{\pgfqpoint{0.704567in}{1.135546in}}{\pgfqpoint{0.699042in}{1.135546in}}%
\pgfpathcurveto{\pgfqpoint{0.693517in}{1.135546in}}{\pgfqpoint{0.688218in}{1.133351in}}{\pgfqpoint{0.684311in}{1.129444in}}%
\pgfpathcurveto{\pgfqpoint{0.680404in}{1.125537in}}{\pgfqpoint{0.678209in}{1.120237in}}{\pgfqpoint{0.678209in}{1.114712in}}%
\pgfpathcurveto{\pgfqpoint{0.678209in}{1.109187in}}{\pgfqpoint{0.680404in}{1.103888in}}{\pgfqpoint{0.684311in}{1.099981in}}%
\pgfpathcurveto{\pgfqpoint{0.688218in}{1.096074in}}{\pgfqpoint{0.693517in}{1.093879in}}{\pgfqpoint{0.699042in}{1.093879in}}%
\pgfpathclose%
\pgfusepath{stroke,fill}%
\end{pgfscope}%
\begin{pgfscope}%
\pgfpathrectangle{\pgfqpoint{0.403630in}{0.251348in}}{\pgfqpoint{1.395000in}{1.359000in}}%
\pgfusepath{clip}%
\pgfsetbuttcap%
\pgfsetroundjoin%
\definecolor{currentfill}{rgb}{0.686275,0.352941,0.313725}%
\pgfsetfillcolor{currentfill}%
\pgfsetlinewidth{0.501875pt}%
\definecolor{currentstroke}{rgb}{0.686275,0.352941,0.313725}%
\pgfsetstrokecolor{currentstroke}%
\pgfsetdash{}{0pt}%
\pgfpathmoveto{\pgfqpoint{0.627925in}{1.072561in}}%
\pgfpathcurveto{\pgfqpoint{0.633450in}{1.072561in}}{\pgfqpoint{0.638749in}{1.074756in}}{\pgfqpoint{0.642656in}{1.078663in}}%
\pgfpathcurveto{\pgfqpoint{0.646563in}{1.082570in}}{\pgfqpoint{0.648758in}{1.087870in}}{\pgfqpoint{0.648758in}{1.093395in}}%
\pgfpathcurveto{\pgfqpoint{0.648758in}{1.098920in}}{\pgfqpoint{0.646563in}{1.104219in}}{\pgfqpoint{0.642656in}{1.108126in}}%
\pgfpathcurveto{\pgfqpoint{0.638749in}{1.112033in}}{\pgfqpoint{0.633450in}{1.114228in}}{\pgfqpoint{0.627925in}{1.114228in}}%
\pgfpathcurveto{\pgfqpoint{0.622399in}{1.114228in}}{\pgfqpoint{0.617100in}{1.112033in}}{\pgfqpoint{0.613193in}{1.108126in}}%
\pgfpathcurveto{\pgfqpoint{0.609286in}{1.104219in}}{\pgfqpoint{0.607091in}{1.098920in}}{\pgfqpoint{0.607091in}{1.093395in}}%
\pgfpathcurveto{\pgfqpoint{0.607091in}{1.087870in}}{\pgfqpoint{0.609286in}{1.082570in}}{\pgfqpoint{0.613193in}{1.078663in}}%
\pgfpathcurveto{\pgfqpoint{0.617100in}{1.074756in}}{\pgfqpoint{0.622399in}{1.072561in}}{\pgfqpoint{0.627925in}{1.072561in}}%
\pgfpathclose%
\pgfusepath{stroke,fill}%
\end{pgfscope}%
\begin{pgfscope}%
\pgfpathrectangle{\pgfqpoint{0.403630in}{0.251348in}}{\pgfqpoint{1.395000in}{1.359000in}}%
\pgfusepath{clip}%
\pgfsetbuttcap%
\pgfsetroundjoin%
\definecolor{currentfill}{rgb}{0.686275,0.352941,0.313725}%
\pgfsetfillcolor{currentfill}%
\pgfsetlinewidth{0.501875pt}%
\definecolor{currentstroke}{rgb}{0.686275,0.352941,0.313725}%
\pgfsetstrokecolor{currentstroke}%
\pgfsetdash{}{0pt}%
\pgfpathmoveto{\pgfqpoint{0.666219in}{0.715491in}}%
\pgfpathcurveto{\pgfqpoint{0.671744in}{0.715491in}}{\pgfqpoint{0.677043in}{0.717686in}}{\pgfqpoint{0.680950in}{0.721593in}}%
\pgfpathcurveto{\pgfqpoint{0.684857in}{0.725500in}}{\pgfqpoint{0.687052in}{0.730799in}}{\pgfqpoint{0.687052in}{0.736324in}}%
\pgfpathcurveto{\pgfqpoint{0.687052in}{0.741849in}}{\pgfqpoint{0.684857in}{0.747149in}}{\pgfqpoint{0.680950in}{0.751056in}}%
\pgfpathcurveto{\pgfqpoint{0.677043in}{0.754962in}}{\pgfqpoint{0.671744in}{0.757157in}}{\pgfqpoint{0.666219in}{0.757157in}}%
\pgfpathcurveto{\pgfqpoint{0.660694in}{0.757157in}}{\pgfqpoint{0.655394in}{0.754962in}}{\pgfqpoint{0.651487in}{0.751056in}}%
\pgfpathcurveto{\pgfqpoint{0.647580in}{0.747149in}}{\pgfqpoint{0.645385in}{0.741849in}}{\pgfqpoint{0.645385in}{0.736324in}}%
\pgfpathcurveto{\pgfqpoint{0.645385in}{0.730799in}}{\pgfqpoint{0.647580in}{0.725500in}}{\pgfqpoint{0.651487in}{0.721593in}}%
\pgfpathcurveto{\pgfqpoint{0.655394in}{0.717686in}}{\pgfqpoint{0.660694in}{0.715491in}}{\pgfqpoint{0.666219in}{0.715491in}}%
\pgfpathclose%
\pgfusepath{stroke,fill}%
\end{pgfscope}%
\begin{pgfscope}%
\pgfpathrectangle{\pgfqpoint{0.403630in}{0.251348in}}{\pgfqpoint{1.395000in}{1.359000in}}%
\pgfusepath{clip}%
\pgfsetbuttcap%
\pgfsetroundjoin%
\definecolor{currentfill}{rgb}{0.686275,0.352941,0.313725}%
\pgfsetfillcolor{currentfill}%
\pgfsetlinewidth{0.501875pt}%
\definecolor{currentstroke}{rgb}{0.686275,0.352941,0.313725}%
\pgfsetstrokecolor{currentstroke}%
\pgfsetdash{}{0pt}%
\pgfpathmoveto{\pgfqpoint{0.627925in}{0.800761in}}%
\pgfpathcurveto{\pgfqpoint{0.633450in}{0.800761in}}{\pgfqpoint{0.638749in}{0.802956in}}{\pgfqpoint{0.642656in}{0.806863in}}%
\pgfpathcurveto{\pgfqpoint{0.646563in}{0.810770in}}{\pgfqpoint{0.648758in}{0.816070in}}{\pgfqpoint{0.648758in}{0.821595in}}%
\pgfpathcurveto{\pgfqpoint{0.648758in}{0.827120in}}{\pgfqpoint{0.646563in}{0.832419in}}{\pgfqpoint{0.642656in}{0.836326in}}%
\pgfpathcurveto{\pgfqpoint{0.638749in}{0.840233in}}{\pgfqpoint{0.633450in}{0.842428in}}{\pgfqpoint{0.627925in}{0.842428in}}%
\pgfpathcurveto{\pgfqpoint{0.622399in}{0.842428in}}{\pgfqpoint{0.617100in}{0.840233in}}{\pgfqpoint{0.613193in}{0.836326in}}%
\pgfpathcurveto{\pgfqpoint{0.609286in}{0.832419in}}{\pgfqpoint{0.607091in}{0.827120in}}{\pgfqpoint{0.607091in}{0.821595in}}%
\pgfpathcurveto{\pgfqpoint{0.607091in}{0.816070in}}{\pgfqpoint{0.609286in}{0.810770in}}{\pgfqpoint{0.613193in}{0.806863in}}%
\pgfpathcurveto{\pgfqpoint{0.617100in}{0.802956in}}{\pgfqpoint{0.622399in}{0.800761in}}{\pgfqpoint{0.627925in}{0.800761in}}%
\pgfpathclose%
\pgfusepath{stroke,fill}%
\end{pgfscope}%
\begin{pgfscope}%
\pgfpathrectangle{\pgfqpoint{0.403630in}{0.251348in}}{\pgfqpoint{1.395000in}{1.359000in}}%
\pgfusepath{clip}%
\pgfsetbuttcap%
\pgfsetroundjoin%
\definecolor{currentfill}{rgb}{0.686275,0.352941,0.313725}%
\pgfsetfillcolor{currentfill}%
\pgfsetlinewidth{0.501875pt}%
\definecolor{currentstroke}{rgb}{0.686275,0.352941,0.313725}%
\pgfsetstrokecolor{currentstroke}%
\pgfsetdash{}{0pt}%
\pgfpathmoveto{\pgfqpoint{0.835807in}{0.496985in}}%
\pgfpathcurveto{\pgfqpoint{0.841332in}{0.496985in}}{\pgfqpoint{0.846631in}{0.499180in}}{\pgfqpoint{0.850538in}{0.503087in}}%
\pgfpathcurveto{\pgfqpoint{0.854445in}{0.506994in}}{\pgfqpoint{0.856640in}{0.512293in}}{\pgfqpoint{0.856640in}{0.517818in}}%
\pgfpathcurveto{\pgfqpoint{0.856640in}{0.523343in}}{\pgfqpoint{0.854445in}{0.528643in}}{\pgfqpoint{0.850538in}{0.532550in}}%
\pgfpathcurveto{\pgfqpoint{0.846631in}{0.536456in}}{\pgfqpoint{0.841332in}{0.538652in}}{\pgfqpoint{0.835807in}{0.538652in}}%
\pgfpathcurveto{\pgfqpoint{0.830282in}{0.538652in}}{\pgfqpoint{0.824982in}{0.536456in}}{\pgfqpoint{0.821075in}{0.532550in}}%
\pgfpathcurveto{\pgfqpoint{0.817169in}{0.528643in}}{\pgfqpoint{0.814974in}{0.523343in}}{\pgfqpoint{0.814974in}{0.517818in}}%
\pgfpathcurveto{\pgfqpoint{0.814974in}{0.512293in}}{\pgfqpoint{0.817169in}{0.506994in}}{\pgfqpoint{0.821075in}{0.503087in}}%
\pgfpathcurveto{\pgfqpoint{0.824982in}{0.499180in}}{\pgfqpoint{0.830282in}{0.496985in}}{\pgfqpoint{0.835807in}{0.496985in}}%
\pgfpathclose%
\pgfusepath{stroke,fill}%
\end{pgfscope}%
\begin{pgfscope}%
\pgfpathrectangle{\pgfqpoint{0.403630in}{0.251348in}}{\pgfqpoint{1.395000in}{1.359000in}}%
\pgfusepath{clip}%
\pgfsetbuttcap%
\pgfsetroundjoin%
\definecolor{currentfill}{rgb}{0.686275,0.352941,0.313725}%
\pgfsetfillcolor{currentfill}%
\pgfsetlinewidth{0.501875pt}%
\definecolor{currentstroke}{rgb}{0.686275,0.352941,0.313725}%
\pgfsetstrokecolor{currentstroke}%
\pgfsetdash{}{0pt}%
\pgfpathmoveto{\pgfqpoint{1.568866in}{0.822079in}}%
\pgfpathcurveto{\pgfqpoint{1.574391in}{0.822079in}}{\pgfqpoint{1.579690in}{0.824274in}}{\pgfqpoint{1.583597in}{0.828181in}}%
\pgfpathcurveto{\pgfqpoint{1.587504in}{0.832088in}}{\pgfqpoint{1.589699in}{0.837387in}}{\pgfqpoint{1.589699in}{0.842912in}}%
\pgfpathcurveto{\pgfqpoint{1.589699in}{0.848437in}}{\pgfqpoint{1.587504in}{0.853737in}}{\pgfqpoint{1.583597in}{0.857644in}}%
\pgfpathcurveto{\pgfqpoint{1.579690in}{0.861551in}}{\pgfqpoint{1.574391in}{0.863746in}}{\pgfqpoint{1.568866in}{0.863746in}}%
\pgfpathcurveto{\pgfqpoint{1.563341in}{0.863746in}}{\pgfqpoint{1.558041in}{0.861551in}}{\pgfqpoint{1.554134in}{0.857644in}}%
\pgfpathcurveto{\pgfqpoint{1.550228in}{0.853737in}}{\pgfqpoint{1.548032in}{0.848437in}}{\pgfqpoint{1.548032in}{0.842912in}}%
\pgfpathcurveto{\pgfqpoint{1.548032in}{0.837387in}}{\pgfqpoint{1.550228in}{0.832088in}}{\pgfqpoint{1.554134in}{0.828181in}}%
\pgfpathcurveto{\pgfqpoint{1.558041in}{0.824274in}}{\pgfqpoint{1.563341in}{0.822079in}}{\pgfqpoint{1.568866in}{0.822079in}}%
\pgfpathclose%
\pgfusepath{stroke,fill}%
\end{pgfscope}%
\begin{pgfscope}%
\pgfpathrectangle{\pgfqpoint{0.403630in}{0.251348in}}{\pgfqpoint{1.395000in}{1.359000in}}%
\pgfusepath{clip}%
\pgfsetbuttcap%
\pgfsetroundjoin%
\definecolor{currentfill}{rgb}{0.686275,0.352941,0.313725}%
\pgfsetfillcolor{currentfill}%
\pgfsetlinewidth{0.501875pt}%
\definecolor{currentstroke}{rgb}{0.686275,0.352941,0.313725}%
\pgfsetstrokecolor{currentstroke}%
\pgfsetdash{}{0pt}%
\pgfpathmoveto{\pgfqpoint{0.748277in}{1.184479in}}%
\pgfpathcurveto{\pgfqpoint{0.753803in}{1.184479in}}{\pgfqpoint{0.759102in}{1.186674in}}{\pgfqpoint{0.763009in}{1.190581in}}%
\pgfpathcurveto{\pgfqpoint{0.766916in}{1.194488in}}{\pgfqpoint{0.769111in}{1.199787in}}{\pgfqpoint{0.769111in}{1.205312in}}%
\pgfpathcurveto{\pgfqpoint{0.769111in}{1.210837in}}{\pgfqpoint{0.766916in}{1.216137in}}{\pgfqpoint{0.763009in}{1.220044in}}%
\pgfpathcurveto{\pgfqpoint{0.759102in}{1.223951in}}{\pgfqpoint{0.753803in}{1.226146in}}{\pgfqpoint{0.748277in}{1.226146in}}%
\pgfpathcurveto{\pgfqpoint{0.742752in}{1.226146in}}{\pgfqpoint{0.737453in}{1.223951in}}{\pgfqpoint{0.733546in}{1.220044in}}%
\pgfpathcurveto{\pgfqpoint{0.729639in}{1.216137in}}{\pgfqpoint{0.727444in}{1.210837in}}{\pgfqpoint{0.727444in}{1.205312in}}%
\pgfpathcurveto{\pgfqpoint{0.727444in}{1.199787in}}{\pgfqpoint{0.729639in}{1.194488in}}{\pgfqpoint{0.733546in}{1.190581in}}%
\pgfpathcurveto{\pgfqpoint{0.737453in}{1.186674in}}{\pgfqpoint{0.742752in}{1.184479in}}{\pgfqpoint{0.748277in}{1.184479in}}%
\pgfpathclose%
\pgfusepath{stroke,fill}%
\end{pgfscope}%
\begin{pgfscope}%
\pgfpathrectangle{\pgfqpoint{0.403630in}{0.251348in}}{\pgfqpoint{1.395000in}{1.359000in}}%
\pgfusepath{clip}%
\pgfsetbuttcap%
\pgfsetroundjoin%
\definecolor{currentfill}{rgb}{0.686275,0.352941,0.313725}%
\pgfsetfillcolor{currentfill}%
\pgfsetlinewidth{0.501875pt}%
\definecolor{currentstroke}{rgb}{0.686275,0.352941,0.313725}%
\pgfsetstrokecolor{currentstroke}%
\pgfsetdash{}{0pt}%
\pgfpathmoveto{\pgfqpoint{1.541513in}{1.029926in}}%
\pgfpathcurveto{\pgfqpoint{1.547038in}{1.029926in}}{\pgfqpoint{1.552337in}{1.032121in}}{\pgfqpoint{1.556244in}{1.036028in}}%
\pgfpathcurveto{\pgfqpoint{1.560151in}{1.039935in}}{\pgfqpoint{1.562346in}{1.045234in}}{\pgfqpoint{1.562346in}{1.050759in}}%
\pgfpathcurveto{\pgfqpoint{1.562346in}{1.056284in}}{\pgfqpoint{1.560151in}{1.061584in}}{\pgfqpoint{1.556244in}{1.065491in}}%
\pgfpathcurveto{\pgfqpoint{1.552337in}{1.069398in}}{\pgfqpoint{1.547038in}{1.071593in}}{\pgfqpoint{1.541513in}{1.071593in}}%
\pgfpathcurveto{\pgfqpoint{1.535988in}{1.071593in}}{\pgfqpoint{1.530688in}{1.069398in}}{\pgfqpoint{1.526781in}{1.065491in}}%
\pgfpathcurveto{\pgfqpoint{1.522875in}{1.061584in}}{\pgfqpoint{1.520679in}{1.056284in}}{\pgfqpoint{1.520679in}{1.050759in}}%
\pgfpathcurveto{\pgfqpoint{1.520679in}{1.045234in}}{\pgfqpoint{1.522875in}{1.039935in}}{\pgfqpoint{1.526781in}{1.036028in}}%
\pgfpathcurveto{\pgfqpoint{1.530688in}{1.032121in}}{\pgfqpoint{1.535988in}{1.029926in}}{\pgfqpoint{1.541513in}{1.029926in}}%
\pgfpathclose%
\pgfusepath{stroke,fill}%
\end{pgfscope}%
\begin{pgfscope}%
\pgfpathrectangle{\pgfqpoint{0.403630in}{0.251348in}}{\pgfqpoint{1.395000in}{1.359000in}}%
\pgfusepath{clip}%
\pgfsetbuttcap%
\pgfsetroundjoin%
\definecolor{currentfill}{rgb}{0.686275,0.352941,0.313725}%
\pgfsetfillcolor{currentfill}%
\pgfsetlinewidth{0.501875pt}%
\definecolor{currentstroke}{rgb}{0.686275,0.352941,0.313725}%
\pgfsetstrokecolor{currentstroke}%
\pgfsetdash{}{0pt}%
\pgfpathmoveto{\pgfqpoint{1.541513in}{1.179150in}}%
\pgfpathcurveto{\pgfqpoint{1.547038in}{1.179150in}}{\pgfqpoint{1.552337in}{1.181345in}}{\pgfqpoint{1.556244in}{1.185252in}}%
\pgfpathcurveto{\pgfqpoint{1.560151in}{1.189158in}}{\pgfqpoint{1.562346in}{1.194458in}}{\pgfqpoint{1.562346in}{1.199983in}}%
\pgfpathcurveto{\pgfqpoint{1.562346in}{1.205508in}}{\pgfqpoint{1.560151in}{1.210808in}}{\pgfqpoint{1.556244in}{1.214714in}}%
\pgfpathcurveto{\pgfqpoint{1.552337in}{1.218621in}}{\pgfqpoint{1.547038in}{1.220816in}}{\pgfqpoint{1.541513in}{1.220816in}}%
\pgfpathcurveto{\pgfqpoint{1.535988in}{1.220816in}}{\pgfqpoint{1.530688in}{1.218621in}}{\pgfqpoint{1.526781in}{1.214714in}}%
\pgfpathcurveto{\pgfqpoint{1.522875in}{1.210808in}}{\pgfqpoint{1.520679in}{1.205508in}}{\pgfqpoint{1.520679in}{1.199983in}}%
\pgfpathcurveto{\pgfqpoint{1.520679in}{1.194458in}}{\pgfqpoint{1.522875in}{1.189158in}}{\pgfqpoint{1.526781in}{1.185252in}}%
\pgfpathcurveto{\pgfqpoint{1.530688in}{1.181345in}}{\pgfqpoint{1.535988in}{1.179150in}}{\pgfqpoint{1.541513in}{1.179150in}}%
\pgfpathclose%
\pgfusepath{stroke,fill}%
\end{pgfscope}%
\begin{pgfscope}%
\pgfpathrectangle{\pgfqpoint{0.403630in}{0.251348in}}{\pgfqpoint{1.395000in}{1.359000in}}%
\pgfusepath{clip}%
\pgfsetbuttcap%
\pgfsetroundjoin%
\definecolor{currentfill}{rgb}{0.686275,0.352941,0.313725}%
\pgfsetfillcolor{currentfill}%
\pgfsetlinewidth{0.501875pt}%
\definecolor{currentstroke}{rgb}{0.686275,0.352941,0.313725}%
\pgfsetstrokecolor{currentstroke}%
\pgfsetdash{}{0pt}%
\pgfpathmoveto{\pgfqpoint{0.644336in}{0.939326in}}%
\pgfpathcurveto{\pgfqpoint{0.649861in}{0.939326in}}{\pgfqpoint{0.655161in}{0.941521in}}{\pgfqpoint{0.659068in}{0.945428in}}%
\pgfpathcurveto{\pgfqpoint{0.662975in}{0.949335in}}{\pgfqpoint{0.665170in}{0.954634in}}{\pgfqpoint{0.665170in}{0.960159in}}%
\pgfpathcurveto{\pgfqpoint{0.665170in}{0.965684in}}{\pgfqpoint{0.662975in}{0.970984in}}{\pgfqpoint{0.659068in}{0.974891in}}%
\pgfpathcurveto{\pgfqpoint{0.655161in}{0.978798in}}{\pgfqpoint{0.649861in}{0.980993in}}{\pgfqpoint{0.644336in}{0.980993in}}%
\pgfpathcurveto{\pgfqpoint{0.638811in}{0.980993in}}{\pgfqpoint{0.633512in}{0.978798in}}{\pgfqpoint{0.629605in}{0.974891in}}%
\pgfpathcurveto{\pgfqpoint{0.625698in}{0.970984in}}{\pgfqpoint{0.623503in}{0.965684in}}{\pgfqpoint{0.623503in}{0.960159in}}%
\pgfpathcurveto{\pgfqpoint{0.623503in}{0.954634in}}{\pgfqpoint{0.625698in}{0.949335in}}{\pgfqpoint{0.629605in}{0.945428in}}%
\pgfpathcurveto{\pgfqpoint{0.633512in}{0.941521in}}{\pgfqpoint{0.638811in}{0.939326in}}{\pgfqpoint{0.644336in}{0.939326in}}%
\pgfpathclose%
\pgfusepath{stroke,fill}%
\end{pgfscope}%
\begin{pgfscope}%
\pgfpathrectangle{\pgfqpoint{0.403630in}{0.251348in}}{\pgfqpoint{1.395000in}{1.359000in}}%
\pgfusepath{clip}%
\pgfsetbuttcap%
\pgfsetroundjoin%
\definecolor{currentfill}{rgb}{0.686275,0.352941,0.313725}%
\pgfsetfillcolor{currentfill}%
\pgfsetlinewidth{0.501875pt}%
\definecolor{currentstroke}{rgb}{0.686275,0.352941,0.313725}%
\pgfsetstrokecolor{currentstroke}%
\pgfsetdash{}{0pt}%
\pgfpathmoveto{\pgfqpoint{0.638866in}{0.918008in}}%
\pgfpathcurveto{\pgfqpoint{0.644391in}{0.918008in}}{\pgfqpoint{0.649690in}{0.920204in}}{\pgfqpoint{0.653597in}{0.924110in}}%
\pgfpathcurveto{\pgfqpoint{0.657504in}{0.928017in}}{\pgfqpoint{0.659699in}{0.933317in}}{\pgfqpoint{0.659699in}{0.938842in}}%
\pgfpathcurveto{\pgfqpoint{0.659699in}{0.944367in}}{\pgfqpoint{0.657504in}{0.949666in}}{\pgfqpoint{0.653597in}{0.953573in}}%
\pgfpathcurveto{\pgfqpoint{0.649690in}{0.957480in}}{\pgfqpoint{0.644391in}{0.959675in}}{\pgfqpoint{0.638866in}{0.959675in}}%
\pgfpathcurveto{\pgfqpoint{0.633341in}{0.959675in}}{\pgfqpoint{0.628041in}{0.957480in}}{\pgfqpoint{0.624134in}{0.953573in}}%
\pgfpathcurveto{\pgfqpoint{0.620228in}{0.949666in}}{\pgfqpoint{0.618032in}{0.944367in}}{\pgfqpoint{0.618032in}{0.938842in}}%
\pgfpathcurveto{\pgfqpoint{0.618032in}{0.933317in}}{\pgfqpoint{0.620228in}{0.928017in}}{\pgfqpoint{0.624134in}{0.924110in}}%
\pgfpathcurveto{\pgfqpoint{0.628041in}{0.920204in}}{\pgfqpoint{0.633341in}{0.918008in}}{\pgfqpoint{0.638866in}{0.918008in}}%
\pgfpathclose%
\pgfusepath{stroke,fill}%
\end{pgfscope}%
\begin{pgfscope}%
\pgfpathrectangle{\pgfqpoint{0.403630in}{0.251348in}}{\pgfqpoint{1.395000in}{1.359000in}}%
\pgfusepath{clip}%
\pgfsetbuttcap%
\pgfsetroundjoin%
\definecolor{currentfill}{rgb}{0.686275,0.352941,0.313725}%
\pgfsetfillcolor{currentfill}%
\pgfsetlinewidth{0.501875pt}%
\definecolor{currentstroke}{rgb}{0.686275,0.352941,0.313725}%
\pgfsetstrokecolor{currentstroke}%
\pgfsetdash{}{0pt}%
\pgfpathmoveto{\pgfqpoint{1.563395in}{0.752797in}}%
\pgfpathcurveto{\pgfqpoint{1.568920in}{0.752797in}}{\pgfqpoint{1.574220in}{0.754992in}}{\pgfqpoint{1.578127in}{0.758899in}}%
\pgfpathcurveto{\pgfqpoint{1.582033in}{0.762805in}}{\pgfqpoint{1.584228in}{0.768105in}}{\pgfqpoint{1.584228in}{0.773630in}}%
\pgfpathcurveto{\pgfqpoint{1.584228in}{0.779155in}}{\pgfqpoint{1.582033in}{0.784455in}}{\pgfqpoint{1.578127in}{0.788361in}}%
\pgfpathcurveto{\pgfqpoint{1.574220in}{0.792268in}}{\pgfqpoint{1.568920in}{0.794463in}}{\pgfqpoint{1.563395in}{0.794463in}}%
\pgfpathcurveto{\pgfqpoint{1.557870in}{0.794463in}}{\pgfqpoint{1.552571in}{0.792268in}}{\pgfqpoint{1.548664in}{0.788361in}}%
\pgfpathcurveto{\pgfqpoint{1.544757in}{0.784455in}}{\pgfqpoint{1.542562in}{0.779155in}}{\pgfqpoint{1.542562in}{0.773630in}}%
\pgfpathcurveto{\pgfqpoint{1.542562in}{0.768105in}}{\pgfqpoint{1.544757in}{0.762805in}}{\pgfqpoint{1.548664in}{0.758899in}}%
\pgfpathcurveto{\pgfqpoint{1.552571in}{0.754992in}}{\pgfqpoint{1.557870in}{0.752797in}}{\pgfqpoint{1.563395in}{0.752797in}}%
\pgfpathclose%
\pgfusepath{stroke,fill}%
\end{pgfscope}%
\begin{pgfscope}%
\pgfpathrectangle{\pgfqpoint{0.403630in}{0.251348in}}{\pgfqpoint{1.395000in}{1.359000in}}%
\pgfusepath{clip}%
\pgfsetbuttcap%
\pgfsetroundjoin%
\definecolor{currentfill}{rgb}{0.686275,0.352941,0.313725}%
\pgfsetfillcolor{currentfill}%
\pgfsetlinewidth{0.501875pt}%
\definecolor{currentstroke}{rgb}{0.686275,0.352941,0.313725}%
\pgfsetstrokecolor{currentstroke}%
\pgfsetdash{}{0pt}%
\pgfpathmoveto{\pgfqpoint{1.207807in}{1.376338in}}%
\pgfpathcurveto{\pgfqpoint{1.213332in}{1.376338in}}{\pgfqpoint{1.218631in}{1.378533in}}{\pgfqpoint{1.222538in}{1.382440in}}%
\pgfpathcurveto{\pgfqpoint{1.226445in}{1.386347in}}{\pgfqpoint{1.228640in}{1.391646in}}{\pgfqpoint{1.228640in}{1.397171in}}%
\pgfpathcurveto{\pgfqpoint{1.228640in}{1.402696in}}{\pgfqpoint{1.226445in}{1.407996in}}{\pgfqpoint{1.222538in}{1.411903in}}%
\pgfpathcurveto{\pgfqpoint{1.218631in}{1.415809in}}{\pgfqpoint{1.213332in}{1.418005in}}{\pgfqpoint{1.207807in}{1.418005in}}%
\pgfpathcurveto{\pgfqpoint{1.202282in}{1.418005in}}{\pgfqpoint{1.196982in}{1.415809in}}{\pgfqpoint{1.193075in}{1.411903in}}%
\pgfpathcurveto{\pgfqpoint{1.189169in}{1.407996in}}{\pgfqpoint{1.186974in}{1.402696in}}{\pgfqpoint{1.186974in}{1.397171in}}%
\pgfpathcurveto{\pgfqpoint{1.186974in}{1.391646in}}{\pgfqpoint{1.189169in}{1.386347in}}{\pgfqpoint{1.193075in}{1.382440in}}%
\pgfpathcurveto{\pgfqpoint{1.196982in}{1.378533in}}{\pgfqpoint{1.202282in}{1.376338in}}{\pgfqpoint{1.207807in}{1.376338in}}%
\pgfpathclose%
\pgfusepath{stroke,fill}%
\end{pgfscope}%
\begin{pgfscope}%
\pgfpathrectangle{\pgfqpoint{0.403630in}{0.251348in}}{\pgfqpoint{1.395000in}{1.359000in}}%
\pgfusepath{clip}%
\pgfsetbuttcap%
\pgfsetroundjoin%
\definecolor{currentfill}{rgb}{0.686275,0.352941,0.313725}%
\pgfsetfillcolor{currentfill}%
\pgfsetlinewidth{0.501875pt}%
\definecolor{currentstroke}{rgb}{0.686275,0.352941,0.313725}%
\pgfsetstrokecolor{currentstroke}%
\pgfsetdash{}{0pt}%
\pgfpathmoveto{\pgfqpoint{1.333630in}{0.496985in}}%
\pgfpathcurveto{\pgfqpoint{1.339155in}{0.496985in}}{\pgfqpoint{1.344455in}{0.499180in}}{\pgfqpoint{1.348362in}{0.503087in}}%
\pgfpathcurveto{\pgfqpoint{1.352269in}{0.506994in}}{\pgfqpoint{1.354464in}{0.512293in}}{\pgfqpoint{1.354464in}{0.517818in}}%
\pgfpathcurveto{\pgfqpoint{1.354464in}{0.523343in}}{\pgfqpoint{1.352269in}{0.528643in}}{\pgfqpoint{1.348362in}{0.532550in}}%
\pgfpathcurveto{\pgfqpoint{1.344455in}{0.536456in}}{\pgfqpoint{1.339155in}{0.538652in}}{\pgfqpoint{1.333630in}{0.538652in}}%
\pgfpathcurveto{\pgfqpoint{1.328105in}{0.538652in}}{\pgfqpoint{1.322806in}{0.536456in}}{\pgfqpoint{1.318899in}{0.532550in}}%
\pgfpathcurveto{\pgfqpoint{1.314992in}{0.528643in}}{\pgfqpoint{1.312797in}{0.523343in}}{\pgfqpoint{1.312797in}{0.517818in}}%
\pgfpathcurveto{\pgfqpoint{1.312797in}{0.512293in}}{\pgfqpoint{1.314992in}{0.506994in}}{\pgfqpoint{1.318899in}{0.503087in}}%
\pgfpathcurveto{\pgfqpoint{1.322806in}{0.499180in}}{\pgfqpoint{1.328105in}{0.496985in}}{\pgfqpoint{1.333630in}{0.496985in}}%
\pgfpathclose%
\pgfusepath{stroke,fill}%
\end{pgfscope}%
\begin{pgfscope}%
\pgfpathrectangle{\pgfqpoint{0.403630in}{0.251348in}}{\pgfqpoint{1.395000in}{1.359000in}}%
\pgfusepath{clip}%
\pgfsetbuttcap%
\pgfsetroundjoin%
\definecolor{currentfill}{rgb}{0.686275,0.352941,0.313725}%
\pgfsetfillcolor{currentfill}%
\pgfsetlinewidth{0.501875pt}%
\definecolor{currentstroke}{rgb}{0.686275,0.352941,0.313725}%
\pgfsetstrokecolor{currentstroke}%
\pgfsetdash{}{0pt}%
\pgfpathmoveto{\pgfqpoint{1.278925in}{0.470338in}}%
\pgfpathcurveto{\pgfqpoint{1.284450in}{0.470338in}}{\pgfqpoint{1.289749in}{0.472533in}}{\pgfqpoint{1.293656in}{0.476440in}}%
\pgfpathcurveto{\pgfqpoint{1.297563in}{0.480347in}}{\pgfqpoint{1.299758in}{0.485646in}}{\pgfqpoint{1.299758in}{0.491171in}}%
\pgfpathcurveto{\pgfqpoint{1.299758in}{0.496696in}}{\pgfqpoint{1.297563in}{0.501996in}}{\pgfqpoint{1.293656in}{0.505903in}}%
\pgfpathcurveto{\pgfqpoint{1.289749in}{0.509809in}}{\pgfqpoint{1.284450in}{0.512005in}}{\pgfqpoint{1.278925in}{0.512005in}}%
\pgfpathcurveto{\pgfqpoint{1.273399in}{0.512005in}}{\pgfqpoint{1.268100in}{0.509809in}}{\pgfqpoint{1.264193in}{0.505903in}}%
\pgfpathcurveto{\pgfqpoint{1.260286in}{0.501996in}}{\pgfqpoint{1.258091in}{0.496696in}}{\pgfqpoint{1.258091in}{0.491171in}}%
\pgfpathcurveto{\pgfqpoint{1.258091in}{0.485646in}}{\pgfqpoint{1.260286in}{0.480347in}}{\pgfqpoint{1.264193in}{0.476440in}}%
\pgfpathcurveto{\pgfqpoint{1.268100in}{0.472533in}}{\pgfqpoint{1.273399in}{0.470338in}}{\pgfqpoint{1.278925in}{0.470338in}}%
\pgfpathclose%
\pgfusepath{stroke,fill}%
\end{pgfscope}%
\begin{pgfscope}%
\pgfpathrectangle{\pgfqpoint{0.403630in}{0.251348in}}{\pgfqpoint{1.395000in}{1.359000in}}%
\pgfusepath{clip}%
\pgfsetbuttcap%
\pgfsetroundjoin%
\definecolor{currentfill}{rgb}{0.686275,0.352941,0.313725}%
\pgfsetfillcolor{currentfill}%
\pgfsetlinewidth{0.501875pt}%
\definecolor{currentstroke}{rgb}{0.686275,0.352941,0.313725}%
\pgfsetstrokecolor{currentstroke}%
\pgfsetdash{}{0pt}%
\pgfpathmoveto{\pgfqpoint{1.574336in}{0.859385in}}%
\pgfpathcurveto{\pgfqpoint{1.579861in}{0.859385in}}{\pgfqpoint{1.585161in}{0.861580in}}{\pgfqpoint{1.589068in}{0.865487in}}%
\pgfpathcurveto{\pgfqpoint{1.592975in}{0.869394in}}{\pgfqpoint{1.595170in}{0.874693in}}{\pgfqpoint{1.595170in}{0.880218in}}%
\pgfpathcurveto{\pgfqpoint{1.595170in}{0.885743in}}{\pgfqpoint{1.592975in}{0.891043in}}{\pgfqpoint{1.589068in}{0.894950in}}%
\pgfpathcurveto{\pgfqpoint{1.585161in}{0.898856in}}{\pgfqpoint{1.579861in}{0.901052in}}{\pgfqpoint{1.574336in}{0.901052in}}%
\pgfpathcurveto{\pgfqpoint{1.568811in}{0.901052in}}{\pgfqpoint{1.563512in}{0.898856in}}{\pgfqpoint{1.559605in}{0.894950in}}%
\pgfpathcurveto{\pgfqpoint{1.555698in}{0.891043in}}{\pgfqpoint{1.553503in}{0.885743in}}{\pgfqpoint{1.553503in}{0.880218in}}%
\pgfpathcurveto{\pgfqpoint{1.553503in}{0.874693in}}{\pgfqpoint{1.555698in}{0.869394in}}{\pgfqpoint{1.559605in}{0.865487in}}%
\pgfpathcurveto{\pgfqpoint{1.563512in}{0.861580in}}{\pgfqpoint{1.568811in}{0.859385in}}{\pgfqpoint{1.574336in}{0.859385in}}%
\pgfpathclose%
\pgfusepath{stroke,fill}%
\end{pgfscope}%
\begin{pgfscope}%
\pgfpathrectangle{\pgfqpoint{0.403630in}{0.251348in}}{\pgfqpoint{1.395000in}{1.359000in}}%
\pgfusepath{clip}%
\pgfsetbuttcap%
\pgfsetroundjoin%
\definecolor{currentfill}{rgb}{0.686275,0.352941,0.313725}%
\pgfsetfillcolor{currentfill}%
\pgfsetlinewidth{0.501875pt}%
\definecolor{currentstroke}{rgb}{0.686275,0.352941,0.313725}%
\pgfsetstrokecolor{currentstroke}%
\pgfsetdash{}{0pt}%
\pgfpathmoveto{\pgfqpoint{1.415689in}{0.512973in}}%
\pgfpathcurveto{\pgfqpoint{1.421214in}{0.512973in}}{\pgfqpoint{1.426514in}{0.515168in}}{\pgfqpoint{1.430421in}{0.519075in}}%
\pgfpathcurveto{\pgfqpoint{1.434327in}{0.522982in}}{\pgfqpoint{1.436523in}{0.528281in}}{\pgfqpoint{1.436523in}{0.533806in}}%
\pgfpathcurveto{\pgfqpoint{1.436523in}{0.539332in}}{\pgfqpoint{1.434327in}{0.544631in}}{\pgfqpoint{1.430421in}{0.548538in}}%
\pgfpathcurveto{\pgfqpoint{1.426514in}{0.552445in}}{\pgfqpoint{1.421214in}{0.554640in}}{\pgfqpoint{1.415689in}{0.554640in}}%
\pgfpathcurveto{\pgfqpoint{1.410164in}{0.554640in}}{\pgfqpoint{1.404865in}{0.552445in}}{\pgfqpoint{1.400958in}{0.548538in}}%
\pgfpathcurveto{\pgfqpoint{1.397051in}{0.544631in}}{\pgfqpoint{1.394856in}{0.539332in}}{\pgfqpoint{1.394856in}{0.533806in}}%
\pgfpathcurveto{\pgfqpoint{1.394856in}{0.528281in}}{\pgfqpoint{1.397051in}{0.522982in}}{\pgfqpoint{1.400958in}{0.519075in}}%
\pgfpathcurveto{\pgfqpoint{1.404865in}{0.515168in}}{\pgfqpoint{1.410164in}{0.512973in}}{\pgfqpoint{1.415689in}{0.512973in}}%
\pgfpathclose%
\pgfusepath{stroke,fill}%
\end{pgfscope}%
\begin{pgfscope}%
\pgfpathrectangle{\pgfqpoint{0.403630in}{0.251348in}}{\pgfqpoint{1.395000in}{1.359000in}}%
\pgfusepath{clip}%
\pgfsetbuttcap%
\pgfsetroundjoin%
\definecolor{currentfill}{rgb}{0.686275,0.352941,0.313725}%
\pgfsetfillcolor{currentfill}%
\pgfsetlinewidth{0.501875pt}%
\definecolor{currentstroke}{rgb}{0.686275,0.352941,0.313725}%
\pgfsetstrokecolor{currentstroke}%
\pgfsetdash{}{0pt}%
\pgfpathmoveto{\pgfqpoint{0.770160in}{1.291067in}}%
\pgfpathcurveto{\pgfqpoint{0.775685in}{1.291067in}}{\pgfqpoint{0.780984in}{1.293262in}}{\pgfqpoint{0.784891in}{1.297169in}}%
\pgfpathcurveto{\pgfqpoint{0.788798in}{1.301076in}}{\pgfqpoint{0.790993in}{1.306376in}}{\pgfqpoint{0.790993in}{1.311901in}}%
\pgfpathcurveto{\pgfqpoint{0.790993in}{1.317426in}}{\pgfqpoint{0.788798in}{1.322725in}}{\pgfqpoint{0.784891in}{1.326632in}}%
\pgfpathcurveto{\pgfqpoint{0.780984in}{1.330539in}}{\pgfqpoint{0.775685in}{1.332734in}}{\pgfqpoint{0.770160in}{1.332734in}}%
\pgfpathcurveto{\pgfqpoint{0.764635in}{1.332734in}}{\pgfqpoint{0.759335in}{1.330539in}}{\pgfqpoint{0.755428in}{1.326632in}}%
\pgfpathcurveto{\pgfqpoint{0.751522in}{1.322725in}}{\pgfqpoint{0.749326in}{1.317426in}}{\pgfqpoint{0.749326in}{1.311901in}}%
\pgfpathcurveto{\pgfqpoint{0.749326in}{1.306376in}}{\pgfqpoint{0.751522in}{1.301076in}}{\pgfqpoint{0.755428in}{1.297169in}}%
\pgfpathcurveto{\pgfqpoint{0.759335in}{1.293262in}}{\pgfqpoint{0.764635in}{1.291067in}}{\pgfqpoint{0.770160in}{1.291067in}}%
\pgfpathclose%
\pgfusepath{stroke,fill}%
\end{pgfscope}%
\begin{pgfscope}%
\pgfpathrectangle{\pgfqpoint{0.403630in}{0.251348in}}{\pgfqpoint{1.395000in}{1.359000in}}%
\pgfusepath{clip}%
\pgfsetbuttcap%
\pgfsetroundjoin%
\definecolor{currentfill}{rgb}{0.686275,0.352941,0.313725}%
\pgfsetfillcolor{currentfill}%
\pgfsetlinewidth{0.501875pt}%
\definecolor{currentstroke}{rgb}{0.686275,0.352941,0.313725}%
\pgfsetstrokecolor{currentstroke}%
\pgfsetdash{}{0pt}%
\pgfpathmoveto{\pgfqpoint{1.508689in}{1.099208in}}%
\pgfpathcurveto{\pgfqpoint{1.514214in}{1.099208in}}{\pgfqpoint{1.519514in}{1.101404in}}{\pgfqpoint{1.523421in}{1.105310in}}%
\pgfpathcurveto{\pgfqpoint{1.527327in}{1.109217in}}{\pgfqpoint{1.529523in}{1.114517in}}{\pgfqpoint{1.529523in}{1.120042in}}%
\pgfpathcurveto{\pgfqpoint{1.529523in}{1.125567in}}{\pgfqpoint{1.527327in}{1.130866in}}{\pgfqpoint{1.523421in}{1.134773in}}%
\pgfpathcurveto{\pgfqpoint{1.519514in}{1.138680in}}{\pgfqpoint{1.514214in}{1.140875in}}{\pgfqpoint{1.508689in}{1.140875in}}%
\pgfpathcurveto{\pgfqpoint{1.503164in}{1.140875in}}{\pgfqpoint{1.497865in}{1.138680in}}{\pgfqpoint{1.493958in}{1.134773in}}%
\pgfpathcurveto{\pgfqpoint{1.490051in}{1.130866in}}{\pgfqpoint{1.487856in}{1.125567in}}{\pgfqpoint{1.487856in}{1.120042in}}%
\pgfpathcurveto{\pgfqpoint{1.487856in}{1.114517in}}{\pgfqpoint{1.490051in}{1.109217in}}{\pgfqpoint{1.493958in}{1.105310in}}%
\pgfpathcurveto{\pgfqpoint{1.497865in}{1.101404in}}{\pgfqpoint{1.503164in}{1.099208in}}{\pgfqpoint{1.508689in}{1.099208in}}%
\pgfpathclose%
\pgfusepath{stroke,fill}%
\end{pgfscope}%
\begin{pgfscope}%
\pgfpathrectangle{\pgfqpoint{0.403630in}{0.251348in}}{\pgfqpoint{1.395000in}{1.359000in}}%
\pgfusepath{clip}%
\pgfsetbuttcap%
\pgfsetroundjoin%
\definecolor{currentfill}{rgb}{0.686275,0.352941,0.313725}%
\pgfsetfillcolor{currentfill}%
\pgfsetlinewidth{0.501875pt}%
\definecolor{currentstroke}{rgb}{0.686275,0.352941,0.313725}%
\pgfsetstrokecolor{currentstroke}%
\pgfsetdash{}{0pt}%
\pgfpathmoveto{\pgfqpoint{0.824866in}{0.512973in}}%
\pgfpathcurveto{\pgfqpoint{0.830391in}{0.512973in}}{\pgfqpoint{0.835690in}{0.515168in}}{\pgfqpoint{0.839597in}{0.519075in}}%
\pgfpathcurveto{\pgfqpoint{0.843504in}{0.522982in}}{\pgfqpoint{0.845699in}{0.528281in}}{\pgfqpoint{0.845699in}{0.533806in}}%
\pgfpathcurveto{\pgfqpoint{0.845699in}{0.539332in}}{\pgfqpoint{0.843504in}{0.544631in}}{\pgfqpoint{0.839597in}{0.548538in}}%
\pgfpathcurveto{\pgfqpoint{0.835690in}{0.552445in}}{\pgfqpoint{0.830391in}{0.554640in}}{\pgfqpoint{0.824866in}{0.554640in}}%
\pgfpathcurveto{\pgfqpoint{0.819341in}{0.554640in}}{\pgfqpoint{0.814041in}{0.552445in}}{\pgfqpoint{0.810134in}{0.548538in}}%
\pgfpathcurveto{\pgfqpoint{0.806228in}{0.544631in}}{\pgfqpoint{0.804032in}{0.539332in}}{\pgfqpoint{0.804032in}{0.533806in}}%
\pgfpathcurveto{\pgfqpoint{0.804032in}{0.528281in}}{\pgfqpoint{0.806228in}{0.522982in}}{\pgfqpoint{0.810134in}{0.519075in}}%
\pgfpathcurveto{\pgfqpoint{0.814041in}{0.515168in}}{\pgfqpoint{0.819341in}{0.512973in}}{\pgfqpoint{0.824866in}{0.512973in}}%
\pgfpathclose%
\pgfusepath{stroke,fill}%
\end{pgfscope}%
\begin{pgfscope}%
\pgfpathrectangle{\pgfqpoint{0.403630in}{0.251348in}}{\pgfqpoint{1.395000in}{1.359000in}}%
\pgfusepath{clip}%
\pgfsetbuttcap%
\pgfsetroundjoin%
\definecolor{currentfill}{rgb}{0.686275,0.352941,0.313725}%
\pgfsetfillcolor{currentfill}%
\pgfsetlinewidth{0.501875pt}%
\definecolor{currentstroke}{rgb}{0.686275,0.352941,0.313725}%
\pgfsetstrokecolor{currentstroke}%
\pgfsetdash{}{0pt}%
\pgfpathmoveto{\pgfqpoint{0.879572in}{0.502314in}}%
\pgfpathcurveto{\pgfqpoint{0.885097in}{0.502314in}}{\pgfqpoint{0.890396in}{0.504509in}}{\pgfqpoint{0.894303in}{0.508416in}}%
\pgfpathcurveto{\pgfqpoint{0.898210in}{0.512323in}}{\pgfqpoint{0.900405in}{0.517623in}}{\pgfqpoint{0.900405in}{0.523148in}}%
\pgfpathcurveto{\pgfqpoint{0.900405in}{0.528673in}}{\pgfqpoint{0.898210in}{0.533972in}}{\pgfqpoint{0.894303in}{0.537879in}}%
\pgfpathcurveto{\pgfqpoint{0.890396in}{0.541786in}}{\pgfqpoint{0.885097in}{0.543981in}}{\pgfqpoint{0.879572in}{0.543981in}}%
\pgfpathcurveto{\pgfqpoint{0.874047in}{0.543981in}}{\pgfqpoint{0.868747in}{0.541786in}}{\pgfqpoint{0.864840in}{0.537879in}}%
\pgfpathcurveto{\pgfqpoint{0.860933in}{0.533972in}}{\pgfqpoint{0.858738in}{0.528673in}}{\pgfqpoint{0.858738in}{0.523148in}}%
\pgfpathcurveto{\pgfqpoint{0.858738in}{0.517623in}}{\pgfqpoint{0.860933in}{0.512323in}}{\pgfqpoint{0.864840in}{0.508416in}}%
\pgfpathcurveto{\pgfqpoint{0.868747in}{0.504509in}}{\pgfqpoint{0.874047in}{0.502314in}}{\pgfqpoint{0.879572in}{0.502314in}}%
\pgfpathclose%
\pgfusepath{stroke,fill}%
\end{pgfscope}%
\begin{pgfscope}%
\pgfpathrectangle{\pgfqpoint{0.403630in}{0.251348in}}{\pgfqpoint{1.395000in}{1.359000in}}%
\pgfusepath{clip}%
\pgfsetbuttcap%
\pgfsetroundjoin%
\definecolor{currentfill}{rgb}{0.686275,0.352941,0.313725}%
\pgfsetfillcolor{currentfill}%
\pgfsetlinewidth{0.501875pt}%
\definecolor{currentstroke}{rgb}{0.686275,0.352941,0.313725}%
\pgfsetstrokecolor{currentstroke}%
\pgfsetdash{}{0pt}%
\pgfpathmoveto{\pgfqpoint{0.633395in}{1.045914in}}%
\pgfpathcurveto{\pgfqpoint{0.638920in}{1.045914in}}{\pgfqpoint{0.644220in}{1.048109in}}{\pgfqpoint{0.648127in}{1.052016in}}%
\pgfpathcurveto{\pgfqpoint{0.652033in}{1.055923in}}{\pgfqpoint{0.654228in}{1.061223in}}{\pgfqpoint{0.654228in}{1.066748in}}%
\pgfpathcurveto{\pgfqpoint{0.654228in}{1.072273in}}{\pgfqpoint{0.652033in}{1.077572in}}{\pgfqpoint{0.648127in}{1.081479in}}%
\pgfpathcurveto{\pgfqpoint{0.644220in}{1.085386in}}{\pgfqpoint{0.638920in}{1.087581in}}{\pgfqpoint{0.633395in}{1.087581in}}%
\pgfpathcurveto{\pgfqpoint{0.627870in}{1.087581in}}{\pgfqpoint{0.622571in}{1.085386in}}{\pgfqpoint{0.618664in}{1.081479in}}%
\pgfpathcurveto{\pgfqpoint{0.614757in}{1.077572in}}{\pgfqpoint{0.612562in}{1.072273in}}{\pgfqpoint{0.612562in}{1.066748in}}%
\pgfpathcurveto{\pgfqpoint{0.612562in}{1.061223in}}{\pgfqpoint{0.614757in}{1.055923in}}{\pgfqpoint{0.618664in}{1.052016in}}%
\pgfpathcurveto{\pgfqpoint{0.622571in}{1.048109in}}{\pgfqpoint{0.627870in}{1.045914in}}{\pgfqpoint{0.633395in}{1.045914in}}%
\pgfpathclose%
\pgfusepath{stroke,fill}%
\end{pgfscope}%
\begin{pgfscope}%
\pgfpathrectangle{\pgfqpoint{0.403630in}{0.251348in}}{\pgfqpoint{1.395000in}{1.359000in}}%
\pgfusepath{clip}%
\pgfsetbuttcap%
\pgfsetroundjoin%
\definecolor{currentfill}{rgb}{0.686275,0.352941,0.313725}%
\pgfsetfillcolor{currentfill}%
\pgfsetlinewidth{0.501875pt}%
\definecolor{currentstroke}{rgb}{0.686275,0.352941,0.313725}%
\pgfsetstrokecolor{currentstroke}%
\pgfsetdash{}{0pt}%
\pgfpathmoveto{\pgfqpoint{1.579807in}{1.013938in}}%
\pgfpathcurveto{\pgfqpoint{1.585332in}{1.013938in}}{\pgfqpoint{1.590631in}{1.016133in}}{\pgfqpoint{1.594538in}{1.020040in}}%
\pgfpathcurveto{\pgfqpoint{1.598445in}{1.023947in}}{\pgfqpoint{1.600640in}{1.029246in}}{\pgfqpoint{1.600640in}{1.034771in}}%
\pgfpathcurveto{\pgfqpoint{1.600640in}{1.040296in}}{\pgfqpoint{1.598445in}{1.045596in}}{\pgfqpoint{1.594538in}{1.049503in}}%
\pgfpathcurveto{\pgfqpoint{1.590631in}{1.053409in}}{\pgfqpoint{1.585332in}{1.055605in}}{\pgfqpoint{1.579807in}{1.055605in}}%
\pgfpathcurveto{\pgfqpoint{1.574282in}{1.055605in}}{\pgfqpoint{1.568982in}{1.053409in}}{\pgfqpoint{1.565075in}{1.049503in}}%
\pgfpathcurveto{\pgfqpoint{1.561169in}{1.045596in}}{\pgfqpoint{1.558974in}{1.040296in}}{\pgfqpoint{1.558974in}{1.034771in}}%
\pgfpathcurveto{\pgfqpoint{1.558974in}{1.029246in}}{\pgfqpoint{1.561169in}{1.023947in}}{\pgfqpoint{1.565075in}{1.020040in}}%
\pgfpathcurveto{\pgfqpoint{1.568982in}{1.016133in}}{\pgfqpoint{1.574282in}{1.013938in}}{\pgfqpoint{1.579807in}{1.013938in}}%
\pgfpathclose%
\pgfusepath{stroke,fill}%
\end{pgfscope}%
\begin{pgfscope}%
\pgfpathrectangle{\pgfqpoint{0.403630in}{0.251348in}}{\pgfqpoint{1.395000in}{1.359000in}}%
\pgfusepath{clip}%
\pgfsetbuttcap%
\pgfsetroundjoin%
\definecolor{currentfill}{rgb}{0.686275,0.352941,0.313725}%
\pgfsetfillcolor{currentfill}%
\pgfsetlinewidth{0.501875pt}%
\definecolor{currentstroke}{rgb}{0.686275,0.352941,0.313725}%
\pgfsetstrokecolor{currentstroke}%
\pgfsetdash{}{0pt}%
\pgfpathmoveto{\pgfqpoint{1.557925in}{0.944655in}}%
\pgfpathcurveto{\pgfqpoint{1.563450in}{0.944655in}}{\pgfqpoint{1.568749in}{0.946851in}}{\pgfqpoint{1.572656in}{0.950757in}}%
\pgfpathcurveto{\pgfqpoint{1.576563in}{0.954664in}}{\pgfqpoint{1.578758in}{0.959964in}}{\pgfqpoint{1.578758in}{0.965489in}}%
\pgfpathcurveto{\pgfqpoint{1.578758in}{0.971014in}}{\pgfqpoint{1.576563in}{0.976313in}}{\pgfqpoint{1.572656in}{0.980220in}}%
\pgfpathcurveto{\pgfqpoint{1.568749in}{0.984127in}}{\pgfqpoint{1.563450in}{0.986322in}}{\pgfqpoint{1.557925in}{0.986322in}}%
\pgfpathcurveto{\pgfqpoint{1.552399in}{0.986322in}}{\pgfqpoint{1.547100in}{0.984127in}}{\pgfqpoint{1.543193in}{0.980220in}}%
\pgfpathcurveto{\pgfqpoint{1.539286in}{0.976313in}}{\pgfqpoint{1.537091in}{0.971014in}}{\pgfqpoint{1.537091in}{0.965489in}}%
\pgfpathcurveto{\pgfqpoint{1.537091in}{0.959964in}}{\pgfqpoint{1.539286in}{0.954664in}}{\pgfqpoint{1.543193in}{0.950757in}}%
\pgfpathcurveto{\pgfqpoint{1.547100in}{0.946851in}}{\pgfqpoint{1.552399in}{0.944655in}}{\pgfqpoint{1.557925in}{0.944655in}}%
\pgfpathclose%
\pgfusepath{stroke,fill}%
\end{pgfscope}%
\begin{pgfscope}%
\pgfpathrectangle{\pgfqpoint{0.403630in}{0.251348in}}{\pgfqpoint{1.395000in}{1.359000in}}%
\pgfusepath{clip}%
\pgfsetbuttcap%
\pgfsetroundjoin%
\definecolor{currentfill}{rgb}{0.686275,0.352941,0.313725}%
\pgfsetfillcolor{currentfill}%
\pgfsetlinewidth{0.501875pt}%
\definecolor{currentstroke}{rgb}{0.686275,0.352941,0.313725}%
\pgfsetstrokecolor{currentstroke}%
\pgfsetdash{}{0pt}%
\pgfpathmoveto{\pgfqpoint{0.912395in}{0.486326in}}%
\pgfpathcurveto{\pgfqpoint{0.917920in}{0.486326in}}{\pgfqpoint{0.923220in}{0.488521in}}{\pgfqpoint{0.927127in}{0.492428in}}%
\pgfpathcurveto{\pgfqpoint{0.931033in}{0.496335in}}{\pgfqpoint{0.933228in}{0.501634in}}{\pgfqpoint{0.933228in}{0.507159in}}%
\pgfpathcurveto{\pgfqpoint{0.933228in}{0.512684in}}{\pgfqpoint{0.931033in}{0.517984in}}{\pgfqpoint{0.927127in}{0.521891in}}%
\pgfpathcurveto{\pgfqpoint{0.923220in}{0.525798in}}{\pgfqpoint{0.917920in}{0.527993in}}{\pgfqpoint{0.912395in}{0.527993in}}%
\pgfpathcurveto{\pgfqpoint{0.906870in}{0.527993in}}{\pgfqpoint{0.901571in}{0.525798in}}{\pgfqpoint{0.897664in}{0.521891in}}%
\pgfpathcurveto{\pgfqpoint{0.893757in}{0.517984in}}{\pgfqpoint{0.891562in}{0.512684in}}{\pgfqpoint{0.891562in}{0.507159in}}%
\pgfpathcurveto{\pgfqpoint{0.891562in}{0.501634in}}{\pgfqpoint{0.893757in}{0.496335in}}{\pgfqpoint{0.897664in}{0.492428in}}%
\pgfpathcurveto{\pgfqpoint{0.901571in}{0.488521in}}{\pgfqpoint{0.906870in}{0.486326in}}{\pgfqpoint{0.912395in}{0.486326in}}%
\pgfpathclose%
\pgfusepath{stroke,fill}%
\end{pgfscope}%
\begin{pgfscope}%
\pgfpathrectangle{\pgfqpoint{0.403630in}{0.251348in}}{\pgfqpoint{1.395000in}{1.359000in}}%
\pgfusepath{clip}%
\pgfsetbuttcap%
\pgfsetroundjoin%
\definecolor{currentfill}{rgb}{0.686275,0.352941,0.313725}%
\pgfsetfillcolor{currentfill}%
\pgfsetlinewidth{0.501875pt}%
\definecolor{currentstroke}{rgb}{0.686275,0.352941,0.313725}%
\pgfsetstrokecolor{currentstroke}%
\pgfsetdash{}{0pt}%
\pgfpathmoveto{\pgfqpoint{1.027277in}{1.381667in}}%
\pgfpathcurveto{\pgfqpoint{1.032803in}{1.381667in}}{\pgfqpoint{1.038102in}{1.383862in}}{\pgfqpoint{1.042009in}{1.387769in}}%
\pgfpathcurveto{\pgfqpoint{1.045916in}{1.391676in}}{\pgfqpoint{1.048111in}{1.396976in}}{\pgfqpoint{1.048111in}{1.402501in}}%
\pgfpathcurveto{\pgfqpoint{1.048111in}{1.408026in}}{\pgfqpoint{1.045916in}{1.413325in}}{\pgfqpoint{1.042009in}{1.417232in}}%
\pgfpathcurveto{\pgfqpoint{1.038102in}{1.421139in}}{\pgfqpoint{1.032803in}{1.423334in}}{\pgfqpoint{1.027277in}{1.423334in}}%
\pgfpathcurveto{\pgfqpoint{1.021752in}{1.423334in}}{\pgfqpoint{1.016453in}{1.421139in}}{\pgfqpoint{1.012546in}{1.417232in}}%
\pgfpathcurveto{\pgfqpoint{1.008639in}{1.413325in}}{\pgfqpoint{1.006444in}{1.408026in}}{\pgfqpoint{1.006444in}{1.402501in}}%
\pgfpathcurveto{\pgfqpoint{1.006444in}{1.396976in}}{\pgfqpoint{1.008639in}{1.391676in}}{\pgfqpoint{1.012546in}{1.387769in}}%
\pgfpathcurveto{\pgfqpoint{1.016453in}{1.383862in}}{\pgfqpoint{1.021752in}{1.381667in}}{\pgfqpoint{1.027277in}{1.381667in}}%
\pgfpathclose%
\pgfusepath{stroke,fill}%
\end{pgfscope}%
\begin{pgfscope}%
\pgfpathrectangle{\pgfqpoint{0.403630in}{0.251348in}}{\pgfqpoint{1.395000in}{1.359000in}}%
\pgfusepath{clip}%
\pgfsetbuttcap%
\pgfsetroundjoin%
\definecolor{currentfill}{rgb}{0.686275,0.352941,0.313725}%
\pgfsetfillcolor{currentfill}%
\pgfsetlinewidth{0.501875pt}%
\definecolor{currentstroke}{rgb}{0.686275,0.352941,0.313725}%
\pgfsetstrokecolor{currentstroke}%
\pgfsetdash{}{0pt}%
\pgfpathmoveto{\pgfqpoint{1.136689in}{0.475667in}}%
\pgfpathcurveto{\pgfqpoint{1.142214in}{0.475667in}}{\pgfqpoint{1.147514in}{0.477862in}}{\pgfqpoint{1.151421in}{0.481769in}}%
\pgfpathcurveto{\pgfqpoint{1.155327in}{0.485676in}}{\pgfqpoint{1.157523in}{0.490976in}}{\pgfqpoint{1.157523in}{0.496501in}}%
\pgfpathcurveto{\pgfqpoint{1.157523in}{0.502026in}}{\pgfqpoint{1.155327in}{0.507325in}}{\pgfqpoint{1.151421in}{0.511232in}}%
\pgfpathcurveto{\pgfqpoint{1.147514in}{0.515139in}}{\pgfqpoint{1.142214in}{0.517334in}}{\pgfqpoint{1.136689in}{0.517334in}}%
\pgfpathcurveto{\pgfqpoint{1.131164in}{0.517334in}}{\pgfqpoint{1.125865in}{0.515139in}}{\pgfqpoint{1.121958in}{0.511232in}}%
\pgfpathcurveto{\pgfqpoint{1.118051in}{0.507325in}}{\pgfqpoint{1.115856in}{0.502026in}}{\pgfqpoint{1.115856in}{0.496501in}}%
\pgfpathcurveto{\pgfqpoint{1.115856in}{0.490976in}}{\pgfqpoint{1.118051in}{0.485676in}}{\pgfqpoint{1.121958in}{0.481769in}}%
\pgfpathcurveto{\pgfqpoint{1.125865in}{0.477862in}}{\pgfqpoint{1.131164in}{0.475667in}}{\pgfqpoint{1.136689in}{0.475667in}}%
\pgfpathclose%
\pgfusepath{stroke,fill}%
\end{pgfscope}%
\begin{pgfscope}%
\pgfpathrectangle{\pgfqpoint{0.403630in}{0.251348in}}{\pgfqpoint{1.395000in}{1.359000in}}%
\pgfusepath{clip}%
\pgfsetbuttcap%
\pgfsetroundjoin%
\definecolor{currentfill}{rgb}{0.686275,0.352941,0.313725}%
\pgfsetfillcolor{currentfill}%
\pgfsetlinewidth{0.501875pt}%
\definecolor{currentstroke}{rgb}{0.686275,0.352941,0.313725}%
\pgfsetstrokecolor{currentstroke}%
\pgfsetdash{}{0pt}%
\pgfpathmoveto{\pgfqpoint{1.027277in}{0.422373in}}%
\pgfpathcurveto{\pgfqpoint{1.032803in}{0.422373in}}{\pgfqpoint{1.038102in}{0.424568in}}{\pgfqpoint{1.042009in}{0.428475in}}%
\pgfpathcurveto{\pgfqpoint{1.045916in}{0.432382in}}{\pgfqpoint{1.048111in}{0.437681in}}{\pgfqpoint{1.048111in}{0.443206in}}%
\pgfpathcurveto{\pgfqpoint{1.048111in}{0.448732in}}{\pgfqpoint{1.045916in}{0.454031in}}{\pgfqpoint{1.042009in}{0.457938in}}%
\pgfpathcurveto{\pgfqpoint{1.038102in}{0.461845in}}{\pgfqpoint{1.032803in}{0.464040in}}{\pgfqpoint{1.027277in}{0.464040in}}%
\pgfpathcurveto{\pgfqpoint{1.021752in}{0.464040in}}{\pgfqpoint{1.016453in}{0.461845in}}{\pgfqpoint{1.012546in}{0.457938in}}%
\pgfpathcurveto{\pgfqpoint{1.008639in}{0.454031in}}{\pgfqpoint{1.006444in}{0.448732in}}{\pgfqpoint{1.006444in}{0.443206in}}%
\pgfpathcurveto{\pgfqpoint{1.006444in}{0.437681in}}{\pgfqpoint{1.008639in}{0.432382in}}{\pgfqpoint{1.012546in}{0.428475in}}%
\pgfpathcurveto{\pgfqpoint{1.016453in}{0.424568in}}{\pgfqpoint{1.021752in}{0.422373in}}{\pgfqpoint{1.027277in}{0.422373in}}%
\pgfpathclose%
\pgfusepath{stroke,fill}%
\end{pgfscope}%
\begin{pgfscope}%
\pgfpathrectangle{\pgfqpoint{0.403630in}{0.251348in}}{\pgfqpoint{1.395000in}{1.359000in}}%
\pgfusepath{clip}%
\pgfsetbuttcap%
\pgfsetroundjoin%
\definecolor{currentfill}{rgb}{0.686275,0.352941,0.313725}%
\pgfsetfillcolor{currentfill}%
\pgfsetlinewidth{0.501875pt}%
\definecolor{currentstroke}{rgb}{0.686275,0.352941,0.313725}%
\pgfsetstrokecolor{currentstroke}%
\pgfsetdash{}{0pt}%
\pgfpathmoveto{\pgfqpoint{1.421160in}{1.269750in}}%
\pgfpathcurveto{\pgfqpoint{1.426685in}{1.269750in}}{\pgfqpoint{1.431984in}{1.271945in}}{\pgfqpoint{1.435891in}{1.275852in}}%
\pgfpathcurveto{\pgfqpoint{1.439798in}{1.279758in}}{\pgfqpoint{1.441993in}{1.285058in}}{\pgfqpoint{1.441993in}{1.290583in}}%
\pgfpathcurveto{\pgfqpoint{1.441993in}{1.296108in}}{\pgfqpoint{1.439798in}{1.301408in}}{\pgfqpoint{1.435891in}{1.305314in}}%
\pgfpathcurveto{\pgfqpoint{1.431984in}{1.309221in}}{\pgfqpoint{1.426685in}{1.311416in}}{\pgfqpoint{1.421160in}{1.311416in}}%
\pgfpathcurveto{\pgfqpoint{1.415635in}{1.311416in}}{\pgfqpoint{1.410335in}{1.309221in}}{\pgfqpoint{1.406428in}{1.305314in}}%
\pgfpathcurveto{\pgfqpoint{1.402522in}{1.301408in}}{\pgfqpoint{1.400326in}{1.296108in}}{\pgfqpoint{1.400326in}{1.290583in}}%
\pgfpathcurveto{\pgfqpoint{1.400326in}{1.285058in}}{\pgfqpoint{1.402522in}{1.279758in}}{\pgfqpoint{1.406428in}{1.275852in}}%
\pgfpathcurveto{\pgfqpoint{1.410335in}{1.271945in}}{\pgfqpoint{1.415635in}{1.269750in}}{\pgfqpoint{1.421160in}{1.269750in}}%
\pgfpathclose%
\pgfusepath{stroke,fill}%
\end{pgfscope}%
\begin{pgfscope}%
\pgfpathrectangle{\pgfqpoint{0.403630in}{0.251348in}}{\pgfqpoint{1.395000in}{1.359000in}}%
\pgfusepath{clip}%
\pgfsetbuttcap%
\pgfsetroundjoin%
\definecolor{currentfill}{rgb}{0.686275,0.352941,0.313725}%
\pgfsetfillcolor{currentfill}%
\pgfsetlinewidth{0.501875pt}%
\definecolor{currentstroke}{rgb}{0.686275,0.352941,0.313725}%
\pgfsetstrokecolor{currentstroke}%
\pgfsetdash{}{0pt}%
\pgfpathmoveto{\pgfqpoint{1.590748in}{0.779444in}}%
\pgfpathcurveto{\pgfqpoint{1.596273in}{0.779444in}}{\pgfqpoint{1.601573in}{0.781639in}}{\pgfqpoint{1.605479in}{0.785546in}}%
\pgfpathcurveto{\pgfqpoint{1.609386in}{0.789452in}}{\pgfqpoint{1.611581in}{0.794752in}}{\pgfqpoint{1.611581in}{0.800277in}}%
\pgfpathcurveto{\pgfqpoint{1.611581in}{0.805802in}}{\pgfqpoint{1.609386in}{0.811102in}}{\pgfqpoint{1.605479in}{0.815008in}}%
\pgfpathcurveto{\pgfqpoint{1.601573in}{0.818915in}}{\pgfqpoint{1.596273in}{0.821110in}}{\pgfqpoint{1.590748in}{0.821110in}}%
\pgfpathcurveto{\pgfqpoint{1.585223in}{0.821110in}}{\pgfqpoint{1.579923in}{0.818915in}}{\pgfqpoint{1.576017in}{0.815008in}}%
\pgfpathcurveto{\pgfqpoint{1.572110in}{0.811102in}}{\pgfqpoint{1.569915in}{0.805802in}}{\pgfqpoint{1.569915in}{0.800277in}}%
\pgfpathcurveto{\pgfqpoint{1.569915in}{0.794752in}}{\pgfqpoint{1.572110in}{0.789452in}}{\pgfqpoint{1.576017in}{0.785546in}}%
\pgfpathcurveto{\pgfqpoint{1.579923in}{0.781639in}}{\pgfqpoint{1.585223in}{0.779444in}}{\pgfqpoint{1.590748in}{0.779444in}}%
\pgfpathclose%
\pgfusepath{stroke,fill}%
\end{pgfscope}%
\begin{pgfscope}%
\pgfpathrectangle{\pgfqpoint{0.403630in}{0.251348in}}{\pgfqpoint{1.395000in}{1.359000in}}%
\pgfusepath{clip}%
\pgfsetbuttcap%
\pgfsetroundjoin%
\definecolor{currentfill}{rgb}{0.686275,0.352941,0.313725}%
\pgfsetfillcolor{currentfill}%
\pgfsetlinewidth{0.501875pt}%
\definecolor{currentstroke}{rgb}{0.686275,0.352941,0.313725}%
\pgfsetstrokecolor{currentstroke}%
\pgfsetdash{}{0pt}%
\pgfpathmoveto{\pgfqpoint{1.530572in}{0.720820in}}%
\pgfpathcurveto{\pgfqpoint{1.536097in}{0.720820in}}{\pgfqpoint{1.541396in}{0.723015in}}{\pgfqpoint{1.545303in}{0.726922in}}%
\pgfpathcurveto{\pgfqpoint{1.549210in}{0.730829in}}{\pgfqpoint{1.551405in}{0.736128in}}{\pgfqpoint{1.551405in}{0.741654in}}%
\pgfpathcurveto{\pgfqpoint{1.551405in}{0.747179in}}{\pgfqpoint{1.549210in}{0.752478in}}{\pgfqpoint{1.545303in}{0.756385in}}%
\pgfpathcurveto{\pgfqpoint{1.541396in}{0.760292in}}{\pgfqpoint{1.536097in}{0.762487in}}{\pgfqpoint{1.530572in}{0.762487in}}%
\pgfpathcurveto{\pgfqpoint{1.525047in}{0.762487in}}{\pgfqpoint{1.519747in}{0.760292in}}{\pgfqpoint{1.515840in}{0.756385in}}%
\pgfpathcurveto{\pgfqpoint{1.511933in}{0.752478in}}{\pgfqpoint{1.509738in}{0.747179in}}{\pgfqpoint{1.509738in}{0.741654in}}%
\pgfpathcurveto{\pgfqpoint{1.509738in}{0.736128in}}{\pgfqpoint{1.511933in}{0.730829in}}{\pgfqpoint{1.515840in}{0.726922in}}%
\pgfpathcurveto{\pgfqpoint{1.519747in}{0.723015in}}{\pgfqpoint{1.525047in}{0.720820in}}{\pgfqpoint{1.530572in}{0.720820in}}%
\pgfpathclose%
\pgfusepath{stroke,fill}%
\end{pgfscope}%
\begin{pgfscope}%
\pgfpathrectangle{\pgfqpoint{0.403630in}{0.251348in}}{\pgfqpoint{1.395000in}{1.359000in}}%
\pgfusepath{clip}%
\pgfsetbuttcap%
\pgfsetroundjoin%
\definecolor{currentfill}{rgb}{0.686275,0.352941,0.313725}%
\pgfsetfillcolor{currentfill}%
\pgfsetlinewidth{0.501875pt}%
\definecolor{currentstroke}{rgb}{0.686275,0.352941,0.313725}%
\pgfsetstrokecolor{currentstroke}%
\pgfsetdash{}{0pt}%
\pgfpathmoveto{\pgfqpoint{1.185925in}{1.408314in}}%
\pgfpathcurveto{\pgfqpoint{1.191450in}{1.408314in}}{\pgfqpoint{1.196749in}{1.410509in}}{\pgfqpoint{1.200656in}{1.414416in}}%
\pgfpathcurveto{\pgfqpoint{1.204563in}{1.418323in}}{\pgfqpoint{1.206758in}{1.423623in}}{\pgfqpoint{1.206758in}{1.429148in}}%
\pgfpathcurveto{\pgfqpoint{1.206758in}{1.434673in}}{\pgfqpoint{1.204563in}{1.439972in}}{\pgfqpoint{1.200656in}{1.443879in}}%
\pgfpathcurveto{\pgfqpoint{1.196749in}{1.447786in}}{\pgfqpoint{1.191450in}{1.449981in}}{\pgfqpoint{1.185925in}{1.449981in}}%
\pgfpathcurveto{\pgfqpoint{1.180399in}{1.449981in}}{\pgfqpoint{1.175100in}{1.447786in}}{\pgfqpoint{1.171193in}{1.443879in}}%
\pgfpathcurveto{\pgfqpoint{1.167286in}{1.439972in}}{\pgfqpoint{1.165091in}{1.434673in}}{\pgfqpoint{1.165091in}{1.429148in}}%
\pgfpathcurveto{\pgfqpoint{1.165091in}{1.423623in}}{\pgfqpoint{1.167286in}{1.418323in}}{\pgfqpoint{1.171193in}{1.414416in}}%
\pgfpathcurveto{\pgfqpoint{1.175100in}{1.410509in}}{\pgfqpoint{1.180399in}{1.408314in}}{\pgfqpoint{1.185925in}{1.408314in}}%
\pgfpathclose%
\pgfusepath{stroke,fill}%
\end{pgfscope}%
\begin{pgfscope}%
\pgfpathrectangle{\pgfqpoint{0.403630in}{0.251348in}}{\pgfqpoint{1.395000in}{1.359000in}}%
\pgfusepath{clip}%
\pgfsetbuttcap%
\pgfsetroundjoin%
\definecolor{currentfill}{rgb}{0.686275,0.352941,0.313725}%
\pgfsetfillcolor{currentfill}%
\pgfsetlinewidth{0.501875pt}%
\definecolor{currentstroke}{rgb}{0.686275,0.352941,0.313725}%
\pgfsetstrokecolor{currentstroke}%
\pgfsetdash{}{0pt}%
\pgfpathmoveto{\pgfqpoint{1.585277in}{0.779444in}}%
\pgfpathcurveto{\pgfqpoint{1.590803in}{0.779444in}}{\pgfqpoint{1.596102in}{0.781639in}}{\pgfqpoint{1.600009in}{0.785546in}}%
\pgfpathcurveto{\pgfqpoint{1.603916in}{0.789452in}}{\pgfqpoint{1.606111in}{0.794752in}}{\pgfqpoint{1.606111in}{0.800277in}}%
\pgfpathcurveto{\pgfqpoint{1.606111in}{0.805802in}}{\pgfqpoint{1.603916in}{0.811102in}}{\pgfqpoint{1.600009in}{0.815008in}}%
\pgfpathcurveto{\pgfqpoint{1.596102in}{0.818915in}}{\pgfqpoint{1.590803in}{0.821110in}}{\pgfqpoint{1.585277in}{0.821110in}}%
\pgfpathcurveto{\pgfqpoint{1.579752in}{0.821110in}}{\pgfqpoint{1.574453in}{0.818915in}}{\pgfqpoint{1.570546in}{0.815008in}}%
\pgfpathcurveto{\pgfqpoint{1.566639in}{0.811102in}}{\pgfqpoint{1.564444in}{0.805802in}}{\pgfqpoint{1.564444in}{0.800277in}}%
\pgfpathcurveto{\pgfqpoint{1.564444in}{0.794752in}}{\pgfqpoint{1.566639in}{0.789452in}}{\pgfqpoint{1.570546in}{0.785546in}}%
\pgfpathcurveto{\pgfqpoint{1.574453in}{0.781639in}}{\pgfqpoint{1.579752in}{0.779444in}}{\pgfqpoint{1.585277in}{0.779444in}}%
\pgfpathclose%
\pgfusepath{stroke,fill}%
\end{pgfscope}%
\begin{pgfscope}%
\pgfpathrectangle{\pgfqpoint{0.403630in}{0.251348in}}{\pgfqpoint{1.395000in}{1.359000in}}%
\pgfusepath{clip}%
\pgfsetbuttcap%
\pgfsetroundjoin%
\definecolor{currentfill}{rgb}{0.686275,0.352941,0.313725}%
\pgfsetfillcolor{currentfill}%
\pgfsetlinewidth{0.501875pt}%
\definecolor{currentstroke}{rgb}{0.686275,0.352941,0.313725}%
\pgfsetstrokecolor{currentstroke}%
\pgfsetdash{}{0pt}%
\pgfpathmoveto{\pgfqpoint{0.715454in}{0.592914in}}%
\pgfpathcurveto{\pgfqpoint{0.720979in}{0.592914in}}{\pgfqpoint{0.726279in}{0.595109in}}{\pgfqpoint{0.730185in}{0.599016in}}%
\pgfpathcurveto{\pgfqpoint{0.734092in}{0.602923in}}{\pgfqpoint{0.736287in}{0.608223in}}{\pgfqpoint{0.736287in}{0.613748in}}%
\pgfpathcurveto{\pgfqpoint{0.736287in}{0.619273in}}{\pgfqpoint{0.734092in}{0.624572in}}{\pgfqpoint{0.730185in}{0.628479in}}%
\pgfpathcurveto{\pgfqpoint{0.726279in}{0.632386in}}{\pgfqpoint{0.720979in}{0.634581in}}{\pgfqpoint{0.715454in}{0.634581in}}%
\pgfpathcurveto{\pgfqpoint{0.709929in}{0.634581in}}{\pgfqpoint{0.704629in}{0.632386in}}{\pgfqpoint{0.700723in}{0.628479in}}%
\pgfpathcurveto{\pgfqpoint{0.696816in}{0.624572in}}{\pgfqpoint{0.694621in}{0.619273in}}{\pgfqpoint{0.694621in}{0.613748in}}%
\pgfpathcurveto{\pgfqpoint{0.694621in}{0.608223in}}{\pgfqpoint{0.696816in}{0.602923in}}{\pgfqpoint{0.700723in}{0.599016in}}%
\pgfpathcurveto{\pgfqpoint{0.704629in}{0.595109in}}{\pgfqpoint{0.709929in}{0.592914in}}{\pgfqpoint{0.715454in}{0.592914in}}%
\pgfpathclose%
\pgfusepath{stroke,fill}%
\end{pgfscope}%
\begin{pgfscope}%
\pgfpathrectangle{\pgfqpoint{0.403630in}{0.251348in}}{\pgfqpoint{1.395000in}{1.359000in}}%
\pgfusepath{clip}%
\pgfsetbuttcap%
\pgfsetroundjoin%
\definecolor{currentfill}{rgb}{0.686275,0.352941,0.313725}%
\pgfsetfillcolor{currentfill}%
\pgfsetlinewidth{0.501875pt}%
\definecolor{currentstroke}{rgb}{0.686275,0.352941,0.313725}%
\pgfsetstrokecolor{currentstroke}%
\pgfsetdash{}{0pt}%
\pgfpathmoveto{\pgfqpoint{1.590748in}{1.024597in}}%
\pgfpathcurveto{\pgfqpoint{1.596273in}{1.024597in}}{\pgfqpoint{1.601573in}{1.026792in}}{\pgfqpoint{1.605479in}{1.030699in}}%
\pgfpathcurveto{\pgfqpoint{1.609386in}{1.034605in}}{\pgfqpoint{1.611581in}{1.039905in}}{\pgfqpoint{1.611581in}{1.045430in}}%
\pgfpathcurveto{\pgfqpoint{1.611581in}{1.050955in}}{\pgfqpoint{1.609386in}{1.056255in}}{\pgfqpoint{1.605479in}{1.060161in}}%
\pgfpathcurveto{\pgfqpoint{1.601573in}{1.064068in}}{\pgfqpoint{1.596273in}{1.066263in}}{\pgfqpoint{1.590748in}{1.066263in}}%
\pgfpathcurveto{\pgfqpoint{1.585223in}{1.066263in}}{\pgfqpoint{1.579923in}{1.064068in}}{\pgfqpoint{1.576017in}{1.060161in}}%
\pgfpathcurveto{\pgfqpoint{1.572110in}{1.056255in}}{\pgfqpoint{1.569915in}{1.050955in}}{\pgfqpoint{1.569915in}{1.045430in}}%
\pgfpathcurveto{\pgfqpoint{1.569915in}{1.039905in}}{\pgfqpoint{1.572110in}{1.034605in}}{\pgfqpoint{1.576017in}{1.030699in}}%
\pgfpathcurveto{\pgfqpoint{1.579923in}{1.026792in}}{\pgfqpoint{1.585223in}{1.024597in}}{\pgfqpoint{1.590748in}{1.024597in}}%
\pgfpathclose%
\pgfusepath{stroke,fill}%
\end{pgfscope}%
\begin{pgfscope}%
\pgfpathrectangle{\pgfqpoint{0.403630in}{0.251348in}}{\pgfqpoint{1.395000in}{1.359000in}}%
\pgfusepath{clip}%
\pgfsetbuttcap%
\pgfsetroundjoin%
\definecolor{currentfill}{rgb}{0.686275,0.352941,0.313725}%
\pgfsetfillcolor{currentfill}%
\pgfsetlinewidth{0.501875pt}%
\definecolor{currentstroke}{rgb}{0.686275,0.352941,0.313725}%
\pgfsetstrokecolor{currentstroke}%
\pgfsetdash{}{0pt}%
\pgfpathmoveto{\pgfqpoint{0.720925in}{1.131185in}}%
\pgfpathcurveto{\pgfqpoint{0.726450in}{1.131185in}}{\pgfqpoint{0.731749in}{1.133380in}}{\pgfqpoint{0.735656in}{1.137287in}}%
\pgfpathcurveto{\pgfqpoint{0.739563in}{1.141194in}}{\pgfqpoint{0.741758in}{1.146493in}}{\pgfqpoint{0.741758in}{1.152018in}}%
\pgfpathcurveto{\pgfqpoint{0.741758in}{1.157543in}}{\pgfqpoint{0.739563in}{1.162843in}}{\pgfqpoint{0.735656in}{1.166750in}}%
\pgfpathcurveto{\pgfqpoint{0.731749in}{1.170656in}}{\pgfqpoint{0.726450in}{1.172852in}}{\pgfqpoint{0.720925in}{1.172852in}}%
\pgfpathcurveto{\pgfqpoint{0.715399in}{1.172852in}}{\pgfqpoint{0.710100in}{1.170656in}}{\pgfqpoint{0.706193in}{1.166750in}}%
\pgfpathcurveto{\pgfqpoint{0.702286in}{1.162843in}}{\pgfqpoint{0.700091in}{1.157543in}}{\pgfqpoint{0.700091in}{1.152018in}}%
\pgfpathcurveto{\pgfqpoint{0.700091in}{1.146493in}}{\pgfqpoint{0.702286in}{1.141194in}}{\pgfqpoint{0.706193in}{1.137287in}}%
\pgfpathcurveto{\pgfqpoint{0.710100in}{1.133380in}}{\pgfqpoint{0.715399in}{1.131185in}}{\pgfqpoint{0.720925in}{1.131185in}}%
\pgfpathclose%
\pgfusepath{stroke,fill}%
\end{pgfscope}%
\begin{pgfscope}%
\pgfpathrectangle{\pgfqpoint{0.403630in}{0.251348in}}{\pgfqpoint{1.395000in}{1.359000in}}%
\pgfusepath{clip}%
\pgfsetbuttcap%
\pgfsetroundjoin%
\definecolor{currentfill}{rgb}{0.686275,0.352941,0.313725}%
\pgfsetfillcolor{currentfill}%
\pgfsetlinewidth{0.501875pt}%
\definecolor{currentstroke}{rgb}{0.686275,0.352941,0.313725}%
\pgfsetstrokecolor{currentstroke}%
\pgfsetdash{}{0pt}%
\pgfpathmoveto{\pgfqpoint{0.677160in}{1.099208in}}%
\pgfpathcurveto{\pgfqpoint{0.682685in}{1.099208in}}{\pgfqpoint{0.687984in}{1.101404in}}{\pgfqpoint{0.691891in}{1.105310in}}%
\pgfpathcurveto{\pgfqpoint{0.695798in}{1.109217in}}{\pgfqpoint{0.697993in}{1.114517in}}{\pgfqpoint{0.697993in}{1.120042in}}%
\pgfpathcurveto{\pgfqpoint{0.697993in}{1.125567in}}{\pgfqpoint{0.695798in}{1.130866in}}{\pgfqpoint{0.691891in}{1.134773in}}%
\pgfpathcurveto{\pgfqpoint{0.687984in}{1.138680in}}{\pgfqpoint{0.682685in}{1.140875in}}{\pgfqpoint{0.677160in}{1.140875in}}%
\pgfpathcurveto{\pgfqpoint{0.671635in}{1.140875in}}{\pgfqpoint{0.666335in}{1.138680in}}{\pgfqpoint{0.662428in}{1.134773in}}%
\pgfpathcurveto{\pgfqpoint{0.658522in}{1.130866in}}{\pgfqpoint{0.656326in}{1.125567in}}{\pgfqpoint{0.656326in}{1.120042in}}%
\pgfpathcurveto{\pgfqpoint{0.656326in}{1.114517in}}{\pgfqpoint{0.658522in}{1.109217in}}{\pgfqpoint{0.662428in}{1.105310in}}%
\pgfpathcurveto{\pgfqpoint{0.666335in}{1.101404in}}{\pgfqpoint{0.671635in}{1.099208in}}{\pgfqpoint{0.677160in}{1.099208in}}%
\pgfpathclose%
\pgfusepath{stroke,fill}%
\end{pgfscope}%
\begin{pgfscope}%
\pgfpathrectangle{\pgfqpoint{0.403630in}{0.251348in}}{\pgfqpoint{1.395000in}{1.359000in}}%
\pgfusepath{clip}%
\pgfsetbuttcap%
\pgfsetroundjoin%
\definecolor{currentfill}{rgb}{0.686275,0.352941,0.313725}%
\pgfsetfillcolor{currentfill}%
\pgfsetlinewidth{0.501875pt}%
\definecolor{currentstroke}{rgb}{0.686275,0.352941,0.313725}%
\pgfsetstrokecolor{currentstroke}%
\pgfsetdash{}{0pt}%
\pgfpathmoveto{\pgfqpoint{1.530572in}{0.811420in}}%
\pgfpathcurveto{\pgfqpoint{1.536097in}{0.811420in}}{\pgfqpoint{1.541396in}{0.813615in}}{\pgfqpoint{1.545303in}{0.817522in}}%
\pgfpathcurveto{\pgfqpoint{1.549210in}{0.821429in}}{\pgfqpoint{1.551405in}{0.826728in}}{\pgfqpoint{1.551405in}{0.832254in}}%
\pgfpathcurveto{\pgfqpoint{1.551405in}{0.837779in}}{\pgfqpoint{1.549210in}{0.843078in}}{\pgfqpoint{1.545303in}{0.846985in}}%
\pgfpathcurveto{\pgfqpoint{1.541396in}{0.850892in}}{\pgfqpoint{1.536097in}{0.853087in}}{\pgfqpoint{1.530572in}{0.853087in}}%
\pgfpathcurveto{\pgfqpoint{1.525047in}{0.853087in}}{\pgfqpoint{1.519747in}{0.850892in}}{\pgfqpoint{1.515840in}{0.846985in}}%
\pgfpathcurveto{\pgfqpoint{1.511933in}{0.843078in}}{\pgfqpoint{1.509738in}{0.837779in}}{\pgfqpoint{1.509738in}{0.832254in}}%
\pgfpathcurveto{\pgfqpoint{1.509738in}{0.826728in}}{\pgfqpoint{1.511933in}{0.821429in}}{\pgfqpoint{1.515840in}{0.817522in}}%
\pgfpathcurveto{\pgfqpoint{1.519747in}{0.813615in}}{\pgfqpoint{1.525047in}{0.811420in}}{\pgfqpoint{1.530572in}{0.811420in}}%
\pgfpathclose%
\pgfusepath{stroke,fill}%
\end{pgfscope}%
\begin{pgfscope}%
\pgfpathrectangle{\pgfqpoint{0.403630in}{0.251348in}}{\pgfqpoint{1.395000in}{1.359000in}}%
\pgfusepath{clip}%
\pgfsetbuttcap%
\pgfsetroundjoin%
\definecolor{currentfill}{rgb}{0.686275,0.352941,0.313725}%
\pgfsetfillcolor{currentfill}%
\pgfsetlinewidth{0.501875pt}%
\definecolor{currentstroke}{rgb}{0.686275,0.352941,0.313725}%
\pgfsetstrokecolor{currentstroke}%
\pgfsetdash{}{0pt}%
\pgfpathmoveto{\pgfqpoint{0.874101in}{1.307055in}}%
\pgfpathcurveto{\pgfqpoint{0.879626in}{1.307055in}}{\pgfqpoint{0.884926in}{1.309251in}}{\pgfqpoint{0.888832in}{1.313157in}}%
\pgfpathcurveto{\pgfqpoint{0.892739in}{1.317064in}}{\pgfqpoint{0.894934in}{1.322364in}}{\pgfqpoint{0.894934in}{1.327889in}}%
\pgfpathcurveto{\pgfqpoint{0.894934in}{1.333414in}}{\pgfqpoint{0.892739in}{1.338713in}}{\pgfqpoint{0.888832in}{1.342620in}}%
\pgfpathcurveto{\pgfqpoint{0.884926in}{1.346527in}}{\pgfqpoint{0.879626in}{1.348722in}}{\pgfqpoint{0.874101in}{1.348722in}}%
\pgfpathcurveto{\pgfqpoint{0.868576in}{1.348722in}}{\pgfqpoint{0.863276in}{1.346527in}}{\pgfqpoint{0.859370in}{1.342620in}}%
\pgfpathcurveto{\pgfqpoint{0.855463in}{1.338713in}}{\pgfqpoint{0.853268in}{1.333414in}}{\pgfqpoint{0.853268in}{1.327889in}}%
\pgfpathcurveto{\pgfqpoint{0.853268in}{1.322364in}}{\pgfqpoint{0.855463in}{1.317064in}}{\pgfqpoint{0.859370in}{1.313157in}}%
\pgfpathcurveto{\pgfqpoint{0.863276in}{1.309251in}}{\pgfqpoint{0.868576in}{1.307055in}}{\pgfqpoint{0.874101in}{1.307055in}}%
\pgfpathclose%
\pgfusepath{stroke,fill}%
\end{pgfscope}%
\begin{pgfscope}%
\pgfpathrectangle{\pgfqpoint{0.403630in}{0.251348in}}{\pgfqpoint{1.395000in}{1.359000in}}%
\pgfusepath{clip}%
\pgfsetbuttcap%
\pgfsetroundjoin%
\definecolor{currentfill}{rgb}{0.686275,0.352941,0.313725}%
\pgfsetfillcolor{currentfill}%
\pgfsetlinewidth{0.501875pt}%
\definecolor{currentstroke}{rgb}{0.686275,0.352941,0.313725}%
\pgfsetstrokecolor{currentstroke}%
\pgfsetdash{}{0pt}%
\pgfpathmoveto{\pgfqpoint{1.224219in}{0.438361in}}%
\pgfpathcurveto{\pgfqpoint{1.229744in}{0.438361in}}{\pgfqpoint{1.235043in}{0.440556in}}{\pgfqpoint{1.238950in}{0.444463in}}%
\pgfpathcurveto{\pgfqpoint{1.242857in}{0.448370in}}{\pgfqpoint{1.245052in}{0.453670in}}{\pgfqpoint{1.245052in}{0.459195in}}%
\pgfpathcurveto{\pgfqpoint{1.245052in}{0.464720in}}{\pgfqpoint{1.242857in}{0.470019in}}{\pgfqpoint{1.238950in}{0.473926in}}%
\pgfpathcurveto{\pgfqpoint{1.235043in}{0.477833in}}{\pgfqpoint{1.229744in}{0.480028in}}{\pgfqpoint{1.224219in}{0.480028in}}%
\pgfpathcurveto{\pgfqpoint{1.218694in}{0.480028in}}{\pgfqpoint{1.213394in}{0.477833in}}{\pgfqpoint{1.209487in}{0.473926in}}%
\pgfpathcurveto{\pgfqpoint{1.205580in}{0.470019in}}{\pgfqpoint{1.203385in}{0.464720in}}{\pgfqpoint{1.203385in}{0.459195in}}%
\pgfpathcurveto{\pgfqpoint{1.203385in}{0.453670in}}{\pgfqpoint{1.205580in}{0.448370in}}{\pgfqpoint{1.209487in}{0.444463in}}%
\pgfpathcurveto{\pgfqpoint{1.213394in}{0.440556in}}{\pgfqpoint{1.218694in}{0.438361in}}{\pgfqpoint{1.224219in}{0.438361in}}%
\pgfpathclose%
\pgfusepath{stroke,fill}%
\end{pgfscope}%
\begin{pgfscope}%
\pgfpathrectangle{\pgfqpoint{0.403630in}{0.251348in}}{\pgfqpoint{1.395000in}{1.359000in}}%
\pgfusepath{clip}%
\pgfsetbuttcap%
\pgfsetroundjoin%
\definecolor{currentfill}{rgb}{0.686275,0.352941,0.313725}%
\pgfsetfillcolor{currentfill}%
\pgfsetlinewidth{0.501875pt}%
\definecolor{currentstroke}{rgb}{0.686275,0.352941,0.313725}%
\pgfsetstrokecolor{currentstroke}%
\pgfsetdash{}{0pt}%
\pgfpathmoveto{\pgfqpoint{0.611513in}{0.806091in}}%
\pgfpathcurveto{\pgfqpoint{0.617038in}{0.806091in}}{\pgfqpoint{0.622337in}{0.808286in}}{\pgfqpoint{0.626244in}{0.812193in}}%
\pgfpathcurveto{\pgfqpoint{0.630151in}{0.816100in}}{\pgfqpoint{0.632346in}{0.821399in}}{\pgfqpoint{0.632346in}{0.826924in}}%
\pgfpathcurveto{\pgfqpoint{0.632346in}{0.832449in}}{\pgfqpoint{0.630151in}{0.837749in}}{\pgfqpoint{0.626244in}{0.841656in}}%
\pgfpathcurveto{\pgfqpoint{0.622337in}{0.845562in}}{\pgfqpoint{0.617038in}{0.847757in}}{\pgfqpoint{0.611513in}{0.847757in}}%
\pgfpathcurveto{\pgfqpoint{0.605988in}{0.847757in}}{\pgfqpoint{0.600688in}{0.845562in}}{\pgfqpoint{0.596781in}{0.841656in}}%
\pgfpathcurveto{\pgfqpoint{0.592875in}{0.837749in}}{\pgfqpoint{0.590679in}{0.832449in}}{\pgfqpoint{0.590679in}{0.826924in}}%
\pgfpathcurveto{\pgfqpoint{0.590679in}{0.821399in}}{\pgfqpoint{0.592875in}{0.816100in}}{\pgfqpoint{0.596781in}{0.812193in}}%
\pgfpathcurveto{\pgfqpoint{0.600688in}{0.808286in}}{\pgfqpoint{0.605988in}{0.806091in}}{\pgfqpoint{0.611513in}{0.806091in}}%
\pgfpathclose%
\pgfusepath{stroke,fill}%
\end{pgfscope}%
\begin{pgfscope}%
\pgfpathrectangle{\pgfqpoint{0.403630in}{0.251348in}}{\pgfqpoint{1.395000in}{1.359000in}}%
\pgfusepath{clip}%
\pgfsetbuttcap%
\pgfsetroundjoin%
\definecolor{currentfill}{rgb}{0.686275,0.352941,0.313725}%
\pgfsetfillcolor{currentfill}%
\pgfsetlinewidth{0.501875pt}%
\definecolor{currentstroke}{rgb}{0.686275,0.352941,0.313725}%
\pgfsetstrokecolor{currentstroke}%
\pgfsetdash{}{0pt}%
\pgfpathmoveto{\pgfqpoint{0.813925in}{1.312385in}}%
\pgfpathcurveto{\pgfqpoint{0.819450in}{1.312385in}}{\pgfqpoint{0.824749in}{1.314580in}}{\pgfqpoint{0.828656in}{1.318487in}}%
\pgfpathcurveto{\pgfqpoint{0.832563in}{1.322394in}}{\pgfqpoint{0.834758in}{1.327693in}}{\pgfqpoint{0.834758in}{1.333218in}}%
\pgfpathcurveto{\pgfqpoint{0.834758in}{1.338743in}}{\pgfqpoint{0.832563in}{1.344043in}}{\pgfqpoint{0.828656in}{1.347950in}}%
\pgfpathcurveto{\pgfqpoint{0.824749in}{1.351856in}}{\pgfqpoint{0.819450in}{1.354052in}}{\pgfqpoint{0.813925in}{1.354052in}}%
\pgfpathcurveto{\pgfqpoint{0.808399in}{1.354052in}}{\pgfqpoint{0.803100in}{1.351856in}}{\pgfqpoint{0.799193in}{1.347950in}}%
\pgfpathcurveto{\pgfqpoint{0.795286in}{1.344043in}}{\pgfqpoint{0.793091in}{1.338743in}}{\pgfqpoint{0.793091in}{1.333218in}}%
\pgfpathcurveto{\pgfqpoint{0.793091in}{1.327693in}}{\pgfqpoint{0.795286in}{1.322394in}}{\pgfqpoint{0.799193in}{1.318487in}}%
\pgfpathcurveto{\pgfqpoint{0.803100in}{1.314580in}}{\pgfqpoint{0.808399in}{1.312385in}}{\pgfqpoint{0.813925in}{1.312385in}}%
\pgfpathclose%
\pgfusepath{stroke,fill}%
\end{pgfscope}%
\begin{pgfscope}%
\pgfpathrectangle{\pgfqpoint{0.403630in}{0.251348in}}{\pgfqpoint{1.395000in}{1.359000in}}%
\pgfusepath{clip}%
\pgfsetbuttcap%
\pgfsetroundjoin%
\definecolor{currentfill}{rgb}{0.686275,0.352941,0.313725}%
\pgfsetfillcolor{currentfill}%
\pgfsetlinewidth{0.501875pt}%
\definecolor{currentstroke}{rgb}{0.686275,0.352941,0.313725}%
\pgfsetstrokecolor{currentstroke}%
\pgfsetdash{}{0pt}%
\pgfpathmoveto{\pgfqpoint{1.224219in}{0.475667in}}%
\pgfpathcurveto{\pgfqpoint{1.229744in}{0.475667in}}{\pgfqpoint{1.235043in}{0.477862in}}{\pgfqpoint{1.238950in}{0.481769in}}%
\pgfpathcurveto{\pgfqpoint{1.242857in}{0.485676in}}{\pgfqpoint{1.245052in}{0.490976in}}{\pgfqpoint{1.245052in}{0.496501in}}%
\pgfpathcurveto{\pgfqpoint{1.245052in}{0.502026in}}{\pgfqpoint{1.242857in}{0.507325in}}{\pgfqpoint{1.238950in}{0.511232in}}%
\pgfpathcurveto{\pgfqpoint{1.235043in}{0.515139in}}{\pgfqpoint{1.229744in}{0.517334in}}{\pgfqpoint{1.224219in}{0.517334in}}%
\pgfpathcurveto{\pgfqpoint{1.218694in}{0.517334in}}{\pgfqpoint{1.213394in}{0.515139in}}{\pgfqpoint{1.209487in}{0.511232in}}%
\pgfpathcurveto{\pgfqpoint{1.205580in}{0.507325in}}{\pgfqpoint{1.203385in}{0.502026in}}{\pgfqpoint{1.203385in}{0.496501in}}%
\pgfpathcurveto{\pgfqpoint{1.203385in}{0.490976in}}{\pgfqpoint{1.205580in}{0.485676in}}{\pgfqpoint{1.209487in}{0.481769in}}%
\pgfpathcurveto{\pgfqpoint{1.213394in}{0.477862in}}{\pgfqpoint{1.218694in}{0.475667in}}{\pgfqpoint{1.224219in}{0.475667in}}%
\pgfpathclose%
\pgfusepath{stroke,fill}%
\end{pgfscope}%
\begin{pgfscope}%
\pgfpathrectangle{\pgfqpoint{0.403630in}{0.251348in}}{\pgfqpoint{1.395000in}{1.359000in}}%
\pgfusepath{clip}%
\pgfsetbuttcap%
\pgfsetroundjoin%
\definecolor{currentfill}{rgb}{0.686275,0.352941,0.313725}%
\pgfsetfillcolor{currentfill}%
\pgfsetlinewidth{0.501875pt}%
\definecolor{currentstroke}{rgb}{0.686275,0.352941,0.313725}%
\pgfsetstrokecolor{currentstroke}%
\pgfsetdash{}{0pt}%
\pgfpathmoveto{\pgfqpoint{1.262513in}{0.438361in}}%
\pgfpathcurveto{\pgfqpoint{1.268038in}{0.438361in}}{\pgfqpoint{1.273337in}{0.440556in}}{\pgfqpoint{1.277244in}{0.444463in}}%
\pgfpathcurveto{\pgfqpoint{1.281151in}{0.448370in}}{\pgfqpoint{1.283346in}{0.453670in}}{\pgfqpoint{1.283346in}{0.459195in}}%
\pgfpathcurveto{\pgfqpoint{1.283346in}{0.464720in}}{\pgfqpoint{1.281151in}{0.470019in}}{\pgfqpoint{1.277244in}{0.473926in}}%
\pgfpathcurveto{\pgfqpoint{1.273337in}{0.477833in}}{\pgfqpoint{1.268038in}{0.480028in}}{\pgfqpoint{1.262513in}{0.480028in}}%
\pgfpathcurveto{\pgfqpoint{1.256988in}{0.480028in}}{\pgfqpoint{1.251688in}{0.477833in}}{\pgfqpoint{1.247781in}{0.473926in}}%
\pgfpathcurveto{\pgfqpoint{1.243875in}{0.470019in}}{\pgfqpoint{1.241679in}{0.464720in}}{\pgfqpoint{1.241679in}{0.459195in}}%
\pgfpathcurveto{\pgfqpoint{1.241679in}{0.453670in}}{\pgfqpoint{1.243875in}{0.448370in}}{\pgfqpoint{1.247781in}{0.444463in}}%
\pgfpathcurveto{\pgfqpoint{1.251688in}{0.440556in}}{\pgfqpoint{1.256988in}{0.438361in}}{\pgfqpoint{1.262513in}{0.438361in}}%
\pgfpathclose%
\pgfusepath{stroke,fill}%
\end{pgfscope}%
\begin{pgfscope}%
\pgfpathrectangle{\pgfqpoint{0.403630in}{0.251348in}}{\pgfqpoint{1.395000in}{1.359000in}}%
\pgfusepath{clip}%
\pgfsetbuttcap%
\pgfsetroundjoin%
\definecolor{currentfill}{rgb}{0.686275,0.352941,0.313725}%
\pgfsetfillcolor{currentfill}%
\pgfsetlinewidth{0.501875pt}%
\definecolor{currentstroke}{rgb}{0.686275,0.352941,0.313725}%
\pgfsetstrokecolor{currentstroke}%
\pgfsetdash{}{0pt}%
\pgfpathmoveto{\pgfqpoint{1.563395in}{0.854055in}}%
\pgfpathcurveto{\pgfqpoint{1.568920in}{0.854055in}}{\pgfqpoint{1.574220in}{0.856251in}}{\pgfqpoint{1.578127in}{0.860157in}}%
\pgfpathcurveto{\pgfqpoint{1.582033in}{0.864064in}}{\pgfqpoint{1.584228in}{0.869364in}}{\pgfqpoint{1.584228in}{0.874889in}}%
\pgfpathcurveto{\pgfqpoint{1.584228in}{0.880414in}}{\pgfqpoint{1.582033in}{0.885713in}}{\pgfqpoint{1.578127in}{0.889620in}}%
\pgfpathcurveto{\pgfqpoint{1.574220in}{0.893527in}}{\pgfqpoint{1.568920in}{0.895722in}}{\pgfqpoint{1.563395in}{0.895722in}}%
\pgfpathcurveto{\pgfqpoint{1.557870in}{0.895722in}}{\pgfqpoint{1.552571in}{0.893527in}}{\pgfqpoint{1.548664in}{0.889620in}}%
\pgfpathcurveto{\pgfqpoint{1.544757in}{0.885713in}}{\pgfqpoint{1.542562in}{0.880414in}}{\pgfqpoint{1.542562in}{0.874889in}}%
\pgfpathcurveto{\pgfqpoint{1.542562in}{0.869364in}}{\pgfqpoint{1.544757in}{0.864064in}}{\pgfqpoint{1.548664in}{0.860157in}}%
\pgfpathcurveto{\pgfqpoint{1.552571in}{0.856251in}}{\pgfqpoint{1.557870in}{0.854055in}}{\pgfqpoint{1.563395in}{0.854055in}}%
\pgfpathclose%
\pgfusepath{stroke,fill}%
\end{pgfscope}%
\begin{pgfscope}%
\pgfpathrectangle{\pgfqpoint{0.403630in}{0.251348in}}{\pgfqpoint{1.395000in}{1.359000in}}%
\pgfusepath{clip}%
\pgfsetbuttcap%
\pgfsetroundjoin%
\definecolor{currentfill}{rgb}{0.686275,0.352941,0.313725}%
\pgfsetfillcolor{currentfill}%
\pgfsetlinewidth{0.501875pt}%
\definecolor{currentstroke}{rgb}{0.686275,0.352941,0.313725}%
\pgfsetstrokecolor{currentstroke}%
\pgfsetdash{}{0pt}%
\pgfpathmoveto{\pgfqpoint{1.601689in}{1.008608in}}%
\pgfpathcurveto{\pgfqpoint{1.607214in}{1.008608in}}{\pgfqpoint{1.612514in}{1.010804in}}{\pgfqpoint{1.616421in}{1.014710in}}%
\pgfpathcurveto{\pgfqpoint{1.620327in}{1.018617in}}{\pgfqpoint{1.622523in}{1.023917in}}{\pgfqpoint{1.622523in}{1.029442in}}%
\pgfpathcurveto{\pgfqpoint{1.622523in}{1.034967in}}{\pgfqpoint{1.620327in}{1.040266in}}{\pgfqpoint{1.616421in}{1.044173in}}%
\pgfpathcurveto{\pgfqpoint{1.612514in}{1.048080in}}{\pgfqpoint{1.607214in}{1.050275in}}{\pgfqpoint{1.601689in}{1.050275in}}%
\pgfpathcurveto{\pgfqpoint{1.596164in}{1.050275in}}{\pgfqpoint{1.590865in}{1.048080in}}{\pgfqpoint{1.586958in}{1.044173in}}%
\pgfpathcurveto{\pgfqpoint{1.583051in}{1.040266in}}{\pgfqpoint{1.580856in}{1.034967in}}{\pgfqpoint{1.580856in}{1.029442in}}%
\pgfpathcurveto{\pgfqpoint{1.580856in}{1.023917in}}{\pgfqpoint{1.583051in}{1.018617in}}{\pgfqpoint{1.586958in}{1.014710in}}%
\pgfpathcurveto{\pgfqpoint{1.590865in}{1.010804in}}{\pgfqpoint{1.596164in}{1.008608in}}{\pgfqpoint{1.601689in}{1.008608in}}%
\pgfpathclose%
\pgfusepath{stroke,fill}%
\end{pgfscope}%
\begin{pgfscope}%
\pgfpathrectangle{\pgfqpoint{0.403630in}{0.251348in}}{\pgfqpoint{1.395000in}{1.359000in}}%
\pgfusepath{clip}%
\pgfsetbuttcap%
\pgfsetroundjoin%
\definecolor{currentfill}{rgb}{0.686275,0.352941,0.313725}%
\pgfsetfillcolor{currentfill}%
\pgfsetlinewidth{0.501875pt}%
\definecolor{currentstroke}{rgb}{0.686275,0.352941,0.313725}%
\pgfsetstrokecolor{currentstroke}%
\pgfsetdash{}{0pt}%
\pgfpathmoveto{\pgfqpoint{0.677160in}{1.179150in}}%
\pgfpathcurveto{\pgfqpoint{0.682685in}{1.179150in}}{\pgfqpoint{0.687984in}{1.181345in}}{\pgfqpoint{0.691891in}{1.185252in}}%
\pgfpathcurveto{\pgfqpoint{0.695798in}{1.189158in}}{\pgfqpoint{0.697993in}{1.194458in}}{\pgfqpoint{0.697993in}{1.199983in}}%
\pgfpathcurveto{\pgfqpoint{0.697993in}{1.205508in}}{\pgfqpoint{0.695798in}{1.210808in}}{\pgfqpoint{0.691891in}{1.214714in}}%
\pgfpathcurveto{\pgfqpoint{0.687984in}{1.218621in}}{\pgfqpoint{0.682685in}{1.220816in}}{\pgfqpoint{0.677160in}{1.220816in}}%
\pgfpathcurveto{\pgfqpoint{0.671635in}{1.220816in}}{\pgfqpoint{0.666335in}{1.218621in}}{\pgfqpoint{0.662428in}{1.214714in}}%
\pgfpathcurveto{\pgfqpoint{0.658522in}{1.210808in}}{\pgfqpoint{0.656326in}{1.205508in}}{\pgfqpoint{0.656326in}{1.199983in}}%
\pgfpathcurveto{\pgfqpoint{0.656326in}{1.194458in}}{\pgfqpoint{0.658522in}{1.189158in}}{\pgfqpoint{0.662428in}{1.185252in}}%
\pgfpathcurveto{\pgfqpoint{0.666335in}{1.181345in}}{\pgfqpoint{0.671635in}{1.179150in}}{\pgfqpoint{0.677160in}{1.179150in}}%
\pgfpathclose%
\pgfusepath{stroke,fill}%
\end{pgfscope}%
\begin{pgfscope}%
\pgfpathrectangle{\pgfqpoint{0.403630in}{0.251348in}}{\pgfqpoint{1.395000in}{1.359000in}}%
\pgfusepath{clip}%
\pgfsetbuttcap%
\pgfsetroundjoin%
\definecolor{currentfill}{rgb}{0.686275,0.352941,0.313725}%
\pgfsetfillcolor{currentfill}%
\pgfsetlinewidth{0.501875pt}%
\definecolor{currentstroke}{rgb}{0.686275,0.352941,0.313725}%
\pgfsetstrokecolor{currentstroke}%
\pgfsetdash{}{0pt}%
\pgfpathmoveto{\pgfqpoint{0.731866in}{0.576926in}}%
\pgfpathcurveto{\pgfqpoint{0.737391in}{0.576926in}}{\pgfqpoint{0.742690in}{0.579121in}}{\pgfqpoint{0.746597in}{0.583028in}}%
\pgfpathcurveto{\pgfqpoint{0.750504in}{0.586935in}}{\pgfqpoint{0.752699in}{0.592234in}}{\pgfqpoint{0.752699in}{0.597759in}}%
\pgfpathcurveto{\pgfqpoint{0.752699in}{0.603284in}}{\pgfqpoint{0.750504in}{0.608584in}}{\pgfqpoint{0.746597in}{0.612491in}}%
\pgfpathcurveto{\pgfqpoint{0.742690in}{0.616398in}}{\pgfqpoint{0.737391in}{0.618593in}}{\pgfqpoint{0.731866in}{0.618593in}}%
\pgfpathcurveto{\pgfqpoint{0.726341in}{0.618593in}}{\pgfqpoint{0.721041in}{0.616398in}}{\pgfqpoint{0.717134in}{0.612491in}}%
\pgfpathcurveto{\pgfqpoint{0.713228in}{0.608584in}}{\pgfqpoint{0.711032in}{0.603284in}}{\pgfqpoint{0.711032in}{0.597759in}}%
\pgfpathcurveto{\pgfqpoint{0.711032in}{0.592234in}}{\pgfqpoint{0.713228in}{0.586935in}}{\pgfqpoint{0.717134in}{0.583028in}}%
\pgfpathcurveto{\pgfqpoint{0.721041in}{0.579121in}}{\pgfqpoint{0.726341in}{0.576926in}}{\pgfqpoint{0.731866in}{0.576926in}}%
\pgfpathclose%
\pgfusepath{stroke,fill}%
\end{pgfscope}%
\begin{pgfscope}%
\pgfpathrectangle{\pgfqpoint{0.403630in}{0.251348in}}{\pgfqpoint{1.395000in}{1.359000in}}%
\pgfusepath{clip}%
\pgfsetbuttcap%
\pgfsetroundjoin%
\definecolor{currentfill}{rgb}{0.686275,0.352941,0.313725}%
\pgfsetfillcolor{currentfill}%
\pgfsetlinewidth{0.501875pt}%
\definecolor{currentstroke}{rgb}{0.686275,0.352941,0.313725}%
\pgfsetstrokecolor{currentstroke}%
\pgfsetdash{}{0pt}%
\pgfpathmoveto{\pgfqpoint{0.644336in}{0.971303in}}%
\pgfpathcurveto{\pgfqpoint{0.649861in}{0.971303in}}{\pgfqpoint{0.655161in}{0.973498in}}{\pgfqpoint{0.659068in}{0.977404in}}%
\pgfpathcurveto{\pgfqpoint{0.662975in}{0.981311in}}{\pgfqpoint{0.665170in}{0.986611in}}{\pgfqpoint{0.665170in}{0.992136in}}%
\pgfpathcurveto{\pgfqpoint{0.665170in}{0.997661in}}{\pgfqpoint{0.662975in}{1.002960in}}{\pgfqpoint{0.659068in}{1.006867in}}%
\pgfpathcurveto{\pgfqpoint{0.655161in}{1.010774in}}{\pgfqpoint{0.649861in}{1.012969in}}{\pgfqpoint{0.644336in}{1.012969in}}%
\pgfpathcurveto{\pgfqpoint{0.638811in}{1.012969in}}{\pgfqpoint{0.633512in}{1.010774in}}{\pgfqpoint{0.629605in}{1.006867in}}%
\pgfpathcurveto{\pgfqpoint{0.625698in}{1.002960in}}{\pgfqpoint{0.623503in}{0.997661in}}{\pgfqpoint{0.623503in}{0.992136in}}%
\pgfpathcurveto{\pgfqpoint{0.623503in}{0.986611in}}{\pgfqpoint{0.625698in}{0.981311in}}{\pgfqpoint{0.629605in}{0.977404in}}%
\pgfpathcurveto{\pgfqpoint{0.633512in}{0.973498in}}{\pgfqpoint{0.638811in}{0.971303in}}{\pgfqpoint{0.644336in}{0.971303in}}%
\pgfpathclose%
\pgfusepath{stroke,fill}%
\end{pgfscope}%
\begin{pgfscope}%
\pgfpathrectangle{\pgfqpoint{0.403630in}{0.251348in}}{\pgfqpoint{1.395000in}{1.359000in}}%
\pgfusepath{clip}%
\pgfsetbuttcap%
\pgfsetroundjoin%
\definecolor{currentfill}{rgb}{0.686275,0.352941,0.313725}%
\pgfsetfillcolor{currentfill}%
\pgfsetlinewidth{0.501875pt}%
\definecolor{currentstroke}{rgb}{0.686275,0.352941,0.313725}%
\pgfsetstrokecolor{currentstroke}%
\pgfsetdash{}{0pt}%
\pgfpathmoveto{\pgfqpoint{1.568866in}{1.131185in}}%
\pgfpathcurveto{\pgfqpoint{1.574391in}{1.131185in}}{\pgfqpoint{1.579690in}{1.133380in}}{\pgfqpoint{1.583597in}{1.137287in}}%
\pgfpathcurveto{\pgfqpoint{1.587504in}{1.141194in}}{\pgfqpoint{1.589699in}{1.146493in}}{\pgfqpoint{1.589699in}{1.152018in}}%
\pgfpathcurveto{\pgfqpoint{1.589699in}{1.157543in}}{\pgfqpoint{1.587504in}{1.162843in}}{\pgfqpoint{1.583597in}{1.166750in}}%
\pgfpathcurveto{\pgfqpoint{1.579690in}{1.170656in}}{\pgfqpoint{1.574391in}{1.172852in}}{\pgfqpoint{1.568866in}{1.172852in}}%
\pgfpathcurveto{\pgfqpoint{1.563341in}{1.172852in}}{\pgfqpoint{1.558041in}{1.170656in}}{\pgfqpoint{1.554134in}{1.166750in}}%
\pgfpathcurveto{\pgfqpoint{1.550228in}{1.162843in}}{\pgfqpoint{1.548032in}{1.157543in}}{\pgfqpoint{1.548032in}{1.152018in}}%
\pgfpathcurveto{\pgfqpoint{1.548032in}{1.146493in}}{\pgfqpoint{1.550228in}{1.141194in}}{\pgfqpoint{1.554134in}{1.137287in}}%
\pgfpathcurveto{\pgfqpoint{1.558041in}{1.133380in}}{\pgfqpoint{1.563341in}{1.131185in}}{\pgfqpoint{1.568866in}{1.131185in}}%
\pgfpathclose%
\pgfusepath{stroke,fill}%
\end{pgfscope}%
\begin{pgfscope}%
\pgfpathrectangle{\pgfqpoint{0.403630in}{0.251348in}}{\pgfqpoint{1.395000in}{1.359000in}}%
\pgfusepath{clip}%
\pgfsetbuttcap%
\pgfsetroundjoin%
\definecolor{currentfill}{rgb}{0.686275,0.352941,0.313725}%
\pgfsetfillcolor{currentfill}%
\pgfsetlinewidth{0.501875pt}%
\definecolor{currentstroke}{rgb}{0.686275,0.352941,0.313725}%
\pgfsetstrokecolor{currentstroke}%
\pgfsetdash{}{0pt}%
\pgfpathmoveto{\pgfqpoint{1.366454in}{0.534291in}}%
\pgfpathcurveto{\pgfqpoint{1.371979in}{0.534291in}}{\pgfqpoint{1.377279in}{0.536486in}}{\pgfqpoint{1.381185in}{0.540393in}}%
\pgfpathcurveto{\pgfqpoint{1.385092in}{0.544300in}}{\pgfqpoint{1.387287in}{0.549599in}}{\pgfqpoint{1.387287in}{0.555124in}}%
\pgfpathcurveto{\pgfqpoint{1.387287in}{0.560649in}}{\pgfqpoint{1.385092in}{0.565949in}}{\pgfqpoint{1.381185in}{0.569856in}}%
\pgfpathcurveto{\pgfqpoint{1.377279in}{0.573762in}}{\pgfqpoint{1.371979in}{0.575957in}}{\pgfqpoint{1.366454in}{0.575957in}}%
\pgfpathcurveto{\pgfqpoint{1.360929in}{0.575957in}}{\pgfqpoint{1.355629in}{0.573762in}}{\pgfqpoint{1.351723in}{0.569856in}}%
\pgfpathcurveto{\pgfqpoint{1.347816in}{0.565949in}}{\pgfqpoint{1.345621in}{0.560649in}}{\pgfqpoint{1.345621in}{0.555124in}}%
\pgfpathcurveto{\pgfqpoint{1.345621in}{0.549599in}}{\pgfqpoint{1.347816in}{0.544300in}}{\pgfqpoint{1.351723in}{0.540393in}}%
\pgfpathcurveto{\pgfqpoint{1.355629in}{0.536486in}}{\pgfqpoint{1.360929in}{0.534291in}}{\pgfqpoint{1.366454in}{0.534291in}}%
\pgfpathclose%
\pgfusepath{stroke,fill}%
\end{pgfscope}%
\begin{pgfscope}%
\pgfpathrectangle{\pgfqpoint{0.403630in}{0.251348in}}{\pgfqpoint{1.395000in}{1.359000in}}%
\pgfusepath{clip}%
\pgfsetbuttcap%
\pgfsetroundjoin%
\definecolor{currentfill}{rgb}{0.686275,0.352941,0.313725}%
\pgfsetfillcolor{currentfill}%
\pgfsetlinewidth{0.501875pt}%
\definecolor{currentstroke}{rgb}{0.686275,0.352941,0.313725}%
\pgfsetstrokecolor{currentstroke}%
\pgfsetdash{}{0pt}%
\pgfpathmoveto{\pgfqpoint{1.596219in}{0.891361in}}%
\pgfpathcurveto{\pgfqpoint{1.601744in}{0.891361in}}{\pgfqpoint{1.607043in}{0.893556in}}{\pgfqpoint{1.610950in}{0.897463in}}%
\pgfpathcurveto{\pgfqpoint{1.614857in}{0.901370in}}{\pgfqpoint{1.617052in}{0.906670in}}{\pgfqpoint{1.617052in}{0.912195in}}%
\pgfpathcurveto{\pgfqpoint{1.617052in}{0.917720in}}{\pgfqpoint{1.614857in}{0.923019in}}{\pgfqpoint{1.610950in}{0.926926in}}%
\pgfpathcurveto{\pgfqpoint{1.607043in}{0.930833in}}{\pgfqpoint{1.601744in}{0.933028in}}{\pgfqpoint{1.596219in}{0.933028in}}%
\pgfpathcurveto{\pgfqpoint{1.590694in}{0.933028in}}{\pgfqpoint{1.585394in}{0.930833in}}{\pgfqpoint{1.581487in}{0.926926in}}%
\pgfpathcurveto{\pgfqpoint{1.577580in}{0.923019in}}{\pgfqpoint{1.575385in}{0.917720in}}{\pgfqpoint{1.575385in}{0.912195in}}%
\pgfpathcurveto{\pgfqpoint{1.575385in}{0.906670in}}{\pgfqpoint{1.577580in}{0.901370in}}{\pgfqpoint{1.581487in}{0.897463in}}%
\pgfpathcurveto{\pgfqpoint{1.585394in}{0.893556in}}{\pgfqpoint{1.590694in}{0.891361in}}{\pgfqpoint{1.596219in}{0.891361in}}%
\pgfpathclose%
\pgfusepath{stroke,fill}%
\end{pgfscope}%
\begin{pgfscope}%
\pgfpathrectangle{\pgfqpoint{0.403630in}{0.251348in}}{\pgfqpoint{1.395000in}{1.359000in}}%
\pgfusepath{clip}%
\pgfsetbuttcap%
\pgfsetroundjoin%
\definecolor{currentfill}{rgb}{0.686275,0.352941,0.313725}%
\pgfsetfillcolor{currentfill}%
\pgfsetlinewidth{0.501875pt}%
\definecolor{currentstroke}{rgb}{0.686275,0.352941,0.313725}%
\pgfsetstrokecolor{currentstroke}%
\pgfsetdash{}{0pt}%
\pgfpathmoveto{\pgfqpoint{0.655277in}{1.008608in}}%
\pgfpathcurveto{\pgfqpoint{0.660803in}{1.008608in}}{\pgfqpoint{0.666102in}{1.010804in}}{\pgfqpoint{0.670009in}{1.014710in}}%
\pgfpathcurveto{\pgfqpoint{0.673916in}{1.018617in}}{\pgfqpoint{0.676111in}{1.023917in}}{\pgfqpoint{0.676111in}{1.029442in}}%
\pgfpathcurveto{\pgfqpoint{0.676111in}{1.034967in}}{\pgfqpoint{0.673916in}{1.040266in}}{\pgfqpoint{0.670009in}{1.044173in}}%
\pgfpathcurveto{\pgfqpoint{0.666102in}{1.048080in}}{\pgfqpoint{0.660803in}{1.050275in}}{\pgfqpoint{0.655277in}{1.050275in}}%
\pgfpathcurveto{\pgfqpoint{0.649752in}{1.050275in}}{\pgfqpoint{0.644453in}{1.048080in}}{\pgfqpoint{0.640546in}{1.044173in}}%
\pgfpathcurveto{\pgfqpoint{0.636639in}{1.040266in}}{\pgfqpoint{0.634444in}{1.034967in}}{\pgfqpoint{0.634444in}{1.029442in}}%
\pgfpathcurveto{\pgfqpoint{0.634444in}{1.023917in}}{\pgfqpoint{0.636639in}{1.018617in}}{\pgfqpoint{0.640546in}{1.014710in}}%
\pgfpathcurveto{\pgfqpoint{0.644453in}{1.010804in}}{\pgfqpoint{0.649752in}{1.008608in}}{\pgfqpoint{0.655277in}{1.008608in}}%
\pgfpathclose%
\pgfusepath{stroke,fill}%
\end{pgfscope}%
\begin{pgfscope}%
\pgfpathrectangle{\pgfqpoint{0.403630in}{0.251348in}}{\pgfqpoint{1.395000in}{1.359000in}}%
\pgfusepath{clip}%
\pgfsetbuttcap%
\pgfsetroundjoin%
\definecolor{currentfill}{rgb}{0.686275,0.352941,0.313725}%
\pgfsetfillcolor{currentfill}%
\pgfsetlinewidth{0.501875pt}%
\definecolor{currentstroke}{rgb}{0.686275,0.352941,0.313725}%
\pgfsetstrokecolor{currentstroke}%
\pgfsetdash{}{0pt}%
\pgfpathmoveto{\pgfqpoint{0.633395in}{0.870044in}}%
\pgfpathcurveto{\pgfqpoint{0.638920in}{0.870044in}}{\pgfqpoint{0.644220in}{0.872239in}}{\pgfqpoint{0.648127in}{0.876146in}}%
\pgfpathcurveto{\pgfqpoint{0.652033in}{0.880052in}}{\pgfqpoint{0.654228in}{0.885352in}}{\pgfqpoint{0.654228in}{0.890877in}}%
\pgfpathcurveto{\pgfqpoint{0.654228in}{0.896402in}}{\pgfqpoint{0.652033in}{0.901702in}}{\pgfqpoint{0.648127in}{0.905608in}}%
\pgfpathcurveto{\pgfqpoint{0.644220in}{0.909515in}}{\pgfqpoint{0.638920in}{0.911710in}}{\pgfqpoint{0.633395in}{0.911710in}}%
\pgfpathcurveto{\pgfqpoint{0.627870in}{0.911710in}}{\pgfqpoint{0.622571in}{0.909515in}}{\pgfqpoint{0.618664in}{0.905608in}}%
\pgfpathcurveto{\pgfqpoint{0.614757in}{0.901702in}}{\pgfqpoint{0.612562in}{0.896402in}}{\pgfqpoint{0.612562in}{0.890877in}}%
\pgfpathcurveto{\pgfqpoint{0.612562in}{0.885352in}}{\pgfqpoint{0.614757in}{0.880052in}}{\pgfqpoint{0.618664in}{0.876146in}}%
\pgfpathcurveto{\pgfqpoint{0.622571in}{0.872239in}}{\pgfqpoint{0.627870in}{0.870044in}}{\pgfqpoint{0.633395in}{0.870044in}}%
\pgfpathclose%
\pgfusepath{stroke,fill}%
\end{pgfscope}%
\begin{pgfscope}%
\pgfpathrectangle{\pgfqpoint{0.403630in}{0.251348in}}{\pgfqpoint{1.395000in}{1.359000in}}%
\pgfusepath{clip}%
\pgfsetbuttcap%
\pgfsetroundjoin%
\definecolor{currentfill}{rgb}{0.686275,0.352941,0.313725}%
\pgfsetfillcolor{currentfill}%
\pgfsetlinewidth{0.501875pt}%
\definecolor{currentstroke}{rgb}{0.686275,0.352941,0.313725}%
\pgfsetstrokecolor{currentstroke}%
\pgfsetdash{}{0pt}%
\pgfpathmoveto{\pgfqpoint{1.262513in}{0.449020in}}%
\pgfpathcurveto{\pgfqpoint{1.268038in}{0.449020in}}{\pgfqpoint{1.273337in}{0.451215in}}{\pgfqpoint{1.277244in}{0.455122in}}%
\pgfpathcurveto{\pgfqpoint{1.281151in}{0.459029in}}{\pgfqpoint{1.283346in}{0.464328in}}{\pgfqpoint{1.283346in}{0.469854in}}%
\pgfpathcurveto{\pgfqpoint{1.283346in}{0.475379in}}{\pgfqpoint{1.281151in}{0.480678in}}{\pgfqpoint{1.277244in}{0.484585in}}%
\pgfpathcurveto{\pgfqpoint{1.273337in}{0.488492in}}{\pgfqpoint{1.268038in}{0.490687in}}{\pgfqpoint{1.262513in}{0.490687in}}%
\pgfpathcurveto{\pgfqpoint{1.256988in}{0.490687in}}{\pgfqpoint{1.251688in}{0.488492in}}{\pgfqpoint{1.247781in}{0.484585in}}%
\pgfpathcurveto{\pgfqpoint{1.243875in}{0.480678in}}{\pgfqpoint{1.241679in}{0.475379in}}{\pgfqpoint{1.241679in}{0.469854in}}%
\pgfpathcurveto{\pgfqpoint{1.241679in}{0.464328in}}{\pgfqpoint{1.243875in}{0.459029in}}{\pgfqpoint{1.247781in}{0.455122in}}%
\pgfpathcurveto{\pgfqpoint{1.251688in}{0.451215in}}{\pgfqpoint{1.256988in}{0.449020in}}{\pgfqpoint{1.262513in}{0.449020in}}%
\pgfpathclose%
\pgfusepath{stroke,fill}%
\end{pgfscope}%
\begin{pgfscope}%
\pgfpathrectangle{\pgfqpoint{0.403630in}{0.251348in}}{\pgfqpoint{1.395000in}{1.359000in}}%
\pgfusepath{clip}%
\pgfsetbuttcap%
\pgfsetroundjoin%
\definecolor{currentfill}{rgb}{0.686275,0.352941,0.313725}%
\pgfsetfillcolor{currentfill}%
\pgfsetlinewidth{0.501875pt}%
\definecolor{currentstroke}{rgb}{0.686275,0.352941,0.313725}%
\pgfsetstrokecolor{currentstroke}%
\pgfsetdash{}{0pt}%
\pgfpathmoveto{\pgfqpoint{1.114807in}{0.433032in}}%
\pgfpathcurveto{\pgfqpoint{1.120332in}{0.433032in}}{\pgfqpoint{1.125631in}{0.435227in}}{\pgfqpoint{1.129538in}{0.439134in}}%
\pgfpathcurveto{\pgfqpoint{1.133445in}{0.443041in}}{\pgfqpoint{1.135640in}{0.448340in}}{\pgfqpoint{1.135640in}{0.453865in}}%
\pgfpathcurveto{\pgfqpoint{1.135640in}{0.459390in}}{\pgfqpoint{1.133445in}{0.464690in}}{\pgfqpoint{1.129538in}{0.468597in}}%
\pgfpathcurveto{\pgfqpoint{1.125631in}{0.472503in}}{\pgfqpoint{1.120332in}{0.474699in}}{\pgfqpoint{1.114807in}{0.474699in}}%
\pgfpathcurveto{\pgfqpoint{1.109282in}{0.474699in}}{\pgfqpoint{1.103982in}{0.472503in}}{\pgfqpoint{1.100075in}{0.468597in}}%
\pgfpathcurveto{\pgfqpoint{1.096169in}{0.464690in}}{\pgfqpoint{1.093974in}{0.459390in}}{\pgfqpoint{1.093974in}{0.453865in}}%
\pgfpathcurveto{\pgfqpoint{1.093974in}{0.448340in}}{\pgfqpoint{1.096169in}{0.443041in}}{\pgfqpoint{1.100075in}{0.439134in}}%
\pgfpathcurveto{\pgfqpoint{1.103982in}{0.435227in}}{\pgfqpoint{1.109282in}{0.433032in}}{\pgfqpoint{1.114807in}{0.433032in}}%
\pgfpathclose%
\pgfusepath{stroke,fill}%
\end{pgfscope}%
\begin{pgfscope}%
\pgfpathrectangle{\pgfqpoint{0.403630in}{0.251348in}}{\pgfqpoint{1.395000in}{1.359000in}}%
\pgfusepath{clip}%
\pgfsetbuttcap%
\pgfsetroundjoin%
\definecolor{currentfill}{rgb}{0.686275,0.352941,0.313725}%
\pgfsetfillcolor{currentfill}%
\pgfsetlinewidth{0.501875pt}%
\definecolor{currentstroke}{rgb}{0.686275,0.352941,0.313725}%
\pgfsetstrokecolor{currentstroke}%
\pgfsetdash{}{0pt}%
\pgfpathmoveto{\pgfqpoint{0.999925in}{0.443691in}}%
\pgfpathcurveto{\pgfqpoint{1.005450in}{0.443691in}}{\pgfqpoint{1.010749in}{0.445886in}}{\pgfqpoint{1.014656in}{0.449793in}}%
\pgfpathcurveto{\pgfqpoint{1.018563in}{0.453700in}}{\pgfqpoint{1.020758in}{0.458999in}}{\pgfqpoint{1.020758in}{0.464524in}}%
\pgfpathcurveto{\pgfqpoint{1.020758in}{0.470049in}}{\pgfqpoint{1.018563in}{0.475349in}}{\pgfqpoint{1.014656in}{0.479256in}}%
\pgfpathcurveto{\pgfqpoint{1.010749in}{0.483162in}}{\pgfqpoint{1.005450in}{0.485357in}}{\pgfqpoint{0.999925in}{0.485357in}}%
\pgfpathcurveto{\pgfqpoint{0.994399in}{0.485357in}}{\pgfqpoint{0.989100in}{0.483162in}}{\pgfqpoint{0.985193in}{0.479256in}}%
\pgfpathcurveto{\pgfqpoint{0.981286in}{0.475349in}}{\pgfqpoint{0.979091in}{0.470049in}}{\pgfqpoint{0.979091in}{0.464524in}}%
\pgfpathcurveto{\pgfqpoint{0.979091in}{0.458999in}}{\pgfqpoint{0.981286in}{0.453700in}}{\pgfqpoint{0.985193in}{0.449793in}}%
\pgfpathcurveto{\pgfqpoint{0.989100in}{0.445886in}}{\pgfqpoint{0.994399in}{0.443691in}}{\pgfqpoint{0.999925in}{0.443691in}}%
\pgfpathclose%
\pgfusepath{stroke,fill}%
\end{pgfscope}%
\begin{pgfscope}%
\pgfpathrectangle{\pgfqpoint{0.403630in}{0.251348in}}{\pgfqpoint{1.395000in}{1.359000in}}%
\pgfusepath{clip}%
\pgfsetbuttcap%
\pgfsetroundjoin%
\definecolor{currentfill}{rgb}{0.686275,0.352941,0.313725}%
\pgfsetfillcolor{currentfill}%
\pgfsetlinewidth{0.501875pt}%
\definecolor{currentstroke}{rgb}{0.686275,0.352941,0.313725}%
\pgfsetstrokecolor{currentstroke}%
\pgfsetdash{}{0pt}%
\pgfpathmoveto{\pgfqpoint{0.841277in}{0.528961in}}%
\pgfpathcurveto{\pgfqpoint{0.846803in}{0.528961in}}{\pgfqpoint{0.852102in}{0.531156in}}{\pgfqpoint{0.856009in}{0.535063in}}%
\pgfpathcurveto{\pgfqpoint{0.859916in}{0.538970in}}{\pgfqpoint{0.862111in}{0.544270in}}{\pgfqpoint{0.862111in}{0.549795in}}%
\pgfpathcurveto{\pgfqpoint{0.862111in}{0.555320in}}{\pgfqpoint{0.859916in}{0.560619in}}{\pgfqpoint{0.856009in}{0.564526in}}%
\pgfpathcurveto{\pgfqpoint{0.852102in}{0.568433in}}{\pgfqpoint{0.846803in}{0.570628in}}{\pgfqpoint{0.841277in}{0.570628in}}%
\pgfpathcurveto{\pgfqpoint{0.835752in}{0.570628in}}{\pgfqpoint{0.830453in}{0.568433in}}{\pgfqpoint{0.826546in}{0.564526in}}%
\pgfpathcurveto{\pgfqpoint{0.822639in}{0.560619in}}{\pgfqpoint{0.820444in}{0.555320in}}{\pgfqpoint{0.820444in}{0.549795in}}%
\pgfpathcurveto{\pgfqpoint{0.820444in}{0.544270in}}{\pgfqpoint{0.822639in}{0.538970in}}{\pgfqpoint{0.826546in}{0.535063in}}%
\pgfpathcurveto{\pgfqpoint{0.830453in}{0.531156in}}{\pgfqpoint{0.835752in}{0.528961in}}{\pgfqpoint{0.841277in}{0.528961in}}%
\pgfpathclose%
\pgfusepath{stroke,fill}%
\end{pgfscope}%
\begin{pgfscope}%
\pgfpathrectangle{\pgfqpoint{0.403630in}{0.251348in}}{\pgfqpoint{1.395000in}{1.359000in}}%
\pgfusepath{clip}%
\pgfsetbuttcap%
\pgfsetroundjoin%
\definecolor{currentfill}{rgb}{0.686275,0.352941,0.313725}%
\pgfsetfillcolor{currentfill}%
\pgfsetlinewidth{0.501875pt}%
\definecolor{currentstroke}{rgb}{0.686275,0.352941,0.313725}%
\pgfsetstrokecolor{currentstroke}%
\pgfsetdash{}{0pt}%
\pgfpathmoveto{\pgfqpoint{1.328160in}{0.496985in}}%
\pgfpathcurveto{\pgfqpoint{1.333685in}{0.496985in}}{\pgfqpoint{1.338984in}{0.499180in}}{\pgfqpoint{1.342891in}{0.503087in}}%
\pgfpathcurveto{\pgfqpoint{1.346798in}{0.506994in}}{\pgfqpoint{1.348993in}{0.512293in}}{\pgfqpoint{1.348993in}{0.517818in}}%
\pgfpathcurveto{\pgfqpoint{1.348993in}{0.523343in}}{\pgfqpoint{1.346798in}{0.528643in}}{\pgfqpoint{1.342891in}{0.532550in}}%
\pgfpathcurveto{\pgfqpoint{1.338984in}{0.536456in}}{\pgfqpoint{1.333685in}{0.538652in}}{\pgfqpoint{1.328160in}{0.538652in}}%
\pgfpathcurveto{\pgfqpoint{1.322635in}{0.538652in}}{\pgfqpoint{1.317335in}{0.536456in}}{\pgfqpoint{1.313428in}{0.532550in}}%
\pgfpathcurveto{\pgfqpoint{1.309522in}{0.528643in}}{\pgfqpoint{1.307326in}{0.523343in}}{\pgfqpoint{1.307326in}{0.517818in}}%
\pgfpathcurveto{\pgfqpoint{1.307326in}{0.512293in}}{\pgfqpoint{1.309522in}{0.506994in}}{\pgfqpoint{1.313428in}{0.503087in}}%
\pgfpathcurveto{\pgfqpoint{1.317335in}{0.499180in}}{\pgfqpoint{1.322635in}{0.496985in}}{\pgfqpoint{1.328160in}{0.496985in}}%
\pgfpathclose%
\pgfusepath{stroke,fill}%
\end{pgfscope}%
\begin{pgfscope}%
\pgfpathrectangle{\pgfqpoint{0.403630in}{0.251348in}}{\pgfqpoint{1.395000in}{1.359000in}}%
\pgfusepath{clip}%
\pgfsetbuttcap%
\pgfsetroundjoin%
\definecolor{currentfill}{rgb}{0.686275,0.352941,0.313725}%
\pgfsetfillcolor{currentfill}%
\pgfsetlinewidth{0.501875pt}%
\definecolor{currentstroke}{rgb}{0.686275,0.352941,0.313725}%
\pgfsetstrokecolor{currentstroke}%
\pgfsetdash{}{0pt}%
\pgfpathmoveto{\pgfqpoint{0.885042in}{0.475667in}}%
\pgfpathcurveto{\pgfqpoint{0.890567in}{0.475667in}}{\pgfqpoint{0.895867in}{0.477862in}}{\pgfqpoint{0.899774in}{0.481769in}}%
\pgfpathcurveto{\pgfqpoint{0.903680in}{0.485676in}}{\pgfqpoint{0.905876in}{0.490976in}}{\pgfqpoint{0.905876in}{0.496501in}}%
\pgfpathcurveto{\pgfqpoint{0.905876in}{0.502026in}}{\pgfqpoint{0.903680in}{0.507325in}}{\pgfqpoint{0.899774in}{0.511232in}}%
\pgfpathcurveto{\pgfqpoint{0.895867in}{0.515139in}}{\pgfqpoint{0.890567in}{0.517334in}}{\pgfqpoint{0.885042in}{0.517334in}}%
\pgfpathcurveto{\pgfqpoint{0.879517in}{0.517334in}}{\pgfqpoint{0.874218in}{0.515139in}}{\pgfqpoint{0.870311in}{0.511232in}}%
\pgfpathcurveto{\pgfqpoint{0.866404in}{0.507325in}}{\pgfqpoint{0.864209in}{0.502026in}}{\pgfqpoint{0.864209in}{0.496501in}}%
\pgfpathcurveto{\pgfqpoint{0.864209in}{0.490976in}}{\pgfqpoint{0.866404in}{0.485676in}}{\pgfqpoint{0.870311in}{0.481769in}}%
\pgfpathcurveto{\pgfqpoint{0.874218in}{0.477862in}}{\pgfqpoint{0.879517in}{0.475667in}}{\pgfqpoint{0.885042in}{0.475667in}}%
\pgfpathclose%
\pgfusepath{stroke,fill}%
\end{pgfscope}%
\begin{pgfscope}%
\pgfpathrectangle{\pgfqpoint{0.403630in}{0.251348in}}{\pgfqpoint{1.395000in}{1.359000in}}%
\pgfusepath{clip}%
\pgfsetbuttcap%
\pgfsetroundjoin%
\definecolor{currentfill}{rgb}{0.686275,0.352941,0.313725}%
\pgfsetfillcolor{currentfill}%
\pgfsetlinewidth{0.501875pt}%
\definecolor{currentstroke}{rgb}{0.686275,0.352941,0.313725}%
\pgfsetstrokecolor{currentstroke}%
\pgfsetdash{}{0pt}%
\pgfpathmoveto{\pgfqpoint{1.530572in}{0.726150in}}%
\pgfpathcurveto{\pgfqpoint{1.536097in}{0.726150in}}{\pgfqpoint{1.541396in}{0.728345in}}{\pgfqpoint{1.545303in}{0.732252in}}%
\pgfpathcurveto{\pgfqpoint{1.549210in}{0.736158in}}{\pgfqpoint{1.551405in}{0.741458in}}{\pgfqpoint{1.551405in}{0.746983in}}%
\pgfpathcurveto{\pgfqpoint{1.551405in}{0.752508in}}{\pgfqpoint{1.549210in}{0.757808in}}{\pgfqpoint{1.545303in}{0.761714in}}%
\pgfpathcurveto{\pgfqpoint{1.541396in}{0.765621in}}{\pgfqpoint{1.536097in}{0.767816in}}{\pgfqpoint{1.530572in}{0.767816in}}%
\pgfpathcurveto{\pgfqpoint{1.525047in}{0.767816in}}{\pgfqpoint{1.519747in}{0.765621in}}{\pgfqpoint{1.515840in}{0.761714in}}%
\pgfpathcurveto{\pgfqpoint{1.511933in}{0.757808in}}{\pgfqpoint{1.509738in}{0.752508in}}{\pgfqpoint{1.509738in}{0.746983in}}%
\pgfpathcurveto{\pgfqpoint{1.509738in}{0.741458in}}{\pgfqpoint{1.511933in}{0.736158in}}{\pgfqpoint{1.515840in}{0.732252in}}%
\pgfpathcurveto{\pgfqpoint{1.519747in}{0.728345in}}{\pgfqpoint{1.525047in}{0.726150in}}{\pgfqpoint{1.530572in}{0.726150in}}%
\pgfpathclose%
\pgfusepath{stroke,fill}%
\end{pgfscope}%
\begin{pgfscope}%
\pgfpathrectangle{\pgfqpoint{0.403630in}{0.251348in}}{\pgfqpoint{1.395000in}{1.359000in}}%
\pgfusepath{clip}%
\pgfsetbuttcap%
\pgfsetroundjoin%
\definecolor{currentfill}{rgb}{0.686275,0.352941,0.313725}%
\pgfsetfillcolor{currentfill}%
\pgfsetlinewidth{0.501875pt}%
\definecolor{currentstroke}{rgb}{0.686275,0.352941,0.313725}%
\pgfsetstrokecolor{currentstroke}%
\pgfsetdash{}{0pt}%
\pgfpathmoveto{\pgfqpoint{0.606042in}{0.774114in}}%
\pgfpathcurveto{\pgfqpoint{0.611567in}{0.774114in}}{\pgfqpoint{0.616867in}{0.776309in}}{\pgfqpoint{0.620774in}{0.780216in}}%
\pgfpathcurveto{\pgfqpoint{0.624680in}{0.784123in}}{\pgfqpoint{0.626876in}{0.789423in}}{\pgfqpoint{0.626876in}{0.794948in}}%
\pgfpathcurveto{\pgfqpoint{0.626876in}{0.800473in}}{\pgfqpoint{0.624680in}{0.805772in}}{\pgfqpoint{0.620774in}{0.809679in}}%
\pgfpathcurveto{\pgfqpoint{0.616867in}{0.813586in}}{\pgfqpoint{0.611567in}{0.815781in}}{\pgfqpoint{0.606042in}{0.815781in}}%
\pgfpathcurveto{\pgfqpoint{0.600517in}{0.815781in}}{\pgfqpoint{0.595218in}{0.813586in}}{\pgfqpoint{0.591311in}{0.809679in}}%
\pgfpathcurveto{\pgfqpoint{0.587404in}{0.805772in}}{\pgfqpoint{0.585209in}{0.800473in}}{\pgfqpoint{0.585209in}{0.794948in}}%
\pgfpathcurveto{\pgfqpoint{0.585209in}{0.789423in}}{\pgfqpoint{0.587404in}{0.784123in}}{\pgfqpoint{0.591311in}{0.780216in}}%
\pgfpathcurveto{\pgfqpoint{0.595218in}{0.776309in}}{\pgfqpoint{0.600517in}{0.774114in}}{\pgfqpoint{0.606042in}{0.774114in}}%
\pgfpathclose%
\pgfusepath{stroke,fill}%
\end{pgfscope}%
\begin{pgfscope}%
\pgfpathrectangle{\pgfqpoint{0.403630in}{0.251348in}}{\pgfqpoint{1.395000in}{1.359000in}}%
\pgfusepath{clip}%
\pgfsetbuttcap%
\pgfsetroundjoin%
\definecolor{currentfill}{rgb}{0.686275,0.352941,0.313725}%
\pgfsetfillcolor{currentfill}%
\pgfsetlinewidth{0.501875pt}%
\definecolor{currentstroke}{rgb}{0.686275,0.352941,0.313725}%
\pgfsetstrokecolor{currentstroke}%
\pgfsetdash{}{0pt}%
\pgfpathmoveto{\pgfqpoint{1.399277in}{0.539620in}}%
\pgfpathcurveto{\pgfqpoint{1.404803in}{0.539620in}}{\pgfqpoint{1.410102in}{0.541815in}}{\pgfqpoint{1.414009in}{0.545722in}}%
\pgfpathcurveto{\pgfqpoint{1.417916in}{0.549629in}}{\pgfqpoint{1.420111in}{0.554928in}}{\pgfqpoint{1.420111in}{0.560454in}}%
\pgfpathcurveto{\pgfqpoint{1.420111in}{0.565979in}}{\pgfqpoint{1.417916in}{0.571278in}}{\pgfqpoint{1.414009in}{0.575185in}}%
\pgfpathcurveto{\pgfqpoint{1.410102in}{0.579092in}}{\pgfqpoint{1.404803in}{0.581287in}}{\pgfqpoint{1.399277in}{0.581287in}}%
\pgfpathcurveto{\pgfqpoint{1.393752in}{0.581287in}}{\pgfqpoint{1.388453in}{0.579092in}}{\pgfqpoint{1.384546in}{0.575185in}}%
\pgfpathcurveto{\pgfqpoint{1.380639in}{0.571278in}}{\pgfqpoint{1.378444in}{0.565979in}}{\pgfqpoint{1.378444in}{0.560454in}}%
\pgfpathcurveto{\pgfqpoint{1.378444in}{0.554928in}}{\pgfqpoint{1.380639in}{0.549629in}}{\pgfqpoint{1.384546in}{0.545722in}}%
\pgfpathcurveto{\pgfqpoint{1.388453in}{0.541815in}}{\pgfqpoint{1.393752in}{0.539620in}}{\pgfqpoint{1.399277in}{0.539620in}}%
\pgfpathclose%
\pgfusepath{stroke,fill}%
\end{pgfscope}%
\begin{pgfscope}%
\pgfpathrectangle{\pgfqpoint{0.403630in}{0.251348in}}{\pgfqpoint{1.395000in}{1.359000in}}%
\pgfusepath{clip}%
\pgfsetbuttcap%
\pgfsetroundjoin%
\definecolor{currentfill}{rgb}{0.686275,0.352941,0.313725}%
\pgfsetfillcolor{currentfill}%
\pgfsetlinewidth{0.501875pt}%
\definecolor{currentstroke}{rgb}{0.686275,0.352941,0.313725}%
\pgfsetstrokecolor{currentstroke}%
\pgfsetdash{}{0pt}%
\pgfpathmoveto{\pgfqpoint{0.655277in}{1.003279in}}%
\pgfpathcurveto{\pgfqpoint{0.660803in}{1.003279in}}{\pgfqpoint{0.666102in}{1.005474in}}{\pgfqpoint{0.670009in}{1.009381in}}%
\pgfpathcurveto{\pgfqpoint{0.673916in}{1.013288in}}{\pgfqpoint{0.676111in}{1.018587in}}{\pgfqpoint{0.676111in}{1.024112in}}%
\pgfpathcurveto{\pgfqpoint{0.676111in}{1.029637in}}{\pgfqpoint{0.673916in}{1.034937in}}{\pgfqpoint{0.670009in}{1.038844in}}%
\pgfpathcurveto{\pgfqpoint{0.666102in}{1.042751in}}{\pgfqpoint{0.660803in}{1.044946in}}{\pgfqpoint{0.655277in}{1.044946in}}%
\pgfpathcurveto{\pgfqpoint{0.649752in}{1.044946in}}{\pgfqpoint{0.644453in}{1.042751in}}{\pgfqpoint{0.640546in}{1.038844in}}%
\pgfpathcurveto{\pgfqpoint{0.636639in}{1.034937in}}{\pgfqpoint{0.634444in}{1.029637in}}{\pgfqpoint{0.634444in}{1.024112in}}%
\pgfpathcurveto{\pgfqpoint{0.634444in}{1.018587in}}{\pgfqpoint{0.636639in}{1.013288in}}{\pgfqpoint{0.640546in}{1.009381in}}%
\pgfpathcurveto{\pgfqpoint{0.644453in}{1.005474in}}{\pgfqpoint{0.649752in}{1.003279in}}{\pgfqpoint{0.655277in}{1.003279in}}%
\pgfpathclose%
\pgfusepath{stroke,fill}%
\end{pgfscope}%
\begin{pgfscope}%
\pgfpathrectangle{\pgfqpoint{0.403630in}{0.251348in}}{\pgfqpoint{1.395000in}{1.359000in}}%
\pgfusepath{clip}%
\pgfsetbuttcap%
\pgfsetroundjoin%
\definecolor{currentfill}{rgb}{0.686275,0.352941,0.313725}%
\pgfsetfillcolor{currentfill}%
\pgfsetlinewidth{0.501875pt}%
\definecolor{currentstroke}{rgb}{0.686275,0.352941,0.313725}%
\pgfsetstrokecolor{currentstroke}%
\pgfsetdash{}{0pt}%
\pgfpathmoveto{\pgfqpoint{0.928807in}{0.486326in}}%
\pgfpathcurveto{\pgfqpoint{0.934332in}{0.486326in}}{\pgfqpoint{0.939631in}{0.488521in}}{\pgfqpoint{0.943538in}{0.492428in}}%
\pgfpathcurveto{\pgfqpoint{0.947445in}{0.496335in}}{\pgfqpoint{0.949640in}{0.501634in}}{\pgfqpoint{0.949640in}{0.507159in}}%
\pgfpathcurveto{\pgfqpoint{0.949640in}{0.512684in}}{\pgfqpoint{0.947445in}{0.517984in}}{\pgfqpoint{0.943538in}{0.521891in}}%
\pgfpathcurveto{\pgfqpoint{0.939631in}{0.525798in}}{\pgfqpoint{0.934332in}{0.527993in}}{\pgfqpoint{0.928807in}{0.527993in}}%
\pgfpathcurveto{\pgfqpoint{0.923282in}{0.527993in}}{\pgfqpoint{0.917982in}{0.525798in}}{\pgfqpoint{0.914075in}{0.521891in}}%
\pgfpathcurveto{\pgfqpoint{0.910169in}{0.517984in}}{\pgfqpoint{0.907974in}{0.512684in}}{\pgfqpoint{0.907974in}{0.507159in}}%
\pgfpathcurveto{\pgfqpoint{0.907974in}{0.501634in}}{\pgfqpoint{0.910169in}{0.496335in}}{\pgfqpoint{0.914075in}{0.492428in}}%
\pgfpathcurveto{\pgfqpoint{0.917982in}{0.488521in}}{\pgfqpoint{0.923282in}{0.486326in}}{\pgfqpoint{0.928807in}{0.486326in}}%
\pgfpathclose%
\pgfusepath{stroke,fill}%
\end{pgfscope}%
\begin{pgfscope}%
\pgfpathrectangle{\pgfqpoint{0.403630in}{0.251348in}}{\pgfqpoint{1.395000in}{1.359000in}}%
\pgfusepath{clip}%
\pgfsetbuttcap%
\pgfsetroundjoin%
\definecolor{currentfill}{rgb}{0.686275,0.352941,0.313725}%
\pgfsetfillcolor{currentfill}%
\pgfsetlinewidth{0.501875pt}%
\definecolor{currentstroke}{rgb}{0.686275,0.352941,0.313725}%
\pgfsetstrokecolor{currentstroke}%
\pgfsetdash{}{0pt}%
\pgfpathmoveto{\pgfqpoint{1.246101in}{1.344361in}}%
\pgfpathcurveto{\pgfqpoint{1.251626in}{1.344361in}}{\pgfqpoint{1.256926in}{1.346556in}}{\pgfqpoint{1.260832in}{1.350463in}}%
\pgfpathcurveto{\pgfqpoint{1.264739in}{1.354370in}}{\pgfqpoint{1.266934in}{1.359670in}}{\pgfqpoint{1.266934in}{1.365195in}}%
\pgfpathcurveto{\pgfqpoint{1.266934in}{1.370720in}}{\pgfqpoint{1.264739in}{1.376019in}}{\pgfqpoint{1.260832in}{1.379926in}}%
\pgfpathcurveto{\pgfqpoint{1.256926in}{1.383833in}}{\pgfqpoint{1.251626in}{1.386028in}}{\pgfqpoint{1.246101in}{1.386028in}}%
\pgfpathcurveto{\pgfqpoint{1.240576in}{1.386028in}}{\pgfqpoint{1.235276in}{1.383833in}}{\pgfqpoint{1.231370in}{1.379926in}}%
\pgfpathcurveto{\pgfqpoint{1.227463in}{1.376019in}}{\pgfqpoint{1.225268in}{1.370720in}}{\pgfqpoint{1.225268in}{1.365195in}}%
\pgfpathcurveto{\pgfqpoint{1.225268in}{1.359670in}}{\pgfqpoint{1.227463in}{1.354370in}}{\pgfqpoint{1.231370in}{1.350463in}}%
\pgfpathcurveto{\pgfqpoint{1.235276in}{1.346556in}}{\pgfqpoint{1.240576in}{1.344361in}}{\pgfqpoint{1.246101in}{1.344361in}}%
\pgfpathclose%
\pgfusepath{stroke,fill}%
\end{pgfscope}%
\begin{pgfscope}%
\pgfpathrectangle{\pgfqpoint{0.403630in}{0.251348in}}{\pgfqpoint{1.395000in}{1.359000in}}%
\pgfusepath{clip}%
\pgfsetbuttcap%
\pgfsetroundjoin%
\definecolor{currentfill}{rgb}{0.686275,0.352941,0.313725}%
\pgfsetfillcolor{currentfill}%
\pgfsetlinewidth{0.501875pt}%
\definecolor{currentstroke}{rgb}{0.686275,0.352941,0.313725}%
\pgfsetstrokecolor{currentstroke}%
\pgfsetdash{}{0pt}%
\pgfpathmoveto{\pgfqpoint{1.596219in}{0.923338in}}%
\pgfpathcurveto{\pgfqpoint{1.601744in}{0.923338in}}{\pgfqpoint{1.607043in}{0.925533in}}{\pgfqpoint{1.610950in}{0.929440in}}%
\pgfpathcurveto{\pgfqpoint{1.614857in}{0.933347in}}{\pgfqpoint{1.617052in}{0.938646in}}{\pgfqpoint{1.617052in}{0.944171in}}%
\pgfpathcurveto{\pgfqpoint{1.617052in}{0.949696in}}{\pgfqpoint{1.614857in}{0.954996in}}{\pgfqpoint{1.610950in}{0.958903in}}%
\pgfpathcurveto{\pgfqpoint{1.607043in}{0.962809in}}{\pgfqpoint{1.601744in}{0.965005in}}{\pgfqpoint{1.596219in}{0.965005in}}%
\pgfpathcurveto{\pgfqpoint{1.590694in}{0.965005in}}{\pgfqpoint{1.585394in}{0.962809in}}{\pgfqpoint{1.581487in}{0.958903in}}%
\pgfpathcurveto{\pgfqpoint{1.577580in}{0.954996in}}{\pgfqpoint{1.575385in}{0.949696in}}{\pgfqpoint{1.575385in}{0.944171in}}%
\pgfpathcurveto{\pgfqpoint{1.575385in}{0.938646in}}{\pgfqpoint{1.577580in}{0.933347in}}{\pgfqpoint{1.581487in}{0.929440in}}%
\pgfpathcurveto{\pgfqpoint{1.585394in}{0.925533in}}{\pgfqpoint{1.590694in}{0.923338in}}{\pgfqpoint{1.596219in}{0.923338in}}%
\pgfpathclose%
\pgfusepath{stroke,fill}%
\end{pgfscope}%
\begin{pgfscope}%
\pgfpathrectangle{\pgfqpoint{0.403630in}{0.251348in}}{\pgfqpoint{1.395000in}{1.359000in}}%
\pgfusepath{clip}%
\pgfsetbuttcap%
\pgfsetroundjoin%
\definecolor{currentfill}{rgb}{0.686275,0.352941,0.313725}%
\pgfsetfillcolor{currentfill}%
\pgfsetlinewidth{0.501875pt}%
\definecolor{currentstroke}{rgb}{0.686275,0.352941,0.313725}%
\pgfsetstrokecolor{currentstroke}%
\pgfsetdash{}{0pt}%
\pgfpathmoveto{\pgfqpoint{1.333630in}{0.539620in}}%
\pgfpathcurveto{\pgfqpoint{1.339155in}{0.539620in}}{\pgfqpoint{1.344455in}{0.541815in}}{\pgfqpoint{1.348362in}{0.545722in}}%
\pgfpathcurveto{\pgfqpoint{1.352269in}{0.549629in}}{\pgfqpoint{1.354464in}{0.554928in}}{\pgfqpoint{1.354464in}{0.560454in}}%
\pgfpathcurveto{\pgfqpoint{1.354464in}{0.565979in}}{\pgfqpoint{1.352269in}{0.571278in}}{\pgfqpoint{1.348362in}{0.575185in}}%
\pgfpathcurveto{\pgfqpoint{1.344455in}{0.579092in}}{\pgfqpoint{1.339155in}{0.581287in}}{\pgfqpoint{1.333630in}{0.581287in}}%
\pgfpathcurveto{\pgfqpoint{1.328105in}{0.581287in}}{\pgfqpoint{1.322806in}{0.579092in}}{\pgfqpoint{1.318899in}{0.575185in}}%
\pgfpathcurveto{\pgfqpoint{1.314992in}{0.571278in}}{\pgfqpoint{1.312797in}{0.565979in}}{\pgfqpoint{1.312797in}{0.560454in}}%
\pgfpathcurveto{\pgfqpoint{1.312797in}{0.554928in}}{\pgfqpoint{1.314992in}{0.549629in}}{\pgfqpoint{1.318899in}{0.545722in}}%
\pgfpathcurveto{\pgfqpoint{1.322806in}{0.541815in}}{\pgfqpoint{1.328105in}{0.539620in}}{\pgfqpoint{1.333630in}{0.539620in}}%
\pgfpathclose%
\pgfusepath{stroke,fill}%
\end{pgfscope}%
\begin{pgfscope}%
\pgfpathrectangle{\pgfqpoint{0.403630in}{0.251348in}}{\pgfqpoint{1.395000in}{1.359000in}}%
\pgfusepath{clip}%
\pgfsetbuttcap%
\pgfsetroundjoin%
\definecolor{currentfill}{rgb}{0.686275,0.352941,0.313725}%
\pgfsetfillcolor{currentfill}%
\pgfsetlinewidth{0.501875pt}%
\definecolor{currentstroke}{rgb}{0.686275,0.352941,0.313725}%
\pgfsetstrokecolor{currentstroke}%
\pgfsetdash{}{0pt}%
\pgfpathmoveto{\pgfqpoint{1.536042in}{0.800761in}}%
\pgfpathcurveto{\pgfqpoint{1.541567in}{0.800761in}}{\pgfqpoint{1.546867in}{0.802956in}}{\pgfqpoint{1.550774in}{0.806863in}}%
\pgfpathcurveto{\pgfqpoint{1.554680in}{0.810770in}}{\pgfqpoint{1.556876in}{0.816070in}}{\pgfqpoint{1.556876in}{0.821595in}}%
\pgfpathcurveto{\pgfqpoint{1.556876in}{0.827120in}}{\pgfqpoint{1.554680in}{0.832419in}}{\pgfqpoint{1.550774in}{0.836326in}}%
\pgfpathcurveto{\pgfqpoint{1.546867in}{0.840233in}}{\pgfqpoint{1.541567in}{0.842428in}}{\pgfqpoint{1.536042in}{0.842428in}}%
\pgfpathcurveto{\pgfqpoint{1.530517in}{0.842428in}}{\pgfqpoint{1.525218in}{0.840233in}}{\pgfqpoint{1.521311in}{0.836326in}}%
\pgfpathcurveto{\pgfqpoint{1.517404in}{0.832419in}}{\pgfqpoint{1.515209in}{0.827120in}}{\pgfqpoint{1.515209in}{0.821595in}}%
\pgfpathcurveto{\pgfqpoint{1.515209in}{0.816070in}}{\pgfqpoint{1.517404in}{0.810770in}}{\pgfqpoint{1.521311in}{0.806863in}}%
\pgfpathcurveto{\pgfqpoint{1.525218in}{0.802956in}}{\pgfqpoint{1.530517in}{0.800761in}}{\pgfqpoint{1.536042in}{0.800761in}}%
\pgfpathclose%
\pgfusepath{stroke,fill}%
\end{pgfscope}%
\begin{pgfscope}%
\pgfpathrectangle{\pgfqpoint{0.403630in}{0.251348in}}{\pgfqpoint{1.395000in}{1.359000in}}%
\pgfusepath{clip}%
\pgfsetbuttcap%
\pgfsetroundjoin%
\definecolor{currentfill}{rgb}{0.686275,0.352941,0.313725}%
\pgfsetfillcolor{currentfill}%
\pgfsetlinewidth{0.501875pt}%
\definecolor{currentstroke}{rgb}{0.686275,0.352941,0.313725}%
\pgfsetstrokecolor{currentstroke}%
\pgfsetdash{}{0pt}%
\pgfpathmoveto{\pgfqpoint{1.497748in}{0.630220in}}%
\pgfpathcurveto{\pgfqpoint{1.503273in}{0.630220in}}{\pgfqpoint{1.508573in}{0.632415in}}{\pgfqpoint{1.512479in}{0.636322in}}%
\pgfpathcurveto{\pgfqpoint{1.516386in}{0.640229in}}{\pgfqpoint{1.518581in}{0.645528in}}{\pgfqpoint{1.518581in}{0.651054in}}%
\pgfpathcurveto{\pgfqpoint{1.518581in}{0.656579in}}{\pgfqpoint{1.516386in}{0.661878in}}{\pgfqpoint{1.512479in}{0.665785in}}%
\pgfpathcurveto{\pgfqpoint{1.508573in}{0.669692in}}{\pgfqpoint{1.503273in}{0.671887in}}{\pgfqpoint{1.497748in}{0.671887in}}%
\pgfpathcurveto{\pgfqpoint{1.492223in}{0.671887in}}{\pgfqpoint{1.486923in}{0.669692in}}{\pgfqpoint{1.483017in}{0.665785in}}%
\pgfpathcurveto{\pgfqpoint{1.479110in}{0.661878in}}{\pgfqpoint{1.476915in}{0.656579in}}{\pgfqpoint{1.476915in}{0.651054in}}%
\pgfpathcurveto{\pgfqpoint{1.476915in}{0.645528in}}{\pgfqpoint{1.479110in}{0.640229in}}{\pgfqpoint{1.483017in}{0.636322in}}%
\pgfpathcurveto{\pgfqpoint{1.486923in}{0.632415in}}{\pgfqpoint{1.492223in}{0.630220in}}{\pgfqpoint{1.497748in}{0.630220in}}%
\pgfpathclose%
\pgfusepath{stroke,fill}%
\end{pgfscope}%
\begin{pgfscope}%
\pgfpathrectangle{\pgfqpoint{0.403630in}{0.251348in}}{\pgfqpoint{1.395000in}{1.359000in}}%
\pgfusepath{clip}%
\pgfsetbuttcap%
\pgfsetroundjoin%
\definecolor{currentfill}{rgb}{0.686275,0.352941,0.313725}%
\pgfsetfillcolor{currentfill}%
\pgfsetlinewidth{0.501875pt}%
\definecolor{currentstroke}{rgb}{0.686275,0.352941,0.313725}%
\pgfsetstrokecolor{currentstroke}%
\pgfsetdash{}{0pt}%
\pgfpathmoveto{\pgfqpoint{0.939748in}{0.502314in}}%
\pgfpathcurveto{\pgfqpoint{0.945273in}{0.502314in}}{\pgfqpoint{0.950573in}{0.504509in}}{\pgfqpoint{0.954479in}{0.508416in}}%
\pgfpathcurveto{\pgfqpoint{0.958386in}{0.512323in}}{\pgfqpoint{0.960581in}{0.517623in}}{\pgfqpoint{0.960581in}{0.523148in}}%
\pgfpathcurveto{\pgfqpoint{0.960581in}{0.528673in}}{\pgfqpoint{0.958386in}{0.533972in}}{\pgfqpoint{0.954479in}{0.537879in}}%
\pgfpathcurveto{\pgfqpoint{0.950573in}{0.541786in}}{\pgfqpoint{0.945273in}{0.543981in}}{\pgfqpoint{0.939748in}{0.543981in}}%
\pgfpathcurveto{\pgfqpoint{0.934223in}{0.543981in}}{\pgfqpoint{0.928923in}{0.541786in}}{\pgfqpoint{0.925017in}{0.537879in}}%
\pgfpathcurveto{\pgfqpoint{0.921110in}{0.533972in}}{\pgfqpoint{0.918915in}{0.528673in}}{\pgfqpoint{0.918915in}{0.523148in}}%
\pgfpathcurveto{\pgfqpoint{0.918915in}{0.517623in}}{\pgfqpoint{0.921110in}{0.512323in}}{\pgfqpoint{0.925017in}{0.508416in}}%
\pgfpathcurveto{\pgfqpoint{0.928923in}{0.504509in}}{\pgfqpoint{0.934223in}{0.502314in}}{\pgfqpoint{0.939748in}{0.502314in}}%
\pgfpathclose%
\pgfusepath{stroke,fill}%
\end{pgfscope}%
\begin{pgfscope}%
\pgfpathrectangle{\pgfqpoint{0.403630in}{0.251348in}}{\pgfqpoint{1.395000in}{1.359000in}}%
\pgfusepath{clip}%
\pgfsetbuttcap%
\pgfsetroundjoin%
\definecolor{currentfill}{rgb}{0.686275,0.352941,0.313725}%
\pgfsetfillcolor{currentfill}%
\pgfsetlinewidth{0.501875pt}%
\definecolor{currentstroke}{rgb}{0.686275,0.352941,0.313725}%
\pgfsetstrokecolor{currentstroke}%
\pgfsetdash{}{0pt}%
\pgfpathmoveto{\pgfqpoint{1.087454in}{0.470338in}}%
\pgfpathcurveto{\pgfqpoint{1.092979in}{0.470338in}}{\pgfqpoint{1.098279in}{0.472533in}}{\pgfqpoint{1.102185in}{0.476440in}}%
\pgfpathcurveto{\pgfqpoint{1.106092in}{0.480347in}}{\pgfqpoint{1.108287in}{0.485646in}}{\pgfqpoint{1.108287in}{0.491171in}}%
\pgfpathcurveto{\pgfqpoint{1.108287in}{0.496696in}}{\pgfqpoint{1.106092in}{0.501996in}}{\pgfqpoint{1.102185in}{0.505903in}}%
\pgfpathcurveto{\pgfqpoint{1.098279in}{0.509809in}}{\pgfqpoint{1.092979in}{0.512005in}}{\pgfqpoint{1.087454in}{0.512005in}}%
\pgfpathcurveto{\pgfqpoint{1.081929in}{0.512005in}}{\pgfqpoint{1.076629in}{0.509809in}}{\pgfqpoint{1.072723in}{0.505903in}}%
\pgfpathcurveto{\pgfqpoint{1.068816in}{0.501996in}}{\pgfqpoint{1.066621in}{0.496696in}}{\pgfqpoint{1.066621in}{0.491171in}}%
\pgfpathcurveto{\pgfqpoint{1.066621in}{0.485646in}}{\pgfqpoint{1.068816in}{0.480347in}}{\pgfqpoint{1.072723in}{0.476440in}}%
\pgfpathcurveto{\pgfqpoint{1.076629in}{0.472533in}}{\pgfqpoint{1.081929in}{0.470338in}}{\pgfqpoint{1.087454in}{0.470338in}}%
\pgfpathclose%
\pgfusepath{stroke,fill}%
\end{pgfscope}%
\begin{pgfscope}%
\pgfpathrectangle{\pgfqpoint{0.403630in}{0.251348in}}{\pgfqpoint{1.395000in}{1.359000in}}%
\pgfusepath{clip}%
\pgfsetbuttcap%
\pgfsetroundjoin%
\definecolor{currentfill}{rgb}{0.686275,0.352941,0.313725}%
\pgfsetfillcolor{currentfill}%
\pgfsetlinewidth{0.501875pt}%
\definecolor{currentstroke}{rgb}{0.686275,0.352941,0.313725}%
\pgfsetstrokecolor{currentstroke}%
\pgfsetdash{}{0pt}%
\pgfpathmoveto{\pgfqpoint{1.574336in}{0.854055in}}%
\pgfpathcurveto{\pgfqpoint{1.579861in}{0.854055in}}{\pgfqpoint{1.585161in}{0.856251in}}{\pgfqpoint{1.589068in}{0.860157in}}%
\pgfpathcurveto{\pgfqpoint{1.592975in}{0.864064in}}{\pgfqpoint{1.595170in}{0.869364in}}{\pgfqpoint{1.595170in}{0.874889in}}%
\pgfpathcurveto{\pgfqpoint{1.595170in}{0.880414in}}{\pgfqpoint{1.592975in}{0.885713in}}{\pgfqpoint{1.589068in}{0.889620in}}%
\pgfpathcurveto{\pgfqpoint{1.585161in}{0.893527in}}{\pgfqpoint{1.579861in}{0.895722in}}{\pgfqpoint{1.574336in}{0.895722in}}%
\pgfpathcurveto{\pgfqpoint{1.568811in}{0.895722in}}{\pgfqpoint{1.563512in}{0.893527in}}{\pgfqpoint{1.559605in}{0.889620in}}%
\pgfpathcurveto{\pgfqpoint{1.555698in}{0.885713in}}{\pgfqpoint{1.553503in}{0.880414in}}{\pgfqpoint{1.553503in}{0.874889in}}%
\pgfpathcurveto{\pgfqpoint{1.553503in}{0.869364in}}{\pgfqpoint{1.555698in}{0.864064in}}{\pgfqpoint{1.559605in}{0.860157in}}%
\pgfpathcurveto{\pgfqpoint{1.563512in}{0.856251in}}{\pgfqpoint{1.568811in}{0.854055in}}{\pgfqpoint{1.574336in}{0.854055in}}%
\pgfpathclose%
\pgfusepath{stroke,fill}%
\end{pgfscope}%
\begin{pgfscope}%
\pgfpathrectangle{\pgfqpoint{0.403630in}{0.251348in}}{\pgfqpoint{1.395000in}{1.359000in}}%
\pgfusepath{clip}%
\pgfsetbuttcap%
\pgfsetroundjoin%
\definecolor{currentfill}{rgb}{0.686275,0.352941,0.313725}%
\pgfsetfillcolor{currentfill}%
\pgfsetlinewidth{0.501875pt}%
\definecolor{currentstroke}{rgb}{0.686275,0.352941,0.313725}%
\pgfsetstrokecolor{currentstroke}%
\pgfsetdash{}{0pt}%
\pgfpathmoveto{\pgfqpoint{0.852219in}{0.502314in}}%
\pgfpathcurveto{\pgfqpoint{0.857744in}{0.502314in}}{\pgfqpoint{0.863043in}{0.504509in}}{\pgfqpoint{0.866950in}{0.508416in}}%
\pgfpathcurveto{\pgfqpoint{0.870857in}{0.512323in}}{\pgfqpoint{0.873052in}{0.517623in}}{\pgfqpoint{0.873052in}{0.523148in}}%
\pgfpathcurveto{\pgfqpoint{0.873052in}{0.528673in}}{\pgfqpoint{0.870857in}{0.533972in}}{\pgfqpoint{0.866950in}{0.537879in}}%
\pgfpathcurveto{\pgfqpoint{0.863043in}{0.541786in}}{\pgfqpoint{0.857744in}{0.543981in}}{\pgfqpoint{0.852219in}{0.543981in}}%
\pgfpathcurveto{\pgfqpoint{0.846694in}{0.543981in}}{\pgfqpoint{0.841394in}{0.541786in}}{\pgfqpoint{0.837487in}{0.537879in}}%
\pgfpathcurveto{\pgfqpoint{0.833580in}{0.533972in}}{\pgfqpoint{0.831385in}{0.528673in}}{\pgfqpoint{0.831385in}{0.523148in}}%
\pgfpathcurveto{\pgfqpoint{0.831385in}{0.517623in}}{\pgfqpoint{0.833580in}{0.512323in}}{\pgfqpoint{0.837487in}{0.508416in}}%
\pgfpathcurveto{\pgfqpoint{0.841394in}{0.504509in}}{\pgfqpoint{0.846694in}{0.502314in}}{\pgfqpoint{0.852219in}{0.502314in}}%
\pgfpathclose%
\pgfusepath{stroke,fill}%
\end{pgfscope}%
\begin{pgfscope}%
\pgfpathrectangle{\pgfqpoint{0.403630in}{0.251348in}}{\pgfqpoint{1.395000in}{1.359000in}}%
\pgfusepath{clip}%
\pgfsetbuttcap%
\pgfsetroundjoin%
\definecolor{currentfill}{rgb}{0.686275,0.352941,0.313725}%
\pgfsetfillcolor{currentfill}%
\pgfsetlinewidth{0.501875pt}%
\definecolor{currentstroke}{rgb}{0.686275,0.352941,0.313725}%
\pgfsetstrokecolor{currentstroke}%
\pgfsetdash{}{0pt}%
\pgfpathmoveto{\pgfqpoint{1.536042in}{0.736808in}}%
\pgfpathcurveto{\pgfqpoint{1.541567in}{0.736808in}}{\pgfqpoint{1.546867in}{0.739004in}}{\pgfqpoint{1.550774in}{0.742910in}}%
\pgfpathcurveto{\pgfqpoint{1.554680in}{0.746817in}}{\pgfqpoint{1.556876in}{0.752117in}}{\pgfqpoint{1.556876in}{0.757642in}}%
\pgfpathcurveto{\pgfqpoint{1.556876in}{0.763167in}}{\pgfqpoint{1.554680in}{0.768466in}}{\pgfqpoint{1.550774in}{0.772373in}}%
\pgfpathcurveto{\pgfqpoint{1.546867in}{0.776280in}}{\pgfqpoint{1.541567in}{0.778475in}}{\pgfqpoint{1.536042in}{0.778475in}}%
\pgfpathcurveto{\pgfqpoint{1.530517in}{0.778475in}}{\pgfqpoint{1.525218in}{0.776280in}}{\pgfqpoint{1.521311in}{0.772373in}}%
\pgfpathcurveto{\pgfqpoint{1.517404in}{0.768466in}}{\pgfqpoint{1.515209in}{0.763167in}}{\pgfqpoint{1.515209in}{0.757642in}}%
\pgfpathcurveto{\pgfqpoint{1.515209in}{0.752117in}}{\pgfqpoint{1.517404in}{0.746817in}}{\pgfqpoint{1.521311in}{0.742910in}}%
\pgfpathcurveto{\pgfqpoint{1.525218in}{0.739004in}}{\pgfqpoint{1.530517in}{0.736808in}}{\pgfqpoint{1.536042in}{0.736808in}}%
\pgfpathclose%
\pgfusepath{stroke,fill}%
\end{pgfscope}%
\begin{pgfscope}%
\pgfpathrectangle{\pgfqpoint{0.403630in}{0.251348in}}{\pgfqpoint{1.395000in}{1.359000in}}%
\pgfusepath{clip}%
\pgfsetbuttcap%
\pgfsetroundjoin%
\definecolor{currentfill}{rgb}{0.686275,0.352941,0.313725}%
\pgfsetfillcolor{currentfill}%
\pgfsetlinewidth{0.501875pt}%
\definecolor{currentstroke}{rgb}{0.686275,0.352941,0.313725}%
\pgfsetstrokecolor{currentstroke}%
\pgfsetdash{}{0pt}%
\pgfpathmoveto{\pgfqpoint{0.890513in}{1.301726in}}%
\pgfpathcurveto{\pgfqpoint{0.896038in}{1.301726in}}{\pgfqpoint{0.901337in}{1.303921in}}{\pgfqpoint{0.905244in}{1.307828in}}%
\pgfpathcurveto{\pgfqpoint{0.909151in}{1.311735in}}{\pgfqpoint{0.911346in}{1.317034in}}{\pgfqpoint{0.911346in}{1.322559in}}%
\pgfpathcurveto{\pgfqpoint{0.911346in}{1.328084in}}{\pgfqpoint{0.909151in}{1.333384in}}{\pgfqpoint{0.905244in}{1.337291in}}%
\pgfpathcurveto{\pgfqpoint{0.901337in}{1.341198in}}{\pgfqpoint{0.896038in}{1.343393in}}{\pgfqpoint{0.890513in}{1.343393in}}%
\pgfpathcurveto{\pgfqpoint{0.884988in}{1.343393in}}{\pgfqpoint{0.879688in}{1.341198in}}{\pgfqpoint{0.875781in}{1.337291in}}%
\pgfpathcurveto{\pgfqpoint{0.871875in}{1.333384in}}{\pgfqpoint{0.869679in}{1.328084in}}{\pgfqpoint{0.869679in}{1.322559in}}%
\pgfpathcurveto{\pgfqpoint{0.869679in}{1.317034in}}{\pgfqpoint{0.871875in}{1.311735in}}{\pgfqpoint{0.875781in}{1.307828in}}%
\pgfpathcurveto{\pgfqpoint{0.879688in}{1.303921in}}{\pgfqpoint{0.884988in}{1.301726in}}{\pgfqpoint{0.890513in}{1.301726in}}%
\pgfpathclose%
\pgfusepath{stroke,fill}%
\end{pgfscope}%
\begin{pgfscope}%
\pgfpathrectangle{\pgfqpoint{0.403630in}{0.251348in}}{\pgfqpoint{1.395000in}{1.359000in}}%
\pgfusepath{clip}%
\pgfsetbuttcap%
\pgfsetroundjoin%
\definecolor{currentfill}{rgb}{0.686275,0.352941,0.313725}%
\pgfsetfillcolor{currentfill}%
\pgfsetlinewidth{0.501875pt}%
\definecolor{currentstroke}{rgb}{0.686275,0.352941,0.313725}%
\pgfsetstrokecolor{currentstroke}%
\pgfsetdash{}{0pt}%
\pgfpathmoveto{\pgfqpoint{0.852219in}{1.291067in}}%
\pgfpathcurveto{\pgfqpoint{0.857744in}{1.291067in}}{\pgfqpoint{0.863043in}{1.293262in}}{\pgfqpoint{0.866950in}{1.297169in}}%
\pgfpathcurveto{\pgfqpoint{0.870857in}{1.301076in}}{\pgfqpoint{0.873052in}{1.306376in}}{\pgfqpoint{0.873052in}{1.311901in}}%
\pgfpathcurveto{\pgfqpoint{0.873052in}{1.317426in}}{\pgfqpoint{0.870857in}{1.322725in}}{\pgfqpoint{0.866950in}{1.326632in}}%
\pgfpathcurveto{\pgfqpoint{0.863043in}{1.330539in}}{\pgfqpoint{0.857744in}{1.332734in}}{\pgfqpoint{0.852219in}{1.332734in}}%
\pgfpathcurveto{\pgfqpoint{0.846694in}{1.332734in}}{\pgfqpoint{0.841394in}{1.330539in}}{\pgfqpoint{0.837487in}{1.326632in}}%
\pgfpathcurveto{\pgfqpoint{0.833580in}{1.322725in}}{\pgfqpoint{0.831385in}{1.317426in}}{\pgfqpoint{0.831385in}{1.311901in}}%
\pgfpathcurveto{\pgfqpoint{0.831385in}{1.306376in}}{\pgfqpoint{0.833580in}{1.301076in}}{\pgfqpoint{0.837487in}{1.297169in}}%
\pgfpathcurveto{\pgfqpoint{0.841394in}{1.293262in}}{\pgfqpoint{0.846694in}{1.291067in}}{\pgfqpoint{0.852219in}{1.291067in}}%
\pgfpathclose%
\pgfusepath{stroke,fill}%
\end{pgfscope}%
\begin{pgfscope}%
\pgfpathrectangle{\pgfqpoint{0.403630in}{0.251348in}}{\pgfqpoint{1.395000in}{1.359000in}}%
\pgfusepath{clip}%
\pgfsetbuttcap%
\pgfsetroundjoin%
\definecolor{currentfill}{rgb}{0.686275,0.352941,0.313725}%
\pgfsetfillcolor{currentfill}%
\pgfsetlinewidth{0.501875pt}%
\definecolor{currentstroke}{rgb}{0.686275,0.352941,0.313725}%
\pgfsetstrokecolor{currentstroke}%
\pgfsetdash{}{0pt}%
\pgfpathmoveto{\pgfqpoint{1.585277in}{0.854055in}}%
\pgfpathcurveto{\pgfqpoint{1.590803in}{0.854055in}}{\pgfqpoint{1.596102in}{0.856251in}}{\pgfqpoint{1.600009in}{0.860157in}}%
\pgfpathcurveto{\pgfqpoint{1.603916in}{0.864064in}}{\pgfqpoint{1.606111in}{0.869364in}}{\pgfqpoint{1.606111in}{0.874889in}}%
\pgfpathcurveto{\pgfqpoint{1.606111in}{0.880414in}}{\pgfqpoint{1.603916in}{0.885713in}}{\pgfqpoint{1.600009in}{0.889620in}}%
\pgfpathcurveto{\pgfqpoint{1.596102in}{0.893527in}}{\pgfqpoint{1.590803in}{0.895722in}}{\pgfqpoint{1.585277in}{0.895722in}}%
\pgfpathcurveto{\pgfqpoint{1.579752in}{0.895722in}}{\pgfqpoint{1.574453in}{0.893527in}}{\pgfqpoint{1.570546in}{0.889620in}}%
\pgfpathcurveto{\pgfqpoint{1.566639in}{0.885713in}}{\pgfqpoint{1.564444in}{0.880414in}}{\pgfqpoint{1.564444in}{0.874889in}}%
\pgfpathcurveto{\pgfqpoint{1.564444in}{0.869364in}}{\pgfqpoint{1.566639in}{0.864064in}}{\pgfqpoint{1.570546in}{0.860157in}}%
\pgfpathcurveto{\pgfqpoint{1.574453in}{0.856251in}}{\pgfqpoint{1.579752in}{0.854055in}}{\pgfqpoint{1.585277in}{0.854055in}}%
\pgfpathclose%
\pgfusepath{stroke,fill}%
\end{pgfscope}%
\begin{pgfscope}%
\pgfpathrectangle{\pgfqpoint{0.403630in}{0.251348in}}{\pgfqpoint{1.395000in}{1.359000in}}%
\pgfusepath{clip}%
\pgfsetbuttcap%
\pgfsetroundjoin%
\definecolor{currentfill}{rgb}{0.686275,0.352941,0.313725}%
\pgfsetfillcolor{currentfill}%
\pgfsetlinewidth{0.501875pt}%
\definecolor{currentstroke}{rgb}{0.686275,0.352941,0.313725}%
\pgfsetstrokecolor{currentstroke}%
\pgfsetdash{}{0pt}%
\pgfpathmoveto{\pgfqpoint{1.590748in}{0.822079in}}%
\pgfpathcurveto{\pgfqpoint{1.596273in}{0.822079in}}{\pgfqpoint{1.601573in}{0.824274in}}{\pgfqpoint{1.605479in}{0.828181in}}%
\pgfpathcurveto{\pgfqpoint{1.609386in}{0.832088in}}{\pgfqpoint{1.611581in}{0.837387in}}{\pgfqpoint{1.611581in}{0.842912in}}%
\pgfpathcurveto{\pgfqpoint{1.611581in}{0.848437in}}{\pgfqpoint{1.609386in}{0.853737in}}{\pgfqpoint{1.605479in}{0.857644in}}%
\pgfpathcurveto{\pgfqpoint{1.601573in}{0.861551in}}{\pgfqpoint{1.596273in}{0.863746in}}{\pgfqpoint{1.590748in}{0.863746in}}%
\pgfpathcurveto{\pgfqpoint{1.585223in}{0.863746in}}{\pgfqpoint{1.579923in}{0.861551in}}{\pgfqpoint{1.576017in}{0.857644in}}%
\pgfpathcurveto{\pgfqpoint{1.572110in}{0.853737in}}{\pgfqpoint{1.569915in}{0.848437in}}{\pgfqpoint{1.569915in}{0.842912in}}%
\pgfpathcurveto{\pgfqpoint{1.569915in}{0.837387in}}{\pgfqpoint{1.572110in}{0.832088in}}{\pgfqpoint{1.576017in}{0.828181in}}%
\pgfpathcurveto{\pgfqpoint{1.579923in}{0.824274in}}{\pgfqpoint{1.585223in}{0.822079in}}{\pgfqpoint{1.590748in}{0.822079in}}%
\pgfpathclose%
\pgfusepath{stroke,fill}%
\end{pgfscope}%
\begin{pgfscope}%
\pgfpathrectangle{\pgfqpoint{0.403630in}{0.251348in}}{\pgfqpoint{1.395000in}{1.359000in}}%
\pgfusepath{clip}%
\pgfsetbuttcap%
\pgfsetroundjoin%
\definecolor{currentfill}{rgb}{0.686275,0.352941,0.313725}%
\pgfsetfillcolor{currentfill}%
\pgfsetlinewidth{0.501875pt}%
\definecolor{currentstroke}{rgb}{0.686275,0.352941,0.313725}%
\pgfsetstrokecolor{currentstroke}%
\pgfsetdash{}{0pt}%
\pgfpathmoveto{\pgfqpoint{0.671689in}{0.699503in}}%
\pgfpathcurveto{\pgfqpoint{0.677214in}{0.699503in}}{\pgfqpoint{0.682514in}{0.701698in}}{\pgfqpoint{0.686421in}{0.705604in}}%
\pgfpathcurveto{\pgfqpoint{0.690327in}{0.709511in}}{\pgfqpoint{0.692523in}{0.714811in}}{\pgfqpoint{0.692523in}{0.720336in}}%
\pgfpathcurveto{\pgfqpoint{0.692523in}{0.725861in}}{\pgfqpoint{0.690327in}{0.731160in}}{\pgfqpoint{0.686421in}{0.735067in}}%
\pgfpathcurveto{\pgfqpoint{0.682514in}{0.738974in}}{\pgfqpoint{0.677214in}{0.741169in}}{\pgfqpoint{0.671689in}{0.741169in}}%
\pgfpathcurveto{\pgfqpoint{0.666164in}{0.741169in}}{\pgfqpoint{0.660865in}{0.738974in}}{\pgfqpoint{0.656958in}{0.735067in}}%
\pgfpathcurveto{\pgfqpoint{0.653051in}{0.731160in}}{\pgfqpoint{0.650856in}{0.725861in}}{\pgfqpoint{0.650856in}{0.720336in}}%
\pgfpathcurveto{\pgfqpoint{0.650856in}{0.714811in}}{\pgfqpoint{0.653051in}{0.709511in}}{\pgfqpoint{0.656958in}{0.705604in}}%
\pgfpathcurveto{\pgfqpoint{0.660865in}{0.701698in}}{\pgfqpoint{0.666164in}{0.699503in}}{\pgfqpoint{0.671689in}{0.699503in}}%
\pgfpathclose%
\pgfusepath{stroke,fill}%
\end{pgfscope}%
\begin{pgfscope}%
\pgfpathrectangle{\pgfqpoint{0.403630in}{0.251348in}}{\pgfqpoint{1.395000in}{1.359000in}}%
\pgfusepath{clip}%
\pgfsetbuttcap%
\pgfsetroundjoin%
\definecolor{currentfill}{rgb}{0.686275,0.352941,0.313725}%
\pgfsetfillcolor{currentfill}%
\pgfsetlinewidth{0.501875pt}%
\definecolor{currentstroke}{rgb}{0.686275,0.352941,0.313725}%
\pgfsetstrokecolor{currentstroke}%
\pgfsetdash{}{0pt}%
\pgfpathmoveto{\pgfqpoint{1.596219in}{0.790103in}}%
\pgfpathcurveto{\pgfqpoint{1.601744in}{0.790103in}}{\pgfqpoint{1.607043in}{0.792298in}}{\pgfqpoint{1.610950in}{0.796204in}}%
\pgfpathcurveto{\pgfqpoint{1.614857in}{0.800111in}}{\pgfqpoint{1.617052in}{0.805411in}}{\pgfqpoint{1.617052in}{0.810936in}}%
\pgfpathcurveto{\pgfqpoint{1.617052in}{0.816461in}}{\pgfqpoint{1.614857in}{0.821760in}}{\pgfqpoint{1.610950in}{0.825667in}}%
\pgfpathcurveto{\pgfqpoint{1.607043in}{0.829574in}}{\pgfqpoint{1.601744in}{0.831769in}}{\pgfqpoint{1.596219in}{0.831769in}}%
\pgfpathcurveto{\pgfqpoint{1.590694in}{0.831769in}}{\pgfqpoint{1.585394in}{0.829574in}}{\pgfqpoint{1.581487in}{0.825667in}}%
\pgfpathcurveto{\pgfqpoint{1.577580in}{0.821760in}}{\pgfqpoint{1.575385in}{0.816461in}}{\pgfqpoint{1.575385in}{0.810936in}}%
\pgfpathcurveto{\pgfqpoint{1.575385in}{0.805411in}}{\pgfqpoint{1.577580in}{0.800111in}}{\pgfqpoint{1.581487in}{0.796204in}}%
\pgfpathcurveto{\pgfqpoint{1.585394in}{0.792298in}}{\pgfqpoint{1.590694in}{0.790103in}}{\pgfqpoint{1.596219in}{0.790103in}}%
\pgfpathclose%
\pgfusepath{stroke,fill}%
\end{pgfscope}%
\begin{pgfscope}%
\pgfpathrectangle{\pgfqpoint{0.403630in}{0.251348in}}{\pgfqpoint{1.395000in}{1.359000in}}%
\pgfusepath{clip}%
\pgfsetbuttcap%
\pgfsetroundjoin%
\definecolor{currentfill}{rgb}{0.686275,0.352941,0.313725}%
\pgfsetfillcolor{currentfill}%
\pgfsetlinewidth{0.501875pt}%
\definecolor{currentstroke}{rgb}{0.686275,0.352941,0.313725}%
\pgfsetstrokecolor{currentstroke}%
\pgfsetdash{}{0pt}%
\pgfpathmoveto{\pgfqpoint{1.382866in}{1.339032in}}%
\pgfpathcurveto{\pgfqpoint{1.388391in}{1.339032in}}{\pgfqpoint{1.393690in}{1.341227in}}{\pgfqpoint{1.397597in}{1.345134in}}%
\pgfpathcurveto{\pgfqpoint{1.401504in}{1.349041in}}{\pgfqpoint{1.403699in}{1.354340in}}{\pgfqpoint{1.403699in}{1.359865in}}%
\pgfpathcurveto{\pgfqpoint{1.403699in}{1.365390in}}{\pgfqpoint{1.401504in}{1.370690in}}{\pgfqpoint{1.397597in}{1.374597in}}%
\pgfpathcurveto{\pgfqpoint{1.393690in}{1.378503in}}{\pgfqpoint{1.388391in}{1.380699in}}{\pgfqpoint{1.382866in}{1.380699in}}%
\pgfpathcurveto{\pgfqpoint{1.377341in}{1.380699in}}{\pgfqpoint{1.372041in}{1.378503in}}{\pgfqpoint{1.368134in}{1.374597in}}%
\pgfpathcurveto{\pgfqpoint{1.364228in}{1.370690in}}{\pgfqpoint{1.362032in}{1.365390in}}{\pgfqpoint{1.362032in}{1.359865in}}%
\pgfpathcurveto{\pgfqpoint{1.362032in}{1.354340in}}{\pgfqpoint{1.364228in}{1.349041in}}{\pgfqpoint{1.368134in}{1.345134in}}%
\pgfpathcurveto{\pgfqpoint{1.372041in}{1.341227in}}{\pgfqpoint{1.377341in}{1.339032in}}{\pgfqpoint{1.382866in}{1.339032in}}%
\pgfpathclose%
\pgfusepath{stroke,fill}%
\end{pgfscope}%
\begin{pgfscope}%
\pgfpathrectangle{\pgfqpoint{0.403630in}{0.251348in}}{\pgfqpoint{1.395000in}{1.359000in}}%
\pgfusepath{clip}%
\pgfsetbuttcap%
\pgfsetroundjoin%
\definecolor{currentfill}{rgb}{0.686275,0.352941,0.313725}%
\pgfsetfillcolor{currentfill}%
\pgfsetlinewidth{0.501875pt}%
\definecolor{currentstroke}{rgb}{0.686275,0.352941,0.313725}%
\pgfsetstrokecolor{currentstroke}%
\pgfsetdash{}{0pt}%
\pgfpathmoveto{\pgfqpoint{1.240630in}{1.323044in}}%
\pgfpathcurveto{\pgfqpoint{1.246155in}{1.323044in}}{\pgfqpoint{1.251455in}{1.325239in}}{\pgfqpoint{1.255362in}{1.329146in}}%
\pgfpathcurveto{\pgfqpoint{1.259269in}{1.333052in}}{\pgfqpoint{1.261464in}{1.338352in}}{\pgfqpoint{1.261464in}{1.343877in}}%
\pgfpathcurveto{\pgfqpoint{1.261464in}{1.349402in}}{\pgfqpoint{1.259269in}{1.354702in}}{\pgfqpoint{1.255362in}{1.358608in}}%
\pgfpathcurveto{\pgfqpoint{1.251455in}{1.362515in}}{\pgfqpoint{1.246155in}{1.364710in}}{\pgfqpoint{1.240630in}{1.364710in}}%
\pgfpathcurveto{\pgfqpoint{1.235105in}{1.364710in}}{\pgfqpoint{1.229806in}{1.362515in}}{\pgfqpoint{1.225899in}{1.358608in}}%
\pgfpathcurveto{\pgfqpoint{1.221992in}{1.354702in}}{\pgfqpoint{1.219797in}{1.349402in}}{\pgfqpoint{1.219797in}{1.343877in}}%
\pgfpathcurveto{\pgfqpoint{1.219797in}{1.338352in}}{\pgfqpoint{1.221992in}{1.333052in}}{\pgfqpoint{1.225899in}{1.329146in}}%
\pgfpathcurveto{\pgfqpoint{1.229806in}{1.325239in}}{\pgfqpoint{1.235105in}{1.323044in}}{\pgfqpoint{1.240630in}{1.323044in}}%
\pgfpathclose%
\pgfusepath{stroke,fill}%
\end{pgfscope}%
\begin{pgfscope}%
\pgfpathrectangle{\pgfqpoint{0.403630in}{0.251348in}}{\pgfqpoint{1.395000in}{1.359000in}}%
\pgfusepath{clip}%
\pgfsetbuttcap%
\pgfsetroundjoin%
\definecolor{currentfill}{rgb}{0.686275,0.352941,0.313725}%
\pgfsetfillcolor{currentfill}%
\pgfsetlinewidth{0.501875pt}%
\definecolor{currentstroke}{rgb}{0.686275,0.352941,0.313725}%
\pgfsetstrokecolor{currentstroke}%
\pgfsetdash{}{0pt}%
\pgfpathmoveto{\pgfqpoint{0.737336in}{1.173820in}}%
\pgfpathcurveto{\pgfqpoint{0.742861in}{1.173820in}}{\pgfqpoint{0.748161in}{1.176015in}}{\pgfqpoint{0.752068in}{1.179922in}}%
\pgfpathcurveto{\pgfqpoint{0.755975in}{1.183829in}}{\pgfqpoint{0.758170in}{1.189128in}}{\pgfqpoint{0.758170in}{1.194654in}}%
\pgfpathcurveto{\pgfqpoint{0.758170in}{1.200179in}}{\pgfqpoint{0.755975in}{1.205478in}}{\pgfqpoint{0.752068in}{1.209385in}}%
\pgfpathcurveto{\pgfqpoint{0.748161in}{1.213292in}}{\pgfqpoint{0.742861in}{1.215487in}}{\pgfqpoint{0.737336in}{1.215487in}}%
\pgfpathcurveto{\pgfqpoint{0.731811in}{1.215487in}}{\pgfqpoint{0.726512in}{1.213292in}}{\pgfqpoint{0.722605in}{1.209385in}}%
\pgfpathcurveto{\pgfqpoint{0.718698in}{1.205478in}}{\pgfqpoint{0.716503in}{1.200179in}}{\pgfqpoint{0.716503in}{1.194654in}}%
\pgfpathcurveto{\pgfqpoint{0.716503in}{1.189128in}}{\pgfqpoint{0.718698in}{1.183829in}}{\pgfqpoint{0.722605in}{1.179922in}}%
\pgfpathcurveto{\pgfqpoint{0.726512in}{1.176015in}}{\pgfqpoint{0.731811in}{1.173820in}}{\pgfqpoint{0.737336in}{1.173820in}}%
\pgfpathclose%
\pgfusepath{stroke,fill}%
\end{pgfscope}%
\begin{pgfscope}%
\pgfpathrectangle{\pgfqpoint{0.403630in}{0.251348in}}{\pgfqpoint{1.395000in}{1.359000in}}%
\pgfusepath{clip}%
\pgfsetbuttcap%
\pgfsetroundjoin%
\definecolor{currentfill}{rgb}{0.686275,0.352941,0.313725}%
\pgfsetfillcolor{currentfill}%
\pgfsetlinewidth{0.501875pt}%
\definecolor{currentstroke}{rgb}{0.686275,0.352941,0.313725}%
\pgfsetstrokecolor{currentstroke}%
\pgfsetdash{}{0pt}%
\pgfpathmoveto{\pgfqpoint{0.852219in}{1.349691in}}%
\pgfpathcurveto{\pgfqpoint{0.857744in}{1.349691in}}{\pgfqpoint{0.863043in}{1.351886in}}{\pgfqpoint{0.866950in}{1.355793in}}%
\pgfpathcurveto{\pgfqpoint{0.870857in}{1.359700in}}{\pgfqpoint{0.873052in}{1.364999in}}{\pgfqpoint{0.873052in}{1.370524in}}%
\pgfpathcurveto{\pgfqpoint{0.873052in}{1.376049in}}{\pgfqpoint{0.870857in}{1.381349in}}{\pgfqpoint{0.866950in}{1.385256in}}%
\pgfpathcurveto{\pgfqpoint{0.863043in}{1.389162in}}{\pgfqpoint{0.857744in}{1.391357in}}{\pgfqpoint{0.852219in}{1.391357in}}%
\pgfpathcurveto{\pgfqpoint{0.846694in}{1.391357in}}{\pgfqpoint{0.841394in}{1.389162in}}{\pgfqpoint{0.837487in}{1.385256in}}%
\pgfpathcurveto{\pgfqpoint{0.833580in}{1.381349in}}{\pgfqpoint{0.831385in}{1.376049in}}{\pgfqpoint{0.831385in}{1.370524in}}%
\pgfpathcurveto{\pgfqpoint{0.831385in}{1.364999in}}{\pgfqpoint{0.833580in}{1.359700in}}{\pgfqpoint{0.837487in}{1.355793in}}%
\pgfpathcurveto{\pgfqpoint{0.841394in}{1.351886in}}{\pgfqpoint{0.846694in}{1.349691in}}{\pgfqpoint{0.852219in}{1.349691in}}%
\pgfpathclose%
\pgfusepath{stroke,fill}%
\end{pgfscope}%
\begin{pgfscope}%
\pgfpathrectangle{\pgfqpoint{0.403630in}{0.251348in}}{\pgfqpoint{1.395000in}{1.359000in}}%
\pgfusepath{clip}%
\pgfsetbuttcap%
\pgfsetroundjoin%
\definecolor{currentfill}{rgb}{0.686275,0.352941,0.313725}%
\pgfsetfillcolor{currentfill}%
\pgfsetlinewidth{0.501875pt}%
\definecolor{currentstroke}{rgb}{0.686275,0.352941,0.313725}%
\pgfsetstrokecolor{currentstroke}%
\pgfsetdash{}{0pt}%
\pgfpathmoveto{\pgfqpoint{1.536042in}{1.003279in}}%
\pgfpathcurveto{\pgfqpoint{1.541567in}{1.003279in}}{\pgfqpoint{1.546867in}{1.005474in}}{\pgfqpoint{1.550774in}{1.009381in}}%
\pgfpathcurveto{\pgfqpoint{1.554680in}{1.013288in}}{\pgfqpoint{1.556876in}{1.018587in}}{\pgfqpoint{1.556876in}{1.024112in}}%
\pgfpathcurveto{\pgfqpoint{1.556876in}{1.029637in}}{\pgfqpoint{1.554680in}{1.034937in}}{\pgfqpoint{1.550774in}{1.038844in}}%
\pgfpathcurveto{\pgfqpoint{1.546867in}{1.042751in}}{\pgfqpoint{1.541567in}{1.044946in}}{\pgfqpoint{1.536042in}{1.044946in}}%
\pgfpathcurveto{\pgfqpoint{1.530517in}{1.044946in}}{\pgfqpoint{1.525218in}{1.042751in}}{\pgfqpoint{1.521311in}{1.038844in}}%
\pgfpathcurveto{\pgfqpoint{1.517404in}{1.034937in}}{\pgfqpoint{1.515209in}{1.029637in}}{\pgfqpoint{1.515209in}{1.024112in}}%
\pgfpathcurveto{\pgfqpoint{1.515209in}{1.018587in}}{\pgfqpoint{1.517404in}{1.013288in}}{\pgfqpoint{1.521311in}{1.009381in}}%
\pgfpathcurveto{\pgfqpoint{1.525218in}{1.005474in}}{\pgfqpoint{1.530517in}{1.003279in}}{\pgfqpoint{1.536042in}{1.003279in}}%
\pgfpathclose%
\pgfusepath{stroke,fill}%
\end{pgfscope}%
\begin{pgfscope}%
\pgfpathrectangle{\pgfqpoint{0.403630in}{0.251348in}}{\pgfqpoint{1.395000in}{1.359000in}}%
\pgfusepath{clip}%
\pgfsetbuttcap%
\pgfsetroundjoin%
\definecolor{currentfill}{rgb}{0.686275,0.352941,0.313725}%
\pgfsetfillcolor{currentfill}%
\pgfsetlinewidth{0.501875pt}%
\definecolor{currentstroke}{rgb}{0.686275,0.352941,0.313725}%
\pgfsetstrokecolor{currentstroke}%
\pgfsetdash{}{0pt}%
\pgfpathmoveto{\pgfqpoint{1.010866in}{0.480997in}}%
\pgfpathcurveto{\pgfqpoint{1.016391in}{0.480997in}}{\pgfqpoint{1.021690in}{0.483192in}}{\pgfqpoint{1.025597in}{0.487099in}}%
\pgfpathcurveto{\pgfqpoint{1.029504in}{0.491005in}}{\pgfqpoint{1.031699in}{0.496305in}}{\pgfqpoint{1.031699in}{0.501830in}}%
\pgfpathcurveto{\pgfqpoint{1.031699in}{0.507355in}}{\pgfqpoint{1.029504in}{0.512655in}}{\pgfqpoint{1.025597in}{0.516561in}}%
\pgfpathcurveto{\pgfqpoint{1.021690in}{0.520468in}}{\pgfqpoint{1.016391in}{0.522663in}}{\pgfqpoint{1.010866in}{0.522663in}}%
\pgfpathcurveto{\pgfqpoint{1.005341in}{0.522663in}}{\pgfqpoint{1.000041in}{0.520468in}}{\pgfqpoint{0.996134in}{0.516561in}}%
\pgfpathcurveto{\pgfqpoint{0.992228in}{0.512655in}}{\pgfqpoint{0.990032in}{0.507355in}}{\pgfqpoint{0.990032in}{0.501830in}}%
\pgfpathcurveto{\pgfqpoint{0.990032in}{0.496305in}}{\pgfqpoint{0.992228in}{0.491005in}}{\pgfqpoint{0.996134in}{0.487099in}}%
\pgfpathcurveto{\pgfqpoint{1.000041in}{0.483192in}}{\pgfqpoint{1.005341in}{0.480997in}}{\pgfqpoint{1.010866in}{0.480997in}}%
\pgfpathclose%
\pgfusepath{stroke,fill}%
\end{pgfscope}%
\begin{pgfscope}%
\pgfpathrectangle{\pgfqpoint{0.403630in}{0.251348in}}{\pgfqpoint{1.395000in}{1.359000in}}%
\pgfusepath{clip}%
\pgfsetbuttcap%
\pgfsetroundjoin%
\definecolor{currentfill}{rgb}{0.686275,0.352941,0.313725}%
\pgfsetfillcolor{currentfill}%
\pgfsetlinewidth{0.501875pt}%
\definecolor{currentstroke}{rgb}{0.686275,0.352941,0.313725}%
\pgfsetstrokecolor{currentstroke}%
\pgfsetdash{}{0pt}%
\pgfpathmoveto{\pgfqpoint{1.470395in}{1.152503in}}%
\pgfpathcurveto{\pgfqpoint{1.475920in}{1.152503in}}{\pgfqpoint{1.481220in}{1.154698in}}{\pgfqpoint{1.485127in}{1.158604in}}%
\pgfpathcurveto{\pgfqpoint{1.489033in}{1.162511in}}{\pgfqpoint{1.491228in}{1.167811in}}{\pgfqpoint{1.491228in}{1.173336in}}%
\pgfpathcurveto{\pgfqpoint{1.491228in}{1.178861in}}{\pgfqpoint{1.489033in}{1.184160in}}{\pgfqpoint{1.485127in}{1.188067in}}%
\pgfpathcurveto{\pgfqpoint{1.481220in}{1.191974in}}{\pgfqpoint{1.475920in}{1.194169in}}{\pgfqpoint{1.470395in}{1.194169in}}%
\pgfpathcurveto{\pgfqpoint{1.464870in}{1.194169in}}{\pgfqpoint{1.459571in}{1.191974in}}{\pgfqpoint{1.455664in}{1.188067in}}%
\pgfpathcurveto{\pgfqpoint{1.451757in}{1.184160in}}{\pgfqpoint{1.449562in}{1.178861in}}{\pgfqpoint{1.449562in}{1.173336in}}%
\pgfpathcurveto{\pgfqpoint{1.449562in}{1.167811in}}{\pgfqpoint{1.451757in}{1.162511in}}{\pgfqpoint{1.455664in}{1.158604in}}%
\pgfpathcurveto{\pgfqpoint{1.459571in}{1.154698in}}{\pgfqpoint{1.464870in}{1.152503in}}{\pgfqpoint{1.470395in}{1.152503in}}%
\pgfpathclose%
\pgfusepath{stroke,fill}%
\end{pgfscope}%
\begin{pgfscope}%
\pgfpathrectangle{\pgfqpoint{0.403630in}{0.251348in}}{\pgfqpoint{1.395000in}{1.359000in}}%
\pgfusepath{clip}%
\pgfsetbuttcap%
\pgfsetroundjoin%
\definecolor{currentfill}{rgb}{0.686275,0.352941,0.313725}%
\pgfsetfillcolor{currentfill}%
\pgfsetlinewidth{0.501875pt}%
\definecolor{currentstroke}{rgb}{0.686275,0.352941,0.313725}%
\pgfsetstrokecolor{currentstroke}%
\pgfsetdash{}{0pt}%
\pgfpathmoveto{\pgfqpoint{1.596219in}{1.067232in}}%
\pgfpathcurveto{\pgfqpoint{1.601744in}{1.067232in}}{\pgfqpoint{1.607043in}{1.069427in}}{\pgfqpoint{1.610950in}{1.073334in}}%
\pgfpathcurveto{\pgfqpoint{1.614857in}{1.077241in}}{\pgfqpoint{1.617052in}{1.082540in}}{\pgfqpoint{1.617052in}{1.088065in}}%
\pgfpathcurveto{\pgfqpoint{1.617052in}{1.093590in}}{\pgfqpoint{1.614857in}{1.098890in}}{\pgfqpoint{1.610950in}{1.102797in}}%
\pgfpathcurveto{\pgfqpoint{1.607043in}{1.106703in}}{\pgfqpoint{1.601744in}{1.108899in}}{\pgfqpoint{1.596219in}{1.108899in}}%
\pgfpathcurveto{\pgfqpoint{1.590694in}{1.108899in}}{\pgfqpoint{1.585394in}{1.106703in}}{\pgfqpoint{1.581487in}{1.102797in}}%
\pgfpathcurveto{\pgfqpoint{1.577580in}{1.098890in}}{\pgfqpoint{1.575385in}{1.093590in}}{\pgfqpoint{1.575385in}{1.088065in}}%
\pgfpathcurveto{\pgfqpoint{1.575385in}{1.082540in}}{\pgfqpoint{1.577580in}{1.077241in}}{\pgfqpoint{1.581487in}{1.073334in}}%
\pgfpathcurveto{\pgfqpoint{1.585394in}{1.069427in}}{\pgfqpoint{1.590694in}{1.067232in}}{\pgfqpoint{1.596219in}{1.067232in}}%
\pgfpathclose%
\pgfusepath{stroke,fill}%
\end{pgfscope}%
\begin{pgfscope}%
\pgfpathrectangle{\pgfqpoint{0.403630in}{0.251348in}}{\pgfqpoint{1.395000in}{1.359000in}}%
\pgfusepath{clip}%
\pgfsetbuttcap%
\pgfsetroundjoin%
\definecolor{currentfill}{rgb}{0.686275,0.352941,0.313725}%
\pgfsetfillcolor{currentfill}%
\pgfsetlinewidth{0.501875pt}%
\definecolor{currentstroke}{rgb}{0.686275,0.352941,0.313725}%
\pgfsetstrokecolor{currentstroke}%
\pgfsetdash{}{0pt}%
\pgfpathmoveto{\pgfqpoint{1.459454in}{1.184479in}}%
\pgfpathcurveto{\pgfqpoint{1.464979in}{1.184479in}}{\pgfqpoint{1.470279in}{1.186674in}}{\pgfqpoint{1.474185in}{1.190581in}}%
\pgfpathcurveto{\pgfqpoint{1.478092in}{1.194488in}}{\pgfqpoint{1.480287in}{1.199787in}}{\pgfqpoint{1.480287in}{1.205312in}}%
\pgfpathcurveto{\pgfqpoint{1.480287in}{1.210837in}}{\pgfqpoint{1.478092in}{1.216137in}}{\pgfqpoint{1.474185in}{1.220044in}}%
\pgfpathcurveto{\pgfqpoint{1.470279in}{1.223951in}}{\pgfqpoint{1.464979in}{1.226146in}}{\pgfqpoint{1.459454in}{1.226146in}}%
\pgfpathcurveto{\pgfqpoint{1.453929in}{1.226146in}}{\pgfqpoint{1.448629in}{1.223951in}}{\pgfqpoint{1.444723in}{1.220044in}}%
\pgfpathcurveto{\pgfqpoint{1.440816in}{1.216137in}}{\pgfqpoint{1.438621in}{1.210837in}}{\pgfqpoint{1.438621in}{1.205312in}}%
\pgfpathcurveto{\pgfqpoint{1.438621in}{1.199787in}}{\pgfqpoint{1.440816in}{1.194488in}}{\pgfqpoint{1.444723in}{1.190581in}}%
\pgfpathcurveto{\pgfqpoint{1.448629in}{1.186674in}}{\pgfqpoint{1.453929in}{1.184479in}}{\pgfqpoint{1.459454in}{1.184479in}}%
\pgfpathclose%
\pgfusepath{stroke,fill}%
\end{pgfscope}%
\begin{pgfscope}%
\pgfpathrectangle{\pgfqpoint{0.403630in}{0.251348in}}{\pgfqpoint{1.395000in}{1.359000in}}%
\pgfusepath{clip}%
\pgfsetbuttcap%
\pgfsetroundjoin%
\definecolor{currentfill}{rgb}{0.686275,0.352941,0.313725}%
\pgfsetfillcolor{currentfill}%
\pgfsetlinewidth{0.501875pt}%
\definecolor{currentstroke}{rgb}{0.686275,0.352941,0.313725}%
\pgfsetstrokecolor{currentstroke}%
\pgfsetdash{}{0pt}%
\pgfpathmoveto{\pgfqpoint{0.819395in}{1.291067in}}%
\pgfpathcurveto{\pgfqpoint{0.824920in}{1.291067in}}{\pgfqpoint{0.830220in}{1.293262in}}{\pgfqpoint{0.834127in}{1.297169in}}%
\pgfpathcurveto{\pgfqpoint{0.838033in}{1.301076in}}{\pgfqpoint{0.840228in}{1.306376in}}{\pgfqpoint{0.840228in}{1.311901in}}%
\pgfpathcurveto{\pgfqpoint{0.840228in}{1.317426in}}{\pgfqpoint{0.838033in}{1.322725in}}{\pgfqpoint{0.834127in}{1.326632in}}%
\pgfpathcurveto{\pgfqpoint{0.830220in}{1.330539in}}{\pgfqpoint{0.824920in}{1.332734in}}{\pgfqpoint{0.819395in}{1.332734in}}%
\pgfpathcurveto{\pgfqpoint{0.813870in}{1.332734in}}{\pgfqpoint{0.808571in}{1.330539in}}{\pgfqpoint{0.804664in}{1.326632in}}%
\pgfpathcurveto{\pgfqpoint{0.800757in}{1.322725in}}{\pgfqpoint{0.798562in}{1.317426in}}{\pgfqpoint{0.798562in}{1.311901in}}%
\pgfpathcurveto{\pgfqpoint{0.798562in}{1.306376in}}{\pgfqpoint{0.800757in}{1.301076in}}{\pgfqpoint{0.804664in}{1.297169in}}%
\pgfpathcurveto{\pgfqpoint{0.808571in}{1.293262in}}{\pgfqpoint{0.813870in}{1.291067in}}{\pgfqpoint{0.819395in}{1.291067in}}%
\pgfpathclose%
\pgfusepath{stroke,fill}%
\end{pgfscope}%
\begin{pgfscope}%
\pgfpathrectangle{\pgfqpoint{0.403630in}{0.251348in}}{\pgfqpoint{1.395000in}{1.359000in}}%
\pgfusepath{clip}%
\pgfsetbuttcap%
\pgfsetroundjoin%
\definecolor{currentfill}{rgb}{0.686275,0.352941,0.313725}%
\pgfsetfillcolor{currentfill}%
\pgfsetlinewidth{0.501875pt}%
\definecolor{currentstroke}{rgb}{0.686275,0.352941,0.313725}%
\pgfsetstrokecolor{currentstroke}%
\pgfsetdash{}{0pt}%
\pgfpathmoveto{\pgfqpoint{0.775630in}{1.259091in}}%
\pgfpathcurveto{\pgfqpoint{0.781155in}{1.259091in}}{\pgfqpoint{0.786455in}{1.261286in}}{\pgfqpoint{0.790362in}{1.265193in}}%
\pgfpathcurveto{\pgfqpoint{0.794269in}{1.269100in}}{\pgfqpoint{0.796464in}{1.274399in}}{\pgfqpoint{0.796464in}{1.279924in}}%
\pgfpathcurveto{\pgfqpoint{0.796464in}{1.285449in}}{\pgfqpoint{0.794269in}{1.290749in}}{\pgfqpoint{0.790362in}{1.294656in}}%
\pgfpathcurveto{\pgfqpoint{0.786455in}{1.298562in}}{\pgfqpoint{0.781155in}{1.300757in}}{\pgfqpoint{0.775630in}{1.300757in}}%
\pgfpathcurveto{\pgfqpoint{0.770105in}{1.300757in}}{\pgfqpoint{0.764806in}{1.298562in}}{\pgfqpoint{0.760899in}{1.294656in}}%
\pgfpathcurveto{\pgfqpoint{0.756992in}{1.290749in}}{\pgfqpoint{0.754797in}{1.285449in}}{\pgfqpoint{0.754797in}{1.279924in}}%
\pgfpathcurveto{\pgfqpoint{0.754797in}{1.274399in}}{\pgfqpoint{0.756992in}{1.269100in}}{\pgfqpoint{0.760899in}{1.265193in}}%
\pgfpathcurveto{\pgfqpoint{0.764806in}{1.261286in}}{\pgfqpoint{0.770105in}{1.259091in}}{\pgfqpoint{0.775630in}{1.259091in}}%
\pgfpathclose%
\pgfusepath{stroke,fill}%
\end{pgfscope}%
\begin{pgfscope}%
\pgfpathrectangle{\pgfqpoint{0.403630in}{0.251348in}}{\pgfqpoint{1.395000in}{1.359000in}}%
\pgfusepath{clip}%
\pgfsetbuttcap%
\pgfsetroundjoin%
\definecolor{currentfill}{rgb}{0.686275,0.352941,0.313725}%
\pgfsetfillcolor{currentfill}%
\pgfsetlinewidth{0.501875pt}%
\definecolor{currentstroke}{rgb}{0.686275,0.352941,0.313725}%
\pgfsetstrokecolor{currentstroke}%
\pgfsetdash{}{0pt}%
\pgfpathmoveto{\pgfqpoint{1.125748in}{0.465008in}}%
\pgfpathcurveto{\pgfqpoint{1.131273in}{0.465008in}}{\pgfqpoint{1.136573in}{0.467204in}}{\pgfqpoint{1.140479in}{0.471110in}}%
\pgfpathcurveto{\pgfqpoint{1.144386in}{0.475017in}}{\pgfqpoint{1.146581in}{0.480317in}}{\pgfqpoint{1.146581in}{0.485842in}}%
\pgfpathcurveto{\pgfqpoint{1.146581in}{0.491367in}}{\pgfqpoint{1.144386in}{0.496666in}}{\pgfqpoint{1.140479in}{0.500573in}}%
\pgfpathcurveto{\pgfqpoint{1.136573in}{0.504480in}}{\pgfqpoint{1.131273in}{0.506675in}}{\pgfqpoint{1.125748in}{0.506675in}}%
\pgfpathcurveto{\pgfqpoint{1.120223in}{0.506675in}}{\pgfqpoint{1.114923in}{0.504480in}}{\pgfqpoint{1.111017in}{0.500573in}}%
\pgfpathcurveto{\pgfqpoint{1.107110in}{0.496666in}}{\pgfqpoint{1.104915in}{0.491367in}}{\pgfqpoint{1.104915in}{0.485842in}}%
\pgfpathcurveto{\pgfqpoint{1.104915in}{0.480317in}}{\pgfqpoint{1.107110in}{0.475017in}}{\pgfqpoint{1.111017in}{0.471110in}}%
\pgfpathcurveto{\pgfqpoint{1.114923in}{0.467204in}}{\pgfqpoint{1.120223in}{0.465008in}}{\pgfqpoint{1.125748in}{0.465008in}}%
\pgfpathclose%
\pgfusepath{stroke,fill}%
\end{pgfscope}%
\begin{pgfscope}%
\pgfpathrectangle{\pgfqpoint{0.403630in}{0.251348in}}{\pgfqpoint{1.395000in}{1.359000in}}%
\pgfusepath{clip}%
\pgfsetbuttcap%
\pgfsetroundjoin%
\definecolor{currentfill}{rgb}{0.686275,0.352941,0.313725}%
\pgfsetfillcolor{currentfill}%
\pgfsetlinewidth{0.501875pt}%
\definecolor{currentstroke}{rgb}{0.686275,0.352941,0.313725}%
\pgfsetstrokecolor{currentstroke}%
\pgfsetdash{}{0pt}%
\pgfpathmoveto{\pgfqpoint{1.508689in}{1.083220in}}%
\pgfpathcurveto{\pgfqpoint{1.514214in}{1.083220in}}{\pgfqpoint{1.519514in}{1.085415in}}{\pgfqpoint{1.523421in}{1.089322in}}%
\pgfpathcurveto{\pgfqpoint{1.527327in}{1.093229in}}{\pgfqpoint{1.529523in}{1.098528in}}{\pgfqpoint{1.529523in}{1.104054in}}%
\pgfpathcurveto{\pgfqpoint{1.529523in}{1.109579in}}{\pgfqpoint{1.527327in}{1.114878in}}{\pgfqpoint{1.523421in}{1.118785in}}%
\pgfpathcurveto{\pgfqpoint{1.519514in}{1.122692in}}{\pgfqpoint{1.514214in}{1.124887in}}{\pgfqpoint{1.508689in}{1.124887in}}%
\pgfpathcurveto{\pgfqpoint{1.503164in}{1.124887in}}{\pgfqpoint{1.497865in}{1.122692in}}{\pgfqpoint{1.493958in}{1.118785in}}%
\pgfpathcurveto{\pgfqpoint{1.490051in}{1.114878in}}{\pgfqpoint{1.487856in}{1.109579in}}{\pgfqpoint{1.487856in}{1.104054in}}%
\pgfpathcurveto{\pgfqpoint{1.487856in}{1.098528in}}{\pgfqpoint{1.490051in}{1.093229in}}{\pgfqpoint{1.493958in}{1.089322in}}%
\pgfpathcurveto{\pgfqpoint{1.497865in}{1.085415in}}{\pgfqpoint{1.503164in}{1.083220in}}{\pgfqpoint{1.508689in}{1.083220in}}%
\pgfpathclose%
\pgfusepath{stroke,fill}%
\end{pgfscope}%
\begin{pgfscope}%
\pgfpathrectangle{\pgfqpoint{0.403630in}{0.251348in}}{\pgfqpoint{1.395000in}{1.359000in}}%
\pgfusepath{clip}%
\pgfsetbuttcap%
\pgfsetroundjoin%
\definecolor{currentfill}{rgb}{0.686275,0.352941,0.313725}%
\pgfsetfillcolor{currentfill}%
\pgfsetlinewidth{0.501875pt}%
\definecolor{currentstroke}{rgb}{0.686275,0.352941,0.313725}%
\pgfsetstrokecolor{currentstroke}%
\pgfsetdash{}{0pt}%
\pgfpathmoveto{\pgfqpoint{1.382866in}{0.512973in}}%
\pgfpathcurveto{\pgfqpoint{1.388391in}{0.512973in}}{\pgfqpoint{1.393690in}{0.515168in}}{\pgfqpoint{1.397597in}{0.519075in}}%
\pgfpathcurveto{\pgfqpoint{1.401504in}{0.522982in}}{\pgfqpoint{1.403699in}{0.528281in}}{\pgfqpoint{1.403699in}{0.533806in}}%
\pgfpathcurveto{\pgfqpoint{1.403699in}{0.539332in}}{\pgfqpoint{1.401504in}{0.544631in}}{\pgfqpoint{1.397597in}{0.548538in}}%
\pgfpathcurveto{\pgfqpoint{1.393690in}{0.552445in}}{\pgfqpoint{1.388391in}{0.554640in}}{\pgfqpoint{1.382866in}{0.554640in}}%
\pgfpathcurveto{\pgfqpoint{1.377341in}{0.554640in}}{\pgfqpoint{1.372041in}{0.552445in}}{\pgfqpoint{1.368134in}{0.548538in}}%
\pgfpathcurveto{\pgfqpoint{1.364228in}{0.544631in}}{\pgfqpoint{1.362032in}{0.539332in}}{\pgfqpoint{1.362032in}{0.533806in}}%
\pgfpathcurveto{\pgfqpoint{1.362032in}{0.528281in}}{\pgfqpoint{1.364228in}{0.522982in}}{\pgfqpoint{1.368134in}{0.519075in}}%
\pgfpathcurveto{\pgfqpoint{1.372041in}{0.515168in}}{\pgfqpoint{1.377341in}{0.512973in}}{\pgfqpoint{1.382866in}{0.512973in}}%
\pgfpathclose%
\pgfusepath{stroke,fill}%
\end{pgfscope}%
\begin{pgfscope}%
\pgfpathrectangle{\pgfqpoint{0.403630in}{0.251348in}}{\pgfqpoint{1.395000in}{1.359000in}}%
\pgfusepath{clip}%
\pgfsetbuttcap%
\pgfsetroundjoin%
\definecolor{currentfill}{rgb}{0.686275,0.352941,0.313725}%
\pgfsetfillcolor{currentfill}%
\pgfsetlinewidth{0.501875pt}%
\definecolor{currentstroke}{rgb}{0.686275,0.352941,0.313725}%
\pgfsetstrokecolor{currentstroke}%
\pgfsetdash{}{0pt}%
\pgfpathmoveto{\pgfqpoint{0.627925in}{0.848726in}}%
\pgfpathcurveto{\pgfqpoint{0.633450in}{0.848726in}}{\pgfqpoint{0.638749in}{0.850921in}}{\pgfqpoint{0.642656in}{0.854828in}}%
\pgfpathcurveto{\pgfqpoint{0.646563in}{0.858735in}}{\pgfqpoint{0.648758in}{0.864034in}}{\pgfqpoint{0.648758in}{0.869559in}}%
\pgfpathcurveto{\pgfqpoint{0.648758in}{0.875084in}}{\pgfqpoint{0.646563in}{0.880384in}}{\pgfqpoint{0.642656in}{0.884291in}}%
\pgfpathcurveto{\pgfqpoint{0.638749in}{0.888198in}}{\pgfqpoint{0.633450in}{0.890393in}}{\pgfqpoint{0.627925in}{0.890393in}}%
\pgfpathcurveto{\pgfqpoint{0.622399in}{0.890393in}}{\pgfqpoint{0.617100in}{0.888198in}}{\pgfqpoint{0.613193in}{0.884291in}}%
\pgfpathcurveto{\pgfqpoint{0.609286in}{0.880384in}}{\pgfqpoint{0.607091in}{0.875084in}}{\pgfqpoint{0.607091in}{0.869559in}}%
\pgfpathcurveto{\pgfqpoint{0.607091in}{0.864034in}}{\pgfqpoint{0.609286in}{0.858735in}}{\pgfqpoint{0.613193in}{0.854828in}}%
\pgfpathcurveto{\pgfqpoint{0.617100in}{0.850921in}}{\pgfqpoint{0.622399in}{0.848726in}}{\pgfqpoint{0.627925in}{0.848726in}}%
\pgfpathclose%
\pgfusepath{stroke,fill}%
\end{pgfscope}%
\begin{pgfscope}%
\pgfpathrectangle{\pgfqpoint{0.403630in}{0.251348in}}{\pgfqpoint{1.395000in}{1.359000in}}%
\pgfusepath{clip}%
\pgfsetbuttcap%
\pgfsetroundjoin%
\definecolor{currentfill}{rgb}{0.686275,0.352941,0.313725}%
\pgfsetfillcolor{currentfill}%
\pgfsetlinewidth{0.501875pt}%
\definecolor{currentstroke}{rgb}{0.686275,0.352941,0.313725}%
\pgfsetstrokecolor{currentstroke}%
\pgfsetdash{}{0pt}%
\pgfpathmoveto{\pgfqpoint{0.808454in}{1.275079in}}%
\pgfpathcurveto{\pgfqpoint{0.813979in}{1.275079in}}{\pgfqpoint{0.819279in}{1.277274in}}{\pgfqpoint{0.823185in}{1.281181in}}%
\pgfpathcurveto{\pgfqpoint{0.827092in}{1.285088in}}{\pgfqpoint{0.829287in}{1.290387in}}{\pgfqpoint{0.829287in}{1.295912in}}%
\pgfpathcurveto{\pgfqpoint{0.829287in}{1.301437in}}{\pgfqpoint{0.827092in}{1.306737in}}{\pgfqpoint{0.823185in}{1.310644in}}%
\pgfpathcurveto{\pgfqpoint{0.819279in}{1.314551in}}{\pgfqpoint{0.813979in}{1.316746in}}{\pgfqpoint{0.808454in}{1.316746in}}%
\pgfpathcurveto{\pgfqpoint{0.802929in}{1.316746in}}{\pgfqpoint{0.797629in}{1.314551in}}{\pgfqpoint{0.793723in}{1.310644in}}%
\pgfpathcurveto{\pgfqpoint{0.789816in}{1.306737in}}{\pgfqpoint{0.787621in}{1.301437in}}{\pgfqpoint{0.787621in}{1.295912in}}%
\pgfpathcurveto{\pgfqpoint{0.787621in}{1.290387in}}{\pgfqpoint{0.789816in}{1.285088in}}{\pgfqpoint{0.793723in}{1.281181in}}%
\pgfpathcurveto{\pgfqpoint{0.797629in}{1.277274in}}{\pgfqpoint{0.802929in}{1.275079in}}{\pgfqpoint{0.808454in}{1.275079in}}%
\pgfpathclose%
\pgfusepath{stroke,fill}%
\end{pgfscope}%
\begin{pgfscope}%
\pgfpathrectangle{\pgfqpoint{0.403630in}{0.251348in}}{\pgfqpoint{1.395000in}{1.359000in}}%
\pgfusepath{clip}%
\pgfsetbuttcap%
\pgfsetroundjoin%
\definecolor{currentfill}{rgb}{0.686275,0.352941,0.313725}%
\pgfsetfillcolor{currentfill}%
\pgfsetlinewidth{0.501875pt}%
\definecolor{currentstroke}{rgb}{0.686275,0.352941,0.313725}%
\pgfsetstrokecolor{currentstroke}%
\pgfsetdash{}{0pt}%
\pgfpathmoveto{\pgfqpoint{1.065572in}{0.411714in}}%
\pgfpathcurveto{\pgfqpoint{1.071097in}{0.411714in}}{\pgfqpoint{1.076396in}{0.413909in}}{\pgfqpoint{1.080303in}{0.417816in}}%
\pgfpathcurveto{\pgfqpoint{1.084210in}{0.421723in}}{\pgfqpoint{1.086405in}{0.427023in}}{\pgfqpoint{1.086405in}{0.432548in}}%
\pgfpathcurveto{\pgfqpoint{1.086405in}{0.438073in}}{\pgfqpoint{1.084210in}{0.443372in}}{\pgfqpoint{1.080303in}{0.447279in}}%
\pgfpathcurveto{\pgfqpoint{1.076396in}{0.451186in}}{\pgfqpoint{1.071097in}{0.453381in}}{\pgfqpoint{1.065572in}{0.453381in}}%
\pgfpathcurveto{\pgfqpoint{1.060047in}{0.453381in}}{\pgfqpoint{1.054747in}{0.451186in}}{\pgfqpoint{1.050840in}{0.447279in}}%
\pgfpathcurveto{\pgfqpoint{1.046933in}{0.443372in}}{\pgfqpoint{1.044738in}{0.438073in}}{\pgfqpoint{1.044738in}{0.432548in}}%
\pgfpathcurveto{\pgfqpoint{1.044738in}{0.427023in}}{\pgfqpoint{1.046933in}{0.421723in}}{\pgfqpoint{1.050840in}{0.417816in}}%
\pgfpathcurveto{\pgfqpoint{1.054747in}{0.413909in}}{\pgfqpoint{1.060047in}{0.411714in}}{\pgfqpoint{1.065572in}{0.411714in}}%
\pgfpathclose%
\pgfusepath{stroke,fill}%
\end{pgfscope}%
\begin{pgfscope}%
\pgfpathrectangle{\pgfqpoint{0.403630in}{0.251348in}}{\pgfqpoint{1.395000in}{1.359000in}}%
\pgfusepath{clip}%
\pgfsetbuttcap%
\pgfsetroundjoin%
\definecolor{currentfill}{rgb}{0.686275,0.352941,0.313725}%
\pgfsetfillcolor{currentfill}%
\pgfsetlinewidth{0.501875pt}%
\definecolor{currentstroke}{rgb}{0.686275,0.352941,0.313725}%
\pgfsetstrokecolor{currentstroke}%
\pgfsetdash{}{0pt}%
\pgfpathmoveto{\pgfqpoint{1.246101in}{0.507644in}}%
\pgfpathcurveto{\pgfqpoint{1.251626in}{0.507644in}}{\pgfqpoint{1.256926in}{0.509839in}}{\pgfqpoint{1.260832in}{0.513746in}}%
\pgfpathcurveto{\pgfqpoint{1.264739in}{0.517652in}}{\pgfqpoint{1.266934in}{0.522952in}}{\pgfqpoint{1.266934in}{0.528477in}}%
\pgfpathcurveto{\pgfqpoint{1.266934in}{0.534002in}}{\pgfqpoint{1.264739in}{0.539302in}}{\pgfqpoint{1.260832in}{0.543208in}}%
\pgfpathcurveto{\pgfqpoint{1.256926in}{0.547115in}}{\pgfqpoint{1.251626in}{0.549310in}}{\pgfqpoint{1.246101in}{0.549310in}}%
\pgfpathcurveto{\pgfqpoint{1.240576in}{0.549310in}}{\pgfqpoint{1.235276in}{0.547115in}}{\pgfqpoint{1.231370in}{0.543208in}}%
\pgfpathcurveto{\pgfqpoint{1.227463in}{0.539302in}}{\pgfqpoint{1.225268in}{0.534002in}}{\pgfqpoint{1.225268in}{0.528477in}}%
\pgfpathcurveto{\pgfqpoint{1.225268in}{0.522952in}}{\pgfqpoint{1.227463in}{0.517652in}}{\pgfqpoint{1.231370in}{0.513746in}}%
\pgfpathcurveto{\pgfqpoint{1.235276in}{0.509839in}}{\pgfqpoint{1.240576in}{0.507644in}}{\pgfqpoint{1.246101in}{0.507644in}}%
\pgfpathclose%
\pgfusepath{stroke,fill}%
\end{pgfscope}%
\begin{pgfscope}%
\pgfpathrectangle{\pgfqpoint{0.403630in}{0.251348in}}{\pgfqpoint{1.395000in}{1.359000in}}%
\pgfusepath{clip}%
\pgfsetbuttcap%
\pgfsetroundjoin%
\definecolor{currentfill}{rgb}{0.686275,0.352941,0.313725}%
\pgfsetfillcolor{currentfill}%
\pgfsetlinewidth{0.501875pt}%
\definecolor{currentstroke}{rgb}{0.686275,0.352941,0.313725}%
\pgfsetstrokecolor{currentstroke}%
\pgfsetdash{}{0pt}%
\pgfpathmoveto{\pgfqpoint{1.541513in}{0.736808in}}%
\pgfpathcurveto{\pgfqpoint{1.547038in}{0.736808in}}{\pgfqpoint{1.552337in}{0.739004in}}{\pgfqpoint{1.556244in}{0.742910in}}%
\pgfpathcurveto{\pgfqpoint{1.560151in}{0.746817in}}{\pgfqpoint{1.562346in}{0.752117in}}{\pgfqpoint{1.562346in}{0.757642in}}%
\pgfpathcurveto{\pgfqpoint{1.562346in}{0.763167in}}{\pgfqpoint{1.560151in}{0.768466in}}{\pgfqpoint{1.556244in}{0.772373in}}%
\pgfpathcurveto{\pgfqpoint{1.552337in}{0.776280in}}{\pgfqpoint{1.547038in}{0.778475in}}{\pgfqpoint{1.541513in}{0.778475in}}%
\pgfpathcurveto{\pgfqpoint{1.535988in}{0.778475in}}{\pgfqpoint{1.530688in}{0.776280in}}{\pgfqpoint{1.526781in}{0.772373in}}%
\pgfpathcurveto{\pgfqpoint{1.522875in}{0.768466in}}{\pgfqpoint{1.520679in}{0.763167in}}{\pgfqpoint{1.520679in}{0.757642in}}%
\pgfpathcurveto{\pgfqpoint{1.520679in}{0.752117in}}{\pgfqpoint{1.522875in}{0.746817in}}{\pgfqpoint{1.526781in}{0.742910in}}%
\pgfpathcurveto{\pgfqpoint{1.530688in}{0.739004in}}{\pgfqpoint{1.535988in}{0.736808in}}{\pgfqpoint{1.541513in}{0.736808in}}%
\pgfpathclose%
\pgfusepath{stroke,fill}%
\end{pgfscope}%
\begin{pgfscope}%
\pgfpathrectangle{\pgfqpoint{0.403630in}{0.251348in}}{\pgfqpoint{1.395000in}{1.359000in}}%
\pgfusepath{clip}%
\pgfsetbuttcap%
\pgfsetroundjoin%
\definecolor{currentfill}{rgb}{0.686275,0.352941,0.313725}%
\pgfsetfillcolor{currentfill}%
\pgfsetlinewidth{0.501875pt}%
\definecolor{currentstroke}{rgb}{0.686275,0.352941,0.313725}%
\pgfsetstrokecolor{currentstroke}%
\pgfsetdash{}{0pt}%
\pgfpathmoveto{\pgfqpoint{1.240630in}{1.355020in}}%
\pgfpathcurveto{\pgfqpoint{1.246155in}{1.355020in}}{\pgfqpoint{1.251455in}{1.357215in}}{\pgfqpoint{1.255362in}{1.361122in}}%
\pgfpathcurveto{\pgfqpoint{1.259269in}{1.365029in}}{\pgfqpoint{1.261464in}{1.370328in}}{\pgfqpoint{1.261464in}{1.375854in}}%
\pgfpathcurveto{\pgfqpoint{1.261464in}{1.381379in}}{\pgfqpoint{1.259269in}{1.386678in}}{\pgfqpoint{1.255362in}{1.390585in}}%
\pgfpathcurveto{\pgfqpoint{1.251455in}{1.394492in}}{\pgfqpoint{1.246155in}{1.396687in}}{\pgfqpoint{1.240630in}{1.396687in}}%
\pgfpathcurveto{\pgfqpoint{1.235105in}{1.396687in}}{\pgfqpoint{1.229806in}{1.394492in}}{\pgfqpoint{1.225899in}{1.390585in}}%
\pgfpathcurveto{\pgfqpoint{1.221992in}{1.386678in}}{\pgfqpoint{1.219797in}{1.381379in}}{\pgfqpoint{1.219797in}{1.375854in}}%
\pgfpathcurveto{\pgfqpoint{1.219797in}{1.370328in}}{\pgfqpoint{1.221992in}{1.365029in}}{\pgfqpoint{1.225899in}{1.361122in}}%
\pgfpathcurveto{\pgfqpoint{1.229806in}{1.357215in}}{\pgfqpoint{1.235105in}{1.355020in}}{\pgfqpoint{1.240630in}{1.355020in}}%
\pgfpathclose%
\pgfusepath{stroke,fill}%
\end{pgfscope}%
\begin{pgfscope}%
\pgfpathrectangle{\pgfqpoint{0.403630in}{0.251348in}}{\pgfqpoint{1.395000in}{1.359000in}}%
\pgfusepath{clip}%
\pgfsetbuttcap%
\pgfsetroundjoin%
\definecolor{currentfill}{rgb}{0.686275,0.352941,0.313725}%
\pgfsetfillcolor{currentfill}%
\pgfsetlinewidth{0.501875pt}%
\definecolor{currentstroke}{rgb}{0.686275,0.352941,0.313725}%
\pgfsetstrokecolor{currentstroke}%
\pgfsetdash{}{0pt}%
\pgfpathmoveto{\pgfqpoint{1.464925in}{1.200467in}}%
\pgfpathcurveto{\pgfqpoint{1.470450in}{1.200467in}}{\pgfqpoint{1.475749in}{1.202662in}}{\pgfqpoint{1.479656in}{1.206569in}}%
\pgfpathcurveto{\pgfqpoint{1.483563in}{1.210476in}}{\pgfqpoint{1.485758in}{1.215776in}}{\pgfqpoint{1.485758in}{1.221301in}}%
\pgfpathcurveto{\pgfqpoint{1.485758in}{1.226826in}}{\pgfqpoint{1.483563in}{1.232125in}}{\pgfqpoint{1.479656in}{1.236032in}}%
\pgfpathcurveto{\pgfqpoint{1.475749in}{1.239939in}}{\pgfqpoint{1.470450in}{1.242134in}}{\pgfqpoint{1.464925in}{1.242134in}}%
\pgfpathcurveto{\pgfqpoint{1.459399in}{1.242134in}}{\pgfqpoint{1.454100in}{1.239939in}}{\pgfqpoint{1.450193in}{1.236032in}}%
\pgfpathcurveto{\pgfqpoint{1.446286in}{1.232125in}}{\pgfqpoint{1.444091in}{1.226826in}}{\pgfqpoint{1.444091in}{1.221301in}}%
\pgfpathcurveto{\pgfqpoint{1.444091in}{1.215776in}}{\pgfqpoint{1.446286in}{1.210476in}}{\pgfqpoint{1.450193in}{1.206569in}}%
\pgfpathcurveto{\pgfqpoint{1.454100in}{1.202662in}}{\pgfqpoint{1.459399in}{1.200467in}}{\pgfqpoint{1.464925in}{1.200467in}}%
\pgfpathclose%
\pgfusepath{stroke,fill}%
\end{pgfscope}%
\begin{pgfscope}%
\pgfpathrectangle{\pgfqpoint{0.403630in}{0.251348in}}{\pgfqpoint{1.395000in}{1.359000in}}%
\pgfusepath{clip}%
\pgfsetbuttcap%
\pgfsetroundjoin%
\definecolor{currentfill}{rgb}{0.686275,0.352941,0.313725}%
\pgfsetfillcolor{currentfill}%
\pgfsetlinewidth{0.501875pt}%
\definecolor{currentstroke}{rgb}{0.686275,0.352941,0.313725}%
\pgfsetstrokecolor{currentstroke}%
\pgfsetdash{}{0pt}%
\pgfpathmoveto{\pgfqpoint{1.081983in}{0.486326in}}%
\pgfpathcurveto{\pgfqpoint{1.087508in}{0.486326in}}{\pgfqpoint{1.092808in}{0.488521in}}{\pgfqpoint{1.096715in}{0.492428in}}%
\pgfpathcurveto{\pgfqpoint{1.100622in}{0.496335in}}{\pgfqpoint{1.102817in}{0.501634in}}{\pgfqpoint{1.102817in}{0.507159in}}%
\pgfpathcurveto{\pgfqpoint{1.102817in}{0.512684in}}{\pgfqpoint{1.100622in}{0.517984in}}{\pgfqpoint{1.096715in}{0.521891in}}%
\pgfpathcurveto{\pgfqpoint{1.092808in}{0.525798in}}{\pgfqpoint{1.087508in}{0.527993in}}{\pgfqpoint{1.081983in}{0.527993in}}%
\pgfpathcurveto{\pgfqpoint{1.076458in}{0.527993in}}{\pgfqpoint{1.071159in}{0.525798in}}{\pgfqpoint{1.067252in}{0.521891in}}%
\pgfpathcurveto{\pgfqpoint{1.063345in}{0.517984in}}{\pgfqpoint{1.061150in}{0.512684in}}{\pgfqpoint{1.061150in}{0.507159in}}%
\pgfpathcurveto{\pgfqpoint{1.061150in}{0.501634in}}{\pgfqpoint{1.063345in}{0.496335in}}{\pgfqpoint{1.067252in}{0.492428in}}%
\pgfpathcurveto{\pgfqpoint{1.071159in}{0.488521in}}{\pgfqpoint{1.076458in}{0.486326in}}{\pgfqpoint{1.081983in}{0.486326in}}%
\pgfpathclose%
\pgfusepath{stroke,fill}%
\end{pgfscope}%
\begin{pgfscope}%
\pgfpathrectangle{\pgfqpoint{0.403630in}{0.251348in}}{\pgfqpoint{1.395000in}{1.359000in}}%
\pgfusepath{clip}%
\pgfsetbuttcap%
\pgfsetroundjoin%
\definecolor{currentfill}{rgb}{0.686275,0.352941,0.313725}%
\pgfsetfillcolor{currentfill}%
\pgfsetlinewidth{0.501875pt}%
\definecolor{currentstroke}{rgb}{0.686275,0.352941,0.313725}%
\pgfsetstrokecolor{currentstroke}%
\pgfsetdash{}{0pt}%
\pgfpathmoveto{\pgfqpoint{1.087454in}{1.349691in}}%
\pgfpathcurveto{\pgfqpoint{1.092979in}{1.349691in}}{\pgfqpoint{1.098279in}{1.351886in}}{\pgfqpoint{1.102185in}{1.355793in}}%
\pgfpathcurveto{\pgfqpoint{1.106092in}{1.359700in}}{\pgfqpoint{1.108287in}{1.364999in}}{\pgfqpoint{1.108287in}{1.370524in}}%
\pgfpathcurveto{\pgfqpoint{1.108287in}{1.376049in}}{\pgfqpoint{1.106092in}{1.381349in}}{\pgfqpoint{1.102185in}{1.385256in}}%
\pgfpathcurveto{\pgfqpoint{1.098279in}{1.389162in}}{\pgfqpoint{1.092979in}{1.391357in}}{\pgfqpoint{1.087454in}{1.391357in}}%
\pgfpathcurveto{\pgfqpoint{1.081929in}{1.391357in}}{\pgfqpoint{1.076629in}{1.389162in}}{\pgfqpoint{1.072723in}{1.385256in}}%
\pgfpathcurveto{\pgfqpoint{1.068816in}{1.381349in}}{\pgfqpoint{1.066621in}{1.376049in}}{\pgfqpoint{1.066621in}{1.370524in}}%
\pgfpathcurveto{\pgfqpoint{1.066621in}{1.364999in}}{\pgfqpoint{1.068816in}{1.359700in}}{\pgfqpoint{1.072723in}{1.355793in}}%
\pgfpathcurveto{\pgfqpoint{1.076629in}{1.351886in}}{\pgfqpoint{1.081929in}{1.349691in}}{\pgfqpoint{1.087454in}{1.349691in}}%
\pgfpathclose%
\pgfusepath{stroke,fill}%
\end{pgfscope}%
\begin{pgfscope}%
\pgfpathrectangle{\pgfqpoint{0.403630in}{0.251348in}}{\pgfqpoint{1.395000in}{1.359000in}}%
\pgfusepath{clip}%
\pgfsetbuttcap%
\pgfsetroundjoin%
\definecolor{currentfill}{rgb}{0.686275,0.352941,0.313725}%
\pgfsetfillcolor{currentfill}%
\pgfsetlinewidth{0.501875pt}%
\definecolor{currentstroke}{rgb}{0.686275,0.352941,0.313725}%
\pgfsetstrokecolor{currentstroke}%
\pgfsetdash{}{0pt}%
\pgfpathmoveto{\pgfqpoint{1.360983in}{1.264420in}}%
\pgfpathcurveto{\pgfqpoint{1.366508in}{1.264420in}}{\pgfqpoint{1.371808in}{1.266615in}}{\pgfqpoint{1.375715in}{1.270522in}}%
\pgfpathcurveto{\pgfqpoint{1.379622in}{1.274429in}}{\pgfqpoint{1.381817in}{1.279728in}}{\pgfqpoint{1.381817in}{1.285254in}}%
\pgfpathcurveto{\pgfqpoint{1.381817in}{1.290779in}}{\pgfqpoint{1.379622in}{1.296078in}}{\pgfqpoint{1.375715in}{1.299985in}}%
\pgfpathcurveto{\pgfqpoint{1.371808in}{1.303892in}}{\pgfqpoint{1.366508in}{1.306087in}}{\pgfqpoint{1.360983in}{1.306087in}}%
\pgfpathcurveto{\pgfqpoint{1.355458in}{1.306087in}}{\pgfqpoint{1.350159in}{1.303892in}}{\pgfqpoint{1.346252in}{1.299985in}}%
\pgfpathcurveto{\pgfqpoint{1.342345in}{1.296078in}}{\pgfqpoint{1.340150in}{1.290779in}}{\pgfqpoint{1.340150in}{1.285254in}}%
\pgfpathcurveto{\pgfqpoint{1.340150in}{1.279728in}}{\pgfqpoint{1.342345in}{1.274429in}}{\pgfqpoint{1.346252in}{1.270522in}}%
\pgfpathcurveto{\pgfqpoint{1.350159in}{1.266615in}}{\pgfqpoint{1.355458in}{1.264420in}}{\pgfqpoint{1.360983in}{1.264420in}}%
\pgfpathclose%
\pgfusepath{stroke,fill}%
\end{pgfscope}%
\begin{pgfscope}%
\pgfpathrectangle{\pgfqpoint{0.403630in}{0.251348in}}{\pgfqpoint{1.395000in}{1.359000in}}%
\pgfusepath{clip}%
\pgfsetbuttcap%
\pgfsetroundjoin%
\definecolor{currentfill}{rgb}{0.686275,0.352941,0.313725}%
\pgfsetfillcolor{currentfill}%
\pgfsetlinewidth{0.501875pt}%
\definecolor{currentstroke}{rgb}{0.686275,0.352941,0.313725}%
\pgfsetstrokecolor{currentstroke}%
\pgfsetdash{}{0pt}%
\pgfpathmoveto{\pgfqpoint{0.616983in}{0.838067in}}%
\pgfpathcurveto{\pgfqpoint{0.622508in}{0.838067in}}{\pgfqpoint{0.627808in}{0.840262in}}{\pgfqpoint{0.631715in}{0.844169in}}%
\pgfpathcurveto{\pgfqpoint{0.635622in}{0.848076in}}{\pgfqpoint{0.637817in}{0.853376in}}{\pgfqpoint{0.637817in}{0.858901in}}%
\pgfpathcurveto{\pgfqpoint{0.637817in}{0.864426in}}{\pgfqpoint{0.635622in}{0.869725in}}{\pgfqpoint{0.631715in}{0.873632in}}%
\pgfpathcurveto{\pgfqpoint{0.627808in}{0.877539in}}{\pgfqpoint{0.622508in}{0.879734in}}{\pgfqpoint{0.616983in}{0.879734in}}%
\pgfpathcurveto{\pgfqpoint{0.611458in}{0.879734in}}{\pgfqpoint{0.606159in}{0.877539in}}{\pgfqpoint{0.602252in}{0.873632in}}%
\pgfpathcurveto{\pgfqpoint{0.598345in}{0.869725in}}{\pgfqpoint{0.596150in}{0.864426in}}{\pgfqpoint{0.596150in}{0.858901in}}%
\pgfpathcurveto{\pgfqpoint{0.596150in}{0.853376in}}{\pgfqpoint{0.598345in}{0.848076in}}{\pgfqpoint{0.602252in}{0.844169in}}%
\pgfpathcurveto{\pgfqpoint{0.606159in}{0.840262in}}{\pgfqpoint{0.611458in}{0.838067in}}{\pgfqpoint{0.616983in}{0.838067in}}%
\pgfpathclose%
\pgfusepath{stroke,fill}%
\end{pgfscope}%
\begin{pgfscope}%
\pgfpathrectangle{\pgfqpoint{0.403630in}{0.251348in}}{\pgfqpoint{1.395000in}{1.359000in}}%
\pgfusepath{clip}%
\pgfsetbuttcap%
\pgfsetroundjoin%
\definecolor{currentfill}{rgb}{0.686275,0.352941,0.313725}%
\pgfsetfillcolor{currentfill}%
\pgfsetlinewidth{0.501875pt}%
\definecolor{currentstroke}{rgb}{0.686275,0.352941,0.313725}%
\pgfsetstrokecolor{currentstroke}%
\pgfsetdash{}{0pt}%
\pgfpathmoveto{\pgfqpoint{1.421160in}{1.312385in}}%
\pgfpathcurveto{\pgfqpoint{1.426685in}{1.312385in}}{\pgfqpoint{1.431984in}{1.314580in}}{\pgfqpoint{1.435891in}{1.318487in}}%
\pgfpathcurveto{\pgfqpoint{1.439798in}{1.322394in}}{\pgfqpoint{1.441993in}{1.327693in}}{\pgfqpoint{1.441993in}{1.333218in}}%
\pgfpathcurveto{\pgfqpoint{1.441993in}{1.338743in}}{\pgfqpoint{1.439798in}{1.344043in}}{\pgfqpoint{1.435891in}{1.347950in}}%
\pgfpathcurveto{\pgfqpoint{1.431984in}{1.351856in}}{\pgfqpoint{1.426685in}{1.354052in}}{\pgfqpoint{1.421160in}{1.354052in}}%
\pgfpathcurveto{\pgfqpoint{1.415635in}{1.354052in}}{\pgfqpoint{1.410335in}{1.351856in}}{\pgfqpoint{1.406428in}{1.347950in}}%
\pgfpathcurveto{\pgfqpoint{1.402522in}{1.344043in}}{\pgfqpoint{1.400326in}{1.338743in}}{\pgfqpoint{1.400326in}{1.333218in}}%
\pgfpathcurveto{\pgfqpoint{1.400326in}{1.327693in}}{\pgfqpoint{1.402522in}{1.322394in}}{\pgfqpoint{1.406428in}{1.318487in}}%
\pgfpathcurveto{\pgfqpoint{1.410335in}{1.314580in}}{\pgfqpoint{1.415635in}{1.312385in}}{\pgfqpoint{1.421160in}{1.312385in}}%
\pgfpathclose%
\pgfusepath{stroke,fill}%
\end{pgfscope}%
\begin{pgfscope}%
\pgfpathrectangle{\pgfqpoint{0.403630in}{0.251348in}}{\pgfqpoint{1.395000in}{1.359000in}}%
\pgfusepath{clip}%
\pgfsetbuttcap%
\pgfsetroundjoin%
\definecolor{currentfill}{rgb}{0.686275,0.352941,0.313725}%
\pgfsetfillcolor{currentfill}%
\pgfsetlinewidth{0.501875pt}%
\definecolor{currentstroke}{rgb}{0.686275,0.352941,0.313725}%
\pgfsetstrokecolor{currentstroke}%
\pgfsetdash{}{0pt}%
\pgfpathmoveto{\pgfqpoint{0.649807in}{0.758126in}}%
\pgfpathcurveto{\pgfqpoint{0.655332in}{0.758126in}}{\pgfqpoint{0.660631in}{0.760321in}}{\pgfqpoint{0.664538in}{0.764228in}}%
\pgfpathcurveto{\pgfqpoint{0.668445in}{0.768135in}}{\pgfqpoint{0.670640in}{0.773434in}}{\pgfqpoint{0.670640in}{0.778959in}}%
\pgfpathcurveto{\pgfqpoint{0.670640in}{0.784484in}}{\pgfqpoint{0.668445in}{0.789784in}}{\pgfqpoint{0.664538in}{0.793691in}}%
\pgfpathcurveto{\pgfqpoint{0.660631in}{0.797598in}}{\pgfqpoint{0.655332in}{0.799793in}}{\pgfqpoint{0.649807in}{0.799793in}}%
\pgfpathcurveto{\pgfqpoint{0.644282in}{0.799793in}}{\pgfqpoint{0.638982in}{0.797598in}}{\pgfqpoint{0.635075in}{0.793691in}}%
\pgfpathcurveto{\pgfqpoint{0.631169in}{0.789784in}}{\pgfqpoint{0.628974in}{0.784484in}}{\pgfqpoint{0.628974in}{0.778959in}}%
\pgfpathcurveto{\pgfqpoint{0.628974in}{0.773434in}}{\pgfqpoint{0.631169in}{0.768135in}}{\pgfqpoint{0.635075in}{0.764228in}}%
\pgfpathcurveto{\pgfqpoint{0.638982in}{0.760321in}}{\pgfqpoint{0.644282in}{0.758126in}}{\pgfqpoint{0.649807in}{0.758126in}}%
\pgfpathclose%
\pgfusepath{stroke,fill}%
\end{pgfscope}%
\begin{pgfscope}%
\pgfpathrectangle{\pgfqpoint{0.403630in}{0.251348in}}{\pgfqpoint{1.395000in}{1.359000in}}%
\pgfusepath{clip}%
\pgfsetbuttcap%
\pgfsetroundjoin%
\definecolor{currentfill}{rgb}{0.686275,0.352941,0.313725}%
\pgfsetfillcolor{currentfill}%
\pgfsetlinewidth{0.501875pt}%
\definecolor{currentstroke}{rgb}{0.686275,0.352941,0.313725}%
\pgfsetstrokecolor{currentstroke}%
\pgfsetdash{}{0pt}%
\pgfpathmoveto{\pgfqpoint{0.677160in}{1.088550in}}%
\pgfpathcurveto{\pgfqpoint{0.682685in}{1.088550in}}{\pgfqpoint{0.687984in}{1.090745in}}{\pgfqpoint{0.691891in}{1.094652in}}%
\pgfpathcurveto{\pgfqpoint{0.695798in}{1.098558in}}{\pgfqpoint{0.697993in}{1.103858in}}{\pgfqpoint{0.697993in}{1.109383in}}%
\pgfpathcurveto{\pgfqpoint{0.697993in}{1.114908in}}{\pgfqpoint{0.695798in}{1.120208in}}{\pgfqpoint{0.691891in}{1.124114in}}%
\pgfpathcurveto{\pgfqpoint{0.687984in}{1.128021in}}{\pgfqpoint{0.682685in}{1.130216in}}{\pgfqpoint{0.677160in}{1.130216in}}%
\pgfpathcurveto{\pgfqpoint{0.671635in}{1.130216in}}{\pgfqpoint{0.666335in}{1.128021in}}{\pgfqpoint{0.662428in}{1.124114in}}%
\pgfpathcurveto{\pgfqpoint{0.658522in}{1.120208in}}{\pgfqpoint{0.656326in}{1.114908in}}{\pgfqpoint{0.656326in}{1.109383in}}%
\pgfpathcurveto{\pgfqpoint{0.656326in}{1.103858in}}{\pgfqpoint{0.658522in}{1.098558in}}{\pgfqpoint{0.662428in}{1.094652in}}%
\pgfpathcurveto{\pgfqpoint{0.666335in}{1.090745in}}{\pgfqpoint{0.671635in}{1.088550in}}{\pgfqpoint{0.677160in}{1.088550in}}%
\pgfpathclose%
\pgfusepath{stroke,fill}%
\end{pgfscope}%
\begin{pgfscope}%
\pgfpathrectangle{\pgfqpoint{0.403630in}{0.251348in}}{\pgfqpoint{1.395000in}{1.359000in}}%
\pgfusepath{clip}%
\pgfsetbuttcap%
\pgfsetroundjoin%
\definecolor{currentfill}{rgb}{0.686275,0.352941,0.313725}%
\pgfsetfillcolor{currentfill}%
\pgfsetlinewidth{0.501875pt}%
\definecolor{currentstroke}{rgb}{0.686275,0.352941,0.313725}%
\pgfsetstrokecolor{currentstroke}%
\pgfsetdash{}{0pt}%
\pgfpathmoveto{\pgfqpoint{1.486807in}{0.678185in}}%
\pgfpathcurveto{\pgfqpoint{1.492332in}{0.678185in}}{\pgfqpoint{1.497631in}{0.680380in}}{\pgfqpoint{1.501538in}{0.684287in}}%
\pgfpathcurveto{\pgfqpoint{1.505445in}{0.688194in}}{\pgfqpoint{1.507640in}{0.693493in}}{\pgfqpoint{1.507640in}{0.699018in}}%
\pgfpathcurveto{\pgfqpoint{1.507640in}{0.704543in}}{\pgfqpoint{1.505445in}{0.709843in}}{\pgfqpoint{1.501538in}{0.713750in}}%
\pgfpathcurveto{\pgfqpoint{1.497631in}{0.717656in}}{\pgfqpoint{1.492332in}{0.719852in}}{\pgfqpoint{1.486807in}{0.719852in}}%
\pgfpathcurveto{\pgfqpoint{1.481282in}{0.719852in}}{\pgfqpoint{1.475982in}{0.717656in}}{\pgfqpoint{1.472075in}{0.713750in}}%
\pgfpathcurveto{\pgfqpoint{1.468169in}{0.709843in}}{\pgfqpoint{1.465974in}{0.704543in}}{\pgfqpoint{1.465974in}{0.699018in}}%
\pgfpathcurveto{\pgfqpoint{1.465974in}{0.693493in}}{\pgfqpoint{1.468169in}{0.688194in}}{\pgfqpoint{1.472075in}{0.684287in}}%
\pgfpathcurveto{\pgfqpoint{1.475982in}{0.680380in}}{\pgfqpoint{1.481282in}{0.678185in}}{\pgfqpoint{1.486807in}{0.678185in}}%
\pgfpathclose%
\pgfusepath{stroke,fill}%
\end{pgfscope}%
\begin{pgfscope}%
\pgfpathrectangle{\pgfqpoint{0.403630in}{0.251348in}}{\pgfqpoint{1.395000in}{1.359000in}}%
\pgfusepath{clip}%
\pgfsetbuttcap%
\pgfsetroundjoin%
\definecolor{currentfill}{rgb}{0.686275,0.352941,0.313725}%
\pgfsetfillcolor{currentfill}%
\pgfsetlinewidth{0.501875pt}%
\definecolor{currentstroke}{rgb}{0.686275,0.352941,0.313725}%
\pgfsetstrokecolor{currentstroke}%
\pgfsetdash{}{0pt}%
\pgfpathmoveto{\pgfqpoint{1.574336in}{1.024597in}}%
\pgfpathcurveto{\pgfqpoint{1.579861in}{1.024597in}}{\pgfqpoint{1.585161in}{1.026792in}}{\pgfqpoint{1.589068in}{1.030699in}}%
\pgfpathcurveto{\pgfqpoint{1.592975in}{1.034605in}}{\pgfqpoint{1.595170in}{1.039905in}}{\pgfqpoint{1.595170in}{1.045430in}}%
\pgfpathcurveto{\pgfqpoint{1.595170in}{1.050955in}}{\pgfqpoint{1.592975in}{1.056255in}}{\pgfqpoint{1.589068in}{1.060161in}}%
\pgfpathcurveto{\pgfqpoint{1.585161in}{1.064068in}}{\pgfqpoint{1.579861in}{1.066263in}}{\pgfqpoint{1.574336in}{1.066263in}}%
\pgfpathcurveto{\pgfqpoint{1.568811in}{1.066263in}}{\pgfqpoint{1.563512in}{1.064068in}}{\pgfqpoint{1.559605in}{1.060161in}}%
\pgfpathcurveto{\pgfqpoint{1.555698in}{1.056255in}}{\pgfqpoint{1.553503in}{1.050955in}}{\pgfqpoint{1.553503in}{1.045430in}}%
\pgfpathcurveto{\pgfqpoint{1.553503in}{1.039905in}}{\pgfqpoint{1.555698in}{1.034605in}}{\pgfqpoint{1.559605in}{1.030699in}}%
\pgfpathcurveto{\pgfqpoint{1.563512in}{1.026792in}}{\pgfqpoint{1.568811in}{1.024597in}}{\pgfqpoint{1.574336in}{1.024597in}}%
\pgfpathclose%
\pgfusepath{stroke,fill}%
\end{pgfscope}%
\begin{pgfscope}%
\pgfpathrectangle{\pgfqpoint{0.403630in}{0.251348in}}{\pgfqpoint{1.395000in}{1.359000in}}%
\pgfusepath{clip}%
\pgfsetbuttcap%
\pgfsetroundjoin%
\definecolor{currentfill}{rgb}{0.686275,0.352941,0.313725}%
\pgfsetfillcolor{currentfill}%
\pgfsetlinewidth{0.501875pt}%
\definecolor{currentstroke}{rgb}{0.686275,0.352941,0.313725}%
\pgfsetstrokecolor{currentstroke}%
\pgfsetdash{}{0pt}%
\pgfpathmoveto{\pgfqpoint{1.536042in}{0.726150in}}%
\pgfpathcurveto{\pgfqpoint{1.541567in}{0.726150in}}{\pgfqpoint{1.546867in}{0.728345in}}{\pgfqpoint{1.550774in}{0.732252in}}%
\pgfpathcurveto{\pgfqpoint{1.554680in}{0.736158in}}{\pgfqpoint{1.556876in}{0.741458in}}{\pgfqpoint{1.556876in}{0.746983in}}%
\pgfpathcurveto{\pgfqpoint{1.556876in}{0.752508in}}{\pgfqpoint{1.554680in}{0.757808in}}{\pgfqpoint{1.550774in}{0.761714in}}%
\pgfpathcurveto{\pgfqpoint{1.546867in}{0.765621in}}{\pgfqpoint{1.541567in}{0.767816in}}{\pgfqpoint{1.536042in}{0.767816in}}%
\pgfpathcurveto{\pgfqpoint{1.530517in}{0.767816in}}{\pgfqpoint{1.525218in}{0.765621in}}{\pgfqpoint{1.521311in}{0.761714in}}%
\pgfpathcurveto{\pgfqpoint{1.517404in}{0.757808in}}{\pgfqpoint{1.515209in}{0.752508in}}{\pgfqpoint{1.515209in}{0.746983in}}%
\pgfpathcurveto{\pgfqpoint{1.515209in}{0.741458in}}{\pgfqpoint{1.517404in}{0.736158in}}{\pgfqpoint{1.521311in}{0.732252in}}%
\pgfpathcurveto{\pgfqpoint{1.525218in}{0.728345in}}{\pgfqpoint{1.530517in}{0.726150in}}{\pgfqpoint{1.536042in}{0.726150in}}%
\pgfpathclose%
\pgfusepath{stroke,fill}%
\end{pgfscope}%
\begin{pgfscope}%
\pgfpathrectangle{\pgfqpoint{0.403630in}{0.251348in}}{\pgfqpoint{1.395000in}{1.359000in}}%
\pgfusepath{clip}%
\pgfsetbuttcap%
\pgfsetroundjoin%
\definecolor{currentfill}{rgb}{0.686275,0.352941,0.313725}%
\pgfsetfillcolor{currentfill}%
\pgfsetlinewidth{0.501875pt}%
\definecolor{currentstroke}{rgb}{0.686275,0.352941,0.313725}%
\pgfsetstrokecolor{currentstroke}%
\pgfsetdash{}{0pt}%
\pgfpathmoveto{\pgfqpoint{1.486807in}{1.232444in}}%
\pgfpathcurveto{\pgfqpoint{1.492332in}{1.232444in}}{\pgfqpoint{1.497631in}{1.234639in}}{\pgfqpoint{1.501538in}{1.238546in}}%
\pgfpathcurveto{\pgfqpoint{1.505445in}{1.242452in}}{\pgfqpoint{1.507640in}{1.247752in}}{\pgfqpoint{1.507640in}{1.253277in}}%
\pgfpathcurveto{\pgfqpoint{1.507640in}{1.258802in}}{\pgfqpoint{1.505445in}{1.264102in}}{\pgfqpoint{1.501538in}{1.268008in}}%
\pgfpathcurveto{\pgfqpoint{1.497631in}{1.271915in}}{\pgfqpoint{1.492332in}{1.274110in}}{\pgfqpoint{1.486807in}{1.274110in}}%
\pgfpathcurveto{\pgfqpoint{1.481282in}{1.274110in}}{\pgfqpoint{1.475982in}{1.271915in}}{\pgfqpoint{1.472075in}{1.268008in}}%
\pgfpathcurveto{\pgfqpoint{1.468169in}{1.264102in}}{\pgfqpoint{1.465974in}{1.258802in}}{\pgfqpoint{1.465974in}{1.253277in}}%
\pgfpathcurveto{\pgfqpoint{1.465974in}{1.247752in}}{\pgfqpoint{1.468169in}{1.242452in}}{\pgfqpoint{1.472075in}{1.238546in}}%
\pgfpathcurveto{\pgfqpoint{1.475982in}{1.234639in}}{\pgfqpoint{1.481282in}{1.232444in}}{\pgfqpoint{1.486807in}{1.232444in}}%
\pgfpathclose%
\pgfusepath{stroke,fill}%
\end{pgfscope}%
\begin{pgfscope}%
\pgfpathrectangle{\pgfqpoint{0.403630in}{0.251348in}}{\pgfqpoint{1.395000in}{1.359000in}}%
\pgfusepath{clip}%
\pgfsetbuttcap%
\pgfsetroundjoin%
\definecolor{currentfill}{rgb}{0.686275,0.352941,0.313725}%
\pgfsetfillcolor{currentfill}%
\pgfsetlinewidth{0.501875pt}%
\definecolor{currentstroke}{rgb}{0.686275,0.352941,0.313725}%
\pgfsetstrokecolor{currentstroke}%
\pgfsetdash{}{0pt}%
\pgfpathmoveto{\pgfqpoint{1.470395in}{0.656867in}}%
\pgfpathcurveto{\pgfqpoint{1.475920in}{0.656867in}}{\pgfqpoint{1.481220in}{0.659062in}}{\pgfqpoint{1.485127in}{0.662969in}}%
\pgfpathcurveto{\pgfqpoint{1.489033in}{0.666876in}}{\pgfqpoint{1.491228in}{0.672176in}}{\pgfqpoint{1.491228in}{0.677701in}}%
\pgfpathcurveto{\pgfqpoint{1.491228in}{0.683226in}}{\pgfqpoint{1.489033in}{0.688525in}}{\pgfqpoint{1.485127in}{0.692432in}}%
\pgfpathcurveto{\pgfqpoint{1.481220in}{0.696339in}}{\pgfqpoint{1.475920in}{0.698534in}}{\pgfqpoint{1.470395in}{0.698534in}}%
\pgfpathcurveto{\pgfqpoint{1.464870in}{0.698534in}}{\pgfqpoint{1.459571in}{0.696339in}}{\pgfqpoint{1.455664in}{0.692432in}}%
\pgfpathcurveto{\pgfqpoint{1.451757in}{0.688525in}}{\pgfqpoint{1.449562in}{0.683226in}}{\pgfqpoint{1.449562in}{0.677701in}}%
\pgfpathcurveto{\pgfqpoint{1.449562in}{0.672176in}}{\pgfqpoint{1.451757in}{0.666876in}}{\pgfqpoint{1.455664in}{0.662969in}}%
\pgfpathcurveto{\pgfqpoint{1.459571in}{0.659062in}}{\pgfqpoint{1.464870in}{0.656867in}}{\pgfqpoint{1.470395in}{0.656867in}}%
\pgfpathclose%
\pgfusepath{stroke,fill}%
\end{pgfscope}%
\begin{pgfscope}%
\pgfpathrectangle{\pgfqpoint{0.403630in}{0.251348in}}{\pgfqpoint{1.395000in}{1.359000in}}%
\pgfusepath{clip}%
\pgfsetbuttcap%
\pgfsetroundjoin%
\definecolor{currentfill}{rgb}{0.686275,0.352941,0.313725}%
\pgfsetfillcolor{currentfill}%
\pgfsetlinewidth{0.501875pt}%
\definecolor{currentstroke}{rgb}{0.686275,0.352941,0.313725}%
\pgfsetstrokecolor{currentstroke}%
\pgfsetdash{}{0pt}%
\pgfpathmoveto{\pgfqpoint{1.278925in}{1.349691in}}%
\pgfpathcurveto{\pgfqpoint{1.284450in}{1.349691in}}{\pgfqpoint{1.289749in}{1.351886in}}{\pgfqpoint{1.293656in}{1.355793in}}%
\pgfpathcurveto{\pgfqpoint{1.297563in}{1.359700in}}{\pgfqpoint{1.299758in}{1.364999in}}{\pgfqpoint{1.299758in}{1.370524in}}%
\pgfpathcurveto{\pgfqpoint{1.299758in}{1.376049in}}{\pgfqpoint{1.297563in}{1.381349in}}{\pgfqpoint{1.293656in}{1.385256in}}%
\pgfpathcurveto{\pgfqpoint{1.289749in}{1.389162in}}{\pgfqpoint{1.284450in}{1.391357in}}{\pgfqpoint{1.278925in}{1.391357in}}%
\pgfpathcurveto{\pgfqpoint{1.273399in}{1.391357in}}{\pgfqpoint{1.268100in}{1.389162in}}{\pgfqpoint{1.264193in}{1.385256in}}%
\pgfpathcurveto{\pgfqpoint{1.260286in}{1.381349in}}{\pgfqpoint{1.258091in}{1.376049in}}{\pgfqpoint{1.258091in}{1.370524in}}%
\pgfpathcurveto{\pgfqpoint{1.258091in}{1.364999in}}{\pgfqpoint{1.260286in}{1.359700in}}{\pgfqpoint{1.264193in}{1.355793in}}%
\pgfpathcurveto{\pgfqpoint{1.268100in}{1.351886in}}{\pgfqpoint{1.273399in}{1.349691in}}{\pgfqpoint{1.278925in}{1.349691in}}%
\pgfpathclose%
\pgfusepath{stroke,fill}%
\end{pgfscope}%
\begin{pgfscope}%
\pgfpathrectangle{\pgfqpoint{0.403630in}{0.251348in}}{\pgfqpoint{1.395000in}{1.359000in}}%
\pgfusepath{clip}%
\pgfsetbuttcap%
\pgfsetroundjoin%
\definecolor{currentfill}{rgb}{0.686275,0.352941,0.313725}%
\pgfsetfillcolor{currentfill}%
\pgfsetlinewidth{0.501875pt}%
\definecolor{currentstroke}{rgb}{0.686275,0.352941,0.313725}%
\pgfsetstrokecolor{currentstroke}%
\pgfsetdash{}{0pt}%
\pgfpathmoveto{\pgfqpoint{0.677160in}{1.168491in}}%
\pgfpathcurveto{\pgfqpoint{0.682685in}{1.168491in}}{\pgfqpoint{0.687984in}{1.170686in}}{\pgfqpoint{0.691891in}{1.174593in}}%
\pgfpathcurveto{\pgfqpoint{0.695798in}{1.178500in}}{\pgfqpoint{0.697993in}{1.183799in}}{\pgfqpoint{0.697993in}{1.189324in}}%
\pgfpathcurveto{\pgfqpoint{0.697993in}{1.194849in}}{\pgfqpoint{0.695798in}{1.200149in}}{\pgfqpoint{0.691891in}{1.204056in}}%
\pgfpathcurveto{\pgfqpoint{0.687984in}{1.207962in}}{\pgfqpoint{0.682685in}{1.210157in}}{\pgfqpoint{0.677160in}{1.210157in}}%
\pgfpathcurveto{\pgfqpoint{0.671635in}{1.210157in}}{\pgfqpoint{0.666335in}{1.207962in}}{\pgfqpoint{0.662428in}{1.204056in}}%
\pgfpathcurveto{\pgfqpoint{0.658522in}{1.200149in}}{\pgfqpoint{0.656326in}{1.194849in}}{\pgfqpoint{0.656326in}{1.189324in}}%
\pgfpathcurveto{\pgfqpoint{0.656326in}{1.183799in}}{\pgfqpoint{0.658522in}{1.178500in}}{\pgfqpoint{0.662428in}{1.174593in}}%
\pgfpathcurveto{\pgfqpoint{0.666335in}{1.170686in}}{\pgfqpoint{0.671635in}{1.168491in}}{\pgfqpoint{0.677160in}{1.168491in}}%
\pgfpathclose%
\pgfusepath{stroke,fill}%
\end{pgfscope}%
\begin{pgfscope}%
\pgfpathrectangle{\pgfqpoint{0.403630in}{0.251348in}}{\pgfqpoint{1.395000in}{1.359000in}}%
\pgfusepath{clip}%
\pgfsetbuttcap%
\pgfsetroundjoin%
\definecolor{currentfill}{rgb}{0.686275,0.352941,0.313725}%
\pgfsetfillcolor{currentfill}%
\pgfsetlinewidth{0.501875pt}%
\definecolor{currentstroke}{rgb}{0.686275,0.352941,0.313725}%
\pgfsetstrokecolor{currentstroke}%
\pgfsetdash{}{0pt}%
\pgfpathmoveto{\pgfqpoint{1.525101in}{0.747467in}}%
\pgfpathcurveto{\pgfqpoint{1.530626in}{0.747467in}}{\pgfqpoint{1.535926in}{0.749662in}}{\pgfqpoint{1.539832in}{0.753569in}}%
\pgfpathcurveto{\pgfqpoint{1.543739in}{0.757476in}}{\pgfqpoint{1.545934in}{0.762776in}}{\pgfqpoint{1.545934in}{0.768301in}}%
\pgfpathcurveto{\pgfqpoint{1.545934in}{0.773826in}}{\pgfqpoint{1.543739in}{0.779125in}}{\pgfqpoint{1.539832in}{0.783032in}}%
\pgfpathcurveto{\pgfqpoint{1.535926in}{0.786939in}}{\pgfqpoint{1.530626in}{0.789134in}}{\pgfqpoint{1.525101in}{0.789134in}}%
\pgfpathcurveto{\pgfqpoint{1.519576in}{0.789134in}}{\pgfqpoint{1.514276in}{0.786939in}}{\pgfqpoint{1.510370in}{0.783032in}}%
\pgfpathcurveto{\pgfqpoint{1.506463in}{0.779125in}}{\pgfqpoint{1.504268in}{0.773826in}}{\pgfqpoint{1.504268in}{0.768301in}}%
\pgfpathcurveto{\pgfqpoint{1.504268in}{0.762776in}}{\pgfqpoint{1.506463in}{0.757476in}}{\pgfqpoint{1.510370in}{0.753569in}}%
\pgfpathcurveto{\pgfqpoint{1.514276in}{0.749662in}}{\pgfqpoint{1.519576in}{0.747467in}}{\pgfqpoint{1.525101in}{0.747467in}}%
\pgfpathclose%
\pgfusepath{stroke,fill}%
\end{pgfscope}%
\begin{pgfscope}%
\pgfpathrectangle{\pgfqpoint{0.403630in}{0.251348in}}{\pgfqpoint{1.395000in}{1.359000in}}%
\pgfusepath{clip}%
\pgfsetbuttcap%
\pgfsetroundjoin%
\definecolor{currentfill}{rgb}{0.686275,0.352941,0.313725}%
\pgfsetfillcolor{currentfill}%
\pgfsetlinewidth{0.501875pt}%
\definecolor{currentstroke}{rgb}{0.686275,0.352941,0.313725}%
\pgfsetstrokecolor{currentstroke}%
\pgfsetdash{}{0pt}%
\pgfpathmoveto{\pgfqpoint{1.607160in}{1.013938in}}%
\pgfpathcurveto{\pgfqpoint{1.612685in}{1.013938in}}{\pgfqpoint{1.617984in}{1.016133in}}{\pgfqpoint{1.621891in}{1.020040in}}%
\pgfpathcurveto{\pgfqpoint{1.625798in}{1.023947in}}{\pgfqpoint{1.627993in}{1.029246in}}{\pgfqpoint{1.627993in}{1.034771in}}%
\pgfpathcurveto{\pgfqpoint{1.627993in}{1.040296in}}{\pgfqpoint{1.625798in}{1.045596in}}{\pgfqpoint{1.621891in}{1.049503in}}%
\pgfpathcurveto{\pgfqpoint{1.617984in}{1.053409in}}{\pgfqpoint{1.612685in}{1.055605in}}{\pgfqpoint{1.607160in}{1.055605in}}%
\pgfpathcurveto{\pgfqpoint{1.601635in}{1.055605in}}{\pgfqpoint{1.596335in}{1.053409in}}{\pgfqpoint{1.592428in}{1.049503in}}%
\pgfpathcurveto{\pgfqpoint{1.588522in}{1.045596in}}{\pgfqpoint{1.586326in}{1.040296in}}{\pgfqpoint{1.586326in}{1.034771in}}%
\pgfpathcurveto{\pgfqpoint{1.586326in}{1.029246in}}{\pgfqpoint{1.588522in}{1.023947in}}{\pgfqpoint{1.592428in}{1.020040in}}%
\pgfpathcurveto{\pgfqpoint{1.596335in}{1.016133in}}{\pgfqpoint{1.601635in}{1.013938in}}{\pgfqpoint{1.607160in}{1.013938in}}%
\pgfpathclose%
\pgfusepath{stroke,fill}%
\end{pgfscope}%
\begin{pgfscope}%
\pgfpathrectangle{\pgfqpoint{0.403630in}{0.251348in}}{\pgfqpoint{1.395000in}{1.359000in}}%
\pgfusepath{clip}%
\pgfsetbuttcap%
\pgfsetroundjoin%
\definecolor{currentfill}{rgb}{0.686275,0.352941,0.313725}%
\pgfsetfillcolor{currentfill}%
\pgfsetlinewidth{0.501875pt}%
\definecolor{currentstroke}{rgb}{0.686275,0.352941,0.313725}%
\pgfsetstrokecolor{currentstroke}%
\pgfsetdash{}{0pt}%
\pgfpathmoveto{\pgfqpoint{1.180454in}{1.365679in}}%
\pgfpathcurveto{\pgfqpoint{1.185979in}{1.365679in}}{\pgfqpoint{1.191279in}{1.367874in}}{\pgfqpoint{1.195185in}{1.371781in}}%
\pgfpathcurveto{\pgfqpoint{1.199092in}{1.375688in}}{\pgfqpoint{1.201287in}{1.380987in}}{\pgfqpoint{1.201287in}{1.386512in}}%
\pgfpathcurveto{\pgfqpoint{1.201287in}{1.392037in}}{\pgfqpoint{1.199092in}{1.397337in}}{\pgfqpoint{1.195185in}{1.401244in}}%
\pgfpathcurveto{\pgfqpoint{1.191279in}{1.405151in}}{\pgfqpoint{1.185979in}{1.407346in}}{\pgfqpoint{1.180454in}{1.407346in}}%
\pgfpathcurveto{\pgfqpoint{1.174929in}{1.407346in}}{\pgfqpoint{1.169629in}{1.405151in}}{\pgfqpoint{1.165723in}{1.401244in}}%
\pgfpathcurveto{\pgfqpoint{1.161816in}{1.397337in}}{\pgfqpoint{1.159621in}{1.392037in}}{\pgfqpoint{1.159621in}{1.386512in}}%
\pgfpathcurveto{\pgfqpoint{1.159621in}{1.380987in}}{\pgfqpoint{1.161816in}{1.375688in}}{\pgfqpoint{1.165723in}{1.371781in}}%
\pgfpathcurveto{\pgfqpoint{1.169629in}{1.367874in}}{\pgfqpoint{1.174929in}{1.365679in}}{\pgfqpoint{1.180454in}{1.365679in}}%
\pgfpathclose%
\pgfusepath{stroke,fill}%
\end{pgfscope}%
\begin{pgfscope}%
\pgfpathrectangle{\pgfqpoint{0.403630in}{0.251348in}}{\pgfqpoint{1.395000in}{1.359000in}}%
\pgfusepath{clip}%
\pgfsetbuttcap%
\pgfsetroundjoin%
\definecolor{currentfill}{rgb}{0.686275,0.352941,0.313725}%
\pgfsetfillcolor{currentfill}%
\pgfsetlinewidth{0.501875pt}%
\definecolor{currentstroke}{rgb}{0.686275,0.352941,0.313725}%
\pgfsetstrokecolor{currentstroke}%
\pgfsetdash{}{0pt}%
\pgfpathmoveto{\pgfqpoint{0.797513in}{1.264420in}}%
\pgfpathcurveto{\pgfqpoint{0.803038in}{1.264420in}}{\pgfqpoint{0.808337in}{1.266615in}}{\pgfqpoint{0.812244in}{1.270522in}}%
\pgfpathcurveto{\pgfqpoint{0.816151in}{1.274429in}}{\pgfqpoint{0.818346in}{1.279728in}}{\pgfqpoint{0.818346in}{1.285254in}}%
\pgfpathcurveto{\pgfqpoint{0.818346in}{1.290779in}}{\pgfqpoint{0.816151in}{1.296078in}}{\pgfqpoint{0.812244in}{1.299985in}}%
\pgfpathcurveto{\pgfqpoint{0.808337in}{1.303892in}}{\pgfqpoint{0.803038in}{1.306087in}}{\pgfqpoint{0.797513in}{1.306087in}}%
\pgfpathcurveto{\pgfqpoint{0.791988in}{1.306087in}}{\pgfqpoint{0.786688in}{1.303892in}}{\pgfqpoint{0.782781in}{1.299985in}}%
\pgfpathcurveto{\pgfqpoint{0.778875in}{1.296078in}}{\pgfqpoint{0.776679in}{1.290779in}}{\pgfqpoint{0.776679in}{1.285254in}}%
\pgfpathcurveto{\pgfqpoint{0.776679in}{1.279728in}}{\pgfqpoint{0.778875in}{1.274429in}}{\pgfqpoint{0.782781in}{1.270522in}}%
\pgfpathcurveto{\pgfqpoint{0.786688in}{1.266615in}}{\pgfqpoint{0.791988in}{1.264420in}}{\pgfqpoint{0.797513in}{1.264420in}}%
\pgfpathclose%
\pgfusepath{stroke,fill}%
\end{pgfscope}%
\begin{pgfscope}%
\pgfpathrectangle{\pgfqpoint{0.403630in}{0.251348in}}{\pgfqpoint{1.395000in}{1.359000in}}%
\pgfusepath{clip}%
\pgfsetbuttcap%
\pgfsetroundjoin%
\definecolor{currentfill}{rgb}{0.686275,0.352941,0.313725}%
\pgfsetfillcolor{currentfill}%
\pgfsetlinewidth{0.501875pt}%
\definecolor{currentstroke}{rgb}{0.686275,0.352941,0.313725}%
\pgfsetstrokecolor{currentstroke}%
\pgfsetdash{}{0pt}%
\pgfpathmoveto{\pgfqpoint{0.660748in}{0.987291in}}%
\pgfpathcurveto{\pgfqpoint{0.666273in}{0.987291in}}{\pgfqpoint{0.671573in}{0.989486in}}{\pgfqpoint{0.675479in}{0.993393in}}%
\pgfpathcurveto{\pgfqpoint{0.679386in}{0.997300in}}{\pgfqpoint{0.681581in}{1.002599in}}{\pgfqpoint{0.681581in}{1.008124in}}%
\pgfpathcurveto{\pgfqpoint{0.681581in}{1.013649in}}{\pgfqpoint{0.679386in}{1.018949in}}{\pgfqpoint{0.675479in}{1.022856in}}%
\pgfpathcurveto{\pgfqpoint{0.671573in}{1.026762in}}{\pgfqpoint{0.666273in}{1.028957in}}{\pgfqpoint{0.660748in}{1.028957in}}%
\pgfpathcurveto{\pgfqpoint{0.655223in}{1.028957in}}{\pgfqpoint{0.649923in}{1.026762in}}{\pgfqpoint{0.646017in}{1.022856in}}%
\pgfpathcurveto{\pgfqpoint{0.642110in}{1.018949in}}{\pgfqpoint{0.639915in}{1.013649in}}{\pgfqpoint{0.639915in}{1.008124in}}%
\pgfpathcurveto{\pgfqpoint{0.639915in}{1.002599in}}{\pgfqpoint{0.642110in}{0.997300in}}{\pgfqpoint{0.646017in}{0.993393in}}%
\pgfpathcurveto{\pgfqpoint{0.649923in}{0.989486in}}{\pgfqpoint{0.655223in}{0.987291in}}{\pgfqpoint{0.660748in}{0.987291in}}%
\pgfpathclose%
\pgfusepath{stroke,fill}%
\end{pgfscope}%
\begin{pgfscope}%
\pgfpathrectangle{\pgfqpoint{0.403630in}{0.251348in}}{\pgfqpoint{1.395000in}{1.359000in}}%
\pgfusepath{clip}%
\pgfsetbuttcap%
\pgfsetroundjoin%
\definecolor{currentfill}{rgb}{0.686275,0.352941,0.313725}%
\pgfsetfillcolor{currentfill}%
\pgfsetlinewidth{0.501875pt}%
\definecolor{currentstroke}{rgb}{0.686275,0.352941,0.313725}%
\pgfsetstrokecolor{currentstroke}%
\pgfsetdash{}{0pt}%
\pgfpathmoveto{\pgfqpoint{1.514160in}{1.211126in}}%
\pgfpathcurveto{\pgfqpoint{1.519685in}{1.211126in}}{\pgfqpoint{1.524984in}{1.213321in}}{\pgfqpoint{1.528891in}{1.217228in}}%
\pgfpathcurveto{\pgfqpoint{1.532798in}{1.221135in}}{\pgfqpoint{1.534993in}{1.226434in}}{\pgfqpoint{1.534993in}{1.231959in}}%
\pgfpathcurveto{\pgfqpoint{1.534993in}{1.237484in}}{\pgfqpoint{1.532798in}{1.242784in}}{\pgfqpoint{1.528891in}{1.246691in}}%
\pgfpathcurveto{\pgfqpoint{1.524984in}{1.250598in}}{\pgfqpoint{1.519685in}{1.252793in}}{\pgfqpoint{1.514160in}{1.252793in}}%
\pgfpathcurveto{\pgfqpoint{1.508635in}{1.252793in}}{\pgfqpoint{1.503335in}{1.250598in}}{\pgfqpoint{1.499428in}{1.246691in}}%
\pgfpathcurveto{\pgfqpoint{1.495522in}{1.242784in}}{\pgfqpoint{1.493326in}{1.237484in}}{\pgfqpoint{1.493326in}{1.231959in}}%
\pgfpathcurveto{\pgfqpoint{1.493326in}{1.226434in}}{\pgfqpoint{1.495522in}{1.221135in}}{\pgfqpoint{1.499428in}{1.217228in}}%
\pgfpathcurveto{\pgfqpoint{1.503335in}{1.213321in}}{\pgfqpoint{1.508635in}{1.211126in}}{\pgfqpoint{1.514160in}{1.211126in}}%
\pgfpathclose%
\pgfusepath{stroke,fill}%
\end{pgfscope}%
\begin{pgfscope}%
\pgfpathrectangle{\pgfqpoint{0.403630in}{0.251348in}}{\pgfqpoint{1.395000in}{1.359000in}}%
\pgfusepath{clip}%
\pgfsetbuttcap%
\pgfsetroundjoin%
\definecolor{currentfill}{rgb}{0.686275,0.352941,0.313725}%
\pgfsetfillcolor{currentfill}%
\pgfsetlinewidth{0.501875pt}%
\definecolor{currentstroke}{rgb}{0.686275,0.352941,0.313725}%
\pgfsetstrokecolor{currentstroke}%
\pgfsetdash{}{0pt}%
\pgfpathmoveto{\pgfqpoint{1.377395in}{0.566267in}}%
\pgfpathcurveto{\pgfqpoint{1.382920in}{0.566267in}}{\pgfqpoint{1.388220in}{0.568462in}}{\pgfqpoint{1.392127in}{0.572369in}}%
\pgfpathcurveto{\pgfqpoint{1.396033in}{0.576276in}}{\pgfqpoint{1.398228in}{0.581576in}}{\pgfqpoint{1.398228in}{0.587101in}}%
\pgfpathcurveto{\pgfqpoint{1.398228in}{0.592626in}}{\pgfqpoint{1.396033in}{0.597925in}}{\pgfqpoint{1.392127in}{0.601832in}}%
\pgfpathcurveto{\pgfqpoint{1.388220in}{0.605739in}}{\pgfqpoint{1.382920in}{0.607934in}}{\pgfqpoint{1.377395in}{0.607934in}}%
\pgfpathcurveto{\pgfqpoint{1.371870in}{0.607934in}}{\pgfqpoint{1.366571in}{0.605739in}}{\pgfqpoint{1.362664in}{0.601832in}}%
\pgfpathcurveto{\pgfqpoint{1.358757in}{0.597925in}}{\pgfqpoint{1.356562in}{0.592626in}}{\pgfqpoint{1.356562in}{0.587101in}}%
\pgfpathcurveto{\pgfqpoint{1.356562in}{0.581576in}}{\pgfqpoint{1.358757in}{0.576276in}}{\pgfqpoint{1.362664in}{0.572369in}}%
\pgfpathcurveto{\pgfqpoint{1.366571in}{0.568462in}}{\pgfqpoint{1.371870in}{0.566267in}}{\pgfqpoint{1.377395in}{0.566267in}}%
\pgfpathclose%
\pgfusepath{stroke,fill}%
\end{pgfscope}%
\begin{pgfscope}%
\pgfpathrectangle{\pgfqpoint{0.403630in}{0.251348in}}{\pgfqpoint{1.395000in}{1.359000in}}%
\pgfusepath{clip}%
\pgfsetbuttcap%
\pgfsetroundjoin%
\definecolor{currentfill}{rgb}{0.686275,0.352941,0.313725}%
\pgfsetfillcolor{currentfill}%
\pgfsetlinewidth{0.501875pt}%
\definecolor{currentstroke}{rgb}{0.686275,0.352941,0.313725}%
\pgfsetstrokecolor{currentstroke}%
\pgfsetdash{}{0pt}%
\pgfpathmoveto{\pgfqpoint{0.595101in}{0.971303in}}%
\pgfpathcurveto{\pgfqpoint{0.600626in}{0.971303in}}{\pgfqpoint{0.605926in}{0.973498in}}{\pgfqpoint{0.609832in}{0.977404in}}%
\pgfpathcurveto{\pgfqpoint{0.613739in}{0.981311in}}{\pgfqpoint{0.615934in}{0.986611in}}{\pgfqpoint{0.615934in}{0.992136in}}%
\pgfpathcurveto{\pgfqpoint{0.615934in}{0.997661in}}{\pgfqpoint{0.613739in}{1.002960in}}{\pgfqpoint{0.609832in}{1.006867in}}%
\pgfpathcurveto{\pgfqpoint{0.605926in}{1.010774in}}{\pgfqpoint{0.600626in}{1.012969in}}{\pgfqpoint{0.595101in}{1.012969in}}%
\pgfpathcurveto{\pgfqpoint{0.589576in}{1.012969in}}{\pgfqpoint{0.584276in}{1.010774in}}{\pgfqpoint{0.580370in}{1.006867in}}%
\pgfpathcurveto{\pgfqpoint{0.576463in}{1.002960in}}{\pgfqpoint{0.574268in}{0.997661in}}{\pgfqpoint{0.574268in}{0.992136in}}%
\pgfpathcurveto{\pgfqpoint{0.574268in}{0.986611in}}{\pgfqpoint{0.576463in}{0.981311in}}{\pgfqpoint{0.580370in}{0.977404in}}%
\pgfpathcurveto{\pgfqpoint{0.584276in}{0.973498in}}{\pgfqpoint{0.589576in}{0.971303in}}{\pgfqpoint{0.595101in}{0.971303in}}%
\pgfpathclose%
\pgfusepath{stroke,fill}%
\end{pgfscope}%
\begin{pgfscope}%
\pgfpathrectangle{\pgfqpoint{0.403630in}{0.251348in}}{\pgfqpoint{1.395000in}{1.359000in}}%
\pgfusepath{clip}%
\pgfsetbuttcap%
\pgfsetroundjoin%
\definecolor{currentfill}{rgb}{0.686275,0.352941,0.313725}%
\pgfsetfillcolor{currentfill}%
\pgfsetlinewidth{0.501875pt}%
\definecolor{currentstroke}{rgb}{0.686275,0.352941,0.313725}%
\pgfsetstrokecolor{currentstroke}%
\pgfsetdash{}{0pt}%
\pgfpathmoveto{\pgfqpoint{1.295336in}{1.333703in}}%
\pgfpathcurveto{\pgfqpoint{1.300861in}{1.333703in}}{\pgfqpoint{1.306161in}{1.335898in}}{\pgfqpoint{1.310068in}{1.339804in}}%
\pgfpathcurveto{\pgfqpoint{1.313975in}{1.343711in}}{\pgfqpoint{1.316170in}{1.349011in}}{\pgfqpoint{1.316170in}{1.354536in}}%
\pgfpathcurveto{\pgfqpoint{1.316170in}{1.360061in}}{\pgfqpoint{1.313975in}{1.365360in}}{\pgfqpoint{1.310068in}{1.369267in}}%
\pgfpathcurveto{\pgfqpoint{1.306161in}{1.373174in}}{\pgfqpoint{1.300861in}{1.375369in}}{\pgfqpoint{1.295336in}{1.375369in}}%
\pgfpathcurveto{\pgfqpoint{1.289811in}{1.375369in}}{\pgfqpoint{1.284512in}{1.373174in}}{\pgfqpoint{1.280605in}{1.369267in}}%
\pgfpathcurveto{\pgfqpoint{1.276698in}{1.365360in}}{\pgfqpoint{1.274503in}{1.360061in}}{\pgfqpoint{1.274503in}{1.354536in}}%
\pgfpathcurveto{\pgfqpoint{1.274503in}{1.349011in}}{\pgfqpoint{1.276698in}{1.343711in}}{\pgfqpoint{1.280605in}{1.339804in}}%
\pgfpathcurveto{\pgfqpoint{1.284512in}{1.335898in}}{\pgfqpoint{1.289811in}{1.333703in}}{\pgfqpoint{1.295336in}{1.333703in}}%
\pgfpathclose%
\pgfusepath{stroke,fill}%
\end{pgfscope}%
\begin{pgfscope}%
\pgfpathrectangle{\pgfqpoint{0.403630in}{0.251348in}}{\pgfqpoint{1.395000in}{1.359000in}}%
\pgfusepath{clip}%
\pgfsetbuttcap%
\pgfsetroundjoin%
\definecolor{currentfill}{rgb}{0.686275,0.352941,0.313725}%
\pgfsetfillcolor{currentfill}%
\pgfsetlinewidth{0.501875pt}%
\definecolor{currentstroke}{rgb}{0.686275,0.352941,0.313725}%
\pgfsetstrokecolor{currentstroke}%
\pgfsetdash{}{0pt}%
\pgfpathmoveto{\pgfqpoint{1.092925in}{1.376338in}}%
\pgfpathcurveto{\pgfqpoint{1.098450in}{1.376338in}}{\pgfqpoint{1.103749in}{1.378533in}}{\pgfqpoint{1.107656in}{1.382440in}}%
\pgfpathcurveto{\pgfqpoint{1.111563in}{1.386347in}}{\pgfqpoint{1.113758in}{1.391646in}}{\pgfqpoint{1.113758in}{1.397171in}}%
\pgfpathcurveto{\pgfqpoint{1.113758in}{1.402696in}}{\pgfqpoint{1.111563in}{1.407996in}}{\pgfqpoint{1.107656in}{1.411903in}}%
\pgfpathcurveto{\pgfqpoint{1.103749in}{1.415809in}}{\pgfqpoint{1.098450in}{1.418005in}}{\pgfqpoint{1.092925in}{1.418005in}}%
\pgfpathcurveto{\pgfqpoint{1.087399in}{1.418005in}}{\pgfqpoint{1.082100in}{1.415809in}}{\pgfqpoint{1.078193in}{1.411903in}}%
\pgfpathcurveto{\pgfqpoint{1.074286in}{1.407996in}}{\pgfqpoint{1.072091in}{1.402696in}}{\pgfqpoint{1.072091in}{1.397171in}}%
\pgfpathcurveto{\pgfqpoint{1.072091in}{1.391646in}}{\pgfqpoint{1.074286in}{1.386347in}}{\pgfqpoint{1.078193in}{1.382440in}}%
\pgfpathcurveto{\pgfqpoint{1.082100in}{1.378533in}}{\pgfqpoint{1.087399in}{1.376338in}}{\pgfqpoint{1.092925in}{1.376338in}}%
\pgfpathclose%
\pgfusepath{stroke,fill}%
\end{pgfscope}%
\begin{pgfscope}%
\pgfpathrectangle{\pgfqpoint{0.403630in}{0.251348in}}{\pgfqpoint{1.395000in}{1.359000in}}%
\pgfusepath{clip}%
\pgfsetbuttcap%
\pgfsetroundjoin%
\definecolor{currentfill}{rgb}{0.686275,0.352941,0.313725}%
\pgfsetfillcolor{currentfill}%
\pgfsetlinewidth{0.501875pt}%
\definecolor{currentstroke}{rgb}{0.686275,0.352941,0.313725}%
\pgfsetstrokecolor{currentstroke}%
\pgfsetdash{}{0pt}%
\pgfpathmoveto{\pgfqpoint{1.415689in}{0.544950in}}%
\pgfpathcurveto{\pgfqpoint{1.421214in}{0.544950in}}{\pgfqpoint{1.426514in}{0.547145in}}{\pgfqpoint{1.430421in}{0.551052in}}%
\pgfpathcurveto{\pgfqpoint{1.434327in}{0.554958in}}{\pgfqpoint{1.436523in}{0.560258in}}{\pgfqpoint{1.436523in}{0.565783in}}%
\pgfpathcurveto{\pgfqpoint{1.436523in}{0.571308in}}{\pgfqpoint{1.434327in}{0.576608in}}{\pgfqpoint{1.430421in}{0.580514in}}%
\pgfpathcurveto{\pgfqpoint{1.426514in}{0.584421in}}{\pgfqpoint{1.421214in}{0.586616in}}{\pgfqpoint{1.415689in}{0.586616in}}%
\pgfpathcurveto{\pgfqpoint{1.410164in}{0.586616in}}{\pgfqpoint{1.404865in}{0.584421in}}{\pgfqpoint{1.400958in}{0.580514in}}%
\pgfpathcurveto{\pgfqpoint{1.397051in}{0.576608in}}{\pgfqpoint{1.394856in}{0.571308in}}{\pgfqpoint{1.394856in}{0.565783in}}%
\pgfpathcurveto{\pgfqpoint{1.394856in}{0.560258in}}{\pgfqpoint{1.397051in}{0.554958in}}{\pgfqpoint{1.400958in}{0.551052in}}%
\pgfpathcurveto{\pgfqpoint{1.404865in}{0.547145in}}{\pgfqpoint{1.410164in}{0.544950in}}{\pgfqpoint{1.415689in}{0.544950in}}%
\pgfpathclose%
\pgfusepath{stroke,fill}%
\end{pgfscope}%
\begin{pgfscope}%
\pgfpathrectangle{\pgfqpoint{0.403630in}{0.251348in}}{\pgfqpoint{1.395000in}{1.359000in}}%
\pgfusepath{clip}%
\pgfsetbuttcap%
\pgfsetroundjoin%
\definecolor{currentfill}{rgb}{0.686275,0.352941,0.313725}%
\pgfsetfillcolor{currentfill}%
\pgfsetlinewidth{0.501875pt}%
\definecolor{currentstroke}{rgb}{0.686275,0.352941,0.313725}%
\pgfsetstrokecolor{currentstroke}%
\pgfsetdash{}{0pt}%
\pgfpathmoveto{\pgfqpoint{0.693572in}{1.131185in}}%
\pgfpathcurveto{\pgfqpoint{0.699097in}{1.131185in}}{\pgfqpoint{0.704396in}{1.133380in}}{\pgfqpoint{0.708303in}{1.137287in}}%
\pgfpathcurveto{\pgfqpoint{0.712210in}{1.141194in}}{\pgfqpoint{0.714405in}{1.146493in}}{\pgfqpoint{0.714405in}{1.152018in}}%
\pgfpathcurveto{\pgfqpoint{0.714405in}{1.157543in}}{\pgfqpoint{0.712210in}{1.162843in}}{\pgfqpoint{0.708303in}{1.166750in}}%
\pgfpathcurveto{\pgfqpoint{0.704396in}{1.170656in}}{\pgfqpoint{0.699097in}{1.172852in}}{\pgfqpoint{0.693572in}{1.172852in}}%
\pgfpathcurveto{\pgfqpoint{0.688047in}{1.172852in}}{\pgfqpoint{0.682747in}{1.170656in}}{\pgfqpoint{0.678840in}{1.166750in}}%
\pgfpathcurveto{\pgfqpoint{0.674933in}{1.162843in}}{\pgfqpoint{0.672738in}{1.157543in}}{\pgfqpoint{0.672738in}{1.152018in}}%
\pgfpathcurveto{\pgfqpoint{0.672738in}{1.146493in}}{\pgfqpoint{0.674933in}{1.141194in}}{\pgfqpoint{0.678840in}{1.137287in}}%
\pgfpathcurveto{\pgfqpoint{0.682747in}{1.133380in}}{\pgfqpoint{0.688047in}{1.131185in}}{\pgfqpoint{0.693572in}{1.131185in}}%
\pgfpathclose%
\pgfusepath{stroke,fill}%
\end{pgfscope}%
\begin{pgfscope}%
\pgfpathrectangle{\pgfqpoint{0.403630in}{0.251348in}}{\pgfqpoint{1.395000in}{1.359000in}}%
\pgfusepath{clip}%
\pgfsetbuttcap%
\pgfsetroundjoin%
\definecolor{currentfill}{rgb}{0.686275,0.352941,0.313725}%
\pgfsetfillcolor{currentfill}%
\pgfsetlinewidth{0.501875pt}%
\definecolor{currentstroke}{rgb}{0.686275,0.352941,0.313725}%
\pgfsetstrokecolor{currentstroke}%
\pgfsetdash{}{0pt}%
\pgfpathmoveto{\pgfqpoint{0.682630in}{1.152503in}}%
\pgfpathcurveto{\pgfqpoint{0.688155in}{1.152503in}}{\pgfqpoint{0.693455in}{1.154698in}}{\pgfqpoint{0.697362in}{1.158604in}}%
\pgfpathcurveto{\pgfqpoint{0.701269in}{1.162511in}}{\pgfqpoint{0.703464in}{1.167811in}}{\pgfqpoint{0.703464in}{1.173336in}}%
\pgfpathcurveto{\pgfqpoint{0.703464in}{1.178861in}}{\pgfqpoint{0.701269in}{1.184160in}}{\pgfqpoint{0.697362in}{1.188067in}}%
\pgfpathcurveto{\pgfqpoint{0.693455in}{1.191974in}}{\pgfqpoint{0.688155in}{1.194169in}}{\pgfqpoint{0.682630in}{1.194169in}}%
\pgfpathcurveto{\pgfqpoint{0.677105in}{1.194169in}}{\pgfqpoint{0.671806in}{1.191974in}}{\pgfqpoint{0.667899in}{1.188067in}}%
\pgfpathcurveto{\pgfqpoint{0.663992in}{1.184160in}}{\pgfqpoint{0.661797in}{1.178861in}}{\pgfqpoint{0.661797in}{1.173336in}}%
\pgfpathcurveto{\pgfqpoint{0.661797in}{1.167811in}}{\pgfqpoint{0.663992in}{1.162511in}}{\pgfqpoint{0.667899in}{1.158604in}}%
\pgfpathcurveto{\pgfqpoint{0.671806in}{1.154698in}}{\pgfqpoint{0.677105in}{1.152503in}}{\pgfqpoint{0.682630in}{1.152503in}}%
\pgfpathclose%
\pgfusepath{stroke,fill}%
\end{pgfscope}%
\begin{pgfscope}%
\pgfpathrectangle{\pgfqpoint{0.403630in}{0.251348in}}{\pgfqpoint{1.395000in}{1.359000in}}%
\pgfusepath{clip}%
\pgfsetbuttcap%
\pgfsetroundjoin%
\definecolor{currentfill}{rgb}{0.686275,0.352941,0.313725}%
\pgfsetfillcolor{currentfill}%
\pgfsetlinewidth{0.501875pt}%
\definecolor{currentstroke}{rgb}{0.686275,0.352941,0.313725}%
\pgfsetstrokecolor{currentstroke}%
\pgfsetdash{}{0pt}%
\pgfpathmoveto{\pgfqpoint{1.169513in}{0.422373in}}%
\pgfpathcurveto{\pgfqpoint{1.175038in}{0.422373in}}{\pgfqpoint{1.180337in}{0.424568in}}{\pgfqpoint{1.184244in}{0.428475in}}%
\pgfpathcurveto{\pgfqpoint{1.188151in}{0.432382in}}{\pgfqpoint{1.190346in}{0.437681in}}{\pgfqpoint{1.190346in}{0.443206in}}%
\pgfpathcurveto{\pgfqpoint{1.190346in}{0.448732in}}{\pgfqpoint{1.188151in}{0.454031in}}{\pgfqpoint{1.184244in}{0.457938in}}%
\pgfpathcurveto{\pgfqpoint{1.180337in}{0.461845in}}{\pgfqpoint{1.175038in}{0.464040in}}{\pgfqpoint{1.169513in}{0.464040in}}%
\pgfpathcurveto{\pgfqpoint{1.163988in}{0.464040in}}{\pgfqpoint{1.158688in}{0.461845in}}{\pgfqpoint{1.154781in}{0.457938in}}%
\pgfpathcurveto{\pgfqpoint{1.150875in}{0.454031in}}{\pgfqpoint{1.148679in}{0.448732in}}{\pgfqpoint{1.148679in}{0.443206in}}%
\pgfpathcurveto{\pgfqpoint{1.148679in}{0.437681in}}{\pgfqpoint{1.150875in}{0.432382in}}{\pgfqpoint{1.154781in}{0.428475in}}%
\pgfpathcurveto{\pgfqpoint{1.158688in}{0.424568in}}{\pgfqpoint{1.163988in}{0.422373in}}{\pgfqpoint{1.169513in}{0.422373in}}%
\pgfpathclose%
\pgfusepath{stroke,fill}%
\end{pgfscope}%
\begin{pgfscope}%
\pgfpathrectangle{\pgfqpoint{0.403630in}{0.251348in}}{\pgfqpoint{1.395000in}{1.359000in}}%
\pgfusepath{clip}%
\pgfsetbuttcap%
\pgfsetroundjoin%
\definecolor{currentfill}{rgb}{0.686275,0.352941,0.313725}%
\pgfsetfillcolor{currentfill}%
\pgfsetlinewidth{0.501875pt}%
\definecolor{currentstroke}{rgb}{0.686275,0.352941,0.313725}%
\pgfsetstrokecolor{currentstroke}%
\pgfsetdash{}{0pt}%
\pgfpathmoveto{\pgfqpoint{1.596219in}{0.859385in}}%
\pgfpathcurveto{\pgfqpoint{1.601744in}{0.859385in}}{\pgfqpoint{1.607043in}{0.861580in}}{\pgfqpoint{1.610950in}{0.865487in}}%
\pgfpathcurveto{\pgfqpoint{1.614857in}{0.869394in}}{\pgfqpoint{1.617052in}{0.874693in}}{\pgfqpoint{1.617052in}{0.880218in}}%
\pgfpathcurveto{\pgfqpoint{1.617052in}{0.885743in}}{\pgfqpoint{1.614857in}{0.891043in}}{\pgfqpoint{1.610950in}{0.894950in}}%
\pgfpathcurveto{\pgfqpoint{1.607043in}{0.898856in}}{\pgfqpoint{1.601744in}{0.901052in}}{\pgfqpoint{1.596219in}{0.901052in}}%
\pgfpathcurveto{\pgfqpoint{1.590694in}{0.901052in}}{\pgfqpoint{1.585394in}{0.898856in}}{\pgfqpoint{1.581487in}{0.894950in}}%
\pgfpathcurveto{\pgfqpoint{1.577580in}{0.891043in}}{\pgfqpoint{1.575385in}{0.885743in}}{\pgfqpoint{1.575385in}{0.880218in}}%
\pgfpathcurveto{\pgfqpoint{1.575385in}{0.874693in}}{\pgfqpoint{1.577580in}{0.869394in}}{\pgfqpoint{1.581487in}{0.865487in}}%
\pgfpathcurveto{\pgfqpoint{1.585394in}{0.861580in}}{\pgfqpoint{1.590694in}{0.859385in}}{\pgfqpoint{1.596219in}{0.859385in}}%
\pgfpathclose%
\pgfusepath{stroke,fill}%
\end{pgfscope}%
\begin{pgfscope}%
\pgfpathrectangle{\pgfqpoint{0.403630in}{0.251348in}}{\pgfqpoint{1.395000in}{1.359000in}}%
\pgfusepath{clip}%
\pgfsetbuttcap%
\pgfsetroundjoin%
\definecolor{currentfill}{rgb}{0.686275,0.352941,0.313725}%
\pgfsetfillcolor{currentfill}%
\pgfsetlinewidth{0.501875pt}%
\definecolor{currentstroke}{rgb}{0.686275,0.352941,0.313725}%
\pgfsetstrokecolor{currentstroke}%
\pgfsetdash{}{0pt}%
\pgfpathmoveto{\pgfqpoint{0.813925in}{0.502314in}}%
\pgfpathcurveto{\pgfqpoint{0.819450in}{0.502314in}}{\pgfqpoint{0.824749in}{0.504509in}}{\pgfqpoint{0.828656in}{0.508416in}}%
\pgfpathcurveto{\pgfqpoint{0.832563in}{0.512323in}}{\pgfqpoint{0.834758in}{0.517623in}}{\pgfqpoint{0.834758in}{0.523148in}}%
\pgfpathcurveto{\pgfqpoint{0.834758in}{0.528673in}}{\pgfqpoint{0.832563in}{0.533972in}}{\pgfqpoint{0.828656in}{0.537879in}}%
\pgfpathcurveto{\pgfqpoint{0.824749in}{0.541786in}}{\pgfqpoint{0.819450in}{0.543981in}}{\pgfqpoint{0.813925in}{0.543981in}}%
\pgfpathcurveto{\pgfqpoint{0.808399in}{0.543981in}}{\pgfqpoint{0.803100in}{0.541786in}}{\pgfqpoint{0.799193in}{0.537879in}}%
\pgfpathcurveto{\pgfqpoint{0.795286in}{0.533972in}}{\pgfqpoint{0.793091in}{0.528673in}}{\pgfqpoint{0.793091in}{0.523148in}}%
\pgfpathcurveto{\pgfqpoint{0.793091in}{0.517623in}}{\pgfqpoint{0.795286in}{0.512323in}}{\pgfqpoint{0.799193in}{0.508416in}}%
\pgfpathcurveto{\pgfqpoint{0.803100in}{0.504509in}}{\pgfqpoint{0.808399in}{0.502314in}}{\pgfqpoint{0.813925in}{0.502314in}}%
\pgfpathclose%
\pgfusepath{stroke,fill}%
\end{pgfscope}%
\begin{pgfscope}%
\pgfpathrectangle{\pgfqpoint{0.403630in}{0.251348in}}{\pgfqpoint{1.395000in}{1.359000in}}%
\pgfusepath{clip}%
\pgfsetbuttcap%
\pgfsetroundjoin%
\definecolor{currentfill}{rgb}{0.686275,0.352941,0.313725}%
\pgfsetfillcolor{currentfill}%
\pgfsetlinewidth{0.501875pt}%
\definecolor{currentstroke}{rgb}{0.686275,0.352941,0.313725}%
\pgfsetstrokecolor{currentstroke}%
\pgfsetdash{}{0pt}%
\pgfpathmoveto{\pgfqpoint{1.049160in}{0.454350in}}%
\pgfpathcurveto{\pgfqpoint{1.054685in}{0.454350in}}{\pgfqpoint{1.059984in}{0.456545in}}{\pgfqpoint{1.063891in}{0.460452in}}%
\pgfpathcurveto{\pgfqpoint{1.067798in}{0.464358in}}{\pgfqpoint{1.069993in}{0.469658in}}{\pgfqpoint{1.069993in}{0.475183in}}%
\pgfpathcurveto{\pgfqpoint{1.069993in}{0.480708in}}{\pgfqpoint{1.067798in}{0.486008in}}{\pgfqpoint{1.063891in}{0.489914in}}%
\pgfpathcurveto{\pgfqpoint{1.059984in}{0.493821in}}{\pgfqpoint{1.054685in}{0.496016in}}{\pgfqpoint{1.049160in}{0.496016in}}%
\pgfpathcurveto{\pgfqpoint{1.043635in}{0.496016in}}{\pgfqpoint{1.038335in}{0.493821in}}{\pgfqpoint{1.034428in}{0.489914in}}%
\pgfpathcurveto{\pgfqpoint{1.030522in}{0.486008in}}{\pgfqpoint{1.028326in}{0.480708in}}{\pgfqpoint{1.028326in}{0.475183in}}%
\pgfpathcurveto{\pgfqpoint{1.028326in}{0.469658in}}{\pgfqpoint{1.030522in}{0.464358in}}{\pgfqpoint{1.034428in}{0.460452in}}%
\pgfpathcurveto{\pgfqpoint{1.038335in}{0.456545in}}{\pgfqpoint{1.043635in}{0.454350in}}{\pgfqpoint{1.049160in}{0.454350in}}%
\pgfpathclose%
\pgfusepath{stroke,fill}%
\end{pgfscope}%
\begin{pgfscope}%
\pgfpathrectangle{\pgfqpoint{0.403630in}{0.251348in}}{\pgfqpoint{1.395000in}{1.359000in}}%
\pgfusepath{clip}%
\pgfsetbuttcap%
\pgfsetroundjoin%
\definecolor{currentfill}{rgb}{0.686275,0.352941,0.313725}%
\pgfsetfillcolor{currentfill}%
\pgfsetlinewidth{0.501875pt}%
\definecolor{currentstroke}{rgb}{0.686275,0.352941,0.313725}%
\pgfsetstrokecolor{currentstroke}%
\pgfsetdash{}{0pt}%
\pgfpathmoveto{\pgfqpoint{1.065572in}{1.371008in}}%
\pgfpathcurveto{\pgfqpoint{1.071097in}{1.371008in}}{\pgfqpoint{1.076396in}{1.373204in}}{\pgfqpoint{1.080303in}{1.377110in}}%
\pgfpathcurveto{\pgfqpoint{1.084210in}{1.381017in}}{\pgfqpoint{1.086405in}{1.386317in}}{\pgfqpoint{1.086405in}{1.391842in}}%
\pgfpathcurveto{\pgfqpoint{1.086405in}{1.397367in}}{\pgfqpoint{1.084210in}{1.402666in}}{\pgfqpoint{1.080303in}{1.406573in}}%
\pgfpathcurveto{\pgfqpoint{1.076396in}{1.410480in}}{\pgfqpoint{1.071097in}{1.412675in}}{\pgfqpoint{1.065572in}{1.412675in}}%
\pgfpathcurveto{\pgfqpoint{1.060047in}{1.412675in}}{\pgfqpoint{1.054747in}{1.410480in}}{\pgfqpoint{1.050840in}{1.406573in}}%
\pgfpathcurveto{\pgfqpoint{1.046933in}{1.402666in}}{\pgfqpoint{1.044738in}{1.397367in}}{\pgfqpoint{1.044738in}{1.391842in}}%
\pgfpathcurveto{\pgfqpoint{1.044738in}{1.386317in}}{\pgfqpoint{1.046933in}{1.381017in}}{\pgfqpoint{1.050840in}{1.377110in}}%
\pgfpathcurveto{\pgfqpoint{1.054747in}{1.373204in}}{\pgfqpoint{1.060047in}{1.371008in}}{\pgfqpoint{1.065572in}{1.371008in}}%
\pgfpathclose%
\pgfusepath{stroke,fill}%
\end{pgfscope}%
\begin{pgfscope}%
\pgfpathrectangle{\pgfqpoint{0.403630in}{0.251348in}}{\pgfqpoint{1.395000in}{1.359000in}}%
\pgfusepath{clip}%
\pgfsetbuttcap%
\pgfsetroundjoin%
\definecolor{currentfill}{rgb}{0.686275,0.352941,0.313725}%
\pgfsetfillcolor{currentfill}%
\pgfsetlinewidth{0.501875pt}%
\definecolor{currentstroke}{rgb}{0.686275,0.352941,0.313725}%
\pgfsetstrokecolor{currentstroke}%
\pgfsetdash{}{0pt}%
\pgfpathmoveto{\pgfqpoint{1.530572in}{0.768785in}}%
\pgfpathcurveto{\pgfqpoint{1.536097in}{0.768785in}}{\pgfqpoint{1.541396in}{0.770980in}}{\pgfqpoint{1.545303in}{0.774887in}}%
\pgfpathcurveto{\pgfqpoint{1.549210in}{0.778794in}}{\pgfqpoint{1.551405in}{0.784093in}}{\pgfqpoint{1.551405in}{0.789618in}}%
\pgfpathcurveto{\pgfqpoint{1.551405in}{0.795143in}}{\pgfqpoint{1.549210in}{0.800443in}}{\pgfqpoint{1.545303in}{0.804350in}}%
\pgfpathcurveto{\pgfqpoint{1.541396in}{0.808256in}}{\pgfqpoint{1.536097in}{0.810452in}}{\pgfqpoint{1.530572in}{0.810452in}}%
\pgfpathcurveto{\pgfqpoint{1.525047in}{0.810452in}}{\pgfqpoint{1.519747in}{0.808256in}}{\pgfqpoint{1.515840in}{0.804350in}}%
\pgfpathcurveto{\pgfqpoint{1.511933in}{0.800443in}}{\pgfqpoint{1.509738in}{0.795143in}}{\pgfqpoint{1.509738in}{0.789618in}}%
\pgfpathcurveto{\pgfqpoint{1.509738in}{0.784093in}}{\pgfqpoint{1.511933in}{0.778794in}}{\pgfqpoint{1.515840in}{0.774887in}}%
\pgfpathcurveto{\pgfqpoint{1.519747in}{0.770980in}}{\pgfqpoint{1.525047in}{0.768785in}}{\pgfqpoint{1.530572in}{0.768785in}}%
\pgfpathclose%
\pgfusepath{stroke,fill}%
\end{pgfscope}%
\begin{pgfscope}%
\pgfpathrectangle{\pgfqpoint{0.403630in}{0.251348in}}{\pgfqpoint{1.395000in}{1.359000in}}%
\pgfusepath{clip}%
\pgfsetbuttcap%
\pgfsetroundjoin%
\definecolor{currentfill}{rgb}{0.686275,0.352941,0.313725}%
\pgfsetfillcolor{currentfill}%
\pgfsetlinewidth{0.501875pt}%
\definecolor{currentstroke}{rgb}{0.686275,0.352941,0.313725}%
\pgfsetstrokecolor{currentstroke}%
\pgfsetdash{}{0pt}%
\pgfpathmoveto{\pgfqpoint{1.109336in}{1.371008in}}%
\pgfpathcurveto{\pgfqpoint{1.114861in}{1.371008in}}{\pgfqpoint{1.120161in}{1.373204in}}{\pgfqpoint{1.124068in}{1.377110in}}%
\pgfpathcurveto{\pgfqpoint{1.127975in}{1.381017in}}{\pgfqpoint{1.130170in}{1.386317in}}{\pgfqpoint{1.130170in}{1.391842in}}%
\pgfpathcurveto{\pgfqpoint{1.130170in}{1.397367in}}{\pgfqpoint{1.127975in}{1.402666in}}{\pgfqpoint{1.124068in}{1.406573in}}%
\pgfpathcurveto{\pgfqpoint{1.120161in}{1.410480in}}{\pgfqpoint{1.114861in}{1.412675in}}{\pgfqpoint{1.109336in}{1.412675in}}%
\pgfpathcurveto{\pgfqpoint{1.103811in}{1.412675in}}{\pgfqpoint{1.098512in}{1.410480in}}{\pgfqpoint{1.094605in}{1.406573in}}%
\pgfpathcurveto{\pgfqpoint{1.090698in}{1.402666in}}{\pgfqpoint{1.088503in}{1.397367in}}{\pgfqpoint{1.088503in}{1.391842in}}%
\pgfpathcurveto{\pgfqpoint{1.088503in}{1.386317in}}{\pgfqpoint{1.090698in}{1.381017in}}{\pgfqpoint{1.094605in}{1.377110in}}%
\pgfpathcurveto{\pgfqpoint{1.098512in}{1.373204in}}{\pgfqpoint{1.103811in}{1.371008in}}{\pgfqpoint{1.109336in}{1.371008in}}%
\pgfpathclose%
\pgfusepath{stroke,fill}%
\end{pgfscope}%
\begin{pgfscope}%
\pgfpathrectangle{\pgfqpoint{0.403630in}{0.251348in}}{\pgfqpoint{1.395000in}{1.359000in}}%
\pgfusepath{clip}%
\pgfsetbuttcap%
\pgfsetroundjoin%
\definecolor{currentfill}{rgb}{0.686275,0.352941,0.313725}%
\pgfsetfillcolor{currentfill}%
\pgfsetlinewidth{0.501875pt}%
\definecolor{currentstroke}{rgb}{0.686275,0.352941,0.313725}%
\pgfsetstrokecolor{currentstroke}%
\pgfsetdash{}{0pt}%
\pgfpathmoveto{\pgfqpoint{0.611513in}{0.928667in}}%
\pgfpathcurveto{\pgfqpoint{0.617038in}{0.928667in}}{\pgfqpoint{0.622337in}{0.930862in}}{\pgfqpoint{0.626244in}{0.934769in}}%
\pgfpathcurveto{\pgfqpoint{0.630151in}{0.938676in}}{\pgfqpoint{0.632346in}{0.943976in}}{\pgfqpoint{0.632346in}{0.949501in}}%
\pgfpathcurveto{\pgfqpoint{0.632346in}{0.955026in}}{\pgfqpoint{0.630151in}{0.960325in}}{\pgfqpoint{0.626244in}{0.964232in}}%
\pgfpathcurveto{\pgfqpoint{0.622337in}{0.968139in}}{\pgfqpoint{0.617038in}{0.970334in}}{\pgfqpoint{0.611513in}{0.970334in}}%
\pgfpathcurveto{\pgfqpoint{0.605988in}{0.970334in}}{\pgfqpoint{0.600688in}{0.968139in}}{\pgfqpoint{0.596781in}{0.964232in}}%
\pgfpathcurveto{\pgfqpoint{0.592875in}{0.960325in}}{\pgfqpoint{0.590679in}{0.955026in}}{\pgfqpoint{0.590679in}{0.949501in}}%
\pgfpathcurveto{\pgfqpoint{0.590679in}{0.943976in}}{\pgfqpoint{0.592875in}{0.938676in}}{\pgfqpoint{0.596781in}{0.934769in}}%
\pgfpathcurveto{\pgfqpoint{0.600688in}{0.930862in}}{\pgfqpoint{0.605988in}{0.928667in}}{\pgfqpoint{0.611513in}{0.928667in}}%
\pgfpathclose%
\pgfusepath{stroke,fill}%
\end{pgfscope}%
\begin{pgfscope}%
\pgfpathrectangle{\pgfqpoint{0.403630in}{0.251348in}}{\pgfqpoint{1.395000in}{1.359000in}}%
\pgfusepath{clip}%
\pgfsetbuttcap%
\pgfsetroundjoin%
\definecolor{currentfill}{rgb}{0.686275,0.352941,0.313725}%
\pgfsetfillcolor{currentfill}%
\pgfsetlinewidth{0.501875pt}%
\definecolor{currentstroke}{rgb}{0.686275,0.352941,0.313725}%
\pgfsetstrokecolor{currentstroke}%
\pgfsetdash{}{0pt}%
\pgfpathmoveto{\pgfqpoint{0.934277in}{0.496985in}}%
\pgfpathcurveto{\pgfqpoint{0.939803in}{0.496985in}}{\pgfqpoint{0.945102in}{0.499180in}}{\pgfqpoint{0.949009in}{0.503087in}}%
\pgfpathcurveto{\pgfqpoint{0.952916in}{0.506994in}}{\pgfqpoint{0.955111in}{0.512293in}}{\pgfqpoint{0.955111in}{0.517818in}}%
\pgfpathcurveto{\pgfqpoint{0.955111in}{0.523343in}}{\pgfqpoint{0.952916in}{0.528643in}}{\pgfqpoint{0.949009in}{0.532550in}}%
\pgfpathcurveto{\pgfqpoint{0.945102in}{0.536456in}}{\pgfqpoint{0.939803in}{0.538652in}}{\pgfqpoint{0.934277in}{0.538652in}}%
\pgfpathcurveto{\pgfqpoint{0.928752in}{0.538652in}}{\pgfqpoint{0.923453in}{0.536456in}}{\pgfqpoint{0.919546in}{0.532550in}}%
\pgfpathcurveto{\pgfqpoint{0.915639in}{0.528643in}}{\pgfqpoint{0.913444in}{0.523343in}}{\pgfqpoint{0.913444in}{0.517818in}}%
\pgfpathcurveto{\pgfqpoint{0.913444in}{0.512293in}}{\pgfqpoint{0.915639in}{0.506994in}}{\pgfqpoint{0.919546in}{0.503087in}}%
\pgfpathcurveto{\pgfqpoint{0.923453in}{0.499180in}}{\pgfqpoint{0.928752in}{0.496985in}}{\pgfqpoint{0.934277in}{0.496985in}}%
\pgfpathclose%
\pgfusepath{stroke,fill}%
\end{pgfscope}%
\begin{pgfscope}%
\pgfpathrectangle{\pgfqpoint{0.403630in}{0.251348in}}{\pgfqpoint{1.395000in}{1.359000in}}%
\pgfusepath{clip}%
\pgfsetbuttcap%
\pgfsetroundjoin%
\definecolor{currentfill}{rgb}{0.686275,0.352941,0.313725}%
\pgfsetfillcolor{currentfill}%
\pgfsetlinewidth{0.501875pt}%
\definecolor{currentstroke}{rgb}{0.686275,0.352941,0.313725}%
\pgfsetstrokecolor{currentstroke}%
\pgfsetdash{}{0pt}%
\pgfpathmoveto{\pgfqpoint{1.459454in}{1.269750in}}%
\pgfpathcurveto{\pgfqpoint{1.464979in}{1.269750in}}{\pgfqpoint{1.470279in}{1.271945in}}{\pgfqpoint{1.474185in}{1.275852in}}%
\pgfpathcurveto{\pgfqpoint{1.478092in}{1.279758in}}{\pgfqpoint{1.480287in}{1.285058in}}{\pgfqpoint{1.480287in}{1.290583in}}%
\pgfpathcurveto{\pgfqpoint{1.480287in}{1.296108in}}{\pgfqpoint{1.478092in}{1.301408in}}{\pgfqpoint{1.474185in}{1.305314in}}%
\pgfpathcurveto{\pgfqpoint{1.470279in}{1.309221in}}{\pgfqpoint{1.464979in}{1.311416in}}{\pgfqpoint{1.459454in}{1.311416in}}%
\pgfpathcurveto{\pgfqpoint{1.453929in}{1.311416in}}{\pgfqpoint{1.448629in}{1.309221in}}{\pgfqpoint{1.444723in}{1.305314in}}%
\pgfpathcurveto{\pgfqpoint{1.440816in}{1.301408in}}{\pgfqpoint{1.438621in}{1.296108in}}{\pgfqpoint{1.438621in}{1.290583in}}%
\pgfpathcurveto{\pgfqpoint{1.438621in}{1.285058in}}{\pgfqpoint{1.440816in}{1.279758in}}{\pgfqpoint{1.444723in}{1.275852in}}%
\pgfpathcurveto{\pgfqpoint{1.448629in}{1.271945in}}{\pgfqpoint{1.453929in}{1.269750in}}{\pgfqpoint{1.459454in}{1.269750in}}%
\pgfpathclose%
\pgfusepath{stroke,fill}%
\end{pgfscope}%
\begin{pgfscope}%
\pgfpathrectangle{\pgfqpoint{0.403630in}{0.251348in}}{\pgfqpoint{1.395000in}{1.359000in}}%
\pgfusepath{clip}%
\pgfsetbuttcap%
\pgfsetroundjoin%
\definecolor{currentfill}{rgb}{0.686275,0.352941,0.313725}%
\pgfsetfillcolor{currentfill}%
\pgfsetlinewidth{0.501875pt}%
\definecolor{currentstroke}{rgb}{0.686275,0.352941,0.313725}%
\pgfsetstrokecolor{currentstroke}%
\pgfsetdash{}{0pt}%
\pgfpathmoveto{\pgfqpoint{0.868630in}{0.507644in}}%
\pgfpathcurveto{\pgfqpoint{0.874155in}{0.507644in}}{\pgfqpoint{0.879455in}{0.509839in}}{\pgfqpoint{0.883362in}{0.513746in}}%
\pgfpathcurveto{\pgfqpoint{0.887269in}{0.517652in}}{\pgfqpoint{0.889464in}{0.522952in}}{\pgfqpoint{0.889464in}{0.528477in}}%
\pgfpathcurveto{\pgfqpoint{0.889464in}{0.534002in}}{\pgfqpoint{0.887269in}{0.539302in}}{\pgfqpoint{0.883362in}{0.543208in}}%
\pgfpathcurveto{\pgfqpoint{0.879455in}{0.547115in}}{\pgfqpoint{0.874155in}{0.549310in}}{\pgfqpoint{0.868630in}{0.549310in}}%
\pgfpathcurveto{\pgfqpoint{0.863105in}{0.549310in}}{\pgfqpoint{0.857806in}{0.547115in}}{\pgfqpoint{0.853899in}{0.543208in}}%
\pgfpathcurveto{\pgfqpoint{0.849992in}{0.539302in}}{\pgfqpoint{0.847797in}{0.534002in}}{\pgfqpoint{0.847797in}{0.528477in}}%
\pgfpathcurveto{\pgfqpoint{0.847797in}{0.522952in}}{\pgfqpoint{0.849992in}{0.517652in}}{\pgfqpoint{0.853899in}{0.513746in}}%
\pgfpathcurveto{\pgfqpoint{0.857806in}{0.509839in}}{\pgfqpoint{0.863105in}{0.507644in}}{\pgfqpoint{0.868630in}{0.507644in}}%
\pgfpathclose%
\pgfusepath{stroke,fill}%
\end{pgfscope}%
\begin{pgfscope}%
\pgfpathrectangle{\pgfqpoint{0.403630in}{0.251348in}}{\pgfqpoint{1.395000in}{1.359000in}}%
\pgfusepath{clip}%
\pgfsetbuttcap%
\pgfsetroundjoin%
\definecolor{currentfill}{rgb}{0.686275,0.352941,0.313725}%
\pgfsetfillcolor{currentfill}%
\pgfsetlinewidth{0.501875pt}%
\definecolor{currentstroke}{rgb}{0.686275,0.352941,0.313725}%
\pgfsetstrokecolor{currentstroke}%
\pgfsetdash{}{0pt}%
\pgfpathmoveto{\pgfqpoint{0.835807in}{1.333703in}}%
\pgfpathcurveto{\pgfqpoint{0.841332in}{1.333703in}}{\pgfqpoint{0.846631in}{1.335898in}}{\pgfqpoint{0.850538in}{1.339804in}}%
\pgfpathcurveto{\pgfqpoint{0.854445in}{1.343711in}}{\pgfqpoint{0.856640in}{1.349011in}}{\pgfqpoint{0.856640in}{1.354536in}}%
\pgfpathcurveto{\pgfqpoint{0.856640in}{1.360061in}}{\pgfqpoint{0.854445in}{1.365360in}}{\pgfqpoint{0.850538in}{1.369267in}}%
\pgfpathcurveto{\pgfqpoint{0.846631in}{1.373174in}}{\pgfqpoint{0.841332in}{1.375369in}}{\pgfqpoint{0.835807in}{1.375369in}}%
\pgfpathcurveto{\pgfqpoint{0.830282in}{1.375369in}}{\pgfqpoint{0.824982in}{1.373174in}}{\pgfqpoint{0.821075in}{1.369267in}}%
\pgfpathcurveto{\pgfqpoint{0.817169in}{1.365360in}}{\pgfqpoint{0.814974in}{1.360061in}}{\pgfqpoint{0.814974in}{1.354536in}}%
\pgfpathcurveto{\pgfqpoint{0.814974in}{1.349011in}}{\pgfqpoint{0.817169in}{1.343711in}}{\pgfqpoint{0.821075in}{1.339804in}}%
\pgfpathcurveto{\pgfqpoint{0.824982in}{1.335898in}}{\pgfqpoint{0.830282in}{1.333703in}}{\pgfqpoint{0.835807in}{1.333703in}}%
\pgfpathclose%
\pgfusepath{stroke,fill}%
\end{pgfscope}%
\begin{pgfscope}%
\pgfpathrectangle{\pgfqpoint{0.403630in}{0.251348in}}{\pgfqpoint{1.395000in}{1.359000in}}%
\pgfusepath{clip}%
\pgfsetbuttcap%
\pgfsetroundjoin%
\definecolor{currentfill}{rgb}{0.686275,0.352941,0.313725}%
\pgfsetfillcolor{currentfill}%
\pgfsetlinewidth{0.501875pt}%
\definecolor{currentstroke}{rgb}{0.686275,0.352941,0.313725}%
\pgfsetstrokecolor{currentstroke}%
\pgfsetdash{}{0pt}%
\pgfpathmoveto{\pgfqpoint{1.607160in}{1.013938in}}%
\pgfpathcurveto{\pgfqpoint{1.612685in}{1.013938in}}{\pgfqpoint{1.617984in}{1.016133in}}{\pgfqpoint{1.621891in}{1.020040in}}%
\pgfpathcurveto{\pgfqpoint{1.625798in}{1.023947in}}{\pgfqpoint{1.627993in}{1.029246in}}{\pgfqpoint{1.627993in}{1.034771in}}%
\pgfpathcurveto{\pgfqpoint{1.627993in}{1.040296in}}{\pgfqpoint{1.625798in}{1.045596in}}{\pgfqpoint{1.621891in}{1.049503in}}%
\pgfpathcurveto{\pgfqpoint{1.617984in}{1.053409in}}{\pgfqpoint{1.612685in}{1.055605in}}{\pgfqpoint{1.607160in}{1.055605in}}%
\pgfpathcurveto{\pgfqpoint{1.601635in}{1.055605in}}{\pgfqpoint{1.596335in}{1.053409in}}{\pgfqpoint{1.592428in}{1.049503in}}%
\pgfpathcurveto{\pgfqpoint{1.588522in}{1.045596in}}{\pgfqpoint{1.586326in}{1.040296in}}{\pgfqpoint{1.586326in}{1.034771in}}%
\pgfpathcurveto{\pgfqpoint{1.586326in}{1.029246in}}{\pgfqpoint{1.588522in}{1.023947in}}{\pgfqpoint{1.592428in}{1.020040in}}%
\pgfpathcurveto{\pgfqpoint{1.596335in}{1.016133in}}{\pgfqpoint{1.601635in}{1.013938in}}{\pgfqpoint{1.607160in}{1.013938in}}%
\pgfpathclose%
\pgfusepath{stroke,fill}%
\end{pgfscope}%
\begin{pgfscope}%
\pgfpathrectangle{\pgfqpoint{0.403630in}{0.251348in}}{\pgfqpoint{1.395000in}{1.359000in}}%
\pgfusepath{clip}%
\pgfsetbuttcap%
\pgfsetroundjoin%
\definecolor{currentfill}{rgb}{0.686275,0.352941,0.313725}%
\pgfsetfillcolor{currentfill}%
\pgfsetlinewidth{0.501875pt}%
\definecolor{currentstroke}{rgb}{0.686275,0.352941,0.313725}%
\pgfsetstrokecolor{currentstroke}%
\pgfsetdash{}{0pt}%
\pgfpathmoveto{\pgfqpoint{0.759219in}{0.587585in}}%
\pgfpathcurveto{\pgfqpoint{0.764744in}{0.587585in}}{\pgfqpoint{0.770043in}{0.589780in}}{\pgfqpoint{0.773950in}{0.593687in}}%
\pgfpathcurveto{\pgfqpoint{0.777857in}{0.597594in}}{\pgfqpoint{0.780052in}{0.602893in}}{\pgfqpoint{0.780052in}{0.608418in}}%
\pgfpathcurveto{\pgfqpoint{0.780052in}{0.613943in}}{\pgfqpoint{0.777857in}{0.619243in}}{\pgfqpoint{0.773950in}{0.623150in}}%
\pgfpathcurveto{\pgfqpoint{0.770043in}{0.627056in}}{\pgfqpoint{0.764744in}{0.629252in}}{\pgfqpoint{0.759219in}{0.629252in}}%
\pgfpathcurveto{\pgfqpoint{0.753694in}{0.629252in}}{\pgfqpoint{0.748394in}{0.627056in}}{\pgfqpoint{0.744487in}{0.623150in}}%
\pgfpathcurveto{\pgfqpoint{0.740580in}{0.619243in}}{\pgfqpoint{0.738385in}{0.613943in}}{\pgfqpoint{0.738385in}{0.608418in}}%
\pgfpathcurveto{\pgfqpoint{0.738385in}{0.602893in}}{\pgfqpoint{0.740580in}{0.597594in}}{\pgfqpoint{0.744487in}{0.593687in}}%
\pgfpathcurveto{\pgfqpoint{0.748394in}{0.589780in}}{\pgfqpoint{0.753694in}{0.587585in}}{\pgfqpoint{0.759219in}{0.587585in}}%
\pgfpathclose%
\pgfusepath{stroke,fill}%
\end{pgfscope}%
\begin{pgfscope}%
\pgfpathrectangle{\pgfqpoint{0.403630in}{0.251348in}}{\pgfqpoint{1.395000in}{1.359000in}}%
\pgfusepath{clip}%
\pgfsetbuttcap%
\pgfsetroundjoin%
\definecolor{currentfill}{rgb}{0.686275,0.352941,0.313725}%
\pgfsetfillcolor{currentfill}%
\pgfsetlinewidth{0.501875pt}%
\definecolor{currentstroke}{rgb}{0.686275,0.352941,0.313725}%
\pgfsetstrokecolor{currentstroke}%
\pgfsetdash{}{0pt}%
\pgfpathmoveto{\pgfqpoint{1.060101in}{0.454350in}}%
\pgfpathcurveto{\pgfqpoint{1.065626in}{0.454350in}}{\pgfqpoint{1.070926in}{0.456545in}}{\pgfqpoint{1.074832in}{0.460452in}}%
\pgfpathcurveto{\pgfqpoint{1.078739in}{0.464358in}}{\pgfqpoint{1.080934in}{0.469658in}}{\pgfqpoint{1.080934in}{0.475183in}}%
\pgfpathcurveto{\pgfqpoint{1.080934in}{0.480708in}}{\pgfqpoint{1.078739in}{0.486008in}}{\pgfqpoint{1.074832in}{0.489914in}}%
\pgfpathcurveto{\pgfqpoint{1.070926in}{0.493821in}}{\pgfqpoint{1.065626in}{0.496016in}}{\pgfqpoint{1.060101in}{0.496016in}}%
\pgfpathcurveto{\pgfqpoint{1.054576in}{0.496016in}}{\pgfqpoint{1.049276in}{0.493821in}}{\pgfqpoint{1.045370in}{0.489914in}}%
\pgfpathcurveto{\pgfqpoint{1.041463in}{0.486008in}}{\pgfqpoint{1.039268in}{0.480708in}}{\pgfqpoint{1.039268in}{0.475183in}}%
\pgfpathcurveto{\pgfqpoint{1.039268in}{0.469658in}}{\pgfqpoint{1.041463in}{0.464358in}}{\pgfqpoint{1.045370in}{0.460452in}}%
\pgfpathcurveto{\pgfqpoint{1.049276in}{0.456545in}}{\pgfqpoint{1.054576in}{0.454350in}}{\pgfqpoint{1.060101in}{0.454350in}}%
\pgfpathclose%
\pgfusepath{stroke,fill}%
\end{pgfscope}%
\begin{pgfscope}%
\pgfpathrectangle{\pgfqpoint{0.403630in}{0.251348in}}{\pgfqpoint{1.395000in}{1.359000in}}%
\pgfusepath{clip}%
\pgfsetbuttcap%
\pgfsetroundjoin%
\definecolor{currentfill}{rgb}{0.686275,0.352941,0.313725}%
\pgfsetfillcolor{currentfill}%
\pgfsetlinewidth{0.501875pt}%
\definecolor{currentstroke}{rgb}{0.686275,0.352941,0.313725}%
\pgfsetstrokecolor{currentstroke}%
\pgfsetdash{}{0pt}%
\pgfpathmoveto{\pgfqpoint{1.514160in}{1.141844in}}%
\pgfpathcurveto{\pgfqpoint{1.519685in}{1.141844in}}{\pgfqpoint{1.524984in}{1.144039in}}{\pgfqpoint{1.528891in}{1.147946in}}%
\pgfpathcurveto{\pgfqpoint{1.532798in}{1.151852in}}{\pgfqpoint{1.534993in}{1.157152in}}{\pgfqpoint{1.534993in}{1.162677in}}%
\pgfpathcurveto{\pgfqpoint{1.534993in}{1.168202in}}{\pgfqpoint{1.532798in}{1.173502in}}{\pgfqpoint{1.528891in}{1.177408in}}%
\pgfpathcurveto{\pgfqpoint{1.524984in}{1.181315in}}{\pgfqpoint{1.519685in}{1.183510in}}{\pgfqpoint{1.514160in}{1.183510in}}%
\pgfpathcurveto{\pgfqpoint{1.508635in}{1.183510in}}{\pgfqpoint{1.503335in}{1.181315in}}{\pgfqpoint{1.499428in}{1.177408in}}%
\pgfpathcurveto{\pgfqpoint{1.495522in}{1.173502in}}{\pgfqpoint{1.493326in}{1.168202in}}{\pgfqpoint{1.493326in}{1.162677in}}%
\pgfpathcurveto{\pgfqpoint{1.493326in}{1.157152in}}{\pgfqpoint{1.495522in}{1.151852in}}{\pgfqpoint{1.499428in}{1.147946in}}%
\pgfpathcurveto{\pgfqpoint{1.503335in}{1.144039in}}{\pgfqpoint{1.508635in}{1.141844in}}{\pgfqpoint{1.514160in}{1.141844in}}%
\pgfpathclose%
\pgfusepath{stroke,fill}%
\end{pgfscope}%
\begin{pgfscope}%
\pgfpathrectangle{\pgfqpoint{0.403630in}{0.251348in}}{\pgfqpoint{1.395000in}{1.359000in}}%
\pgfusepath{clip}%
\pgfsetbuttcap%
\pgfsetroundjoin%
\definecolor{currentfill}{rgb}{0.686275,0.352941,0.313725}%
\pgfsetfillcolor{currentfill}%
\pgfsetlinewidth{0.501875pt}%
\definecolor{currentstroke}{rgb}{0.686275,0.352941,0.313725}%
\pgfsetstrokecolor{currentstroke}%
\pgfsetdash{}{0pt}%
\pgfpathmoveto{\pgfqpoint{0.923336in}{0.512973in}}%
\pgfpathcurveto{\pgfqpoint{0.928861in}{0.512973in}}{\pgfqpoint{0.934161in}{0.515168in}}{\pgfqpoint{0.938068in}{0.519075in}}%
\pgfpathcurveto{\pgfqpoint{0.941975in}{0.522982in}}{\pgfqpoint{0.944170in}{0.528281in}}{\pgfqpoint{0.944170in}{0.533806in}}%
\pgfpathcurveto{\pgfqpoint{0.944170in}{0.539332in}}{\pgfqpoint{0.941975in}{0.544631in}}{\pgfqpoint{0.938068in}{0.548538in}}%
\pgfpathcurveto{\pgfqpoint{0.934161in}{0.552445in}}{\pgfqpoint{0.928861in}{0.554640in}}{\pgfqpoint{0.923336in}{0.554640in}}%
\pgfpathcurveto{\pgfqpoint{0.917811in}{0.554640in}}{\pgfqpoint{0.912512in}{0.552445in}}{\pgfqpoint{0.908605in}{0.548538in}}%
\pgfpathcurveto{\pgfqpoint{0.904698in}{0.544631in}}{\pgfqpoint{0.902503in}{0.539332in}}{\pgfqpoint{0.902503in}{0.533806in}}%
\pgfpathcurveto{\pgfqpoint{0.902503in}{0.528281in}}{\pgfqpoint{0.904698in}{0.522982in}}{\pgfqpoint{0.908605in}{0.519075in}}%
\pgfpathcurveto{\pgfqpoint{0.912512in}{0.515168in}}{\pgfqpoint{0.917811in}{0.512973in}}{\pgfqpoint{0.923336in}{0.512973in}}%
\pgfpathclose%
\pgfusepath{stroke,fill}%
\end{pgfscope}%
\begin{pgfscope}%
\pgfpathrectangle{\pgfqpoint{0.403630in}{0.251348in}}{\pgfqpoint{1.395000in}{1.359000in}}%
\pgfusepath{clip}%
\pgfsetbuttcap%
\pgfsetroundjoin%
\definecolor{currentfill}{rgb}{0.686275,0.352941,0.313725}%
\pgfsetfillcolor{currentfill}%
\pgfsetlinewidth{0.501875pt}%
\definecolor{currentstroke}{rgb}{0.686275,0.352941,0.313725}%
\pgfsetstrokecolor{currentstroke}%
\pgfsetdash{}{0pt}%
\pgfpathmoveto{\pgfqpoint{1.147630in}{0.433032in}}%
\pgfpathcurveto{\pgfqpoint{1.153155in}{0.433032in}}{\pgfqpoint{1.158455in}{0.435227in}}{\pgfqpoint{1.162362in}{0.439134in}}%
\pgfpathcurveto{\pgfqpoint{1.166269in}{0.443041in}}{\pgfqpoint{1.168464in}{0.448340in}}{\pgfqpoint{1.168464in}{0.453865in}}%
\pgfpathcurveto{\pgfqpoint{1.168464in}{0.459390in}}{\pgfqpoint{1.166269in}{0.464690in}}{\pgfqpoint{1.162362in}{0.468597in}}%
\pgfpathcurveto{\pgfqpoint{1.158455in}{0.472503in}}{\pgfqpoint{1.153155in}{0.474699in}}{\pgfqpoint{1.147630in}{0.474699in}}%
\pgfpathcurveto{\pgfqpoint{1.142105in}{0.474699in}}{\pgfqpoint{1.136806in}{0.472503in}}{\pgfqpoint{1.132899in}{0.468597in}}%
\pgfpathcurveto{\pgfqpoint{1.128992in}{0.464690in}}{\pgfqpoint{1.126797in}{0.459390in}}{\pgfqpoint{1.126797in}{0.453865in}}%
\pgfpathcurveto{\pgfqpoint{1.126797in}{0.448340in}}{\pgfqpoint{1.128992in}{0.443041in}}{\pgfqpoint{1.132899in}{0.439134in}}%
\pgfpathcurveto{\pgfqpoint{1.136806in}{0.435227in}}{\pgfqpoint{1.142105in}{0.433032in}}{\pgfqpoint{1.147630in}{0.433032in}}%
\pgfpathclose%
\pgfusepath{stroke,fill}%
\end{pgfscope}%
\begin{pgfscope}%
\pgfpathrectangle{\pgfqpoint{0.403630in}{0.251348in}}{\pgfqpoint{1.395000in}{1.359000in}}%
\pgfusepath{clip}%
\pgfsetbuttcap%
\pgfsetroundjoin%
\definecolor{currentfill}{rgb}{0.686275,0.352941,0.313725}%
\pgfsetfillcolor{currentfill}%
\pgfsetlinewidth{0.501875pt}%
\definecolor{currentstroke}{rgb}{0.686275,0.352941,0.313725}%
\pgfsetstrokecolor{currentstroke}%
\pgfsetdash{}{0pt}%
\pgfpathmoveto{\pgfqpoint{1.213277in}{0.496985in}}%
\pgfpathcurveto{\pgfqpoint{1.218803in}{0.496985in}}{\pgfqpoint{1.224102in}{0.499180in}}{\pgfqpoint{1.228009in}{0.503087in}}%
\pgfpathcurveto{\pgfqpoint{1.231916in}{0.506994in}}{\pgfqpoint{1.234111in}{0.512293in}}{\pgfqpoint{1.234111in}{0.517818in}}%
\pgfpathcurveto{\pgfqpoint{1.234111in}{0.523343in}}{\pgfqpoint{1.231916in}{0.528643in}}{\pgfqpoint{1.228009in}{0.532550in}}%
\pgfpathcurveto{\pgfqpoint{1.224102in}{0.536456in}}{\pgfqpoint{1.218803in}{0.538652in}}{\pgfqpoint{1.213277in}{0.538652in}}%
\pgfpathcurveto{\pgfqpoint{1.207752in}{0.538652in}}{\pgfqpoint{1.202453in}{0.536456in}}{\pgfqpoint{1.198546in}{0.532550in}}%
\pgfpathcurveto{\pgfqpoint{1.194639in}{0.528643in}}{\pgfqpoint{1.192444in}{0.523343in}}{\pgfqpoint{1.192444in}{0.517818in}}%
\pgfpathcurveto{\pgfqpoint{1.192444in}{0.512293in}}{\pgfqpoint{1.194639in}{0.506994in}}{\pgfqpoint{1.198546in}{0.503087in}}%
\pgfpathcurveto{\pgfqpoint{1.202453in}{0.499180in}}{\pgfqpoint{1.207752in}{0.496985in}}{\pgfqpoint{1.213277in}{0.496985in}}%
\pgfpathclose%
\pgfusepath{stroke,fill}%
\end{pgfscope}%
\begin{pgfscope}%
\pgfpathrectangle{\pgfqpoint{0.403630in}{0.251348in}}{\pgfqpoint{1.395000in}{1.359000in}}%
\pgfusepath{clip}%
\pgfsetbuttcap%
\pgfsetroundjoin%
\definecolor{currentfill}{rgb}{0.686275,0.352941,0.313725}%
\pgfsetfillcolor{currentfill}%
\pgfsetlinewidth{0.501875pt}%
\definecolor{currentstroke}{rgb}{0.686275,0.352941,0.313725}%
\pgfsetstrokecolor{currentstroke}%
\pgfsetdash{}{0pt}%
\pgfpathmoveto{\pgfqpoint{1.514160in}{1.088550in}}%
\pgfpathcurveto{\pgfqpoint{1.519685in}{1.088550in}}{\pgfqpoint{1.524984in}{1.090745in}}{\pgfqpoint{1.528891in}{1.094652in}}%
\pgfpathcurveto{\pgfqpoint{1.532798in}{1.098558in}}{\pgfqpoint{1.534993in}{1.103858in}}{\pgfqpoint{1.534993in}{1.109383in}}%
\pgfpathcurveto{\pgfqpoint{1.534993in}{1.114908in}}{\pgfqpoint{1.532798in}{1.120208in}}{\pgfqpoint{1.528891in}{1.124114in}}%
\pgfpathcurveto{\pgfqpoint{1.524984in}{1.128021in}}{\pgfqpoint{1.519685in}{1.130216in}}{\pgfqpoint{1.514160in}{1.130216in}}%
\pgfpathcurveto{\pgfqpoint{1.508635in}{1.130216in}}{\pgfqpoint{1.503335in}{1.128021in}}{\pgfqpoint{1.499428in}{1.124114in}}%
\pgfpathcurveto{\pgfqpoint{1.495522in}{1.120208in}}{\pgfqpoint{1.493326in}{1.114908in}}{\pgfqpoint{1.493326in}{1.109383in}}%
\pgfpathcurveto{\pgfqpoint{1.493326in}{1.103858in}}{\pgfqpoint{1.495522in}{1.098558in}}{\pgfqpoint{1.499428in}{1.094652in}}%
\pgfpathcurveto{\pgfqpoint{1.503335in}{1.090745in}}{\pgfqpoint{1.508635in}{1.088550in}}{\pgfqpoint{1.514160in}{1.088550in}}%
\pgfpathclose%
\pgfusepath{stroke,fill}%
\end{pgfscope}%
\begin{pgfscope}%
\pgfpathrectangle{\pgfqpoint{0.403630in}{0.251348in}}{\pgfqpoint{1.395000in}{1.359000in}}%
\pgfusepath{clip}%
\pgfsetbuttcap%
\pgfsetroundjoin%
\definecolor{currentfill}{rgb}{0.686275,0.352941,0.313725}%
\pgfsetfillcolor{currentfill}%
\pgfsetlinewidth{0.501875pt}%
\definecolor{currentstroke}{rgb}{0.686275,0.352941,0.313725}%
\pgfsetstrokecolor{currentstroke}%
\pgfsetdash{}{0pt}%
\pgfpathmoveto{\pgfqpoint{1.010866in}{0.475667in}}%
\pgfpathcurveto{\pgfqpoint{1.016391in}{0.475667in}}{\pgfqpoint{1.021690in}{0.477862in}}{\pgfqpoint{1.025597in}{0.481769in}}%
\pgfpathcurveto{\pgfqpoint{1.029504in}{0.485676in}}{\pgfqpoint{1.031699in}{0.490976in}}{\pgfqpoint{1.031699in}{0.496501in}}%
\pgfpathcurveto{\pgfqpoint{1.031699in}{0.502026in}}{\pgfqpoint{1.029504in}{0.507325in}}{\pgfqpoint{1.025597in}{0.511232in}}%
\pgfpathcurveto{\pgfqpoint{1.021690in}{0.515139in}}{\pgfqpoint{1.016391in}{0.517334in}}{\pgfqpoint{1.010866in}{0.517334in}}%
\pgfpathcurveto{\pgfqpoint{1.005341in}{0.517334in}}{\pgfqpoint{1.000041in}{0.515139in}}{\pgfqpoint{0.996134in}{0.511232in}}%
\pgfpathcurveto{\pgfqpoint{0.992228in}{0.507325in}}{\pgfqpoint{0.990032in}{0.502026in}}{\pgfqpoint{0.990032in}{0.496501in}}%
\pgfpathcurveto{\pgfqpoint{0.990032in}{0.490976in}}{\pgfqpoint{0.992228in}{0.485676in}}{\pgfqpoint{0.996134in}{0.481769in}}%
\pgfpathcurveto{\pgfqpoint{1.000041in}{0.477862in}}{\pgfqpoint{1.005341in}{0.475667in}}{\pgfqpoint{1.010866in}{0.475667in}}%
\pgfpathclose%
\pgfusepath{stroke,fill}%
\end{pgfscope}%
\begin{pgfscope}%
\pgfpathrectangle{\pgfqpoint{0.403630in}{0.251348in}}{\pgfqpoint{1.395000in}{1.359000in}}%
\pgfusepath{clip}%
\pgfsetbuttcap%
\pgfsetroundjoin%
\definecolor{currentfill}{rgb}{0.686275,0.352941,0.313725}%
\pgfsetfillcolor{currentfill}%
\pgfsetlinewidth{0.501875pt}%
\definecolor{currentstroke}{rgb}{0.686275,0.352941,0.313725}%
\pgfsetstrokecolor{currentstroke}%
\pgfsetdash{}{0pt}%
\pgfpathmoveto{\pgfqpoint{1.388336in}{1.280408in}}%
\pgfpathcurveto{\pgfqpoint{1.393861in}{1.280408in}}{\pgfqpoint{1.399161in}{1.282604in}}{\pgfqpoint{1.403068in}{1.286510in}}%
\pgfpathcurveto{\pgfqpoint{1.406975in}{1.290417in}}{\pgfqpoint{1.409170in}{1.295717in}}{\pgfqpoint{1.409170in}{1.301242in}}%
\pgfpathcurveto{\pgfqpoint{1.409170in}{1.306767in}}{\pgfqpoint{1.406975in}{1.312066in}}{\pgfqpoint{1.403068in}{1.315973in}}%
\pgfpathcurveto{\pgfqpoint{1.399161in}{1.319880in}}{\pgfqpoint{1.393861in}{1.322075in}}{\pgfqpoint{1.388336in}{1.322075in}}%
\pgfpathcurveto{\pgfqpoint{1.382811in}{1.322075in}}{\pgfqpoint{1.377512in}{1.319880in}}{\pgfqpoint{1.373605in}{1.315973in}}%
\pgfpathcurveto{\pgfqpoint{1.369698in}{1.312066in}}{\pgfqpoint{1.367503in}{1.306767in}}{\pgfqpoint{1.367503in}{1.301242in}}%
\pgfpathcurveto{\pgfqpoint{1.367503in}{1.295717in}}{\pgfqpoint{1.369698in}{1.290417in}}{\pgfqpoint{1.373605in}{1.286510in}}%
\pgfpathcurveto{\pgfqpoint{1.377512in}{1.282604in}}{\pgfqpoint{1.382811in}{1.280408in}}{\pgfqpoint{1.388336in}{1.280408in}}%
\pgfpathclose%
\pgfusepath{stroke,fill}%
\end{pgfscope}%
\begin{pgfscope}%
\pgfpathrectangle{\pgfqpoint{0.403630in}{0.251348in}}{\pgfqpoint{1.395000in}{1.359000in}}%
\pgfusepath{clip}%
\pgfsetbuttcap%
\pgfsetroundjoin%
\definecolor{currentfill}{rgb}{0.686275,0.352941,0.313725}%
\pgfsetfillcolor{currentfill}%
\pgfsetlinewidth{0.501875pt}%
\definecolor{currentstroke}{rgb}{0.686275,0.352941,0.313725}%
\pgfsetstrokecolor{currentstroke}%
\pgfsetdash{}{0pt}%
\pgfpathmoveto{\pgfqpoint{0.688101in}{0.662197in}}%
\pgfpathcurveto{\pgfqpoint{0.693626in}{0.662197in}}{\pgfqpoint{0.698926in}{0.664392in}}{\pgfqpoint{0.702832in}{0.668299in}}%
\pgfpathcurveto{\pgfqpoint{0.706739in}{0.672205in}}{\pgfqpoint{0.708934in}{0.677505in}}{\pgfqpoint{0.708934in}{0.683030in}}%
\pgfpathcurveto{\pgfqpoint{0.708934in}{0.688555in}}{\pgfqpoint{0.706739in}{0.693855in}}{\pgfqpoint{0.702832in}{0.697761in}}%
\pgfpathcurveto{\pgfqpoint{0.698926in}{0.701668in}}{\pgfqpoint{0.693626in}{0.703863in}}{\pgfqpoint{0.688101in}{0.703863in}}%
\pgfpathcurveto{\pgfqpoint{0.682576in}{0.703863in}}{\pgfqpoint{0.677276in}{0.701668in}}{\pgfqpoint{0.673370in}{0.697761in}}%
\pgfpathcurveto{\pgfqpoint{0.669463in}{0.693855in}}{\pgfqpoint{0.667268in}{0.688555in}}{\pgfqpoint{0.667268in}{0.683030in}}%
\pgfpathcurveto{\pgfqpoint{0.667268in}{0.677505in}}{\pgfqpoint{0.669463in}{0.672205in}}{\pgfqpoint{0.673370in}{0.668299in}}%
\pgfpathcurveto{\pgfqpoint{0.677276in}{0.664392in}}{\pgfqpoint{0.682576in}{0.662197in}}{\pgfqpoint{0.688101in}{0.662197in}}%
\pgfpathclose%
\pgfusepath{stroke,fill}%
\end{pgfscope}%
\begin{pgfscope}%
\pgfpathrectangle{\pgfqpoint{0.403630in}{0.251348in}}{\pgfqpoint{1.395000in}{1.359000in}}%
\pgfusepath{clip}%
\pgfsetbuttcap%
\pgfsetroundjoin%
\definecolor{currentfill}{rgb}{0.686275,0.352941,0.313725}%
\pgfsetfillcolor{currentfill}%
\pgfsetlinewidth{0.501875pt}%
\definecolor{currentstroke}{rgb}{0.686275,0.352941,0.313725}%
\pgfsetstrokecolor{currentstroke}%
\pgfsetdash{}{0pt}%
\pgfpathmoveto{\pgfqpoint{1.448513in}{1.189808in}}%
\pgfpathcurveto{\pgfqpoint{1.454038in}{1.189808in}}{\pgfqpoint{1.459337in}{1.192004in}}{\pgfqpoint{1.463244in}{1.195910in}}%
\pgfpathcurveto{\pgfqpoint{1.467151in}{1.199817in}}{\pgfqpoint{1.469346in}{1.205117in}}{\pgfqpoint{1.469346in}{1.210642in}}%
\pgfpathcurveto{\pgfqpoint{1.469346in}{1.216167in}}{\pgfqpoint{1.467151in}{1.221466in}}{\pgfqpoint{1.463244in}{1.225373in}}%
\pgfpathcurveto{\pgfqpoint{1.459337in}{1.229280in}}{\pgfqpoint{1.454038in}{1.231475in}}{\pgfqpoint{1.448513in}{1.231475in}}%
\pgfpathcurveto{\pgfqpoint{1.442988in}{1.231475in}}{\pgfqpoint{1.437688in}{1.229280in}}{\pgfqpoint{1.433781in}{1.225373in}}%
\pgfpathcurveto{\pgfqpoint{1.429875in}{1.221466in}}{\pgfqpoint{1.427679in}{1.216167in}}{\pgfqpoint{1.427679in}{1.210642in}}%
\pgfpathcurveto{\pgfqpoint{1.427679in}{1.205117in}}{\pgfqpoint{1.429875in}{1.199817in}}{\pgfqpoint{1.433781in}{1.195910in}}%
\pgfpathcurveto{\pgfqpoint{1.437688in}{1.192004in}}{\pgfqpoint{1.442988in}{1.189808in}}{\pgfqpoint{1.448513in}{1.189808in}}%
\pgfpathclose%
\pgfusepath{stroke,fill}%
\end{pgfscope}%
\begin{pgfscope}%
\pgfpathrectangle{\pgfqpoint{0.403630in}{0.251348in}}{\pgfqpoint{1.395000in}{1.359000in}}%
\pgfusepath{clip}%
\pgfsetbuttcap%
\pgfsetroundjoin%
\definecolor{currentfill}{rgb}{0.686275,0.352941,0.313725}%
\pgfsetfillcolor{currentfill}%
\pgfsetlinewidth{0.501875pt}%
\definecolor{currentstroke}{rgb}{0.686275,0.352941,0.313725}%
\pgfsetstrokecolor{currentstroke}%
\pgfsetdash{}{0pt}%
\pgfpathmoveto{\pgfqpoint{1.251572in}{0.507644in}}%
\pgfpathcurveto{\pgfqpoint{1.257097in}{0.507644in}}{\pgfqpoint{1.262396in}{0.509839in}}{\pgfqpoint{1.266303in}{0.513746in}}%
\pgfpathcurveto{\pgfqpoint{1.270210in}{0.517652in}}{\pgfqpoint{1.272405in}{0.522952in}}{\pgfqpoint{1.272405in}{0.528477in}}%
\pgfpathcurveto{\pgfqpoint{1.272405in}{0.534002in}}{\pgfqpoint{1.270210in}{0.539302in}}{\pgfqpoint{1.266303in}{0.543208in}}%
\pgfpathcurveto{\pgfqpoint{1.262396in}{0.547115in}}{\pgfqpoint{1.257097in}{0.549310in}}{\pgfqpoint{1.251572in}{0.549310in}}%
\pgfpathcurveto{\pgfqpoint{1.246047in}{0.549310in}}{\pgfqpoint{1.240747in}{0.547115in}}{\pgfqpoint{1.236840in}{0.543208in}}%
\pgfpathcurveto{\pgfqpoint{1.232933in}{0.539302in}}{\pgfqpoint{1.230738in}{0.534002in}}{\pgfqpoint{1.230738in}{0.528477in}}%
\pgfpathcurveto{\pgfqpoint{1.230738in}{0.522952in}}{\pgfqpoint{1.232933in}{0.517652in}}{\pgfqpoint{1.236840in}{0.513746in}}%
\pgfpathcurveto{\pgfqpoint{1.240747in}{0.509839in}}{\pgfqpoint{1.246047in}{0.507644in}}{\pgfqpoint{1.251572in}{0.507644in}}%
\pgfpathclose%
\pgfusepath{stroke,fill}%
\end{pgfscope}%
\begin{pgfscope}%
\pgfpathrectangle{\pgfqpoint{0.403630in}{0.251348in}}{\pgfqpoint{1.395000in}{1.359000in}}%
\pgfusepath{clip}%
\pgfsetbuttcap%
\pgfsetroundjoin%
\definecolor{currentfill}{rgb}{0.686275,0.352941,0.313725}%
\pgfsetfillcolor{currentfill}%
\pgfsetlinewidth{0.501875pt}%
\definecolor{currentstroke}{rgb}{0.686275,0.352941,0.313725}%
\pgfsetstrokecolor{currentstroke}%
\pgfsetdash{}{0pt}%
\pgfpathmoveto{\pgfqpoint{1.503219in}{1.168491in}}%
\pgfpathcurveto{\pgfqpoint{1.508744in}{1.168491in}}{\pgfqpoint{1.514043in}{1.170686in}}{\pgfqpoint{1.517950in}{1.174593in}}%
\pgfpathcurveto{\pgfqpoint{1.521857in}{1.178500in}}{\pgfqpoint{1.524052in}{1.183799in}}{\pgfqpoint{1.524052in}{1.189324in}}%
\pgfpathcurveto{\pgfqpoint{1.524052in}{1.194849in}}{\pgfqpoint{1.521857in}{1.200149in}}{\pgfqpoint{1.517950in}{1.204056in}}%
\pgfpathcurveto{\pgfqpoint{1.514043in}{1.207962in}}{\pgfqpoint{1.508744in}{1.210157in}}{\pgfqpoint{1.503219in}{1.210157in}}%
\pgfpathcurveto{\pgfqpoint{1.497694in}{1.210157in}}{\pgfqpoint{1.492394in}{1.207962in}}{\pgfqpoint{1.488487in}{1.204056in}}%
\pgfpathcurveto{\pgfqpoint{1.484580in}{1.200149in}}{\pgfqpoint{1.482385in}{1.194849in}}{\pgfqpoint{1.482385in}{1.189324in}}%
\pgfpathcurveto{\pgfqpoint{1.482385in}{1.183799in}}{\pgfqpoint{1.484580in}{1.178500in}}{\pgfqpoint{1.488487in}{1.174593in}}%
\pgfpathcurveto{\pgfqpoint{1.492394in}{1.170686in}}{\pgfqpoint{1.497694in}{1.168491in}}{\pgfqpoint{1.503219in}{1.168491in}}%
\pgfpathclose%
\pgfusepath{stroke,fill}%
\end{pgfscope}%
\begin{pgfscope}%
\pgfpathrectangle{\pgfqpoint{0.403630in}{0.251348in}}{\pgfqpoint{1.395000in}{1.359000in}}%
\pgfusepath{clip}%
\pgfsetbuttcap%
\pgfsetroundjoin%
\definecolor{currentfill}{rgb}{0.686275,0.352941,0.313725}%
\pgfsetfillcolor{currentfill}%
\pgfsetlinewidth{0.501875pt}%
\definecolor{currentstroke}{rgb}{0.686275,0.352941,0.313725}%
\pgfsetstrokecolor{currentstroke}%
\pgfsetdash{}{0pt}%
\pgfpathmoveto{\pgfqpoint{1.366454in}{0.555608in}}%
\pgfpathcurveto{\pgfqpoint{1.371979in}{0.555608in}}{\pgfqpoint{1.377279in}{0.557804in}}{\pgfqpoint{1.381185in}{0.561710in}}%
\pgfpathcurveto{\pgfqpoint{1.385092in}{0.565617in}}{\pgfqpoint{1.387287in}{0.570917in}}{\pgfqpoint{1.387287in}{0.576442in}}%
\pgfpathcurveto{\pgfqpoint{1.387287in}{0.581967in}}{\pgfqpoint{1.385092in}{0.587266in}}{\pgfqpoint{1.381185in}{0.591173in}}%
\pgfpathcurveto{\pgfqpoint{1.377279in}{0.595080in}}{\pgfqpoint{1.371979in}{0.597275in}}{\pgfqpoint{1.366454in}{0.597275in}}%
\pgfpathcurveto{\pgfqpoint{1.360929in}{0.597275in}}{\pgfqpoint{1.355629in}{0.595080in}}{\pgfqpoint{1.351723in}{0.591173in}}%
\pgfpathcurveto{\pgfqpoint{1.347816in}{0.587266in}}{\pgfqpoint{1.345621in}{0.581967in}}{\pgfqpoint{1.345621in}{0.576442in}}%
\pgfpathcurveto{\pgfqpoint{1.345621in}{0.570917in}}{\pgfqpoint{1.347816in}{0.565617in}}{\pgfqpoint{1.351723in}{0.561710in}}%
\pgfpathcurveto{\pgfqpoint{1.355629in}{0.557804in}}{\pgfqpoint{1.360929in}{0.555608in}}{\pgfqpoint{1.366454in}{0.555608in}}%
\pgfpathclose%
\pgfusepath{stroke,fill}%
\end{pgfscope}%
\begin{pgfscope}%
\pgfpathrectangle{\pgfqpoint{0.403630in}{0.251348in}}{\pgfqpoint{1.395000in}{1.359000in}}%
\pgfusepath{clip}%
\pgfsetbuttcap%
\pgfsetroundjoin%
\definecolor{currentfill}{rgb}{0.686275,0.352941,0.313725}%
\pgfsetfillcolor{currentfill}%
\pgfsetlinewidth{0.501875pt}%
\definecolor{currentstroke}{rgb}{0.686275,0.352941,0.313725}%
\pgfsetstrokecolor{currentstroke}%
\pgfsetdash{}{0pt}%
\pgfpathmoveto{\pgfqpoint{0.950689in}{1.328373in}}%
\pgfpathcurveto{\pgfqpoint{0.956214in}{1.328373in}}{\pgfqpoint{0.961514in}{1.330568in}}{\pgfqpoint{0.965421in}{1.334475in}}%
\pgfpathcurveto{\pgfqpoint{0.969327in}{1.338382in}}{\pgfqpoint{0.971523in}{1.343681in}}{\pgfqpoint{0.971523in}{1.349206in}}%
\pgfpathcurveto{\pgfqpoint{0.971523in}{1.354732in}}{\pgfqpoint{0.969327in}{1.360031in}}{\pgfqpoint{0.965421in}{1.363938in}}%
\pgfpathcurveto{\pgfqpoint{0.961514in}{1.367845in}}{\pgfqpoint{0.956214in}{1.370040in}}{\pgfqpoint{0.950689in}{1.370040in}}%
\pgfpathcurveto{\pgfqpoint{0.945164in}{1.370040in}}{\pgfqpoint{0.939865in}{1.367845in}}{\pgfqpoint{0.935958in}{1.363938in}}%
\pgfpathcurveto{\pgfqpoint{0.932051in}{1.360031in}}{\pgfqpoint{0.929856in}{1.354732in}}{\pgfqpoint{0.929856in}{1.349206in}}%
\pgfpathcurveto{\pgfqpoint{0.929856in}{1.343681in}}{\pgfqpoint{0.932051in}{1.338382in}}{\pgfqpoint{0.935958in}{1.334475in}}%
\pgfpathcurveto{\pgfqpoint{0.939865in}{1.330568in}}{\pgfqpoint{0.945164in}{1.328373in}}{\pgfqpoint{0.950689in}{1.328373in}}%
\pgfpathclose%
\pgfusepath{stroke,fill}%
\end{pgfscope}%
\begin{pgfscope}%
\pgfpathrectangle{\pgfqpoint{0.403630in}{0.251348in}}{\pgfqpoint{1.395000in}{1.359000in}}%
\pgfusepath{clip}%
\pgfsetbuttcap%
\pgfsetroundjoin%
\definecolor{currentfill}{rgb}{0.686275,0.352941,0.313725}%
\pgfsetfillcolor{currentfill}%
\pgfsetlinewidth{0.501875pt}%
\definecolor{currentstroke}{rgb}{0.686275,0.352941,0.313725}%
\pgfsetstrokecolor{currentstroke}%
\pgfsetdash{}{0pt}%
\pgfpathmoveto{\pgfqpoint{0.775630in}{0.555608in}}%
\pgfpathcurveto{\pgfqpoint{0.781155in}{0.555608in}}{\pgfqpoint{0.786455in}{0.557804in}}{\pgfqpoint{0.790362in}{0.561710in}}%
\pgfpathcurveto{\pgfqpoint{0.794269in}{0.565617in}}{\pgfqpoint{0.796464in}{0.570917in}}{\pgfqpoint{0.796464in}{0.576442in}}%
\pgfpathcurveto{\pgfqpoint{0.796464in}{0.581967in}}{\pgfqpoint{0.794269in}{0.587266in}}{\pgfqpoint{0.790362in}{0.591173in}}%
\pgfpathcurveto{\pgfqpoint{0.786455in}{0.595080in}}{\pgfqpoint{0.781155in}{0.597275in}}{\pgfqpoint{0.775630in}{0.597275in}}%
\pgfpathcurveto{\pgfqpoint{0.770105in}{0.597275in}}{\pgfqpoint{0.764806in}{0.595080in}}{\pgfqpoint{0.760899in}{0.591173in}}%
\pgfpathcurveto{\pgfqpoint{0.756992in}{0.587266in}}{\pgfqpoint{0.754797in}{0.581967in}}{\pgfqpoint{0.754797in}{0.576442in}}%
\pgfpathcurveto{\pgfqpoint{0.754797in}{0.570917in}}{\pgfqpoint{0.756992in}{0.565617in}}{\pgfqpoint{0.760899in}{0.561710in}}%
\pgfpathcurveto{\pgfqpoint{0.764806in}{0.557804in}}{\pgfqpoint{0.770105in}{0.555608in}}{\pgfqpoint{0.775630in}{0.555608in}}%
\pgfpathclose%
\pgfusepath{stroke,fill}%
\end{pgfscope}%
\begin{pgfscope}%
\pgfpathrectangle{\pgfqpoint{0.403630in}{0.251348in}}{\pgfqpoint{1.395000in}{1.359000in}}%
\pgfusepath{clip}%
\pgfsetbuttcap%
\pgfsetroundjoin%
\definecolor{currentfill}{rgb}{0.686275,0.352941,0.313725}%
\pgfsetfillcolor{currentfill}%
\pgfsetlinewidth{0.501875pt}%
\definecolor{currentstroke}{rgb}{0.686275,0.352941,0.313725}%
\pgfsetstrokecolor{currentstroke}%
\pgfsetdash{}{0pt}%
\pgfpathmoveto{\pgfqpoint{1.317219in}{0.507644in}}%
\pgfpathcurveto{\pgfqpoint{1.322744in}{0.507644in}}{\pgfqpoint{1.328043in}{0.509839in}}{\pgfqpoint{1.331950in}{0.513746in}}%
\pgfpathcurveto{\pgfqpoint{1.335857in}{0.517652in}}{\pgfqpoint{1.338052in}{0.522952in}}{\pgfqpoint{1.338052in}{0.528477in}}%
\pgfpathcurveto{\pgfqpoint{1.338052in}{0.534002in}}{\pgfqpoint{1.335857in}{0.539302in}}{\pgfqpoint{1.331950in}{0.543208in}}%
\pgfpathcurveto{\pgfqpoint{1.328043in}{0.547115in}}{\pgfqpoint{1.322744in}{0.549310in}}{\pgfqpoint{1.317219in}{0.549310in}}%
\pgfpathcurveto{\pgfqpoint{1.311694in}{0.549310in}}{\pgfqpoint{1.306394in}{0.547115in}}{\pgfqpoint{1.302487in}{0.543208in}}%
\pgfpathcurveto{\pgfqpoint{1.298580in}{0.539302in}}{\pgfqpoint{1.296385in}{0.534002in}}{\pgfqpoint{1.296385in}{0.528477in}}%
\pgfpathcurveto{\pgfqpoint{1.296385in}{0.522952in}}{\pgfqpoint{1.298580in}{0.517652in}}{\pgfqpoint{1.302487in}{0.513746in}}%
\pgfpathcurveto{\pgfqpoint{1.306394in}{0.509839in}}{\pgfqpoint{1.311694in}{0.507644in}}{\pgfqpoint{1.317219in}{0.507644in}}%
\pgfpathclose%
\pgfusepath{stroke,fill}%
\end{pgfscope}%
\begin{pgfscope}%
\pgfpathrectangle{\pgfqpoint{0.403630in}{0.251348in}}{\pgfqpoint{1.395000in}{1.359000in}}%
\pgfusepath{clip}%
\pgfsetbuttcap%
\pgfsetroundjoin%
\definecolor{currentfill}{rgb}{0.686275,0.352941,0.313725}%
\pgfsetfillcolor{currentfill}%
\pgfsetlinewidth{0.501875pt}%
\definecolor{currentstroke}{rgb}{0.686275,0.352941,0.313725}%
\pgfsetstrokecolor{currentstroke}%
\pgfsetdash{}{0pt}%
\pgfpathmoveto{\pgfqpoint{1.443042in}{1.269750in}}%
\pgfpathcurveto{\pgfqpoint{1.448567in}{1.269750in}}{\pgfqpoint{1.453867in}{1.271945in}}{\pgfqpoint{1.457774in}{1.275852in}}%
\pgfpathcurveto{\pgfqpoint{1.461680in}{1.279758in}}{\pgfqpoint{1.463876in}{1.285058in}}{\pgfqpoint{1.463876in}{1.290583in}}%
\pgfpathcurveto{\pgfqpoint{1.463876in}{1.296108in}}{\pgfqpoint{1.461680in}{1.301408in}}{\pgfqpoint{1.457774in}{1.305314in}}%
\pgfpathcurveto{\pgfqpoint{1.453867in}{1.309221in}}{\pgfqpoint{1.448567in}{1.311416in}}{\pgfqpoint{1.443042in}{1.311416in}}%
\pgfpathcurveto{\pgfqpoint{1.437517in}{1.311416in}}{\pgfqpoint{1.432218in}{1.309221in}}{\pgfqpoint{1.428311in}{1.305314in}}%
\pgfpathcurveto{\pgfqpoint{1.424404in}{1.301408in}}{\pgfqpoint{1.422209in}{1.296108in}}{\pgfqpoint{1.422209in}{1.290583in}}%
\pgfpathcurveto{\pgfqpoint{1.422209in}{1.285058in}}{\pgfqpoint{1.424404in}{1.279758in}}{\pgfqpoint{1.428311in}{1.275852in}}%
\pgfpathcurveto{\pgfqpoint{1.432218in}{1.271945in}}{\pgfqpoint{1.437517in}{1.269750in}}{\pgfqpoint{1.443042in}{1.269750in}}%
\pgfpathclose%
\pgfusepath{stroke,fill}%
\end{pgfscope}%
\begin{pgfscope}%
\pgfpathrectangle{\pgfqpoint{0.403630in}{0.251348in}}{\pgfqpoint{1.395000in}{1.359000in}}%
\pgfusepath{clip}%
\pgfsetbuttcap%
\pgfsetroundjoin%
\definecolor{currentfill}{rgb}{0.686275,0.352941,0.313725}%
\pgfsetfillcolor{currentfill}%
\pgfsetlinewidth{0.501875pt}%
\definecolor{currentstroke}{rgb}{0.686275,0.352941,0.313725}%
\pgfsetstrokecolor{currentstroke}%
\pgfsetdash{}{0pt}%
\pgfpathmoveto{\pgfqpoint{1.339101in}{0.480997in}}%
\pgfpathcurveto{\pgfqpoint{1.344626in}{0.480997in}}{\pgfqpoint{1.349926in}{0.483192in}}{\pgfqpoint{1.353832in}{0.487099in}}%
\pgfpathcurveto{\pgfqpoint{1.357739in}{0.491005in}}{\pgfqpoint{1.359934in}{0.496305in}}{\pgfqpoint{1.359934in}{0.501830in}}%
\pgfpathcurveto{\pgfqpoint{1.359934in}{0.507355in}}{\pgfqpoint{1.357739in}{0.512655in}}{\pgfqpoint{1.353832in}{0.516561in}}%
\pgfpathcurveto{\pgfqpoint{1.349926in}{0.520468in}}{\pgfqpoint{1.344626in}{0.522663in}}{\pgfqpoint{1.339101in}{0.522663in}}%
\pgfpathcurveto{\pgfqpoint{1.333576in}{0.522663in}}{\pgfqpoint{1.328276in}{0.520468in}}{\pgfqpoint{1.324370in}{0.516561in}}%
\pgfpathcurveto{\pgfqpoint{1.320463in}{0.512655in}}{\pgfqpoint{1.318268in}{0.507355in}}{\pgfqpoint{1.318268in}{0.501830in}}%
\pgfpathcurveto{\pgfqpoint{1.318268in}{0.496305in}}{\pgfqpoint{1.320463in}{0.491005in}}{\pgfqpoint{1.324370in}{0.487099in}}%
\pgfpathcurveto{\pgfqpoint{1.328276in}{0.483192in}}{\pgfqpoint{1.333576in}{0.480997in}}{\pgfqpoint{1.339101in}{0.480997in}}%
\pgfpathclose%
\pgfusepath{stroke,fill}%
\end{pgfscope}%
\begin{pgfscope}%
\pgfpathrectangle{\pgfqpoint{0.403630in}{0.251348in}}{\pgfqpoint{1.395000in}{1.359000in}}%
\pgfusepath{clip}%
\pgfsetbuttcap%
\pgfsetroundjoin%
\definecolor{currentfill}{rgb}{0.686275,0.352941,0.313725}%
\pgfsetfillcolor{currentfill}%
\pgfsetlinewidth{0.501875pt}%
\definecolor{currentstroke}{rgb}{0.686275,0.352941,0.313725}%
\pgfsetstrokecolor{currentstroke}%
\pgfsetdash{}{0pt}%
\pgfpathmoveto{\pgfqpoint{0.709983in}{1.227114in}}%
\pgfpathcurveto{\pgfqpoint{0.715508in}{1.227114in}}{\pgfqpoint{0.720808in}{1.229309in}}{\pgfqpoint{0.724715in}{1.233216in}}%
\pgfpathcurveto{\pgfqpoint{0.728622in}{1.237123in}}{\pgfqpoint{0.730817in}{1.242423in}}{\pgfqpoint{0.730817in}{1.247948in}}%
\pgfpathcurveto{\pgfqpoint{0.730817in}{1.253473in}}{\pgfqpoint{0.728622in}{1.258772in}}{\pgfqpoint{0.724715in}{1.262679in}}%
\pgfpathcurveto{\pgfqpoint{0.720808in}{1.266586in}}{\pgfqpoint{0.715508in}{1.268781in}}{\pgfqpoint{0.709983in}{1.268781in}}%
\pgfpathcurveto{\pgfqpoint{0.704458in}{1.268781in}}{\pgfqpoint{0.699159in}{1.266586in}}{\pgfqpoint{0.695252in}{1.262679in}}%
\pgfpathcurveto{\pgfqpoint{0.691345in}{1.258772in}}{\pgfqpoint{0.689150in}{1.253473in}}{\pgfqpoint{0.689150in}{1.247948in}}%
\pgfpathcurveto{\pgfqpoint{0.689150in}{1.242423in}}{\pgfqpoint{0.691345in}{1.237123in}}{\pgfqpoint{0.695252in}{1.233216in}}%
\pgfpathcurveto{\pgfqpoint{0.699159in}{1.229309in}}{\pgfqpoint{0.704458in}{1.227114in}}{\pgfqpoint{0.709983in}{1.227114in}}%
\pgfpathclose%
\pgfusepath{stroke,fill}%
\end{pgfscope}%
\begin{pgfscope}%
\pgfpathrectangle{\pgfqpoint{0.403630in}{0.251348in}}{\pgfqpoint{1.395000in}{1.359000in}}%
\pgfusepath{clip}%
\pgfsetbuttcap%
\pgfsetmiterjoin%
\pgfsetlinewidth{0.501875pt}%
\definecolor{currentstroke}{rgb}{0.000000,0.356863,0.509804}%
\pgfsetstrokecolor{currentstroke}%
\pgfsetdash{}{0pt}%
\pgfpathmoveto{\pgfqpoint{1.103866in}{0.613748in}}%
\pgfpathcurveto{\pgfqpoint{1.190915in}{0.613748in}}{\pgfqpoint{1.274410in}{0.647440in}}{\pgfqpoint{1.335963in}{0.707405in}}%
\pgfpathcurveto{\pgfqpoint{1.397516in}{0.767369in}}{\pgfqpoint{1.432101in}{0.848710in}}{\pgfqpoint{1.432101in}{0.933512in}}%
\pgfpathcurveto{\pgfqpoint{1.432101in}{1.018315in}}{\pgfqpoint{1.397516in}{1.099656in}}{\pgfqpoint{1.335963in}{1.159620in}}%
\pgfpathcurveto{\pgfqpoint{1.274410in}{1.219585in}}{\pgfqpoint{1.190915in}{1.253277in}}{\pgfqpoint{1.103866in}{1.253277in}}%
\pgfpathcurveto{\pgfqpoint{1.016817in}{1.253277in}}{\pgfqpoint{0.933321in}{1.219585in}}{\pgfqpoint{0.871768in}{1.159620in}}%
\pgfpathcurveto{\pgfqpoint{0.810215in}{1.099656in}}{\pgfqpoint{0.775630in}{1.018315in}}{\pgfqpoint{0.775630in}{0.933512in}}%
\pgfpathcurveto{\pgfqpoint{0.775630in}{0.848710in}}{\pgfqpoint{0.810215in}{0.767369in}}{\pgfqpoint{0.871768in}{0.707405in}}%
\pgfpathcurveto{\pgfqpoint{0.933321in}{0.647440in}}{\pgfqpoint{1.016817in}{0.613748in}}{\pgfqpoint{1.103866in}{0.613748in}}%
\pgfpathclose%
\pgfusepath{stroke}%
\end{pgfscope}%
\begin{pgfscope}%
\pgfsetrectcap%
\pgfsetmiterjoin%
\pgfsetlinewidth{0.501875pt}%
\definecolor{currentstroke}{rgb}{0.317647,0.317647,0.317647}%
\pgfsetstrokecolor{currentstroke}%
\pgfsetdash{}{0pt}%
\pgfpathmoveto{\pgfqpoint{0.403630in}{0.251348in}}%
\pgfpathlineto{\pgfqpoint{0.403630in}{1.610348in}}%
\pgfusepath{stroke}%
\end{pgfscope}%
\begin{pgfscope}%
\pgfsetrectcap%
\pgfsetmiterjoin%
\pgfsetlinewidth{0.501875pt}%
\definecolor{currentstroke}{rgb}{0.317647,0.317647,0.317647}%
\pgfsetstrokecolor{currentstroke}%
\pgfsetdash{}{0pt}%
\pgfpathmoveto{\pgfqpoint{0.403630in}{0.251348in}}%
\pgfpathlineto{\pgfqpoint{1.798630in}{0.251348in}}%
\pgfusepath{stroke}%
\end{pgfscope}%
\begin{pgfscope}%
\pgfsetbuttcap%
\pgfsetroundjoin%
\definecolor{currentfill}{rgb}{0.333333,0.333333,0.333333}%
\pgfsetfillcolor{currentfill}%
\pgfsetlinewidth{0.501875pt}%
\definecolor{currentstroke}{rgb}{0.333333,0.333333,0.333333}%
\pgfsetstrokecolor{currentstroke}%
\pgfsetdash{}{0pt}%
\pgfpathmoveto{\pgfqpoint{1.600617in}{1.504190in}}%
\pgfpathcurveto{\pgfqpoint{1.606142in}{1.504190in}}{\pgfqpoint{1.611441in}{1.506385in}}{\pgfqpoint{1.615348in}{1.510292in}}%
\pgfpathcurveto{\pgfqpoint{1.619255in}{1.514198in}}{\pgfqpoint{1.621450in}{1.519498in}}{\pgfqpoint{1.621450in}{1.525023in}}%
\pgfpathcurveto{\pgfqpoint{1.621450in}{1.530548in}}{\pgfqpoint{1.619255in}{1.535848in}}{\pgfqpoint{1.615348in}{1.539754in}}%
\pgfpathcurveto{\pgfqpoint{1.611441in}{1.543661in}}{\pgfqpoint{1.606142in}{1.545856in}}{\pgfqpoint{1.600617in}{1.545856in}}%
\pgfpathcurveto{\pgfqpoint{1.595091in}{1.545856in}}{\pgfqpoint{1.589792in}{1.543661in}}{\pgfqpoint{1.585885in}{1.539754in}}%
\pgfpathcurveto{\pgfqpoint{1.581978in}{1.535848in}}{\pgfqpoint{1.579783in}{1.530548in}}{\pgfqpoint{1.579783in}{1.525023in}}%
\pgfpathcurveto{\pgfqpoint{1.579783in}{1.519498in}}{\pgfqpoint{1.581978in}{1.514198in}}{\pgfqpoint{1.585885in}{1.510292in}}%
\pgfpathcurveto{\pgfqpoint{1.589792in}{1.506385in}}{\pgfqpoint{1.595091in}{1.504190in}}{\pgfqpoint{1.600617in}{1.504190in}}%
\pgfpathclose%
\pgfusepath{stroke,fill}%
\end{pgfscope}%
\begin{pgfscope}%
\definecolor{textcolor}{rgb}{0.000000,0.000000,0.000000}%
\pgfsetstrokecolor{textcolor}%
\pgfsetfillcolor{textcolor}%
\pgftext[x=1.704742in,y=1.494653in,left,base]{\color{textcolor}\rmfamily\fontsize{8.330000}{9.996000}\selectfont class 0}%
\end{pgfscope}%
\begin{pgfscope}%
\pgfsetbuttcap%
\pgfsetroundjoin%
\definecolor{currentfill}{rgb}{0.686275,0.352941,0.313725}%
\pgfsetfillcolor{currentfill}%
\pgfsetlinewidth{0.501875pt}%
\definecolor{currentstroke}{rgb}{0.686275,0.352941,0.313725}%
\pgfsetstrokecolor{currentstroke}%
\pgfsetdash{}{0pt}%
\pgfpathmoveto{\pgfqpoint{1.600617in}{1.354481in}}%
\pgfpathcurveto{\pgfqpoint{1.606142in}{1.354481in}}{\pgfqpoint{1.611441in}{1.356676in}}{\pgfqpoint{1.615348in}{1.360583in}}%
\pgfpathcurveto{\pgfqpoint{1.619255in}{1.364490in}}{\pgfqpoint{1.621450in}{1.369789in}}{\pgfqpoint{1.621450in}{1.375315in}}%
\pgfpathcurveto{\pgfqpoint{1.621450in}{1.380840in}}{\pgfqpoint{1.619255in}{1.386139in}}{\pgfqpoint{1.615348in}{1.390046in}}%
\pgfpathcurveto{\pgfqpoint{1.611441in}{1.393953in}}{\pgfqpoint{1.606142in}{1.396148in}}{\pgfqpoint{1.600617in}{1.396148in}}%
\pgfpathcurveto{\pgfqpoint{1.595091in}{1.396148in}}{\pgfqpoint{1.589792in}{1.393953in}}{\pgfqpoint{1.585885in}{1.390046in}}%
\pgfpathcurveto{\pgfqpoint{1.581978in}{1.386139in}}{\pgfqpoint{1.579783in}{1.380840in}}{\pgfqpoint{1.579783in}{1.375315in}}%
\pgfpathcurveto{\pgfqpoint{1.579783in}{1.369789in}}{\pgfqpoint{1.581978in}{1.364490in}}{\pgfqpoint{1.585885in}{1.360583in}}%
\pgfpathcurveto{\pgfqpoint{1.589792in}{1.356676in}}{\pgfqpoint{1.595091in}{1.354481in}}{\pgfqpoint{1.600617in}{1.354481in}}%
\pgfpathclose%
\pgfusepath{stroke,fill}%
\end{pgfscope}%
\begin{pgfscope}%
\definecolor{textcolor}{rgb}{0.000000,0.000000,0.000000}%
\pgfsetstrokecolor{textcolor}%
\pgfsetfillcolor{textcolor}%
\pgftext[x=1.704742in,y=1.344945in,left,base]{\color{textcolor}\rmfamily\fontsize{8.330000}{9.996000}\selectfont class 1}%
\end{pgfscope}%
\end{pgfpicture}%
\makeatother%
\endgroup%

	\caption[Circles data set.]{The Circles data set has two classes (\textit{red} and \textit{black}) which can be separated by a decision boundary (\textit{blue}).} 
	\label{circlestasksketch}
	\vspace{-1cm}
\end{wrapfigure}
The Circles data set describes a set of points $p = p(x,y)$ in a two-dimensional plane, which are in either of two disjunct rings, each representing a class
\begin{align}
\text{class}(p) =
\begin{cases}
0 ,&\quad \quad r_{\text{inner}}^2 < x^2 + y^2 < r_{\text{outer}}^2, \\
1 ,&\quad \quad R_{\text{inner}}^2 < x^2 + y^2 < R_{\text{outer}}^2,
\end{cases}
\end{align}
with the confining inner and outer radii of the first and second ring $r_{\text{inner}}$, $r_{\text{outer}}$ and  $R_{\text{inner}}$, $R_{\text{outer}}$ respectively. The goal of the task is to find a decision boundary that successfully separates the both rings as shown in \cref{circlestasksketch}.

The Circles task is slightly modified to reduce the implementation effort on the \gls{ppu} and to make it easier to find a decision boundary. The class $0$ is reduced to one confining circle by setting the inner radius to zero. The outer radius of the second class is replaced by the \SI{8}{\bit} limitations of the input. The remaining two boundaries are set far apart to $r_{\text{outer}} = \sqrt{8000}$ and $R_{\text{inner}} = \sqrt{13000}$. The area in between the classes is where the network should place the decision boundary. The more space is available, the easier it is for the network to succeed. 

\begin{figure}[b!]
	\begin{subfigure}[c]{0.5\textwidth}
		\centering
		%		\caption{}
		\inputpgf{figures}{nu_x_input.pgf}
		%		\label{nuxinput}
	\end{subfigure}
	\begin{subfigure}[c]{0.5\textwidth}
		\centering
		%		\caption{}
		\inputpgf{figures}{nu_y_input.pgf}	
		%		\label{nuyinput}
	\end{subfigure}
	\caption[Input coding for the Circles experiment.]{Input coding for the Circles experiment. Each point $p(x,y)$ of a representative data set with 100 points per class is mapped to two input frequencies each depending on one of the coordinates. In the left figure, the $x$-dependency is shown and to the right the $y$-dependency.}
	\label{circlesinputs}
\end{figure}

With a rate-based coding, the data points need to be translated into firing rates first. A point $p$ is represented by two signed \SI{8}{\bit} integers which correspond to the coordinates $x$ and $y$ with $x, y \in [-128,127]$. The corresponding firing rate of a coordinate $c \in \{x,y\}$ is then given by
\begin{equation}\label{inputfrequency}
\nu_{\text{input, c}}(c) = \nu_\text{max} \cdot \frac{c + 128}{255},
\end{equation}
with the maximum firing rate $\nu_\text{max}$ set by a respective spike train source. A representative data set of 100 samples per class illustrates in \cref{circlesinputs} how the points of the modified task are distributed. Moreover, the translation of a point into an input rate is shown. For instance, a point from the lower right corner p(100,100) is associated with two input rates: one depends on the x coordinate $\nu_{\text{in, x}}(100) \approx \SI{450}{\kilo \Hz}$ and one on the y coordinate $\nu_{\text{in, y}}(-100) \approx \SI{55}{\kilo \Hz}$.

As for a supervised learning method, the input data is labeled. To identify the class of an input pattern in the output layer, a single readout neuron is trained to yield a target rate $\nu^*$
\begin{align}
\nu^*(p) =
\begin{cases}
\nu_0^* = \SI{32.6}{\kilo \Hz} ,&\quad \quad \text{class}(p) = 0,\\
\nu_1^* = \SI{93.5}{\kilo \Hz} ,&\quad \quad \text{class}(p) = 1.
\end{cases}
\label{circlestarget}
\end{align}
The mismatch between the target rate and firing rate of the readout neuron \gls{nuout} then determines the error of the output layer $e^{(o)}$
\begin{equation*}
e^{(o)}(p) = \gls{nuout}(p) - \nu^*(p).
\label{circleserror}
\end{equation*}
The decision boundary, the rate separating both classes, is chosen to be the mean of both target rates
\begin{equation*}
\nu_\text{DB} = \frac{\nu_0^* + \nu_1^*}{2} = \SI{63.05}{\kilo \Hz}.
\label{circlesdb}
\end{equation*}

%A reasonable choice of the target rates is crucial for the training success. If the rates are either too close or too far from each other, the network will have troubles to solve the task. 

\section{Poisson Spike Train Generator}
\label{poissonspiketrains}
The on-chip implementation of the experiment requires the generation of spikes for the input data and noise spikes for sigmoid activation function. The latter relies on spike trains with an underlying Poisson distribution of the spikes. With the lack of dedicated spike generators on-chip, the \gls{ppu} has to fulfill the task.

One way to numerically generate a Poisson spike train is to repeatedly perform a Bernoulli process on a short time interval $\Delta t$ over a period $T$. The Bernoulli process is equivalent to an unfair coin flip with probability $p$, which is set depending on the desired firing rate $\nu$ by $p(\nu) = \nu \cdot \Delta t$. Given that the probability equals one, i.e. at every time interval a spike will be fired, the firing rate reaches its maximum at $\nu_\text{max} = \nicefrac{1}{\Delta t}$. However, the spike train is no longer Poisson-based for high firing rates but rather fires with a fixed \glsfirst{isi}. Therefore, the spike generator will only be used at low frequencies for the noise generation ($\nu_\text{noise} \ll \nu_\text{max}$). The input spike trains on the other hand do not rely on pure Poisson spike trains and thus can use the full frequency range of the spike train generator.

With respect to hardware limitations, the maximum fire rate of the spike train generator $\nu_\text{max, ppu}$ is defined by the shortest possible \gls{isi} on the \gls{ppu} which is in turn limited by the time required to generate a single spike $\Delta t_\text{spike}$
\begin{align*}
\nu_\text{max, ppu} = \frac{1}{\Delta t_\text{spike}} = \frac{1}{\SI{0.44}{\mu \s}} = \SI{2.27}{\mega \Hz}.
\end{align*}
The duration of the spike generation $\Delta t_\text{spike}$ on the \gls{ppu} is determined by recording a constant spiking pattern with the highest possible frequency. The total number of spikes is counted by on-chip spike counters and the duration is simply measured by recording the membrane potential with an oscilloscope.

% Stopped here, rewrite again not schlüssig yet...
%The use case of the spike generator in the final setup is to generate noise spike trains for the sigmoid activation function and to translate the Circles data set into input rates. Only the latter requires high input rates up to $\SI{500}{\kilo \Hz}$, the noise rate is only in the order of a few $\si{\kilo \Hz}$.

%In theory the maximum fire rate of a \gls{lif} neuron $\nu_\text{max}$ is give by $\nu_\text{max} = \nicefrac{1}{\gls{refrac}}$.
%In the final experiment setup, the output rate of an individual neuron is further constraint by a chosen 8-bit resolution of the spike counts and a measurement period of a few milliseconds to a measurable rate in order of few $\SI{100}{\kilo \Hz}$.
%will be divided into four independent channels, each producing at max a fourth of $\nu_\text{max, hw}$.
On the \gls{ppu} the Bernoulli process is implemented by comparing the frequency dependent probability $p(\nu)$ with a randomly drawn number. A popular method to generate random numbers on a system with limited memory and computational resources is the \emph{xorshift} (\citealp{marsaglia2003xorshift}). A random number is thereby generated by repeatedly applying the XOR operator on a seed variable and a bit-shifted versions of itself. %The \emph{bit shift} of a variable $x$ to the left by $n$, written as $x\;{\scriptstyle<<}\;n$, corresponds to the multiplication by $2^n$. In analogy, a shift to the right, $x\;{\scriptstyle>>}\;n$, is equivalent to the division by $2^n$. 


%The xorshift can be implemented in C as follows.
%\begin{minted}{C}
%uint32_t xorshift32(uint32_t* seed)
%{
%*seed ^= *seed << 13;y
%*seed ^= *seed >> 17;
%*seed ^= *seed << 5;
%return *seed;
%}
%\end{minted}

\section{Activation Function on Chip}
The theoretical background of how the activation function of a \gls{lif} neuron becomes sigmoidal has been motivated in \cref{ratecoding}. The main tool to create such an activation function is the continuous stimulation of the neuron with Poisson noise spike trains. Thereby a broad Gaussian distribution of the membrane potential is established, which in turn yields a sigmoid activation function. In \cref{vleak_w_noise}, the recorded free membrane potential on the \gls{dls} is shown while being stimulated by excitatory and inhibitory noise spike trains with a frequency of \SI{70}{\kilo \Hz} and a synaptic strength of $w_\text{noise}= 15$. The overall activation function then depends on several parameters such as the synaptic weights and frequencies of the spike sources, the neuron potentials as well as the time constants \gls{tau_m} and \gls{refrac}. 

%The continuous stimulation with Poisson spike trains leads to a Gaussian free membrane potential distribution $f_{\gls{v_mem}}$ centered around the resting potential \gls{v_leak} as depicted in \cref{vleak_w_noise}). In a naive approach, the part of the distribution that exceeds a certain threshold potential correlates to the number of fired spikes. This neglects non vanishing effects from the fire dynamics of the membrane which have been investigated in more detail by \citealp{petrovici12phdthesis}. The impact of these dynamics can be reduced by the use of very short time constants for the synaptic input \gls{tau_syn} and the membrane \gls{tau_m}.
%
%However, despite the strongly simplified picture, this view still offers a correct intuition how the threshold and leak potential as well as the strength of the noise effect the free membrane potential and in turn change the shape of the transfer function: more noise leads to a broader distribution and thus a more gently incline of the sigmoid; synaptic input moves the distribution to either to a lower or higher mean value; moving the threshold corresponds to an additional bias term.
\begin{figure}
	\begin{center}
		%% Creator: Matplotlib, PGF backend
%%
%% To include the figure in your LaTeX document, write
%%   \input{<filename>.pgf}
%%
%% Make sure the required packages are loaded in your preamble
%%   \usepackage{pgf}
%%
%% Figures using additional raster images can only be included by \input if
%% they are in the same directory as the main LaTeX file. For loading figures
%% from other directories you can use the `import` package
%%   \usepackage{import}
%% and then include the figures with
%%   \import{<path to file>}{<filename>.pgf}
%%
%% Matplotlib used the following preamble
%%   \usepackage{amsmath} \usepackage{pifont} \usepackage{xcolor} \definecolor{green}{HTML}{467821} \definecolor{red}{HTML}{CF4457} \usepackage[detect-all]{siunitx}
%%   \usepackage{fontspec}
%%
\begingroup%
\makeatletter%
\begin{pgfpicture}%
\pgfpathrectangle{\pgfpointorigin}{\pgfqpoint{4.501905in}{2.793578in}}%
\pgfusepath{use as bounding box, clip}%
\begin{pgfscope}%
\pgfsetbuttcap%
\pgfsetmiterjoin%
\pgfsetlinewidth{0.000000pt}%
\definecolor{currentstroke}{rgb}{0.000000,0.000000,0.000000}%
\pgfsetstrokecolor{currentstroke}%
\pgfsetstrokeopacity{0.000000}%
\pgfsetdash{}{0pt}%
\pgfpathmoveto{\pgfqpoint{0.000000in}{0.000000in}}%
\pgfpathlineto{\pgfqpoint{4.501905in}{0.000000in}}%
\pgfpathlineto{\pgfqpoint{4.501905in}{2.793578in}}%
\pgfpathlineto{\pgfqpoint{0.000000in}{2.793578in}}%
\pgfpathclose%
\pgfusepath{}%
\end{pgfscope}%
\begin{pgfscope}%
\pgfsetbuttcap%
\pgfsetmiterjoin%
\pgfsetlinewidth{0.000000pt}%
\definecolor{currentstroke}{rgb}{0.000000,0.000000,0.000000}%
\pgfsetstrokecolor{currentstroke}%
\pgfsetstrokeopacity{0.000000}%
\pgfsetdash{}{0pt}%
\pgfpathmoveto{\pgfqpoint{0.526905in}{0.383578in}}%
\pgfpathlineto{\pgfqpoint{4.401905in}{0.383578in}}%
\pgfpathlineto{\pgfqpoint{4.401905in}{2.693578in}}%
\pgfpathlineto{\pgfqpoint{0.526905in}{2.693578in}}%
\pgfpathclose%
\pgfusepath{}%
\end{pgfscope}%
\begin{pgfscope}%
\pgfsetbuttcap%
\pgfsetroundjoin%
\definecolor{currentfill}{rgb}{0.317647,0.317647,0.317647}%
\pgfsetfillcolor{currentfill}%
\pgfsetlinewidth{0.501875pt}%
\definecolor{currentstroke}{rgb}{0.317647,0.317647,0.317647}%
\pgfsetstrokecolor{currentstroke}%
\pgfsetdash{}{0pt}%
\pgfsys@defobject{currentmarker}{\pgfqpoint{0.000000in}{-0.020833in}}{\pgfqpoint{0.000000in}{0.000000in}}{%
\pgfpathmoveto{\pgfqpoint{0.000000in}{0.000000in}}%
\pgfpathlineto{\pgfqpoint{0.000000in}{-0.020833in}}%
\pgfusepath{stroke,fill}%
}%
\begin{pgfscope}%
\pgfsys@transformshift{0.543128in}{0.383578in}%
\pgfsys@useobject{currentmarker}{}%
\end{pgfscope}%
\end{pgfscope}%
\begin{pgfscope}%
\definecolor{textcolor}{rgb}{0.317647,0.317647,0.317647}%
\pgfsetstrokecolor{textcolor}%
\pgfsetfillcolor{textcolor}%
\pgftext[x=0.543128in,y=0.334967in,,top]{\color{textcolor}\rmfamily\fontsize{6.664000}{7.996800}\selectfont \(\displaystyle 400\)}%
\end{pgfscope}%
\begin{pgfscope}%
\pgfsetbuttcap%
\pgfsetroundjoin%
\definecolor{currentfill}{rgb}{0.317647,0.317647,0.317647}%
\pgfsetfillcolor{currentfill}%
\pgfsetlinewidth{0.501875pt}%
\definecolor{currentstroke}{rgb}{0.317647,0.317647,0.317647}%
\pgfsetstrokecolor{currentstroke}%
\pgfsetdash{}{0pt}%
\pgfsys@defobject{currentmarker}{\pgfqpoint{0.000000in}{-0.020833in}}{\pgfqpoint{0.000000in}{0.000000in}}{%
\pgfpathmoveto{\pgfqpoint{0.000000in}{0.000000in}}%
\pgfpathlineto{\pgfqpoint{0.000000in}{-0.020833in}}%
\pgfusepath{stroke,fill}%
}%
\begin{pgfscope}%
\pgfsys@transformshift{1.228682in}{0.383578in}%
\pgfsys@useobject{currentmarker}{}%
\end{pgfscope}%
\end{pgfscope}%
\begin{pgfscope}%
\definecolor{textcolor}{rgb}{0.317647,0.317647,0.317647}%
\pgfsetstrokecolor{textcolor}%
\pgfsetfillcolor{textcolor}%
\pgftext[x=1.228682in,y=0.334967in,,top]{\color{textcolor}\rmfamily\fontsize{6.664000}{7.996800}\selectfont \(\displaystyle 420\)}%
\end{pgfscope}%
\begin{pgfscope}%
\pgfsetbuttcap%
\pgfsetroundjoin%
\definecolor{currentfill}{rgb}{0.317647,0.317647,0.317647}%
\pgfsetfillcolor{currentfill}%
\pgfsetlinewidth{0.501875pt}%
\definecolor{currentstroke}{rgb}{0.317647,0.317647,0.317647}%
\pgfsetstrokecolor{currentstroke}%
\pgfsetdash{}{0pt}%
\pgfsys@defobject{currentmarker}{\pgfqpoint{0.000000in}{-0.020833in}}{\pgfqpoint{0.000000in}{0.000000in}}{%
\pgfpathmoveto{\pgfqpoint{0.000000in}{0.000000in}}%
\pgfpathlineto{\pgfqpoint{0.000000in}{-0.020833in}}%
\pgfusepath{stroke,fill}%
}%
\begin{pgfscope}%
\pgfsys@transformshift{1.914237in}{0.383578in}%
\pgfsys@useobject{currentmarker}{}%
\end{pgfscope}%
\end{pgfscope}%
\begin{pgfscope}%
\definecolor{textcolor}{rgb}{0.317647,0.317647,0.317647}%
\pgfsetstrokecolor{textcolor}%
\pgfsetfillcolor{textcolor}%
\pgftext[x=1.914237in,y=0.334967in,,top]{\color{textcolor}\rmfamily\fontsize{6.664000}{7.996800}\selectfont \(\displaystyle 440\)}%
\end{pgfscope}%
\begin{pgfscope}%
\pgfsetbuttcap%
\pgfsetroundjoin%
\definecolor{currentfill}{rgb}{0.317647,0.317647,0.317647}%
\pgfsetfillcolor{currentfill}%
\pgfsetlinewidth{0.501875pt}%
\definecolor{currentstroke}{rgb}{0.317647,0.317647,0.317647}%
\pgfsetstrokecolor{currentstroke}%
\pgfsetdash{}{0pt}%
\pgfsys@defobject{currentmarker}{\pgfqpoint{0.000000in}{-0.020833in}}{\pgfqpoint{0.000000in}{0.000000in}}{%
\pgfpathmoveto{\pgfqpoint{0.000000in}{0.000000in}}%
\pgfpathlineto{\pgfqpoint{0.000000in}{-0.020833in}}%
\pgfusepath{stroke,fill}%
}%
\begin{pgfscope}%
\pgfsys@transformshift{2.599792in}{0.383578in}%
\pgfsys@useobject{currentmarker}{}%
\end{pgfscope}%
\end{pgfscope}%
\begin{pgfscope}%
\definecolor{textcolor}{rgb}{0.317647,0.317647,0.317647}%
\pgfsetstrokecolor{textcolor}%
\pgfsetfillcolor{textcolor}%
\pgftext[x=2.599792in,y=0.334967in,,top]{\color{textcolor}\rmfamily\fontsize{6.664000}{7.996800}\selectfont \(\displaystyle 460\)}%
\end{pgfscope}%
\begin{pgfscope}%
\pgfsetbuttcap%
\pgfsetroundjoin%
\definecolor{currentfill}{rgb}{0.317647,0.317647,0.317647}%
\pgfsetfillcolor{currentfill}%
\pgfsetlinewidth{0.501875pt}%
\definecolor{currentstroke}{rgb}{0.317647,0.317647,0.317647}%
\pgfsetstrokecolor{currentstroke}%
\pgfsetdash{}{0pt}%
\pgfsys@defobject{currentmarker}{\pgfqpoint{0.000000in}{-0.020833in}}{\pgfqpoint{0.000000in}{0.000000in}}{%
\pgfpathmoveto{\pgfqpoint{0.000000in}{0.000000in}}%
\pgfpathlineto{\pgfqpoint{0.000000in}{-0.020833in}}%
\pgfusepath{stroke,fill}%
}%
\begin{pgfscope}%
\pgfsys@transformshift{3.285346in}{0.383578in}%
\pgfsys@useobject{currentmarker}{}%
\end{pgfscope}%
\end{pgfscope}%
\begin{pgfscope}%
\definecolor{textcolor}{rgb}{0.317647,0.317647,0.317647}%
\pgfsetstrokecolor{textcolor}%
\pgfsetfillcolor{textcolor}%
\pgftext[x=3.285346in,y=0.334967in,,top]{\color{textcolor}\rmfamily\fontsize{6.664000}{7.996800}\selectfont \(\displaystyle 480\)}%
\end{pgfscope}%
\begin{pgfscope}%
\pgfsetbuttcap%
\pgfsetroundjoin%
\definecolor{currentfill}{rgb}{0.317647,0.317647,0.317647}%
\pgfsetfillcolor{currentfill}%
\pgfsetlinewidth{0.501875pt}%
\definecolor{currentstroke}{rgb}{0.317647,0.317647,0.317647}%
\pgfsetstrokecolor{currentstroke}%
\pgfsetdash{}{0pt}%
\pgfsys@defobject{currentmarker}{\pgfqpoint{0.000000in}{-0.020833in}}{\pgfqpoint{0.000000in}{0.000000in}}{%
\pgfpathmoveto{\pgfqpoint{0.000000in}{0.000000in}}%
\pgfpathlineto{\pgfqpoint{0.000000in}{-0.020833in}}%
\pgfusepath{stroke,fill}%
}%
\begin{pgfscope}%
\pgfsys@transformshift{3.970901in}{0.383578in}%
\pgfsys@useobject{currentmarker}{}%
\end{pgfscope}%
\end{pgfscope}%
\begin{pgfscope}%
\definecolor{textcolor}{rgb}{0.317647,0.317647,0.317647}%
\pgfsetstrokecolor{textcolor}%
\pgfsetfillcolor{textcolor}%
\pgftext[x=3.970901in,y=0.334967in,,top]{\color{textcolor}\rmfamily\fontsize{6.664000}{7.996800}\selectfont \(\displaystyle 500\)}%
\end{pgfscope}%
\begin{pgfscope}%
\definecolor{textcolor}{rgb}{0.317647,0.317647,0.317647}%
\pgfsetstrokecolor{textcolor}%
\pgfsetfillcolor{textcolor}%
\pgftext[x=2.464405in,y=0.197222in,,top]{\color{textcolor}\rmfamily\fontsize{6.664000}{7.996800}\selectfont \(\displaystyle V_\mathrm{m} \; (\si{\milli \V})\)}%
\end{pgfscope}%
\begin{pgfscope}%
\pgfsetbuttcap%
\pgfsetroundjoin%
\definecolor{currentfill}{rgb}{0.317647,0.317647,0.317647}%
\pgfsetfillcolor{currentfill}%
\pgfsetlinewidth{0.501875pt}%
\definecolor{currentstroke}{rgb}{0.317647,0.317647,0.317647}%
\pgfsetstrokecolor{currentstroke}%
\pgfsetdash{}{0pt}%
\pgfsys@defobject{currentmarker}{\pgfqpoint{-0.020833in}{0.000000in}}{\pgfqpoint{0.000000in}{0.000000in}}{%
\pgfpathmoveto{\pgfqpoint{0.000000in}{0.000000in}}%
\pgfpathlineto{\pgfqpoint{-0.020833in}{0.000000in}}%
\pgfusepath{stroke,fill}%
}%
\begin{pgfscope}%
\pgfsys@transformshift{0.526905in}{0.383578in}%
\pgfsys@useobject{currentmarker}{}%
\end{pgfscope}%
\end{pgfscope}%
\begin{pgfscope}%
\definecolor{textcolor}{rgb}{0.317647,0.317647,0.317647}%
\pgfsetstrokecolor{textcolor}%
\pgfsetfillcolor{textcolor}%
\pgftext[x=0.237745in,y=0.351461in,left,base]{\color{textcolor}\rmfamily\fontsize{6.664000}{7.996800}\selectfont \(\displaystyle 0.000\)}%
\end{pgfscope}%
\begin{pgfscope}%
\pgfsetbuttcap%
\pgfsetroundjoin%
\definecolor{currentfill}{rgb}{0.317647,0.317647,0.317647}%
\pgfsetfillcolor{currentfill}%
\pgfsetlinewidth{0.501875pt}%
\definecolor{currentstroke}{rgb}{0.317647,0.317647,0.317647}%
\pgfsetstrokecolor{currentstroke}%
\pgfsetdash{}{0pt}%
\pgfsys@defobject{currentmarker}{\pgfqpoint{-0.020833in}{0.000000in}}{\pgfqpoint{0.000000in}{0.000000in}}{%
\pgfpathmoveto{\pgfqpoint{0.000000in}{0.000000in}}%
\pgfpathlineto{\pgfqpoint{-0.020833in}{0.000000in}}%
\pgfusepath{stroke,fill}%
}%
\begin{pgfscope}%
\pgfsys@transformshift{0.526905in}{0.759929in}%
\pgfsys@useobject{currentmarker}{}%
\end{pgfscope}%
\end{pgfscope}%
\begin{pgfscope}%
\definecolor{textcolor}{rgb}{0.317647,0.317647,0.317647}%
\pgfsetstrokecolor{textcolor}%
\pgfsetfillcolor{textcolor}%
\pgftext[x=0.237745in,y=0.727812in,left,base]{\color{textcolor}\rmfamily\fontsize{6.664000}{7.996800}\selectfont \(\displaystyle 0.005\)}%
\end{pgfscope}%
\begin{pgfscope}%
\pgfsetbuttcap%
\pgfsetroundjoin%
\definecolor{currentfill}{rgb}{0.317647,0.317647,0.317647}%
\pgfsetfillcolor{currentfill}%
\pgfsetlinewidth{0.501875pt}%
\definecolor{currentstroke}{rgb}{0.317647,0.317647,0.317647}%
\pgfsetstrokecolor{currentstroke}%
\pgfsetdash{}{0pt}%
\pgfsys@defobject{currentmarker}{\pgfqpoint{-0.020833in}{0.000000in}}{\pgfqpoint{0.000000in}{0.000000in}}{%
\pgfpathmoveto{\pgfqpoint{0.000000in}{0.000000in}}%
\pgfpathlineto{\pgfqpoint{-0.020833in}{0.000000in}}%
\pgfusepath{stroke,fill}%
}%
\begin{pgfscope}%
\pgfsys@transformshift{0.526905in}{1.136280in}%
\pgfsys@useobject{currentmarker}{}%
\end{pgfscope}%
\end{pgfscope}%
\begin{pgfscope}%
\definecolor{textcolor}{rgb}{0.317647,0.317647,0.317647}%
\pgfsetstrokecolor{textcolor}%
\pgfsetfillcolor{textcolor}%
\pgftext[x=0.237745in,y=1.104163in,left,base]{\color{textcolor}\rmfamily\fontsize{6.664000}{7.996800}\selectfont \(\displaystyle 0.010\)}%
\end{pgfscope}%
\begin{pgfscope}%
\pgfsetbuttcap%
\pgfsetroundjoin%
\definecolor{currentfill}{rgb}{0.317647,0.317647,0.317647}%
\pgfsetfillcolor{currentfill}%
\pgfsetlinewidth{0.501875pt}%
\definecolor{currentstroke}{rgb}{0.317647,0.317647,0.317647}%
\pgfsetstrokecolor{currentstroke}%
\pgfsetdash{}{0pt}%
\pgfsys@defobject{currentmarker}{\pgfqpoint{-0.020833in}{0.000000in}}{\pgfqpoint{0.000000in}{0.000000in}}{%
\pgfpathmoveto{\pgfqpoint{0.000000in}{0.000000in}}%
\pgfpathlineto{\pgfqpoint{-0.020833in}{0.000000in}}%
\pgfusepath{stroke,fill}%
}%
\begin{pgfscope}%
\pgfsys@transformshift{0.526905in}{1.512630in}%
\pgfsys@useobject{currentmarker}{}%
\end{pgfscope}%
\end{pgfscope}%
\begin{pgfscope}%
\definecolor{textcolor}{rgb}{0.317647,0.317647,0.317647}%
\pgfsetstrokecolor{textcolor}%
\pgfsetfillcolor{textcolor}%
\pgftext[x=0.237745in,y=1.480513in,left,base]{\color{textcolor}\rmfamily\fontsize{6.664000}{7.996800}\selectfont \(\displaystyle 0.015\)}%
\end{pgfscope}%
\begin{pgfscope}%
\pgfsetbuttcap%
\pgfsetroundjoin%
\definecolor{currentfill}{rgb}{0.317647,0.317647,0.317647}%
\pgfsetfillcolor{currentfill}%
\pgfsetlinewidth{0.501875pt}%
\definecolor{currentstroke}{rgb}{0.317647,0.317647,0.317647}%
\pgfsetstrokecolor{currentstroke}%
\pgfsetdash{}{0pt}%
\pgfsys@defobject{currentmarker}{\pgfqpoint{-0.020833in}{0.000000in}}{\pgfqpoint{0.000000in}{0.000000in}}{%
\pgfpathmoveto{\pgfqpoint{0.000000in}{0.000000in}}%
\pgfpathlineto{\pgfqpoint{-0.020833in}{0.000000in}}%
\pgfusepath{stroke,fill}%
}%
\begin{pgfscope}%
\pgfsys@transformshift{0.526905in}{1.888981in}%
\pgfsys@useobject{currentmarker}{}%
\end{pgfscope}%
\end{pgfscope}%
\begin{pgfscope}%
\definecolor{textcolor}{rgb}{0.317647,0.317647,0.317647}%
\pgfsetstrokecolor{textcolor}%
\pgfsetfillcolor{textcolor}%
\pgftext[x=0.237745in,y=1.856864in,left,base]{\color{textcolor}\rmfamily\fontsize{6.664000}{7.996800}\selectfont \(\displaystyle 0.020\)}%
\end{pgfscope}%
\begin{pgfscope}%
\pgfsetbuttcap%
\pgfsetroundjoin%
\definecolor{currentfill}{rgb}{0.317647,0.317647,0.317647}%
\pgfsetfillcolor{currentfill}%
\pgfsetlinewidth{0.501875pt}%
\definecolor{currentstroke}{rgb}{0.317647,0.317647,0.317647}%
\pgfsetstrokecolor{currentstroke}%
\pgfsetdash{}{0pt}%
\pgfsys@defobject{currentmarker}{\pgfqpoint{-0.020833in}{0.000000in}}{\pgfqpoint{0.000000in}{0.000000in}}{%
\pgfpathmoveto{\pgfqpoint{0.000000in}{0.000000in}}%
\pgfpathlineto{\pgfqpoint{-0.020833in}{0.000000in}}%
\pgfusepath{stroke,fill}%
}%
\begin{pgfscope}%
\pgfsys@transformshift{0.526905in}{2.265332in}%
\pgfsys@useobject{currentmarker}{}%
\end{pgfscope}%
\end{pgfscope}%
\begin{pgfscope}%
\definecolor{textcolor}{rgb}{0.317647,0.317647,0.317647}%
\pgfsetstrokecolor{textcolor}%
\pgfsetfillcolor{textcolor}%
\pgftext[x=0.237745in,y=2.233215in,left,base]{\color{textcolor}\rmfamily\fontsize{6.664000}{7.996800}\selectfont \(\displaystyle 0.025\)}%
\end{pgfscope}%
\begin{pgfscope}%
\pgfsetbuttcap%
\pgfsetroundjoin%
\definecolor{currentfill}{rgb}{0.317647,0.317647,0.317647}%
\pgfsetfillcolor{currentfill}%
\pgfsetlinewidth{0.501875pt}%
\definecolor{currentstroke}{rgb}{0.317647,0.317647,0.317647}%
\pgfsetstrokecolor{currentstroke}%
\pgfsetdash{}{0pt}%
\pgfsys@defobject{currentmarker}{\pgfqpoint{-0.020833in}{0.000000in}}{\pgfqpoint{0.000000in}{0.000000in}}{%
\pgfpathmoveto{\pgfqpoint{0.000000in}{0.000000in}}%
\pgfpathlineto{\pgfqpoint{-0.020833in}{0.000000in}}%
\pgfusepath{stroke,fill}%
}%
\begin{pgfscope}%
\pgfsys@transformshift{0.526905in}{2.641682in}%
\pgfsys@useobject{currentmarker}{}%
\end{pgfscope}%
\end{pgfscope}%
\begin{pgfscope}%
\definecolor{textcolor}{rgb}{0.317647,0.317647,0.317647}%
\pgfsetstrokecolor{textcolor}%
\pgfsetfillcolor{textcolor}%
\pgftext[x=0.237745in,y=2.609566in,left,base]{\color{textcolor}\rmfamily\fontsize{6.664000}{7.996800}\selectfont \(\displaystyle 0.030\)}%
\end{pgfscope}%
\begin{pgfscope}%
\definecolor{textcolor}{rgb}{0.317647,0.317647,0.317647}%
\pgfsetstrokecolor{textcolor}%
\pgfsetfillcolor{textcolor}%
\pgftext[x=0.182189in,y=1.538578in,,bottom,rotate=90.000000]{\color{textcolor}\rmfamily\fontsize{6.664000}{7.996800}\selectfont Density}%
\end{pgfscope}%
\begin{pgfscope}%
\pgfpathrectangle{\pgfqpoint{0.526905in}{0.383578in}}{\pgfqpoint{3.875000in}{2.310000in}}%
\pgfusepath{clip}%
\pgfsetbuttcap%
\pgfsetmiterjoin%
\definecolor{currentfill}{rgb}{0.686275,0.352941,0.313725}%
\pgfsetfillcolor{currentfill}%
\pgfsetfillopacity{0.300000}%
\pgfsetlinewidth{0.000000pt}%
\definecolor{currentstroke}{rgb}{0.000000,0.000000,0.000000}%
\pgfsetstrokecolor{currentstroke}%
\pgfsetstrokeopacity{0.300000}%
\pgfsetdash{}{0pt}%
\pgfpathmoveto{\pgfqpoint{0.703041in}{0.383578in}}%
\pgfpathlineto{\pgfqpoint{0.720630in}{0.383578in}}%
\pgfpathlineto{\pgfqpoint{0.720630in}{0.389190in}}%
\pgfpathlineto{\pgfqpoint{0.703041in}{0.389190in}}%
\pgfpathclose%
\pgfusepath{fill}%
\end{pgfscope}%
\begin{pgfscope}%
\pgfpathrectangle{\pgfqpoint{0.526905in}{0.383578in}}{\pgfqpoint{3.875000in}{2.310000in}}%
\pgfusepath{clip}%
\pgfsetbuttcap%
\pgfsetmiterjoin%
\definecolor{currentfill}{rgb}{0.686275,0.352941,0.313725}%
\pgfsetfillcolor{currentfill}%
\pgfsetfillopacity{0.300000}%
\pgfsetlinewidth{0.000000pt}%
\definecolor{currentstroke}{rgb}{0.000000,0.000000,0.000000}%
\pgfsetstrokecolor{currentstroke}%
\pgfsetstrokeopacity{0.300000}%
\pgfsetdash{}{0pt}%
\pgfpathmoveto{\pgfqpoint{0.720630in}{0.383578in}}%
\pgfpathlineto{\pgfqpoint{0.738219in}{0.383578in}}%
\pgfpathlineto{\pgfqpoint{0.738219in}{0.383578in}}%
\pgfpathlineto{\pgfqpoint{0.720630in}{0.383578in}}%
\pgfpathclose%
\pgfusepath{fill}%
\end{pgfscope}%
\begin{pgfscope}%
\pgfpathrectangle{\pgfqpoint{0.526905in}{0.383578in}}{\pgfqpoint{3.875000in}{2.310000in}}%
\pgfusepath{clip}%
\pgfsetbuttcap%
\pgfsetmiterjoin%
\definecolor{currentfill}{rgb}{0.686275,0.352941,0.313725}%
\pgfsetfillcolor{currentfill}%
\pgfsetfillopacity{0.300000}%
\pgfsetlinewidth{0.000000pt}%
\definecolor{currentstroke}{rgb}{0.000000,0.000000,0.000000}%
\pgfsetstrokecolor{currentstroke}%
\pgfsetstrokeopacity{0.300000}%
\pgfsetdash{}{0pt}%
\pgfpathmoveto{\pgfqpoint{0.738219in}{0.383578in}}%
\pgfpathlineto{\pgfqpoint{0.755808in}{0.383578in}}%
\pgfpathlineto{\pgfqpoint{0.755808in}{0.394803in}}%
\pgfpathlineto{\pgfqpoint{0.738219in}{0.394803in}}%
\pgfpathclose%
\pgfusepath{fill}%
\end{pgfscope}%
\begin{pgfscope}%
\pgfpathrectangle{\pgfqpoint{0.526905in}{0.383578in}}{\pgfqpoint{3.875000in}{2.310000in}}%
\pgfusepath{clip}%
\pgfsetbuttcap%
\pgfsetmiterjoin%
\definecolor{currentfill}{rgb}{0.686275,0.352941,0.313725}%
\pgfsetfillcolor{currentfill}%
\pgfsetfillopacity{0.300000}%
\pgfsetlinewidth{0.000000pt}%
\definecolor{currentstroke}{rgb}{0.000000,0.000000,0.000000}%
\pgfsetstrokecolor{currentstroke}%
\pgfsetstrokeopacity{0.300000}%
\pgfsetdash{}{0pt}%
\pgfpathmoveto{\pgfqpoint{0.755808in}{0.383578in}}%
\pgfpathlineto{\pgfqpoint{0.773397in}{0.383578in}}%
\pgfpathlineto{\pgfqpoint{0.773397in}{0.394803in}}%
\pgfpathlineto{\pgfqpoint{0.755808in}{0.394803in}}%
\pgfpathclose%
\pgfusepath{fill}%
\end{pgfscope}%
\begin{pgfscope}%
\pgfpathrectangle{\pgfqpoint{0.526905in}{0.383578in}}{\pgfqpoint{3.875000in}{2.310000in}}%
\pgfusepath{clip}%
\pgfsetbuttcap%
\pgfsetmiterjoin%
\definecolor{currentfill}{rgb}{0.686275,0.352941,0.313725}%
\pgfsetfillcolor{currentfill}%
\pgfsetfillopacity{0.300000}%
\pgfsetlinewidth{0.000000pt}%
\definecolor{currentstroke}{rgb}{0.000000,0.000000,0.000000}%
\pgfsetstrokecolor{currentstroke}%
\pgfsetstrokeopacity{0.300000}%
\pgfsetdash{}{0pt}%
\pgfpathmoveto{\pgfqpoint{0.773397in}{0.383578in}}%
\pgfpathlineto{\pgfqpoint{0.790986in}{0.383578in}}%
\pgfpathlineto{\pgfqpoint{0.790986in}{0.389190in}}%
\pgfpathlineto{\pgfqpoint{0.773397in}{0.389190in}}%
\pgfpathclose%
\pgfusepath{fill}%
\end{pgfscope}%
\begin{pgfscope}%
\pgfpathrectangle{\pgfqpoint{0.526905in}{0.383578in}}{\pgfqpoint{3.875000in}{2.310000in}}%
\pgfusepath{clip}%
\pgfsetbuttcap%
\pgfsetmiterjoin%
\definecolor{currentfill}{rgb}{0.686275,0.352941,0.313725}%
\pgfsetfillcolor{currentfill}%
\pgfsetfillopacity{0.300000}%
\pgfsetlinewidth{0.000000pt}%
\definecolor{currentstroke}{rgb}{0.000000,0.000000,0.000000}%
\pgfsetstrokecolor{currentstroke}%
\pgfsetstrokeopacity{0.300000}%
\pgfsetdash{}{0pt}%
\pgfpathmoveto{\pgfqpoint{0.790986in}{0.383578in}}%
\pgfpathlineto{\pgfqpoint{0.808575in}{0.383578in}}%
\pgfpathlineto{\pgfqpoint{0.808575in}{0.389190in}}%
\pgfpathlineto{\pgfqpoint{0.790986in}{0.389190in}}%
\pgfpathclose%
\pgfusepath{fill}%
\end{pgfscope}%
\begin{pgfscope}%
\pgfpathrectangle{\pgfqpoint{0.526905in}{0.383578in}}{\pgfqpoint{3.875000in}{2.310000in}}%
\pgfusepath{clip}%
\pgfsetbuttcap%
\pgfsetmiterjoin%
\definecolor{currentfill}{rgb}{0.686275,0.352941,0.313725}%
\pgfsetfillcolor{currentfill}%
\pgfsetfillopacity{0.300000}%
\pgfsetlinewidth{0.000000pt}%
\definecolor{currentstroke}{rgb}{0.000000,0.000000,0.000000}%
\pgfsetstrokecolor{currentstroke}%
\pgfsetstrokeopacity{0.300000}%
\pgfsetdash{}{0pt}%
\pgfpathmoveto{\pgfqpoint{0.808575in}{0.383578in}}%
\pgfpathlineto{\pgfqpoint{0.826164in}{0.383578in}}%
\pgfpathlineto{\pgfqpoint{0.826164in}{0.394803in}}%
\pgfpathlineto{\pgfqpoint{0.808575in}{0.394803in}}%
\pgfpathclose%
\pgfusepath{fill}%
\end{pgfscope}%
\begin{pgfscope}%
\pgfpathrectangle{\pgfqpoint{0.526905in}{0.383578in}}{\pgfqpoint{3.875000in}{2.310000in}}%
\pgfusepath{clip}%
\pgfsetbuttcap%
\pgfsetmiterjoin%
\definecolor{currentfill}{rgb}{0.686275,0.352941,0.313725}%
\pgfsetfillcolor{currentfill}%
\pgfsetfillopacity{0.300000}%
\pgfsetlinewidth{0.000000pt}%
\definecolor{currentstroke}{rgb}{0.000000,0.000000,0.000000}%
\pgfsetstrokecolor{currentstroke}%
\pgfsetstrokeopacity{0.300000}%
\pgfsetdash{}{0pt}%
\pgfpathmoveto{\pgfqpoint{0.826164in}{0.383578in}}%
\pgfpathlineto{\pgfqpoint{0.843754in}{0.383578in}}%
\pgfpathlineto{\pgfqpoint{0.843754in}{0.389190in}}%
\pgfpathlineto{\pgfqpoint{0.826164in}{0.389190in}}%
\pgfpathclose%
\pgfusepath{fill}%
\end{pgfscope}%
\begin{pgfscope}%
\pgfpathrectangle{\pgfqpoint{0.526905in}{0.383578in}}{\pgfqpoint{3.875000in}{2.310000in}}%
\pgfusepath{clip}%
\pgfsetbuttcap%
\pgfsetmiterjoin%
\definecolor{currentfill}{rgb}{0.686275,0.352941,0.313725}%
\pgfsetfillcolor{currentfill}%
\pgfsetfillopacity{0.300000}%
\pgfsetlinewidth{0.000000pt}%
\definecolor{currentstroke}{rgb}{0.000000,0.000000,0.000000}%
\pgfsetstrokecolor{currentstroke}%
\pgfsetstrokeopacity{0.300000}%
\pgfsetdash{}{0pt}%
\pgfpathmoveto{\pgfqpoint{0.843754in}{0.383578in}}%
\pgfpathlineto{\pgfqpoint{0.861343in}{0.383578in}}%
\pgfpathlineto{\pgfqpoint{0.861343in}{0.383578in}}%
\pgfpathlineto{\pgfqpoint{0.843754in}{0.383578in}}%
\pgfpathclose%
\pgfusepath{fill}%
\end{pgfscope}%
\begin{pgfscope}%
\pgfpathrectangle{\pgfqpoint{0.526905in}{0.383578in}}{\pgfqpoint{3.875000in}{2.310000in}}%
\pgfusepath{clip}%
\pgfsetbuttcap%
\pgfsetmiterjoin%
\definecolor{currentfill}{rgb}{0.686275,0.352941,0.313725}%
\pgfsetfillcolor{currentfill}%
\pgfsetfillopacity{0.300000}%
\pgfsetlinewidth{0.000000pt}%
\definecolor{currentstroke}{rgb}{0.000000,0.000000,0.000000}%
\pgfsetstrokecolor{currentstroke}%
\pgfsetstrokeopacity{0.300000}%
\pgfsetdash{}{0pt}%
\pgfpathmoveto{\pgfqpoint{0.861343in}{0.383578in}}%
\pgfpathlineto{\pgfqpoint{0.878932in}{0.383578in}}%
\pgfpathlineto{\pgfqpoint{0.878932in}{0.394803in}}%
\pgfpathlineto{\pgfqpoint{0.861343in}{0.394803in}}%
\pgfpathclose%
\pgfusepath{fill}%
\end{pgfscope}%
\begin{pgfscope}%
\pgfpathrectangle{\pgfqpoint{0.526905in}{0.383578in}}{\pgfqpoint{3.875000in}{2.310000in}}%
\pgfusepath{clip}%
\pgfsetbuttcap%
\pgfsetmiterjoin%
\definecolor{currentfill}{rgb}{0.686275,0.352941,0.313725}%
\pgfsetfillcolor{currentfill}%
\pgfsetfillopacity{0.300000}%
\pgfsetlinewidth{0.000000pt}%
\definecolor{currentstroke}{rgb}{0.000000,0.000000,0.000000}%
\pgfsetstrokecolor{currentstroke}%
\pgfsetstrokeopacity{0.300000}%
\pgfsetdash{}{0pt}%
\pgfpathmoveto{\pgfqpoint{0.878932in}{0.383578in}}%
\pgfpathlineto{\pgfqpoint{0.896521in}{0.383578in}}%
\pgfpathlineto{\pgfqpoint{0.896521in}{0.400415in}}%
\pgfpathlineto{\pgfqpoint{0.878932in}{0.400415in}}%
\pgfpathclose%
\pgfusepath{fill}%
\end{pgfscope}%
\begin{pgfscope}%
\pgfpathrectangle{\pgfqpoint{0.526905in}{0.383578in}}{\pgfqpoint{3.875000in}{2.310000in}}%
\pgfusepath{clip}%
\pgfsetbuttcap%
\pgfsetmiterjoin%
\definecolor{currentfill}{rgb}{0.686275,0.352941,0.313725}%
\pgfsetfillcolor{currentfill}%
\pgfsetfillopacity{0.300000}%
\pgfsetlinewidth{0.000000pt}%
\definecolor{currentstroke}{rgb}{0.000000,0.000000,0.000000}%
\pgfsetstrokecolor{currentstroke}%
\pgfsetstrokeopacity{0.300000}%
\pgfsetdash{}{0pt}%
\pgfpathmoveto{\pgfqpoint{0.896521in}{0.383578in}}%
\pgfpathlineto{\pgfqpoint{0.914110in}{0.383578in}}%
\pgfpathlineto{\pgfqpoint{0.914110in}{0.389190in}}%
\pgfpathlineto{\pgfqpoint{0.896521in}{0.389190in}}%
\pgfpathclose%
\pgfusepath{fill}%
\end{pgfscope}%
\begin{pgfscope}%
\pgfpathrectangle{\pgfqpoint{0.526905in}{0.383578in}}{\pgfqpoint{3.875000in}{2.310000in}}%
\pgfusepath{clip}%
\pgfsetbuttcap%
\pgfsetmiterjoin%
\definecolor{currentfill}{rgb}{0.686275,0.352941,0.313725}%
\pgfsetfillcolor{currentfill}%
\pgfsetfillopacity{0.300000}%
\pgfsetlinewidth{0.000000pt}%
\definecolor{currentstroke}{rgb}{0.000000,0.000000,0.000000}%
\pgfsetstrokecolor{currentstroke}%
\pgfsetstrokeopacity{0.300000}%
\pgfsetdash{}{0pt}%
\pgfpathmoveto{\pgfqpoint{0.914110in}{0.383578in}}%
\pgfpathlineto{\pgfqpoint{0.931699in}{0.383578in}}%
\pgfpathlineto{\pgfqpoint{0.931699in}{0.383578in}}%
\pgfpathlineto{\pgfqpoint{0.914110in}{0.383578in}}%
\pgfpathclose%
\pgfusepath{fill}%
\end{pgfscope}%
\begin{pgfscope}%
\pgfpathrectangle{\pgfqpoint{0.526905in}{0.383578in}}{\pgfqpoint{3.875000in}{2.310000in}}%
\pgfusepath{clip}%
\pgfsetbuttcap%
\pgfsetmiterjoin%
\definecolor{currentfill}{rgb}{0.686275,0.352941,0.313725}%
\pgfsetfillcolor{currentfill}%
\pgfsetfillopacity{0.300000}%
\pgfsetlinewidth{0.000000pt}%
\definecolor{currentstroke}{rgb}{0.000000,0.000000,0.000000}%
\pgfsetstrokecolor{currentstroke}%
\pgfsetstrokeopacity{0.300000}%
\pgfsetdash{}{0pt}%
\pgfpathmoveto{\pgfqpoint{0.931699in}{0.383578in}}%
\pgfpathlineto{\pgfqpoint{0.949288in}{0.383578in}}%
\pgfpathlineto{\pgfqpoint{0.949288in}{0.406027in}}%
\pgfpathlineto{\pgfqpoint{0.931699in}{0.406027in}}%
\pgfpathclose%
\pgfusepath{fill}%
\end{pgfscope}%
\begin{pgfscope}%
\pgfpathrectangle{\pgfqpoint{0.526905in}{0.383578in}}{\pgfqpoint{3.875000in}{2.310000in}}%
\pgfusepath{clip}%
\pgfsetbuttcap%
\pgfsetmiterjoin%
\definecolor{currentfill}{rgb}{0.686275,0.352941,0.313725}%
\pgfsetfillcolor{currentfill}%
\pgfsetfillopacity{0.300000}%
\pgfsetlinewidth{0.000000pt}%
\definecolor{currentstroke}{rgb}{0.000000,0.000000,0.000000}%
\pgfsetstrokecolor{currentstroke}%
\pgfsetstrokeopacity{0.300000}%
\pgfsetdash{}{0pt}%
\pgfpathmoveto{\pgfqpoint{0.949288in}{0.383578in}}%
\pgfpathlineto{\pgfqpoint{0.966877in}{0.383578in}}%
\pgfpathlineto{\pgfqpoint{0.966877in}{0.394803in}}%
\pgfpathlineto{\pgfqpoint{0.949288in}{0.394803in}}%
\pgfpathclose%
\pgfusepath{fill}%
\end{pgfscope}%
\begin{pgfscope}%
\pgfpathrectangle{\pgfqpoint{0.526905in}{0.383578in}}{\pgfqpoint{3.875000in}{2.310000in}}%
\pgfusepath{clip}%
\pgfsetbuttcap%
\pgfsetmiterjoin%
\definecolor{currentfill}{rgb}{0.686275,0.352941,0.313725}%
\pgfsetfillcolor{currentfill}%
\pgfsetfillopacity{0.300000}%
\pgfsetlinewidth{0.000000pt}%
\definecolor{currentstroke}{rgb}{0.000000,0.000000,0.000000}%
\pgfsetstrokecolor{currentstroke}%
\pgfsetstrokeopacity{0.300000}%
\pgfsetdash{}{0pt}%
\pgfpathmoveto{\pgfqpoint{0.966877in}{0.383578in}}%
\pgfpathlineto{\pgfqpoint{0.984466in}{0.383578in}}%
\pgfpathlineto{\pgfqpoint{0.984466in}{0.400415in}}%
\pgfpathlineto{\pgfqpoint{0.966877in}{0.400415in}}%
\pgfpathclose%
\pgfusepath{fill}%
\end{pgfscope}%
\begin{pgfscope}%
\pgfpathrectangle{\pgfqpoint{0.526905in}{0.383578in}}{\pgfqpoint{3.875000in}{2.310000in}}%
\pgfusepath{clip}%
\pgfsetbuttcap%
\pgfsetmiterjoin%
\definecolor{currentfill}{rgb}{0.686275,0.352941,0.313725}%
\pgfsetfillcolor{currentfill}%
\pgfsetfillopacity{0.300000}%
\pgfsetlinewidth{0.000000pt}%
\definecolor{currentstroke}{rgb}{0.000000,0.000000,0.000000}%
\pgfsetstrokecolor{currentstroke}%
\pgfsetstrokeopacity{0.300000}%
\pgfsetdash{}{0pt}%
\pgfpathmoveto{\pgfqpoint{0.984466in}{0.383578in}}%
\pgfpathlineto{\pgfqpoint{1.002055in}{0.383578in}}%
\pgfpathlineto{\pgfqpoint{1.002055in}{0.411639in}}%
\pgfpathlineto{\pgfqpoint{0.984466in}{0.411639in}}%
\pgfpathclose%
\pgfusepath{fill}%
\end{pgfscope}%
\begin{pgfscope}%
\pgfpathrectangle{\pgfqpoint{0.526905in}{0.383578in}}{\pgfqpoint{3.875000in}{2.310000in}}%
\pgfusepath{clip}%
\pgfsetbuttcap%
\pgfsetmiterjoin%
\definecolor{currentfill}{rgb}{0.686275,0.352941,0.313725}%
\pgfsetfillcolor{currentfill}%
\pgfsetfillopacity{0.300000}%
\pgfsetlinewidth{0.000000pt}%
\definecolor{currentstroke}{rgb}{0.000000,0.000000,0.000000}%
\pgfsetstrokecolor{currentstroke}%
\pgfsetstrokeopacity{0.300000}%
\pgfsetdash{}{0pt}%
\pgfpathmoveto{\pgfqpoint{1.002055in}{0.383578in}}%
\pgfpathlineto{\pgfqpoint{1.019644in}{0.383578in}}%
\pgfpathlineto{\pgfqpoint{1.019644in}{0.389190in}}%
\pgfpathlineto{\pgfqpoint{1.002055in}{0.389190in}}%
\pgfpathclose%
\pgfusepath{fill}%
\end{pgfscope}%
\begin{pgfscope}%
\pgfpathrectangle{\pgfqpoint{0.526905in}{0.383578in}}{\pgfqpoint{3.875000in}{2.310000in}}%
\pgfusepath{clip}%
\pgfsetbuttcap%
\pgfsetmiterjoin%
\definecolor{currentfill}{rgb}{0.686275,0.352941,0.313725}%
\pgfsetfillcolor{currentfill}%
\pgfsetfillopacity{0.300000}%
\pgfsetlinewidth{0.000000pt}%
\definecolor{currentstroke}{rgb}{0.000000,0.000000,0.000000}%
\pgfsetstrokecolor{currentstroke}%
\pgfsetstrokeopacity{0.300000}%
\pgfsetdash{}{0pt}%
\pgfpathmoveto{\pgfqpoint{1.019644in}{0.383578in}}%
\pgfpathlineto{\pgfqpoint{1.037233in}{0.383578in}}%
\pgfpathlineto{\pgfqpoint{1.037233in}{0.422864in}}%
\pgfpathlineto{\pgfqpoint{1.019644in}{0.422864in}}%
\pgfpathclose%
\pgfusepath{fill}%
\end{pgfscope}%
\begin{pgfscope}%
\pgfpathrectangle{\pgfqpoint{0.526905in}{0.383578in}}{\pgfqpoint{3.875000in}{2.310000in}}%
\pgfusepath{clip}%
\pgfsetbuttcap%
\pgfsetmiterjoin%
\definecolor{currentfill}{rgb}{0.686275,0.352941,0.313725}%
\pgfsetfillcolor{currentfill}%
\pgfsetfillopacity{0.300000}%
\pgfsetlinewidth{0.000000pt}%
\definecolor{currentstroke}{rgb}{0.000000,0.000000,0.000000}%
\pgfsetstrokecolor{currentstroke}%
\pgfsetstrokeopacity{0.300000}%
\pgfsetdash{}{0pt}%
\pgfpathmoveto{\pgfqpoint{1.037233in}{0.383578in}}%
\pgfpathlineto{\pgfqpoint{1.054822in}{0.383578in}}%
\pgfpathlineto{\pgfqpoint{1.054822in}{0.406027in}}%
\pgfpathlineto{\pgfqpoint{1.037233in}{0.406027in}}%
\pgfpathclose%
\pgfusepath{fill}%
\end{pgfscope}%
\begin{pgfscope}%
\pgfpathrectangle{\pgfqpoint{0.526905in}{0.383578in}}{\pgfqpoint{3.875000in}{2.310000in}}%
\pgfusepath{clip}%
\pgfsetbuttcap%
\pgfsetmiterjoin%
\definecolor{currentfill}{rgb}{0.686275,0.352941,0.313725}%
\pgfsetfillcolor{currentfill}%
\pgfsetfillopacity{0.300000}%
\pgfsetlinewidth{0.000000pt}%
\definecolor{currentstroke}{rgb}{0.000000,0.000000,0.000000}%
\pgfsetstrokecolor{currentstroke}%
\pgfsetstrokeopacity{0.300000}%
\pgfsetdash{}{0pt}%
\pgfpathmoveto{\pgfqpoint{1.054822in}{0.383578in}}%
\pgfpathlineto{\pgfqpoint{1.072411in}{0.383578in}}%
\pgfpathlineto{\pgfqpoint{1.072411in}{0.406027in}}%
\pgfpathlineto{\pgfqpoint{1.054822in}{0.406027in}}%
\pgfpathclose%
\pgfusepath{fill}%
\end{pgfscope}%
\begin{pgfscope}%
\pgfpathrectangle{\pgfqpoint{0.526905in}{0.383578in}}{\pgfqpoint{3.875000in}{2.310000in}}%
\pgfusepath{clip}%
\pgfsetbuttcap%
\pgfsetmiterjoin%
\definecolor{currentfill}{rgb}{0.686275,0.352941,0.313725}%
\pgfsetfillcolor{currentfill}%
\pgfsetfillopacity{0.300000}%
\pgfsetlinewidth{0.000000pt}%
\definecolor{currentstroke}{rgb}{0.000000,0.000000,0.000000}%
\pgfsetstrokecolor{currentstroke}%
\pgfsetstrokeopacity{0.300000}%
\pgfsetdash{}{0pt}%
\pgfpathmoveto{\pgfqpoint{1.072411in}{0.383578in}}%
\pgfpathlineto{\pgfqpoint{1.090000in}{0.383578in}}%
\pgfpathlineto{\pgfqpoint{1.090000in}{0.400415in}}%
\pgfpathlineto{\pgfqpoint{1.072411in}{0.400415in}}%
\pgfpathclose%
\pgfusepath{fill}%
\end{pgfscope}%
\begin{pgfscope}%
\pgfpathrectangle{\pgfqpoint{0.526905in}{0.383578in}}{\pgfqpoint{3.875000in}{2.310000in}}%
\pgfusepath{clip}%
\pgfsetbuttcap%
\pgfsetmiterjoin%
\definecolor{currentfill}{rgb}{0.686275,0.352941,0.313725}%
\pgfsetfillcolor{currentfill}%
\pgfsetfillopacity{0.300000}%
\pgfsetlinewidth{0.000000pt}%
\definecolor{currentstroke}{rgb}{0.000000,0.000000,0.000000}%
\pgfsetstrokecolor{currentstroke}%
\pgfsetstrokeopacity{0.300000}%
\pgfsetdash{}{0pt}%
\pgfpathmoveto{\pgfqpoint{1.090000in}{0.383578in}}%
\pgfpathlineto{\pgfqpoint{1.107589in}{0.383578in}}%
\pgfpathlineto{\pgfqpoint{1.107589in}{0.417252in}}%
\pgfpathlineto{\pgfqpoint{1.090000in}{0.417252in}}%
\pgfpathclose%
\pgfusepath{fill}%
\end{pgfscope}%
\begin{pgfscope}%
\pgfpathrectangle{\pgfqpoint{0.526905in}{0.383578in}}{\pgfqpoint{3.875000in}{2.310000in}}%
\pgfusepath{clip}%
\pgfsetbuttcap%
\pgfsetmiterjoin%
\definecolor{currentfill}{rgb}{0.686275,0.352941,0.313725}%
\pgfsetfillcolor{currentfill}%
\pgfsetfillopacity{0.300000}%
\pgfsetlinewidth{0.000000pt}%
\definecolor{currentstroke}{rgb}{0.000000,0.000000,0.000000}%
\pgfsetstrokecolor{currentstroke}%
\pgfsetstrokeopacity{0.300000}%
\pgfsetdash{}{0pt}%
\pgfpathmoveto{\pgfqpoint{1.107589in}{0.383578in}}%
\pgfpathlineto{\pgfqpoint{1.125178in}{0.383578in}}%
\pgfpathlineto{\pgfqpoint{1.125178in}{0.450925in}}%
\pgfpathlineto{\pgfqpoint{1.107589in}{0.450925in}}%
\pgfpathclose%
\pgfusepath{fill}%
\end{pgfscope}%
\begin{pgfscope}%
\pgfpathrectangle{\pgfqpoint{0.526905in}{0.383578in}}{\pgfqpoint{3.875000in}{2.310000in}}%
\pgfusepath{clip}%
\pgfsetbuttcap%
\pgfsetmiterjoin%
\definecolor{currentfill}{rgb}{0.686275,0.352941,0.313725}%
\pgfsetfillcolor{currentfill}%
\pgfsetfillopacity{0.300000}%
\pgfsetlinewidth{0.000000pt}%
\definecolor{currentstroke}{rgb}{0.000000,0.000000,0.000000}%
\pgfsetstrokecolor{currentstroke}%
\pgfsetstrokeopacity{0.300000}%
\pgfsetdash{}{0pt}%
\pgfpathmoveto{\pgfqpoint{1.125178in}{0.383578in}}%
\pgfpathlineto{\pgfqpoint{1.142767in}{0.383578in}}%
\pgfpathlineto{\pgfqpoint{1.142767in}{0.406027in}}%
\pgfpathlineto{\pgfqpoint{1.125178in}{0.406027in}}%
\pgfpathclose%
\pgfusepath{fill}%
\end{pgfscope}%
\begin{pgfscope}%
\pgfpathrectangle{\pgfqpoint{0.526905in}{0.383578in}}{\pgfqpoint{3.875000in}{2.310000in}}%
\pgfusepath{clip}%
\pgfsetbuttcap%
\pgfsetmiterjoin%
\definecolor{currentfill}{rgb}{0.686275,0.352941,0.313725}%
\pgfsetfillcolor{currentfill}%
\pgfsetfillopacity{0.300000}%
\pgfsetlinewidth{0.000000pt}%
\definecolor{currentstroke}{rgb}{0.000000,0.000000,0.000000}%
\pgfsetstrokecolor{currentstroke}%
\pgfsetstrokeopacity{0.300000}%
\pgfsetdash{}{0pt}%
\pgfpathmoveto{\pgfqpoint{1.142767in}{0.383578in}}%
\pgfpathlineto{\pgfqpoint{1.160356in}{0.383578in}}%
\pgfpathlineto{\pgfqpoint{1.160356in}{0.411639in}}%
\pgfpathlineto{\pgfqpoint{1.142767in}{0.411639in}}%
\pgfpathclose%
\pgfusepath{fill}%
\end{pgfscope}%
\begin{pgfscope}%
\pgfpathrectangle{\pgfqpoint{0.526905in}{0.383578in}}{\pgfqpoint{3.875000in}{2.310000in}}%
\pgfusepath{clip}%
\pgfsetbuttcap%
\pgfsetmiterjoin%
\definecolor{currentfill}{rgb}{0.686275,0.352941,0.313725}%
\pgfsetfillcolor{currentfill}%
\pgfsetfillopacity{0.300000}%
\pgfsetlinewidth{0.000000pt}%
\definecolor{currentstroke}{rgb}{0.000000,0.000000,0.000000}%
\pgfsetstrokecolor{currentstroke}%
\pgfsetstrokeopacity{0.300000}%
\pgfsetdash{}{0pt}%
\pgfpathmoveto{\pgfqpoint{1.160356in}{0.383578in}}%
\pgfpathlineto{\pgfqpoint{1.177946in}{0.383578in}}%
\pgfpathlineto{\pgfqpoint{1.177946in}{0.428476in}}%
\pgfpathlineto{\pgfqpoint{1.160356in}{0.428476in}}%
\pgfpathclose%
\pgfusepath{fill}%
\end{pgfscope}%
\begin{pgfscope}%
\pgfpathrectangle{\pgfqpoint{0.526905in}{0.383578in}}{\pgfqpoint{3.875000in}{2.310000in}}%
\pgfusepath{clip}%
\pgfsetbuttcap%
\pgfsetmiterjoin%
\definecolor{currentfill}{rgb}{0.686275,0.352941,0.313725}%
\pgfsetfillcolor{currentfill}%
\pgfsetfillopacity{0.300000}%
\pgfsetlinewidth{0.000000pt}%
\definecolor{currentstroke}{rgb}{0.000000,0.000000,0.000000}%
\pgfsetstrokecolor{currentstroke}%
\pgfsetstrokeopacity{0.300000}%
\pgfsetdash{}{0pt}%
\pgfpathmoveto{\pgfqpoint{1.177946in}{0.383578in}}%
\pgfpathlineto{\pgfqpoint{1.195535in}{0.383578in}}%
\pgfpathlineto{\pgfqpoint{1.195535in}{0.439701in}}%
\pgfpathlineto{\pgfqpoint{1.177946in}{0.439701in}}%
\pgfpathclose%
\pgfusepath{fill}%
\end{pgfscope}%
\begin{pgfscope}%
\pgfpathrectangle{\pgfqpoint{0.526905in}{0.383578in}}{\pgfqpoint{3.875000in}{2.310000in}}%
\pgfusepath{clip}%
\pgfsetbuttcap%
\pgfsetmiterjoin%
\definecolor{currentfill}{rgb}{0.686275,0.352941,0.313725}%
\pgfsetfillcolor{currentfill}%
\pgfsetfillopacity{0.300000}%
\pgfsetlinewidth{0.000000pt}%
\definecolor{currentstroke}{rgb}{0.000000,0.000000,0.000000}%
\pgfsetstrokecolor{currentstroke}%
\pgfsetstrokeopacity{0.300000}%
\pgfsetdash{}{0pt}%
\pgfpathmoveto{\pgfqpoint{1.195535in}{0.383578in}}%
\pgfpathlineto{\pgfqpoint{1.213124in}{0.383578in}}%
\pgfpathlineto{\pgfqpoint{1.213124in}{0.434088in}}%
\pgfpathlineto{\pgfqpoint{1.195535in}{0.434088in}}%
\pgfpathclose%
\pgfusepath{fill}%
\end{pgfscope}%
\begin{pgfscope}%
\pgfpathrectangle{\pgfqpoint{0.526905in}{0.383578in}}{\pgfqpoint{3.875000in}{2.310000in}}%
\pgfusepath{clip}%
\pgfsetbuttcap%
\pgfsetmiterjoin%
\definecolor{currentfill}{rgb}{0.686275,0.352941,0.313725}%
\pgfsetfillcolor{currentfill}%
\pgfsetfillopacity{0.300000}%
\pgfsetlinewidth{0.000000pt}%
\definecolor{currentstroke}{rgb}{0.000000,0.000000,0.000000}%
\pgfsetstrokecolor{currentstroke}%
\pgfsetstrokeopacity{0.300000}%
\pgfsetdash{}{0pt}%
\pgfpathmoveto{\pgfqpoint{1.213124in}{0.383578in}}%
\pgfpathlineto{\pgfqpoint{1.230713in}{0.383578in}}%
\pgfpathlineto{\pgfqpoint{1.230713in}{0.473374in}}%
\pgfpathlineto{\pgfqpoint{1.213124in}{0.473374in}}%
\pgfpathclose%
\pgfusepath{fill}%
\end{pgfscope}%
\begin{pgfscope}%
\pgfpathrectangle{\pgfqpoint{0.526905in}{0.383578in}}{\pgfqpoint{3.875000in}{2.310000in}}%
\pgfusepath{clip}%
\pgfsetbuttcap%
\pgfsetmiterjoin%
\definecolor{currentfill}{rgb}{0.686275,0.352941,0.313725}%
\pgfsetfillcolor{currentfill}%
\pgfsetfillopacity{0.300000}%
\pgfsetlinewidth{0.000000pt}%
\definecolor{currentstroke}{rgb}{0.000000,0.000000,0.000000}%
\pgfsetstrokecolor{currentstroke}%
\pgfsetstrokeopacity{0.300000}%
\pgfsetdash{}{0pt}%
\pgfpathmoveto{\pgfqpoint{1.230713in}{0.383578in}}%
\pgfpathlineto{\pgfqpoint{1.248302in}{0.383578in}}%
\pgfpathlineto{\pgfqpoint{1.248302in}{0.439701in}}%
\pgfpathlineto{\pgfqpoint{1.230713in}{0.439701in}}%
\pgfpathclose%
\pgfusepath{fill}%
\end{pgfscope}%
\begin{pgfscope}%
\pgfpathrectangle{\pgfqpoint{0.526905in}{0.383578in}}{\pgfqpoint{3.875000in}{2.310000in}}%
\pgfusepath{clip}%
\pgfsetbuttcap%
\pgfsetmiterjoin%
\definecolor{currentfill}{rgb}{0.686275,0.352941,0.313725}%
\pgfsetfillcolor{currentfill}%
\pgfsetfillopacity{0.300000}%
\pgfsetlinewidth{0.000000pt}%
\definecolor{currentstroke}{rgb}{0.000000,0.000000,0.000000}%
\pgfsetstrokecolor{currentstroke}%
\pgfsetstrokeopacity{0.300000}%
\pgfsetdash{}{0pt}%
\pgfpathmoveto{\pgfqpoint{1.248302in}{0.383578in}}%
\pgfpathlineto{\pgfqpoint{1.265891in}{0.383578in}}%
\pgfpathlineto{\pgfqpoint{1.265891in}{0.450925in}}%
\pgfpathlineto{\pgfqpoint{1.248302in}{0.450925in}}%
\pgfpathclose%
\pgfusepath{fill}%
\end{pgfscope}%
\begin{pgfscope}%
\pgfpathrectangle{\pgfqpoint{0.526905in}{0.383578in}}{\pgfqpoint{3.875000in}{2.310000in}}%
\pgfusepath{clip}%
\pgfsetbuttcap%
\pgfsetmiterjoin%
\definecolor{currentfill}{rgb}{0.686275,0.352941,0.313725}%
\pgfsetfillcolor{currentfill}%
\pgfsetfillopacity{0.300000}%
\pgfsetlinewidth{0.000000pt}%
\definecolor{currentstroke}{rgb}{0.000000,0.000000,0.000000}%
\pgfsetstrokecolor{currentstroke}%
\pgfsetstrokeopacity{0.300000}%
\pgfsetdash{}{0pt}%
\pgfpathmoveto{\pgfqpoint{1.265891in}{0.383578in}}%
\pgfpathlineto{\pgfqpoint{1.283480in}{0.383578in}}%
\pgfpathlineto{\pgfqpoint{1.283480in}{0.422864in}}%
\pgfpathlineto{\pgfqpoint{1.265891in}{0.422864in}}%
\pgfpathclose%
\pgfusepath{fill}%
\end{pgfscope}%
\begin{pgfscope}%
\pgfpathrectangle{\pgfqpoint{0.526905in}{0.383578in}}{\pgfqpoint{3.875000in}{2.310000in}}%
\pgfusepath{clip}%
\pgfsetbuttcap%
\pgfsetmiterjoin%
\definecolor{currentfill}{rgb}{0.686275,0.352941,0.313725}%
\pgfsetfillcolor{currentfill}%
\pgfsetfillopacity{0.300000}%
\pgfsetlinewidth{0.000000pt}%
\definecolor{currentstroke}{rgb}{0.000000,0.000000,0.000000}%
\pgfsetstrokecolor{currentstroke}%
\pgfsetstrokeopacity{0.300000}%
\pgfsetdash{}{0pt}%
\pgfpathmoveto{\pgfqpoint{1.283480in}{0.383578in}}%
\pgfpathlineto{\pgfqpoint{1.301069in}{0.383578in}}%
\pgfpathlineto{\pgfqpoint{1.301069in}{0.484598in}}%
\pgfpathlineto{\pgfqpoint{1.283480in}{0.484598in}}%
\pgfpathclose%
\pgfusepath{fill}%
\end{pgfscope}%
\begin{pgfscope}%
\pgfpathrectangle{\pgfqpoint{0.526905in}{0.383578in}}{\pgfqpoint{3.875000in}{2.310000in}}%
\pgfusepath{clip}%
\pgfsetbuttcap%
\pgfsetmiterjoin%
\definecolor{currentfill}{rgb}{0.686275,0.352941,0.313725}%
\pgfsetfillcolor{currentfill}%
\pgfsetfillopacity{0.300000}%
\pgfsetlinewidth{0.000000pt}%
\definecolor{currentstroke}{rgb}{0.000000,0.000000,0.000000}%
\pgfsetstrokecolor{currentstroke}%
\pgfsetstrokeopacity{0.300000}%
\pgfsetdash{}{0pt}%
\pgfpathmoveto{\pgfqpoint{1.301069in}{0.383578in}}%
\pgfpathlineto{\pgfqpoint{1.318658in}{0.383578in}}%
\pgfpathlineto{\pgfqpoint{1.318658in}{0.422864in}}%
\pgfpathlineto{\pgfqpoint{1.301069in}{0.422864in}}%
\pgfpathclose%
\pgfusepath{fill}%
\end{pgfscope}%
\begin{pgfscope}%
\pgfpathrectangle{\pgfqpoint{0.526905in}{0.383578in}}{\pgfqpoint{3.875000in}{2.310000in}}%
\pgfusepath{clip}%
\pgfsetbuttcap%
\pgfsetmiterjoin%
\definecolor{currentfill}{rgb}{0.686275,0.352941,0.313725}%
\pgfsetfillcolor{currentfill}%
\pgfsetfillopacity{0.300000}%
\pgfsetlinewidth{0.000000pt}%
\definecolor{currentstroke}{rgb}{0.000000,0.000000,0.000000}%
\pgfsetstrokecolor{currentstroke}%
\pgfsetstrokeopacity{0.300000}%
\pgfsetdash{}{0pt}%
\pgfpathmoveto{\pgfqpoint{1.318658in}{0.383578in}}%
\pgfpathlineto{\pgfqpoint{1.336247in}{0.383578in}}%
\pgfpathlineto{\pgfqpoint{1.336247in}{0.490211in}}%
\pgfpathlineto{\pgfqpoint{1.318658in}{0.490211in}}%
\pgfpathclose%
\pgfusepath{fill}%
\end{pgfscope}%
\begin{pgfscope}%
\pgfpathrectangle{\pgfqpoint{0.526905in}{0.383578in}}{\pgfqpoint{3.875000in}{2.310000in}}%
\pgfusepath{clip}%
\pgfsetbuttcap%
\pgfsetmiterjoin%
\definecolor{currentfill}{rgb}{0.686275,0.352941,0.313725}%
\pgfsetfillcolor{currentfill}%
\pgfsetfillopacity{0.300000}%
\pgfsetlinewidth{0.000000pt}%
\definecolor{currentstroke}{rgb}{0.000000,0.000000,0.000000}%
\pgfsetstrokecolor{currentstroke}%
\pgfsetstrokeopacity{0.300000}%
\pgfsetdash{}{0pt}%
\pgfpathmoveto{\pgfqpoint{1.336247in}{0.383578in}}%
\pgfpathlineto{\pgfqpoint{1.353836in}{0.383578in}}%
\pgfpathlineto{\pgfqpoint{1.353836in}{0.484598in}}%
\pgfpathlineto{\pgfqpoint{1.336247in}{0.484598in}}%
\pgfpathclose%
\pgfusepath{fill}%
\end{pgfscope}%
\begin{pgfscope}%
\pgfpathrectangle{\pgfqpoint{0.526905in}{0.383578in}}{\pgfqpoint{3.875000in}{2.310000in}}%
\pgfusepath{clip}%
\pgfsetbuttcap%
\pgfsetmiterjoin%
\definecolor{currentfill}{rgb}{0.686275,0.352941,0.313725}%
\pgfsetfillcolor{currentfill}%
\pgfsetfillopacity{0.300000}%
\pgfsetlinewidth{0.000000pt}%
\definecolor{currentstroke}{rgb}{0.000000,0.000000,0.000000}%
\pgfsetstrokecolor{currentstroke}%
\pgfsetstrokeopacity{0.300000}%
\pgfsetdash{}{0pt}%
\pgfpathmoveto{\pgfqpoint{1.353836in}{0.383578in}}%
\pgfpathlineto{\pgfqpoint{1.371425in}{0.383578in}}%
\pgfpathlineto{\pgfqpoint{1.371425in}{0.478986in}}%
\pgfpathlineto{\pgfqpoint{1.353836in}{0.478986in}}%
\pgfpathclose%
\pgfusepath{fill}%
\end{pgfscope}%
\begin{pgfscope}%
\pgfpathrectangle{\pgfqpoint{0.526905in}{0.383578in}}{\pgfqpoint{3.875000in}{2.310000in}}%
\pgfusepath{clip}%
\pgfsetbuttcap%
\pgfsetmiterjoin%
\definecolor{currentfill}{rgb}{0.686275,0.352941,0.313725}%
\pgfsetfillcolor{currentfill}%
\pgfsetfillopacity{0.300000}%
\pgfsetlinewidth{0.000000pt}%
\definecolor{currentstroke}{rgb}{0.000000,0.000000,0.000000}%
\pgfsetstrokecolor{currentstroke}%
\pgfsetstrokeopacity{0.300000}%
\pgfsetdash{}{0pt}%
\pgfpathmoveto{\pgfqpoint{1.371425in}{0.383578in}}%
\pgfpathlineto{\pgfqpoint{1.389014in}{0.383578in}}%
\pgfpathlineto{\pgfqpoint{1.389014in}{0.467762in}}%
\pgfpathlineto{\pgfqpoint{1.371425in}{0.467762in}}%
\pgfpathclose%
\pgfusepath{fill}%
\end{pgfscope}%
\begin{pgfscope}%
\pgfpathrectangle{\pgfqpoint{0.526905in}{0.383578in}}{\pgfqpoint{3.875000in}{2.310000in}}%
\pgfusepath{clip}%
\pgfsetbuttcap%
\pgfsetmiterjoin%
\definecolor{currentfill}{rgb}{0.686275,0.352941,0.313725}%
\pgfsetfillcolor{currentfill}%
\pgfsetfillopacity{0.300000}%
\pgfsetlinewidth{0.000000pt}%
\definecolor{currentstroke}{rgb}{0.000000,0.000000,0.000000}%
\pgfsetstrokecolor{currentstroke}%
\pgfsetstrokeopacity{0.300000}%
\pgfsetdash{}{0pt}%
\pgfpathmoveto{\pgfqpoint{1.389014in}{0.383578in}}%
\pgfpathlineto{\pgfqpoint{1.406603in}{0.383578in}}%
\pgfpathlineto{\pgfqpoint{1.406603in}{0.557558in}}%
\pgfpathlineto{\pgfqpoint{1.389014in}{0.557558in}}%
\pgfpathclose%
\pgfusepath{fill}%
\end{pgfscope}%
\begin{pgfscope}%
\pgfpathrectangle{\pgfqpoint{0.526905in}{0.383578in}}{\pgfqpoint{3.875000in}{2.310000in}}%
\pgfusepath{clip}%
\pgfsetbuttcap%
\pgfsetmiterjoin%
\definecolor{currentfill}{rgb}{0.686275,0.352941,0.313725}%
\pgfsetfillcolor{currentfill}%
\pgfsetfillopacity{0.300000}%
\pgfsetlinewidth{0.000000pt}%
\definecolor{currentstroke}{rgb}{0.000000,0.000000,0.000000}%
\pgfsetstrokecolor{currentstroke}%
\pgfsetstrokeopacity{0.300000}%
\pgfsetdash{}{0pt}%
\pgfpathmoveto{\pgfqpoint{1.406603in}{0.383578in}}%
\pgfpathlineto{\pgfqpoint{1.424192in}{0.383578in}}%
\pgfpathlineto{\pgfqpoint{1.424192in}{0.557558in}}%
\pgfpathlineto{\pgfqpoint{1.406603in}{0.557558in}}%
\pgfpathclose%
\pgfusepath{fill}%
\end{pgfscope}%
\begin{pgfscope}%
\pgfpathrectangle{\pgfqpoint{0.526905in}{0.383578in}}{\pgfqpoint{3.875000in}{2.310000in}}%
\pgfusepath{clip}%
\pgfsetbuttcap%
\pgfsetmiterjoin%
\definecolor{currentfill}{rgb}{0.686275,0.352941,0.313725}%
\pgfsetfillcolor{currentfill}%
\pgfsetfillopacity{0.300000}%
\pgfsetlinewidth{0.000000pt}%
\definecolor{currentstroke}{rgb}{0.000000,0.000000,0.000000}%
\pgfsetstrokecolor{currentstroke}%
\pgfsetstrokeopacity{0.300000}%
\pgfsetdash{}{0pt}%
\pgfpathmoveto{\pgfqpoint{1.424192in}{0.383578in}}%
\pgfpathlineto{\pgfqpoint{1.441781in}{0.383578in}}%
\pgfpathlineto{\pgfqpoint{1.441781in}{0.478986in}}%
\pgfpathlineto{\pgfqpoint{1.424192in}{0.478986in}}%
\pgfpathclose%
\pgfusepath{fill}%
\end{pgfscope}%
\begin{pgfscope}%
\pgfpathrectangle{\pgfqpoint{0.526905in}{0.383578in}}{\pgfqpoint{3.875000in}{2.310000in}}%
\pgfusepath{clip}%
\pgfsetbuttcap%
\pgfsetmiterjoin%
\definecolor{currentfill}{rgb}{0.686275,0.352941,0.313725}%
\pgfsetfillcolor{currentfill}%
\pgfsetfillopacity{0.300000}%
\pgfsetlinewidth{0.000000pt}%
\definecolor{currentstroke}{rgb}{0.000000,0.000000,0.000000}%
\pgfsetstrokecolor{currentstroke}%
\pgfsetstrokeopacity{0.300000}%
\pgfsetdash{}{0pt}%
\pgfpathmoveto{\pgfqpoint{1.441781in}{0.383578in}}%
\pgfpathlineto{\pgfqpoint{1.459370in}{0.383578in}}%
\pgfpathlineto{\pgfqpoint{1.459370in}{0.535109in}}%
\pgfpathlineto{\pgfqpoint{1.441781in}{0.535109in}}%
\pgfpathclose%
\pgfusepath{fill}%
\end{pgfscope}%
\begin{pgfscope}%
\pgfpathrectangle{\pgfqpoint{0.526905in}{0.383578in}}{\pgfqpoint{3.875000in}{2.310000in}}%
\pgfusepath{clip}%
\pgfsetbuttcap%
\pgfsetmiterjoin%
\definecolor{currentfill}{rgb}{0.686275,0.352941,0.313725}%
\pgfsetfillcolor{currentfill}%
\pgfsetfillopacity{0.300000}%
\pgfsetlinewidth{0.000000pt}%
\definecolor{currentstroke}{rgb}{0.000000,0.000000,0.000000}%
\pgfsetstrokecolor{currentstroke}%
\pgfsetstrokeopacity{0.300000}%
\pgfsetdash{}{0pt}%
\pgfpathmoveto{\pgfqpoint{1.459370in}{0.383578in}}%
\pgfpathlineto{\pgfqpoint{1.476959in}{0.383578in}}%
\pgfpathlineto{\pgfqpoint{1.476959in}{0.568782in}}%
\pgfpathlineto{\pgfqpoint{1.459370in}{0.568782in}}%
\pgfpathclose%
\pgfusepath{fill}%
\end{pgfscope}%
\begin{pgfscope}%
\pgfpathrectangle{\pgfqpoint{0.526905in}{0.383578in}}{\pgfqpoint{3.875000in}{2.310000in}}%
\pgfusepath{clip}%
\pgfsetbuttcap%
\pgfsetmiterjoin%
\definecolor{currentfill}{rgb}{0.686275,0.352941,0.313725}%
\pgfsetfillcolor{currentfill}%
\pgfsetfillopacity{0.300000}%
\pgfsetlinewidth{0.000000pt}%
\definecolor{currentstroke}{rgb}{0.000000,0.000000,0.000000}%
\pgfsetstrokecolor{currentstroke}%
\pgfsetstrokeopacity{0.300000}%
\pgfsetdash{}{0pt}%
\pgfpathmoveto{\pgfqpoint{1.476959in}{0.383578in}}%
\pgfpathlineto{\pgfqpoint{1.494548in}{0.383578in}}%
\pgfpathlineto{\pgfqpoint{1.494548in}{0.568782in}}%
\pgfpathlineto{\pgfqpoint{1.476959in}{0.568782in}}%
\pgfpathclose%
\pgfusepath{fill}%
\end{pgfscope}%
\begin{pgfscope}%
\pgfpathrectangle{\pgfqpoint{0.526905in}{0.383578in}}{\pgfqpoint{3.875000in}{2.310000in}}%
\pgfusepath{clip}%
\pgfsetbuttcap%
\pgfsetmiterjoin%
\definecolor{currentfill}{rgb}{0.686275,0.352941,0.313725}%
\pgfsetfillcolor{currentfill}%
\pgfsetfillopacity{0.300000}%
\pgfsetlinewidth{0.000000pt}%
\definecolor{currentstroke}{rgb}{0.000000,0.000000,0.000000}%
\pgfsetstrokecolor{currentstroke}%
\pgfsetstrokeopacity{0.300000}%
\pgfsetdash{}{0pt}%
\pgfpathmoveto{\pgfqpoint{1.494548in}{0.383578in}}%
\pgfpathlineto{\pgfqpoint{1.512138in}{0.383578in}}%
\pgfpathlineto{\pgfqpoint{1.512138in}{0.619292in}}%
\pgfpathlineto{\pgfqpoint{1.494548in}{0.619292in}}%
\pgfpathclose%
\pgfusepath{fill}%
\end{pgfscope}%
\begin{pgfscope}%
\pgfpathrectangle{\pgfqpoint{0.526905in}{0.383578in}}{\pgfqpoint{3.875000in}{2.310000in}}%
\pgfusepath{clip}%
\pgfsetbuttcap%
\pgfsetmiterjoin%
\definecolor{currentfill}{rgb}{0.686275,0.352941,0.313725}%
\pgfsetfillcolor{currentfill}%
\pgfsetfillopacity{0.300000}%
\pgfsetlinewidth{0.000000pt}%
\definecolor{currentstroke}{rgb}{0.000000,0.000000,0.000000}%
\pgfsetstrokecolor{currentstroke}%
\pgfsetstrokeopacity{0.300000}%
\pgfsetdash{}{0pt}%
\pgfpathmoveto{\pgfqpoint{1.512138in}{0.383578in}}%
\pgfpathlineto{\pgfqpoint{1.529727in}{0.383578in}}%
\pgfpathlineto{\pgfqpoint{1.529727in}{0.591231in}}%
\pgfpathlineto{\pgfqpoint{1.512138in}{0.591231in}}%
\pgfpathclose%
\pgfusepath{fill}%
\end{pgfscope}%
\begin{pgfscope}%
\pgfpathrectangle{\pgfqpoint{0.526905in}{0.383578in}}{\pgfqpoint{3.875000in}{2.310000in}}%
\pgfusepath{clip}%
\pgfsetbuttcap%
\pgfsetmiterjoin%
\definecolor{currentfill}{rgb}{0.686275,0.352941,0.313725}%
\pgfsetfillcolor{currentfill}%
\pgfsetfillopacity{0.300000}%
\pgfsetlinewidth{0.000000pt}%
\definecolor{currentstroke}{rgb}{0.000000,0.000000,0.000000}%
\pgfsetstrokecolor{currentstroke}%
\pgfsetstrokeopacity{0.300000}%
\pgfsetdash{}{0pt}%
\pgfpathmoveto{\pgfqpoint{1.529727in}{0.383578in}}%
\pgfpathlineto{\pgfqpoint{1.547316in}{0.383578in}}%
\pgfpathlineto{\pgfqpoint{1.547316in}{0.529496in}}%
\pgfpathlineto{\pgfqpoint{1.529727in}{0.529496in}}%
\pgfpathclose%
\pgfusepath{fill}%
\end{pgfscope}%
\begin{pgfscope}%
\pgfpathrectangle{\pgfqpoint{0.526905in}{0.383578in}}{\pgfqpoint{3.875000in}{2.310000in}}%
\pgfusepath{clip}%
\pgfsetbuttcap%
\pgfsetmiterjoin%
\definecolor{currentfill}{rgb}{0.686275,0.352941,0.313725}%
\pgfsetfillcolor{currentfill}%
\pgfsetfillopacity{0.300000}%
\pgfsetlinewidth{0.000000pt}%
\definecolor{currentstroke}{rgb}{0.000000,0.000000,0.000000}%
\pgfsetstrokecolor{currentstroke}%
\pgfsetstrokeopacity{0.300000}%
\pgfsetdash{}{0pt}%
\pgfpathmoveto{\pgfqpoint{1.547316in}{0.383578in}}%
\pgfpathlineto{\pgfqpoint{1.564905in}{0.383578in}}%
\pgfpathlineto{\pgfqpoint{1.564905in}{0.568782in}}%
\pgfpathlineto{\pgfqpoint{1.547316in}{0.568782in}}%
\pgfpathclose%
\pgfusepath{fill}%
\end{pgfscope}%
\begin{pgfscope}%
\pgfpathrectangle{\pgfqpoint{0.526905in}{0.383578in}}{\pgfqpoint{3.875000in}{2.310000in}}%
\pgfusepath{clip}%
\pgfsetbuttcap%
\pgfsetmiterjoin%
\definecolor{currentfill}{rgb}{0.686275,0.352941,0.313725}%
\pgfsetfillcolor{currentfill}%
\pgfsetfillopacity{0.300000}%
\pgfsetlinewidth{0.000000pt}%
\definecolor{currentstroke}{rgb}{0.000000,0.000000,0.000000}%
\pgfsetstrokecolor{currentstroke}%
\pgfsetstrokeopacity{0.300000}%
\pgfsetdash{}{0pt}%
\pgfpathmoveto{\pgfqpoint{1.564905in}{0.383578in}}%
\pgfpathlineto{\pgfqpoint{1.582494in}{0.383578in}}%
\pgfpathlineto{\pgfqpoint{1.582494in}{0.669803in}}%
\pgfpathlineto{\pgfqpoint{1.564905in}{0.669803in}}%
\pgfpathclose%
\pgfusepath{fill}%
\end{pgfscope}%
\begin{pgfscope}%
\pgfpathrectangle{\pgfqpoint{0.526905in}{0.383578in}}{\pgfqpoint{3.875000in}{2.310000in}}%
\pgfusepath{clip}%
\pgfsetbuttcap%
\pgfsetmiterjoin%
\definecolor{currentfill}{rgb}{0.686275,0.352941,0.313725}%
\pgfsetfillcolor{currentfill}%
\pgfsetfillopacity{0.300000}%
\pgfsetlinewidth{0.000000pt}%
\definecolor{currentstroke}{rgb}{0.000000,0.000000,0.000000}%
\pgfsetstrokecolor{currentstroke}%
\pgfsetstrokeopacity{0.300000}%
\pgfsetdash{}{0pt}%
\pgfpathmoveto{\pgfqpoint{1.582494in}{0.383578in}}%
\pgfpathlineto{\pgfqpoint{1.600083in}{0.383578in}}%
\pgfpathlineto{\pgfqpoint{1.600083in}{0.658578in}}%
\pgfpathlineto{\pgfqpoint{1.582494in}{0.658578in}}%
\pgfpathclose%
\pgfusepath{fill}%
\end{pgfscope}%
\begin{pgfscope}%
\pgfpathrectangle{\pgfqpoint{0.526905in}{0.383578in}}{\pgfqpoint{3.875000in}{2.310000in}}%
\pgfusepath{clip}%
\pgfsetbuttcap%
\pgfsetmiterjoin%
\definecolor{currentfill}{rgb}{0.686275,0.352941,0.313725}%
\pgfsetfillcolor{currentfill}%
\pgfsetfillopacity{0.300000}%
\pgfsetlinewidth{0.000000pt}%
\definecolor{currentstroke}{rgb}{0.000000,0.000000,0.000000}%
\pgfsetstrokecolor{currentstroke}%
\pgfsetstrokeopacity{0.300000}%
\pgfsetdash{}{0pt}%
\pgfpathmoveto{\pgfqpoint{1.600083in}{0.383578in}}%
\pgfpathlineto{\pgfqpoint{1.617672in}{0.383578in}}%
\pgfpathlineto{\pgfqpoint{1.617672in}{0.585619in}}%
\pgfpathlineto{\pgfqpoint{1.600083in}{0.585619in}}%
\pgfpathclose%
\pgfusepath{fill}%
\end{pgfscope}%
\begin{pgfscope}%
\pgfpathrectangle{\pgfqpoint{0.526905in}{0.383578in}}{\pgfqpoint{3.875000in}{2.310000in}}%
\pgfusepath{clip}%
\pgfsetbuttcap%
\pgfsetmiterjoin%
\definecolor{currentfill}{rgb}{0.686275,0.352941,0.313725}%
\pgfsetfillcolor{currentfill}%
\pgfsetfillopacity{0.300000}%
\pgfsetlinewidth{0.000000pt}%
\definecolor{currentstroke}{rgb}{0.000000,0.000000,0.000000}%
\pgfsetstrokecolor{currentstroke}%
\pgfsetstrokeopacity{0.300000}%
\pgfsetdash{}{0pt}%
\pgfpathmoveto{\pgfqpoint{1.617672in}{0.383578in}}%
\pgfpathlineto{\pgfqpoint{1.635261in}{0.383578in}}%
\pgfpathlineto{\pgfqpoint{1.635261in}{0.703476in}}%
\pgfpathlineto{\pgfqpoint{1.617672in}{0.703476in}}%
\pgfpathclose%
\pgfusepath{fill}%
\end{pgfscope}%
\begin{pgfscope}%
\pgfpathrectangle{\pgfqpoint{0.526905in}{0.383578in}}{\pgfqpoint{3.875000in}{2.310000in}}%
\pgfusepath{clip}%
\pgfsetbuttcap%
\pgfsetmiterjoin%
\definecolor{currentfill}{rgb}{0.686275,0.352941,0.313725}%
\pgfsetfillcolor{currentfill}%
\pgfsetfillopacity{0.300000}%
\pgfsetlinewidth{0.000000pt}%
\definecolor{currentstroke}{rgb}{0.000000,0.000000,0.000000}%
\pgfsetstrokecolor{currentstroke}%
\pgfsetstrokeopacity{0.300000}%
\pgfsetdash{}{0pt}%
\pgfpathmoveto{\pgfqpoint{1.635261in}{0.383578in}}%
\pgfpathlineto{\pgfqpoint{1.652850in}{0.383578in}}%
\pgfpathlineto{\pgfqpoint{1.652850in}{0.692252in}}%
\pgfpathlineto{\pgfqpoint{1.635261in}{0.692252in}}%
\pgfpathclose%
\pgfusepath{fill}%
\end{pgfscope}%
\begin{pgfscope}%
\pgfpathrectangle{\pgfqpoint{0.526905in}{0.383578in}}{\pgfqpoint{3.875000in}{2.310000in}}%
\pgfusepath{clip}%
\pgfsetbuttcap%
\pgfsetmiterjoin%
\definecolor{currentfill}{rgb}{0.686275,0.352941,0.313725}%
\pgfsetfillcolor{currentfill}%
\pgfsetfillopacity{0.300000}%
\pgfsetlinewidth{0.000000pt}%
\definecolor{currentstroke}{rgb}{0.000000,0.000000,0.000000}%
\pgfsetstrokecolor{currentstroke}%
\pgfsetstrokeopacity{0.300000}%
\pgfsetdash{}{0pt}%
\pgfpathmoveto{\pgfqpoint{1.652850in}{0.383578in}}%
\pgfpathlineto{\pgfqpoint{1.670439in}{0.383578in}}%
\pgfpathlineto{\pgfqpoint{1.670439in}{0.681027in}}%
\pgfpathlineto{\pgfqpoint{1.652850in}{0.681027in}}%
\pgfpathclose%
\pgfusepath{fill}%
\end{pgfscope}%
\begin{pgfscope}%
\pgfpathrectangle{\pgfqpoint{0.526905in}{0.383578in}}{\pgfqpoint{3.875000in}{2.310000in}}%
\pgfusepath{clip}%
\pgfsetbuttcap%
\pgfsetmiterjoin%
\definecolor{currentfill}{rgb}{0.686275,0.352941,0.313725}%
\pgfsetfillcolor{currentfill}%
\pgfsetfillopacity{0.300000}%
\pgfsetlinewidth{0.000000pt}%
\definecolor{currentstroke}{rgb}{0.000000,0.000000,0.000000}%
\pgfsetstrokecolor{currentstroke}%
\pgfsetstrokeopacity{0.300000}%
\pgfsetdash{}{0pt}%
\pgfpathmoveto{\pgfqpoint{1.670439in}{0.383578in}}%
\pgfpathlineto{\pgfqpoint{1.688028in}{0.383578in}}%
\pgfpathlineto{\pgfqpoint{1.688028in}{0.681027in}}%
\pgfpathlineto{\pgfqpoint{1.670439in}{0.681027in}}%
\pgfpathclose%
\pgfusepath{fill}%
\end{pgfscope}%
\begin{pgfscope}%
\pgfpathrectangle{\pgfqpoint{0.526905in}{0.383578in}}{\pgfqpoint{3.875000in}{2.310000in}}%
\pgfusepath{clip}%
\pgfsetbuttcap%
\pgfsetmiterjoin%
\definecolor{currentfill}{rgb}{0.686275,0.352941,0.313725}%
\pgfsetfillcolor{currentfill}%
\pgfsetfillopacity{0.300000}%
\pgfsetlinewidth{0.000000pt}%
\definecolor{currentstroke}{rgb}{0.000000,0.000000,0.000000}%
\pgfsetstrokecolor{currentstroke}%
\pgfsetstrokeopacity{0.300000}%
\pgfsetdash{}{0pt}%
\pgfpathmoveto{\pgfqpoint{1.688028in}{0.383578in}}%
\pgfpathlineto{\pgfqpoint{1.705617in}{0.383578in}}%
\pgfpathlineto{\pgfqpoint{1.705617in}{0.793272in}}%
\pgfpathlineto{\pgfqpoint{1.688028in}{0.793272in}}%
\pgfpathclose%
\pgfusepath{fill}%
\end{pgfscope}%
\begin{pgfscope}%
\pgfpathrectangle{\pgfqpoint{0.526905in}{0.383578in}}{\pgfqpoint{3.875000in}{2.310000in}}%
\pgfusepath{clip}%
\pgfsetbuttcap%
\pgfsetmiterjoin%
\definecolor{currentfill}{rgb}{0.686275,0.352941,0.313725}%
\pgfsetfillcolor{currentfill}%
\pgfsetfillopacity{0.300000}%
\pgfsetlinewidth{0.000000pt}%
\definecolor{currentstroke}{rgb}{0.000000,0.000000,0.000000}%
\pgfsetstrokecolor{currentstroke}%
\pgfsetstrokeopacity{0.300000}%
\pgfsetdash{}{0pt}%
\pgfpathmoveto{\pgfqpoint{1.705617in}{0.383578in}}%
\pgfpathlineto{\pgfqpoint{1.723206in}{0.383578in}}%
\pgfpathlineto{\pgfqpoint{1.723206in}{0.798884in}}%
\pgfpathlineto{\pgfqpoint{1.705617in}{0.798884in}}%
\pgfpathclose%
\pgfusepath{fill}%
\end{pgfscope}%
\begin{pgfscope}%
\pgfpathrectangle{\pgfqpoint{0.526905in}{0.383578in}}{\pgfqpoint{3.875000in}{2.310000in}}%
\pgfusepath{clip}%
\pgfsetbuttcap%
\pgfsetmiterjoin%
\definecolor{currentfill}{rgb}{0.686275,0.352941,0.313725}%
\pgfsetfillcolor{currentfill}%
\pgfsetfillopacity{0.300000}%
\pgfsetlinewidth{0.000000pt}%
\definecolor{currentstroke}{rgb}{0.000000,0.000000,0.000000}%
\pgfsetstrokecolor{currentstroke}%
\pgfsetstrokeopacity{0.300000}%
\pgfsetdash{}{0pt}%
\pgfpathmoveto{\pgfqpoint{1.723206in}{0.383578in}}%
\pgfpathlineto{\pgfqpoint{1.740795in}{0.383578in}}%
\pgfpathlineto{\pgfqpoint{1.740795in}{0.714701in}}%
\pgfpathlineto{\pgfqpoint{1.723206in}{0.714701in}}%
\pgfpathclose%
\pgfusepath{fill}%
\end{pgfscope}%
\begin{pgfscope}%
\pgfpathrectangle{\pgfqpoint{0.526905in}{0.383578in}}{\pgfqpoint{3.875000in}{2.310000in}}%
\pgfusepath{clip}%
\pgfsetbuttcap%
\pgfsetmiterjoin%
\definecolor{currentfill}{rgb}{0.686275,0.352941,0.313725}%
\pgfsetfillcolor{currentfill}%
\pgfsetfillopacity{0.300000}%
\pgfsetlinewidth{0.000000pt}%
\definecolor{currentstroke}{rgb}{0.000000,0.000000,0.000000}%
\pgfsetstrokecolor{currentstroke}%
\pgfsetstrokeopacity{0.300000}%
\pgfsetdash{}{0pt}%
\pgfpathmoveto{\pgfqpoint{1.740795in}{0.383578in}}%
\pgfpathlineto{\pgfqpoint{1.758384in}{0.383578in}}%
\pgfpathlineto{\pgfqpoint{1.758384in}{0.697864in}}%
\pgfpathlineto{\pgfqpoint{1.740795in}{0.697864in}}%
\pgfpathclose%
\pgfusepath{fill}%
\end{pgfscope}%
\begin{pgfscope}%
\pgfpathrectangle{\pgfqpoint{0.526905in}{0.383578in}}{\pgfqpoint{3.875000in}{2.310000in}}%
\pgfusepath{clip}%
\pgfsetbuttcap%
\pgfsetmiterjoin%
\definecolor{currentfill}{rgb}{0.686275,0.352941,0.313725}%
\pgfsetfillcolor{currentfill}%
\pgfsetfillopacity{0.300000}%
\pgfsetlinewidth{0.000000pt}%
\definecolor{currentstroke}{rgb}{0.000000,0.000000,0.000000}%
\pgfsetstrokecolor{currentstroke}%
\pgfsetstrokeopacity{0.300000}%
\pgfsetdash{}{0pt}%
\pgfpathmoveto{\pgfqpoint{1.758384in}{0.383578in}}%
\pgfpathlineto{\pgfqpoint{1.775973in}{0.383578in}}%
\pgfpathlineto{\pgfqpoint{1.775973in}{0.787660in}}%
\pgfpathlineto{\pgfqpoint{1.758384in}{0.787660in}}%
\pgfpathclose%
\pgfusepath{fill}%
\end{pgfscope}%
\begin{pgfscope}%
\pgfpathrectangle{\pgfqpoint{0.526905in}{0.383578in}}{\pgfqpoint{3.875000in}{2.310000in}}%
\pgfusepath{clip}%
\pgfsetbuttcap%
\pgfsetmiterjoin%
\definecolor{currentfill}{rgb}{0.686275,0.352941,0.313725}%
\pgfsetfillcolor{currentfill}%
\pgfsetfillopacity{0.300000}%
\pgfsetlinewidth{0.000000pt}%
\definecolor{currentstroke}{rgb}{0.000000,0.000000,0.000000}%
\pgfsetstrokecolor{currentstroke}%
\pgfsetstrokeopacity{0.300000}%
\pgfsetdash{}{0pt}%
\pgfpathmoveto{\pgfqpoint{1.775973in}{0.383578in}}%
\pgfpathlineto{\pgfqpoint{1.793562in}{0.383578in}}%
\pgfpathlineto{\pgfqpoint{1.793562in}{0.843782in}}%
\pgfpathlineto{\pgfqpoint{1.775973in}{0.843782in}}%
\pgfpathclose%
\pgfusepath{fill}%
\end{pgfscope}%
\begin{pgfscope}%
\pgfpathrectangle{\pgfqpoint{0.526905in}{0.383578in}}{\pgfqpoint{3.875000in}{2.310000in}}%
\pgfusepath{clip}%
\pgfsetbuttcap%
\pgfsetmiterjoin%
\definecolor{currentfill}{rgb}{0.686275,0.352941,0.313725}%
\pgfsetfillcolor{currentfill}%
\pgfsetfillopacity{0.300000}%
\pgfsetlinewidth{0.000000pt}%
\definecolor{currentstroke}{rgb}{0.000000,0.000000,0.000000}%
\pgfsetstrokecolor{currentstroke}%
\pgfsetstrokeopacity{0.300000}%
\pgfsetdash{}{0pt}%
\pgfpathmoveto{\pgfqpoint{1.793562in}{0.383578in}}%
\pgfpathlineto{\pgfqpoint{1.811151in}{0.383578in}}%
\pgfpathlineto{\pgfqpoint{1.811151in}{0.888680in}}%
\pgfpathlineto{\pgfqpoint{1.793562in}{0.888680in}}%
\pgfpathclose%
\pgfusepath{fill}%
\end{pgfscope}%
\begin{pgfscope}%
\pgfpathrectangle{\pgfqpoint{0.526905in}{0.383578in}}{\pgfqpoint{3.875000in}{2.310000in}}%
\pgfusepath{clip}%
\pgfsetbuttcap%
\pgfsetmiterjoin%
\definecolor{currentfill}{rgb}{0.686275,0.352941,0.313725}%
\pgfsetfillcolor{currentfill}%
\pgfsetfillopacity{0.300000}%
\pgfsetlinewidth{0.000000pt}%
\definecolor{currentstroke}{rgb}{0.000000,0.000000,0.000000}%
\pgfsetstrokecolor{currentstroke}%
\pgfsetstrokeopacity{0.300000}%
\pgfsetdash{}{0pt}%
\pgfpathmoveto{\pgfqpoint{1.811151in}{0.383578in}}%
\pgfpathlineto{\pgfqpoint{1.828741in}{0.383578in}}%
\pgfpathlineto{\pgfqpoint{1.828741in}{0.939190in}}%
\pgfpathlineto{\pgfqpoint{1.811151in}{0.939190in}}%
\pgfpathclose%
\pgfusepath{fill}%
\end{pgfscope}%
\begin{pgfscope}%
\pgfpathrectangle{\pgfqpoint{0.526905in}{0.383578in}}{\pgfqpoint{3.875000in}{2.310000in}}%
\pgfusepath{clip}%
\pgfsetbuttcap%
\pgfsetmiterjoin%
\definecolor{currentfill}{rgb}{0.686275,0.352941,0.313725}%
\pgfsetfillcolor{currentfill}%
\pgfsetfillopacity{0.300000}%
\pgfsetlinewidth{0.000000pt}%
\definecolor{currentstroke}{rgb}{0.000000,0.000000,0.000000}%
\pgfsetstrokecolor{currentstroke}%
\pgfsetstrokeopacity{0.300000}%
\pgfsetdash{}{0pt}%
\pgfpathmoveto{\pgfqpoint{1.828741in}{0.383578in}}%
\pgfpathlineto{\pgfqpoint{1.846330in}{0.383578in}}%
\pgfpathlineto{\pgfqpoint{1.846330in}{0.821333in}}%
\pgfpathlineto{\pgfqpoint{1.828741in}{0.821333in}}%
\pgfpathclose%
\pgfusepath{fill}%
\end{pgfscope}%
\begin{pgfscope}%
\pgfpathrectangle{\pgfqpoint{0.526905in}{0.383578in}}{\pgfqpoint{3.875000in}{2.310000in}}%
\pgfusepath{clip}%
\pgfsetbuttcap%
\pgfsetmiterjoin%
\definecolor{currentfill}{rgb}{0.686275,0.352941,0.313725}%
\pgfsetfillcolor{currentfill}%
\pgfsetfillopacity{0.300000}%
\pgfsetlinewidth{0.000000pt}%
\definecolor{currentstroke}{rgb}{0.000000,0.000000,0.000000}%
\pgfsetstrokecolor{currentstroke}%
\pgfsetstrokeopacity{0.300000}%
\pgfsetdash{}{0pt}%
\pgfpathmoveto{\pgfqpoint{1.846330in}{0.383578in}}%
\pgfpathlineto{\pgfqpoint{1.863919in}{0.383578in}}%
\pgfpathlineto{\pgfqpoint{1.863919in}{0.871843in}}%
\pgfpathlineto{\pgfqpoint{1.846330in}{0.871843in}}%
\pgfpathclose%
\pgfusepath{fill}%
\end{pgfscope}%
\begin{pgfscope}%
\pgfpathrectangle{\pgfqpoint{0.526905in}{0.383578in}}{\pgfqpoint{3.875000in}{2.310000in}}%
\pgfusepath{clip}%
\pgfsetbuttcap%
\pgfsetmiterjoin%
\definecolor{currentfill}{rgb}{0.686275,0.352941,0.313725}%
\pgfsetfillcolor{currentfill}%
\pgfsetfillopacity{0.300000}%
\pgfsetlinewidth{0.000000pt}%
\definecolor{currentstroke}{rgb}{0.000000,0.000000,0.000000}%
\pgfsetstrokecolor{currentstroke}%
\pgfsetstrokeopacity{0.300000}%
\pgfsetdash{}{0pt}%
\pgfpathmoveto{\pgfqpoint{1.863919in}{0.383578in}}%
\pgfpathlineto{\pgfqpoint{1.881508in}{0.383578in}}%
\pgfpathlineto{\pgfqpoint{1.881508in}{0.888680in}}%
\pgfpathlineto{\pgfqpoint{1.863919in}{0.888680in}}%
\pgfpathclose%
\pgfusepath{fill}%
\end{pgfscope}%
\begin{pgfscope}%
\pgfpathrectangle{\pgfqpoint{0.526905in}{0.383578in}}{\pgfqpoint{3.875000in}{2.310000in}}%
\pgfusepath{clip}%
\pgfsetbuttcap%
\pgfsetmiterjoin%
\definecolor{currentfill}{rgb}{0.686275,0.352941,0.313725}%
\pgfsetfillcolor{currentfill}%
\pgfsetfillopacity{0.300000}%
\pgfsetlinewidth{0.000000pt}%
\definecolor{currentstroke}{rgb}{0.000000,0.000000,0.000000}%
\pgfsetstrokecolor{currentstroke}%
\pgfsetstrokeopacity{0.300000}%
\pgfsetdash{}{0pt}%
\pgfpathmoveto{\pgfqpoint{1.881508in}{0.383578in}}%
\pgfpathlineto{\pgfqpoint{1.899097in}{0.383578in}}%
\pgfpathlineto{\pgfqpoint{1.899097in}{1.040211in}}%
\pgfpathlineto{\pgfqpoint{1.881508in}{1.040211in}}%
\pgfpathclose%
\pgfusepath{fill}%
\end{pgfscope}%
\begin{pgfscope}%
\pgfpathrectangle{\pgfqpoint{0.526905in}{0.383578in}}{\pgfqpoint{3.875000in}{2.310000in}}%
\pgfusepath{clip}%
\pgfsetbuttcap%
\pgfsetmiterjoin%
\definecolor{currentfill}{rgb}{0.686275,0.352941,0.313725}%
\pgfsetfillcolor{currentfill}%
\pgfsetfillopacity{0.300000}%
\pgfsetlinewidth{0.000000pt}%
\definecolor{currentstroke}{rgb}{0.000000,0.000000,0.000000}%
\pgfsetstrokecolor{currentstroke}%
\pgfsetstrokeopacity{0.300000}%
\pgfsetdash{}{0pt}%
\pgfpathmoveto{\pgfqpoint{1.899097in}{0.383578in}}%
\pgfpathlineto{\pgfqpoint{1.916686in}{0.383578in}}%
\pgfpathlineto{\pgfqpoint{1.916686in}{1.068272in}}%
\pgfpathlineto{\pgfqpoint{1.899097in}{1.068272in}}%
\pgfpathclose%
\pgfusepath{fill}%
\end{pgfscope}%
\begin{pgfscope}%
\pgfpathrectangle{\pgfqpoint{0.526905in}{0.383578in}}{\pgfqpoint{3.875000in}{2.310000in}}%
\pgfusepath{clip}%
\pgfsetbuttcap%
\pgfsetmiterjoin%
\definecolor{currentfill}{rgb}{0.686275,0.352941,0.313725}%
\pgfsetfillcolor{currentfill}%
\pgfsetfillopacity{0.300000}%
\pgfsetlinewidth{0.000000pt}%
\definecolor{currentstroke}{rgb}{0.000000,0.000000,0.000000}%
\pgfsetstrokecolor{currentstroke}%
\pgfsetstrokeopacity{0.300000}%
\pgfsetdash{}{0pt}%
\pgfpathmoveto{\pgfqpoint{1.916686in}{0.383578in}}%
\pgfpathlineto{\pgfqpoint{1.934275in}{0.383578in}}%
\pgfpathlineto{\pgfqpoint{1.934275in}{1.101945in}}%
\pgfpathlineto{\pgfqpoint{1.916686in}{1.101945in}}%
\pgfpathclose%
\pgfusepath{fill}%
\end{pgfscope}%
\begin{pgfscope}%
\pgfpathrectangle{\pgfqpoint{0.526905in}{0.383578in}}{\pgfqpoint{3.875000in}{2.310000in}}%
\pgfusepath{clip}%
\pgfsetbuttcap%
\pgfsetmiterjoin%
\definecolor{currentfill}{rgb}{0.686275,0.352941,0.313725}%
\pgfsetfillcolor{currentfill}%
\pgfsetfillopacity{0.300000}%
\pgfsetlinewidth{0.000000pt}%
\definecolor{currentstroke}{rgb}{0.000000,0.000000,0.000000}%
\pgfsetstrokecolor{currentstroke}%
\pgfsetstrokeopacity{0.300000}%
\pgfsetdash{}{0pt}%
\pgfpathmoveto{\pgfqpoint{1.934275in}{0.383578in}}%
\pgfpathlineto{\pgfqpoint{1.951864in}{0.383578in}}%
\pgfpathlineto{\pgfqpoint{1.951864in}{0.922354in}}%
\pgfpathlineto{\pgfqpoint{1.934275in}{0.922354in}}%
\pgfpathclose%
\pgfusepath{fill}%
\end{pgfscope}%
\begin{pgfscope}%
\pgfpathrectangle{\pgfqpoint{0.526905in}{0.383578in}}{\pgfqpoint{3.875000in}{2.310000in}}%
\pgfusepath{clip}%
\pgfsetbuttcap%
\pgfsetmiterjoin%
\definecolor{currentfill}{rgb}{0.686275,0.352941,0.313725}%
\pgfsetfillcolor{currentfill}%
\pgfsetfillopacity{0.300000}%
\pgfsetlinewidth{0.000000pt}%
\definecolor{currentstroke}{rgb}{0.000000,0.000000,0.000000}%
\pgfsetstrokecolor{currentstroke}%
\pgfsetstrokeopacity{0.300000}%
\pgfsetdash{}{0pt}%
\pgfpathmoveto{\pgfqpoint{1.951864in}{0.383578in}}%
\pgfpathlineto{\pgfqpoint{1.969453in}{0.383578in}}%
\pgfpathlineto{\pgfqpoint{1.969453in}{1.186129in}}%
\pgfpathlineto{\pgfqpoint{1.951864in}{1.186129in}}%
\pgfpathclose%
\pgfusepath{fill}%
\end{pgfscope}%
\begin{pgfscope}%
\pgfpathrectangle{\pgfqpoint{0.526905in}{0.383578in}}{\pgfqpoint{3.875000in}{2.310000in}}%
\pgfusepath{clip}%
\pgfsetbuttcap%
\pgfsetmiterjoin%
\definecolor{currentfill}{rgb}{0.686275,0.352941,0.313725}%
\pgfsetfillcolor{currentfill}%
\pgfsetfillopacity{0.300000}%
\pgfsetlinewidth{0.000000pt}%
\definecolor{currentstroke}{rgb}{0.000000,0.000000,0.000000}%
\pgfsetstrokecolor{currentstroke}%
\pgfsetstrokeopacity{0.300000}%
\pgfsetdash{}{0pt}%
\pgfpathmoveto{\pgfqpoint{1.969453in}{0.383578in}}%
\pgfpathlineto{\pgfqpoint{1.987042in}{0.383578in}}%
\pgfpathlineto{\pgfqpoint{1.987042in}{1.180517in}}%
\pgfpathlineto{\pgfqpoint{1.969453in}{1.180517in}}%
\pgfpathclose%
\pgfusepath{fill}%
\end{pgfscope}%
\begin{pgfscope}%
\pgfpathrectangle{\pgfqpoint{0.526905in}{0.383578in}}{\pgfqpoint{3.875000in}{2.310000in}}%
\pgfusepath{clip}%
\pgfsetbuttcap%
\pgfsetmiterjoin%
\definecolor{currentfill}{rgb}{0.686275,0.352941,0.313725}%
\pgfsetfillcolor{currentfill}%
\pgfsetfillopacity{0.300000}%
\pgfsetlinewidth{0.000000pt}%
\definecolor{currentstroke}{rgb}{0.000000,0.000000,0.000000}%
\pgfsetstrokecolor{currentstroke}%
\pgfsetstrokeopacity{0.300000}%
\pgfsetdash{}{0pt}%
\pgfpathmoveto{\pgfqpoint{1.987042in}{0.383578in}}%
\pgfpathlineto{\pgfqpoint{2.004631in}{0.383578in}}%
\pgfpathlineto{\pgfqpoint{2.004631in}{1.158068in}}%
\pgfpathlineto{\pgfqpoint{1.987042in}{1.158068in}}%
\pgfpathclose%
\pgfusepath{fill}%
\end{pgfscope}%
\begin{pgfscope}%
\pgfpathrectangle{\pgfqpoint{0.526905in}{0.383578in}}{\pgfqpoint{3.875000in}{2.310000in}}%
\pgfusepath{clip}%
\pgfsetbuttcap%
\pgfsetmiterjoin%
\definecolor{currentfill}{rgb}{0.686275,0.352941,0.313725}%
\pgfsetfillcolor{currentfill}%
\pgfsetfillopacity{0.300000}%
\pgfsetlinewidth{0.000000pt}%
\definecolor{currentstroke}{rgb}{0.000000,0.000000,0.000000}%
\pgfsetstrokecolor{currentstroke}%
\pgfsetstrokeopacity{0.300000}%
\pgfsetdash{}{0pt}%
\pgfpathmoveto{\pgfqpoint{2.004631in}{0.383578in}}%
\pgfpathlineto{\pgfqpoint{2.022220in}{0.383578in}}%
\pgfpathlineto{\pgfqpoint{2.022220in}{1.191741in}}%
\pgfpathlineto{\pgfqpoint{2.004631in}{1.191741in}}%
\pgfpathclose%
\pgfusepath{fill}%
\end{pgfscope}%
\begin{pgfscope}%
\pgfpathrectangle{\pgfqpoint{0.526905in}{0.383578in}}{\pgfqpoint{3.875000in}{2.310000in}}%
\pgfusepath{clip}%
\pgfsetbuttcap%
\pgfsetmiterjoin%
\definecolor{currentfill}{rgb}{0.686275,0.352941,0.313725}%
\pgfsetfillcolor{currentfill}%
\pgfsetfillopacity{0.300000}%
\pgfsetlinewidth{0.000000pt}%
\definecolor{currentstroke}{rgb}{0.000000,0.000000,0.000000}%
\pgfsetstrokecolor{currentstroke}%
\pgfsetstrokeopacity{0.300000}%
\pgfsetdash{}{0pt}%
\pgfpathmoveto{\pgfqpoint{2.022220in}{0.383578in}}%
\pgfpathlineto{\pgfqpoint{2.039809in}{0.383578in}}%
\pgfpathlineto{\pgfqpoint{2.039809in}{1.186129in}}%
\pgfpathlineto{\pgfqpoint{2.022220in}{1.186129in}}%
\pgfpathclose%
\pgfusepath{fill}%
\end{pgfscope}%
\begin{pgfscope}%
\pgfpathrectangle{\pgfqpoint{0.526905in}{0.383578in}}{\pgfqpoint{3.875000in}{2.310000in}}%
\pgfusepath{clip}%
\pgfsetbuttcap%
\pgfsetmiterjoin%
\definecolor{currentfill}{rgb}{0.686275,0.352941,0.313725}%
\pgfsetfillcolor{currentfill}%
\pgfsetfillopacity{0.300000}%
\pgfsetlinewidth{0.000000pt}%
\definecolor{currentstroke}{rgb}{0.000000,0.000000,0.000000}%
\pgfsetstrokecolor{currentstroke}%
\pgfsetstrokeopacity{0.300000}%
\pgfsetdash{}{0pt}%
\pgfpathmoveto{\pgfqpoint{2.039809in}{0.383578in}}%
\pgfpathlineto{\pgfqpoint{2.057398in}{0.383578in}}%
\pgfpathlineto{\pgfqpoint{2.057398in}{1.225415in}}%
\pgfpathlineto{\pgfqpoint{2.039809in}{1.225415in}}%
\pgfpathclose%
\pgfusepath{fill}%
\end{pgfscope}%
\begin{pgfscope}%
\pgfpathrectangle{\pgfqpoint{0.526905in}{0.383578in}}{\pgfqpoint{3.875000in}{2.310000in}}%
\pgfusepath{clip}%
\pgfsetbuttcap%
\pgfsetmiterjoin%
\definecolor{currentfill}{rgb}{0.686275,0.352941,0.313725}%
\pgfsetfillcolor{currentfill}%
\pgfsetfillopacity{0.300000}%
\pgfsetlinewidth{0.000000pt}%
\definecolor{currentstroke}{rgb}{0.000000,0.000000,0.000000}%
\pgfsetstrokecolor{currentstroke}%
\pgfsetstrokeopacity{0.300000}%
\pgfsetdash{}{0pt}%
\pgfpathmoveto{\pgfqpoint{2.057398in}{0.383578in}}%
\pgfpathlineto{\pgfqpoint{2.074987in}{0.383578in}}%
\pgfpathlineto{\pgfqpoint{2.074987in}{1.247864in}}%
\pgfpathlineto{\pgfqpoint{2.057398in}{1.247864in}}%
\pgfpathclose%
\pgfusepath{fill}%
\end{pgfscope}%
\begin{pgfscope}%
\pgfpathrectangle{\pgfqpoint{0.526905in}{0.383578in}}{\pgfqpoint{3.875000in}{2.310000in}}%
\pgfusepath{clip}%
\pgfsetbuttcap%
\pgfsetmiterjoin%
\definecolor{currentfill}{rgb}{0.686275,0.352941,0.313725}%
\pgfsetfillcolor{currentfill}%
\pgfsetfillopacity{0.300000}%
\pgfsetlinewidth{0.000000pt}%
\definecolor{currentstroke}{rgb}{0.000000,0.000000,0.000000}%
\pgfsetstrokecolor{currentstroke}%
\pgfsetstrokeopacity{0.300000}%
\pgfsetdash{}{0pt}%
\pgfpathmoveto{\pgfqpoint{2.074987in}{0.383578in}}%
\pgfpathlineto{\pgfqpoint{2.092576in}{0.383578in}}%
\pgfpathlineto{\pgfqpoint{2.092576in}{1.416231in}}%
\pgfpathlineto{\pgfqpoint{2.074987in}{1.416231in}}%
\pgfpathclose%
\pgfusepath{fill}%
\end{pgfscope}%
\begin{pgfscope}%
\pgfpathrectangle{\pgfqpoint{0.526905in}{0.383578in}}{\pgfqpoint{3.875000in}{2.310000in}}%
\pgfusepath{clip}%
\pgfsetbuttcap%
\pgfsetmiterjoin%
\definecolor{currentfill}{rgb}{0.686275,0.352941,0.313725}%
\pgfsetfillcolor{currentfill}%
\pgfsetfillopacity{0.300000}%
\pgfsetlinewidth{0.000000pt}%
\definecolor{currentstroke}{rgb}{0.000000,0.000000,0.000000}%
\pgfsetstrokecolor{currentstroke}%
\pgfsetstrokeopacity{0.300000}%
\pgfsetdash{}{0pt}%
\pgfpathmoveto{\pgfqpoint{2.092576in}{0.383578in}}%
\pgfpathlineto{\pgfqpoint{2.110165in}{0.383578in}}%
\pgfpathlineto{\pgfqpoint{2.110165in}{1.231027in}}%
\pgfpathlineto{\pgfqpoint{2.092576in}{1.231027in}}%
\pgfpathclose%
\pgfusepath{fill}%
\end{pgfscope}%
\begin{pgfscope}%
\pgfpathrectangle{\pgfqpoint{0.526905in}{0.383578in}}{\pgfqpoint{3.875000in}{2.310000in}}%
\pgfusepath{clip}%
\pgfsetbuttcap%
\pgfsetmiterjoin%
\definecolor{currentfill}{rgb}{0.686275,0.352941,0.313725}%
\pgfsetfillcolor{currentfill}%
\pgfsetfillopacity{0.300000}%
\pgfsetlinewidth{0.000000pt}%
\definecolor{currentstroke}{rgb}{0.000000,0.000000,0.000000}%
\pgfsetstrokecolor{currentstroke}%
\pgfsetstrokeopacity{0.300000}%
\pgfsetdash{}{0pt}%
\pgfpathmoveto{\pgfqpoint{2.110165in}{0.383578in}}%
\pgfpathlineto{\pgfqpoint{2.127754in}{0.383578in}}%
\pgfpathlineto{\pgfqpoint{2.127754in}{1.410619in}}%
\pgfpathlineto{\pgfqpoint{2.110165in}{1.410619in}}%
\pgfpathclose%
\pgfusepath{fill}%
\end{pgfscope}%
\begin{pgfscope}%
\pgfpathrectangle{\pgfqpoint{0.526905in}{0.383578in}}{\pgfqpoint{3.875000in}{2.310000in}}%
\pgfusepath{clip}%
\pgfsetbuttcap%
\pgfsetmiterjoin%
\definecolor{currentfill}{rgb}{0.686275,0.352941,0.313725}%
\pgfsetfillcolor{currentfill}%
\pgfsetfillopacity{0.300000}%
\pgfsetlinewidth{0.000000pt}%
\definecolor{currentstroke}{rgb}{0.000000,0.000000,0.000000}%
\pgfsetstrokecolor{currentstroke}%
\pgfsetstrokeopacity{0.300000}%
\pgfsetdash{}{0pt}%
\pgfpathmoveto{\pgfqpoint{2.127754in}{0.383578in}}%
\pgfpathlineto{\pgfqpoint{2.145343in}{0.383578in}}%
\pgfpathlineto{\pgfqpoint{2.145343in}{1.360109in}}%
\pgfpathlineto{\pgfqpoint{2.127754in}{1.360109in}}%
\pgfpathclose%
\pgfusepath{fill}%
\end{pgfscope}%
\begin{pgfscope}%
\pgfpathrectangle{\pgfqpoint{0.526905in}{0.383578in}}{\pgfqpoint{3.875000in}{2.310000in}}%
\pgfusepath{clip}%
\pgfsetbuttcap%
\pgfsetmiterjoin%
\definecolor{currentfill}{rgb}{0.686275,0.352941,0.313725}%
\pgfsetfillcolor{currentfill}%
\pgfsetfillopacity{0.300000}%
\pgfsetlinewidth{0.000000pt}%
\definecolor{currentstroke}{rgb}{0.000000,0.000000,0.000000}%
\pgfsetstrokecolor{currentstroke}%
\pgfsetstrokeopacity{0.300000}%
\pgfsetdash{}{0pt}%
\pgfpathmoveto{\pgfqpoint{2.145343in}{0.383578in}}%
\pgfpathlineto{\pgfqpoint{2.162933in}{0.383578in}}%
\pgfpathlineto{\pgfqpoint{2.162933in}{1.483578in}}%
\pgfpathlineto{\pgfqpoint{2.145343in}{1.483578in}}%
\pgfpathclose%
\pgfusepath{fill}%
\end{pgfscope}%
\begin{pgfscope}%
\pgfpathrectangle{\pgfqpoint{0.526905in}{0.383578in}}{\pgfqpoint{3.875000in}{2.310000in}}%
\pgfusepath{clip}%
\pgfsetbuttcap%
\pgfsetmiterjoin%
\definecolor{currentfill}{rgb}{0.686275,0.352941,0.313725}%
\pgfsetfillcolor{currentfill}%
\pgfsetfillopacity{0.300000}%
\pgfsetlinewidth{0.000000pt}%
\definecolor{currentstroke}{rgb}{0.000000,0.000000,0.000000}%
\pgfsetstrokecolor{currentstroke}%
\pgfsetstrokeopacity{0.300000}%
\pgfsetdash{}{0pt}%
\pgfpathmoveto{\pgfqpoint{2.162933in}{0.383578in}}%
\pgfpathlineto{\pgfqpoint{2.180522in}{0.383578in}}%
\pgfpathlineto{\pgfqpoint{2.180522in}{1.405007in}}%
\pgfpathlineto{\pgfqpoint{2.162933in}{1.405007in}}%
\pgfpathclose%
\pgfusepath{fill}%
\end{pgfscope}%
\begin{pgfscope}%
\pgfpathrectangle{\pgfqpoint{0.526905in}{0.383578in}}{\pgfqpoint{3.875000in}{2.310000in}}%
\pgfusepath{clip}%
\pgfsetbuttcap%
\pgfsetmiterjoin%
\definecolor{currentfill}{rgb}{0.686275,0.352941,0.313725}%
\pgfsetfillcolor{currentfill}%
\pgfsetfillopacity{0.300000}%
\pgfsetlinewidth{0.000000pt}%
\definecolor{currentstroke}{rgb}{0.000000,0.000000,0.000000}%
\pgfsetstrokecolor{currentstroke}%
\pgfsetstrokeopacity{0.300000}%
\pgfsetdash{}{0pt}%
\pgfpathmoveto{\pgfqpoint{2.180522in}{0.383578in}}%
\pgfpathlineto{\pgfqpoint{2.198111in}{0.383578in}}%
\pgfpathlineto{\pgfqpoint{2.198111in}{1.573374in}}%
\pgfpathlineto{\pgfqpoint{2.180522in}{1.573374in}}%
\pgfpathclose%
\pgfusepath{fill}%
\end{pgfscope}%
\begin{pgfscope}%
\pgfpathrectangle{\pgfqpoint{0.526905in}{0.383578in}}{\pgfqpoint{3.875000in}{2.310000in}}%
\pgfusepath{clip}%
\pgfsetbuttcap%
\pgfsetmiterjoin%
\definecolor{currentfill}{rgb}{0.686275,0.352941,0.313725}%
\pgfsetfillcolor{currentfill}%
\pgfsetfillopacity{0.300000}%
\pgfsetlinewidth{0.000000pt}%
\definecolor{currentstroke}{rgb}{0.000000,0.000000,0.000000}%
\pgfsetstrokecolor{currentstroke}%
\pgfsetstrokeopacity{0.300000}%
\pgfsetdash{}{0pt}%
\pgfpathmoveto{\pgfqpoint{2.198111in}{0.383578in}}%
\pgfpathlineto{\pgfqpoint{2.215700in}{0.383578in}}%
\pgfpathlineto{\pgfqpoint{2.215700in}{1.528476in}}%
\pgfpathlineto{\pgfqpoint{2.198111in}{1.528476in}}%
\pgfpathclose%
\pgfusepath{fill}%
\end{pgfscope}%
\begin{pgfscope}%
\pgfpathrectangle{\pgfqpoint{0.526905in}{0.383578in}}{\pgfqpoint{3.875000in}{2.310000in}}%
\pgfusepath{clip}%
\pgfsetbuttcap%
\pgfsetmiterjoin%
\definecolor{currentfill}{rgb}{0.686275,0.352941,0.313725}%
\pgfsetfillcolor{currentfill}%
\pgfsetfillopacity{0.300000}%
\pgfsetlinewidth{0.000000pt}%
\definecolor{currentstroke}{rgb}{0.000000,0.000000,0.000000}%
\pgfsetstrokecolor{currentstroke}%
\pgfsetstrokeopacity{0.300000}%
\pgfsetdash{}{0pt}%
\pgfpathmoveto{\pgfqpoint{2.215700in}{0.383578in}}%
\pgfpathlineto{\pgfqpoint{2.233289in}{0.383578in}}%
\pgfpathlineto{\pgfqpoint{2.233289in}{1.629496in}}%
\pgfpathlineto{\pgfqpoint{2.215700in}{1.629496in}}%
\pgfpathclose%
\pgfusepath{fill}%
\end{pgfscope}%
\begin{pgfscope}%
\pgfpathrectangle{\pgfqpoint{0.526905in}{0.383578in}}{\pgfqpoint{3.875000in}{2.310000in}}%
\pgfusepath{clip}%
\pgfsetbuttcap%
\pgfsetmiterjoin%
\definecolor{currentfill}{rgb}{0.686275,0.352941,0.313725}%
\pgfsetfillcolor{currentfill}%
\pgfsetfillopacity{0.300000}%
\pgfsetlinewidth{0.000000pt}%
\definecolor{currentstroke}{rgb}{0.000000,0.000000,0.000000}%
\pgfsetstrokecolor{currentstroke}%
\pgfsetstrokeopacity{0.300000}%
\pgfsetdash{}{0pt}%
\pgfpathmoveto{\pgfqpoint{2.233289in}{0.383578in}}%
\pgfpathlineto{\pgfqpoint{2.250878in}{0.383578in}}%
\pgfpathlineto{\pgfqpoint{2.250878in}{1.545313in}}%
\pgfpathlineto{\pgfqpoint{2.233289in}{1.545313in}}%
\pgfpathclose%
\pgfusepath{fill}%
\end{pgfscope}%
\begin{pgfscope}%
\pgfpathrectangle{\pgfqpoint{0.526905in}{0.383578in}}{\pgfqpoint{3.875000in}{2.310000in}}%
\pgfusepath{clip}%
\pgfsetbuttcap%
\pgfsetmiterjoin%
\definecolor{currentfill}{rgb}{0.686275,0.352941,0.313725}%
\pgfsetfillcolor{currentfill}%
\pgfsetfillopacity{0.300000}%
\pgfsetlinewidth{0.000000pt}%
\definecolor{currentstroke}{rgb}{0.000000,0.000000,0.000000}%
\pgfsetstrokecolor{currentstroke}%
\pgfsetstrokeopacity{0.300000}%
\pgfsetdash{}{0pt}%
\pgfpathmoveto{\pgfqpoint{2.250878in}{0.383578in}}%
\pgfpathlineto{\pgfqpoint{2.268467in}{0.383578in}}%
\pgfpathlineto{\pgfqpoint{2.268467in}{1.618272in}}%
\pgfpathlineto{\pgfqpoint{2.250878in}{1.618272in}}%
\pgfpathclose%
\pgfusepath{fill}%
\end{pgfscope}%
\begin{pgfscope}%
\pgfpathrectangle{\pgfqpoint{0.526905in}{0.383578in}}{\pgfqpoint{3.875000in}{2.310000in}}%
\pgfusepath{clip}%
\pgfsetbuttcap%
\pgfsetmiterjoin%
\definecolor{currentfill}{rgb}{0.686275,0.352941,0.313725}%
\pgfsetfillcolor{currentfill}%
\pgfsetfillopacity{0.300000}%
\pgfsetlinewidth{0.000000pt}%
\definecolor{currentstroke}{rgb}{0.000000,0.000000,0.000000}%
\pgfsetstrokecolor{currentstroke}%
\pgfsetstrokeopacity{0.300000}%
\pgfsetdash{}{0pt}%
\pgfpathmoveto{\pgfqpoint{2.268467in}{0.383578in}}%
\pgfpathlineto{\pgfqpoint{2.286056in}{0.383578in}}%
\pgfpathlineto{\pgfqpoint{2.286056in}{1.708068in}}%
\pgfpathlineto{\pgfqpoint{2.268467in}{1.708068in}}%
\pgfpathclose%
\pgfusepath{fill}%
\end{pgfscope}%
\begin{pgfscope}%
\pgfpathrectangle{\pgfqpoint{0.526905in}{0.383578in}}{\pgfqpoint{3.875000in}{2.310000in}}%
\pgfusepath{clip}%
\pgfsetbuttcap%
\pgfsetmiterjoin%
\definecolor{currentfill}{rgb}{0.686275,0.352941,0.313725}%
\pgfsetfillcolor{currentfill}%
\pgfsetfillopacity{0.300000}%
\pgfsetlinewidth{0.000000pt}%
\definecolor{currentstroke}{rgb}{0.000000,0.000000,0.000000}%
\pgfsetstrokecolor{currentstroke}%
\pgfsetstrokeopacity{0.300000}%
\pgfsetdash{}{0pt}%
\pgfpathmoveto{\pgfqpoint{2.286056in}{0.383578in}}%
\pgfpathlineto{\pgfqpoint{2.303645in}{0.383578in}}%
\pgfpathlineto{\pgfqpoint{2.303645in}{1.848374in}}%
\pgfpathlineto{\pgfqpoint{2.286056in}{1.848374in}}%
\pgfpathclose%
\pgfusepath{fill}%
\end{pgfscope}%
\begin{pgfscope}%
\pgfpathrectangle{\pgfqpoint{0.526905in}{0.383578in}}{\pgfqpoint{3.875000in}{2.310000in}}%
\pgfusepath{clip}%
\pgfsetbuttcap%
\pgfsetmiterjoin%
\definecolor{currentfill}{rgb}{0.686275,0.352941,0.313725}%
\pgfsetfillcolor{currentfill}%
\pgfsetfillopacity{0.300000}%
\pgfsetlinewidth{0.000000pt}%
\definecolor{currentstroke}{rgb}{0.000000,0.000000,0.000000}%
\pgfsetstrokecolor{currentstroke}%
\pgfsetstrokeopacity{0.300000}%
\pgfsetdash{}{0pt}%
\pgfpathmoveto{\pgfqpoint{2.303645in}{0.383578in}}%
\pgfpathlineto{\pgfqpoint{2.321234in}{0.383578in}}%
\pgfpathlineto{\pgfqpoint{2.321234in}{1.724905in}}%
\pgfpathlineto{\pgfqpoint{2.303645in}{1.724905in}}%
\pgfpathclose%
\pgfusepath{fill}%
\end{pgfscope}%
\begin{pgfscope}%
\pgfpathrectangle{\pgfqpoint{0.526905in}{0.383578in}}{\pgfqpoint{3.875000in}{2.310000in}}%
\pgfusepath{clip}%
\pgfsetbuttcap%
\pgfsetmiterjoin%
\definecolor{currentfill}{rgb}{0.686275,0.352941,0.313725}%
\pgfsetfillcolor{currentfill}%
\pgfsetfillopacity{0.300000}%
\pgfsetlinewidth{0.000000pt}%
\definecolor{currentstroke}{rgb}{0.000000,0.000000,0.000000}%
\pgfsetstrokecolor{currentstroke}%
\pgfsetstrokeopacity{0.300000}%
\pgfsetdash{}{0pt}%
\pgfpathmoveto{\pgfqpoint{2.321234in}{0.383578in}}%
\pgfpathlineto{\pgfqpoint{2.338823in}{0.383578in}}%
\pgfpathlineto{\pgfqpoint{2.338823in}{1.752966in}}%
\pgfpathlineto{\pgfqpoint{2.321234in}{1.752966in}}%
\pgfpathclose%
\pgfusepath{fill}%
\end{pgfscope}%
\begin{pgfscope}%
\pgfpathrectangle{\pgfqpoint{0.526905in}{0.383578in}}{\pgfqpoint{3.875000in}{2.310000in}}%
\pgfusepath{clip}%
\pgfsetbuttcap%
\pgfsetmiterjoin%
\definecolor{currentfill}{rgb}{0.686275,0.352941,0.313725}%
\pgfsetfillcolor{currentfill}%
\pgfsetfillopacity{0.300000}%
\pgfsetlinewidth{0.000000pt}%
\definecolor{currentstroke}{rgb}{0.000000,0.000000,0.000000}%
\pgfsetstrokecolor{currentstroke}%
\pgfsetstrokeopacity{0.300000}%
\pgfsetdash{}{0pt}%
\pgfpathmoveto{\pgfqpoint{2.338823in}{0.383578in}}%
\pgfpathlineto{\pgfqpoint{2.356412in}{0.383578in}}%
\pgfpathlineto{\pgfqpoint{2.356412in}{1.983068in}}%
\pgfpathlineto{\pgfqpoint{2.338823in}{1.983068in}}%
\pgfpathclose%
\pgfusepath{fill}%
\end{pgfscope}%
\begin{pgfscope}%
\pgfpathrectangle{\pgfqpoint{0.526905in}{0.383578in}}{\pgfqpoint{3.875000in}{2.310000in}}%
\pgfusepath{clip}%
\pgfsetbuttcap%
\pgfsetmiterjoin%
\definecolor{currentfill}{rgb}{0.686275,0.352941,0.313725}%
\pgfsetfillcolor{currentfill}%
\pgfsetfillopacity{0.300000}%
\pgfsetlinewidth{0.000000pt}%
\definecolor{currentstroke}{rgb}{0.000000,0.000000,0.000000}%
\pgfsetstrokecolor{currentstroke}%
\pgfsetstrokeopacity{0.300000}%
\pgfsetdash{}{0pt}%
\pgfpathmoveto{\pgfqpoint{2.356412in}{0.383578in}}%
\pgfpathlineto{\pgfqpoint{2.374001in}{0.383578in}}%
\pgfpathlineto{\pgfqpoint{2.374001in}{2.005517in}}%
\pgfpathlineto{\pgfqpoint{2.356412in}{2.005517in}}%
\pgfpathclose%
\pgfusepath{fill}%
\end{pgfscope}%
\begin{pgfscope}%
\pgfpathrectangle{\pgfqpoint{0.526905in}{0.383578in}}{\pgfqpoint{3.875000in}{2.310000in}}%
\pgfusepath{clip}%
\pgfsetbuttcap%
\pgfsetmiterjoin%
\definecolor{currentfill}{rgb}{0.686275,0.352941,0.313725}%
\pgfsetfillcolor{currentfill}%
\pgfsetfillopacity{0.300000}%
\pgfsetlinewidth{0.000000pt}%
\definecolor{currentstroke}{rgb}{0.000000,0.000000,0.000000}%
\pgfsetstrokecolor{currentstroke}%
\pgfsetstrokeopacity{0.300000}%
\pgfsetdash{}{0pt}%
\pgfpathmoveto{\pgfqpoint{2.374001in}{0.383578in}}%
\pgfpathlineto{\pgfqpoint{2.391590in}{0.383578in}}%
\pgfpathlineto{\pgfqpoint{2.391590in}{1.876435in}}%
\pgfpathlineto{\pgfqpoint{2.374001in}{1.876435in}}%
\pgfpathclose%
\pgfusepath{fill}%
\end{pgfscope}%
\begin{pgfscope}%
\pgfpathrectangle{\pgfqpoint{0.526905in}{0.383578in}}{\pgfqpoint{3.875000in}{2.310000in}}%
\pgfusepath{clip}%
\pgfsetbuttcap%
\pgfsetmiterjoin%
\definecolor{currentfill}{rgb}{0.686275,0.352941,0.313725}%
\pgfsetfillcolor{currentfill}%
\pgfsetfillopacity{0.300000}%
\pgfsetlinewidth{0.000000pt}%
\definecolor{currentstroke}{rgb}{0.000000,0.000000,0.000000}%
\pgfsetstrokecolor{currentstroke}%
\pgfsetstrokeopacity{0.300000}%
\pgfsetdash{}{0pt}%
\pgfpathmoveto{\pgfqpoint{2.391590in}{0.383578in}}%
\pgfpathlineto{\pgfqpoint{2.409179in}{0.383578in}}%
\pgfpathlineto{\pgfqpoint{2.409179in}{2.123374in}}%
\pgfpathlineto{\pgfqpoint{2.391590in}{2.123374in}}%
\pgfpathclose%
\pgfusepath{fill}%
\end{pgfscope}%
\begin{pgfscope}%
\pgfpathrectangle{\pgfqpoint{0.526905in}{0.383578in}}{\pgfqpoint{3.875000in}{2.310000in}}%
\pgfusepath{clip}%
\pgfsetbuttcap%
\pgfsetmiterjoin%
\definecolor{currentfill}{rgb}{0.686275,0.352941,0.313725}%
\pgfsetfillcolor{currentfill}%
\pgfsetfillopacity{0.300000}%
\pgfsetlinewidth{0.000000pt}%
\definecolor{currentstroke}{rgb}{0.000000,0.000000,0.000000}%
\pgfsetstrokecolor{currentstroke}%
\pgfsetstrokeopacity{0.300000}%
\pgfsetdash{}{0pt}%
\pgfpathmoveto{\pgfqpoint{2.409179in}{0.383578in}}%
\pgfpathlineto{\pgfqpoint{2.426768in}{0.383578in}}%
\pgfpathlineto{\pgfqpoint{2.426768in}{2.089701in}}%
\pgfpathlineto{\pgfqpoint{2.409179in}{2.089701in}}%
\pgfpathclose%
\pgfusepath{fill}%
\end{pgfscope}%
\begin{pgfscope}%
\pgfpathrectangle{\pgfqpoint{0.526905in}{0.383578in}}{\pgfqpoint{3.875000in}{2.310000in}}%
\pgfusepath{clip}%
\pgfsetbuttcap%
\pgfsetmiterjoin%
\definecolor{currentfill}{rgb}{0.686275,0.352941,0.313725}%
\pgfsetfillcolor{currentfill}%
\pgfsetfillopacity{0.300000}%
\pgfsetlinewidth{0.000000pt}%
\definecolor{currentstroke}{rgb}{0.000000,0.000000,0.000000}%
\pgfsetstrokecolor{currentstroke}%
\pgfsetstrokeopacity{0.300000}%
\pgfsetdash{}{0pt}%
\pgfpathmoveto{\pgfqpoint{2.426768in}{0.383578in}}%
\pgfpathlineto{\pgfqpoint{2.444357in}{0.383578in}}%
\pgfpathlineto{\pgfqpoint{2.444357in}{2.117762in}}%
\pgfpathlineto{\pgfqpoint{2.426768in}{2.117762in}}%
\pgfpathclose%
\pgfusepath{fill}%
\end{pgfscope}%
\begin{pgfscope}%
\pgfpathrectangle{\pgfqpoint{0.526905in}{0.383578in}}{\pgfqpoint{3.875000in}{2.310000in}}%
\pgfusepath{clip}%
\pgfsetbuttcap%
\pgfsetmiterjoin%
\definecolor{currentfill}{rgb}{0.686275,0.352941,0.313725}%
\pgfsetfillcolor{currentfill}%
\pgfsetfillopacity{0.300000}%
\pgfsetlinewidth{0.000000pt}%
\definecolor{currentstroke}{rgb}{0.000000,0.000000,0.000000}%
\pgfsetstrokecolor{currentstroke}%
\pgfsetstrokeopacity{0.300000}%
\pgfsetdash{}{0pt}%
\pgfpathmoveto{\pgfqpoint{2.444357in}{0.383578in}}%
\pgfpathlineto{\pgfqpoint{2.461946in}{0.383578in}}%
\pgfpathlineto{\pgfqpoint{2.461946in}{2.128986in}}%
\pgfpathlineto{\pgfqpoint{2.444357in}{2.128986in}}%
\pgfpathclose%
\pgfusepath{fill}%
\end{pgfscope}%
\begin{pgfscope}%
\pgfpathrectangle{\pgfqpoint{0.526905in}{0.383578in}}{\pgfqpoint{3.875000in}{2.310000in}}%
\pgfusepath{clip}%
\pgfsetbuttcap%
\pgfsetmiterjoin%
\definecolor{currentfill}{rgb}{0.686275,0.352941,0.313725}%
\pgfsetfillcolor{currentfill}%
\pgfsetfillopacity{0.300000}%
\pgfsetlinewidth{0.000000pt}%
\definecolor{currentstroke}{rgb}{0.000000,0.000000,0.000000}%
\pgfsetstrokecolor{currentstroke}%
\pgfsetstrokeopacity{0.300000}%
\pgfsetdash{}{0pt}%
\pgfpathmoveto{\pgfqpoint{2.461946in}{0.383578in}}%
\pgfpathlineto{\pgfqpoint{2.479535in}{0.383578in}}%
\pgfpathlineto{\pgfqpoint{2.479535in}{2.420823in}}%
\pgfpathlineto{\pgfqpoint{2.461946in}{2.420823in}}%
\pgfpathclose%
\pgfusepath{fill}%
\end{pgfscope}%
\begin{pgfscope}%
\pgfpathrectangle{\pgfqpoint{0.526905in}{0.383578in}}{\pgfqpoint{3.875000in}{2.310000in}}%
\pgfusepath{clip}%
\pgfsetbuttcap%
\pgfsetmiterjoin%
\definecolor{currentfill}{rgb}{0.686275,0.352941,0.313725}%
\pgfsetfillcolor{currentfill}%
\pgfsetfillopacity{0.300000}%
\pgfsetlinewidth{0.000000pt}%
\definecolor{currentstroke}{rgb}{0.000000,0.000000,0.000000}%
\pgfsetstrokecolor{currentstroke}%
\pgfsetstrokeopacity{0.300000}%
\pgfsetdash{}{0pt}%
\pgfpathmoveto{\pgfqpoint{2.479535in}{0.383578in}}%
\pgfpathlineto{\pgfqpoint{2.497125in}{0.383578in}}%
\pgfpathlineto{\pgfqpoint{2.497125in}{2.235619in}}%
\pgfpathlineto{\pgfqpoint{2.479535in}{2.235619in}}%
\pgfpathclose%
\pgfusepath{fill}%
\end{pgfscope}%
\begin{pgfscope}%
\pgfpathrectangle{\pgfqpoint{0.526905in}{0.383578in}}{\pgfqpoint{3.875000in}{2.310000in}}%
\pgfusepath{clip}%
\pgfsetbuttcap%
\pgfsetmiterjoin%
\definecolor{currentfill}{rgb}{0.686275,0.352941,0.313725}%
\pgfsetfillcolor{currentfill}%
\pgfsetfillopacity{0.300000}%
\pgfsetlinewidth{0.000000pt}%
\definecolor{currentstroke}{rgb}{0.000000,0.000000,0.000000}%
\pgfsetstrokecolor{currentstroke}%
\pgfsetstrokeopacity{0.300000}%
\pgfsetdash{}{0pt}%
\pgfpathmoveto{\pgfqpoint{2.497125in}{0.383578in}}%
\pgfpathlineto{\pgfqpoint{2.514714in}{0.383578in}}%
\pgfpathlineto{\pgfqpoint{2.514714in}{2.381537in}}%
\pgfpathlineto{\pgfqpoint{2.497125in}{2.381537in}}%
\pgfpathclose%
\pgfusepath{fill}%
\end{pgfscope}%
\begin{pgfscope}%
\pgfpathrectangle{\pgfqpoint{0.526905in}{0.383578in}}{\pgfqpoint{3.875000in}{2.310000in}}%
\pgfusepath{clip}%
\pgfsetbuttcap%
\pgfsetmiterjoin%
\definecolor{currentfill}{rgb}{0.686275,0.352941,0.313725}%
\pgfsetfillcolor{currentfill}%
\pgfsetfillopacity{0.300000}%
\pgfsetlinewidth{0.000000pt}%
\definecolor{currentstroke}{rgb}{0.000000,0.000000,0.000000}%
\pgfsetstrokecolor{currentstroke}%
\pgfsetstrokeopacity{0.300000}%
\pgfsetdash{}{0pt}%
\pgfpathmoveto{\pgfqpoint{2.514714in}{0.383578in}}%
\pgfpathlineto{\pgfqpoint{2.532303in}{0.383578in}}%
\pgfpathlineto{\pgfqpoint{2.532303in}{2.246843in}}%
\pgfpathlineto{\pgfqpoint{2.514714in}{2.246843in}}%
\pgfpathclose%
\pgfusepath{fill}%
\end{pgfscope}%
\begin{pgfscope}%
\pgfpathrectangle{\pgfqpoint{0.526905in}{0.383578in}}{\pgfqpoint{3.875000in}{2.310000in}}%
\pgfusepath{clip}%
\pgfsetbuttcap%
\pgfsetmiterjoin%
\definecolor{currentfill}{rgb}{0.686275,0.352941,0.313725}%
\pgfsetfillcolor{currentfill}%
\pgfsetfillopacity{0.300000}%
\pgfsetlinewidth{0.000000pt}%
\definecolor{currentstroke}{rgb}{0.000000,0.000000,0.000000}%
\pgfsetstrokecolor{currentstroke}%
\pgfsetstrokeopacity{0.300000}%
\pgfsetdash{}{0pt}%
\pgfpathmoveto{\pgfqpoint{2.532303in}{0.383578in}}%
\pgfpathlineto{\pgfqpoint{2.549892in}{0.383578in}}%
\pgfpathlineto{\pgfqpoint{2.549892in}{2.319803in}}%
\pgfpathlineto{\pgfqpoint{2.532303in}{2.319803in}}%
\pgfpathclose%
\pgfusepath{fill}%
\end{pgfscope}%
\begin{pgfscope}%
\pgfpathrectangle{\pgfqpoint{0.526905in}{0.383578in}}{\pgfqpoint{3.875000in}{2.310000in}}%
\pgfusepath{clip}%
\pgfsetbuttcap%
\pgfsetmiterjoin%
\definecolor{currentfill}{rgb}{0.686275,0.352941,0.313725}%
\pgfsetfillcolor{currentfill}%
\pgfsetfillopacity{0.300000}%
\pgfsetlinewidth{0.000000pt}%
\definecolor{currentstroke}{rgb}{0.000000,0.000000,0.000000}%
\pgfsetstrokecolor{currentstroke}%
\pgfsetstrokeopacity{0.300000}%
\pgfsetdash{}{0pt}%
\pgfpathmoveto{\pgfqpoint{2.549892in}{0.383578in}}%
\pgfpathlineto{\pgfqpoint{2.567481in}{0.383578in}}%
\pgfpathlineto{\pgfqpoint{2.567481in}{2.230007in}}%
\pgfpathlineto{\pgfqpoint{2.549892in}{2.230007in}}%
\pgfpathclose%
\pgfusepath{fill}%
\end{pgfscope}%
\begin{pgfscope}%
\pgfpathrectangle{\pgfqpoint{0.526905in}{0.383578in}}{\pgfqpoint{3.875000in}{2.310000in}}%
\pgfusepath{clip}%
\pgfsetbuttcap%
\pgfsetmiterjoin%
\definecolor{currentfill}{rgb}{0.686275,0.352941,0.313725}%
\pgfsetfillcolor{currentfill}%
\pgfsetfillopacity{0.300000}%
\pgfsetlinewidth{0.000000pt}%
\definecolor{currentstroke}{rgb}{0.000000,0.000000,0.000000}%
\pgfsetstrokecolor{currentstroke}%
\pgfsetstrokeopacity{0.300000}%
\pgfsetdash{}{0pt}%
\pgfpathmoveto{\pgfqpoint{2.567481in}{0.383578in}}%
\pgfpathlineto{\pgfqpoint{2.585070in}{0.383578in}}%
\pgfpathlineto{\pgfqpoint{2.585070in}{2.302966in}}%
\pgfpathlineto{\pgfqpoint{2.567481in}{2.302966in}}%
\pgfpathclose%
\pgfusepath{fill}%
\end{pgfscope}%
\begin{pgfscope}%
\pgfpathrectangle{\pgfqpoint{0.526905in}{0.383578in}}{\pgfqpoint{3.875000in}{2.310000in}}%
\pgfusepath{clip}%
\pgfsetbuttcap%
\pgfsetmiterjoin%
\definecolor{currentfill}{rgb}{0.686275,0.352941,0.313725}%
\pgfsetfillcolor{currentfill}%
\pgfsetfillopacity{0.300000}%
\pgfsetlinewidth{0.000000pt}%
\definecolor{currentstroke}{rgb}{0.000000,0.000000,0.000000}%
\pgfsetstrokecolor{currentstroke}%
\pgfsetstrokeopacity{0.300000}%
\pgfsetdash{}{0pt}%
\pgfpathmoveto{\pgfqpoint{2.585070in}{0.383578in}}%
\pgfpathlineto{\pgfqpoint{2.602659in}{0.383578in}}%
\pgfpathlineto{\pgfqpoint{2.602659in}{2.516231in}}%
\pgfpathlineto{\pgfqpoint{2.585070in}{2.516231in}}%
\pgfpathclose%
\pgfusepath{fill}%
\end{pgfscope}%
\begin{pgfscope}%
\pgfpathrectangle{\pgfqpoint{0.526905in}{0.383578in}}{\pgfqpoint{3.875000in}{2.310000in}}%
\pgfusepath{clip}%
\pgfsetbuttcap%
\pgfsetmiterjoin%
\definecolor{currentfill}{rgb}{0.686275,0.352941,0.313725}%
\pgfsetfillcolor{currentfill}%
\pgfsetfillopacity{0.300000}%
\pgfsetlinewidth{0.000000pt}%
\definecolor{currentstroke}{rgb}{0.000000,0.000000,0.000000}%
\pgfsetstrokecolor{currentstroke}%
\pgfsetstrokeopacity{0.300000}%
\pgfsetdash{}{0pt}%
\pgfpathmoveto{\pgfqpoint{2.602659in}{0.383578in}}%
\pgfpathlineto{\pgfqpoint{2.620248in}{0.383578in}}%
\pgfpathlineto{\pgfqpoint{2.620248in}{2.375925in}}%
\pgfpathlineto{\pgfqpoint{2.602659in}{2.375925in}}%
\pgfpathclose%
\pgfusepath{fill}%
\end{pgfscope}%
\begin{pgfscope}%
\pgfpathrectangle{\pgfqpoint{0.526905in}{0.383578in}}{\pgfqpoint{3.875000in}{2.310000in}}%
\pgfusepath{clip}%
\pgfsetbuttcap%
\pgfsetmiterjoin%
\definecolor{currentfill}{rgb}{0.686275,0.352941,0.313725}%
\pgfsetfillcolor{currentfill}%
\pgfsetfillopacity{0.300000}%
\pgfsetlinewidth{0.000000pt}%
\definecolor{currentstroke}{rgb}{0.000000,0.000000,0.000000}%
\pgfsetstrokecolor{currentstroke}%
\pgfsetstrokeopacity{0.300000}%
\pgfsetdash{}{0pt}%
\pgfpathmoveto{\pgfqpoint{2.620248in}{0.383578in}}%
\pgfpathlineto{\pgfqpoint{2.637837in}{0.383578in}}%
\pgfpathlineto{\pgfqpoint{2.637837in}{2.353476in}}%
\pgfpathlineto{\pgfqpoint{2.620248in}{2.353476in}}%
\pgfpathclose%
\pgfusepath{fill}%
\end{pgfscope}%
\begin{pgfscope}%
\pgfpathrectangle{\pgfqpoint{0.526905in}{0.383578in}}{\pgfqpoint{3.875000in}{2.310000in}}%
\pgfusepath{clip}%
\pgfsetbuttcap%
\pgfsetmiterjoin%
\definecolor{currentfill}{rgb}{0.686275,0.352941,0.313725}%
\pgfsetfillcolor{currentfill}%
\pgfsetfillopacity{0.300000}%
\pgfsetlinewidth{0.000000pt}%
\definecolor{currentstroke}{rgb}{0.000000,0.000000,0.000000}%
\pgfsetstrokecolor{currentstroke}%
\pgfsetstrokeopacity{0.300000}%
\pgfsetdash{}{0pt}%
\pgfpathmoveto{\pgfqpoint{2.637837in}{0.383578in}}%
\pgfpathlineto{\pgfqpoint{2.655426in}{0.383578in}}%
\pgfpathlineto{\pgfqpoint{2.655426in}{2.572354in}}%
\pgfpathlineto{\pgfqpoint{2.637837in}{2.572354in}}%
\pgfpathclose%
\pgfusepath{fill}%
\end{pgfscope}%
\begin{pgfscope}%
\pgfpathrectangle{\pgfqpoint{0.526905in}{0.383578in}}{\pgfqpoint{3.875000in}{2.310000in}}%
\pgfusepath{clip}%
\pgfsetbuttcap%
\pgfsetmiterjoin%
\definecolor{currentfill}{rgb}{0.686275,0.352941,0.313725}%
\pgfsetfillcolor{currentfill}%
\pgfsetfillopacity{0.300000}%
\pgfsetlinewidth{0.000000pt}%
\definecolor{currentstroke}{rgb}{0.000000,0.000000,0.000000}%
\pgfsetstrokecolor{currentstroke}%
\pgfsetstrokeopacity{0.300000}%
\pgfsetdash{}{0pt}%
\pgfpathmoveto{\pgfqpoint{2.655426in}{0.383578in}}%
\pgfpathlineto{\pgfqpoint{2.673015in}{0.383578in}}%
\pgfpathlineto{\pgfqpoint{2.673015in}{2.432047in}}%
\pgfpathlineto{\pgfqpoint{2.655426in}{2.432047in}}%
\pgfpathclose%
\pgfusepath{fill}%
\end{pgfscope}%
\begin{pgfscope}%
\pgfpathrectangle{\pgfqpoint{0.526905in}{0.383578in}}{\pgfqpoint{3.875000in}{2.310000in}}%
\pgfusepath{clip}%
\pgfsetbuttcap%
\pgfsetmiterjoin%
\definecolor{currentfill}{rgb}{0.686275,0.352941,0.313725}%
\pgfsetfillcolor{currentfill}%
\pgfsetfillopacity{0.300000}%
\pgfsetlinewidth{0.000000pt}%
\definecolor{currentstroke}{rgb}{0.000000,0.000000,0.000000}%
\pgfsetstrokecolor{currentstroke}%
\pgfsetstrokeopacity{0.300000}%
\pgfsetdash{}{0pt}%
\pgfpathmoveto{\pgfqpoint{2.673015in}{0.383578in}}%
\pgfpathlineto{\pgfqpoint{2.690604in}{0.383578in}}%
\pgfpathlineto{\pgfqpoint{2.690604in}{2.482558in}}%
\pgfpathlineto{\pgfqpoint{2.673015in}{2.482558in}}%
\pgfpathclose%
\pgfusepath{fill}%
\end{pgfscope}%
\begin{pgfscope}%
\pgfpathrectangle{\pgfqpoint{0.526905in}{0.383578in}}{\pgfqpoint{3.875000in}{2.310000in}}%
\pgfusepath{clip}%
\pgfsetbuttcap%
\pgfsetmiterjoin%
\definecolor{currentfill}{rgb}{0.686275,0.352941,0.313725}%
\pgfsetfillcolor{currentfill}%
\pgfsetfillopacity{0.300000}%
\pgfsetlinewidth{0.000000pt}%
\definecolor{currentstroke}{rgb}{0.000000,0.000000,0.000000}%
\pgfsetstrokecolor{currentstroke}%
\pgfsetstrokeopacity{0.300000}%
\pgfsetdash{}{0pt}%
\pgfpathmoveto{\pgfqpoint{2.690604in}{0.383578in}}%
\pgfpathlineto{\pgfqpoint{2.708193in}{0.383578in}}%
\pgfpathlineto{\pgfqpoint{2.708193in}{2.488170in}}%
\pgfpathlineto{\pgfqpoint{2.690604in}{2.488170in}}%
\pgfpathclose%
\pgfusepath{fill}%
\end{pgfscope}%
\begin{pgfscope}%
\pgfpathrectangle{\pgfqpoint{0.526905in}{0.383578in}}{\pgfqpoint{3.875000in}{2.310000in}}%
\pgfusepath{clip}%
\pgfsetbuttcap%
\pgfsetmiterjoin%
\definecolor{currentfill}{rgb}{0.686275,0.352941,0.313725}%
\pgfsetfillcolor{currentfill}%
\pgfsetfillopacity{0.300000}%
\pgfsetlinewidth{0.000000pt}%
\definecolor{currentstroke}{rgb}{0.000000,0.000000,0.000000}%
\pgfsetstrokecolor{currentstroke}%
\pgfsetstrokeopacity{0.300000}%
\pgfsetdash{}{0pt}%
\pgfpathmoveto{\pgfqpoint{2.708193in}{0.383578in}}%
\pgfpathlineto{\pgfqpoint{2.725782in}{0.383578in}}%
\pgfpathlineto{\pgfqpoint{2.725782in}{2.583578in}}%
\pgfpathlineto{\pgfqpoint{2.708193in}{2.583578in}}%
\pgfpathclose%
\pgfusepath{fill}%
\end{pgfscope}%
\begin{pgfscope}%
\pgfpathrectangle{\pgfqpoint{0.526905in}{0.383578in}}{\pgfqpoint{3.875000in}{2.310000in}}%
\pgfusepath{clip}%
\pgfsetbuttcap%
\pgfsetmiterjoin%
\definecolor{currentfill}{rgb}{0.686275,0.352941,0.313725}%
\pgfsetfillcolor{currentfill}%
\pgfsetfillopacity{0.300000}%
\pgfsetlinewidth{0.000000pt}%
\definecolor{currentstroke}{rgb}{0.000000,0.000000,0.000000}%
\pgfsetstrokecolor{currentstroke}%
\pgfsetstrokeopacity{0.300000}%
\pgfsetdash{}{0pt}%
\pgfpathmoveto{\pgfqpoint{2.725782in}{0.383578in}}%
\pgfpathlineto{\pgfqpoint{2.743371in}{0.383578in}}%
\pgfpathlineto{\pgfqpoint{2.743371in}{2.544292in}}%
\pgfpathlineto{\pgfqpoint{2.725782in}{2.544292in}}%
\pgfpathclose%
\pgfusepath{fill}%
\end{pgfscope}%
\begin{pgfscope}%
\pgfpathrectangle{\pgfqpoint{0.526905in}{0.383578in}}{\pgfqpoint{3.875000in}{2.310000in}}%
\pgfusepath{clip}%
\pgfsetbuttcap%
\pgfsetmiterjoin%
\definecolor{currentfill}{rgb}{0.686275,0.352941,0.313725}%
\pgfsetfillcolor{currentfill}%
\pgfsetfillopacity{0.300000}%
\pgfsetlinewidth{0.000000pt}%
\definecolor{currentstroke}{rgb}{0.000000,0.000000,0.000000}%
\pgfsetstrokecolor{currentstroke}%
\pgfsetstrokeopacity{0.300000}%
\pgfsetdash{}{0pt}%
\pgfpathmoveto{\pgfqpoint{2.743371in}{0.383578in}}%
\pgfpathlineto{\pgfqpoint{2.760960in}{0.383578in}}%
\pgfpathlineto{\pgfqpoint{2.760960in}{2.432047in}}%
\pgfpathlineto{\pgfqpoint{2.743371in}{2.432047in}}%
\pgfpathclose%
\pgfusepath{fill}%
\end{pgfscope}%
\begin{pgfscope}%
\pgfpathrectangle{\pgfqpoint{0.526905in}{0.383578in}}{\pgfqpoint{3.875000in}{2.310000in}}%
\pgfusepath{clip}%
\pgfsetbuttcap%
\pgfsetmiterjoin%
\definecolor{currentfill}{rgb}{0.686275,0.352941,0.313725}%
\pgfsetfillcolor{currentfill}%
\pgfsetfillopacity{0.300000}%
\pgfsetlinewidth{0.000000pt}%
\definecolor{currentstroke}{rgb}{0.000000,0.000000,0.000000}%
\pgfsetstrokecolor{currentstroke}%
\pgfsetstrokeopacity{0.300000}%
\pgfsetdash{}{0pt}%
\pgfpathmoveto{\pgfqpoint{2.760960in}{0.383578in}}%
\pgfpathlineto{\pgfqpoint{2.778549in}{0.383578in}}%
\pgfpathlineto{\pgfqpoint{2.778549in}{2.426435in}}%
\pgfpathlineto{\pgfqpoint{2.760960in}{2.426435in}}%
\pgfpathclose%
\pgfusepath{fill}%
\end{pgfscope}%
\begin{pgfscope}%
\pgfpathrectangle{\pgfqpoint{0.526905in}{0.383578in}}{\pgfqpoint{3.875000in}{2.310000in}}%
\pgfusepath{clip}%
\pgfsetbuttcap%
\pgfsetmiterjoin%
\definecolor{currentfill}{rgb}{0.686275,0.352941,0.313725}%
\pgfsetfillcolor{currentfill}%
\pgfsetfillopacity{0.300000}%
\pgfsetlinewidth{0.000000pt}%
\definecolor{currentstroke}{rgb}{0.000000,0.000000,0.000000}%
\pgfsetstrokecolor{currentstroke}%
\pgfsetstrokeopacity{0.300000}%
\pgfsetdash{}{0pt}%
\pgfpathmoveto{\pgfqpoint{2.778549in}{0.383578in}}%
\pgfpathlineto{\pgfqpoint{2.796138in}{0.383578in}}%
\pgfpathlineto{\pgfqpoint{2.796138in}{2.375925in}}%
\pgfpathlineto{\pgfqpoint{2.778549in}{2.375925in}}%
\pgfpathclose%
\pgfusepath{fill}%
\end{pgfscope}%
\begin{pgfscope}%
\pgfpathrectangle{\pgfqpoint{0.526905in}{0.383578in}}{\pgfqpoint{3.875000in}{2.310000in}}%
\pgfusepath{clip}%
\pgfsetbuttcap%
\pgfsetmiterjoin%
\definecolor{currentfill}{rgb}{0.686275,0.352941,0.313725}%
\pgfsetfillcolor{currentfill}%
\pgfsetfillopacity{0.300000}%
\pgfsetlinewidth{0.000000pt}%
\definecolor{currentstroke}{rgb}{0.000000,0.000000,0.000000}%
\pgfsetstrokecolor{currentstroke}%
\pgfsetstrokeopacity{0.300000}%
\pgfsetdash{}{0pt}%
\pgfpathmoveto{\pgfqpoint{2.796138in}{0.383578in}}%
\pgfpathlineto{\pgfqpoint{2.813728in}{0.383578in}}%
\pgfpathlineto{\pgfqpoint{2.813728in}{2.465721in}}%
\pgfpathlineto{\pgfqpoint{2.796138in}{2.465721in}}%
\pgfpathclose%
\pgfusepath{fill}%
\end{pgfscope}%
\begin{pgfscope}%
\pgfpathrectangle{\pgfqpoint{0.526905in}{0.383578in}}{\pgfqpoint{3.875000in}{2.310000in}}%
\pgfusepath{clip}%
\pgfsetbuttcap%
\pgfsetmiterjoin%
\definecolor{currentfill}{rgb}{0.686275,0.352941,0.313725}%
\pgfsetfillcolor{currentfill}%
\pgfsetfillopacity{0.300000}%
\pgfsetlinewidth{0.000000pt}%
\definecolor{currentstroke}{rgb}{0.000000,0.000000,0.000000}%
\pgfsetstrokecolor{currentstroke}%
\pgfsetstrokeopacity{0.300000}%
\pgfsetdash{}{0pt}%
\pgfpathmoveto{\pgfqpoint{2.813728in}{0.383578in}}%
\pgfpathlineto{\pgfqpoint{2.831317in}{0.383578in}}%
\pgfpathlineto{\pgfqpoint{2.831317in}{2.415211in}}%
\pgfpathlineto{\pgfqpoint{2.813728in}{2.415211in}}%
\pgfpathclose%
\pgfusepath{fill}%
\end{pgfscope}%
\begin{pgfscope}%
\pgfpathrectangle{\pgfqpoint{0.526905in}{0.383578in}}{\pgfqpoint{3.875000in}{2.310000in}}%
\pgfusepath{clip}%
\pgfsetbuttcap%
\pgfsetmiterjoin%
\definecolor{currentfill}{rgb}{0.686275,0.352941,0.313725}%
\pgfsetfillcolor{currentfill}%
\pgfsetfillopacity{0.300000}%
\pgfsetlinewidth{0.000000pt}%
\definecolor{currentstroke}{rgb}{0.000000,0.000000,0.000000}%
\pgfsetstrokecolor{currentstroke}%
\pgfsetstrokeopacity{0.300000}%
\pgfsetdash{}{0pt}%
\pgfpathmoveto{\pgfqpoint{2.831317in}{0.383578in}}%
\pgfpathlineto{\pgfqpoint{2.848906in}{0.383578in}}%
\pgfpathlineto{\pgfqpoint{2.848906in}{2.555517in}}%
\pgfpathlineto{\pgfqpoint{2.831317in}{2.555517in}}%
\pgfpathclose%
\pgfusepath{fill}%
\end{pgfscope}%
\begin{pgfscope}%
\pgfpathrectangle{\pgfqpoint{0.526905in}{0.383578in}}{\pgfqpoint{3.875000in}{2.310000in}}%
\pgfusepath{clip}%
\pgfsetbuttcap%
\pgfsetmiterjoin%
\definecolor{currentfill}{rgb}{0.686275,0.352941,0.313725}%
\pgfsetfillcolor{currentfill}%
\pgfsetfillopacity{0.300000}%
\pgfsetlinewidth{0.000000pt}%
\definecolor{currentstroke}{rgb}{0.000000,0.000000,0.000000}%
\pgfsetstrokecolor{currentstroke}%
\pgfsetstrokeopacity{0.300000}%
\pgfsetdash{}{0pt}%
\pgfpathmoveto{\pgfqpoint{2.848906in}{0.383578in}}%
\pgfpathlineto{\pgfqpoint{2.866495in}{0.383578in}}%
\pgfpathlineto{\pgfqpoint{2.866495in}{2.555517in}}%
\pgfpathlineto{\pgfqpoint{2.848906in}{2.555517in}}%
\pgfpathclose%
\pgfusepath{fill}%
\end{pgfscope}%
\begin{pgfscope}%
\pgfpathrectangle{\pgfqpoint{0.526905in}{0.383578in}}{\pgfqpoint{3.875000in}{2.310000in}}%
\pgfusepath{clip}%
\pgfsetbuttcap%
\pgfsetmiterjoin%
\definecolor{currentfill}{rgb}{0.686275,0.352941,0.313725}%
\pgfsetfillcolor{currentfill}%
\pgfsetfillopacity{0.300000}%
\pgfsetlinewidth{0.000000pt}%
\definecolor{currentstroke}{rgb}{0.000000,0.000000,0.000000}%
\pgfsetstrokecolor{currentstroke}%
\pgfsetstrokeopacity{0.300000}%
\pgfsetdash{}{0pt}%
\pgfpathmoveto{\pgfqpoint{2.866495in}{0.383578in}}%
\pgfpathlineto{\pgfqpoint{2.884084in}{0.383578in}}%
\pgfpathlineto{\pgfqpoint{2.884084in}{2.420823in}}%
\pgfpathlineto{\pgfqpoint{2.866495in}{2.420823in}}%
\pgfpathclose%
\pgfusepath{fill}%
\end{pgfscope}%
\begin{pgfscope}%
\pgfpathrectangle{\pgfqpoint{0.526905in}{0.383578in}}{\pgfqpoint{3.875000in}{2.310000in}}%
\pgfusepath{clip}%
\pgfsetbuttcap%
\pgfsetmiterjoin%
\definecolor{currentfill}{rgb}{0.686275,0.352941,0.313725}%
\pgfsetfillcolor{currentfill}%
\pgfsetfillopacity{0.300000}%
\pgfsetlinewidth{0.000000pt}%
\definecolor{currentstroke}{rgb}{0.000000,0.000000,0.000000}%
\pgfsetstrokecolor{currentstroke}%
\pgfsetstrokeopacity{0.300000}%
\pgfsetdash{}{0pt}%
\pgfpathmoveto{\pgfqpoint{2.884084in}{0.383578in}}%
\pgfpathlineto{\pgfqpoint{2.901673in}{0.383578in}}%
\pgfpathlineto{\pgfqpoint{2.901673in}{2.398374in}}%
\pgfpathlineto{\pgfqpoint{2.884084in}{2.398374in}}%
\pgfpathclose%
\pgfusepath{fill}%
\end{pgfscope}%
\begin{pgfscope}%
\pgfpathrectangle{\pgfqpoint{0.526905in}{0.383578in}}{\pgfqpoint{3.875000in}{2.310000in}}%
\pgfusepath{clip}%
\pgfsetbuttcap%
\pgfsetmiterjoin%
\definecolor{currentfill}{rgb}{0.686275,0.352941,0.313725}%
\pgfsetfillcolor{currentfill}%
\pgfsetfillopacity{0.300000}%
\pgfsetlinewidth{0.000000pt}%
\definecolor{currentstroke}{rgb}{0.000000,0.000000,0.000000}%
\pgfsetstrokecolor{currentstroke}%
\pgfsetstrokeopacity{0.300000}%
\pgfsetdash{}{0pt}%
\pgfpathmoveto{\pgfqpoint{2.901673in}{0.383578in}}%
\pgfpathlineto{\pgfqpoint{2.919262in}{0.383578in}}%
\pgfpathlineto{\pgfqpoint{2.919262in}{2.510619in}}%
\pgfpathlineto{\pgfqpoint{2.901673in}{2.510619in}}%
\pgfpathclose%
\pgfusepath{fill}%
\end{pgfscope}%
\begin{pgfscope}%
\pgfpathrectangle{\pgfqpoint{0.526905in}{0.383578in}}{\pgfqpoint{3.875000in}{2.310000in}}%
\pgfusepath{clip}%
\pgfsetbuttcap%
\pgfsetmiterjoin%
\definecolor{currentfill}{rgb}{0.686275,0.352941,0.313725}%
\pgfsetfillcolor{currentfill}%
\pgfsetfillopacity{0.300000}%
\pgfsetlinewidth{0.000000pt}%
\definecolor{currentstroke}{rgb}{0.000000,0.000000,0.000000}%
\pgfsetstrokecolor{currentstroke}%
\pgfsetstrokeopacity{0.300000}%
\pgfsetdash{}{0pt}%
\pgfpathmoveto{\pgfqpoint{2.919262in}{0.383578in}}%
\pgfpathlineto{\pgfqpoint{2.936851in}{0.383578in}}%
\pgfpathlineto{\pgfqpoint{2.936851in}{2.347864in}}%
\pgfpathlineto{\pgfqpoint{2.919262in}{2.347864in}}%
\pgfpathclose%
\pgfusepath{fill}%
\end{pgfscope}%
\begin{pgfscope}%
\pgfpathrectangle{\pgfqpoint{0.526905in}{0.383578in}}{\pgfqpoint{3.875000in}{2.310000in}}%
\pgfusepath{clip}%
\pgfsetbuttcap%
\pgfsetmiterjoin%
\definecolor{currentfill}{rgb}{0.686275,0.352941,0.313725}%
\pgfsetfillcolor{currentfill}%
\pgfsetfillopacity{0.300000}%
\pgfsetlinewidth{0.000000pt}%
\definecolor{currentstroke}{rgb}{0.000000,0.000000,0.000000}%
\pgfsetstrokecolor{currentstroke}%
\pgfsetstrokeopacity{0.300000}%
\pgfsetdash{}{0pt}%
\pgfpathmoveto{\pgfqpoint{2.936851in}{0.383578in}}%
\pgfpathlineto{\pgfqpoint{2.954440in}{0.383578in}}%
\pgfpathlineto{\pgfqpoint{2.954440in}{2.398374in}}%
\pgfpathlineto{\pgfqpoint{2.936851in}{2.398374in}}%
\pgfpathclose%
\pgfusepath{fill}%
\end{pgfscope}%
\begin{pgfscope}%
\pgfpathrectangle{\pgfqpoint{0.526905in}{0.383578in}}{\pgfqpoint{3.875000in}{2.310000in}}%
\pgfusepath{clip}%
\pgfsetbuttcap%
\pgfsetmiterjoin%
\definecolor{currentfill}{rgb}{0.686275,0.352941,0.313725}%
\pgfsetfillcolor{currentfill}%
\pgfsetfillopacity{0.300000}%
\pgfsetlinewidth{0.000000pt}%
\definecolor{currentstroke}{rgb}{0.000000,0.000000,0.000000}%
\pgfsetstrokecolor{currentstroke}%
\pgfsetstrokeopacity{0.300000}%
\pgfsetdash{}{0pt}%
\pgfpathmoveto{\pgfqpoint{2.954440in}{0.383578in}}%
\pgfpathlineto{\pgfqpoint{2.972029in}{0.383578in}}%
\pgfpathlineto{\pgfqpoint{2.972029in}{2.465721in}}%
\pgfpathlineto{\pgfqpoint{2.954440in}{2.465721in}}%
\pgfpathclose%
\pgfusepath{fill}%
\end{pgfscope}%
\begin{pgfscope}%
\pgfpathrectangle{\pgfqpoint{0.526905in}{0.383578in}}{\pgfqpoint{3.875000in}{2.310000in}}%
\pgfusepath{clip}%
\pgfsetbuttcap%
\pgfsetmiterjoin%
\definecolor{currentfill}{rgb}{0.686275,0.352941,0.313725}%
\pgfsetfillcolor{currentfill}%
\pgfsetfillopacity{0.300000}%
\pgfsetlinewidth{0.000000pt}%
\definecolor{currentstroke}{rgb}{0.000000,0.000000,0.000000}%
\pgfsetstrokecolor{currentstroke}%
\pgfsetstrokeopacity{0.300000}%
\pgfsetdash{}{0pt}%
\pgfpathmoveto{\pgfqpoint{2.972029in}{0.383578in}}%
\pgfpathlineto{\pgfqpoint{2.989618in}{0.383578in}}%
\pgfpathlineto{\pgfqpoint{2.989618in}{2.218782in}}%
\pgfpathlineto{\pgfqpoint{2.972029in}{2.218782in}}%
\pgfpathclose%
\pgfusepath{fill}%
\end{pgfscope}%
\begin{pgfscope}%
\pgfpathrectangle{\pgfqpoint{0.526905in}{0.383578in}}{\pgfqpoint{3.875000in}{2.310000in}}%
\pgfusepath{clip}%
\pgfsetbuttcap%
\pgfsetmiterjoin%
\definecolor{currentfill}{rgb}{0.686275,0.352941,0.313725}%
\pgfsetfillcolor{currentfill}%
\pgfsetfillopacity{0.300000}%
\pgfsetlinewidth{0.000000pt}%
\definecolor{currentstroke}{rgb}{0.000000,0.000000,0.000000}%
\pgfsetstrokecolor{currentstroke}%
\pgfsetstrokeopacity{0.300000}%
\pgfsetdash{}{0pt}%
\pgfpathmoveto{\pgfqpoint{2.989618in}{0.383578in}}%
\pgfpathlineto{\pgfqpoint{3.007207in}{0.383578in}}%
\pgfpathlineto{\pgfqpoint{3.007207in}{2.201945in}}%
\pgfpathlineto{\pgfqpoint{2.989618in}{2.201945in}}%
\pgfpathclose%
\pgfusepath{fill}%
\end{pgfscope}%
\begin{pgfscope}%
\pgfpathrectangle{\pgfqpoint{0.526905in}{0.383578in}}{\pgfqpoint{3.875000in}{2.310000in}}%
\pgfusepath{clip}%
\pgfsetbuttcap%
\pgfsetmiterjoin%
\definecolor{currentfill}{rgb}{0.686275,0.352941,0.313725}%
\pgfsetfillcolor{currentfill}%
\pgfsetfillopacity{0.300000}%
\pgfsetlinewidth{0.000000pt}%
\definecolor{currentstroke}{rgb}{0.000000,0.000000,0.000000}%
\pgfsetstrokecolor{currentstroke}%
\pgfsetstrokeopacity{0.300000}%
\pgfsetdash{}{0pt}%
\pgfpathmoveto{\pgfqpoint{3.007207in}{0.383578in}}%
\pgfpathlineto{\pgfqpoint{3.024796in}{0.383578in}}%
\pgfpathlineto{\pgfqpoint{3.024796in}{2.426435in}}%
\pgfpathlineto{\pgfqpoint{3.007207in}{2.426435in}}%
\pgfpathclose%
\pgfusepath{fill}%
\end{pgfscope}%
\begin{pgfscope}%
\pgfpathrectangle{\pgfqpoint{0.526905in}{0.383578in}}{\pgfqpoint{3.875000in}{2.310000in}}%
\pgfusepath{clip}%
\pgfsetbuttcap%
\pgfsetmiterjoin%
\definecolor{currentfill}{rgb}{0.686275,0.352941,0.313725}%
\pgfsetfillcolor{currentfill}%
\pgfsetfillopacity{0.300000}%
\pgfsetlinewidth{0.000000pt}%
\definecolor{currentstroke}{rgb}{0.000000,0.000000,0.000000}%
\pgfsetstrokecolor{currentstroke}%
\pgfsetstrokeopacity{0.300000}%
\pgfsetdash{}{0pt}%
\pgfpathmoveto{\pgfqpoint{3.024796in}{0.383578in}}%
\pgfpathlineto{\pgfqpoint{3.042385in}{0.383578in}}%
\pgfpathlineto{\pgfqpoint{3.042385in}{2.235619in}}%
\pgfpathlineto{\pgfqpoint{3.024796in}{2.235619in}}%
\pgfpathclose%
\pgfusepath{fill}%
\end{pgfscope}%
\begin{pgfscope}%
\pgfpathrectangle{\pgfqpoint{0.526905in}{0.383578in}}{\pgfqpoint{3.875000in}{2.310000in}}%
\pgfusepath{clip}%
\pgfsetbuttcap%
\pgfsetmiterjoin%
\definecolor{currentfill}{rgb}{0.686275,0.352941,0.313725}%
\pgfsetfillcolor{currentfill}%
\pgfsetfillopacity{0.300000}%
\pgfsetlinewidth{0.000000pt}%
\definecolor{currentstroke}{rgb}{0.000000,0.000000,0.000000}%
\pgfsetstrokecolor{currentstroke}%
\pgfsetstrokeopacity{0.300000}%
\pgfsetdash{}{0pt}%
\pgfpathmoveto{\pgfqpoint{3.042385in}{0.383578in}}%
\pgfpathlineto{\pgfqpoint{3.059974in}{0.383578in}}%
\pgfpathlineto{\pgfqpoint{3.059974in}{1.983068in}}%
\pgfpathlineto{\pgfqpoint{3.042385in}{1.983068in}}%
\pgfpathclose%
\pgfusepath{fill}%
\end{pgfscope}%
\begin{pgfscope}%
\pgfpathrectangle{\pgfqpoint{0.526905in}{0.383578in}}{\pgfqpoint{3.875000in}{2.310000in}}%
\pgfusepath{clip}%
\pgfsetbuttcap%
\pgfsetmiterjoin%
\definecolor{currentfill}{rgb}{0.686275,0.352941,0.313725}%
\pgfsetfillcolor{currentfill}%
\pgfsetfillopacity{0.300000}%
\pgfsetlinewidth{0.000000pt}%
\definecolor{currentstroke}{rgb}{0.000000,0.000000,0.000000}%
\pgfsetstrokecolor{currentstroke}%
\pgfsetstrokeopacity{0.300000}%
\pgfsetdash{}{0pt}%
\pgfpathmoveto{\pgfqpoint{3.059974in}{0.383578in}}%
\pgfpathlineto{\pgfqpoint{3.077563in}{0.383578in}}%
\pgfpathlineto{\pgfqpoint{3.077563in}{1.994292in}}%
\pgfpathlineto{\pgfqpoint{3.059974in}{1.994292in}}%
\pgfpathclose%
\pgfusepath{fill}%
\end{pgfscope}%
\begin{pgfscope}%
\pgfpathrectangle{\pgfqpoint{0.526905in}{0.383578in}}{\pgfqpoint{3.875000in}{2.310000in}}%
\pgfusepath{clip}%
\pgfsetbuttcap%
\pgfsetmiterjoin%
\definecolor{currentfill}{rgb}{0.686275,0.352941,0.313725}%
\pgfsetfillcolor{currentfill}%
\pgfsetfillopacity{0.300000}%
\pgfsetlinewidth{0.000000pt}%
\definecolor{currentstroke}{rgb}{0.000000,0.000000,0.000000}%
\pgfsetstrokecolor{currentstroke}%
\pgfsetstrokeopacity{0.300000}%
\pgfsetdash{}{0pt}%
\pgfpathmoveto{\pgfqpoint{3.077563in}{0.383578in}}%
\pgfpathlineto{\pgfqpoint{3.095152in}{0.383578in}}%
\pgfpathlineto{\pgfqpoint{3.095152in}{1.960619in}}%
\pgfpathlineto{\pgfqpoint{3.077563in}{1.960619in}}%
\pgfpathclose%
\pgfusepath{fill}%
\end{pgfscope}%
\begin{pgfscope}%
\pgfpathrectangle{\pgfqpoint{0.526905in}{0.383578in}}{\pgfqpoint{3.875000in}{2.310000in}}%
\pgfusepath{clip}%
\pgfsetbuttcap%
\pgfsetmiterjoin%
\definecolor{currentfill}{rgb}{0.686275,0.352941,0.313725}%
\pgfsetfillcolor{currentfill}%
\pgfsetfillopacity{0.300000}%
\pgfsetlinewidth{0.000000pt}%
\definecolor{currentstroke}{rgb}{0.000000,0.000000,0.000000}%
\pgfsetstrokecolor{currentstroke}%
\pgfsetstrokeopacity{0.300000}%
\pgfsetdash{}{0pt}%
\pgfpathmoveto{\pgfqpoint{3.095152in}{0.383578in}}%
\pgfpathlineto{\pgfqpoint{3.112741in}{0.383578in}}%
\pgfpathlineto{\pgfqpoint{3.112741in}{1.932558in}}%
\pgfpathlineto{\pgfqpoint{3.095152in}{1.932558in}}%
\pgfpathclose%
\pgfusepath{fill}%
\end{pgfscope}%
\begin{pgfscope}%
\pgfpathrectangle{\pgfqpoint{0.526905in}{0.383578in}}{\pgfqpoint{3.875000in}{2.310000in}}%
\pgfusepath{clip}%
\pgfsetbuttcap%
\pgfsetmiterjoin%
\definecolor{currentfill}{rgb}{0.686275,0.352941,0.313725}%
\pgfsetfillcolor{currentfill}%
\pgfsetfillopacity{0.300000}%
\pgfsetlinewidth{0.000000pt}%
\definecolor{currentstroke}{rgb}{0.000000,0.000000,0.000000}%
\pgfsetstrokecolor{currentstroke}%
\pgfsetstrokeopacity{0.300000}%
\pgfsetdash{}{0pt}%
\pgfpathmoveto{\pgfqpoint{3.112741in}{0.383578in}}%
\pgfpathlineto{\pgfqpoint{3.130330in}{0.383578in}}%
\pgfpathlineto{\pgfqpoint{3.130330in}{1.926945in}}%
\pgfpathlineto{\pgfqpoint{3.112741in}{1.926945in}}%
\pgfpathclose%
\pgfusepath{fill}%
\end{pgfscope}%
\begin{pgfscope}%
\pgfpathrectangle{\pgfqpoint{0.526905in}{0.383578in}}{\pgfqpoint{3.875000in}{2.310000in}}%
\pgfusepath{clip}%
\pgfsetbuttcap%
\pgfsetmiterjoin%
\definecolor{currentfill}{rgb}{0.686275,0.352941,0.313725}%
\pgfsetfillcolor{currentfill}%
\pgfsetfillopacity{0.300000}%
\pgfsetlinewidth{0.000000pt}%
\definecolor{currentstroke}{rgb}{0.000000,0.000000,0.000000}%
\pgfsetstrokecolor{currentstroke}%
\pgfsetstrokeopacity{0.300000}%
\pgfsetdash{}{0pt}%
\pgfpathmoveto{\pgfqpoint{3.130330in}{0.383578in}}%
\pgfpathlineto{\pgfqpoint{3.147920in}{0.383578in}}%
\pgfpathlineto{\pgfqpoint{3.147920in}{1.971843in}}%
\pgfpathlineto{\pgfqpoint{3.130330in}{1.971843in}}%
\pgfpathclose%
\pgfusepath{fill}%
\end{pgfscope}%
\begin{pgfscope}%
\pgfpathrectangle{\pgfqpoint{0.526905in}{0.383578in}}{\pgfqpoint{3.875000in}{2.310000in}}%
\pgfusepath{clip}%
\pgfsetbuttcap%
\pgfsetmiterjoin%
\definecolor{currentfill}{rgb}{0.686275,0.352941,0.313725}%
\pgfsetfillcolor{currentfill}%
\pgfsetfillopacity{0.300000}%
\pgfsetlinewidth{0.000000pt}%
\definecolor{currentstroke}{rgb}{0.000000,0.000000,0.000000}%
\pgfsetstrokecolor{currentstroke}%
\pgfsetstrokeopacity{0.300000}%
\pgfsetdash{}{0pt}%
\pgfpathmoveto{\pgfqpoint{3.147920in}{0.383578in}}%
\pgfpathlineto{\pgfqpoint{3.165509in}{0.383578in}}%
\pgfpathlineto{\pgfqpoint{3.165509in}{1.769803in}}%
\pgfpathlineto{\pgfqpoint{3.147920in}{1.769803in}}%
\pgfpathclose%
\pgfusepath{fill}%
\end{pgfscope}%
\begin{pgfscope}%
\pgfpathrectangle{\pgfqpoint{0.526905in}{0.383578in}}{\pgfqpoint{3.875000in}{2.310000in}}%
\pgfusepath{clip}%
\pgfsetbuttcap%
\pgfsetmiterjoin%
\definecolor{currentfill}{rgb}{0.686275,0.352941,0.313725}%
\pgfsetfillcolor{currentfill}%
\pgfsetfillopacity{0.300000}%
\pgfsetlinewidth{0.000000pt}%
\definecolor{currentstroke}{rgb}{0.000000,0.000000,0.000000}%
\pgfsetstrokecolor{currentstroke}%
\pgfsetstrokeopacity{0.300000}%
\pgfsetdash{}{0pt}%
\pgfpathmoveto{\pgfqpoint{3.165509in}{0.383578in}}%
\pgfpathlineto{\pgfqpoint{3.183098in}{0.383578in}}%
\pgfpathlineto{\pgfqpoint{3.183098in}{1.719292in}}%
\pgfpathlineto{\pgfqpoint{3.165509in}{1.719292in}}%
\pgfpathclose%
\pgfusepath{fill}%
\end{pgfscope}%
\begin{pgfscope}%
\pgfpathrectangle{\pgfqpoint{0.526905in}{0.383578in}}{\pgfqpoint{3.875000in}{2.310000in}}%
\pgfusepath{clip}%
\pgfsetbuttcap%
\pgfsetmiterjoin%
\definecolor{currentfill}{rgb}{0.686275,0.352941,0.313725}%
\pgfsetfillcolor{currentfill}%
\pgfsetfillopacity{0.300000}%
\pgfsetlinewidth{0.000000pt}%
\definecolor{currentstroke}{rgb}{0.000000,0.000000,0.000000}%
\pgfsetstrokecolor{currentstroke}%
\pgfsetstrokeopacity{0.300000}%
\pgfsetdash{}{0pt}%
\pgfpathmoveto{\pgfqpoint{3.183098in}{0.383578in}}%
\pgfpathlineto{\pgfqpoint{3.200687in}{0.383578in}}%
\pgfpathlineto{\pgfqpoint{3.200687in}{1.623884in}}%
\pgfpathlineto{\pgfqpoint{3.183098in}{1.623884in}}%
\pgfpathclose%
\pgfusepath{fill}%
\end{pgfscope}%
\begin{pgfscope}%
\pgfpathrectangle{\pgfqpoint{0.526905in}{0.383578in}}{\pgfqpoint{3.875000in}{2.310000in}}%
\pgfusepath{clip}%
\pgfsetbuttcap%
\pgfsetmiterjoin%
\definecolor{currentfill}{rgb}{0.686275,0.352941,0.313725}%
\pgfsetfillcolor{currentfill}%
\pgfsetfillopacity{0.300000}%
\pgfsetlinewidth{0.000000pt}%
\definecolor{currentstroke}{rgb}{0.000000,0.000000,0.000000}%
\pgfsetstrokecolor{currentstroke}%
\pgfsetstrokeopacity{0.300000}%
\pgfsetdash{}{0pt}%
\pgfpathmoveto{\pgfqpoint{3.200687in}{0.383578in}}%
\pgfpathlineto{\pgfqpoint{3.218276in}{0.383578in}}%
\pgfpathlineto{\pgfqpoint{3.218276in}{1.494803in}}%
\pgfpathlineto{\pgfqpoint{3.200687in}{1.494803in}}%
\pgfpathclose%
\pgfusepath{fill}%
\end{pgfscope}%
\begin{pgfscope}%
\pgfpathrectangle{\pgfqpoint{0.526905in}{0.383578in}}{\pgfqpoint{3.875000in}{2.310000in}}%
\pgfusepath{clip}%
\pgfsetbuttcap%
\pgfsetmiterjoin%
\definecolor{currentfill}{rgb}{0.686275,0.352941,0.313725}%
\pgfsetfillcolor{currentfill}%
\pgfsetfillopacity{0.300000}%
\pgfsetlinewidth{0.000000pt}%
\definecolor{currentstroke}{rgb}{0.000000,0.000000,0.000000}%
\pgfsetstrokecolor{currentstroke}%
\pgfsetstrokeopacity{0.300000}%
\pgfsetdash{}{0pt}%
\pgfpathmoveto{\pgfqpoint{3.218276in}{0.383578in}}%
\pgfpathlineto{\pgfqpoint{3.235865in}{0.383578in}}%
\pgfpathlineto{\pgfqpoint{3.235865in}{1.657558in}}%
\pgfpathlineto{\pgfqpoint{3.218276in}{1.657558in}}%
\pgfpathclose%
\pgfusepath{fill}%
\end{pgfscope}%
\begin{pgfscope}%
\pgfpathrectangle{\pgfqpoint{0.526905in}{0.383578in}}{\pgfqpoint{3.875000in}{2.310000in}}%
\pgfusepath{clip}%
\pgfsetbuttcap%
\pgfsetmiterjoin%
\definecolor{currentfill}{rgb}{0.686275,0.352941,0.313725}%
\pgfsetfillcolor{currentfill}%
\pgfsetfillopacity{0.300000}%
\pgfsetlinewidth{0.000000pt}%
\definecolor{currentstroke}{rgb}{0.000000,0.000000,0.000000}%
\pgfsetstrokecolor{currentstroke}%
\pgfsetstrokeopacity{0.300000}%
\pgfsetdash{}{0pt}%
\pgfpathmoveto{\pgfqpoint{3.235865in}{0.383578in}}%
\pgfpathlineto{\pgfqpoint{3.253454in}{0.383578in}}%
\pgfpathlineto{\pgfqpoint{3.253454in}{1.477966in}}%
\pgfpathlineto{\pgfqpoint{3.235865in}{1.477966in}}%
\pgfpathclose%
\pgfusepath{fill}%
\end{pgfscope}%
\begin{pgfscope}%
\pgfpathrectangle{\pgfqpoint{0.526905in}{0.383578in}}{\pgfqpoint{3.875000in}{2.310000in}}%
\pgfusepath{clip}%
\pgfsetbuttcap%
\pgfsetmiterjoin%
\definecolor{currentfill}{rgb}{0.686275,0.352941,0.313725}%
\pgfsetfillcolor{currentfill}%
\pgfsetfillopacity{0.300000}%
\pgfsetlinewidth{0.000000pt}%
\definecolor{currentstroke}{rgb}{0.000000,0.000000,0.000000}%
\pgfsetstrokecolor{currentstroke}%
\pgfsetstrokeopacity{0.300000}%
\pgfsetdash{}{0pt}%
\pgfpathmoveto{\pgfqpoint{3.253454in}{0.383578in}}%
\pgfpathlineto{\pgfqpoint{3.271043in}{0.383578in}}%
\pgfpathlineto{\pgfqpoint{3.271043in}{1.483578in}}%
\pgfpathlineto{\pgfqpoint{3.253454in}{1.483578in}}%
\pgfpathclose%
\pgfusepath{fill}%
\end{pgfscope}%
\begin{pgfscope}%
\pgfpathrectangle{\pgfqpoint{0.526905in}{0.383578in}}{\pgfqpoint{3.875000in}{2.310000in}}%
\pgfusepath{clip}%
\pgfsetbuttcap%
\pgfsetmiterjoin%
\definecolor{currentfill}{rgb}{0.686275,0.352941,0.313725}%
\pgfsetfillcolor{currentfill}%
\pgfsetfillopacity{0.300000}%
\pgfsetlinewidth{0.000000pt}%
\definecolor{currentstroke}{rgb}{0.000000,0.000000,0.000000}%
\pgfsetstrokecolor{currentstroke}%
\pgfsetstrokeopacity{0.300000}%
\pgfsetdash{}{0pt}%
\pgfpathmoveto{\pgfqpoint{3.271043in}{0.383578in}}%
\pgfpathlineto{\pgfqpoint{3.288632in}{0.383578in}}%
\pgfpathlineto{\pgfqpoint{3.288632in}{1.360109in}}%
\pgfpathlineto{\pgfqpoint{3.271043in}{1.360109in}}%
\pgfpathclose%
\pgfusepath{fill}%
\end{pgfscope}%
\begin{pgfscope}%
\pgfpathrectangle{\pgfqpoint{0.526905in}{0.383578in}}{\pgfqpoint{3.875000in}{2.310000in}}%
\pgfusepath{clip}%
\pgfsetbuttcap%
\pgfsetmiterjoin%
\definecolor{currentfill}{rgb}{0.686275,0.352941,0.313725}%
\pgfsetfillcolor{currentfill}%
\pgfsetfillopacity{0.300000}%
\pgfsetlinewidth{0.000000pt}%
\definecolor{currentstroke}{rgb}{0.000000,0.000000,0.000000}%
\pgfsetstrokecolor{currentstroke}%
\pgfsetstrokeopacity{0.300000}%
\pgfsetdash{}{0pt}%
\pgfpathmoveto{\pgfqpoint{3.288632in}{0.383578in}}%
\pgfpathlineto{\pgfqpoint{3.306221in}{0.383578in}}%
\pgfpathlineto{\pgfqpoint{3.306221in}{1.225415in}}%
\pgfpathlineto{\pgfqpoint{3.288632in}{1.225415in}}%
\pgfpathclose%
\pgfusepath{fill}%
\end{pgfscope}%
\begin{pgfscope}%
\pgfpathrectangle{\pgfqpoint{0.526905in}{0.383578in}}{\pgfqpoint{3.875000in}{2.310000in}}%
\pgfusepath{clip}%
\pgfsetbuttcap%
\pgfsetmiterjoin%
\definecolor{currentfill}{rgb}{0.686275,0.352941,0.313725}%
\pgfsetfillcolor{currentfill}%
\pgfsetfillopacity{0.300000}%
\pgfsetlinewidth{0.000000pt}%
\definecolor{currentstroke}{rgb}{0.000000,0.000000,0.000000}%
\pgfsetstrokecolor{currentstroke}%
\pgfsetstrokeopacity{0.300000}%
\pgfsetdash{}{0pt}%
\pgfpathmoveto{\pgfqpoint{3.306221in}{0.383578in}}%
\pgfpathlineto{\pgfqpoint{3.323810in}{0.383578in}}%
\pgfpathlineto{\pgfqpoint{3.323810in}{1.174905in}}%
\pgfpathlineto{\pgfqpoint{3.306221in}{1.174905in}}%
\pgfpathclose%
\pgfusepath{fill}%
\end{pgfscope}%
\begin{pgfscope}%
\pgfpathrectangle{\pgfqpoint{0.526905in}{0.383578in}}{\pgfqpoint{3.875000in}{2.310000in}}%
\pgfusepath{clip}%
\pgfsetbuttcap%
\pgfsetmiterjoin%
\definecolor{currentfill}{rgb}{0.686275,0.352941,0.313725}%
\pgfsetfillcolor{currentfill}%
\pgfsetfillopacity{0.300000}%
\pgfsetlinewidth{0.000000pt}%
\definecolor{currentstroke}{rgb}{0.000000,0.000000,0.000000}%
\pgfsetstrokecolor{currentstroke}%
\pgfsetstrokeopacity{0.300000}%
\pgfsetdash{}{0pt}%
\pgfpathmoveto{\pgfqpoint{3.323810in}{0.383578in}}%
\pgfpathlineto{\pgfqpoint{3.341399in}{0.383578in}}%
\pgfpathlineto{\pgfqpoint{3.341399in}{1.107558in}}%
\pgfpathlineto{\pgfqpoint{3.323810in}{1.107558in}}%
\pgfpathclose%
\pgfusepath{fill}%
\end{pgfscope}%
\begin{pgfscope}%
\pgfpathrectangle{\pgfqpoint{0.526905in}{0.383578in}}{\pgfqpoint{3.875000in}{2.310000in}}%
\pgfusepath{clip}%
\pgfsetbuttcap%
\pgfsetmiterjoin%
\definecolor{currentfill}{rgb}{0.686275,0.352941,0.313725}%
\pgfsetfillcolor{currentfill}%
\pgfsetfillopacity{0.300000}%
\pgfsetlinewidth{0.000000pt}%
\definecolor{currentstroke}{rgb}{0.000000,0.000000,0.000000}%
\pgfsetstrokecolor{currentstroke}%
\pgfsetstrokeopacity{0.300000}%
\pgfsetdash{}{0pt}%
\pgfpathmoveto{\pgfqpoint{3.341399in}{0.383578in}}%
\pgfpathlineto{\pgfqpoint{3.358988in}{0.383578in}}%
\pgfpathlineto{\pgfqpoint{3.358988in}{0.995313in}}%
\pgfpathlineto{\pgfqpoint{3.341399in}{0.995313in}}%
\pgfpathclose%
\pgfusepath{fill}%
\end{pgfscope}%
\begin{pgfscope}%
\pgfpathrectangle{\pgfqpoint{0.526905in}{0.383578in}}{\pgfqpoint{3.875000in}{2.310000in}}%
\pgfusepath{clip}%
\pgfsetbuttcap%
\pgfsetmiterjoin%
\definecolor{currentfill}{rgb}{0.686275,0.352941,0.313725}%
\pgfsetfillcolor{currentfill}%
\pgfsetfillopacity{0.300000}%
\pgfsetlinewidth{0.000000pt}%
\definecolor{currentstroke}{rgb}{0.000000,0.000000,0.000000}%
\pgfsetstrokecolor{currentstroke}%
\pgfsetstrokeopacity{0.300000}%
\pgfsetdash{}{0pt}%
\pgfpathmoveto{\pgfqpoint{3.358988in}{0.383578in}}%
\pgfpathlineto{\pgfqpoint{3.376577in}{0.383578in}}%
\pgfpathlineto{\pgfqpoint{3.376577in}{1.090721in}}%
\pgfpathlineto{\pgfqpoint{3.358988in}{1.090721in}}%
\pgfpathclose%
\pgfusepath{fill}%
\end{pgfscope}%
\begin{pgfscope}%
\pgfpathrectangle{\pgfqpoint{0.526905in}{0.383578in}}{\pgfqpoint{3.875000in}{2.310000in}}%
\pgfusepath{clip}%
\pgfsetbuttcap%
\pgfsetmiterjoin%
\definecolor{currentfill}{rgb}{0.686275,0.352941,0.313725}%
\pgfsetfillcolor{currentfill}%
\pgfsetfillopacity{0.300000}%
\pgfsetlinewidth{0.000000pt}%
\definecolor{currentstroke}{rgb}{0.000000,0.000000,0.000000}%
\pgfsetstrokecolor{currentstroke}%
\pgfsetstrokeopacity{0.300000}%
\pgfsetdash{}{0pt}%
\pgfpathmoveto{\pgfqpoint{3.376577in}{0.383578in}}%
\pgfpathlineto{\pgfqpoint{3.394166in}{0.383578in}}%
\pgfpathlineto{\pgfqpoint{3.394166in}{1.045823in}}%
\pgfpathlineto{\pgfqpoint{3.376577in}{1.045823in}}%
\pgfpathclose%
\pgfusepath{fill}%
\end{pgfscope}%
\begin{pgfscope}%
\pgfpathrectangle{\pgfqpoint{0.526905in}{0.383578in}}{\pgfqpoint{3.875000in}{2.310000in}}%
\pgfusepath{clip}%
\pgfsetbuttcap%
\pgfsetmiterjoin%
\definecolor{currentfill}{rgb}{0.686275,0.352941,0.313725}%
\pgfsetfillcolor{currentfill}%
\pgfsetfillopacity{0.300000}%
\pgfsetlinewidth{0.000000pt}%
\definecolor{currentstroke}{rgb}{0.000000,0.000000,0.000000}%
\pgfsetstrokecolor{currentstroke}%
\pgfsetstrokeopacity{0.300000}%
\pgfsetdash{}{0pt}%
\pgfpathmoveto{\pgfqpoint{3.394166in}{0.383578in}}%
\pgfpathlineto{\pgfqpoint{3.411755in}{0.383578in}}%
\pgfpathlineto{\pgfqpoint{3.411755in}{1.017762in}}%
\pgfpathlineto{\pgfqpoint{3.394166in}{1.017762in}}%
\pgfpathclose%
\pgfusepath{fill}%
\end{pgfscope}%
\begin{pgfscope}%
\pgfpathrectangle{\pgfqpoint{0.526905in}{0.383578in}}{\pgfqpoint{3.875000in}{2.310000in}}%
\pgfusepath{clip}%
\pgfsetbuttcap%
\pgfsetmiterjoin%
\definecolor{currentfill}{rgb}{0.686275,0.352941,0.313725}%
\pgfsetfillcolor{currentfill}%
\pgfsetfillopacity{0.300000}%
\pgfsetlinewidth{0.000000pt}%
\definecolor{currentstroke}{rgb}{0.000000,0.000000,0.000000}%
\pgfsetstrokecolor{currentstroke}%
\pgfsetstrokeopacity{0.300000}%
\pgfsetdash{}{0pt}%
\pgfpathmoveto{\pgfqpoint{3.411755in}{0.383578in}}%
\pgfpathlineto{\pgfqpoint{3.429344in}{0.383578in}}%
\pgfpathlineto{\pgfqpoint{3.429344in}{1.113170in}}%
\pgfpathlineto{\pgfqpoint{3.411755in}{1.113170in}}%
\pgfpathclose%
\pgfusepath{fill}%
\end{pgfscope}%
\begin{pgfscope}%
\pgfpathrectangle{\pgfqpoint{0.526905in}{0.383578in}}{\pgfqpoint{3.875000in}{2.310000in}}%
\pgfusepath{clip}%
\pgfsetbuttcap%
\pgfsetmiterjoin%
\definecolor{currentfill}{rgb}{0.686275,0.352941,0.313725}%
\pgfsetfillcolor{currentfill}%
\pgfsetfillopacity{0.300000}%
\pgfsetlinewidth{0.000000pt}%
\definecolor{currentstroke}{rgb}{0.000000,0.000000,0.000000}%
\pgfsetstrokecolor{currentstroke}%
\pgfsetstrokeopacity{0.300000}%
\pgfsetdash{}{0pt}%
\pgfpathmoveto{\pgfqpoint{3.429344in}{0.383578in}}%
\pgfpathlineto{\pgfqpoint{3.446933in}{0.383578in}}%
\pgfpathlineto{\pgfqpoint{3.446933in}{1.040211in}}%
\pgfpathlineto{\pgfqpoint{3.429344in}{1.040211in}}%
\pgfpathclose%
\pgfusepath{fill}%
\end{pgfscope}%
\begin{pgfscope}%
\pgfpathrectangle{\pgfqpoint{0.526905in}{0.383578in}}{\pgfqpoint{3.875000in}{2.310000in}}%
\pgfusepath{clip}%
\pgfsetbuttcap%
\pgfsetmiterjoin%
\definecolor{currentfill}{rgb}{0.686275,0.352941,0.313725}%
\pgfsetfillcolor{currentfill}%
\pgfsetfillopacity{0.300000}%
\pgfsetlinewidth{0.000000pt}%
\definecolor{currentstroke}{rgb}{0.000000,0.000000,0.000000}%
\pgfsetstrokecolor{currentstroke}%
\pgfsetstrokeopacity{0.300000}%
\pgfsetdash{}{0pt}%
\pgfpathmoveto{\pgfqpoint{3.446933in}{0.383578in}}%
\pgfpathlineto{\pgfqpoint{3.464522in}{0.383578in}}%
\pgfpathlineto{\pgfqpoint{3.464522in}{0.989701in}}%
\pgfpathlineto{\pgfqpoint{3.446933in}{0.989701in}}%
\pgfpathclose%
\pgfusepath{fill}%
\end{pgfscope}%
\begin{pgfscope}%
\pgfpathrectangle{\pgfqpoint{0.526905in}{0.383578in}}{\pgfqpoint{3.875000in}{2.310000in}}%
\pgfusepath{clip}%
\pgfsetbuttcap%
\pgfsetmiterjoin%
\definecolor{currentfill}{rgb}{0.686275,0.352941,0.313725}%
\pgfsetfillcolor{currentfill}%
\pgfsetfillopacity{0.300000}%
\pgfsetlinewidth{0.000000pt}%
\definecolor{currentstroke}{rgb}{0.000000,0.000000,0.000000}%
\pgfsetstrokecolor{currentstroke}%
\pgfsetstrokeopacity{0.300000}%
\pgfsetdash{}{0pt}%
\pgfpathmoveto{\pgfqpoint{3.464522in}{0.383578in}}%
\pgfpathlineto{\pgfqpoint{3.482112in}{0.383578in}}%
\pgfpathlineto{\pgfqpoint{3.482112in}{0.984088in}}%
\pgfpathlineto{\pgfqpoint{3.464522in}{0.984088in}}%
\pgfpathclose%
\pgfusepath{fill}%
\end{pgfscope}%
\begin{pgfscope}%
\pgfpathrectangle{\pgfqpoint{0.526905in}{0.383578in}}{\pgfqpoint{3.875000in}{2.310000in}}%
\pgfusepath{clip}%
\pgfsetbuttcap%
\pgfsetmiterjoin%
\definecolor{currentfill}{rgb}{0.686275,0.352941,0.313725}%
\pgfsetfillcolor{currentfill}%
\pgfsetfillopacity{0.300000}%
\pgfsetlinewidth{0.000000pt}%
\definecolor{currentstroke}{rgb}{0.000000,0.000000,0.000000}%
\pgfsetstrokecolor{currentstroke}%
\pgfsetstrokeopacity{0.300000}%
\pgfsetdash{}{0pt}%
\pgfpathmoveto{\pgfqpoint{3.482112in}{0.383578in}}%
\pgfpathlineto{\pgfqpoint{3.499701in}{0.383578in}}%
\pgfpathlineto{\pgfqpoint{3.499701in}{0.888680in}}%
\pgfpathlineto{\pgfqpoint{3.482112in}{0.888680in}}%
\pgfpathclose%
\pgfusepath{fill}%
\end{pgfscope}%
\begin{pgfscope}%
\pgfpathrectangle{\pgfqpoint{0.526905in}{0.383578in}}{\pgfqpoint{3.875000in}{2.310000in}}%
\pgfusepath{clip}%
\pgfsetbuttcap%
\pgfsetmiterjoin%
\definecolor{currentfill}{rgb}{0.686275,0.352941,0.313725}%
\pgfsetfillcolor{currentfill}%
\pgfsetfillopacity{0.300000}%
\pgfsetlinewidth{0.000000pt}%
\definecolor{currentstroke}{rgb}{0.000000,0.000000,0.000000}%
\pgfsetstrokecolor{currentstroke}%
\pgfsetstrokeopacity{0.300000}%
\pgfsetdash{}{0pt}%
\pgfpathmoveto{\pgfqpoint{3.499701in}{0.383578in}}%
\pgfpathlineto{\pgfqpoint{3.517290in}{0.383578in}}%
\pgfpathlineto{\pgfqpoint{3.517290in}{0.883068in}}%
\pgfpathlineto{\pgfqpoint{3.499701in}{0.883068in}}%
\pgfpathclose%
\pgfusepath{fill}%
\end{pgfscope}%
\begin{pgfscope}%
\pgfpathrectangle{\pgfqpoint{0.526905in}{0.383578in}}{\pgfqpoint{3.875000in}{2.310000in}}%
\pgfusepath{clip}%
\pgfsetbuttcap%
\pgfsetmiterjoin%
\definecolor{currentfill}{rgb}{0.686275,0.352941,0.313725}%
\pgfsetfillcolor{currentfill}%
\pgfsetfillopacity{0.300000}%
\pgfsetlinewidth{0.000000pt}%
\definecolor{currentstroke}{rgb}{0.000000,0.000000,0.000000}%
\pgfsetstrokecolor{currentstroke}%
\pgfsetstrokeopacity{0.300000}%
\pgfsetdash{}{0pt}%
\pgfpathmoveto{\pgfqpoint{3.517290in}{0.383578in}}%
\pgfpathlineto{\pgfqpoint{3.534879in}{0.383578in}}%
\pgfpathlineto{\pgfqpoint{3.534879in}{0.838170in}}%
\pgfpathlineto{\pgfqpoint{3.517290in}{0.838170in}}%
\pgfpathclose%
\pgfusepath{fill}%
\end{pgfscope}%
\begin{pgfscope}%
\pgfpathrectangle{\pgfqpoint{0.526905in}{0.383578in}}{\pgfqpoint{3.875000in}{2.310000in}}%
\pgfusepath{clip}%
\pgfsetbuttcap%
\pgfsetmiterjoin%
\definecolor{currentfill}{rgb}{0.686275,0.352941,0.313725}%
\pgfsetfillcolor{currentfill}%
\pgfsetfillopacity{0.300000}%
\pgfsetlinewidth{0.000000pt}%
\definecolor{currentstroke}{rgb}{0.000000,0.000000,0.000000}%
\pgfsetstrokecolor{currentstroke}%
\pgfsetstrokeopacity{0.300000}%
\pgfsetdash{}{0pt}%
\pgfpathmoveto{\pgfqpoint{3.534879in}{0.383578in}}%
\pgfpathlineto{\pgfqpoint{3.552468in}{0.383578in}}%
\pgfpathlineto{\pgfqpoint{3.552468in}{0.748374in}}%
\pgfpathlineto{\pgfqpoint{3.534879in}{0.748374in}}%
\pgfpathclose%
\pgfusepath{fill}%
\end{pgfscope}%
\begin{pgfscope}%
\pgfpathrectangle{\pgfqpoint{0.526905in}{0.383578in}}{\pgfqpoint{3.875000in}{2.310000in}}%
\pgfusepath{clip}%
\pgfsetbuttcap%
\pgfsetmiterjoin%
\definecolor{currentfill}{rgb}{0.686275,0.352941,0.313725}%
\pgfsetfillcolor{currentfill}%
\pgfsetfillopacity{0.300000}%
\pgfsetlinewidth{0.000000pt}%
\definecolor{currentstroke}{rgb}{0.000000,0.000000,0.000000}%
\pgfsetstrokecolor{currentstroke}%
\pgfsetstrokeopacity{0.300000}%
\pgfsetdash{}{0pt}%
\pgfpathmoveto{\pgfqpoint{3.552468in}{0.383578in}}%
\pgfpathlineto{\pgfqpoint{3.570057in}{0.383578in}}%
\pgfpathlineto{\pgfqpoint{3.570057in}{0.770823in}}%
\pgfpathlineto{\pgfqpoint{3.552468in}{0.770823in}}%
\pgfpathclose%
\pgfusepath{fill}%
\end{pgfscope}%
\begin{pgfscope}%
\pgfpathrectangle{\pgfqpoint{0.526905in}{0.383578in}}{\pgfqpoint{3.875000in}{2.310000in}}%
\pgfusepath{clip}%
\pgfsetbuttcap%
\pgfsetmiterjoin%
\definecolor{currentfill}{rgb}{0.686275,0.352941,0.313725}%
\pgfsetfillcolor{currentfill}%
\pgfsetfillopacity{0.300000}%
\pgfsetlinewidth{0.000000pt}%
\definecolor{currentstroke}{rgb}{0.000000,0.000000,0.000000}%
\pgfsetstrokecolor{currentstroke}%
\pgfsetstrokeopacity{0.300000}%
\pgfsetdash{}{0pt}%
\pgfpathmoveto{\pgfqpoint{3.570057in}{0.383578in}}%
\pgfpathlineto{\pgfqpoint{3.587646in}{0.383578in}}%
\pgfpathlineto{\pgfqpoint{3.587646in}{0.759598in}}%
\pgfpathlineto{\pgfqpoint{3.570057in}{0.759598in}}%
\pgfpathclose%
\pgfusepath{fill}%
\end{pgfscope}%
\begin{pgfscope}%
\pgfpathrectangle{\pgfqpoint{0.526905in}{0.383578in}}{\pgfqpoint{3.875000in}{2.310000in}}%
\pgfusepath{clip}%
\pgfsetbuttcap%
\pgfsetmiterjoin%
\definecolor{currentfill}{rgb}{0.686275,0.352941,0.313725}%
\pgfsetfillcolor{currentfill}%
\pgfsetfillopacity{0.300000}%
\pgfsetlinewidth{0.000000pt}%
\definecolor{currentstroke}{rgb}{0.000000,0.000000,0.000000}%
\pgfsetstrokecolor{currentstroke}%
\pgfsetstrokeopacity{0.300000}%
\pgfsetdash{}{0pt}%
\pgfpathmoveto{\pgfqpoint{3.587646in}{0.383578in}}%
\pgfpathlineto{\pgfqpoint{3.605235in}{0.383578in}}%
\pgfpathlineto{\pgfqpoint{3.605235in}{0.709088in}}%
\pgfpathlineto{\pgfqpoint{3.587646in}{0.709088in}}%
\pgfpathclose%
\pgfusepath{fill}%
\end{pgfscope}%
\begin{pgfscope}%
\pgfpathrectangle{\pgfqpoint{0.526905in}{0.383578in}}{\pgfqpoint{3.875000in}{2.310000in}}%
\pgfusepath{clip}%
\pgfsetbuttcap%
\pgfsetmiterjoin%
\definecolor{currentfill}{rgb}{0.686275,0.352941,0.313725}%
\pgfsetfillcolor{currentfill}%
\pgfsetfillopacity{0.300000}%
\pgfsetlinewidth{0.000000pt}%
\definecolor{currentstroke}{rgb}{0.000000,0.000000,0.000000}%
\pgfsetstrokecolor{currentstroke}%
\pgfsetstrokeopacity{0.300000}%
\pgfsetdash{}{0pt}%
\pgfpathmoveto{\pgfqpoint{3.605235in}{0.383578in}}%
\pgfpathlineto{\pgfqpoint{3.622824in}{0.383578in}}%
\pgfpathlineto{\pgfqpoint{3.622824in}{0.720313in}}%
\pgfpathlineto{\pgfqpoint{3.605235in}{0.720313in}}%
\pgfpathclose%
\pgfusepath{fill}%
\end{pgfscope}%
\begin{pgfscope}%
\pgfpathrectangle{\pgfqpoint{0.526905in}{0.383578in}}{\pgfqpoint{3.875000in}{2.310000in}}%
\pgfusepath{clip}%
\pgfsetbuttcap%
\pgfsetmiterjoin%
\definecolor{currentfill}{rgb}{0.686275,0.352941,0.313725}%
\pgfsetfillcolor{currentfill}%
\pgfsetfillopacity{0.300000}%
\pgfsetlinewidth{0.000000pt}%
\definecolor{currentstroke}{rgb}{0.000000,0.000000,0.000000}%
\pgfsetstrokecolor{currentstroke}%
\pgfsetstrokeopacity{0.300000}%
\pgfsetdash{}{0pt}%
\pgfpathmoveto{\pgfqpoint{3.622824in}{0.383578in}}%
\pgfpathlineto{\pgfqpoint{3.640413in}{0.383578in}}%
\pgfpathlineto{\pgfqpoint{3.640413in}{0.608068in}}%
\pgfpathlineto{\pgfqpoint{3.622824in}{0.608068in}}%
\pgfpathclose%
\pgfusepath{fill}%
\end{pgfscope}%
\begin{pgfscope}%
\pgfpathrectangle{\pgfqpoint{0.526905in}{0.383578in}}{\pgfqpoint{3.875000in}{2.310000in}}%
\pgfusepath{clip}%
\pgfsetbuttcap%
\pgfsetmiterjoin%
\definecolor{currentfill}{rgb}{0.686275,0.352941,0.313725}%
\pgfsetfillcolor{currentfill}%
\pgfsetfillopacity{0.300000}%
\pgfsetlinewidth{0.000000pt}%
\definecolor{currentstroke}{rgb}{0.000000,0.000000,0.000000}%
\pgfsetstrokecolor{currentstroke}%
\pgfsetstrokeopacity{0.300000}%
\pgfsetdash{}{0pt}%
\pgfpathmoveto{\pgfqpoint{3.640413in}{0.383578in}}%
\pgfpathlineto{\pgfqpoint{3.658002in}{0.383578in}}%
\pgfpathlineto{\pgfqpoint{3.658002in}{0.697864in}}%
\pgfpathlineto{\pgfqpoint{3.640413in}{0.697864in}}%
\pgfpathclose%
\pgfusepath{fill}%
\end{pgfscope}%
\begin{pgfscope}%
\pgfpathrectangle{\pgfqpoint{0.526905in}{0.383578in}}{\pgfqpoint{3.875000in}{2.310000in}}%
\pgfusepath{clip}%
\pgfsetbuttcap%
\pgfsetmiterjoin%
\definecolor{currentfill}{rgb}{0.686275,0.352941,0.313725}%
\pgfsetfillcolor{currentfill}%
\pgfsetfillopacity{0.300000}%
\pgfsetlinewidth{0.000000pt}%
\definecolor{currentstroke}{rgb}{0.000000,0.000000,0.000000}%
\pgfsetstrokecolor{currentstroke}%
\pgfsetstrokeopacity{0.300000}%
\pgfsetdash{}{0pt}%
\pgfpathmoveto{\pgfqpoint{3.658002in}{0.383578in}}%
\pgfpathlineto{\pgfqpoint{3.675591in}{0.383578in}}%
\pgfpathlineto{\pgfqpoint{3.675591in}{0.602456in}}%
\pgfpathlineto{\pgfqpoint{3.658002in}{0.602456in}}%
\pgfpathclose%
\pgfusepath{fill}%
\end{pgfscope}%
\begin{pgfscope}%
\pgfpathrectangle{\pgfqpoint{0.526905in}{0.383578in}}{\pgfqpoint{3.875000in}{2.310000in}}%
\pgfusepath{clip}%
\pgfsetbuttcap%
\pgfsetmiterjoin%
\definecolor{currentfill}{rgb}{0.686275,0.352941,0.313725}%
\pgfsetfillcolor{currentfill}%
\pgfsetfillopacity{0.300000}%
\pgfsetlinewidth{0.000000pt}%
\definecolor{currentstroke}{rgb}{0.000000,0.000000,0.000000}%
\pgfsetstrokecolor{currentstroke}%
\pgfsetstrokeopacity{0.300000}%
\pgfsetdash{}{0pt}%
\pgfpathmoveto{\pgfqpoint{3.675591in}{0.383578in}}%
\pgfpathlineto{\pgfqpoint{3.693180in}{0.383578in}}%
\pgfpathlineto{\pgfqpoint{3.693180in}{0.613680in}}%
\pgfpathlineto{\pgfqpoint{3.675591in}{0.613680in}}%
\pgfpathclose%
\pgfusepath{fill}%
\end{pgfscope}%
\begin{pgfscope}%
\pgfpathrectangle{\pgfqpoint{0.526905in}{0.383578in}}{\pgfqpoint{3.875000in}{2.310000in}}%
\pgfusepath{clip}%
\pgfsetbuttcap%
\pgfsetmiterjoin%
\definecolor{currentfill}{rgb}{0.686275,0.352941,0.313725}%
\pgfsetfillcolor{currentfill}%
\pgfsetfillopacity{0.300000}%
\pgfsetlinewidth{0.000000pt}%
\definecolor{currentstroke}{rgb}{0.000000,0.000000,0.000000}%
\pgfsetstrokecolor{currentstroke}%
\pgfsetstrokeopacity{0.300000}%
\pgfsetdash{}{0pt}%
\pgfpathmoveto{\pgfqpoint{3.693180in}{0.383578in}}%
\pgfpathlineto{\pgfqpoint{3.710769in}{0.383578in}}%
\pgfpathlineto{\pgfqpoint{3.710769in}{0.529496in}}%
\pgfpathlineto{\pgfqpoint{3.693180in}{0.529496in}}%
\pgfpathclose%
\pgfusepath{fill}%
\end{pgfscope}%
\begin{pgfscope}%
\pgfpathrectangle{\pgfqpoint{0.526905in}{0.383578in}}{\pgfqpoint{3.875000in}{2.310000in}}%
\pgfusepath{clip}%
\pgfsetbuttcap%
\pgfsetmiterjoin%
\definecolor{currentfill}{rgb}{0.686275,0.352941,0.313725}%
\pgfsetfillcolor{currentfill}%
\pgfsetfillopacity{0.300000}%
\pgfsetlinewidth{0.000000pt}%
\definecolor{currentstroke}{rgb}{0.000000,0.000000,0.000000}%
\pgfsetstrokecolor{currentstroke}%
\pgfsetstrokeopacity{0.300000}%
\pgfsetdash{}{0pt}%
\pgfpathmoveto{\pgfqpoint{3.710769in}{0.383578in}}%
\pgfpathlineto{\pgfqpoint{3.728358in}{0.383578in}}%
\pgfpathlineto{\pgfqpoint{3.728358in}{0.551945in}}%
\pgfpathlineto{\pgfqpoint{3.710769in}{0.551945in}}%
\pgfpathclose%
\pgfusepath{fill}%
\end{pgfscope}%
\begin{pgfscope}%
\pgfpathrectangle{\pgfqpoint{0.526905in}{0.383578in}}{\pgfqpoint{3.875000in}{2.310000in}}%
\pgfusepath{clip}%
\pgfsetbuttcap%
\pgfsetmiterjoin%
\definecolor{currentfill}{rgb}{0.686275,0.352941,0.313725}%
\pgfsetfillcolor{currentfill}%
\pgfsetfillopacity{0.300000}%
\pgfsetlinewidth{0.000000pt}%
\definecolor{currentstroke}{rgb}{0.000000,0.000000,0.000000}%
\pgfsetstrokecolor{currentstroke}%
\pgfsetstrokeopacity{0.300000}%
\pgfsetdash{}{0pt}%
\pgfpathmoveto{\pgfqpoint{3.728358in}{0.383578in}}%
\pgfpathlineto{\pgfqpoint{3.745947in}{0.383578in}}%
\pgfpathlineto{\pgfqpoint{3.745947in}{0.563170in}}%
\pgfpathlineto{\pgfqpoint{3.728358in}{0.563170in}}%
\pgfpathclose%
\pgfusepath{fill}%
\end{pgfscope}%
\begin{pgfscope}%
\pgfpathrectangle{\pgfqpoint{0.526905in}{0.383578in}}{\pgfqpoint{3.875000in}{2.310000in}}%
\pgfusepath{clip}%
\pgfsetbuttcap%
\pgfsetmiterjoin%
\definecolor{currentfill}{rgb}{0.686275,0.352941,0.313725}%
\pgfsetfillcolor{currentfill}%
\pgfsetfillopacity{0.300000}%
\pgfsetlinewidth{0.000000pt}%
\definecolor{currentstroke}{rgb}{0.000000,0.000000,0.000000}%
\pgfsetstrokecolor{currentstroke}%
\pgfsetstrokeopacity{0.300000}%
\pgfsetdash{}{0pt}%
\pgfpathmoveto{\pgfqpoint{3.745947in}{0.383578in}}%
\pgfpathlineto{\pgfqpoint{3.763536in}{0.383578in}}%
\pgfpathlineto{\pgfqpoint{3.763536in}{0.518272in}}%
\pgfpathlineto{\pgfqpoint{3.745947in}{0.518272in}}%
\pgfpathclose%
\pgfusepath{fill}%
\end{pgfscope}%
\begin{pgfscope}%
\pgfpathrectangle{\pgfqpoint{0.526905in}{0.383578in}}{\pgfqpoint{3.875000in}{2.310000in}}%
\pgfusepath{clip}%
\pgfsetbuttcap%
\pgfsetmiterjoin%
\definecolor{currentfill}{rgb}{0.686275,0.352941,0.313725}%
\pgfsetfillcolor{currentfill}%
\pgfsetfillopacity{0.300000}%
\pgfsetlinewidth{0.000000pt}%
\definecolor{currentstroke}{rgb}{0.000000,0.000000,0.000000}%
\pgfsetstrokecolor{currentstroke}%
\pgfsetstrokeopacity{0.300000}%
\pgfsetdash{}{0pt}%
\pgfpathmoveto{\pgfqpoint{3.763536in}{0.383578in}}%
\pgfpathlineto{\pgfqpoint{3.781125in}{0.383578in}}%
\pgfpathlineto{\pgfqpoint{3.781125in}{0.563170in}}%
\pgfpathlineto{\pgfqpoint{3.763536in}{0.563170in}}%
\pgfpathclose%
\pgfusepath{fill}%
\end{pgfscope}%
\begin{pgfscope}%
\pgfpathrectangle{\pgfqpoint{0.526905in}{0.383578in}}{\pgfqpoint{3.875000in}{2.310000in}}%
\pgfusepath{clip}%
\pgfsetbuttcap%
\pgfsetmiterjoin%
\definecolor{currentfill}{rgb}{0.686275,0.352941,0.313725}%
\pgfsetfillcolor{currentfill}%
\pgfsetfillopacity{0.300000}%
\pgfsetlinewidth{0.000000pt}%
\definecolor{currentstroke}{rgb}{0.000000,0.000000,0.000000}%
\pgfsetstrokecolor{currentstroke}%
\pgfsetstrokeopacity{0.300000}%
\pgfsetdash{}{0pt}%
\pgfpathmoveto{\pgfqpoint{3.781125in}{0.383578in}}%
\pgfpathlineto{\pgfqpoint{3.798715in}{0.383578in}}%
\pgfpathlineto{\pgfqpoint{3.798715in}{0.495823in}}%
\pgfpathlineto{\pgfqpoint{3.781125in}{0.495823in}}%
\pgfpathclose%
\pgfusepath{fill}%
\end{pgfscope}%
\begin{pgfscope}%
\pgfpathrectangle{\pgfqpoint{0.526905in}{0.383578in}}{\pgfqpoint{3.875000in}{2.310000in}}%
\pgfusepath{clip}%
\pgfsetbuttcap%
\pgfsetmiterjoin%
\definecolor{currentfill}{rgb}{0.686275,0.352941,0.313725}%
\pgfsetfillcolor{currentfill}%
\pgfsetfillopacity{0.300000}%
\pgfsetlinewidth{0.000000pt}%
\definecolor{currentstroke}{rgb}{0.000000,0.000000,0.000000}%
\pgfsetstrokecolor{currentstroke}%
\pgfsetstrokeopacity{0.300000}%
\pgfsetdash{}{0pt}%
\pgfpathmoveto{\pgfqpoint{3.798715in}{0.383578in}}%
\pgfpathlineto{\pgfqpoint{3.816304in}{0.383578in}}%
\pgfpathlineto{\pgfqpoint{3.816304in}{0.540721in}}%
\pgfpathlineto{\pgfqpoint{3.798715in}{0.540721in}}%
\pgfpathclose%
\pgfusepath{fill}%
\end{pgfscope}%
\begin{pgfscope}%
\pgfpathrectangle{\pgfqpoint{0.526905in}{0.383578in}}{\pgfqpoint{3.875000in}{2.310000in}}%
\pgfusepath{clip}%
\pgfsetbuttcap%
\pgfsetmiterjoin%
\definecolor{currentfill}{rgb}{0.686275,0.352941,0.313725}%
\pgfsetfillcolor{currentfill}%
\pgfsetfillopacity{0.300000}%
\pgfsetlinewidth{0.000000pt}%
\definecolor{currentstroke}{rgb}{0.000000,0.000000,0.000000}%
\pgfsetstrokecolor{currentstroke}%
\pgfsetstrokeopacity{0.300000}%
\pgfsetdash{}{0pt}%
\pgfpathmoveto{\pgfqpoint{3.816304in}{0.383578in}}%
\pgfpathlineto{\pgfqpoint{3.833893in}{0.383578in}}%
\pgfpathlineto{\pgfqpoint{3.833893in}{0.462149in}}%
\pgfpathlineto{\pgfqpoint{3.816304in}{0.462149in}}%
\pgfpathclose%
\pgfusepath{fill}%
\end{pgfscope}%
\begin{pgfscope}%
\pgfpathrectangle{\pgfqpoint{0.526905in}{0.383578in}}{\pgfqpoint{3.875000in}{2.310000in}}%
\pgfusepath{clip}%
\pgfsetbuttcap%
\pgfsetmiterjoin%
\definecolor{currentfill}{rgb}{0.686275,0.352941,0.313725}%
\pgfsetfillcolor{currentfill}%
\pgfsetfillopacity{0.300000}%
\pgfsetlinewidth{0.000000pt}%
\definecolor{currentstroke}{rgb}{0.000000,0.000000,0.000000}%
\pgfsetstrokecolor{currentstroke}%
\pgfsetstrokeopacity{0.300000}%
\pgfsetdash{}{0pt}%
\pgfpathmoveto{\pgfqpoint{3.833893in}{0.383578in}}%
\pgfpathlineto{\pgfqpoint{3.851482in}{0.383578in}}%
\pgfpathlineto{\pgfqpoint{3.851482in}{0.484598in}}%
\pgfpathlineto{\pgfqpoint{3.833893in}{0.484598in}}%
\pgfpathclose%
\pgfusepath{fill}%
\end{pgfscope}%
\begin{pgfscope}%
\pgfpathrectangle{\pgfqpoint{0.526905in}{0.383578in}}{\pgfqpoint{3.875000in}{2.310000in}}%
\pgfusepath{clip}%
\pgfsetbuttcap%
\pgfsetmiterjoin%
\definecolor{currentfill}{rgb}{0.686275,0.352941,0.313725}%
\pgfsetfillcolor{currentfill}%
\pgfsetfillopacity{0.300000}%
\pgfsetlinewidth{0.000000pt}%
\definecolor{currentstroke}{rgb}{0.000000,0.000000,0.000000}%
\pgfsetstrokecolor{currentstroke}%
\pgfsetstrokeopacity{0.300000}%
\pgfsetdash{}{0pt}%
\pgfpathmoveto{\pgfqpoint{3.851482in}{0.383578in}}%
\pgfpathlineto{\pgfqpoint{3.869071in}{0.383578in}}%
\pgfpathlineto{\pgfqpoint{3.869071in}{0.490211in}}%
\pgfpathlineto{\pgfqpoint{3.851482in}{0.490211in}}%
\pgfpathclose%
\pgfusepath{fill}%
\end{pgfscope}%
\begin{pgfscope}%
\pgfpathrectangle{\pgfqpoint{0.526905in}{0.383578in}}{\pgfqpoint{3.875000in}{2.310000in}}%
\pgfusepath{clip}%
\pgfsetbuttcap%
\pgfsetmiterjoin%
\definecolor{currentfill}{rgb}{0.686275,0.352941,0.313725}%
\pgfsetfillcolor{currentfill}%
\pgfsetfillopacity{0.300000}%
\pgfsetlinewidth{0.000000pt}%
\definecolor{currentstroke}{rgb}{0.000000,0.000000,0.000000}%
\pgfsetstrokecolor{currentstroke}%
\pgfsetstrokeopacity{0.300000}%
\pgfsetdash{}{0pt}%
\pgfpathmoveto{\pgfqpoint{3.869071in}{0.383578in}}%
\pgfpathlineto{\pgfqpoint{3.886660in}{0.383578in}}%
\pgfpathlineto{\pgfqpoint{3.886660in}{0.434088in}}%
\pgfpathlineto{\pgfqpoint{3.869071in}{0.434088in}}%
\pgfpathclose%
\pgfusepath{fill}%
\end{pgfscope}%
\begin{pgfscope}%
\pgfpathrectangle{\pgfqpoint{0.526905in}{0.383578in}}{\pgfqpoint{3.875000in}{2.310000in}}%
\pgfusepath{clip}%
\pgfsetbuttcap%
\pgfsetmiterjoin%
\definecolor{currentfill}{rgb}{0.686275,0.352941,0.313725}%
\pgfsetfillcolor{currentfill}%
\pgfsetfillopacity{0.300000}%
\pgfsetlinewidth{0.000000pt}%
\definecolor{currentstroke}{rgb}{0.000000,0.000000,0.000000}%
\pgfsetstrokecolor{currentstroke}%
\pgfsetstrokeopacity{0.300000}%
\pgfsetdash{}{0pt}%
\pgfpathmoveto{\pgfqpoint{3.886660in}{0.383578in}}%
\pgfpathlineto{\pgfqpoint{3.904249in}{0.383578in}}%
\pgfpathlineto{\pgfqpoint{3.904249in}{0.462149in}}%
\pgfpathlineto{\pgfqpoint{3.886660in}{0.462149in}}%
\pgfpathclose%
\pgfusepath{fill}%
\end{pgfscope}%
\begin{pgfscope}%
\pgfpathrectangle{\pgfqpoint{0.526905in}{0.383578in}}{\pgfqpoint{3.875000in}{2.310000in}}%
\pgfusepath{clip}%
\pgfsetbuttcap%
\pgfsetmiterjoin%
\definecolor{currentfill}{rgb}{0.686275,0.352941,0.313725}%
\pgfsetfillcolor{currentfill}%
\pgfsetfillopacity{0.300000}%
\pgfsetlinewidth{0.000000pt}%
\definecolor{currentstroke}{rgb}{0.000000,0.000000,0.000000}%
\pgfsetstrokecolor{currentstroke}%
\pgfsetstrokeopacity{0.300000}%
\pgfsetdash{}{0pt}%
\pgfpathmoveto{\pgfqpoint{3.904249in}{0.383578in}}%
\pgfpathlineto{\pgfqpoint{3.921838in}{0.383578in}}%
\pgfpathlineto{\pgfqpoint{3.921838in}{0.422864in}}%
\pgfpathlineto{\pgfqpoint{3.904249in}{0.422864in}}%
\pgfpathclose%
\pgfusepath{fill}%
\end{pgfscope}%
\begin{pgfscope}%
\pgfpathrectangle{\pgfqpoint{0.526905in}{0.383578in}}{\pgfqpoint{3.875000in}{2.310000in}}%
\pgfusepath{clip}%
\pgfsetbuttcap%
\pgfsetmiterjoin%
\definecolor{currentfill}{rgb}{0.686275,0.352941,0.313725}%
\pgfsetfillcolor{currentfill}%
\pgfsetfillopacity{0.300000}%
\pgfsetlinewidth{0.000000pt}%
\definecolor{currentstroke}{rgb}{0.000000,0.000000,0.000000}%
\pgfsetstrokecolor{currentstroke}%
\pgfsetstrokeopacity{0.300000}%
\pgfsetdash{}{0pt}%
\pgfpathmoveto{\pgfqpoint{3.921838in}{0.383578in}}%
\pgfpathlineto{\pgfqpoint{3.939427in}{0.383578in}}%
\pgfpathlineto{\pgfqpoint{3.939427in}{0.406027in}}%
\pgfpathlineto{\pgfqpoint{3.921838in}{0.406027in}}%
\pgfpathclose%
\pgfusepath{fill}%
\end{pgfscope}%
\begin{pgfscope}%
\pgfpathrectangle{\pgfqpoint{0.526905in}{0.383578in}}{\pgfqpoint{3.875000in}{2.310000in}}%
\pgfusepath{clip}%
\pgfsetbuttcap%
\pgfsetmiterjoin%
\definecolor{currentfill}{rgb}{0.686275,0.352941,0.313725}%
\pgfsetfillcolor{currentfill}%
\pgfsetfillopacity{0.300000}%
\pgfsetlinewidth{0.000000pt}%
\definecolor{currentstroke}{rgb}{0.000000,0.000000,0.000000}%
\pgfsetstrokecolor{currentstroke}%
\pgfsetstrokeopacity{0.300000}%
\pgfsetdash{}{0pt}%
\pgfpathmoveto{\pgfqpoint{3.939427in}{0.383578in}}%
\pgfpathlineto{\pgfqpoint{3.957016in}{0.383578in}}%
\pgfpathlineto{\pgfqpoint{3.957016in}{0.417252in}}%
\pgfpathlineto{\pgfqpoint{3.939427in}{0.417252in}}%
\pgfpathclose%
\pgfusepath{fill}%
\end{pgfscope}%
\begin{pgfscope}%
\pgfpathrectangle{\pgfqpoint{0.526905in}{0.383578in}}{\pgfqpoint{3.875000in}{2.310000in}}%
\pgfusepath{clip}%
\pgfsetbuttcap%
\pgfsetmiterjoin%
\definecolor{currentfill}{rgb}{0.686275,0.352941,0.313725}%
\pgfsetfillcolor{currentfill}%
\pgfsetfillopacity{0.300000}%
\pgfsetlinewidth{0.000000pt}%
\definecolor{currentstroke}{rgb}{0.000000,0.000000,0.000000}%
\pgfsetstrokecolor{currentstroke}%
\pgfsetstrokeopacity{0.300000}%
\pgfsetdash{}{0pt}%
\pgfpathmoveto{\pgfqpoint{3.957016in}{0.383578in}}%
\pgfpathlineto{\pgfqpoint{3.974605in}{0.383578in}}%
\pgfpathlineto{\pgfqpoint{3.974605in}{0.422864in}}%
\pgfpathlineto{\pgfqpoint{3.957016in}{0.422864in}}%
\pgfpathclose%
\pgfusepath{fill}%
\end{pgfscope}%
\begin{pgfscope}%
\pgfpathrectangle{\pgfqpoint{0.526905in}{0.383578in}}{\pgfqpoint{3.875000in}{2.310000in}}%
\pgfusepath{clip}%
\pgfsetbuttcap%
\pgfsetmiterjoin%
\definecolor{currentfill}{rgb}{0.686275,0.352941,0.313725}%
\pgfsetfillcolor{currentfill}%
\pgfsetfillopacity{0.300000}%
\pgfsetlinewidth{0.000000pt}%
\definecolor{currentstroke}{rgb}{0.000000,0.000000,0.000000}%
\pgfsetstrokecolor{currentstroke}%
\pgfsetstrokeopacity{0.300000}%
\pgfsetdash{}{0pt}%
\pgfpathmoveto{\pgfqpoint{3.974605in}{0.383578in}}%
\pgfpathlineto{\pgfqpoint{3.992194in}{0.383578in}}%
\pgfpathlineto{\pgfqpoint{3.992194in}{0.422864in}}%
\pgfpathlineto{\pgfqpoint{3.974605in}{0.422864in}}%
\pgfpathclose%
\pgfusepath{fill}%
\end{pgfscope}%
\begin{pgfscope}%
\pgfpathrectangle{\pgfqpoint{0.526905in}{0.383578in}}{\pgfqpoint{3.875000in}{2.310000in}}%
\pgfusepath{clip}%
\pgfsetbuttcap%
\pgfsetmiterjoin%
\definecolor{currentfill}{rgb}{0.686275,0.352941,0.313725}%
\pgfsetfillcolor{currentfill}%
\pgfsetfillopacity{0.300000}%
\pgfsetlinewidth{0.000000pt}%
\definecolor{currentstroke}{rgb}{0.000000,0.000000,0.000000}%
\pgfsetstrokecolor{currentstroke}%
\pgfsetstrokeopacity{0.300000}%
\pgfsetdash{}{0pt}%
\pgfpathmoveto{\pgfqpoint{3.992194in}{0.383578in}}%
\pgfpathlineto{\pgfqpoint{4.009783in}{0.383578in}}%
\pgfpathlineto{\pgfqpoint{4.009783in}{0.417252in}}%
\pgfpathlineto{\pgfqpoint{3.992194in}{0.417252in}}%
\pgfpathclose%
\pgfusepath{fill}%
\end{pgfscope}%
\begin{pgfscope}%
\pgfpathrectangle{\pgfqpoint{0.526905in}{0.383578in}}{\pgfqpoint{3.875000in}{2.310000in}}%
\pgfusepath{clip}%
\pgfsetbuttcap%
\pgfsetmiterjoin%
\definecolor{currentfill}{rgb}{0.686275,0.352941,0.313725}%
\pgfsetfillcolor{currentfill}%
\pgfsetfillopacity{0.300000}%
\pgfsetlinewidth{0.000000pt}%
\definecolor{currentstroke}{rgb}{0.000000,0.000000,0.000000}%
\pgfsetstrokecolor{currentstroke}%
\pgfsetstrokeopacity{0.300000}%
\pgfsetdash{}{0pt}%
\pgfpathmoveto{\pgfqpoint{4.009783in}{0.383578in}}%
\pgfpathlineto{\pgfqpoint{4.027372in}{0.383578in}}%
\pgfpathlineto{\pgfqpoint{4.027372in}{0.422864in}}%
\pgfpathlineto{\pgfqpoint{4.009783in}{0.422864in}}%
\pgfpathclose%
\pgfusepath{fill}%
\end{pgfscope}%
\begin{pgfscope}%
\pgfpathrectangle{\pgfqpoint{0.526905in}{0.383578in}}{\pgfqpoint{3.875000in}{2.310000in}}%
\pgfusepath{clip}%
\pgfsetbuttcap%
\pgfsetmiterjoin%
\definecolor{currentfill}{rgb}{0.686275,0.352941,0.313725}%
\pgfsetfillcolor{currentfill}%
\pgfsetfillopacity{0.300000}%
\pgfsetlinewidth{0.000000pt}%
\definecolor{currentstroke}{rgb}{0.000000,0.000000,0.000000}%
\pgfsetstrokecolor{currentstroke}%
\pgfsetstrokeopacity{0.300000}%
\pgfsetdash{}{0pt}%
\pgfpathmoveto{\pgfqpoint{4.027372in}{0.383578in}}%
\pgfpathlineto{\pgfqpoint{4.044961in}{0.383578in}}%
\pgfpathlineto{\pgfqpoint{4.044961in}{0.428476in}}%
\pgfpathlineto{\pgfqpoint{4.027372in}{0.428476in}}%
\pgfpathclose%
\pgfusepath{fill}%
\end{pgfscope}%
\begin{pgfscope}%
\pgfpathrectangle{\pgfqpoint{0.526905in}{0.383578in}}{\pgfqpoint{3.875000in}{2.310000in}}%
\pgfusepath{clip}%
\pgfsetbuttcap%
\pgfsetmiterjoin%
\definecolor{currentfill}{rgb}{0.686275,0.352941,0.313725}%
\pgfsetfillcolor{currentfill}%
\pgfsetfillopacity{0.300000}%
\pgfsetlinewidth{0.000000pt}%
\definecolor{currentstroke}{rgb}{0.000000,0.000000,0.000000}%
\pgfsetstrokecolor{currentstroke}%
\pgfsetstrokeopacity{0.300000}%
\pgfsetdash{}{0pt}%
\pgfpathmoveto{\pgfqpoint{4.044961in}{0.383578in}}%
\pgfpathlineto{\pgfqpoint{4.062550in}{0.383578in}}%
\pgfpathlineto{\pgfqpoint{4.062550in}{0.394803in}}%
\pgfpathlineto{\pgfqpoint{4.044961in}{0.394803in}}%
\pgfpathclose%
\pgfusepath{fill}%
\end{pgfscope}%
\begin{pgfscope}%
\pgfpathrectangle{\pgfqpoint{0.526905in}{0.383578in}}{\pgfqpoint{3.875000in}{2.310000in}}%
\pgfusepath{clip}%
\pgfsetbuttcap%
\pgfsetmiterjoin%
\definecolor{currentfill}{rgb}{0.686275,0.352941,0.313725}%
\pgfsetfillcolor{currentfill}%
\pgfsetfillopacity{0.300000}%
\pgfsetlinewidth{0.000000pt}%
\definecolor{currentstroke}{rgb}{0.000000,0.000000,0.000000}%
\pgfsetstrokecolor{currentstroke}%
\pgfsetstrokeopacity{0.300000}%
\pgfsetdash{}{0pt}%
\pgfpathmoveto{\pgfqpoint{4.062550in}{0.383578in}}%
\pgfpathlineto{\pgfqpoint{4.080139in}{0.383578in}}%
\pgfpathlineto{\pgfqpoint{4.080139in}{0.400415in}}%
\pgfpathlineto{\pgfqpoint{4.062550in}{0.400415in}}%
\pgfpathclose%
\pgfusepath{fill}%
\end{pgfscope}%
\begin{pgfscope}%
\pgfpathrectangle{\pgfqpoint{0.526905in}{0.383578in}}{\pgfqpoint{3.875000in}{2.310000in}}%
\pgfusepath{clip}%
\pgfsetbuttcap%
\pgfsetmiterjoin%
\definecolor{currentfill}{rgb}{0.686275,0.352941,0.313725}%
\pgfsetfillcolor{currentfill}%
\pgfsetfillopacity{0.300000}%
\pgfsetlinewidth{0.000000pt}%
\definecolor{currentstroke}{rgb}{0.000000,0.000000,0.000000}%
\pgfsetstrokecolor{currentstroke}%
\pgfsetstrokeopacity{0.300000}%
\pgfsetdash{}{0pt}%
\pgfpathmoveto{\pgfqpoint{4.080139in}{0.383578in}}%
\pgfpathlineto{\pgfqpoint{4.097728in}{0.383578in}}%
\pgfpathlineto{\pgfqpoint{4.097728in}{0.411639in}}%
\pgfpathlineto{\pgfqpoint{4.080139in}{0.411639in}}%
\pgfpathclose%
\pgfusepath{fill}%
\end{pgfscope}%
\begin{pgfscope}%
\pgfpathrectangle{\pgfqpoint{0.526905in}{0.383578in}}{\pgfqpoint{3.875000in}{2.310000in}}%
\pgfusepath{clip}%
\pgfsetbuttcap%
\pgfsetmiterjoin%
\definecolor{currentfill}{rgb}{0.686275,0.352941,0.313725}%
\pgfsetfillcolor{currentfill}%
\pgfsetfillopacity{0.300000}%
\pgfsetlinewidth{0.000000pt}%
\definecolor{currentstroke}{rgb}{0.000000,0.000000,0.000000}%
\pgfsetstrokecolor{currentstroke}%
\pgfsetstrokeopacity{0.300000}%
\pgfsetdash{}{0pt}%
\pgfpathmoveto{\pgfqpoint{4.097728in}{0.383578in}}%
\pgfpathlineto{\pgfqpoint{4.115317in}{0.383578in}}%
\pgfpathlineto{\pgfqpoint{4.115317in}{0.400415in}}%
\pgfpathlineto{\pgfqpoint{4.097728in}{0.400415in}}%
\pgfpathclose%
\pgfusepath{fill}%
\end{pgfscope}%
\begin{pgfscope}%
\pgfpathrectangle{\pgfqpoint{0.526905in}{0.383578in}}{\pgfqpoint{3.875000in}{2.310000in}}%
\pgfusepath{clip}%
\pgfsetbuttcap%
\pgfsetmiterjoin%
\definecolor{currentfill}{rgb}{0.686275,0.352941,0.313725}%
\pgfsetfillcolor{currentfill}%
\pgfsetfillopacity{0.300000}%
\pgfsetlinewidth{0.000000pt}%
\definecolor{currentstroke}{rgb}{0.000000,0.000000,0.000000}%
\pgfsetstrokecolor{currentstroke}%
\pgfsetstrokeopacity{0.300000}%
\pgfsetdash{}{0pt}%
\pgfpathmoveto{\pgfqpoint{4.115317in}{0.383578in}}%
\pgfpathlineto{\pgfqpoint{4.132907in}{0.383578in}}%
\pgfpathlineto{\pgfqpoint{4.132907in}{0.394803in}}%
\pgfpathlineto{\pgfqpoint{4.115317in}{0.394803in}}%
\pgfpathclose%
\pgfusepath{fill}%
\end{pgfscope}%
\begin{pgfscope}%
\pgfpathrectangle{\pgfqpoint{0.526905in}{0.383578in}}{\pgfqpoint{3.875000in}{2.310000in}}%
\pgfusepath{clip}%
\pgfsetbuttcap%
\pgfsetmiterjoin%
\definecolor{currentfill}{rgb}{0.686275,0.352941,0.313725}%
\pgfsetfillcolor{currentfill}%
\pgfsetfillopacity{0.300000}%
\pgfsetlinewidth{0.000000pt}%
\definecolor{currentstroke}{rgb}{0.000000,0.000000,0.000000}%
\pgfsetstrokecolor{currentstroke}%
\pgfsetstrokeopacity{0.300000}%
\pgfsetdash{}{0pt}%
\pgfpathmoveto{\pgfqpoint{4.132907in}{0.383578in}}%
\pgfpathlineto{\pgfqpoint{4.150496in}{0.383578in}}%
\pgfpathlineto{\pgfqpoint{4.150496in}{0.394803in}}%
\pgfpathlineto{\pgfqpoint{4.132907in}{0.394803in}}%
\pgfpathclose%
\pgfusepath{fill}%
\end{pgfscope}%
\begin{pgfscope}%
\pgfpathrectangle{\pgfqpoint{0.526905in}{0.383578in}}{\pgfqpoint{3.875000in}{2.310000in}}%
\pgfusepath{clip}%
\pgfsetbuttcap%
\pgfsetmiterjoin%
\definecolor{currentfill}{rgb}{0.686275,0.352941,0.313725}%
\pgfsetfillcolor{currentfill}%
\pgfsetfillopacity{0.300000}%
\pgfsetlinewidth{0.000000pt}%
\definecolor{currentstroke}{rgb}{0.000000,0.000000,0.000000}%
\pgfsetstrokecolor{currentstroke}%
\pgfsetstrokeopacity{0.300000}%
\pgfsetdash{}{0pt}%
\pgfpathmoveto{\pgfqpoint{4.150496in}{0.383578in}}%
\pgfpathlineto{\pgfqpoint{4.168085in}{0.383578in}}%
\pgfpathlineto{\pgfqpoint{4.168085in}{0.389190in}}%
\pgfpathlineto{\pgfqpoint{4.150496in}{0.389190in}}%
\pgfpathclose%
\pgfusepath{fill}%
\end{pgfscope}%
\begin{pgfscope}%
\pgfpathrectangle{\pgfqpoint{0.526905in}{0.383578in}}{\pgfqpoint{3.875000in}{2.310000in}}%
\pgfusepath{clip}%
\pgfsetbuttcap%
\pgfsetmiterjoin%
\definecolor{currentfill}{rgb}{0.686275,0.352941,0.313725}%
\pgfsetfillcolor{currentfill}%
\pgfsetfillopacity{0.300000}%
\pgfsetlinewidth{0.000000pt}%
\definecolor{currentstroke}{rgb}{0.000000,0.000000,0.000000}%
\pgfsetstrokecolor{currentstroke}%
\pgfsetstrokeopacity{0.300000}%
\pgfsetdash{}{0pt}%
\pgfpathmoveto{\pgfqpoint{4.168085in}{0.383578in}}%
\pgfpathlineto{\pgfqpoint{4.185674in}{0.383578in}}%
\pgfpathlineto{\pgfqpoint{4.185674in}{0.383578in}}%
\pgfpathlineto{\pgfqpoint{4.168085in}{0.383578in}}%
\pgfpathclose%
\pgfusepath{fill}%
\end{pgfscope}%
\begin{pgfscope}%
\pgfpathrectangle{\pgfqpoint{0.526905in}{0.383578in}}{\pgfqpoint{3.875000in}{2.310000in}}%
\pgfusepath{clip}%
\pgfsetbuttcap%
\pgfsetmiterjoin%
\definecolor{currentfill}{rgb}{0.686275,0.352941,0.313725}%
\pgfsetfillcolor{currentfill}%
\pgfsetfillopacity{0.300000}%
\pgfsetlinewidth{0.000000pt}%
\definecolor{currentstroke}{rgb}{0.000000,0.000000,0.000000}%
\pgfsetstrokecolor{currentstroke}%
\pgfsetstrokeopacity{0.300000}%
\pgfsetdash{}{0pt}%
\pgfpathmoveto{\pgfqpoint{4.185674in}{0.383578in}}%
\pgfpathlineto{\pgfqpoint{4.203263in}{0.383578in}}%
\pgfpathlineto{\pgfqpoint{4.203263in}{0.389190in}}%
\pgfpathlineto{\pgfqpoint{4.185674in}{0.389190in}}%
\pgfpathclose%
\pgfusepath{fill}%
\end{pgfscope}%
\begin{pgfscope}%
\pgfpathrectangle{\pgfqpoint{0.526905in}{0.383578in}}{\pgfqpoint{3.875000in}{2.310000in}}%
\pgfusepath{clip}%
\pgfsetbuttcap%
\pgfsetmiterjoin%
\definecolor{currentfill}{rgb}{0.686275,0.352941,0.313725}%
\pgfsetfillcolor{currentfill}%
\pgfsetfillopacity{0.300000}%
\pgfsetlinewidth{0.000000pt}%
\definecolor{currentstroke}{rgb}{0.000000,0.000000,0.000000}%
\pgfsetstrokecolor{currentstroke}%
\pgfsetstrokeopacity{0.300000}%
\pgfsetdash{}{0pt}%
\pgfpathmoveto{\pgfqpoint{4.203263in}{0.383578in}}%
\pgfpathlineto{\pgfqpoint{4.220852in}{0.383578in}}%
\pgfpathlineto{\pgfqpoint{4.220852in}{0.389190in}}%
\pgfpathlineto{\pgfqpoint{4.203263in}{0.389190in}}%
\pgfpathclose%
\pgfusepath{fill}%
\end{pgfscope}%
\begin{pgfscope}%
\pgfpathrectangle{\pgfqpoint{0.526905in}{0.383578in}}{\pgfqpoint{3.875000in}{2.310000in}}%
\pgfusepath{clip}%
\pgfsetbuttcap%
\pgfsetmiterjoin%
\definecolor{currentfill}{rgb}{0.333333,0.333333,0.333333}%
\pgfsetfillcolor{currentfill}%
\pgfsetlinewidth{0.000000pt}%
\definecolor{currentstroke}{rgb}{0.000000,0.000000,0.000000}%
\pgfsetstrokecolor{currentstroke}%
\pgfsetstrokeopacity{0.000000}%
\pgfsetdash{}{0pt}%
\pgfpathmoveto{\pgfqpoint{3.336483in}{0.383578in}}%
\pgfpathlineto{\pgfqpoint{3.363905in}{0.383578in}}%
\pgfpathlineto{\pgfqpoint{3.363905in}{0.995313in}}%
\pgfpathlineto{\pgfqpoint{3.336483in}{0.995313in}}%
\pgfpathclose%
\pgfusepath{fill}%
\end{pgfscope}%
\begin{pgfscope}%
\pgfpathrectangle{\pgfqpoint{0.526905in}{0.383578in}}{\pgfqpoint{3.875000in}{2.310000in}}%
\pgfusepath{clip}%
\pgfsetbuttcap%
\pgfsetmiterjoin%
\definecolor{currentfill}{rgb}{0.333333,0.333333,0.333333}%
\pgfsetfillcolor{currentfill}%
\pgfsetlinewidth{0.000000pt}%
\definecolor{currentstroke}{rgb}{0.000000,0.000000,0.000000}%
\pgfsetstrokecolor{currentstroke}%
\pgfsetstrokeopacity{0.000000}%
\pgfsetdash{}{0pt}%
\pgfpathmoveto{\pgfqpoint{3.354072in}{0.383578in}}%
\pgfpathlineto{\pgfqpoint{3.381494in}{0.383578in}}%
\pgfpathlineto{\pgfqpoint{3.381494in}{1.090721in}}%
\pgfpathlineto{\pgfqpoint{3.354072in}{1.090721in}}%
\pgfpathclose%
\pgfusepath{fill}%
\end{pgfscope}%
\begin{pgfscope}%
\pgfpathrectangle{\pgfqpoint{0.526905in}{0.383578in}}{\pgfqpoint{3.875000in}{2.310000in}}%
\pgfusepath{clip}%
\pgfsetbuttcap%
\pgfsetmiterjoin%
\definecolor{currentfill}{rgb}{0.333333,0.333333,0.333333}%
\pgfsetfillcolor{currentfill}%
\pgfsetlinewidth{0.000000pt}%
\definecolor{currentstroke}{rgb}{0.000000,0.000000,0.000000}%
\pgfsetstrokecolor{currentstroke}%
\pgfsetstrokeopacity{0.000000}%
\pgfsetdash{}{0pt}%
\pgfpathmoveto{\pgfqpoint{3.371661in}{0.383578in}}%
\pgfpathlineto{\pgfqpoint{3.399083in}{0.383578in}}%
\pgfpathlineto{\pgfqpoint{3.399083in}{1.045823in}}%
\pgfpathlineto{\pgfqpoint{3.371661in}{1.045823in}}%
\pgfpathclose%
\pgfusepath{fill}%
\end{pgfscope}%
\begin{pgfscope}%
\pgfpathrectangle{\pgfqpoint{0.526905in}{0.383578in}}{\pgfqpoint{3.875000in}{2.310000in}}%
\pgfusepath{clip}%
\pgfsetbuttcap%
\pgfsetmiterjoin%
\definecolor{currentfill}{rgb}{0.333333,0.333333,0.333333}%
\pgfsetfillcolor{currentfill}%
\pgfsetlinewidth{0.000000pt}%
\definecolor{currentstroke}{rgb}{0.000000,0.000000,0.000000}%
\pgfsetstrokecolor{currentstroke}%
\pgfsetstrokeopacity{0.000000}%
\pgfsetdash{}{0pt}%
\pgfpathmoveto{\pgfqpoint{3.389250in}{0.383578in}}%
\pgfpathlineto{\pgfqpoint{3.416672in}{0.383578in}}%
\pgfpathlineto{\pgfqpoint{3.416672in}{1.017762in}}%
\pgfpathlineto{\pgfqpoint{3.389250in}{1.017762in}}%
\pgfpathclose%
\pgfusepath{fill}%
\end{pgfscope}%
\begin{pgfscope}%
\pgfpathrectangle{\pgfqpoint{0.526905in}{0.383578in}}{\pgfqpoint{3.875000in}{2.310000in}}%
\pgfusepath{clip}%
\pgfsetbuttcap%
\pgfsetmiterjoin%
\definecolor{currentfill}{rgb}{0.333333,0.333333,0.333333}%
\pgfsetfillcolor{currentfill}%
\pgfsetlinewidth{0.000000pt}%
\definecolor{currentstroke}{rgb}{0.000000,0.000000,0.000000}%
\pgfsetstrokecolor{currentstroke}%
\pgfsetstrokeopacity{0.000000}%
\pgfsetdash{}{0pt}%
\pgfpathmoveto{\pgfqpoint{3.406839in}{0.383578in}}%
\pgfpathlineto{\pgfqpoint{3.434261in}{0.383578in}}%
\pgfpathlineto{\pgfqpoint{3.434261in}{1.113170in}}%
\pgfpathlineto{\pgfqpoint{3.406839in}{1.113170in}}%
\pgfpathclose%
\pgfusepath{fill}%
\end{pgfscope}%
\begin{pgfscope}%
\pgfpathrectangle{\pgfqpoint{0.526905in}{0.383578in}}{\pgfqpoint{3.875000in}{2.310000in}}%
\pgfusepath{clip}%
\pgfsetbuttcap%
\pgfsetmiterjoin%
\definecolor{currentfill}{rgb}{0.333333,0.333333,0.333333}%
\pgfsetfillcolor{currentfill}%
\pgfsetlinewidth{0.000000pt}%
\definecolor{currentstroke}{rgb}{0.000000,0.000000,0.000000}%
\pgfsetstrokecolor{currentstroke}%
\pgfsetstrokeopacity{0.000000}%
\pgfsetdash{}{0pt}%
\pgfpathmoveto{\pgfqpoint{3.424428in}{0.383578in}}%
\pgfpathlineto{\pgfqpoint{3.451850in}{0.383578in}}%
\pgfpathlineto{\pgfqpoint{3.451850in}{1.040211in}}%
\pgfpathlineto{\pgfqpoint{3.424428in}{1.040211in}}%
\pgfpathclose%
\pgfusepath{fill}%
\end{pgfscope}%
\begin{pgfscope}%
\pgfpathrectangle{\pgfqpoint{0.526905in}{0.383578in}}{\pgfqpoint{3.875000in}{2.310000in}}%
\pgfusepath{clip}%
\pgfsetbuttcap%
\pgfsetmiterjoin%
\definecolor{currentfill}{rgb}{0.333333,0.333333,0.333333}%
\pgfsetfillcolor{currentfill}%
\pgfsetlinewidth{0.000000pt}%
\definecolor{currentstroke}{rgb}{0.000000,0.000000,0.000000}%
\pgfsetstrokecolor{currentstroke}%
\pgfsetstrokeopacity{0.000000}%
\pgfsetdash{}{0pt}%
\pgfpathmoveto{\pgfqpoint{3.442017in}{0.383578in}}%
\pgfpathlineto{\pgfqpoint{3.469439in}{0.383578in}}%
\pgfpathlineto{\pgfqpoint{3.469439in}{0.989701in}}%
\pgfpathlineto{\pgfqpoint{3.442017in}{0.989701in}}%
\pgfpathclose%
\pgfusepath{fill}%
\end{pgfscope}%
\begin{pgfscope}%
\pgfpathrectangle{\pgfqpoint{0.526905in}{0.383578in}}{\pgfqpoint{3.875000in}{2.310000in}}%
\pgfusepath{clip}%
\pgfsetbuttcap%
\pgfsetmiterjoin%
\definecolor{currentfill}{rgb}{0.333333,0.333333,0.333333}%
\pgfsetfillcolor{currentfill}%
\pgfsetlinewidth{0.000000pt}%
\definecolor{currentstroke}{rgb}{0.000000,0.000000,0.000000}%
\pgfsetstrokecolor{currentstroke}%
\pgfsetstrokeopacity{0.000000}%
\pgfsetdash{}{0pt}%
\pgfpathmoveto{\pgfqpoint{3.459606in}{0.383578in}}%
\pgfpathlineto{\pgfqpoint{3.487028in}{0.383578in}}%
\pgfpathlineto{\pgfqpoint{3.487028in}{0.984088in}}%
\pgfpathlineto{\pgfqpoint{3.459606in}{0.984088in}}%
\pgfpathclose%
\pgfusepath{fill}%
\end{pgfscope}%
\begin{pgfscope}%
\pgfpathrectangle{\pgfqpoint{0.526905in}{0.383578in}}{\pgfqpoint{3.875000in}{2.310000in}}%
\pgfusepath{clip}%
\pgfsetbuttcap%
\pgfsetmiterjoin%
\definecolor{currentfill}{rgb}{0.333333,0.333333,0.333333}%
\pgfsetfillcolor{currentfill}%
\pgfsetlinewidth{0.000000pt}%
\definecolor{currentstroke}{rgb}{0.000000,0.000000,0.000000}%
\pgfsetstrokecolor{currentstroke}%
\pgfsetstrokeopacity{0.000000}%
\pgfsetdash{}{0pt}%
\pgfpathmoveto{\pgfqpoint{3.477195in}{0.383578in}}%
\pgfpathlineto{\pgfqpoint{3.504617in}{0.383578in}}%
\pgfpathlineto{\pgfqpoint{3.504617in}{0.888680in}}%
\pgfpathlineto{\pgfqpoint{3.477195in}{0.888680in}}%
\pgfpathclose%
\pgfusepath{fill}%
\end{pgfscope}%
\begin{pgfscope}%
\pgfpathrectangle{\pgfqpoint{0.526905in}{0.383578in}}{\pgfqpoint{3.875000in}{2.310000in}}%
\pgfusepath{clip}%
\pgfsetbuttcap%
\pgfsetmiterjoin%
\definecolor{currentfill}{rgb}{0.333333,0.333333,0.333333}%
\pgfsetfillcolor{currentfill}%
\pgfsetlinewidth{0.000000pt}%
\definecolor{currentstroke}{rgb}{0.000000,0.000000,0.000000}%
\pgfsetstrokecolor{currentstroke}%
\pgfsetstrokeopacity{0.000000}%
\pgfsetdash{}{0pt}%
\pgfpathmoveto{\pgfqpoint{3.494784in}{0.383578in}}%
\pgfpathlineto{\pgfqpoint{3.522206in}{0.383578in}}%
\pgfpathlineto{\pgfqpoint{3.522206in}{0.883068in}}%
\pgfpathlineto{\pgfqpoint{3.494784in}{0.883068in}}%
\pgfpathclose%
\pgfusepath{fill}%
\end{pgfscope}%
\begin{pgfscope}%
\pgfpathrectangle{\pgfqpoint{0.526905in}{0.383578in}}{\pgfqpoint{3.875000in}{2.310000in}}%
\pgfusepath{clip}%
\pgfsetbuttcap%
\pgfsetmiterjoin%
\definecolor{currentfill}{rgb}{0.333333,0.333333,0.333333}%
\pgfsetfillcolor{currentfill}%
\pgfsetlinewidth{0.000000pt}%
\definecolor{currentstroke}{rgb}{0.000000,0.000000,0.000000}%
\pgfsetstrokecolor{currentstroke}%
\pgfsetstrokeopacity{0.000000}%
\pgfsetdash{}{0pt}%
\pgfpathmoveto{\pgfqpoint{3.512373in}{0.383578in}}%
\pgfpathlineto{\pgfqpoint{3.539795in}{0.383578in}}%
\pgfpathlineto{\pgfqpoint{3.539795in}{0.838170in}}%
\pgfpathlineto{\pgfqpoint{3.512373in}{0.838170in}}%
\pgfpathclose%
\pgfusepath{fill}%
\end{pgfscope}%
\begin{pgfscope}%
\pgfpathrectangle{\pgfqpoint{0.526905in}{0.383578in}}{\pgfqpoint{3.875000in}{2.310000in}}%
\pgfusepath{clip}%
\pgfsetbuttcap%
\pgfsetmiterjoin%
\definecolor{currentfill}{rgb}{0.333333,0.333333,0.333333}%
\pgfsetfillcolor{currentfill}%
\pgfsetlinewidth{0.000000pt}%
\definecolor{currentstroke}{rgb}{0.000000,0.000000,0.000000}%
\pgfsetstrokecolor{currentstroke}%
\pgfsetstrokeopacity{0.000000}%
\pgfsetdash{}{0pt}%
\pgfpathmoveto{\pgfqpoint{3.529962in}{0.383578in}}%
\pgfpathlineto{\pgfqpoint{3.557384in}{0.383578in}}%
\pgfpathlineto{\pgfqpoint{3.557384in}{0.748374in}}%
\pgfpathlineto{\pgfqpoint{3.529962in}{0.748374in}}%
\pgfpathclose%
\pgfusepath{fill}%
\end{pgfscope}%
\begin{pgfscope}%
\pgfpathrectangle{\pgfqpoint{0.526905in}{0.383578in}}{\pgfqpoint{3.875000in}{2.310000in}}%
\pgfusepath{clip}%
\pgfsetbuttcap%
\pgfsetmiterjoin%
\definecolor{currentfill}{rgb}{0.333333,0.333333,0.333333}%
\pgfsetfillcolor{currentfill}%
\pgfsetlinewidth{0.000000pt}%
\definecolor{currentstroke}{rgb}{0.000000,0.000000,0.000000}%
\pgfsetstrokecolor{currentstroke}%
\pgfsetstrokeopacity{0.000000}%
\pgfsetdash{}{0pt}%
\pgfpathmoveto{\pgfqpoint{3.547551in}{0.383578in}}%
\pgfpathlineto{\pgfqpoint{3.574973in}{0.383578in}}%
\pgfpathlineto{\pgfqpoint{3.574973in}{0.770823in}}%
\pgfpathlineto{\pgfqpoint{3.547551in}{0.770823in}}%
\pgfpathclose%
\pgfusepath{fill}%
\end{pgfscope}%
\begin{pgfscope}%
\pgfpathrectangle{\pgfqpoint{0.526905in}{0.383578in}}{\pgfqpoint{3.875000in}{2.310000in}}%
\pgfusepath{clip}%
\pgfsetbuttcap%
\pgfsetmiterjoin%
\definecolor{currentfill}{rgb}{0.333333,0.333333,0.333333}%
\pgfsetfillcolor{currentfill}%
\pgfsetlinewidth{0.000000pt}%
\definecolor{currentstroke}{rgb}{0.000000,0.000000,0.000000}%
\pgfsetstrokecolor{currentstroke}%
\pgfsetstrokeopacity{0.000000}%
\pgfsetdash{}{0pt}%
\pgfpathmoveto{\pgfqpoint{3.565140in}{0.383578in}}%
\pgfpathlineto{\pgfqpoint{3.592562in}{0.383578in}}%
\pgfpathlineto{\pgfqpoint{3.592562in}{0.759598in}}%
\pgfpathlineto{\pgfqpoint{3.565140in}{0.759598in}}%
\pgfpathclose%
\pgfusepath{fill}%
\end{pgfscope}%
\begin{pgfscope}%
\pgfpathrectangle{\pgfqpoint{0.526905in}{0.383578in}}{\pgfqpoint{3.875000in}{2.310000in}}%
\pgfusepath{clip}%
\pgfsetbuttcap%
\pgfsetmiterjoin%
\definecolor{currentfill}{rgb}{0.333333,0.333333,0.333333}%
\pgfsetfillcolor{currentfill}%
\pgfsetlinewidth{0.000000pt}%
\definecolor{currentstroke}{rgb}{0.000000,0.000000,0.000000}%
\pgfsetstrokecolor{currentstroke}%
\pgfsetstrokeopacity{0.000000}%
\pgfsetdash{}{0pt}%
\pgfpathmoveto{\pgfqpoint{3.582729in}{0.383578in}}%
\pgfpathlineto{\pgfqpoint{3.610151in}{0.383578in}}%
\pgfpathlineto{\pgfqpoint{3.610151in}{0.709088in}}%
\pgfpathlineto{\pgfqpoint{3.582729in}{0.709088in}}%
\pgfpathclose%
\pgfusepath{fill}%
\end{pgfscope}%
\begin{pgfscope}%
\pgfpathrectangle{\pgfqpoint{0.526905in}{0.383578in}}{\pgfqpoint{3.875000in}{2.310000in}}%
\pgfusepath{clip}%
\pgfsetbuttcap%
\pgfsetmiterjoin%
\definecolor{currentfill}{rgb}{0.333333,0.333333,0.333333}%
\pgfsetfillcolor{currentfill}%
\pgfsetlinewidth{0.000000pt}%
\definecolor{currentstroke}{rgb}{0.000000,0.000000,0.000000}%
\pgfsetstrokecolor{currentstroke}%
\pgfsetstrokeopacity{0.000000}%
\pgfsetdash{}{0pt}%
\pgfpathmoveto{\pgfqpoint{3.600318in}{0.383578in}}%
\pgfpathlineto{\pgfqpoint{3.627741in}{0.383578in}}%
\pgfpathlineto{\pgfqpoint{3.627741in}{0.720313in}}%
\pgfpathlineto{\pgfqpoint{3.600318in}{0.720313in}}%
\pgfpathclose%
\pgfusepath{fill}%
\end{pgfscope}%
\begin{pgfscope}%
\pgfpathrectangle{\pgfqpoint{0.526905in}{0.383578in}}{\pgfqpoint{3.875000in}{2.310000in}}%
\pgfusepath{clip}%
\pgfsetbuttcap%
\pgfsetmiterjoin%
\definecolor{currentfill}{rgb}{0.333333,0.333333,0.333333}%
\pgfsetfillcolor{currentfill}%
\pgfsetlinewidth{0.000000pt}%
\definecolor{currentstroke}{rgb}{0.000000,0.000000,0.000000}%
\pgfsetstrokecolor{currentstroke}%
\pgfsetstrokeopacity{0.000000}%
\pgfsetdash{}{0pt}%
\pgfpathmoveto{\pgfqpoint{3.617907in}{0.383578in}}%
\pgfpathlineto{\pgfqpoint{3.645330in}{0.383578in}}%
\pgfpathlineto{\pgfqpoint{3.645330in}{0.608068in}}%
\pgfpathlineto{\pgfqpoint{3.617907in}{0.608068in}}%
\pgfpathclose%
\pgfusepath{fill}%
\end{pgfscope}%
\begin{pgfscope}%
\pgfpathrectangle{\pgfqpoint{0.526905in}{0.383578in}}{\pgfqpoint{3.875000in}{2.310000in}}%
\pgfusepath{clip}%
\pgfsetbuttcap%
\pgfsetmiterjoin%
\definecolor{currentfill}{rgb}{0.333333,0.333333,0.333333}%
\pgfsetfillcolor{currentfill}%
\pgfsetlinewidth{0.000000pt}%
\definecolor{currentstroke}{rgb}{0.000000,0.000000,0.000000}%
\pgfsetstrokecolor{currentstroke}%
\pgfsetstrokeopacity{0.000000}%
\pgfsetdash{}{0pt}%
\pgfpathmoveto{\pgfqpoint{3.635496in}{0.383578in}}%
\pgfpathlineto{\pgfqpoint{3.662919in}{0.383578in}}%
\pgfpathlineto{\pgfqpoint{3.662919in}{0.697864in}}%
\pgfpathlineto{\pgfqpoint{3.635496in}{0.697864in}}%
\pgfpathclose%
\pgfusepath{fill}%
\end{pgfscope}%
\begin{pgfscope}%
\pgfpathrectangle{\pgfqpoint{0.526905in}{0.383578in}}{\pgfqpoint{3.875000in}{2.310000in}}%
\pgfusepath{clip}%
\pgfsetbuttcap%
\pgfsetmiterjoin%
\definecolor{currentfill}{rgb}{0.333333,0.333333,0.333333}%
\pgfsetfillcolor{currentfill}%
\pgfsetlinewidth{0.000000pt}%
\definecolor{currentstroke}{rgb}{0.000000,0.000000,0.000000}%
\pgfsetstrokecolor{currentstroke}%
\pgfsetstrokeopacity{0.000000}%
\pgfsetdash{}{0pt}%
\pgfpathmoveto{\pgfqpoint{3.653086in}{0.383578in}}%
\pgfpathlineto{\pgfqpoint{3.680508in}{0.383578in}}%
\pgfpathlineto{\pgfqpoint{3.680508in}{0.602456in}}%
\pgfpathlineto{\pgfqpoint{3.653086in}{0.602456in}}%
\pgfpathclose%
\pgfusepath{fill}%
\end{pgfscope}%
\begin{pgfscope}%
\pgfpathrectangle{\pgfqpoint{0.526905in}{0.383578in}}{\pgfqpoint{3.875000in}{2.310000in}}%
\pgfusepath{clip}%
\pgfsetbuttcap%
\pgfsetmiterjoin%
\definecolor{currentfill}{rgb}{0.333333,0.333333,0.333333}%
\pgfsetfillcolor{currentfill}%
\pgfsetlinewidth{0.000000pt}%
\definecolor{currentstroke}{rgb}{0.000000,0.000000,0.000000}%
\pgfsetstrokecolor{currentstroke}%
\pgfsetstrokeopacity{0.000000}%
\pgfsetdash{}{0pt}%
\pgfpathmoveto{\pgfqpoint{3.670675in}{0.383578in}}%
\pgfpathlineto{\pgfqpoint{3.698097in}{0.383578in}}%
\pgfpathlineto{\pgfqpoint{3.698097in}{0.613680in}}%
\pgfpathlineto{\pgfqpoint{3.670675in}{0.613680in}}%
\pgfpathclose%
\pgfusepath{fill}%
\end{pgfscope}%
\begin{pgfscope}%
\pgfpathrectangle{\pgfqpoint{0.526905in}{0.383578in}}{\pgfqpoint{3.875000in}{2.310000in}}%
\pgfusepath{clip}%
\pgfsetbuttcap%
\pgfsetmiterjoin%
\definecolor{currentfill}{rgb}{0.333333,0.333333,0.333333}%
\pgfsetfillcolor{currentfill}%
\pgfsetlinewidth{0.000000pt}%
\definecolor{currentstroke}{rgb}{0.000000,0.000000,0.000000}%
\pgfsetstrokecolor{currentstroke}%
\pgfsetstrokeopacity{0.000000}%
\pgfsetdash{}{0pt}%
\pgfpathmoveto{\pgfqpoint{3.688264in}{0.383578in}}%
\pgfpathlineto{\pgfqpoint{3.715686in}{0.383578in}}%
\pgfpathlineto{\pgfqpoint{3.715686in}{0.529496in}}%
\pgfpathlineto{\pgfqpoint{3.688264in}{0.529496in}}%
\pgfpathclose%
\pgfusepath{fill}%
\end{pgfscope}%
\begin{pgfscope}%
\pgfpathrectangle{\pgfqpoint{0.526905in}{0.383578in}}{\pgfqpoint{3.875000in}{2.310000in}}%
\pgfusepath{clip}%
\pgfsetbuttcap%
\pgfsetmiterjoin%
\definecolor{currentfill}{rgb}{0.333333,0.333333,0.333333}%
\pgfsetfillcolor{currentfill}%
\pgfsetlinewidth{0.000000pt}%
\definecolor{currentstroke}{rgb}{0.000000,0.000000,0.000000}%
\pgfsetstrokecolor{currentstroke}%
\pgfsetstrokeopacity{0.000000}%
\pgfsetdash{}{0pt}%
\pgfpathmoveto{\pgfqpoint{3.705853in}{0.383578in}}%
\pgfpathlineto{\pgfqpoint{3.733275in}{0.383578in}}%
\pgfpathlineto{\pgfqpoint{3.733275in}{0.551945in}}%
\pgfpathlineto{\pgfqpoint{3.705853in}{0.551945in}}%
\pgfpathclose%
\pgfusepath{fill}%
\end{pgfscope}%
\begin{pgfscope}%
\pgfpathrectangle{\pgfqpoint{0.526905in}{0.383578in}}{\pgfqpoint{3.875000in}{2.310000in}}%
\pgfusepath{clip}%
\pgfsetbuttcap%
\pgfsetmiterjoin%
\definecolor{currentfill}{rgb}{0.333333,0.333333,0.333333}%
\pgfsetfillcolor{currentfill}%
\pgfsetlinewidth{0.000000pt}%
\definecolor{currentstroke}{rgb}{0.000000,0.000000,0.000000}%
\pgfsetstrokecolor{currentstroke}%
\pgfsetstrokeopacity{0.000000}%
\pgfsetdash{}{0pt}%
\pgfpathmoveto{\pgfqpoint{3.723442in}{0.383578in}}%
\pgfpathlineto{\pgfqpoint{3.750864in}{0.383578in}}%
\pgfpathlineto{\pgfqpoint{3.750864in}{0.563170in}}%
\pgfpathlineto{\pgfqpoint{3.723442in}{0.563170in}}%
\pgfpathclose%
\pgfusepath{fill}%
\end{pgfscope}%
\begin{pgfscope}%
\pgfpathrectangle{\pgfqpoint{0.526905in}{0.383578in}}{\pgfqpoint{3.875000in}{2.310000in}}%
\pgfusepath{clip}%
\pgfsetbuttcap%
\pgfsetmiterjoin%
\definecolor{currentfill}{rgb}{0.333333,0.333333,0.333333}%
\pgfsetfillcolor{currentfill}%
\pgfsetlinewidth{0.000000pt}%
\definecolor{currentstroke}{rgb}{0.000000,0.000000,0.000000}%
\pgfsetstrokecolor{currentstroke}%
\pgfsetstrokeopacity{0.000000}%
\pgfsetdash{}{0pt}%
\pgfpathmoveto{\pgfqpoint{3.741031in}{0.383578in}}%
\pgfpathlineto{\pgfqpoint{3.768453in}{0.383578in}}%
\pgfpathlineto{\pgfqpoint{3.768453in}{0.518272in}}%
\pgfpathlineto{\pgfqpoint{3.741031in}{0.518272in}}%
\pgfpathclose%
\pgfusepath{fill}%
\end{pgfscope}%
\begin{pgfscope}%
\pgfpathrectangle{\pgfqpoint{0.526905in}{0.383578in}}{\pgfqpoint{3.875000in}{2.310000in}}%
\pgfusepath{clip}%
\pgfsetbuttcap%
\pgfsetmiterjoin%
\definecolor{currentfill}{rgb}{0.333333,0.333333,0.333333}%
\pgfsetfillcolor{currentfill}%
\pgfsetlinewidth{0.000000pt}%
\definecolor{currentstroke}{rgb}{0.000000,0.000000,0.000000}%
\pgfsetstrokecolor{currentstroke}%
\pgfsetstrokeopacity{0.000000}%
\pgfsetdash{}{0pt}%
\pgfpathmoveto{\pgfqpoint{3.758620in}{0.383578in}}%
\pgfpathlineto{\pgfqpoint{3.786042in}{0.383578in}}%
\pgfpathlineto{\pgfqpoint{3.786042in}{0.563170in}}%
\pgfpathlineto{\pgfqpoint{3.758620in}{0.563170in}}%
\pgfpathclose%
\pgfusepath{fill}%
\end{pgfscope}%
\begin{pgfscope}%
\pgfpathrectangle{\pgfqpoint{0.526905in}{0.383578in}}{\pgfqpoint{3.875000in}{2.310000in}}%
\pgfusepath{clip}%
\pgfsetbuttcap%
\pgfsetmiterjoin%
\definecolor{currentfill}{rgb}{0.333333,0.333333,0.333333}%
\pgfsetfillcolor{currentfill}%
\pgfsetlinewidth{0.000000pt}%
\definecolor{currentstroke}{rgb}{0.000000,0.000000,0.000000}%
\pgfsetstrokecolor{currentstroke}%
\pgfsetstrokeopacity{0.000000}%
\pgfsetdash{}{0pt}%
\pgfpathmoveto{\pgfqpoint{3.776209in}{0.383578in}}%
\pgfpathlineto{\pgfqpoint{3.803631in}{0.383578in}}%
\pgfpathlineto{\pgfqpoint{3.803631in}{0.495823in}}%
\pgfpathlineto{\pgfqpoint{3.776209in}{0.495823in}}%
\pgfpathclose%
\pgfusepath{fill}%
\end{pgfscope}%
\begin{pgfscope}%
\pgfpathrectangle{\pgfqpoint{0.526905in}{0.383578in}}{\pgfqpoint{3.875000in}{2.310000in}}%
\pgfusepath{clip}%
\pgfsetbuttcap%
\pgfsetmiterjoin%
\definecolor{currentfill}{rgb}{0.333333,0.333333,0.333333}%
\pgfsetfillcolor{currentfill}%
\pgfsetlinewidth{0.000000pt}%
\definecolor{currentstroke}{rgb}{0.000000,0.000000,0.000000}%
\pgfsetstrokecolor{currentstroke}%
\pgfsetstrokeopacity{0.000000}%
\pgfsetdash{}{0pt}%
\pgfpathmoveto{\pgfqpoint{3.793798in}{0.383578in}}%
\pgfpathlineto{\pgfqpoint{3.821220in}{0.383578in}}%
\pgfpathlineto{\pgfqpoint{3.821220in}{0.540721in}}%
\pgfpathlineto{\pgfqpoint{3.793798in}{0.540721in}}%
\pgfpathclose%
\pgfusepath{fill}%
\end{pgfscope}%
\begin{pgfscope}%
\pgfpathrectangle{\pgfqpoint{0.526905in}{0.383578in}}{\pgfqpoint{3.875000in}{2.310000in}}%
\pgfusepath{clip}%
\pgfsetbuttcap%
\pgfsetmiterjoin%
\definecolor{currentfill}{rgb}{0.333333,0.333333,0.333333}%
\pgfsetfillcolor{currentfill}%
\pgfsetlinewidth{0.000000pt}%
\definecolor{currentstroke}{rgb}{0.000000,0.000000,0.000000}%
\pgfsetstrokecolor{currentstroke}%
\pgfsetstrokeopacity{0.000000}%
\pgfsetdash{}{0pt}%
\pgfpathmoveto{\pgfqpoint{3.811387in}{0.383578in}}%
\pgfpathlineto{\pgfqpoint{3.838809in}{0.383578in}}%
\pgfpathlineto{\pgfqpoint{3.838809in}{0.462149in}}%
\pgfpathlineto{\pgfqpoint{3.811387in}{0.462149in}}%
\pgfpathclose%
\pgfusepath{fill}%
\end{pgfscope}%
\begin{pgfscope}%
\pgfpathrectangle{\pgfqpoint{0.526905in}{0.383578in}}{\pgfqpoint{3.875000in}{2.310000in}}%
\pgfusepath{clip}%
\pgfsetbuttcap%
\pgfsetmiterjoin%
\definecolor{currentfill}{rgb}{0.333333,0.333333,0.333333}%
\pgfsetfillcolor{currentfill}%
\pgfsetlinewidth{0.000000pt}%
\definecolor{currentstroke}{rgb}{0.000000,0.000000,0.000000}%
\pgfsetstrokecolor{currentstroke}%
\pgfsetstrokeopacity{0.000000}%
\pgfsetdash{}{0pt}%
\pgfpathmoveto{\pgfqpoint{3.828976in}{0.383578in}}%
\pgfpathlineto{\pgfqpoint{3.856398in}{0.383578in}}%
\pgfpathlineto{\pgfqpoint{3.856398in}{0.484598in}}%
\pgfpathlineto{\pgfqpoint{3.828976in}{0.484598in}}%
\pgfpathclose%
\pgfusepath{fill}%
\end{pgfscope}%
\begin{pgfscope}%
\pgfpathrectangle{\pgfqpoint{0.526905in}{0.383578in}}{\pgfqpoint{3.875000in}{2.310000in}}%
\pgfusepath{clip}%
\pgfsetbuttcap%
\pgfsetmiterjoin%
\definecolor{currentfill}{rgb}{0.333333,0.333333,0.333333}%
\pgfsetfillcolor{currentfill}%
\pgfsetlinewidth{0.000000pt}%
\definecolor{currentstroke}{rgb}{0.000000,0.000000,0.000000}%
\pgfsetstrokecolor{currentstroke}%
\pgfsetstrokeopacity{0.000000}%
\pgfsetdash{}{0pt}%
\pgfpathmoveto{\pgfqpoint{3.846565in}{0.383578in}}%
\pgfpathlineto{\pgfqpoint{3.873987in}{0.383578in}}%
\pgfpathlineto{\pgfqpoint{3.873987in}{0.490211in}}%
\pgfpathlineto{\pgfqpoint{3.846565in}{0.490211in}}%
\pgfpathclose%
\pgfusepath{fill}%
\end{pgfscope}%
\begin{pgfscope}%
\pgfpathrectangle{\pgfqpoint{0.526905in}{0.383578in}}{\pgfqpoint{3.875000in}{2.310000in}}%
\pgfusepath{clip}%
\pgfsetbuttcap%
\pgfsetmiterjoin%
\definecolor{currentfill}{rgb}{0.333333,0.333333,0.333333}%
\pgfsetfillcolor{currentfill}%
\pgfsetlinewidth{0.000000pt}%
\definecolor{currentstroke}{rgb}{0.000000,0.000000,0.000000}%
\pgfsetstrokecolor{currentstroke}%
\pgfsetstrokeopacity{0.000000}%
\pgfsetdash{}{0pt}%
\pgfpathmoveto{\pgfqpoint{3.864154in}{0.383578in}}%
\pgfpathlineto{\pgfqpoint{3.891576in}{0.383578in}}%
\pgfpathlineto{\pgfqpoint{3.891576in}{0.434088in}}%
\pgfpathlineto{\pgfqpoint{3.864154in}{0.434088in}}%
\pgfpathclose%
\pgfusepath{fill}%
\end{pgfscope}%
\begin{pgfscope}%
\pgfpathrectangle{\pgfqpoint{0.526905in}{0.383578in}}{\pgfqpoint{3.875000in}{2.310000in}}%
\pgfusepath{clip}%
\pgfsetbuttcap%
\pgfsetmiterjoin%
\definecolor{currentfill}{rgb}{0.333333,0.333333,0.333333}%
\pgfsetfillcolor{currentfill}%
\pgfsetlinewidth{0.000000pt}%
\definecolor{currentstroke}{rgb}{0.000000,0.000000,0.000000}%
\pgfsetstrokecolor{currentstroke}%
\pgfsetstrokeopacity{0.000000}%
\pgfsetdash{}{0pt}%
\pgfpathmoveto{\pgfqpoint{3.881743in}{0.383578in}}%
\pgfpathlineto{\pgfqpoint{3.909165in}{0.383578in}}%
\pgfpathlineto{\pgfqpoint{3.909165in}{0.462149in}}%
\pgfpathlineto{\pgfqpoint{3.881743in}{0.462149in}}%
\pgfpathclose%
\pgfusepath{fill}%
\end{pgfscope}%
\begin{pgfscope}%
\pgfpathrectangle{\pgfqpoint{0.526905in}{0.383578in}}{\pgfqpoint{3.875000in}{2.310000in}}%
\pgfusepath{clip}%
\pgfsetbuttcap%
\pgfsetmiterjoin%
\definecolor{currentfill}{rgb}{0.333333,0.333333,0.333333}%
\pgfsetfillcolor{currentfill}%
\pgfsetlinewidth{0.000000pt}%
\definecolor{currentstroke}{rgb}{0.000000,0.000000,0.000000}%
\pgfsetstrokecolor{currentstroke}%
\pgfsetstrokeopacity{0.000000}%
\pgfsetdash{}{0pt}%
\pgfpathmoveto{\pgfqpoint{3.899332in}{0.383578in}}%
\pgfpathlineto{\pgfqpoint{3.926754in}{0.383578in}}%
\pgfpathlineto{\pgfqpoint{3.926754in}{0.422864in}}%
\pgfpathlineto{\pgfqpoint{3.899332in}{0.422864in}}%
\pgfpathclose%
\pgfusepath{fill}%
\end{pgfscope}%
\begin{pgfscope}%
\pgfpathrectangle{\pgfqpoint{0.526905in}{0.383578in}}{\pgfqpoint{3.875000in}{2.310000in}}%
\pgfusepath{clip}%
\pgfsetbuttcap%
\pgfsetmiterjoin%
\definecolor{currentfill}{rgb}{0.333333,0.333333,0.333333}%
\pgfsetfillcolor{currentfill}%
\pgfsetlinewidth{0.000000pt}%
\definecolor{currentstroke}{rgb}{0.000000,0.000000,0.000000}%
\pgfsetstrokecolor{currentstroke}%
\pgfsetstrokeopacity{0.000000}%
\pgfsetdash{}{0pt}%
\pgfpathmoveto{\pgfqpoint{3.916921in}{0.383578in}}%
\pgfpathlineto{\pgfqpoint{3.944344in}{0.383578in}}%
\pgfpathlineto{\pgfqpoint{3.944344in}{0.406027in}}%
\pgfpathlineto{\pgfqpoint{3.916921in}{0.406027in}}%
\pgfpathclose%
\pgfusepath{fill}%
\end{pgfscope}%
\begin{pgfscope}%
\pgfpathrectangle{\pgfqpoint{0.526905in}{0.383578in}}{\pgfqpoint{3.875000in}{2.310000in}}%
\pgfusepath{clip}%
\pgfsetbuttcap%
\pgfsetmiterjoin%
\definecolor{currentfill}{rgb}{0.333333,0.333333,0.333333}%
\pgfsetfillcolor{currentfill}%
\pgfsetlinewidth{0.000000pt}%
\definecolor{currentstroke}{rgb}{0.000000,0.000000,0.000000}%
\pgfsetstrokecolor{currentstroke}%
\pgfsetstrokeopacity{0.000000}%
\pgfsetdash{}{0pt}%
\pgfpathmoveto{\pgfqpoint{3.934510in}{0.383578in}}%
\pgfpathlineto{\pgfqpoint{3.961933in}{0.383578in}}%
\pgfpathlineto{\pgfqpoint{3.961933in}{0.417252in}}%
\pgfpathlineto{\pgfqpoint{3.934510in}{0.417252in}}%
\pgfpathclose%
\pgfusepath{fill}%
\end{pgfscope}%
\begin{pgfscope}%
\pgfpathrectangle{\pgfqpoint{0.526905in}{0.383578in}}{\pgfqpoint{3.875000in}{2.310000in}}%
\pgfusepath{clip}%
\pgfsetbuttcap%
\pgfsetmiterjoin%
\definecolor{currentfill}{rgb}{0.333333,0.333333,0.333333}%
\pgfsetfillcolor{currentfill}%
\pgfsetlinewidth{0.000000pt}%
\definecolor{currentstroke}{rgb}{0.000000,0.000000,0.000000}%
\pgfsetstrokecolor{currentstroke}%
\pgfsetstrokeopacity{0.000000}%
\pgfsetdash{}{0pt}%
\pgfpathmoveto{\pgfqpoint{3.952099in}{0.383578in}}%
\pgfpathlineto{\pgfqpoint{3.979522in}{0.383578in}}%
\pgfpathlineto{\pgfqpoint{3.979522in}{0.422864in}}%
\pgfpathlineto{\pgfqpoint{3.952099in}{0.422864in}}%
\pgfpathclose%
\pgfusepath{fill}%
\end{pgfscope}%
\begin{pgfscope}%
\pgfpathrectangle{\pgfqpoint{0.526905in}{0.383578in}}{\pgfqpoint{3.875000in}{2.310000in}}%
\pgfusepath{clip}%
\pgfsetbuttcap%
\pgfsetmiterjoin%
\definecolor{currentfill}{rgb}{0.333333,0.333333,0.333333}%
\pgfsetfillcolor{currentfill}%
\pgfsetlinewidth{0.000000pt}%
\definecolor{currentstroke}{rgb}{0.000000,0.000000,0.000000}%
\pgfsetstrokecolor{currentstroke}%
\pgfsetstrokeopacity{0.000000}%
\pgfsetdash{}{0pt}%
\pgfpathmoveto{\pgfqpoint{3.969688in}{0.383578in}}%
\pgfpathlineto{\pgfqpoint{3.997111in}{0.383578in}}%
\pgfpathlineto{\pgfqpoint{3.997111in}{0.422864in}}%
\pgfpathlineto{\pgfqpoint{3.969688in}{0.422864in}}%
\pgfpathclose%
\pgfusepath{fill}%
\end{pgfscope}%
\begin{pgfscope}%
\pgfpathrectangle{\pgfqpoint{0.526905in}{0.383578in}}{\pgfqpoint{3.875000in}{2.310000in}}%
\pgfusepath{clip}%
\pgfsetbuttcap%
\pgfsetmiterjoin%
\definecolor{currentfill}{rgb}{0.333333,0.333333,0.333333}%
\pgfsetfillcolor{currentfill}%
\pgfsetlinewidth{0.000000pt}%
\definecolor{currentstroke}{rgb}{0.000000,0.000000,0.000000}%
\pgfsetstrokecolor{currentstroke}%
\pgfsetstrokeopacity{0.000000}%
\pgfsetdash{}{0pt}%
\pgfpathmoveto{\pgfqpoint{3.987278in}{0.383578in}}%
\pgfpathlineto{\pgfqpoint{4.014700in}{0.383578in}}%
\pgfpathlineto{\pgfqpoint{4.014700in}{0.417252in}}%
\pgfpathlineto{\pgfqpoint{3.987278in}{0.417252in}}%
\pgfpathclose%
\pgfusepath{fill}%
\end{pgfscope}%
\begin{pgfscope}%
\pgfpathrectangle{\pgfqpoint{0.526905in}{0.383578in}}{\pgfqpoint{3.875000in}{2.310000in}}%
\pgfusepath{clip}%
\pgfsetbuttcap%
\pgfsetmiterjoin%
\definecolor{currentfill}{rgb}{0.333333,0.333333,0.333333}%
\pgfsetfillcolor{currentfill}%
\pgfsetlinewidth{0.000000pt}%
\definecolor{currentstroke}{rgb}{0.000000,0.000000,0.000000}%
\pgfsetstrokecolor{currentstroke}%
\pgfsetstrokeopacity{0.000000}%
\pgfsetdash{}{0pt}%
\pgfpathmoveto{\pgfqpoint{4.004867in}{0.383578in}}%
\pgfpathlineto{\pgfqpoint{4.032289in}{0.383578in}}%
\pgfpathlineto{\pgfqpoint{4.032289in}{0.422864in}}%
\pgfpathlineto{\pgfqpoint{4.004867in}{0.422864in}}%
\pgfpathclose%
\pgfusepath{fill}%
\end{pgfscope}%
\begin{pgfscope}%
\pgfpathrectangle{\pgfqpoint{0.526905in}{0.383578in}}{\pgfqpoint{3.875000in}{2.310000in}}%
\pgfusepath{clip}%
\pgfsetbuttcap%
\pgfsetmiterjoin%
\definecolor{currentfill}{rgb}{0.333333,0.333333,0.333333}%
\pgfsetfillcolor{currentfill}%
\pgfsetlinewidth{0.000000pt}%
\definecolor{currentstroke}{rgb}{0.000000,0.000000,0.000000}%
\pgfsetstrokecolor{currentstroke}%
\pgfsetstrokeopacity{0.000000}%
\pgfsetdash{}{0pt}%
\pgfpathmoveto{\pgfqpoint{4.022456in}{0.383578in}}%
\pgfpathlineto{\pgfqpoint{4.049878in}{0.383578in}}%
\pgfpathlineto{\pgfqpoint{4.049878in}{0.428476in}}%
\pgfpathlineto{\pgfqpoint{4.022456in}{0.428476in}}%
\pgfpathclose%
\pgfusepath{fill}%
\end{pgfscope}%
\begin{pgfscope}%
\pgfpathrectangle{\pgfqpoint{0.526905in}{0.383578in}}{\pgfqpoint{3.875000in}{2.310000in}}%
\pgfusepath{clip}%
\pgfsetbuttcap%
\pgfsetmiterjoin%
\definecolor{currentfill}{rgb}{0.333333,0.333333,0.333333}%
\pgfsetfillcolor{currentfill}%
\pgfsetlinewidth{0.000000pt}%
\definecolor{currentstroke}{rgb}{0.000000,0.000000,0.000000}%
\pgfsetstrokecolor{currentstroke}%
\pgfsetstrokeopacity{0.000000}%
\pgfsetdash{}{0pt}%
\pgfpathmoveto{\pgfqpoint{4.040045in}{0.383578in}}%
\pgfpathlineto{\pgfqpoint{4.067467in}{0.383578in}}%
\pgfpathlineto{\pgfqpoint{4.067467in}{0.394803in}}%
\pgfpathlineto{\pgfqpoint{4.040045in}{0.394803in}}%
\pgfpathclose%
\pgfusepath{fill}%
\end{pgfscope}%
\begin{pgfscope}%
\pgfpathrectangle{\pgfqpoint{0.526905in}{0.383578in}}{\pgfqpoint{3.875000in}{2.310000in}}%
\pgfusepath{clip}%
\pgfsetbuttcap%
\pgfsetmiterjoin%
\definecolor{currentfill}{rgb}{0.333333,0.333333,0.333333}%
\pgfsetfillcolor{currentfill}%
\pgfsetlinewidth{0.000000pt}%
\definecolor{currentstroke}{rgb}{0.000000,0.000000,0.000000}%
\pgfsetstrokecolor{currentstroke}%
\pgfsetstrokeopacity{0.000000}%
\pgfsetdash{}{0pt}%
\pgfpathmoveto{\pgfqpoint{4.057634in}{0.383578in}}%
\pgfpathlineto{\pgfqpoint{4.085056in}{0.383578in}}%
\pgfpathlineto{\pgfqpoint{4.085056in}{0.400415in}}%
\pgfpathlineto{\pgfqpoint{4.057634in}{0.400415in}}%
\pgfpathclose%
\pgfusepath{fill}%
\end{pgfscope}%
\begin{pgfscope}%
\pgfpathrectangle{\pgfqpoint{0.526905in}{0.383578in}}{\pgfqpoint{3.875000in}{2.310000in}}%
\pgfusepath{clip}%
\pgfsetbuttcap%
\pgfsetmiterjoin%
\definecolor{currentfill}{rgb}{0.333333,0.333333,0.333333}%
\pgfsetfillcolor{currentfill}%
\pgfsetlinewidth{0.000000pt}%
\definecolor{currentstroke}{rgb}{0.000000,0.000000,0.000000}%
\pgfsetstrokecolor{currentstroke}%
\pgfsetstrokeopacity{0.000000}%
\pgfsetdash{}{0pt}%
\pgfpathmoveto{\pgfqpoint{4.075223in}{0.383578in}}%
\pgfpathlineto{\pgfqpoint{4.102645in}{0.383578in}}%
\pgfpathlineto{\pgfqpoint{4.102645in}{0.411639in}}%
\pgfpathlineto{\pgfqpoint{4.075223in}{0.411639in}}%
\pgfpathclose%
\pgfusepath{fill}%
\end{pgfscope}%
\begin{pgfscope}%
\pgfpathrectangle{\pgfqpoint{0.526905in}{0.383578in}}{\pgfqpoint{3.875000in}{2.310000in}}%
\pgfusepath{clip}%
\pgfsetbuttcap%
\pgfsetmiterjoin%
\definecolor{currentfill}{rgb}{0.333333,0.333333,0.333333}%
\pgfsetfillcolor{currentfill}%
\pgfsetlinewidth{0.000000pt}%
\definecolor{currentstroke}{rgb}{0.000000,0.000000,0.000000}%
\pgfsetstrokecolor{currentstroke}%
\pgfsetstrokeopacity{0.000000}%
\pgfsetdash{}{0pt}%
\pgfpathmoveto{\pgfqpoint{4.092812in}{0.383578in}}%
\pgfpathlineto{\pgfqpoint{4.120234in}{0.383578in}}%
\pgfpathlineto{\pgfqpoint{4.120234in}{0.400415in}}%
\pgfpathlineto{\pgfqpoint{4.092812in}{0.400415in}}%
\pgfpathclose%
\pgfusepath{fill}%
\end{pgfscope}%
\begin{pgfscope}%
\pgfpathrectangle{\pgfqpoint{0.526905in}{0.383578in}}{\pgfqpoint{3.875000in}{2.310000in}}%
\pgfusepath{clip}%
\pgfsetbuttcap%
\pgfsetmiterjoin%
\definecolor{currentfill}{rgb}{0.333333,0.333333,0.333333}%
\pgfsetfillcolor{currentfill}%
\pgfsetlinewidth{0.000000pt}%
\definecolor{currentstroke}{rgb}{0.000000,0.000000,0.000000}%
\pgfsetstrokecolor{currentstroke}%
\pgfsetstrokeopacity{0.000000}%
\pgfsetdash{}{0pt}%
\pgfpathmoveto{\pgfqpoint{4.110401in}{0.383578in}}%
\pgfpathlineto{\pgfqpoint{4.137823in}{0.383578in}}%
\pgfpathlineto{\pgfqpoint{4.137823in}{0.394803in}}%
\pgfpathlineto{\pgfqpoint{4.110401in}{0.394803in}}%
\pgfpathclose%
\pgfusepath{fill}%
\end{pgfscope}%
\begin{pgfscope}%
\pgfpathrectangle{\pgfqpoint{0.526905in}{0.383578in}}{\pgfqpoint{3.875000in}{2.310000in}}%
\pgfusepath{clip}%
\pgfsetbuttcap%
\pgfsetmiterjoin%
\definecolor{currentfill}{rgb}{0.333333,0.333333,0.333333}%
\pgfsetfillcolor{currentfill}%
\pgfsetlinewidth{0.000000pt}%
\definecolor{currentstroke}{rgb}{0.000000,0.000000,0.000000}%
\pgfsetstrokecolor{currentstroke}%
\pgfsetstrokeopacity{0.000000}%
\pgfsetdash{}{0pt}%
\pgfpathmoveto{\pgfqpoint{4.127990in}{0.383578in}}%
\pgfpathlineto{\pgfqpoint{4.155412in}{0.383578in}}%
\pgfpathlineto{\pgfqpoint{4.155412in}{0.394803in}}%
\pgfpathlineto{\pgfqpoint{4.127990in}{0.394803in}}%
\pgfpathclose%
\pgfusepath{fill}%
\end{pgfscope}%
\begin{pgfscope}%
\pgfpathrectangle{\pgfqpoint{0.526905in}{0.383578in}}{\pgfqpoint{3.875000in}{2.310000in}}%
\pgfusepath{clip}%
\pgfsetbuttcap%
\pgfsetmiterjoin%
\definecolor{currentfill}{rgb}{0.333333,0.333333,0.333333}%
\pgfsetfillcolor{currentfill}%
\pgfsetlinewidth{0.000000pt}%
\definecolor{currentstroke}{rgb}{0.000000,0.000000,0.000000}%
\pgfsetstrokecolor{currentstroke}%
\pgfsetstrokeopacity{0.000000}%
\pgfsetdash{}{0pt}%
\pgfpathmoveto{\pgfqpoint{4.145579in}{0.383578in}}%
\pgfpathlineto{\pgfqpoint{4.173001in}{0.383578in}}%
\pgfpathlineto{\pgfqpoint{4.173001in}{0.389190in}}%
\pgfpathlineto{\pgfqpoint{4.145579in}{0.389190in}}%
\pgfpathclose%
\pgfusepath{fill}%
\end{pgfscope}%
\begin{pgfscope}%
\pgfpathrectangle{\pgfqpoint{0.526905in}{0.383578in}}{\pgfqpoint{3.875000in}{2.310000in}}%
\pgfusepath{clip}%
\pgfsetbuttcap%
\pgfsetmiterjoin%
\definecolor{currentfill}{rgb}{0.333333,0.333333,0.333333}%
\pgfsetfillcolor{currentfill}%
\pgfsetlinewidth{0.000000pt}%
\definecolor{currentstroke}{rgb}{0.000000,0.000000,0.000000}%
\pgfsetstrokecolor{currentstroke}%
\pgfsetstrokeopacity{0.000000}%
\pgfsetdash{}{0pt}%
\pgfpathmoveto{\pgfqpoint{4.163168in}{0.383578in}}%
\pgfpathlineto{\pgfqpoint{4.190590in}{0.383578in}}%
\pgfpathlineto{\pgfqpoint{4.190590in}{0.383578in}}%
\pgfpathlineto{\pgfqpoint{4.163168in}{0.383578in}}%
\pgfpathclose%
\pgfusepath{fill}%
\end{pgfscope}%
\begin{pgfscope}%
\pgfpathrectangle{\pgfqpoint{0.526905in}{0.383578in}}{\pgfqpoint{3.875000in}{2.310000in}}%
\pgfusepath{clip}%
\pgfsetbuttcap%
\pgfsetmiterjoin%
\definecolor{currentfill}{rgb}{0.333333,0.333333,0.333333}%
\pgfsetfillcolor{currentfill}%
\pgfsetlinewidth{0.000000pt}%
\definecolor{currentstroke}{rgb}{0.000000,0.000000,0.000000}%
\pgfsetstrokecolor{currentstroke}%
\pgfsetstrokeopacity{0.000000}%
\pgfsetdash{}{0pt}%
\pgfpathmoveto{\pgfqpoint{4.180757in}{0.383578in}}%
\pgfpathlineto{\pgfqpoint{4.208179in}{0.383578in}}%
\pgfpathlineto{\pgfqpoint{4.208179in}{0.389190in}}%
\pgfpathlineto{\pgfqpoint{4.180757in}{0.389190in}}%
\pgfpathclose%
\pgfusepath{fill}%
\end{pgfscope}%
\begin{pgfscope}%
\pgfpathrectangle{\pgfqpoint{0.526905in}{0.383578in}}{\pgfqpoint{3.875000in}{2.310000in}}%
\pgfusepath{clip}%
\pgfsetbuttcap%
\pgfsetmiterjoin%
\definecolor{currentfill}{rgb}{0.333333,0.333333,0.333333}%
\pgfsetfillcolor{currentfill}%
\pgfsetlinewidth{0.000000pt}%
\definecolor{currentstroke}{rgb}{0.000000,0.000000,0.000000}%
\pgfsetstrokecolor{currentstroke}%
\pgfsetstrokeopacity{0.000000}%
\pgfsetdash{}{0pt}%
\pgfpathmoveto{\pgfqpoint{4.198346in}{0.383578in}}%
\pgfpathlineto{\pgfqpoint{4.225768in}{0.383578in}}%
\pgfpathlineto{\pgfqpoint{4.225768in}{0.389190in}}%
\pgfpathlineto{\pgfqpoint{4.198346in}{0.389190in}}%
\pgfpathclose%
\pgfusepath{fill}%
\end{pgfscope}%
\begin{pgfscope}%
\pgfpathrectangle{\pgfqpoint{0.526905in}{0.383578in}}{\pgfqpoint{3.875000in}{2.310000in}}%
\pgfusepath{clip}%
\pgfsetrectcap%
\pgfsetroundjoin%
\pgfsetlinewidth{0.803000pt}%
\definecolor{currentstroke}{rgb}{0.333333,0.333333,0.333333}%
\pgfsetstrokecolor{currentstroke}%
\pgfsetdash{}{0pt}%
\pgfpathmoveto{\pgfqpoint{0.711836in}{0.384563in}}%
\pgfpathlineto{\pgfqpoint{0.870137in}{0.386816in}}%
\pgfpathlineto{\pgfqpoint{0.975671in}{0.390348in}}%
\pgfpathlineto{\pgfqpoint{1.063617in}{0.395678in}}%
\pgfpathlineto{\pgfqpoint{1.133973in}{0.402405in}}%
\pgfpathlineto{\pgfqpoint{1.186740in}{0.409467in}}%
\pgfpathlineto{\pgfqpoint{1.239507in}{0.418782in}}%
\pgfpathlineto{\pgfqpoint{1.292274in}{0.430917in}}%
\pgfpathlineto{\pgfqpoint{1.327452in}{0.440893in}}%
\pgfpathlineto{\pgfqpoint{1.362631in}{0.452627in}}%
\pgfpathlineto{\pgfqpoint{1.397809in}{0.466352in}}%
\pgfpathlineto{\pgfqpoint{1.432987in}{0.482312in}}%
\pgfpathlineto{\pgfqpoint{1.468165in}{0.500765in}}%
\pgfpathlineto{\pgfqpoint{1.503343in}{0.521978in}}%
\pgfpathlineto{\pgfqpoint{1.538521in}{0.546220in}}%
\pgfpathlineto{\pgfqpoint{1.573699in}{0.573761in}}%
\pgfpathlineto{\pgfqpoint{1.608877in}{0.604863in}}%
\pgfpathlineto{\pgfqpoint{1.644055in}{0.639774in}}%
\pgfpathlineto{\pgfqpoint{1.679234in}{0.678723in}}%
\pgfpathlineto{\pgfqpoint{1.714412in}{0.721906in}}%
\pgfpathlineto{\pgfqpoint{1.749590in}{0.769485in}}%
\pgfpathlineto{\pgfqpoint{1.784768in}{0.821572in}}%
\pgfpathlineto{\pgfqpoint{1.819946in}{0.878224in}}%
\pgfpathlineto{\pgfqpoint{1.855124in}{0.939434in}}%
\pgfpathlineto{\pgfqpoint{1.890302in}{1.005122in}}%
\pgfpathlineto{\pgfqpoint{1.925480in}{1.075125in}}%
\pgfpathlineto{\pgfqpoint{1.960658in}{1.149198in}}%
\pgfpathlineto{\pgfqpoint{2.013426in}{1.267172in}}%
\pgfpathlineto{\pgfqpoint{2.066193in}{1.391982in}}%
\pgfpathlineto{\pgfqpoint{2.136549in}{1.565498in}}%
\pgfpathlineto{\pgfqpoint{2.259672in}{1.871149in}}%
\pgfpathlineto{\pgfqpoint{2.312439in}{1.994936in}}%
\pgfpathlineto{\pgfqpoint{2.347618in}{2.072604in}}%
\pgfpathlineto{\pgfqpoint{2.382796in}{2.145237in}}%
\pgfpathlineto{\pgfqpoint{2.417974in}{2.211883in}}%
\pgfpathlineto{\pgfqpoint{2.453152in}{2.271641in}}%
\pgfpathlineto{\pgfqpoint{2.488330in}{2.323684in}}%
\pgfpathlineto{\pgfqpoint{2.505919in}{2.346579in}}%
\pgfpathlineto{\pgfqpoint{2.523508in}{2.367277in}}%
\pgfpathlineto{\pgfqpoint{2.541097in}{2.385703in}}%
\pgfpathlineto{\pgfqpoint{2.558686in}{2.401791in}}%
\pgfpathlineto{\pgfqpoint{2.576275in}{2.415482in}}%
\pgfpathlineto{\pgfqpoint{2.593864in}{2.426725in}}%
\pgfpathlineto{\pgfqpoint{2.611453in}{2.435479in}}%
\pgfpathlineto{\pgfqpoint{2.629042in}{2.441711in}}%
\pgfpathlineto{\pgfqpoint{2.646632in}{2.445398in}}%
\pgfpathlineto{\pgfqpoint{2.664221in}{2.446526in}}%
\pgfpathlineto{\pgfqpoint{2.681810in}{2.445091in}}%
\pgfpathlineto{\pgfqpoint{2.699399in}{2.441099in}}%
\pgfpathlineto{\pgfqpoint{2.716988in}{2.434564in}}%
\pgfpathlineto{\pgfqpoint{2.734577in}{2.425511in}}%
\pgfpathlineto{\pgfqpoint{2.752166in}{2.413973in}}%
\pgfpathlineto{\pgfqpoint{2.769755in}{2.399993in}}%
\pgfpathlineto{\pgfqpoint{2.787344in}{2.383622in}}%
\pgfpathlineto{\pgfqpoint{2.804933in}{2.364920in}}%
\pgfpathlineto{\pgfqpoint{2.822522in}{2.343956in}}%
\pgfpathlineto{\pgfqpoint{2.840111in}{2.320804in}}%
\pgfpathlineto{\pgfqpoint{2.857700in}{2.295548in}}%
\pgfpathlineto{\pgfqpoint{2.892878in}{2.239089in}}%
\pgfpathlineto{\pgfqpoint{2.928056in}{2.175369in}}%
\pgfpathlineto{\pgfqpoint{2.963234in}{2.105257in}}%
\pgfpathlineto{\pgfqpoint{2.998413in}{2.029686in}}%
\pgfpathlineto{\pgfqpoint{3.033591in}{1.949627in}}%
\pgfpathlineto{\pgfqpoint{3.086358in}{1.823299in}}%
\pgfpathlineto{\pgfqpoint{3.174303in}{1.604270in}}%
\pgfpathlineto{\pgfqpoint{3.262248in}{1.386901in}}%
\pgfpathlineto{\pgfqpoint{3.315016in}{1.262328in}}%
\pgfpathlineto{\pgfqpoint{3.367783in}{1.144662in}}%
\pgfpathlineto{\pgfqpoint{3.402961in}{1.070824in}}%
\pgfpathlineto{\pgfqpoint{3.438139in}{1.001072in}}%
\pgfpathlineto{\pgfqpoint{3.473317in}{0.935648in}}%
\pgfpathlineto{\pgfqpoint{3.508495in}{0.874709in}}%
\pgfpathlineto{\pgfqpoint{3.543673in}{0.818330in}}%
\pgfpathlineto{\pgfqpoint{3.578851in}{0.766515in}}%
\pgfpathlineto{\pgfqpoint{3.614029in}{0.719203in}}%
\pgfpathlineto{\pgfqpoint{3.649208in}{0.676277in}}%
\pgfpathlineto{\pgfqpoint{3.684386in}{0.637576in}}%
\pgfpathlineto{\pgfqpoint{3.719564in}{0.602899in}}%
\pgfpathlineto{\pgfqpoint{3.754742in}{0.572017in}}%
\pgfpathlineto{\pgfqpoint{3.789920in}{0.544681in}}%
\pgfpathlineto{\pgfqpoint{3.825098in}{0.520628in}}%
\pgfpathlineto{\pgfqpoint{3.860276in}{0.499587in}}%
\pgfpathlineto{\pgfqpoint{3.895454in}{0.481290in}}%
\pgfpathlineto{\pgfqpoint{3.930632in}{0.465471in}}%
\pgfpathlineto{\pgfqpoint{3.965811in}{0.451872in}}%
\pgfpathlineto{\pgfqpoint{4.000989in}{0.440249in}}%
\pgfpathlineto{\pgfqpoint{4.036167in}{0.430371in}}%
\pgfpathlineto{\pgfqpoint{4.088934in}{0.418361in}}%
\pgfpathlineto{\pgfqpoint{4.141701in}{0.409146in}}%
\pgfpathlineto{\pgfqpoint{4.194468in}{0.402163in}}%
\pgfpathlineto{\pgfqpoint{4.212057in}{0.400247in}}%
\pgfpathlineto{\pgfqpoint{4.212057in}{0.400247in}}%
\pgfusepath{stroke}%
\end{pgfscope}%
\begin{pgfscope}%
\pgfpathrectangle{\pgfqpoint{0.526905in}{0.383578in}}{\pgfqpoint{3.875000in}{2.310000in}}%
\pgfusepath{clip}%
\pgfsetbuttcap%
\pgfsetroundjoin%
\pgfsetlinewidth{0.803000pt}%
\definecolor{currentstroke}{rgb}{0.333333,0.333333,0.333333}%
\pgfsetstrokecolor{currentstroke}%
\pgfsetdash{{2.960000pt}{1.280000pt}}{0.000000pt}%
\pgfpathmoveto{\pgfqpoint{2.663169in}{0.383578in}}%
\pgfpathlineto{\pgfqpoint{2.663169in}{2.693578in}}%
\pgfusepath{stroke}%
\end{pgfscope}%
\begin{pgfscope}%
\pgfpathrectangle{\pgfqpoint{0.526905in}{0.383578in}}{\pgfqpoint{3.875000in}{2.310000in}}%
\pgfusepath{clip}%
\pgfsetbuttcap%
\pgfsetroundjoin%
\pgfsetlinewidth{0.803000pt}%
\definecolor{currentstroke}{rgb}{1.000000,0.000000,0.000000}%
\pgfsetstrokecolor{currentstroke}%
\pgfsetdash{{2.960000pt}{1.280000pt}}{0.000000pt}%
\pgfpathmoveto{\pgfqpoint{3.332605in}{0.383578in}}%
\pgfpathlineto{\pgfqpoint{3.332605in}{2.693578in}}%
\pgfusepath{stroke}%
\end{pgfscope}%
\begin{pgfscope}%
\pgfsetrectcap%
\pgfsetmiterjoin%
\pgfsetlinewidth{0.501875pt}%
\definecolor{currentstroke}{rgb}{0.317647,0.317647,0.317647}%
\pgfsetstrokecolor{currentstroke}%
\pgfsetdash{}{0pt}%
\pgfpathmoveto{\pgfqpoint{0.526905in}{0.383578in}}%
\pgfpathlineto{\pgfqpoint{0.526905in}{2.693578in}}%
\pgfusepath{stroke}%
\end{pgfscope}%
\begin{pgfscope}%
\pgfsetrectcap%
\pgfsetmiterjoin%
\pgfsetlinewidth{0.501875pt}%
\definecolor{currentstroke}{rgb}{0.317647,0.317647,0.317647}%
\pgfsetstrokecolor{currentstroke}%
\pgfsetdash{}{0pt}%
\pgfpathmoveto{\pgfqpoint{0.526905in}{0.383578in}}%
\pgfpathlineto{\pgfqpoint{4.401905in}{0.383578in}}%
\pgfusepath{stroke}%
\end{pgfscope}%
\begin{pgfscope}%
\definecolor{textcolor}{rgb}{0.000000,0.000000,0.000000}%
\pgfsetstrokecolor{textcolor}%
\pgfsetfillcolor{textcolor}%
\pgftext[x=2.731725in,y=1.888981in,left,base]{\color{textcolor}\rmfamily\fontsize{8.000000}{9.600000}\selectfont \(\displaystyle V_{\mathrm{leak}}\)}%
\end{pgfscope}%
\begin{pgfscope}%
\definecolor{textcolor}{rgb}{0.000000,0.000000,0.000000}%
\pgfsetstrokecolor{textcolor}%
\pgfsetfillcolor{textcolor}%
\pgftext[x=3.401160in,y=1.888981in,left,base]{\color{textcolor}\rmfamily\fontsize{8.000000}{9.600000}\selectfont \(\displaystyle \vartheta\)}%
\end{pgfscope}%
\end{pgfpicture}%
\makeatother%
\endgroup%

	\end{center}
	\caption[Gaussian free membrane potential distribution on \gls{dls}.]{Gaussian free membrane potential distribution on \gls{dls}. In a naive approach, the part of the distribution that exceeds a certain threshold potential correlates to the number of fired spikes, which yields a perfect sigmoid for the activation function in return. The width of the distribution correlates to amount and impact of the injected noise spikes.}
	\label{vleak_w_noise}
\end{figure}



%As shown in the simulation of the activation function in \cref{ratecoding}, the slope and of the activation function and its alignment along the x-axis can be easily modified.

%A slope is tuned by 

%\begin{equation}
%\gls{transfer} = \gls{transfer}(\gls{nuin}, b, \gls{refrac}, \gls{tau_m}, \gls{v_reset}, f_{\gls{v_mem}}),
%\end{equation}
%with the free membrane distribution $f_{\gls{v_mem}} = f_{\gls{v_mem}}(V; \gls{v_leak}, \nu_\text{noise}, w_\text{noise})$. 
%The bias term $b$ in the neuron's activation is again replaced by the relative distance $\delta V$ between the resting potential and the threshold, since $b \propto \delta V = \gls{v_leak} - \gls{thres}$. 
%
%Some parameters will be kept fixed, whereas others are required to operate within a certain range such as the threshold or the input frequency. The setup is hand tuned with the support of parameter sweeps to find a suitable configuration of the fixed parameters. The final set of parameters is listed in \cref{hardwarevssimulationtable}
%
\paragraph{Calibration}\label{calibration}
The manufacturing process of analog neuromorphic hardware causes systematic stochastic deviations in the neuron and synapse parameters. The thereby induced heterogeneity between neurons and synapses is referred to as \emph{fixed-pattern noise} and is constant in time. In an uncalibrated state, the neuron circuits exhibit activation functions which do not align with regards to their maximum rate and alignment along the x-axis (see \cref{transferfunction_wout_calib}). Despite the detuned parameters, it has been shown that plasticity rules can correct the intrinsic imperfections of the analog hardware up to a certain degree (\citealp{wunderlich2019advantages}). However, the dynamic range of the individual neurons on the chip is limited and to ensure properly overlapping activation function a calibration of the respective parameters is inevitable.

\begin{figure}
	%\captionsetup[subfigure]{justification=centering}
	\centering
	\begin{subfigure}[b]{0.49\textwidth}
		\caption{}
		%% Creator: Matplotlib, PGF backend
%%
%% To include the figure in your LaTeX document, write
%%   \input{<filename>.pgf}
%%
%% Make sure the required packages are loaded in your preamble
%%   \usepackage{pgf}
%%
%% Figures using additional raster images can only be included by \input if
%% they are in the same directory as the main LaTeX file. For loading figures
%% from other directories you can use the `import` package
%%   \usepackage{import}
%% and then include the figures with
%%   \import{<path to file>}{<filename>.pgf}
%%
%% Matplotlib used the following preamble
%%   \usepackage{amsmath} \usepackage{pifont} \usepackage{xcolor} \definecolor{green}{HTML}{467821} \definecolor{red}{HTML}{CF4457} \usepackage[detect-all]{siunitx}
%%   \usepackage{fontspec}
%%
\begingroup%
\makeatletter%
\begin{pgfpicture}%
\pgfpathrectangle{\pgfpointorigin}{\pgfqpoint{2.931438in}{2.408578in}}%
\pgfusepath{use as bounding box, clip}%
\begin{pgfscope}%
\pgfsetbuttcap%
\pgfsetmiterjoin%
\pgfsetlinewidth{0.000000pt}%
\definecolor{currentstroke}{rgb}{0.000000,0.000000,0.000000}%
\pgfsetstrokecolor{currentstroke}%
\pgfsetstrokeopacity{0.000000}%
\pgfsetdash{}{0pt}%
\pgfpathmoveto{\pgfqpoint{0.000000in}{0.000000in}}%
\pgfpathlineto{\pgfqpoint{2.931438in}{0.000000in}}%
\pgfpathlineto{\pgfqpoint{2.931438in}{2.408578in}}%
\pgfpathlineto{\pgfqpoint{0.000000in}{2.408578in}}%
\pgfpathclose%
\pgfusepath{}%
\end{pgfscope}%
\begin{pgfscope}%
\pgfsetbuttcap%
\pgfsetmiterjoin%
\pgfsetlinewidth{0.000000pt}%
\definecolor{currentstroke}{rgb}{0.000000,0.000000,0.000000}%
\pgfsetstrokecolor{currentstroke}%
\pgfsetstrokeopacity{0.000000}%
\pgfsetdash{}{0pt}%
\pgfpathmoveto{\pgfqpoint{0.453589in}{0.383578in}}%
\pgfpathlineto{\pgfqpoint{2.778589in}{0.383578in}}%
\pgfpathlineto{\pgfqpoint{2.778589in}{2.308578in}}%
\pgfpathlineto{\pgfqpoint{0.453589in}{2.308578in}}%
\pgfpathclose%
\pgfusepath{}%
\end{pgfscope}%
\begin{pgfscope}%
\pgfsetbuttcap%
\pgfsetroundjoin%
\definecolor{currentfill}{rgb}{0.317647,0.317647,0.317647}%
\pgfsetfillcolor{currentfill}%
\pgfsetlinewidth{0.501875pt}%
\definecolor{currentstroke}{rgb}{0.317647,0.317647,0.317647}%
\pgfsetstrokecolor{currentstroke}%
\pgfsetdash{}{0pt}%
\pgfsys@defobject{currentmarker}{\pgfqpoint{0.000000in}{-0.020833in}}{\pgfqpoint{0.000000in}{0.000000in}}{%
\pgfpathmoveto{\pgfqpoint{0.000000in}{0.000000in}}%
\pgfpathlineto{\pgfqpoint{0.000000in}{-0.020833in}}%
\pgfusepath{stroke,fill}%
}%
\begin{pgfscope}%
\pgfsys@transformshift{0.483784in}{0.383578in}%
\pgfsys@useobject{currentmarker}{}%
\end{pgfscope}%
\end{pgfscope}%
\begin{pgfscope}%
\definecolor{textcolor}{rgb}{0.317647,0.317647,0.317647}%
\pgfsetstrokecolor{textcolor}%
\pgfsetfillcolor{textcolor}%
\pgftext[x=0.483784in,y=0.334967in,,top]{\color{textcolor}\rmfamily\fontsize{6.664000}{7.996800}\selectfont \(\displaystyle -600\)}%
\end{pgfscope}%
\begin{pgfscope}%
\pgfsetbuttcap%
\pgfsetroundjoin%
\definecolor{currentfill}{rgb}{0.317647,0.317647,0.317647}%
\pgfsetfillcolor{currentfill}%
\pgfsetlinewidth{0.501875pt}%
\definecolor{currentstroke}{rgb}{0.317647,0.317647,0.317647}%
\pgfsetstrokecolor{currentstroke}%
\pgfsetdash{}{0pt}%
\pgfsys@defobject{currentmarker}{\pgfqpoint{0.000000in}{-0.020833in}}{\pgfqpoint{0.000000in}{0.000000in}}{%
\pgfpathmoveto{\pgfqpoint{0.000000in}{0.000000in}}%
\pgfpathlineto{\pgfqpoint{0.000000in}{-0.020833in}}%
\pgfusepath{stroke,fill}%
}%
\begin{pgfscope}%
\pgfsys@transformshift{0.861219in}{0.383578in}%
\pgfsys@useobject{currentmarker}{}%
\end{pgfscope}%
\end{pgfscope}%
\begin{pgfscope}%
\definecolor{textcolor}{rgb}{0.317647,0.317647,0.317647}%
\pgfsetstrokecolor{textcolor}%
\pgfsetfillcolor{textcolor}%
\pgftext[x=0.861219in,y=0.334967in,,top]{\color{textcolor}\rmfamily\fontsize{6.664000}{7.996800}\selectfont \(\displaystyle -400\)}%
\end{pgfscope}%
\begin{pgfscope}%
\pgfsetbuttcap%
\pgfsetroundjoin%
\definecolor{currentfill}{rgb}{0.317647,0.317647,0.317647}%
\pgfsetfillcolor{currentfill}%
\pgfsetlinewidth{0.501875pt}%
\definecolor{currentstroke}{rgb}{0.317647,0.317647,0.317647}%
\pgfsetstrokecolor{currentstroke}%
\pgfsetdash{}{0pt}%
\pgfsys@defobject{currentmarker}{\pgfqpoint{0.000000in}{-0.020833in}}{\pgfqpoint{0.000000in}{0.000000in}}{%
\pgfpathmoveto{\pgfqpoint{0.000000in}{0.000000in}}%
\pgfpathlineto{\pgfqpoint{0.000000in}{-0.020833in}}%
\pgfusepath{stroke,fill}%
}%
\begin{pgfscope}%
\pgfsys@transformshift{1.238654in}{0.383578in}%
\pgfsys@useobject{currentmarker}{}%
\end{pgfscope}%
\end{pgfscope}%
\begin{pgfscope}%
\definecolor{textcolor}{rgb}{0.317647,0.317647,0.317647}%
\pgfsetstrokecolor{textcolor}%
\pgfsetfillcolor{textcolor}%
\pgftext[x=1.238654in,y=0.334967in,,top]{\color{textcolor}\rmfamily\fontsize{6.664000}{7.996800}\selectfont \(\displaystyle -200\)}%
\end{pgfscope}%
\begin{pgfscope}%
\pgfsetbuttcap%
\pgfsetroundjoin%
\definecolor{currentfill}{rgb}{0.317647,0.317647,0.317647}%
\pgfsetfillcolor{currentfill}%
\pgfsetlinewidth{0.501875pt}%
\definecolor{currentstroke}{rgb}{0.317647,0.317647,0.317647}%
\pgfsetstrokecolor{currentstroke}%
\pgfsetdash{}{0pt}%
\pgfsys@defobject{currentmarker}{\pgfqpoint{0.000000in}{-0.020833in}}{\pgfqpoint{0.000000in}{0.000000in}}{%
\pgfpathmoveto{\pgfqpoint{0.000000in}{0.000000in}}%
\pgfpathlineto{\pgfqpoint{0.000000in}{-0.020833in}}%
\pgfusepath{stroke,fill}%
}%
\begin{pgfscope}%
\pgfsys@transformshift{1.616089in}{0.383578in}%
\pgfsys@useobject{currentmarker}{}%
\end{pgfscope}%
\end{pgfscope}%
\begin{pgfscope}%
\definecolor{textcolor}{rgb}{0.317647,0.317647,0.317647}%
\pgfsetstrokecolor{textcolor}%
\pgfsetfillcolor{textcolor}%
\pgftext[x=1.616089in,y=0.334967in,,top]{\color{textcolor}\rmfamily\fontsize{6.664000}{7.996800}\selectfont \(\displaystyle 0\)}%
\end{pgfscope}%
\begin{pgfscope}%
\pgfsetbuttcap%
\pgfsetroundjoin%
\definecolor{currentfill}{rgb}{0.317647,0.317647,0.317647}%
\pgfsetfillcolor{currentfill}%
\pgfsetlinewidth{0.501875pt}%
\definecolor{currentstroke}{rgb}{0.317647,0.317647,0.317647}%
\pgfsetstrokecolor{currentstroke}%
\pgfsetdash{}{0pt}%
\pgfsys@defobject{currentmarker}{\pgfqpoint{0.000000in}{-0.020833in}}{\pgfqpoint{0.000000in}{0.000000in}}{%
\pgfpathmoveto{\pgfqpoint{0.000000in}{0.000000in}}%
\pgfpathlineto{\pgfqpoint{0.000000in}{-0.020833in}}%
\pgfusepath{stroke,fill}%
}%
\begin{pgfscope}%
\pgfsys@transformshift{1.993524in}{0.383578in}%
\pgfsys@useobject{currentmarker}{}%
\end{pgfscope}%
\end{pgfscope}%
\begin{pgfscope}%
\definecolor{textcolor}{rgb}{0.317647,0.317647,0.317647}%
\pgfsetstrokecolor{textcolor}%
\pgfsetfillcolor{textcolor}%
\pgftext[x=1.993524in,y=0.334967in,,top]{\color{textcolor}\rmfamily\fontsize{6.664000}{7.996800}\selectfont \(\displaystyle 200\)}%
\end{pgfscope}%
\begin{pgfscope}%
\pgfsetbuttcap%
\pgfsetroundjoin%
\definecolor{currentfill}{rgb}{0.317647,0.317647,0.317647}%
\pgfsetfillcolor{currentfill}%
\pgfsetlinewidth{0.501875pt}%
\definecolor{currentstroke}{rgb}{0.317647,0.317647,0.317647}%
\pgfsetstrokecolor{currentstroke}%
\pgfsetdash{}{0pt}%
\pgfsys@defobject{currentmarker}{\pgfqpoint{0.000000in}{-0.020833in}}{\pgfqpoint{0.000000in}{0.000000in}}{%
\pgfpathmoveto{\pgfqpoint{0.000000in}{0.000000in}}%
\pgfpathlineto{\pgfqpoint{0.000000in}{-0.020833in}}%
\pgfusepath{stroke,fill}%
}%
\begin{pgfscope}%
\pgfsys@transformshift{2.370959in}{0.383578in}%
\pgfsys@useobject{currentmarker}{}%
\end{pgfscope}%
\end{pgfscope}%
\begin{pgfscope}%
\definecolor{textcolor}{rgb}{0.317647,0.317647,0.317647}%
\pgfsetstrokecolor{textcolor}%
\pgfsetfillcolor{textcolor}%
\pgftext[x=2.370959in,y=0.334967in,,top]{\color{textcolor}\rmfamily\fontsize{6.664000}{7.996800}\selectfont \(\displaystyle 400\)}%
\end{pgfscope}%
\begin{pgfscope}%
\pgfsetbuttcap%
\pgfsetroundjoin%
\definecolor{currentfill}{rgb}{0.317647,0.317647,0.317647}%
\pgfsetfillcolor{currentfill}%
\pgfsetlinewidth{0.501875pt}%
\definecolor{currentstroke}{rgb}{0.317647,0.317647,0.317647}%
\pgfsetstrokecolor{currentstroke}%
\pgfsetdash{}{0pt}%
\pgfsys@defobject{currentmarker}{\pgfqpoint{0.000000in}{-0.020833in}}{\pgfqpoint{0.000000in}{0.000000in}}{%
\pgfpathmoveto{\pgfqpoint{0.000000in}{0.000000in}}%
\pgfpathlineto{\pgfqpoint{0.000000in}{-0.020833in}}%
\pgfusepath{stroke,fill}%
}%
\begin{pgfscope}%
\pgfsys@transformshift{2.748394in}{0.383578in}%
\pgfsys@useobject{currentmarker}{}%
\end{pgfscope}%
\end{pgfscope}%
\begin{pgfscope}%
\definecolor{textcolor}{rgb}{0.317647,0.317647,0.317647}%
\pgfsetstrokecolor{textcolor}%
\pgfsetfillcolor{textcolor}%
\pgftext[x=2.748394in,y=0.334967in,,top]{\color{textcolor}\rmfamily\fontsize{6.664000}{7.996800}\selectfont \(\displaystyle 600\)}%
\end{pgfscope}%
\begin{pgfscope}%
\definecolor{textcolor}{rgb}{0.317647,0.317647,0.317647}%
\pgfsetstrokecolor{textcolor}%
\pgfsetfillcolor{textcolor}%
\pgftext[x=1.616089in,y=0.197222in,,top]{\color{textcolor}\rmfamily\fontsize{6.664000}{7.996800}\selectfont input frequency \(\displaystyle \nu_\mathrm{in} \; (\si{\kilo \Hz})\)}%
\end{pgfscope}%
\begin{pgfscope}%
\pgfsetbuttcap%
\pgfsetroundjoin%
\definecolor{currentfill}{rgb}{0.317647,0.317647,0.317647}%
\pgfsetfillcolor{currentfill}%
\pgfsetlinewidth{0.501875pt}%
\definecolor{currentstroke}{rgb}{0.317647,0.317647,0.317647}%
\pgfsetstrokecolor{currentstroke}%
\pgfsetdash{}{0pt}%
\pgfsys@defobject{currentmarker}{\pgfqpoint{-0.020833in}{0.000000in}}{\pgfqpoint{0.000000in}{0.000000in}}{%
\pgfpathmoveto{\pgfqpoint{0.000000in}{0.000000in}}%
\pgfpathlineto{\pgfqpoint{-0.020833in}{0.000000in}}%
\pgfusepath{stroke,fill}%
}%
\begin{pgfscope}%
\pgfsys@transformshift{0.453589in}{0.471078in}%
\pgfsys@useobject{currentmarker}{}%
\end{pgfscope}%
\end{pgfscope}%
\begin{pgfscope}%
\definecolor{textcolor}{rgb}{0.317647,0.317647,0.317647}%
\pgfsetstrokecolor{textcolor}%
\pgfsetfillcolor{textcolor}%
\pgftext[x=0.363504in,y=0.438961in,left,base]{\color{textcolor}\rmfamily\fontsize{6.664000}{7.996800}\selectfont \(\displaystyle 0\)}%
\end{pgfscope}%
\begin{pgfscope}%
\pgfsetbuttcap%
\pgfsetroundjoin%
\definecolor{currentfill}{rgb}{0.317647,0.317647,0.317647}%
\pgfsetfillcolor{currentfill}%
\pgfsetlinewidth{0.501875pt}%
\definecolor{currentstroke}{rgb}{0.317647,0.317647,0.317647}%
\pgfsetstrokecolor{currentstroke}%
\pgfsetdash{}{0pt}%
\pgfsys@defobject{currentmarker}{\pgfqpoint{-0.020833in}{0.000000in}}{\pgfqpoint{0.000000in}{0.000000in}}{%
\pgfpathmoveto{\pgfqpoint{0.000000in}{0.000000in}}%
\pgfpathlineto{\pgfqpoint{-0.020833in}{0.000000in}}%
\pgfusepath{stroke,fill}%
}%
\begin{pgfscope}%
\pgfsys@transformshift{0.453589in}{0.786764in}%
\pgfsys@useobject{currentmarker}{}%
\end{pgfscope}%
\end{pgfscope}%
\begin{pgfscope}%
\definecolor{textcolor}{rgb}{0.317647,0.317647,0.317647}%
\pgfsetstrokecolor{textcolor}%
\pgfsetfillcolor{textcolor}%
\pgftext[x=0.308141in,y=0.754648in,left,base]{\color{textcolor}\rmfamily\fontsize{6.664000}{7.996800}\selectfont \(\displaystyle 20\)}%
\end{pgfscope}%
\begin{pgfscope}%
\pgfsetbuttcap%
\pgfsetroundjoin%
\definecolor{currentfill}{rgb}{0.317647,0.317647,0.317647}%
\pgfsetfillcolor{currentfill}%
\pgfsetlinewidth{0.501875pt}%
\definecolor{currentstroke}{rgb}{0.317647,0.317647,0.317647}%
\pgfsetstrokecolor{currentstroke}%
\pgfsetdash{}{0pt}%
\pgfsys@defobject{currentmarker}{\pgfqpoint{-0.020833in}{0.000000in}}{\pgfqpoint{0.000000in}{0.000000in}}{%
\pgfpathmoveto{\pgfqpoint{0.000000in}{0.000000in}}%
\pgfpathlineto{\pgfqpoint{-0.020833in}{0.000000in}}%
\pgfusepath{stroke,fill}%
}%
\begin{pgfscope}%
\pgfsys@transformshift{0.453589in}{1.102451in}%
\pgfsys@useobject{currentmarker}{}%
\end{pgfscope}%
\end{pgfscope}%
\begin{pgfscope}%
\definecolor{textcolor}{rgb}{0.317647,0.317647,0.317647}%
\pgfsetstrokecolor{textcolor}%
\pgfsetfillcolor{textcolor}%
\pgftext[x=0.308141in,y=1.070334in,left,base]{\color{textcolor}\rmfamily\fontsize{6.664000}{7.996800}\selectfont \(\displaystyle 40\)}%
\end{pgfscope}%
\begin{pgfscope}%
\pgfsetbuttcap%
\pgfsetroundjoin%
\definecolor{currentfill}{rgb}{0.317647,0.317647,0.317647}%
\pgfsetfillcolor{currentfill}%
\pgfsetlinewidth{0.501875pt}%
\definecolor{currentstroke}{rgb}{0.317647,0.317647,0.317647}%
\pgfsetstrokecolor{currentstroke}%
\pgfsetdash{}{0pt}%
\pgfsys@defobject{currentmarker}{\pgfqpoint{-0.020833in}{0.000000in}}{\pgfqpoint{0.000000in}{0.000000in}}{%
\pgfpathmoveto{\pgfqpoint{0.000000in}{0.000000in}}%
\pgfpathlineto{\pgfqpoint{-0.020833in}{0.000000in}}%
\pgfusepath{stroke,fill}%
}%
\begin{pgfscope}%
\pgfsys@transformshift{0.453589in}{1.418137in}%
\pgfsys@useobject{currentmarker}{}%
\end{pgfscope}%
\end{pgfscope}%
\begin{pgfscope}%
\definecolor{textcolor}{rgb}{0.317647,0.317647,0.317647}%
\pgfsetstrokecolor{textcolor}%
\pgfsetfillcolor{textcolor}%
\pgftext[x=0.308141in,y=1.386020in,left,base]{\color{textcolor}\rmfamily\fontsize{6.664000}{7.996800}\selectfont \(\displaystyle 60\)}%
\end{pgfscope}%
\begin{pgfscope}%
\pgfsetbuttcap%
\pgfsetroundjoin%
\definecolor{currentfill}{rgb}{0.317647,0.317647,0.317647}%
\pgfsetfillcolor{currentfill}%
\pgfsetlinewidth{0.501875pt}%
\definecolor{currentstroke}{rgb}{0.317647,0.317647,0.317647}%
\pgfsetstrokecolor{currentstroke}%
\pgfsetdash{}{0pt}%
\pgfsys@defobject{currentmarker}{\pgfqpoint{-0.020833in}{0.000000in}}{\pgfqpoint{0.000000in}{0.000000in}}{%
\pgfpathmoveto{\pgfqpoint{0.000000in}{0.000000in}}%
\pgfpathlineto{\pgfqpoint{-0.020833in}{0.000000in}}%
\pgfusepath{stroke,fill}%
}%
\begin{pgfscope}%
\pgfsys@transformshift{0.453589in}{1.733823in}%
\pgfsys@useobject{currentmarker}{}%
\end{pgfscope}%
\end{pgfscope}%
\begin{pgfscope}%
\definecolor{textcolor}{rgb}{0.317647,0.317647,0.317647}%
\pgfsetstrokecolor{textcolor}%
\pgfsetfillcolor{textcolor}%
\pgftext[x=0.308141in,y=1.701706in,left,base]{\color{textcolor}\rmfamily\fontsize{6.664000}{7.996800}\selectfont \(\displaystyle 80\)}%
\end{pgfscope}%
\begin{pgfscope}%
\pgfsetbuttcap%
\pgfsetroundjoin%
\definecolor{currentfill}{rgb}{0.317647,0.317647,0.317647}%
\pgfsetfillcolor{currentfill}%
\pgfsetlinewidth{0.501875pt}%
\definecolor{currentstroke}{rgb}{0.317647,0.317647,0.317647}%
\pgfsetstrokecolor{currentstroke}%
\pgfsetdash{}{0pt}%
\pgfsys@defobject{currentmarker}{\pgfqpoint{-0.020833in}{0.000000in}}{\pgfqpoint{0.000000in}{0.000000in}}{%
\pgfpathmoveto{\pgfqpoint{0.000000in}{0.000000in}}%
\pgfpathlineto{\pgfqpoint{-0.020833in}{0.000000in}}%
\pgfusepath{stroke,fill}%
}%
\begin{pgfscope}%
\pgfsys@transformshift{0.453589in}{2.049509in}%
\pgfsys@useobject{currentmarker}{}%
\end{pgfscope}%
\end{pgfscope}%
\begin{pgfscope}%
\definecolor{textcolor}{rgb}{0.317647,0.317647,0.317647}%
\pgfsetstrokecolor{textcolor}%
\pgfsetfillcolor{textcolor}%
\pgftext[x=0.252778in,y=2.017393in,left,base]{\color{textcolor}\rmfamily\fontsize{6.664000}{7.996800}\selectfont \(\displaystyle 100\)}%
\end{pgfscope}%
\begin{pgfscope}%
\definecolor{textcolor}{rgb}{0.317647,0.317647,0.317647}%
\pgfsetstrokecolor{textcolor}%
\pgfsetfillcolor{textcolor}%
\pgftext[x=0.197222in,y=1.346078in,,bottom,rotate=90.000000]{\color{textcolor}\rmfamily\fontsize{6.664000}{7.996800}\selectfont output frequency \(\displaystyle \nu_\mathrm{out} \; (\si{\kilo \Hz})\)}%
\end{pgfscope}%
\begin{pgfscope}%
\pgfpathrectangle{\pgfqpoint{0.453589in}{0.383578in}}{\pgfqpoint{2.325000in}{1.925000in}}%
\pgfusepath{clip}%
\pgfsetrectcap%
\pgfsetroundjoin%
\pgfsetlinewidth{0.803000pt}%
\definecolor{currentstroke}{rgb}{0.333333,0.333333,0.333333}%
\pgfsetstrokecolor{currentstroke}%
\pgfsetdash{}{0pt}%
\pgfpathmoveto{\pgfqpoint{0.559271in}{0.471078in}}%
\pgfpathlineto{\pgfqpoint{0.617983in}{0.471078in}}%
\pgfpathlineto{\pgfqpoint{0.676695in}{0.471078in}}%
\pgfpathlineto{\pgfqpoint{0.735407in}{0.471078in}}%
\pgfpathlineto{\pgfqpoint{0.794119in}{0.471078in}}%
\pgfpathlineto{\pgfqpoint{0.852831in}{0.471078in}}%
\pgfpathlineto{\pgfqpoint{0.911543in}{0.471078in}}%
\pgfpathlineto{\pgfqpoint{0.970255in}{0.471078in}}%
\pgfpathlineto{\pgfqpoint{1.028968in}{0.471078in}}%
\pgfpathlineto{\pgfqpoint{1.087680in}{0.471078in}}%
\pgfpathlineto{\pgfqpoint{1.146392in}{0.471078in}}%
\pgfpathlineto{\pgfqpoint{1.205104in}{0.471078in}}%
\pgfpathlineto{\pgfqpoint{1.263816in}{0.471078in}}%
\pgfpathlineto{\pgfqpoint{1.322528in}{0.471078in}}%
\pgfpathlineto{\pgfqpoint{1.381240in}{0.471078in}}%
\pgfpathlineto{\pgfqpoint{1.439952in}{0.471078in}}%
\pgfpathlineto{\pgfqpoint{1.498665in}{0.471078in}}%
\pgfpathlineto{\pgfqpoint{1.557377in}{0.471078in}}%
\pgfpathlineto{\pgfqpoint{1.616089in}{0.471078in}}%
\pgfpathlineto{\pgfqpoint{1.674801in}{0.471078in}}%
\pgfpathlineto{\pgfqpoint{1.733513in}{0.477941in}}%
\pgfpathlineto{\pgfqpoint{1.792225in}{0.512255in}}%
\pgfpathlineto{\pgfqpoint{1.850937in}{0.580882in}}%
\pgfpathlineto{\pgfqpoint{1.909649in}{0.766176in}}%
\pgfpathlineto{\pgfqpoint{1.968361in}{0.958333in}}%
\pgfpathlineto{\pgfqpoint{2.027074in}{1.219117in}}%
\pgfpathlineto{\pgfqpoint{2.085786in}{1.466176in}}%
\pgfpathlineto{\pgfqpoint{2.144498in}{1.699509in}}%
\pgfpathlineto{\pgfqpoint{2.203210in}{1.850490in}}%
\pgfpathlineto{\pgfqpoint{2.261922in}{1.974019in}}%
\pgfpathlineto{\pgfqpoint{2.320634in}{2.001470in}}%
\pgfpathlineto{\pgfqpoint{2.379346in}{2.035784in}}%
\pgfpathlineto{\pgfqpoint{2.438058in}{2.049509in}}%
\pgfpathlineto{\pgfqpoint{2.496771in}{2.049509in}}%
\pgfpathlineto{\pgfqpoint{2.555483in}{2.063235in}}%
\pgfpathlineto{\pgfqpoint{2.614195in}{2.070098in}}%
\pgfpathlineto{\pgfqpoint{2.672907in}{2.070098in}}%
\pgfusepath{stroke}%
\end{pgfscope}%
\begin{pgfscope}%
\pgfpathrectangle{\pgfqpoint{0.453589in}{0.383578in}}{\pgfqpoint{2.325000in}{1.925000in}}%
\pgfusepath{clip}%
\pgfsetrectcap%
\pgfsetroundjoin%
\pgfsetlinewidth{0.803000pt}%
\definecolor{currentstroke}{rgb}{0.686275,0.352941,0.313725}%
\pgfsetstrokecolor{currentstroke}%
\pgfsetdash{}{0pt}%
\pgfpathmoveto{\pgfqpoint{0.559271in}{0.471078in}}%
\pgfpathlineto{\pgfqpoint{0.617983in}{0.471078in}}%
\pgfpathlineto{\pgfqpoint{0.676695in}{0.471078in}}%
\pgfpathlineto{\pgfqpoint{0.735407in}{0.471078in}}%
\pgfpathlineto{\pgfqpoint{0.794119in}{0.471078in}}%
\pgfpathlineto{\pgfqpoint{0.852831in}{0.471078in}}%
\pgfpathlineto{\pgfqpoint{0.911543in}{0.471078in}}%
\pgfpathlineto{\pgfqpoint{0.970255in}{0.477941in}}%
\pgfpathlineto{\pgfqpoint{1.028968in}{0.477941in}}%
\pgfpathlineto{\pgfqpoint{1.087680in}{0.491666in}}%
\pgfpathlineto{\pgfqpoint{1.146392in}{0.574019in}}%
\pgfpathlineto{\pgfqpoint{1.205104in}{0.670098in}}%
\pgfpathlineto{\pgfqpoint{1.263816in}{1.061274in}}%
\pgfpathlineto{\pgfqpoint{1.322528in}{1.287745in}}%
\pgfpathlineto{\pgfqpoint{1.381240in}{1.562255in}}%
\pgfpathlineto{\pgfqpoint{1.439952in}{1.809313in}}%
\pgfpathlineto{\pgfqpoint{1.498665in}{2.049509in}}%
\pgfpathlineto{\pgfqpoint{1.557377in}{2.193627in}}%
\pgfpathlineto{\pgfqpoint{1.616089in}{2.221078in}}%
\pgfpathlineto{\pgfqpoint{1.674801in}{2.221078in}}%
\pgfpathlineto{\pgfqpoint{1.733513in}{2.221078in}}%
\pgfpathlineto{\pgfqpoint{1.792225in}{2.221078in}}%
\pgfpathlineto{\pgfqpoint{1.850937in}{2.221078in}}%
\pgfpathlineto{\pgfqpoint{1.909649in}{2.221078in}}%
\pgfpathlineto{\pgfqpoint{1.968361in}{2.221078in}}%
\pgfpathlineto{\pgfqpoint{2.027074in}{2.221078in}}%
\pgfpathlineto{\pgfqpoint{2.085786in}{2.221078in}}%
\pgfpathlineto{\pgfqpoint{2.144498in}{2.221078in}}%
\pgfpathlineto{\pgfqpoint{2.203210in}{2.221078in}}%
\pgfpathlineto{\pgfqpoint{2.261922in}{2.221078in}}%
\pgfpathlineto{\pgfqpoint{2.320634in}{2.221078in}}%
\pgfpathlineto{\pgfqpoint{2.379346in}{2.221078in}}%
\pgfpathlineto{\pgfqpoint{2.438058in}{2.221078in}}%
\pgfpathlineto{\pgfqpoint{2.496771in}{2.221078in}}%
\pgfpathlineto{\pgfqpoint{2.555483in}{2.221078in}}%
\pgfpathlineto{\pgfqpoint{2.614195in}{2.221078in}}%
\pgfpathlineto{\pgfqpoint{2.672907in}{2.221078in}}%
\pgfusepath{stroke}%
\end{pgfscope}%
\begin{pgfscope}%
\pgfpathrectangle{\pgfqpoint{0.453589in}{0.383578in}}{\pgfqpoint{2.325000in}{1.925000in}}%
\pgfusepath{clip}%
\pgfsetrectcap%
\pgfsetroundjoin%
\pgfsetlinewidth{0.803000pt}%
\definecolor{currentstroke}{rgb}{0.000000,0.356863,0.509804}%
\pgfsetstrokecolor{currentstroke}%
\pgfsetdash{}{0pt}%
\pgfpathmoveto{\pgfqpoint{0.559271in}{0.471078in}}%
\pgfpathlineto{\pgfqpoint{0.617983in}{0.471078in}}%
\pgfpathlineto{\pgfqpoint{0.676695in}{0.471078in}}%
\pgfpathlineto{\pgfqpoint{0.735407in}{0.471078in}}%
\pgfpathlineto{\pgfqpoint{0.794119in}{0.471078in}}%
\pgfpathlineto{\pgfqpoint{0.852831in}{0.471078in}}%
\pgfpathlineto{\pgfqpoint{0.911543in}{0.471078in}}%
\pgfpathlineto{\pgfqpoint{0.970255in}{0.471078in}}%
\pgfpathlineto{\pgfqpoint{1.028968in}{0.471078in}}%
\pgfpathlineto{\pgfqpoint{1.087680in}{0.471078in}}%
\pgfpathlineto{\pgfqpoint{1.146392in}{0.471078in}}%
\pgfpathlineto{\pgfqpoint{1.205104in}{0.471078in}}%
\pgfpathlineto{\pgfqpoint{1.263816in}{0.471078in}}%
\pgfpathlineto{\pgfqpoint{1.322528in}{0.471078in}}%
\pgfpathlineto{\pgfqpoint{1.381240in}{0.471078in}}%
\pgfpathlineto{\pgfqpoint{1.439952in}{0.471078in}}%
\pgfpathlineto{\pgfqpoint{1.498665in}{0.471078in}}%
\pgfpathlineto{\pgfqpoint{1.557377in}{0.471078in}}%
\pgfpathlineto{\pgfqpoint{1.616089in}{0.471078in}}%
\pgfpathlineto{\pgfqpoint{1.674801in}{0.471078in}}%
\pgfpathlineto{\pgfqpoint{1.733513in}{0.491666in}}%
\pgfpathlineto{\pgfqpoint{1.792225in}{0.553431in}}%
\pgfpathlineto{\pgfqpoint{1.850937in}{0.725000in}}%
\pgfpathlineto{\pgfqpoint{1.909649in}{0.972058in}}%
\pgfpathlineto{\pgfqpoint{1.968361in}{1.342647in}}%
\pgfpathlineto{\pgfqpoint{2.027074in}{1.617156in}}%
\pgfpathlineto{\pgfqpoint{2.085786in}{1.775000in}}%
\pgfpathlineto{\pgfqpoint{2.144498in}{1.925980in}}%
\pgfpathlineto{\pgfqpoint{2.203210in}{2.028921in}}%
\pgfpathlineto{\pgfqpoint{2.261922in}{2.076960in}}%
\pgfpathlineto{\pgfqpoint{2.320634in}{2.097549in}}%
\pgfpathlineto{\pgfqpoint{2.379346in}{2.104411in}}%
\pgfpathlineto{\pgfqpoint{2.438058in}{2.118137in}}%
\pgfpathlineto{\pgfqpoint{2.496771in}{2.118137in}}%
\pgfpathlineto{\pgfqpoint{2.555483in}{2.118137in}}%
\pgfpathlineto{\pgfqpoint{2.614195in}{2.125000in}}%
\pgfpathlineto{\pgfqpoint{2.672907in}{2.125000in}}%
\pgfusepath{stroke}%
\end{pgfscope}%
\begin{pgfscope}%
\pgfpathrectangle{\pgfqpoint{0.453589in}{0.383578in}}{\pgfqpoint{2.325000in}{1.925000in}}%
\pgfusepath{clip}%
\pgfsetrectcap%
\pgfsetroundjoin%
\pgfsetlinewidth{0.803000pt}%
\definecolor{currentstroke}{rgb}{0.490196,0.588235,0.431373}%
\pgfsetstrokecolor{currentstroke}%
\pgfsetdash{}{0pt}%
\pgfpathmoveto{\pgfqpoint{0.559271in}{0.471078in}}%
\pgfpathlineto{\pgfqpoint{0.617983in}{0.471078in}}%
\pgfpathlineto{\pgfqpoint{0.676695in}{0.471078in}}%
\pgfpathlineto{\pgfqpoint{0.735407in}{0.471078in}}%
\pgfpathlineto{\pgfqpoint{0.794119in}{0.471078in}}%
\pgfpathlineto{\pgfqpoint{0.852831in}{0.471078in}}%
\pgfpathlineto{\pgfqpoint{0.911543in}{0.471078in}}%
\pgfpathlineto{\pgfqpoint{0.970255in}{0.471078in}}%
\pgfpathlineto{\pgfqpoint{1.028968in}{0.471078in}}%
\pgfpathlineto{\pgfqpoint{1.087680in}{0.471078in}}%
\pgfpathlineto{\pgfqpoint{1.146392in}{0.471078in}}%
\pgfpathlineto{\pgfqpoint{1.205104in}{0.471078in}}%
\pgfpathlineto{\pgfqpoint{1.263816in}{0.471078in}}%
\pgfpathlineto{\pgfqpoint{1.322528in}{0.471078in}}%
\pgfpathlineto{\pgfqpoint{1.381240in}{0.471078in}}%
\pgfpathlineto{\pgfqpoint{1.439952in}{0.471078in}}%
\pgfpathlineto{\pgfqpoint{1.498665in}{0.471078in}}%
\pgfpathlineto{\pgfqpoint{1.557377in}{0.471078in}}%
\pgfpathlineto{\pgfqpoint{1.616089in}{0.471078in}}%
\pgfpathlineto{\pgfqpoint{1.674801in}{0.471078in}}%
\pgfpathlineto{\pgfqpoint{1.733513in}{0.477941in}}%
\pgfpathlineto{\pgfqpoint{1.792225in}{0.512255in}}%
\pgfpathlineto{\pgfqpoint{1.850937in}{0.574019in}}%
\pgfpathlineto{\pgfqpoint{1.909649in}{0.752451in}}%
\pgfpathlineto{\pgfqpoint{1.968361in}{0.972058in}}%
\pgfpathlineto{\pgfqpoint{2.027074in}{1.376960in}}%
\pgfpathlineto{\pgfqpoint{2.085786in}{1.651470in}}%
\pgfpathlineto{\pgfqpoint{2.144498in}{2.056372in}}%
\pgfpathlineto{\pgfqpoint{2.203210in}{2.214215in}}%
\pgfpathlineto{\pgfqpoint{2.261922in}{2.221078in}}%
\pgfpathlineto{\pgfqpoint{2.320634in}{2.221078in}}%
\pgfpathlineto{\pgfqpoint{2.379346in}{2.221078in}}%
\pgfpathlineto{\pgfqpoint{2.438058in}{2.221078in}}%
\pgfpathlineto{\pgfqpoint{2.496771in}{2.221078in}}%
\pgfpathlineto{\pgfqpoint{2.555483in}{2.221078in}}%
\pgfpathlineto{\pgfqpoint{2.614195in}{2.221078in}}%
\pgfpathlineto{\pgfqpoint{2.672907in}{2.221078in}}%
\pgfusepath{stroke}%
\end{pgfscope}%
\begin{pgfscope}%
\pgfpathrectangle{\pgfqpoint{0.453589in}{0.383578in}}{\pgfqpoint{2.325000in}{1.925000in}}%
\pgfusepath{clip}%
\pgfsetrectcap%
\pgfsetroundjoin%
\pgfsetlinewidth{0.803000pt}%
\definecolor{currentstroke}{rgb}{0.843137,0.666667,0.313725}%
\pgfsetstrokecolor{currentstroke}%
\pgfsetdash{}{0pt}%
\pgfpathmoveto{\pgfqpoint{0.559271in}{0.471078in}}%
\pgfpathlineto{\pgfqpoint{0.617983in}{0.471078in}}%
\pgfpathlineto{\pgfqpoint{0.676695in}{0.471078in}}%
\pgfpathlineto{\pgfqpoint{0.735407in}{0.471078in}}%
\pgfpathlineto{\pgfqpoint{0.794119in}{0.471078in}}%
\pgfpathlineto{\pgfqpoint{0.852831in}{0.471078in}}%
\pgfpathlineto{\pgfqpoint{0.911543in}{0.471078in}}%
\pgfpathlineto{\pgfqpoint{0.970255in}{0.471078in}}%
\pgfpathlineto{\pgfqpoint{1.028968in}{0.477941in}}%
\pgfpathlineto{\pgfqpoint{1.087680in}{0.491666in}}%
\pgfpathlineto{\pgfqpoint{1.146392in}{0.553431in}}%
\pgfpathlineto{\pgfqpoint{1.205104in}{0.642647in}}%
\pgfpathlineto{\pgfqpoint{1.263816in}{1.033823in}}%
\pgfpathlineto{\pgfqpoint{1.322528in}{1.225980in}}%
\pgfpathlineto{\pgfqpoint{1.381240in}{1.438725in}}%
\pgfpathlineto{\pgfqpoint{1.439952in}{1.672058in}}%
\pgfpathlineto{\pgfqpoint{1.498665in}{1.871078in}}%
\pgfpathlineto{\pgfqpoint{1.557377in}{1.980882in}}%
\pgfpathlineto{\pgfqpoint{1.616089in}{2.035784in}}%
\pgfpathlineto{\pgfqpoint{1.674801in}{2.049509in}}%
\pgfpathlineto{\pgfqpoint{1.733513in}{2.049509in}}%
\pgfpathlineto{\pgfqpoint{1.792225in}{2.063235in}}%
\pgfpathlineto{\pgfqpoint{1.850937in}{2.063235in}}%
\pgfpathlineto{\pgfqpoint{1.909649in}{2.070098in}}%
\pgfpathlineto{\pgfqpoint{1.968361in}{2.076960in}}%
\pgfpathlineto{\pgfqpoint{2.027074in}{2.083823in}}%
\pgfpathlineto{\pgfqpoint{2.085786in}{2.083823in}}%
\pgfpathlineto{\pgfqpoint{2.144498in}{2.083823in}}%
\pgfpathlineto{\pgfqpoint{2.203210in}{2.097549in}}%
\pgfpathlineto{\pgfqpoint{2.261922in}{2.097549in}}%
\pgfpathlineto{\pgfqpoint{2.320634in}{2.104411in}}%
\pgfpathlineto{\pgfqpoint{2.379346in}{2.104411in}}%
\pgfpathlineto{\pgfqpoint{2.438058in}{2.104411in}}%
\pgfpathlineto{\pgfqpoint{2.496771in}{2.104411in}}%
\pgfpathlineto{\pgfqpoint{2.555483in}{2.097549in}}%
\pgfpathlineto{\pgfqpoint{2.614195in}{2.104411in}}%
\pgfpathlineto{\pgfqpoint{2.672907in}{2.104411in}}%
\pgfusepath{stroke}%
\end{pgfscope}%
\begin{pgfscope}%
\pgfpathrectangle{\pgfqpoint{0.453589in}{0.383578in}}{\pgfqpoint{2.325000in}{1.925000in}}%
\pgfusepath{clip}%
\pgfsetrectcap%
\pgfsetroundjoin%
\pgfsetlinewidth{0.803000pt}%
\definecolor{currentstroke}{rgb}{0.333333,0.333333,0.333333}%
\pgfsetstrokecolor{currentstroke}%
\pgfsetdash{}{0pt}%
\pgfpathmoveto{\pgfqpoint{0.559271in}{0.471078in}}%
\pgfpathlineto{\pgfqpoint{0.617983in}{0.471078in}}%
\pgfpathlineto{\pgfqpoint{0.676695in}{0.471078in}}%
\pgfpathlineto{\pgfqpoint{0.735407in}{0.471078in}}%
\pgfpathlineto{\pgfqpoint{0.794119in}{0.471078in}}%
\pgfpathlineto{\pgfqpoint{0.852831in}{0.471078in}}%
\pgfpathlineto{\pgfqpoint{0.911543in}{0.471078in}}%
\pgfpathlineto{\pgfqpoint{0.970255in}{0.471078in}}%
\pgfpathlineto{\pgfqpoint{1.028968in}{0.471078in}}%
\pgfpathlineto{\pgfqpoint{1.087680in}{0.471078in}}%
\pgfpathlineto{\pgfqpoint{1.146392in}{0.471078in}}%
\pgfpathlineto{\pgfqpoint{1.205104in}{0.471078in}}%
\pgfpathlineto{\pgfqpoint{1.263816in}{0.471078in}}%
\pgfpathlineto{\pgfqpoint{1.322528in}{0.471078in}}%
\pgfpathlineto{\pgfqpoint{1.381240in}{0.477941in}}%
\pgfpathlineto{\pgfqpoint{1.439952in}{0.471078in}}%
\pgfpathlineto{\pgfqpoint{1.498665in}{0.498529in}}%
\pgfpathlineto{\pgfqpoint{1.557377in}{0.553431in}}%
\pgfpathlineto{\pgfqpoint{1.616089in}{0.608333in}}%
\pgfpathlineto{\pgfqpoint{1.674801in}{1.239706in}}%
\pgfpathlineto{\pgfqpoint{1.733513in}{1.610294in}}%
\pgfpathlineto{\pgfqpoint{1.792225in}{1.967156in}}%
\pgfpathlineto{\pgfqpoint{1.850937in}{2.214215in}}%
\pgfpathlineto{\pgfqpoint{1.909649in}{2.221078in}}%
\pgfpathlineto{\pgfqpoint{1.968361in}{2.221078in}}%
\pgfpathlineto{\pgfqpoint{2.027074in}{2.221078in}}%
\pgfpathlineto{\pgfqpoint{2.085786in}{2.221078in}}%
\pgfpathlineto{\pgfqpoint{2.144498in}{2.221078in}}%
\pgfpathlineto{\pgfqpoint{2.203210in}{2.221078in}}%
\pgfpathlineto{\pgfqpoint{2.261922in}{2.221078in}}%
\pgfpathlineto{\pgfqpoint{2.320634in}{2.221078in}}%
\pgfpathlineto{\pgfqpoint{2.379346in}{2.221078in}}%
\pgfpathlineto{\pgfqpoint{2.438058in}{2.221078in}}%
\pgfpathlineto{\pgfqpoint{2.496771in}{2.221078in}}%
\pgfpathlineto{\pgfqpoint{2.555483in}{2.221078in}}%
\pgfpathlineto{\pgfqpoint{2.614195in}{2.221078in}}%
\pgfpathlineto{\pgfqpoint{2.672907in}{2.221078in}}%
\pgfusepath{stroke}%
\end{pgfscope}%
\begin{pgfscope}%
\pgfpathrectangle{\pgfqpoint{0.453589in}{0.383578in}}{\pgfqpoint{2.325000in}{1.925000in}}%
\pgfusepath{clip}%
\pgfsetrectcap%
\pgfsetroundjoin%
\pgfsetlinewidth{0.803000pt}%
\definecolor{currentstroke}{rgb}{0.686275,0.352941,0.313725}%
\pgfsetstrokecolor{currentstroke}%
\pgfsetdash{}{0pt}%
\pgfpathmoveto{\pgfqpoint{0.559271in}{0.471078in}}%
\pgfpathlineto{\pgfqpoint{0.617983in}{0.471078in}}%
\pgfpathlineto{\pgfqpoint{0.676695in}{0.471078in}}%
\pgfpathlineto{\pgfqpoint{0.735407in}{0.471078in}}%
\pgfpathlineto{\pgfqpoint{0.794119in}{0.471078in}}%
\pgfpathlineto{\pgfqpoint{0.852831in}{0.471078in}}%
\pgfpathlineto{\pgfqpoint{0.911543in}{0.471078in}}%
\pgfpathlineto{\pgfqpoint{0.970255in}{0.471078in}}%
\pgfpathlineto{\pgfqpoint{1.028968in}{0.471078in}}%
\pgfpathlineto{\pgfqpoint{1.087680in}{0.471078in}}%
\pgfpathlineto{\pgfqpoint{1.146392in}{0.471078in}}%
\pgfpathlineto{\pgfqpoint{1.205104in}{0.471078in}}%
\pgfpathlineto{\pgfqpoint{1.263816in}{0.471078in}}%
\pgfpathlineto{\pgfqpoint{1.322528in}{0.471078in}}%
\pgfpathlineto{\pgfqpoint{1.381240in}{0.471078in}}%
\pgfpathlineto{\pgfqpoint{1.439952in}{0.471078in}}%
\pgfpathlineto{\pgfqpoint{1.498665in}{0.471078in}}%
\pgfpathlineto{\pgfqpoint{1.557377in}{0.484804in}}%
\pgfpathlineto{\pgfqpoint{1.616089in}{0.477941in}}%
\pgfpathlineto{\pgfqpoint{1.674801in}{0.766176in}}%
\pgfpathlineto{\pgfqpoint{1.733513in}{1.081862in}}%
\pgfpathlineto{\pgfqpoint{1.792225in}{1.514215in}}%
\pgfpathlineto{\pgfqpoint{1.850937in}{1.788725in}}%
\pgfpathlineto{\pgfqpoint{1.909649in}{1.932843in}}%
\pgfpathlineto{\pgfqpoint{1.968361in}{2.015196in}}%
\pgfpathlineto{\pgfqpoint{2.027074in}{2.090686in}}%
\pgfpathlineto{\pgfqpoint{2.085786in}{2.125000in}}%
\pgfpathlineto{\pgfqpoint{2.144498in}{2.145588in}}%
\pgfpathlineto{\pgfqpoint{2.203210in}{2.159313in}}%
\pgfpathlineto{\pgfqpoint{2.261922in}{2.166176in}}%
\pgfpathlineto{\pgfqpoint{2.320634in}{2.173039in}}%
\pgfpathlineto{\pgfqpoint{2.379346in}{2.179902in}}%
\pgfpathlineto{\pgfqpoint{2.438058in}{2.179902in}}%
\pgfpathlineto{\pgfqpoint{2.496771in}{2.179902in}}%
\pgfpathlineto{\pgfqpoint{2.555483in}{2.179902in}}%
\pgfpathlineto{\pgfqpoint{2.614195in}{2.186764in}}%
\pgfpathlineto{\pgfqpoint{2.672907in}{2.179902in}}%
\pgfusepath{stroke}%
\end{pgfscope}%
\begin{pgfscope}%
\pgfpathrectangle{\pgfqpoint{0.453589in}{0.383578in}}{\pgfqpoint{2.325000in}{1.925000in}}%
\pgfusepath{clip}%
\pgfsetrectcap%
\pgfsetroundjoin%
\pgfsetlinewidth{0.803000pt}%
\definecolor{currentstroke}{rgb}{0.000000,0.356863,0.509804}%
\pgfsetstrokecolor{currentstroke}%
\pgfsetdash{}{0pt}%
\pgfpathmoveto{\pgfqpoint{0.559271in}{0.471078in}}%
\pgfpathlineto{\pgfqpoint{0.617983in}{0.471078in}}%
\pgfpathlineto{\pgfqpoint{0.676695in}{0.471078in}}%
\pgfpathlineto{\pgfqpoint{0.735407in}{0.471078in}}%
\pgfpathlineto{\pgfqpoint{0.794119in}{0.471078in}}%
\pgfpathlineto{\pgfqpoint{0.852831in}{0.471078in}}%
\pgfpathlineto{\pgfqpoint{0.911543in}{0.471078in}}%
\pgfpathlineto{\pgfqpoint{0.970255in}{0.471078in}}%
\pgfpathlineto{\pgfqpoint{1.028968in}{0.471078in}}%
\pgfpathlineto{\pgfqpoint{1.087680in}{0.471078in}}%
\pgfpathlineto{\pgfqpoint{1.146392in}{0.471078in}}%
\pgfpathlineto{\pgfqpoint{1.205104in}{0.471078in}}%
\pgfpathlineto{\pgfqpoint{1.263816in}{0.471078in}}%
\pgfpathlineto{\pgfqpoint{1.322528in}{0.471078in}}%
\pgfpathlineto{\pgfqpoint{1.381240in}{0.471078in}}%
\pgfpathlineto{\pgfqpoint{1.439952in}{0.471078in}}%
\pgfpathlineto{\pgfqpoint{1.498665in}{0.471078in}}%
\pgfpathlineto{\pgfqpoint{1.557377in}{0.471078in}}%
\pgfpathlineto{\pgfqpoint{1.616089in}{0.471078in}}%
\pgfpathlineto{\pgfqpoint{1.674801in}{0.471078in}}%
\pgfpathlineto{\pgfqpoint{1.733513in}{0.471078in}}%
\pgfpathlineto{\pgfqpoint{1.792225in}{0.471078in}}%
\pgfpathlineto{\pgfqpoint{1.850937in}{0.471078in}}%
\pgfpathlineto{\pgfqpoint{1.909649in}{0.512255in}}%
\pgfpathlineto{\pgfqpoint{1.968361in}{0.574019in}}%
\pgfpathlineto{\pgfqpoint{2.027074in}{0.731862in}}%
\pgfpathlineto{\pgfqpoint{2.085786in}{0.855392in}}%
\pgfpathlineto{\pgfqpoint{2.144498in}{1.205392in}}%
\pgfpathlineto{\pgfqpoint{2.203210in}{1.452451in}}%
\pgfpathlineto{\pgfqpoint{2.261922in}{1.713235in}}%
\pgfpathlineto{\pgfqpoint{2.320634in}{1.843627in}}%
\pgfpathlineto{\pgfqpoint{2.379346in}{1.919117in}}%
\pgfpathlineto{\pgfqpoint{2.438058in}{1.946568in}}%
\pgfpathlineto{\pgfqpoint{2.496771in}{1.953431in}}%
\pgfpathlineto{\pgfqpoint{2.555483in}{1.960294in}}%
\pgfpathlineto{\pgfqpoint{2.614195in}{1.960294in}}%
\pgfpathlineto{\pgfqpoint{2.672907in}{1.967156in}}%
\pgfusepath{stroke}%
\end{pgfscope}%
\begin{pgfscope}%
\pgfpathrectangle{\pgfqpoint{0.453589in}{0.383578in}}{\pgfqpoint{2.325000in}{1.925000in}}%
\pgfusepath{clip}%
\pgfsetrectcap%
\pgfsetroundjoin%
\pgfsetlinewidth{0.803000pt}%
\definecolor{currentstroke}{rgb}{0.490196,0.588235,0.431373}%
\pgfsetstrokecolor{currentstroke}%
\pgfsetdash{}{0pt}%
\pgfpathmoveto{\pgfqpoint{0.559271in}{0.471078in}}%
\pgfpathlineto{\pgfqpoint{0.617983in}{0.471078in}}%
\pgfpathlineto{\pgfqpoint{0.676695in}{0.471078in}}%
\pgfpathlineto{\pgfqpoint{0.735407in}{0.471078in}}%
\pgfpathlineto{\pgfqpoint{0.794119in}{0.471078in}}%
\pgfpathlineto{\pgfqpoint{0.852831in}{0.471078in}}%
\pgfpathlineto{\pgfqpoint{0.911543in}{0.471078in}}%
\pgfpathlineto{\pgfqpoint{0.970255in}{0.471078in}}%
\pgfpathlineto{\pgfqpoint{1.028968in}{0.471078in}}%
\pgfpathlineto{\pgfqpoint{1.087680in}{0.471078in}}%
\pgfpathlineto{\pgfqpoint{1.146392in}{0.477941in}}%
\pgfpathlineto{\pgfqpoint{1.205104in}{0.477941in}}%
\pgfpathlineto{\pgfqpoint{1.263816in}{0.532843in}}%
\pgfpathlineto{\pgfqpoint{1.322528in}{0.676960in}}%
\pgfpathlineto{\pgfqpoint{1.381240in}{0.807353in}}%
\pgfpathlineto{\pgfqpoint{1.439952in}{0.958333in}}%
\pgfpathlineto{\pgfqpoint{1.498665in}{1.198529in}}%
\pgfpathlineto{\pgfqpoint{1.557377in}{1.548529in}}%
\pgfpathlineto{\pgfqpoint{1.616089in}{1.925980in}}%
\pgfpathlineto{\pgfqpoint{1.674801in}{2.056372in}}%
\pgfpathlineto{\pgfqpoint{1.733513in}{2.104411in}}%
\pgfpathlineto{\pgfqpoint{1.792225in}{2.145588in}}%
\pgfpathlineto{\pgfqpoint{1.850937in}{2.159313in}}%
\pgfpathlineto{\pgfqpoint{1.909649in}{2.179902in}}%
\pgfpathlineto{\pgfqpoint{1.968361in}{2.193627in}}%
\pgfpathlineto{\pgfqpoint{2.027074in}{2.200490in}}%
\pgfpathlineto{\pgfqpoint{2.085786in}{2.207353in}}%
\pgfpathlineto{\pgfqpoint{2.144498in}{2.214215in}}%
\pgfpathlineto{\pgfqpoint{2.203210in}{2.221078in}}%
\pgfpathlineto{\pgfqpoint{2.261922in}{2.221078in}}%
\pgfpathlineto{\pgfqpoint{2.320634in}{2.221078in}}%
\pgfpathlineto{\pgfqpoint{2.379346in}{2.221078in}}%
\pgfpathlineto{\pgfqpoint{2.438058in}{2.221078in}}%
\pgfpathlineto{\pgfqpoint{2.496771in}{2.221078in}}%
\pgfpathlineto{\pgfqpoint{2.555483in}{2.221078in}}%
\pgfpathlineto{\pgfqpoint{2.614195in}{2.221078in}}%
\pgfpathlineto{\pgfqpoint{2.672907in}{2.221078in}}%
\pgfusepath{stroke}%
\end{pgfscope}%
\begin{pgfscope}%
\pgfpathrectangle{\pgfqpoint{0.453589in}{0.383578in}}{\pgfqpoint{2.325000in}{1.925000in}}%
\pgfusepath{clip}%
\pgfsetrectcap%
\pgfsetroundjoin%
\pgfsetlinewidth{0.803000pt}%
\definecolor{currentstroke}{rgb}{0.843137,0.666667,0.313725}%
\pgfsetstrokecolor{currentstroke}%
\pgfsetdash{}{0pt}%
\pgfpathmoveto{\pgfqpoint{0.559271in}{0.471078in}}%
\pgfpathlineto{\pgfqpoint{0.617983in}{0.471078in}}%
\pgfpathlineto{\pgfqpoint{0.676695in}{0.471078in}}%
\pgfpathlineto{\pgfqpoint{0.735407in}{0.471078in}}%
\pgfpathlineto{\pgfqpoint{0.794119in}{0.471078in}}%
\pgfpathlineto{\pgfqpoint{0.852831in}{0.491666in}}%
\pgfpathlineto{\pgfqpoint{0.911543in}{0.532843in}}%
\pgfpathlineto{\pgfqpoint{0.970255in}{0.718137in}}%
\pgfpathlineto{\pgfqpoint{1.028968in}{0.855392in}}%
\pgfpathlineto{\pgfqpoint{1.087680in}{1.184804in}}%
\pgfpathlineto{\pgfqpoint{1.146392in}{1.411274in}}%
\pgfpathlineto{\pgfqpoint{1.205104in}{1.624019in}}%
\pgfpathlineto{\pgfqpoint{1.263816in}{1.919117in}}%
\pgfpathlineto{\pgfqpoint{1.322528in}{1.960294in}}%
\pgfpathlineto{\pgfqpoint{1.381240in}{2.022058in}}%
\pgfpathlineto{\pgfqpoint{1.439952in}{2.056372in}}%
\pgfpathlineto{\pgfqpoint{1.498665in}{2.070098in}}%
\pgfpathlineto{\pgfqpoint{1.557377in}{2.083823in}}%
\pgfpathlineto{\pgfqpoint{1.616089in}{2.097549in}}%
\pgfpathlineto{\pgfqpoint{1.674801in}{2.104411in}}%
\pgfpathlineto{\pgfqpoint{1.733513in}{2.104411in}}%
\pgfpathlineto{\pgfqpoint{1.792225in}{2.104411in}}%
\pgfpathlineto{\pgfqpoint{1.850937in}{2.111274in}}%
\pgfpathlineto{\pgfqpoint{1.909649in}{2.111274in}}%
\pgfpathlineto{\pgfqpoint{1.968361in}{2.111274in}}%
\pgfpathlineto{\pgfqpoint{2.027074in}{2.118137in}}%
\pgfpathlineto{\pgfqpoint{2.085786in}{2.118137in}}%
\pgfpathlineto{\pgfqpoint{2.144498in}{2.125000in}}%
\pgfpathlineto{\pgfqpoint{2.203210in}{2.125000in}}%
\pgfpathlineto{\pgfqpoint{2.261922in}{2.125000in}}%
\pgfpathlineto{\pgfqpoint{2.320634in}{2.131862in}}%
\pgfpathlineto{\pgfqpoint{2.379346in}{2.125000in}}%
\pgfpathlineto{\pgfqpoint{2.438058in}{2.125000in}}%
\pgfpathlineto{\pgfqpoint{2.496771in}{2.125000in}}%
\pgfpathlineto{\pgfqpoint{2.555483in}{2.125000in}}%
\pgfpathlineto{\pgfqpoint{2.614195in}{2.125000in}}%
\pgfpathlineto{\pgfqpoint{2.672907in}{2.125000in}}%
\pgfusepath{stroke}%
\end{pgfscope}%
\begin{pgfscope}%
\pgfpathrectangle{\pgfqpoint{0.453589in}{0.383578in}}{\pgfqpoint{2.325000in}{1.925000in}}%
\pgfusepath{clip}%
\pgfsetrectcap%
\pgfsetroundjoin%
\pgfsetlinewidth{0.803000pt}%
\definecolor{currentstroke}{rgb}{0.333333,0.333333,0.333333}%
\pgfsetstrokecolor{currentstroke}%
\pgfsetdash{}{0pt}%
\pgfpathmoveto{\pgfqpoint{0.559271in}{0.477941in}}%
\pgfpathlineto{\pgfqpoint{0.617983in}{0.471078in}}%
\pgfpathlineto{\pgfqpoint{0.676695in}{0.471078in}}%
\pgfpathlineto{\pgfqpoint{0.735407in}{0.471078in}}%
\pgfpathlineto{\pgfqpoint{0.794119in}{0.471078in}}%
\pgfpathlineto{\pgfqpoint{0.852831in}{0.471078in}}%
\pgfpathlineto{\pgfqpoint{0.911543in}{0.477941in}}%
\pgfpathlineto{\pgfqpoint{0.970255in}{0.498529in}}%
\pgfpathlineto{\pgfqpoint{1.028968in}{0.532843in}}%
\pgfpathlineto{\pgfqpoint{1.087680in}{0.608333in}}%
\pgfpathlineto{\pgfqpoint{1.146392in}{0.731862in}}%
\pgfpathlineto{\pgfqpoint{1.205104in}{0.862255in}}%
\pgfpathlineto{\pgfqpoint{1.263816in}{1.397549in}}%
\pgfpathlineto{\pgfqpoint{1.322528in}{1.651470in}}%
\pgfpathlineto{\pgfqpoint{1.381240in}{1.877941in}}%
\pgfpathlineto{\pgfqpoint{1.439952in}{2.008333in}}%
\pgfpathlineto{\pgfqpoint{1.498665in}{2.097549in}}%
\pgfpathlineto{\pgfqpoint{1.557377in}{2.152451in}}%
\pgfpathlineto{\pgfqpoint{1.616089in}{2.179902in}}%
\pgfpathlineto{\pgfqpoint{1.674801in}{2.186764in}}%
\pgfpathlineto{\pgfqpoint{1.733513in}{2.193627in}}%
\pgfpathlineto{\pgfqpoint{1.792225in}{2.207353in}}%
\pgfpathlineto{\pgfqpoint{1.850937in}{2.207353in}}%
\pgfpathlineto{\pgfqpoint{1.909649in}{2.214215in}}%
\pgfpathlineto{\pgfqpoint{1.968361in}{2.214215in}}%
\pgfpathlineto{\pgfqpoint{2.027074in}{2.221078in}}%
\pgfpathlineto{\pgfqpoint{2.085786in}{2.221078in}}%
\pgfpathlineto{\pgfqpoint{2.144498in}{2.221078in}}%
\pgfpathlineto{\pgfqpoint{2.203210in}{2.221078in}}%
\pgfpathlineto{\pgfqpoint{2.261922in}{2.221078in}}%
\pgfpathlineto{\pgfqpoint{2.320634in}{2.221078in}}%
\pgfpathlineto{\pgfqpoint{2.379346in}{2.221078in}}%
\pgfpathlineto{\pgfqpoint{2.438058in}{2.221078in}}%
\pgfpathlineto{\pgfqpoint{2.496771in}{2.221078in}}%
\pgfpathlineto{\pgfqpoint{2.555483in}{2.221078in}}%
\pgfpathlineto{\pgfqpoint{2.614195in}{2.221078in}}%
\pgfpathlineto{\pgfqpoint{2.672907in}{2.221078in}}%
\pgfusepath{stroke}%
\end{pgfscope}%
\begin{pgfscope}%
\pgfpathrectangle{\pgfqpoint{0.453589in}{0.383578in}}{\pgfqpoint{2.325000in}{1.925000in}}%
\pgfusepath{clip}%
\pgfsetrectcap%
\pgfsetroundjoin%
\pgfsetlinewidth{0.803000pt}%
\definecolor{currentstroke}{rgb}{0.686275,0.352941,0.313725}%
\pgfsetstrokecolor{currentstroke}%
\pgfsetdash{}{0pt}%
\pgfpathmoveto{\pgfqpoint{0.559271in}{0.471078in}}%
\pgfpathlineto{\pgfqpoint{0.617983in}{0.471078in}}%
\pgfpathlineto{\pgfqpoint{0.676695in}{0.471078in}}%
\pgfpathlineto{\pgfqpoint{0.735407in}{0.471078in}}%
\pgfpathlineto{\pgfqpoint{0.794119in}{0.471078in}}%
\pgfpathlineto{\pgfqpoint{0.852831in}{0.471078in}}%
\pgfpathlineto{\pgfqpoint{0.911543in}{0.471078in}}%
\pgfpathlineto{\pgfqpoint{0.970255in}{0.471078in}}%
\pgfpathlineto{\pgfqpoint{1.028968in}{0.471078in}}%
\pgfpathlineto{\pgfqpoint{1.087680in}{0.471078in}}%
\pgfpathlineto{\pgfqpoint{1.146392in}{0.471078in}}%
\pgfpathlineto{\pgfqpoint{1.205104in}{0.471078in}}%
\pgfpathlineto{\pgfqpoint{1.263816in}{0.471078in}}%
\pgfpathlineto{\pgfqpoint{1.322528in}{0.471078in}}%
\pgfpathlineto{\pgfqpoint{1.381240in}{0.471078in}}%
\pgfpathlineto{\pgfqpoint{1.439952in}{0.471078in}}%
\pgfpathlineto{\pgfqpoint{1.498665in}{0.471078in}}%
\pgfpathlineto{\pgfqpoint{1.557377in}{0.471078in}}%
\pgfpathlineto{\pgfqpoint{1.616089in}{0.477941in}}%
\pgfpathlineto{\pgfqpoint{1.674801in}{0.800490in}}%
\pgfpathlineto{\pgfqpoint{1.733513in}{1.157353in}}%
\pgfpathlineto{\pgfqpoint{1.792225in}{1.486764in}}%
\pgfpathlineto{\pgfqpoint{1.850937in}{1.809313in}}%
\pgfpathlineto{\pgfqpoint{1.909649in}{2.001470in}}%
\pgfpathlineto{\pgfqpoint{1.968361in}{2.070098in}}%
\pgfpathlineto{\pgfqpoint{2.027074in}{2.159313in}}%
\pgfpathlineto{\pgfqpoint{2.085786in}{2.193627in}}%
\pgfpathlineto{\pgfqpoint{2.144498in}{2.214215in}}%
\pgfpathlineto{\pgfqpoint{2.203210in}{2.221078in}}%
\pgfpathlineto{\pgfqpoint{2.261922in}{2.221078in}}%
\pgfpathlineto{\pgfqpoint{2.320634in}{2.221078in}}%
\pgfpathlineto{\pgfqpoint{2.379346in}{2.221078in}}%
\pgfpathlineto{\pgfqpoint{2.438058in}{2.221078in}}%
\pgfpathlineto{\pgfqpoint{2.496771in}{2.221078in}}%
\pgfpathlineto{\pgfqpoint{2.555483in}{2.221078in}}%
\pgfpathlineto{\pgfqpoint{2.614195in}{2.221078in}}%
\pgfpathlineto{\pgfqpoint{2.672907in}{2.221078in}}%
\pgfusepath{stroke}%
\end{pgfscope}%
\begin{pgfscope}%
\pgfsetrectcap%
\pgfsetmiterjoin%
\pgfsetlinewidth{0.501875pt}%
\definecolor{currentstroke}{rgb}{0.317647,0.317647,0.317647}%
\pgfsetstrokecolor{currentstroke}%
\pgfsetdash{}{0pt}%
\pgfpathmoveto{\pgfqpoint{0.453589in}{0.383578in}}%
\pgfpathlineto{\pgfqpoint{0.453589in}{2.308578in}}%
\pgfusepath{stroke}%
\end{pgfscope}%
\begin{pgfscope}%
\pgfsetrectcap%
\pgfsetmiterjoin%
\pgfsetlinewidth{0.501875pt}%
\definecolor{currentstroke}{rgb}{0.317647,0.317647,0.317647}%
\pgfsetstrokecolor{currentstroke}%
\pgfsetdash{}{0pt}%
\pgfpathmoveto{\pgfqpoint{0.453589in}{0.383578in}}%
\pgfpathlineto{\pgfqpoint{2.778589in}{0.383578in}}%
\pgfusepath{stroke}%
\end{pgfscope}%
\end{pgfpicture}%
\makeatother%
\endgroup%

		\label{transferfunction_wout_calib}
	\end{subfigure}
	\begin{subfigure}[b]{0.49\textwidth}		
		\caption{}
		%% Creator: Matplotlib, PGF backend
%%
%% To include the figure in your LaTeX document, write
%%   \input{<filename>.pgf}
%%
%% Make sure the required packages are loaded in your preamble
%%   \usepackage{pgf}
%%
%% Figures using additional raster images can only be included by \input if
%% they are in the same directory as the main LaTeX file. For loading figures
%% from other directories you can use the `import` package
%%   \usepackage{import}
%% and then include the figures with
%%   \import{<path to file>}{<filename>.pgf}
%%
%% Matplotlib used the following preamble
%%   \usepackage{amsmath} \usepackage{pifont} \usepackage{xcolor} \definecolor{green}{HTML}{467821} \definecolor{red}{HTML}{CF4457} \usepackage[detect-all]{siunitx}
%%   \usepackage{fontspec}
%%
\begingroup%
\makeatletter%
\begin{pgfpicture}%
\pgfpathrectangle{\pgfpointorigin}{\pgfqpoint{2.933591in}{2.795730in}}%
\pgfusepath{use as bounding box, clip}%
\begin{pgfscope}%
\pgfsetbuttcap%
\pgfsetmiterjoin%
\pgfsetlinewidth{0.000000pt}%
\definecolor{currentstroke}{rgb}{0.000000,0.000000,0.000000}%
\pgfsetstrokecolor{currentstroke}%
\pgfsetstrokeopacity{0.000000}%
\pgfsetdash{}{0pt}%
\pgfpathmoveto{\pgfqpoint{0.000000in}{0.000000in}}%
\pgfpathlineto{\pgfqpoint{2.933591in}{0.000000in}}%
\pgfpathlineto{\pgfqpoint{2.933591in}{2.795730in}}%
\pgfpathlineto{\pgfqpoint{0.000000in}{2.795730in}}%
\pgfpathclose%
\pgfusepath{}%
\end{pgfscope}%
\begin{pgfscope}%
\pgfsetbuttcap%
\pgfsetmiterjoin%
\pgfsetlinewidth{0.000000pt}%
\definecolor{currentstroke}{rgb}{0.000000,0.000000,0.000000}%
\pgfsetstrokecolor{currentstroke}%
\pgfsetstrokeopacity{0.000000}%
\pgfsetdash{}{0pt}%
\pgfpathmoveto{\pgfqpoint{0.455741in}{0.385730in}}%
\pgfpathlineto{\pgfqpoint{2.780741in}{0.385730in}}%
\pgfpathlineto{\pgfqpoint{2.780741in}{2.695730in}}%
\pgfpathlineto{\pgfqpoint{0.455741in}{2.695730in}}%
\pgfpathclose%
\pgfusepath{}%
\end{pgfscope}%
\begin{pgfscope}%
\pgfsetbuttcap%
\pgfsetroundjoin%
\definecolor{currentfill}{rgb}{0.317647,0.317647,0.317647}%
\pgfsetfillcolor{currentfill}%
\pgfsetlinewidth{0.501875pt}%
\definecolor{currentstroke}{rgb}{0.317647,0.317647,0.317647}%
\pgfsetstrokecolor{currentstroke}%
\pgfsetdash{}{0pt}%
\pgfsys@defobject{currentmarker}{\pgfqpoint{0.000000in}{-0.020833in}}{\pgfqpoint{0.000000in}{0.000000in}}{%
\pgfpathmoveto{\pgfqpoint{0.000000in}{0.000000in}}%
\pgfpathlineto{\pgfqpoint{0.000000in}{-0.020833in}}%
\pgfusepath{stroke,fill}%
}%
\begin{pgfscope}%
\pgfsys@transformshift{0.485936in}{0.385730in}%
\pgfsys@useobject{currentmarker}{}%
\end{pgfscope}%
\end{pgfscope}%
\begin{pgfscope}%
\definecolor{textcolor}{rgb}{0.317647,0.317647,0.317647}%
\pgfsetstrokecolor{textcolor}%
\pgfsetfillcolor{textcolor}%
\pgftext[x=0.485936in,y=0.337119in,,top]{\color{textcolor}\rmfamily\fontsize{6.664000}{7.996800}\selectfont \(\displaystyle -600\)}%
\end{pgfscope}%
\begin{pgfscope}%
\pgfsetbuttcap%
\pgfsetroundjoin%
\definecolor{currentfill}{rgb}{0.317647,0.317647,0.317647}%
\pgfsetfillcolor{currentfill}%
\pgfsetlinewidth{0.501875pt}%
\definecolor{currentstroke}{rgb}{0.317647,0.317647,0.317647}%
\pgfsetstrokecolor{currentstroke}%
\pgfsetdash{}{0pt}%
\pgfsys@defobject{currentmarker}{\pgfqpoint{0.000000in}{-0.020833in}}{\pgfqpoint{0.000000in}{0.000000in}}{%
\pgfpathmoveto{\pgfqpoint{0.000000in}{0.000000in}}%
\pgfpathlineto{\pgfqpoint{0.000000in}{-0.020833in}}%
\pgfusepath{stroke,fill}%
}%
\begin{pgfscope}%
\pgfsys@transformshift{0.863371in}{0.385730in}%
\pgfsys@useobject{currentmarker}{}%
\end{pgfscope}%
\end{pgfscope}%
\begin{pgfscope}%
\definecolor{textcolor}{rgb}{0.317647,0.317647,0.317647}%
\pgfsetstrokecolor{textcolor}%
\pgfsetfillcolor{textcolor}%
\pgftext[x=0.863371in,y=0.337119in,,top]{\color{textcolor}\rmfamily\fontsize{6.664000}{7.996800}\selectfont \(\displaystyle -400\)}%
\end{pgfscope}%
\begin{pgfscope}%
\pgfsetbuttcap%
\pgfsetroundjoin%
\definecolor{currentfill}{rgb}{0.317647,0.317647,0.317647}%
\pgfsetfillcolor{currentfill}%
\pgfsetlinewidth{0.501875pt}%
\definecolor{currentstroke}{rgb}{0.317647,0.317647,0.317647}%
\pgfsetstrokecolor{currentstroke}%
\pgfsetdash{}{0pt}%
\pgfsys@defobject{currentmarker}{\pgfqpoint{0.000000in}{-0.020833in}}{\pgfqpoint{0.000000in}{0.000000in}}{%
\pgfpathmoveto{\pgfqpoint{0.000000in}{0.000000in}}%
\pgfpathlineto{\pgfqpoint{0.000000in}{-0.020833in}}%
\pgfusepath{stroke,fill}%
}%
\begin{pgfscope}%
\pgfsys@transformshift{1.240806in}{0.385730in}%
\pgfsys@useobject{currentmarker}{}%
\end{pgfscope}%
\end{pgfscope}%
\begin{pgfscope}%
\definecolor{textcolor}{rgb}{0.317647,0.317647,0.317647}%
\pgfsetstrokecolor{textcolor}%
\pgfsetfillcolor{textcolor}%
\pgftext[x=1.240806in,y=0.337119in,,top]{\color{textcolor}\rmfamily\fontsize{6.664000}{7.996800}\selectfont \(\displaystyle -200\)}%
\end{pgfscope}%
\begin{pgfscope}%
\pgfsetbuttcap%
\pgfsetroundjoin%
\definecolor{currentfill}{rgb}{0.317647,0.317647,0.317647}%
\pgfsetfillcolor{currentfill}%
\pgfsetlinewidth{0.501875pt}%
\definecolor{currentstroke}{rgb}{0.317647,0.317647,0.317647}%
\pgfsetstrokecolor{currentstroke}%
\pgfsetdash{}{0pt}%
\pgfsys@defobject{currentmarker}{\pgfqpoint{0.000000in}{-0.020833in}}{\pgfqpoint{0.000000in}{0.000000in}}{%
\pgfpathmoveto{\pgfqpoint{0.000000in}{0.000000in}}%
\pgfpathlineto{\pgfqpoint{0.000000in}{-0.020833in}}%
\pgfusepath{stroke,fill}%
}%
\begin{pgfscope}%
\pgfsys@transformshift{1.618241in}{0.385730in}%
\pgfsys@useobject{currentmarker}{}%
\end{pgfscope}%
\end{pgfscope}%
\begin{pgfscope}%
\definecolor{textcolor}{rgb}{0.317647,0.317647,0.317647}%
\pgfsetstrokecolor{textcolor}%
\pgfsetfillcolor{textcolor}%
\pgftext[x=1.618241in,y=0.337119in,,top]{\color{textcolor}\rmfamily\fontsize{6.664000}{7.996800}\selectfont \(\displaystyle 0\)}%
\end{pgfscope}%
\begin{pgfscope}%
\pgfsetbuttcap%
\pgfsetroundjoin%
\definecolor{currentfill}{rgb}{0.317647,0.317647,0.317647}%
\pgfsetfillcolor{currentfill}%
\pgfsetlinewidth{0.501875pt}%
\definecolor{currentstroke}{rgb}{0.317647,0.317647,0.317647}%
\pgfsetstrokecolor{currentstroke}%
\pgfsetdash{}{0pt}%
\pgfsys@defobject{currentmarker}{\pgfqpoint{0.000000in}{-0.020833in}}{\pgfqpoint{0.000000in}{0.000000in}}{%
\pgfpathmoveto{\pgfqpoint{0.000000in}{0.000000in}}%
\pgfpathlineto{\pgfqpoint{0.000000in}{-0.020833in}}%
\pgfusepath{stroke,fill}%
}%
\begin{pgfscope}%
\pgfsys@transformshift{1.995676in}{0.385730in}%
\pgfsys@useobject{currentmarker}{}%
\end{pgfscope}%
\end{pgfscope}%
\begin{pgfscope}%
\definecolor{textcolor}{rgb}{0.317647,0.317647,0.317647}%
\pgfsetstrokecolor{textcolor}%
\pgfsetfillcolor{textcolor}%
\pgftext[x=1.995676in,y=0.337119in,,top]{\color{textcolor}\rmfamily\fontsize{6.664000}{7.996800}\selectfont \(\displaystyle 200\)}%
\end{pgfscope}%
\begin{pgfscope}%
\pgfsetbuttcap%
\pgfsetroundjoin%
\definecolor{currentfill}{rgb}{0.317647,0.317647,0.317647}%
\pgfsetfillcolor{currentfill}%
\pgfsetlinewidth{0.501875pt}%
\definecolor{currentstroke}{rgb}{0.317647,0.317647,0.317647}%
\pgfsetstrokecolor{currentstroke}%
\pgfsetdash{}{0pt}%
\pgfsys@defobject{currentmarker}{\pgfqpoint{0.000000in}{-0.020833in}}{\pgfqpoint{0.000000in}{0.000000in}}{%
\pgfpathmoveto{\pgfqpoint{0.000000in}{0.000000in}}%
\pgfpathlineto{\pgfqpoint{0.000000in}{-0.020833in}}%
\pgfusepath{stroke,fill}%
}%
\begin{pgfscope}%
\pgfsys@transformshift{2.373111in}{0.385730in}%
\pgfsys@useobject{currentmarker}{}%
\end{pgfscope}%
\end{pgfscope}%
\begin{pgfscope}%
\definecolor{textcolor}{rgb}{0.317647,0.317647,0.317647}%
\pgfsetstrokecolor{textcolor}%
\pgfsetfillcolor{textcolor}%
\pgftext[x=2.373111in,y=0.337119in,,top]{\color{textcolor}\rmfamily\fontsize{6.664000}{7.996800}\selectfont \(\displaystyle 400\)}%
\end{pgfscope}%
\begin{pgfscope}%
\pgfsetbuttcap%
\pgfsetroundjoin%
\definecolor{currentfill}{rgb}{0.317647,0.317647,0.317647}%
\pgfsetfillcolor{currentfill}%
\pgfsetlinewidth{0.501875pt}%
\definecolor{currentstroke}{rgb}{0.317647,0.317647,0.317647}%
\pgfsetstrokecolor{currentstroke}%
\pgfsetdash{}{0pt}%
\pgfsys@defobject{currentmarker}{\pgfqpoint{0.000000in}{-0.020833in}}{\pgfqpoint{0.000000in}{0.000000in}}{%
\pgfpathmoveto{\pgfqpoint{0.000000in}{0.000000in}}%
\pgfpathlineto{\pgfqpoint{0.000000in}{-0.020833in}}%
\pgfusepath{stroke,fill}%
}%
\begin{pgfscope}%
\pgfsys@transformshift{2.750546in}{0.385730in}%
\pgfsys@useobject{currentmarker}{}%
\end{pgfscope}%
\end{pgfscope}%
\begin{pgfscope}%
\definecolor{textcolor}{rgb}{0.317647,0.317647,0.317647}%
\pgfsetstrokecolor{textcolor}%
\pgfsetfillcolor{textcolor}%
\pgftext[x=2.750546in,y=0.337119in,,top]{\color{textcolor}\rmfamily\fontsize{6.664000}{7.996800}\selectfont \(\displaystyle 600\)}%
\end{pgfscope}%
\begin{pgfscope}%
\definecolor{textcolor}{rgb}{0.317647,0.317647,0.317647}%
\pgfsetstrokecolor{textcolor}%
\pgfsetfillcolor{textcolor}%
\pgftext[x=1.618241in,y=0.199375in,,top]{\color{textcolor}\rmfamily\fontsize{6.664000}{7.996800}\selectfont \(\displaystyle \nu_\mathrm{input} \; (\si{\kilo \Hz})\)}%
\end{pgfscope}%
\begin{pgfscope}%
\pgfsetbuttcap%
\pgfsetroundjoin%
\definecolor{currentfill}{rgb}{0.317647,0.317647,0.317647}%
\pgfsetfillcolor{currentfill}%
\pgfsetlinewidth{0.501875pt}%
\definecolor{currentstroke}{rgb}{0.317647,0.317647,0.317647}%
\pgfsetstrokecolor{currentstroke}%
\pgfsetdash{}{0pt}%
\pgfsys@defobject{currentmarker}{\pgfqpoint{-0.020833in}{0.000000in}}{\pgfqpoint{0.000000in}{0.000000in}}{%
\pgfpathmoveto{\pgfqpoint{0.000000in}{0.000000in}}%
\pgfpathlineto{\pgfqpoint{-0.020833in}{0.000000in}}%
\pgfusepath{stroke,fill}%
}%
\begin{pgfscope}%
\pgfsys@transformshift{0.455741in}{0.490730in}%
\pgfsys@useobject{currentmarker}{}%
\end{pgfscope}%
\end{pgfscope}%
\begin{pgfscope}%
\definecolor{textcolor}{rgb}{0.317647,0.317647,0.317647}%
\pgfsetstrokecolor{textcolor}%
\pgfsetfillcolor{textcolor}%
\pgftext[x=0.365656in,y=0.458614in,left,base]{\color{textcolor}\rmfamily\fontsize{6.664000}{7.996800}\selectfont \(\displaystyle 0\)}%
\end{pgfscope}%
\begin{pgfscope}%
\pgfsetbuttcap%
\pgfsetroundjoin%
\definecolor{currentfill}{rgb}{0.317647,0.317647,0.317647}%
\pgfsetfillcolor{currentfill}%
\pgfsetlinewidth{0.501875pt}%
\definecolor{currentstroke}{rgb}{0.317647,0.317647,0.317647}%
\pgfsetstrokecolor{currentstroke}%
\pgfsetdash{}{0pt}%
\pgfsys@defobject{currentmarker}{\pgfqpoint{-0.020833in}{0.000000in}}{\pgfqpoint{0.000000in}{0.000000in}}{%
\pgfpathmoveto{\pgfqpoint{0.000000in}{0.000000in}}%
\pgfpathlineto{\pgfqpoint{-0.020833in}{0.000000in}}%
\pgfusepath{stroke,fill}%
}%
\begin{pgfscope}%
\pgfsys@transformshift{0.455741in}{0.869554in}%
\pgfsys@useobject{currentmarker}{}%
\end{pgfscope}%
\end{pgfscope}%
\begin{pgfscope}%
\definecolor{textcolor}{rgb}{0.317647,0.317647,0.317647}%
\pgfsetstrokecolor{textcolor}%
\pgfsetfillcolor{textcolor}%
\pgftext[x=0.310293in,y=0.837437in,left,base]{\color{textcolor}\rmfamily\fontsize{6.664000}{7.996800}\selectfont \(\displaystyle 20\)}%
\end{pgfscope}%
\begin{pgfscope}%
\pgfsetbuttcap%
\pgfsetroundjoin%
\definecolor{currentfill}{rgb}{0.317647,0.317647,0.317647}%
\pgfsetfillcolor{currentfill}%
\pgfsetlinewidth{0.501875pt}%
\definecolor{currentstroke}{rgb}{0.317647,0.317647,0.317647}%
\pgfsetstrokecolor{currentstroke}%
\pgfsetdash{}{0pt}%
\pgfsys@defobject{currentmarker}{\pgfqpoint{-0.020833in}{0.000000in}}{\pgfqpoint{0.000000in}{0.000000in}}{%
\pgfpathmoveto{\pgfqpoint{0.000000in}{0.000000in}}%
\pgfpathlineto{\pgfqpoint{-0.020833in}{0.000000in}}%
\pgfusepath{stroke,fill}%
}%
\begin{pgfscope}%
\pgfsys@transformshift{0.455741in}{1.248377in}%
\pgfsys@useobject{currentmarker}{}%
\end{pgfscope}%
\end{pgfscope}%
\begin{pgfscope}%
\definecolor{textcolor}{rgb}{0.317647,0.317647,0.317647}%
\pgfsetstrokecolor{textcolor}%
\pgfsetfillcolor{textcolor}%
\pgftext[x=0.310293in,y=1.216261in,left,base]{\color{textcolor}\rmfamily\fontsize{6.664000}{7.996800}\selectfont \(\displaystyle 40\)}%
\end{pgfscope}%
\begin{pgfscope}%
\pgfsetbuttcap%
\pgfsetroundjoin%
\definecolor{currentfill}{rgb}{0.317647,0.317647,0.317647}%
\pgfsetfillcolor{currentfill}%
\pgfsetlinewidth{0.501875pt}%
\definecolor{currentstroke}{rgb}{0.317647,0.317647,0.317647}%
\pgfsetstrokecolor{currentstroke}%
\pgfsetdash{}{0pt}%
\pgfsys@defobject{currentmarker}{\pgfqpoint{-0.020833in}{0.000000in}}{\pgfqpoint{0.000000in}{0.000000in}}{%
\pgfpathmoveto{\pgfqpoint{0.000000in}{0.000000in}}%
\pgfpathlineto{\pgfqpoint{-0.020833in}{0.000000in}}%
\pgfusepath{stroke,fill}%
}%
\begin{pgfscope}%
\pgfsys@transformshift{0.455741in}{1.627201in}%
\pgfsys@useobject{currentmarker}{}%
\end{pgfscope}%
\end{pgfscope}%
\begin{pgfscope}%
\definecolor{textcolor}{rgb}{0.317647,0.317647,0.317647}%
\pgfsetstrokecolor{textcolor}%
\pgfsetfillcolor{textcolor}%
\pgftext[x=0.310293in,y=1.595084in,left,base]{\color{textcolor}\rmfamily\fontsize{6.664000}{7.996800}\selectfont \(\displaystyle 60\)}%
\end{pgfscope}%
\begin{pgfscope}%
\pgfsetbuttcap%
\pgfsetroundjoin%
\definecolor{currentfill}{rgb}{0.317647,0.317647,0.317647}%
\pgfsetfillcolor{currentfill}%
\pgfsetlinewidth{0.501875pt}%
\definecolor{currentstroke}{rgb}{0.317647,0.317647,0.317647}%
\pgfsetstrokecolor{currentstroke}%
\pgfsetdash{}{0pt}%
\pgfsys@defobject{currentmarker}{\pgfqpoint{-0.020833in}{0.000000in}}{\pgfqpoint{0.000000in}{0.000000in}}{%
\pgfpathmoveto{\pgfqpoint{0.000000in}{0.000000in}}%
\pgfpathlineto{\pgfqpoint{-0.020833in}{0.000000in}}%
\pgfusepath{stroke,fill}%
}%
\begin{pgfscope}%
\pgfsys@transformshift{0.455741in}{2.006025in}%
\pgfsys@useobject{currentmarker}{}%
\end{pgfscope}%
\end{pgfscope}%
\begin{pgfscope}%
\definecolor{textcolor}{rgb}{0.317647,0.317647,0.317647}%
\pgfsetstrokecolor{textcolor}%
\pgfsetfillcolor{textcolor}%
\pgftext[x=0.310293in,y=1.973908in,left,base]{\color{textcolor}\rmfamily\fontsize{6.664000}{7.996800}\selectfont \(\displaystyle 80\)}%
\end{pgfscope}%
\begin{pgfscope}%
\pgfsetbuttcap%
\pgfsetroundjoin%
\definecolor{currentfill}{rgb}{0.317647,0.317647,0.317647}%
\pgfsetfillcolor{currentfill}%
\pgfsetlinewidth{0.501875pt}%
\definecolor{currentstroke}{rgb}{0.317647,0.317647,0.317647}%
\pgfsetstrokecolor{currentstroke}%
\pgfsetdash{}{0pt}%
\pgfsys@defobject{currentmarker}{\pgfqpoint{-0.020833in}{0.000000in}}{\pgfqpoint{0.000000in}{0.000000in}}{%
\pgfpathmoveto{\pgfqpoint{0.000000in}{0.000000in}}%
\pgfpathlineto{\pgfqpoint{-0.020833in}{0.000000in}}%
\pgfusepath{stroke,fill}%
}%
\begin{pgfscope}%
\pgfsys@transformshift{0.455741in}{2.384848in}%
\pgfsys@useobject{currentmarker}{}%
\end{pgfscope}%
\end{pgfscope}%
\begin{pgfscope}%
\definecolor{textcolor}{rgb}{0.317647,0.317647,0.317647}%
\pgfsetstrokecolor{textcolor}%
\pgfsetfillcolor{textcolor}%
\pgftext[x=0.254930in,y=2.352731in,left,base]{\color{textcolor}\rmfamily\fontsize{6.664000}{7.996800}\selectfont \(\displaystyle 100\)}%
\end{pgfscope}%
\begin{pgfscope}%
\definecolor{textcolor}{rgb}{0.317647,0.317647,0.317647}%
\pgfsetstrokecolor{textcolor}%
\pgfsetfillcolor{textcolor}%
\pgftext[x=0.199375in,y=1.540730in,,bottom,rotate=90.000000]{\color{textcolor}\rmfamily\fontsize{6.664000}{7.996800}\selectfont \(\displaystyle \nu_\mathrm{output} \; (\si{\kilo \Hz})\)}%
\end{pgfscope}%
\begin{pgfscope}%
\pgfpathrectangle{\pgfqpoint{0.455741in}{0.385730in}}{\pgfqpoint{2.325000in}{2.310000in}}%
\pgfusepath{clip}%
\pgfsetrectcap%
\pgfsetroundjoin%
\pgfsetlinewidth{0.803000pt}%
\definecolor{currentstroke}{rgb}{0.333333,0.333333,0.333333}%
\pgfsetstrokecolor{currentstroke}%
\pgfsetdash{}{0pt}%
\pgfpathmoveto{\pgfqpoint{0.561423in}{0.490730in}}%
\pgfpathlineto{\pgfqpoint{0.620135in}{0.490730in}}%
\pgfpathlineto{\pgfqpoint{0.678847in}{0.490730in}}%
\pgfpathlineto{\pgfqpoint{0.737559in}{0.490730in}}%
\pgfpathlineto{\pgfqpoint{0.796271in}{0.490730in}}%
\pgfpathlineto{\pgfqpoint{0.854984in}{0.490730in}}%
\pgfpathlineto{\pgfqpoint{0.913696in}{0.490730in}}%
\pgfpathlineto{\pgfqpoint{0.972408in}{0.490730in}}%
\pgfpathlineto{\pgfqpoint{1.031120in}{0.490730in}}%
\pgfpathlineto{\pgfqpoint{1.089832in}{0.490730in}}%
\pgfpathlineto{\pgfqpoint{1.148544in}{0.490730in}}%
\pgfpathlineto{\pgfqpoint{1.207256in}{0.490730in}}%
\pgfpathlineto{\pgfqpoint{1.265968in}{0.490730in}}%
\pgfpathlineto{\pgfqpoint{1.324681in}{0.498966in}}%
\pgfpathlineto{\pgfqpoint{1.383393in}{0.515436in}}%
\pgfpathlineto{\pgfqpoint{1.442105in}{0.523672in}}%
\pgfpathlineto{\pgfqpoint{1.500817in}{0.581319in}}%
\pgfpathlineto{\pgfqpoint{1.559529in}{0.811907in}}%
\pgfpathlineto{\pgfqpoint{1.618241in}{0.910730in}}%
\pgfpathlineto{\pgfqpoint{1.676953in}{1.454260in}}%
\pgfpathlineto{\pgfqpoint{1.735665in}{2.014260in}}%
\pgfpathlineto{\pgfqpoint{1.794377in}{2.220142in}}%
\pgfpathlineto{\pgfqpoint{1.853090in}{2.368377in}}%
\pgfpathlineto{\pgfqpoint{1.911802in}{2.458966in}}%
\pgfpathlineto{\pgfqpoint{1.970514in}{2.508377in}}%
\pgfpathlineto{\pgfqpoint{2.029226in}{2.524848in}}%
\pgfpathlineto{\pgfqpoint{2.087938in}{2.549554in}}%
\pgfpathlineto{\pgfqpoint{2.146650in}{2.549554in}}%
\pgfpathlineto{\pgfqpoint{2.205362in}{2.557789in}}%
\pgfpathlineto{\pgfqpoint{2.264074in}{2.574260in}}%
\pgfpathlineto{\pgfqpoint{2.322787in}{2.582495in}}%
\pgfpathlineto{\pgfqpoint{2.381499in}{2.582495in}}%
\pgfpathlineto{\pgfqpoint{2.440211in}{2.590730in}}%
\pgfpathlineto{\pgfqpoint{2.498923in}{2.590730in}}%
\pgfpathlineto{\pgfqpoint{2.557635in}{2.590730in}}%
\pgfpathlineto{\pgfqpoint{2.616347in}{2.590730in}}%
\pgfpathlineto{\pgfqpoint{2.675059in}{2.590730in}}%
\pgfusepath{stroke}%
\end{pgfscope}%
\begin{pgfscope}%
\pgfpathrectangle{\pgfqpoint{0.455741in}{0.385730in}}{\pgfqpoint{2.325000in}{2.310000in}}%
\pgfusepath{clip}%
\pgfsetrectcap%
\pgfsetroundjoin%
\pgfsetlinewidth{0.803000pt}%
\definecolor{currentstroke}{rgb}{0.686275,0.352941,0.313725}%
\pgfsetstrokecolor{currentstroke}%
\pgfsetdash{}{0pt}%
\pgfpathmoveto{\pgfqpoint{0.561423in}{0.490730in}}%
\pgfpathlineto{\pgfqpoint{0.620135in}{0.490730in}}%
\pgfpathlineto{\pgfqpoint{0.678847in}{0.490730in}}%
\pgfpathlineto{\pgfqpoint{0.737559in}{0.490730in}}%
\pgfpathlineto{\pgfqpoint{0.796271in}{0.490730in}}%
\pgfpathlineto{\pgfqpoint{0.854984in}{0.490730in}}%
\pgfpathlineto{\pgfqpoint{0.913696in}{0.490730in}}%
\pgfpathlineto{\pgfqpoint{0.972408in}{0.490730in}}%
\pgfpathlineto{\pgfqpoint{1.031120in}{0.490730in}}%
\pgfpathlineto{\pgfqpoint{1.089832in}{0.490730in}}%
\pgfpathlineto{\pgfqpoint{1.148544in}{0.490730in}}%
\pgfpathlineto{\pgfqpoint{1.207256in}{0.490730in}}%
\pgfpathlineto{\pgfqpoint{1.265968in}{0.490730in}}%
\pgfpathlineto{\pgfqpoint{1.324681in}{0.507201in}}%
\pgfpathlineto{\pgfqpoint{1.383393in}{0.515436in}}%
\pgfpathlineto{\pgfqpoint{1.442105in}{0.531907in}}%
\pgfpathlineto{\pgfqpoint{1.500817in}{0.581319in}}%
\pgfpathlineto{\pgfqpoint{1.559529in}{0.729554in}}%
\pgfpathlineto{\pgfqpoint{1.618241in}{1.001319in}}%
\pgfpathlineto{\pgfqpoint{1.676953in}{1.627201in}}%
\pgfpathlineto{\pgfqpoint{1.735665in}{2.137789in}}%
\pgfpathlineto{\pgfqpoint{1.794377in}{2.253083in}}%
\pgfpathlineto{\pgfqpoint{1.853090in}{2.393083in}}%
\pgfpathlineto{\pgfqpoint{1.911802in}{2.467201in}}%
\pgfpathlineto{\pgfqpoint{1.970514in}{2.500142in}}%
\pgfpathlineto{\pgfqpoint{2.029226in}{2.524848in}}%
\pgfpathlineto{\pgfqpoint{2.087938in}{2.541319in}}%
\pgfpathlineto{\pgfqpoint{2.146650in}{2.549554in}}%
\pgfpathlineto{\pgfqpoint{2.205362in}{2.557789in}}%
\pgfpathlineto{\pgfqpoint{2.264074in}{2.574260in}}%
\pgfpathlineto{\pgfqpoint{2.322787in}{2.582495in}}%
\pgfpathlineto{\pgfqpoint{2.381499in}{2.574260in}}%
\pgfpathlineto{\pgfqpoint{2.440211in}{2.582495in}}%
\pgfpathlineto{\pgfqpoint{2.498923in}{2.590730in}}%
\pgfpathlineto{\pgfqpoint{2.557635in}{2.590730in}}%
\pgfpathlineto{\pgfqpoint{2.616347in}{2.582495in}}%
\pgfpathlineto{\pgfqpoint{2.675059in}{2.582495in}}%
\pgfusepath{stroke}%
\end{pgfscope}%
\begin{pgfscope}%
\pgfpathrectangle{\pgfqpoint{0.455741in}{0.385730in}}{\pgfqpoint{2.325000in}{2.310000in}}%
\pgfusepath{clip}%
\pgfsetrectcap%
\pgfsetroundjoin%
\pgfsetlinewidth{0.803000pt}%
\definecolor{currentstroke}{rgb}{0.000000,0.356863,0.509804}%
\pgfsetstrokecolor{currentstroke}%
\pgfsetdash{}{0pt}%
\pgfpathmoveto{\pgfqpoint{0.561423in}{0.490730in}}%
\pgfpathlineto{\pgfqpoint{0.620135in}{0.490730in}}%
\pgfpathlineto{\pgfqpoint{0.678847in}{0.490730in}}%
\pgfpathlineto{\pgfqpoint{0.737559in}{0.490730in}}%
\pgfpathlineto{\pgfqpoint{0.796271in}{0.490730in}}%
\pgfpathlineto{\pgfqpoint{0.854984in}{0.490730in}}%
\pgfpathlineto{\pgfqpoint{0.913696in}{0.490730in}}%
\pgfpathlineto{\pgfqpoint{0.972408in}{0.490730in}}%
\pgfpathlineto{\pgfqpoint{1.031120in}{0.490730in}}%
\pgfpathlineto{\pgfqpoint{1.089832in}{0.490730in}}%
\pgfpathlineto{\pgfqpoint{1.148544in}{0.490730in}}%
\pgfpathlineto{\pgfqpoint{1.207256in}{0.490730in}}%
\pgfpathlineto{\pgfqpoint{1.265968in}{0.490730in}}%
\pgfpathlineto{\pgfqpoint{1.324681in}{0.515436in}}%
\pgfpathlineto{\pgfqpoint{1.383393in}{0.523672in}}%
\pgfpathlineto{\pgfqpoint{1.442105in}{0.556613in}}%
\pgfpathlineto{\pgfqpoint{1.500817in}{0.663672in}}%
\pgfpathlineto{\pgfqpoint{1.559529in}{0.803672in}}%
\pgfpathlineto{\pgfqpoint{1.618241in}{1.058966in}}%
\pgfpathlineto{\pgfqpoint{1.676953in}{1.824848in}}%
\pgfpathlineto{\pgfqpoint{1.735665in}{2.220142in}}%
\pgfpathlineto{\pgfqpoint{1.794377in}{2.343672in}}%
\pgfpathlineto{\pgfqpoint{1.853090in}{2.401319in}}%
\pgfpathlineto{\pgfqpoint{1.911802in}{2.467201in}}%
\pgfpathlineto{\pgfqpoint{1.970514in}{2.508377in}}%
\pgfpathlineto{\pgfqpoint{2.029226in}{2.524848in}}%
\pgfpathlineto{\pgfqpoint{2.087938in}{2.541319in}}%
\pgfpathlineto{\pgfqpoint{2.146650in}{2.549554in}}%
\pgfpathlineto{\pgfqpoint{2.205362in}{2.557789in}}%
\pgfpathlineto{\pgfqpoint{2.264074in}{2.566025in}}%
\pgfpathlineto{\pgfqpoint{2.322787in}{2.566025in}}%
\pgfpathlineto{\pgfqpoint{2.381499in}{2.582495in}}%
\pgfpathlineto{\pgfqpoint{2.440211in}{2.582495in}}%
\pgfpathlineto{\pgfqpoint{2.498923in}{2.590730in}}%
\pgfpathlineto{\pgfqpoint{2.557635in}{2.590730in}}%
\pgfpathlineto{\pgfqpoint{2.616347in}{2.582495in}}%
\pgfpathlineto{\pgfqpoint{2.675059in}{2.574260in}}%
\pgfusepath{stroke}%
\end{pgfscope}%
\begin{pgfscope}%
\pgfpathrectangle{\pgfqpoint{0.455741in}{0.385730in}}{\pgfqpoint{2.325000in}{2.310000in}}%
\pgfusepath{clip}%
\pgfsetrectcap%
\pgfsetroundjoin%
\pgfsetlinewidth{0.803000pt}%
\definecolor{currentstroke}{rgb}{0.490196,0.588235,0.431373}%
\pgfsetstrokecolor{currentstroke}%
\pgfsetdash{}{0pt}%
\pgfpathmoveto{\pgfqpoint{0.561423in}{0.490730in}}%
\pgfpathlineto{\pgfqpoint{0.620135in}{0.490730in}}%
\pgfpathlineto{\pgfqpoint{0.678847in}{0.490730in}}%
\pgfpathlineto{\pgfqpoint{0.737559in}{0.490730in}}%
\pgfpathlineto{\pgfqpoint{0.796271in}{0.490730in}}%
\pgfpathlineto{\pgfqpoint{0.854984in}{0.490730in}}%
\pgfpathlineto{\pgfqpoint{0.913696in}{0.490730in}}%
\pgfpathlineto{\pgfqpoint{0.972408in}{0.490730in}}%
\pgfpathlineto{\pgfqpoint{1.031120in}{0.490730in}}%
\pgfpathlineto{\pgfqpoint{1.089832in}{0.490730in}}%
\pgfpathlineto{\pgfqpoint{1.148544in}{0.490730in}}%
\pgfpathlineto{\pgfqpoint{1.207256in}{0.490730in}}%
\pgfpathlineto{\pgfqpoint{1.265968in}{0.490730in}}%
\pgfpathlineto{\pgfqpoint{1.324681in}{0.498966in}}%
\pgfpathlineto{\pgfqpoint{1.383393in}{0.507201in}}%
\pgfpathlineto{\pgfqpoint{1.442105in}{0.523672in}}%
\pgfpathlineto{\pgfqpoint{1.500817in}{0.622495in}}%
\pgfpathlineto{\pgfqpoint{1.559529in}{0.778966in}}%
\pgfpathlineto{\pgfqpoint{1.618241in}{1.001319in}}%
\pgfpathlineto{\pgfqpoint{1.676953in}{1.569554in}}%
\pgfpathlineto{\pgfqpoint{1.735665in}{2.113083in}}%
\pgfpathlineto{\pgfqpoint{1.794377in}{2.211907in}}%
\pgfpathlineto{\pgfqpoint{1.853090in}{2.401319in}}%
\pgfpathlineto{\pgfqpoint{1.911802in}{2.458966in}}%
\pgfpathlineto{\pgfqpoint{1.970514in}{2.500142in}}%
\pgfpathlineto{\pgfqpoint{2.029226in}{2.516613in}}%
\pgfpathlineto{\pgfqpoint{2.087938in}{2.533083in}}%
\pgfpathlineto{\pgfqpoint{2.146650in}{2.541319in}}%
\pgfpathlineto{\pgfqpoint{2.205362in}{2.549554in}}%
\pgfpathlineto{\pgfqpoint{2.264074in}{2.557789in}}%
\pgfpathlineto{\pgfqpoint{2.322787in}{2.566025in}}%
\pgfpathlineto{\pgfqpoint{2.381499in}{2.566025in}}%
\pgfpathlineto{\pgfqpoint{2.440211in}{2.566025in}}%
\pgfpathlineto{\pgfqpoint{2.498923in}{2.574260in}}%
\pgfpathlineto{\pgfqpoint{2.557635in}{2.574260in}}%
\pgfpathlineto{\pgfqpoint{2.616347in}{2.566025in}}%
\pgfpathlineto{\pgfqpoint{2.675059in}{2.574260in}}%
\pgfusepath{stroke}%
\end{pgfscope}%
\begin{pgfscope}%
\pgfpathrectangle{\pgfqpoint{0.455741in}{0.385730in}}{\pgfqpoint{2.325000in}{2.310000in}}%
\pgfusepath{clip}%
\pgfsetrectcap%
\pgfsetroundjoin%
\pgfsetlinewidth{0.803000pt}%
\definecolor{currentstroke}{rgb}{0.843137,0.666667,0.313725}%
\pgfsetstrokecolor{currentstroke}%
\pgfsetdash{}{0pt}%
\pgfpathmoveto{\pgfqpoint{0.561423in}{0.490730in}}%
\pgfpathlineto{\pgfqpoint{0.620135in}{0.490730in}}%
\pgfpathlineto{\pgfqpoint{0.678847in}{0.490730in}}%
\pgfpathlineto{\pgfqpoint{0.737559in}{0.490730in}}%
\pgfpathlineto{\pgfqpoint{0.796271in}{0.490730in}}%
\pgfpathlineto{\pgfqpoint{0.854984in}{0.490730in}}%
\pgfpathlineto{\pgfqpoint{0.913696in}{0.490730in}}%
\pgfpathlineto{\pgfqpoint{0.972408in}{0.490730in}}%
\pgfpathlineto{\pgfqpoint{1.031120in}{0.490730in}}%
\pgfpathlineto{\pgfqpoint{1.089832in}{0.490730in}}%
\pgfpathlineto{\pgfqpoint{1.148544in}{0.490730in}}%
\pgfpathlineto{\pgfqpoint{1.207256in}{0.490730in}}%
\pgfpathlineto{\pgfqpoint{1.265968in}{0.490730in}}%
\pgfpathlineto{\pgfqpoint{1.324681in}{0.498966in}}%
\pgfpathlineto{\pgfqpoint{1.383393in}{0.507201in}}%
\pgfpathlineto{\pgfqpoint{1.442105in}{0.523672in}}%
\pgfpathlineto{\pgfqpoint{1.500817in}{0.573083in}}%
\pgfpathlineto{\pgfqpoint{1.559529in}{0.737789in}}%
\pgfpathlineto{\pgfqpoint{1.618241in}{0.943672in}}%
\pgfpathlineto{\pgfqpoint{1.676953in}{1.586025in}}%
\pgfpathlineto{\pgfqpoint{1.735665in}{2.121319in}}%
\pgfpathlineto{\pgfqpoint{1.794377in}{2.286025in}}%
\pgfpathlineto{\pgfqpoint{1.853090in}{2.384848in}}%
\pgfpathlineto{\pgfqpoint{1.911802in}{2.458966in}}%
\pgfpathlineto{\pgfqpoint{1.970514in}{2.491907in}}%
\pgfpathlineto{\pgfqpoint{2.029226in}{2.508377in}}%
\pgfpathlineto{\pgfqpoint{2.087938in}{2.541319in}}%
\pgfpathlineto{\pgfqpoint{2.146650in}{2.549554in}}%
\pgfpathlineto{\pgfqpoint{2.205362in}{2.549554in}}%
\pgfpathlineto{\pgfqpoint{2.264074in}{2.566025in}}%
\pgfpathlineto{\pgfqpoint{2.322787in}{2.582495in}}%
\pgfpathlineto{\pgfqpoint{2.381499in}{2.582495in}}%
\pgfpathlineto{\pgfqpoint{2.440211in}{2.590730in}}%
\pgfpathlineto{\pgfqpoint{2.498923in}{2.590730in}}%
\pgfpathlineto{\pgfqpoint{2.557635in}{2.590730in}}%
\pgfpathlineto{\pgfqpoint{2.616347in}{2.590730in}}%
\pgfpathlineto{\pgfqpoint{2.675059in}{2.590730in}}%
\pgfusepath{stroke}%
\end{pgfscope}%
\begin{pgfscope}%
\pgfpathrectangle{\pgfqpoint{0.455741in}{0.385730in}}{\pgfqpoint{2.325000in}{2.310000in}}%
\pgfusepath{clip}%
\pgfsetrectcap%
\pgfsetroundjoin%
\pgfsetlinewidth{0.803000pt}%
\definecolor{currentstroke}{rgb}{0.333333,0.333333,0.333333}%
\pgfsetstrokecolor{currentstroke}%
\pgfsetdash{}{0pt}%
\pgfpathmoveto{\pgfqpoint{0.561423in}{0.490730in}}%
\pgfpathlineto{\pgfqpoint{0.620135in}{0.490730in}}%
\pgfpathlineto{\pgfqpoint{0.678847in}{0.490730in}}%
\pgfpathlineto{\pgfqpoint{0.737559in}{0.490730in}}%
\pgfpathlineto{\pgfqpoint{0.796271in}{0.490730in}}%
\pgfpathlineto{\pgfqpoint{0.854984in}{0.490730in}}%
\pgfpathlineto{\pgfqpoint{0.913696in}{0.490730in}}%
\pgfpathlineto{\pgfqpoint{0.972408in}{0.490730in}}%
\pgfpathlineto{\pgfqpoint{1.031120in}{0.490730in}}%
\pgfpathlineto{\pgfqpoint{1.089832in}{0.490730in}}%
\pgfpathlineto{\pgfqpoint{1.148544in}{0.490730in}}%
\pgfpathlineto{\pgfqpoint{1.207256in}{0.490730in}}%
\pgfpathlineto{\pgfqpoint{1.265968in}{0.490730in}}%
\pgfpathlineto{\pgfqpoint{1.324681in}{0.507201in}}%
\pgfpathlineto{\pgfqpoint{1.383393in}{0.540142in}}%
\pgfpathlineto{\pgfqpoint{1.442105in}{0.564848in}}%
\pgfpathlineto{\pgfqpoint{1.500817in}{0.597789in}}%
\pgfpathlineto{\pgfqpoint{1.559529in}{0.754260in}}%
\pgfpathlineto{\pgfqpoint{1.618241in}{0.993083in}}%
\pgfpathlineto{\pgfqpoint{1.676953in}{1.594260in}}%
\pgfpathlineto{\pgfqpoint{1.735665in}{2.096613in}}%
\pgfpathlineto{\pgfqpoint{1.794377in}{2.236613in}}%
\pgfpathlineto{\pgfqpoint{1.853090in}{2.384848in}}%
\pgfpathlineto{\pgfqpoint{1.911802in}{2.467201in}}%
\pgfpathlineto{\pgfqpoint{1.970514in}{2.508377in}}%
\pgfpathlineto{\pgfqpoint{2.029226in}{2.516613in}}%
\pgfpathlineto{\pgfqpoint{2.087938in}{2.541319in}}%
\pgfpathlineto{\pgfqpoint{2.146650in}{2.549554in}}%
\pgfpathlineto{\pgfqpoint{2.205362in}{2.557789in}}%
\pgfpathlineto{\pgfqpoint{2.264074in}{2.557789in}}%
\pgfpathlineto{\pgfqpoint{2.322787in}{2.574260in}}%
\pgfpathlineto{\pgfqpoint{2.381499in}{2.574260in}}%
\pgfpathlineto{\pgfqpoint{2.440211in}{2.566025in}}%
\pgfpathlineto{\pgfqpoint{2.498923in}{2.582495in}}%
\pgfpathlineto{\pgfqpoint{2.557635in}{2.582495in}}%
\pgfpathlineto{\pgfqpoint{2.616347in}{2.566025in}}%
\pgfpathlineto{\pgfqpoint{2.675059in}{2.574260in}}%
\pgfusepath{stroke}%
\end{pgfscope}%
\begin{pgfscope}%
\pgfpathrectangle{\pgfqpoint{0.455741in}{0.385730in}}{\pgfqpoint{2.325000in}{2.310000in}}%
\pgfusepath{clip}%
\pgfsetrectcap%
\pgfsetroundjoin%
\pgfsetlinewidth{0.803000pt}%
\definecolor{currentstroke}{rgb}{0.686275,0.352941,0.313725}%
\pgfsetstrokecolor{currentstroke}%
\pgfsetdash{}{0pt}%
\pgfpathmoveto{\pgfqpoint{0.561423in}{0.490730in}}%
\pgfpathlineto{\pgfqpoint{0.620135in}{0.490730in}}%
\pgfpathlineto{\pgfqpoint{0.678847in}{0.490730in}}%
\pgfpathlineto{\pgfqpoint{0.737559in}{0.490730in}}%
\pgfpathlineto{\pgfqpoint{0.796271in}{0.490730in}}%
\pgfpathlineto{\pgfqpoint{0.854984in}{0.490730in}}%
\pgfpathlineto{\pgfqpoint{0.913696in}{0.490730in}}%
\pgfpathlineto{\pgfqpoint{0.972408in}{0.490730in}}%
\pgfpathlineto{\pgfqpoint{1.031120in}{0.490730in}}%
\pgfpathlineto{\pgfqpoint{1.089832in}{0.490730in}}%
\pgfpathlineto{\pgfqpoint{1.148544in}{0.490730in}}%
\pgfpathlineto{\pgfqpoint{1.207256in}{0.490730in}}%
\pgfpathlineto{\pgfqpoint{1.265968in}{0.490730in}}%
\pgfpathlineto{\pgfqpoint{1.324681in}{0.507201in}}%
\pgfpathlineto{\pgfqpoint{1.383393in}{0.531907in}}%
\pgfpathlineto{\pgfqpoint{1.442105in}{0.573083in}}%
\pgfpathlineto{\pgfqpoint{1.500817in}{0.663672in}}%
\pgfpathlineto{\pgfqpoint{1.559529in}{0.960142in}}%
\pgfpathlineto{\pgfqpoint{1.618241in}{1.198966in}}%
\pgfpathlineto{\pgfqpoint{1.676953in}{1.874260in}}%
\pgfpathlineto{\pgfqpoint{1.735665in}{2.211907in}}%
\pgfpathlineto{\pgfqpoint{1.794377in}{2.318966in}}%
\pgfpathlineto{\pgfqpoint{1.853090in}{2.426025in}}%
\pgfpathlineto{\pgfqpoint{1.911802in}{2.483672in}}%
\pgfpathlineto{\pgfqpoint{1.970514in}{2.516613in}}%
\pgfpathlineto{\pgfqpoint{2.029226in}{2.541319in}}%
\pgfpathlineto{\pgfqpoint{2.087938in}{2.549554in}}%
\pgfpathlineto{\pgfqpoint{2.146650in}{2.557789in}}%
\pgfpathlineto{\pgfqpoint{2.205362in}{2.566025in}}%
\pgfpathlineto{\pgfqpoint{2.264074in}{2.574260in}}%
\pgfpathlineto{\pgfqpoint{2.322787in}{2.574260in}}%
\pgfpathlineto{\pgfqpoint{2.381499in}{2.582495in}}%
\pgfpathlineto{\pgfqpoint{2.440211in}{2.582495in}}%
\pgfpathlineto{\pgfqpoint{2.498923in}{2.590730in}}%
\pgfpathlineto{\pgfqpoint{2.557635in}{2.590730in}}%
\pgfpathlineto{\pgfqpoint{2.616347in}{2.574260in}}%
\pgfpathlineto{\pgfqpoint{2.675059in}{2.582495in}}%
\pgfusepath{stroke}%
\end{pgfscope}%
\begin{pgfscope}%
\pgfpathrectangle{\pgfqpoint{0.455741in}{0.385730in}}{\pgfqpoint{2.325000in}{2.310000in}}%
\pgfusepath{clip}%
\pgfsetrectcap%
\pgfsetroundjoin%
\pgfsetlinewidth{0.803000pt}%
\definecolor{currentstroke}{rgb}{0.000000,0.356863,0.509804}%
\pgfsetstrokecolor{currentstroke}%
\pgfsetdash{}{0pt}%
\pgfpathmoveto{\pgfqpoint{0.561423in}{0.490730in}}%
\pgfpathlineto{\pgfqpoint{0.620135in}{0.490730in}}%
\pgfpathlineto{\pgfqpoint{0.678847in}{0.490730in}}%
\pgfpathlineto{\pgfqpoint{0.737559in}{0.490730in}}%
\pgfpathlineto{\pgfqpoint{0.796271in}{0.490730in}}%
\pgfpathlineto{\pgfqpoint{0.854984in}{0.490730in}}%
\pgfpathlineto{\pgfqpoint{0.913696in}{0.490730in}}%
\pgfpathlineto{\pgfqpoint{0.972408in}{0.490730in}}%
\pgfpathlineto{\pgfqpoint{1.031120in}{0.490730in}}%
\pgfpathlineto{\pgfqpoint{1.089832in}{0.490730in}}%
\pgfpathlineto{\pgfqpoint{1.148544in}{0.490730in}}%
\pgfpathlineto{\pgfqpoint{1.207256in}{0.490730in}}%
\pgfpathlineto{\pgfqpoint{1.265968in}{0.490730in}}%
\pgfpathlineto{\pgfqpoint{1.324681in}{0.507201in}}%
\pgfpathlineto{\pgfqpoint{1.383393in}{0.515436in}}%
\pgfpathlineto{\pgfqpoint{1.442105in}{0.531907in}}%
\pgfpathlineto{\pgfqpoint{1.500817in}{0.630730in}}%
\pgfpathlineto{\pgfqpoint{1.559529in}{0.737789in}}%
\pgfpathlineto{\pgfqpoint{1.618241in}{1.116613in}}%
\pgfpathlineto{\pgfqpoint{1.676953in}{1.643672in}}%
\pgfpathlineto{\pgfqpoint{1.735665in}{2.088377in}}%
\pgfpathlineto{\pgfqpoint{1.794377in}{2.269554in}}%
\pgfpathlineto{\pgfqpoint{1.853090in}{2.393083in}}%
\pgfpathlineto{\pgfqpoint{1.911802in}{2.475436in}}%
\pgfpathlineto{\pgfqpoint{1.970514in}{2.500142in}}%
\pgfpathlineto{\pgfqpoint{2.029226in}{2.524848in}}%
\pgfpathlineto{\pgfqpoint{2.087938in}{2.541319in}}%
\pgfpathlineto{\pgfqpoint{2.146650in}{2.557789in}}%
\pgfpathlineto{\pgfqpoint{2.205362in}{2.566025in}}%
\pgfpathlineto{\pgfqpoint{2.264074in}{2.566025in}}%
\pgfpathlineto{\pgfqpoint{2.322787in}{2.574260in}}%
\pgfpathlineto{\pgfqpoint{2.381499in}{2.582495in}}%
\pgfpathlineto{\pgfqpoint{2.440211in}{2.590730in}}%
\pgfpathlineto{\pgfqpoint{2.498923in}{2.590730in}}%
\pgfpathlineto{\pgfqpoint{2.557635in}{2.590730in}}%
\pgfpathlineto{\pgfqpoint{2.616347in}{2.582495in}}%
\pgfpathlineto{\pgfqpoint{2.675059in}{2.590730in}}%
\pgfusepath{stroke}%
\end{pgfscope}%
\begin{pgfscope}%
\pgfpathrectangle{\pgfqpoint{0.455741in}{0.385730in}}{\pgfqpoint{2.325000in}{2.310000in}}%
\pgfusepath{clip}%
\pgfsetrectcap%
\pgfsetroundjoin%
\pgfsetlinewidth{0.803000pt}%
\definecolor{currentstroke}{rgb}{0.490196,0.588235,0.431373}%
\pgfsetstrokecolor{currentstroke}%
\pgfsetdash{}{0pt}%
\pgfpathmoveto{\pgfqpoint{0.561423in}{0.490730in}}%
\pgfpathlineto{\pgfqpoint{0.620135in}{0.490730in}}%
\pgfpathlineto{\pgfqpoint{0.678847in}{0.490730in}}%
\pgfpathlineto{\pgfqpoint{0.737559in}{0.490730in}}%
\pgfpathlineto{\pgfqpoint{0.796271in}{0.490730in}}%
\pgfpathlineto{\pgfqpoint{0.854984in}{0.490730in}}%
\pgfpathlineto{\pgfqpoint{0.913696in}{0.490730in}}%
\pgfpathlineto{\pgfqpoint{0.972408in}{0.490730in}}%
\pgfpathlineto{\pgfqpoint{1.031120in}{0.490730in}}%
\pgfpathlineto{\pgfqpoint{1.089832in}{0.490730in}}%
\pgfpathlineto{\pgfqpoint{1.148544in}{0.490730in}}%
\pgfpathlineto{\pgfqpoint{1.207256in}{0.490730in}}%
\pgfpathlineto{\pgfqpoint{1.265968in}{0.490730in}}%
\pgfpathlineto{\pgfqpoint{1.324681in}{0.498966in}}%
\pgfpathlineto{\pgfqpoint{1.383393in}{0.507201in}}%
\pgfpathlineto{\pgfqpoint{1.442105in}{0.523672in}}%
\pgfpathlineto{\pgfqpoint{1.500817in}{0.564848in}}%
\pgfpathlineto{\pgfqpoint{1.559529in}{0.746025in}}%
\pgfpathlineto{\pgfqpoint{1.618241in}{0.877789in}}%
\pgfpathlineto{\pgfqpoint{1.676953in}{1.462495in}}%
\pgfpathlineto{\pgfqpoint{1.735665in}{1.973083in}}%
\pgfpathlineto{\pgfqpoint{1.794377in}{2.203672in}}%
\pgfpathlineto{\pgfqpoint{1.853090in}{2.351907in}}%
\pgfpathlineto{\pgfqpoint{1.911802in}{2.458966in}}%
\pgfpathlineto{\pgfqpoint{1.970514in}{2.491907in}}%
\pgfpathlineto{\pgfqpoint{2.029226in}{2.516613in}}%
\pgfpathlineto{\pgfqpoint{2.087938in}{2.533083in}}%
\pgfpathlineto{\pgfqpoint{2.146650in}{2.541319in}}%
\pgfpathlineto{\pgfqpoint{2.205362in}{2.549554in}}%
\pgfpathlineto{\pgfqpoint{2.264074in}{2.557789in}}%
\pgfpathlineto{\pgfqpoint{2.322787in}{2.566025in}}%
\pgfpathlineto{\pgfqpoint{2.381499in}{2.574260in}}%
\pgfpathlineto{\pgfqpoint{2.440211in}{2.574260in}}%
\pgfpathlineto{\pgfqpoint{2.498923in}{2.574260in}}%
\pgfpathlineto{\pgfqpoint{2.557635in}{2.582495in}}%
\pgfpathlineto{\pgfqpoint{2.616347in}{2.574260in}}%
\pgfpathlineto{\pgfqpoint{2.675059in}{2.574260in}}%
\pgfusepath{stroke}%
\end{pgfscope}%
\begin{pgfscope}%
\pgfpathrectangle{\pgfqpoint{0.455741in}{0.385730in}}{\pgfqpoint{2.325000in}{2.310000in}}%
\pgfusepath{clip}%
\pgfsetrectcap%
\pgfsetroundjoin%
\pgfsetlinewidth{0.803000pt}%
\definecolor{currentstroke}{rgb}{0.843137,0.666667,0.313725}%
\pgfsetstrokecolor{currentstroke}%
\pgfsetdash{}{0pt}%
\pgfpathmoveto{\pgfqpoint{0.561423in}{0.490730in}}%
\pgfpathlineto{\pgfqpoint{0.620135in}{0.490730in}}%
\pgfpathlineto{\pgfqpoint{0.678847in}{0.490730in}}%
\pgfpathlineto{\pgfqpoint{0.737559in}{0.490730in}}%
\pgfpathlineto{\pgfqpoint{0.796271in}{0.490730in}}%
\pgfpathlineto{\pgfqpoint{0.854984in}{0.490730in}}%
\pgfpathlineto{\pgfqpoint{0.913696in}{0.490730in}}%
\pgfpathlineto{\pgfqpoint{0.972408in}{0.490730in}}%
\pgfpathlineto{\pgfqpoint{1.031120in}{0.490730in}}%
\pgfpathlineto{\pgfqpoint{1.089832in}{0.490730in}}%
\pgfpathlineto{\pgfqpoint{1.148544in}{0.490730in}}%
\pgfpathlineto{\pgfqpoint{1.207256in}{0.490730in}}%
\pgfpathlineto{\pgfqpoint{1.265968in}{0.490730in}}%
\pgfpathlineto{\pgfqpoint{1.324681in}{0.507201in}}%
\pgfpathlineto{\pgfqpoint{1.383393in}{0.515436in}}%
\pgfpathlineto{\pgfqpoint{1.442105in}{0.548377in}}%
\pgfpathlineto{\pgfqpoint{1.500817in}{0.647201in}}%
\pgfpathlineto{\pgfqpoint{1.559529in}{0.696613in}}%
\pgfpathlineto{\pgfqpoint{1.618241in}{0.910730in}}%
\pgfpathlineto{\pgfqpoint{1.676953in}{1.594260in}}%
\pgfpathlineto{\pgfqpoint{1.735665in}{2.088377in}}%
\pgfpathlineto{\pgfqpoint{1.794377in}{2.261319in}}%
\pgfpathlineto{\pgfqpoint{1.853090in}{2.393083in}}%
\pgfpathlineto{\pgfqpoint{1.911802in}{2.467201in}}%
\pgfpathlineto{\pgfqpoint{1.970514in}{2.508377in}}%
\pgfpathlineto{\pgfqpoint{2.029226in}{2.524848in}}%
\pgfpathlineto{\pgfqpoint{2.087938in}{2.533083in}}%
\pgfpathlineto{\pgfqpoint{2.146650in}{2.549554in}}%
\pgfpathlineto{\pgfqpoint{2.205362in}{2.549554in}}%
\pgfpathlineto{\pgfqpoint{2.264074in}{2.566025in}}%
\pgfpathlineto{\pgfqpoint{2.322787in}{2.566025in}}%
\pgfpathlineto{\pgfqpoint{2.381499in}{2.574260in}}%
\pgfpathlineto{\pgfqpoint{2.440211in}{2.574260in}}%
\pgfpathlineto{\pgfqpoint{2.498923in}{2.574260in}}%
\pgfpathlineto{\pgfqpoint{2.557635in}{2.582495in}}%
\pgfpathlineto{\pgfqpoint{2.616347in}{2.574260in}}%
\pgfpathlineto{\pgfqpoint{2.675059in}{2.566025in}}%
\pgfusepath{stroke}%
\end{pgfscope}%
\begin{pgfscope}%
\pgfpathrectangle{\pgfqpoint{0.455741in}{0.385730in}}{\pgfqpoint{2.325000in}{2.310000in}}%
\pgfusepath{clip}%
\pgfsetrectcap%
\pgfsetroundjoin%
\pgfsetlinewidth{0.803000pt}%
\definecolor{currentstroke}{rgb}{0.333333,0.333333,0.333333}%
\pgfsetstrokecolor{currentstroke}%
\pgfsetdash{}{0pt}%
\pgfpathmoveto{\pgfqpoint{0.561423in}{0.490730in}}%
\pgfpathlineto{\pgfqpoint{0.620135in}{0.490730in}}%
\pgfpathlineto{\pgfqpoint{0.678847in}{0.490730in}}%
\pgfpathlineto{\pgfqpoint{0.737559in}{0.490730in}}%
\pgfpathlineto{\pgfqpoint{0.796271in}{0.490730in}}%
\pgfpathlineto{\pgfqpoint{0.854984in}{0.490730in}}%
\pgfpathlineto{\pgfqpoint{0.913696in}{0.490730in}}%
\pgfpathlineto{\pgfqpoint{0.972408in}{0.490730in}}%
\pgfpathlineto{\pgfqpoint{1.031120in}{0.490730in}}%
\pgfpathlineto{\pgfqpoint{1.089832in}{0.490730in}}%
\pgfpathlineto{\pgfqpoint{1.148544in}{0.490730in}}%
\pgfpathlineto{\pgfqpoint{1.207256in}{0.490730in}}%
\pgfpathlineto{\pgfqpoint{1.265968in}{0.490730in}}%
\pgfpathlineto{\pgfqpoint{1.324681in}{0.507201in}}%
\pgfpathlineto{\pgfqpoint{1.383393in}{0.515436in}}%
\pgfpathlineto{\pgfqpoint{1.442105in}{0.523672in}}%
\pgfpathlineto{\pgfqpoint{1.500817in}{0.548377in}}%
\pgfpathlineto{\pgfqpoint{1.559529in}{0.655436in}}%
\pgfpathlineto{\pgfqpoint{1.618241in}{0.894260in}}%
\pgfpathlineto{\pgfqpoint{1.676953in}{1.602495in}}%
\pgfpathlineto{\pgfqpoint{1.735665in}{2.121319in}}%
\pgfpathlineto{\pgfqpoint{1.794377in}{2.261319in}}%
\pgfpathlineto{\pgfqpoint{1.853090in}{2.393083in}}%
\pgfpathlineto{\pgfqpoint{1.911802in}{2.467201in}}%
\pgfpathlineto{\pgfqpoint{1.970514in}{2.500142in}}%
\pgfpathlineto{\pgfqpoint{2.029226in}{2.524848in}}%
\pgfpathlineto{\pgfqpoint{2.087938in}{2.541319in}}%
\pgfpathlineto{\pgfqpoint{2.146650in}{2.549554in}}%
\pgfpathlineto{\pgfqpoint{2.205362in}{2.549554in}}%
\pgfpathlineto{\pgfqpoint{2.264074in}{2.566025in}}%
\pgfpathlineto{\pgfqpoint{2.322787in}{2.566025in}}%
\pgfpathlineto{\pgfqpoint{2.381499in}{2.566025in}}%
\pgfpathlineto{\pgfqpoint{2.440211in}{2.566025in}}%
\pgfpathlineto{\pgfqpoint{2.498923in}{2.566025in}}%
\pgfpathlineto{\pgfqpoint{2.557635in}{2.574260in}}%
\pgfpathlineto{\pgfqpoint{2.616347in}{2.566025in}}%
\pgfpathlineto{\pgfqpoint{2.675059in}{2.566025in}}%
\pgfusepath{stroke}%
\end{pgfscope}%
\begin{pgfscope}%
\pgfpathrectangle{\pgfqpoint{0.455741in}{0.385730in}}{\pgfqpoint{2.325000in}{2.310000in}}%
\pgfusepath{clip}%
\pgfsetrectcap%
\pgfsetroundjoin%
\pgfsetlinewidth{0.803000pt}%
\definecolor{currentstroke}{rgb}{0.686275,0.352941,0.313725}%
\pgfsetstrokecolor{currentstroke}%
\pgfsetdash{}{0pt}%
\pgfpathmoveto{\pgfqpoint{0.561423in}{0.490730in}}%
\pgfpathlineto{\pgfqpoint{0.620135in}{0.490730in}}%
\pgfpathlineto{\pgfqpoint{0.678847in}{0.490730in}}%
\pgfpathlineto{\pgfqpoint{0.737559in}{0.490730in}}%
\pgfpathlineto{\pgfqpoint{0.796271in}{0.490730in}}%
\pgfpathlineto{\pgfqpoint{0.854984in}{0.490730in}}%
\pgfpathlineto{\pgfqpoint{0.913696in}{0.490730in}}%
\pgfpathlineto{\pgfqpoint{0.972408in}{0.490730in}}%
\pgfpathlineto{\pgfqpoint{1.031120in}{0.490730in}}%
\pgfpathlineto{\pgfqpoint{1.089832in}{0.490730in}}%
\pgfpathlineto{\pgfqpoint{1.148544in}{0.490730in}}%
\pgfpathlineto{\pgfqpoint{1.207256in}{0.490730in}}%
\pgfpathlineto{\pgfqpoint{1.265968in}{0.490730in}}%
\pgfpathlineto{\pgfqpoint{1.324681in}{0.498966in}}%
\pgfpathlineto{\pgfqpoint{1.383393in}{0.515436in}}%
\pgfpathlineto{\pgfqpoint{1.442105in}{0.523672in}}%
\pgfpathlineto{\pgfqpoint{1.500817in}{0.597789in}}%
\pgfpathlineto{\pgfqpoint{1.559529in}{0.803672in}}%
\pgfpathlineto{\pgfqpoint{1.618241in}{0.943672in}}%
\pgfpathlineto{\pgfqpoint{1.676953in}{1.577789in}}%
\pgfpathlineto{\pgfqpoint{1.735665in}{2.104848in}}%
\pgfpathlineto{\pgfqpoint{1.794377in}{2.261319in}}%
\pgfpathlineto{\pgfqpoint{1.853090in}{2.384848in}}%
\pgfpathlineto{\pgfqpoint{1.911802in}{2.467201in}}%
\pgfpathlineto{\pgfqpoint{1.970514in}{2.508377in}}%
\pgfpathlineto{\pgfqpoint{2.029226in}{2.516613in}}%
\pgfpathlineto{\pgfqpoint{2.087938in}{2.541319in}}%
\pgfpathlineto{\pgfqpoint{2.146650in}{2.549554in}}%
\pgfpathlineto{\pgfqpoint{2.205362in}{2.566025in}}%
\pgfpathlineto{\pgfqpoint{2.264074in}{2.566025in}}%
\pgfpathlineto{\pgfqpoint{2.322787in}{2.574260in}}%
\pgfpathlineto{\pgfqpoint{2.381499in}{2.574260in}}%
\pgfpathlineto{\pgfqpoint{2.440211in}{2.574260in}}%
\pgfpathlineto{\pgfqpoint{2.498923in}{2.582495in}}%
\pgfpathlineto{\pgfqpoint{2.557635in}{2.582495in}}%
\pgfpathlineto{\pgfqpoint{2.616347in}{2.574260in}}%
\pgfpathlineto{\pgfqpoint{2.675059in}{2.574260in}}%
\pgfusepath{stroke}%
\end{pgfscope}%
\begin{pgfscope}%
\pgfsetrectcap%
\pgfsetmiterjoin%
\pgfsetlinewidth{0.501875pt}%
\definecolor{currentstroke}{rgb}{0.317647,0.317647,0.317647}%
\pgfsetstrokecolor{currentstroke}%
\pgfsetdash{}{0pt}%
\pgfpathmoveto{\pgfqpoint{0.455741in}{0.385730in}}%
\pgfpathlineto{\pgfqpoint{0.455741in}{2.695730in}}%
\pgfusepath{stroke}%
\end{pgfscope}%
\begin{pgfscope}%
\pgfsetrectcap%
\pgfsetmiterjoin%
\pgfsetlinewidth{0.501875pt}%
\definecolor{currentstroke}{rgb}{0.317647,0.317647,0.317647}%
\pgfsetstrokecolor{currentstroke}%
\pgfsetdash{}{0pt}%
\pgfpathmoveto{\pgfqpoint{0.455741in}{0.385730in}}%
\pgfpathlineto{\pgfqpoint{2.780741in}{0.385730in}}%
\pgfusepath{stroke}%
\end{pgfscope}%
\end{pgfpicture}%
\makeatother%
\endgroup%

		\label{transferfunction_w_calib}
	\end{subfigure}
	\caption[Calibration of the sigmoid activation function on \acrshort{dls}.]{Calibration of the activation function on \acrshort{dls}. \textbf{(\subref{transferfunction_wout_calib})}: In the uncalibrated state the maximum rate (\gls{refrac}) and center alignment (\gls{v_leak}) deviates due to imperfection in the manufacturing process of the analog hardware.  \textbf{(\subref{transferfunction_w_calib})}: The calibration aligns the dynamic range of individual neurons, which helps the training.}
\end{figure}

A collection of simple experiment-specific \gls{ppu}-based calibration routines is implemented to maintain a pure on-chip implementation. The calibration is based on a binary search algorithm to find the suitable \gls{dac}-values for the analog neuron parameters. In general, the binary search algorithm compares a target value to the middle value of a given sorted array. Depending on the outcome of the comparison the lower or the upper half is eliminated from the search space and the search is repeated with new boundaries which are set by the remaining half. In a worst case scenario, the algorithm finds the target value after $\mathcal{O}(\log(n))$ repetitions, given the searched array has $n$ entries  (\citealp{binarysearchsource}).

For the calibration of the activation function the sorted array is replaced with the \SI{10}{\bit} range of the targeted analog parameter. Instead of the direct comparison with a target value, a condition is chosen which resembles the desired calibration outcome. For instance, the maximum output frequency of the neuron is calibrated
by changing the \gls{dac} value of the refractory period \gls{refrac}, since in the limit of high frequencies the output rate mainly depends on the refractory period, c.f. with \cref{fireratehigh}. As condition for the binary search, the maximum output rate \gls{nuout} is measured at a high input rate and compared to a target rate $\gls{nuout}^*$. Depending on the outcome, either half of the parameter range is eliminated from the search space.
For the experiment, the target rate is set to \SI{111.3}{\kilo \Hz} which corresponds to $256$ recorded spikes over a measurement period of $T=\SI{2.3}{\milli \s}$.

The calibration of the alignment along the x-axis is slightly more complicated. The activation function is centered if its inflection point is positioned at zero. In theory, the firing rate at the inflection point equals half the maximum rate. As shown in the simulation of the activation function in \cref{ratecoding}, the curve can be shifted along the x-axis by changing the potential difference $\delta V$ between the resting potential and the threshold. Since later on the threshold is used as the bias parameter, the \gls{dac}-value of the resting potential will be calibrated. As condition for the binary search, the rate of the activation function is measured with no additional input except for the noise sources and is compared to the half of the maximum rate.

During the calibration of the resting potential with respect to a centered inflection point of the activation function, the positioning of the sigmoid only depends on the relative potential difference $\delta V$. The choice of a work point for the absolute threshold and resting potential should in theory be arbitrary, as the whole operating range can be shifted. On analog hardware, this does not hold true, since some circuits exhibit limited operating ranges. As a consequence the work point is fixed by a threshold potential of \SI{0.54}{\V}.

%In a practical observation, the center of the sigmoid activation function could be placed at zero input by balancing a fixed position of the reset potential with a minor correction of the bias. This was achieved by formulating the condition of the binary search for the bias related parameter in a way that at zero input the activation function returns half the maximum rate. For a higher output rate, the bias had to be reduced and vice versa. On chip, a change of the bias is applied by adapting the \gls{dac} value of the resting potential while keeping the threshold fixed at a chosen work point of approximately \SI{0.54}{\V}.
%The choice of a work point for the absolute threshold and resting potential should in theory be arbitrary, as the whole operating range can be shifted. On analog hardware, this does not hold true, since some circuits exhibit limited operating ranges.

\textbf{STOPPED HERE}

The calibration of the refractory period and the resting potential is repeated and interleaved several times. This is done to reduce the potential influence of the other yet uncalibrated parameter. In \cref{transferfunction_w_calib} the final calibrated state is shown. 

In the context of analog circuits it is very unlikely to find the exact target \gls{dac}-value due to a temporal variability during the measurements. However, the algorithm continuously approximates the searched value under the chosen condition. The 


\paragraph{Bias and Synaptic Weights} As shown in the simulation of the activation function in \cref{ratecoding}, the slope and of the activation function and its alignment along the x-axis can be easily modified. ...

\begin{figure}
	\begin{subfigure}[c]{0.5\textwidth}
		\centering
		\caption{}
		%% Creator: Matplotlib, PGF backend
%%
%% To include the figure in your LaTeX document, write
%%   \input{<filename>.pgf}
%%
%% Make sure the required packages are loaded in your preamble
%%   \usepackage{pgf}
%%
%% Figures using additional raster images can only be included by \input if
%% they are in the same directory as the main LaTeX file. For loading figures
%% from other directories you can use the `import` package
%%   \usepackage{import}
%% and then include the figures with
%%   \import{<path to file>}{<filename>.pgf}
%%
%% Matplotlib used the following preamble
%%   \usepackage{amsmath} \usepackage{pifont} \usepackage{xcolor} \definecolor{green}{HTML}{467821} \definecolor{red}{HTML}{CF4457} \usepackage[detect-all]{siunitx}
%%   \usepackage{fontspec}
%%
\begingroup%
\makeatletter%
\begin{pgfpicture}%
\pgfpathrectangle{\pgfpointorigin}{\pgfqpoint{2.931438in}{2.408578in}}%
\pgfusepath{use as bounding box, clip}%
\begin{pgfscope}%
\pgfsetbuttcap%
\pgfsetmiterjoin%
\pgfsetlinewidth{0.000000pt}%
\definecolor{currentstroke}{rgb}{0.000000,0.000000,0.000000}%
\pgfsetstrokecolor{currentstroke}%
\pgfsetstrokeopacity{0.000000}%
\pgfsetdash{}{0pt}%
\pgfpathmoveto{\pgfqpoint{0.000000in}{0.000000in}}%
\pgfpathlineto{\pgfqpoint{2.931438in}{0.000000in}}%
\pgfpathlineto{\pgfqpoint{2.931438in}{2.408578in}}%
\pgfpathlineto{\pgfqpoint{0.000000in}{2.408578in}}%
\pgfpathclose%
\pgfusepath{}%
\end{pgfscope}%
\begin{pgfscope}%
\pgfsetbuttcap%
\pgfsetmiterjoin%
\pgfsetlinewidth{0.000000pt}%
\definecolor{currentstroke}{rgb}{0.000000,0.000000,0.000000}%
\pgfsetstrokecolor{currentstroke}%
\pgfsetstrokeopacity{0.000000}%
\pgfsetdash{}{0pt}%
\pgfpathmoveto{\pgfqpoint{0.453589in}{0.383578in}}%
\pgfpathlineto{\pgfqpoint{2.778589in}{0.383578in}}%
\pgfpathlineto{\pgfqpoint{2.778589in}{2.308578in}}%
\pgfpathlineto{\pgfqpoint{0.453589in}{2.308578in}}%
\pgfpathclose%
\pgfusepath{}%
\end{pgfscope}%
\begin{pgfscope}%
\pgfsetbuttcap%
\pgfsetroundjoin%
\definecolor{currentfill}{rgb}{0.317647,0.317647,0.317647}%
\pgfsetfillcolor{currentfill}%
\pgfsetlinewidth{0.501875pt}%
\definecolor{currentstroke}{rgb}{0.317647,0.317647,0.317647}%
\pgfsetstrokecolor{currentstroke}%
\pgfsetdash{}{0pt}%
\pgfsys@defobject{currentmarker}{\pgfqpoint{0.000000in}{-0.020833in}}{\pgfqpoint{0.000000in}{0.000000in}}{%
\pgfpathmoveto{\pgfqpoint{0.000000in}{0.000000in}}%
\pgfpathlineto{\pgfqpoint{0.000000in}{-0.020833in}}%
\pgfusepath{stroke,fill}%
}%
\begin{pgfscope}%
\pgfsys@transformshift{0.483784in}{0.383578in}%
\pgfsys@useobject{currentmarker}{}%
\end{pgfscope}%
\end{pgfscope}%
\begin{pgfscope}%
\definecolor{textcolor}{rgb}{0.317647,0.317647,0.317647}%
\pgfsetstrokecolor{textcolor}%
\pgfsetfillcolor{textcolor}%
\pgftext[x=0.483784in,y=0.334967in,,top]{\color{textcolor}\rmfamily\fontsize{6.664000}{7.996800}\selectfont \(\displaystyle -600\)}%
\end{pgfscope}%
\begin{pgfscope}%
\pgfsetbuttcap%
\pgfsetroundjoin%
\definecolor{currentfill}{rgb}{0.317647,0.317647,0.317647}%
\pgfsetfillcolor{currentfill}%
\pgfsetlinewidth{0.501875pt}%
\definecolor{currentstroke}{rgb}{0.317647,0.317647,0.317647}%
\pgfsetstrokecolor{currentstroke}%
\pgfsetdash{}{0pt}%
\pgfsys@defobject{currentmarker}{\pgfqpoint{0.000000in}{-0.020833in}}{\pgfqpoint{0.000000in}{0.000000in}}{%
\pgfpathmoveto{\pgfqpoint{0.000000in}{0.000000in}}%
\pgfpathlineto{\pgfqpoint{0.000000in}{-0.020833in}}%
\pgfusepath{stroke,fill}%
}%
\begin{pgfscope}%
\pgfsys@transformshift{0.861219in}{0.383578in}%
\pgfsys@useobject{currentmarker}{}%
\end{pgfscope}%
\end{pgfscope}%
\begin{pgfscope}%
\definecolor{textcolor}{rgb}{0.317647,0.317647,0.317647}%
\pgfsetstrokecolor{textcolor}%
\pgfsetfillcolor{textcolor}%
\pgftext[x=0.861219in,y=0.334967in,,top]{\color{textcolor}\rmfamily\fontsize{6.664000}{7.996800}\selectfont \(\displaystyle -400\)}%
\end{pgfscope}%
\begin{pgfscope}%
\pgfsetbuttcap%
\pgfsetroundjoin%
\definecolor{currentfill}{rgb}{0.317647,0.317647,0.317647}%
\pgfsetfillcolor{currentfill}%
\pgfsetlinewidth{0.501875pt}%
\definecolor{currentstroke}{rgb}{0.317647,0.317647,0.317647}%
\pgfsetstrokecolor{currentstroke}%
\pgfsetdash{}{0pt}%
\pgfsys@defobject{currentmarker}{\pgfqpoint{0.000000in}{-0.020833in}}{\pgfqpoint{0.000000in}{0.000000in}}{%
\pgfpathmoveto{\pgfqpoint{0.000000in}{0.000000in}}%
\pgfpathlineto{\pgfqpoint{0.000000in}{-0.020833in}}%
\pgfusepath{stroke,fill}%
}%
\begin{pgfscope}%
\pgfsys@transformshift{1.238654in}{0.383578in}%
\pgfsys@useobject{currentmarker}{}%
\end{pgfscope}%
\end{pgfscope}%
\begin{pgfscope}%
\definecolor{textcolor}{rgb}{0.317647,0.317647,0.317647}%
\pgfsetstrokecolor{textcolor}%
\pgfsetfillcolor{textcolor}%
\pgftext[x=1.238654in,y=0.334967in,,top]{\color{textcolor}\rmfamily\fontsize{6.664000}{7.996800}\selectfont \(\displaystyle -200\)}%
\end{pgfscope}%
\begin{pgfscope}%
\pgfsetbuttcap%
\pgfsetroundjoin%
\definecolor{currentfill}{rgb}{0.317647,0.317647,0.317647}%
\pgfsetfillcolor{currentfill}%
\pgfsetlinewidth{0.501875pt}%
\definecolor{currentstroke}{rgb}{0.317647,0.317647,0.317647}%
\pgfsetstrokecolor{currentstroke}%
\pgfsetdash{}{0pt}%
\pgfsys@defobject{currentmarker}{\pgfqpoint{0.000000in}{-0.020833in}}{\pgfqpoint{0.000000in}{0.000000in}}{%
\pgfpathmoveto{\pgfqpoint{0.000000in}{0.000000in}}%
\pgfpathlineto{\pgfqpoint{0.000000in}{-0.020833in}}%
\pgfusepath{stroke,fill}%
}%
\begin{pgfscope}%
\pgfsys@transformshift{1.616089in}{0.383578in}%
\pgfsys@useobject{currentmarker}{}%
\end{pgfscope}%
\end{pgfscope}%
\begin{pgfscope}%
\definecolor{textcolor}{rgb}{0.317647,0.317647,0.317647}%
\pgfsetstrokecolor{textcolor}%
\pgfsetfillcolor{textcolor}%
\pgftext[x=1.616089in,y=0.334967in,,top]{\color{textcolor}\rmfamily\fontsize{6.664000}{7.996800}\selectfont \(\displaystyle 0\)}%
\end{pgfscope}%
\begin{pgfscope}%
\pgfsetbuttcap%
\pgfsetroundjoin%
\definecolor{currentfill}{rgb}{0.317647,0.317647,0.317647}%
\pgfsetfillcolor{currentfill}%
\pgfsetlinewidth{0.501875pt}%
\definecolor{currentstroke}{rgb}{0.317647,0.317647,0.317647}%
\pgfsetstrokecolor{currentstroke}%
\pgfsetdash{}{0pt}%
\pgfsys@defobject{currentmarker}{\pgfqpoint{0.000000in}{-0.020833in}}{\pgfqpoint{0.000000in}{0.000000in}}{%
\pgfpathmoveto{\pgfqpoint{0.000000in}{0.000000in}}%
\pgfpathlineto{\pgfqpoint{0.000000in}{-0.020833in}}%
\pgfusepath{stroke,fill}%
}%
\begin{pgfscope}%
\pgfsys@transformshift{1.993524in}{0.383578in}%
\pgfsys@useobject{currentmarker}{}%
\end{pgfscope}%
\end{pgfscope}%
\begin{pgfscope}%
\definecolor{textcolor}{rgb}{0.317647,0.317647,0.317647}%
\pgfsetstrokecolor{textcolor}%
\pgfsetfillcolor{textcolor}%
\pgftext[x=1.993524in,y=0.334967in,,top]{\color{textcolor}\rmfamily\fontsize{6.664000}{7.996800}\selectfont \(\displaystyle 200\)}%
\end{pgfscope}%
\begin{pgfscope}%
\pgfsetbuttcap%
\pgfsetroundjoin%
\definecolor{currentfill}{rgb}{0.317647,0.317647,0.317647}%
\pgfsetfillcolor{currentfill}%
\pgfsetlinewidth{0.501875pt}%
\definecolor{currentstroke}{rgb}{0.317647,0.317647,0.317647}%
\pgfsetstrokecolor{currentstroke}%
\pgfsetdash{}{0pt}%
\pgfsys@defobject{currentmarker}{\pgfqpoint{0.000000in}{-0.020833in}}{\pgfqpoint{0.000000in}{0.000000in}}{%
\pgfpathmoveto{\pgfqpoint{0.000000in}{0.000000in}}%
\pgfpathlineto{\pgfqpoint{0.000000in}{-0.020833in}}%
\pgfusepath{stroke,fill}%
}%
\begin{pgfscope}%
\pgfsys@transformshift{2.370959in}{0.383578in}%
\pgfsys@useobject{currentmarker}{}%
\end{pgfscope}%
\end{pgfscope}%
\begin{pgfscope}%
\definecolor{textcolor}{rgb}{0.317647,0.317647,0.317647}%
\pgfsetstrokecolor{textcolor}%
\pgfsetfillcolor{textcolor}%
\pgftext[x=2.370959in,y=0.334967in,,top]{\color{textcolor}\rmfamily\fontsize{6.664000}{7.996800}\selectfont \(\displaystyle 400\)}%
\end{pgfscope}%
\begin{pgfscope}%
\pgfsetbuttcap%
\pgfsetroundjoin%
\definecolor{currentfill}{rgb}{0.317647,0.317647,0.317647}%
\pgfsetfillcolor{currentfill}%
\pgfsetlinewidth{0.501875pt}%
\definecolor{currentstroke}{rgb}{0.317647,0.317647,0.317647}%
\pgfsetstrokecolor{currentstroke}%
\pgfsetdash{}{0pt}%
\pgfsys@defobject{currentmarker}{\pgfqpoint{0.000000in}{-0.020833in}}{\pgfqpoint{0.000000in}{0.000000in}}{%
\pgfpathmoveto{\pgfqpoint{0.000000in}{0.000000in}}%
\pgfpathlineto{\pgfqpoint{0.000000in}{-0.020833in}}%
\pgfusepath{stroke,fill}%
}%
\begin{pgfscope}%
\pgfsys@transformshift{2.748394in}{0.383578in}%
\pgfsys@useobject{currentmarker}{}%
\end{pgfscope}%
\end{pgfscope}%
\begin{pgfscope}%
\definecolor{textcolor}{rgb}{0.317647,0.317647,0.317647}%
\pgfsetstrokecolor{textcolor}%
\pgfsetfillcolor{textcolor}%
\pgftext[x=2.748394in,y=0.334967in,,top]{\color{textcolor}\rmfamily\fontsize{6.664000}{7.996800}\selectfont \(\displaystyle 600\)}%
\end{pgfscope}%
\begin{pgfscope}%
\definecolor{textcolor}{rgb}{0.317647,0.317647,0.317647}%
\pgfsetstrokecolor{textcolor}%
\pgfsetfillcolor{textcolor}%
\pgftext[x=1.616089in,y=0.197222in,,top]{\color{textcolor}\rmfamily\fontsize{6.664000}{7.996800}\selectfont input frequency \(\displaystyle \nu_\mathrm{in} \; (\si{\kilo \Hz})\)}%
\end{pgfscope}%
\begin{pgfscope}%
\pgfsetbuttcap%
\pgfsetroundjoin%
\definecolor{currentfill}{rgb}{0.317647,0.317647,0.317647}%
\pgfsetfillcolor{currentfill}%
\pgfsetlinewidth{0.501875pt}%
\definecolor{currentstroke}{rgb}{0.317647,0.317647,0.317647}%
\pgfsetstrokecolor{currentstroke}%
\pgfsetdash{}{0pt}%
\pgfsys@defobject{currentmarker}{\pgfqpoint{-0.020833in}{0.000000in}}{\pgfqpoint{0.000000in}{0.000000in}}{%
\pgfpathmoveto{\pgfqpoint{0.000000in}{0.000000in}}%
\pgfpathlineto{\pgfqpoint{-0.020833in}{0.000000in}}%
\pgfusepath{stroke,fill}%
}%
\begin{pgfscope}%
\pgfsys@transformshift{0.453589in}{0.451840in}%
\pgfsys@useobject{currentmarker}{}%
\end{pgfscope}%
\end{pgfscope}%
\begin{pgfscope}%
\definecolor{textcolor}{rgb}{0.317647,0.317647,0.317647}%
\pgfsetstrokecolor{textcolor}%
\pgfsetfillcolor{textcolor}%
\pgftext[x=0.363504in,y=0.419724in,left,base]{\color{textcolor}\rmfamily\fontsize{6.664000}{7.996800}\selectfont \(\displaystyle 0\)}%
\end{pgfscope}%
\begin{pgfscope}%
\pgfsetbuttcap%
\pgfsetroundjoin%
\definecolor{currentfill}{rgb}{0.317647,0.317647,0.317647}%
\pgfsetfillcolor{currentfill}%
\pgfsetlinewidth{0.501875pt}%
\definecolor{currentstroke}{rgb}{0.317647,0.317647,0.317647}%
\pgfsetstrokecolor{currentstroke}%
\pgfsetdash{}{0pt}%
\pgfsys@defobject{currentmarker}{\pgfqpoint{-0.020833in}{0.000000in}}{\pgfqpoint{0.000000in}{0.000000in}}{%
\pgfpathmoveto{\pgfqpoint{0.000000in}{0.000000in}}%
\pgfpathlineto{\pgfqpoint{-0.020833in}{0.000000in}}%
\pgfusepath{stroke,fill}%
}%
\begin{pgfscope}%
\pgfsys@transformshift{0.453589in}{0.724890in}%
\pgfsys@useobject{currentmarker}{}%
\end{pgfscope}%
\end{pgfscope}%
\begin{pgfscope}%
\definecolor{textcolor}{rgb}{0.317647,0.317647,0.317647}%
\pgfsetstrokecolor{textcolor}%
\pgfsetfillcolor{textcolor}%
\pgftext[x=0.308141in,y=0.692773in,left,base]{\color{textcolor}\rmfamily\fontsize{6.664000}{7.996800}\selectfont \(\displaystyle 20\)}%
\end{pgfscope}%
\begin{pgfscope}%
\pgfsetbuttcap%
\pgfsetroundjoin%
\definecolor{currentfill}{rgb}{0.317647,0.317647,0.317647}%
\pgfsetfillcolor{currentfill}%
\pgfsetlinewidth{0.501875pt}%
\definecolor{currentstroke}{rgb}{0.317647,0.317647,0.317647}%
\pgfsetstrokecolor{currentstroke}%
\pgfsetdash{}{0pt}%
\pgfsys@defobject{currentmarker}{\pgfqpoint{-0.020833in}{0.000000in}}{\pgfqpoint{0.000000in}{0.000000in}}{%
\pgfpathmoveto{\pgfqpoint{0.000000in}{0.000000in}}%
\pgfpathlineto{\pgfqpoint{-0.020833in}{0.000000in}}%
\pgfusepath{stroke,fill}%
}%
\begin{pgfscope}%
\pgfsys@transformshift{0.453589in}{0.997940in}%
\pgfsys@useobject{currentmarker}{}%
\end{pgfscope}%
\end{pgfscope}%
\begin{pgfscope}%
\definecolor{textcolor}{rgb}{0.317647,0.317647,0.317647}%
\pgfsetstrokecolor{textcolor}%
\pgfsetfillcolor{textcolor}%
\pgftext[x=0.308141in,y=0.965823in,left,base]{\color{textcolor}\rmfamily\fontsize{6.664000}{7.996800}\selectfont \(\displaystyle 40\)}%
\end{pgfscope}%
\begin{pgfscope}%
\pgfsetbuttcap%
\pgfsetroundjoin%
\definecolor{currentfill}{rgb}{0.317647,0.317647,0.317647}%
\pgfsetfillcolor{currentfill}%
\pgfsetlinewidth{0.501875pt}%
\definecolor{currentstroke}{rgb}{0.317647,0.317647,0.317647}%
\pgfsetstrokecolor{currentstroke}%
\pgfsetdash{}{0pt}%
\pgfsys@defobject{currentmarker}{\pgfqpoint{-0.020833in}{0.000000in}}{\pgfqpoint{0.000000in}{0.000000in}}{%
\pgfpathmoveto{\pgfqpoint{0.000000in}{0.000000in}}%
\pgfpathlineto{\pgfqpoint{-0.020833in}{0.000000in}}%
\pgfusepath{stroke,fill}%
}%
\begin{pgfscope}%
\pgfsys@transformshift{0.453589in}{1.270989in}%
\pgfsys@useobject{currentmarker}{}%
\end{pgfscope}%
\end{pgfscope}%
\begin{pgfscope}%
\definecolor{textcolor}{rgb}{0.317647,0.317647,0.317647}%
\pgfsetstrokecolor{textcolor}%
\pgfsetfillcolor{textcolor}%
\pgftext[x=0.308141in,y=1.238873in,left,base]{\color{textcolor}\rmfamily\fontsize{6.664000}{7.996800}\selectfont \(\displaystyle 60\)}%
\end{pgfscope}%
\begin{pgfscope}%
\pgfsetbuttcap%
\pgfsetroundjoin%
\definecolor{currentfill}{rgb}{0.317647,0.317647,0.317647}%
\pgfsetfillcolor{currentfill}%
\pgfsetlinewidth{0.501875pt}%
\definecolor{currentstroke}{rgb}{0.317647,0.317647,0.317647}%
\pgfsetstrokecolor{currentstroke}%
\pgfsetdash{}{0pt}%
\pgfsys@defobject{currentmarker}{\pgfqpoint{-0.020833in}{0.000000in}}{\pgfqpoint{0.000000in}{0.000000in}}{%
\pgfpathmoveto{\pgfqpoint{0.000000in}{0.000000in}}%
\pgfpathlineto{\pgfqpoint{-0.020833in}{0.000000in}}%
\pgfusepath{stroke,fill}%
}%
\begin{pgfscope}%
\pgfsys@transformshift{0.453589in}{1.544039in}%
\pgfsys@useobject{currentmarker}{}%
\end{pgfscope}%
\end{pgfscope}%
\begin{pgfscope}%
\definecolor{textcolor}{rgb}{0.317647,0.317647,0.317647}%
\pgfsetstrokecolor{textcolor}%
\pgfsetfillcolor{textcolor}%
\pgftext[x=0.308141in,y=1.511922in,left,base]{\color{textcolor}\rmfamily\fontsize{6.664000}{7.996800}\selectfont \(\displaystyle 80\)}%
\end{pgfscope}%
\begin{pgfscope}%
\pgfsetbuttcap%
\pgfsetroundjoin%
\definecolor{currentfill}{rgb}{0.317647,0.317647,0.317647}%
\pgfsetfillcolor{currentfill}%
\pgfsetlinewidth{0.501875pt}%
\definecolor{currentstroke}{rgb}{0.317647,0.317647,0.317647}%
\pgfsetstrokecolor{currentstroke}%
\pgfsetdash{}{0pt}%
\pgfsys@defobject{currentmarker}{\pgfqpoint{-0.020833in}{0.000000in}}{\pgfqpoint{0.000000in}{0.000000in}}{%
\pgfpathmoveto{\pgfqpoint{0.000000in}{0.000000in}}%
\pgfpathlineto{\pgfqpoint{-0.020833in}{0.000000in}}%
\pgfusepath{stroke,fill}%
}%
\begin{pgfscope}%
\pgfsys@transformshift{0.453589in}{1.817089in}%
\pgfsys@useobject{currentmarker}{}%
\end{pgfscope}%
\end{pgfscope}%
\begin{pgfscope}%
\definecolor{textcolor}{rgb}{0.317647,0.317647,0.317647}%
\pgfsetstrokecolor{textcolor}%
\pgfsetfillcolor{textcolor}%
\pgftext[x=0.252778in,y=1.784972in,left,base]{\color{textcolor}\rmfamily\fontsize{6.664000}{7.996800}\selectfont \(\displaystyle 100\)}%
\end{pgfscope}%
\begin{pgfscope}%
\pgfsetbuttcap%
\pgfsetroundjoin%
\definecolor{currentfill}{rgb}{0.317647,0.317647,0.317647}%
\pgfsetfillcolor{currentfill}%
\pgfsetlinewidth{0.501875pt}%
\definecolor{currentstroke}{rgb}{0.317647,0.317647,0.317647}%
\pgfsetstrokecolor{currentstroke}%
\pgfsetdash{}{0pt}%
\pgfsys@defobject{currentmarker}{\pgfqpoint{-0.020833in}{0.000000in}}{\pgfqpoint{0.000000in}{0.000000in}}{%
\pgfpathmoveto{\pgfqpoint{0.000000in}{0.000000in}}%
\pgfpathlineto{\pgfqpoint{-0.020833in}{0.000000in}}%
\pgfusepath{stroke,fill}%
}%
\begin{pgfscope}%
\pgfsys@transformshift{0.453589in}{2.090138in}%
\pgfsys@useobject{currentmarker}{}%
\end{pgfscope}%
\end{pgfscope}%
\begin{pgfscope}%
\definecolor{textcolor}{rgb}{0.317647,0.317647,0.317647}%
\pgfsetstrokecolor{textcolor}%
\pgfsetfillcolor{textcolor}%
\pgftext[x=0.252778in,y=2.058022in,left,base]{\color{textcolor}\rmfamily\fontsize{6.664000}{7.996800}\selectfont \(\displaystyle 120\)}%
\end{pgfscope}%
\begin{pgfscope}%
\definecolor{textcolor}{rgb}{0.317647,0.317647,0.317647}%
\pgfsetstrokecolor{textcolor}%
\pgfsetfillcolor{textcolor}%
\pgftext[x=0.197222in,y=1.346078in,,bottom,rotate=90.000000]{\color{textcolor}\rmfamily\fontsize{6.664000}{7.996800}\selectfont output frequency \(\displaystyle \nu_\mathrm{out} \; (\si{\kilo \Hz})\)}%
\end{pgfscope}%
\begin{pgfscope}%
\pgfpathrectangle{\pgfqpoint{0.453589in}{0.383578in}}{\pgfqpoint{2.325000in}{1.925000in}}%
\pgfusepath{clip}%
\pgfsetbuttcap%
\pgfsetroundjoin%
\definecolor{currentfill}{rgb}{0.333333,0.333333,0.333333}%
\pgfsetfillcolor{currentfill}%
\pgfsetlinewidth{1.003750pt}%
\definecolor{currentstroke}{rgb}{0.333333,0.333333,0.333333}%
\pgfsetstrokecolor{currentstroke}%
\pgfsetdash{}{0pt}%
\pgfsys@defobject{currentmarker}{\pgfqpoint{-0.010417in}{-0.010417in}}{\pgfqpoint{0.010417in}{0.010417in}}{%
\pgfpathmoveto{\pgfqpoint{0.000000in}{-0.010417in}}%
\pgfpathcurveto{\pgfqpoint{0.002763in}{-0.010417in}}{\pgfqpoint{0.005412in}{-0.009319in}}{\pgfqpoint{0.007366in}{-0.007366in}}%
\pgfpathcurveto{\pgfqpoint{0.009319in}{-0.005412in}}{\pgfqpoint{0.010417in}{-0.002763in}}{\pgfqpoint{0.010417in}{0.000000in}}%
\pgfpathcurveto{\pgfqpoint{0.010417in}{0.002763in}}{\pgfqpoint{0.009319in}{0.005412in}}{\pgfqpoint{0.007366in}{0.007366in}}%
\pgfpathcurveto{\pgfqpoint{0.005412in}{0.009319in}}{\pgfqpoint{0.002763in}{0.010417in}}{\pgfqpoint{0.000000in}{0.010417in}}%
\pgfpathcurveto{\pgfqpoint{-0.002763in}{0.010417in}}{\pgfqpoint{-0.005412in}{0.009319in}}{\pgfqpoint{-0.007366in}{0.007366in}}%
\pgfpathcurveto{\pgfqpoint{-0.009319in}{0.005412in}}{\pgfqpoint{-0.010417in}{0.002763in}}{\pgfqpoint{-0.010417in}{0.000000in}}%
\pgfpathcurveto{\pgfqpoint{-0.010417in}{-0.002763in}}{\pgfqpoint{-0.009319in}{-0.005412in}}{\pgfqpoint{-0.007366in}{-0.007366in}}%
\pgfpathcurveto{\pgfqpoint{-0.005412in}{-0.009319in}}{\pgfqpoint{-0.002763in}{-0.010417in}}{\pgfqpoint{0.000000in}{-0.010417in}}%
\pgfpathclose%
\pgfusepath{stroke,fill}%
}%
\begin{pgfscope}%
\pgfsys@transformshift{0.559271in}{0.463712in}%
\pgfsys@useobject{currentmarker}{}%
\end{pgfscope}%
\begin{pgfscope}%
\pgfsys@transformshift{0.617983in}{0.457776in}%
\pgfsys@useobject{currentmarker}{}%
\end{pgfscope}%
\begin{pgfscope}%
\pgfsys@transformshift{0.676695in}{0.457776in}%
\pgfsys@useobject{currentmarker}{}%
\end{pgfscope}%
\begin{pgfscope}%
\pgfsys@transformshift{0.735407in}{0.463712in}%
\pgfsys@useobject{currentmarker}{}%
\end{pgfscope}%
\begin{pgfscope}%
\pgfsys@transformshift{0.794119in}{0.469648in}%
\pgfsys@useobject{currentmarker}{}%
\end{pgfscope}%
\begin{pgfscope}%
\pgfsys@transformshift{0.852831in}{0.475584in}%
\pgfsys@useobject{currentmarker}{}%
\end{pgfscope}%
\begin{pgfscope}%
\pgfsys@transformshift{0.911543in}{0.499327in}%
\pgfsys@useobject{currentmarker}{}%
\end{pgfscope}%
\begin{pgfscope}%
\pgfsys@transformshift{0.970255in}{0.499327in}%
\pgfsys@useobject{currentmarker}{}%
\end{pgfscope}%
\begin{pgfscope}%
\pgfsys@transformshift{1.028968in}{0.511199in}%
\pgfsys@useobject{currentmarker}{}%
\end{pgfscope}%
\begin{pgfscope}%
\pgfsys@transformshift{1.087680in}{0.511199in}%
\pgfsys@useobject{currentmarker}{}%
\end{pgfscope}%
\begin{pgfscope}%
\pgfsys@transformshift{1.146392in}{0.564622in}%
\pgfsys@useobject{currentmarker}{}%
\end{pgfscope}%
\begin{pgfscope}%
\pgfsys@transformshift{1.205104in}{0.576494in}%
\pgfsys@useobject{currentmarker}{}%
\end{pgfscope}%
\begin{pgfscope}%
\pgfsys@transformshift{1.263816in}{0.606173in}%
\pgfsys@useobject{currentmarker}{}%
\end{pgfscope}%
\begin{pgfscope}%
\pgfsys@transformshift{1.322528in}{0.695211in}%
\pgfsys@useobject{currentmarker}{}%
\end{pgfscope}%
\begin{pgfscope}%
\pgfsys@transformshift{1.381240in}{0.689275in}%
\pgfsys@useobject{currentmarker}{}%
\end{pgfscope}%
\begin{pgfscope}%
\pgfsys@transformshift{1.439952in}{0.707083in}%
\pgfsys@useobject{currentmarker}{}%
\end{pgfscope}%
\begin{pgfscope}%
\pgfsys@transformshift{1.498665in}{0.784249in}%
\pgfsys@useobject{currentmarker}{}%
\end{pgfscope}%
\begin{pgfscope}%
\pgfsys@transformshift{1.557377in}{0.885158in}%
\pgfsys@useobject{currentmarker}{}%
\end{pgfscope}%
\begin{pgfscope}%
\pgfsys@transformshift{1.616089in}{0.897030in}%
\pgfsys@useobject{currentmarker}{}%
\end{pgfscope}%
\begin{pgfscope}%
\pgfsys@transformshift{1.674801in}{1.021683in}%
\pgfsys@useobject{currentmarker}{}%
\end{pgfscope}%
\begin{pgfscope}%
\pgfsys@transformshift{1.733513in}{0.968260in}%
\pgfsys@useobject{currentmarker}{}%
\end{pgfscope}%
\begin{pgfscope}%
\pgfsys@transformshift{1.792225in}{1.134465in}%
\pgfsys@useobject{currentmarker}{}%
\end{pgfscope}%
\begin{pgfscope}%
\pgfsys@transformshift{1.850937in}{1.294733in}%
\pgfsys@useobject{currentmarker}{}%
\end{pgfscope}%
\begin{pgfscope}%
\pgfsys@transformshift{1.909649in}{1.443129in}%
\pgfsys@useobject{currentmarker}{}%
\end{pgfscope}%
\begin{pgfscope}%
\pgfsys@transformshift{1.968361in}{1.401578in}%
\pgfsys@useobject{currentmarker}{}%
\end{pgfscope}%
\begin{pgfscope}%
\pgfsys@transformshift{2.027074in}{1.478745in}%
\pgfsys@useobject{currentmarker}{}%
\end{pgfscope}%
\begin{pgfscope}%
\pgfsys@transformshift{2.085786in}{1.603398in}%
\pgfsys@useobject{currentmarker}{}%
\end{pgfscope}%
\begin{pgfscope}%
\pgfsys@transformshift{2.144498in}{1.674628in}%
\pgfsys@useobject{currentmarker}{}%
\end{pgfscope}%
\begin{pgfscope}%
\pgfsys@transformshift{2.203210in}{1.692436in}%
\pgfsys@useobject{currentmarker}{}%
\end{pgfscope}%
\begin{pgfscope}%
\pgfsys@transformshift{2.261922in}{1.745858in}%
\pgfsys@useobject{currentmarker}{}%
\end{pgfscope}%
\begin{pgfscope}%
\pgfsys@transformshift{2.320634in}{1.757730in}%
\pgfsys@useobject{currentmarker}{}%
\end{pgfscope}%
\begin{pgfscope}%
\pgfsys@transformshift{2.379346in}{1.739922in}%
\pgfsys@useobject{currentmarker}{}%
\end{pgfscope}%
\begin{pgfscope}%
\pgfsys@transformshift{2.438058in}{1.769602in}%
\pgfsys@useobject{currentmarker}{}%
\end{pgfscope}%
\begin{pgfscope}%
\pgfsys@transformshift{2.496771in}{1.781474in}%
\pgfsys@useobject{currentmarker}{}%
\end{pgfscope}%
\begin{pgfscope}%
\pgfsys@transformshift{2.555483in}{1.775538in}%
\pgfsys@useobject{currentmarker}{}%
\end{pgfscope}%
\begin{pgfscope}%
\pgfsys@transformshift{2.614195in}{1.787409in}%
\pgfsys@useobject{currentmarker}{}%
\end{pgfscope}%
\begin{pgfscope}%
\pgfsys@transformshift{2.672907in}{1.817089in}%
\pgfsys@useobject{currentmarker}{}%
\end{pgfscope}%
\end{pgfscope}%
\begin{pgfscope}%
\pgfpathrectangle{\pgfqpoint{0.453589in}{0.383578in}}{\pgfqpoint{2.325000in}{1.925000in}}%
\pgfusepath{clip}%
\pgfsetrectcap%
\pgfsetroundjoin%
\pgfsetlinewidth{0.803000pt}%
\definecolor{currentstroke}{rgb}{0.333333,0.333333,0.333333}%
\pgfsetstrokecolor{currentstroke}%
\pgfsetdash{}{0pt}%
\pgfpathmoveto{\pgfqpoint{0.559271in}{0.459024in}}%
\pgfpathlineto{\pgfqpoint{0.617983in}{0.461102in}}%
\pgfpathlineto{\pgfqpoint{0.676695in}{0.463775in}}%
\pgfpathlineto{\pgfqpoint{0.735407in}{0.467212in}}%
\pgfpathlineto{\pgfqpoint{0.794119in}{0.471624in}}%
\pgfpathlineto{\pgfqpoint{0.852831in}{0.477279in}}%
\pgfpathlineto{\pgfqpoint{0.911543in}{0.484512in}}%
\pgfpathlineto{\pgfqpoint{0.970255in}{0.493739in}}%
\pgfpathlineto{\pgfqpoint{1.028968in}{0.505470in}}%
\pgfpathlineto{\pgfqpoint{1.087680in}{0.520320in}}%
\pgfpathlineto{\pgfqpoint{1.146392in}{0.539015in}}%
\pgfpathlineto{\pgfqpoint{1.205104in}{0.562391in}}%
\pgfpathlineto{\pgfqpoint{1.263816in}{0.591367in}}%
\pgfpathlineto{\pgfqpoint{1.322528in}{0.626907in}}%
\pgfpathlineto{\pgfqpoint{1.381240in}{0.669932in}}%
\pgfpathlineto{\pgfqpoint{1.439952in}{0.721205in}}%
\pgfpathlineto{\pgfqpoint{1.498665in}{0.781170in}}%
\pgfpathlineto{\pgfqpoint{1.557377in}{0.849783in}}%
\pgfpathlineto{\pgfqpoint{1.616089in}{0.926351in}}%
\pgfpathlineto{\pgfqpoint{1.674801in}{1.009446in}}%
\pgfpathlineto{\pgfqpoint{1.733513in}{1.096942in}}%
\pgfpathlineto{\pgfqpoint{1.792225in}{1.186187in}}%
\pgfpathlineto{\pgfqpoint{1.850937in}{1.274312in}}%
\pgfpathlineto{\pgfqpoint{1.909649in}{1.358590in}}%
\pgfpathlineto{\pgfqpoint{1.968361in}{1.436756in}}%
\pgfpathlineto{\pgfqpoint{2.027074in}{1.507219in}}%
\pgfpathlineto{\pgfqpoint{2.085786in}{1.569128in}}%
\pgfpathlineto{\pgfqpoint{2.144498in}{1.622305in}}%
\pgfpathlineto{\pgfqpoint{2.203210in}{1.667103in}}%
\pgfpathlineto{\pgfqpoint{2.261922in}{1.704227in}}%
\pgfpathlineto{\pgfqpoint{2.320634in}{1.734577in}}%
\pgfpathlineto{\pgfqpoint{2.379346in}{1.759113in}}%
\pgfpathlineto{\pgfqpoint{2.438058in}{1.778772in}}%
\pgfpathlineto{\pgfqpoint{2.496771in}{1.794408in}}%
\pgfpathlineto{\pgfqpoint{2.555483in}{1.806774in}}%
\pgfpathlineto{\pgfqpoint{2.614195in}{1.816509in}}%
\pgfpathlineto{\pgfqpoint{2.672907in}{1.824146in}}%
\pgfusepath{stroke}%
\end{pgfscope}%
\begin{pgfscope}%
\pgfpathrectangle{\pgfqpoint{0.453589in}{0.383578in}}{\pgfqpoint{2.325000in}{1.925000in}}%
\pgfusepath{clip}%
\pgfsetbuttcap%
\pgfsetroundjoin%
\definecolor{currentfill}{rgb}{0.686275,0.352941,0.313725}%
\pgfsetfillcolor{currentfill}%
\pgfsetlinewidth{1.003750pt}%
\definecolor{currentstroke}{rgb}{0.686275,0.352941,0.313725}%
\pgfsetstrokecolor{currentstroke}%
\pgfsetdash{}{0pt}%
\pgfsys@defobject{currentmarker}{\pgfqpoint{-0.010417in}{-0.010417in}}{\pgfqpoint{0.010417in}{0.010417in}}{%
\pgfpathmoveto{\pgfqpoint{0.000000in}{-0.010417in}}%
\pgfpathcurveto{\pgfqpoint{0.002763in}{-0.010417in}}{\pgfqpoint{0.005412in}{-0.009319in}}{\pgfqpoint{0.007366in}{-0.007366in}}%
\pgfpathcurveto{\pgfqpoint{0.009319in}{-0.005412in}}{\pgfqpoint{0.010417in}{-0.002763in}}{\pgfqpoint{0.010417in}{0.000000in}}%
\pgfpathcurveto{\pgfqpoint{0.010417in}{0.002763in}}{\pgfqpoint{0.009319in}{0.005412in}}{\pgfqpoint{0.007366in}{0.007366in}}%
\pgfpathcurveto{\pgfqpoint{0.005412in}{0.009319in}}{\pgfqpoint{0.002763in}{0.010417in}}{\pgfqpoint{0.000000in}{0.010417in}}%
\pgfpathcurveto{\pgfqpoint{-0.002763in}{0.010417in}}{\pgfqpoint{-0.005412in}{0.009319in}}{\pgfqpoint{-0.007366in}{0.007366in}}%
\pgfpathcurveto{\pgfqpoint{-0.009319in}{0.005412in}}{\pgfqpoint{-0.010417in}{0.002763in}}{\pgfqpoint{-0.010417in}{0.000000in}}%
\pgfpathcurveto{\pgfqpoint{-0.010417in}{-0.002763in}}{\pgfqpoint{-0.009319in}{-0.005412in}}{\pgfqpoint{-0.007366in}{-0.007366in}}%
\pgfpathcurveto{\pgfqpoint{-0.005412in}{-0.009319in}}{\pgfqpoint{-0.002763in}{-0.010417in}}{\pgfqpoint{0.000000in}{-0.010417in}}%
\pgfpathclose%
\pgfusepath{stroke,fill}%
}%
\begin{pgfscope}%
\pgfsys@transformshift{0.559271in}{0.451840in}%
\pgfsys@useobject{currentmarker}{}%
\end{pgfscope}%
\begin{pgfscope}%
\pgfsys@transformshift{0.617983in}{0.451840in}%
\pgfsys@useobject{currentmarker}{}%
\end{pgfscope}%
\begin{pgfscope}%
\pgfsys@transformshift{0.676695in}{0.451840in}%
\pgfsys@useobject{currentmarker}{}%
\end{pgfscope}%
\begin{pgfscope}%
\pgfsys@transformshift{0.735407in}{0.451840in}%
\pgfsys@useobject{currentmarker}{}%
\end{pgfscope}%
\begin{pgfscope}%
\pgfsys@transformshift{0.794119in}{0.451840in}%
\pgfsys@useobject{currentmarker}{}%
\end{pgfscope}%
\begin{pgfscope}%
\pgfsys@transformshift{0.852831in}{0.451840in}%
\pgfsys@useobject{currentmarker}{}%
\end{pgfscope}%
\begin{pgfscope}%
\pgfsys@transformshift{0.911543in}{0.451840in}%
\pgfsys@useobject{currentmarker}{}%
\end{pgfscope}%
\begin{pgfscope}%
\pgfsys@transformshift{0.970255in}{0.451840in}%
\pgfsys@useobject{currentmarker}{}%
\end{pgfscope}%
\begin{pgfscope}%
\pgfsys@transformshift{1.028968in}{0.451840in}%
\pgfsys@useobject{currentmarker}{}%
\end{pgfscope}%
\begin{pgfscope}%
\pgfsys@transformshift{1.087680in}{0.451840in}%
\pgfsys@useobject{currentmarker}{}%
\end{pgfscope}%
\begin{pgfscope}%
\pgfsys@transformshift{1.146392in}{0.451840in}%
\pgfsys@useobject{currentmarker}{}%
\end{pgfscope}%
\begin{pgfscope}%
\pgfsys@transformshift{1.205104in}{0.457776in}%
\pgfsys@useobject{currentmarker}{}%
\end{pgfscope}%
\begin{pgfscope}%
\pgfsys@transformshift{1.263816in}{0.463712in}%
\pgfsys@useobject{currentmarker}{}%
\end{pgfscope}%
\begin{pgfscope}%
\pgfsys@transformshift{1.322528in}{0.493391in}%
\pgfsys@useobject{currentmarker}{}%
\end{pgfscope}%
\begin{pgfscope}%
\pgfsys@transformshift{1.381240in}{0.523071in}%
\pgfsys@useobject{currentmarker}{}%
\end{pgfscope}%
\begin{pgfscope}%
\pgfsys@transformshift{1.439952in}{0.534943in}%
\pgfsys@useobject{currentmarker}{}%
\end{pgfscope}%
\begin{pgfscope}%
\pgfsys@transformshift{1.498665in}{0.588365in}%
\pgfsys@useobject{currentmarker}{}%
\end{pgfscope}%
\begin{pgfscope}%
\pgfsys@transformshift{1.557377in}{0.689275in}%
\pgfsys@useobject{currentmarker}{}%
\end{pgfscope}%
\begin{pgfscope}%
\pgfsys@transformshift{1.616089in}{0.879223in}%
\pgfsys@useobject{currentmarker}{}%
\end{pgfscope}%
\begin{pgfscope}%
\pgfsys@transformshift{1.674801in}{1.223503in}%
\pgfsys@useobject{currentmarker}{}%
\end{pgfscope}%
\begin{pgfscope}%
\pgfsys@transformshift{1.733513in}{1.449065in}%
\pgfsys@useobject{currentmarker}{}%
\end{pgfscope}%
\begin{pgfscope}%
\pgfsys@transformshift{1.792225in}{1.425322in}%
\pgfsys@useobject{currentmarker}{}%
\end{pgfscope}%
\begin{pgfscope}%
\pgfsys@transformshift{1.850937in}{1.639013in}%
\pgfsys@useobject{currentmarker}{}%
\end{pgfscope}%
\begin{pgfscope}%
\pgfsys@transformshift{1.909649in}{1.745858in}%
\pgfsys@useobject{currentmarker}{}%
\end{pgfscope}%
\begin{pgfscope}%
\pgfsys@transformshift{1.968361in}{1.781474in}%
\pgfsys@useobject{currentmarker}{}%
\end{pgfscope}%
\begin{pgfscope}%
\pgfsys@transformshift{2.027074in}{1.823025in}%
\pgfsys@useobject{currentmarker}{}%
\end{pgfscope}%
\begin{pgfscope}%
\pgfsys@transformshift{2.085786in}{1.852704in}%
\pgfsys@useobject{currentmarker}{}%
\end{pgfscope}%
\begin{pgfscope}%
\pgfsys@transformshift{2.144498in}{1.876447in}%
\pgfsys@useobject{currentmarker}{}%
\end{pgfscope}%
\begin{pgfscope}%
\pgfsys@transformshift{2.203210in}{1.876447in}%
\pgfsys@useobject{currentmarker}{}%
\end{pgfscope}%
\begin{pgfscope}%
\pgfsys@transformshift{2.261922in}{1.894255in}%
\pgfsys@useobject{currentmarker}{}%
\end{pgfscope}%
\begin{pgfscope}%
\pgfsys@transformshift{2.320634in}{1.900191in}%
\pgfsys@useobject{currentmarker}{}%
\end{pgfscope}%
\begin{pgfscope}%
\pgfsys@transformshift{2.379346in}{1.906127in}%
\pgfsys@useobject{currentmarker}{}%
\end{pgfscope}%
\begin{pgfscope}%
\pgfsys@transformshift{2.438058in}{1.906127in}%
\pgfsys@useobject{currentmarker}{}%
\end{pgfscope}%
\begin{pgfscope}%
\pgfsys@transformshift{2.496771in}{1.912062in}%
\pgfsys@useobject{currentmarker}{}%
\end{pgfscope}%
\begin{pgfscope}%
\pgfsys@transformshift{2.555483in}{1.912062in}%
\pgfsys@useobject{currentmarker}{}%
\end{pgfscope}%
\begin{pgfscope}%
\pgfsys@transformshift{2.614195in}{1.917998in}%
\pgfsys@useobject{currentmarker}{}%
\end{pgfscope}%
\begin{pgfscope}%
\pgfsys@transformshift{2.672907in}{1.917998in}%
\pgfsys@useobject{currentmarker}{}%
\end{pgfscope}%
\end{pgfscope}%
\begin{pgfscope}%
\pgfpathrectangle{\pgfqpoint{0.453589in}{0.383578in}}{\pgfqpoint{2.325000in}{1.925000in}}%
\pgfusepath{clip}%
\pgfsetrectcap%
\pgfsetroundjoin%
\pgfsetlinewidth{0.803000pt}%
\definecolor{currentstroke}{rgb}{0.686275,0.352941,0.313725}%
\pgfsetstrokecolor{currentstroke}%
\pgfsetdash{}{0pt}%
\pgfpathmoveto{\pgfqpoint{0.559271in}{0.451853in}}%
\pgfpathlineto{\pgfqpoint{0.617983in}{0.451864in}}%
\pgfpathlineto{\pgfqpoint{0.676695in}{0.451883in}}%
\pgfpathlineto{\pgfqpoint{0.735407in}{0.451919in}}%
\pgfpathlineto{\pgfqpoint{0.794119in}{0.451984in}}%
\pgfpathlineto{\pgfqpoint{0.852831in}{0.452103in}}%
\pgfpathlineto{\pgfqpoint{0.911543in}{0.452321in}}%
\pgfpathlineto{\pgfqpoint{0.970255in}{0.452721in}}%
\pgfpathlineto{\pgfqpoint{1.028968in}{0.453453in}}%
\pgfpathlineto{\pgfqpoint{1.087680in}{0.454792in}}%
\pgfpathlineto{\pgfqpoint{1.146392in}{0.457238in}}%
\pgfpathlineto{\pgfqpoint{1.205104in}{0.461698in}}%
\pgfpathlineto{\pgfqpoint{1.263816in}{0.469796in}}%
\pgfpathlineto{\pgfqpoint{1.322528in}{0.484396in}}%
\pgfpathlineto{\pgfqpoint{1.381240in}{0.510377in}}%
\pgfpathlineto{\pgfqpoint{1.439952in}{0.555569in}}%
\pgfpathlineto{\pgfqpoint{1.498665in}{0.631129in}}%
\pgfpathlineto{\pgfqpoint{1.557377in}{0.749487in}}%
\pgfpathlineto{\pgfqpoint{1.616089in}{0.917186in}}%
\pgfpathlineto{\pgfqpoint{1.674801in}{1.123884in}}%
\pgfpathlineto{\pgfqpoint{1.733513in}{1.338999in}}%
\pgfpathlineto{\pgfqpoint{1.792225in}{1.526840in}}%
\pgfpathlineto{\pgfqpoint{1.850937in}{1.667331in}}%
\pgfpathlineto{\pgfqpoint{1.909649in}{1.760709in}}%
\pgfpathlineto{\pgfqpoint{1.968361in}{1.818002in}}%
\pgfpathlineto{\pgfqpoint{2.027074in}{1.851446in}}%
\pgfpathlineto{\pgfqpoint{2.085786in}{1.870403in}}%
\pgfpathlineto{\pgfqpoint{2.144498in}{1.880970in}}%
\pgfpathlineto{\pgfqpoint{2.203210in}{1.886805in}}%
\pgfpathlineto{\pgfqpoint{2.261922in}{1.890011in}}%
\pgfpathlineto{\pgfqpoint{2.320634in}{1.891767in}}%
\pgfpathlineto{\pgfqpoint{2.379346in}{1.892727in}}%
\pgfpathlineto{\pgfqpoint{2.438058in}{1.893252in}}%
\pgfpathlineto{\pgfqpoint{2.496771in}{1.893539in}}%
\pgfpathlineto{\pgfqpoint{2.555483in}{1.893695in}}%
\pgfpathlineto{\pgfqpoint{2.614195in}{1.893781in}}%
\pgfpathlineto{\pgfqpoint{2.672907in}{1.893827in}}%
\pgfusepath{stroke}%
\end{pgfscope}%
\begin{pgfscope}%
\pgfpathrectangle{\pgfqpoint{0.453589in}{0.383578in}}{\pgfqpoint{2.325000in}{1.925000in}}%
\pgfusepath{clip}%
\pgfsetbuttcap%
\pgfsetroundjoin%
\definecolor{currentfill}{rgb}{0.000000,0.356863,0.509804}%
\pgfsetfillcolor{currentfill}%
\pgfsetlinewidth{1.003750pt}%
\definecolor{currentstroke}{rgb}{0.000000,0.356863,0.509804}%
\pgfsetstrokecolor{currentstroke}%
\pgfsetdash{}{0pt}%
\pgfsys@defobject{currentmarker}{\pgfqpoint{-0.010417in}{-0.010417in}}{\pgfqpoint{0.010417in}{0.010417in}}{%
\pgfpathmoveto{\pgfqpoint{0.000000in}{-0.010417in}}%
\pgfpathcurveto{\pgfqpoint{0.002763in}{-0.010417in}}{\pgfqpoint{0.005412in}{-0.009319in}}{\pgfqpoint{0.007366in}{-0.007366in}}%
\pgfpathcurveto{\pgfqpoint{0.009319in}{-0.005412in}}{\pgfqpoint{0.010417in}{-0.002763in}}{\pgfqpoint{0.010417in}{0.000000in}}%
\pgfpathcurveto{\pgfqpoint{0.010417in}{0.002763in}}{\pgfqpoint{0.009319in}{0.005412in}}{\pgfqpoint{0.007366in}{0.007366in}}%
\pgfpathcurveto{\pgfqpoint{0.005412in}{0.009319in}}{\pgfqpoint{0.002763in}{0.010417in}}{\pgfqpoint{0.000000in}{0.010417in}}%
\pgfpathcurveto{\pgfqpoint{-0.002763in}{0.010417in}}{\pgfqpoint{-0.005412in}{0.009319in}}{\pgfqpoint{-0.007366in}{0.007366in}}%
\pgfpathcurveto{\pgfqpoint{-0.009319in}{0.005412in}}{\pgfqpoint{-0.010417in}{0.002763in}}{\pgfqpoint{-0.010417in}{0.000000in}}%
\pgfpathcurveto{\pgfqpoint{-0.010417in}{-0.002763in}}{\pgfqpoint{-0.009319in}{-0.005412in}}{\pgfqpoint{-0.007366in}{-0.007366in}}%
\pgfpathcurveto{\pgfqpoint{-0.005412in}{-0.009319in}}{\pgfqpoint{-0.002763in}{-0.010417in}}{\pgfqpoint{0.000000in}{-0.010417in}}%
\pgfpathclose%
\pgfusepath{stroke,fill}%
}%
\begin{pgfscope}%
\pgfsys@transformshift{0.559271in}{0.451840in}%
\pgfsys@useobject{currentmarker}{}%
\end{pgfscope}%
\begin{pgfscope}%
\pgfsys@transformshift{0.617983in}{0.451840in}%
\pgfsys@useobject{currentmarker}{}%
\end{pgfscope}%
\begin{pgfscope}%
\pgfsys@transformshift{0.676695in}{0.451840in}%
\pgfsys@useobject{currentmarker}{}%
\end{pgfscope}%
\begin{pgfscope}%
\pgfsys@transformshift{0.735407in}{0.451840in}%
\pgfsys@useobject{currentmarker}{}%
\end{pgfscope}%
\begin{pgfscope}%
\pgfsys@transformshift{0.794119in}{0.451840in}%
\pgfsys@useobject{currentmarker}{}%
\end{pgfscope}%
\begin{pgfscope}%
\pgfsys@transformshift{0.852831in}{0.451840in}%
\pgfsys@useobject{currentmarker}{}%
\end{pgfscope}%
\begin{pgfscope}%
\pgfsys@transformshift{0.911543in}{0.451840in}%
\pgfsys@useobject{currentmarker}{}%
\end{pgfscope}%
\begin{pgfscope}%
\pgfsys@transformshift{0.970255in}{0.451840in}%
\pgfsys@useobject{currentmarker}{}%
\end{pgfscope}%
\begin{pgfscope}%
\pgfsys@transformshift{1.028968in}{0.451840in}%
\pgfsys@useobject{currentmarker}{}%
\end{pgfscope}%
\begin{pgfscope}%
\pgfsys@transformshift{1.087680in}{0.451840in}%
\pgfsys@useobject{currentmarker}{}%
\end{pgfscope}%
\begin{pgfscope}%
\pgfsys@transformshift{1.146392in}{0.451840in}%
\pgfsys@useobject{currentmarker}{}%
\end{pgfscope}%
\begin{pgfscope}%
\pgfsys@transformshift{1.205104in}{0.451840in}%
\pgfsys@useobject{currentmarker}{}%
\end{pgfscope}%
\begin{pgfscope}%
\pgfsys@transformshift{1.263816in}{0.451840in}%
\pgfsys@useobject{currentmarker}{}%
\end{pgfscope}%
\begin{pgfscope}%
\pgfsys@transformshift{1.322528in}{0.463712in}%
\pgfsys@useobject{currentmarker}{}%
\end{pgfscope}%
\begin{pgfscope}%
\pgfsys@transformshift{1.381240in}{0.469648in}%
\pgfsys@useobject{currentmarker}{}%
\end{pgfscope}%
\begin{pgfscope}%
\pgfsys@transformshift{1.439952in}{0.493391in}%
\pgfsys@useobject{currentmarker}{}%
\end{pgfscope}%
\begin{pgfscope}%
\pgfsys@transformshift{1.498665in}{0.564622in}%
\pgfsys@useobject{currentmarker}{}%
\end{pgfscope}%
\begin{pgfscope}%
\pgfsys@transformshift{1.557377in}{0.600237in}%
\pgfsys@useobject{currentmarker}{}%
\end{pgfscope}%
\begin{pgfscope}%
\pgfsys@transformshift{1.616089in}{0.754569in}%
\pgfsys@useobject{currentmarker}{}%
\end{pgfscope}%
\begin{pgfscope}%
\pgfsys@transformshift{1.674801in}{1.247246in}%
\pgfsys@useobject{currentmarker}{}%
\end{pgfscope}%
\begin{pgfscope}%
\pgfsys@transformshift{1.733513in}{1.603398in}%
\pgfsys@useobject{currentmarker}{}%
\end{pgfscope}%
\begin{pgfscope}%
\pgfsys@transformshift{1.792225in}{1.728051in}%
\pgfsys@useobject{currentmarker}{}%
\end{pgfscope}%
\begin{pgfscope}%
\pgfsys@transformshift{1.850937in}{1.823025in}%
\pgfsys@useobject{currentmarker}{}%
\end{pgfscope}%
\begin{pgfscope}%
\pgfsys@transformshift{1.909649in}{1.876447in}%
\pgfsys@useobject{currentmarker}{}%
\end{pgfscope}%
\begin{pgfscope}%
\pgfsys@transformshift{1.968361in}{1.906127in}%
\pgfsys@useobject{currentmarker}{}%
\end{pgfscope}%
\begin{pgfscope}%
\pgfsys@transformshift{2.027074in}{1.917998in}%
\pgfsys@useobject{currentmarker}{}%
\end{pgfscope}%
\begin{pgfscope}%
\pgfsys@transformshift{2.085786in}{1.923934in}%
\pgfsys@useobject{currentmarker}{}%
\end{pgfscope}%
\begin{pgfscope}%
\pgfsys@transformshift{2.144498in}{1.935806in}%
\pgfsys@useobject{currentmarker}{}%
\end{pgfscope}%
\begin{pgfscope}%
\pgfsys@transformshift{2.203210in}{1.935806in}%
\pgfsys@useobject{currentmarker}{}%
\end{pgfscope}%
\begin{pgfscope}%
\pgfsys@transformshift{2.261922in}{1.947678in}%
\pgfsys@useobject{currentmarker}{}%
\end{pgfscope}%
\begin{pgfscope}%
\pgfsys@transformshift{2.320634in}{1.947678in}%
\pgfsys@useobject{currentmarker}{}%
\end{pgfscope}%
\begin{pgfscope}%
\pgfsys@transformshift{2.379346in}{1.953614in}%
\pgfsys@useobject{currentmarker}{}%
\end{pgfscope}%
\begin{pgfscope}%
\pgfsys@transformshift{2.438058in}{1.953614in}%
\pgfsys@useobject{currentmarker}{}%
\end{pgfscope}%
\begin{pgfscope}%
\pgfsys@transformshift{2.496771in}{1.953614in}%
\pgfsys@useobject{currentmarker}{}%
\end{pgfscope}%
\begin{pgfscope}%
\pgfsys@transformshift{2.555483in}{1.959549in}%
\pgfsys@useobject{currentmarker}{}%
\end{pgfscope}%
\begin{pgfscope}%
\pgfsys@transformshift{2.614195in}{1.953614in}%
\pgfsys@useobject{currentmarker}{}%
\end{pgfscope}%
\begin{pgfscope}%
\pgfsys@transformshift{2.672907in}{1.947678in}%
\pgfsys@useobject{currentmarker}{}%
\end{pgfscope}%
\end{pgfscope}%
\begin{pgfscope}%
\pgfpathrectangle{\pgfqpoint{0.453589in}{0.383578in}}{\pgfqpoint{2.325000in}{1.925000in}}%
\pgfusepath{clip}%
\pgfsetrectcap%
\pgfsetroundjoin%
\pgfsetlinewidth{0.803000pt}%
\definecolor{currentstroke}{rgb}{0.000000,0.356863,0.509804}%
\pgfsetstrokecolor{currentstroke}%
\pgfsetdash{}{0pt}%
\pgfpathmoveto{\pgfqpoint{0.559271in}{0.451840in}}%
\pgfpathlineto{\pgfqpoint{0.617983in}{0.451840in}}%
\pgfpathlineto{\pgfqpoint{0.676695in}{0.451840in}}%
\pgfpathlineto{\pgfqpoint{0.735407in}{0.451841in}}%
\pgfpathlineto{\pgfqpoint{0.794119in}{0.451841in}}%
\pgfpathlineto{\pgfqpoint{0.852831in}{0.451841in}}%
\pgfpathlineto{\pgfqpoint{0.911543in}{0.451842in}}%
\pgfpathlineto{\pgfqpoint{0.970255in}{0.451845in}}%
\pgfpathlineto{\pgfqpoint{1.028968in}{0.451853in}}%
\pgfpathlineto{\pgfqpoint{1.087680in}{0.451876in}}%
\pgfpathlineto{\pgfqpoint{1.146392in}{0.451943in}}%
\pgfpathlineto{\pgfqpoint{1.205104in}{0.452138in}}%
\pgfpathlineto{\pgfqpoint{1.263816in}{0.452705in}}%
\pgfpathlineto{\pgfqpoint{1.322528in}{0.454345in}}%
\pgfpathlineto{\pgfqpoint{1.381240in}{0.459084in}}%
\pgfpathlineto{\pgfqpoint{1.439952in}{0.472664in}}%
\pgfpathlineto{\pgfqpoint{1.498665in}{0.510688in}}%
\pgfpathlineto{\pgfqpoint{1.557377in}{0.610607in}}%
\pgfpathlineto{\pgfqpoint{1.616089in}{0.834584in}}%
\pgfpathlineto{\pgfqpoint{1.674801in}{1.196788in}}%
\pgfpathlineto{\pgfqpoint{1.733513in}{1.557363in}}%
\pgfpathlineto{\pgfqpoint{1.792225in}{1.778714in}}%
\pgfpathlineto{\pgfqpoint{1.850937in}{1.877065in}}%
\pgfpathlineto{\pgfqpoint{1.909649in}{1.914429in}}%
\pgfpathlineto{\pgfqpoint{1.968361in}{1.927764in}}%
\pgfpathlineto{\pgfqpoint{2.027074in}{1.932417in}}%
\pgfpathlineto{\pgfqpoint{2.085786in}{1.934027in}}%
\pgfpathlineto{\pgfqpoint{2.144498in}{1.934583in}}%
\pgfpathlineto{\pgfqpoint{2.203210in}{1.934775in}}%
\pgfpathlineto{\pgfqpoint{2.261922in}{1.934841in}}%
\pgfpathlineto{\pgfqpoint{2.320634in}{1.934864in}}%
\pgfpathlineto{\pgfqpoint{2.379346in}{1.934872in}}%
\pgfpathlineto{\pgfqpoint{2.438058in}{1.934874in}}%
\pgfpathlineto{\pgfqpoint{2.496771in}{1.934875in}}%
\pgfpathlineto{\pgfqpoint{2.555483in}{1.934876in}}%
\pgfpathlineto{\pgfqpoint{2.614195in}{1.934876in}}%
\pgfpathlineto{\pgfqpoint{2.672907in}{1.934876in}}%
\pgfusepath{stroke}%
\end{pgfscope}%
\begin{pgfscope}%
\pgfpathrectangle{\pgfqpoint{0.453589in}{0.383578in}}{\pgfqpoint{2.325000in}{1.925000in}}%
\pgfusepath{clip}%
\pgfsetbuttcap%
\pgfsetroundjoin%
\definecolor{currentfill}{rgb}{0.490196,0.588235,0.431373}%
\pgfsetfillcolor{currentfill}%
\pgfsetlinewidth{1.003750pt}%
\definecolor{currentstroke}{rgb}{0.490196,0.588235,0.431373}%
\pgfsetstrokecolor{currentstroke}%
\pgfsetdash{}{0pt}%
\pgfsys@defobject{currentmarker}{\pgfqpoint{-0.010417in}{-0.010417in}}{\pgfqpoint{0.010417in}{0.010417in}}{%
\pgfpathmoveto{\pgfqpoint{0.000000in}{-0.010417in}}%
\pgfpathcurveto{\pgfqpoint{0.002763in}{-0.010417in}}{\pgfqpoint{0.005412in}{-0.009319in}}{\pgfqpoint{0.007366in}{-0.007366in}}%
\pgfpathcurveto{\pgfqpoint{0.009319in}{-0.005412in}}{\pgfqpoint{0.010417in}{-0.002763in}}{\pgfqpoint{0.010417in}{0.000000in}}%
\pgfpathcurveto{\pgfqpoint{0.010417in}{0.002763in}}{\pgfqpoint{0.009319in}{0.005412in}}{\pgfqpoint{0.007366in}{0.007366in}}%
\pgfpathcurveto{\pgfqpoint{0.005412in}{0.009319in}}{\pgfqpoint{0.002763in}{0.010417in}}{\pgfqpoint{0.000000in}{0.010417in}}%
\pgfpathcurveto{\pgfqpoint{-0.002763in}{0.010417in}}{\pgfqpoint{-0.005412in}{0.009319in}}{\pgfqpoint{-0.007366in}{0.007366in}}%
\pgfpathcurveto{\pgfqpoint{-0.009319in}{0.005412in}}{\pgfqpoint{-0.010417in}{0.002763in}}{\pgfqpoint{-0.010417in}{0.000000in}}%
\pgfpathcurveto{\pgfqpoint{-0.010417in}{-0.002763in}}{\pgfqpoint{-0.009319in}{-0.005412in}}{\pgfqpoint{-0.007366in}{-0.007366in}}%
\pgfpathcurveto{\pgfqpoint{-0.005412in}{-0.009319in}}{\pgfqpoint{-0.002763in}{-0.010417in}}{\pgfqpoint{0.000000in}{-0.010417in}}%
\pgfpathclose%
\pgfusepath{stroke,fill}%
}%
\begin{pgfscope}%
\pgfsys@transformshift{0.559271in}{0.451840in}%
\pgfsys@useobject{currentmarker}{}%
\end{pgfscope}%
\begin{pgfscope}%
\pgfsys@transformshift{0.617983in}{0.451840in}%
\pgfsys@useobject{currentmarker}{}%
\end{pgfscope}%
\begin{pgfscope}%
\pgfsys@transformshift{0.676695in}{0.451840in}%
\pgfsys@useobject{currentmarker}{}%
\end{pgfscope}%
\begin{pgfscope}%
\pgfsys@transformshift{0.735407in}{0.451840in}%
\pgfsys@useobject{currentmarker}{}%
\end{pgfscope}%
\begin{pgfscope}%
\pgfsys@transformshift{0.794119in}{0.451840in}%
\pgfsys@useobject{currentmarker}{}%
\end{pgfscope}%
\begin{pgfscope}%
\pgfsys@transformshift{0.852831in}{0.451840in}%
\pgfsys@useobject{currentmarker}{}%
\end{pgfscope}%
\begin{pgfscope}%
\pgfsys@transformshift{0.911543in}{0.451840in}%
\pgfsys@useobject{currentmarker}{}%
\end{pgfscope}%
\begin{pgfscope}%
\pgfsys@transformshift{0.970255in}{0.451840in}%
\pgfsys@useobject{currentmarker}{}%
\end{pgfscope}%
\begin{pgfscope}%
\pgfsys@transformshift{1.028968in}{0.451840in}%
\pgfsys@useobject{currentmarker}{}%
\end{pgfscope}%
\begin{pgfscope}%
\pgfsys@transformshift{1.087680in}{0.451840in}%
\pgfsys@useobject{currentmarker}{}%
\end{pgfscope}%
\begin{pgfscope}%
\pgfsys@transformshift{1.146392in}{0.451840in}%
\pgfsys@useobject{currentmarker}{}%
\end{pgfscope}%
\begin{pgfscope}%
\pgfsys@transformshift{1.205104in}{0.451840in}%
\pgfsys@useobject{currentmarker}{}%
\end{pgfscope}%
\begin{pgfscope}%
\pgfsys@transformshift{1.263816in}{0.451840in}%
\pgfsys@useobject{currentmarker}{}%
\end{pgfscope}%
\begin{pgfscope}%
\pgfsys@transformshift{1.322528in}{0.451840in}%
\pgfsys@useobject{currentmarker}{}%
\end{pgfscope}%
\begin{pgfscope}%
\pgfsys@transformshift{1.381240in}{0.451840in}%
\pgfsys@useobject{currentmarker}{}%
\end{pgfscope}%
\begin{pgfscope}%
\pgfsys@transformshift{1.439952in}{0.457776in}%
\pgfsys@useobject{currentmarker}{}%
\end{pgfscope}%
\begin{pgfscope}%
\pgfsys@transformshift{1.498665in}{0.499327in}%
\pgfsys@useobject{currentmarker}{}%
\end{pgfscope}%
\begin{pgfscope}%
\pgfsys@transformshift{1.557377in}{0.606173in}%
\pgfsys@useobject{currentmarker}{}%
\end{pgfscope}%
\begin{pgfscope}%
\pgfsys@transformshift{1.616089in}{0.730826in}%
\pgfsys@useobject{currentmarker}{}%
\end{pgfscope}%
\begin{pgfscope}%
\pgfsys@transformshift{1.674801in}{1.401578in}%
\pgfsys@useobject{currentmarker}{}%
\end{pgfscope}%
\begin{pgfscope}%
\pgfsys@transformshift{1.733513in}{1.728051in}%
\pgfsys@useobject{currentmarker}{}%
\end{pgfscope}%
\begin{pgfscope}%
\pgfsys@transformshift{1.792225in}{1.846768in}%
\pgfsys@useobject{currentmarker}{}%
\end{pgfscope}%
\begin{pgfscope}%
\pgfsys@transformshift{1.850937in}{1.923934in}%
\pgfsys@useobject{currentmarker}{}%
\end{pgfscope}%
\begin{pgfscope}%
\pgfsys@transformshift{1.909649in}{1.941742in}%
\pgfsys@useobject{currentmarker}{}%
\end{pgfscope}%
\begin{pgfscope}%
\pgfsys@transformshift{1.968361in}{1.953614in}%
\pgfsys@useobject{currentmarker}{}%
\end{pgfscope}%
\begin{pgfscope}%
\pgfsys@transformshift{2.027074in}{1.959549in}%
\pgfsys@useobject{currentmarker}{}%
\end{pgfscope}%
\begin{pgfscope}%
\pgfsys@transformshift{2.085786in}{1.965485in}%
\pgfsys@useobject{currentmarker}{}%
\end{pgfscope}%
\begin{pgfscope}%
\pgfsys@transformshift{2.144498in}{1.965485in}%
\pgfsys@useobject{currentmarker}{}%
\end{pgfscope}%
\begin{pgfscope}%
\pgfsys@transformshift{2.203210in}{1.965485in}%
\pgfsys@useobject{currentmarker}{}%
\end{pgfscope}%
\begin{pgfscope}%
\pgfsys@transformshift{2.261922in}{1.959549in}%
\pgfsys@useobject{currentmarker}{}%
\end{pgfscope}%
\begin{pgfscope}%
\pgfsys@transformshift{2.320634in}{1.959549in}%
\pgfsys@useobject{currentmarker}{}%
\end{pgfscope}%
\begin{pgfscope}%
\pgfsys@transformshift{2.379346in}{1.965485in}%
\pgfsys@useobject{currentmarker}{}%
\end{pgfscope}%
\begin{pgfscope}%
\pgfsys@transformshift{2.438058in}{1.959549in}%
\pgfsys@useobject{currentmarker}{}%
\end{pgfscope}%
\begin{pgfscope}%
\pgfsys@transformshift{2.496771in}{1.965485in}%
\pgfsys@useobject{currentmarker}{}%
\end{pgfscope}%
\begin{pgfscope}%
\pgfsys@transformshift{2.555483in}{1.965485in}%
\pgfsys@useobject{currentmarker}{}%
\end{pgfscope}%
\begin{pgfscope}%
\pgfsys@transformshift{2.614195in}{1.959549in}%
\pgfsys@useobject{currentmarker}{}%
\end{pgfscope}%
\begin{pgfscope}%
\pgfsys@transformshift{2.672907in}{1.959549in}%
\pgfsys@useobject{currentmarker}{}%
\end{pgfscope}%
\end{pgfscope}%
\begin{pgfscope}%
\pgfpathrectangle{\pgfqpoint{0.453589in}{0.383578in}}{\pgfqpoint{2.325000in}{1.925000in}}%
\pgfusepath{clip}%
\pgfsetrectcap%
\pgfsetroundjoin%
\pgfsetlinewidth{0.803000pt}%
\definecolor{currentstroke}{rgb}{0.490196,0.588235,0.431373}%
\pgfsetstrokecolor{currentstroke}%
\pgfsetdash{}{0pt}%
\pgfpathmoveto{\pgfqpoint{0.559271in}{0.451840in}}%
\pgfpathlineto{\pgfqpoint{0.617983in}{0.451840in}}%
\pgfpathlineto{\pgfqpoint{0.676695in}{0.451840in}}%
\pgfpathlineto{\pgfqpoint{0.735407in}{0.451840in}}%
\pgfpathlineto{\pgfqpoint{0.794119in}{0.451840in}}%
\pgfpathlineto{\pgfqpoint{0.852831in}{0.451840in}}%
\pgfpathlineto{\pgfqpoint{0.911543in}{0.451840in}}%
\pgfpathlineto{\pgfqpoint{0.970255in}{0.451840in}}%
\pgfpathlineto{\pgfqpoint{1.028968in}{0.451841in}}%
\pgfpathlineto{\pgfqpoint{1.087680in}{0.451841in}}%
\pgfpathlineto{\pgfqpoint{1.146392in}{0.451843in}}%
\pgfpathlineto{\pgfqpoint{1.205104in}{0.451853in}}%
\pgfpathlineto{\pgfqpoint{1.263816in}{0.451898in}}%
\pgfpathlineto{\pgfqpoint{1.322528in}{0.452099in}}%
\pgfpathlineto{\pgfqpoint{1.381240in}{0.453003in}}%
\pgfpathlineto{\pgfqpoint{1.439952in}{0.457053in}}%
\pgfpathlineto{\pgfqpoint{1.498665in}{0.475003in}}%
\pgfpathlineto{\pgfqpoint{1.557377in}{0.550684in}}%
\pgfpathlineto{\pgfqpoint{1.616089in}{0.813274in}}%
\pgfpathlineto{\pgfqpoint{1.674801in}{1.334912in}}%
\pgfpathlineto{\pgfqpoint{1.733513in}{1.752222in}}%
\pgfpathlineto{\pgfqpoint{1.792225in}{1.904907in}}%
\pgfpathlineto{\pgfqpoint{1.850937in}{1.943860in}}%
\pgfpathlineto{\pgfqpoint{1.909649in}{1.952807in}}%
\pgfpathlineto{\pgfqpoint{1.968361in}{1.954811in}}%
\pgfpathlineto{\pgfqpoint{2.027074in}{1.955258in}}%
\pgfpathlineto{\pgfqpoint{2.085786in}{1.955357in}}%
\pgfpathlineto{\pgfqpoint{2.144498in}{1.955379in}}%
\pgfpathlineto{\pgfqpoint{2.203210in}{1.955384in}}%
\pgfpathlineto{\pgfqpoint{2.261922in}{1.955385in}}%
\pgfpathlineto{\pgfqpoint{2.320634in}{1.955385in}}%
\pgfpathlineto{\pgfqpoint{2.379346in}{1.955385in}}%
\pgfpathlineto{\pgfqpoint{2.438058in}{1.955385in}}%
\pgfpathlineto{\pgfqpoint{2.496771in}{1.955385in}}%
\pgfpathlineto{\pgfqpoint{2.555483in}{1.955385in}}%
\pgfpathlineto{\pgfqpoint{2.614195in}{1.955385in}}%
\pgfpathlineto{\pgfqpoint{2.672907in}{1.955385in}}%
\pgfusepath{stroke}%
\end{pgfscope}%
\begin{pgfscope}%
\pgfsetrectcap%
\pgfsetmiterjoin%
\pgfsetlinewidth{0.501875pt}%
\definecolor{currentstroke}{rgb}{0.317647,0.317647,0.317647}%
\pgfsetstrokecolor{currentstroke}%
\pgfsetdash{}{0pt}%
\pgfpathmoveto{\pgfqpoint{0.453589in}{0.383578in}}%
\pgfpathlineto{\pgfqpoint{0.453589in}{2.308578in}}%
\pgfusepath{stroke}%
\end{pgfscope}%
\begin{pgfscope}%
\pgfsetrectcap%
\pgfsetmiterjoin%
\pgfsetlinewidth{0.501875pt}%
\definecolor{currentstroke}{rgb}{0.317647,0.317647,0.317647}%
\pgfsetstrokecolor{currentstroke}%
\pgfsetdash{}{0pt}%
\pgfpathmoveto{\pgfqpoint{0.453589in}{0.383578in}}%
\pgfpathlineto{\pgfqpoint{2.778589in}{0.383578in}}%
\pgfusepath{stroke}%
\end{pgfscope}%
\begin{pgfscope}%
\pgfsetbuttcap%
\pgfsetroundjoin%
\definecolor{currentfill}{rgb}{0.333333,0.333333,0.333333}%
\pgfsetfillcolor{currentfill}%
\pgfsetlinewidth{1.003750pt}%
\definecolor{currentstroke}{rgb}{0.333333,0.333333,0.333333}%
\pgfsetstrokecolor{currentstroke}%
\pgfsetdash{}{0pt}%
\pgfsys@defobject{currentmarker}{\pgfqpoint{-0.010417in}{-0.010417in}}{\pgfqpoint{0.010417in}{0.010417in}}{%
\pgfpathmoveto{\pgfqpoint{0.000000in}{-0.010417in}}%
\pgfpathcurveto{\pgfqpoint{0.002763in}{-0.010417in}}{\pgfqpoint{0.005412in}{-0.009319in}}{\pgfqpoint{0.007366in}{-0.007366in}}%
\pgfpathcurveto{\pgfqpoint{0.009319in}{-0.005412in}}{\pgfqpoint{0.010417in}{-0.002763in}}{\pgfqpoint{0.010417in}{0.000000in}}%
\pgfpathcurveto{\pgfqpoint{0.010417in}{0.002763in}}{\pgfqpoint{0.009319in}{0.005412in}}{\pgfqpoint{0.007366in}{0.007366in}}%
\pgfpathcurveto{\pgfqpoint{0.005412in}{0.009319in}}{\pgfqpoint{0.002763in}{0.010417in}}{\pgfqpoint{0.000000in}{0.010417in}}%
\pgfpathcurveto{\pgfqpoint{-0.002763in}{0.010417in}}{\pgfqpoint{-0.005412in}{0.009319in}}{\pgfqpoint{-0.007366in}{0.007366in}}%
\pgfpathcurveto{\pgfqpoint{-0.009319in}{0.005412in}}{\pgfqpoint{-0.010417in}{0.002763in}}{\pgfqpoint{-0.010417in}{0.000000in}}%
\pgfpathcurveto{\pgfqpoint{-0.010417in}{-0.002763in}}{\pgfqpoint{-0.009319in}{-0.005412in}}{\pgfqpoint{-0.007366in}{-0.007366in}}%
\pgfpathcurveto{\pgfqpoint{-0.005412in}{-0.009319in}}{\pgfqpoint{-0.002763in}{-0.010417in}}{\pgfqpoint{0.000000in}{-0.010417in}}%
\pgfpathclose%
\pgfusepath{stroke,fill}%
}%
\begin{pgfscope}%
\pgfsys@transformshift{0.518378in}{2.248417in}%
\pgfsys@useobject{currentmarker}{}%
\end{pgfscope}%
\end{pgfscope}%
\begin{pgfscope}%
\definecolor{textcolor}{rgb}{0.000000,0.000000,0.000000}%
\pgfsetstrokecolor{textcolor}%
\pgfsetfillcolor{textcolor}%
\pgftext[x=0.601678in,y=2.216022in,left,base]{\color{textcolor}\rmfamily\fontsize{6.664000}{7.996800}\selectfont \(\displaystyle b \propto \delta V = \vartheta = \SI{5}{\V}\)}%
\end{pgfscope}%
\begin{pgfscope}%
\pgfsetbuttcap%
\pgfsetroundjoin%
\definecolor{currentfill}{rgb}{0.686275,0.352941,0.313725}%
\pgfsetfillcolor{currentfill}%
\pgfsetlinewidth{1.003750pt}%
\definecolor{currentstroke}{rgb}{0.686275,0.352941,0.313725}%
\pgfsetstrokecolor{currentstroke}%
\pgfsetdash{}{0pt}%
\pgfsys@defobject{currentmarker}{\pgfqpoint{-0.010417in}{-0.010417in}}{\pgfqpoint{0.010417in}{0.010417in}}{%
\pgfpathmoveto{\pgfqpoint{0.000000in}{-0.010417in}}%
\pgfpathcurveto{\pgfqpoint{0.002763in}{-0.010417in}}{\pgfqpoint{0.005412in}{-0.009319in}}{\pgfqpoint{0.007366in}{-0.007366in}}%
\pgfpathcurveto{\pgfqpoint{0.009319in}{-0.005412in}}{\pgfqpoint{0.010417in}{-0.002763in}}{\pgfqpoint{0.010417in}{0.000000in}}%
\pgfpathcurveto{\pgfqpoint{0.010417in}{0.002763in}}{\pgfqpoint{0.009319in}{0.005412in}}{\pgfqpoint{0.007366in}{0.007366in}}%
\pgfpathcurveto{\pgfqpoint{0.005412in}{0.009319in}}{\pgfqpoint{0.002763in}{0.010417in}}{\pgfqpoint{0.000000in}{0.010417in}}%
\pgfpathcurveto{\pgfqpoint{-0.002763in}{0.010417in}}{\pgfqpoint{-0.005412in}{0.009319in}}{\pgfqpoint{-0.007366in}{0.007366in}}%
\pgfpathcurveto{\pgfqpoint{-0.009319in}{0.005412in}}{\pgfqpoint{-0.010417in}{0.002763in}}{\pgfqpoint{-0.010417in}{0.000000in}}%
\pgfpathcurveto{\pgfqpoint{-0.010417in}{-0.002763in}}{\pgfqpoint{-0.009319in}{-0.005412in}}{\pgfqpoint{-0.007366in}{-0.007366in}}%
\pgfpathcurveto{\pgfqpoint{-0.005412in}{-0.009319in}}{\pgfqpoint{-0.002763in}{-0.010417in}}{\pgfqpoint{0.000000in}{-0.010417in}}%
\pgfpathclose%
\pgfusepath{stroke,fill}%
}%
\begin{pgfscope}%
\pgfsys@transformshift{0.518378in}{2.128650in}%
\pgfsys@useobject{currentmarker}{}%
\end{pgfscope}%
\end{pgfscope}%
\begin{pgfscope}%
\definecolor{textcolor}{rgb}{0.000000,0.000000,0.000000}%
\pgfsetstrokecolor{textcolor}%
\pgfsetfillcolor{textcolor}%
\pgftext[x=0.601678in,y=2.096256in,left,base]{\color{textcolor}\rmfamily\fontsize{6.664000}{7.996800}\selectfont \(\displaystyle b \propto \delta V = \vartheta = \SI{15}{\V}\)}%
\end{pgfscope}%
\begin{pgfscope}%
\pgfsetbuttcap%
\pgfsetroundjoin%
\definecolor{currentfill}{rgb}{0.000000,0.356863,0.509804}%
\pgfsetfillcolor{currentfill}%
\pgfsetlinewidth{1.003750pt}%
\definecolor{currentstroke}{rgb}{0.000000,0.356863,0.509804}%
\pgfsetstrokecolor{currentstroke}%
\pgfsetdash{}{0pt}%
\pgfsys@defobject{currentmarker}{\pgfqpoint{-0.010417in}{-0.010417in}}{\pgfqpoint{0.010417in}{0.010417in}}{%
\pgfpathmoveto{\pgfqpoint{0.000000in}{-0.010417in}}%
\pgfpathcurveto{\pgfqpoint{0.002763in}{-0.010417in}}{\pgfqpoint{0.005412in}{-0.009319in}}{\pgfqpoint{0.007366in}{-0.007366in}}%
\pgfpathcurveto{\pgfqpoint{0.009319in}{-0.005412in}}{\pgfqpoint{0.010417in}{-0.002763in}}{\pgfqpoint{0.010417in}{0.000000in}}%
\pgfpathcurveto{\pgfqpoint{0.010417in}{0.002763in}}{\pgfqpoint{0.009319in}{0.005412in}}{\pgfqpoint{0.007366in}{0.007366in}}%
\pgfpathcurveto{\pgfqpoint{0.005412in}{0.009319in}}{\pgfqpoint{0.002763in}{0.010417in}}{\pgfqpoint{0.000000in}{0.010417in}}%
\pgfpathcurveto{\pgfqpoint{-0.002763in}{0.010417in}}{\pgfqpoint{-0.005412in}{0.009319in}}{\pgfqpoint{-0.007366in}{0.007366in}}%
\pgfpathcurveto{\pgfqpoint{-0.009319in}{0.005412in}}{\pgfqpoint{-0.010417in}{0.002763in}}{\pgfqpoint{-0.010417in}{0.000000in}}%
\pgfpathcurveto{\pgfqpoint{-0.010417in}{-0.002763in}}{\pgfqpoint{-0.009319in}{-0.005412in}}{\pgfqpoint{-0.007366in}{-0.007366in}}%
\pgfpathcurveto{\pgfqpoint{-0.005412in}{-0.009319in}}{\pgfqpoint{-0.002763in}{-0.010417in}}{\pgfqpoint{0.000000in}{-0.010417in}}%
\pgfpathclose%
\pgfusepath{stroke,fill}%
}%
\begin{pgfscope}%
\pgfsys@transformshift{0.518378in}{2.008884in}%
\pgfsys@useobject{currentmarker}{}%
\end{pgfscope}%
\end{pgfscope}%
\begin{pgfscope}%
\definecolor{textcolor}{rgb}{0.000000,0.000000,0.000000}%
\pgfsetstrokecolor{textcolor}%
\pgfsetfillcolor{textcolor}%
\pgftext[x=0.601678in,y=1.976489in,left,base]{\color{textcolor}\rmfamily\fontsize{6.664000}{7.996800}\selectfont \(\displaystyle b \propto \delta V = \vartheta = \SI{30}{\V}\)}%
\end{pgfscope}%
\begin{pgfscope}%
\pgfsetbuttcap%
\pgfsetroundjoin%
\definecolor{currentfill}{rgb}{0.490196,0.588235,0.431373}%
\pgfsetfillcolor{currentfill}%
\pgfsetlinewidth{1.003750pt}%
\definecolor{currentstroke}{rgb}{0.490196,0.588235,0.431373}%
\pgfsetstrokecolor{currentstroke}%
\pgfsetdash{}{0pt}%
\pgfsys@defobject{currentmarker}{\pgfqpoint{-0.010417in}{-0.010417in}}{\pgfqpoint{0.010417in}{0.010417in}}{%
\pgfpathmoveto{\pgfqpoint{0.000000in}{-0.010417in}}%
\pgfpathcurveto{\pgfqpoint{0.002763in}{-0.010417in}}{\pgfqpoint{0.005412in}{-0.009319in}}{\pgfqpoint{0.007366in}{-0.007366in}}%
\pgfpathcurveto{\pgfqpoint{0.009319in}{-0.005412in}}{\pgfqpoint{0.010417in}{-0.002763in}}{\pgfqpoint{0.010417in}{0.000000in}}%
\pgfpathcurveto{\pgfqpoint{0.010417in}{0.002763in}}{\pgfqpoint{0.009319in}{0.005412in}}{\pgfqpoint{0.007366in}{0.007366in}}%
\pgfpathcurveto{\pgfqpoint{0.005412in}{0.009319in}}{\pgfqpoint{0.002763in}{0.010417in}}{\pgfqpoint{0.000000in}{0.010417in}}%
\pgfpathcurveto{\pgfqpoint{-0.002763in}{0.010417in}}{\pgfqpoint{-0.005412in}{0.009319in}}{\pgfqpoint{-0.007366in}{0.007366in}}%
\pgfpathcurveto{\pgfqpoint{-0.009319in}{0.005412in}}{\pgfqpoint{-0.010417in}{0.002763in}}{\pgfqpoint{-0.010417in}{0.000000in}}%
\pgfpathcurveto{\pgfqpoint{-0.010417in}{-0.002763in}}{\pgfqpoint{-0.009319in}{-0.005412in}}{\pgfqpoint{-0.007366in}{-0.007366in}}%
\pgfpathcurveto{\pgfqpoint{-0.005412in}{-0.009319in}}{\pgfqpoint{-0.002763in}{-0.010417in}}{\pgfqpoint{0.000000in}{-0.010417in}}%
\pgfpathclose%
\pgfusepath{stroke,fill}%
}%
\begin{pgfscope}%
\pgfsys@transformshift{0.518378in}{1.889117in}%
\pgfsys@useobject{currentmarker}{}%
\end{pgfscope}%
\end{pgfscope}%
\begin{pgfscope}%
\definecolor{textcolor}{rgb}{0.000000,0.000000,0.000000}%
\pgfsetstrokecolor{textcolor}%
\pgfsetfillcolor{textcolor}%
\pgftext[x=0.601678in,y=1.856722in,left,base]{\color{textcolor}\rmfamily\fontsize{6.664000}{7.996800}\selectfont \(\displaystyle b \propto \delta V = \vartheta = \SI{60}{\V}\)}%
\end{pgfscope}%
\end{pgfpicture}%
\makeatother%
\endgroup%

		\label{dlsactivationfunctionweight}
	\end{subfigure}	
	\begin{subfigure}[c]{0.5\textwidth}
		\centering
		\caption{}
		%% Creator: Matplotlib, PGF backend
%%
%% To include the figure in your LaTeX document, write
%%   \input{<filename>.pgf}
%%
%% Make sure the required packages are loaded in your preamble
%%   \usepackage{pgf}
%%
%% Figures using additional raster images can only be included by \input if
%% they are in the same directory as the main LaTeX file. For loading figures
%% from other directories you can use the `import` package
%%   \usepackage{import}
%% and then include the figures with
%%   \import{<path to file>}{<filename>.pgf}
%%
%% Matplotlib used the following preamble
%%   \usepackage{amsmath} \usepackage{pifont} \usepackage{xcolor} \definecolor{green}{HTML}{467821} \definecolor{red}{HTML}{CF4457} \usepackage[detect-all]{siunitx}
%%   \usepackage{fontspec}
%%
\begingroup%
\makeatletter%
\begin{pgfpicture}%
\pgfpathrectangle{\pgfpointorigin}{\pgfqpoint{4.430741in}{2.795730in}}%
\pgfusepath{use as bounding box, clip}%
\begin{pgfscope}%
\pgfsetbuttcap%
\pgfsetmiterjoin%
\pgfsetlinewidth{0.000000pt}%
\definecolor{currentstroke}{rgb}{0.000000,0.000000,0.000000}%
\pgfsetstrokecolor{currentstroke}%
\pgfsetstrokeopacity{0.000000}%
\pgfsetdash{}{0pt}%
\pgfpathmoveto{\pgfqpoint{0.000000in}{0.000000in}}%
\pgfpathlineto{\pgfqpoint{4.430741in}{0.000000in}}%
\pgfpathlineto{\pgfqpoint{4.430741in}{2.795730in}}%
\pgfpathlineto{\pgfqpoint{0.000000in}{2.795730in}}%
\pgfpathclose%
\pgfusepath{}%
\end{pgfscope}%
\begin{pgfscope}%
\pgfsetbuttcap%
\pgfsetmiterjoin%
\pgfsetlinewidth{0.000000pt}%
\definecolor{currentstroke}{rgb}{0.000000,0.000000,0.000000}%
\pgfsetstrokecolor{currentstroke}%
\pgfsetstrokeopacity{0.000000}%
\pgfsetdash{}{0pt}%
\pgfpathmoveto{\pgfqpoint{0.455741in}{0.385730in}}%
\pgfpathlineto{\pgfqpoint{4.330741in}{0.385730in}}%
\pgfpathlineto{\pgfqpoint{4.330741in}{2.695730in}}%
\pgfpathlineto{\pgfqpoint{0.455741in}{2.695730in}}%
\pgfpathclose%
\pgfusepath{}%
\end{pgfscope}%
\begin{pgfscope}%
\pgfsetbuttcap%
\pgfsetroundjoin%
\definecolor{currentfill}{rgb}{0.317647,0.317647,0.317647}%
\pgfsetfillcolor{currentfill}%
\pgfsetlinewidth{0.501875pt}%
\definecolor{currentstroke}{rgb}{0.317647,0.317647,0.317647}%
\pgfsetstrokecolor{currentstroke}%
\pgfsetdash{}{0pt}%
\pgfsys@defobject{currentmarker}{\pgfqpoint{0.000000in}{-0.020833in}}{\pgfqpoint{0.000000in}{0.000000in}}{%
\pgfpathmoveto{\pgfqpoint{0.000000in}{0.000000in}}%
\pgfpathlineto{\pgfqpoint{0.000000in}{-0.020833in}}%
\pgfusepath{stroke,fill}%
}%
\begin{pgfscope}%
\pgfsys@transformshift{1.025723in}{0.385730in}%
\pgfsys@useobject{currentmarker}{}%
\end{pgfscope}%
\end{pgfscope}%
\begin{pgfscope}%
\definecolor{textcolor}{rgb}{0.317647,0.317647,0.317647}%
\pgfsetstrokecolor{textcolor}%
\pgfsetfillcolor{textcolor}%
\pgftext[x=1.025723in,y=0.337119in,,top]{\color{textcolor}\rmfamily\fontsize{6.664000}{7.996800}\selectfont \(\displaystyle -1000\)}%
\end{pgfscope}%
\begin{pgfscope}%
\pgfsetbuttcap%
\pgfsetroundjoin%
\definecolor{currentfill}{rgb}{0.317647,0.317647,0.317647}%
\pgfsetfillcolor{currentfill}%
\pgfsetlinewidth{0.501875pt}%
\definecolor{currentstroke}{rgb}{0.317647,0.317647,0.317647}%
\pgfsetstrokecolor{currentstroke}%
\pgfsetdash{}{0pt}%
\pgfsys@defobject{currentmarker}{\pgfqpoint{0.000000in}{-0.020833in}}{\pgfqpoint{0.000000in}{0.000000in}}{%
\pgfpathmoveto{\pgfqpoint{0.000000in}{0.000000in}}%
\pgfpathlineto{\pgfqpoint{0.000000in}{-0.020833in}}%
\pgfusepath{stroke,fill}%
}%
\begin{pgfscope}%
\pgfsys@transformshift{1.709482in}{0.385730in}%
\pgfsys@useobject{currentmarker}{}%
\end{pgfscope}%
\end{pgfscope}%
\begin{pgfscope}%
\definecolor{textcolor}{rgb}{0.317647,0.317647,0.317647}%
\pgfsetstrokecolor{textcolor}%
\pgfsetfillcolor{textcolor}%
\pgftext[x=1.709482in,y=0.337119in,,top]{\color{textcolor}\rmfamily\fontsize{6.664000}{7.996800}\selectfont \(\displaystyle -500\)}%
\end{pgfscope}%
\begin{pgfscope}%
\pgfsetbuttcap%
\pgfsetroundjoin%
\definecolor{currentfill}{rgb}{0.317647,0.317647,0.317647}%
\pgfsetfillcolor{currentfill}%
\pgfsetlinewidth{0.501875pt}%
\definecolor{currentstroke}{rgb}{0.317647,0.317647,0.317647}%
\pgfsetstrokecolor{currentstroke}%
\pgfsetdash{}{0pt}%
\pgfsys@defobject{currentmarker}{\pgfqpoint{0.000000in}{-0.020833in}}{\pgfqpoint{0.000000in}{0.000000in}}{%
\pgfpathmoveto{\pgfqpoint{0.000000in}{0.000000in}}%
\pgfpathlineto{\pgfqpoint{0.000000in}{-0.020833in}}%
\pgfusepath{stroke,fill}%
}%
\begin{pgfscope}%
\pgfsys@transformshift{2.393241in}{0.385730in}%
\pgfsys@useobject{currentmarker}{}%
\end{pgfscope}%
\end{pgfscope}%
\begin{pgfscope}%
\definecolor{textcolor}{rgb}{0.317647,0.317647,0.317647}%
\pgfsetstrokecolor{textcolor}%
\pgfsetfillcolor{textcolor}%
\pgftext[x=2.393241in,y=0.337119in,,top]{\color{textcolor}\rmfamily\fontsize{6.664000}{7.996800}\selectfont \(\displaystyle 0\)}%
\end{pgfscope}%
\begin{pgfscope}%
\pgfsetbuttcap%
\pgfsetroundjoin%
\definecolor{currentfill}{rgb}{0.317647,0.317647,0.317647}%
\pgfsetfillcolor{currentfill}%
\pgfsetlinewidth{0.501875pt}%
\definecolor{currentstroke}{rgb}{0.317647,0.317647,0.317647}%
\pgfsetstrokecolor{currentstroke}%
\pgfsetdash{}{0pt}%
\pgfsys@defobject{currentmarker}{\pgfqpoint{0.000000in}{-0.020833in}}{\pgfqpoint{0.000000in}{0.000000in}}{%
\pgfpathmoveto{\pgfqpoint{0.000000in}{0.000000in}}%
\pgfpathlineto{\pgfqpoint{0.000000in}{-0.020833in}}%
\pgfusepath{stroke,fill}%
}%
\begin{pgfscope}%
\pgfsys@transformshift{3.077000in}{0.385730in}%
\pgfsys@useobject{currentmarker}{}%
\end{pgfscope}%
\end{pgfscope}%
\begin{pgfscope}%
\definecolor{textcolor}{rgb}{0.317647,0.317647,0.317647}%
\pgfsetstrokecolor{textcolor}%
\pgfsetfillcolor{textcolor}%
\pgftext[x=3.077000in,y=0.337119in,,top]{\color{textcolor}\rmfamily\fontsize{6.664000}{7.996800}\selectfont \(\displaystyle 500\)}%
\end{pgfscope}%
\begin{pgfscope}%
\pgfsetbuttcap%
\pgfsetroundjoin%
\definecolor{currentfill}{rgb}{0.317647,0.317647,0.317647}%
\pgfsetfillcolor{currentfill}%
\pgfsetlinewidth{0.501875pt}%
\definecolor{currentstroke}{rgb}{0.317647,0.317647,0.317647}%
\pgfsetstrokecolor{currentstroke}%
\pgfsetdash{}{0pt}%
\pgfsys@defobject{currentmarker}{\pgfqpoint{0.000000in}{-0.020833in}}{\pgfqpoint{0.000000in}{0.000000in}}{%
\pgfpathmoveto{\pgfqpoint{0.000000in}{0.000000in}}%
\pgfpathlineto{\pgfqpoint{0.000000in}{-0.020833in}}%
\pgfusepath{stroke,fill}%
}%
\begin{pgfscope}%
\pgfsys@transformshift{3.760759in}{0.385730in}%
\pgfsys@useobject{currentmarker}{}%
\end{pgfscope}%
\end{pgfscope}%
\begin{pgfscope}%
\definecolor{textcolor}{rgb}{0.317647,0.317647,0.317647}%
\pgfsetstrokecolor{textcolor}%
\pgfsetfillcolor{textcolor}%
\pgftext[x=3.760759in,y=0.337119in,,top]{\color{textcolor}\rmfamily\fontsize{6.664000}{7.996800}\selectfont \(\displaystyle 1000\)}%
\end{pgfscope}%
\begin{pgfscope}%
\definecolor{textcolor}{rgb}{0.317647,0.317647,0.317647}%
\pgfsetstrokecolor{textcolor}%
\pgfsetfillcolor{textcolor}%
\pgftext[x=2.393241in,y=0.199375in,,top]{\color{textcolor}\rmfamily\fontsize{6.664000}{7.996800}\selectfont \(\displaystyle \nu_\mathrm{input} \; (\si{\kilo \Hz})\)}%
\end{pgfscope}%
\begin{pgfscope}%
\pgfsetbuttcap%
\pgfsetroundjoin%
\definecolor{currentfill}{rgb}{0.317647,0.317647,0.317647}%
\pgfsetfillcolor{currentfill}%
\pgfsetlinewidth{0.501875pt}%
\definecolor{currentstroke}{rgb}{0.317647,0.317647,0.317647}%
\pgfsetstrokecolor{currentstroke}%
\pgfsetdash{}{0pt}%
\pgfsys@defobject{currentmarker}{\pgfqpoint{-0.020833in}{0.000000in}}{\pgfqpoint{0.000000in}{0.000000in}}{%
\pgfpathmoveto{\pgfqpoint{0.000000in}{0.000000in}}%
\pgfpathlineto{\pgfqpoint{-0.020833in}{0.000000in}}%
\pgfusepath{stroke,fill}%
}%
\begin{pgfscope}%
\pgfsys@transformshift{0.455741in}{0.490730in}%
\pgfsys@useobject{currentmarker}{}%
\end{pgfscope}%
\end{pgfscope}%
\begin{pgfscope}%
\definecolor{textcolor}{rgb}{0.317647,0.317647,0.317647}%
\pgfsetstrokecolor{textcolor}%
\pgfsetfillcolor{textcolor}%
\pgftext[x=0.365656in,y=0.458614in,left,base]{\color{textcolor}\rmfamily\fontsize{6.664000}{7.996800}\selectfont \(\displaystyle 0\)}%
\end{pgfscope}%
\begin{pgfscope}%
\pgfsetbuttcap%
\pgfsetroundjoin%
\definecolor{currentfill}{rgb}{0.317647,0.317647,0.317647}%
\pgfsetfillcolor{currentfill}%
\pgfsetlinewidth{0.501875pt}%
\definecolor{currentstroke}{rgb}{0.317647,0.317647,0.317647}%
\pgfsetstrokecolor{currentstroke}%
\pgfsetdash{}{0pt}%
\pgfsys@defobject{currentmarker}{\pgfqpoint{-0.020833in}{0.000000in}}{\pgfqpoint{0.000000in}{0.000000in}}{%
\pgfpathmoveto{\pgfqpoint{0.000000in}{0.000000in}}%
\pgfpathlineto{\pgfqpoint{-0.020833in}{0.000000in}}%
\pgfusepath{stroke,fill}%
}%
\begin{pgfscope}%
\pgfsys@transformshift{0.455741in}{0.869554in}%
\pgfsys@useobject{currentmarker}{}%
\end{pgfscope}%
\end{pgfscope}%
\begin{pgfscope}%
\definecolor{textcolor}{rgb}{0.317647,0.317647,0.317647}%
\pgfsetstrokecolor{textcolor}%
\pgfsetfillcolor{textcolor}%
\pgftext[x=0.310293in,y=0.837437in,left,base]{\color{textcolor}\rmfamily\fontsize{6.664000}{7.996800}\selectfont \(\displaystyle 20\)}%
\end{pgfscope}%
\begin{pgfscope}%
\pgfsetbuttcap%
\pgfsetroundjoin%
\definecolor{currentfill}{rgb}{0.317647,0.317647,0.317647}%
\pgfsetfillcolor{currentfill}%
\pgfsetlinewidth{0.501875pt}%
\definecolor{currentstroke}{rgb}{0.317647,0.317647,0.317647}%
\pgfsetstrokecolor{currentstroke}%
\pgfsetdash{}{0pt}%
\pgfsys@defobject{currentmarker}{\pgfqpoint{-0.020833in}{0.000000in}}{\pgfqpoint{0.000000in}{0.000000in}}{%
\pgfpathmoveto{\pgfqpoint{0.000000in}{0.000000in}}%
\pgfpathlineto{\pgfqpoint{-0.020833in}{0.000000in}}%
\pgfusepath{stroke,fill}%
}%
\begin{pgfscope}%
\pgfsys@transformshift{0.455741in}{1.248377in}%
\pgfsys@useobject{currentmarker}{}%
\end{pgfscope}%
\end{pgfscope}%
\begin{pgfscope}%
\definecolor{textcolor}{rgb}{0.317647,0.317647,0.317647}%
\pgfsetstrokecolor{textcolor}%
\pgfsetfillcolor{textcolor}%
\pgftext[x=0.310293in,y=1.216261in,left,base]{\color{textcolor}\rmfamily\fontsize{6.664000}{7.996800}\selectfont \(\displaystyle 40\)}%
\end{pgfscope}%
\begin{pgfscope}%
\pgfsetbuttcap%
\pgfsetroundjoin%
\definecolor{currentfill}{rgb}{0.317647,0.317647,0.317647}%
\pgfsetfillcolor{currentfill}%
\pgfsetlinewidth{0.501875pt}%
\definecolor{currentstroke}{rgb}{0.317647,0.317647,0.317647}%
\pgfsetstrokecolor{currentstroke}%
\pgfsetdash{}{0pt}%
\pgfsys@defobject{currentmarker}{\pgfqpoint{-0.020833in}{0.000000in}}{\pgfqpoint{0.000000in}{0.000000in}}{%
\pgfpathmoveto{\pgfqpoint{0.000000in}{0.000000in}}%
\pgfpathlineto{\pgfqpoint{-0.020833in}{0.000000in}}%
\pgfusepath{stroke,fill}%
}%
\begin{pgfscope}%
\pgfsys@transformshift{0.455741in}{1.627201in}%
\pgfsys@useobject{currentmarker}{}%
\end{pgfscope}%
\end{pgfscope}%
\begin{pgfscope}%
\definecolor{textcolor}{rgb}{0.317647,0.317647,0.317647}%
\pgfsetstrokecolor{textcolor}%
\pgfsetfillcolor{textcolor}%
\pgftext[x=0.310293in,y=1.595084in,left,base]{\color{textcolor}\rmfamily\fontsize{6.664000}{7.996800}\selectfont \(\displaystyle 60\)}%
\end{pgfscope}%
\begin{pgfscope}%
\pgfsetbuttcap%
\pgfsetroundjoin%
\definecolor{currentfill}{rgb}{0.317647,0.317647,0.317647}%
\pgfsetfillcolor{currentfill}%
\pgfsetlinewidth{0.501875pt}%
\definecolor{currentstroke}{rgb}{0.317647,0.317647,0.317647}%
\pgfsetstrokecolor{currentstroke}%
\pgfsetdash{}{0pt}%
\pgfsys@defobject{currentmarker}{\pgfqpoint{-0.020833in}{0.000000in}}{\pgfqpoint{0.000000in}{0.000000in}}{%
\pgfpathmoveto{\pgfqpoint{0.000000in}{0.000000in}}%
\pgfpathlineto{\pgfqpoint{-0.020833in}{0.000000in}}%
\pgfusepath{stroke,fill}%
}%
\begin{pgfscope}%
\pgfsys@transformshift{0.455741in}{2.006025in}%
\pgfsys@useobject{currentmarker}{}%
\end{pgfscope}%
\end{pgfscope}%
\begin{pgfscope}%
\definecolor{textcolor}{rgb}{0.317647,0.317647,0.317647}%
\pgfsetstrokecolor{textcolor}%
\pgfsetfillcolor{textcolor}%
\pgftext[x=0.310293in,y=1.973908in,left,base]{\color{textcolor}\rmfamily\fontsize{6.664000}{7.996800}\selectfont \(\displaystyle 80\)}%
\end{pgfscope}%
\begin{pgfscope}%
\pgfsetbuttcap%
\pgfsetroundjoin%
\definecolor{currentfill}{rgb}{0.317647,0.317647,0.317647}%
\pgfsetfillcolor{currentfill}%
\pgfsetlinewidth{0.501875pt}%
\definecolor{currentstroke}{rgb}{0.317647,0.317647,0.317647}%
\pgfsetstrokecolor{currentstroke}%
\pgfsetdash{}{0pt}%
\pgfsys@defobject{currentmarker}{\pgfqpoint{-0.020833in}{0.000000in}}{\pgfqpoint{0.000000in}{0.000000in}}{%
\pgfpathmoveto{\pgfqpoint{0.000000in}{0.000000in}}%
\pgfpathlineto{\pgfqpoint{-0.020833in}{0.000000in}}%
\pgfusepath{stroke,fill}%
}%
\begin{pgfscope}%
\pgfsys@transformshift{0.455741in}{2.384848in}%
\pgfsys@useobject{currentmarker}{}%
\end{pgfscope}%
\end{pgfscope}%
\begin{pgfscope}%
\definecolor{textcolor}{rgb}{0.317647,0.317647,0.317647}%
\pgfsetstrokecolor{textcolor}%
\pgfsetfillcolor{textcolor}%
\pgftext[x=0.254930in,y=2.352731in,left,base]{\color{textcolor}\rmfamily\fontsize{6.664000}{7.996800}\selectfont \(\displaystyle 100\)}%
\end{pgfscope}%
\begin{pgfscope}%
\definecolor{textcolor}{rgb}{0.317647,0.317647,0.317647}%
\pgfsetstrokecolor{textcolor}%
\pgfsetfillcolor{textcolor}%
\pgftext[x=0.199375in,y=1.540730in,,bottom,rotate=90.000000]{\color{textcolor}\rmfamily\fontsize{6.664000}{7.996800}\selectfont \(\displaystyle \nu_\mathrm{output} \; (\si{\kilo \Hz})\)}%
\end{pgfscope}%
\begin{pgfscope}%
\pgfpathrectangle{\pgfqpoint{0.455741in}{0.385730in}}{\pgfqpoint{3.875000in}{2.310000in}}%
\pgfusepath{clip}%
\pgfsetbuttcap%
\pgfsetroundjoin%
\pgfsetlinewidth{0.803000pt}%
\definecolor{currentstroke}{rgb}{0.333333,0.333333,0.333333}%
\pgfsetstrokecolor{currentstroke}%
\pgfsetdash{{2.960000pt}{1.280000pt}}{0.000000pt}%
\pgfpathmoveto{\pgfqpoint{0.631877in}{0.498966in}}%
\pgfpathlineto{\pgfqpoint{0.729731in}{0.490730in}}%
\pgfpathlineto{\pgfqpoint{0.827585in}{0.490730in}}%
\pgfpathlineto{\pgfqpoint{0.925438in}{0.490730in}}%
\pgfpathlineto{\pgfqpoint{1.023292in}{0.507201in}}%
\pgfpathlineto{\pgfqpoint{1.121145in}{0.540142in}}%
\pgfpathlineto{\pgfqpoint{1.218999in}{0.770730in}}%
\pgfpathlineto{\pgfqpoint{1.316852in}{1.264848in}}%
\pgfpathlineto{\pgfqpoint{1.414706in}{1.561319in}}%
\pgfpathlineto{\pgfqpoint{1.512559in}{1.981319in}}%
\pgfpathlineto{\pgfqpoint{1.610413in}{2.129554in}}%
\pgfpathlineto{\pgfqpoint{1.708266in}{2.368377in}}%
\pgfpathlineto{\pgfqpoint{1.806120in}{2.442495in}}%
\pgfpathlineto{\pgfqpoint{1.903973in}{2.491907in}}%
\pgfpathlineto{\pgfqpoint{2.001827in}{2.516613in}}%
\pgfpathlineto{\pgfqpoint{2.099681in}{2.541319in}}%
\pgfpathlineto{\pgfqpoint{2.197534in}{2.557789in}}%
\pgfpathlineto{\pgfqpoint{2.295388in}{2.566025in}}%
\pgfpathlineto{\pgfqpoint{2.393241in}{2.582495in}}%
\pgfpathlineto{\pgfqpoint{2.491095in}{2.582495in}}%
\pgfpathlineto{\pgfqpoint{2.588948in}{2.590730in}}%
\pgfpathlineto{\pgfqpoint{2.686802in}{2.590730in}}%
\pgfpathlineto{\pgfqpoint{2.784655in}{2.590730in}}%
\pgfpathlineto{\pgfqpoint{2.882509in}{2.590730in}}%
\pgfpathlineto{\pgfqpoint{2.980362in}{2.590730in}}%
\pgfpathlineto{\pgfqpoint{3.078216in}{2.590730in}}%
\pgfpathlineto{\pgfqpoint{3.176069in}{2.590730in}}%
\pgfpathlineto{\pgfqpoint{3.273923in}{2.590730in}}%
\pgfpathlineto{\pgfqpoint{3.371776in}{2.590730in}}%
\pgfpathlineto{\pgfqpoint{3.469630in}{2.590730in}}%
\pgfpathlineto{\pgfqpoint{3.567484in}{2.590730in}}%
\pgfpathlineto{\pgfqpoint{3.665337in}{2.590730in}}%
\pgfpathlineto{\pgfqpoint{3.763191in}{2.590730in}}%
\pgfpathlineto{\pgfqpoint{3.861044in}{2.590730in}}%
\pgfpathlineto{\pgfqpoint{3.958898in}{2.590730in}}%
\pgfpathlineto{\pgfqpoint{4.056751in}{2.590730in}}%
\pgfpathlineto{\pgfqpoint{4.154605in}{2.590730in}}%
\pgfusepath{stroke}%
\end{pgfscope}%
\begin{pgfscope}%
\pgfpathrectangle{\pgfqpoint{0.455741in}{0.385730in}}{\pgfqpoint{3.875000in}{2.310000in}}%
\pgfusepath{clip}%
\pgfsetbuttcap%
\pgfsetroundjoin%
\pgfsetlinewidth{0.803000pt}%
\definecolor{currentstroke}{rgb}{0.686275,0.352941,0.313725}%
\pgfsetstrokecolor{currentstroke}%
\pgfsetdash{{2.960000pt}{1.280000pt}}{0.000000pt}%
\pgfpathmoveto{\pgfqpoint{0.631877in}{0.498966in}}%
\pgfpathlineto{\pgfqpoint{0.729731in}{0.490730in}}%
\pgfpathlineto{\pgfqpoint{0.827585in}{0.490730in}}%
\pgfpathlineto{\pgfqpoint{0.925438in}{0.490730in}}%
\pgfpathlineto{\pgfqpoint{1.023292in}{0.507201in}}%
\pgfpathlineto{\pgfqpoint{1.121145in}{0.531907in}}%
\pgfpathlineto{\pgfqpoint{1.218999in}{0.663672in}}%
\pgfpathlineto{\pgfqpoint{1.316852in}{1.058966in}}%
\pgfpathlineto{\pgfqpoint{1.414706in}{1.330730in}}%
\pgfpathlineto{\pgfqpoint{1.512559in}{1.684848in}}%
\pgfpathlineto{\pgfqpoint{1.610413in}{1.923672in}}%
\pgfpathlineto{\pgfqpoint{1.708266in}{2.187201in}}%
\pgfpathlineto{\pgfqpoint{1.806120in}{2.360142in}}%
\pgfpathlineto{\pgfqpoint{1.903973in}{2.450730in}}%
\pgfpathlineto{\pgfqpoint{2.001827in}{2.483672in}}%
\pgfpathlineto{\pgfqpoint{2.099681in}{2.524848in}}%
\pgfpathlineto{\pgfqpoint{2.197534in}{2.549554in}}%
\pgfpathlineto{\pgfqpoint{2.295388in}{2.557789in}}%
\pgfpathlineto{\pgfqpoint{2.393241in}{2.566025in}}%
\pgfpathlineto{\pgfqpoint{2.491095in}{2.574260in}}%
\pgfpathlineto{\pgfqpoint{2.588948in}{2.582495in}}%
\pgfpathlineto{\pgfqpoint{2.686802in}{2.590730in}}%
\pgfpathlineto{\pgfqpoint{2.784655in}{2.590730in}}%
\pgfpathlineto{\pgfqpoint{2.882509in}{2.590730in}}%
\pgfpathlineto{\pgfqpoint{2.980362in}{2.590730in}}%
\pgfpathlineto{\pgfqpoint{3.078216in}{2.590730in}}%
\pgfpathlineto{\pgfqpoint{3.176069in}{2.590730in}}%
\pgfpathlineto{\pgfqpoint{3.273923in}{2.590730in}}%
\pgfpathlineto{\pgfqpoint{3.371776in}{2.590730in}}%
\pgfpathlineto{\pgfqpoint{3.469630in}{2.590730in}}%
\pgfpathlineto{\pgfqpoint{3.567484in}{2.590730in}}%
\pgfpathlineto{\pgfqpoint{3.665337in}{2.590730in}}%
\pgfpathlineto{\pgfqpoint{3.763191in}{2.590730in}}%
\pgfpathlineto{\pgfqpoint{3.861044in}{2.590730in}}%
\pgfpathlineto{\pgfqpoint{3.958898in}{2.590730in}}%
\pgfpathlineto{\pgfqpoint{4.056751in}{2.590730in}}%
\pgfpathlineto{\pgfqpoint{4.154605in}{2.590730in}}%
\pgfusepath{stroke}%
\end{pgfscope}%
\begin{pgfscope}%
\pgfpathrectangle{\pgfqpoint{0.455741in}{0.385730in}}{\pgfqpoint{3.875000in}{2.310000in}}%
\pgfusepath{clip}%
\pgfsetbuttcap%
\pgfsetroundjoin%
\pgfsetlinewidth{0.803000pt}%
\definecolor{currentstroke}{rgb}{0.000000,0.356863,0.509804}%
\pgfsetstrokecolor{currentstroke}%
\pgfsetdash{{2.960000pt}{1.280000pt}}{0.000000pt}%
\pgfpathmoveto{\pgfqpoint{0.631877in}{0.498966in}}%
\pgfpathlineto{\pgfqpoint{0.729731in}{0.490730in}}%
\pgfpathlineto{\pgfqpoint{0.827585in}{0.490730in}}%
\pgfpathlineto{\pgfqpoint{0.925438in}{0.490730in}}%
\pgfpathlineto{\pgfqpoint{1.023292in}{0.490730in}}%
\pgfpathlineto{\pgfqpoint{1.121145in}{0.490730in}}%
\pgfpathlineto{\pgfqpoint{1.218999in}{0.540142in}}%
\pgfpathlineto{\pgfqpoint{1.316852in}{0.680142in}}%
\pgfpathlineto{\pgfqpoint{1.414706in}{0.803672in}}%
\pgfpathlineto{\pgfqpoint{1.512559in}{1.058966in}}%
\pgfpathlineto{\pgfqpoint{1.610413in}{1.347201in}}%
\pgfpathlineto{\pgfqpoint{1.708266in}{1.709554in}}%
\pgfpathlineto{\pgfqpoint{1.806120in}{2.080142in}}%
\pgfpathlineto{\pgfqpoint{1.903973in}{2.294260in}}%
\pgfpathlineto{\pgfqpoint{2.001827in}{2.376613in}}%
\pgfpathlineto{\pgfqpoint{2.099681in}{2.475436in}}%
\pgfpathlineto{\pgfqpoint{2.197534in}{2.500142in}}%
\pgfpathlineto{\pgfqpoint{2.295388in}{2.516613in}}%
\pgfpathlineto{\pgfqpoint{2.393241in}{2.533083in}}%
\pgfpathlineto{\pgfqpoint{2.491095in}{2.541319in}}%
\pgfpathlineto{\pgfqpoint{2.588948in}{2.549554in}}%
\pgfpathlineto{\pgfqpoint{2.686802in}{2.557789in}}%
\pgfpathlineto{\pgfqpoint{2.784655in}{2.566025in}}%
\pgfpathlineto{\pgfqpoint{2.882509in}{2.574260in}}%
\pgfpathlineto{\pgfqpoint{2.980362in}{2.574260in}}%
\pgfpathlineto{\pgfqpoint{3.078216in}{2.582495in}}%
\pgfpathlineto{\pgfqpoint{3.176069in}{2.582495in}}%
\pgfpathlineto{\pgfqpoint{3.273923in}{2.590730in}}%
\pgfpathlineto{\pgfqpoint{3.371776in}{2.590730in}}%
\pgfpathlineto{\pgfqpoint{3.469630in}{2.590730in}}%
\pgfpathlineto{\pgfqpoint{3.567484in}{2.590730in}}%
\pgfpathlineto{\pgfqpoint{3.665337in}{2.590730in}}%
\pgfpathlineto{\pgfqpoint{3.763191in}{2.590730in}}%
\pgfpathlineto{\pgfqpoint{3.861044in}{2.590730in}}%
\pgfpathlineto{\pgfqpoint{3.958898in}{2.590730in}}%
\pgfpathlineto{\pgfqpoint{4.056751in}{2.590730in}}%
\pgfpathlineto{\pgfqpoint{4.154605in}{2.590730in}}%
\pgfusepath{stroke}%
\end{pgfscope}%
\begin{pgfscope}%
\pgfpathrectangle{\pgfqpoint{0.455741in}{0.385730in}}{\pgfqpoint{3.875000in}{2.310000in}}%
\pgfusepath{clip}%
\pgfsetbuttcap%
\pgfsetroundjoin%
\pgfsetlinewidth{0.803000pt}%
\definecolor{currentstroke}{rgb}{0.490196,0.588235,0.431373}%
\pgfsetstrokecolor{currentstroke}%
\pgfsetdash{{2.960000pt}{1.280000pt}}{0.000000pt}%
\pgfpathmoveto{\pgfqpoint{0.631877in}{0.498966in}}%
\pgfpathlineto{\pgfqpoint{0.729731in}{0.490730in}}%
\pgfpathlineto{\pgfqpoint{0.827585in}{0.490730in}}%
\pgfpathlineto{\pgfqpoint{0.925438in}{0.490730in}}%
\pgfpathlineto{\pgfqpoint{1.023292in}{0.531907in}}%
\pgfpathlineto{\pgfqpoint{1.121145in}{0.581319in}}%
\pgfpathlineto{\pgfqpoint{1.218999in}{0.918966in}}%
\pgfpathlineto{\pgfqpoint{1.316852in}{1.380142in}}%
\pgfpathlineto{\pgfqpoint{1.414706in}{1.577789in}}%
\pgfpathlineto{\pgfqpoint{1.512559in}{2.014260in}}%
\pgfpathlineto{\pgfqpoint{1.610413in}{2.121319in}}%
\pgfpathlineto{\pgfqpoint{1.708266in}{2.343672in}}%
\pgfpathlineto{\pgfqpoint{1.806120in}{2.442495in}}%
\pgfpathlineto{\pgfqpoint{1.903973in}{2.500142in}}%
\pgfpathlineto{\pgfqpoint{2.001827in}{2.516613in}}%
\pgfpathlineto{\pgfqpoint{2.099681in}{2.541319in}}%
\pgfpathlineto{\pgfqpoint{2.197534in}{2.549554in}}%
\pgfpathlineto{\pgfqpoint{2.295388in}{2.574260in}}%
\pgfpathlineto{\pgfqpoint{2.393241in}{2.574260in}}%
\pgfpathlineto{\pgfqpoint{2.491095in}{2.582495in}}%
\pgfpathlineto{\pgfqpoint{2.588948in}{2.590730in}}%
\pgfpathlineto{\pgfqpoint{2.686802in}{2.582495in}}%
\pgfpathlineto{\pgfqpoint{2.784655in}{2.590730in}}%
\pgfpathlineto{\pgfqpoint{2.882509in}{2.590730in}}%
\pgfpathlineto{\pgfqpoint{2.980362in}{2.590730in}}%
\pgfpathlineto{\pgfqpoint{3.078216in}{2.590730in}}%
\pgfpathlineto{\pgfqpoint{3.176069in}{2.590730in}}%
\pgfpathlineto{\pgfqpoint{3.273923in}{2.590730in}}%
\pgfpathlineto{\pgfqpoint{3.371776in}{2.590730in}}%
\pgfpathlineto{\pgfqpoint{3.469630in}{2.590730in}}%
\pgfpathlineto{\pgfqpoint{3.567484in}{2.590730in}}%
\pgfpathlineto{\pgfqpoint{3.665337in}{2.590730in}}%
\pgfpathlineto{\pgfqpoint{3.763191in}{2.590730in}}%
\pgfpathlineto{\pgfqpoint{3.861044in}{2.590730in}}%
\pgfpathlineto{\pgfqpoint{3.958898in}{2.590730in}}%
\pgfpathlineto{\pgfqpoint{4.056751in}{2.590730in}}%
\pgfpathlineto{\pgfqpoint{4.154605in}{2.590730in}}%
\pgfusepath{stroke}%
\end{pgfscope}%
\begin{pgfscope}%
\pgfpathrectangle{\pgfqpoint{0.455741in}{0.385730in}}{\pgfqpoint{3.875000in}{2.310000in}}%
\pgfusepath{clip}%
\pgfsetbuttcap%
\pgfsetroundjoin%
\pgfsetlinewidth{0.803000pt}%
\definecolor{currentstroke}{rgb}{0.843137,0.666667,0.313725}%
\pgfsetstrokecolor{currentstroke}%
\pgfsetdash{{2.960000pt}{1.280000pt}}{0.000000pt}%
\pgfpathmoveto{\pgfqpoint{0.631877in}{0.498966in}}%
\pgfpathlineto{\pgfqpoint{0.729731in}{0.490730in}}%
\pgfpathlineto{\pgfqpoint{0.827585in}{0.490730in}}%
\pgfpathlineto{\pgfqpoint{0.925438in}{0.490730in}}%
\pgfpathlineto{\pgfqpoint{1.023292in}{0.490730in}}%
\pgfpathlineto{\pgfqpoint{1.121145in}{0.507201in}}%
\pgfpathlineto{\pgfqpoint{1.218999in}{0.589554in}}%
\pgfpathlineto{\pgfqpoint{1.316852in}{0.861319in}}%
\pgfpathlineto{\pgfqpoint{1.414706in}{0.976613in}}%
\pgfpathlineto{\pgfqpoint{1.512559in}{1.371907in}}%
\pgfpathlineto{\pgfqpoint{1.610413in}{1.767201in}}%
\pgfpathlineto{\pgfqpoint{1.708266in}{2.055436in}}%
\pgfpathlineto{\pgfqpoint{1.806120in}{2.269554in}}%
\pgfpathlineto{\pgfqpoint{1.903973in}{2.417789in}}%
\pgfpathlineto{\pgfqpoint{2.001827in}{2.483672in}}%
\pgfpathlineto{\pgfqpoint{2.099681in}{2.541319in}}%
\pgfpathlineto{\pgfqpoint{2.197534in}{2.549554in}}%
\pgfpathlineto{\pgfqpoint{2.295388in}{2.574260in}}%
\pgfpathlineto{\pgfqpoint{2.393241in}{2.582495in}}%
\pgfpathlineto{\pgfqpoint{2.491095in}{2.590730in}}%
\pgfpathlineto{\pgfqpoint{2.588948in}{2.590730in}}%
\pgfpathlineto{\pgfqpoint{2.686802in}{2.590730in}}%
\pgfpathlineto{\pgfqpoint{2.784655in}{2.590730in}}%
\pgfpathlineto{\pgfqpoint{2.882509in}{2.590730in}}%
\pgfpathlineto{\pgfqpoint{2.980362in}{2.590730in}}%
\pgfpathlineto{\pgfqpoint{3.078216in}{2.590730in}}%
\pgfpathlineto{\pgfqpoint{3.176069in}{2.590730in}}%
\pgfpathlineto{\pgfqpoint{3.273923in}{2.590730in}}%
\pgfpathlineto{\pgfqpoint{3.371776in}{2.590730in}}%
\pgfpathlineto{\pgfqpoint{3.469630in}{2.590730in}}%
\pgfpathlineto{\pgfqpoint{3.567484in}{2.590730in}}%
\pgfpathlineto{\pgfqpoint{3.665337in}{2.590730in}}%
\pgfpathlineto{\pgfqpoint{3.763191in}{2.590730in}}%
\pgfpathlineto{\pgfqpoint{3.861044in}{2.590730in}}%
\pgfpathlineto{\pgfqpoint{3.958898in}{2.590730in}}%
\pgfpathlineto{\pgfqpoint{4.056751in}{2.590730in}}%
\pgfpathlineto{\pgfqpoint{4.154605in}{2.590730in}}%
\pgfusepath{stroke}%
\end{pgfscope}%
\begin{pgfscope}%
\pgfpathrectangle{\pgfqpoint{0.455741in}{0.385730in}}{\pgfqpoint{3.875000in}{2.310000in}}%
\pgfusepath{clip}%
\pgfsetbuttcap%
\pgfsetroundjoin%
\pgfsetlinewidth{0.803000pt}%
\definecolor{currentstroke}{rgb}{0.333333,0.333333,0.333333}%
\pgfsetstrokecolor{currentstroke}%
\pgfsetdash{{2.960000pt}{1.280000pt}}{0.000000pt}%
\pgfpathmoveto{\pgfqpoint{0.631877in}{0.498966in}}%
\pgfpathlineto{\pgfqpoint{0.729731in}{0.490730in}}%
\pgfpathlineto{\pgfqpoint{0.827585in}{0.490730in}}%
\pgfpathlineto{\pgfqpoint{0.925438in}{0.490730in}}%
\pgfpathlineto{\pgfqpoint{1.023292in}{0.490730in}}%
\pgfpathlineto{\pgfqpoint{1.121145in}{0.507201in}}%
\pgfpathlineto{\pgfqpoint{1.218999in}{0.573083in}}%
\pgfpathlineto{\pgfqpoint{1.316852in}{0.861319in}}%
\pgfpathlineto{\pgfqpoint{1.414706in}{1.042495in}}%
\pgfpathlineto{\pgfqpoint{1.512559in}{1.404848in}}%
\pgfpathlineto{\pgfqpoint{1.610413in}{1.709554in}}%
\pgfpathlineto{\pgfqpoint{1.708266in}{2.080142in}}%
\pgfpathlineto{\pgfqpoint{1.806120in}{2.261319in}}%
\pgfpathlineto{\pgfqpoint{1.903973in}{2.393083in}}%
\pgfpathlineto{\pgfqpoint{2.001827in}{2.450730in}}%
\pgfpathlineto{\pgfqpoint{2.099681in}{2.500142in}}%
\pgfpathlineto{\pgfqpoint{2.197534in}{2.516613in}}%
\pgfpathlineto{\pgfqpoint{2.295388in}{2.533083in}}%
\pgfpathlineto{\pgfqpoint{2.393241in}{2.541319in}}%
\pgfpathlineto{\pgfqpoint{2.491095in}{2.549554in}}%
\pgfpathlineto{\pgfqpoint{2.588948in}{2.557789in}}%
\pgfpathlineto{\pgfqpoint{2.686802in}{2.566025in}}%
\pgfpathlineto{\pgfqpoint{2.784655in}{2.566025in}}%
\pgfpathlineto{\pgfqpoint{2.882509in}{2.574260in}}%
\pgfpathlineto{\pgfqpoint{2.980362in}{2.582495in}}%
\pgfpathlineto{\pgfqpoint{3.078216in}{2.582495in}}%
\pgfpathlineto{\pgfqpoint{3.176069in}{2.582495in}}%
\pgfpathlineto{\pgfqpoint{3.273923in}{2.590730in}}%
\pgfpathlineto{\pgfqpoint{3.371776in}{2.590730in}}%
\pgfpathlineto{\pgfqpoint{3.469630in}{2.590730in}}%
\pgfpathlineto{\pgfqpoint{3.567484in}{2.590730in}}%
\pgfpathlineto{\pgfqpoint{3.665337in}{2.590730in}}%
\pgfpathlineto{\pgfqpoint{3.763191in}{2.590730in}}%
\pgfpathlineto{\pgfqpoint{3.861044in}{2.590730in}}%
\pgfpathlineto{\pgfqpoint{3.958898in}{2.590730in}}%
\pgfpathlineto{\pgfqpoint{4.056751in}{2.590730in}}%
\pgfpathlineto{\pgfqpoint{4.154605in}{2.590730in}}%
\pgfusepath{stroke}%
\end{pgfscope}%
\begin{pgfscope}%
\pgfpathrectangle{\pgfqpoint{0.455741in}{0.385730in}}{\pgfqpoint{3.875000in}{2.310000in}}%
\pgfusepath{clip}%
\pgfsetbuttcap%
\pgfsetroundjoin%
\pgfsetlinewidth{0.803000pt}%
\definecolor{currentstroke}{rgb}{0.686275,0.352941,0.313725}%
\pgfsetstrokecolor{currentstroke}%
\pgfsetdash{{2.960000pt}{1.280000pt}}{0.000000pt}%
\pgfpathmoveto{\pgfqpoint{0.631877in}{0.498966in}}%
\pgfpathlineto{\pgfqpoint{0.729731in}{0.490730in}}%
\pgfpathlineto{\pgfqpoint{0.827585in}{0.490730in}}%
\pgfpathlineto{\pgfqpoint{0.925438in}{0.490730in}}%
\pgfpathlineto{\pgfqpoint{1.023292in}{0.490730in}}%
\pgfpathlineto{\pgfqpoint{1.121145in}{0.490730in}}%
\pgfpathlineto{\pgfqpoint{1.218999in}{0.515436in}}%
\pgfpathlineto{\pgfqpoint{1.316852in}{0.614260in}}%
\pgfpathlineto{\pgfqpoint{1.414706in}{0.663672in}}%
\pgfpathlineto{\pgfqpoint{1.512559in}{0.968377in}}%
\pgfpathlineto{\pgfqpoint{1.610413in}{1.314260in}}%
\pgfpathlineto{\pgfqpoint{1.708266in}{1.717789in}}%
\pgfpathlineto{\pgfqpoint{1.806120in}{1.973083in}}%
\pgfpathlineto{\pgfqpoint{1.903973in}{2.277789in}}%
\pgfpathlineto{\pgfqpoint{2.001827in}{2.393083in}}%
\pgfpathlineto{\pgfqpoint{2.099681in}{2.516613in}}%
\pgfpathlineto{\pgfqpoint{2.197534in}{2.541319in}}%
\pgfpathlineto{\pgfqpoint{2.295388in}{2.557789in}}%
\pgfpathlineto{\pgfqpoint{2.393241in}{2.574260in}}%
\pgfpathlineto{\pgfqpoint{2.491095in}{2.582495in}}%
\pgfpathlineto{\pgfqpoint{2.588948in}{2.582495in}}%
\pgfpathlineto{\pgfqpoint{2.686802in}{2.590730in}}%
\pgfpathlineto{\pgfqpoint{2.784655in}{2.582495in}}%
\pgfpathlineto{\pgfqpoint{2.882509in}{2.590730in}}%
\pgfpathlineto{\pgfqpoint{2.980362in}{2.590730in}}%
\pgfpathlineto{\pgfqpoint{3.078216in}{2.590730in}}%
\pgfpathlineto{\pgfqpoint{3.176069in}{2.590730in}}%
\pgfpathlineto{\pgfqpoint{3.273923in}{2.590730in}}%
\pgfpathlineto{\pgfqpoint{3.371776in}{2.590730in}}%
\pgfpathlineto{\pgfqpoint{3.469630in}{2.590730in}}%
\pgfpathlineto{\pgfqpoint{3.567484in}{2.590730in}}%
\pgfpathlineto{\pgfqpoint{3.665337in}{2.590730in}}%
\pgfpathlineto{\pgfqpoint{3.763191in}{2.590730in}}%
\pgfpathlineto{\pgfqpoint{3.861044in}{2.590730in}}%
\pgfpathlineto{\pgfqpoint{3.958898in}{2.590730in}}%
\pgfpathlineto{\pgfqpoint{4.056751in}{2.590730in}}%
\pgfpathlineto{\pgfqpoint{4.154605in}{2.590730in}}%
\pgfusepath{stroke}%
\end{pgfscope}%
\begin{pgfscope}%
\pgfpathrectangle{\pgfqpoint{0.455741in}{0.385730in}}{\pgfqpoint{3.875000in}{2.310000in}}%
\pgfusepath{clip}%
\pgfsetbuttcap%
\pgfsetroundjoin%
\pgfsetlinewidth{0.803000pt}%
\definecolor{currentstroke}{rgb}{0.000000,0.356863,0.509804}%
\pgfsetstrokecolor{currentstroke}%
\pgfsetdash{{2.960000pt}{1.280000pt}}{0.000000pt}%
\pgfpathmoveto{\pgfqpoint{0.631877in}{0.498966in}}%
\pgfpathlineto{\pgfqpoint{0.729731in}{0.490730in}}%
\pgfpathlineto{\pgfqpoint{0.827585in}{0.490730in}}%
\pgfpathlineto{\pgfqpoint{0.925438in}{0.490730in}}%
\pgfpathlineto{\pgfqpoint{1.023292in}{0.490730in}}%
\pgfpathlineto{\pgfqpoint{1.121145in}{0.498966in}}%
\pgfpathlineto{\pgfqpoint{1.218999in}{0.581319in}}%
\pgfpathlineto{\pgfqpoint{1.316852in}{0.886025in}}%
\pgfpathlineto{\pgfqpoint{1.414706in}{0.968377in}}%
\pgfpathlineto{\pgfqpoint{1.512559in}{1.396613in}}%
\pgfpathlineto{\pgfqpoint{1.610413in}{1.693083in}}%
\pgfpathlineto{\pgfqpoint{1.708266in}{2.055436in}}%
\pgfpathlineto{\pgfqpoint{1.806120in}{2.277789in}}%
\pgfpathlineto{\pgfqpoint{1.903973in}{2.417789in}}%
\pgfpathlineto{\pgfqpoint{2.001827in}{2.467201in}}%
\pgfpathlineto{\pgfqpoint{2.099681in}{2.516613in}}%
\pgfpathlineto{\pgfqpoint{2.197534in}{2.533083in}}%
\pgfpathlineto{\pgfqpoint{2.295388in}{2.549554in}}%
\pgfpathlineto{\pgfqpoint{2.393241in}{2.566025in}}%
\pgfpathlineto{\pgfqpoint{2.491095in}{2.574260in}}%
\pgfpathlineto{\pgfqpoint{2.588948in}{2.582495in}}%
\pgfpathlineto{\pgfqpoint{2.686802in}{2.582495in}}%
\pgfpathlineto{\pgfqpoint{2.784655in}{2.590730in}}%
\pgfpathlineto{\pgfqpoint{2.882509in}{2.582495in}}%
\pgfpathlineto{\pgfqpoint{2.980362in}{2.590730in}}%
\pgfpathlineto{\pgfqpoint{3.078216in}{2.590730in}}%
\pgfpathlineto{\pgfqpoint{3.176069in}{2.590730in}}%
\pgfpathlineto{\pgfqpoint{3.273923in}{2.590730in}}%
\pgfpathlineto{\pgfqpoint{3.371776in}{2.590730in}}%
\pgfpathlineto{\pgfqpoint{3.469630in}{2.590730in}}%
\pgfpathlineto{\pgfqpoint{3.567484in}{2.590730in}}%
\pgfpathlineto{\pgfqpoint{3.665337in}{2.590730in}}%
\pgfpathlineto{\pgfqpoint{3.763191in}{2.590730in}}%
\pgfpathlineto{\pgfqpoint{3.861044in}{2.590730in}}%
\pgfpathlineto{\pgfqpoint{3.958898in}{2.590730in}}%
\pgfpathlineto{\pgfqpoint{4.056751in}{2.590730in}}%
\pgfpathlineto{\pgfqpoint{4.154605in}{2.590730in}}%
\pgfusepath{stroke}%
\end{pgfscope}%
\begin{pgfscope}%
\pgfpathrectangle{\pgfqpoint{0.455741in}{0.385730in}}{\pgfqpoint{3.875000in}{2.310000in}}%
\pgfusepath{clip}%
\pgfsetbuttcap%
\pgfsetroundjoin%
\pgfsetlinewidth{0.803000pt}%
\definecolor{currentstroke}{rgb}{0.490196,0.588235,0.431373}%
\pgfsetstrokecolor{currentstroke}%
\pgfsetdash{{2.960000pt}{1.280000pt}}{0.000000pt}%
\pgfpathmoveto{\pgfqpoint{0.631877in}{0.498966in}}%
\pgfpathlineto{\pgfqpoint{0.729731in}{0.490730in}}%
\pgfpathlineto{\pgfqpoint{0.827585in}{0.490730in}}%
\pgfpathlineto{\pgfqpoint{0.925438in}{0.490730in}}%
\pgfpathlineto{\pgfqpoint{1.023292in}{0.498966in}}%
\pgfpathlineto{\pgfqpoint{1.121145in}{0.515436in}}%
\pgfpathlineto{\pgfqpoint{1.218999in}{0.614260in}}%
\pgfpathlineto{\pgfqpoint{1.316852in}{0.918966in}}%
\pgfpathlineto{\pgfqpoint{1.414706in}{1.075436in}}%
\pgfpathlineto{\pgfqpoint{1.512559in}{1.520142in}}%
\pgfpathlineto{\pgfqpoint{1.610413in}{1.767201in}}%
\pgfpathlineto{\pgfqpoint{1.708266in}{2.129554in}}%
\pgfpathlineto{\pgfqpoint{1.806120in}{2.318966in}}%
\pgfpathlineto{\pgfqpoint{1.903973in}{2.434260in}}%
\pgfpathlineto{\pgfqpoint{2.001827in}{2.458966in}}%
\pgfpathlineto{\pgfqpoint{2.099681in}{2.516613in}}%
\pgfpathlineto{\pgfqpoint{2.197534in}{2.533083in}}%
\pgfpathlineto{\pgfqpoint{2.295388in}{2.549554in}}%
\pgfpathlineto{\pgfqpoint{2.393241in}{2.566025in}}%
\pgfpathlineto{\pgfqpoint{2.491095in}{2.574260in}}%
\pgfpathlineto{\pgfqpoint{2.588948in}{2.574260in}}%
\pgfpathlineto{\pgfqpoint{2.686802in}{2.574260in}}%
\pgfpathlineto{\pgfqpoint{2.784655in}{2.582495in}}%
\pgfpathlineto{\pgfqpoint{2.882509in}{2.590730in}}%
\pgfpathlineto{\pgfqpoint{2.980362in}{2.590730in}}%
\pgfpathlineto{\pgfqpoint{3.078216in}{2.590730in}}%
\pgfpathlineto{\pgfqpoint{3.176069in}{2.590730in}}%
\pgfpathlineto{\pgfqpoint{3.273923in}{2.590730in}}%
\pgfpathlineto{\pgfqpoint{3.371776in}{2.590730in}}%
\pgfpathlineto{\pgfqpoint{3.469630in}{2.590730in}}%
\pgfpathlineto{\pgfqpoint{3.567484in}{2.590730in}}%
\pgfpathlineto{\pgfqpoint{3.665337in}{2.590730in}}%
\pgfpathlineto{\pgfqpoint{3.763191in}{2.590730in}}%
\pgfpathlineto{\pgfqpoint{3.861044in}{2.590730in}}%
\pgfpathlineto{\pgfqpoint{3.958898in}{2.590730in}}%
\pgfpathlineto{\pgfqpoint{4.056751in}{2.590730in}}%
\pgfpathlineto{\pgfqpoint{4.154605in}{2.590730in}}%
\pgfusepath{stroke}%
\end{pgfscope}%
\begin{pgfscope}%
\pgfpathrectangle{\pgfqpoint{0.455741in}{0.385730in}}{\pgfqpoint{3.875000in}{2.310000in}}%
\pgfusepath{clip}%
\pgfsetbuttcap%
\pgfsetroundjoin%
\pgfsetlinewidth{0.803000pt}%
\definecolor{currentstroke}{rgb}{0.843137,0.666667,0.313725}%
\pgfsetstrokecolor{currentstroke}%
\pgfsetdash{{2.960000pt}{1.280000pt}}{0.000000pt}%
\pgfpathmoveto{\pgfqpoint{0.631877in}{0.498966in}}%
\pgfpathlineto{\pgfqpoint{0.729731in}{0.490730in}}%
\pgfpathlineto{\pgfqpoint{0.827585in}{0.490730in}}%
\pgfpathlineto{\pgfqpoint{0.925438in}{0.490730in}}%
\pgfpathlineto{\pgfqpoint{1.023292in}{0.515436in}}%
\pgfpathlineto{\pgfqpoint{1.121145in}{0.564848in}}%
\pgfpathlineto{\pgfqpoint{1.218999in}{0.820142in}}%
\pgfpathlineto{\pgfqpoint{1.316852in}{1.273083in}}%
\pgfpathlineto{\pgfqpoint{1.414706in}{1.511907in}}%
\pgfpathlineto{\pgfqpoint{1.512559in}{1.931907in}}%
\pgfpathlineto{\pgfqpoint{1.610413in}{2.113083in}}%
\pgfpathlineto{\pgfqpoint{1.708266in}{2.327201in}}%
\pgfpathlineto{\pgfqpoint{1.806120in}{2.426025in}}%
\pgfpathlineto{\pgfqpoint{1.903973in}{2.475436in}}%
\pgfpathlineto{\pgfqpoint{2.001827in}{2.508377in}}%
\pgfpathlineto{\pgfqpoint{2.099681in}{2.533083in}}%
\pgfpathlineto{\pgfqpoint{2.197534in}{2.549554in}}%
\pgfpathlineto{\pgfqpoint{2.295388in}{2.566025in}}%
\pgfpathlineto{\pgfqpoint{2.393241in}{2.574260in}}%
\pgfpathlineto{\pgfqpoint{2.491095in}{2.574260in}}%
\pgfpathlineto{\pgfqpoint{2.588948in}{2.582495in}}%
\pgfpathlineto{\pgfqpoint{2.686802in}{2.582495in}}%
\pgfpathlineto{\pgfqpoint{2.784655in}{2.590730in}}%
\pgfpathlineto{\pgfqpoint{2.882509in}{2.590730in}}%
\pgfpathlineto{\pgfqpoint{2.980362in}{2.590730in}}%
\pgfpathlineto{\pgfqpoint{3.078216in}{2.590730in}}%
\pgfpathlineto{\pgfqpoint{3.176069in}{2.590730in}}%
\pgfpathlineto{\pgfqpoint{3.273923in}{2.590730in}}%
\pgfpathlineto{\pgfqpoint{3.371776in}{2.590730in}}%
\pgfpathlineto{\pgfqpoint{3.469630in}{2.590730in}}%
\pgfpathlineto{\pgfqpoint{3.567484in}{2.590730in}}%
\pgfpathlineto{\pgfqpoint{3.665337in}{2.590730in}}%
\pgfpathlineto{\pgfqpoint{3.763191in}{2.590730in}}%
\pgfpathlineto{\pgfqpoint{3.861044in}{2.590730in}}%
\pgfpathlineto{\pgfqpoint{3.958898in}{2.590730in}}%
\pgfpathlineto{\pgfqpoint{4.056751in}{2.590730in}}%
\pgfpathlineto{\pgfqpoint{4.154605in}{2.590730in}}%
\pgfusepath{stroke}%
\end{pgfscope}%
\begin{pgfscope}%
\pgfpathrectangle{\pgfqpoint{0.455741in}{0.385730in}}{\pgfqpoint{3.875000in}{2.310000in}}%
\pgfusepath{clip}%
\pgfsetbuttcap%
\pgfsetroundjoin%
\pgfsetlinewidth{0.803000pt}%
\definecolor{currentstroke}{rgb}{0.333333,0.333333,0.333333}%
\pgfsetstrokecolor{currentstroke}%
\pgfsetdash{{2.960000pt}{1.280000pt}}{0.000000pt}%
\pgfpathmoveto{\pgfqpoint{0.631877in}{0.498966in}}%
\pgfpathlineto{\pgfqpoint{0.729731in}{0.490730in}}%
\pgfpathlineto{\pgfqpoint{0.827585in}{0.490730in}}%
\pgfpathlineto{\pgfqpoint{0.925438in}{0.490730in}}%
\pgfpathlineto{\pgfqpoint{1.023292in}{0.507201in}}%
\pgfpathlineto{\pgfqpoint{1.121145in}{0.548377in}}%
\pgfpathlineto{\pgfqpoint{1.218999in}{0.729554in}}%
\pgfpathlineto{\pgfqpoint{1.316852in}{1.116613in}}%
\pgfpathlineto{\pgfqpoint{1.414706in}{1.330730in}}%
\pgfpathlineto{\pgfqpoint{1.512559in}{1.734260in}}%
\pgfpathlineto{\pgfqpoint{1.610413in}{1.973083in}}%
\pgfpathlineto{\pgfqpoint{1.708266in}{2.236613in}}%
\pgfpathlineto{\pgfqpoint{1.806120in}{2.360142in}}%
\pgfpathlineto{\pgfqpoint{1.903973in}{2.450730in}}%
\pgfpathlineto{\pgfqpoint{2.001827in}{2.483672in}}%
\pgfpathlineto{\pgfqpoint{2.099681in}{2.516613in}}%
\pgfpathlineto{\pgfqpoint{2.197534in}{2.541319in}}%
\pgfpathlineto{\pgfqpoint{2.295388in}{2.557789in}}%
\pgfpathlineto{\pgfqpoint{2.393241in}{2.574260in}}%
\pgfpathlineto{\pgfqpoint{2.491095in}{2.574260in}}%
\pgfpathlineto{\pgfqpoint{2.588948in}{2.582495in}}%
\pgfpathlineto{\pgfqpoint{2.686802in}{2.590730in}}%
\pgfpathlineto{\pgfqpoint{2.784655in}{2.590730in}}%
\pgfpathlineto{\pgfqpoint{2.882509in}{2.590730in}}%
\pgfpathlineto{\pgfqpoint{2.980362in}{2.590730in}}%
\pgfpathlineto{\pgfqpoint{3.078216in}{2.590730in}}%
\pgfpathlineto{\pgfqpoint{3.176069in}{2.590730in}}%
\pgfpathlineto{\pgfqpoint{3.273923in}{2.590730in}}%
\pgfpathlineto{\pgfqpoint{3.371776in}{2.590730in}}%
\pgfpathlineto{\pgfqpoint{3.469630in}{2.590730in}}%
\pgfpathlineto{\pgfqpoint{3.567484in}{2.590730in}}%
\pgfpathlineto{\pgfqpoint{3.665337in}{2.590730in}}%
\pgfpathlineto{\pgfqpoint{3.763191in}{2.590730in}}%
\pgfpathlineto{\pgfqpoint{3.861044in}{2.590730in}}%
\pgfpathlineto{\pgfqpoint{3.958898in}{2.590730in}}%
\pgfpathlineto{\pgfqpoint{4.056751in}{2.590730in}}%
\pgfpathlineto{\pgfqpoint{4.154605in}{2.590730in}}%
\pgfusepath{stroke}%
\end{pgfscope}%
\begin{pgfscope}%
\pgfpathrectangle{\pgfqpoint{0.455741in}{0.385730in}}{\pgfqpoint{3.875000in}{2.310000in}}%
\pgfusepath{clip}%
\pgfsetbuttcap%
\pgfsetroundjoin%
\pgfsetlinewidth{0.803000pt}%
\definecolor{currentstroke}{rgb}{0.686275,0.352941,0.313725}%
\pgfsetstrokecolor{currentstroke}%
\pgfsetdash{{2.960000pt}{1.280000pt}}{0.000000pt}%
\pgfpathmoveto{\pgfqpoint{0.631877in}{0.490730in}}%
\pgfpathlineto{\pgfqpoint{0.729731in}{0.490730in}}%
\pgfpathlineto{\pgfqpoint{0.827585in}{0.490730in}}%
\pgfpathlineto{\pgfqpoint{0.925438in}{0.490730in}}%
\pgfpathlineto{\pgfqpoint{1.023292in}{0.490730in}}%
\pgfpathlineto{\pgfqpoint{1.121145in}{0.498966in}}%
\pgfpathlineto{\pgfqpoint{1.218999in}{0.564848in}}%
\pgfpathlineto{\pgfqpoint{1.316852in}{0.787201in}}%
\pgfpathlineto{\pgfqpoint{1.414706in}{0.902495in}}%
\pgfpathlineto{\pgfqpoint{1.512559in}{1.273083in}}%
\pgfpathlineto{\pgfqpoint{1.610413in}{1.643672in}}%
\pgfpathlineto{\pgfqpoint{1.708266in}{1.989554in}}%
\pgfpathlineto{\pgfqpoint{1.806120in}{2.195436in}}%
\pgfpathlineto{\pgfqpoint{1.903973in}{2.360142in}}%
\pgfpathlineto{\pgfqpoint{2.001827in}{2.434260in}}%
\pgfpathlineto{\pgfqpoint{2.099681in}{2.500142in}}%
\pgfpathlineto{\pgfqpoint{2.197534in}{2.524848in}}%
\pgfpathlineto{\pgfqpoint{2.295388in}{2.541319in}}%
\pgfpathlineto{\pgfqpoint{2.393241in}{2.557789in}}%
\pgfpathlineto{\pgfqpoint{2.491095in}{2.566025in}}%
\pgfpathlineto{\pgfqpoint{2.588948in}{2.574260in}}%
\pgfpathlineto{\pgfqpoint{2.686802in}{2.574260in}}%
\pgfpathlineto{\pgfqpoint{2.784655in}{2.582495in}}%
\pgfpathlineto{\pgfqpoint{2.882509in}{2.582495in}}%
\pgfpathlineto{\pgfqpoint{2.980362in}{2.590730in}}%
\pgfpathlineto{\pgfqpoint{3.078216in}{2.590730in}}%
\pgfpathlineto{\pgfqpoint{3.176069in}{2.590730in}}%
\pgfpathlineto{\pgfqpoint{3.273923in}{2.590730in}}%
\pgfpathlineto{\pgfqpoint{3.371776in}{2.590730in}}%
\pgfpathlineto{\pgfqpoint{3.469630in}{2.590730in}}%
\pgfpathlineto{\pgfqpoint{3.567484in}{2.590730in}}%
\pgfpathlineto{\pgfqpoint{3.665337in}{2.590730in}}%
\pgfpathlineto{\pgfqpoint{3.763191in}{2.590730in}}%
\pgfpathlineto{\pgfqpoint{3.861044in}{2.590730in}}%
\pgfpathlineto{\pgfqpoint{3.958898in}{2.590730in}}%
\pgfpathlineto{\pgfqpoint{4.056751in}{2.590730in}}%
\pgfpathlineto{\pgfqpoint{4.154605in}{2.590730in}}%
\pgfusepath{stroke}%
\end{pgfscope}%
\begin{pgfscope}%
\pgfpathrectangle{\pgfqpoint{0.455741in}{0.385730in}}{\pgfqpoint{3.875000in}{2.310000in}}%
\pgfusepath{clip}%
\pgfsetrectcap%
\pgfsetroundjoin%
\pgfsetlinewidth{0.803000pt}%
\definecolor{currentstroke}{rgb}{0.000000,0.356863,0.509804}%
\pgfsetstrokecolor{currentstroke}%
\pgfsetdash{}{0pt}%
\pgfpathmoveto{\pgfqpoint{0.631877in}{0.490730in}}%
\pgfpathlineto{\pgfqpoint{0.729731in}{0.490730in}}%
\pgfpathlineto{\pgfqpoint{0.827585in}{0.490730in}}%
\pgfpathlineto{\pgfqpoint{0.925438in}{0.490730in}}%
\pgfpathlineto{\pgfqpoint{1.023292in}{0.490730in}}%
\pgfpathlineto{\pgfqpoint{1.121145in}{0.490730in}}%
\pgfpathlineto{\pgfqpoint{1.218999in}{0.490730in}}%
\pgfpathlineto{\pgfqpoint{1.316852in}{0.490730in}}%
\pgfpathlineto{\pgfqpoint{1.414706in}{0.490730in}}%
\pgfpathlineto{\pgfqpoint{1.512559in}{0.490730in}}%
\pgfpathlineto{\pgfqpoint{1.610413in}{0.490730in}}%
\pgfpathlineto{\pgfqpoint{1.708266in}{0.490730in}}%
\pgfpathlineto{\pgfqpoint{1.806120in}{0.490730in}}%
\pgfpathlineto{\pgfqpoint{1.903973in}{0.490730in}}%
\pgfpathlineto{\pgfqpoint{2.001827in}{0.515436in}}%
\pgfpathlineto{\pgfqpoint{2.099681in}{0.540142in}}%
\pgfpathlineto{\pgfqpoint{2.197534in}{0.680142in}}%
\pgfpathlineto{\pgfqpoint{2.295388in}{0.836613in}}%
\pgfpathlineto{\pgfqpoint{2.393241in}{1.174260in}}%
\pgfpathlineto{\pgfqpoint{2.491095in}{1.676613in}}%
\pgfpathlineto{\pgfqpoint{2.588948in}{1.973083in}}%
\pgfpathlineto{\pgfqpoint{2.686802in}{2.302495in}}%
\pgfpathlineto{\pgfqpoint{2.784655in}{2.417789in}}%
\pgfpathlineto{\pgfqpoint{2.882509in}{2.500142in}}%
\pgfpathlineto{\pgfqpoint{2.980362in}{2.516613in}}%
\pgfpathlineto{\pgfqpoint{3.078216in}{2.541319in}}%
\pgfpathlineto{\pgfqpoint{3.176069in}{2.557789in}}%
\pgfpathlineto{\pgfqpoint{3.273923in}{2.566025in}}%
\pgfpathlineto{\pgfqpoint{3.371776in}{2.574260in}}%
\pgfpathlineto{\pgfqpoint{3.469630in}{2.582495in}}%
\pgfpathlineto{\pgfqpoint{3.567484in}{2.590730in}}%
\pgfpathlineto{\pgfqpoint{3.665337in}{2.590730in}}%
\pgfpathlineto{\pgfqpoint{3.763191in}{2.590730in}}%
\pgfpathlineto{\pgfqpoint{3.861044in}{2.590730in}}%
\pgfpathlineto{\pgfqpoint{3.958898in}{2.590730in}}%
\pgfpathlineto{\pgfqpoint{4.056751in}{2.590730in}}%
\pgfpathlineto{\pgfqpoint{4.154605in}{2.590730in}}%
\pgfusepath{stroke}%
\end{pgfscope}%
\begin{pgfscope}%
\pgfpathrectangle{\pgfqpoint{0.455741in}{0.385730in}}{\pgfqpoint{3.875000in}{2.310000in}}%
\pgfusepath{clip}%
\pgfsetrectcap%
\pgfsetroundjoin%
\pgfsetlinewidth{0.803000pt}%
\definecolor{currentstroke}{rgb}{0.490196,0.588235,0.431373}%
\pgfsetstrokecolor{currentstroke}%
\pgfsetdash{}{0pt}%
\pgfpathmoveto{\pgfqpoint{0.631877in}{0.490730in}}%
\pgfpathlineto{\pgfqpoint{0.729731in}{0.490730in}}%
\pgfpathlineto{\pgfqpoint{0.827585in}{0.490730in}}%
\pgfpathlineto{\pgfqpoint{0.925438in}{0.490730in}}%
\pgfpathlineto{\pgfqpoint{1.023292in}{0.490730in}}%
\pgfpathlineto{\pgfqpoint{1.121145in}{0.490730in}}%
\pgfpathlineto{\pgfqpoint{1.218999in}{0.490730in}}%
\pgfpathlineto{\pgfqpoint{1.316852in}{0.490730in}}%
\pgfpathlineto{\pgfqpoint{1.414706in}{0.490730in}}%
\pgfpathlineto{\pgfqpoint{1.512559in}{0.490730in}}%
\pgfpathlineto{\pgfqpoint{1.610413in}{0.490730in}}%
\pgfpathlineto{\pgfqpoint{1.708266in}{0.490730in}}%
\pgfpathlineto{\pgfqpoint{1.806120in}{0.490730in}}%
\pgfpathlineto{\pgfqpoint{1.903973in}{0.490730in}}%
\pgfpathlineto{\pgfqpoint{2.001827in}{0.507201in}}%
\pgfpathlineto{\pgfqpoint{2.099681in}{0.523672in}}%
\pgfpathlineto{\pgfqpoint{2.197534in}{0.581319in}}%
\pgfpathlineto{\pgfqpoint{2.295388in}{0.655436in}}%
\pgfpathlineto{\pgfqpoint{2.393241in}{0.927201in}}%
\pgfpathlineto{\pgfqpoint{2.491095in}{1.594260in}}%
\pgfpathlineto{\pgfqpoint{2.588948in}{1.874260in}}%
\pgfpathlineto{\pgfqpoint{2.686802in}{2.162495in}}%
\pgfpathlineto{\pgfqpoint{2.784655in}{2.376613in}}%
\pgfpathlineto{\pgfqpoint{2.882509in}{2.458966in}}%
\pgfpathlineto{\pgfqpoint{2.980362in}{2.491907in}}%
\pgfpathlineto{\pgfqpoint{3.078216in}{2.541319in}}%
\pgfpathlineto{\pgfqpoint{3.176069in}{2.557789in}}%
\pgfpathlineto{\pgfqpoint{3.273923in}{2.574260in}}%
\pgfpathlineto{\pgfqpoint{3.371776in}{2.582495in}}%
\pgfpathlineto{\pgfqpoint{3.469630in}{2.590730in}}%
\pgfpathlineto{\pgfqpoint{3.567484in}{2.590730in}}%
\pgfpathlineto{\pgfqpoint{3.665337in}{2.590730in}}%
\pgfpathlineto{\pgfqpoint{3.763191in}{2.590730in}}%
\pgfpathlineto{\pgfqpoint{3.861044in}{2.590730in}}%
\pgfpathlineto{\pgfqpoint{3.958898in}{2.590730in}}%
\pgfpathlineto{\pgfqpoint{4.056751in}{2.590730in}}%
\pgfpathlineto{\pgfqpoint{4.154605in}{2.590730in}}%
\pgfusepath{stroke}%
\end{pgfscope}%
\begin{pgfscope}%
\pgfpathrectangle{\pgfqpoint{0.455741in}{0.385730in}}{\pgfqpoint{3.875000in}{2.310000in}}%
\pgfusepath{clip}%
\pgfsetrectcap%
\pgfsetroundjoin%
\pgfsetlinewidth{0.803000pt}%
\definecolor{currentstroke}{rgb}{0.843137,0.666667,0.313725}%
\pgfsetstrokecolor{currentstroke}%
\pgfsetdash{}{0pt}%
\pgfpathmoveto{\pgfqpoint{0.631877in}{0.490730in}}%
\pgfpathlineto{\pgfqpoint{0.729731in}{0.490730in}}%
\pgfpathlineto{\pgfqpoint{0.827585in}{0.490730in}}%
\pgfpathlineto{\pgfqpoint{0.925438in}{0.490730in}}%
\pgfpathlineto{\pgfqpoint{1.023292in}{0.490730in}}%
\pgfpathlineto{\pgfqpoint{1.121145in}{0.490730in}}%
\pgfpathlineto{\pgfqpoint{1.218999in}{0.490730in}}%
\pgfpathlineto{\pgfqpoint{1.316852in}{0.490730in}}%
\pgfpathlineto{\pgfqpoint{1.414706in}{0.490730in}}%
\pgfpathlineto{\pgfqpoint{1.512559in}{0.490730in}}%
\pgfpathlineto{\pgfqpoint{1.610413in}{0.490730in}}%
\pgfpathlineto{\pgfqpoint{1.708266in}{0.490730in}}%
\pgfpathlineto{\pgfqpoint{1.806120in}{0.490730in}}%
\pgfpathlineto{\pgfqpoint{1.903973in}{0.498966in}}%
\pgfpathlineto{\pgfqpoint{2.001827in}{0.540142in}}%
\pgfpathlineto{\pgfqpoint{2.099681in}{0.581319in}}%
\pgfpathlineto{\pgfqpoint{2.197534in}{0.713083in}}%
\pgfpathlineto{\pgfqpoint{2.295388in}{0.844848in}}%
\pgfpathlineto{\pgfqpoint{2.393241in}{1.124848in}}%
\pgfpathlineto{\pgfqpoint{2.491095in}{1.701319in}}%
\pgfpathlineto{\pgfqpoint{2.588948in}{2.047201in}}%
\pgfpathlineto{\pgfqpoint{2.686802in}{2.351907in}}%
\pgfpathlineto{\pgfqpoint{2.784655in}{2.426025in}}%
\pgfpathlineto{\pgfqpoint{2.882509in}{2.491907in}}%
\pgfpathlineto{\pgfqpoint{2.980362in}{2.524848in}}%
\pgfpathlineto{\pgfqpoint{3.078216in}{2.549554in}}%
\pgfpathlineto{\pgfqpoint{3.176069in}{2.557789in}}%
\pgfpathlineto{\pgfqpoint{3.273923in}{2.574260in}}%
\pgfpathlineto{\pgfqpoint{3.371776in}{2.582495in}}%
\pgfpathlineto{\pgfqpoint{3.469630in}{2.590730in}}%
\pgfpathlineto{\pgfqpoint{3.567484in}{2.590730in}}%
\pgfpathlineto{\pgfqpoint{3.665337in}{2.590730in}}%
\pgfpathlineto{\pgfqpoint{3.763191in}{2.590730in}}%
\pgfpathlineto{\pgfqpoint{3.861044in}{2.590730in}}%
\pgfpathlineto{\pgfqpoint{3.958898in}{2.590730in}}%
\pgfpathlineto{\pgfqpoint{4.056751in}{2.590730in}}%
\pgfpathlineto{\pgfqpoint{4.154605in}{2.590730in}}%
\pgfusepath{stroke}%
\end{pgfscope}%
\begin{pgfscope}%
\pgfpathrectangle{\pgfqpoint{0.455741in}{0.385730in}}{\pgfqpoint{3.875000in}{2.310000in}}%
\pgfusepath{clip}%
\pgfsetrectcap%
\pgfsetroundjoin%
\pgfsetlinewidth{0.803000pt}%
\definecolor{currentstroke}{rgb}{0.333333,0.333333,0.333333}%
\pgfsetstrokecolor{currentstroke}%
\pgfsetdash{}{0pt}%
\pgfpathmoveto{\pgfqpoint{0.631877in}{0.490730in}}%
\pgfpathlineto{\pgfqpoint{0.729731in}{0.490730in}}%
\pgfpathlineto{\pgfqpoint{0.827585in}{0.490730in}}%
\pgfpathlineto{\pgfqpoint{0.925438in}{0.490730in}}%
\pgfpathlineto{\pgfqpoint{1.023292in}{0.490730in}}%
\pgfpathlineto{\pgfqpoint{1.121145in}{0.490730in}}%
\pgfpathlineto{\pgfqpoint{1.218999in}{0.490730in}}%
\pgfpathlineto{\pgfqpoint{1.316852in}{0.490730in}}%
\pgfpathlineto{\pgfqpoint{1.414706in}{0.490730in}}%
\pgfpathlineto{\pgfqpoint{1.512559in}{0.490730in}}%
\pgfpathlineto{\pgfqpoint{1.610413in}{0.490730in}}%
\pgfpathlineto{\pgfqpoint{1.708266in}{0.490730in}}%
\pgfpathlineto{\pgfqpoint{1.806120in}{0.490730in}}%
\pgfpathlineto{\pgfqpoint{1.903973in}{0.490730in}}%
\pgfpathlineto{\pgfqpoint{2.001827in}{0.523672in}}%
\pgfpathlineto{\pgfqpoint{2.099681in}{0.556613in}}%
\pgfpathlineto{\pgfqpoint{2.197534in}{0.680142in}}%
\pgfpathlineto{\pgfqpoint{2.295388in}{0.820142in}}%
\pgfpathlineto{\pgfqpoint{2.393241in}{1.124848in}}%
\pgfpathlineto{\pgfqpoint{2.491095in}{1.791907in}}%
\pgfpathlineto{\pgfqpoint{2.588948in}{2.187201in}}%
\pgfpathlineto{\pgfqpoint{2.686802in}{2.376613in}}%
\pgfpathlineto{\pgfqpoint{2.784655in}{2.434260in}}%
\pgfpathlineto{\pgfqpoint{2.882509in}{2.491907in}}%
\pgfpathlineto{\pgfqpoint{2.980362in}{2.516613in}}%
\pgfpathlineto{\pgfqpoint{3.078216in}{2.541319in}}%
\pgfpathlineto{\pgfqpoint{3.176069in}{2.566025in}}%
\pgfpathlineto{\pgfqpoint{3.273923in}{2.566025in}}%
\pgfpathlineto{\pgfqpoint{3.371776in}{2.574260in}}%
\pgfpathlineto{\pgfqpoint{3.469630in}{2.574260in}}%
\pgfpathlineto{\pgfqpoint{3.567484in}{2.574260in}}%
\pgfpathlineto{\pgfqpoint{3.665337in}{2.590730in}}%
\pgfpathlineto{\pgfqpoint{3.763191in}{2.590730in}}%
\pgfpathlineto{\pgfqpoint{3.861044in}{2.590730in}}%
\pgfpathlineto{\pgfqpoint{3.958898in}{2.590730in}}%
\pgfpathlineto{\pgfqpoint{4.056751in}{2.590730in}}%
\pgfpathlineto{\pgfqpoint{4.154605in}{2.590730in}}%
\pgfusepath{stroke}%
\end{pgfscope}%
\begin{pgfscope}%
\pgfpathrectangle{\pgfqpoint{0.455741in}{0.385730in}}{\pgfqpoint{3.875000in}{2.310000in}}%
\pgfusepath{clip}%
\pgfsetrectcap%
\pgfsetroundjoin%
\pgfsetlinewidth{0.803000pt}%
\definecolor{currentstroke}{rgb}{0.686275,0.352941,0.313725}%
\pgfsetstrokecolor{currentstroke}%
\pgfsetdash{}{0pt}%
\pgfpathmoveto{\pgfqpoint{0.631877in}{0.490730in}}%
\pgfpathlineto{\pgfqpoint{0.729731in}{0.490730in}}%
\pgfpathlineto{\pgfqpoint{0.827585in}{0.490730in}}%
\pgfpathlineto{\pgfqpoint{0.925438in}{0.490730in}}%
\pgfpathlineto{\pgfqpoint{1.023292in}{0.490730in}}%
\pgfpathlineto{\pgfqpoint{1.121145in}{0.490730in}}%
\pgfpathlineto{\pgfqpoint{1.218999in}{0.490730in}}%
\pgfpathlineto{\pgfqpoint{1.316852in}{0.490730in}}%
\pgfpathlineto{\pgfqpoint{1.414706in}{0.490730in}}%
\pgfpathlineto{\pgfqpoint{1.512559in}{0.490730in}}%
\pgfpathlineto{\pgfqpoint{1.610413in}{0.490730in}}%
\pgfpathlineto{\pgfqpoint{1.708266in}{0.490730in}}%
\pgfpathlineto{\pgfqpoint{1.806120in}{0.490730in}}%
\pgfpathlineto{\pgfqpoint{1.903973in}{0.490730in}}%
\pgfpathlineto{\pgfqpoint{2.001827in}{0.498966in}}%
\pgfpathlineto{\pgfqpoint{2.099681in}{0.507201in}}%
\pgfpathlineto{\pgfqpoint{2.197534in}{0.573083in}}%
\pgfpathlineto{\pgfqpoint{2.295388in}{0.778966in}}%
\pgfpathlineto{\pgfqpoint{2.393241in}{1.091907in}}%
\pgfpathlineto{\pgfqpoint{2.491095in}{1.701319in}}%
\pgfpathlineto{\pgfqpoint{2.588948in}{1.931907in}}%
\pgfpathlineto{\pgfqpoint{2.686802in}{2.195436in}}%
\pgfpathlineto{\pgfqpoint{2.784655in}{2.343672in}}%
\pgfpathlineto{\pgfqpoint{2.882509in}{2.450730in}}%
\pgfpathlineto{\pgfqpoint{2.980362in}{2.483672in}}%
\pgfpathlineto{\pgfqpoint{3.078216in}{2.524848in}}%
\pgfpathlineto{\pgfqpoint{3.176069in}{2.549554in}}%
\pgfpathlineto{\pgfqpoint{3.273923in}{2.566025in}}%
\pgfpathlineto{\pgfqpoint{3.371776in}{2.574260in}}%
\pgfpathlineto{\pgfqpoint{3.469630in}{2.590730in}}%
\pgfpathlineto{\pgfqpoint{3.567484in}{2.590730in}}%
\pgfpathlineto{\pgfqpoint{3.665337in}{2.590730in}}%
\pgfpathlineto{\pgfqpoint{3.763191in}{2.590730in}}%
\pgfpathlineto{\pgfqpoint{3.861044in}{2.590730in}}%
\pgfpathlineto{\pgfqpoint{3.958898in}{2.590730in}}%
\pgfpathlineto{\pgfqpoint{4.056751in}{2.590730in}}%
\pgfpathlineto{\pgfqpoint{4.154605in}{2.590730in}}%
\pgfusepath{stroke}%
\end{pgfscope}%
\begin{pgfscope}%
\pgfpathrectangle{\pgfqpoint{0.455741in}{0.385730in}}{\pgfqpoint{3.875000in}{2.310000in}}%
\pgfusepath{clip}%
\pgfsetrectcap%
\pgfsetroundjoin%
\pgfsetlinewidth{0.803000pt}%
\definecolor{currentstroke}{rgb}{0.000000,0.356863,0.509804}%
\pgfsetstrokecolor{currentstroke}%
\pgfsetdash{}{0pt}%
\pgfpathmoveto{\pgfqpoint{0.631877in}{0.490730in}}%
\pgfpathlineto{\pgfqpoint{0.729731in}{0.490730in}}%
\pgfpathlineto{\pgfqpoint{0.827585in}{0.490730in}}%
\pgfpathlineto{\pgfqpoint{0.925438in}{0.490730in}}%
\pgfpathlineto{\pgfqpoint{1.023292in}{0.490730in}}%
\pgfpathlineto{\pgfqpoint{1.121145in}{0.490730in}}%
\pgfpathlineto{\pgfqpoint{1.218999in}{0.490730in}}%
\pgfpathlineto{\pgfqpoint{1.316852in}{0.490730in}}%
\pgfpathlineto{\pgfqpoint{1.414706in}{0.490730in}}%
\pgfpathlineto{\pgfqpoint{1.512559in}{0.490730in}}%
\pgfpathlineto{\pgfqpoint{1.610413in}{0.490730in}}%
\pgfpathlineto{\pgfqpoint{1.708266in}{0.490730in}}%
\pgfpathlineto{\pgfqpoint{1.806120in}{0.490730in}}%
\pgfpathlineto{\pgfqpoint{1.903973in}{0.498966in}}%
\pgfpathlineto{\pgfqpoint{2.001827in}{0.531907in}}%
\pgfpathlineto{\pgfqpoint{2.099681in}{0.573083in}}%
\pgfpathlineto{\pgfqpoint{2.197534in}{0.729554in}}%
\pgfpathlineto{\pgfqpoint{2.295388in}{0.910730in}}%
\pgfpathlineto{\pgfqpoint{2.393241in}{1.248377in}}%
\pgfpathlineto{\pgfqpoint{2.491095in}{1.824848in}}%
\pgfpathlineto{\pgfqpoint{2.588948in}{2.162495in}}%
\pgfpathlineto{\pgfqpoint{2.686802in}{2.343672in}}%
\pgfpathlineto{\pgfqpoint{2.784655in}{2.417789in}}%
\pgfpathlineto{\pgfqpoint{2.882509in}{2.500142in}}%
\pgfpathlineto{\pgfqpoint{2.980362in}{2.524848in}}%
\pgfpathlineto{\pgfqpoint{3.078216in}{2.541319in}}%
\pgfpathlineto{\pgfqpoint{3.176069in}{2.557789in}}%
\pgfpathlineto{\pgfqpoint{3.273923in}{2.566025in}}%
\pgfpathlineto{\pgfqpoint{3.371776in}{2.574260in}}%
\pgfpathlineto{\pgfqpoint{3.469630in}{2.582495in}}%
\pgfpathlineto{\pgfqpoint{3.567484in}{2.590730in}}%
\pgfpathlineto{\pgfqpoint{3.665337in}{2.590730in}}%
\pgfpathlineto{\pgfqpoint{3.763191in}{2.590730in}}%
\pgfpathlineto{\pgfqpoint{3.861044in}{2.590730in}}%
\pgfpathlineto{\pgfqpoint{3.958898in}{2.590730in}}%
\pgfpathlineto{\pgfqpoint{4.056751in}{2.590730in}}%
\pgfpathlineto{\pgfqpoint{4.154605in}{2.590730in}}%
\pgfusepath{stroke}%
\end{pgfscope}%
\begin{pgfscope}%
\pgfpathrectangle{\pgfqpoint{0.455741in}{0.385730in}}{\pgfqpoint{3.875000in}{2.310000in}}%
\pgfusepath{clip}%
\pgfsetrectcap%
\pgfsetroundjoin%
\pgfsetlinewidth{0.803000pt}%
\definecolor{currentstroke}{rgb}{0.490196,0.588235,0.431373}%
\pgfsetstrokecolor{currentstroke}%
\pgfsetdash{}{0pt}%
\pgfpathmoveto{\pgfqpoint{0.631877in}{0.490730in}}%
\pgfpathlineto{\pgfqpoint{0.729731in}{0.490730in}}%
\pgfpathlineto{\pgfqpoint{0.827585in}{0.490730in}}%
\pgfpathlineto{\pgfqpoint{0.925438in}{0.490730in}}%
\pgfpathlineto{\pgfqpoint{1.023292in}{0.490730in}}%
\pgfpathlineto{\pgfqpoint{1.121145in}{0.490730in}}%
\pgfpathlineto{\pgfqpoint{1.218999in}{0.490730in}}%
\pgfpathlineto{\pgfqpoint{1.316852in}{0.490730in}}%
\pgfpathlineto{\pgfqpoint{1.414706in}{0.490730in}}%
\pgfpathlineto{\pgfqpoint{1.512559in}{0.490730in}}%
\pgfpathlineto{\pgfqpoint{1.610413in}{0.490730in}}%
\pgfpathlineto{\pgfqpoint{1.708266in}{0.490730in}}%
\pgfpathlineto{\pgfqpoint{1.806120in}{0.490730in}}%
\pgfpathlineto{\pgfqpoint{1.903973in}{0.490730in}}%
\pgfpathlineto{\pgfqpoint{2.001827in}{0.523672in}}%
\pgfpathlineto{\pgfqpoint{2.099681in}{0.540142in}}%
\pgfpathlineto{\pgfqpoint{2.197534in}{0.713083in}}%
\pgfpathlineto{\pgfqpoint{2.295388in}{0.877789in}}%
\pgfpathlineto{\pgfqpoint{2.393241in}{1.371907in}}%
\pgfpathlineto{\pgfqpoint{2.491095in}{1.898966in}}%
\pgfpathlineto{\pgfqpoint{2.588948in}{2.162495in}}%
\pgfpathlineto{\pgfqpoint{2.686802in}{2.343672in}}%
\pgfpathlineto{\pgfqpoint{2.784655in}{2.417789in}}%
\pgfpathlineto{\pgfqpoint{2.882509in}{2.491907in}}%
\pgfpathlineto{\pgfqpoint{2.980362in}{2.516613in}}%
\pgfpathlineto{\pgfqpoint{3.078216in}{2.549554in}}%
\pgfpathlineto{\pgfqpoint{3.176069in}{2.557789in}}%
\pgfpathlineto{\pgfqpoint{3.273923in}{2.574260in}}%
\pgfpathlineto{\pgfqpoint{3.371776in}{2.582495in}}%
\pgfpathlineto{\pgfqpoint{3.469630in}{2.590730in}}%
\pgfpathlineto{\pgfqpoint{3.567484in}{2.590730in}}%
\pgfpathlineto{\pgfqpoint{3.665337in}{2.590730in}}%
\pgfpathlineto{\pgfqpoint{3.763191in}{2.590730in}}%
\pgfpathlineto{\pgfqpoint{3.861044in}{2.590730in}}%
\pgfpathlineto{\pgfqpoint{3.958898in}{2.590730in}}%
\pgfpathlineto{\pgfqpoint{4.056751in}{2.590730in}}%
\pgfpathlineto{\pgfqpoint{4.154605in}{2.590730in}}%
\pgfusepath{stroke}%
\end{pgfscope}%
\begin{pgfscope}%
\pgfpathrectangle{\pgfqpoint{0.455741in}{0.385730in}}{\pgfqpoint{3.875000in}{2.310000in}}%
\pgfusepath{clip}%
\pgfsetrectcap%
\pgfsetroundjoin%
\pgfsetlinewidth{0.803000pt}%
\definecolor{currentstroke}{rgb}{0.843137,0.666667,0.313725}%
\pgfsetstrokecolor{currentstroke}%
\pgfsetdash{}{0pt}%
\pgfpathmoveto{\pgfqpoint{0.631877in}{0.490730in}}%
\pgfpathlineto{\pgfqpoint{0.729731in}{0.490730in}}%
\pgfpathlineto{\pgfqpoint{0.827585in}{0.490730in}}%
\pgfpathlineto{\pgfqpoint{0.925438in}{0.490730in}}%
\pgfpathlineto{\pgfqpoint{1.023292in}{0.490730in}}%
\pgfpathlineto{\pgfqpoint{1.121145in}{0.490730in}}%
\pgfpathlineto{\pgfqpoint{1.218999in}{0.490730in}}%
\pgfpathlineto{\pgfqpoint{1.316852in}{0.490730in}}%
\pgfpathlineto{\pgfqpoint{1.414706in}{0.490730in}}%
\pgfpathlineto{\pgfqpoint{1.512559in}{0.490730in}}%
\pgfpathlineto{\pgfqpoint{1.610413in}{0.490730in}}%
\pgfpathlineto{\pgfqpoint{1.708266in}{0.490730in}}%
\pgfpathlineto{\pgfqpoint{1.806120in}{0.490730in}}%
\pgfpathlineto{\pgfqpoint{1.903973in}{0.490730in}}%
\pgfpathlineto{\pgfqpoint{2.001827in}{0.523672in}}%
\pgfpathlineto{\pgfqpoint{2.099681in}{0.556613in}}%
\pgfpathlineto{\pgfqpoint{2.197534in}{0.713083in}}%
\pgfpathlineto{\pgfqpoint{2.295388in}{0.803672in}}%
\pgfpathlineto{\pgfqpoint{2.393241in}{1.338966in}}%
\pgfpathlineto{\pgfqpoint{2.491095in}{1.890730in}}%
\pgfpathlineto{\pgfqpoint{2.588948in}{2.006025in}}%
\pgfpathlineto{\pgfqpoint{2.686802in}{2.302495in}}%
\pgfpathlineto{\pgfqpoint{2.784655in}{2.409554in}}%
\pgfpathlineto{\pgfqpoint{2.882509in}{2.500142in}}%
\pgfpathlineto{\pgfqpoint{2.980362in}{2.524848in}}%
\pgfpathlineto{\pgfqpoint{3.078216in}{2.541319in}}%
\pgfpathlineto{\pgfqpoint{3.176069in}{2.557789in}}%
\pgfpathlineto{\pgfqpoint{3.273923in}{2.574260in}}%
\pgfpathlineto{\pgfqpoint{3.371776in}{2.582495in}}%
\pgfpathlineto{\pgfqpoint{3.469630in}{2.582495in}}%
\pgfpathlineto{\pgfqpoint{3.567484in}{2.590730in}}%
\pgfpathlineto{\pgfqpoint{3.665337in}{2.590730in}}%
\pgfpathlineto{\pgfqpoint{3.763191in}{2.590730in}}%
\pgfpathlineto{\pgfqpoint{3.861044in}{2.590730in}}%
\pgfpathlineto{\pgfqpoint{3.958898in}{2.590730in}}%
\pgfpathlineto{\pgfqpoint{4.056751in}{2.590730in}}%
\pgfpathlineto{\pgfqpoint{4.154605in}{2.590730in}}%
\pgfusepath{stroke}%
\end{pgfscope}%
\begin{pgfscope}%
\pgfpathrectangle{\pgfqpoint{0.455741in}{0.385730in}}{\pgfqpoint{3.875000in}{2.310000in}}%
\pgfusepath{clip}%
\pgfsetrectcap%
\pgfsetroundjoin%
\pgfsetlinewidth{0.803000pt}%
\definecolor{currentstroke}{rgb}{0.333333,0.333333,0.333333}%
\pgfsetstrokecolor{currentstroke}%
\pgfsetdash{}{0pt}%
\pgfpathmoveto{\pgfqpoint{0.631877in}{0.490730in}}%
\pgfpathlineto{\pgfqpoint{0.729731in}{0.490730in}}%
\pgfpathlineto{\pgfqpoint{0.827585in}{0.490730in}}%
\pgfpathlineto{\pgfqpoint{0.925438in}{0.490730in}}%
\pgfpathlineto{\pgfqpoint{1.023292in}{0.490730in}}%
\pgfpathlineto{\pgfqpoint{1.121145in}{0.490730in}}%
\pgfpathlineto{\pgfqpoint{1.218999in}{0.490730in}}%
\pgfpathlineto{\pgfqpoint{1.316852in}{0.490730in}}%
\pgfpathlineto{\pgfqpoint{1.414706in}{0.490730in}}%
\pgfpathlineto{\pgfqpoint{1.512559in}{0.490730in}}%
\pgfpathlineto{\pgfqpoint{1.610413in}{0.490730in}}%
\pgfpathlineto{\pgfqpoint{1.708266in}{0.490730in}}%
\pgfpathlineto{\pgfqpoint{1.806120in}{0.490730in}}%
\pgfpathlineto{\pgfqpoint{1.903973in}{0.490730in}}%
\pgfpathlineto{\pgfqpoint{2.001827in}{0.498966in}}%
\pgfpathlineto{\pgfqpoint{2.099681in}{0.507201in}}%
\pgfpathlineto{\pgfqpoint{2.197534in}{0.548377in}}%
\pgfpathlineto{\pgfqpoint{2.295388in}{0.713083in}}%
\pgfpathlineto{\pgfqpoint{2.393241in}{0.935436in}}%
\pgfpathlineto{\pgfqpoint{2.491095in}{1.487201in}}%
\pgfpathlineto{\pgfqpoint{2.588948in}{1.841319in}}%
\pgfpathlineto{\pgfqpoint{2.686802in}{2.220142in}}%
\pgfpathlineto{\pgfqpoint{2.784655in}{2.343672in}}%
\pgfpathlineto{\pgfqpoint{2.882509in}{2.417789in}}%
\pgfpathlineto{\pgfqpoint{2.980362in}{2.475436in}}%
\pgfpathlineto{\pgfqpoint{3.078216in}{2.516613in}}%
\pgfpathlineto{\pgfqpoint{3.176069in}{2.549554in}}%
\pgfpathlineto{\pgfqpoint{3.273923in}{2.557789in}}%
\pgfpathlineto{\pgfqpoint{3.371776in}{2.574260in}}%
\pgfpathlineto{\pgfqpoint{3.469630in}{2.574260in}}%
\pgfpathlineto{\pgfqpoint{3.567484in}{2.582495in}}%
\pgfpathlineto{\pgfqpoint{3.665337in}{2.590730in}}%
\pgfpathlineto{\pgfqpoint{3.763191in}{2.590730in}}%
\pgfpathlineto{\pgfqpoint{3.861044in}{2.590730in}}%
\pgfpathlineto{\pgfqpoint{3.958898in}{2.590730in}}%
\pgfpathlineto{\pgfqpoint{4.056751in}{2.590730in}}%
\pgfpathlineto{\pgfqpoint{4.154605in}{2.590730in}}%
\pgfusepath{stroke}%
\end{pgfscope}%
\begin{pgfscope}%
\pgfpathrectangle{\pgfqpoint{0.455741in}{0.385730in}}{\pgfqpoint{3.875000in}{2.310000in}}%
\pgfusepath{clip}%
\pgfsetrectcap%
\pgfsetroundjoin%
\pgfsetlinewidth{0.803000pt}%
\definecolor{currentstroke}{rgb}{0.686275,0.352941,0.313725}%
\pgfsetstrokecolor{currentstroke}%
\pgfsetdash{}{0pt}%
\pgfpathmoveto{\pgfqpoint{0.631877in}{0.490730in}}%
\pgfpathlineto{\pgfqpoint{0.729731in}{0.490730in}}%
\pgfpathlineto{\pgfqpoint{0.827585in}{0.490730in}}%
\pgfpathlineto{\pgfqpoint{0.925438in}{0.490730in}}%
\pgfpathlineto{\pgfqpoint{1.023292in}{0.490730in}}%
\pgfpathlineto{\pgfqpoint{1.121145in}{0.490730in}}%
\pgfpathlineto{\pgfqpoint{1.218999in}{0.490730in}}%
\pgfpathlineto{\pgfqpoint{1.316852in}{0.490730in}}%
\pgfpathlineto{\pgfqpoint{1.414706in}{0.490730in}}%
\pgfpathlineto{\pgfqpoint{1.512559in}{0.490730in}}%
\pgfpathlineto{\pgfqpoint{1.610413in}{0.490730in}}%
\pgfpathlineto{\pgfqpoint{1.708266in}{0.490730in}}%
\pgfpathlineto{\pgfqpoint{1.806120in}{0.490730in}}%
\pgfpathlineto{\pgfqpoint{1.903973in}{0.490730in}}%
\pgfpathlineto{\pgfqpoint{2.001827in}{0.507201in}}%
\pgfpathlineto{\pgfqpoint{2.099681in}{0.531907in}}%
\pgfpathlineto{\pgfqpoint{2.197534in}{0.630730in}}%
\pgfpathlineto{\pgfqpoint{2.295388in}{0.770730in}}%
\pgfpathlineto{\pgfqpoint{2.393241in}{1.207201in}}%
\pgfpathlineto{\pgfqpoint{2.491095in}{1.783672in}}%
\pgfpathlineto{\pgfqpoint{2.588948in}{2.104848in}}%
\pgfpathlineto{\pgfqpoint{2.686802in}{2.343672in}}%
\pgfpathlineto{\pgfqpoint{2.784655in}{2.409554in}}%
\pgfpathlineto{\pgfqpoint{2.882509in}{2.483672in}}%
\pgfpathlineto{\pgfqpoint{2.980362in}{2.491907in}}%
\pgfpathlineto{\pgfqpoint{3.078216in}{2.524848in}}%
\pgfpathlineto{\pgfqpoint{3.176069in}{2.549554in}}%
\pgfpathlineto{\pgfqpoint{3.273923in}{2.566025in}}%
\pgfpathlineto{\pgfqpoint{3.371776in}{2.574260in}}%
\pgfpathlineto{\pgfqpoint{3.469630in}{2.582495in}}%
\pgfpathlineto{\pgfqpoint{3.567484in}{2.582495in}}%
\pgfpathlineto{\pgfqpoint{3.665337in}{2.590730in}}%
\pgfpathlineto{\pgfqpoint{3.763191in}{2.590730in}}%
\pgfpathlineto{\pgfqpoint{3.861044in}{2.590730in}}%
\pgfpathlineto{\pgfqpoint{3.958898in}{2.590730in}}%
\pgfpathlineto{\pgfqpoint{4.056751in}{2.590730in}}%
\pgfpathlineto{\pgfqpoint{4.154605in}{2.590730in}}%
\pgfusepath{stroke}%
\end{pgfscope}%
\begin{pgfscope}%
\pgfpathrectangle{\pgfqpoint{0.455741in}{0.385730in}}{\pgfqpoint{3.875000in}{2.310000in}}%
\pgfusepath{clip}%
\pgfsetrectcap%
\pgfsetroundjoin%
\pgfsetlinewidth{0.803000pt}%
\definecolor{currentstroke}{rgb}{0.000000,0.356863,0.509804}%
\pgfsetstrokecolor{currentstroke}%
\pgfsetdash{}{0pt}%
\pgfpathmoveto{\pgfqpoint{0.631877in}{0.490730in}}%
\pgfpathlineto{\pgfqpoint{0.729731in}{0.490730in}}%
\pgfpathlineto{\pgfqpoint{0.827585in}{0.490730in}}%
\pgfpathlineto{\pgfqpoint{0.925438in}{0.490730in}}%
\pgfpathlineto{\pgfqpoint{1.023292in}{0.490730in}}%
\pgfpathlineto{\pgfqpoint{1.121145in}{0.490730in}}%
\pgfpathlineto{\pgfqpoint{1.218999in}{0.490730in}}%
\pgfpathlineto{\pgfqpoint{1.316852in}{0.490730in}}%
\pgfpathlineto{\pgfqpoint{1.414706in}{0.490730in}}%
\pgfpathlineto{\pgfqpoint{1.512559in}{0.490730in}}%
\pgfpathlineto{\pgfqpoint{1.610413in}{0.490730in}}%
\pgfpathlineto{\pgfqpoint{1.708266in}{0.490730in}}%
\pgfpathlineto{\pgfqpoint{1.806120in}{0.490730in}}%
\pgfpathlineto{\pgfqpoint{1.903973in}{0.490730in}}%
\pgfpathlineto{\pgfqpoint{2.001827in}{0.515436in}}%
\pgfpathlineto{\pgfqpoint{2.099681in}{0.531907in}}%
\pgfpathlineto{\pgfqpoint{2.197534in}{0.589554in}}%
\pgfpathlineto{\pgfqpoint{2.295388in}{0.729554in}}%
\pgfpathlineto{\pgfqpoint{2.393241in}{1.091907in}}%
\pgfpathlineto{\pgfqpoint{2.491095in}{1.742495in}}%
\pgfpathlineto{\pgfqpoint{2.588948in}{2.022495in}}%
\pgfpathlineto{\pgfqpoint{2.686802in}{2.310730in}}%
\pgfpathlineto{\pgfqpoint{2.784655in}{2.409554in}}%
\pgfpathlineto{\pgfqpoint{2.882509in}{2.475436in}}%
\pgfpathlineto{\pgfqpoint{2.980362in}{2.500142in}}%
\pgfpathlineto{\pgfqpoint{3.078216in}{2.533083in}}%
\pgfpathlineto{\pgfqpoint{3.176069in}{2.549554in}}%
\pgfpathlineto{\pgfqpoint{3.273923in}{2.557789in}}%
\pgfpathlineto{\pgfqpoint{3.371776in}{2.574260in}}%
\pgfpathlineto{\pgfqpoint{3.469630in}{2.582495in}}%
\pgfpathlineto{\pgfqpoint{3.567484in}{2.574260in}}%
\pgfpathlineto{\pgfqpoint{3.665337in}{2.590730in}}%
\pgfpathlineto{\pgfqpoint{3.763191in}{2.590730in}}%
\pgfpathlineto{\pgfqpoint{3.861044in}{2.590730in}}%
\pgfpathlineto{\pgfqpoint{3.958898in}{2.582495in}}%
\pgfpathlineto{\pgfqpoint{4.056751in}{2.590730in}}%
\pgfpathlineto{\pgfqpoint{4.154605in}{2.582495in}}%
\pgfusepath{stroke}%
\end{pgfscope}%
\begin{pgfscope}%
\pgfpathrectangle{\pgfqpoint{0.455741in}{0.385730in}}{\pgfqpoint{3.875000in}{2.310000in}}%
\pgfusepath{clip}%
\pgfsetrectcap%
\pgfsetroundjoin%
\pgfsetlinewidth{0.803000pt}%
\definecolor{currentstroke}{rgb}{0.490196,0.588235,0.431373}%
\pgfsetstrokecolor{currentstroke}%
\pgfsetdash{}{0pt}%
\pgfpathmoveto{\pgfqpoint{0.631877in}{0.490730in}}%
\pgfpathlineto{\pgfqpoint{0.729731in}{0.490730in}}%
\pgfpathlineto{\pgfqpoint{0.827585in}{0.490730in}}%
\pgfpathlineto{\pgfqpoint{0.925438in}{0.490730in}}%
\pgfpathlineto{\pgfqpoint{1.023292in}{0.490730in}}%
\pgfpathlineto{\pgfqpoint{1.121145in}{0.490730in}}%
\pgfpathlineto{\pgfqpoint{1.218999in}{0.490730in}}%
\pgfpathlineto{\pgfqpoint{1.316852in}{0.490730in}}%
\pgfpathlineto{\pgfqpoint{1.414706in}{0.490730in}}%
\pgfpathlineto{\pgfqpoint{1.512559in}{0.490730in}}%
\pgfpathlineto{\pgfqpoint{1.610413in}{0.490730in}}%
\pgfpathlineto{\pgfqpoint{1.708266in}{0.490730in}}%
\pgfpathlineto{\pgfqpoint{1.806120in}{0.490730in}}%
\pgfpathlineto{\pgfqpoint{1.903973in}{0.490730in}}%
\pgfpathlineto{\pgfqpoint{2.001827in}{0.507201in}}%
\pgfpathlineto{\pgfqpoint{2.099681in}{0.523672in}}%
\pgfpathlineto{\pgfqpoint{2.197534in}{0.630730in}}%
\pgfpathlineto{\pgfqpoint{2.295388in}{0.754260in}}%
\pgfpathlineto{\pgfqpoint{2.393241in}{1.034260in}}%
\pgfpathlineto{\pgfqpoint{2.491095in}{1.618966in}}%
\pgfpathlineto{\pgfqpoint{2.588948in}{2.022495in}}%
\pgfpathlineto{\pgfqpoint{2.686802in}{2.253083in}}%
\pgfpathlineto{\pgfqpoint{2.784655in}{2.409554in}}%
\pgfpathlineto{\pgfqpoint{2.882509in}{2.483672in}}%
\pgfpathlineto{\pgfqpoint{2.980362in}{2.508377in}}%
\pgfpathlineto{\pgfqpoint{3.078216in}{2.533083in}}%
\pgfpathlineto{\pgfqpoint{3.176069in}{2.557789in}}%
\pgfpathlineto{\pgfqpoint{3.273923in}{2.574260in}}%
\pgfpathlineto{\pgfqpoint{3.371776in}{2.582495in}}%
\pgfpathlineto{\pgfqpoint{3.469630in}{2.582495in}}%
\pgfpathlineto{\pgfqpoint{3.567484in}{2.590730in}}%
\pgfpathlineto{\pgfqpoint{3.665337in}{2.590730in}}%
\pgfpathlineto{\pgfqpoint{3.763191in}{2.590730in}}%
\pgfpathlineto{\pgfqpoint{3.861044in}{2.590730in}}%
\pgfpathlineto{\pgfqpoint{3.958898in}{2.590730in}}%
\pgfpathlineto{\pgfqpoint{4.056751in}{2.590730in}}%
\pgfpathlineto{\pgfqpoint{4.154605in}{2.590730in}}%
\pgfusepath{stroke}%
\end{pgfscope}%
\begin{pgfscope}%
\pgfpathrectangle{\pgfqpoint{0.455741in}{0.385730in}}{\pgfqpoint{3.875000in}{2.310000in}}%
\pgfusepath{clip}%
\pgfsetbuttcap%
\pgfsetroundjoin%
\pgfsetlinewidth{0.803000pt}%
\definecolor{currentstroke}{rgb}{0.843137,0.666667,0.313725}%
\pgfsetstrokecolor{currentstroke}%
\pgfsetdash{{0.800000pt}{1.320000pt}}{0.000000pt}%
\pgfpathmoveto{\pgfqpoint{0.631877in}{0.490730in}}%
\pgfpathlineto{\pgfqpoint{0.729731in}{0.490730in}}%
\pgfpathlineto{\pgfqpoint{0.827585in}{0.490730in}}%
\pgfpathlineto{\pgfqpoint{0.925438in}{0.490730in}}%
\pgfpathlineto{\pgfqpoint{1.023292in}{0.490730in}}%
\pgfpathlineto{\pgfqpoint{1.121145in}{0.490730in}}%
\pgfpathlineto{\pgfqpoint{1.218999in}{0.490730in}}%
\pgfpathlineto{\pgfqpoint{1.316852in}{0.490730in}}%
\pgfpathlineto{\pgfqpoint{1.414706in}{0.490730in}}%
\pgfpathlineto{\pgfqpoint{1.512559in}{0.490730in}}%
\pgfpathlineto{\pgfqpoint{1.610413in}{0.490730in}}%
\pgfpathlineto{\pgfqpoint{1.708266in}{0.490730in}}%
\pgfpathlineto{\pgfqpoint{1.806120in}{0.490730in}}%
\pgfpathlineto{\pgfqpoint{1.903973in}{0.490730in}}%
\pgfpathlineto{\pgfqpoint{2.001827in}{0.490730in}}%
\pgfpathlineto{\pgfqpoint{2.099681in}{0.490730in}}%
\pgfpathlineto{\pgfqpoint{2.197534in}{0.490730in}}%
\pgfpathlineto{\pgfqpoint{2.295388in}{0.490730in}}%
\pgfpathlineto{\pgfqpoint{2.393241in}{0.490730in}}%
\pgfpathlineto{\pgfqpoint{2.491095in}{0.490730in}}%
\pgfpathlineto{\pgfqpoint{2.588948in}{0.490730in}}%
\pgfpathlineto{\pgfqpoint{2.686802in}{0.490730in}}%
\pgfpathlineto{\pgfqpoint{2.784655in}{0.490730in}}%
\pgfpathlineto{\pgfqpoint{2.882509in}{0.490730in}}%
\pgfpathlineto{\pgfqpoint{2.980362in}{0.490730in}}%
\pgfpathlineto{\pgfqpoint{3.078216in}{0.490730in}}%
\pgfpathlineto{\pgfqpoint{3.176069in}{0.515436in}}%
\pgfpathlineto{\pgfqpoint{3.273923in}{0.622495in}}%
\pgfpathlineto{\pgfqpoint{3.371776in}{0.803672in}}%
\pgfpathlineto{\pgfqpoint{3.469630in}{1.141319in}}%
\pgfpathlineto{\pgfqpoint{3.567484in}{1.594260in}}%
\pgfpathlineto{\pgfqpoint{3.665337in}{1.915436in}}%
\pgfpathlineto{\pgfqpoint{3.763191in}{2.269554in}}%
\pgfpathlineto{\pgfqpoint{3.861044in}{2.434260in}}%
\pgfpathlineto{\pgfqpoint{3.958898in}{2.516613in}}%
\pgfpathlineto{\pgfqpoint{4.056751in}{2.541319in}}%
\pgfpathlineto{\pgfqpoint{4.154605in}{2.541319in}}%
\pgfusepath{stroke}%
\end{pgfscope}%
\begin{pgfscope}%
\pgfpathrectangle{\pgfqpoint{0.455741in}{0.385730in}}{\pgfqpoint{3.875000in}{2.310000in}}%
\pgfusepath{clip}%
\pgfsetbuttcap%
\pgfsetroundjoin%
\pgfsetlinewidth{0.803000pt}%
\definecolor{currentstroke}{rgb}{0.333333,0.333333,0.333333}%
\pgfsetstrokecolor{currentstroke}%
\pgfsetdash{{0.800000pt}{1.320000pt}}{0.000000pt}%
\pgfpathmoveto{\pgfqpoint{0.631877in}{0.490730in}}%
\pgfpathlineto{\pgfqpoint{0.729731in}{0.490730in}}%
\pgfpathlineto{\pgfqpoint{0.827585in}{0.490730in}}%
\pgfpathlineto{\pgfqpoint{0.925438in}{0.490730in}}%
\pgfpathlineto{\pgfqpoint{1.023292in}{0.490730in}}%
\pgfpathlineto{\pgfqpoint{1.121145in}{0.490730in}}%
\pgfpathlineto{\pgfqpoint{1.218999in}{0.490730in}}%
\pgfpathlineto{\pgfqpoint{1.316852in}{0.490730in}}%
\pgfpathlineto{\pgfqpoint{1.414706in}{0.490730in}}%
\pgfpathlineto{\pgfqpoint{1.512559in}{0.490730in}}%
\pgfpathlineto{\pgfqpoint{1.610413in}{0.490730in}}%
\pgfpathlineto{\pgfqpoint{1.708266in}{0.490730in}}%
\pgfpathlineto{\pgfqpoint{1.806120in}{0.490730in}}%
\pgfpathlineto{\pgfqpoint{1.903973in}{0.490730in}}%
\pgfpathlineto{\pgfqpoint{2.001827in}{0.490730in}}%
\pgfpathlineto{\pgfqpoint{2.099681in}{0.490730in}}%
\pgfpathlineto{\pgfqpoint{2.197534in}{0.490730in}}%
\pgfpathlineto{\pgfqpoint{2.295388in}{0.490730in}}%
\pgfpathlineto{\pgfqpoint{2.393241in}{0.490730in}}%
\pgfpathlineto{\pgfqpoint{2.491095in}{0.490730in}}%
\pgfpathlineto{\pgfqpoint{2.588948in}{0.490730in}}%
\pgfpathlineto{\pgfqpoint{2.686802in}{0.490730in}}%
\pgfpathlineto{\pgfqpoint{2.784655in}{0.490730in}}%
\pgfpathlineto{\pgfqpoint{2.882509in}{0.498966in}}%
\pgfpathlineto{\pgfqpoint{2.980362in}{0.548377in}}%
\pgfpathlineto{\pgfqpoint{3.078216in}{0.655436in}}%
\pgfpathlineto{\pgfqpoint{3.176069in}{0.861319in}}%
\pgfpathlineto{\pgfqpoint{3.273923in}{1.174260in}}%
\pgfpathlineto{\pgfqpoint{3.371776in}{1.717789in}}%
\pgfpathlineto{\pgfqpoint{3.469630in}{2.129554in}}%
\pgfpathlineto{\pgfqpoint{3.567484in}{2.351907in}}%
\pgfpathlineto{\pgfqpoint{3.665337in}{2.458966in}}%
\pgfpathlineto{\pgfqpoint{3.763191in}{2.500142in}}%
\pgfpathlineto{\pgfqpoint{3.861044in}{2.533083in}}%
\pgfpathlineto{\pgfqpoint{3.958898in}{2.541319in}}%
\pgfpathlineto{\pgfqpoint{4.056751in}{2.541319in}}%
\pgfpathlineto{\pgfqpoint{4.154605in}{2.549554in}}%
\pgfusepath{stroke}%
\end{pgfscope}%
\begin{pgfscope}%
\pgfpathrectangle{\pgfqpoint{0.455741in}{0.385730in}}{\pgfqpoint{3.875000in}{2.310000in}}%
\pgfusepath{clip}%
\pgfsetbuttcap%
\pgfsetroundjoin%
\pgfsetlinewidth{0.803000pt}%
\definecolor{currentstroke}{rgb}{0.686275,0.352941,0.313725}%
\pgfsetstrokecolor{currentstroke}%
\pgfsetdash{{0.800000pt}{1.320000pt}}{0.000000pt}%
\pgfpathmoveto{\pgfqpoint{0.631877in}{0.490730in}}%
\pgfpathlineto{\pgfqpoint{0.729731in}{0.490730in}}%
\pgfpathlineto{\pgfqpoint{0.827585in}{0.490730in}}%
\pgfpathlineto{\pgfqpoint{0.925438in}{0.490730in}}%
\pgfpathlineto{\pgfqpoint{1.023292in}{0.490730in}}%
\pgfpathlineto{\pgfqpoint{1.121145in}{0.490730in}}%
\pgfpathlineto{\pgfqpoint{1.218999in}{0.490730in}}%
\pgfpathlineto{\pgfqpoint{1.316852in}{0.490730in}}%
\pgfpathlineto{\pgfqpoint{1.414706in}{0.490730in}}%
\pgfpathlineto{\pgfqpoint{1.512559in}{0.490730in}}%
\pgfpathlineto{\pgfqpoint{1.610413in}{0.490730in}}%
\pgfpathlineto{\pgfqpoint{1.708266in}{0.490730in}}%
\pgfpathlineto{\pgfqpoint{1.806120in}{0.490730in}}%
\pgfpathlineto{\pgfqpoint{1.903973in}{0.490730in}}%
\pgfpathlineto{\pgfqpoint{2.001827in}{0.490730in}}%
\pgfpathlineto{\pgfqpoint{2.099681in}{0.490730in}}%
\pgfpathlineto{\pgfqpoint{2.197534in}{0.490730in}}%
\pgfpathlineto{\pgfqpoint{2.295388in}{0.490730in}}%
\pgfpathlineto{\pgfqpoint{2.393241in}{0.490730in}}%
\pgfpathlineto{\pgfqpoint{2.491095in}{0.490730in}}%
\pgfpathlineto{\pgfqpoint{2.588948in}{0.498966in}}%
\pgfpathlineto{\pgfqpoint{2.686802in}{0.548377in}}%
\pgfpathlineto{\pgfqpoint{2.784655in}{0.680142in}}%
\pgfpathlineto{\pgfqpoint{2.882509in}{0.836613in}}%
\pgfpathlineto{\pgfqpoint{2.980362in}{1.133083in}}%
\pgfpathlineto{\pgfqpoint{3.078216in}{1.487201in}}%
\pgfpathlineto{\pgfqpoint{3.176069in}{1.915436in}}%
\pgfpathlineto{\pgfqpoint{3.273923in}{2.104848in}}%
\pgfpathlineto{\pgfqpoint{3.371776in}{2.360142in}}%
\pgfpathlineto{\pgfqpoint{3.469630in}{2.475436in}}%
\pgfpathlineto{\pgfqpoint{3.567484in}{2.524848in}}%
\pgfpathlineto{\pgfqpoint{3.665337in}{2.549554in}}%
\pgfpathlineto{\pgfqpoint{3.763191in}{2.566025in}}%
\pgfpathlineto{\pgfqpoint{3.861044in}{2.574260in}}%
\pgfpathlineto{\pgfqpoint{3.958898in}{2.574260in}}%
\pgfpathlineto{\pgfqpoint{4.056751in}{2.574260in}}%
\pgfpathlineto{\pgfqpoint{4.154605in}{2.574260in}}%
\pgfusepath{stroke}%
\end{pgfscope}%
\begin{pgfscope}%
\pgfpathrectangle{\pgfqpoint{0.455741in}{0.385730in}}{\pgfqpoint{3.875000in}{2.310000in}}%
\pgfusepath{clip}%
\pgfsetbuttcap%
\pgfsetroundjoin%
\pgfsetlinewidth{0.803000pt}%
\definecolor{currentstroke}{rgb}{0.000000,0.356863,0.509804}%
\pgfsetstrokecolor{currentstroke}%
\pgfsetdash{{0.800000pt}{1.320000pt}}{0.000000pt}%
\pgfpathmoveto{\pgfqpoint{0.631877in}{0.490730in}}%
\pgfpathlineto{\pgfqpoint{0.729731in}{0.490730in}}%
\pgfpathlineto{\pgfqpoint{0.827585in}{0.490730in}}%
\pgfpathlineto{\pgfqpoint{0.925438in}{0.490730in}}%
\pgfpathlineto{\pgfqpoint{1.023292in}{0.490730in}}%
\pgfpathlineto{\pgfqpoint{1.121145in}{0.490730in}}%
\pgfpathlineto{\pgfqpoint{1.218999in}{0.490730in}}%
\pgfpathlineto{\pgfqpoint{1.316852in}{0.490730in}}%
\pgfpathlineto{\pgfqpoint{1.414706in}{0.490730in}}%
\pgfpathlineto{\pgfqpoint{1.512559in}{0.490730in}}%
\pgfpathlineto{\pgfqpoint{1.610413in}{0.490730in}}%
\pgfpathlineto{\pgfqpoint{1.708266in}{0.490730in}}%
\pgfpathlineto{\pgfqpoint{1.806120in}{0.490730in}}%
\pgfpathlineto{\pgfqpoint{1.903973in}{0.490730in}}%
\pgfpathlineto{\pgfqpoint{2.001827in}{0.490730in}}%
\pgfpathlineto{\pgfqpoint{2.099681in}{0.490730in}}%
\pgfpathlineto{\pgfqpoint{2.197534in}{0.490730in}}%
\pgfpathlineto{\pgfqpoint{2.295388in}{0.490730in}}%
\pgfpathlineto{\pgfqpoint{2.393241in}{0.490730in}}%
\pgfpathlineto{\pgfqpoint{2.491095in}{0.490730in}}%
\pgfpathlineto{\pgfqpoint{2.588948in}{0.490730in}}%
\pgfpathlineto{\pgfqpoint{2.686802in}{0.490730in}}%
\pgfpathlineto{\pgfqpoint{2.784655in}{0.490730in}}%
\pgfpathlineto{\pgfqpoint{2.882509in}{0.490730in}}%
\pgfpathlineto{\pgfqpoint{2.980362in}{0.490730in}}%
\pgfpathlineto{\pgfqpoint{3.078216in}{0.490730in}}%
\pgfpathlineto{\pgfqpoint{3.176069in}{0.498966in}}%
\pgfpathlineto{\pgfqpoint{3.273923in}{0.589554in}}%
\pgfpathlineto{\pgfqpoint{3.371776in}{0.729554in}}%
\pgfpathlineto{\pgfqpoint{3.469630in}{0.976613in}}%
\pgfpathlineto{\pgfqpoint{3.567484in}{1.437789in}}%
\pgfpathlineto{\pgfqpoint{3.665337in}{1.824848in}}%
\pgfpathlineto{\pgfqpoint{3.763191in}{2.104848in}}%
\pgfpathlineto{\pgfqpoint{3.861044in}{2.351907in}}%
\pgfpathlineto{\pgfqpoint{3.958898in}{2.442495in}}%
\pgfpathlineto{\pgfqpoint{4.056751in}{2.467201in}}%
\pgfpathlineto{\pgfqpoint{4.154605in}{2.475436in}}%
\pgfusepath{stroke}%
\end{pgfscope}%
\begin{pgfscope}%
\pgfpathrectangle{\pgfqpoint{0.455741in}{0.385730in}}{\pgfqpoint{3.875000in}{2.310000in}}%
\pgfusepath{clip}%
\pgfsetbuttcap%
\pgfsetroundjoin%
\pgfsetlinewidth{0.803000pt}%
\definecolor{currentstroke}{rgb}{0.490196,0.588235,0.431373}%
\pgfsetstrokecolor{currentstroke}%
\pgfsetdash{{0.800000pt}{1.320000pt}}{0.000000pt}%
\pgfpathmoveto{\pgfqpoint{0.631877in}{0.490730in}}%
\pgfpathlineto{\pgfqpoint{0.729731in}{0.490730in}}%
\pgfpathlineto{\pgfqpoint{0.827585in}{0.490730in}}%
\pgfpathlineto{\pgfqpoint{0.925438in}{0.490730in}}%
\pgfpathlineto{\pgfqpoint{1.023292in}{0.490730in}}%
\pgfpathlineto{\pgfqpoint{1.121145in}{0.490730in}}%
\pgfpathlineto{\pgfqpoint{1.218999in}{0.490730in}}%
\pgfpathlineto{\pgfqpoint{1.316852in}{0.490730in}}%
\pgfpathlineto{\pgfqpoint{1.414706in}{0.490730in}}%
\pgfpathlineto{\pgfqpoint{1.512559in}{0.490730in}}%
\pgfpathlineto{\pgfqpoint{1.610413in}{0.490730in}}%
\pgfpathlineto{\pgfqpoint{1.708266in}{0.490730in}}%
\pgfpathlineto{\pgfqpoint{1.806120in}{0.490730in}}%
\pgfpathlineto{\pgfqpoint{1.903973in}{0.490730in}}%
\pgfpathlineto{\pgfqpoint{2.001827in}{0.490730in}}%
\pgfpathlineto{\pgfqpoint{2.099681in}{0.490730in}}%
\pgfpathlineto{\pgfqpoint{2.197534in}{0.490730in}}%
\pgfpathlineto{\pgfqpoint{2.295388in}{0.490730in}}%
\pgfpathlineto{\pgfqpoint{2.393241in}{0.490730in}}%
\pgfpathlineto{\pgfqpoint{2.491095in}{0.490730in}}%
\pgfpathlineto{\pgfqpoint{2.588948in}{0.490730in}}%
\pgfpathlineto{\pgfqpoint{2.686802in}{0.490730in}}%
\pgfpathlineto{\pgfqpoint{2.784655in}{0.490730in}}%
\pgfpathlineto{\pgfqpoint{2.882509in}{0.490730in}}%
\pgfpathlineto{\pgfqpoint{2.980362in}{0.490730in}}%
\pgfpathlineto{\pgfqpoint{3.078216in}{0.498966in}}%
\pgfpathlineto{\pgfqpoint{3.176069in}{0.540142in}}%
\pgfpathlineto{\pgfqpoint{3.273923in}{0.647201in}}%
\pgfpathlineto{\pgfqpoint{3.371776in}{0.828377in}}%
\pgfpathlineto{\pgfqpoint{3.469630in}{1.223672in}}%
\pgfpathlineto{\pgfqpoint{3.567484in}{1.594260in}}%
\pgfpathlineto{\pgfqpoint{3.665337in}{2.022495in}}%
\pgfpathlineto{\pgfqpoint{3.763191in}{2.277789in}}%
\pgfpathlineto{\pgfqpoint{3.861044in}{2.434260in}}%
\pgfpathlineto{\pgfqpoint{3.958898in}{2.500142in}}%
\pgfpathlineto{\pgfqpoint{4.056751in}{2.516613in}}%
\pgfpathlineto{\pgfqpoint{4.154605in}{2.524848in}}%
\pgfusepath{stroke}%
\end{pgfscope}%
\begin{pgfscope}%
\pgfpathrectangle{\pgfqpoint{0.455741in}{0.385730in}}{\pgfqpoint{3.875000in}{2.310000in}}%
\pgfusepath{clip}%
\pgfsetbuttcap%
\pgfsetroundjoin%
\pgfsetlinewidth{0.803000pt}%
\definecolor{currentstroke}{rgb}{0.843137,0.666667,0.313725}%
\pgfsetstrokecolor{currentstroke}%
\pgfsetdash{{0.800000pt}{1.320000pt}}{0.000000pt}%
\pgfpathmoveto{\pgfqpoint{0.631877in}{0.490730in}}%
\pgfpathlineto{\pgfqpoint{0.729731in}{0.490730in}}%
\pgfpathlineto{\pgfqpoint{0.827585in}{0.490730in}}%
\pgfpathlineto{\pgfqpoint{0.925438in}{0.490730in}}%
\pgfpathlineto{\pgfqpoint{1.023292in}{0.490730in}}%
\pgfpathlineto{\pgfqpoint{1.121145in}{0.490730in}}%
\pgfpathlineto{\pgfqpoint{1.218999in}{0.490730in}}%
\pgfpathlineto{\pgfqpoint{1.316852in}{0.490730in}}%
\pgfpathlineto{\pgfqpoint{1.414706in}{0.490730in}}%
\pgfpathlineto{\pgfqpoint{1.512559in}{0.490730in}}%
\pgfpathlineto{\pgfqpoint{1.610413in}{0.490730in}}%
\pgfpathlineto{\pgfqpoint{1.708266in}{0.490730in}}%
\pgfpathlineto{\pgfqpoint{1.806120in}{0.490730in}}%
\pgfpathlineto{\pgfqpoint{1.903973in}{0.490730in}}%
\pgfpathlineto{\pgfqpoint{2.001827in}{0.490730in}}%
\pgfpathlineto{\pgfqpoint{2.099681in}{0.490730in}}%
\pgfpathlineto{\pgfqpoint{2.197534in}{0.490730in}}%
\pgfpathlineto{\pgfqpoint{2.295388in}{0.490730in}}%
\pgfpathlineto{\pgfqpoint{2.393241in}{0.490730in}}%
\pgfpathlineto{\pgfqpoint{2.491095in}{0.490730in}}%
\pgfpathlineto{\pgfqpoint{2.588948in}{0.490730in}}%
\pgfpathlineto{\pgfqpoint{2.686802in}{0.490730in}}%
\pgfpathlineto{\pgfqpoint{2.784655in}{0.507201in}}%
\pgfpathlineto{\pgfqpoint{2.882509in}{0.540142in}}%
\pgfpathlineto{\pgfqpoint{2.980362in}{0.630730in}}%
\pgfpathlineto{\pgfqpoint{3.078216in}{0.811907in}}%
\pgfpathlineto{\pgfqpoint{3.176069in}{1.083672in}}%
\pgfpathlineto{\pgfqpoint{3.273923in}{1.528377in}}%
\pgfpathlineto{\pgfqpoint{3.371776in}{1.898966in}}%
\pgfpathlineto{\pgfqpoint{3.469630in}{2.195436in}}%
\pgfpathlineto{\pgfqpoint{3.567484in}{2.384848in}}%
\pgfpathlineto{\pgfqpoint{3.665337in}{2.467201in}}%
\pgfpathlineto{\pgfqpoint{3.763191in}{2.508377in}}%
\pgfpathlineto{\pgfqpoint{3.861044in}{2.516613in}}%
\pgfpathlineto{\pgfqpoint{3.958898in}{2.524848in}}%
\pgfpathlineto{\pgfqpoint{4.056751in}{2.533083in}}%
\pgfpathlineto{\pgfqpoint{4.154605in}{2.533083in}}%
\pgfusepath{stroke}%
\end{pgfscope}%
\begin{pgfscope}%
\pgfpathrectangle{\pgfqpoint{0.455741in}{0.385730in}}{\pgfqpoint{3.875000in}{2.310000in}}%
\pgfusepath{clip}%
\pgfsetbuttcap%
\pgfsetroundjoin%
\pgfsetlinewidth{0.803000pt}%
\definecolor{currentstroke}{rgb}{0.333333,0.333333,0.333333}%
\pgfsetstrokecolor{currentstroke}%
\pgfsetdash{{0.800000pt}{1.320000pt}}{0.000000pt}%
\pgfpathmoveto{\pgfqpoint{0.631877in}{0.490730in}}%
\pgfpathlineto{\pgfqpoint{0.729731in}{0.490730in}}%
\pgfpathlineto{\pgfqpoint{0.827585in}{0.490730in}}%
\pgfpathlineto{\pgfqpoint{0.925438in}{0.490730in}}%
\pgfpathlineto{\pgfqpoint{1.023292in}{0.490730in}}%
\pgfpathlineto{\pgfqpoint{1.121145in}{0.490730in}}%
\pgfpathlineto{\pgfqpoint{1.218999in}{0.490730in}}%
\pgfpathlineto{\pgfqpoint{1.316852in}{0.490730in}}%
\pgfpathlineto{\pgfqpoint{1.414706in}{0.490730in}}%
\pgfpathlineto{\pgfqpoint{1.512559in}{0.490730in}}%
\pgfpathlineto{\pgfqpoint{1.610413in}{0.490730in}}%
\pgfpathlineto{\pgfqpoint{1.708266in}{0.490730in}}%
\pgfpathlineto{\pgfqpoint{1.806120in}{0.490730in}}%
\pgfpathlineto{\pgfqpoint{1.903973in}{0.490730in}}%
\pgfpathlineto{\pgfqpoint{2.001827in}{0.490730in}}%
\pgfpathlineto{\pgfqpoint{2.099681in}{0.490730in}}%
\pgfpathlineto{\pgfqpoint{2.197534in}{0.490730in}}%
\pgfpathlineto{\pgfqpoint{2.295388in}{0.490730in}}%
\pgfpathlineto{\pgfqpoint{2.393241in}{0.490730in}}%
\pgfpathlineto{\pgfqpoint{2.491095in}{0.490730in}}%
\pgfpathlineto{\pgfqpoint{2.588948in}{0.490730in}}%
\pgfpathlineto{\pgfqpoint{2.686802in}{0.490730in}}%
\pgfpathlineto{\pgfqpoint{2.784655in}{0.490730in}}%
\pgfpathlineto{\pgfqpoint{2.882509in}{0.498966in}}%
\pgfpathlineto{\pgfqpoint{2.980362in}{0.531907in}}%
\pgfpathlineto{\pgfqpoint{3.078216in}{0.622495in}}%
\pgfpathlineto{\pgfqpoint{3.176069in}{0.787201in}}%
\pgfpathlineto{\pgfqpoint{3.273923in}{1.124848in}}%
\pgfpathlineto{\pgfqpoint{3.371776in}{1.478966in}}%
\pgfpathlineto{\pgfqpoint{3.469630in}{1.874260in}}%
\pgfpathlineto{\pgfqpoint{3.567484in}{2.253083in}}%
\pgfpathlineto{\pgfqpoint{3.665337in}{2.401319in}}%
\pgfpathlineto{\pgfqpoint{3.763191in}{2.450730in}}%
\pgfpathlineto{\pgfqpoint{3.861044in}{2.491907in}}%
\pgfpathlineto{\pgfqpoint{3.958898in}{2.500142in}}%
\pgfpathlineto{\pgfqpoint{4.056751in}{2.508377in}}%
\pgfpathlineto{\pgfqpoint{4.154605in}{2.516613in}}%
\pgfusepath{stroke}%
\end{pgfscope}%
\begin{pgfscope}%
\pgfpathrectangle{\pgfqpoint{0.455741in}{0.385730in}}{\pgfqpoint{3.875000in}{2.310000in}}%
\pgfusepath{clip}%
\pgfsetbuttcap%
\pgfsetroundjoin%
\pgfsetlinewidth{0.803000pt}%
\definecolor{currentstroke}{rgb}{0.686275,0.352941,0.313725}%
\pgfsetstrokecolor{currentstroke}%
\pgfsetdash{{0.800000pt}{1.320000pt}}{0.000000pt}%
\pgfpathmoveto{\pgfqpoint{0.631877in}{0.490730in}}%
\pgfpathlineto{\pgfqpoint{0.729731in}{0.490730in}}%
\pgfpathlineto{\pgfqpoint{0.827585in}{0.490730in}}%
\pgfpathlineto{\pgfqpoint{0.925438in}{0.490730in}}%
\pgfpathlineto{\pgfqpoint{1.023292in}{0.490730in}}%
\pgfpathlineto{\pgfqpoint{1.121145in}{0.490730in}}%
\pgfpathlineto{\pgfqpoint{1.218999in}{0.490730in}}%
\pgfpathlineto{\pgfqpoint{1.316852in}{0.490730in}}%
\pgfpathlineto{\pgfqpoint{1.414706in}{0.490730in}}%
\pgfpathlineto{\pgfqpoint{1.512559in}{0.490730in}}%
\pgfpathlineto{\pgfqpoint{1.610413in}{0.490730in}}%
\pgfpathlineto{\pgfqpoint{1.708266in}{0.490730in}}%
\pgfpathlineto{\pgfqpoint{1.806120in}{0.490730in}}%
\pgfpathlineto{\pgfqpoint{1.903973in}{0.490730in}}%
\pgfpathlineto{\pgfqpoint{2.001827in}{0.490730in}}%
\pgfpathlineto{\pgfqpoint{2.099681in}{0.490730in}}%
\pgfpathlineto{\pgfqpoint{2.197534in}{0.490730in}}%
\pgfpathlineto{\pgfqpoint{2.295388in}{0.490730in}}%
\pgfpathlineto{\pgfqpoint{2.393241in}{0.490730in}}%
\pgfpathlineto{\pgfqpoint{2.491095in}{0.490730in}}%
\pgfpathlineto{\pgfqpoint{2.588948in}{0.490730in}}%
\pgfpathlineto{\pgfqpoint{2.686802in}{0.490730in}}%
\pgfpathlineto{\pgfqpoint{2.784655in}{0.498966in}}%
\pgfpathlineto{\pgfqpoint{2.882509in}{0.531907in}}%
\pgfpathlineto{\pgfqpoint{2.980362in}{0.622495in}}%
\pgfpathlineto{\pgfqpoint{3.078216in}{0.746025in}}%
\pgfpathlineto{\pgfqpoint{3.176069in}{1.001319in}}%
\pgfpathlineto{\pgfqpoint{3.273923in}{1.396613in}}%
\pgfpathlineto{\pgfqpoint{3.371776in}{1.849554in}}%
\pgfpathlineto{\pgfqpoint{3.469630in}{2.187201in}}%
\pgfpathlineto{\pgfqpoint{3.567484in}{2.376613in}}%
\pgfpathlineto{\pgfqpoint{3.665337in}{2.483672in}}%
\pgfpathlineto{\pgfqpoint{3.763191in}{2.524848in}}%
\pgfpathlineto{\pgfqpoint{3.861044in}{2.549554in}}%
\pgfpathlineto{\pgfqpoint{3.958898in}{2.557789in}}%
\pgfpathlineto{\pgfqpoint{4.056751in}{2.557789in}}%
\pgfpathlineto{\pgfqpoint{4.154605in}{2.557789in}}%
\pgfusepath{stroke}%
\end{pgfscope}%
\begin{pgfscope}%
\pgfpathrectangle{\pgfqpoint{0.455741in}{0.385730in}}{\pgfqpoint{3.875000in}{2.310000in}}%
\pgfusepath{clip}%
\pgfsetbuttcap%
\pgfsetroundjoin%
\pgfsetlinewidth{0.803000pt}%
\definecolor{currentstroke}{rgb}{0.000000,0.356863,0.509804}%
\pgfsetstrokecolor{currentstroke}%
\pgfsetdash{{0.800000pt}{1.320000pt}}{0.000000pt}%
\pgfpathmoveto{\pgfqpoint{0.631877in}{0.490730in}}%
\pgfpathlineto{\pgfqpoint{0.729731in}{0.490730in}}%
\pgfpathlineto{\pgfqpoint{0.827585in}{0.490730in}}%
\pgfpathlineto{\pgfqpoint{0.925438in}{0.490730in}}%
\pgfpathlineto{\pgfqpoint{1.023292in}{0.490730in}}%
\pgfpathlineto{\pgfqpoint{1.121145in}{0.490730in}}%
\pgfpathlineto{\pgfqpoint{1.218999in}{0.490730in}}%
\pgfpathlineto{\pgfqpoint{1.316852in}{0.490730in}}%
\pgfpathlineto{\pgfqpoint{1.414706in}{0.490730in}}%
\pgfpathlineto{\pgfqpoint{1.512559in}{0.490730in}}%
\pgfpathlineto{\pgfqpoint{1.610413in}{0.490730in}}%
\pgfpathlineto{\pgfqpoint{1.708266in}{0.490730in}}%
\pgfpathlineto{\pgfqpoint{1.806120in}{0.490730in}}%
\pgfpathlineto{\pgfqpoint{1.903973in}{0.490730in}}%
\pgfpathlineto{\pgfqpoint{2.001827in}{0.490730in}}%
\pgfpathlineto{\pgfqpoint{2.099681in}{0.490730in}}%
\pgfpathlineto{\pgfqpoint{2.197534in}{0.490730in}}%
\pgfpathlineto{\pgfqpoint{2.295388in}{0.490730in}}%
\pgfpathlineto{\pgfqpoint{2.393241in}{0.490730in}}%
\pgfpathlineto{\pgfqpoint{2.491095in}{0.490730in}}%
\pgfpathlineto{\pgfqpoint{2.588948in}{0.490730in}}%
\pgfpathlineto{\pgfqpoint{2.686802in}{0.490730in}}%
\pgfpathlineto{\pgfqpoint{2.784655in}{0.490730in}}%
\pgfpathlineto{\pgfqpoint{2.882509in}{0.490730in}}%
\pgfpathlineto{\pgfqpoint{2.980362in}{0.490730in}}%
\pgfpathlineto{\pgfqpoint{3.078216in}{0.498966in}}%
\pgfpathlineto{\pgfqpoint{3.176069in}{0.548377in}}%
\pgfpathlineto{\pgfqpoint{3.273923in}{0.721319in}}%
\pgfpathlineto{\pgfqpoint{3.371776in}{1.001319in}}%
\pgfpathlineto{\pgfqpoint{3.469630in}{1.314260in}}%
\pgfpathlineto{\pgfqpoint{3.567484in}{1.750730in}}%
\pgfpathlineto{\pgfqpoint{3.665337in}{2.129554in}}%
\pgfpathlineto{\pgfqpoint{3.763191in}{2.302495in}}%
\pgfpathlineto{\pgfqpoint{3.861044in}{2.409554in}}%
\pgfpathlineto{\pgfqpoint{3.958898in}{2.434260in}}%
\pgfpathlineto{\pgfqpoint{4.056751in}{2.434260in}}%
\pgfpathlineto{\pgfqpoint{4.154605in}{2.442495in}}%
\pgfusepath{stroke}%
\end{pgfscope}%
\begin{pgfscope}%
\pgfpathrectangle{\pgfqpoint{0.455741in}{0.385730in}}{\pgfqpoint{3.875000in}{2.310000in}}%
\pgfusepath{clip}%
\pgfsetbuttcap%
\pgfsetroundjoin%
\pgfsetlinewidth{0.803000pt}%
\definecolor{currentstroke}{rgb}{0.490196,0.588235,0.431373}%
\pgfsetstrokecolor{currentstroke}%
\pgfsetdash{{0.800000pt}{1.320000pt}}{0.000000pt}%
\pgfpathmoveto{\pgfqpoint{0.631877in}{0.490730in}}%
\pgfpathlineto{\pgfqpoint{0.729731in}{0.490730in}}%
\pgfpathlineto{\pgfqpoint{0.827585in}{0.490730in}}%
\pgfpathlineto{\pgfqpoint{0.925438in}{0.490730in}}%
\pgfpathlineto{\pgfqpoint{1.023292in}{0.490730in}}%
\pgfpathlineto{\pgfqpoint{1.121145in}{0.490730in}}%
\pgfpathlineto{\pgfqpoint{1.218999in}{0.490730in}}%
\pgfpathlineto{\pgfqpoint{1.316852in}{0.490730in}}%
\pgfpathlineto{\pgfqpoint{1.414706in}{0.490730in}}%
\pgfpathlineto{\pgfqpoint{1.512559in}{0.490730in}}%
\pgfpathlineto{\pgfqpoint{1.610413in}{0.490730in}}%
\pgfpathlineto{\pgfqpoint{1.708266in}{0.490730in}}%
\pgfpathlineto{\pgfqpoint{1.806120in}{0.490730in}}%
\pgfpathlineto{\pgfqpoint{1.903973in}{0.490730in}}%
\pgfpathlineto{\pgfqpoint{2.001827in}{0.490730in}}%
\pgfpathlineto{\pgfqpoint{2.099681in}{0.490730in}}%
\pgfpathlineto{\pgfqpoint{2.197534in}{0.490730in}}%
\pgfpathlineto{\pgfqpoint{2.295388in}{0.490730in}}%
\pgfpathlineto{\pgfqpoint{2.393241in}{0.490730in}}%
\pgfpathlineto{\pgfqpoint{2.491095in}{0.490730in}}%
\pgfpathlineto{\pgfqpoint{2.588948in}{0.490730in}}%
\pgfpathlineto{\pgfqpoint{2.686802in}{0.490730in}}%
\pgfpathlineto{\pgfqpoint{2.784655in}{0.490730in}}%
\pgfpathlineto{\pgfqpoint{2.882509in}{0.490730in}}%
\pgfpathlineto{\pgfqpoint{2.980362in}{0.507201in}}%
\pgfpathlineto{\pgfqpoint{3.078216in}{0.564848in}}%
\pgfpathlineto{\pgfqpoint{3.176069in}{0.688377in}}%
\pgfpathlineto{\pgfqpoint{3.273923in}{0.960142in}}%
\pgfpathlineto{\pgfqpoint{3.371776in}{1.413083in}}%
\pgfpathlineto{\pgfqpoint{3.469630in}{1.849554in}}%
\pgfpathlineto{\pgfqpoint{3.567484in}{2.154260in}}%
\pgfpathlineto{\pgfqpoint{3.665337in}{2.368377in}}%
\pgfpathlineto{\pgfqpoint{3.763191in}{2.434260in}}%
\pgfpathlineto{\pgfqpoint{3.861044in}{2.475436in}}%
\pgfpathlineto{\pgfqpoint{3.958898in}{2.491907in}}%
\pgfpathlineto{\pgfqpoint{4.056751in}{2.491907in}}%
\pgfpathlineto{\pgfqpoint{4.154605in}{2.500142in}}%
\pgfusepath{stroke}%
\end{pgfscope}%
\begin{pgfscope}%
\pgfpathrectangle{\pgfqpoint{0.455741in}{0.385730in}}{\pgfqpoint{3.875000in}{2.310000in}}%
\pgfusepath{clip}%
\pgfsetbuttcap%
\pgfsetroundjoin%
\pgfsetlinewidth{0.803000pt}%
\definecolor{currentstroke}{rgb}{0.843137,0.666667,0.313725}%
\pgfsetstrokecolor{currentstroke}%
\pgfsetdash{{0.800000pt}{1.320000pt}}{0.000000pt}%
\pgfpathmoveto{\pgfqpoint{0.631877in}{0.490730in}}%
\pgfpathlineto{\pgfqpoint{0.729731in}{0.490730in}}%
\pgfpathlineto{\pgfqpoint{0.827585in}{0.490730in}}%
\pgfpathlineto{\pgfqpoint{0.925438in}{0.490730in}}%
\pgfpathlineto{\pgfqpoint{1.023292in}{0.490730in}}%
\pgfpathlineto{\pgfqpoint{1.121145in}{0.490730in}}%
\pgfpathlineto{\pgfqpoint{1.218999in}{0.490730in}}%
\pgfpathlineto{\pgfqpoint{1.316852in}{0.490730in}}%
\pgfpathlineto{\pgfqpoint{1.414706in}{0.490730in}}%
\pgfpathlineto{\pgfqpoint{1.512559in}{0.490730in}}%
\pgfpathlineto{\pgfqpoint{1.610413in}{0.490730in}}%
\pgfpathlineto{\pgfqpoint{1.708266in}{0.490730in}}%
\pgfpathlineto{\pgfqpoint{1.806120in}{0.490730in}}%
\pgfpathlineto{\pgfqpoint{1.903973in}{0.490730in}}%
\pgfpathlineto{\pgfqpoint{2.001827in}{0.490730in}}%
\pgfpathlineto{\pgfqpoint{2.099681in}{0.490730in}}%
\pgfpathlineto{\pgfqpoint{2.197534in}{0.490730in}}%
\pgfpathlineto{\pgfqpoint{2.295388in}{0.490730in}}%
\pgfpathlineto{\pgfqpoint{2.393241in}{0.490730in}}%
\pgfpathlineto{\pgfqpoint{2.491095in}{0.490730in}}%
\pgfpathlineto{\pgfqpoint{2.588948in}{0.490730in}}%
\pgfpathlineto{\pgfqpoint{2.686802in}{0.490730in}}%
\pgfpathlineto{\pgfqpoint{2.784655in}{0.490730in}}%
\pgfpathlineto{\pgfqpoint{2.882509in}{0.490730in}}%
\pgfpathlineto{\pgfqpoint{2.980362in}{0.490730in}}%
\pgfpathlineto{\pgfqpoint{3.078216in}{0.507201in}}%
\pgfpathlineto{\pgfqpoint{3.176069in}{0.556613in}}%
\pgfpathlineto{\pgfqpoint{3.273923in}{0.688377in}}%
\pgfpathlineto{\pgfqpoint{3.371776in}{1.009554in}}%
\pgfpathlineto{\pgfqpoint{3.469630in}{1.437789in}}%
\pgfpathlineto{\pgfqpoint{3.567484in}{1.808377in}}%
\pgfpathlineto{\pgfqpoint{3.665337in}{2.088377in}}%
\pgfpathlineto{\pgfqpoint{3.763191in}{2.220142in}}%
\pgfpathlineto{\pgfqpoint{3.861044in}{2.360142in}}%
\pgfpathlineto{\pgfqpoint{3.958898in}{2.409554in}}%
\pgfpathlineto{\pgfqpoint{4.056751in}{2.426025in}}%
\pgfpathlineto{\pgfqpoint{4.154605in}{2.450730in}}%
\pgfusepath{stroke}%
\end{pgfscope}%
\begin{pgfscope}%
\pgfpathrectangle{\pgfqpoint{0.455741in}{0.385730in}}{\pgfqpoint{3.875000in}{2.310000in}}%
\pgfusepath{clip}%
\pgfsetbuttcap%
\pgfsetroundjoin%
\pgfsetlinewidth{0.803000pt}%
\definecolor{currentstroke}{rgb}{0.333333,0.333333,0.333333}%
\pgfsetstrokecolor{currentstroke}%
\pgfsetdash{{0.800000pt}{1.320000pt}}{0.000000pt}%
\pgfpathmoveto{\pgfqpoint{0.631877in}{0.490730in}}%
\pgfpathlineto{\pgfqpoint{0.729731in}{0.490730in}}%
\pgfpathlineto{\pgfqpoint{0.827585in}{0.490730in}}%
\pgfpathlineto{\pgfqpoint{0.925438in}{0.490730in}}%
\pgfpathlineto{\pgfqpoint{1.023292in}{0.490730in}}%
\pgfpathlineto{\pgfqpoint{1.121145in}{0.490730in}}%
\pgfpathlineto{\pgfqpoint{1.218999in}{0.490730in}}%
\pgfpathlineto{\pgfqpoint{1.316852in}{0.490730in}}%
\pgfpathlineto{\pgfqpoint{1.414706in}{0.490730in}}%
\pgfpathlineto{\pgfqpoint{1.512559in}{0.490730in}}%
\pgfpathlineto{\pgfqpoint{1.610413in}{0.490730in}}%
\pgfpathlineto{\pgfqpoint{1.708266in}{0.490730in}}%
\pgfpathlineto{\pgfqpoint{1.806120in}{0.490730in}}%
\pgfpathlineto{\pgfqpoint{1.903973in}{0.490730in}}%
\pgfpathlineto{\pgfqpoint{2.001827in}{0.490730in}}%
\pgfpathlineto{\pgfqpoint{2.099681in}{0.490730in}}%
\pgfpathlineto{\pgfqpoint{2.197534in}{0.490730in}}%
\pgfpathlineto{\pgfqpoint{2.295388in}{0.490730in}}%
\pgfpathlineto{\pgfqpoint{2.393241in}{0.490730in}}%
\pgfpathlineto{\pgfqpoint{2.491095in}{0.490730in}}%
\pgfpathlineto{\pgfqpoint{2.588948in}{0.490730in}}%
\pgfpathlineto{\pgfqpoint{2.686802in}{0.490730in}}%
\pgfpathlineto{\pgfqpoint{2.784655in}{0.490730in}}%
\pgfpathlineto{\pgfqpoint{2.882509in}{0.498966in}}%
\pgfpathlineto{\pgfqpoint{2.980362in}{0.523672in}}%
\pgfpathlineto{\pgfqpoint{3.078216in}{0.614260in}}%
\pgfpathlineto{\pgfqpoint{3.176069in}{0.844848in}}%
\pgfpathlineto{\pgfqpoint{3.273923in}{1.207201in}}%
\pgfpathlineto{\pgfqpoint{3.371776in}{1.627201in}}%
\pgfpathlineto{\pgfqpoint{3.469630in}{1.989554in}}%
\pgfpathlineto{\pgfqpoint{3.567484in}{2.277789in}}%
\pgfpathlineto{\pgfqpoint{3.665337in}{2.417789in}}%
\pgfpathlineto{\pgfqpoint{3.763191in}{2.475436in}}%
\pgfpathlineto{\pgfqpoint{3.861044in}{2.508377in}}%
\pgfpathlineto{\pgfqpoint{3.958898in}{2.516613in}}%
\pgfpathlineto{\pgfqpoint{4.056751in}{2.524848in}}%
\pgfpathlineto{\pgfqpoint{4.154605in}{2.524848in}}%
\pgfusepath{stroke}%
\end{pgfscope}%
\begin{pgfscope}%
\pgfsetrectcap%
\pgfsetmiterjoin%
\pgfsetlinewidth{0.501875pt}%
\definecolor{currentstroke}{rgb}{0.317647,0.317647,0.317647}%
\pgfsetstrokecolor{currentstroke}%
\pgfsetdash{}{0pt}%
\pgfpathmoveto{\pgfqpoint{0.455741in}{0.385730in}}%
\pgfpathlineto{\pgfqpoint{0.455741in}{2.695730in}}%
\pgfusepath{stroke}%
\end{pgfscope}%
\begin{pgfscope}%
\pgfsetrectcap%
\pgfsetmiterjoin%
\pgfsetlinewidth{0.501875pt}%
\definecolor{currentstroke}{rgb}{0.317647,0.317647,0.317647}%
\pgfsetstrokecolor{currentstroke}%
\pgfsetdash{}{0pt}%
\pgfpathmoveto{\pgfqpoint{0.455741in}{0.385730in}}%
\pgfpathlineto{\pgfqpoint{4.330741in}{0.385730in}}%
\pgfusepath{stroke}%
\end{pgfscope}%
\begin{pgfscope}%
\pgfsetbuttcap%
\pgfsetroundjoin%
\pgfsetlinewidth{0.803000pt}%
\definecolor{currentstroke}{rgb}{0.000000,0.000000,0.000000}%
\pgfsetstrokecolor{currentstroke}%
\pgfsetdash{{2.960000pt}{1.280000pt}}{0.000000pt}%
\pgfpathmoveto{\pgfqpoint{0.483508in}{2.635569in}}%
\pgfpathlineto{\pgfqpoint{0.557552in}{2.635569in}}%
\pgfusepath{stroke}%
\end{pgfscope}%
\begin{pgfscope}%
\definecolor{textcolor}{rgb}{0.000000,0.000000,0.000000}%
\pgfsetstrokecolor{textcolor}%
\pgfsetfillcolor{textcolor}%
\pgftext[x=0.603830in,y=2.603175in,left,base]{\color{textcolor}\rmfamily\fontsize{6.664000}{7.996800}\selectfont \(\displaystyle b \propto  \delta V = \SI{52.5}{\milli \V}\)}%
\end{pgfscope}%
\begin{pgfscope}%
\pgfsetrectcap%
\pgfsetroundjoin%
\pgfsetlinewidth{0.803000pt}%
\definecolor{currentstroke}{rgb}{0.000000,0.000000,0.000000}%
\pgfsetstrokecolor{currentstroke}%
\pgfsetdash{}{0pt}%
\pgfpathmoveto{\pgfqpoint{0.483508in}{2.515803in}}%
\pgfpathlineto{\pgfqpoint{0.557552in}{2.515803in}}%
\pgfusepath{stroke}%
\end{pgfscope}%
\begin{pgfscope}%
\definecolor{textcolor}{rgb}{0.000000,0.000000,0.000000}%
\pgfsetstrokecolor{textcolor}%
\pgfsetfillcolor{textcolor}%
\pgftext[x=0.603830in,y=2.483408in,left,base]{\color{textcolor}\rmfamily\fontsize{6.664000}{7.996800}\selectfont \(\displaystyle b \propto  \delta V = \SI{0.0}{\milli \V}\)}%
\end{pgfscope}%
\begin{pgfscope}%
\pgfsetbuttcap%
\pgfsetroundjoin%
\pgfsetlinewidth{0.803000pt}%
\definecolor{currentstroke}{rgb}{0.000000,0.000000,0.000000}%
\pgfsetstrokecolor{currentstroke}%
\pgfsetdash{{0.800000pt}{1.320000pt}}{0.000000pt}%
\pgfpathmoveto{\pgfqpoint{0.483508in}{2.396036in}}%
\pgfpathlineto{\pgfqpoint{0.557552in}{2.396036in}}%
\pgfusepath{stroke}%
\end{pgfscope}%
\begin{pgfscope}%
\definecolor{textcolor}{rgb}{0.000000,0.000000,0.000000}%
\pgfsetstrokecolor{textcolor}%
\pgfsetfillcolor{textcolor}%
\pgftext[x=0.603830in,y=2.363642in,left,base]{\color{textcolor}\rmfamily\fontsize{6.664000}{7.996800}\selectfont \(\displaystyle b \propto  \delta V = \SI{-52.4}{\milli \V}\)}%
\end{pgfscope}%
\end{pgfpicture}%
\makeatother%
\endgroup%

		\label{dlsactivationfunctionbias}
	\end{subfigure}
	\caption[Measurement of the sigmoid activation function on \gls{dls}]{Measurement of the sigmoid activation function on \gls{dls}. A sigmoid activation function can be achieved by applying additional Poisson noise spike trains onto the membrane. \textbf{(\subref{dlsactivationfunctionweight})}: The dependency of the input weight is given due to its direct proportionality to the synaptic input current $\gls{isyn} \propto w \nu_\text{in}$. \textbf{(\subref{dlsactivationfunctionbias})}: Changing the value of \gls{thres} and thereby the potnetial difference $\delta V$ to resting potential is interchangeable with adding a bias to the synaptic input of the neuron. The activation function is thereby shifted in either direction along the x-axis. This has been measured for all neurons in use for the Circles task. The shifted activation functions do not need to align per neuron, since the threshold parameter is set individually and can be easily adapted during the training.}
\end{figure}
%
%\begin{figure}
%	\begin{center}
%		%% Creator: Matplotlib, PGF backend
%%
%% To include the figure in your LaTeX document, write
%%   \input{<filename>.pgf}
%%
%% Make sure the required packages are loaded in your preamble
%%   \usepackage{pgf}
%%
%% Figures using additional raster images can only be included by \input if
%% they are in the same directory as the main LaTeX file. For loading figures
%% from other directories you can use the `import` package
%%   \usepackage{import}
%% and then include the figures with
%%   \import{<path to file>}{<filename>.pgf}
%%
%% Matplotlib used the following preamble
%%   \usepackage{amsmath} \usepackage{pifont} \usepackage{xcolor} \definecolor{green}{HTML}{467821} \definecolor{red}{HTML}{CF4457} \usepackage[detect-all]{siunitx}
%%   \usepackage{fontspec}
%%
\begingroup%
\makeatletter%
\begin{pgfpicture}%
\pgfpathrectangle{\pgfpointorigin}{\pgfqpoint{4.430741in}{2.795730in}}%
\pgfusepath{use as bounding box, clip}%
\begin{pgfscope}%
\pgfsetbuttcap%
\pgfsetmiterjoin%
\pgfsetlinewidth{0.000000pt}%
\definecolor{currentstroke}{rgb}{0.000000,0.000000,0.000000}%
\pgfsetstrokecolor{currentstroke}%
\pgfsetstrokeopacity{0.000000}%
\pgfsetdash{}{0pt}%
\pgfpathmoveto{\pgfqpoint{0.000000in}{0.000000in}}%
\pgfpathlineto{\pgfqpoint{4.430741in}{0.000000in}}%
\pgfpathlineto{\pgfqpoint{4.430741in}{2.795730in}}%
\pgfpathlineto{\pgfqpoint{0.000000in}{2.795730in}}%
\pgfpathclose%
\pgfusepath{}%
\end{pgfscope}%
\begin{pgfscope}%
\pgfsetbuttcap%
\pgfsetmiterjoin%
\pgfsetlinewidth{0.000000pt}%
\definecolor{currentstroke}{rgb}{0.000000,0.000000,0.000000}%
\pgfsetstrokecolor{currentstroke}%
\pgfsetstrokeopacity{0.000000}%
\pgfsetdash{}{0pt}%
\pgfpathmoveto{\pgfqpoint{0.455741in}{0.385730in}}%
\pgfpathlineto{\pgfqpoint{4.330741in}{0.385730in}}%
\pgfpathlineto{\pgfqpoint{4.330741in}{2.695730in}}%
\pgfpathlineto{\pgfqpoint{0.455741in}{2.695730in}}%
\pgfpathclose%
\pgfusepath{}%
\end{pgfscope}%
\begin{pgfscope}%
\pgfsetbuttcap%
\pgfsetroundjoin%
\definecolor{currentfill}{rgb}{0.317647,0.317647,0.317647}%
\pgfsetfillcolor{currentfill}%
\pgfsetlinewidth{0.501875pt}%
\definecolor{currentstroke}{rgb}{0.317647,0.317647,0.317647}%
\pgfsetstrokecolor{currentstroke}%
\pgfsetdash{}{0pt}%
\pgfsys@defobject{currentmarker}{\pgfqpoint{0.000000in}{-0.020833in}}{\pgfqpoint{0.000000in}{0.000000in}}{%
\pgfpathmoveto{\pgfqpoint{0.000000in}{0.000000in}}%
\pgfpathlineto{\pgfqpoint{0.000000in}{-0.020833in}}%
\pgfusepath{stroke,fill}%
}%
\begin{pgfscope}%
\pgfsys@transformshift{1.025723in}{0.385730in}%
\pgfsys@useobject{currentmarker}{}%
\end{pgfscope}%
\end{pgfscope}%
\begin{pgfscope}%
\definecolor{textcolor}{rgb}{0.317647,0.317647,0.317647}%
\pgfsetstrokecolor{textcolor}%
\pgfsetfillcolor{textcolor}%
\pgftext[x=1.025723in,y=0.337119in,,top]{\color{textcolor}\rmfamily\fontsize{6.664000}{7.996800}\selectfont \(\displaystyle -1000\)}%
\end{pgfscope}%
\begin{pgfscope}%
\pgfsetbuttcap%
\pgfsetroundjoin%
\definecolor{currentfill}{rgb}{0.317647,0.317647,0.317647}%
\pgfsetfillcolor{currentfill}%
\pgfsetlinewidth{0.501875pt}%
\definecolor{currentstroke}{rgb}{0.317647,0.317647,0.317647}%
\pgfsetstrokecolor{currentstroke}%
\pgfsetdash{}{0pt}%
\pgfsys@defobject{currentmarker}{\pgfqpoint{0.000000in}{-0.020833in}}{\pgfqpoint{0.000000in}{0.000000in}}{%
\pgfpathmoveto{\pgfqpoint{0.000000in}{0.000000in}}%
\pgfpathlineto{\pgfqpoint{0.000000in}{-0.020833in}}%
\pgfusepath{stroke,fill}%
}%
\begin{pgfscope}%
\pgfsys@transformshift{1.709482in}{0.385730in}%
\pgfsys@useobject{currentmarker}{}%
\end{pgfscope}%
\end{pgfscope}%
\begin{pgfscope}%
\definecolor{textcolor}{rgb}{0.317647,0.317647,0.317647}%
\pgfsetstrokecolor{textcolor}%
\pgfsetfillcolor{textcolor}%
\pgftext[x=1.709482in,y=0.337119in,,top]{\color{textcolor}\rmfamily\fontsize{6.664000}{7.996800}\selectfont \(\displaystyle -500\)}%
\end{pgfscope}%
\begin{pgfscope}%
\pgfsetbuttcap%
\pgfsetroundjoin%
\definecolor{currentfill}{rgb}{0.317647,0.317647,0.317647}%
\pgfsetfillcolor{currentfill}%
\pgfsetlinewidth{0.501875pt}%
\definecolor{currentstroke}{rgb}{0.317647,0.317647,0.317647}%
\pgfsetstrokecolor{currentstroke}%
\pgfsetdash{}{0pt}%
\pgfsys@defobject{currentmarker}{\pgfqpoint{0.000000in}{-0.020833in}}{\pgfqpoint{0.000000in}{0.000000in}}{%
\pgfpathmoveto{\pgfqpoint{0.000000in}{0.000000in}}%
\pgfpathlineto{\pgfqpoint{0.000000in}{-0.020833in}}%
\pgfusepath{stroke,fill}%
}%
\begin{pgfscope}%
\pgfsys@transformshift{2.393241in}{0.385730in}%
\pgfsys@useobject{currentmarker}{}%
\end{pgfscope}%
\end{pgfscope}%
\begin{pgfscope}%
\definecolor{textcolor}{rgb}{0.317647,0.317647,0.317647}%
\pgfsetstrokecolor{textcolor}%
\pgfsetfillcolor{textcolor}%
\pgftext[x=2.393241in,y=0.337119in,,top]{\color{textcolor}\rmfamily\fontsize{6.664000}{7.996800}\selectfont \(\displaystyle 0\)}%
\end{pgfscope}%
\begin{pgfscope}%
\pgfsetbuttcap%
\pgfsetroundjoin%
\definecolor{currentfill}{rgb}{0.317647,0.317647,0.317647}%
\pgfsetfillcolor{currentfill}%
\pgfsetlinewidth{0.501875pt}%
\definecolor{currentstroke}{rgb}{0.317647,0.317647,0.317647}%
\pgfsetstrokecolor{currentstroke}%
\pgfsetdash{}{0pt}%
\pgfsys@defobject{currentmarker}{\pgfqpoint{0.000000in}{-0.020833in}}{\pgfqpoint{0.000000in}{0.000000in}}{%
\pgfpathmoveto{\pgfqpoint{0.000000in}{0.000000in}}%
\pgfpathlineto{\pgfqpoint{0.000000in}{-0.020833in}}%
\pgfusepath{stroke,fill}%
}%
\begin{pgfscope}%
\pgfsys@transformshift{3.077000in}{0.385730in}%
\pgfsys@useobject{currentmarker}{}%
\end{pgfscope}%
\end{pgfscope}%
\begin{pgfscope}%
\definecolor{textcolor}{rgb}{0.317647,0.317647,0.317647}%
\pgfsetstrokecolor{textcolor}%
\pgfsetfillcolor{textcolor}%
\pgftext[x=3.077000in,y=0.337119in,,top]{\color{textcolor}\rmfamily\fontsize{6.664000}{7.996800}\selectfont \(\displaystyle 500\)}%
\end{pgfscope}%
\begin{pgfscope}%
\pgfsetbuttcap%
\pgfsetroundjoin%
\definecolor{currentfill}{rgb}{0.317647,0.317647,0.317647}%
\pgfsetfillcolor{currentfill}%
\pgfsetlinewidth{0.501875pt}%
\definecolor{currentstroke}{rgb}{0.317647,0.317647,0.317647}%
\pgfsetstrokecolor{currentstroke}%
\pgfsetdash{}{0pt}%
\pgfsys@defobject{currentmarker}{\pgfqpoint{0.000000in}{-0.020833in}}{\pgfqpoint{0.000000in}{0.000000in}}{%
\pgfpathmoveto{\pgfqpoint{0.000000in}{0.000000in}}%
\pgfpathlineto{\pgfqpoint{0.000000in}{-0.020833in}}%
\pgfusepath{stroke,fill}%
}%
\begin{pgfscope}%
\pgfsys@transformshift{3.760759in}{0.385730in}%
\pgfsys@useobject{currentmarker}{}%
\end{pgfscope}%
\end{pgfscope}%
\begin{pgfscope}%
\definecolor{textcolor}{rgb}{0.317647,0.317647,0.317647}%
\pgfsetstrokecolor{textcolor}%
\pgfsetfillcolor{textcolor}%
\pgftext[x=3.760759in,y=0.337119in,,top]{\color{textcolor}\rmfamily\fontsize{6.664000}{7.996800}\selectfont \(\displaystyle 1000\)}%
\end{pgfscope}%
\begin{pgfscope}%
\definecolor{textcolor}{rgb}{0.317647,0.317647,0.317647}%
\pgfsetstrokecolor{textcolor}%
\pgfsetfillcolor{textcolor}%
\pgftext[x=2.393241in,y=0.199375in,,top]{\color{textcolor}\rmfamily\fontsize{6.664000}{7.996800}\selectfont \(\displaystyle \nu_\mathrm{input} \; (\si{\kilo \Hz})\)}%
\end{pgfscope}%
\begin{pgfscope}%
\pgfsetbuttcap%
\pgfsetroundjoin%
\definecolor{currentfill}{rgb}{0.317647,0.317647,0.317647}%
\pgfsetfillcolor{currentfill}%
\pgfsetlinewidth{0.501875pt}%
\definecolor{currentstroke}{rgb}{0.317647,0.317647,0.317647}%
\pgfsetstrokecolor{currentstroke}%
\pgfsetdash{}{0pt}%
\pgfsys@defobject{currentmarker}{\pgfqpoint{-0.020833in}{0.000000in}}{\pgfqpoint{0.000000in}{0.000000in}}{%
\pgfpathmoveto{\pgfqpoint{0.000000in}{0.000000in}}%
\pgfpathlineto{\pgfqpoint{-0.020833in}{0.000000in}}%
\pgfusepath{stroke,fill}%
}%
\begin{pgfscope}%
\pgfsys@transformshift{0.455741in}{0.490730in}%
\pgfsys@useobject{currentmarker}{}%
\end{pgfscope}%
\end{pgfscope}%
\begin{pgfscope}%
\definecolor{textcolor}{rgb}{0.317647,0.317647,0.317647}%
\pgfsetstrokecolor{textcolor}%
\pgfsetfillcolor{textcolor}%
\pgftext[x=0.365656in,y=0.458614in,left,base]{\color{textcolor}\rmfamily\fontsize{6.664000}{7.996800}\selectfont \(\displaystyle 0\)}%
\end{pgfscope}%
\begin{pgfscope}%
\pgfsetbuttcap%
\pgfsetroundjoin%
\definecolor{currentfill}{rgb}{0.317647,0.317647,0.317647}%
\pgfsetfillcolor{currentfill}%
\pgfsetlinewidth{0.501875pt}%
\definecolor{currentstroke}{rgb}{0.317647,0.317647,0.317647}%
\pgfsetstrokecolor{currentstroke}%
\pgfsetdash{}{0pt}%
\pgfsys@defobject{currentmarker}{\pgfqpoint{-0.020833in}{0.000000in}}{\pgfqpoint{0.000000in}{0.000000in}}{%
\pgfpathmoveto{\pgfqpoint{0.000000in}{0.000000in}}%
\pgfpathlineto{\pgfqpoint{-0.020833in}{0.000000in}}%
\pgfusepath{stroke,fill}%
}%
\begin{pgfscope}%
\pgfsys@transformshift{0.455741in}{0.869554in}%
\pgfsys@useobject{currentmarker}{}%
\end{pgfscope}%
\end{pgfscope}%
\begin{pgfscope}%
\definecolor{textcolor}{rgb}{0.317647,0.317647,0.317647}%
\pgfsetstrokecolor{textcolor}%
\pgfsetfillcolor{textcolor}%
\pgftext[x=0.310293in,y=0.837437in,left,base]{\color{textcolor}\rmfamily\fontsize{6.664000}{7.996800}\selectfont \(\displaystyle 20\)}%
\end{pgfscope}%
\begin{pgfscope}%
\pgfsetbuttcap%
\pgfsetroundjoin%
\definecolor{currentfill}{rgb}{0.317647,0.317647,0.317647}%
\pgfsetfillcolor{currentfill}%
\pgfsetlinewidth{0.501875pt}%
\definecolor{currentstroke}{rgb}{0.317647,0.317647,0.317647}%
\pgfsetstrokecolor{currentstroke}%
\pgfsetdash{}{0pt}%
\pgfsys@defobject{currentmarker}{\pgfqpoint{-0.020833in}{0.000000in}}{\pgfqpoint{0.000000in}{0.000000in}}{%
\pgfpathmoveto{\pgfqpoint{0.000000in}{0.000000in}}%
\pgfpathlineto{\pgfqpoint{-0.020833in}{0.000000in}}%
\pgfusepath{stroke,fill}%
}%
\begin{pgfscope}%
\pgfsys@transformshift{0.455741in}{1.248377in}%
\pgfsys@useobject{currentmarker}{}%
\end{pgfscope}%
\end{pgfscope}%
\begin{pgfscope}%
\definecolor{textcolor}{rgb}{0.317647,0.317647,0.317647}%
\pgfsetstrokecolor{textcolor}%
\pgfsetfillcolor{textcolor}%
\pgftext[x=0.310293in,y=1.216261in,left,base]{\color{textcolor}\rmfamily\fontsize{6.664000}{7.996800}\selectfont \(\displaystyle 40\)}%
\end{pgfscope}%
\begin{pgfscope}%
\pgfsetbuttcap%
\pgfsetroundjoin%
\definecolor{currentfill}{rgb}{0.317647,0.317647,0.317647}%
\pgfsetfillcolor{currentfill}%
\pgfsetlinewidth{0.501875pt}%
\definecolor{currentstroke}{rgb}{0.317647,0.317647,0.317647}%
\pgfsetstrokecolor{currentstroke}%
\pgfsetdash{}{0pt}%
\pgfsys@defobject{currentmarker}{\pgfqpoint{-0.020833in}{0.000000in}}{\pgfqpoint{0.000000in}{0.000000in}}{%
\pgfpathmoveto{\pgfqpoint{0.000000in}{0.000000in}}%
\pgfpathlineto{\pgfqpoint{-0.020833in}{0.000000in}}%
\pgfusepath{stroke,fill}%
}%
\begin{pgfscope}%
\pgfsys@transformshift{0.455741in}{1.627201in}%
\pgfsys@useobject{currentmarker}{}%
\end{pgfscope}%
\end{pgfscope}%
\begin{pgfscope}%
\definecolor{textcolor}{rgb}{0.317647,0.317647,0.317647}%
\pgfsetstrokecolor{textcolor}%
\pgfsetfillcolor{textcolor}%
\pgftext[x=0.310293in,y=1.595084in,left,base]{\color{textcolor}\rmfamily\fontsize{6.664000}{7.996800}\selectfont \(\displaystyle 60\)}%
\end{pgfscope}%
\begin{pgfscope}%
\pgfsetbuttcap%
\pgfsetroundjoin%
\definecolor{currentfill}{rgb}{0.317647,0.317647,0.317647}%
\pgfsetfillcolor{currentfill}%
\pgfsetlinewidth{0.501875pt}%
\definecolor{currentstroke}{rgb}{0.317647,0.317647,0.317647}%
\pgfsetstrokecolor{currentstroke}%
\pgfsetdash{}{0pt}%
\pgfsys@defobject{currentmarker}{\pgfqpoint{-0.020833in}{0.000000in}}{\pgfqpoint{0.000000in}{0.000000in}}{%
\pgfpathmoveto{\pgfqpoint{0.000000in}{0.000000in}}%
\pgfpathlineto{\pgfqpoint{-0.020833in}{0.000000in}}%
\pgfusepath{stroke,fill}%
}%
\begin{pgfscope}%
\pgfsys@transformshift{0.455741in}{2.006025in}%
\pgfsys@useobject{currentmarker}{}%
\end{pgfscope}%
\end{pgfscope}%
\begin{pgfscope}%
\definecolor{textcolor}{rgb}{0.317647,0.317647,0.317647}%
\pgfsetstrokecolor{textcolor}%
\pgfsetfillcolor{textcolor}%
\pgftext[x=0.310293in,y=1.973908in,left,base]{\color{textcolor}\rmfamily\fontsize{6.664000}{7.996800}\selectfont \(\displaystyle 80\)}%
\end{pgfscope}%
\begin{pgfscope}%
\pgfsetbuttcap%
\pgfsetroundjoin%
\definecolor{currentfill}{rgb}{0.317647,0.317647,0.317647}%
\pgfsetfillcolor{currentfill}%
\pgfsetlinewidth{0.501875pt}%
\definecolor{currentstroke}{rgb}{0.317647,0.317647,0.317647}%
\pgfsetstrokecolor{currentstroke}%
\pgfsetdash{}{0pt}%
\pgfsys@defobject{currentmarker}{\pgfqpoint{-0.020833in}{0.000000in}}{\pgfqpoint{0.000000in}{0.000000in}}{%
\pgfpathmoveto{\pgfqpoint{0.000000in}{0.000000in}}%
\pgfpathlineto{\pgfqpoint{-0.020833in}{0.000000in}}%
\pgfusepath{stroke,fill}%
}%
\begin{pgfscope}%
\pgfsys@transformshift{0.455741in}{2.384848in}%
\pgfsys@useobject{currentmarker}{}%
\end{pgfscope}%
\end{pgfscope}%
\begin{pgfscope}%
\definecolor{textcolor}{rgb}{0.317647,0.317647,0.317647}%
\pgfsetstrokecolor{textcolor}%
\pgfsetfillcolor{textcolor}%
\pgftext[x=0.254930in,y=2.352731in,left,base]{\color{textcolor}\rmfamily\fontsize{6.664000}{7.996800}\selectfont \(\displaystyle 100\)}%
\end{pgfscope}%
\begin{pgfscope}%
\definecolor{textcolor}{rgb}{0.317647,0.317647,0.317647}%
\pgfsetstrokecolor{textcolor}%
\pgfsetfillcolor{textcolor}%
\pgftext[x=0.199375in,y=1.540730in,,bottom,rotate=90.000000]{\color{textcolor}\rmfamily\fontsize{6.664000}{7.996800}\selectfont \(\displaystyle \nu_\mathrm{output} \; (\si{\kilo \Hz})\)}%
\end{pgfscope}%
\begin{pgfscope}%
\pgfpathrectangle{\pgfqpoint{0.455741in}{0.385730in}}{\pgfqpoint{3.875000in}{2.310000in}}%
\pgfusepath{clip}%
\pgfsetbuttcap%
\pgfsetroundjoin%
\pgfsetlinewidth{0.803000pt}%
\definecolor{currentstroke}{rgb}{0.333333,0.333333,0.333333}%
\pgfsetstrokecolor{currentstroke}%
\pgfsetdash{{2.960000pt}{1.280000pt}}{0.000000pt}%
\pgfpathmoveto{\pgfqpoint{0.631877in}{0.498966in}}%
\pgfpathlineto{\pgfqpoint{0.729731in}{0.490730in}}%
\pgfpathlineto{\pgfqpoint{0.827585in}{0.490730in}}%
\pgfpathlineto{\pgfqpoint{0.925438in}{0.490730in}}%
\pgfpathlineto{\pgfqpoint{1.023292in}{0.507201in}}%
\pgfpathlineto{\pgfqpoint{1.121145in}{0.540142in}}%
\pgfpathlineto{\pgfqpoint{1.218999in}{0.770730in}}%
\pgfpathlineto{\pgfqpoint{1.316852in}{1.264848in}}%
\pgfpathlineto{\pgfqpoint{1.414706in}{1.561319in}}%
\pgfpathlineto{\pgfqpoint{1.512559in}{1.981319in}}%
\pgfpathlineto{\pgfqpoint{1.610413in}{2.129554in}}%
\pgfpathlineto{\pgfqpoint{1.708266in}{2.368377in}}%
\pgfpathlineto{\pgfqpoint{1.806120in}{2.442495in}}%
\pgfpathlineto{\pgfqpoint{1.903973in}{2.491907in}}%
\pgfpathlineto{\pgfqpoint{2.001827in}{2.516613in}}%
\pgfpathlineto{\pgfqpoint{2.099681in}{2.541319in}}%
\pgfpathlineto{\pgfqpoint{2.197534in}{2.557789in}}%
\pgfpathlineto{\pgfqpoint{2.295388in}{2.566025in}}%
\pgfpathlineto{\pgfqpoint{2.393241in}{2.582495in}}%
\pgfpathlineto{\pgfqpoint{2.491095in}{2.582495in}}%
\pgfpathlineto{\pgfqpoint{2.588948in}{2.590730in}}%
\pgfpathlineto{\pgfqpoint{2.686802in}{2.590730in}}%
\pgfpathlineto{\pgfqpoint{2.784655in}{2.590730in}}%
\pgfpathlineto{\pgfqpoint{2.882509in}{2.590730in}}%
\pgfpathlineto{\pgfqpoint{2.980362in}{2.590730in}}%
\pgfpathlineto{\pgfqpoint{3.078216in}{2.590730in}}%
\pgfpathlineto{\pgfqpoint{3.176069in}{2.590730in}}%
\pgfpathlineto{\pgfqpoint{3.273923in}{2.590730in}}%
\pgfpathlineto{\pgfqpoint{3.371776in}{2.590730in}}%
\pgfpathlineto{\pgfqpoint{3.469630in}{2.590730in}}%
\pgfpathlineto{\pgfqpoint{3.567484in}{2.590730in}}%
\pgfpathlineto{\pgfqpoint{3.665337in}{2.590730in}}%
\pgfpathlineto{\pgfqpoint{3.763191in}{2.590730in}}%
\pgfpathlineto{\pgfqpoint{3.861044in}{2.590730in}}%
\pgfpathlineto{\pgfqpoint{3.958898in}{2.590730in}}%
\pgfpathlineto{\pgfqpoint{4.056751in}{2.590730in}}%
\pgfpathlineto{\pgfqpoint{4.154605in}{2.590730in}}%
\pgfusepath{stroke}%
\end{pgfscope}%
\begin{pgfscope}%
\pgfpathrectangle{\pgfqpoint{0.455741in}{0.385730in}}{\pgfqpoint{3.875000in}{2.310000in}}%
\pgfusepath{clip}%
\pgfsetbuttcap%
\pgfsetroundjoin%
\pgfsetlinewidth{0.803000pt}%
\definecolor{currentstroke}{rgb}{0.686275,0.352941,0.313725}%
\pgfsetstrokecolor{currentstroke}%
\pgfsetdash{{2.960000pt}{1.280000pt}}{0.000000pt}%
\pgfpathmoveto{\pgfqpoint{0.631877in}{0.498966in}}%
\pgfpathlineto{\pgfqpoint{0.729731in}{0.490730in}}%
\pgfpathlineto{\pgfqpoint{0.827585in}{0.490730in}}%
\pgfpathlineto{\pgfqpoint{0.925438in}{0.490730in}}%
\pgfpathlineto{\pgfqpoint{1.023292in}{0.507201in}}%
\pgfpathlineto{\pgfqpoint{1.121145in}{0.531907in}}%
\pgfpathlineto{\pgfqpoint{1.218999in}{0.663672in}}%
\pgfpathlineto{\pgfqpoint{1.316852in}{1.058966in}}%
\pgfpathlineto{\pgfqpoint{1.414706in}{1.330730in}}%
\pgfpathlineto{\pgfqpoint{1.512559in}{1.684848in}}%
\pgfpathlineto{\pgfqpoint{1.610413in}{1.923672in}}%
\pgfpathlineto{\pgfqpoint{1.708266in}{2.187201in}}%
\pgfpathlineto{\pgfqpoint{1.806120in}{2.360142in}}%
\pgfpathlineto{\pgfqpoint{1.903973in}{2.450730in}}%
\pgfpathlineto{\pgfqpoint{2.001827in}{2.483672in}}%
\pgfpathlineto{\pgfqpoint{2.099681in}{2.524848in}}%
\pgfpathlineto{\pgfqpoint{2.197534in}{2.549554in}}%
\pgfpathlineto{\pgfqpoint{2.295388in}{2.557789in}}%
\pgfpathlineto{\pgfqpoint{2.393241in}{2.566025in}}%
\pgfpathlineto{\pgfqpoint{2.491095in}{2.574260in}}%
\pgfpathlineto{\pgfqpoint{2.588948in}{2.582495in}}%
\pgfpathlineto{\pgfqpoint{2.686802in}{2.590730in}}%
\pgfpathlineto{\pgfqpoint{2.784655in}{2.590730in}}%
\pgfpathlineto{\pgfqpoint{2.882509in}{2.590730in}}%
\pgfpathlineto{\pgfqpoint{2.980362in}{2.590730in}}%
\pgfpathlineto{\pgfqpoint{3.078216in}{2.590730in}}%
\pgfpathlineto{\pgfqpoint{3.176069in}{2.590730in}}%
\pgfpathlineto{\pgfqpoint{3.273923in}{2.590730in}}%
\pgfpathlineto{\pgfqpoint{3.371776in}{2.590730in}}%
\pgfpathlineto{\pgfqpoint{3.469630in}{2.590730in}}%
\pgfpathlineto{\pgfqpoint{3.567484in}{2.590730in}}%
\pgfpathlineto{\pgfqpoint{3.665337in}{2.590730in}}%
\pgfpathlineto{\pgfqpoint{3.763191in}{2.590730in}}%
\pgfpathlineto{\pgfqpoint{3.861044in}{2.590730in}}%
\pgfpathlineto{\pgfqpoint{3.958898in}{2.590730in}}%
\pgfpathlineto{\pgfqpoint{4.056751in}{2.590730in}}%
\pgfpathlineto{\pgfqpoint{4.154605in}{2.590730in}}%
\pgfusepath{stroke}%
\end{pgfscope}%
\begin{pgfscope}%
\pgfpathrectangle{\pgfqpoint{0.455741in}{0.385730in}}{\pgfqpoint{3.875000in}{2.310000in}}%
\pgfusepath{clip}%
\pgfsetbuttcap%
\pgfsetroundjoin%
\pgfsetlinewidth{0.803000pt}%
\definecolor{currentstroke}{rgb}{0.000000,0.356863,0.509804}%
\pgfsetstrokecolor{currentstroke}%
\pgfsetdash{{2.960000pt}{1.280000pt}}{0.000000pt}%
\pgfpathmoveto{\pgfqpoint{0.631877in}{0.498966in}}%
\pgfpathlineto{\pgfqpoint{0.729731in}{0.490730in}}%
\pgfpathlineto{\pgfqpoint{0.827585in}{0.490730in}}%
\pgfpathlineto{\pgfqpoint{0.925438in}{0.490730in}}%
\pgfpathlineto{\pgfqpoint{1.023292in}{0.490730in}}%
\pgfpathlineto{\pgfqpoint{1.121145in}{0.490730in}}%
\pgfpathlineto{\pgfqpoint{1.218999in}{0.540142in}}%
\pgfpathlineto{\pgfqpoint{1.316852in}{0.680142in}}%
\pgfpathlineto{\pgfqpoint{1.414706in}{0.803672in}}%
\pgfpathlineto{\pgfqpoint{1.512559in}{1.058966in}}%
\pgfpathlineto{\pgfqpoint{1.610413in}{1.347201in}}%
\pgfpathlineto{\pgfqpoint{1.708266in}{1.709554in}}%
\pgfpathlineto{\pgfqpoint{1.806120in}{2.080142in}}%
\pgfpathlineto{\pgfqpoint{1.903973in}{2.294260in}}%
\pgfpathlineto{\pgfqpoint{2.001827in}{2.376613in}}%
\pgfpathlineto{\pgfqpoint{2.099681in}{2.475436in}}%
\pgfpathlineto{\pgfqpoint{2.197534in}{2.500142in}}%
\pgfpathlineto{\pgfqpoint{2.295388in}{2.516613in}}%
\pgfpathlineto{\pgfqpoint{2.393241in}{2.533083in}}%
\pgfpathlineto{\pgfqpoint{2.491095in}{2.541319in}}%
\pgfpathlineto{\pgfqpoint{2.588948in}{2.549554in}}%
\pgfpathlineto{\pgfqpoint{2.686802in}{2.557789in}}%
\pgfpathlineto{\pgfqpoint{2.784655in}{2.566025in}}%
\pgfpathlineto{\pgfqpoint{2.882509in}{2.574260in}}%
\pgfpathlineto{\pgfqpoint{2.980362in}{2.574260in}}%
\pgfpathlineto{\pgfqpoint{3.078216in}{2.582495in}}%
\pgfpathlineto{\pgfqpoint{3.176069in}{2.582495in}}%
\pgfpathlineto{\pgfqpoint{3.273923in}{2.590730in}}%
\pgfpathlineto{\pgfqpoint{3.371776in}{2.590730in}}%
\pgfpathlineto{\pgfqpoint{3.469630in}{2.590730in}}%
\pgfpathlineto{\pgfqpoint{3.567484in}{2.590730in}}%
\pgfpathlineto{\pgfqpoint{3.665337in}{2.590730in}}%
\pgfpathlineto{\pgfqpoint{3.763191in}{2.590730in}}%
\pgfpathlineto{\pgfqpoint{3.861044in}{2.590730in}}%
\pgfpathlineto{\pgfqpoint{3.958898in}{2.590730in}}%
\pgfpathlineto{\pgfqpoint{4.056751in}{2.590730in}}%
\pgfpathlineto{\pgfqpoint{4.154605in}{2.590730in}}%
\pgfusepath{stroke}%
\end{pgfscope}%
\begin{pgfscope}%
\pgfpathrectangle{\pgfqpoint{0.455741in}{0.385730in}}{\pgfqpoint{3.875000in}{2.310000in}}%
\pgfusepath{clip}%
\pgfsetbuttcap%
\pgfsetroundjoin%
\pgfsetlinewidth{0.803000pt}%
\definecolor{currentstroke}{rgb}{0.490196,0.588235,0.431373}%
\pgfsetstrokecolor{currentstroke}%
\pgfsetdash{{2.960000pt}{1.280000pt}}{0.000000pt}%
\pgfpathmoveto{\pgfqpoint{0.631877in}{0.498966in}}%
\pgfpathlineto{\pgfqpoint{0.729731in}{0.490730in}}%
\pgfpathlineto{\pgfqpoint{0.827585in}{0.490730in}}%
\pgfpathlineto{\pgfqpoint{0.925438in}{0.490730in}}%
\pgfpathlineto{\pgfqpoint{1.023292in}{0.531907in}}%
\pgfpathlineto{\pgfqpoint{1.121145in}{0.581319in}}%
\pgfpathlineto{\pgfqpoint{1.218999in}{0.918966in}}%
\pgfpathlineto{\pgfqpoint{1.316852in}{1.380142in}}%
\pgfpathlineto{\pgfqpoint{1.414706in}{1.577789in}}%
\pgfpathlineto{\pgfqpoint{1.512559in}{2.014260in}}%
\pgfpathlineto{\pgfqpoint{1.610413in}{2.121319in}}%
\pgfpathlineto{\pgfqpoint{1.708266in}{2.343672in}}%
\pgfpathlineto{\pgfqpoint{1.806120in}{2.442495in}}%
\pgfpathlineto{\pgfqpoint{1.903973in}{2.500142in}}%
\pgfpathlineto{\pgfqpoint{2.001827in}{2.516613in}}%
\pgfpathlineto{\pgfqpoint{2.099681in}{2.541319in}}%
\pgfpathlineto{\pgfqpoint{2.197534in}{2.549554in}}%
\pgfpathlineto{\pgfqpoint{2.295388in}{2.574260in}}%
\pgfpathlineto{\pgfqpoint{2.393241in}{2.574260in}}%
\pgfpathlineto{\pgfqpoint{2.491095in}{2.582495in}}%
\pgfpathlineto{\pgfqpoint{2.588948in}{2.590730in}}%
\pgfpathlineto{\pgfqpoint{2.686802in}{2.582495in}}%
\pgfpathlineto{\pgfqpoint{2.784655in}{2.590730in}}%
\pgfpathlineto{\pgfqpoint{2.882509in}{2.590730in}}%
\pgfpathlineto{\pgfqpoint{2.980362in}{2.590730in}}%
\pgfpathlineto{\pgfqpoint{3.078216in}{2.590730in}}%
\pgfpathlineto{\pgfqpoint{3.176069in}{2.590730in}}%
\pgfpathlineto{\pgfqpoint{3.273923in}{2.590730in}}%
\pgfpathlineto{\pgfqpoint{3.371776in}{2.590730in}}%
\pgfpathlineto{\pgfqpoint{3.469630in}{2.590730in}}%
\pgfpathlineto{\pgfqpoint{3.567484in}{2.590730in}}%
\pgfpathlineto{\pgfqpoint{3.665337in}{2.590730in}}%
\pgfpathlineto{\pgfqpoint{3.763191in}{2.590730in}}%
\pgfpathlineto{\pgfqpoint{3.861044in}{2.590730in}}%
\pgfpathlineto{\pgfqpoint{3.958898in}{2.590730in}}%
\pgfpathlineto{\pgfqpoint{4.056751in}{2.590730in}}%
\pgfpathlineto{\pgfqpoint{4.154605in}{2.590730in}}%
\pgfusepath{stroke}%
\end{pgfscope}%
\begin{pgfscope}%
\pgfpathrectangle{\pgfqpoint{0.455741in}{0.385730in}}{\pgfqpoint{3.875000in}{2.310000in}}%
\pgfusepath{clip}%
\pgfsetbuttcap%
\pgfsetroundjoin%
\pgfsetlinewidth{0.803000pt}%
\definecolor{currentstroke}{rgb}{0.843137,0.666667,0.313725}%
\pgfsetstrokecolor{currentstroke}%
\pgfsetdash{{2.960000pt}{1.280000pt}}{0.000000pt}%
\pgfpathmoveto{\pgfqpoint{0.631877in}{0.498966in}}%
\pgfpathlineto{\pgfqpoint{0.729731in}{0.490730in}}%
\pgfpathlineto{\pgfqpoint{0.827585in}{0.490730in}}%
\pgfpathlineto{\pgfqpoint{0.925438in}{0.490730in}}%
\pgfpathlineto{\pgfqpoint{1.023292in}{0.490730in}}%
\pgfpathlineto{\pgfqpoint{1.121145in}{0.507201in}}%
\pgfpathlineto{\pgfqpoint{1.218999in}{0.589554in}}%
\pgfpathlineto{\pgfqpoint{1.316852in}{0.861319in}}%
\pgfpathlineto{\pgfqpoint{1.414706in}{0.976613in}}%
\pgfpathlineto{\pgfqpoint{1.512559in}{1.371907in}}%
\pgfpathlineto{\pgfqpoint{1.610413in}{1.767201in}}%
\pgfpathlineto{\pgfqpoint{1.708266in}{2.055436in}}%
\pgfpathlineto{\pgfqpoint{1.806120in}{2.269554in}}%
\pgfpathlineto{\pgfqpoint{1.903973in}{2.417789in}}%
\pgfpathlineto{\pgfqpoint{2.001827in}{2.483672in}}%
\pgfpathlineto{\pgfqpoint{2.099681in}{2.541319in}}%
\pgfpathlineto{\pgfqpoint{2.197534in}{2.549554in}}%
\pgfpathlineto{\pgfqpoint{2.295388in}{2.574260in}}%
\pgfpathlineto{\pgfqpoint{2.393241in}{2.582495in}}%
\pgfpathlineto{\pgfqpoint{2.491095in}{2.590730in}}%
\pgfpathlineto{\pgfqpoint{2.588948in}{2.590730in}}%
\pgfpathlineto{\pgfqpoint{2.686802in}{2.590730in}}%
\pgfpathlineto{\pgfqpoint{2.784655in}{2.590730in}}%
\pgfpathlineto{\pgfqpoint{2.882509in}{2.590730in}}%
\pgfpathlineto{\pgfqpoint{2.980362in}{2.590730in}}%
\pgfpathlineto{\pgfqpoint{3.078216in}{2.590730in}}%
\pgfpathlineto{\pgfqpoint{3.176069in}{2.590730in}}%
\pgfpathlineto{\pgfqpoint{3.273923in}{2.590730in}}%
\pgfpathlineto{\pgfqpoint{3.371776in}{2.590730in}}%
\pgfpathlineto{\pgfqpoint{3.469630in}{2.590730in}}%
\pgfpathlineto{\pgfqpoint{3.567484in}{2.590730in}}%
\pgfpathlineto{\pgfqpoint{3.665337in}{2.590730in}}%
\pgfpathlineto{\pgfqpoint{3.763191in}{2.590730in}}%
\pgfpathlineto{\pgfqpoint{3.861044in}{2.590730in}}%
\pgfpathlineto{\pgfqpoint{3.958898in}{2.590730in}}%
\pgfpathlineto{\pgfqpoint{4.056751in}{2.590730in}}%
\pgfpathlineto{\pgfqpoint{4.154605in}{2.590730in}}%
\pgfusepath{stroke}%
\end{pgfscope}%
\begin{pgfscope}%
\pgfpathrectangle{\pgfqpoint{0.455741in}{0.385730in}}{\pgfqpoint{3.875000in}{2.310000in}}%
\pgfusepath{clip}%
\pgfsetbuttcap%
\pgfsetroundjoin%
\pgfsetlinewidth{0.803000pt}%
\definecolor{currentstroke}{rgb}{0.333333,0.333333,0.333333}%
\pgfsetstrokecolor{currentstroke}%
\pgfsetdash{{2.960000pt}{1.280000pt}}{0.000000pt}%
\pgfpathmoveto{\pgfqpoint{0.631877in}{0.498966in}}%
\pgfpathlineto{\pgfqpoint{0.729731in}{0.490730in}}%
\pgfpathlineto{\pgfqpoint{0.827585in}{0.490730in}}%
\pgfpathlineto{\pgfqpoint{0.925438in}{0.490730in}}%
\pgfpathlineto{\pgfqpoint{1.023292in}{0.490730in}}%
\pgfpathlineto{\pgfqpoint{1.121145in}{0.507201in}}%
\pgfpathlineto{\pgfqpoint{1.218999in}{0.573083in}}%
\pgfpathlineto{\pgfqpoint{1.316852in}{0.861319in}}%
\pgfpathlineto{\pgfqpoint{1.414706in}{1.042495in}}%
\pgfpathlineto{\pgfqpoint{1.512559in}{1.404848in}}%
\pgfpathlineto{\pgfqpoint{1.610413in}{1.709554in}}%
\pgfpathlineto{\pgfqpoint{1.708266in}{2.080142in}}%
\pgfpathlineto{\pgfqpoint{1.806120in}{2.261319in}}%
\pgfpathlineto{\pgfqpoint{1.903973in}{2.393083in}}%
\pgfpathlineto{\pgfqpoint{2.001827in}{2.450730in}}%
\pgfpathlineto{\pgfqpoint{2.099681in}{2.500142in}}%
\pgfpathlineto{\pgfqpoint{2.197534in}{2.516613in}}%
\pgfpathlineto{\pgfqpoint{2.295388in}{2.533083in}}%
\pgfpathlineto{\pgfqpoint{2.393241in}{2.541319in}}%
\pgfpathlineto{\pgfqpoint{2.491095in}{2.549554in}}%
\pgfpathlineto{\pgfqpoint{2.588948in}{2.557789in}}%
\pgfpathlineto{\pgfqpoint{2.686802in}{2.566025in}}%
\pgfpathlineto{\pgfqpoint{2.784655in}{2.566025in}}%
\pgfpathlineto{\pgfqpoint{2.882509in}{2.574260in}}%
\pgfpathlineto{\pgfqpoint{2.980362in}{2.582495in}}%
\pgfpathlineto{\pgfqpoint{3.078216in}{2.582495in}}%
\pgfpathlineto{\pgfqpoint{3.176069in}{2.582495in}}%
\pgfpathlineto{\pgfqpoint{3.273923in}{2.590730in}}%
\pgfpathlineto{\pgfqpoint{3.371776in}{2.590730in}}%
\pgfpathlineto{\pgfqpoint{3.469630in}{2.590730in}}%
\pgfpathlineto{\pgfqpoint{3.567484in}{2.590730in}}%
\pgfpathlineto{\pgfqpoint{3.665337in}{2.590730in}}%
\pgfpathlineto{\pgfqpoint{3.763191in}{2.590730in}}%
\pgfpathlineto{\pgfqpoint{3.861044in}{2.590730in}}%
\pgfpathlineto{\pgfqpoint{3.958898in}{2.590730in}}%
\pgfpathlineto{\pgfqpoint{4.056751in}{2.590730in}}%
\pgfpathlineto{\pgfqpoint{4.154605in}{2.590730in}}%
\pgfusepath{stroke}%
\end{pgfscope}%
\begin{pgfscope}%
\pgfpathrectangle{\pgfqpoint{0.455741in}{0.385730in}}{\pgfqpoint{3.875000in}{2.310000in}}%
\pgfusepath{clip}%
\pgfsetbuttcap%
\pgfsetroundjoin%
\pgfsetlinewidth{0.803000pt}%
\definecolor{currentstroke}{rgb}{0.686275,0.352941,0.313725}%
\pgfsetstrokecolor{currentstroke}%
\pgfsetdash{{2.960000pt}{1.280000pt}}{0.000000pt}%
\pgfpathmoveto{\pgfqpoint{0.631877in}{0.498966in}}%
\pgfpathlineto{\pgfqpoint{0.729731in}{0.490730in}}%
\pgfpathlineto{\pgfqpoint{0.827585in}{0.490730in}}%
\pgfpathlineto{\pgfqpoint{0.925438in}{0.490730in}}%
\pgfpathlineto{\pgfqpoint{1.023292in}{0.490730in}}%
\pgfpathlineto{\pgfqpoint{1.121145in}{0.490730in}}%
\pgfpathlineto{\pgfqpoint{1.218999in}{0.515436in}}%
\pgfpathlineto{\pgfqpoint{1.316852in}{0.614260in}}%
\pgfpathlineto{\pgfqpoint{1.414706in}{0.663672in}}%
\pgfpathlineto{\pgfqpoint{1.512559in}{0.968377in}}%
\pgfpathlineto{\pgfqpoint{1.610413in}{1.314260in}}%
\pgfpathlineto{\pgfqpoint{1.708266in}{1.717789in}}%
\pgfpathlineto{\pgfqpoint{1.806120in}{1.973083in}}%
\pgfpathlineto{\pgfqpoint{1.903973in}{2.277789in}}%
\pgfpathlineto{\pgfqpoint{2.001827in}{2.393083in}}%
\pgfpathlineto{\pgfqpoint{2.099681in}{2.516613in}}%
\pgfpathlineto{\pgfqpoint{2.197534in}{2.541319in}}%
\pgfpathlineto{\pgfqpoint{2.295388in}{2.557789in}}%
\pgfpathlineto{\pgfqpoint{2.393241in}{2.574260in}}%
\pgfpathlineto{\pgfqpoint{2.491095in}{2.582495in}}%
\pgfpathlineto{\pgfqpoint{2.588948in}{2.582495in}}%
\pgfpathlineto{\pgfqpoint{2.686802in}{2.590730in}}%
\pgfpathlineto{\pgfqpoint{2.784655in}{2.582495in}}%
\pgfpathlineto{\pgfqpoint{2.882509in}{2.590730in}}%
\pgfpathlineto{\pgfqpoint{2.980362in}{2.590730in}}%
\pgfpathlineto{\pgfqpoint{3.078216in}{2.590730in}}%
\pgfpathlineto{\pgfqpoint{3.176069in}{2.590730in}}%
\pgfpathlineto{\pgfqpoint{3.273923in}{2.590730in}}%
\pgfpathlineto{\pgfqpoint{3.371776in}{2.590730in}}%
\pgfpathlineto{\pgfqpoint{3.469630in}{2.590730in}}%
\pgfpathlineto{\pgfqpoint{3.567484in}{2.590730in}}%
\pgfpathlineto{\pgfqpoint{3.665337in}{2.590730in}}%
\pgfpathlineto{\pgfqpoint{3.763191in}{2.590730in}}%
\pgfpathlineto{\pgfqpoint{3.861044in}{2.590730in}}%
\pgfpathlineto{\pgfqpoint{3.958898in}{2.590730in}}%
\pgfpathlineto{\pgfqpoint{4.056751in}{2.590730in}}%
\pgfpathlineto{\pgfqpoint{4.154605in}{2.590730in}}%
\pgfusepath{stroke}%
\end{pgfscope}%
\begin{pgfscope}%
\pgfpathrectangle{\pgfqpoint{0.455741in}{0.385730in}}{\pgfqpoint{3.875000in}{2.310000in}}%
\pgfusepath{clip}%
\pgfsetbuttcap%
\pgfsetroundjoin%
\pgfsetlinewidth{0.803000pt}%
\definecolor{currentstroke}{rgb}{0.000000,0.356863,0.509804}%
\pgfsetstrokecolor{currentstroke}%
\pgfsetdash{{2.960000pt}{1.280000pt}}{0.000000pt}%
\pgfpathmoveto{\pgfqpoint{0.631877in}{0.498966in}}%
\pgfpathlineto{\pgfqpoint{0.729731in}{0.490730in}}%
\pgfpathlineto{\pgfqpoint{0.827585in}{0.490730in}}%
\pgfpathlineto{\pgfqpoint{0.925438in}{0.490730in}}%
\pgfpathlineto{\pgfqpoint{1.023292in}{0.490730in}}%
\pgfpathlineto{\pgfqpoint{1.121145in}{0.498966in}}%
\pgfpathlineto{\pgfqpoint{1.218999in}{0.581319in}}%
\pgfpathlineto{\pgfqpoint{1.316852in}{0.886025in}}%
\pgfpathlineto{\pgfqpoint{1.414706in}{0.968377in}}%
\pgfpathlineto{\pgfqpoint{1.512559in}{1.396613in}}%
\pgfpathlineto{\pgfqpoint{1.610413in}{1.693083in}}%
\pgfpathlineto{\pgfqpoint{1.708266in}{2.055436in}}%
\pgfpathlineto{\pgfqpoint{1.806120in}{2.277789in}}%
\pgfpathlineto{\pgfqpoint{1.903973in}{2.417789in}}%
\pgfpathlineto{\pgfqpoint{2.001827in}{2.467201in}}%
\pgfpathlineto{\pgfqpoint{2.099681in}{2.516613in}}%
\pgfpathlineto{\pgfqpoint{2.197534in}{2.533083in}}%
\pgfpathlineto{\pgfqpoint{2.295388in}{2.549554in}}%
\pgfpathlineto{\pgfqpoint{2.393241in}{2.566025in}}%
\pgfpathlineto{\pgfqpoint{2.491095in}{2.574260in}}%
\pgfpathlineto{\pgfqpoint{2.588948in}{2.582495in}}%
\pgfpathlineto{\pgfqpoint{2.686802in}{2.582495in}}%
\pgfpathlineto{\pgfqpoint{2.784655in}{2.590730in}}%
\pgfpathlineto{\pgfqpoint{2.882509in}{2.582495in}}%
\pgfpathlineto{\pgfqpoint{2.980362in}{2.590730in}}%
\pgfpathlineto{\pgfqpoint{3.078216in}{2.590730in}}%
\pgfpathlineto{\pgfqpoint{3.176069in}{2.590730in}}%
\pgfpathlineto{\pgfqpoint{3.273923in}{2.590730in}}%
\pgfpathlineto{\pgfqpoint{3.371776in}{2.590730in}}%
\pgfpathlineto{\pgfqpoint{3.469630in}{2.590730in}}%
\pgfpathlineto{\pgfqpoint{3.567484in}{2.590730in}}%
\pgfpathlineto{\pgfqpoint{3.665337in}{2.590730in}}%
\pgfpathlineto{\pgfqpoint{3.763191in}{2.590730in}}%
\pgfpathlineto{\pgfqpoint{3.861044in}{2.590730in}}%
\pgfpathlineto{\pgfqpoint{3.958898in}{2.590730in}}%
\pgfpathlineto{\pgfqpoint{4.056751in}{2.590730in}}%
\pgfpathlineto{\pgfqpoint{4.154605in}{2.590730in}}%
\pgfusepath{stroke}%
\end{pgfscope}%
\begin{pgfscope}%
\pgfpathrectangle{\pgfqpoint{0.455741in}{0.385730in}}{\pgfqpoint{3.875000in}{2.310000in}}%
\pgfusepath{clip}%
\pgfsetbuttcap%
\pgfsetroundjoin%
\pgfsetlinewidth{0.803000pt}%
\definecolor{currentstroke}{rgb}{0.490196,0.588235,0.431373}%
\pgfsetstrokecolor{currentstroke}%
\pgfsetdash{{2.960000pt}{1.280000pt}}{0.000000pt}%
\pgfpathmoveto{\pgfqpoint{0.631877in}{0.498966in}}%
\pgfpathlineto{\pgfqpoint{0.729731in}{0.490730in}}%
\pgfpathlineto{\pgfqpoint{0.827585in}{0.490730in}}%
\pgfpathlineto{\pgfqpoint{0.925438in}{0.490730in}}%
\pgfpathlineto{\pgfqpoint{1.023292in}{0.498966in}}%
\pgfpathlineto{\pgfqpoint{1.121145in}{0.515436in}}%
\pgfpathlineto{\pgfqpoint{1.218999in}{0.614260in}}%
\pgfpathlineto{\pgfqpoint{1.316852in}{0.918966in}}%
\pgfpathlineto{\pgfqpoint{1.414706in}{1.075436in}}%
\pgfpathlineto{\pgfqpoint{1.512559in}{1.520142in}}%
\pgfpathlineto{\pgfqpoint{1.610413in}{1.767201in}}%
\pgfpathlineto{\pgfqpoint{1.708266in}{2.129554in}}%
\pgfpathlineto{\pgfqpoint{1.806120in}{2.318966in}}%
\pgfpathlineto{\pgfqpoint{1.903973in}{2.434260in}}%
\pgfpathlineto{\pgfqpoint{2.001827in}{2.458966in}}%
\pgfpathlineto{\pgfqpoint{2.099681in}{2.516613in}}%
\pgfpathlineto{\pgfqpoint{2.197534in}{2.533083in}}%
\pgfpathlineto{\pgfqpoint{2.295388in}{2.549554in}}%
\pgfpathlineto{\pgfqpoint{2.393241in}{2.566025in}}%
\pgfpathlineto{\pgfqpoint{2.491095in}{2.574260in}}%
\pgfpathlineto{\pgfqpoint{2.588948in}{2.574260in}}%
\pgfpathlineto{\pgfqpoint{2.686802in}{2.574260in}}%
\pgfpathlineto{\pgfqpoint{2.784655in}{2.582495in}}%
\pgfpathlineto{\pgfqpoint{2.882509in}{2.590730in}}%
\pgfpathlineto{\pgfqpoint{2.980362in}{2.590730in}}%
\pgfpathlineto{\pgfqpoint{3.078216in}{2.590730in}}%
\pgfpathlineto{\pgfqpoint{3.176069in}{2.590730in}}%
\pgfpathlineto{\pgfqpoint{3.273923in}{2.590730in}}%
\pgfpathlineto{\pgfqpoint{3.371776in}{2.590730in}}%
\pgfpathlineto{\pgfqpoint{3.469630in}{2.590730in}}%
\pgfpathlineto{\pgfqpoint{3.567484in}{2.590730in}}%
\pgfpathlineto{\pgfqpoint{3.665337in}{2.590730in}}%
\pgfpathlineto{\pgfqpoint{3.763191in}{2.590730in}}%
\pgfpathlineto{\pgfqpoint{3.861044in}{2.590730in}}%
\pgfpathlineto{\pgfqpoint{3.958898in}{2.590730in}}%
\pgfpathlineto{\pgfqpoint{4.056751in}{2.590730in}}%
\pgfpathlineto{\pgfqpoint{4.154605in}{2.590730in}}%
\pgfusepath{stroke}%
\end{pgfscope}%
\begin{pgfscope}%
\pgfpathrectangle{\pgfqpoint{0.455741in}{0.385730in}}{\pgfqpoint{3.875000in}{2.310000in}}%
\pgfusepath{clip}%
\pgfsetbuttcap%
\pgfsetroundjoin%
\pgfsetlinewidth{0.803000pt}%
\definecolor{currentstroke}{rgb}{0.843137,0.666667,0.313725}%
\pgfsetstrokecolor{currentstroke}%
\pgfsetdash{{2.960000pt}{1.280000pt}}{0.000000pt}%
\pgfpathmoveto{\pgfqpoint{0.631877in}{0.498966in}}%
\pgfpathlineto{\pgfqpoint{0.729731in}{0.490730in}}%
\pgfpathlineto{\pgfqpoint{0.827585in}{0.490730in}}%
\pgfpathlineto{\pgfqpoint{0.925438in}{0.490730in}}%
\pgfpathlineto{\pgfqpoint{1.023292in}{0.515436in}}%
\pgfpathlineto{\pgfqpoint{1.121145in}{0.564848in}}%
\pgfpathlineto{\pgfqpoint{1.218999in}{0.820142in}}%
\pgfpathlineto{\pgfqpoint{1.316852in}{1.273083in}}%
\pgfpathlineto{\pgfqpoint{1.414706in}{1.511907in}}%
\pgfpathlineto{\pgfqpoint{1.512559in}{1.931907in}}%
\pgfpathlineto{\pgfqpoint{1.610413in}{2.113083in}}%
\pgfpathlineto{\pgfqpoint{1.708266in}{2.327201in}}%
\pgfpathlineto{\pgfqpoint{1.806120in}{2.426025in}}%
\pgfpathlineto{\pgfqpoint{1.903973in}{2.475436in}}%
\pgfpathlineto{\pgfqpoint{2.001827in}{2.508377in}}%
\pgfpathlineto{\pgfqpoint{2.099681in}{2.533083in}}%
\pgfpathlineto{\pgfqpoint{2.197534in}{2.549554in}}%
\pgfpathlineto{\pgfqpoint{2.295388in}{2.566025in}}%
\pgfpathlineto{\pgfqpoint{2.393241in}{2.574260in}}%
\pgfpathlineto{\pgfqpoint{2.491095in}{2.574260in}}%
\pgfpathlineto{\pgfqpoint{2.588948in}{2.582495in}}%
\pgfpathlineto{\pgfqpoint{2.686802in}{2.582495in}}%
\pgfpathlineto{\pgfqpoint{2.784655in}{2.590730in}}%
\pgfpathlineto{\pgfqpoint{2.882509in}{2.590730in}}%
\pgfpathlineto{\pgfqpoint{2.980362in}{2.590730in}}%
\pgfpathlineto{\pgfqpoint{3.078216in}{2.590730in}}%
\pgfpathlineto{\pgfqpoint{3.176069in}{2.590730in}}%
\pgfpathlineto{\pgfqpoint{3.273923in}{2.590730in}}%
\pgfpathlineto{\pgfqpoint{3.371776in}{2.590730in}}%
\pgfpathlineto{\pgfqpoint{3.469630in}{2.590730in}}%
\pgfpathlineto{\pgfqpoint{3.567484in}{2.590730in}}%
\pgfpathlineto{\pgfqpoint{3.665337in}{2.590730in}}%
\pgfpathlineto{\pgfqpoint{3.763191in}{2.590730in}}%
\pgfpathlineto{\pgfqpoint{3.861044in}{2.590730in}}%
\pgfpathlineto{\pgfqpoint{3.958898in}{2.590730in}}%
\pgfpathlineto{\pgfqpoint{4.056751in}{2.590730in}}%
\pgfpathlineto{\pgfqpoint{4.154605in}{2.590730in}}%
\pgfusepath{stroke}%
\end{pgfscope}%
\begin{pgfscope}%
\pgfpathrectangle{\pgfqpoint{0.455741in}{0.385730in}}{\pgfqpoint{3.875000in}{2.310000in}}%
\pgfusepath{clip}%
\pgfsetbuttcap%
\pgfsetroundjoin%
\pgfsetlinewidth{0.803000pt}%
\definecolor{currentstroke}{rgb}{0.333333,0.333333,0.333333}%
\pgfsetstrokecolor{currentstroke}%
\pgfsetdash{{2.960000pt}{1.280000pt}}{0.000000pt}%
\pgfpathmoveto{\pgfqpoint{0.631877in}{0.498966in}}%
\pgfpathlineto{\pgfqpoint{0.729731in}{0.490730in}}%
\pgfpathlineto{\pgfqpoint{0.827585in}{0.490730in}}%
\pgfpathlineto{\pgfqpoint{0.925438in}{0.490730in}}%
\pgfpathlineto{\pgfqpoint{1.023292in}{0.507201in}}%
\pgfpathlineto{\pgfqpoint{1.121145in}{0.548377in}}%
\pgfpathlineto{\pgfqpoint{1.218999in}{0.729554in}}%
\pgfpathlineto{\pgfqpoint{1.316852in}{1.116613in}}%
\pgfpathlineto{\pgfqpoint{1.414706in}{1.330730in}}%
\pgfpathlineto{\pgfqpoint{1.512559in}{1.734260in}}%
\pgfpathlineto{\pgfqpoint{1.610413in}{1.973083in}}%
\pgfpathlineto{\pgfqpoint{1.708266in}{2.236613in}}%
\pgfpathlineto{\pgfqpoint{1.806120in}{2.360142in}}%
\pgfpathlineto{\pgfqpoint{1.903973in}{2.450730in}}%
\pgfpathlineto{\pgfqpoint{2.001827in}{2.483672in}}%
\pgfpathlineto{\pgfqpoint{2.099681in}{2.516613in}}%
\pgfpathlineto{\pgfqpoint{2.197534in}{2.541319in}}%
\pgfpathlineto{\pgfqpoint{2.295388in}{2.557789in}}%
\pgfpathlineto{\pgfqpoint{2.393241in}{2.574260in}}%
\pgfpathlineto{\pgfqpoint{2.491095in}{2.574260in}}%
\pgfpathlineto{\pgfqpoint{2.588948in}{2.582495in}}%
\pgfpathlineto{\pgfqpoint{2.686802in}{2.590730in}}%
\pgfpathlineto{\pgfqpoint{2.784655in}{2.590730in}}%
\pgfpathlineto{\pgfqpoint{2.882509in}{2.590730in}}%
\pgfpathlineto{\pgfqpoint{2.980362in}{2.590730in}}%
\pgfpathlineto{\pgfqpoint{3.078216in}{2.590730in}}%
\pgfpathlineto{\pgfqpoint{3.176069in}{2.590730in}}%
\pgfpathlineto{\pgfqpoint{3.273923in}{2.590730in}}%
\pgfpathlineto{\pgfqpoint{3.371776in}{2.590730in}}%
\pgfpathlineto{\pgfqpoint{3.469630in}{2.590730in}}%
\pgfpathlineto{\pgfqpoint{3.567484in}{2.590730in}}%
\pgfpathlineto{\pgfqpoint{3.665337in}{2.590730in}}%
\pgfpathlineto{\pgfqpoint{3.763191in}{2.590730in}}%
\pgfpathlineto{\pgfqpoint{3.861044in}{2.590730in}}%
\pgfpathlineto{\pgfqpoint{3.958898in}{2.590730in}}%
\pgfpathlineto{\pgfqpoint{4.056751in}{2.590730in}}%
\pgfpathlineto{\pgfqpoint{4.154605in}{2.590730in}}%
\pgfusepath{stroke}%
\end{pgfscope}%
\begin{pgfscope}%
\pgfpathrectangle{\pgfqpoint{0.455741in}{0.385730in}}{\pgfqpoint{3.875000in}{2.310000in}}%
\pgfusepath{clip}%
\pgfsetbuttcap%
\pgfsetroundjoin%
\pgfsetlinewidth{0.803000pt}%
\definecolor{currentstroke}{rgb}{0.686275,0.352941,0.313725}%
\pgfsetstrokecolor{currentstroke}%
\pgfsetdash{{2.960000pt}{1.280000pt}}{0.000000pt}%
\pgfpathmoveto{\pgfqpoint{0.631877in}{0.490730in}}%
\pgfpathlineto{\pgfqpoint{0.729731in}{0.490730in}}%
\pgfpathlineto{\pgfqpoint{0.827585in}{0.490730in}}%
\pgfpathlineto{\pgfqpoint{0.925438in}{0.490730in}}%
\pgfpathlineto{\pgfqpoint{1.023292in}{0.490730in}}%
\pgfpathlineto{\pgfqpoint{1.121145in}{0.498966in}}%
\pgfpathlineto{\pgfqpoint{1.218999in}{0.564848in}}%
\pgfpathlineto{\pgfqpoint{1.316852in}{0.787201in}}%
\pgfpathlineto{\pgfqpoint{1.414706in}{0.902495in}}%
\pgfpathlineto{\pgfqpoint{1.512559in}{1.273083in}}%
\pgfpathlineto{\pgfqpoint{1.610413in}{1.643672in}}%
\pgfpathlineto{\pgfqpoint{1.708266in}{1.989554in}}%
\pgfpathlineto{\pgfqpoint{1.806120in}{2.195436in}}%
\pgfpathlineto{\pgfqpoint{1.903973in}{2.360142in}}%
\pgfpathlineto{\pgfqpoint{2.001827in}{2.434260in}}%
\pgfpathlineto{\pgfqpoint{2.099681in}{2.500142in}}%
\pgfpathlineto{\pgfqpoint{2.197534in}{2.524848in}}%
\pgfpathlineto{\pgfqpoint{2.295388in}{2.541319in}}%
\pgfpathlineto{\pgfqpoint{2.393241in}{2.557789in}}%
\pgfpathlineto{\pgfqpoint{2.491095in}{2.566025in}}%
\pgfpathlineto{\pgfqpoint{2.588948in}{2.574260in}}%
\pgfpathlineto{\pgfqpoint{2.686802in}{2.574260in}}%
\pgfpathlineto{\pgfqpoint{2.784655in}{2.582495in}}%
\pgfpathlineto{\pgfqpoint{2.882509in}{2.582495in}}%
\pgfpathlineto{\pgfqpoint{2.980362in}{2.590730in}}%
\pgfpathlineto{\pgfqpoint{3.078216in}{2.590730in}}%
\pgfpathlineto{\pgfqpoint{3.176069in}{2.590730in}}%
\pgfpathlineto{\pgfqpoint{3.273923in}{2.590730in}}%
\pgfpathlineto{\pgfqpoint{3.371776in}{2.590730in}}%
\pgfpathlineto{\pgfqpoint{3.469630in}{2.590730in}}%
\pgfpathlineto{\pgfqpoint{3.567484in}{2.590730in}}%
\pgfpathlineto{\pgfqpoint{3.665337in}{2.590730in}}%
\pgfpathlineto{\pgfqpoint{3.763191in}{2.590730in}}%
\pgfpathlineto{\pgfqpoint{3.861044in}{2.590730in}}%
\pgfpathlineto{\pgfqpoint{3.958898in}{2.590730in}}%
\pgfpathlineto{\pgfqpoint{4.056751in}{2.590730in}}%
\pgfpathlineto{\pgfqpoint{4.154605in}{2.590730in}}%
\pgfusepath{stroke}%
\end{pgfscope}%
\begin{pgfscope}%
\pgfpathrectangle{\pgfqpoint{0.455741in}{0.385730in}}{\pgfqpoint{3.875000in}{2.310000in}}%
\pgfusepath{clip}%
\pgfsetrectcap%
\pgfsetroundjoin%
\pgfsetlinewidth{0.803000pt}%
\definecolor{currentstroke}{rgb}{0.000000,0.356863,0.509804}%
\pgfsetstrokecolor{currentstroke}%
\pgfsetdash{}{0pt}%
\pgfpathmoveto{\pgfqpoint{0.631877in}{0.490730in}}%
\pgfpathlineto{\pgfqpoint{0.729731in}{0.490730in}}%
\pgfpathlineto{\pgfqpoint{0.827585in}{0.490730in}}%
\pgfpathlineto{\pgfqpoint{0.925438in}{0.490730in}}%
\pgfpathlineto{\pgfqpoint{1.023292in}{0.490730in}}%
\pgfpathlineto{\pgfqpoint{1.121145in}{0.490730in}}%
\pgfpathlineto{\pgfqpoint{1.218999in}{0.490730in}}%
\pgfpathlineto{\pgfqpoint{1.316852in}{0.490730in}}%
\pgfpathlineto{\pgfqpoint{1.414706in}{0.490730in}}%
\pgfpathlineto{\pgfqpoint{1.512559in}{0.490730in}}%
\pgfpathlineto{\pgfqpoint{1.610413in}{0.490730in}}%
\pgfpathlineto{\pgfqpoint{1.708266in}{0.490730in}}%
\pgfpathlineto{\pgfqpoint{1.806120in}{0.490730in}}%
\pgfpathlineto{\pgfqpoint{1.903973in}{0.490730in}}%
\pgfpathlineto{\pgfqpoint{2.001827in}{0.515436in}}%
\pgfpathlineto{\pgfqpoint{2.099681in}{0.540142in}}%
\pgfpathlineto{\pgfqpoint{2.197534in}{0.680142in}}%
\pgfpathlineto{\pgfqpoint{2.295388in}{0.836613in}}%
\pgfpathlineto{\pgfqpoint{2.393241in}{1.174260in}}%
\pgfpathlineto{\pgfqpoint{2.491095in}{1.676613in}}%
\pgfpathlineto{\pgfqpoint{2.588948in}{1.973083in}}%
\pgfpathlineto{\pgfqpoint{2.686802in}{2.302495in}}%
\pgfpathlineto{\pgfqpoint{2.784655in}{2.417789in}}%
\pgfpathlineto{\pgfqpoint{2.882509in}{2.500142in}}%
\pgfpathlineto{\pgfqpoint{2.980362in}{2.516613in}}%
\pgfpathlineto{\pgfqpoint{3.078216in}{2.541319in}}%
\pgfpathlineto{\pgfqpoint{3.176069in}{2.557789in}}%
\pgfpathlineto{\pgfqpoint{3.273923in}{2.566025in}}%
\pgfpathlineto{\pgfqpoint{3.371776in}{2.574260in}}%
\pgfpathlineto{\pgfqpoint{3.469630in}{2.582495in}}%
\pgfpathlineto{\pgfqpoint{3.567484in}{2.590730in}}%
\pgfpathlineto{\pgfqpoint{3.665337in}{2.590730in}}%
\pgfpathlineto{\pgfqpoint{3.763191in}{2.590730in}}%
\pgfpathlineto{\pgfqpoint{3.861044in}{2.590730in}}%
\pgfpathlineto{\pgfqpoint{3.958898in}{2.590730in}}%
\pgfpathlineto{\pgfqpoint{4.056751in}{2.590730in}}%
\pgfpathlineto{\pgfqpoint{4.154605in}{2.590730in}}%
\pgfusepath{stroke}%
\end{pgfscope}%
\begin{pgfscope}%
\pgfpathrectangle{\pgfqpoint{0.455741in}{0.385730in}}{\pgfqpoint{3.875000in}{2.310000in}}%
\pgfusepath{clip}%
\pgfsetrectcap%
\pgfsetroundjoin%
\pgfsetlinewidth{0.803000pt}%
\definecolor{currentstroke}{rgb}{0.490196,0.588235,0.431373}%
\pgfsetstrokecolor{currentstroke}%
\pgfsetdash{}{0pt}%
\pgfpathmoveto{\pgfqpoint{0.631877in}{0.490730in}}%
\pgfpathlineto{\pgfqpoint{0.729731in}{0.490730in}}%
\pgfpathlineto{\pgfqpoint{0.827585in}{0.490730in}}%
\pgfpathlineto{\pgfqpoint{0.925438in}{0.490730in}}%
\pgfpathlineto{\pgfqpoint{1.023292in}{0.490730in}}%
\pgfpathlineto{\pgfqpoint{1.121145in}{0.490730in}}%
\pgfpathlineto{\pgfqpoint{1.218999in}{0.490730in}}%
\pgfpathlineto{\pgfqpoint{1.316852in}{0.490730in}}%
\pgfpathlineto{\pgfqpoint{1.414706in}{0.490730in}}%
\pgfpathlineto{\pgfqpoint{1.512559in}{0.490730in}}%
\pgfpathlineto{\pgfqpoint{1.610413in}{0.490730in}}%
\pgfpathlineto{\pgfqpoint{1.708266in}{0.490730in}}%
\pgfpathlineto{\pgfqpoint{1.806120in}{0.490730in}}%
\pgfpathlineto{\pgfqpoint{1.903973in}{0.490730in}}%
\pgfpathlineto{\pgfqpoint{2.001827in}{0.507201in}}%
\pgfpathlineto{\pgfqpoint{2.099681in}{0.523672in}}%
\pgfpathlineto{\pgfqpoint{2.197534in}{0.581319in}}%
\pgfpathlineto{\pgfqpoint{2.295388in}{0.655436in}}%
\pgfpathlineto{\pgfqpoint{2.393241in}{0.927201in}}%
\pgfpathlineto{\pgfqpoint{2.491095in}{1.594260in}}%
\pgfpathlineto{\pgfqpoint{2.588948in}{1.874260in}}%
\pgfpathlineto{\pgfqpoint{2.686802in}{2.162495in}}%
\pgfpathlineto{\pgfqpoint{2.784655in}{2.376613in}}%
\pgfpathlineto{\pgfqpoint{2.882509in}{2.458966in}}%
\pgfpathlineto{\pgfqpoint{2.980362in}{2.491907in}}%
\pgfpathlineto{\pgfqpoint{3.078216in}{2.541319in}}%
\pgfpathlineto{\pgfqpoint{3.176069in}{2.557789in}}%
\pgfpathlineto{\pgfqpoint{3.273923in}{2.574260in}}%
\pgfpathlineto{\pgfqpoint{3.371776in}{2.582495in}}%
\pgfpathlineto{\pgfqpoint{3.469630in}{2.590730in}}%
\pgfpathlineto{\pgfqpoint{3.567484in}{2.590730in}}%
\pgfpathlineto{\pgfqpoint{3.665337in}{2.590730in}}%
\pgfpathlineto{\pgfqpoint{3.763191in}{2.590730in}}%
\pgfpathlineto{\pgfqpoint{3.861044in}{2.590730in}}%
\pgfpathlineto{\pgfqpoint{3.958898in}{2.590730in}}%
\pgfpathlineto{\pgfqpoint{4.056751in}{2.590730in}}%
\pgfpathlineto{\pgfqpoint{4.154605in}{2.590730in}}%
\pgfusepath{stroke}%
\end{pgfscope}%
\begin{pgfscope}%
\pgfpathrectangle{\pgfqpoint{0.455741in}{0.385730in}}{\pgfqpoint{3.875000in}{2.310000in}}%
\pgfusepath{clip}%
\pgfsetrectcap%
\pgfsetroundjoin%
\pgfsetlinewidth{0.803000pt}%
\definecolor{currentstroke}{rgb}{0.843137,0.666667,0.313725}%
\pgfsetstrokecolor{currentstroke}%
\pgfsetdash{}{0pt}%
\pgfpathmoveto{\pgfqpoint{0.631877in}{0.490730in}}%
\pgfpathlineto{\pgfqpoint{0.729731in}{0.490730in}}%
\pgfpathlineto{\pgfqpoint{0.827585in}{0.490730in}}%
\pgfpathlineto{\pgfqpoint{0.925438in}{0.490730in}}%
\pgfpathlineto{\pgfqpoint{1.023292in}{0.490730in}}%
\pgfpathlineto{\pgfqpoint{1.121145in}{0.490730in}}%
\pgfpathlineto{\pgfqpoint{1.218999in}{0.490730in}}%
\pgfpathlineto{\pgfqpoint{1.316852in}{0.490730in}}%
\pgfpathlineto{\pgfqpoint{1.414706in}{0.490730in}}%
\pgfpathlineto{\pgfqpoint{1.512559in}{0.490730in}}%
\pgfpathlineto{\pgfqpoint{1.610413in}{0.490730in}}%
\pgfpathlineto{\pgfqpoint{1.708266in}{0.490730in}}%
\pgfpathlineto{\pgfqpoint{1.806120in}{0.490730in}}%
\pgfpathlineto{\pgfqpoint{1.903973in}{0.498966in}}%
\pgfpathlineto{\pgfqpoint{2.001827in}{0.540142in}}%
\pgfpathlineto{\pgfqpoint{2.099681in}{0.581319in}}%
\pgfpathlineto{\pgfqpoint{2.197534in}{0.713083in}}%
\pgfpathlineto{\pgfqpoint{2.295388in}{0.844848in}}%
\pgfpathlineto{\pgfqpoint{2.393241in}{1.124848in}}%
\pgfpathlineto{\pgfqpoint{2.491095in}{1.701319in}}%
\pgfpathlineto{\pgfqpoint{2.588948in}{2.047201in}}%
\pgfpathlineto{\pgfqpoint{2.686802in}{2.351907in}}%
\pgfpathlineto{\pgfqpoint{2.784655in}{2.426025in}}%
\pgfpathlineto{\pgfqpoint{2.882509in}{2.491907in}}%
\pgfpathlineto{\pgfqpoint{2.980362in}{2.524848in}}%
\pgfpathlineto{\pgfqpoint{3.078216in}{2.549554in}}%
\pgfpathlineto{\pgfqpoint{3.176069in}{2.557789in}}%
\pgfpathlineto{\pgfqpoint{3.273923in}{2.574260in}}%
\pgfpathlineto{\pgfqpoint{3.371776in}{2.582495in}}%
\pgfpathlineto{\pgfqpoint{3.469630in}{2.590730in}}%
\pgfpathlineto{\pgfqpoint{3.567484in}{2.590730in}}%
\pgfpathlineto{\pgfqpoint{3.665337in}{2.590730in}}%
\pgfpathlineto{\pgfqpoint{3.763191in}{2.590730in}}%
\pgfpathlineto{\pgfqpoint{3.861044in}{2.590730in}}%
\pgfpathlineto{\pgfqpoint{3.958898in}{2.590730in}}%
\pgfpathlineto{\pgfqpoint{4.056751in}{2.590730in}}%
\pgfpathlineto{\pgfqpoint{4.154605in}{2.590730in}}%
\pgfusepath{stroke}%
\end{pgfscope}%
\begin{pgfscope}%
\pgfpathrectangle{\pgfqpoint{0.455741in}{0.385730in}}{\pgfqpoint{3.875000in}{2.310000in}}%
\pgfusepath{clip}%
\pgfsetrectcap%
\pgfsetroundjoin%
\pgfsetlinewidth{0.803000pt}%
\definecolor{currentstroke}{rgb}{0.333333,0.333333,0.333333}%
\pgfsetstrokecolor{currentstroke}%
\pgfsetdash{}{0pt}%
\pgfpathmoveto{\pgfqpoint{0.631877in}{0.490730in}}%
\pgfpathlineto{\pgfqpoint{0.729731in}{0.490730in}}%
\pgfpathlineto{\pgfqpoint{0.827585in}{0.490730in}}%
\pgfpathlineto{\pgfqpoint{0.925438in}{0.490730in}}%
\pgfpathlineto{\pgfqpoint{1.023292in}{0.490730in}}%
\pgfpathlineto{\pgfqpoint{1.121145in}{0.490730in}}%
\pgfpathlineto{\pgfqpoint{1.218999in}{0.490730in}}%
\pgfpathlineto{\pgfqpoint{1.316852in}{0.490730in}}%
\pgfpathlineto{\pgfqpoint{1.414706in}{0.490730in}}%
\pgfpathlineto{\pgfqpoint{1.512559in}{0.490730in}}%
\pgfpathlineto{\pgfqpoint{1.610413in}{0.490730in}}%
\pgfpathlineto{\pgfqpoint{1.708266in}{0.490730in}}%
\pgfpathlineto{\pgfqpoint{1.806120in}{0.490730in}}%
\pgfpathlineto{\pgfqpoint{1.903973in}{0.490730in}}%
\pgfpathlineto{\pgfqpoint{2.001827in}{0.523672in}}%
\pgfpathlineto{\pgfqpoint{2.099681in}{0.556613in}}%
\pgfpathlineto{\pgfqpoint{2.197534in}{0.680142in}}%
\pgfpathlineto{\pgfqpoint{2.295388in}{0.820142in}}%
\pgfpathlineto{\pgfqpoint{2.393241in}{1.124848in}}%
\pgfpathlineto{\pgfqpoint{2.491095in}{1.791907in}}%
\pgfpathlineto{\pgfqpoint{2.588948in}{2.187201in}}%
\pgfpathlineto{\pgfqpoint{2.686802in}{2.376613in}}%
\pgfpathlineto{\pgfqpoint{2.784655in}{2.434260in}}%
\pgfpathlineto{\pgfqpoint{2.882509in}{2.491907in}}%
\pgfpathlineto{\pgfqpoint{2.980362in}{2.516613in}}%
\pgfpathlineto{\pgfqpoint{3.078216in}{2.541319in}}%
\pgfpathlineto{\pgfqpoint{3.176069in}{2.566025in}}%
\pgfpathlineto{\pgfqpoint{3.273923in}{2.566025in}}%
\pgfpathlineto{\pgfqpoint{3.371776in}{2.574260in}}%
\pgfpathlineto{\pgfqpoint{3.469630in}{2.574260in}}%
\pgfpathlineto{\pgfqpoint{3.567484in}{2.574260in}}%
\pgfpathlineto{\pgfqpoint{3.665337in}{2.590730in}}%
\pgfpathlineto{\pgfqpoint{3.763191in}{2.590730in}}%
\pgfpathlineto{\pgfqpoint{3.861044in}{2.590730in}}%
\pgfpathlineto{\pgfqpoint{3.958898in}{2.590730in}}%
\pgfpathlineto{\pgfqpoint{4.056751in}{2.590730in}}%
\pgfpathlineto{\pgfqpoint{4.154605in}{2.590730in}}%
\pgfusepath{stroke}%
\end{pgfscope}%
\begin{pgfscope}%
\pgfpathrectangle{\pgfqpoint{0.455741in}{0.385730in}}{\pgfqpoint{3.875000in}{2.310000in}}%
\pgfusepath{clip}%
\pgfsetrectcap%
\pgfsetroundjoin%
\pgfsetlinewidth{0.803000pt}%
\definecolor{currentstroke}{rgb}{0.686275,0.352941,0.313725}%
\pgfsetstrokecolor{currentstroke}%
\pgfsetdash{}{0pt}%
\pgfpathmoveto{\pgfqpoint{0.631877in}{0.490730in}}%
\pgfpathlineto{\pgfqpoint{0.729731in}{0.490730in}}%
\pgfpathlineto{\pgfqpoint{0.827585in}{0.490730in}}%
\pgfpathlineto{\pgfqpoint{0.925438in}{0.490730in}}%
\pgfpathlineto{\pgfqpoint{1.023292in}{0.490730in}}%
\pgfpathlineto{\pgfqpoint{1.121145in}{0.490730in}}%
\pgfpathlineto{\pgfqpoint{1.218999in}{0.490730in}}%
\pgfpathlineto{\pgfqpoint{1.316852in}{0.490730in}}%
\pgfpathlineto{\pgfqpoint{1.414706in}{0.490730in}}%
\pgfpathlineto{\pgfqpoint{1.512559in}{0.490730in}}%
\pgfpathlineto{\pgfqpoint{1.610413in}{0.490730in}}%
\pgfpathlineto{\pgfqpoint{1.708266in}{0.490730in}}%
\pgfpathlineto{\pgfqpoint{1.806120in}{0.490730in}}%
\pgfpathlineto{\pgfqpoint{1.903973in}{0.490730in}}%
\pgfpathlineto{\pgfqpoint{2.001827in}{0.498966in}}%
\pgfpathlineto{\pgfqpoint{2.099681in}{0.507201in}}%
\pgfpathlineto{\pgfqpoint{2.197534in}{0.573083in}}%
\pgfpathlineto{\pgfqpoint{2.295388in}{0.778966in}}%
\pgfpathlineto{\pgfqpoint{2.393241in}{1.091907in}}%
\pgfpathlineto{\pgfqpoint{2.491095in}{1.701319in}}%
\pgfpathlineto{\pgfqpoint{2.588948in}{1.931907in}}%
\pgfpathlineto{\pgfqpoint{2.686802in}{2.195436in}}%
\pgfpathlineto{\pgfqpoint{2.784655in}{2.343672in}}%
\pgfpathlineto{\pgfqpoint{2.882509in}{2.450730in}}%
\pgfpathlineto{\pgfqpoint{2.980362in}{2.483672in}}%
\pgfpathlineto{\pgfqpoint{3.078216in}{2.524848in}}%
\pgfpathlineto{\pgfqpoint{3.176069in}{2.549554in}}%
\pgfpathlineto{\pgfqpoint{3.273923in}{2.566025in}}%
\pgfpathlineto{\pgfqpoint{3.371776in}{2.574260in}}%
\pgfpathlineto{\pgfqpoint{3.469630in}{2.590730in}}%
\pgfpathlineto{\pgfqpoint{3.567484in}{2.590730in}}%
\pgfpathlineto{\pgfqpoint{3.665337in}{2.590730in}}%
\pgfpathlineto{\pgfqpoint{3.763191in}{2.590730in}}%
\pgfpathlineto{\pgfqpoint{3.861044in}{2.590730in}}%
\pgfpathlineto{\pgfqpoint{3.958898in}{2.590730in}}%
\pgfpathlineto{\pgfqpoint{4.056751in}{2.590730in}}%
\pgfpathlineto{\pgfqpoint{4.154605in}{2.590730in}}%
\pgfusepath{stroke}%
\end{pgfscope}%
\begin{pgfscope}%
\pgfpathrectangle{\pgfqpoint{0.455741in}{0.385730in}}{\pgfqpoint{3.875000in}{2.310000in}}%
\pgfusepath{clip}%
\pgfsetrectcap%
\pgfsetroundjoin%
\pgfsetlinewidth{0.803000pt}%
\definecolor{currentstroke}{rgb}{0.000000,0.356863,0.509804}%
\pgfsetstrokecolor{currentstroke}%
\pgfsetdash{}{0pt}%
\pgfpathmoveto{\pgfqpoint{0.631877in}{0.490730in}}%
\pgfpathlineto{\pgfqpoint{0.729731in}{0.490730in}}%
\pgfpathlineto{\pgfqpoint{0.827585in}{0.490730in}}%
\pgfpathlineto{\pgfqpoint{0.925438in}{0.490730in}}%
\pgfpathlineto{\pgfqpoint{1.023292in}{0.490730in}}%
\pgfpathlineto{\pgfqpoint{1.121145in}{0.490730in}}%
\pgfpathlineto{\pgfqpoint{1.218999in}{0.490730in}}%
\pgfpathlineto{\pgfqpoint{1.316852in}{0.490730in}}%
\pgfpathlineto{\pgfqpoint{1.414706in}{0.490730in}}%
\pgfpathlineto{\pgfqpoint{1.512559in}{0.490730in}}%
\pgfpathlineto{\pgfqpoint{1.610413in}{0.490730in}}%
\pgfpathlineto{\pgfqpoint{1.708266in}{0.490730in}}%
\pgfpathlineto{\pgfqpoint{1.806120in}{0.490730in}}%
\pgfpathlineto{\pgfqpoint{1.903973in}{0.498966in}}%
\pgfpathlineto{\pgfqpoint{2.001827in}{0.531907in}}%
\pgfpathlineto{\pgfqpoint{2.099681in}{0.573083in}}%
\pgfpathlineto{\pgfqpoint{2.197534in}{0.729554in}}%
\pgfpathlineto{\pgfqpoint{2.295388in}{0.910730in}}%
\pgfpathlineto{\pgfqpoint{2.393241in}{1.248377in}}%
\pgfpathlineto{\pgfqpoint{2.491095in}{1.824848in}}%
\pgfpathlineto{\pgfqpoint{2.588948in}{2.162495in}}%
\pgfpathlineto{\pgfqpoint{2.686802in}{2.343672in}}%
\pgfpathlineto{\pgfqpoint{2.784655in}{2.417789in}}%
\pgfpathlineto{\pgfqpoint{2.882509in}{2.500142in}}%
\pgfpathlineto{\pgfqpoint{2.980362in}{2.524848in}}%
\pgfpathlineto{\pgfqpoint{3.078216in}{2.541319in}}%
\pgfpathlineto{\pgfqpoint{3.176069in}{2.557789in}}%
\pgfpathlineto{\pgfqpoint{3.273923in}{2.566025in}}%
\pgfpathlineto{\pgfqpoint{3.371776in}{2.574260in}}%
\pgfpathlineto{\pgfqpoint{3.469630in}{2.582495in}}%
\pgfpathlineto{\pgfqpoint{3.567484in}{2.590730in}}%
\pgfpathlineto{\pgfqpoint{3.665337in}{2.590730in}}%
\pgfpathlineto{\pgfqpoint{3.763191in}{2.590730in}}%
\pgfpathlineto{\pgfqpoint{3.861044in}{2.590730in}}%
\pgfpathlineto{\pgfqpoint{3.958898in}{2.590730in}}%
\pgfpathlineto{\pgfqpoint{4.056751in}{2.590730in}}%
\pgfpathlineto{\pgfqpoint{4.154605in}{2.590730in}}%
\pgfusepath{stroke}%
\end{pgfscope}%
\begin{pgfscope}%
\pgfpathrectangle{\pgfqpoint{0.455741in}{0.385730in}}{\pgfqpoint{3.875000in}{2.310000in}}%
\pgfusepath{clip}%
\pgfsetrectcap%
\pgfsetroundjoin%
\pgfsetlinewidth{0.803000pt}%
\definecolor{currentstroke}{rgb}{0.490196,0.588235,0.431373}%
\pgfsetstrokecolor{currentstroke}%
\pgfsetdash{}{0pt}%
\pgfpathmoveto{\pgfqpoint{0.631877in}{0.490730in}}%
\pgfpathlineto{\pgfqpoint{0.729731in}{0.490730in}}%
\pgfpathlineto{\pgfqpoint{0.827585in}{0.490730in}}%
\pgfpathlineto{\pgfqpoint{0.925438in}{0.490730in}}%
\pgfpathlineto{\pgfqpoint{1.023292in}{0.490730in}}%
\pgfpathlineto{\pgfqpoint{1.121145in}{0.490730in}}%
\pgfpathlineto{\pgfqpoint{1.218999in}{0.490730in}}%
\pgfpathlineto{\pgfqpoint{1.316852in}{0.490730in}}%
\pgfpathlineto{\pgfqpoint{1.414706in}{0.490730in}}%
\pgfpathlineto{\pgfqpoint{1.512559in}{0.490730in}}%
\pgfpathlineto{\pgfqpoint{1.610413in}{0.490730in}}%
\pgfpathlineto{\pgfqpoint{1.708266in}{0.490730in}}%
\pgfpathlineto{\pgfqpoint{1.806120in}{0.490730in}}%
\pgfpathlineto{\pgfqpoint{1.903973in}{0.490730in}}%
\pgfpathlineto{\pgfqpoint{2.001827in}{0.523672in}}%
\pgfpathlineto{\pgfqpoint{2.099681in}{0.540142in}}%
\pgfpathlineto{\pgfqpoint{2.197534in}{0.713083in}}%
\pgfpathlineto{\pgfqpoint{2.295388in}{0.877789in}}%
\pgfpathlineto{\pgfqpoint{2.393241in}{1.371907in}}%
\pgfpathlineto{\pgfqpoint{2.491095in}{1.898966in}}%
\pgfpathlineto{\pgfqpoint{2.588948in}{2.162495in}}%
\pgfpathlineto{\pgfqpoint{2.686802in}{2.343672in}}%
\pgfpathlineto{\pgfqpoint{2.784655in}{2.417789in}}%
\pgfpathlineto{\pgfqpoint{2.882509in}{2.491907in}}%
\pgfpathlineto{\pgfqpoint{2.980362in}{2.516613in}}%
\pgfpathlineto{\pgfqpoint{3.078216in}{2.549554in}}%
\pgfpathlineto{\pgfqpoint{3.176069in}{2.557789in}}%
\pgfpathlineto{\pgfqpoint{3.273923in}{2.574260in}}%
\pgfpathlineto{\pgfqpoint{3.371776in}{2.582495in}}%
\pgfpathlineto{\pgfqpoint{3.469630in}{2.590730in}}%
\pgfpathlineto{\pgfqpoint{3.567484in}{2.590730in}}%
\pgfpathlineto{\pgfqpoint{3.665337in}{2.590730in}}%
\pgfpathlineto{\pgfqpoint{3.763191in}{2.590730in}}%
\pgfpathlineto{\pgfqpoint{3.861044in}{2.590730in}}%
\pgfpathlineto{\pgfqpoint{3.958898in}{2.590730in}}%
\pgfpathlineto{\pgfqpoint{4.056751in}{2.590730in}}%
\pgfpathlineto{\pgfqpoint{4.154605in}{2.590730in}}%
\pgfusepath{stroke}%
\end{pgfscope}%
\begin{pgfscope}%
\pgfpathrectangle{\pgfqpoint{0.455741in}{0.385730in}}{\pgfqpoint{3.875000in}{2.310000in}}%
\pgfusepath{clip}%
\pgfsetrectcap%
\pgfsetroundjoin%
\pgfsetlinewidth{0.803000pt}%
\definecolor{currentstroke}{rgb}{0.843137,0.666667,0.313725}%
\pgfsetstrokecolor{currentstroke}%
\pgfsetdash{}{0pt}%
\pgfpathmoveto{\pgfqpoint{0.631877in}{0.490730in}}%
\pgfpathlineto{\pgfqpoint{0.729731in}{0.490730in}}%
\pgfpathlineto{\pgfqpoint{0.827585in}{0.490730in}}%
\pgfpathlineto{\pgfqpoint{0.925438in}{0.490730in}}%
\pgfpathlineto{\pgfqpoint{1.023292in}{0.490730in}}%
\pgfpathlineto{\pgfqpoint{1.121145in}{0.490730in}}%
\pgfpathlineto{\pgfqpoint{1.218999in}{0.490730in}}%
\pgfpathlineto{\pgfqpoint{1.316852in}{0.490730in}}%
\pgfpathlineto{\pgfqpoint{1.414706in}{0.490730in}}%
\pgfpathlineto{\pgfqpoint{1.512559in}{0.490730in}}%
\pgfpathlineto{\pgfqpoint{1.610413in}{0.490730in}}%
\pgfpathlineto{\pgfqpoint{1.708266in}{0.490730in}}%
\pgfpathlineto{\pgfqpoint{1.806120in}{0.490730in}}%
\pgfpathlineto{\pgfqpoint{1.903973in}{0.490730in}}%
\pgfpathlineto{\pgfqpoint{2.001827in}{0.523672in}}%
\pgfpathlineto{\pgfqpoint{2.099681in}{0.556613in}}%
\pgfpathlineto{\pgfqpoint{2.197534in}{0.713083in}}%
\pgfpathlineto{\pgfqpoint{2.295388in}{0.803672in}}%
\pgfpathlineto{\pgfqpoint{2.393241in}{1.338966in}}%
\pgfpathlineto{\pgfqpoint{2.491095in}{1.890730in}}%
\pgfpathlineto{\pgfqpoint{2.588948in}{2.006025in}}%
\pgfpathlineto{\pgfqpoint{2.686802in}{2.302495in}}%
\pgfpathlineto{\pgfqpoint{2.784655in}{2.409554in}}%
\pgfpathlineto{\pgfqpoint{2.882509in}{2.500142in}}%
\pgfpathlineto{\pgfqpoint{2.980362in}{2.524848in}}%
\pgfpathlineto{\pgfqpoint{3.078216in}{2.541319in}}%
\pgfpathlineto{\pgfqpoint{3.176069in}{2.557789in}}%
\pgfpathlineto{\pgfqpoint{3.273923in}{2.574260in}}%
\pgfpathlineto{\pgfqpoint{3.371776in}{2.582495in}}%
\pgfpathlineto{\pgfqpoint{3.469630in}{2.582495in}}%
\pgfpathlineto{\pgfqpoint{3.567484in}{2.590730in}}%
\pgfpathlineto{\pgfqpoint{3.665337in}{2.590730in}}%
\pgfpathlineto{\pgfqpoint{3.763191in}{2.590730in}}%
\pgfpathlineto{\pgfqpoint{3.861044in}{2.590730in}}%
\pgfpathlineto{\pgfqpoint{3.958898in}{2.590730in}}%
\pgfpathlineto{\pgfqpoint{4.056751in}{2.590730in}}%
\pgfpathlineto{\pgfqpoint{4.154605in}{2.590730in}}%
\pgfusepath{stroke}%
\end{pgfscope}%
\begin{pgfscope}%
\pgfpathrectangle{\pgfqpoint{0.455741in}{0.385730in}}{\pgfqpoint{3.875000in}{2.310000in}}%
\pgfusepath{clip}%
\pgfsetrectcap%
\pgfsetroundjoin%
\pgfsetlinewidth{0.803000pt}%
\definecolor{currentstroke}{rgb}{0.333333,0.333333,0.333333}%
\pgfsetstrokecolor{currentstroke}%
\pgfsetdash{}{0pt}%
\pgfpathmoveto{\pgfqpoint{0.631877in}{0.490730in}}%
\pgfpathlineto{\pgfqpoint{0.729731in}{0.490730in}}%
\pgfpathlineto{\pgfqpoint{0.827585in}{0.490730in}}%
\pgfpathlineto{\pgfqpoint{0.925438in}{0.490730in}}%
\pgfpathlineto{\pgfqpoint{1.023292in}{0.490730in}}%
\pgfpathlineto{\pgfqpoint{1.121145in}{0.490730in}}%
\pgfpathlineto{\pgfqpoint{1.218999in}{0.490730in}}%
\pgfpathlineto{\pgfqpoint{1.316852in}{0.490730in}}%
\pgfpathlineto{\pgfqpoint{1.414706in}{0.490730in}}%
\pgfpathlineto{\pgfqpoint{1.512559in}{0.490730in}}%
\pgfpathlineto{\pgfqpoint{1.610413in}{0.490730in}}%
\pgfpathlineto{\pgfqpoint{1.708266in}{0.490730in}}%
\pgfpathlineto{\pgfqpoint{1.806120in}{0.490730in}}%
\pgfpathlineto{\pgfqpoint{1.903973in}{0.490730in}}%
\pgfpathlineto{\pgfqpoint{2.001827in}{0.498966in}}%
\pgfpathlineto{\pgfqpoint{2.099681in}{0.507201in}}%
\pgfpathlineto{\pgfqpoint{2.197534in}{0.548377in}}%
\pgfpathlineto{\pgfqpoint{2.295388in}{0.713083in}}%
\pgfpathlineto{\pgfqpoint{2.393241in}{0.935436in}}%
\pgfpathlineto{\pgfqpoint{2.491095in}{1.487201in}}%
\pgfpathlineto{\pgfqpoint{2.588948in}{1.841319in}}%
\pgfpathlineto{\pgfqpoint{2.686802in}{2.220142in}}%
\pgfpathlineto{\pgfqpoint{2.784655in}{2.343672in}}%
\pgfpathlineto{\pgfqpoint{2.882509in}{2.417789in}}%
\pgfpathlineto{\pgfqpoint{2.980362in}{2.475436in}}%
\pgfpathlineto{\pgfqpoint{3.078216in}{2.516613in}}%
\pgfpathlineto{\pgfqpoint{3.176069in}{2.549554in}}%
\pgfpathlineto{\pgfqpoint{3.273923in}{2.557789in}}%
\pgfpathlineto{\pgfqpoint{3.371776in}{2.574260in}}%
\pgfpathlineto{\pgfqpoint{3.469630in}{2.574260in}}%
\pgfpathlineto{\pgfqpoint{3.567484in}{2.582495in}}%
\pgfpathlineto{\pgfqpoint{3.665337in}{2.590730in}}%
\pgfpathlineto{\pgfqpoint{3.763191in}{2.590730in}}%
\pgfpathlineto{\pgfqpoint{3.861044in}{2.590730in}}%
\pgfpathlineto{\pgfqpoint{3.958898in}{2.590730in}}%
\pgfpathlineto{\pgfqpoint{4.056751in}{2.590730in}}%
\pgfpathlineto{\pgfqpoint{4.154605in}{2.590730in}}%
\pgfusepath{stroke}%
\end{pgfscope}%
\begin{pgfscope}%
\pgfpathrectangle{\pgfqpoint{0.455741in}{0.385730in}}{\pgfqpoint{3.875000in}{2.310000in}}%
\pgfusepath{clip}%
\pgfsetrectcap%
\pgfsetroundjoin%
\pgfsetlinewidth{0.803000pt}%
\definecolor{currentstroke}{rgb}{0.686275,0.352941,0.313725}%
\pgfsetstrokecolor{currentstroke}%
\pgfsetdash{}{0pt}%
\pgfpathmoveto{\pgfqpoint{0.631877in}{0.490730in}}%
\pgfpathlineto{\pgfqpoint{0.729731in}{0.490730in}}%
\pgfpathlineto{\pgfqpoint{0.827585in}{0.490730in}}%
\pgfpathlineto{\pgfqpoint{0.925438in}{0.490730in}}%
\pgfpathlineto{\pgfqpoint{1.023292in}{0.490730in}}%
\pgfpathlineto{\pgfqpoint{1.121145in}{0.490730in}}%
\pgfpathlineto{\pgfqpoint{1.218999in}{0.490730in}}%
\pgfpathlineto{\pgfqpoint{1.316852in}{0.490730in}}%
\pgfpathlineto{\pgfqpoint{1.414706in}{0.490730in}}%
\pgfpathlineto{\pgfqpoint{1.512559in}{0.490730in}}%
\pgfpathlineto{\pgfqpoint{1.610413in}{0.490730in}}%
\pgfpathlineto{\pgfqpoint{1.708266in}{0.490730in}}%
\pgfpathlineto{\pgfqpoint{1.806120in}{0.490730in}}%
\pgfpathlineto{\pgfqpoint{1.903973in}{0.490730in}}%
\pgfpathlineto{\pgfqpoint{2.001827in}{0.507201in}}%
\pgfpathlineto{\pgfqpoint{2.099681in}{0.531907in}}%
\pgfpathlineto{\pgfqpoint{2.197534in}{0.630730in}}%
\pgfpathlineto{\pgfqpoint{2.295388in}{0.770730in}}%
\pgfpathlineto{\pgfqpoint{2.393241in}{1.207201in}}%
\pgfpathlineto{\pgfqpoint{2.491095in}{1.783672in}}%
\pgfpathlineto{\pgfqpoint{2.588948in}{2.104848in}}%
\pgfpathlineto{\pgfqpoint{2.686802in}{2.343672in}}%
\pgfpathlineto{\pgfqpoint{2.784655in}{2.409554in}}%
\pgfpathlineto{\pgfqpoint{2.882509in}{2.483672in}}%
\pgfpathlineto{\pgfqpoint{2.980362in}{2.491907in}}%
\pgfpathlineto{\pgfqpoint{3.078216in}{2.524848in}}%
\pgfpathlineto{\pgfqpoint{3.176069in}{2.549554in}}%
\pgfpathlineto{\pgfqpoint{3.273923in}{2.566025in}}%
\pgfpathlineto{\pgfqpoint{3.371776in}{2.574260in}}%
\pgfpathlineto{\pgfqpoint{3.469630in}{2.582495in}}%
\pgfpathlineto{\pgfqpoint{3.567484in}{2.582495in}}%
\pgfpathlineto{\pgfqpoint{3.665337in}{2.590730in}}%
\pgfpathlineto{\pgfqpoint{3.763191in}{2.590730in}}%
\pgfpathlineto{\pgfqpoint{3.861044in}{2.590730in}}%
\pgfpathlineto{\pgfqpoint{3.958898in}{2.590730in}}%
\pgfpathlineto{\pgfqpoint{4.056751in}{2.590730in}}%
\pgfpathlineto{\pgfqpoint{4.154605in}{2.590730in}}%
\pgfusepath{stroke}%
\end{pgfscope}%
\begin{pgfscope}%
\pgfpathrectangle{\pgfqpoint{0.455741in}{0.385730in}}{\pgfqpoint{3.875000in}{2.310000in}}%
\pgfusepath{clip}%
\pgfsetrectcap%
\pgfsetroundjoin%
\pgfsetlinewidth{0.803000pt}%
\definecolor{currentstroke}{rgb}{0.000000,0.356863,0.509804}%
\pgfsetstrokecolor{currentstroke}%
\pgfsetdash{}{0pt}%
\pgfpathmoveto{\pgfqpoint{0.631877in}{0.490730in}}%
\pgfpathlineto{\pgfqpoint{0.729731in}{0.490730in}}%
\pgfpathlineto{\pgfqpoint{0.827585in}{0.490730in}}%
\pgfpathlineto{\pgfqpoint{0.925438in}{0.490730in}}%
\pgfpathlineto{\pgfqpoint{1.023292in}{0.490730in}}%
\pgfpathlineto{\pgfqpoint{1.121145in}{0.490730in}}%
\pgfpathlineto{\pgfqpoint{1.218999in}{0.490730in}}%
\pgfpathlineto{\pgfqpoint{1.316852in}{0.490730in}}%
\pgfpathlineto{\pgfqpoint{1.414706in}{0.490730in}}%
\pgfpathlineto{\pgfqpoint{1.512559in}{0.490730in}}%
\pgfpathlineto{\pgfqpoint{1.610413in}{0.490730in}}%
\pgfpathlineto{\pgfqpoint{1.708266in}{0.490730in}}%
\pgfpathlineto{\pgfqpoint{1.806120in}{0.490730in}}%
\pgfpathlineto{\pgfqpoint{1.903973in}{0.490730in}}%
\pgfpathlineto{\pgfqpoint{2.001827in}{0.515436in}}%
\pgfpathlineto{\pgfqpoint{2.099681in}{0.531907in}}%
\pgfpathlineto{\pgfqpoint{2.197534in}{0.589554in}}%
\pgfpathlineto{\pgfqpoint{2.295388in}{0.729554in}}%
\pgfpathlineto{\pgfqpoint{2.393241in}{1.091907in}}%
\pgfpathlineto{\pgfqpoint{2.491095in}{1.742495in}}%
\pgfpathlineto{\pgfqpoint{2.588948in}{2.022495in}}%
\pgfpathlineto{\pgfqpoint{2.686802in}{2.310730in}}%
\pgfpathlineto{\pgfqpoint{2.784655in}{2.409554in}}%
\pgfpathlineto{\pgfqpoint{2.882509in}{2.475436in}}%
\pgfpathlineto{\pgfqpoint{2.980362in}{2.500142in}}%
\pgfpathlineto{\pgfqpoint{3.078216in}{2.533083in}}%
\pgfpathlineto{\pgfqpoint{3.176069in}{2.549554in}}%
\pgfpathlineto{\pgfqpoint{3.273923in}{2.557789in}}%
\pgfpathlineto{\pgfqpoint{3.371776in}{2.574260in}}%
\pgfpathlineto{\pgfqpoint{3.469630in}{2.582495in}}%
\pgfpathlineto{\pgfqpoint{3.567484in}{2.574260in}}%
\pgfpathlineto{\pgfqpoint{3.665337in}{2.590730in}}%
\pgfpathlineto{\pgfqpoint{3.763191in}{2.590730in}}%
\pgfpathlineto{\pgfqpoint{3.861044in}{2.590730in}}%
\pgfpathlineto{\pgfqpoint{3.958898in}{2.582495in}}%
\pgfpathlineto{\pgfqpoint{4.056751in}{2.590730in}}%
\pgfpathlineto{\pgfqpoint{4.154605in}{2.582495in}}%
\pgfusepath{stroke}%
\end{pgfscope}%
\begin{pgfscope}%
\pgfpathrectangle{\pgfqpoint{0.455741in}{0.385730in}}{\pgfqpoint{3.875000in}{2.310000in}}%
\pgfusepath{clip}%
\pgfsetrectcap%
\pgfsetroundjoin%
\pgfsetlinewidth{0.803000pt}%
\definecolor{currentstroke}{rgb}{0.490196,0.588235,0.431373}%
\pgfsetstrokecolor{currentstroke}%
\pgfsetdash{}{0pt}%
\pgfpathmoveto{\pgfqpoint{0.631877in}{0.490730in}}%
\pgfpathlineto{\pgfqpoint{0.729731in}{0.490730in}}%
\pgfpathlineto{\pgfqpoint{0.827585in}{0.490730in}}%
\pgfpathlineto{\pgfqpoint{0.925438in}{0.490730in}}%
\pgfpathlineto{\pgfqpoint{1.023292in}{0.490730in}}%
\pgfpathlineto{\pgfqpoint{1.121145in}{0.490730in}}%
\pgfpathlineto{\pgfqpoint{1.218999in}{0.490730in}}%
\pgfpathlineto{\pgfqpoint{1.316852in}{0.490730in}}%
\pgfpathlineto{\pgfqpoint{1.414706in}{0.490730in}}%
\pgfpathlineto{\pgfqpoint{1.512559in}{0.490730in}}%
\pgfpathlineto{\pgfqpoint{1.610413in}{0.490730in}}%
\pgfpathlineto{\pgfqpoint{1.708266in}{0.490730in}}%
\pgfpathlineto{\pgfqpoint{1.806120in}{0.490730in}}%
\pgfpathlineto{\pgfqpoint{1.903973in}{0.490730in}}%
\pgfpathlineto{\pgfqpoint{2.001827in}{0.507201in}}%
\pgfpathlineto{\pgfqpoint{2.099681in}{0.523672in}}%
\pgfpathlineto{\pgfqpoint{2.197534in}{0.630730in}}%
\pgfpathlineto{\pgfqpoint{2.295388in}{0.754260in}}%
\pgfpathlineto{\pgfqpoint{2.393241in}{1.034260in}}%
\pgfpathlineto{\pgfqpoint{2.491095in}{1.618966in}}%
\pgfpathlineto{\pgfqpoint{2.588948in}{2.022495in}}%
\pgfpathlineto{\pgfqpoint{2.686802in}{2.253083in}}%
\pgfpathlineto{\pgfqpoint{2.784655in}{2.409554in}}%
\pgfpathlineto{\pgfqpoint{2.882509in}{2.483672in}}%
\pgfpathlineto{\pgfqpoint{2.980362in}{2.508377in}}%
\pgfpathlineto{\pgfqpoint{3.078216in}{2.533083in}}%
\pgfpathlineto{\pgfqpoint{3.176069in}{2.557789in}}%
\pgfpathlineto{\pgfqpoint{3.273923in}{2.574260in}}%
\pgfpathlineto{\pgfqpoint{3.371776in}{2.582495in}}%
\pgfpathlineto{\pgfqpoint{3.469630in}{2.582495in}}%
\pgfpathlineto{\pgfqpoint{3.567484in}{2.590730in}}%
\pgfpathlineto{\pgfqpoint{3.665337in}{2.590730in}}%
\pgfpathlineto{\pgfqpoint{3.763191in}{2.590730in}}%
\pgfpathlineto{\pgfqpoint{3.861044in}{2.590730in}}%
\pgfpathlineto{\pgfqpoint{3.958898in}{2.590730in}}%
\pgfpathlineto{\pgfqpoint{4.056751in}{2.590730in}}%
\pgfpathlineto{\pgfqpoint{4.154605in}{2.590730in}}%
\pgfusepath{stroke}%
\end{pgfscope}%
\begin{pgfscope}%
\pgfpathrectangle{\pgfqpoint{0.455741in}{0.385730in}}{\pgfqpoint{3.875000in}{2.310000in}}%
\pgfusepath{clip}%
\pgfsetbuttcap%
\pgfsetroundjoin%
\pgfsetlinewidth{0.803000pt}%
\definecolor{currentstroke}{rgb}{0.843137,0.666667,0.313725}%
\pgfsetstrokecolor{currentstroke}%
\pgfsetdash{{0.800000pt}{1.320000pt}}{0.000000pt}%
\pgfpathmoveto{\pgfqpoint{0.631877in}{0.490730in}}%
\pgfpathlineto{\pgfqpoint{0.729731in}{0.490730in}}%
\pgfpathlineto{\pgfqpoint{0.827585in}{0.490730in}}%
\pgfpathlineto{\pgfqpoint{0.925438in}{0.490730in}}%
\pgfpathlineto{\pgfqpoint{1.023292in}{0.490730in}}%
\pgfpathlineto{\pgfqpoint{1.121145in}{0.490730in}}%
\pgfpathlineto{\pgfqpoint{1.218999in}{0.490730in}}%
\pgfpathlineto{\pgfqpoint{1.316852in}{0.490730in}}%
\pgfpathlineto{\pgfqpoint{1.414706in}{0.490730in}}%
\pgfpathlineto{\pgfqpoint{1.512559in}{0.490730in}}%
\pgfpathlineto{\pgfqpoint{1.610413in}{0.490730in}}%
\pgfpathlineto{\pgfqpoint{1.708266in}{0.490730in}}%
\pgfpathlineto{\pgfqpoint{1.806120in}{0.490730in}}%
\pgfpathlineto{\pgfqpoint{1.903973in}{0.490730in}}%
\pgfpathlineto{\pgfqpoint{2.001827in}{0.490730in}}%
\pgfpathlineto{\pgfqpoint{2.099681in}{0.490730in}}%
\pgfpathlineto{\pgfqpoint{2.197534in}{0.490730in}}%
\pgfpathlineto{\pgfqpoint{2.295388in}{0.490730in}}%
\pgfpathlineto{\pgfqpoint{2.393241in}{0.490730in}}%
\pgfpathlineto{\pgfqpoint{2.491095in}{0.490730in}}%
\pgfpathlineto{\pgfqpoint{2.588948in}{0.490730in}}%
\pgfpathlineto{\pgfqpoint{2.686802in}{0.490730in}}%
\pgfpathlineto{\pgfqpoint{2.784655in}{0.490730in}}%
\pgfpathlineto{\pgfqpoint{2.882509in}{0.490730in}}%
\pgfpathlineto{\pgfqpoint{2.980362in}{0.490730in}}%
\pgfpathlineto{\pgfqpoint{3.078216in}{0.490730in}}%
\pgfpathlineto{\pgfqpoint{3.176069in}{0.515436in}}%
\pgfpathlineto{\pgfqpoint{3.273923in}{0.622495in}}%
\pgfpathlineto{\pgfqpoint{3.371776in}{0.803672in}}%
\pgfpathlineto{\pgfqpoint{3.469630in}{1.141319in}}%
\pgfpathlineto{\pgfqpoint{3.567484in}{1.594260in}}%
\pgfpathlineto{\pgfqpoint{3.665337in}{1.915436in}}%
\pgfpathlineto{\pgfqpoint{3.763191in}{2.269554in}}%
\pgfpathlineto{\pgfqpoint{3.861044in}{2.434260in}}%
\pgfpathlineto{\pgfqpoint{3.958898in}{2.516613in}}%
\pgfpathlineto{\pgfqpoint{4.056751in}{2.541319in}}%
\pgfpathlineto{\pgfqpoint{4.154605in}{2.541319in}}%
\pgfusepath{stroke}%
\end{pgfscope}%
\begin{pgfscope}%
\pgfpathrectangle{\pgfqpoint{0.455741in}{0.385730in}}{\pgfqpoint{3.875000in}{2.310000in}}%
\pgfusepath{clip}%
\pgfsetbuttcap%
\pgfsetroundjoin%
\pgfsetlinewidth{0.803000pt}%
\definecolor{currentstroke}{rgb}{0.333333,0.333333,0.333333}%
\pgfsetstrokecolor{currentstroke}%
\pgfsetdash{{0.800000pt}{1.320000pt}}{0.000000pt}%
\pgfpathmoveto{\pgfqpoint{0.631877in}{0.490730in}}%
\pgfpathlineto{\pgfqpoint{0.729731in}{0.490730in}}%
\pgfpathlineto{\pgfqpoint{0.827585in}{0.490730in}}%
\pgfpathlineto{\pgfqpoint{0.925438in}{0.490730in}}%
\pgfpathlineto{\pgfqpoint{1.023292in}{0.490730in}}%
\pgfpathlineto{\pgfqpoint{1.121145in}{0.490730in}}%
\pgfpathlineto{\pgfqpoint{1.218999in}{0.490730in}}%
\pgfpathlineto{\pgfqpoint{1.316852in}{0.490730in}}%
\pgfpathlineto{\pgfqpoint{1.414706in}{0.490730in}}%
\pgfpathlineto{\pgfqpoint{1.512559in}{0.490730in}}%
\pgfpathlineto{\pgfqpoint{1.610413in}{0.490730in}}%
\pgfpathlineto{\pgfqpoint{1.708266in}{0.490730in}}%
\pgfpathlineto{\pgfqpoint{1.806120in}{0.490730in}}%
\pgfpathlineto{\pgfqpoint{1.903973in}{0.490730in}}%
\pgfpathlineto{\pgfqpoint{2.001827in}{0.490730in}}%
\pgfpathlineto{\pgfqpoint{2.099681in}{0.490730in}}%
\pgfpathlineto{\pgfqpoint{2.197534in}{0.490730in}}%
\pgfpathlineto{\pgfqpoint{2.295388in}{0.490730in}}%
\pgfpathlineto{\pgfqpoint{2.393241in}{0.490730in}}%
\pgfpathlineto{\pgfqpoint{2.491095in}{0.490730in}}%
\pgfpathlineto{\pgfqpoint{2.588948in}{0.490730in}}%
\pgfpathlineto{\pgfqpoint{2.686802in}{0.490730in}}%
\pgfpathlineto{\pgfqpoint{2.784655in}{0.490730in}}%
\pgfpathlineto{\pgfqpoint{2.882509in}{0.498966in}}%
\pgfpathlineto{\pgfqpoint{2.980362in}{0.548377in}}%
\pgfpathlineto{\pgfqpoint{3.078216in}{0.655436in}}%
\pgfpathlineto{\pgfqpoint{3.176069in}{0.861319in}}%
\pgfpathlineto{\pgfqpoint{3.273923in}{1.174260in}}%
\pgfpathlineto{\pgfqpoint{3.371776in}{1.717789in}}%
\pgfpathlineto{\pgfqpoint{3.469630in}{2.129554in}}%
\pgfpathlineto{\pgfqpoint{3.567484in}{2.351907in}}%
\pgfpathlineto{\pgfqpoint{3.665337in}{2.458966in}}%
\pgfpathlineto{\pgfqpoint{3.763191in}{2.500142in}}%
\pgfpathlineto{\pgfqpoint{3.861044in}{2.533083in}}%
\pgfpathlineto{\pgfqpoint{3.958898in}{2.541319in}}%
\pgfpathlineto{\pgfqpoint{4.056751in}{2.541319in}}%
\pgfpathlineto{\pgfqpoint{4.154605in}{2.549554in}}%
\pgfusepath{stroke}%
\end{pgfscope}%
\begin{pgfscope}%
\pgfpathrectangle{\pgfqpoint{0.455741in}{0.385730in}}{\pgfqpoint{3.875000in}{2.310000in}}%
\pgfusepath{clip}%
\pgfsetbuttcap%
\pgfsetroundjoin%
\pgfsetlinewidth{0.803000pt}%
\definecolor{currentstroke}{rgb}{0.686275,0.352941,0.313725}%
\pgfsetstrokecolor{currentstroke}%
\pgfsetdash{{0.800000pt}{1.320000pt}}{0.000000pt}%
\pgfpathmoveto{\pgfqpoint{0.631877in}{0.490730in}}%
\pgfpathlineto{\pgfqpoint{0.729731in}{0.490730in}}%
\pgfpathlineto{\pgfqpoint{0.827585in}{0.490730in}}%
\pgfpathlineto{\pgfqpoint{0.925438in}{0.490730in}}%
\pgfpathlineto{\pgfqpoint{1.023292in}{0.490730in}}%
\pgfpathlineto{\pgfqpoint{1.121145in}{0.490730in}}%
\pgfpathlineto{\pgfqpoint{1.218999in}{0.490730in}}%
\pgfpathlineto{\pgfqpoint{1.316852in}{0.490730in}}%
\pgfpathlineto{\pgfqpoint{1.414706in}{0.490730in}}%
\pgfpathlineto{\pgfqpoint{1.512559in}{0.490730in}}%
\pgfpathlineto{\pgfqpoint{1.610413in}{0.490730in}}%
\pgfpathlineto{\pgfqpoint{1.708266in}{0.490730in}}%
\pgfpathlineto{\pgfqpoint{1.806120in}{0.490730in}}%
\pgfpathlineto{\pgfqpoint{1.903973in}{0.490730in}}%
\pgfpathlineto{\pgfqpoint{2.001827in}{0.490730in}}%
\pgfpathlineto{\pgfqpoint{2.099681in}{0.490730in}}%
\pgfpathlineto{\pgfqpoint{2.197534in}{0.490730in}}%
\pgfpathlineto{\pgfqpoint{2.295388in}{0.490730in}}%
\pgfpathlineto{\pgfqpoint{2.393241in}{0.490730in}}%
\pgfpathlineto{\pgfqpoint{2.491095in}{0.490730in}}%
\pgfpathlineto{\pgfqpoint{2.588948in}{0.498966in}}%
\pgfpathlineto{\pgfqpoint{2.686802in}{0.548377in}}%
\pgfpathlineto{\pgfqpoint{2.784655in}{0.680142in}}%
\pgfpathlineto{\pgfqpoint{2.882509in}{0.836613in}}%
\pgfpathlineto{\pgfqpoint{2.980362in}{1.133083in}}%
\pgfpathlineto{\pgfqpoint{3.078216in}{1.487201in}}%
\pgfpathlineto{\pgfqpoint{3.176069in}{1.915436in}}%
\pgfpathlineto{\pgfqpoint{3.273923in}{2.104848in}}%
\pgfpathlineto{\pgfqpoint{3.371776in}{2.360142in}}%
\pgfpathlineto{\pgfqpoint{3.469630in}{2.475436in}}%
\pgfpathlineto{\pgfqpoint{3.567484in}{2.524848in}}%
\pgfpathlineto{\pgfqpoint{3.665337in}{2.549554in}}%
\pgfpathlineto{\pgfqpoint{3.763191in}{2.566025in}}%
\pgfpathlineto{\pgfqpoint{3.861044in}{2.574260in}}%
\pgfpathlineto{\pgfqpoint{3.958898in}{2.574260in}}%
\pgfpathlineto{\pgfqpoint{4.056751in}{2.574260in}}%
\pgfpathlineto{\pgfqpoint{4.154605in}{2.574260in}}%
\pgfusepath{stroke}%
\end{pgfscope}%
\begin{pgfscope}%
\pgfpathrectangle{\pgfqpoint{0.455741in}{0.385730in}}{\pgfqpoint{3.875000in}{2.310000in}}%
\pgfusepath{clip}%
\pgfsetbuttcap%
\pgfsetroundjoin%
\pgfsetlinewidth{0.803000pt}%
\definecolor{currentstroke}{rgb}{0.000000,0.356863,0.509804}%
\pgfsetstrokecolor{currentstroke}%
\pgfsetdash{{0.800000pt}{1.320000pt}}{0.000000pt}%
\pgfpathmoveto{\pgfqpoint{0.631877in}{0.490730in}}%
\pgfpathlineto{\pgfqpoint{0.729731in}{0.490730in}}%
\pgfpathlineto{\pgfqpoint{0.827585in}{0.490730in}}%
\pgfpathlineto{\pgfqpoint{0.925438in}{0.490730in}}%
\pgfpathlineto{\pgfqpoint{1.023292in}{0.490730in}}%
\pgfpathlineto{\pgfqpoint{1.121145in}{0.490730in}}%
\pgfpathlineto{\pgfqpoint{1.218999in}{0.490730in}}%
\pgfpathlineto{\pgfqpoint{1.316852in}{0.490730in}}%
\pgfpathlineto{\pgfqpoint{1.414706in}{0.490730in}}%
\pgfpathlineto{\pgfqpoint{1.512559in}{0.490730in}}%
\pgfpathlineto{\pgfqpoint{1.610413in}{0.490730in}}%
\pgfpathlineto{\pgfqpoint{1.708266in}{0.490730in}}%
\pgfpathlineto{\pgfqpoint{1.806120in}{0.490730in}}%
\pgfpathlineto{\pgfqpoint{1.903973in}{0.490730in}}%
\pgfpathlineto{\pgfqpoint{2.001827in}{0.490730in}}%
\pgfpathlineto{\pgfqpoint{2.099681in}{0.490730in}}%
\pgfpathlineto{\pgfqpoint{2.197534in}{0.490730in}}%
\pgfpathlineto{\pgfqpoint{2.295388in}{0.490730in}}%
\pgfpathlineto{\pgfqpoint{2.393241in}{0.490730in}}%
\pgfpathlineto{\pgfqpoint{2.491095in}{0.490730in}}%
\pgfpathlineto{\pgfqpoint{2.588948in}{0.490730in}}%
\pgfpathlineto{\pgfqpoint{2.686802in}{0.490730in}}%
\pgfpathlineto{\pgfqpoint{2.784655in}{0.490730in}}%
\pgfpathlineto{\pgfqpoint{2.882509in}{0.490730in}}%
\pgfpathlineto{\pgfqpoint{2.980362in}{0.490730in}}%
\pgfpathlineto{\pgfqpoint{3.078216in}{0.490730in}}%
\pgfpathlineto{\pgfqpoint{3.176069in}{0.498966in}}%
\pgfpathlineto{\pgfqpoint{3.273923in}{0.589554in}}%
\pgfpathlineto{\pgfqpoint{3.371776in}{0.729554in}}%
\pgfpathlineto{\pgfqpoint{3.469630in}{0.976613in}}%
\pgfpathlineto{\pgfqpoint{3.567484in}{1.437789in}}%
\pgfpathlineto{\pgfqpoint{3.665337in}{1.824848in}}%
\pgfpathlineto{\pgfqpoint{3.763191in}{2.104848in}}%
\pgfpathlineto{\pgfqpoint{3.861044in}{2.351907in}}%
\pgfpathlineto{\pgfqpoint{3.958898in}{2.442495in}}%
\pgfpathlineto{\pgfqpoint{4.056751in}{2.467201in}}%
\pgfpathlineto{\pgfqpoint{4.154605in}{2.475436in}}%
\pgfusepath{stroke}%
\end{pgfscope}%
\begin{pgfscope}%
\pgfpathrectangle{\pgfqpoint{0.455741in}{0.385730in}}{\pgfqpoint{3.875000in}{2.310000in}}%
\pgfusepath{clip}%
\pgfsetbuttcap%
\pgfsetroundjoin%
\pgfsetlinewidth{0.803000pt}%
\definecolor{currentstroke}{rgb}{0.490196,0.588235,0.431373}%
\pgfsetstrokecolor{currentstroke}%
\pgfsetdash{{0.800000pt}{1.320000pt}}{0.000000pt}%
\pgfpathmoveto{\pgfqpoint{0.631877in}{0.490730in}}%
\pgfpathlineto{\pgfqpoint{0.729731in}{0.490730in}}%
\pgfpathlineto{\pgfqpoint{0.827585in}{0.490730in}}%
\pgfpathlineto{\pgfqpoint{0.925438in}{0.490730in}}%
\pgfpathlineto{\pgfqpoint{1.023292in}{0.490730in}}%
\pgfpathlineto{\pgfqpoint{1.121145in}{0.490730in}}%
\pgfpathlineto{\pgfqpoint{1.218999in}{0.490730in}}%
\pgfpathlineto{\pgfqpoint{1.316852in}{0.490730in}}%
\pgfpathlineto{\pgfqpoint{1.414706in}{0.490730in}}%
\pgfpathlineto{\pgfqpoint{1.512559in}{0.490730in}}%
\pgfpathlineto{\pgfqpoint{1.610413in}{0.490730in}}%
\pgfpathlineto{\pgfqpoint{1.708266in}{0.490730in}}%
\pgfpathlineto{\pgfqpoint{1.806120in}{0.490730in}}%
\pgfpathlineto{\pgfqpoint{1.903973in}{0.490730in}}%
\pgfpathlineto{\pgfqpoint{2.001827in}{0.490730in}}%
\pgfpathlineto{\pgfqpoint{2.099681in}{0.490730in}}%
\pgfpathlineto{\pgfqpoint{2.197534in}{0.490730in}}%
\pgfpathlineto{\pgfqpoint{2.295388in}{0.490730in}}%
\pgfpathlineto{\pgfqpoint{2.393241in}{0.490730in}}%
\pgfpathlineto{\pgfqpoint{2.491095in}{0.490730in}}%
\pgfpathlineto{\pgfqpoint{2.588948in}{0.490730in}}%
\pgfpathlineto{\pgfqpoint{2.686802in}{0.490730in}}%
\pgfpathlineto{\pgfqpoint{2.784655in}{0.490730in}}%
\pgfpathlineto{\pgfqpoint{2.882509in}{0.490730in}}%
\pgfpathlineto{\pgfqpoint{2.980362in}{0.490730in}}%
\pgfpathlineto{\pgfqpoint{3.078216in}{0.498966in}}%
\pgfpathlineto{\pgfqpoint{3.176069in}{0.540142in}}%
\pgfpathlineto{\pgfqpoint{3.273923in}{0.647201in}}%
\pgfpathlineto{\pgfqpoint{3.371776in}{0.828377in}}%
\pgfpathlineto{\pgfqpoint{3.469630in}{1.223672in}}%
\pgfpathlineto{\pgfqpoint{3.567484in}{1.594260in}}%
\pgfpathlineto{\pgfqpoint{3.665337in}{2.022495in}}%
\pgfpathlineto{\pgfqpoint{3.763191in}{2.277789in}}%
\pgfpathlineto{\pgfqpoint{3.861044in}{2.434260in}}%
\pgfpathlineto{\pgfqpoint{3.958898in}{2.500142in}}%
\pgfpathlineto{\pgfqpoint{4.056751in}{2.516613in}}%
\pgfpathlineto{\pgfqpoint{4.154605in}{2.524848in}}%
\pgfusepath{stroke}%
\end{pgfscope}%
\begin{pgfscope}%
\pgfpathrectangle{\pgfqpoint{0.455741in}{0.385730in}}{\pgfqpoint{3.875000in}{2.310000in}}%
\pgfusepath{clip}%
\pgfsetbuttcap%
\pgfsetroundjoin%
\pgfsetlinewidth{0.803000pt}%
\definecolor{currentstroke}{rgb}{0.843137,0.666667,0.313725}%
\pgfsetstrokecolor{currentstroke}%
\pgfsetdash{{0.800000pt}{1.320000pt}}{0.000000pt}%
\pgfpathmoveto{\pgfqpoint{0.631877in}{0.490730in}}%
\pgfpathlineto{\pgfqpoint{0.729731in}{0.490730in}}%
\pgfpathlineto{\pgfqpoint{0.827585in}{0.490730in}}%
\pgfpathlineto{\pgfqpoint{0.925438in}{0.490730in}}%
\pgfpathlineto{\pgfqpoint{1.023292in}{0.490730in}}%
\pgfpathlineto{\pgfqpoint{1.121145in}{0.490730in}}%
\pgfpathlineto{\pgfqpoint{1.218999in}{0.490730in}}%
\pgfpathlineto{\pgfqpoint{1.316852in}{0.490730in}}%
\pgfpathlineto{\pgfqpoint{1.414706in}{0.490730in}}%
\pgfpathlineto{\pgfqpoint{1.512559in}{0.490730in}}%
\pgfpathlineto{\pgfqpoint{1.610413in}{0.490730in}}%
\pgfpathlineto{\pgfqpoint{1.708266in}{0.490730in}}%
\pgfpathlineto{\pgfqpoint{1.806120in}{0.490730in}}%
\pgfpathlineto{\pgfqpoint{1.903973in}{0.490730in}}%
\pgfpathlineto{\pgfqpoint{2.001827in}{0.490730in}}%
\pgfpathlineto{\pgfqpoint{2.099681in}{0.490730in}}%
\pgfpathlineto{\pgfqpoint{2.197534in}{0.490730in}}%
\pgfpathlineto{\pgfqpoint{2.295388in}{0.490730in}}%
\pgfpathlineto{\pgfqpoint{2.393241in}{0.490730in}}%
\pgfpathlineto{\pgfqpoint{2.491095in}{0.490730in}}%
\pgfpathlineto{\pgfqpoint{2.588948in}{0.490730in}}%
\pgfpathlineto{\pgfqpoint{2.686802in}{0.490730in}}%
\pgfpathlineto{\pgfqpoint{2.784655in}{0.507201in}}%
\pgfpathlineto{\pgfqpoint{2.882509in}{0.540142in}}%
\pgfpathlineto{\pgfqpoint{2.980362in}{0.630730in}}%
\pgfpathlineto{\pgfqpoint{3.078216in}{0.811907in}}%
\pgfpathlineto{\pgfqpoint{3.176069in}{1.083672in}}%
\pgfpathlineto{\pgfqpoint{3.273923in}{1.528377in}}%
\pgfpathlineto{\pgfqpoint{3.371776in}{1.898966in}}%
\pgfpathlineto{\pgfqpoint{3.469630in}{2.195436in}}%
\pgfpathlineto{\pgfqpoint{3.567484in}{2.384848in}}%
\pgfpathlineto{\pgfqpoint{3.665337in}{2.467201in}}%
\pgfpathlineto{\pgfqpoint{3.763191in}{2.508377in}}%
\pgfpathlineto{\pgfqpoint{3.861044in}{2.516613in}}%
\pgfpathlineto{\pgfqpoint{3.958898in}{2.524848in}}%
\pgfpathlineto{\pgfqpoint{4.056751in}{2.533083in}}%
\pgfpathlineto{\pgfqpoint{4.154605in}{2.533083in}}%
\pgfusepath{stroke}%
\end{pgfscope}%
\begin{pgfscope}%
\pgfpathrectangle{\pgfqpoint{0.455741in}{0.385730in}}{\pgfqpoint{3.875000in}{2.310000in}}%
\pgfusepath{clip}%
\pgfsetbuttcap%
\pgfsetroundjoin%
\pgfsetlinewidth{0.803000pt}%
\definecolor{currentstroke}{rgb}{0.333333,0.333333,0.333333}%
\pgfsetstrokecolor{currentstroke}%
\pgfsetdash{{0.800000pt}{1.320000pt}}{0.000000pt}%
\pgfpathmoveto{\pgfqpoint{0.631877in}{0.490730in}}%
\pgfpathlineto{\pgfqpoint{0.729731in}{0.490730in}}%
\pgfpathlineto{\pgfqpoint{0.827585in}{0.490730in}}%
\pgfpathlineto{\pgfqpoint{0.925438in}{0.490730in}}%
\pgfpathlineto{\pgfqpoint{1.023292in}{0.490730in}}%
\pgfpathlineto{\pgfqpoint{1.121145in}{0.490730in}}%
\pgfpathlineto{\pgfqpoint{1.218999in}{0.490730in}}%
\pgfpathlineto{\pgfqpoint{1.316852in}{0.490730in}}%
\pgfpathlineto{\pgfqpoint{1.414706in}{0.490730in}}%
\pgfpathlineto{\pgfqpoint{1.512559in}{0.490730in}}%
\pgfpathlineto{\pgfqpoint{1.610413in}{0.490730in}}%
\pgfpathlineto{\pgfqpoint{1.708266in}{0.490730in}}%
\pgfpathlineto{\pgfqpoint{1.806120in}{0.490730in}}%
\pgfpathlineto{\pgfqpoint{1.903973in}{0.490730in}}%
\pgfpathlineto{\pgfqpoint{2.001827in}{0.490730in}}%
\pgfpathlineto{\pgfqpoint{2.099681in}{0.490730in}}%
\pgfpathlineto{\pgfqpoint{2.197534in}{0.490730in}}%
\pgfpathlineto{\pgfqpoint{2.295388in}{0.490730in}}%
\pgfpathlineto{\pgfqpoint{2.393241in}{0.490730in}}%
\pgfpathlineto{\pgfqpoint{2.491095in}{0.490730in}}%
\pgfpathlineto{\pgfqpoint{2.588948in}{0.490730in}}%
\pgfpathlineto{\pgfqpoint{2.686802in}{0.490730in}}%
\pgfpathlineto{\pgfqpoint{2.784655in}{0.490730in}}%
\pgfpathlineto{\pgfqpoint{2.882509in}{0.498966in}}%
\pgfpathlineto{\pgfqpoint{2.980362in}{0.531907in}}%
\pgfpathlineto{\pgfqpoint{3.078216in}{0.622495in}}%
\pgfpathlineto{\pgfqpoint{3.176069in}{0.787201in}}%
\pgfpathlineto{\pgfqpoint{3.273923in}{1.124848in}}%
\pgfpathlineto{\pgfqpoint{3.371776in}{1.478966in}}%
\pgfpathlineto{\pgfqpoint{3.469630in}{1.874260in}}%
\pgfpathlineto{\pgfqpoint{3.567484in}{2.253083in}}%
\pgfpathlineto{\pgfqpoint{3.665337in}{2.401319in}}%
\pgfpathlineto{\pgfqpoint{3.763191in}{2.450730in}}%
\pgfpathlineto{\pgfqpoint{3.861044in}{2.491907in}}%
\pgfpathlineto{\pgfqpoint{3.958898in}{2.500142in}}%
\pgfpathlineto{\pgfqpoint{4.056751in}{2.508377in}}%
\pgfpathlineto{\pgfqpoint{4.154605in}{2.516613in}}%
\pgfusepath{stroke}%
\end{pgfscope}%
\begin{pgfscope}%
\pgfpathrectangle{\pgfqpoint{0.455741in}{0.385730in}}{\pgfqpoint{3.875000in}{2.310000in}}%
\pgfusepath{clip}%
\pgfsetbuttcap%
\pgfsetroundjoin%
\pgfsetlinewidth{0.803000pt}%
\definecolor{currentstroke}{rgb}{0.686275,0.352941,0.313725}%
\pgfsetstrokecolor{currentstroke}%
\pgfsetdash{{0.800000pt}{1.320000pt}}{0.000000pt}%
\pgfpathmoveto{\pgfqpoint{0.631877in}{0.490730in}}%
\pgfpathlineto{\pgfqpoint{0.729731in}{0.490730in}}%
\pgfpathlineto{\pgfqpoint{0.827585in}{0.490730in}}%
\pgfpathlineto{\pgfqpoint{0.925438in}{0.490730in}}%
\pgfpathlineto{\pgfqpoint{1.023292in}{0.490730in}}%
\pgfpathlineto{\pgfqpoint{1.121145in}{0.490730in}}%
\pgfpathlineto{\pgfqpoint{1.218999in}{0.490730in}}%
\pgfpathlineto{\pgfqpoint{1.316852in}{0.490730in}}%
\pgfpathlineto{\pgfqpoint{1.414706in}{0.490730in}}%
\pgfpathlineto{\pgfqpoint{1.512559in}{0.490730in}}%
\pgfpathlineto{\pgfqpoint{1.610413in}{0.490730in}}%
\pgfpathlineto{\pgfqpoint{1.708266in}{0.490730in}}%
\pgfpathlineto{\pgfqpoint{1.806120in}{0.490730in}}%
\pgfpathlineto{\pgfqpoint{1.903973in}{0.490730in}}%
\pgfpathlineto{\pgfqpoint{2.001827in}{0.490730in}}%
\pgfpathlineto{\pgfqpoint{2.099681in}{0.490730in}}%
\pgfpathlineto{\pgfqpoint{2.197534in}{0.490730in}}%
\pgfpathlineto{\pgfqpoint{2.295388in}{0.490730in}}%
\pgfpathlineto{\pgfqpoint{2.393241in}{0.490730in}}%
\pgfpathlineto{\pgfqpoint{2.491095in}{0.490730in}}%
\pgfpathlineto{\pgfqpoint{2.588948in}{0.490730in}}%
\pgfpathlineto{\pgfqpoint{2.686802in}{0.490730in}}%
\pgfpathlineto{\pgfqpoint{2.784655in}{0.498966in}}%
\pgfpathlineto{\pgfqpoint{2.882509in}{0.531907in}}%
\pgfpathlineto{\pgfqpoint{2.980362in}{0.622495in}}%
\pgfpathlineto{\pgfqpoint{3.078216in}{0.746025in}}%
\pgfpathlineto{\pgfqpoint{3.176069in}{1.001319in}}%
\pgfpathlineto{\pgfqpoint{3.273923in}{1.396613in}}%
\pgfpathlineto{\pgfqpoint{3.371776in}{1.849554in}}%
\pgfpathlineto{\pgfqpoint{3.469630in}{2.187201in}}%
\pgfpathlineto{\pgfqpoint{3.567484in}{2.376613in}}%
\pgfpathlineto{\pgfqpoint{3.665337in}{2.483672in}}%
\pgfpathlineto{\pgfqpoint{3.763191in}{2.524848in}}%
\pgfpathlineto{\pgfqpoint{3.861044in}{2.549554in}}%
\pgfpathlineto{\pgfqpoint{3.958898in}{2.557789in}}%
\pgfpathlineto{\pgfqpoint{4.056751in}{2.557789in}}%
\pgfpathlineto{\pgfqpoint{4.154605in}{2.557789in}}%
\pgfusepath{stroke}%
\end{pgfscope}%
\begin{pgfscope}%
\pgfpathrectangle{\pgfqpoint{0.455741in}{0.385730in}}{\pgfqpoint{3.875000in}{2.310000in}}%
\pgfusepath{clip}%
\pgfsetbuttcap%
\pgfsetroundjoin%
\pgfsetlinewidth{0.803000pt}%
\definecolor{currentstroke}{rgb}{0.000000,0.356863,0.509804}%
\pgfsetstrokecolor{currentstroke}%
\pgfsetdash{{0.800000pt}{1.320000pt}}{0.000000pt}%
\pgfpathmoveto{\pgfqpoint{0.631877in}{0.490730in}}%
\pgfpathlineto{\pgfqpoint{0.729731in}{0.490730in}}%
\pgfpathlineto{\pgfqpoint{0.827585in}{0.490730in}}%
\pgfpathlineto{\pgfqpoint{0.925438in}{0.490730in}}%
\pgfpathlineto{\pgfqpoint{1.023292in}{0.490730in}}%
\pgfpathlineto{\pgfqpoint{1.121145in}{0.490730in}}%
\pgfpathlineto{\pgfqpoint{1.218999in}{0.490730in}}%
\pgfpathlineto{\pgfqpoint{1.316852in}{0.490730in}}%
\pgfpathlineto{\pgfqpoint{1.414706in}{0.490730in}}%
\pgfpathlineto{\pgfqpoint{1.512559in}{0.490730in}}%
\pgfpathlineto{\pgfqpoint{1.610413in}{0.490730in}}%
\pgfpathlineto{\pgfqpoint{1.708266in}{0.490730in}}%
\pgfpathlineto{\pgfqpoint{1.806120in}{0.490730in}}%
\pgfpathlineto{\pgfqpoint{1.903973in}{0.490730in}}%
\pgfpathlineto{\pgfqpoint{2.001827in}{0.490730in}}%
\pgfpathlineto{\pgfqpoint{2.099681in}{0.490730in}}%
\pgfpathlineto{\pgfqpoint{2.197534in}{0.490730in}}%
\pgfpathlineto{\pgfqpoint{2.295388in}{0.490730in}}%
\pgfpathlineto{\pgfqpoint{2.393241in}{0.490730in}}%
\pgfpathlineto{\pgfqpoint{2.491095in}{0.490730in}}%
\pgfpathlineto{\pgfqpoint{2.588948in}{0.490730in}}%
\pgfpathlineto{\pgfqpoint{2.686802in}{0.490730in}}%
\pgfpathlineto{\pgfqpoint{2.784655in}{0.490730in}}%
\pgfpathlineto{\pgfqpoint{2.882509in}{0.490730in}}%
\pgfpathlineto{\pgfqpoint{2.980362in}{0.490730in}}%
\pgfpathlineto{\pgfqpoint{3.078216in}{0.498966in}}%
\pgfpathlineto{\pgfqpoint{3.176069in}{0.548377in}}%
\pgfpathlineto{\pgfqpoint{3.273923in}{0.721319in}}%
\pgfpathlineto{\pgfqpoint{3.371776in}{1.001319in}}%
\pgfpathlineto{\pgfqpoint{3.469630in}{1.314260in}}%
\pgfpathlineto{\pgfqpoint{3.567484in}{1.750730in}}%
\pgfpathlineto{\pgfqpoint{3.665337in}{2.129554in}}%
\pgfpathlineto{\pgfqpoint{3.763191in}{2.302495in}}%
\pgfpathlineto{\pgfqpoint{3.861044in}{2.409554in}}%
\pgfpathlineto{\pgfqpoint{3.958898in}{2.434260in}}%
\pgfpathlineto{\pgfqpoint{4.056751in}{2.434260in}}%
\pgfpathlineto{\pgfqpoint{4.154605in}{2.442495in}}%
\pgfusepath{stroke}%
\end{pgfscope}%
\begin{pgfscope}%
\pgfpathrectangle{\pgfqpoint{0.455741in}{0.385730in}}{\pgfqpoint{3.875000in}{2.310000in}}%
\pgfusepath{clip}%
\pgfsetbuttcap%
\pgfsetroundjoin%
\pgfsetlinewidth{0.803000pt}%
\definecolor{currentstroke}{rgb}{0.490196,0.588235,0.431373}%
\pgfsetstrokecolor{currentstroke}%
\pgfsetdash{{0.800000pt}{1.320000pt}}{0.000000pt}%
\pgfpathmoveto{\pgfqpoint{0.631877in}{0.490730in}}%
\pgfpathlineto{\pgfqpoint{0.729731in}{0.490730in}}%
\pgfpathlineto{\pgfqpoint{0.827585in}{0.490730in}}%
\pgfpathlineto{\pgfqpoint{0.925438in}{0.490730in}}%
\pgfpathlineto{\pgfqpoint{1.023292in}{0.490730in}}%
\pgfpathlineto{\pgfqpoint{1.121145in}{0.490730in}}%
\pgfpathlineto{\pgfqpoint{1.218999in}{0.490730in}}%
\pgfpathlineto{\pgfqpoint{1.316852in}{0.490730in}}%
\pgfpathlineto{\pgfqpoint{1.414706in}{0.490730in}}%
\pgfpathlineto{\pgfqpoint{1.512559in}{0.490730in}}%
\pgfpathlineto{\pgfqpoint{1.610413in}{0.490730in}}%
\pgfpathlineto{\pgfqpoint{1.708266in}{0.490730in}}%
\pgfpathlineto{\pgfqpoint{1.806120in}{0.490730in}}%
\pgfpathlineto{\pgfqpoint{1.903973in}{0.490730in}}%
\pgfpathlineto{\pgfqpoint{2.001827in}{0.490730in}}%
\pgfpathlineto{\pgfqpoint{2.099681in}{0.490730in}}%
\pgfpathlineto{\pgfqpoint{2.197534in}{0.490730in}}%
\pgfpathlineto{\pgfqpoint{2.295388in}{0.490730in}}%
\pgfpathlineto{\pgfqpoint{2.393241in}{0.490730in}}%
\pgfpathlineto{\pgfqpoint{2.491095in}{0.490730in}}%
\pgfpathlineto{\pgfqpoint{2.588948in}{0.490730in}}%
\pgfpathlineto{\pgfqpoint{2.686802in}{0.490730in}}%
\pgfpathlineto{\pgfqpoint{2.784655in}{0.490730in}}%
\pgfpathlineto{\pgfqpoint{2.882509in}{0.490730in}}%
\pgfpathlineto{\pgfqpoint{2.980362in}{0.507201in}}%
\pgfpathlineto{\pgfqpoint{3.078216in}{0.564848in}}%
\pgfpathlineto{\pgfqpoint{3.176069in}{0.688377in}}%
\pgfpathlineto{\pgfqpoint{3.273923in}{0.960142in}}%
\pgfpathlineto{\pgfqpoint{3.371776in}{1.413083in}}%
\pgfpathlineto{\pgfqpoint{3.469630in}{1.849554in}}%
\pgfpathlineto{\pgfqpoint{3.567484in}{2.154260in}}%
\pgfpathlineto{\pgfqpoint{3.665337in}{2.368377in}}%
\pgfpathlineto{\pgfqpoint{3.763191in}{2.434260in}}%
\pgfpathlineto{\pgfqpoint{3.861044in}{2.475436in}}%
\pgfpathlineto{\pgfqpoint{3.958898in}{2.491907in}}%
\pgfpathlineto{\pgfqpoint{4.056751in}{2.491907in}}%
\pgfpathlineto{\pgfqpoint{4.154605in}{2.500142in}}%
\pgfusepath{stroke}%
\end{pgfscope}%
\begin{pgfscope}%
\pgfpathrectangle{\pgfqpoint{0.455741in}{0.385730in}}{\pgfqpoint{3.875000in}{2.310000in}}%
\pgfusepath{clip}%
\pgfsetbuttcap%
\pgfsetroundjoin%
\pgfsetlinewidth{0.803000pt}%
\definecolor{currentstroke}{rgb}{0.843137,0.666667,0.313725}%
\pgfsetstrokecolor{currentstroke}%
\pgfsetdash{{0.800000pt}{1.320000pt}}{0.000000pt}%
\pgfpathmoveto{\pgfqpoint{0.631877in}{0.490730in}}%
\pgfpathlineto{\pgfqpoint{0.729731in}{0.490730in}}%
\pgfpathlineto{\pgfqpoint{0.827585in}{0.490730in}}%
\pgfpathlineto{\pgfqpoint{0.925438in}{0.490730in}}%
\pgfpathlineto{\pgfqpoint{1.023292in}{0.490730in}}%
\pgfpathlineto{\pgfqpoint{1.121145in}{0.490730in}}%
\pgfpathlineto{\pgfqpoint{1.218999in}{0.490730in}}%
\pgfpathlineto{\pgfqpoint{1.316852in}{0.490730in}}%
\pgfpathlineto{\pgfqpoint{1.414706in}{0.490730in}}%
\pgfpathlineto{\pgfqpoint{1.512559in}{0.490730in}}%
\pgfpathlineto{\pgfqpoint{1.610413in}{0.490730in}}%
\pgfpathlineto{\pgfqpoint{1.708266in}{0.490730in}}%
\pgfpathlineto{\pgfqpoint{1.806120in}{0.490730in}}%
\pgfpathlineto{\pgfqpoint{1.903973in}{0.490730in}}%
\pgfpathlineto{\pgfqpoint{2.001827in}{0.490730in}}%
\pgfpathlineto{\pgfqpoint{2.099681in}{0.490730in}}%
\pgfpathlineto{\pgfqpoint{2.197534in}{0.490730in}}%
\pgfpathlineto{\pgfqpoint{2.295388in}{0.490730in}}%
\pgfpathlineto{\pgfqpoint{2.393241in}{0.490730in}}%
\pgfpathlineto{\pgfqpoint{2.491095in}{0.490730in}}%
\pgfpathlineto{\pgfqpoint{2.588948in}{0.490730in}}%
\pgfpathlineto{\pgfqpoint{2.686802in}{0.490730in}}%
\pgfpathlineto{\pgfqpoint{2.784655in}{0.490730in}}%
\pgfpathlineto{\pgfqpoint{2.882509in}{0.490730in}}%
\pgfpathlineto{\pgfqpoint{2.980362in}{0.490730in}}%
\pgfpathlineto{\pgfqpoint{3.078216in}{0.507201in}}%
\pgfpathlineto{\pgfqpoint{3.176069in}{0.556613in}}%
\pgfpathlineto{\pgfqpoint{3.273923in}{0.688377in}}%
\pgfpathlineto{\pgfqpoint{3.371776in}{1.009554in}}%
\pgfpathlineto{\pgfqpoint{3.469630in}{1.437789in}}%
\pgfpathlineto{\pgfqpoint{3.567484in}{1.808377in}}%
\pgfpathlineto{\pgfqpoint{3.665337in}{2.088377in}}%
\pgfpathlineto{\pgfqpoint{3.763191in}{2.220142in}}%
\pgfpathlineto{\pgfqpoint{3.861044in}{2.360142in}}%
\pgfpathlineto{\pgfqpoint{3.958898in}{2.409554in}}%
\pgfpathlineto{\pgfqpoint{4.056751in}{2.426025in}}%
\pgfpathlineto{\pgfqpoint{4.154605in}{2.450730in}}%
\pgfusepath{stroke}%
\end{pgfscope}%
\begin{pgfscope}%
\pgfpathrectangle{\pgfqpoint{0.455741in}{0.385730in}}{\pgfqpoint{3.875000in}{2.310000in}}%
\pgfusepath{clip}%
\pgfsetbuttcap%
\pgfsetroundjoin%
\pgfsetlinewidth{0.803000pt}%
\definecolor{currentstroke}{rgb}{0.333333,0.333333,0.333333}%
\pgfsetstrokecolor{currentstroke}%
\pgfsetdash{{0.800000pt}{1.320000pt}}{0.000000pt}%
\pgfpathmoveto{\pgfqpoint{0.631877in}{0.490730in}}%
\pgfpathlineto{\pgfqpoint{0.729731in}{0.490730in}}%
\pgfpathlineto{\pgfqpoint{0.827585in}{0.490730in}}%
\pgfpathlineto{\pgfqpoint{0.925438in}{0.490730in}}%
\pgfpathlineto{\pgfqpoint{1.023292in}{0.490730in}}%
\pgfpathlineto{\pgfqpoint{1.121145in}{0.490730in}}%
\pgfpathlineto{\pgfqpoint{1.218999in}{0.490730in}}%
\pgfpathlineto{\pgfqpoint{1.316852in}{0.490730in}}%
\pgfpathlineto{\pgfqpoint{1.414706in}{0.490730in}}%
\pgfpathlineto{\pgfqpoint{1.512559in}{0.490730in}}%
\pgfpathlineto{\pgfqpoint{1.610413in}{0.490730in}}%
\pgfpathlineto{\pgfqpoint{1.708266in}{0.490730in}}%
\pgfpathlineto{\pgfqpoint{1.806120in}{0.490730in}}%
\pgfpathlineto{\pgfqpoint{1.903973in}{0.490730in}}%
\pgfpathlineto{\pgfqpoint{2.001827in}{0.490730in}}%
\pgfpathlineto{\pgfqpoint{2.099681in}{0.490730in}}%
\pgfpathlineto{\pgfqpoint{2.197534in}{0.490730in}}%
\pgfpathlineto{\pgfqpoint{2.295388in}{0.490730in}}%
\pgfpathlineto{\pgfqpoint{2.393241in}{0.490730in}}%
\pgfpathlineto{\pgfqpoint{2.491095in}{0.490730in}}%
\pgfpathlineto{\pgfqpoint{2.588948in}{0.490730in}}%
\pgfpathlineto{\pgfqpoint{2.686802in}{0.490730in}}%
\pgfpathlineto{\pgfqpoint{2.784655in}{0.490730in}}%
\pgfpathlineto{\pgfqpoint{2.882509in}{0.498966in}}%
\pgfpathlineto{\pgfqpoint{2.980362in}{0.523672in}}%
\pgfpathlineto{\pgfqpoint{3.078216in}{0.614260in}}%
\pgfpathlineto{\pgfqpoint{3.176069in}{0.844848in}}%
\pgfpathlineto{\pgfqpoint{3.273923in}{1.207201in}}%
\pgfpathlineto{\pgfqpoint{3.371776in}{1.627201in}}%
\pgfpathlineto{\pgfqpoint{3.469630in}{1.989554in}}%
\pgfpathlineto{\pgfqpoint{3.567484in}{2.277789in}}%
\pgfpathlineto{\pgfqpoint{3.665337in}{2.417789in}}%
\pgfpathlineto{\pgfqpoint{3.763191in}{2.475436in}}%
\pgfpathlineto{\pgfqpoint{3.861044in}{2.508377in}}%
\pgfpathlineto{\pgfqpoint{3.958898in}{2.516613in}}%
\pgfpathlineto{\pgfqpoint{4.056751in}{2.524848in}}%
\pgfpathlineto{\pgfqpoint{4.154605in}{2.524848in}}%
\pgfusepath{stroke}%
\end{pgfscope}%
\begin{pgfscope}%
\pgfsetrectcap%
\pgfsetmiterjoin%
\pgfsetlinewidth{0.501875pt}%
\definecolor{currentstroke}{rgb}{0.317647,0.317647,0.317647}%
\pgfsetstrokecolor{currentstroke}%
\pgfsetdash{}{0pt}%
\pgfpathmoveto{\pgfqpoint{0.455741in}{0.385730in}}%
\pgfpathlineto{\pgfqpoint{0.455741in}{2.695730in}}%
\pgfusepath{stroke}%
\end{pgfscope}%
\begin{pgfscope}%
\pgfsetrectcap%
\pgfsetmiterjoin%
\pgfsetlinewidth{0.501875pt}%
\definecolor{currentstroke}{rgb}{0.317647,0.317647,0.317647}%
\pgfsetstrokecolor{currentstroke}%
\pgfsetdash{}{0pt}%
\pgfpathmoveto{\pgfqpoint{0.455741in}{0.385730in}}%
\pgfpathlineto{\pgfqpoint{4.330741in}{0.385730in}}%
\pgfusepath{stroke}%
\end{pgfscope}%
\begin{pgfscope}%
\pgfsetbuttcap%
\pgfsetroundjoin%
\pgfsetlinewidth{0.803000pt}%
\definecolor{currentstroke}{rgb}{0.000000,0.000000,0.000000}%
\pgfsetstrokecolor{currentstroke}%
\pgfsetdash{{2.960000pt}{1.280000pt}}{0.000000pt}%
\pgfpathmoveto{\pgfqpoint{0.483508in}{2.635569in}}%
\pgfpathlineto{\pgfqpoint{0.557552in}{2.635569in}}%
\pgfusepath{stroke}%
\end{pgfscope}%
\begin{pgfscope}%
\definecolor{textcolor}{rgb}{0.000000,0.000000,0.000000}%
\pgfsetstrokecolor{textcolor}%
\pgfsetfillcolor{textcolor}%
\pgftext[x=0.603830in,y=2.603175in,left,base]{\color{textcolor}\rmfamily\fontsize{6.664000}{7.996800}\selectfont \(\displaystyle b \propto  \delta V = \SI{52.5}{\milli \V}\)}%
\end{pgfscope}%
\begin{pgfscope}%
\pgfsetrectcap%
\pgfsetroundjoin%
\pgfsetlinewidth{0.803000pt}%
\definecolor{currentstroke}{rgb}{0.000000,0.000000,0.000000}%
\pgfsetstrokecolor{currentstroke}%
\pgfsetdash{}{0pt}%
\pgfpathmoveto{\pgfqpoint{0.483508in}{2.515803in}}%
\pgfpathlineto{\pgfqpoint{0.557552in}{2.515803in}}%
\pgfusepath{stroke}%
\end{pgfscope}%
\begin{pgfscope}%
\definecolor{textcolor}{rgb}{0.000000,0.000000,0.000000}%
\pgfsetstrokecolor{textcolor}%
\pgfsetfillcolor{textcolor}%
\pgftext[x=0.603830in,y=2.483408in,left,base]{\color{textcolor}\rmfamily\fontsize{6.664000}{7.996800}\selectfont \(\displaystyle b \propto  \delta V = \SI{0.0}{\milli \V}\)}%
\end{pgfscope}%
\begin{pgfscope}%
\pgfsetbuttcap%
\pgfsetroundjoin%
\pgfsetlinewidth{0.803000pt}%
\definecolor{currentstroke}{rgb}{0.000000,0.000000,0.000000}%
\pgfsetstrokecolor{currentstroke}%
\pgfsetdash{{0.800000pt}{1.320000pt}}{0.000000pt}%
\pgfpathmoveto{\pgfqpoint{0.483508in}{2.396036in}}%
\pgfpathlineto{\pgfqpoint{0.557552in}{2.396036in}}%
\pgfusepath{stroke}%
\end{pgfscope}%
\begin{pgfscope}%
\definecolor{textcolor}{rgb}{0.000000,0.000000,0.000000}%
\pgfsetstrokecolor{textcolor}%
\pgfsetfillcolor{textcolor}%
\pgftext[x=0.603830in,y=2.363642in,left,base]{\color{textcolor}\rmfamily\fontsize{6.664000}{7.996800}\selectfont \(\displaystyle b \propto  \delta V = \SI{-52.4}{\milli \V}\)}%
\end{pgfscope}%
\end{pgfpicture}%
\makeatother%
\endgroup%

%	\end{center}
%	\caption[Sigmoid transfer function on \gls{dls}.]{Sigmoid transfer function on \gls{dls}. The resulting transfer function of the broadened free membrane distribution is set to  Shifting \gls{thres} and thereby changing the relative potential difference $\delta V$ is interchangeable with adding a bias term to the \gls{lif} neuron's activation.}
%	\label{transferfunction_with_bias}
%\end{figure}

The bias term $b$ is again replaced by the relative distance $\delta V$ between the resting potential and the threshold $b \propto \delta V$. As the analog core brings more and less favorable operating points, the resting potential is kept constant and only the threshold is shifted to change the bias (c.f. \cref{transferfunction_with_bias}).



\section{Experiment Setup on \gls{dls}}
\label{circlesimplementation}
For the classification of the Circles dataset a single hidden layer network is trained on-chip with \glsfirst{sgd}. The \gls{ppu} executes both the forward and a backward pass. This also includes generating the dataset and the input spikes. After a proper initialization routine, the training loop is started: the corresponding input rates of a randomly drawn data point $p(x,y)$ are fed into the network and the resulting output rates are evaluated by the \gls{sgd} routine on the \gls{ppu}. The computed parameter updates are then applied before the next iteration can start.

\paragraph{Forward Pass}

\begin{wrapfigure}{R}{0.35\textwidth}
	\centering
	\vspace{-1cm}
	\tikzset{block/.style={font={\sffamily\footnotesize},align=center}}%
\tikzset{box/.style={draw=black!90}}%
\tikzset{block label/.style={fill=white,font={\sffamily\footnotesize},inner sep=0.05cm}}%
\begin{tikzpicture}[
baseline=(anc.south),
font={\sffamily \small},
scale=1.,
>=latex,
transform shape,
line width=1.0\pgflinewidth,
anchor=center
]
\pgfdeclarelayer{background layer}
\pgfsetlayers{background layer,main}

\node[block,box,minimum width=3.0cm,minimum height=2.0cm] (syn) at (0,0) {};
\node[block,minimum width=3.0cm,minimum height=0.5cm,below=0.2cm of syn] (nrn) {};
\node[block,box,minimum width=3.0cm,minimum height=0.5cm,below=0.1cm of nrn] (cm) {parameter storage};

\node[block,minimum width=0.5cm,minimum height=2.0cm,left=0.05cm of syn] (drv) {};
\node[block,minimum width=0.5cm,minimum height=.0cm,left=0.17cm of syn] (drvname) {};
\node[block,minimum width=0.2cm,minimum height=2.0cm,left=0.05cm of drv] (padi) {};

\node[block,minimum width=3.0cm,minimum height=0.3cm,above=0.1cm of syn] (cadc) {};

% fitted analog core
\node[anchor=north west,gray,rotate=90] (anc) at (cm.south east) {};
\node[block,box,transparent,fit={(padi) (cadc) (anc)},inner sep=0.1cm] (anc bound) {};

\foreach \x in {0,1,...,5} {
	\draw[] (nrn.south west) ++ (\x*0.5,0.0) ++ (0.25,0.25) circle (0.2cm);
	\draw[gray,latex-] (syn.south west) ++ (\x*0.5+0.25,-0.22) -- ++(0.0,2.22);
	
	\foreach \y in {0,1,...,3}
	\draw[gray] (syn.south west) ++ (\x*0.5+0.25,0.25+\y*0.5) circle (0.02cm);
}

\foreach \y in {0,1,...,3} {
	\draw[] (drv.south west) ++ (0.0,\y*0.5) ++ (0.1,0.05) -- ++(0.0,0.4) -- ++(0.3,-0.2) -- cycle;
	\draw[gray] (syn.south west) ++ (-0.1,\y*0.5+0.25) -- ++(3.0,0.0) -- ++(0.1,0.0);
}

% arrows

% labels
\node[block label] at (syn) {synapse array};
\node[block label, rotate=90] at (drvname.south west) {synapse driver};
\node[block label] at (nrn) {neurons};
\end{tikzpicture}
	\caption[Overview of the synapse array.]{Overview of the synapse array. Figure adapted from \cite{billaudelle2019versatile}.} 
	\label{synapsearraysketch}
	\vspace{-1cm}
\end{wrapfigure}
The neuromorphic core of the \gls{dls} provides 32 analog neuron circuits which are interconnected by synapse array of $32 \times 32$ synapses (c.f. \cref{synapseschematics}). Any input spike to the network is injected row-wise by dedicated synapse drivers as either excitatory or inhibitory input current. At a column, the label of an incoming spike is compared with local address field and in case of a match the spike is relayed down to the designated neuron.

As for \glspl{ann}, the synapse must be able to seamlessly evolve from an inhibitory weight to an excitatory weight and vice versa. By design a single synapse cannot alternate between the signs of the input without changing the synapse driver's configuration which is time-consuming and can only be done row-wise. Instead of a single input row, two rows are used for the same input - one in excitatory and one inhibitory mode. In each column only one of the double-synapses is active and corresponds to the current sign of the synaptic weight. As the weight changes the sign, the active synapse is silenced by setting its weight to zero and the other one is reactivated. 

The first layer of the network, the input layer, is connected to the hidden units by using a total of four rows and eleven columns. A dedicated event router injects the generated spikes by the hidden neurons back into the network, with each hidden neuron requiring another double-row. The recurrent connections of the hidden layer are then summed by a single column over the respective $11 \times 2$ rows into the output unit. The weight of the noise sources remains fixed and thus they only require a single row, one for excitatory and one for inhibitory noise spikes.

A Poisson spike train generator on the \gls{ppu} provides a stream of random spikes over a certain measurement period. The provided spike resources are split into four branches, two for the excitatory and inhibitory noise source and two for both input units. Each noise branch has to supply twelve neurons with Poisson spike trains. In an attempt to reduce correlated input between the individual neurons the branches are further divided into eight channels and thus only four will share the noise with other ones.

%With the double-row structure, a total of six rows is used for the various inputs: two for the noise generation and four for the two-dimensional input. The recurrent connections of the hidden layer require another $22$ rows. In \cref{network_structure} the exact row assignment is sketched. Other than during the calibration process, where each neuron was calibrated individually, now the noise source needs to provide Poisson spike trains for all neurons in the hidden and output layer. As a compromise the source is divided further into eight channels. Four neurons will share the noise with other ones. The frequency of the noise in both layers is set to $\nu_\text{noise} \approx \SI{1.1}{\kilo \Hz}$ using weights with $w_\text{noise} = \pm 15$. The received input frequency varies for each layer. The input layer provides an input rate of up to $2 \times \SI{500}{\kilo \Hz}$ for each hidden unit. The output frequency of hidden units is limited by the spike counter and measurement period to $\gls{nuout} \approx \SI{111.3}{\kilo \Hz}$, leading to a theoretical maximum of $\gls{nuin}= 11 \times \gls{nuout} \approx 1219.6$ for the output layer. 
%A presynaptic spike train, e.g. generated by the \gls{ppu}, is injected row-wise into the $32\times32$ synapse array. Then, the synapses compare a dedicated address label of the incoming spike with their own. Once they match, the spike is weighted and forwarded to the column-wise connected neurons at the bottom as either inhibitory or excitatory current. Recurrent connections are made available by the use of a spike router, which routes the postsynaptic spike trains generated by a neuron back into the synapse array.
%A single input row cannot provide alternating negative and positive inputs to the same neuron. Changing the synapses configuration simply costs too much time. To ensure a continuous evolution of the weight from negative to positive, two rows are used simultaneously for the same input - one in excitatory and one inhibitory mode. As the weight changes sign, the unused row is set silent. At the start of the experiment the weights are randomly initialized using a uniform distribution limited to $w_{ij} \in [-25, 25]$.
%The initial conditions of the network have been proven to crucial for its learning success. A ``silent" hidden units has a hard to time become active. Adding a weight depending term to the initial zero bias $b_i^\text{init} \propto \sum_j w_{ij}$ establishes a certain initial fire rate and solves the problem for most cases.

\paragraph{Backward Pass}
The output rates obtained by the forward pass are then used to compute the updates for weights and biases according to \glsfirst{sgd}.

The general-purpose unit of the \gls{ppu} operates with a 32-bit architecture without hardware-support to floating point types. To increase the precision of the parameter updates, the computation is bit-shifted to the left. The calculated results are then stochastically rounded and shifted back right before being applied to the hardware. Stochastic rounding has been proven to be a viable workaround for deep learning with parameters of limited precision (\citealp{limitedprecisionpaper}). Typical methods of directed rounding to an integer $x$ are rounding up $\lceil x\rceil$, down $\lfloor x\rfloor$ or to the nearest neighbor $\lceil\frac{\lfloor2x\rfloor}{2}\rceil$. Stochastic rounding on the other hand rounds $x$ with a probability $p$ corresponding to the proximity to the upper or lower neighbor
\begin{equation*}
x_\text{stoch}(x) = 
\begin{cases}
\lfloor x \rfloor \quad \quad &\text{with } p = 1 - (x - \lfloor x \rfloor), \\
\lfloor x \rfloor + 1 &\text{with } p = 1 - \lfloor x \rfloor. \\
\end{cases}
\end{equation*}

Direct access to the analog weight parameters from the general-purpose unit is time-consuming and computation intensive due to the lack of parallelizations. With feedback alignment, the need of the current weight information becomes irrelevant to the plasticity rule. Only the computed weight update needs to be applied somehow. This can be implemented by using the vector unit. The \gls{simd} instructions allow parallel access to all weights and therefore it is not only used to efficiently apply the weight updates but also to silence and reactivated the respective synapses in the double-row structure of the synapse array. The \emph{assembly} code-base for this implementation has been co-written by Sebastian Billaudelle and Benjamin Cramer.

During training the weights can max out, due to their limited \SI{6}{\bit} range. The implementation of a simple weight regularization using the convenient parallel access of the vector unit solves this issue. The regularization of both, excitatory and inhibitory weights $w_{ij}^\text{inh},\; w_{ij}^\text{exc}\equiv w_{ij} \in [0,2^6)$ is implemented using bit shifts to minimize the implementation effort and given by
\begin{align*}
\text{Reg}(w_{ij}) &= w_{ij} - \left(\left(w_{ij} \ll 1\right) \gg 6\right) \nonumber \\
&= w_{ij} - \left\lfloor \frac{\lfloor 2 \cdot w_{ij}  \rfloor}{2^{6}} \right\rfloor.					
%	&= w_{ij} - \left\lfloor \lfloor 2 w_{ij} \rfloor \cdot 2^{-6} \right\rfloor
\end{align*}
The impact of the regularization is fine tuned by applying it only with a probability $p=\nicefrac{1}{4}$.

The bias is set by lowering or raising the \gls{dac}-value of the threshold potential. The resolution of the capacitive parameter memory equals to roughly \SI{1.7}{\milli \V}, which in turn translates to a change of the input frequency of about \SI{9.9}{\kilo \Hz}. As a comparison the maximum input frequency of a neuron is around \SI{1}{\mega \Hz}, but the magnitude of the typical presynaptic activity will only be at a few \SI{100}{\kilo \Hz}. Hence, the bias has a rather coarse resolution.

However, compared to the speed of the accelerated neuromorphic core, the \acrlong{dac} is slow. Instead of an instant and large change of the bias, a continuous and slow implementation can reduce potential negative effects of the coarse bias resolution, especially in combination with a low learning rate.

\paragraph{Initial Conditions and Hyperparameters}
The final experiment setup involves the setting of initial conditions, neuron parameters and the tuning of hyper-parameters. The latter describe a set of parameters, that is not changed during the training process but their choice often determines whether or not the training succeeds (\citealp{Goodfellow-et-al-2016}). A subset of the most relevant parameters is listed in \cref{circlesinitparameters}.

\begin{table}[t!]\centering\ra{1.3}
	\begin{tabular}{@{}rlll@{}}\toprule
		& Parameter								& 	hidden neurons 			& 	output neurons 	\\ \midrule
		& initial weights $w_{ij}^\text{init}$	& 	 $\in[-25, 25]$			&	$\in[-25, 25]$	\\
		& initial bias $b_{i}^\text{init}$		& $\propto \frac{1}{2} \sum_j w_{ij}$ &	0		\\
		& learning rate of bias  $\eta_\text{w}$&	$165e4$					& 	$5.5e4$			\\
		& learning rate of bias  $\eta_\text{b}$&	$2.3e4$					& 	$0.23e4$		\\
		& resting potential \gls{v_leak}		&	$\SI{0.54}{\V}$ 		& 	$\SI{0.54}{\V}$	\\
		& reset potential \gls{v_reset}			&	$\SI{0.01}{\V}$			& 	$\SI{0.01}{\V}$	\\
		& refractory period \gls{refrac}		&	$\SI{9}{\micro \s}$			& 	$\SI{0.01}{\V}$	\\
		& range of threshold \gls{thres} 				&	\SIrange{0.45}{0.63}{\V}&	\SIrange{0.45}{0.63}{\V}\\
		& max input frequency $\nu_{\text{in}}$	&	$2 \times \SI{500}{\kilo \Hz}$	& $11 \times \SI{111.3}{\kilo\Hz}$	\\
		\bottomrule
	\end{tabular}
	\caption[Initial, hyper and neuron parameters per layer.]{Initial, hyper and neuron parameters per layer. Some parameters require a different setting for the hidden and the output layer.}
	\label{circlesinitparameters}
\end{table}

At the beginning of the experiment all weights of both layers are randomly initialized using a uniform distribution ranging from $-25$ to $25$. 

The initial bias of both layers is set to zero, i.e. $\gls{thres} = \gls{v_leak}$. In this setting it has been observed that a ``silent" hidden unit at the start of the training most likely remains inactive throughout the training process. As a workaround, a weight dependent term is added to the zero bias, such that $b_i^\text{init} \propto \sum_j w_{ij}$. This ensures an initial firing rate of every neuron at the beginning of the training.

In the final setting, the incline of the activation function needed to be raised by a fair amount resulting in a noise frequency $\nu_\text{noise}$ of roughly \SI{1.1}{\kilo \Hz}, which corresponds only to a fraction of the available resource. The impact of the noise spikes is tuned by the synaptic weight $w_\text{noise}$ which is set to $\pm 15$ for excitatory and inhibitory spikes respectively.

The input layer, on the other hand, exploits almost the maximum with firing rates ranging from \SIrange{0}{500}{\kilo \Hz}. The maximum input rate of a hidden unit is then given by the combined input of both input units. The same applies to the output layer where the output rates of the hidden units are combined to a maximum input of $11 \times \SI{111.3}{\kilo \Hz}$.

The learning rates may appear unusually high, which is mainly due to the lack of normalization. The normalization was not done beforehand in order to reduce the risk of eliminating small contributions of variables in the on-chip computation of the parameter updates. Furthermore the learning rates are divided into layer and parameter, to cope with the individual magnitudes. Even without the necessity to norm resulting parameter updates, it is not unusual to tune learning rates separately (\citealp{Goodfellow-et-al-2016}). The final rates are set to provide a fair amount of activity for both layers and parameters (c.f. \cref{monitoringplots}).

\section{Training}
The implementation of a full on-chip training is achieved by generating the required training dataset on the fly. In this way, each forward pass is performed with a randomly drawn data point from either of the two rings from \cref{circlestask}. The backward pass is performed directly after the first training sample, i.e. the minibatch size is one.

\begin{figure}
	\begin{center}
		%% Creator: Matplotlib, PGF backend
%%
%% To include the figure in your LaTeX document, write
%%   \input{<filename>.pgf}
%%
%% Make sure the required packages are loaded in your preamble
%%   \usepackage{pgf}
%%
%% Figures using additional raster images can only be included by \input if
%% they are in the same directory as the main LaTeX file. For loading figures
%% from other directories you can use the `import` package
%%   \usepackage{import}
%% and then include the figures with
%%   \import{<path to file>}{<filename>.pgf}
%%
%% Matplotlib used the following preamble
%%   \usepackage{amsmath} \usepackage{pifont} \usepackage{xcolor} \definecolor{green}{HTML}{467821} \definecolor{red}{HTML}{CF4457} \usepackage[detect-all]{siunitx}
%%   \usepackage{fontspec}
%%
\begingroup%
\makeatletter%
\begin{pgfpicture}%
\pgfpathrectangle{\pgfpointorigin}{\pgfqpoint{5.197336in}{3.530797in}}%
\pgfusepath{use as bounding box, clip}%
\begin{pgfscope}%
\pgfsetbuttcap%
\pgfsetmiterjoin%
\pgfsetlinewidth{0.000000pt}%
\definecolor{currentstroke}{rgb}{0.000000,0.000000,0.000000}%
\pgfsetstrokecolor{currentstroke}%
\pgfsetstrokeopacity{0.000000}%
\pgfsetdash{}{0pt}%
\pgfpathmoveto{\pgfqpoint{0.000000in}{0.000000in}}%
\pgfpathlineto{\pgfqpoint{5.197336in}{0.000000in}}%
\pgfpathlineto{\pgfqpoint{5.197336in}{3.530797in}}%
\pgfpathlineto{\pgfqpoint{0.000000in}{3.530797in}}%
\pgfpathclose%
\pgfusepath{}%
\end{pgfscope}%
\begin{pgfscope}%
\pgfsetbuttcap%
\pgfsetmiterjoin%
\pgfsetlinewidth{0.000000pt}%
\definecolor{currentstroke}{rgb}{0.000000,0.000000,0.000000}%
\pgfsetstrokecolor{currentstroke}%
\pgfsetstrokeopacity{0.000000}%
\pgfsetdash{}{0pt}%
\pgfpathmoveto{\pgfqpoint{0.447336in}{2.026146in}}%
\pgfpathlineto{\pgfqpoint{5.097336in}{2.026146in}}%
\pgfpathlineto{\pgfqpoint{5.097336in}{3.430797in}}%
\pgfpathlineto{\pgfqpoint{0.447336in}{3.430797in}}%
\pgfpathclose%
\pgfusepath{}%
\end{pgfscope}%
\begin{pgfscope}%
\pgfsetbuttcap%
\pgfsetroundjoin%
\definecolor{currentfill}{rgb}{0.317647,0.317647,0.317647}%
\pgfsetfillcolor{currentfill}%
\pgfsetlinewidth{0.501875pt}%
\definecolor{currentstroke}{rgb}{0.317647,0.317647,0.317647}%
\pgfsetstrokecolor{currentstroke}%
\pgfsetdash{}{0pt}%
\pgfsys@defobject{currentmarker}{\pgfqpoint{0.000000in}{-0.020833in}}{\pgfqpoint{0.000000in}{0.000000in}}{%
\pgfpathmoveto{\pgfqpoint{0.000000in}{0.000000in}}%
\pgfpathlineto{\pgfqpoint{0.000000in}{-0.020833in}}%
\pgfusepath{stroke,fill}%
}%
\begin{pgfscope}%
\pgfsys@transformshift{0.658700in}{2.026146in}%
\pgfsys@useobject{currentmarker}{}%
\end{pgfscope}%
\end{pgfscope}%
\begin{pgfscope}%
\pgfsetbuttcap%
\pgfsetroundjoin%
\definecolor{currentfill}{rgb}{0.317647,0.317647,0.317647}%
\pgfsetfillcolor{currentfill}%
\pgfsetlinewidth{0.501875pt}%
\definecolor{currentstroke}{rgb}{0.317647,0.317647,0.317647}%
\pgfsetstrokecolor{currentstroke}%
\pgfsetdash{}{0pt}%
\pgfsys@defobject{currentmarker}{\pgfqpoint{0.000000in}{-0.020833in}}{\pgfqpoint{0.000000in}{0.000000in}}{%
\pgfpathmoveto{\pgfqpoint{0.000000in}{0.000000in}}%
\pgfpathlineto{\pgfqpoint{0.000000in}{-0.020833in}}%
\pgfusepath{stroke,fill}%
}%
\begin{pgfscope}%
\pgfsys@transformshift{1.187770in}{2.026146in}%
\pgfsys@useobject{currentmarker}{}%
\end{pgfscope}%
\end{pgfscope}%
\begin{pgfscope}%
\pgfsetbuttcap%
\pgfsetroundjoin%
\definecolor{currentfill}{rgb}{0.317647,0.317647,0.317647}%
\pgfsetfillcolor{currentfill}%
\pgfsetlinewidth{0.501875pt}%
\definecolor{currentstroke}{rgb}{0.317647,0.317647,0.317647}%
\pgfsetstrokecolor{currentstroke}%
\pgfsetdash{}{0pt}%
\pgfsys@defobject{currentmarker}{\pgfqpoint{0.000000in}{-0.020833in}}{\pgfqpoint{0.000000in}{0.000000in}}{%
\pgfpathmoveto{\pgfqpoint{0.000000in}{0.000000in}}%
\pgfpathlineto{\pgfqpoint{0.000000in}{-0.020833in}}%
\pgfusepath{stroke,fill}%
}%
\begin{pgfscope}%
\pgfsys@transformshift{1.716841in}{2.026146in}%
\pgfsys@useobject{currentmarker}{}%
\end{pgfscope}%
\end{pgfscope}%
\begin{pgfscope}%
\pgfsetbuttcap%
\pgfsetroundjoin%
\definecolor{currentfill}{rgb}{0.317647,0.317647,0.317647}%
\pgfsetfillcolor{currentfill}%
\pgfsetlinewidth{0.501875pt}%
\definecolor{currentstroke}{rgb}{0.317647,0.317647,0.317647}%
\pgfsetstrokecolor{currentstroke}%
\pgfsetdash{}{0pt}%
\pgfsys@defobject{currentmarker}{\pgfqpoint{0.000000in}{-0.020833in}}{\pgfqpoint{0.000000in}{0.000000in}}{%
\pgfpathmoveto{\pgfqpoint{0.000000in}{0.000000in}}%
\pgfpathlineto{\pgfqpoint{0.000000in}{-0.020833in}}%
\pgfusepath{stroke,fill}%
}%
\begin{pgfscope}%
\pgfsys@transformshift{2.245911in}{2.026146in}%
\pgfsys@useobject{currentmarker}{}%
\end{pgfscope}%
\end{pgfscope}%
\begin{pgfscope}%
\pgfsetbuttcap%
\pgfsetroundjoin%
\definecolor{currentfill}{rgb}{0.317647,0.317647,0.317647}%
\pgfsetfillcolor{currentfill}%
\pgfsetlinewidth{0.501875pt}%
\definecolor{currentstroke}{rgb}{0.317647,0.317647,0.317647}%
\pgfsetstrokecolor{currentstroke}%
\pgfsetdash{}{0pt}%
\pgfsys@defobject{currentmarker}{\pgfqpoint{0.000000in}{-0.020833in}}{\pgfqpoint{0.000000in}{0.000000in}}{%
\pgfpathmoveto{\pgfqpoint{0.000000in}{0.000000in}}%
\pgfpathlineto{\pgfqpoint{0.000000in}{-0.020833in}}%
\pgfusepath{stroke,fill}%
}%
\begin{pgfscope}%
\pgfsys@transformshift{2.774981in}{2.026146in}%
\pgfsys@useobject{currentmarker}{}%
\end{pgfscope}%
\end{pgfscope}%
\begin{pgfscope}%
\pgfsetbuttcap%
\pgfsetroundjoin%
\definecolor{currentfill}{rgb}{0.317647,0.317647,0.317647}%
\pgfsetfillcolor{currentfill}%
\pgfsetlinewidth{0.501875pt}%
\definecolor{currentstroke}{rgb}{0.317647,0.317647,0.317647}%
\pgfsetstrokecolor{currentstroke}%
\pgfsetdash{}{0pt}%
\pgfsys@defobject{currentmarker}{\pgfqpoint{0.000000in}{-0.020833in}}{\pgfqpoint{0.000000in}{0.000000in}}{%
\pgfpathmoveto{\pgfqpoint{0.000000in}{0.000000in}}%
\pgfpathlineto{\pgfqpoint{0.000000in}{-0.020833in}}%
\pgfusepath{stroke,fill}%
}%
\begin{pgfscope}%
\pgfsys@transformshift{3.304052in}{2.026146in}%
\pgfsys@useobject{currentmarker}{}%
\end{pgfscope}%
\end{pgfscope}%
\begin{pgfscope}%
\pgfsetbuttcap%
\pgfsetroundjoin%
\definecolor{currentfill}{rgb}{0.317647,0.317647,0.317647}%
\pgfsetfillcolor{currentfill}%
\pgfsetlinewidth{0.501875pt}%
\definecolor{currentstroke}{rgb}{0.317647,0.317647,0.317647}%
\pgfsetstrokecolor{currentstroke}%
\pgfsetdash{}{0pt}%
\pgfsys@defobject{currentmarker}{\pgfqpoint{0.000000in}{-0.020833in}}{\pgfqpoint{0.000000in}{0.000000in}}{%
\pgfpathmoveto{\pgfqpoint{0.000000in}{0.000000in}}%
\pgfpathlineto{\pgfqpoint{0.000000in}{-0.020833in}}%
\pgfusepath{stroke,fill}%
}%
\begin{pgfscope}%
\pgfsys@transformshift{3.833122in}{2.026146in}%
\pgfsys@useobject{currentmarker}{}%
\end{pgfscope}%
\end{pgfscope}%
\begin{pgfscope}%
\pgfsetbuttcap%
\pgfsetroundjoin%
\definecolor{currentfill}{rgb}{0.317647,0.317647,0.317647}%
\pgfsetfillcolor{currentfill}%
\pgfsetlinewidth{0.501875pt}%
\definecolor{currentstroke}{rgb}{0.317647,0.317647,0.317647}%
\pgfsetstrokecolor{currentstroke}%
\pgfsetdash{}{0pt}%
\pgfsys@defobject{currentmarker}{\pgfqpoint{0.000000in}{-0.020833in}}{\pgfqpoint{0.000000in}{0.000000in}}{%
\pgfpathmoveto{\pgfqpoint{0.000000in}{0.000000in}}%
\pgfpathlineto{\pgfqpoint{0.000000in}{-0.020833in}}%
\pgfusepath{stroke,fill}%
}%
\begin{pgfscope}%
\pgfsys@transformshift{4.362193in}{2.026146in}%
\pgfsys@useobject{currentmarker}{}%
\end{pgfscope}%
\end{pgfscope}%
\begin{pgfscope}%
\pgfsetbuttcap%
\pgfsetroundjoin%
\definecolor{currentfill}{rgb}{0.317647,0.317647,0.317647}%
\pgfsetfillcolor{currentfill}%
\pgfsetlinewidth{0.501875pt}%
\definecolor{currentstroke}{rgb}{0.317647,0.317647,0.317647}%
\pgfsetstrokecolor{currentstroke}%
\pgfsetdash{}{0pt}%
\pgfsys@defobject{currentmarker}{\pgfqpoint{0.000000in}{-0.020833in}}{\pgfqpoint{0.000000in}{0.000000in}}{%
\pgfpathmoveto{\pgfqpoint{0.000000in}{0.000000in}}%
\pgfpathlineto{\pgfqpoint{0.000000in}{-0.020833in}}%
\pgfusepath{stroke,fill}%
}%
\begin{pgfscope}%
\pgfsys@transformshift{4.891263in}{2.026146in}%
\pgfsys@useobject{currentmarker}{}%
\end{pgfscope}%
\end{pgfscope}%
\begin{pgfscope}%
\pgfsetbuttcap%
\pgfsetroundjoin%
\definecolor{currentfill}{rgb}{0.317647,0.317647,0.317647}%
\pgfsetfillcolor{currentfill}%
\pgfsetlinewidth{0.501875pt}%
\definecolor{currentstroke}{rgb}{0.317647,0.317647,0.317647}%
\pgfsetstrokecolor{currentstroke}%
\pgfsetdash{}{0pt}%
\pgfsys@defobject{currentmarker}{\pgfqpoint{-0.020833in}{0.000000in}}{\pgfqpoint{0.000000in}{0.000000in}}{%
\pgfpathmoveto{\pgfqpoint{0.000000in}{0.000000in}}%
\pgfpathlineto{\pgfqpoint{-0.020833in}{0.000000in}}%
\pgfusepath{stroke,fill}%
}%
\begin{pgfscope}%
\pgfsys@transformshift{0.447336in}{2.165108in}%
\pgfsys@useobject{currentmarker}{}%
\end{pgfscope}%
\end{pgfscope}%
\begin{pgfscope}%
\definecolor{textcolor}{rgb}{0.317647,0.317647,0.317647}%
\pgfsetstrokecolor{textcolor}%
\pgfsetfillcolor{textcolor}%
\pgftext[x=0.261763in,y=2.124962in,left,base]{\color{textcolor}\rmfamily\fontsize{8.330000}{9.996000}\selectfont \(\displaystyle 0.2\)}%
\end{pgfscope}%
\begin{pgfscope}%
\pgfsetbuttcap%
\pgfsetroundjoin%
\definecolor{currentfill}{rgb}{0.317647,0.317647,0.317647}%
\pgfsetfillcolor{currentfill}%
\pgfsetlinewidth{0.501875pt}%
\definecolor{currentstroke}{rgb}{0.317647,0.317647,0.317647}%
\pgfsetstrokecolor{currentstroke}%
\pgfsetdash{}{0pt}%
\pgfsys@defobject{currentmarker}{\pgfqpoint{-0.020833in}{0.000000in}}{\pgfqpoint{0.000000in}{0.000000in}}{%
\pgfpathmoveto{\pgfqpoint{0.000000in}{0.000000in}}%
\pgfpathlineto{\pgfqpoint{-0.020833in}{0.000000in}}%
\pgfusepath{stroke,fill}%
}%
\begin{pgfscope}%
\pgfsys@transformshift{0.447336in}{2.465568in}%
\pgfsys@useobject{currentmarker}{}%
\end{pgfscope}%
\end{pgfscope}%
\begin{pgfscope}%
\definecolor{textcolor}{rgb}{0.317647,0.317647,0.317647}%
\pgfsetstrokecolor{textcolor}%
\pgfsetfillcolor{textcolor}%
\pgftext[x=0.261763in,y=2.425422in,left,base]{\color{textcolor}\rmfamily\fontsize{8.330000}{9.996000}\selectfont \(\displaystyle 0.4\)}%
\end{pgfscope}%
\begin{pgfscope}%
\pgfsetbuttcap%
\pgfsetroundjoin%
\definecolor{currentfill}{rgb}{0.317647,0.317647,0.317647}%
\pgfsetfillcolor{currentfill}%
\pgfsetlinewidth{0.501875pt}%
\definecolor{currentstroke}{rgb}{0.317647,0.317647,0.317647}%
\pgfsetstrokecolor{currentstroke}%
\pgfsetdash{}{0pt}%
\pgfsys@defobject{currentmarker}{\pgfqpoint{-0.020833in}{0.000000in}}{\pgfqpoint{0.000000in}{0.000000in}}{%
\pgfpathmoveto{\pgfqpoint{0.000000in}{0.000000in}}%
\pgfpathlineto{\pgfqpoint{-0.020833in}{0.000000in}}%
\pgfusepath{stroke,fill}%
}%
\begin{pgfscope}%
\pgfsys@transformshift{0.447336in}{2.766029in}%
\pgfsys@useobject{currentmarker}{}%
\end{pgfscope}%
\end{pgfscope}%
\begin{pgfscope}%
\definecolor{textcolor}{rgb}{0.317647,0.317647,0.317647}%
\pgfsetstrokecolor{textcolor}%
\pgfsetfillcolor{textcolor}%
\pgftext[x=0.261763in,y=2.725883in,left,base]{\color{textcolor}\rmfamily\fontsize{8.330000}{9.996000}\selectfont \(\displaystyle 0.6\)}%
\end{pgfscope}%
\begin{pgfscope}%
\pgfsetbuttcap%
\pgfsetroundjoin%
\definecolor{currentfill}{rgb}{0.317647,0.317647,0.317647}%
\pgfsetfillcolor{currentfill}%
\pgfsetlinewidth{0.501875pt}%
\definecolor{currentstroke}{rgb}{0.317647,0.317647,0.317647}%
\pgfsetstrokecolor{currentstroke}%
\pgfsetdash{}{0pt}%
\pgfsys@defobject{currentmarker}{\pgfqpoint{-0.020833in}{0.000000in}}{\pgfqpoint{0.000000in}{0.000000in}}{%
\pgfpathmoveto{\pgfqpoint{0.000000in}{0.000000in}}%
\pgfpathlineto{\pgfqpoint{-0.020833in}{0.000000in}}%
\pgfusepath{stroke,fill}%
}%
\begin{pgfscope}%
\pgfsys@transformshift{0.447336in}{3.066489in}%
\pgfsys@useobject{currentmarker}{}%
\end{pgfscope}%
\end{pgfscope}%
\begin{pgfscope}%
\definecolor{textcolor}{rgb}{0.317647,0.317647,0.317647}%
\pgfsetstrokecolor{textcolor}%
\pgfsetfillcolor{textcolor}%
\pgftext[x=0.261763in,y=3.026343in,left,base]{\color{textcolor}\rmfamily\fontsize{8.330000}{9.996000}\selectfont \(\displaystyle 0.8\)}%
\end{pgfscope}%
\begin{pgfscope}%
\pgfsetbuttcap%
\pgfsetroundjoin%
\definecolor{currentfill}{rgb}{0.317647,0.317647,0.317647}%
\pgfsetfillcolor{currentfill}%
\pgfsetlinewidth{0.501875pt}%
\definecolor{currentstroke}{rgb}{0.317647,0.317647,0.317647}%
\pgfsetstrokecolor{currentstroke}%
\pgfsetdash{}{0pt}%
\pgfsys@defobject{currentmarker}{\pgfqpoint{-0.020833in}{0.000000in}}{\pgfqpoint{0.000000in}{0.000000in}}{%
\pgfpathmoveto{\pgfqpoint{0.000000in}{0.000000in}}%
\pgfpathlineto{\pgfqpoint{-0.020833in}{0.000000in}}%
\pgfusepath{stroke,fill}%
}%
\begin{pgfscope}%
\pgfsys@transformshift{0.447336in}{3.366949in}%
\pgfsys@useobject{currentmarker}{}%
\end{pgfscope}%
\end{pgfscope}%
\begin{pgfscope}%
\definecolor{textcolor}{rgb}{0.317647,0.317647,0.317647}%
\pgfsetstrokecolor{textcolor}%
\pgfsetfillcolor{textcolor}%
\pgftext[x=0.261763in,y=3.326803in,left,base]{\color{textcolor}\rmfamily\fontsize{8.330000}{9.996000}\selectfont \(\displaystyle 1.0\)}%
\end{pgfscope}%
\begin{pgfscope}%
\definecolor{textcolor}{rgb}{0.317647,0.317647,0.317647}%
\pgfsetstrokecolor{textcolor}%
\pgfsetfillcolor{textcolor}%
\pgftext[x=0.206207in,y=2.728471in,,bottom,rotate=90.000000]{\color{textcolor}\rmfamily\fontsize{8.330000}{9.996000}\selectfont Accuracy}%
\end{pgfscope}%
\begin{pgfscope}%
\pgfpathrectangle{\pgfqpoint{0.447336in}{2.026146in}}{\pgfqpoint{4.650000in}{1.404651in}}%
\pgfusepath{clip}%
\pgfsetrectcap%
\pgfsetroundjoin%
\pgfsetlinewidth{0.803000pt}%
\definecolor{currentstroke}{rgb}{0.333333,0.333333,0.333333}%
\pgfsetstrokecolor{currentstroke}%
\pgfsetdash{}{0pt}%
\pgfpathmoveto{\pgfqpoint{0.658700in}{2.593264in}}%
\pgfpathlineto{\pgfqpoint{0.663990in}{2.615799in}}%
\pgfpathlineto{\pgfqpoint{0.669281in}{2.623310in}}%
\pgfpathlineto{\pgfqpoint{0.674572in}{2.653356in}}%
\pgfpathlineto{\pgfqpoint{0.679863in}{2.510637in}}%
\pgfpathlineto{\pgfqpoint{0.685153in}{2.728471in}}%
\pgfpathlineto{\pgfqpoint{0.690444in}{2.698425in}}%
\pgfpathlineto{\pgfqpoint{0.695735in}{2.713448in}}%
\pgfpathlineto{\pgfqpoint{0.701025in}{2.698425in}}%
\pgfpathlineto{\pgfqpoint{0.711607in}{2.713448in}}%
\pgfpathlineto{\pgfqpoint{0.716897in}{2.735983in}}%
\pgfpathlineto{\pgfqpoint{0.722188in}{2.713448in}}%
\pgfpathlineto{\pgfqpoint{0.727479in}{2.623310in}}%
\pgfpathlineto{\pgfqpoint{0.732770in}{2.315338in}}%
\pgfpathlineto{\pgfqpoint{0.738060in}{2.578241in}}%
\pgfpathlineto{\pgfqpoint{0.743351in}{2.660868in}}%
\pgfpathlineto{\pgfqpoint{0.748642in}{2.089993in}}%
\pgfpathlineto{\pgfqpoint{0.753932in}{2.142574in}}%
\pgfpathlineto{\pgfqpoint{0.759223in}{2.638333in}}%
\pgfpathlineto{\pgfqpoint{0.764514in}{2.428011in}}%
\pgfpathlineto{\pgfqpoint{0.769805in}{2.728471in}}%
\pgfpathlineto{\pgfqpoint{0.775095in}{2.480591in}}%
\pgfpathlineto{\pgfqpoint{0.780386in}{2.683402in}}%
\pgfpathlineto{\pgfqpoint{0.785677in}{2.788563in}}%
\pgfpathlineto{\pgfqpoint{0.790967in}{2.563218in}}%
\pgfpathlineto{\pgfqpoint{0.796258in}{2.773540in}}%
\pgfpathlineto{\pgfqpoint{0.801549in}{2.330361in}}%
\pgfpathlineto{\pgfqpoint{0.806839in}{2.225200in}}%
\pgfpathlineto{\pgfqpoint{0.812130in}{2.292804in}}%
\pgfpathlineto{\pgfqpoint{0.817421in}{2.833632in}}%
\pgfpathlineto{\pgfqpoint{0.822712in}{2.826121in}}%
\pgfpathlineto{\pgfqpoint{0.828002in}{2.968839in}}%
\pgfpathlineto{\pgfqpoint{0.833293in}{3.028931in}}%
\pgfpathlineto{\pgfqpoint{0.843874in}{3.028931in}}%
\pgfpathlineto{\pgfqpoint{0.849165in}{2.563218in}}%
\pgfpathlineto{\pgfqpoint{0.854456in}{3.021420in}}%
\pgfpathlineto{\pgfqpoint{0.859747in}{2.923770in}}%
\pgfpathlineto{\pgfqpoint{0.865037in}{2.878701in}}%
\pgfpathlineto{\pgfqpoint{0.870328in}{2.968839in}}%
\pgfpathlineto{\pgfqpoint{0.875619in}{2.908747in}}%
\pgfpathlineto{\pgfqpoint{0.886200in}{2.428011in}}%
\pgfpathlineto{\pgfqpoint{0.891491in}{2.367919in}}%
\pgfpathlineto{\pgfqpoint{0.896781in}{2.428011in}}%
\pgfpathlineto{\pgfqpoint{0.902072in}{2.412988in}}%
\pgfpathlineto{\pgfqpoint{0.907363in}{2.480591in}}%
\pgfpathlineto{\pgfqpoint{0.912654in}{2.713448in}}%
\pgfpathlineto{\pgfqpoint{0.923235in}{2.938793in}}%
\pgfpathlineto{\pgfqpoint{0.928526in}{2.946305in}}%
\pgfpathlineto{\pgfqpoint{0.933816in}{3.171650in}}%
\pgfpathlineto{\pgfqpoint{0.939107in}{2.976351in}}%
\pgfpathlineto{\pgfqpoint{0.944398in}{2.848655in}}%
\pgfpathlineto{\pgfqpoint{0.949688in}{2.886213in}}%
\pgfpathlineto{\pgfqpoint{0.954979in}{3.171650in}}%
\pgfpathlineto{\pgfqpoint{0.960270in}{3.149115in}}%
\pgfpathlineto{\pgfqpoint{0.965561in}{3.043954in}}%
\pgfpathlineto{\pgfqpoint{0.970851in}{3.021420in}}%
\pgfpathlineto{\pgfqpoint{0.976142in}{3.028931in}}%
\pgfpathlineto{\pgfqpoint{0.981433in}{3.051466in}}%
\pgfpathlineto{\pgfqpoint{0.986723in}{3.104046in}}%
\pgfpathlineto{\pgfqpoint{0.992014in}{2.946305in}}%
\pgfpathlineto{\pgfqpoint{0.997305in}{3.141604in}}%
\pgfpathlineto{\pgfqpoint{1.002596in}{3.104046in}}%
\pgfpathlineto{\pgfqpoint{1.007886in}{3.089023in}}%
\pgfpathlineto{\pgfqpoint{1.013177in}{3.066489in}}%
\pgfpathlineto{\pgfqpoint{1.018468in}{3.134092in}}%
\pgfpathlineto{\pgfqpoint{1.023758in}{3.111558in}}%
\pgfpathlineto{\pgfqpoint{1.029049in}{3.104046in}}%
\pgfpathlineto{\pgfqpoint{1.034340in}{3.104046in}}%
\pgfpathlineto{\pgfqpoint{1.039630in}{3.171650in}}%
\pgfpathlineto{\pgfqpoint{1.044921in}{3.164138in}}%
\pgfpathlineto{\pgfqpoint{1.050212in}{3.119069in}}%
\pgfpathlineto{\pgfqpoint{1.055503in}{3.194184in}}%
\pgfpathlineto{\pgfqpoint{1.060793in}{3.186673in}}%
\pgfpathlineto{\pgfqpoint{1.066084in}{3.171650in}}%
\pgfpathlineto{\pgfqpoint{1.071375in}{2.946305in}}%
\pgfpathlineto{\pgfqpoint{1.076665in}{3.134092in}}%
\pgfpathlineto{\pgfqpoint{1.081956in}{3.149115in}}%
\pgfpathlineto{\pgfqpoint{1.087247in}{3.126581in}}%
\pgfpathlineto{\pgfqpoint{1.092537in}{3.089023in}}%
\pgfpathlineto{\pgfqpoint{1.097828in}{2.998885in}}%
\pgfpathlineto{\pgfqpoint{1.103119in}{3.066489in}}%
\pgfpathlineto{\pgfqpoint{1.108410in}{3.089023in}}%
\pgfpathlineto{\pgfqpoint{1.113700in}{3.051466in}}%
\pgfpathlineto{\pgfqpoint{1.118991in}{3.058977in}}%
\pgfpathlineto{\pgfqpoint{1.124282in}{3.104046in}}%
\pgfpathlineto{\pgfqpoint{1.129572in}{2.886213in}}%
\pgfpathlineto{\pgfqpoint{1.134863in}{3.104046in}}%
\pgfpathlineto{\pgfqpoint{1.140154in}{3.141604in}}%
\pgfpathlineto{\pgfqpoint{1.145445in}{2.923770in}}%
\pgfpathlineto{\pgfqpoint{1.150735in}{3.089023in}}%
\pgfpathlineto{\pgfqpoint{1.156026in}{3.104046in}}%
\pgfpathlineto{\pgfqpoint{1.161317in}{3.149115in}}%
\pgfpathlineto{\pgfqpoint{1.166607in}{2.856167in}}%
\pgfpathlineto{\pgfqpoint{1.171898in}{3.126581in}}%
\pgfpathlineto{\pgfqpoint{1.177189in}{3.141604in}}%
\pgfpathlineto{\pgfqpoint{1.182479in}{3.164138in}}%
\pgfpathlineto{\pgfqpoint{1.187770in}{3.104046in}}%
\pgfpathlineto{\pgfqpoint{1.193061in}{2.991374in}}%
\pgfpathlineto{\pgfqpoint{1.198352in}{2.931282in}}%
\pgfpathlineto{\pgfqpoint{1.203642in}{2.826121in}}%
\pgfpathlineto{\pgfqpoint{1.208933in}{2.803586in}}%
\pgfpathlineto{\pgfqpoint{1.214224in}{2.871190in}}%
\pgfpathlineto{\pgfqpoint{1.219514in}{2.871190in}}%
\pgfpathlineto{\pgfqpoint{1.224805in}{2.773540in}}%
\pgfpathlineto{\pgfqpoint{1.230096in}{2.781052in}}%
\pgfpathlineto{\pgfqpoint{1.235387in}{2.796075in}}%
\pgfpathlineto{\pgfqpoint{1.240677in}{2.871190in}}%
\pgfpathlineto{\pgfqpoint{1.245968in}{2.818609in}}%
\pgfpathlineto{\pgfqpoint{1.251259in}{2.818609in}}%
\pgfpathlineto{\pgfqpoint{1.256549in}{2.803586in}}%
\pgfpathlineto{\pgfqpoint{1.261840in}{2.878701in}}%
\pgfpathlineto{\pgfqpoint{1.267131in}{2.923770in}}%
\pgfpathlineto{\pgfqpoint{1.272421in}{2.833632in}}%
\pgfpathlineto{\pgfqpoint{1.277712in}{2.833632in}}%
\pgfpathlineto{\pgfqpoint{1.283003in}{2.953816in}}%
\pgfpathlineto{\pgfqpoint{1.288294in}{2.923770in}}%
\pgfpathlineto{\pgfqpoint{1.293584in}{3.074000in}}%
\pgfpathlineto{\pgfqpoint{1.298875in}{3.013908in}}%
\pgfpathlineto{\pgfqpoint{1.304166in}{2.856167in}}%
\pgfpathlineto{\pgfqpoint{1.309456in}{3.051466in}}%
\pgfpathlineto{\pgfqpoint{1.314747in}{3.021420in}}%
\pgfpathlineto{\pgfqpoint{1.320038in}{2.878701in}}%
\pgfpathlineto{\pgfqpoint{1.325328in}{3.021420in}}%
\pgfpathlineto{\pgfqpoint{1.330619in}{2.893724in}}%
\pgfpathlineto{\pgfqpoint{1.335910in}{3.081512in}}%
\pgfpathlineto{\pgfqpoint{1.341201in}{3.043954in}}%
\pgfpathlineto{\pgfqpoint{1.346491in}{2.908747in}}%
\pgfpathlineto{\pgfqpoint{1.351782in}{2.833632in}}%
\pgfpathlineto{\pgfqpoint{1.357073in}{2.848655in}}%
\pgfpathlineto{\pgfqpoint{1.362363in}{2.946305in}}%
\pgfpathlineto{\pgfqpoint{1.367654in}{2.946305in}}%
\pgfpathlineto{\pgfqpoint{1.372945in}{2.901236in}}%
\pgfpathlineto{\pgfqpoint{1.378236in}{3.036443in}}%
\pgfpathlineto{\pgfqpoint{1.383526in}{3.043954in}}%
\pgfpathlineto{\pgfqpoint{1.388817in}{3.021420in}}%
\pgfpathlineto{\pgfqpoint{1.394108in}{3.006397in}}%
\pgfpathlineto{\pgfqpoint{1.399398in}{3.021420in}}%
\pgfpathlineto{\pgfqpoint{1.404689in}{3.081512in}}%
\pgfpathlineto{\pgfqpoint{1.409980in}{3.051466in}}%
\pgfpathlineto{\pgfqpoint{1.415270in}{3.104046in}}%
\pgfpathlineto{\pgfqpoint{1.420561in}{3.111558in}}%
\pgfpathlineto{\pgfqpoint{1.425852in}{3.111558in}}%
\pgfpathlineto{\pgfqpoint{1.431143in}{3.126581in}}%
\pgfpathlineto{\pgfqpoint{1.436433in}{3.074000in}}%
\pgfpathlineto{\pgfqpoint{1.441724in}{3.141604in}}%
\pgfpathlineto{\pgfqpoint{1.447015in}{3.126581in}}%
\pgfpathlineto{\pgfqpoint{1.452305in}{3.126581in}}%
\pgfpathlineto{\pgfqpoint{1.457596in}{3.134092in}}%
\pgfpathlineto{\pgfqpoint{1.462887in}{3.096535in}}%
\pgfpathlineto{\pgfqpoint{1.468178in}{3.081512in}}%
\pgfpathlineto{\pgfqpoint{1.473468in}{3.036443in}}%
\pgfpathlineto{\pgfqpoint{1.478759in}{3.081512in}}%
\pgfpathlineto{\pgfqpoint{1.484050in}{3.036443in}}%
\pgfpathlineto{\pgfqpoint{1.494631in}{3.104046in}}%
\pgfpathlineto{\pgfqpoint{1.499922in}{3.201696in}}%
\pgfpathlineto{\pgfqpoint{1.505212in}{3.164138in}}%
\pgfpathlineto{\pgfqpoint{1.510503in}{3.216719in}}%
\pgfpathlineto{\pgfqpoint{1.515794in}{3.141604in}}%
\pgfpathlineto{\pgfqpoint{1.521085in}{3.104046in}}%
\pgfpathlineto{\pgfqpoint{1.526375in}{3.104046in}}%
\pgfpathlineto{\pgfqpoint{1.531666in}{3.149115in}}%
\pgfpathlineto{\pgfqpoint{1.536957in}{3.179161in}}%
\pgfpathlineto{\pgfqpoint{1.542247in}{2.570730in}}%
\pgfpathlineto{\pgfqpoint{1.547538in}{3.209207in}}%
\pgfpathlineto{\pgfqpoint{1.552829in}{3.141604in}}%
\pgfpathlineto{\pgfqpoint{1.558119in}{3.096535in}}%
\pgfpathlineto{\pgfqpoint{1.563410in}{3.164138in}}%
\pgfpathlineto{\pgfqpoint{1.568701in}{3.156627in}}%
\pgfpathlineto{\pgfqpoint{1.573992in}{3.081512in}}%
\pgfpathlineto{\pgfqpoint{1.579282in}{3.141604in}}%
\pgfpathlineto{\pgfqpoint{1.584573in}{3.028931in}}%
\pgfpathlineto{\pgfqpoint{1.589864in}{3.119069in}}%
\pgfpathlineto{\pgfqpoint{1.595154in}{3.111558in}}%
\pgfpathlineto{\pgfqpoint{1.600445in}{3.134092in}}%
\pgfpathlineto{\pgfqpoint{1.611027in}{3.134092in}}%
\pgfpathlineto{\pgfqpoint{1.616317in}{3.149115in}}%
\pgfpathlineto{\pgfqpoint{1.621608in}{3.156627in}}%
\pgfpathlineto{\pgfqpoint{1.626899in}{3.051466in}}%
\pgfpathlineto{\pgfqpoint{1.632189in}{3.149115in}}%
\pgfpathlineto{\pgfqpoint{1.637480in}{2.638333in}}%
\pgfpathlineto{\pgfqpoint{1.642771in}{3.179161in}}%
\pgfpathlineto{\pgfqpoint{1.663934in}{3.149115in}}%
\pgfpathlineto{\pgfqpoint{1.669224in}{3.119069in}}%
\pgfpathlineto{\pgfqpoint{1.674515in}{3.134092in}}%
\pgfpathlineto{\pgfqpoint{1.679806in}{3.134092in}}%
\pgfpathlineto{\pgfqpoint{1.685096in}{3.186673in}}%
\pgfpathlineto{\pgfqpoint{1.690387in}{2.961328in}}%
\pgfpathlineto{\pgfqpoint{1.700968in}{3.201696in}}%
\pgfpathlineto{\pgfqpoint{1.706259in}{3.141604in}}%
\pgfpathlineto{\pgfqpoint{1.711550in}{3.201696in}}%
\pgfpathlineto{\pgfqpoint{1.716841in}{3.171650in}}%
\pgfpathlineto{\pgfqpoint{1.722131in}{3.164138in}}%
\pgfpathlineto{\pgfqpoint{1.727422in}{3.149115in}}%
\pgfpathlineto{\pgfqpoint{1.732713in}{3.096535in}}%
\pgfpathlineto{\pgfqpoint{1.738003in}{3.171650in}}%
\pgfpathlineto{\pgfqpoint{1.743294in}{3.156627in}}%
\pgfpathlineto{\pgfqpoint{1.748585in}{3.209207in}}%
\pgfpathlineto{\pgfqpoint{1.753876in}{3.186673in}}%
\pgfpathlineto{\pgfqpoint{1.759166in}{3.194184in}}%
\pgfpathlineto{\pgfqpoint{1.764457in}{3.171650in}}%
\pgfpathlineto{\pgfqpoint{1.769748in}{3.164138in}}%
\pgfpathlineto{\pgfqpoint{1.775038in}{3.261788in}}%
\pgfpathlineto{\pgfqpoint{1.780329in}{3.224230in}}%
\pgfpathlineto{\pgfqpoint{1.785620in}{3.126581in}}%
\pgfpathlineto{\pgfqpoint{1.790910in}{3.186673in}}%
\pgfpathlineto{\pgfqpoint{1.796201in}{3.164138in}}%
\pgfpathlineto{\pgfqpoint{1.801492in}{3.194184in}}%
\pgfpathlineto{\pgfqpoint{1.806783in}{3.201696in}}%
\pgfpathlineto{\pgfqpoint{1.812073in}{3.141604in}}%
\pgfpathlineto{\pgfqpoint{1.817364in}{3.246765in}}%
\pgfpathlineto{\pgfqpoint{1.822655in}{3.261788in}}%
\pgfpathlineto{\pgfqpoint{1.833236in}{3.231742in}}%
\pgfpathlineto{\pgfqpoint{1.838527in}{3.186673in}}%
\pgfpathlineto{\pgfqpoint{1.843818in}{3.239253in}}%
\pgfpathlineto{\pgfqpoint{1.849108in}{3.224230in}}%
\pgfpathlineto{\pgfqpoint{1.859690in}{2.938793in}}%
\pgfpathlineto{\pgfqpoint{1.864980in}{3.149115in}}%
\pgfpathlineto{\pgfqpoint{1.870271in}{3.164138in}}%
\pgfpathlineto{\pgfqpoint{1.875562in}{3.164138in}}%
\pgfpathlineto{\pgfqpoint{1.880852in}{3.171650in}}%
\pgfpathlineto{\pgfqpoint{1.886143in}{3.134092in}}%
\pgfpathlineto{\pgfqpoint{1.891434in}{3.156627in}}%
\pgfpathlineto{\pgfqpoint{1.896725in}{3.164138in}}%
\pgfpathlineto{\pgfqpoint{1.902015in}{3.156627in}}%
\pgfpathlineto{\pgfqpoint{1.907306in}{3.171650in}}%
\pgfpathlineto{\pgfqpoint{1.912597in}{3.119069in}}%
\pgfpathlineto{\pgfqpoint{1.917887in}{3.179161in}}%
\pgfpathlineto{\pgfqpoint{1.923178in}{3.179161in}}%
\pgfpathlineto{\pgfqpoint{1.928469in}{3.134092in}}%
\pgfpathlineto{\pgfqpoint{1.933759in}{3.209207in}}%
\pgfpathlineto{\pgfqpoint{1.939050in}{3.119069in}}%
\pgfpathlineto{\pgfqpoint{1.944341in}{3.201696in}}%
\pgfpathlineto{\pgfqpoint{1.949632in}{3.216719in}}%
\pgfpathlineto{\pgfqpoint{1.954922in}{3.186673in}}%
\pgfpathlineto{\pgfqpoint{1.965504in}{3.171650in}}%
\pgfpathlineto{\pgfqpoint{1.976085in}{3.171650in}}%
\pgfpathlineto{\pgfqpoint{1.981376in}{3.156627in}}%
\pgfpathlineto{\pgfqpoint{1.986667in}{3.186673in}}%
\pgfpathlineto{\pgfqpoint{1.991957in}{3.179161in}}%
\pgfpathlineto{\pgfqpoint{1.997248in}{3.156627in}}%
\pgfpathlineto{\pgfqpoint{2.002539in}{3.171650in}}%
\pgfpathlineto{\pgfqpoint{2.007829in}{3.164138in}}%
\pgfpathlineto{\pgfqpoint{2.013120in}{3.164138in}}%
\pgfpathlineto{\pgfqpoint{2.018411in}{3.216719in}}%
\pgfpathlineto{\pgfqpoint{2.028992in}{3.216719in}}%
\pgfpathlineto{\pgfqpoint{2.034283in}{3.149115in}}%
\pgfpathlineto{\pgfqpoint{2.039574in}{3.104046in}}%
\pgfpathlineto{\pgfqpoint{2.044864in}{3.149115in}}%
\pgfpathlineto{\pgfqpoint{2.050155in}{2.998885in}}%
\pgfpathlineto{\pgfqpoint{2.055446in}{3.186673in}}%
\pgfpathlineto{\pgfqpoint{2.060736in}{3.231742in}}%
\pgfpathlineto{\pgfqpoint{2.066027in}{3.209207in}}%
\pgfpathlineto{\pgfqpoint{2.081899in}{3.209207in}}%
\pgfpathlineto{\pgfqpoint{2.092481in}{3.194184in}}%
\pgfpathlineto{\pgfqpoint{2.097771in}{3.156627in}}%
\pgfpathlineto{\pgfqpoint{2.103062in}{3.134092in}}%
\pgfpathlineto{\pgfqpoint{2.108353in}{3.164138in}}%
\pgfpathlineto{\pgfqpoint{2.113643in}{3.156627in}}%
\pgfpathlineto{\pgfqpoint{2.118934in}{3.171650in}}%
\pgfpathlineto{\pgfqpoint{2.124225in}{3.239253in}}%
\pgfpathlineto{\pgfqpoint{2.129516in}{3.231742in}}%
\pgfpathlineto{\pgfqpoint{2.134806in}{3.201696in}}%
\pgfpathlineto{\pgfqpoint{2.140097in}{3.209207in}}%
\pgfpathlineto{\pgfqpoint{2.145388in}{3.209207in}}%
\pgfpathlineto{\pgfqpoint{2.150678in}{3.194184in}}%
\pgfpathlineto{\pgfqpoint{2.155969in}{3.043954in}}%
\pgfpathlineto{\pgfqpoint{2.161260in}{3.231742in}}%
\pgfpathlineto{\pgfqpoint{2.166550in}{3.224230in}}%
\pgfpathlineto{\pgfqpoint{2.171841in}{3.254276in}}%
\pgfpathlineto{\pgfqpoint{2.177132in}{3.239253in}}%
\pgfpathlineto{\pgfqpoint{2.182423in}{3.276811in}}%
\pgfpathlineto{\pgfqpoint{2.187713in}{3.149115in}}%
\pgfpathlineto{\pgfqpoint{2.193004in}{3.254276in}}%
\pgfpathlineto{\pgfqpoint{2.198295in}{3.291834in}}%
\pgfpathlineto{\pgfqpoint{2.203585in}{3.276811in}}%
\pgfpathlineto{\pgfqpoint{2.208876in}{3.246765in}}%
\pgfpathlineto{\pgfqpoint{2.219458in}{3.231742in}}%
\pgfpathlineto{\pgfqpoint{2.224748in}{3.254276in}}%
\pgfpathlineto{\pgfqpoint{2.230039in}{3.231742in}}%
\pgfpathlineto{\pgfqpoint{2.235330in}{3.239253in}}%
\pgfpathlineto{\pgfqpoint{2.240620in}{3.276811in}}%
\pgfpathlineto{\pgfqpoint{2.245911in}{3.231742in}}%
\pgfpathlineto{\pgfqpoint{2.251202in}{3.269299in}}%
\pgfpathlineto{\pgfqpoint{2.261783in}{3.254276in}}%
\pgfpathlineto{\pgfqpoint{2.267074in}{3.261788in}}%
\pgfpathlineto{\pgfqpoint{2.272365in}{3.276811in}}%
\pgfpathlineto{\pgfqpoint{2.277655in}{3.224230in}}%
\pgfpathlineto{\pgfqpoint{2.282946in}{3.224230in}}%
\pgfpathlineto{\pgfqpoint{2.304109in}{3.194184in}}%
\pgfpathlineto{\pgfqpoint{2.309399in}{2.811098in}}%
\pgfpathlineto{\pgfqpoint{2.314690in}{3.224230in}}%
\pgfpathlineto{\pgfqpoint{2.319981in}{3.201696in}}%
\pgfpathlineto{\pgfqpoint{2.325272in}{3.216719in}}%
\pgfpathlineto{\pgfqpoint{2.330562in}{3.209207in}}%
\pgfpathlineto{\pgfqpoint{2.335853in}{3.216719in}}%
\pgfpathlineto{\pgfqpoint{2.341144in}{3.058977in}}%
\pgfpathlineto{\pgfqpoint{2.346434in}{2.968839in}}%
\pgfpathlineto{\pgfqpoint{2.351725in}{3.051466in}}%
\pgfpathlineto{\pgfqpoint{2.357016in}{3.028931in}}%
\pgfpathlineto{\pgfqpoint{2.362307in}{3.171650in}}%
\pgfpathlineto{\pgfqpoint{2.367597in}{3.179161in}}%
\pgfpathlineto{\pgfqpoint{2.372888in}{3.194184in}}%
\pgfpathlineto{\pgfqpoint{2.378179in}{3.194184in}}%
\pgfpathlineto{\pgfqpoint{2.383469in}{3.171650in}}%
\pgfpathlineto{\pgfqpoint{2.399341in}{3.126581in}}%
\pgfpathlineto{\pgfqpoint{2.404632in}{3.171650in}}%
\pgfpathlineto{\pgfqpoint{2.409923in}{3.186673in}}%
\pgfpathlineto{\pgfqpoint{2.415214in}{3.194184in}}%
\pgfpathlineto{\pgfqpoint{2.420504in}{3.216719in}}%
\pgfpathlineto{\pgfqpoint{2.425795in}{3.231742in}}%
\pgfpathlineto{\pgfqpoint{2.431086in}{3.209207in}}%
\pgfpathlineto{\pgfqpoint{2.436376in}{3.239253in}}%
\pgfpathlineto{\pgfqpoint{2.441667in}{3.209207in}}%
\pgfpathlineto{\pgfqpoint{2.446958in}{3.201696in}}%
\pgfpathlineto{\pgfqpoint{2.452249in}{3.179161in}}%
\pgfpathlineto{\pgfqpoint{2.457539in}{3.179161in}}%
\pgfpathlineto{\pgfqpoint{2.462830in}{3.194184in}}%
\pgfpathlineto{\pgfqpoint{2.468121in}{3.164138in}}%
\pgfpathlineto{\pgfqpoint{2.473411in}{3.201696in}}%
\pgfpathlineto{\pgfqpoint{2.478702in}{3.194184in}}%
\pgfpathlineto{\pgfqpoint{2.483993in}{3.194184in}}%
\pgfpathlineto{\pgfqpoint{2.489283in}{3.201696in}}%
\pgfpathlineto{\pgfqpoint{2.494574in}{3.231742in}}%
\pgfpathlineto{\pgfqpoint{2.499865in}{3.224230in}}%
\pgfpathlineto{\pgfqpoint{2.505156in}{3.254276in}}%
\pgfpathlineto{\pgfqpoint{2.510446in}{3.299345in}}%
\pgfpathlineto{\pgfqpoint{2.521028in}{3.209207in}}%
\pgfpathlineto{\pgfqpoint{2.526318in}{3.194184in}}%
\pgfpathlineto{\pgfqpoint{2.536900in}{3.194184in}}%
\pgfpathlineto{\pgfqpoint{2.542190in}{3.201696in}}%
\pgfpathlineto{\pgfqpoint{2.547481in}{3.201696in}}%
\pgfpathlineto{\pgfqpoint{2.552772in}{2.788563in}}%
\pgfpathlineto{\pgfqpoint{2.558063in}{2.886213in}}%
\pgfpathlineto{\pgfqpoint{2.563353in}{3.186673in}}%
\pgfpathlineto{\pgfqpoint{2.568644in}{3.186673in}}%
\pgfpathlineto{\pgfqpoint{2.573935in}{3.194184in}}%
\pgfpathlineto{\pgfqpoint{2.579225in}{3.194184in}}%
\pgfpathlineto{\pgfqpoint{2.589807in}{3.179161in}}%
\pgfpathlineto{\pgfqpoint{2.595098in}{3.209207in}}%
\pgfpathlineto{\pgfqpoint{2.600388in}{3.186673in}}%
\pgfpathlineto{\pgfqpoint{2.605679in}{3.194184in}}%
\pgfpathlineto{\pgfqpoint{2.610970in}{3.194184in}}%
\pgfpathlineto{\pgfqpoint{2.616260in}{3.201696in}}%
\pgfpathlineto{\pgfqpoint{2.621551in}{3.186673in}}%
\pgfpathlineto{\pgfqpoint{2.626842in}{3.179161in}}%
\pgfpathlineto{\pgfqpoint{2.632132in}{3.201696in}}%
\pgfpathlineto{\pgfqpoint{2.637423in}{3.171650in}}%
\pgfpathlineto{\pgfqpoint{2.642714in}{3.201696in}}%
\pgfpathlineto{\pgfqpoint{2.648005in}{3.186673in}}%
\pgfpathlineto{\pgfqpoint{2.653295in}{3.201696in}}%
\pgfpathlineto{\pgfqpoint{2.658586in}{3.239253in}}%
\pgfpathlineto{\pgfqpoint{2.663877in}{3.239253in}}%
\pgfpathlineto{\pgfqpoint{2.669167in}{3.284322in}}%
\pgfpathlineto{\pgfqpoint{2.674458in}{3.209207in}}%
\pgfpathlineto{\pgfqpoint{2.679749in}{3.224230in}}%
\pgfpathlineto{\pgfqpoint{2.690330in}{3.194184in}}%
\pgfpathlineto{\pgfqpoint{2.695621in}{3.201696in}}%
\pgfpathlineto{\pgfqpoint{2.700912in}{3.171650in}}%
\pgfpathlineto{\pgfqpoint{2.706202in}{3.239253in}}%
\pgfpathlineto{\pgfqpoint{2.711493in}{3.201696in}}%
\pgfpathlineto{\pgfqpoint{2.716784in}{3.321880in}}%
\pgfpathlineto{\pgfqpoint{2.727365in}{3.006397in}}%
\pgfpathlineto{\pgfqpoint{2.732656in}{3.351926in}}%
\pgfpathlineto{\pgfqpoint{2.737947in}{3.344414in}}%
\pgfpathlineto{\pgfqpoint{2.748528in}{3.359437in}}%
\pgfpathlineto{\pgfqpoint{2.753819in}{3.344414in}}%
\pgfpathlineto{\pgfqpoint{2.759109in}{3.336903in}}%
\pgfpathlineto{\pgfqpoint{2.764400in}{3.366949in}}%
\pgfpathlineto{\pgfqpoint{2.769691in}{3.276811in}}%
\pgfpathlineto{\pgfqpoint{2.774981in}{3.366949in}}%
\pgfpathlineto{\pgfqpoint{2.806726in}{3.366949in}}%
\pgfpathlineto{\pgfqpoint{2.812016in}{3.344414in}}%
\pgfpathlineto{\pgfqpoint{2.817307in}{3.231742in}}%
\pgfpathlineto{\pgfqpoint{2.822598in}{3.224230in}}%
\pgfpathlineto{\pgfqpoint{2.827889in}{3.119069in}}%
\pgfpathlineto{\pgfqpoint{2.833179in}{3.254276in}}%
\pgfpathlineto{\pgfqpoint{2.838470in}{3.209207in}}%
\pgfpathlineto{\pgfqpoint{2.843761in}{3.269299in}}%
\pgfpathlineto{\pgfqpoint{2.849051in}{3.254276in}}%
\pgfpathlineto{\pgfqpoint{2.854342in}{3.284322in}}%
\pgfpathlineto{\pgfqpoint{2.859633in}{3.269299in}}%
\pgfpathlineto{\pgfqpoint{2.864923in}{3.366949in}}%
\pgfpathlineto{\pgfqpoint{2.875505in}{3.276811in}}%
\pgfpathlineto{\pgfqpoint{2.880796in}{3.329391in}}%
\pgfpathlineto{\pgfqpoint{2.886086in}{3.351926in}}%
\pgfpathlineto{\pgfqpoint{2.896668in}{3.366949in}}%
\pgfpathlineto{\pgfqpoint{2.901958in}{3.336903in}}%
\pgfpathlineto{\pgfqpoint{2.907249in}{3.269299in}}%
\pgfpathlineto{\pgfqpoint{2.912540in}{3.359437in}}%
\pgfpathlineto{\pgfqpoint{2.917830in}{3.351926in}}%
\pgfpathlineto{\pgfqpoint{2.923121in}{3.366949in}}%
\pgfpathlineto{\pgfqpoint{2.933703in}{3.351926in}}%
\pgfpathlineto{\pgfqpoint{2.938993in}{3.284322in}}%
\pgfpathlineto{\pgfqpoint{2.944284in}{3.366949in}}%
\pgfpathlineto{\pgfqpoint{2.949575in}{3.359437in}}%
\pgfpathlineto{\pgfqpoint{2.954865in}{3.359437in}}%
\pgfpathlineto{\pgfqpoint{2.960156in}{3.344414in}}%
\pgfpathlineto{\pgfqpoint{2.965447in}{3.284322in}}%
\pgfpathlineto{\pgfqpoint{2.970738in}{3.291834in}}%
\pgfpathlineto{\pgfqpoint{2.976028in}{3.276811in}}%
\pgfpathlineto{\pgfqpoint{2.981319in}{3.306857in}}%
\pgfpathlineto{\pgfqpoint{2.986610in}{3.306857in}}%
\pgfpathlineto{\pgfqpoint{2.991900in}{3.329391in}}%
\pgfpathlineto{\pgfqpoint{2.997191in}{3.359437in}}%
\pgfpathlineto{\pgfqpoint{3.002482in}{3.366949in}}%
\pgfpathlineto{\pgfqpoint{3.007772in}{3.359437in}}%
\pgfpathlineto{\pgfqpoint{3.013063in}{3.306857in}}%
\pgfpathlineto{\pgfqpoint{3.023645in}{3.291834in}}%
\pgfpathlineto{\pgfqpoint{3.028935in}{3.321880in}}%
\pgfpathlineto{\pgfqpoint{3.039517in}{3.351926in}}%
\pgfpathlineto{\pgfqpoint{3.044807in}{3.329391in}}%
\pgfpathlineto{\pgfqpoint{3.050098in}{3.314368in}}%
\pgfpathlineto{\pgfqpoint{3.055389in}{3.359437in}}%
\pgfpathlineto{\pgfqpoint{3.060679in}{3.336903in}}%
\pgfpathlineto{\pgfqpoint{3.065970in}{3.359437in}}%
\pgfpathlineto{\pgfqpoint{3.071261in}{3.366949in}}%
\pgfpathlineto{\pgfqpoint{3.076552in}{3.314368in}}%
\pgfpathlineto{\pgfqpoint{3.081842in}{3.366949in}}%
\pgfpathlineto{\pgfqpoint{3.087133in}{3.366949in}}%
\pgfpathlineto{\pgfqpoint{3.092424in}{3.359437in}}%
\pgfpathlineto{\pgfqpoint{3.097714in}{3.366949in}}%
\pgfpathlineto{\pgfqpoint{3.108296in}{3.366949in}}%
\pgfpathlineto{\pgfqpoint{3.113587in}{3.359437in}}%
\pgfpathlineto{\pgfqpoint{3.118877in}{3.344414in}}%
\pgfpathlineto{\pgfqpoint{3.124168in}{3.359437in}}%
\pgfpathlineto{\pgfqpoint{3.129459in}{3.366949in}}%
\pgfpathlineto{\pgfqpoint{3.134749in}{3.366949in}}%
\pgfpathlineto{\pgfqpoint{3.140040in}{3.359437in}}%
\pgfpathlineto{\pgfqpoint{3.161203in}{3.359437in}}%
\pgfpathlineto{\pgfqpoint{3.166494in}{3.366949in}}%
\pgfpathlineto{\pgfqpoint{3.219401in}{3.366949in}}%
\pgfpathlineto{\pgfqpoint{3.224691in}{3.359437in}}%
\pgfpathlineto{\pgfqpoint{3.229982in}{3.366949in}}%
\pgfpathlineto{\pgfqpoint{3.240563in}{3.366949in}}%
\pgfpathlineto{\pgfqpoint{3.245854in}{3.359437in}}%
\pgfpathlineto{\pgfqpoint{3.261726in}{3.359437in}}%
\pgfpathlineto{\pgfqpoint{3.267017in}{3.366949in}}%
\pgfpathlineto{\pgfqpoint{3.335796in}{3.366949in}}%
\pgfpathlineto{\pgfqpoint{3.341087in}{3.359437in}}%
\pgfpathlineto{\pgfqpoint{3.346378in}{3.366949in}}%
\pgfpathlineto{\pgfqpoint{3.351668in}{3.359437in}}%
\pgfpathlineto{\pgfqpoint{3.356959in}{3.366949in}}%
\pgfpathlineto{\pgfqpoint{3.399285in}{3.366949in}}%
\pgfpathlineto{\pgfqpoint{3.409866in}{3.336903in}}%
\pgfpathlineto{\pgfqpoint{3.415157in}{3.366949in}}%
\pgfpathlineto{\pgfqpoint{3.431029in}{3.366949in}}%
\pgfpathlineto{\pgfqpoint{3.436319in}{3.351926in}}%
\pgfpathlineto{\pgfqpoint{3.441610in}{3.351926in}}%
\pgfpathlineto{\pgfqpoint{3.446901in}{3.366949in}}%
\pgfpathlineto{\pgfqpoint{3.452192in}{3.366949in}}%
\pgfpathlineto{\pgfqpoint{3.457482in}{3.344414in}}%
\pgfpathlineto{\pgfqpoint{3.473354in}{3.366949in}}%
\pgfpathlineto{\pgfqpoint{3.515680in}{3.366949in}}%
\pgfpathlineto{\pgfqpoint{3.520971in}{3.359437in}}%
\pgfpathlineto{\pgfqpoint{3.526261in}{3.366949in}}%
\pgfpathlineto{\pgfqpoint{3.621494in}{3.366949in}}%
\pgfpathlineto{\pgfqpoint{3.626785in}{3.359437in}}%
\pgfpathlineto{\pgfqpoint{3.632076in}{3.366949in}}%
\pgfpathlineto{\pgfqpoint{3.637366in}{3.254276in}}%
\pgfpathlineto{\pgfqpoint{3.642657in}{2.653356in}}%
\pgfpathlineto{\pgfqpoint{3.647948in}{3.366949in}}%
\pgfpathlineto{\pgfqpoint{3.954809in}{3.366949in}}%
\pgfpathlineto{\pgfqpoint{3.960099in}{3.351926in}}%
\pgfpathlineto{\pgfqpoint{3.965390in}{3.366949in}}%
\pgfpathlineto{\pgfqpoint{4.013006in}{3.366949in}}%
\pgfpathlineto{\pgfqpoint{4.018297in}{3.359437in}}%
\pgfpathlineto{\pgfqpoint{4.023588in}{3.366949in}}%
\pgfpathlineto{\pgfqpoint{4.028878in}{3.366949in}}%
\pgfpathlineto{\pgfqpoint{4.034169in}{3.276811in}}%
\pgfpathlineto{\pgfqpoint{4.044750in}{3.164138in}}%
\pgfpathlineto{\pgfqpoint{4.050041in}{3.366949in}}%
\pgfpathlineto{\pgfqpoint{4.055332in}{3.359437in}}%
\pgfpathlineto{\pgfqpoint{4.060623in}{3.366949in}}%
\pgfpathlineto{\pgfqpoint{4.071204in}{3.366949in}}%
\pgfpathlineto{\pgfqpoint{4.076495in}{3.359437in}}%
\pgfpathlineto{\pgfqpoint{4.081785in}{3.359437in}}%
\pgfpathlineto{\pgfqpoint{4.087076in}{3.366949in}}%
\pgfpathlineto{\pgfqpoint{4.092367in}{3.366949in}}%
\pgfpathlineto{\pgfqpoint{4.097658in}{3.359437in}}%
\pgfpathlineto{\pgfqpoint{4.102948in}{3.359437in}}%
\pgfpathlineto{\pgfqpoint{4.108239in}{3.321880in}}%
\pgfpathlineto{\pgfqpoint{4.113530in}{3.366949in}}%
\pgfpathlineto{\pgfqpoint{4.118820in}{3.359437in}}%
\pgfpathlineto{\pgfqpoint{4.124111in}{3.366949in}}%
\pgfpathlineto{\pgfqpoint{4.129402in}{3.359437in}}%
\pgfpathlineto{\pgfqpoint{4.134692in}{3.366949in}}%
\pgfpathlineto{\pgfqpoint{4.139983in}{3.366949in}}%
\pgfpathlineto{\pgfqpoint{4.145274in}{3.359437in}}%
\pgfpathlineto{\pgfqpoint{4.150565in}{3.366949in}}%
\pgfpathlineto{\pgfqpoint{4.166437in}{3.366949in}}%
\pgfpathlineto{\pgfqpoint{4.171727in}{3.359437in}}%
\pgfpathlineto{\pgfqpoint{4.177018in}{3.366949in}}%
\pgfpathlineto{\pgfqpoint{4.198181in}{3.366949in}}%
\pgfpathlineto{\pgfqpoint{4.203472in}{3.336903in}}%
\pgfpathlineto{\pgfqpoint{4.208762in}{3.366949in}}%
\pgfpathlineto{\pgfqpoint{4.214053in}{3.359437in}}%
\pgfpathlineto{\pgfqpoint{4.219344in}{3.366949in}}%
\pgfpathlineto{\pgfqpoint{4.224634in}{3.336903in}}%
\pgfpathlineto{\pgfqpoint{4.229925in}{3.351926in}}%
\pgfpathlineto{\pgfqpoint{4.235216in}{3.359437in}}%
\pgfpathlineto{\pgfqpoint{4.240507in}{3.359437in}}%
\pgfpathlineto{\pgfqpoint{4.245797in}{3.366949in}}%
\pgfpathlineto{\pgfqpoint{4.251088in}{3.359437in}}%
\pgfpathlineto{\pgfqpoint{4.256379in}{3.359437in}}%
\pgfpathlineto{\pgfqpoint{4.261669in}{3.351926in}}%
\pgfpathlineto{\pgfqpoint{4.272251in}{3.366949in}}%
\pgfpathlineto{\pgfqpoint{4.277541in}{3.359437in}}%
\pgfpathlineto{\pgfqpoint{4.282832in}{3.366949in}}%
\pgfpathlineto{\pgfqpoint{4.346321in}{3.366949in}}%
\pgfpathlineto{\pgfqpoint{4.351611in}{3.359437in}}%
\pgfpathlineto{\pgfqpoint{4.356902in}{3.366949in}}%
\pgfpathlineto{\pgfqpoint{4.362193in}{3.329391in}}%
\pgfpathlineto{\pgfqpoint{4.367483in}{3.336903in}}%
\pgfpathlineto{\pgfqpoint{4.372774in}{3.366949in}}%
\pgfpathlineto{\pgfqpoint{4.393937in}{3.366949in}}%
\pgfpathlineto{\pgfqpoint{4.399228in}{3.359437in}}%
\pgfpathlineto{\pgfqpoint{4.404518in}{3.366949in}}%
\pgfpathlineto{\pgfqpoint{4.409809in}{3.359437in}}%
\pgfpathlineto{\pgfqpoint{4.415100in}{3.359437in}}%
\pgfpathlineto{\pgfqpoint{4.420390in}{3.351926in}}%
\pgfpathlineto{\pgfqpoint{4.430972in}{3.366949in}}%
\pgfpathlineto{\pgfqpoint{4.436263in}{3.359437in}}%
\pgfpathlineto{\pgfqpoint{4.441553in}{3.359437in}}%
\pgfpathlineto{\pgfqpoint{4.446844in}{3.314368in}}%
\pgfpathlineto{\pgfqpoint{4.452135in}{3.359437in}}%
\pgfpathlineto{\pgfqpoint{4.457425in}{3.366949in}}%
\pgfpathlineto{\pgfqpoint{4.478588in}{3.366949in}}%
\pgfpathlineto{\pgfqpoint{4.483879in}{3.359437in}}%
\pgfpathlineto{\pgfqpoint{4.489170in}{3.366949in}}%
\pgfpathlineto{\pgfqpoint{4.494460in}{3.366949in}}%
\pgfpathlineto{\pgfqpoint{4.505042in}{3.336903in}}%
\pgfpathlineto{\pgfqpoint{4.510332in}{3.344414in}}%
\pgfpathlineto{\pgfqpoint{4.515623in}{3.359437in}}%
\pgfpathlineto{\pgfqpoint{4.520914in}{3.366949in}}%
\pgfpathlineto{\pgfqpoint{4.526205in}{3.351926in}}%
\pgfpathlineto{\pgfqpoint{4.531495in}{3.366949in}}%
\pgfpathlineto{\pgfqpoint{4.536786in}{3.344414in}}%
\pgfpathlineto{\pgfqpoint{4.542077in}{3.336903in}}%
\pgfpathlineto{\pgfqpoint{4.547367in}{3.344414in}}%
\pgfpathlineto{\pgfqpoint{4.552658in}{3.344414in}}%
\pgfpathlineto{\pgfqpoint{4.557949in}{3.321880in}}%
\pgfpathlineto{\pgfqpoint{4.568530in}{3.351926in}}%
\pgfpathlineto{\pgfqpoint{4.573821in}{3.359437in}}%
\pgfpathlineto{\pgfqpoint{4.579112in}{3.336903in}}%
\pgfpathlineto{\pgfqpoint{4.584402in}{3.359437in}}%
\pgfpathlineto{\pgfqpoint{4.594984in}{3.344414in}}%
\pgfpathlineto{\pgfqpoint{4.600274in}{3.351926in}}%
\pgfpathlineto{\pgfqpoint{4.605565in}{3.344414in}}%
\pgfpathlineto{\pgfqpoint{4.610856in}{3.366949in}}%
\pgfpathlineto{\pgfqpoint{4.616147in}{3.359437in}}%
\pgfpathlineto{\pgfqpoint{4.621437in}{3.344414in}}%
\pgfpathlineto{\pgfqpoint{4.626728in}{3.299345in}}%
\pgfpathlineto{\pgfqpoint{4.632019in}{3.336903in}}%
\pgfpathlineto{\pgfqpoint{4.637309in}{3.321880in}}%
\pgfpathlineto{\pgfqpoint{4.642600in}{3.171650in}}%
\pgfpathlineto{\pgfqpoint{4.647891in}{3.209207in}}%
\pgfpathlineto{\pgfqpoint{4.653181in}{3.366949in}}%
\pgfpathlineto{\pgfqpoint{4.669054in}{3.366949in}}%
\pgfpathlineto{\pgfqpoint{4.674344in}{3.359437in}}%
\pgfpathlineto{\pgfqpoint{4.679635in}{3.329391in}}%
\pgfpathlineto{\pgfqpoint{4.684926in}{3.366949in}}%
\pgfpathlineto{\pgfqpoint{4.700798in}{3.366949in}}%
\pgfpathlineto{\pgfqpoint{4.706089in}{3.336903in}}%
\pgfpathlineto{\pgfqpoint{4.711379in}{3.336903in}}%
\pgfpathlineto{\pgfqpoint{4.716670in}{3.359437in}}%
\pgfpathlineto{\pgfqpoint{4.721961in}{3.351926in}}%
\pgfpathlineto{\pgfqpoint{4.727251in}{3.336903in}}%
\pgfpathlineto{\pgfqpoint{4.732542in}{3.359437in}}%
\pgfpathlineto{\pgfqpoint{4.737833in}{3.344414in}}%
\pgfpathlineto{\pgfqpoint{4.743123in}{3.336903in}}%
\pgfpathlineto{\pgfqpoint{4.753705in}{3.306857in}}%
\pgfpathlineto{\pgfqpoint{4.758996in}{3.284322in}}%
\pgfpathlineto{\pgfqpoint{4.764286in}{3.291834in}}%
\pgfpathlineto{\pgfqpoint{4.769577in}{3.291834in}}%
\pgfpathlineto{\pgfqpoint{4.774868in}{3.261788in}}%
\pgfpathlineto{\pgfqpoint{4.780158in}{3.246765in}}%
\pgfpathlineto{\pgfqpoint{4.785449in}{3.299345in}}%
\pgfpathlineto{\pgfqpoint{4.790740in}{3.299345in}}%
\pgfpathlineto{\pgfqpoint{4.796031in}{3.261788in}}%
\pgfpathlineto{\pgfqpoint{4.801321in}{3.246765in}}%
\pgfpathlineto{\pgfqpoint{4.806612in}{3.284322in}}%
\pgfpathlineto{\pgfqpoint{4.811903in}{3.216719in}}%
\pgfpathlineto{\pgfqpoint{4.817193in}{3.254276in}}%
\pgfpathlineto{\pgfqpoint{4.822484in}{3.201696in}}%
\pgfpathlineto{\pgfqpoint{4.827775in}{3.284322in}}%
\pgfpathlineto{\pgfqpoint{4.833065in}{3.291834in}}%
\pgfpathlineto{\pgfqpoint{4.838356in}{3.321880in}}%
\pgfpathlineto{\pgfqpoint{4.843647in}{3.269299in}}%
\pgfpathlineto{\pgfqpoint{4.848938in}{3.366949in}}%
\pgfpathlineto{\pgfqpoint{4.854228in}{3.321880in}}%
\pgfpathlineto{\pgfqpoint{4.859519in}{3.351926in}}%
\pgfpathlineto{\pgfqpoint{4.864810in}{3.359437in}}%
\pgfpathlineto{\pgfqpoint{4.870100in}{3.314368in}}%
\pgfpathlineto{\pgfqpoint{4.875391in}{3.321880in}}%
\pgfpathlineto{\pgfqpoint{4.880682in}{3.351926in}}%
\pgfpathlineto{\pgfqpoint{4.885972in}{3.359437in}}%
\pgfpathlineto{\pgfqpoint{4.885972in}{3.359437in}}%
\pgfusepath{stroke}%
\end{pgfscope}%
\begin{pgfscope}%
\pgfsetrectcap%
\pgfsetmiterjoin%
\pgfsetlinewidth{0.501875pt}%
\definecolor{currentstroke}{rgb}{0.317647,0.317647,0.317647}%
\pgfsetstrokecolor{currentstroke}%
\pgfsetdash{}{0pt}%
\pgfpathmoveto{\pgfqpoint{0.447336in}{2.026146in}}%
\pgfpathlineto{\pgfqpoint{0.447336in}{3.430797in}}%
\pgfusepath{stroke}%
\end{pgfscope}%
\begin{pgfscope}%
\pgfsetrectcap%
\pgfsetmiterjoin%
\pgfsetlinewidth{0.501875pt}%
\definecolor{currentstroke}{rgb}{0.317647,0.317647,0.317647}%
\pgfsetstrokecolor{currentstroke}%
\pgfsetdash{}{0pt}%
\pgfpathmoveto{\pgfqpoint{0.447336in}{2.026146in}}%
\pgfpathlineto{\pgfqpoint{5.097336in}{2.026146in}}%
\pgfusepath{stroke}%
\end{pgfscope}%
\begin{pgfscope}%
\pgfsetbuttcap%
\pgfsetmiterjoin%
\pgfsetlinewidth{0.000000pt}%
\definecolor{currentstroke}{rgb}{0.000000,0.000000,0.000000}%
\pgfsetstrokecolor{currentstroke}%
\pgfsetstrokeopacity{0.000000}%
\pgfsetdash{}{0pt}%
\pgfpathmoveto{\pgfqpoint{0.447336in}{0.410797in}}%
\pgfpathlineto{\pgfqpoint{5.097336in}{0.410797in}}%
\pgfpathlineto{\pgfqpoint{5.097336in}{1.815448in}}%
\pgfpathlineto{\pgfqpoint{0.447336in}{1.815448in}}%
\pgfpathclose%
\pgfusepath{}%
\end{pgfscope}%
\begin{pgfscope}%
\pgfsetbuttcap%
\pgfsetroundjoin%
\definecolor{currentfill}{rgb}{0.317647,0.317647,0.317647}%
\pgfsetfillcolor{currentfill}%
\pgfsetlinewidth{0.501875pt}%
\definecolor{currentstroke}{rgb}{0.317647,0.317647,0.317647}%
\pgfsetstrokecolor{currentstroke}%
\pgfsetdash{}{0pt}%
\pgfsys@defobject{currentmarker}{\pgfqpoint{0.000000in}{-0.020833in}}{\pgfqpoint{0.000000in}{0.000000in}}{%
\pgfpathmoveto{\pgfqpoint{0.000000in}{0.000000in}}%
\pgfpathlineto{\pgfqpoint{0.000000in}{-0.020833in}}%
\pgfusepath{stroke,fill}%
}%
\begin{pgfscope}%
\pgfsys@transformshift{0.658700in}{0.410797in}%
\pgfsys@useobject{currentmarker}{}%
\end{pgfscope}%
\end{pgfscope}%
\begin{pgfscope}%
\definecolor{textcolor}{rgb}{0.317647,0.317647,0.317647}%
\pgfsetstrokecolor{textcolor}%
\pgfsetfillcolor{textcolor}%
\pgftext[x=0.658700in,y=0.362186in,,top]{\color{textcolor}\rmfamily\fontsize{8.330000}{9.996000}\selectfont \(\displaystyle 0\)}%
\end{pgfscope}%
\begin{pgfscope}%
\pgfsetbuttcap%
\pgfsetroundjoin%
\definecolor{currentfill}{rgb}{0.317647,0.317647,0.317647}%
\pgfsetfillcolor{currentfill}%
\pgfsetlinewidth{0.501875pt}%
\definecolor{currentstroke}{rgb}{0.317647,0.317647,0.317647}%
\pgfsetstrokecolor{currentstroke}%
\pgfsetdash{}{0pt}%
\pgfsys@defobject{currentmarker}{\pgfqpoint{0.000000in}{-0.020833in}}{\pgfqpoint{0.000000in}{0.000000in}}{%
\pgfpathmoveto{\pgfqpoint{0.000000in}{0.000000in}}%
\pgfpathlineto{\pgfqpoint{0.000000in}{-0.020833in}}%
\pgfusepath{stroke,fill}%
}%
\begin{pgfscope}%
\pgfsys@transformshift{1.187770in}{0.410797in}%
\pgfsys@useobject{currentmarker}{}%
\end{pgfscope}%
\end{pgfscope}%
\begin{pgfscope}%
\definecolor{textcolor}{rgb}{0.317647,0.317647,0.317647}%
\pgfsetstrokecolor{textcolor}%
\pgfsetfillcolor{textcolor}%
\pgftext[x=1.187770in,y=0.362186in,,top]{\color{textcolor}\rmfamily\fontsize{8.330000}{9.996000}\selectfont \(\displaystyle 500\)}%
\end{pgfscope}%
\begin{pgfscope}%
\pgfsetbuttcap%
\pgfsetroundjoin%
\definecolor{currentfill}{rgb}{0.317647,0.317647,0.317647}%
\pgfsetfillcolor{currentfill}%
\pgfsetlinewidth{0.501875pt}%
\definecolor{currentstroke}{rgb}{0.317647,0.317647,0.317647}%
\pgfsetstrokecolor{currentstroke}%
\pgfsetdash{}{0pt}%
\pgfsys@defobject{currentmarker}{\pgfqpoint{0.000000in}{-0.020833in}}{\pgfqpoint{0.000000in}{0.000000in}}{%
\pgfpathmoveto{\pgfqpoint{0.000000in}{0.000000in}}%
\pgfpathlineto{\pgfqpoint{0.000000in}{-0.020833in}}%
\pgfusepath{stroke,fill}%
}%
\begin{pgfscope}%
\pgfsys@transformshift{1.716841in}{0.410797in}%
\pgfsys@useobject{currentmarker}{}%
\end{pgfscope}%
\end{pgfscope}%
\begin{pgfscope}%
\definecolor{textcolor}{rgb}{0.317647,0.317647,0.317647}%
\pgfsetstrokecolor{textcolor}%
\pgfsetfillcolor{textcolor}%
\pgftext[x=1.716841in,y=0.362186in,,top]{\color{textcolor}\rmfamily\fontsize{8.330000}{9.996000}\selectfont \(\displaystyle 1000\)}%
\end{pgfscope}%
\begin{pgfscope}%
\pgfsetbuttcap%
\pgfsetroundjoin%
\definecolor{currentfill}{rgb}{0.317647,0.317647,0.317647}%
\pgfsetfillcolor{currentfill}%
\pgfsetlinewidth{0.501875pt}%
\definecolor{currentstroke}{rgb}{0.317647,0.317647,0.317647}%
\pgfsetstrokecolor{currentstroke}%
\pgfsetdash{}{0pt}%
\pgfsys@defobject{currentmarker}{\pgfqpoint{0.000000in}{-0.020833in}}{\pgfqpoint{0.000000in}{0.000000in}}{%
\pgfpathmoveto{\pgfqpoint{0.000000in}{0.000000in}}%
\pgfpathlineto{\pgfqpoint{0.000000in}{-0.020833in}}%
\pgfusepath{stroke,fill}%
}%
\begin{pgfscope}%
\pgfsys@transformshift{2.245911in}{0.410797in}%
\pgfsys@useobject{currentmarker}{}%
\end{pgfscope}%
\end{pgfscope}%
\begin{pgfscope}%
\definecolor{textcolor}{rgb}{0.317647,0.317647,0.317647}%
\pgfsetstrokecolor{textcolor}%
\pgfsetfillcolor{textcolor}%
\pgftext[x=2.245911in,y=0.362186in,,top]{\color{textcolor}\rmfamily\fontsize{8.330000}{9.996000}\selectfont \(\displaystyle 1500\)}%
\end{pgfscope}%
\begin{pgfscope}%
\pgfsetbuttcap%
\pgfsetroundjoin%
\definecolor{currentfill}{rgb}{0.317647,0.317647,0.317647}%
\pgfsetfillcolor{currentfill}%
\pgfsetlinewidth{0.501875pt}%
\definecolor{currentstroke}{rgb}{0.317647,0.317647,0.317647}%
\pgfsetstrokecolor{currentstroke}%
\pgfsetdash{}{0pt}%
\pgfsys@defobject{currentmarker}{\pgfqpoint{0.000000in}{-0.020833in}}{\pgfqpoint{0.000000in}{0.000000in}}{%
\pgfpathmoveto{\pgfqpoint{0.000000in}{0.000000in}}%
\pgfpathlineto{\pgfqpoint{0.000000in}{-0.020833in}}%
\pgfusepath{stroke,fill}%
}%
\begin{pgfscope}%
\pgfsys@transformshift{2.774981in}{0.410797in}%
\pgfsys@useobject{currentmarker}{}%
\end{pgfscope}%
\end{pgfscope}%
\begin{pgfscope}%
\definecolor{textcolor}{rgb}{0.317647,0.317647,0.317647}%
\pgfsetstrokecolor{textcolor}%
\pgfsetfillcolor{textcolor}%
\pgftext[x=2.774981in,y=0.362186in,,top]{\color{textcolor}\rmfamily\fontsize{8.330000}{9.996000}\selectfont \(\displaystyle 2000\)}%
\end{pgfscope}%
\begin{pgfscope}%
\pgfsetbuttcap%
\pgfsetroundjoin%
\definecolor{currentfill}{rgb}{0.317647,0.317647,0.317647}%
\pgfsetfillcolor{currentfill}%
\pgfsetlinewidth{0.501875pt}%
\definecolor{currentstroke}{rgb}{0.317647,0.317647,0.317647}%
\pgfsetstrokecolor{currentstroke}%
\pgfsetdash{}{0pt}%
\pgfsys@defobject{currentmarker}{\pgfqpoint{0.000000in}{-0.020833in}}{\pgfqpoint{0.000000in}{0.000000in}}{%
\pgfpathmoveto{\pgfqpoint{0.000000in}{0.000000in}}%
\pgfpathlineto{\pgfqpoint{0.000000in}{-0.020833in}}%
\pgfusepath{stroke,fill}%
}%
\begin{pgfscope}%
\pgfsys@transformshift{3.304052in}{0.410797in}%
\pgfsys@useobject{currentmarker}{}%
\end{pgfscope}%
\end{pgfscope}%
\begin{pgfscope}%
\definecolor{textcolor}{rgb}{0.317647,0.317647,0.317647}%
\pgfsetstrokecolor{textcolor}%
\pgfsetfillcolor{textcolor}%
\pgftext[x=3.304052in,y=0.362186in,,top]{\color{textcolor}\rmfamily\fontsize{8.330000}{9.996000}\selectfont \(\displaystyle 2500\)}%
\end{pgfscope}%
\begin{pgfscope}%
\pgfsetbuttcap%
\pgfsetroundjoin%
\definecolor{currentfill}{rgb}{0.317647,0.317647,0.317647}%
\pgfsetfillcolor{currentfill}%
\pgfsetlinewidth{0.501875pt}%
\definecolor{currentstroke}{rgb}{0.317647,0.317647,0.317647}%
\pgfsetstrokecolor{currentstroke}%
\pgfsetdash{}{0pt}%
\pgfsys@defobject{currentmarker}{\pgfqpoint{0.000000in}{-0.020833in}}{\pgfqpoint{0.000000in}{0.000000in}}{%
\pgfpathmoveto{\pgfqpoint{0.000000in}{0.000000in}}%
\pgfpathlineto{\pgfqpoint{0.000000in}{-0.020833in}}%
\pgfusepath{stroke,fill}%
}%
\begin{pgfscope}%
\pgfsys@transformshift{3.833122in}{0.410797in}%
\pgfsys@useobject{currentmarker}{}%
\end{pgfscope}%
\end{pgfscope}%
\begin{pgfscope}%
\definecolor{textcolor}{rgb}{0.317647,0.317647,0.317647}%
\pgfsetstrokecolor{textcolor}%
\pgfsetfillcolor{textcolor}%
\pgftext[x=3.833122in,y=0.362186in,,top]{\color{textcolor}\rmfamily\fontsize{8.330000}{9.996000}\selectfont \(\displaystyle 3000\)}%
\end{pgfscope}%
\begin{pgfscope}%
\pgfsetbuttcap%
\pgfsetroundjoin%
\definecolor{currentfill}{rgb}{0.317647,0.317647,0.317647}%
\pgfsetfillcolor{currentfill}%
\pgfsetlinewidth{0.501875pt}%
\definecolor{currentstroke}{rgb}{0.317647,0.317647,0.317647}%
\pgfsetstrokecolor{currentstroke}%
\pgfsetdash{}{0pt}%
\pgfsys@defobject{currentmarker}{\pgfqpoint{0.000000in}{-0.020833in}}{\pgfqpoint{0.000000in}{0.000000in}}{%
\pgfpathmoveto{\pgfqpoint{0.000000in}{0.000000in}}%
\pgfpathlineto{\pgfqpoint{0.000000in}{-0.020833in}}%
\pgfusepath{stroke,fill}%
}%
\begin{pgfscope}%
\pgfsys@transformshift{4.362193in}{0.410797in}%
\pgfsys@useobject{currentmarker}{}%
\end{pgfscope}%
\end{pgfscope}%
\begin{pgfscope}%
\definecolor{textcolor}{rgb}{0.317647,0.317647,0.317647}%
\pgfsetstrokecolor{textcolor}%
\pgfsetfillcolor{textcolor}%
\pgftext[x=4.362193in,y=0.362186in,,top]{\color{textcolor}\rmfamily\fontsize{8.330000}{9.996000}\selectfont \(\displaystyle 3500\)}%
\end{pgfscope}%
\begin{pgfscope}%
\pgfsetbuttcap%
\pgfsetroundjoin%
\definecolor{currentfill}{rgb}{0.317647,0.317647,0.317647}%
\pgfsetfillcolor{currentfill}%
\pgfsetlinewidth{0.501875pt}%
\definecolor{currentstroke}{rgb}{0.317647,0.317647,0.317647}%
\pgfsetstrokecolor{currentstroke}%
\pgfsetdash{}{0pt}%
\pgfsys@defobject{currentmarker}{\pgfqpoint{0.000000in}{-0.020833in}}{\pgfqpoint{0.000000in}{0.000000in}}{%
\pgfpathmoveto{\pgfqpoint{0.000000in}{0.000000in}}%
\pgfpathlineto{\pgfqpoint{0.000000in}{-0.020833in}}%
\pgfusepath{stroke,fill}%
}%
\begin{pgfscope}%
\pgfsys@transformshift{4.891263in}{0.410797in}%
\pgfsys@useobject{currentmarker}{}%
\end{pgfscope}%
\end{pgfscope}%
\begin{pgfscope}%
\definecolor{textcolor}{rgb}{0.317647,0.317647,0.317647}%
\pgfsetstrokecolor{textcolor}%
\pgfsetfillcolor{textcolor}%
\pgftext[x=4.891263in,y=0.362186in,,top]{\color{textcolor}\rmfamily\fontsize{8.330000}{9.996000}\selectfont \(\displaystyle 4000\)}%
\end{pgfscope}%
\begin{pgfscope}%
\definecolor{textcolor}{rgb}{0.317647,0.317647,0.317647}%
\pgfsetstrokecolor{textcolor}%
\pgfsetfillcolor{textcolor}%
\pgftext[x=2.772336in,y=0.203893in,,top]{\color{textcolor}\rmfamily\fontsize{8.330000}{9.996000}\selectfont Steps}%
\end{pgfscope}%
\begin{pgfscope}%
\pgfsetbuttcap%
\pgfsetroundjoin%
\definecolor{currentfill}{rgb}{0.317647,0.317647,0.317647}%
\pgfsetfillcolor{currentfill}%
\pgfsetlinewidth{0.501875pt}%
\definecolor{currentstroke}{rgb}{0.317647,0.317647,0.317647}%
\pgfsetstrokecolor{currentstroke}%
\pgfsetdash{}{0pt}%
\pgfsys@defobject{currentmarker}{\pgfqpoint{-0.020833in}{0.000000in}}{\pgfqpoint{0.000000in}{0.000000in}}{%
\pgfpathmoveto{\pgfqpoint{0.000000in}{0.000000in}}%
\pgfpathlineto{\pgfqpoint{-0.020833in}{0.000000in}}%
\pgfusepath{stroke,fill}%
}%
\begin{pgfscope}%
\pgfsys@transformshift{0.447336in}{0.771145in}%
\pgfsys@useobject{currentmarker}{}%
\end{pgfscope}%
\end{pgfscope}%
\begin{pgfscope}%
\definecolor{textcolor}{rgb}{0.317647,0.317647,0.317647}%
\pgfsetstrokecolor{textcolor}%
\pgfsetfillcolor{textcolor}%
\pgftext[x=0.294557in,y=0.730999in,left,base]{\color{textcolor}\rmfamily\fontsize{8.330000}{9.996000}\selectfont \(\displaystyle 20\)}%
\end{pgfscope}%
\begin{pgfscope}%
\pgfsetbuttcap%
\pgfsetroundjoin%
\definecolor{currentfill}{rgb}{0.317647,0.317647,0.317647}%
\pgfsetfillcolor{currentfill}%
\pgfsetlinewidth{0.501875pt}%
\definecolor{currentstroke}{rgb}{0.317647,0.317647,0.317647}%
\pgfsetstrokecolor{currentstroke}%
\pgfsetdash{}{0pt}%
\pgfsys@defobject{currentmarker}{\pgfqpoint{-0.020833in}{0.000000in}}{\pgfqpoint{0.000000in}{0.000000in}}{%
\pgfpathmoveto{\pgfqpoint{0.000000in}{0.000000in}}%
\pgfpathlineto{\pgfqpoint{-0.020833in}{0.000000in}}%
\pgfusepath{stroke,fill}%
}%
\begin{pgfscope}%
\pgfsys@transformshift{0.447336in}{1.185688in}%
\pgfsys@useobject{currentmarker}{}%
\end{pgfscope}%
\end{pgfscope}%
\begin{pgfscope}%
\definecolor{textcolor}{rgb}{0.317647,0.317647,0.317647}%
\pgfsetstrokecolor{textcolor}%
\pgfsetfillcolor{textcolor}%
\pgftext[x=0.294557in,y=1.145543in,left,base]{\color{textcolor}\rmfamily\fontsize{8.330000}{9.996000}\selectfont \(\displaystyle 40\)}%
\end{pgfscope}%
\begin{pgfscope}%
\pgfsetbuttcap%
\pgfsetroundjoin%
\definecolor{currentfill}{rgb}{0.317647,0.317647,0.317647}%
\pgfsetfillcolor{currentfill}%
\pgfsetlinewidth{0.501875pt}%
\definecolor{currentstroke}{rgb}{0.317647,0.317647,0.317647}%
\pgfsetstrokecolor{currentstroke}%
\pgfsetdash{}{0pt}%
\pgfsys@defobject{currentmarker}{\pgfqpoint{-0.020833in}{0.000000in}}{\pgfqpoint{0.000000in}{0.000000in}}{%
\pgfpathmoveto{\pgfqpoint{0.000000in}{0.000000in}}%
\pgfpathlineto{\pgfqpoint{-0.020833in}{0.000000in}}%
\pgfusepath{stroke,fill}%
}%
\begin{pgfscope}%
\pgfsys@transformshift{0.447336in}{1.600232in}%
\pgfsys@useobject{currentmarker}{}%
\end{pgfscope}%
\end{pgfscope}%
\begin{pgfscope}%
\definecolor{textcolor}{rgb}{0.317647,0.317647,0.317647}%
\pgfsetstrokecolor{textcolor}%
\pgfsetfillcolor{textcolor}%
\pgftext[x=0.294557in,y=1.560086in,left,base]{\color{textcolor}\rmfamily\fontsize{8.330000}{9.996000}\selectfont \(\displaystyle 60\)}%
\end{pgfscope}%
\begin{pgfscope}%
\definecolor{textcolor}{rgb}{0.317647,0.317647,0.317647}%
\pgfsetstrokecolor{textcolor}%
\pgfsetfillcolor{textcolor}%
\pgftext[x=0.239001in,y=1.113122in,,bottom,rotate=90.000000]{\color{textcolor}\rmfamily\fontsize{8.330000}{9.996000}\selectfont RMSE (kHz)}%
\end{pgfscope}%
\begin{pgfscope}%
\pgfpathrectangle{\pgfqpoint{0.447336in}{0.410797in}}{\pgfqpoint{4.650000in}{1.404651in}}%
\pgfusepath{clip}%
\pgfsetrectcap%
\pgfsetroundjoin%
\pgfsetlinewidth{0.803000pt}%
\definecolor{currentstroke}{rgb}{0.333333,0.333333,0.333333}%
\pgfsetstrokecolor{currentstroke}%
\pgfsetdash{}{0pt}%
\pgfpathmoveto{\pgfqpoint{0.658700in}{1.751600in}}%
\pgfpathlineto{\pgfqpoint{0.669281in}{1.572548in}}%
\pgfpathlineto{\pgfqpoint{0.674572in}{1.563947in}}%
\pgfpathlineto{\pgfqpoint{0.679863in}{1.608741in}}%
\pgfpathlineto{\pgfqpoint{0.685153in}{1.607122in}}%
\pgfpathlineto{\pgfqpoint{0.690444in}{1.628251in}}%
\pgfpathlineto{\pgfqpoint{0.695735in}{1.624331in}}%
\pgfpathlineto{\pgfqpoint{0.701025in}{1.627089in}}%
\pgfpathlineto{\pgfqpoint{0.706316in}{1.642170in}}%
\pgfpathlineto{\pgfqpoint{0.711607in}{1.591781in}}%
\pgfpathlineto{\pgfqpoint{0.716897in}{1.574563in}}%
\pgfpathlineto{\pgfqpoint{0.722188in}{1.571965in}}%
\pgfpathlineto{\pgfqpoint{0.727479in}{1.541700in}}%
\pgfpathlineto{\pgfqpoint{0.732770in}{1.550488in}}%
\pgfpathlineto{\pgfqpoint{0.738060in}{1.555036in}}%
\pgfpathlineto{\pgfqpoint{0.743351in}{1.582601in}}%
\pgfpathlineto{\pgfqpoint{0.748642in}{1.505167in}}%
\pgfpathlineto{\pgfqpoint{0.753932in}{1.551351in}}%
\pgfpathlineto{\pgfqpoint{0.759223in}{1.506355in}}%
\pgfpathlineto{\pgfqpoint{0.764514in}{1.433127in}}%
\pgfpathlineto{\pgfqpoint{0.769805in}{1.479489in}}%
\pgfpathlineto{\pgfqpoint{0.775095in}{1.251358in}}%
\pgfpathlineto{\pgfqpoint{0.780386in}{1.322636in}}%
\pgfpathlineto{\pgfqpoint{0.785677in}{1.319769in}}%
\pgfpathlineto{\pgfqpoint{0.790967in}{1.320185in}}%
\pgfpathlineto{\pgfqpoint{0.796258in}{1.313070in}}%
\pgfpathlineto{\pgfqpoint{0.801549in}{1.230852in}}%
\pgfpathlineto{\pgfqpoint{0.812130in}{1.282986in}}%
\pgfpathlineto{\pgfqpoint{0.817421in}{1.330274in}}%
\pgfpathlineto{\pgfqpoint{0.822712in}{1.303525in}}%
\pgfpathlineto{\pgfqpoint{0.828002in}{1.036472in}}%
\pgfpathlineto{\pgfqpoint{0.833293in}{1.016526in}}%
\pgfpathlineto{\pgfqpoint{0.838584in}{1.038372in}}%
\pgfpathlineto{\pgfqpoint{0.843874in}{1.041606in}}%
\pgfpathlineto{\pgfqpoint{0.849165in}{1.262828in}}%
\pgfpathlineto{\pgfqpoint{0.854456in}{1.037330in}}%
\pgfpathlineto{\pgfqpoint{0.859747in}{1.073239in}}%
\pgfpathlineto{\pgfqpoint{0.865037in}{1.067947in}}%
\pgfpathlineto{\pgfqpoint{0.870328in}{1.045853in}}%
\pgfpathlineto{\pgfqpoint{0.875619in}{1.138341in}}%
\pgfpathlineto{\pgfqpoint{0.880909in}{1.184345in}}%
\pgfpathlineto{\pgfqpoint{0.886200in}{1.292892in}}%
\pgfpathlineto{\pgfqpoint{0.891491in}{1.224975in}}%
\pgfpathlineto{\pgfqpoint{0.896781in}{1.238613in}}%
\pgfpathlineto{\pgfqpoint{0.902072in}{1.216208in}}%
\pgfpathlineto{\pgfqpoint{0.912654in}{1.097055in}}%
\pgfpathlineto{\pgfqpoint{0.917944in}{1.062814in}}%
\pgfpathlineto{\pgfqpoint{0.923235in}{1.006838in}}%
\pgfpathlineto{\pgfqpoint{0.928526in}{1.177611in}}%
\pgfpathlineto{\pgfqpoint{0.933816in}{0.974905in}}%
\pgfpathlineto{\pgfqpoint{0.944398in}{1.026076in}}%
\pgfpathlineto{\pgfqpoint{0.949688in}{1.031231in}}%
\pgfpathlineto{\pgfqpoint{0.954979in}{0.941371in}}%
\pgfpathlineto{\pgfqpoint{0.960270in}{0.942466in}}%
\pgfpathlineto{\pgfqpoint{0.965561in}{1.001895in}}%
\pgfpathlineto{\pgfqpoint{0.970851in}{1.107355in}}%
\pgfpathlineto{\pgfqpoint{0.976142in}{1.016361in}}%
\pgfpathlineto{\pgfqpoint{0.981433in}{1.017974in}}%
\pgfpathlineto{\pgfqpoint{0.986723in}{0.987363in}}%
\pgfpathlineto{\pgfqpoint{0.992014in}{1.011336in}}%
\pgfpathlineto{\pgfqpoint{0.997305in}{0.949939in}}%
\pgfpathlineto{\pgfqpoint{1.002596in}{0.953477in}}%
\pgfpathlineto{\pgfqpoint{1.007886in}{0.952819in}}%
\pgfpathlineto{\pgfqpoint{1.013177in}{1.002050in}}%
\pgfpathlineto{\pgfqpoint{1.018468in}{0.959194in}}%
\pgfpathlineto{\pgfqpoint{1.023758in}{0.951006in}}%
\pgfpathlineto{\pgfqpoint{1.029049in}{0.953271in}}%
\pgfpathlineto{\pgfqpoint{1.034340in}{0.941715in}}%
\pgfpathlineto{\pgfqpoint{1.044921in}{0.900820in}}%
\pgfpathlineto{\pgfqpoint{1.050212in}{0.899721in}}%
\pgfpathlineto{\pgfqpoint{1.055503in}{0.878717in}}%
\pgfpathlineto{\pgfqpoint{1.060793in}{0.846080in}}%
\pgfpathlineto{\pgfqpoint{1.066084in}{0.864632in}}%
\pgfpathlineto{\pgfqpoint{1.071375in}{1.151805in}}%
\pgfpathlineto{\pgfqpoint{1.076665in}{0.865145in}}%
\pgfpathlineto{\pgfqpoint{1.081956in}{0.846388in}}%
\pgfpathlineto{\pgfqpoint{1.087247in}{0.871472in}}%
\pgfpathlineto{\pgfqpoint{1.097828in}{1.042889in}}%
\pgfpathlineto{\pgfqpoint{1.103119in}{0.944555in}}%
\pgfpathlineto{\pgfqpoint{1.108410in}{0.924032in}}%
\pgfpathlineto{\pgfqpoint{1.113700in}{0.945136in}}%
\pgfpathlineto{\pgfqpoint{1.118991in}{0.978789in}}%
\pgfpathlineto{\pgfqpoint{1.124282in}{0.902655in}}%
\pgfpathlineto{\pgfqpoint{1.129572in}{1.187987in}}%
\pgfpathlineto{\pgfqpoint{1.134863in}{0.900519in}}%
\pgfpathlineto{\pgfqpoint{1.140154in}{0.887102in}}%
\pgfpathlineto{\pgfqpoint{1.145445in}{1.112773in}}%
\pgfpathlineto{\pgfqpoint{1.150735in}{0.934171in}}%
\pgfpathlineto{\pgfqpoint{1.156026in}{0.948981in}}%
\pgfpathlineto{\pgfqpoint{1.161317in}{0.854660in}}%
\pgfpathlineto{\pgfqpoint{1.166607in}{1.177278in}}%
\pgfpathlineto{\pgfqpoint{1.171898in}{0.863763in}}%
\pgfpathlineto{\pgfqpoint{1.177189in}{0.876832in}}%
\pgfpathlineto{\pgfqpoint{1.182479in}{0.826276in}}%
\pgfpathlineto{\pgfqpoint{1.203642in}{1.183539in}}%
\pgfpathlineto{\pgfqpoint{1.208933in}{1.194445in}}%
\pgfpathlineto{\pgfqpoint{1.214224in}{1.148493in}}%
\pgfpathlineto{\pgfqpoint{1.219514in}{1.139812in}}%
\pgfpathlineto{\pgfqpoint{1.224805in}{1.332071in}}%
\pgfpathlineto{\pgfqpoint{1.230096in}{1.340623in}}%
\pgfpathlineto{\pgfqpoint{1.235387in}{1.239633in}}%
\pgfpathlineto{\pgfqpoint{1.240677in}{1.091602in}}%
\pgfpathlineto{\pgfqpoint{1.245968in}{1.168748in}}%
\pgfpathlineto{\pgfqpoint{1.251259in}{1.268868in}}%
\pgfpathlineto{\pgfqpoint{1.256549in}{1.253435in}}%
\pgfpathlineto{\pgfqpoint{1.261840in}{1.068093in}}%
\pgfpathlineto{\pgfqpoint{1.267131in}{1.021320in}}%
\pgfpathlineto{\pgfqpoint{1.272421in}{1.127213in}}%
\pgfpathlineto{\pgfqpoint{1.277712in}{1.138170in}}%
\pgfpathlineto{\pgfqpoint{1.283003in}{1.002238in}}%
\pgfpathlineto{\pgfqpoint{1.288294in}{1.033500in}}%
\pgfpathlineto{\pgfqpoint{1.293584in}{0.873070in}}%
\pgfpathlineto{\pgfqpoint{1.298875in}{0.900320in}}%
\pgfpathlineto{\pgfqpoint{1.304166in}{1.125281in}}%
\pgfpathlineto{\pgfqpoint{1.309456in}{0.978475in}}%
\pgfpathlineto{\pgfqpoint{1.314747in}{1.010319in}}%
\pgfpathlineto{\pgfqpoint{1.320038in}{1.121870in}}%
\pgfpathlineto{\pgfqpoint{1.325328in}{1.052583in}}%
\pgfpathlineto{\pgfqpoint{1.330619in}{1.008637in}}%
\pgfpathlineto{\pgfqpoint{1.335910in}{0.902142in}}%
\pgfpathlineto{\pgfqpoint{1.341201in}{0.991510in}}%
\pgfpathlineto{\pgfqpoint{1.346491in}{1.150793in}}%
\pgfpathlineto{\pgfqpoint{1.351782in}{1.190515in}}%
\pgfpathlineto{\pgfqpoint{1.357073in}{1.170990in}}%
\pgfpathlineto{\pgfqpoint{1.362363in}{1.023887in}}%
\pgfpathlineto{\pgfqpoint{1.367654in}{1.002145in}}%
\pgfpathlineto{\pgfqpoint{1.372945in}{1.099589in}}%
\pgfpathlineto{\pgfqpoint{1.378236in}{0.996351in}}%
\pgfpathlineto{\pgfqpoint{1.383526in}{0.985944in}}%
\pgfpathlineto{\pgfqpoint{1.388817in}{0.987105in}}%
\pgfpathlineto{\pgfqpoint{1.394108in}{0.984349in}}%
\pgfpathlineto{\pgfqpoint{1.399398in}{1.041521in}}%
\pgfpathlineto{\pgfqpoint{1.404689in}{0.923964in}}%
\pgfpathlineto{\pgfqpoint{1.409980in}{1.021629in}}%
\pgfpathlineto{\pgfqpoint{1.415270in}{0.965818in}}%
\pgfpathlineto{\pgfqpoint{1.420561in}{0.938819in}}%
\pgfpathlineto{\pgfqpoint{1.425852in}{0.902348in}}%
\pgfpathlineto{\pgfqpoint{1.431143in}{0.989730in}}%
\pgfpathlineto{\pgfqpoint{1.436433in}{1.014035in}}%
\pgfpathlineto{\pgfqpoint{1.441724in}{0.858440in}}%
\pgfpathlineto{\pgfqpoint{1.452305in}{0.945039in}}%
\pgfpathlineto{\pgfqpoint{1.457596in}{0.856499in}}%
\pgfpathlineto{\pgfqpoint{1.468178in}{0.939931in}}%
\pgfpathlineto{\pgfqpoint{1.473468in}{0.999872in}}%
\pgfpathlineto{\pgfqpoint{1.478759in}{0.801564in}}%
\pgfpathlineto{\pgfqpoint{1.484050in}{0.843503in}}%
\pgfpathlineto{\pgfqpoint{1.489340in}{0.834910in}}%
\pgfpathlineto{\pgfqpoint{1.499922in}{0.733772in}}%
\pgfpathlineto{\pgfqpoint{1.505212in}{0.818789in}}%
\pgfpathlineto{\pgfqpoint{1.510503in}{0.813121in}}%
\pgfpathlineto{\pgfqpoint{1.515794in}{1.002880in}}%
\pgfpathlineto{\pgfqpoint{1.521085in}{1.010250in}}%
\pgfpathlineto{\pgfqpoint{1.526375in}{1.031043in}}%
\pgfpathlineto{\pgfqpoint{1.531666in}{0.942570in}}%
\pgfpathlineto{\pgfqpoint{1.536957in}{0.916280in}}%
\pgfpathlineto{\pgfqpoint{1.542247in}{1.032413in}}%
\pgfpathlineto{\pgfqpoint{1.547538in}{0.857716in}}%
\pgfpathlineto{\pgfqpoint{1.552829in}{0.861038in}}%
\pgfpathlineto{\pgfqpoint{1.558119in}{0.835919in}}%
\pgfpathlineto{\pgfqpoint{1.563410in}{0.834286in}}%
\pgfpathlineto{\pgfqpoint{1.568701in}{0.868183in}}%
\pgfpathlineto{\pgfqpoint{1.573992in}{0.997136in}}%
\pgfpathlineto{\pgfqpoint{1.579282in}{1.034006in}}%
\pgfpathlineto{\pgfqpoint{1.584573in}{1.046284in}}%
\pgfpathlineto{\pgfqpoint{1.589864in}{0.887513in}}%
\pgfpathlineto{\pgfqpoint{1.595154in}{1.041504in}}%
\pgfpathlineto{\pgfqpoint{1.600445in}{1.033521in}}%
\pgfpathlineto{\pgfqpoint{1.605736in}{1.094021in}}%
\pgfpathlineto{\pgfqpoint{1.611027in}{1.074460in}}%
\pgfpathlineto{\pgfqpoint{1.616317in}{1.033636in}}%
\pgfpathlineto{\pgfqpoint{1.621608in}{1.072051in}}%
\pgfpathlineto{\pgfqpoint{1.626899in}{1.189017in}}%
\pgfpathlineto{\pgfqpoint{1.632189in}{1.061046in}}%
\pgfpathlineto{\pgfqpoint{1.637480in}{1.030427in}}%
\pgfpathlineto{\pgfqpoint{1.642771in}{0.974016in}}%
\pgfpathlineto{\pgfqpoint{1.648061in}{0.988738in}}%
\pgfpathlineto{\pgfqpoint{1.658643in}{1.001979in}}%
\pgfpathlineto{\pgfqpoint{1.663934in}{1.026991in}}%
\pgfpathlineto{\pgfqpoint{1.669224in}{1.042176in}}%
\pgfpathlineto{\pgfqpoint{1.679806in}{0.992659in}}%
\pgfpathlineto{\pgfqpoint{1.685096in}{0.906873in}}%
\pgfpathlineto{\pgfqpoint{1.690387in}{0.857432in}}%
\pgfpathlineto{\pgfqpoint{1.695678in}{0.830622in}}%
\pgfpathlineto{\pgfqpoint{1.700968in}{0.789888in}}%
\pgfpathlineto{\pgfqpoint{1.706259in}{0.948452in}}%
\pgfpathlineto{\pgfqpoint{1.711550in}{0.830554in}}%
\pgfpathlineto{\pgfqpoint{1.716841in}{0.891250in}}%
\pgfpathlineto{\pgfqpoint{1.722131in}{0.891753in}}%
\pgfpathlineto{\pgfqpoint{1.727422in}{1.035177in}}%
\pgfpathlineto{\pgfqpoint{1.732713in}{1.079870in}}%
\pgfpathlineto{\pgfqpoint{1.738003in}{0.916323in}}%
\pgfpathlineto{\pgfqpoint{1.743294in}{1.001312in}}%
\pgfpathlineto{\pgfqpoint{1.748585in}{0.828121in}}%
\pgfpathlineto{\pgfqpoint{1.753876in}{0.854565in}}%
\pgfpathlineto{\pgfqpoint{1.759166in}{0.846865in}}%
\pgfpathlineto{\pgfqpoint{1.764457in}{0.873827in}}%
\pgfpathlineto{\pgfqpoint{1.769748in}{0.870779in}}%
\pgfpathlineto{\pgfqpoint{1.775038in}{0.757464in}}%
\pgfpathlineto{\pgfqpoint{1.780329in}{0.791506in}}%
\pgfpathlineto{\pgfqpoint{1.785620in}{1.042085in}}%
\pgfpathlineto{\pgfqpoint{1.790910in}{0.962243in}}%
\pgfpathlineto{\pgfqpoint{1.796201in}{0.943799in}}%
\pgfpathlineto{\pgfqpoint{1.801492in}{0.935718in}}%
\pgfpathlineto{\pgfqpoint{1.806783in}{0.753952in}}%
\pgfpathlineto{\pgfqpoint{1.812073in}{0.768795in}}%
\pgfpathlineto{\pgfqpoint{1.822655in}{0.741917in}}%
\pgfpathlineto{\pgfqpoint{1.827945in}{0.747817in}}%
\pgfpathlineto{\pgfqpoint{1.833236in}{0.748928in}}%
\pgfpathlineto{\pgfqpoint{1.838527in}{0.795425in}}%
\pgfpathlineto{\pgfqpoint{1.843818in}{0.776301in}}%
\pgfpathlineto{\pgfqpoint{1.849108in}{0.777870in}}%
\pgfpathlineto{\pgfqpoint{1.854399in}{0.835383in}}%
\pgfpathlineto{\pgfqpoint{1.859690in}{0.874961in}}%
\pgfpathlineto{\pgfqpoint{1.864980in}{0.825531in}}%
\pgfpathlineto{\pgfqpoint{1.870271in}{0.904693in}}%
\pgfpathlineto{\pgfqpoint{1.875562in}{0.885827in}}%
\pgfpathlineto{\pgfqpoint{1.880852in}{0.923868in}}%
\pgfpathlineto{\pgfqpoint{1.886143in}{0.935392in}}%
\pgfpathlineto{\pgfqpoint{1.891434in}{0.926886in}}%
\pgfpathlineto{\pgfqpoint{1.896725in}{0.961683in}}%
\pgfpathlineto{\pgfqpoint{1.902015in}{0.966764in}}%
\pgfpathlineto{\pgfqpoint{1.907306in}{0.960529in}}%
\pgfpathlineto{\pgfqpoint{1.912597in}{0.929828in}}%
\pgfpathlineto{\pgfqpoint{1.917887in}{0.953462in}}%
\pgfpathlineto{\pgfqpoint{1.923178in}{0.966801in}}%
\pgfpathlineto{\pgfqpoint{1.928469in}{1.003940in}}%
\pgfpathlineto{\pgfqpoint{1.933759in}{0.831299in}}%
\pgfpathlineto{\pgfqpoint{1.939050in}{0.932239in}}%
\pgfpathlineto{\pgfqpoint{1.944341in}{0.910124in}}%
\pgfpathlineto{\pgfqpoint{1.949632in}{0.839139in}}%
\pgfpathlineto{\pgfqpoint{1.960213in}{0.949287in}}%
\pgfpathlineto{\pgfqpoint{1.965504in}{0.923285in}}%
\pgfpathlineto{\pgfqpoint{1.970794in}{0.966019in}}%
\pgfpathlineto{\pgfqpoint{1.981376in}{1.017446in}}%
\pgfpathlineto{\pgfqpoint{1.986667in}{0.936038in}}%
\pgfpathlineto{\pgfqpoint{1.991957in}{0.950512in}}%
\pgfpathlineto{\pgfqpoint{1.997248in}{1.010129in}}%
\pgfpathlineto{\pgfqpoint{2.002539in}{0.984632in}}%
\pgfpathlineto{\pgfqpoint{2.007829in}{1.020677in}}%
\pgfpathlineto{\pgfqpoint{2.013120in}{0.955549in}}%
\pgfpathlineto{\pgfqpoint{2.018411in}{0.835030in}}%
\pgfpathlineto{\pgfqpoint{2.023701in}{0.841160in}}%
\pgfpathlineto{\pgfqpoint{2.028992in}{0.833909in}}%
\pgfpathlineto{\pgfqpoint{2.034283in}{1.025403in}}%
\pgfpathlineto{\pgfqpoint{2.039574in}{0.990799in}}%
\pgfpathlineto{\pgfqpoint{2.044864in}{1.020743in}}%
\pgfpathlineto{\pgfqpoint{2.050155in}{1.025194in}}%
\pgfpathlineto{\pgfqpoint{2.055446in}{0.872467in}}%
\pgfpathlineto{\pgfqpoint{2.060736in}{0.822884in}}%
\pgfpathlineto{\pgfqpoint{2.066027in}{0.853450in}}%
\pgfpathlineto{\pgfqpoint{2.071318in}{0.833818in}}%
\pgfpathlineto{\pgfqpoint{2.076608in}{0.843374in}}%
\pgfpathlineto{\pgfqpoint{2.081899in}{0.856958in}}%
\pgfpathlineto{\pgfqpoint{2.087190in}{0.862059in}}%
\pgfpathlineto{\pgfqpoint{2.092481in}{0.889752in}}%
\pgfpathlineto{\pgfqpoint{2.097771in}{0.992348in}}%
\pgfpathlineto{\pgfqpoint{2.103062in}{1.012517in}}%
\pgfpathlineto{\pgfqpoint{2.108353in}{1.012167in}}%
\pgfpathlineto{\pgfqpoint{2.113643in}{1.032453in}}%
\pgfpathlineto{\pgfqpoint{2.118934in}{0.966851in}}%
\pgfpathlineto{\pgfqpoint{2.124225in}{0.805423in}}%
\pgfpathlineto{\pgfqpoint{2.129516in}{0.806098in}}%
\pgfpathlineto{\pgfqpoint{2.134806in}{0.888504in}}%
\pgfpathlineto{\pgfqpoint{2.140097in}{0.861277in}}%
\pgfpathlineto{\pgfqpoint{2.145388in}{0.861871in}}%
\pgfpathlineto{\pgfqpoint{2.150678in}{0.881152in}}%
\pgfpathlineto{\pgfqpoint{2.155969in}{1.146223in}}%
\pgfpathlineto{\pgfqpoint{2.161260in}{0.800084in}}%
\pgfpathlineto{\pgfqpoint{2.166550in}{0.761784in}}%
\pgfpathlineto{\pgfqpoint{2.171841in}{0.774802in}}%
\pgfpathlineto{\pgfqpoint{2.177132in}{0.781373in}}%
\pgfpathlineto{\pgfqpoint{2.182423in}{0.698894in}}%
\pgfpathlineto{\pgfqpoint{2.187713in}{0.904679in}}%
\pgfpathlineto{\pgfqpoint{2.193004in}{0.704449in}}%
\pgfpathlineto{\pgfqpoint{2.198295in}{0.694051in}}%
\pgfpathlineto{\pgfqpoint{2.203585in}{0.722516in}}%
\pgfpathlineto{\pgfqpoint{2.208876in}{0.807938in}}%
\pgfpathlineto{\pgfqpoint{2.214167in}{0.771893in}}%
\pgfpathlineto{\pgfqpoint{2.219458in}{0.805620in}}%
\pgfpathlineto{\pgfqpoint{2.224748in}{0.740781in}}%
\pgfpathlineto{\pgfqpoint{2.230039in}{0.760877in}}%
\pgfpathlineto{\pgfqpoint{2.235330in}{0.828968in}}%
\pgfpathlineto{\pgfqpoint{2.240620in}{0.729666in}}%
\pgfpathlineto{\pgfqpoint{2.245911in}{0.842690in}}%
\pgfpathlineto{\pgfqpoint{2.251202in}{0.776900in}}%
\pgfpathlineto{\pgfqpoint{2.256492in}{0.770417in}}%
\pgfpathlineto{\pgfqpoint{2.261783in}{0.740224in}}%
\pgfpathlineto{\pgfqpoint{2.267074in}{0.723283in}}%
\pgfpathlineto{\pgfqpoint{2.272365in}{0.733168in}}%
\pgfpathlineto{\pgfqpoint{2.277655in}{0.791702in}}%
\pgfpathlineto{\pgfqpoint{2.282946in}{0.784531in}}%
\pgfpathlineto{\pgfqpoint{2.288237in}{0.800454in}}%
\pgfpathlineto{\pgfqpoint{2.293527in}{0.823967in}}%
\pgfpathlineto{\pgfqpoint{2.298818in}{0.830130in}}%
\pgfpathlineto{\pgfqpoint{2.304109in}{0.829391in}}%
\pgfpathlineto{\pgfqpoint{2.309399in}{1.308248in}}%
\pgfpathlineto{\pgfqpoint{2.314690in}{0.805585in}}%
\pgfpathlineto{\pgfqpoint{2.319981in}{0.844662in}}%
\pgfpathlineto{\pgfqpoint{2.330562in}{0.794544in}}%
\pgfpathlineto{\pgfqpoint{2.335853in}{0.785276in}}%
\pgfpathlineto{\pgfqpoint{2.346434in}{0.857246in}}%
\pgfpathlineto{\pgfqpoint{2.351725in}{0.826394in}}%
\pgfpathlineto{\pgfqpoint{2.357016in}{1.192072in}}%
\pgfpathlineto{\pgfqpoint{2.362307in}{0.787284in}}%
\pgfpathlineto{\pgfqpoint{2.367597in}{0.728852in}}%
\pgfpathlineto{\pgfqpoint{2.372888in}{0.770776in}}%
\pgfpathlineto{\pgfqpoint{2.378179in}{0.723599in}}%
\pgfpathlineto{\pgfqpoint{2.383469in}{0.877804in}}%
\pgfpathlineto{\pgfqpoint{2.388760in}{0.946704in}}%
\pgfpathlineto{\pgfqpoint{2.394051in}{0.799575in}}%
\pgfpathlineto{\pgfqpoint{2.399341in}{0.790263in}}%
\pgfpathlineto{\pgfqpoint{2.404632in}{0.788775in}}%
\pgfpathlineto{\pgfqpoint{2.409923in}{0.743237in}}%
\pgfpathlineto{\pgfqpoint{2.415214in}{0.757070in}}%
\pgfpathlineto{\pgfqpoint{2.420504in}{0.741343in}}%
\pgfpathlineto{\pgfqpoint{2.425795in}{0.708813in}}%
\pgfpathlineto{\pgfqpoint{2.431086in}{0.720178in}}%
\pgfpathlineto{\pgfqpoint{2.436376in}{0.699063in}}%
\pgfpathlineto{\pgfqpoint{2.441667in}{0.733045in}}%
\pgfpathlineto{\pgfqpoint{2.446958in}{0.717164in}}%
\pgfpathlineto{\pgfqpoint{2.452249in}{0.839704in}}%
\pgfpathlineto{\pgfqpoint{2.457539in}{0.854856in}}%
\pgfpathlineto{\pgfqpoint{2.462830in}{0.818199in}}%
\pgfpathlineto{\pgfqpoint{2.468121in}{0.999867in}}%
\pgfpathlineto{\pgfqpoint{2.473411in}{0.841139in}}%
\pgfpathlineto{\pgfqpoint{2.478702in}{0.843853in}}%
\pgfpathlineto{\pgfqpoint{2.483993in}{0.821663in}}%
\pgfpathlineto{\pgfqpoint{2.489283in}{0.838497in}}%
\pgfpathlineto{\pgfqpoint{2.494574in}{0.686142in}}%
\pgfpathlineto{\pgfqpoint{2.499865in}{0.709281in}}%
\pgfpathlineto{\pgfqpoint{2.510446in}{0.670002in}}%
\pgfpathlineto{\pgfqpoint{2.515737in}{0.709362in}}%
\pgfpathlineto{\pgfqpoint{2.521028in}{0.816636in}}%
\pgfpathlineto{\pgfqpoint{2.526318in}{0.881493in}}%
\pgfpathlineto{\pgfqpoint{2.531609in}{0.917784in}}%
\pgfpathlineto{\pgfqpoint{2.536900in}{0.857102in}}%
\pgfpathlineto{\pgfqpoint{2.542190in}{0.817623in}}%
\pgfpathlineto{\pgfqpoint{2.552772in}{0.891776in}}%
\pgfpathlineto{\pgfqpoint{2.558063in}{0.881785in}}%
\pgfpathlineto{\pgfqpoint{2.563353in}{0.811155in}}%
\pgfpathlineto{\pgfqpoint{2.568644in}{0.812073in}}%
\pgfpathlineto{\pgfqpoint{2.573935in}{0.748577in}}%
\pgfpathlineto{\pgfqpoint{2.579225in}{0.776716in}}%
\pgfpathlineto{\pgfqpoint{2.584516in}{0.901198in}}%
\pgfpathlineto{\pgfqpoint{2.595098in}{0.770542in}}%
\pgfpathlineto{\pgfqpoint{2.600388in}{0.809066in}}%
\pgfpathlineto{\pgfqpoint{2.610970in}{0.858686in}}%
\pgfpathlineto{\pgfqpoint{2.616260in}{0.830437in}}%
\pgfpathlineto{\pgfqpoint{2.621551in}{0.845853in}}%
\pgfpathlineto{\pgfqpoint{2.626842in}{0.855964in}}%
\pgfpathlineto{\pgfqpoint{2.632132in}{0.784388in}}%
\pgfpathlineto{\pgfqpoint{2.637423in}{0.803279in}}%
\pgfpathlineto{\pgfqpoint{2.642714in}{0.781364in}}%
\pgfpathlineto{\pgfqpoint{2.648005in}{0.786714in}}%
\pgfpathlineto{\pgfqpoint{2.653295in}{0.729010in}}%
\pgfpathlineto{\pgfqpoint{2.669167in}{0.633849in}}%
\pgfpathlineto{\pgfqpoint{2.674458in}{0.757772in}}%
\pgfpathlineto{\pgfqpoint{2.679749in}{0.707802in}}%
\pgfpathlineto{\pgfqpoint{2.685039in}{0.676909in}}%
\pgfpathlineto{\pgfqpoint{2.690330in}{0.718845in}}%
\pgfpathlineto{\pgfqpoint{2.700912in}{0.831526in}}%
\pgfpathlineto{\pgfqpoint{2.706202in}{0.725807in}}%
\pgfpathlineto{\pgfqpoint{2.711493in}{0.803760in}}%
\pgfpathlineto{\pgfqpoint{2.716784in}{0.653346in}}%
\pgfpathlineto{\pgfqpoint{2.722074in}{0.820322in}}%
\pgfpathlineto{\pgfqpoint{2.727365in}{0.833128in}}%
\pgfpathlineto{\pgfqpoint{2.732656in}{0.597053in}}%
\pgfpathlineto{\pgfqpoint{2.737947in}{0.587898in}}%
\pgfpathlineto{\pgfqpoint{2.743237in}{0.590978in}}%
\pgfpathlineto{\pgfqpoint{2.748528in}{0.550151in}}%
\pgfpathlineto{\pgfqpoint{2.753819in}{0.633156in}}%
\pgfpathlineto{\pgfqpoint{2.759109in}{0.608953in}}%
\pgfpathlineto{\pgfqpoint{2.764400in}{0.571274in}}%
\pgfpathlineto{\pgfqpoint{2.769691in}{0.686243in}}%
\pgfpathlineto{\pgfqpoint{2.774981in}{0.522687in}}%
\pgfpathlineto{\pgfqpoint{2.780272in}{0.564669in}}%
\pgfpathlineto{\pgfqpoint{2.785563in}{0.531003in}}%
\pgfpathlineto{\pgfqpoint{2.790854in}{0.581235in}}%
\pgfpathlineto{\pgfqpoint{2.796144in}{0.533898in}}%
\pgfpathlineto{\pgfqpoint{2.801435in}{0.529464in}}%
\pgfpathlineto{\pgfqpoint{2.806726in}{0.518807in}}%
\pgfpathlineto{\pgfqpoint{2.812016in}{0.581228in}}%
\pgfpathlineto{\pgfqpoint{2.817307in}{0.759965in}}%
\pgfpathlineto{\pgfqpoint{2.822598in}{0.771485in}}%
\pgfpathlineto{\pgfqpoint{2.827889in}{0.875205in}}%
\pgfpathlineto{\pgfqpoint{2.833179in}{0.766825in}}%
\pgfpathlineto{\pgfqpoint{2.838470in}{0.848756in}}%
\pgfpathlineto{\pgfqpoint{2.843761in}{0.695180in}}%
\pgfpathlineto{\pgfqpoint{2.849051in}{0.727115in}}%
\pgfpathlineto{\pgfqpoint{2.854342in}{0.638471in}}%
\pgfpathlineto{\pgfqpoint{2.859633in}{0.671006in}}%
\pgfpathlineto{\pgfqpoint{2.864923in}{0.582097in}}%
\pgfpathlineto{\pgfqpoint{2.870214in}{0.604898in}}%
\pgfpathlineto{\pgfqpoint{2.875505in}{0.645599in}}%
\pgfpathlineto{\pgfqpoint{2.880796in}{0.607156in}}%
\pgfpathlineto{\pgfqpoint{2.886086in}{0.546527in}}%
\pgfpathlineto{\pgfqpoint{2.891377in}{0.563306in}}%
\pgfpathlineto{\pgfqpoint{2.896668in}{0.512690in}}%
\pgfpathlineto{\pgfqpoint{2.901958in}{0.657514in}}%
\pgfpathlineto{\pgfqpoint{2.907249in}{0.738326in}}%
\pgfpathlineto{\pgfqpoint{2.912540in}{0.564719in}}%
\pgfpathlineto{\pgfqpoint{2.917830in}{0.604527in}}%
\pgfpathlineto{\pgfqpoint{2.923121in}{0.536772in}}%
\pgfpathlineto{\pgfqpoint{2.928412in}{0.523964in}}%
\pgfpathlineto{\pgfqpoint{2.933703in}{0.570257in}}%
\pgfpathlineto{\pgfqpoint{2.938993in}{0.749266in}}%
\pgfpathlineto{\pgfqpoint{2.944284in}{0.536677in}}%
\pgfpathlineto{\pgfqpoint{2.954865in}{0.562974in}}%
\pgfpathlineto{\pgfqpoint{2.965447in}{0.679889in}}%
\pgfpathlineto{\pgfqpoint{2.970738in}{0.697060in}}%
\pgfpathlineto{\pgfqpoint{2.976028in}{0.701889in}}%
\pgfpathlineto{\pgfqpoint{2.981319in}{0.594763in}}%
\pgfpathlineto{\pgfqpoint{2.986610in}{0.637047in}}%
\pgfpathlineto{\pgfqpoint{2.991900in}{0.605715in}}%
\pgfpathlineto{\pgfqpoint{2.997191in}{0.543146in}}%
\pgfpathlineto{\pgfqpoint{3.002482in}{0.565112in}}%
\pgfpathlineto{\pgfqpoint{3.007772in}{0.562664in}}%
\pgfpathlineto{\pgfqpoint{3.013063in}{0.670411in}}%
\pgfpathlineto{\pgfqpoint{3.018354in}{0.615482in}}%
\pgfpathlineto{\pgfqpoint{3.023645in}{0.623208in}}%
\pgfpathlineto{\pgfqpoint{3.028935in}{0.601706in}}%
\pgfpathlineto{\pgfqpoint{3.034226in}{0.592273in}}%
\pgfpathlineto{\pgfqpoint{3.039517in}{0.572612in}}%
\pgfpathlineto{\pgfqpoint{3.050098in}{0.629067in}}%
\pgfpathlineto{\pgfqpoint{3.055389in}{0.551842in}}%
\pgfpathlineto{\pgfqpoint{3.060679in}{0.560359in}}%
\pgfpathlineto{\pgfqpoint{3.065970in}{0.544105in}}%
\pgfpathlineto{\pgfqpoint{3.071261in}{0.541257in}}%
\pgfpathlineto{\pgfqpoint{3.076552in}{0.629890in}}%
\pgfpathlineto{\pgfqpoint{3.081842in}{0.536397in}}%
\pgfpathlineto{\pgfqpoint{3.087133in}{0.534281in}}%
\pgfpathlineto{\pgfqpoint{3.092424in}{0.534947in}}%
\pgfpathlineto{\pgfqpoint{3.097714in}{0.555806in}}%
\pgfpathlineto{\pgfqpoint{3.103005in}{0.523005in}}%
\pgfpathlineto{\pgfqpoint{3.108296in}{0.573365in}}%
\pgfpathlineto{\pgfqpoint{3.113587in}{0.532222in}}%
\pgfpathlineto{\pgfqpoint{3.118877in}{0.535958in}}%
\pgfpathlineto{\pgfqpoint{3.124168in}{0.565651in}}%
\pgfpathlineto{\pgfqpoint{3.129459in}{0.579895in}}%
\pgfpathlineto{\pgfqpoint{3.134749in}{0.539416in}}%
\pgfpathlineto{\pgfqpoint{3.140040in}{0.549155in}}%
\pgfpathlineto{\pgfqpoint{3.145331in}{0.598720in}}%
\pgfpathlineto{\pgfqpoint{3.150621in}{0.533977in}}%
\pgfpathlineto{\pgfqpoint{3.155912in}{0.585637in}}%
\pgfpathlineto{\pgfqpoint{3.161203in}{0.535759in}}%
\pgfpathlineto{\pgfqpoint{3.166494in}{0.538530in}}%
\pgfpathlineto{\pgfqpoint{3.171784in}{0.535349in}}%
\pgfpathlineto{\pgfqpoint{3.177075in}{0.523162in}}%
\pgfpathlineto{\pgfqpoint{3.187656in}{0.539996in}}%
\pgfpathlineto{\pgfqpoint{3.192947in}{0.506458in}}%
\pgfpathlineto{\pgfqpoint{3.198238in}{0.591511in}}%
\pgfpathlineto{\pgfqpoint{3.203529in}{0.527742in}}%
\pgfpathlineto{\pgfqpoint{3.208819in}{0.562712in}}%
\pgfpathlineto{\pgfqpoint{3.214110in}{0.561158in}}%
\pgfpathlineto{\pgfqpoint{3.219401in}{0.572946in}}%
\pgfpathlineto{\pgfqpoint{3.224691in}{0.526302in}}%
\pgfpathlineto{\pgfqpoint{3.229982in}{0.625017in}}%
\pgfpathlineto{\pgfqpoint{3.235273in}{0.587870in}}%
\pgfpathlineto{\pgfqpoint{3.240563in}{0.614021in}}%
\pgfpathlineto{\pgfqpoint{3.245854in}{0.580596in}}%
\pgfpathlineto{\pgfqpoint{3.251145in}{0.522353in}}%
\pgfpathlineto{\pgfqpoint{3.256436in}{0.538010in}}%
\pgfpathlineto{\pgfqpoint{3.261726in}{0.535269in}}%
\pgfpathlineto{\pgfqpoint{3.267017in}{0.569817in}}%
\pgfpathlineto{\pgfqpoint{3.272308in}{0.559900in}}%
\pgfpathlineto{\pgfqpoint{3.277598in}{0.539934in}}%
\pgfpathlineto{\pgfqpoint{3.282889in}{0.529298in}}%
\pgfpathlineto{\pgfqpoint{3.288180in}{0.551912in}}%
\pgfpathlineto{\pgfqpoint{3.293470in}{0.514888in}}%
\pgfpathlineto{\pgfqpoint{3.298761in}{0.564691in}}%
\pgfpathlineto{\pgfqpoint{3.304052in}{0.532680in}}%
\pgfpathlineto{\pgfqpoint{3.309343in}{0.564719in}}%
\pgfpathlineto{\pgfqpoint{3.314633in}{0.514665in}}%
\pgfpathlineto{\pgfqpoint{3.319924in}{0.522884in}}%
\pgfpathlineto{\pgfqpoint{3.325215in}{0.503675in}}%
\pgfpathlineto{\pgfqpoint{3.335796in}{0.495705in}}%
\pgfpathlineto{\pgfqpoint{3.341087in}{0.581149in}}%
\pgfpathlineto{\pgfqpoint{3.346378in}{0.532745in}}%
\pgfpathlineto{\pgfqpoint{3.351668in}{0.528957in}}%
\pgfpathlineto{\pgfqpoint{3.356959in}{0.517536in}}%
\pgfpathlineto{\pgfqpoint{3.367540in}{0.567753in}}%
\pgfpathlineto{\pgfqpoint{3.372831in}{0.552634in}}%
\pgfpathlineto{\pgfqpoint{3.378122in}{0.492254in}}%
\pgfpathlineto{\pgfqpoint{3.383412in}{0.500363in}}%
\pgfpathlineto{\pgfqpoint{3.388703in}{0.592364in}}%
\pgfpathlineto{\pgfqpoint{3.393994in}{0.487782in}}%
\pgfpathlineto{\pgfqpoint{3.399285in}{0.503503in}}%
\pgfpathlineto{\pgfqpoint{3.409866in}{0.571300in}}%
\pgfpathlineto{\pgfqpoint{3.415157in}{0.515642in}}%
\pgfpathlineto{\pgfqpoint{3.420447in}{0.545531in}}%
\pgfpathlineto{\pgfqpoint{3.425738in}{0.501092in}}%
\pgfpathlineto{\pgfqpoint{3.431029in}{0.515870in}}%
\pgfpathlineto{\pgfqpoint{3.436319in}{0.609385in}}%
\pgfpathlineto{\pgfqpoint{3.441610in}{0.609733in}}%
\pgfpathlineto{\pgfqpoint{3.446901in}{0.530931in}}%
\pgfpathlineto{\pgfqpoint{3.452192in}{0.516928in}}%
\pgfpathlineto{\pgfqpoint{3.457482in}{0.571387in}}%
\pgfpathlineto{\pgfqpoint{3.462773in}{0.568811in}}%
\pgfpathlineto{\pgfqpoint{3.468064in}{0.538299in}}%
\pgfpathlineto{\pgfqpoint{3.473354in}{0.519266in}}%
\pgfpathlineto{\pgfqpoint{3.478645in}{0.533319in}}%
\pgfpathlineto{\pgfqpoint{3.483936in}{0.527915in}}%
\pgfpathlineto{\pgfqpoint{3.489227in}{0.545521in}}%
\pgfpathlineto{\pgfqpoint{3.494517in}{0.514924in}}%
\pgfpathlineto{\pgfqpoint{3.499808in}{0.508051in}}%
\pgfpathlineto{\pgfqpoint{3.505099in}{0.563876in}}%
\pgfpathlineto{\pgfqpoint{3.510389in}{0.537200in}}%
\pgfpathlineto{\pgfqpoint{3.515680in}{0.487524in}}%
\pgfpathlineto{\pgfqpoint{3.520971in}{0.533548in}}%
\pgfpathlineto{\pgfqpoint{3.526261in}{0.476675in}}%
\pgfpathlineto{\pgfqpoint{3.531552in}{0.488327in}}%
\pgfpathlineto{\pgfqpoint{3.536843in}{0.475351in}}%
\pgfpathlineto{\pgfqpoint{3.542134in}{0.479874in}}%
\pgfpathlineto{\pgfqpoint{3.547424in}{0.477790in}}%
\pgfpathlineto{\pgfqpoint{3.552715in}{0.480984in}}%
\pgfpathlineto{\pgfqpoint{3.558006in}{0.563157in}}%
\pgfpathlineto{\pgfqpoint{3.563296in}{0.548991in}}%
\pgfpathlineto{\pgfqpoint{3.568587in}{0.522518in}}%
\pgfpathlineto{\pgfqpoint{3.573878in}{0.542243in}}%
\pgfpathlineto{\pgfqpoint{3.579169in}{0.555271in}}%
\pgfpathlineto{\pgfqpoint{3.584459in}{0.474644in}}%
\pgfpathlineto{\pgfqpoint{3.589750in}{0.516018in}}%
\pgfpathlineto{\pgfqpoint{3.595041in}{0.667680in}}%
\pgfpathlineto{\pgfqpoint{3.600331in}{0.646658in}}%
\pgfpathlineto{\pgfqpoint{3.605622in}{0.632874in}}%
\pgfpathlineto{\pgfqpoint{3.610913in}{0.517058in}}%
\pgfpathlineto{\pgfqpoint{3.616203in}{0.489803in}}%
\pgfpathlineto{\pgfqpoint{3.621494in}{0.487102in}}%
\pgfpathlineto{\pgfqpoint{3.626785in}{0.554482in}}%
\pgfpathlineto{\pgfqpoint{3.632076in}{0.495243in}}%
\pgfpathlineto{\pgfqpoint{3.637366in}{0.700825in}}%
\pgfpathlineto{\pgfqpoint{3.642657in}{1.374187in}}%
\pgfpathlineto{\pgfqpoint{3.647948in}{0.493749in}}%
\pgfpathlineto{\pgfqpoint{3.653238in}{0.494173in}}%
\pgfpathlineto{\pgfqpoint{3.658529in}{0.491083in}}%
\pgfpathlineto{\pgfqpoint{3.663820in}{0.481551in}}%
\pgfpathlineto{\pgfqpoint{3.674401in}{0.518251in}}%
\pgfpathlineto{\pgfqpoint{3.679692in}{0.544901in}}%
\pgfpathlineto{\pgfqpoint{3.684983in}{0.483084in}}%
\pgfpathlineto{\pgfqpoint{3.690273in}{0.483297in}}%
\pgfpathlineto{\pgfqpoint{3.695564in}{0.566078in}}%
\pgfpathlineto{\pgfqpoint{3.700855in}{0.558700in}}%
\pgfpathlineto{\pgfqpoint{3.706145in}{0.529140in}}%
\pgfpathlineto{\pgfqpoint{3.711436in}{0.526035in}}%
\pgfpathlineto{\pgfqpoint{3.716727in}{0.601321in}}%
\pgfpathlineto{\pgfqpoint{3.722018in}{0.583186in}}%
\pgfpathlineto{\pgfqpoint{3.727308in}{0.605652in}}%
\pgfpathlineto{\pgfqpoint{3.732599in}{0.514316in}}%
\pgfpathlineto{\pgfqpoint{3.737890in}{0.490978in}}%
\pgfpathlineto{\pgfqpoint{3.743180in}{0.590439in}}%
\pgfpathlineto{\pgfqpoint{3.748471in}{0.510600in}}%
\pgfpathlineto{\pgfqpoint{3.753762in}{0.501468in}}%
\pgfpathlineto{\pgfqpoint{3.759052in}{0.484935in}}%
\pgfpathlineto{\pgfqpoint{3.764343in}{0.508809in}}%
\pgfpathlineto{\pgfqpoint{3.769634in}{0.509533in}}%
\pgfpathlineto{\pgfqpoint{3.774925in}{0.530322in}}%
\pgfpathlineto{\pgfqpoint{3.780215in}{0.540404in}}%
\pgfpathlineto{\pgfqpoint{3.785506in}{0.636393in}}%
\pgfpathlineto{\pgfqpoint{3.790797in}{0.604020in}}%
\pgfpathlineto{\pgfqpoint{3.796087in}{0.625013in}}%
\pgfpathlineto{\pgfqpoint{3.801378in}{0.595273in}}%
\pgfpathlineto{\pgfqpoint{3.806669in}{0.616673in}}%
\pgfpathlineto{\pgfqpoint{3.811960in}{0.574676in}}%
\pgfpathlineto{\pgfqpoint{3.817250in}{0.704649in}}%
\pgfpathlineto{\pgfqpoint{3.822541in}{0.611215in}}%
\pgfpathlineto{\pgfqpoint{3.827832in}{0.581256in}}%
\pgfpathlineto{\pgfqpoint{3.833122in}{0.598133in}}%
\pgfpathlineto{\pgfqpoint{3.838413in}{0.621943in}}%
\pgfpathlineto{\pgfqpoint{3.848994in}{0.530392in}}%
\pgfpathlineto{\pgfqpoint{3.854285in}{0.557086in}}%
\pgfpathlineto{\pgfqpoint{3.859576in}{0.561470in}}%
\pgfpathlineto{\pgfqpoint{3.864867in}{0.523898in}}%
\pgfpathlineto{\pgfqpoint{3.870157in}{0.523392in}}%
\pgfpathlineto{\pgfqpoint{3.875448in}{0.526861in}}%
\pgfpathlineto{\pgfqpoint{3.880739in}{0.514053in}}%
\pgfpathlineto{\pgfqpoint{3.886029in}{0.521581in}}%
\pgfpathlineto{\pgfqpoint{3.891320in}{0.580991in}}%
\pgfpathlineto{\pgfqpoint{3.896611in}{0.590818in}}%
\pgfpathlineto{\pgfqpoint{3.901901in}{0.529711in}}%
\pgfpathlineto{\pgfqpoint{3.907192in}{0.527001in}}%
\pgfpathlineto{\pgfqpoint{3.912483in}{0.561789in}}%
\pgfpathlineto{\pgfqpoint{3.917774in}{0.561432in}}%
\pgfpathlineto{\pgfqpoint{3.923064in}{0.600203in}}%
\pgfpathlineto{\pgfqpoint{3.928355in}{0.559662in}}%
\pgfpathlineto{\pgfqpoint{3.933646in}{0.602875in}}%
\pgfpathlineto{\pgfqpoint{3.938936in}{0.605683in}}%
\pgfpathlineto{\pgfqpoint{3.944227in}{0.571770in}}%
\pgfpathlineto{\pgfqpoint{3.949518in}{0.561991in}}%
\pgfpathlineto{\pgfqpoint{3.954809in}{0.535169in}}%
\pgfpathlineto{\pgfqpoint{3.960099in}{0.549184in}}%
\pgfpathlineto{\pgfqpoint{3.965390in}{0.584658in}}%
\pgfpathlineto{\pgfqpoint{3.970681in}{0.578658in}}%
\pgfpathlineto{\pgfqpoint{3.981262in}{0.505681in}}%
\pgfpathlineto{\pgfqpoint{3.986553in}{0.512559in}}%
\pgfpathlineto{\pgfqpoint{3.991843in}{0.527953in}}%
\pgfpathlineto{\pgfqpoint{3.997134in}{0.602622in}}%
\pgfpathlineto{\pgfqpoint{4.002425in}{0.591314in}}%
\pgfpathlineto{\pgfqpoint{4.007716in}{0.574197in}}%
\pgfpathlineto{\pgfqpoint{4.013006in}{0.527623in}}%
\pgfpathlineto{\pgfqpoint{4.018297in}{0.524246in}}%
\pgfpathlineto{\pgfqpoint{4.023588in}{0.589953in}}%
\pgfpathlineto{\pgfqpoint{4.028878in}{0.519977in}}%
\pgfpathlineto{\pgfqpoint{4.034169in}{0.655685in}}%
\pgfpathlineto{\pgfqpoint{4.044750in}{0.725021in}}%
\pgfpathlineto{\pgfqpoint{4.050041in}{0.528334in}}%
\pgfpathlineto{\pgfqpoint{4.060623in}{0.485334in}}%
\pgfpathlineto{\pgfqpoint{4.065913in}{0.522266in}}%
\pgfpathlineto{\pgfqpoint{4.071204in}{0.523139in}}%
\pgfpathlineto{\pgfqpoint{4.076495in}{0.536084in}}%
\pgfpathlineto{\pgfqpoint{4.081785in}{0.518941in}}%
\pgfpathlineto{\pgfqpoint{4.087076in}{0.514663in}}%
\pgfpathlineto{\pgfqpoint{4.092367in}{0.515197in}}%
\pgfpathlineto{\pgfqpoint{4.097658in}{0.528264in}}%
\pgfpathlineto{\pgfqpoint{4.102948in}{0.507613in}}%
\pgfpathlineto{\pgfqpoint{4.108239in}{0.603205in}}%
\pgfpathlineto{\pgfqpoint{4.113530in}{0.524053in}}%
\pgfpathlineto{\pgfqpoint{4.118820in}{0.558510in}}%
\pgfpathlineto{\pgfqpoint{4.124111in}{0.524751in}}%
\pgfpathlineto{\pgfqpoint{4.134692in}{0.550750in}}%
\pgfpathlineto{\pgfqpoint{4.145274in}{0.622149in}}%
\pgfpathlineto{\pgfqpoint{4.150565in}{0.611822in}}%
\pgfpathlineto{\pgfqpoint{4.155855in}{0.551240in}}%
\pgfpathlineto{\pgfqpoint{4.161146in}{0.522687in}}%
\pgfpathlineto{\pgfqpoint{4.166437in}{0.526469in}}%
\pgfpathlineto{\pgfqpoint{4.171727in}{0.535643in}}%
\pgfpathlineto{\pgfqpoint{4.177018in}{0.529885in}}%
\pgfpathlineto{\pgfqpoint{4.182309in}{0.548435in}}%
\pgfpathlineto{\pgfqpoint{4.187600in}{0.560384in}}%
\pgfpathlineto{\pgfqpoint{4.192890in}{0.521151in}}%
\pgfpathlineto{\pgfqpoint{4.198181in}{0.523293in}}%
\pgfpathlineto{\pgfqpoint{4.203472in}{0.571338in}}%
\pgfpathlineto{\pgfqpoint{4.208762in}{0.532776in}}%
\pgfpathlineto{\pgfqpoint{4.214053in}{0.534524in}}%
\pgfpathlineto{\pgfqpoint{4.219344in}{0.531243in}}%
\pgfpathlineto{\pgfqpoint{4.224634in}{0.609510in}}%
\pgfpathlineto{\pgfqpoint{4.229925in}{0.541684in}}%
\pgfpathlineto{\pgfqpoint{4.235216in}{0.545883in}}%
\pgfpathlineto{\pgfqpoint{4.240507in}{0.538030in}}%
\pgfpathlineto{\pgfqpoint{4.245797in}{0.552039in}}%
\pgfpathlineto{\pgfqpoint{4.251088in}{0.537553in}}%
\pgfpathlineto{\pgfqpoint{4.261669in}{0.552645in}}%
\pgfpathlineto{\pgfqpoint{4.266960in}{0.569371in}}%
\pgfpathlineto{\pgfqpoint{4.272251in}{0.513190in}}%
\pgfpathlineto{\pgfqpoint{4.277541in}{0.541716in}}%
\pgfpathlineto{\pgfqpoint{4.288123in}{0.499357in}}%
\pgfpathlineto{\pgfqpoint{4.298704in}{0.512373in}}%
\pgfpathlineto{\pgfqpoint{4.303995in}{0.506767in}}%
\pgfpathlineto{\pgfqpoint{4.309286in}{0.517160in}}%
\pgfpathlineto{\pgfqpoint{4.314576in}{0.505983in}}%
\pgfpathlineto{\pgfqpoint{4.319867in}{0.534104in}}%
\pgfpathlineto{\pgfqpoint{4.325158in}{0.516136in}}%
\pgfpathlineto{\pgfqpoint{4.335739in}{0.514711in}}%
\pgfpathlineto{\pgfqpoint{4.341030in}{0.509176in}}%
\pgfpathlineto{\pgfqpoint{4.346321in}{0.540614in}}%
\pgfpathlineto{\pgfqpoint{4.351611in}{0.528225in}}%
\pgfpathlineto{\pgfqpoint{4.356902in}{0.534982in}}%
\pgfpathlineto{\pgfqpoint{4.362193in}{0.607628in}}%
\pgfpathlineto{\pgfqpoint{4.367483in}{0.587906in}}%
\pgfpathlineto{\pgfqpoint{4.372774in}{0.545140in}}%
\pgfpathlineto{\pgfqpoint{4.378065in}{0.540881in}}%
\pgfpathlineto{\pgfqpoint{4.383356in}{0.509647in}}%
\pgfpathlineto{\pgfqpoint{4.388646in}{0.536734in}}%
\pgfpathlineto{\pgfqpoint{4.393937in}{0.574020in}}%
\pgfpathlineto{\pgfqpoint{4.399228in}{0.579781in}}%
\pgfpathlineto{\pgfqpoint{4.404518in}{0.569267in}}%
\pgfpathlineto{\pgfqpoint{4.409809in}{0.537151in}}%
\pgfpathlineto{\pgfqpoint{4.415100in}{0.533283in}}%
\pgfpathlineto{\pgfqpoint{4.420390in}{0.599039in}}%
\pgfpathlineto{\pgfqpoint{4.425681in}{0.567604in}}%
\pgfpathlineto{\pgfqpoint{4.430972in}{0.522054in}}%
\pgfpathlineto{\pgfqpoint{4.436263in}{0.527829in}}%
\pgfpathlineto{\pgfqpoint{4.441553in}{0.541989in}}%
\pgfpathlineto{\pgfqpoint{4.446844in}{0.617616in}}%
\pgfpathlineto{\pgfqpoint{4.452135in}{0.547154in}}%
\pgfpathlineto{\pgfqpoint{4.457425in}{0.538441in}}%
\pgfpathlineto{\pgfqpoint{4.462716in}{0.572205in}}%
\pgfpathlineto{\pgfqpoint{4.468007in}{0.566413in}}%
\pgfpathlineto{\pgfqpoint{4.473298in}{0.548932in}}%
\pgfpathlineto{\pgfqpoint{4.478588in}{0.547429in}}%
\pgfpathlineto{\pgfqpoint{4.483879in}{0.562468in}}%
\pgfpathlineto{\pgfqpoint{4.489170in}{0.542422in}}%
\pgfpathlineto{\pgfqpoint{4.494460in}{0.541325in}}%
\pgfpathlineto{\pgfqpoint{4.505042in}{0.580589in}}%
\pgfpathlineto{\pgfqpoint{4.515623in}{0.569088in}}%
\pgfpathlineto{\pgfqpoint{4.520914in}{0.600310in}}%
\pgfpathlineto{\pgfqpoint{4.526205in}{0.561432in}}%
\pgfpathlineto{\pgfqpoint{4.531495in}{0.619047in}}%
\pgfpathlineto{\pgfqpoint{4.536786in}{0.621344in}}%
\pgfpathlineto{\pgfqpoint{4.542077in}{0.647006in}}%
\pgfpathlineto{\pgfqpoint{4.547367in}{0.623028in}}%
\pgfpathlineto{\pgfqpoint{4.552658in}{0.608640in}}%
\pgfpathlineto{\pgfqpoint{4.557949in}{0.624851in}}%
\pgfpathlineto{\pgfqpoint{4.563240in}{0.614187in}}%
\pgfpathlineto{\pgfqpoint{4.568530in}{0.599796in}}%
\pgfpathlineto{\pgfqpoint{4.573821in}{0.597056in}}%
\pgfpathlineto{\pgfqpoint{4.579112in}{0.605973in}}%
\pgfpathlineto{\pgfqpoint{4.584402in}{0.597515in}}%
\pgfpathlineto{\pgfqpoint{4.589693in}{0.596517in}}%
\pgfpathlineto{\pgfqpoint{4.594984in}{0.600055in}}%
\pgfpathlineto{\pgfqpoint{4.600274in}{0.587128in}}%
\pgfpathlineto{\pgfqpoint{4.605565in}{0.587920in}}%
\pgfpathlineto{\pgfqpoint{4.610856in}{0.560265in}}%
\pgfpathlineto{\pgfqpoint{4.616147in}{0.562578in}}%
\pgfpathlineto{\pgfqpoint{4.621437in}{0.598922in}}%
\pgfpathlineto{\pgfqpoint{4.626728in}{0.648270in}}%
\pgfpathlineto{\pgfqpoint{4.632019in}{0.615455in}}%
\pgfpathlineto{\pgfqpoint{4.637309in}{0.624928in}}%
\pgfpathlineto{\pgfqpoint{4.642600in}{0.720011in}}%
\pgfpathlineto{\pgfqpoint{4.647891in}{0.713378in}}%
\pgfpathlineto{\pgfqpoint{4.653181in}{0.552331in}}%
\pgfpathlineto{\pgfqpoint{4.658472in}{0.552266in}}%
\pgfpathlineto{\pgfqpoint{4.663763in}{0.557797in}}%
\pgfpathlineto{\pgfqpoint{4.669054in}{0.538300in}}%
\pgfpathlineto{\pgfqpoint{4.674344in}{0.533361in}}%
\pgfpathlineto{\pgfqpoint{4.679635in}{0.599002in}}%
\pgfpathlineto{\pgfqpoint{4.684926in}{0.521384in}}%
\pgfpathlineto{\pgfqpoint{4.690216in}{0.527684in}}%
\pgfpathlineto{\pgfqpoint{4.700798in}{0.572112in}}%
\pgfpathlineto{\pgfqpoint{4.706089in}{0.589911in}}%
\pgfpathlineto{\pgfqpoint{4.711379in}{0.591253in}}%
\pgfpathlineto{\pgfqpoint{4.716670in}{0.533477in}}%
\pgfpathlineto{\pgfqpoint{4.727251in}{0.584658in}}%
\pgfpathlineto{\pgfqpoint{4.732542in}{0.558344in}}%
\pgfpathlineto{\pgfqpoint{4.737833in}{0.579118in}}%
\pgfpathlineto{\pgfqpoint{4.743123in}{0.589503in}}%
\pgfpathlineto{\pgfqpoint{4.748414in}{0.589010in}}%
\pgfpathlineto{\pgfqpoint{4.758996in}{0.658844in}}%
\pgfpathlineto{\pgfqpoint{4.764286in}{0.647537in}}%
\pgfpathlineto{\pgfqpoint{4.769577in}{0.644663in}}%
\pgfpathlineto{\pgfqpoint{4.774868in}{0.676585in}}%
\pgfpathlineto{\pgfqpoint{4.780158in}{0.674936in}}%
\pgfpathlineto{\pgfqpoint{4.785449in}{0.634070in}}%
\pgfpathlineto{\pgfqpoint{4.796031in}{0.673211in}}%
\pgfpathlineto{\pgfqpoint{4.801321in}{0.672801in}}%
\pgfpathlineto{\pgfqpoint{4.806612in}{0.682092in}}%
\pgfpathlineto{\pgfqpoint{4.811903in}{0.710924in}}%
\pgfpathlineto{\pgfqpoint{4.817193in}{0.674162in}}%
\pgfpathlineto{\pgfqpoint{4.822484in}{0.759501in}}%
\pgfpathlineto{\pgfqpoint{4.827775in}{0.664332in}}%
\pgfpathlineto{\pgfqpoint{4.833065in}{0.644581in}}%
\pgfpathlineto{\pgfqpoint{4.838356in}{0.604546in}}%
\pgfpathlineto{\pgfqpoint{4.843647in}{0.664159in}}%
\pgfpathlineto{\pgfqpoint{4.848938in}{0.556732in}}%
\pgfpathlineto{\pgfqpoint{4.854228in}{0.584271in}}%
\pgfpathlineto{\pgfqpoint{4.859519in}{0.549229in}}%
\pgfpathlineto{\pgfqpoint{4.870100in}{0.627151in}}%
\pgfpathlineto{\pgfqpoint{4.875391in}{0.634809in}}%
\pgfpathlineto{\pgfqpoint{4.880682in}{0.555174in}}%
\pgfpathlineto{\pgfqpoint{4.885972in}{0.530346in}}%
\pgfpathlineto{\pgfqpoint{4.885972in}{0.530346in}}%
\pgfusepath{stroke}%
\end{pgfscope}%
\begin{pgfscope}%
\pgfsetrectcap%
\pgfsetmiterjoin%
\pgfsetlinewidth{0.501875pt}%
\definecolor{currentstroke}{rgb}{0.317647,0.317647,0.317647}%
\pgfsetstrokecolor{currentstroke}%
\pgfsetdash{}{0pt}%
\pgfpathmoveto{\pgfqpoint{0.447336in}{0.410797in}}%
\pgfpathlineto{\pgfqpoint{0.447336in}{1.815448in}}%
\pgfusepath{stroke}%
\end{pgfscope}%
\begin{pgfscope}%
\pgfsetrectcap%
\pgfsetmiterjoin%
\pgfsetlinewidth{0.501875pt}%
\definecolor{currentstroke}{rgb}{0.317647,0.317647,0.317647}%
\pgfsetstrokecolor{currentstroke}%
\pgfsetdash{}{0pt}%
\pgfpathmoveto{\pgfqpoint{0.447336in}{0.410797in}}%
\pgfpathlineto{\pgfqpoint{5.097336in}{0.410797in}}%
\pgfusepath{stroke}%
\end{pgfscope}%
\end{pgfpicture}%
\makeatother%
\endgroup%

	\end{center}
	\caption[Monitoring of the training performance.]{Monitoring of the training performance. The performance is monitored by the accuracy (left) and the \acrfull{rmse} (right).}
	\label{circles_acc}
\end{figure}

In the following the network is trained for 2500 iterations. For a better illustration of the training process, a balanced \emph{validation} dataset with $100$ randomly drawn data points per class is chosen to validate the current state of the network after every fifth iteration. The limitation to every fifth step was chosen to reduce the measurement time to a few hours and to cope with the increasing instability of the \gls{fpga} during long measurements.

The same dataset is also used to monitor the performance of the training process by computing the \gls{rmse} and the accuracy. The \gls{rmse} is given by
\begin{equation}
\text{\gls{rmse}} = \sqrt{\frac{\sum_p e^{(o)}(p)^2}{n_\text{points}}},
\end{equation}
with the error $e^{(o)}$ from \cref{circleserror} and the number of data points $n_\text{points}$ in the validation dataset and the accuracy yields
\begin{equation}
\text{Accuracy} = \frac{n_\text{true}}{n_\text{points}},
\end{equation}
with $n_\text{true}$ the number of correctly classified points with respect to decision boundary from \cref{circlesdb}.

%\begin{figure}[htb!]
%	\centering
%	\include{learning_process_s5}
%	\vspace{-0.3in}
%	\caption[Initial state of the deep network.]{Initial state of the deep network. For simplicity only the hidden and output layer are displayed. The arrows connecting the hidden units with output units resemble the synaptic connections, with red being an excitatory synapse and the thickness of the arrow corresponding to the synaptic strength.}
%	%\label{learning_process_s5}
%\end{figure}

\begin{figure}[t!]
	\begin{subfigure}{\textwidth}
		\caption{}
		\vspace{-0.1in}
		\centering
		\include{learning_process_s5}
		\label{learning_process_s5}
		\vspace{-.5in}
	\end{subfigure}
	\begin{subfigure}{\textwidth}
		\caption{}
		\vspace{-0.1in}
		\centering
		\begin{tikzpicture}[scale=1.4] 
\node (hidden0) at (0.0,3.0) {\includegraphics[width=1.6cm]{mfp/learning_process/h_neuron_500_1.png}}; 
\node (hidden1) at (1.0,3.0) {\includegraphics[width=1.6cm]{mfp/learning_process/h_neuron_500_2.png}}; 
\node (hidden2) at (2.0,3.0) {\includegraphics[width=1.6cm]{mfp/learning_process/h_neuron_500_3.png}}; 
\node (hidden3) at (3.0,3.0) {\includegraphics[width=1.6cm]{mfp/learning_process/h_neuron_500_4.png}}; 
\node (hidden4) at (4.0,3.0) {\includegraphics[width=1.6cm]{mfp/learning_process/h_neuron_500_5.png}}; 
\node (hidden5) at (5.0,3.0) {\includegraphics[width=1.6cm]{mfp/learning_process/h_neuron_500_6.png}}; 
\node (hidden6) at (6.0,3.0) {\includegraphics[width=1.6cm]{mfp/learning_process/h_neuron_500_7.png}}; 
\node (hidden7) at (7.0,3.0) {\includegraphics[width=1.6cm]{mfp/learning_process/h_neuron_500_8.png}}; 
\node (hidden8) at (8.0,3.0) {\includegraphics[width=1.6cm]{mfp/learning_process/h_neuron_500_9.png}}; 
\node (hidden9) at (9.0,3.0) {\includegraphics[width=1.6cm]{mfp/learning_process/h_neuron_500_10.png}}; 
\node (hidden10) at (10.0,3.0) {\includegraphics[width=1.6cm]{mfp/learning_process/h_neuron_500_11.png}}; 
\node (out500) at (5.2,0.0) {\includegraphics[width=3.6cm]{mfp/learning_process/output_neuron_500.png}}; 

\draw[-stealth,line width=1.5pt, color=red ] (hidden0.south) -- (out500); 
\draw[-stealth,line width=0.1pt, color=blue ] (hidden1.south) -- (out500); 
\draw[-stealth,line width=0.8pt, color=red ] (hidden2.south) -- (out500); 
\draw[-stealth,line width=0.7pt, color=red ] (hidden3.south) -- (out500); 
\draw[-stealth,line width=1.2pt, color=red ] (hidden4.south) -- (out500); 
\draw[-stealth,line width=2.0pt, color=red ] (hidden5.south) -- (out500); 
\draw[-stealth,line width=2.3pt, color=red ] (hidden6.south) -- (out500); 
\draw[-stealth,line width=0.9pt, color=red ] (hidden7.south) -- (out500); 
\draw[-stealth,line width=0.5pt, color=blue ] (hidden8.south) -- (out500); 
\draw[-stealth,line width=0.2pt, color=red ] (hidden9.south) -- (out500); 
\draw[-stealth,line width=2.2pt, color=red ] (hidden10.south) -- (out500); 
\end{tikzpicture} 
		\label{learning_process_s500}
		\vspace{-.5in}
	\end{subfigure}
	\begin{subfigure}{\textwidth}
		\caption{}
		\vspace{-0.1in}
		\centering
		\begin{tikzpicture}[scale=1.4] 
\node (hidden0) at (0.0,3.0) {\includegraphics[width=1.6cm]{mfp/learning_process/h_neuron_2500_1.png}}; 
\node (hidden1) at (1.0,3.0) {\includegraphics[width=1.6cm]{mfp/learning_process/h_neuron_2500_2.png}}; 
\node (hidden2) at (2.0,3.0) {\includegraphics[width=1.6cm]{mfp/learning_process/h_neuron_2500_3.png}}; 
\node (hidden3) at (3.0,3.0) {\includegraphics[width=1.6cm]{mfp/learning_process/h_neuron_2500_4.png}}; 
\node (hidden4) at (4.0,3.0) {\includegraphics[width=1.6cm]{mfp/learning_process/h_neuron_2500_5.png}}; 
\node (hidden5) at (5.0,3.0) {\includegraphics[width=1.6cm]{mfp/learning_process/h_neuron_2500_6.png}}; 
\node (hidden6) at (6.0,3.0) {\includegraphics[width=1.6cm]{mfp/learning_process/h_neuron_2500_7.png}}; 
\node (hidden7) at (7.0,3.0) {\includegraphics[width=1.6cm]{mfp/learning_process/h_neuron_2500_8.png}}; 
\node (hidden8) at (8.0,3.0) {\includegraphics[width=1.6cm]{mfp/learning_process/h_neuron_2500_9.png}}; 
\node (hidden9) at (9.0,3.0) {\includegraphics[width=1.6cm]{mfp/learning_process/h_neuron_2500_10.png}}; 
\node (hidden10) at (10.0,3.0) {\includegraphics[width=1.6cm]{mfp/learning_process/h_neuron_2500_11.png}}; 
\node (out5) at (5.2,0.0) {\includegraphics[width=3.6cm]{mfp/learning_process/output_neuron_2500.png}}; 

\draw[-stealth,line width=0.4pt, color=red ] (hidden0.south) -- (out5); 
\draw[-stealth,line width=2.0pt, color=blue ] (hidden1.south) -- (out5); 
\draw[-stealth,line width=1.9pt, color=red ] (hidden2.south) -- (out5); 
\draw[-stealth,line width=1.1pt, color=red ] (hidden3.south) -- (out5); 
\draw[-stealth,line width=2.1pt, color=red ] (hidden4.south) -- (out5); 
\draw[-stealth,line width=2.2pt, color=red ] (hidden5.south) -- (out5); 
\draw[-stealth,line width=1.4pt, color=red ] (hidden6.south) -- (out5); 
\draw[-stealth,line width=0.1pt, color=red ] (hidden7.south) -- (out5); 
\draw[-stealth,line width=2.2pt, color=blue ] (hidden8.south) -- (out5); 
\draw[-stealth,line width=0.7pt, color=red ] (hidden9.south) -- (out5); 
\draw[-stealth,line width=0.7pt, color=red ] (hidden10.south) -- (out5); 
\end{tikzpicture} 
		\label{learning_process_s2500}
	\end{subfigure}
	\vspace{-.4in}
	\caption[Evolution of the network during training.]{Evolution of the network during training. For simplicity only the hidden and output layer are displayed. The connecting arrows resemble the excitatory (\emph{red}) and inhibotry (\emph{blue}) synaptic connections and their thickness corresponds to the synaptic strength. \textbf{(\subref{learning_process_s5})} Initial untrained state of the network. \textbf{(\subref{learning_process_s500})} Network state after 500 iterations. With a decision boundary of roughly \SI{60}{\kilo \Hz}, the output unit starts to correctly identify the two classes. \textbf{(\subref{learning_process_s2500})} Final trained state of the network. After 2500 iterations the network has succeed to separate both classes with an accuracy of almost 100 \%.}
\end{figure}

At the beginning of the training process, each hidden unit is initialized with a linear combination of the input schemes from \cref{circlesinputs} which is then transformed into a non-linear output by the activation function. The new representations of the input layer are again linearly combined into a net input for the output unit where another non-linear transformation is applied. The initial state of the network is shown in \cref{learning_process_s5}. At this stage of the training process the output unit cannot classify the task correctly. 

%\begin{figure}
%	\label{learning_process_s500}
%	\begin{tikzpicture}[scale=1.4] 
\node (hidden0) at (0.0,3.0) {\includegraphics[width=1.6cm]{mfp/learning_process/h_neuron_500_1.png}}; 
\node (hidden1) at (1.0,3.0) {\includegraphics[width=1.6cm]{mfp/learning_process/h_neuron_500_2.png}}; 
\node (hidden2) at (2.0,3.0) {\includegraphics[width=1.6cm]{mfp/learning_process/h_neuron_500_3.png}}; 
\node (hidden3) at (3.0,3.0) {\includegraphics[width=1.6cm]{mfp/learning_process/h_neuron_500_4.png}}; 
\node (hidden4) at (4.0,3.0) {\includegraphics[width=1.6cm]{mfp/learning_process/h_neuron_500_5.png}}; 
\node (hidden5) at (5.0,3.0) {\includegraphics[width=1.6cm]{mfp/learning_process/h_neuron_500_6.png}}; 
\node (hidden6) at (6.0,3.0) {\includegraphics[width=1.6cm]{mfp/learning_process/h_neuron_500_7.png}}; 
\node (hidden7) at (7.0,3.0) {\includegraphics[width=1.6cm]{mfp/learning_process/h_neuron_500_8.png}}; 
\node (hidden8) at (8.0,3.0) {\includegraphics[width=1.6cm]{mfp/learning_process/h_neuron_500_9.png}}; 
\node (hidden9) at (9.0,3.0) {\includegraphics[width=1.6cm]{mfp/learning_process/h_neuron_500_10.png}}; 
\node (hidden10) at (10.0,3.0) {\includegraphics[width=1.6cm]{mfp/learning_process/h_neuron_500_11.png}}; 
\node (out500) at (5.2,0.0) {\includegraphics[width=3.6cm]{mfp/learning_process/output_neuron_500.png}}; 

\draw[-stealth,line width=1.5pt, color=red ] (hidden0.south) -- (out500); 
\draw[-stealth,line width=0.1pt, color=blue ] (hidden1.south) -- (out500); 
\draw[-stealth,line width=0.8pt, color=red ] (hidden2.south) -- (out500); 
\draw[-stealth,line width=0.7pt, color=red ] (hidden3.south) -- (out500); 
\draw[-stealth,line width=1.2pt, color=red ] (hidden4.south) -- (out500); 
\draw[-stealth,line width=2.0pt, color=red ] (hidden5.south) -- (out500); 
\draw[-stealth,line width=2.3pt, color=red ] (hidden6.south) -- (out500); 
\draw[-stealth,line width=0.9pt, color=red ] (hidden7.south) -- (out500); 
\draw[-stealth,line width=0.5pt, color=blue ] (hidden8.south) -- (out500); 
\draw[-stealth,line width=0.2pt, color=red ] (hidden9.south) -- (out500); 
\draw[-stealth,line width=2.2pt, color=red ] (hidden10.south) -- (out500); 
\end{tikzpicture} 
%	\caption[Network state after 500 iterations.]{Network state after 500 iterations. With a decision boundary of roughly \SI{60}{\kilo \Hz}, the output unit starts to correctly identify the two classes.}
%\end{figure}

The evolved network after 500 iterations is depicted in \cref{learning_process_s500} and shows first signs of improvement. The \gls{rmse} has already been cut in half and also the accuracy has improved as shown in \cref{circles_acc}. Less useful representation in the hidden layer are further modified by changing the bias or the weights. The beneficial ones are rewarded with a stronger connection to the output layer and further refinement of their bias and input weights.

At 2500 iterations, see \cref{learning_process_s2500}, the network has converged. The accuracy has also stabilized at almost 100 \%. The average over the last 500 iterations yields a mean accuracy of $(98.2 \pm 2.9)\,\%$ and the average over the last 250 iterations improves to  $(99.6 \pm 0.8)\,\%$. The \gls{rmse} is also reduced to a minimum, i.e. the output neuron resembles the target well. The average output error over the last 250 iterations yields $(9.6 \pm 1.4)\,\si{\kilo \Hz}$.


%For further monitoring purposes, the evolution of the weights and biases are tracked for the hidden and output layer (see \cref{network_monitoring}).
%\begin{figure}
%	\centering
%    %% Creator: Matplotlib, PGF backend
%%
%% To include the figure in your LaTeX document, write
%%   \input{<filename>.pgf}
%%
%% Make sure the required packages are loaded in your preamble
%%   \usepackage{pgf}
%%
%% Figures using additional raster images can only be included by \input if
%% they are in the same directory as the main LaTeX file. For loading figures
%% from other directories you can use the `import` package
%%   \usepackage{import}
%% and then include the figures with
%%   \import{<path to file>}{<filename>.pgf}
%%
%% Matplotlib used the following preamble
%%   \usepackage{amsmath} \usepackage{pifont} \usepackage{xcolor} \definecolor{green}{HTML}{467821} \definecolor{red}{HTML}{CF4457} \usepackage[detect-all]{siunitx}
%%   \usepackage{fontspec}
%%
\begingroup%
\makeatletter%
\begin{pgfpicture}%
\pgfpathrectangle{\pgfpointorigin}{\pgfqpoint{6.176722in}{3.548545in}}%
\pgfusepath{use as bounding box, clip}%
\begin{pgfscope}%
\pgfsetbuttcap%
\pgfsetmiterjoin%
\pgfsetlinewidth{0.000000pt}%
\definecolor{currentstroke}{rgb}{0.000000,0.000000,0.000000}%
\pgfsetstrokecolor{currentstroke}%
\pgfsetstrokeopacity{0.000000}%
\pgfsetdash{}{0pt}%
\pgfpathmoveto{\pgfqpoint{0.000000in}{0.000000in}}%
\pgfpathlineto{\pgfqpoint{6.176722in}{0.000000in}}%
\pgfpathlineto{\pgfqpoint{6.176722in}{3.548545in}}%
\pgfpathlineto{\pgfqpoint{0.000000in}{3.548545in}}%
\pgfpathclose%
\pgfusepath{}%
\end{pgfscope}%
\begin{pgfscope}%
\pgfsetbuttcap%
\pgfsetmiterjoin%
\pgfsetlinewidth{0.000000pt}%
\definecolor{currentstroke}{rgb}{0.000000,0.000000,0.000000}%
\pgfsetstrokecolor{currentstroke}%
\pgfsetstrokeopacity{0.000000}%
\pgfsetdash{}{0pt}%
\pgfpathmoveto{\pgfqpoint{0.488751in}{1.946106in}}%
\pgfpathlineto{\pgfqpoint{2.749168in}{1.946106in}}%
\pgfpathlineto{\pgfqpoint{2.749168in}{3.448545in}}%
\pgfpathlineto{\pgfqpoint{0.488751in}{3.448545in}}%
\pgfpathclose%
\pgfusepath{}%
\end{pgfscope}%
\begin{pgfscope}%
\pgfsetbuttcap%
\pgfsetroundjoin%
\definecolor{currentfill}{rgb}{0.317647,0.317647,0.317647}%
\pgfsetfillcolor{currentfill}%
\pgfsetlinewidth{0.501875pt}%
\definecolor{currentstroke}{rgb}{0.317647,0.317647,0.317647}%
\pgfsetstrokecolor{currentstroke}%
\pgfsetdash{}{0pt}%
\pgfsys@defobject{currentmarker}{\pgfqpoint{0.000000in}{-0.020833in}}{\pgfqpoint{0.000000in}{0.000000in}}{%
\pgfpathmoveto{\pgfqpoint{0.000000in}{0.000000in}}%
\pgfpathlineto{\pgfqpoint{0.000000in}{-0.020833in}}%
\pgfusepath{stroke,fill}%
}%
\begin{pgfscope}%
\pgfsys@transformshift{0.591497in}{1.946106in}%
\pgfsys@useobject{currentmarker}{}%
\end{pgfscope}%
\end{pgfscope}%
\begin{pgfscope}%
\pgfsetbuttcap%
\pgfsetroundjoin%
\definecolor{currentfill}{rgb}{0.317647,0.317647,0.317647}%
\pgfsetfillcolor{currentfill}%
\pgfsetlinewidth{0.501875pt}%
\definecolor{currentstroke}{rgb}{0.317647,0.317647,0.317647}%
\pgfsetstrokecolor{currentstroke}%
\pgfsetdash{}{0pt}%
\pgfsys@defobject{currentmarker}{\pgfqpoint{0.000000in}{-0.020833in}}{\pgfqpoint{0.000000in}{0.000000in}}{%
\pgfpathmoveto{\pgfqpoint{0.000000in}{0.000000in}}%
\pgfpathlineto{\pgfqpoint{0.000000in}{-0.020833in}}%
\pgfusepath{stroke,fill}%
}%
\begin{pgfscope}%
\pgfsys@transformshift{1.105871in}{1.946106in}%
\pgfsys@useobject{currentmarker}{}%
\end{pgfscope}%
\end{pgfscope}%
\begin{pgfscope}%
\pgfsetbuttcap%
\pgfsetroundjoin%
\definecolor{currentfill}{rgb}{0.317647,0.317647,0.317647}%
\pgfsetfillcolor{currentfill}%
\pgfsetlinewidth{0.501875pt}%
\definecolor{currentstroke}{rgb}{0.317647,0.317647,0.317647}%
\pgfsetstrokecolor{currentstroke}%
\pgfsetdash{}{0pt}%
\pgfsys@defobject{currentmarker}{\pgfqpoint{0.000000in}{-0.020833in}}{\pgfqpoint{0.000000in}{0.000000in}}{%
\pgfpathmoveto{\pgfqpoint{0.000000in}{0.000000in}}%
\pgfpathlineto{\pgfqpoint{0.000000in}{-0.020833in}}%
\pgfusepath{stroke,fill}%
}%
\begin{pgfscope}%
\pgfsys@transformshift{1.620245in}{1.946106in}%
\pgfsys@useobject{currentmarker}{}%
\end{pgfscope}%
\end{pgfscope}%
\begin{pgfscope}%
\pgfsetbuttcap%
\pgfsetroundjoin%
\definecolor{currentfill}{rgb}{0.317647,0.317647,0.317647}%
\pgfsetfillcolor{currentfill}%
\pgfsetlinewidth{0.501875pt}%
\definecolor{currentstroke}{rgb}{0.317647,0.317647,0.317647}%
\pgfsetstrokecolor{currentstroke}%
\pgfsetdash{}{0pt}%
\pgfsys@defobject{currentmarker}{\pgfqpoint{0.000000in}{-0.020833in}}{\pgfqpoint{0.000000in}{0.000000in}}{%
\pgfpathmoveto{\pgfqpoint{0.000000in}{0.000000in}}%
\pgfpathlineto{\pgfqpoint{0.000000in}{-0.020833in}}%
\pgfusepath{stroke,fill}%
}%
\begin{pgfscope}%
\pgfsys@transformshift{2.134619in}{1.946106in}%
\pgfsys@useobject{currentmarker}{}%
\end{pgfscope}%
\end{pgfscope}%
\begin{pgfscope}%
\pgfsetbuttcap%
\pgfsetroundjoin%
\definecolor{currentfill}{rgb}{0.317647,0.317647,0.317647}%
\pgfsetfillcolor{currentfill}%
\pgfsetlinewidth{0.501875pt}%
\definecolor{currentstroke}{rgb}{0.317647,0.317647,0.317647}%
\pgfsetstrokecolor{currentstroke}%
\pgfsetdash{}{0pt}%
\pgfsys@defobject{currentmarker}{\pgfqpoint{0.000000in}{-0.020833in}}{\pgfqpoint{0.000000in}{0.000000in}}{%
\pgfpathmoveto{\pgfqpoint{0.000000in}{0.000000in}}%
\pgfpathlineto{\pgfqpoint{0.000000in}{-0.020833in}}%
\pgfusepath{stroke,fill}%
}%
\begin{pgfscope}%
\pgfsys@transformshift{2.648993in}{1.946106in}%
\pgfsys@useobject{currentmarker}{}%
\end{pgfscope}%
\end{pgfscope}%
\begin{pgfscope}%
\pgfsetbuttcap%
\pgfsetroundjoin%
\definecolor{currentfill}{rgb}{0.317647,0.317647,0.317647}%
\pgfsetfillcolor{currentfill}%
\pgfsetlinewidth{0.501875pt}%
\definecolor{currentstroke}{rgb}{0.317647,0.317647,0.317647}%
\pgfsetstrokecolor{currentstroke}%
\pgfsetdash{}{0pt}%
\pgfsys@defobject{currentmarker}{\pgfqpoint{-0.020833in}{0.000000in}}{\pgfqpoint{0.000000in}{0.000000in}}{%
\pgfpathmoveto{\pgfqpoint{0.000000in}{0.000000in}}%
\pgfpathlineto{\pgfqpoint{-0.020833in}{0.000000in}}%
\pgfusepath{stroke,fill}%
}%
\begin{pgfscope}%
\pgfsys@transformshift{0.488751in}{2.261240in}%
\pgfsys@useobject{currentmarker}{}%
\end{pgfscope}%
\end{pgfscope}%
\begin{pgfscope}%
\definecolor{textcolor}{rgb}{0.317647,0.317647,0.317647}%
\pgfsetstrokecolor{textcolor}%
\pgfsetfillcolor{textcolor}%
\pgftext[x=0.256497in,y=2.229123in,left,base]{\color{textcolor}\rmfamily\fontsize{6.664000}{7.996800}\selectfont \(\displaystyle -20\)}%
\end{pgfscope}%
\begin{pgfscope}%
\pgfsetbuttcap%
\pgfsetroundjoin%
\definecolor{currentfill}{rgb}{0.317647,0.317647,0.317647}%
\pgfsetfillcolor{currentfill}%
\pgfsetlinewidth{0.501875pt}%
\definecolor{currentstroke}{rgb}{0.317647,0.317647,0.317647}%
\pgfsetstrokecolor{currentstroke}%
\pgfsetdash{}{0pt}%
\pgfsys@defobject{currentmarker}{\pgfqpoint{-0.020833in}{0.000000in}}{\pgfqpoint{0.000000in}{0.000000in}}{%
\pgfpathmoveto{\pgfqpoint{0.000000in}{0.000000in}}%
\pgfpathlineto{\pgfqpoint{-0.020833in}{0.000000in}}%
\pgfusepath{stroke,fill}%
}%
\begin{pgfscope}%
\pgfsys@transformshift{0.488751in}{2.590361in}%
\pgfsys@useobject{currentmarker}{}%
\end{pgfscope}%
\end{pgfscope}%
\begin{pgfscope}%
\definecolor{textcolor}{rgb}{0.317647,0.317647,0.317647}%
\pgfsetstrokecolor{textcolor}%
\pgfsetfillcolor{textcolor}%
\pgftext[x=0.398666in,y=2.558244in,left,base]{\color{textcolor}\rmfamily\fontsize{6.664000}{7.996800}\selectfont \(\displaystyle 0\)}%
\end{pgfscope}%
\begin{pgfscope}%
\pgfsetbuttcap%
\pgfsetroundjoin%
\definecolor{currentfill}{rgb}{0.317647,0.317647,0.317647}%
\pgfsetfillcolor{currentfill}%
\pgfsetlinewidth{0.501875pt}%
\definecolor{currentstroke}{rgb}{0.317647,0.317647,0.317647}%
\pgfsetstrokecolor{currentstroke}%
\pgfsetdash{}{0pt}%
\pgfsys@defobject{currentmarker}{\pgfqpoint{-0.020833in}{0.000000in}}{\pgfqpoint{0.000000in}{0.000000in}}{%
\pgfpathmoveto{\pgfqpoint{0.000000in}{0.000000in}}%
\pgfpathlineto{\pgfqpoint{-0.020833in}{0.000000in}}%
\pgfusepath{stroke,fill}%
}%
\begin{pgfscope}%
\pgfsys@transformshift{0.488751in}{2.919482in}%
\pgfsys@useobject{currentmarker}{}%
\end{pgfscope}%
\end{pgfscope}%
\begin{pgfscope}%
\definecolor{textcolor}{rgb}{0.317647,0.317647,0.317647}%
\pgfsetstrokecolor{textcolor}%
\pgfsetfillcolor{textcolor}%
\pgftext[x=0.343303in,y=2.887366in,left,base]{\color{textcolor}\rmfamily\fontsize{6.664000}{7.996800}\selectfont \(\displaystyle 20\)}%
\end{pgfscope}%
\begin{pgfscope}%
\pgfsetbuttcap%
\pgfsetroundjoin%
\definecolor{currentfill}{rgb}{0.317647,0.317647,0.317647}%
\pgfsetfillcolor{currentfill}%
\pgfsetlinewidth{0.501875pt}%
\definecolor{currentstroke}{rgb}{0.317647,0.317647,0.317647}%
\pgfsetstrokecolor{currentstroke}%
\pgfsetdash{}{0pt}%
\pgfsys@defobject{currentmarker}{\pgfqpoint{-0.020833in}{0.000000in}}{\pgfqpoint{0.000000in}{0.000000in}}{%
\pgfpathmoveto{\pgfqpoint{0.000000in}{0.000000in}}%
\pgfpathlineto{\pgfqpoint{-0.020833in}{0.000000in}}%
\pgfusepath{stroke,fill}%
}%
\begin{pgfscope}%
\pgfsys@transformshift{0.488751in}{3.248604in}%
\pgfsys@useobject{currentmarker}{}%
\end{pgfscope}%
\end{pgfscope}%
\begin{pgfscope}%
\definecolor{textcolor}{rgb}{0.317647,0.317647,0.317647}%
\pgfsetstrokecolor{textcolor}%
\pgfsetfillcolor{textcolor}%
\pgftext[x=0.343303in,y=3.216487in,left,base]{\color{textcolor}\rmfamily\fontsize{6.664000}{7.996800}\selectfont \(\displaystyle 40\)}%
\end{pgfscope}%
\begin{pgfscope}%
\definecolor{textcolor}{rgb}{0.317647,0.317647,0.317647}%
\pgfsetstrokecolor{textcolor}%
\pgfsetfillcolor{textcolor}%
\pgftext[x=0.200942in,y=2.697325in,,bottom,rotate=90.000000]{\color{textcolor}\rmfamily\fontsize{6.664000}{7.996800}\selectfont \(\displaystyle W^{(\mathrm{h})}\)}%
\end{pgfscope}%
\begin{pgfscope}%
\pgfpathrectangle{\pgfqpoint{0.488751in}{1.946106in}}{\pgfqpoint{2.260417in}{1.502439in}}%
\pgfusepath{clip}%
\pgfsetrectcap%
\pgfsetroundjoin%
\pgfsetlinewidth{0.803000pt}%
\definecolor{currentstroke}{rgb}{0.333333,0.333333,0.333333}%
\pgfsetstrokecolor{currentstroke}%
\pgfsetdash{}{0pt}%
\pgfpathmoveto{\pgfqpoint{0.591497in}{2.935938in}}%
\pgfpathlineto{\pgfqpoint{0.594069in}{2.985307in}}%
\pgfpathlineto{\pgfqpoint{0.596641in}{2.985307in}}%
\pgfpathlineto{\pgfqpoint{0.599213in}{2.968851in}}%
\pgfpathlineto{\pgfqpoint{0.604356in}{3.084043in}}%
\pgfpathlineto{\pgfqpoint{0.606928in}{3.051131in}}%
\pgfpathlineto{\pgfqpoint{0.609500in}{3.034675in}}%
\pgfpathlineto{\pgfqpoint{0.612072in}{3.034675in}}%
\pgfpathlineto{\pgfqpoint{0.614644in}{3.100499in}}%
\pgfpathlineto{\pgfqpoint{0.619788in}{3.100499in}}%
\pgfpathlineto{\pgfqpoint{0.622359in}{3.084043in}}%
\pgfpathlineto{\pgfqpoint{0.624931in}{3.149867in}}%
\pgfpathlineto{\pgfqpoint{0.627503in}{3.133411in}}%
\pgfpathlineto{\pgfqpoint{0.630075in}{3.067587in}}%
\pgfpathlineto{\pgfqpoint{0.632647in}{3.100499in}}%
\pgfpathlineto{\pgfqpoint{0.637791in}{3.182779in}}%
\pgfpathlineto{\pgfqpoint{0.640363in}{3.051131in}}%
\pgfpathlineto{\pgfqpoint{0.642934in}{3.067587in}}%
\pgfpathlineto{\pgfqpoint{0.648078in}{2.968851in}}%
\pgfpathlineto{\pgfqpoint{0.650650in}{2.985307in}}%
\pgfpathlineto{\pgfqpoint{0.653222in}{2.952395in}}%
\pgfpathlineto{\pgfqpoint{0.655794in}{2.952395in}}%
\pgfpathlineto{\pgfqpoint{0.663509in}{2.903026in}}%
\pgfpathlineto{\pgfqpoint{0.666081in}{2.919482in}}%
\pgfpathlineto{\pgfqpoint{0.668653in}{2.886570in}}%
\pgfpathlineto{\pgfqpoint{0.671225in}{2.903026in}}%
\pgfpathlineto{\pgfqpoint{0.673797in}{2.886570in}}%
\pgfpathlineto{\pgfqpoint{0.676369in}{2.886570in}}%
\pgfpathlineto{\pgfqpoint{0.678941in}{2.903026in}}%
\pgfpathlineto{\pgfqpoint{0.684084in}{2.903026in}}%
\pgfpathlineto{\pgfqpoint{0.686656in}{2.919482in}}%
\pgfpathlineto{\pgfqpoint{0.689228in}{2.886570in}}%
\pgfpathlineto{\pgfqpoint{0.702087in}{2.886570in}}%
\pgfpathlineto{\pgfqpoint{0.704659in}{2.903026in}}%
\pgfpathlineto{\pgfqpoint{0.707231in}{2.903026in}}%
\pgfpathlineto{\pgfqpoint{0.712375in}{2.870114in}}%
\pgfpathlineto{\pgfqpoint{0.714947in}{2.886570in}}%
\pgfpathlineto{\pgfqpoint{0.725234in}{2.886570in}}%
\pgfpathlineto{\pgfqpoint{0.727806in}{2.935938in}}%
\pgfpathlineto{\pgfqpoint{0.730378in}{2.952395in}}%
\pgfpathlineto{\pgfqpoint{0.735522in}{2.952395in}}%
\pgfpathlineto{\pgfqpoint{0.738094in}{2.935938in}}%
\pgfpathlineto{\pgfqpoint{0.740666in}{2.903026in}}%
\pgfpathlineto{\pgfqpoint{0.743237in}{2.903026in}}%
\pgfpathlineto{\pgfqpoint{0.745809in}{2.935938in}}%
\pgfpathlineto{\pgfqpoint{0.748381in}{2.935938in}}%
\pgfpathlineto{\pgfqpoint{0.750953in}{2.952395in}}%
\pgfpathlineto{\pgfqpoint{0.753525in}{2.985307in}}%
\pgfpathlineto{\pgfqpoint{0.756097in}{2.968851in}}%
\pgfpathlineto{\pgfqpoint{0.774100in}{2.968851in}}%
\pgfpathlineto{\pgfqpoint{0.779244in}{3.001763in}}%
\pgfpathlineto{\pgfqpoint{0.781815in}{2.968851in}}%
\pgfpathlineto{\pgfqpoint{0.784387in}{3.001763in}}%
\pgfpathlineto{\pgfqpoint{0.786959in}{3.018219in}}%
\pgfpathlineto{\pgfqpoint{0.789531in}{3.018219in}}%
\pgfpathlineto{\pgfqpoint{0.792103in}{3.001763in}}%
\pgfpathlineto{\pgfqpoint{0.794675in}{3.001763in}}%
\pgfpathlineto{\pgfqpoint{0.797247in}{3.018219in}}%
\pgfpathlineto{\pgfqpoint{0.799819in}{3.018219in}}%
\pgfpathlineto{\pgfqpoint{0.804962in}{2.985307in}}%
\pgfpathlineto{\pgfqpoint{0.807534in}{2.985307in}}%
\pgfpathlineto{\pgfqpoint{0.810106in}{2.968851in}}%
\pgfpathlineto{\pgfqpoint{0.833253in}{2.968851in}}%
\pgfpathlineto{\pgfqpoint{0.835825in}{2.952395in}}%
\pgfpathlineto{\pgfqpoint{0.840968in}{2.952395in}}%
\pgfpathlineto{\pgfqpoint{0.843540in}{2.935938in}}%
\pgfpathlineto{\pgfqpoint{0.846112in}{2.952395in}}%
\pgfpathlineto{\pgfqpoint{0.848684in}{2.935938in}}%
\pgfpathlineto{\pgfqpoint{0.889834in}{2.935938in}}%
\pgfpathlineto{\pgfqpoint{0.892406in}{2.952395in}}%
\pgfpathlineto{\pgfqpoint{0.897550in}{2.952395in}}%
\pgfpathlineto{\pgfqpoint{0.900121in}{2.985307in}}%
\pgfpathlineto{\pgfqpoint{0.902693in}{2.985307in}}%
\pgfpathlineto{\pgfqpoint{0.905265in}{2.968851in}}%
\pgfpathlineto{\pgfqpoint{0.915553in}{2.968851in}}%
\pgfpathlineto{\pgfqpoint{0.918125in}{2.985307in}}%
\pgfpathlineto{\pgfqpoint{0.920696in}{2.985307in}}%
\pgfpathlineto{\pgfqpoint{0.923268in}{2.968851in}}%
\pgfpathlineto{\pgfqpoint{0.925840in}{2.919482in}}%
\pgfpathlineto{\pgfqpoint{0.928412in}{2.903026in}}%
\pgfpathlineto{\pgfqpoint{0.951559in}{2.903026in}}%
\pgfpathlineto{\pgfqpoint{0.954131in}{2.919482in}}%
\pgfpathlineto{\pgfqpoint{0.959274in}{2.919482in}}%
\pgfpathlineto{\pgfqpoint{0.961846in}{2.952395in}}%
\pgfpathlineto{\pgfqpoint{0.964418in}{2.919482in}}%
\pgfpathlineto{\pgfqpoint{0.969562in}{2.919482in}}%
\pgfpathlineto{\pgfqpoint{0.972134in}{2.935938in}}%
\pgfpathlineto{\pgfqpoint{0.982421in}{2.935938in}}%
\pgfpathlineto{\pgfqpoint{0.984993in}{2.952395in}}%
\pgfpathlineto{\pgfqpoint{0.987565in}{2.952395in}}%
\pgfpathlineto{\pgfqpoint{0.990137in}{2.985307in}}%
\pgfpathlineto{\pgfqpoint{1.000424in}{2.985307in}}%
\pgfpathlineto{\pgfqpoint{1.002996in}{2.952395in}}%
\pgfpathlineto{\pgfqpoint{1.005568in}{2.952395in}}%
\pgfpathlineto{\pgfqpoint{1.008140in}{2.935938in}}%
\pgfpathlineto{\pgfqpoint{1.010712in}{2.952395in}}%
\pgfpathlineto{\pgfqpoint{1.013284in}{2.952395in}}%
\pgfpathlineto{\pgfqpoint{1.015856in}{2.968851in}}%
\pgfpathlineto{\pgfqpoint{1.036431in}{2.968851in}}%
\pgfpathlineto{\pgfqpoint{1.039002in}{2.935938in}}%
\pgfpathlineto{\pgfqpoint{1.041574in}{2.952395in}}%
\pgfpathlineto{\pgfqpoint{1.044146in}{2.952395in}}%
\pgfpathlineto{\pgfqpoint{1.046718in}{2.919482in}}%
\pgfpathlineto{\pgfqpoint{1.059577in}{2.919482in}}%
\pgfpathlineto{\pgfqpoint{1.062149in}{2.935938in}}%
\pgfpathlineto{\pgfqpoint{1.085296in}{2.935938in}}%
\pgfpathlineto{\pgfqpoint{1.093012in}{2.985307in}}%
\pgfpathlineto{\pgfqpoint{1.108443in}{2.985307in}}%
\pgfpathlineto{\pgfqpoint{1.111015in}{2.968851in}}%
\pgfpathlineto{\pgfqpoint{1.118730in}{2.968851in}}%
\pgfpathlineto{\pgfqpoint{1.121302in}{3.001763in}}%
\pgfpathlineto{\pgfqpoint{1.136734in}{3.001763in}}%
\pgfpathlineto{\pgfqpoint{1.139305in}{2.968851in}}%
\pgfpathlineto{\pgfqpoint{1.147021in}{2.968851in}}%
\pgfpathlineto{\pgfqpoint{1.152165in}{3.034675in}}%
\pgfpathlineto{\pgfqpoint{1.154737in}{3.018219in}}%
\pgfpathlineto{\pgfqpoint{1.157308in}{3.018219in}}%
\pgfpathlineto{\pgfqpoint{1.159880in}{3.001763in}}%
\pgfpathlineto{\pgfqpoint{1.165024in}{3.001763in}}%
\pgfpathlineto{\pgfqpoint{1.167596in}{2.985307in}}%
\pgfpathlineto{\pgfqpoint{1.170168in}{3.001763in}}%
\pgfpathlineto{\pgfqpoint{1.172740in}{3.001763in}}%
\pgfpathlineto{\pgfqpoint{1.177883in}{2.968851in}}%
\pgfpathlineto{\pgfqpoint{1.180455in}{2.935938in}}%
\pgfpathlineto{\pgfqpoint{1.198458in}{2.935938in}}%
\pgfpathlineto{\pgfqpoint{1.201030in}{2.952395in}}%
\pgfpathlineto{\pgfqpoint{1.208746in}{2.952395in}}%
\pgfpathlineto{\pgfqpoint{1.211318in}{2.985307in}}%
\pgfpathlineto{\pgfqpoint{1.219033in}{2.985307in}}%
\pgfpathlineto{\pgfqpoint{1.221605in}{2.952395in}}%
\pgfpathlineto{\pgfqpoint{1.224177in}{2.952395in}}%
\pgfpathlineto{\pgfqpoint{1.226749in}{2.935938in}}%
\pgfpathlineto{\pgfqpoint{1.229321in}{2.935938in}}%
\pgfpathlineto{\pgfqpoint{1.231893in}{2.919482in}}%
\pgfpathlineto{\pgfqpoint{1.234465in}{2.935938in}}%
\pgfpathlineto{\pgfqpoint{1.237036in}{2.968851in}}%
\pgfpathlineto{\pgfqpoint{1.249896in}{2.968851in}}%
\pgfpathlineto{\pgfqpoint{1.252468in}{2.985307in}}%
\pgfpathlineto{\pgfqpoint{1.257611in}{2.985307in}}%
\pgfpathlineto{\pgfqpoint{1.260183in}{2.952395in}}%
\pgfpathlineto{\pgfqpoint{1.267899in}{2.952395in}}%
\pgfpathlineto{\pgfqpoint{1.270471in}{2.985307in}}%
\pgfpathlineto{\pgfqpoint{1.273043in}{3.001763in}}%
\pgfpathlineto{\pgfqpoint{1.285902in}{3.001763in}}%
\pgfpathlineto{\pgfqpoint{1.288474in}{2.985307in}}%
\pgfpathlineto{\pgfqpoint{1.291046in}{2.952395in}}%
\pgfpathlineto{\pgfqpoint{1.293618in}{2.935938in}}%
\pgfpathlineto{\pgfqpoint{1.298761in}{2.935938in}}%
\pgfpathlineto{\pgfqpoint{1.303905in}{3.001763in}}%
\pgfpathlineto{\pgfqpoint{1.319336in}{3.001763in}}%
\pgfpathlineto{\pgfqpoint{1.321908in}{3.034675in}}%
\pgfpathlineto{\pgfqpoint{1.329624in}{3.034675in}}%
\pgfpathlineto{\pgfqpoint{1.332196in}{3.051131in}}%
\pgfpathlineto{\pgfqpoint{1.334768in}{3.051131in}}%
\pgfpathlineto{\pgfqpoint{1.337339in}{3.067587in}}%
\pgfpathlineto{\pgfqpoint{1.342483in}{3.067587in}}%
\pgfpathlineto{\pgfqpoint{1.345055in}{3.051131in}}%
\pgfpathlineto{\pgfqpoint{1.347627in}{3.067587in}}%
\pgfpathlineto{\pgfqpoint{1.355342in}{3.067587in}}%
\pgfpathlineto{\pgfqpoint{1.357914in}{3.034675in}}%
\pgfpathlineto{\pgfqpoint{1.360486in}{3.051131in}}%
\pgfpathlineto{\pgfqpoint{1.363058in}{3.034675in}}%
\pgfpathlineto{\pgfqpoint{1.375917in}{3.034675in}}%
\pgfpathlineto{\pgfqpoint{1.381061in}{3.001763in}}%
\pgfpathlineto{\pgfqpoint{1.406780in}{3.001763in}}%
\pgfpathlineto{\pgfqpoint{1.409352in}{2.968851in}}%
\pgfpathlineto{\pgfqpoint{1.419639in}{2.968851in}}%
\pgfpathlineto{\pgfqpoint{1.422211in}{2.952395in}}%
\pgfpathlineto{\pgfqpoint{1.432499in}{2.952395in}}%
\pgfpathlineto{\pgfqpoint{1.435070in}{2.935938in}}%
\pgfpathlineto{\pgfqpoint{1.437642in}{2.952395in}}%
\pgfpathlineto{\pgfqpoint{1.453074in}{2.952395in}}%
\pgfpathlineto{\pgfqpoint{1.455645in}{2.968851in}}%
\pgfpathlineto{\pgfqpoint{1.489080in}{2.968851in}}%
\pgfpathlineto{\pgfqpoint{1.491652in}{2.985307in}}%
\pgfpathlineto{\pgfqpoint{1.494223in}{2.968851in}}%
\pgfpathlineto{\pgfqpoint{1.496795in}{2.968851in}}%
\pgfpathlineto{\pgfqpoint{1.499367in}{2.935938in}}%
\pgfpathlineto{\pgfqpoint{1.509655in}{2.935938in}}%
\pgfpathlineto{\pgfqpoint{1.512227in}{2.903026in}}%
\pgfpathlineto{\pgfqpoint{1.514798in}{2.903026in}}%
\pgfpathlineto{\pgfqpoint{1.517370in}{2.886570in}}%
\pgfpathlineto{\pgfqpoint{1.576523in}{2.886570in}}%
\pgfpathlineto{\pgfqpoint{1.579095in}{2.903026in}}%
\pgfpathlineto{\pgfqpoint{1.584239in}{2.903026in}}%
\pgfpathlineto{\pgfqpoint{1.586811in}{2.952395in}}%
\pgfpathlineto{\pgfqpoint{1.589383in}{2.952395in}}%
\pgfpathlineto{\pgfqpoint{1.591955in}{2.985307in}}%
\pgfpathlineto{\pgfqpoint{1.594526in}{2.968851in}}%
\pgfpathlineto{\pgfqpoint{1.597098in}{3.018219in}}%
\pgfpathlineto{\pgfqpoint{1.599670in}{3.018219in}}%
\pgfpathlineto{\pgfqpoint{1.602242in}{3.051131in}}%
\pgfpathlineto{\pgfqpoint{1.607386in}{3.084043in}}%
\pgfpathlineto{\pgfqpoint{1.609958in}{3.067587in}}%
\pgfpathlineto{\pgfqpoint{1.612529in}{3.100499in}}%
\pgfpathlineto{\pgfqpoint{1.615101in}{3.084043in}}%
\pgfpathlineto{\pgfqpoint{1.617673in}{3.100499in}}%
\pgfpathlineto{\pgfqpoint{1.622817in}{3.067587in}}%
\pgfpathlineto{\pgfqpoint{1.633104in}{3.067587in}}%
\pgfpathlineto{\pgfqpoint{1.638248in}{3.100499in}}%
\pgfpathlineto{\pgfqpoint{1.640820in}{3.133411in}}%
\pgfpathlineto{\pgfqpoint{1.643392in}{3.149867in}}%
\pgfpathlineto{\pgfqpoint{1.645964in}{3.116955in}}%
\pgfpathlineto{\pgfqpoint{1.648536in}{3.067587in}}%
\pgfpathlineto{\pgfqpoint{1.651108in}{3.067587in}}%
\pgfpathlineto{\pgfqpoint{1.653679in}{3.100499in}}%
\pgfpathlineto{\pgfqpoint{1.656251in}{3.100499in}}%
\pgfpathlineto{\pgfqpoint{1.658823in}{3.133411in}}%
\pgfpathlineto{\pgfqpoint{1.661395in}{3.084043in}}%
\pgfpathlineto{\pgfqpoint{1.663967in}{3.100499in}}%
\pgfpathlineto{\pgfqpoint{1.674254in}{3.100499in}}%
\pgfpathlineto{\pgfqpoint{1.676826in}{3.133411in}}%
\pgfpathlineto{\pgfqpoint{1.679398in}{3.100499in}}%
\pgfpathlineto{\pgfqpoint{1.681970in}{3.100499in}}%
\pgfpathlineto{\pgfqpoint{1.687114in}{3.133411in}}%
\pgfpathlineto{\pgfqpoint{1.689686in}{3.100499in}}%
\pgfpathlineto{\pgfqpoint{1.697401in}{3.100499in}}%
\pgfpathlineto{\pgfqpoint{1.699973in}{3.116955in}}%
\pgfpathlineto{\pgfqpoint{1.702545in}{3.100499in}}%
\pgfpathlineto{\pgfqpoint{1.705117in}{3.149867in}}%
\pgfpathlineto{\pgfqpoint{1.707689in}{3.133411in}}%
\pgfpathlineto{\pgfqpoint{1.710261in}{3.133411in}}%
\pgfpathlineto{\pgfqpoint{1.712832in}{3.100499in}}%
\pgfpathlineto{\pgfqpoint{1.715404in}{3.100499in}}%
\pgfpathlineto{\pgfqpoint{1.717976in}{3.116955in}}%
\pgfpathlineto{\pgfqpoint{1.720548in}{3.149867in}}%
\pgfpathlineto{\pgfqpoint{1.723120in}{3.149867in}}%
\pgfpathlineto{\pgfqpoint{1.725692in}{3.182779in}}%
\pgfpathlineto{\pgfqpoint{1.728264in}{3.149867in}}%
\pgfpathlineto{\pgfqpoint{1.730836in}{3.100499in}}%
\pgfpathlineto{\pgfqpoint{1.733407in}{3.100499in}}%
\pgfpathlineto{\pgfqpoint{1.735979in}{3.084043in}}%
\pgfpathlineto{\pgfqpoint{1.738551in}{3.100499in}}%
\pgfpathlineto{\pgfqpoint{1.743695in}{3.100499in}}%
\pgfpathlineto{\pgfqpoint{1.746267in}{3.067587in}}%
\pgfpathlineto{\pgfqpoint{1.751410in}{3.067587in}}%
\pgfpathlineto{\pgfqpoint{1.753982in}{3.084043in}}%
\pgfpathlineto{\pgfqpoint{1.756554in}{3.067587in}}%
\pgfpathlineto{\pgfqpoint{1.759126in}{3.084043in}}%
\pgfpathlineto{\pgfqpoint{1.764270in}{3.051131in}}%
\pgfpathlineto{\pgfqpoint{1.766842in}{3.084043in}}%
\pgfpathlineto{\pgfqpoint{1.769414in}{3.051131in}}%
\pgfpathlineto{\pgfqpoint{1.771985in}{3.034675in}}%
\pgfpathlineto{\pgfqpoint{1.774557in}{3.034675in}}%
\pgfpathlineto{\pgfqpoint{1.777129in}{3.001763in}}%
\pgfpathlineto{\pgfqpoint{1.779701in}{3.034675in}}%
\pgfpathlineto{\pgfqpoint{1.782273in}{3.018219in}}%
\pgfpathlineto{\pgfqpoint{1.784845in}{3.034675in}}%
\pgfpathlineto{\pgfqpoint{1.787417in}{3.034675in}}%
\pgfpathlineto{\pgfqpoint{1.789989in}{3.018219in}}%
\pgfpathlineto{\pgfqpoint{1.792560in}{3.018219in}}%
\pgfpathlineto{\pgfqpoint{1.797704in}{3.051131in}}%
\pgfpathlineto{\pgfqpoint{1.800276in}{3.018219in}}%
\pgfpathlineto{\pgfqpoint{1.802848in}{3.018219in}}%
\pgfpathlineto{\pgfqpoint{1.805420in}{3.034675in}}%
\pgfpathlineto{\pgfqpoint{1.807992in}{3.018219in}}%
\pgfpathlineto{\pgfqpoint{1.810563in}{3.018219in}}%
\pgfpathlineto{\pgfqpoint{1.813135in}{3.034675in}}%
\pgfpathlineto{\pgfqpoint{1.815707in}{3.018219in}}%
\pgfpathlineto{\pgfqpoint{1.823423in}{3.018219in}}%
\pgfpathlineto{\pgfqpoint{1.825995in}{3.001763in}}%
\pgfpathlineto{\pgfqpoint{1.838854in}{3.001763in}}%
\pgfpathlineto{\pgfqpoint{1.841426in}{2.985307in}}%
\pgfpathlineto{\pgfqpoint{1.849142in}{2.985307in}}%
\pgfpathlineto{\pgfqpoint{1.854285in}{3.018219in}}%
\pgfpathlineto{\pgfqpoint{1.856857in}{3.018219in}}%
\pgfpathlineto{\pgfqpoint{1.859429in}{3.001763in}}%
\pgfpathlineto{\pgfqpoint{1.867145in}{3.001763in}}%
\pgfpathlineto{\pgfqpoint{1.869717in}{2.985307in}}%
\pgfpathlineto{\pgfqpoint{1.872288in}{3.001763in}}%
\pgfpathlineto{\pgfqpoint{1.880004in}{3.001763in}}%
\pgfpathlineto{\pgfqpoint{1.882576in}{3.018219in}}%
\pgfpathlineto{\pgfqpoint{1.885148in}{3.018219in}}%
\pgfpathlineto{\pgfqpoint{1.887720in}{3.034675in}}%
\pgfpathlineto{\pgfqpoint{1.892863in}{3.034675in}}%
\pgfpathlineto{\pgfqpoint{1.895435in}{3.018219in}}%
\pgfpathlineto{\pgfqpoint{1.903151in}{3.018219in}}%
\pgfpathlineto{\pgfqpoint{1.905723in}{3.034675in}}%
\pgfpathlineto{\pgfqpoint{1.908295in}{3.034675in}}%
\pgfpathlineto{\pgfqpoint{1.910866in}{3.018219in}}%
\pgfpathlineto{\pgfqpoint{1.913438in}{3.018219in}}%
\pgfpathlineto{\pgfqpoint{1.916010in}{3.034675in}}%
\pgfpathlineto{\pgfqpoint{1.921154in}{3.034675in}}%
\pgfpathlineto{\pgfqpoint{1.926298in}{3.067587in}}%
\pgfpathlineto{\pgfqpoint{1.928870in}{3.067587in}}%
\pgfpathlineto{\pgfqpoint{1.931441in}{3.034675in}}%
\pgfpathlineto{\pgfqpoint{1.934013in}{3.034675in}}%
\pgfpathlineto{\pgfqpoint{1.936585in}{3.051131in}}%
\pgfpathlineto{\pgfqpoint{1.941729in}{3.051131in}}%
\pgfpathlineto{\pgfqpoint{1.944301in}{3.034675in}}%
\pgfpathlineto{\pgfqpoint{1.946873in}{3.051131in}}%
\pgfpathlineto{\pgfqpoint{1.957160in}{3.051131in}}%
\pgfpathlineto{\pgfqpoint{1.959732in}{3.034675in}}%
\pgfpathlineto{\pgfqpoint{1.967448in}{3.034675in}}%
\pgfpathlineto{\pgfqpoint{1.970019in}{3.051131in}}%
\pgfpathlineto{\pgfqpoint{1.972591in}{3.051131in}}%
\pgfpathlineto{\pgfqpoint{1.975163in}{3.034675in}}%
\pgfpathlineto{\pgfqpoint{1.977735in}{3.051131in}}%
\pgfpathlineto{\pgfqpoint{1.980307in}{3.034675in}}%
\pgfpathlineto{\pgfqpoint{1.982879in}{3.034675in}}%
\pgfpathlineto{\pgfqpoint{1.985451in}{3.018219in}}%
\pgfpathlineto{\pgfqpoint{1.998310in}{3.018219in}}%
\pgfpathlineto{\pgfqpoint{2.003454in}{2.985307in}}%
\pgfpathlineto{\pgfqpoint{2.011169in}{2.985307in}}%
\pgfpathlineto{\pgfqpoint{2.013741in}{3.001763in}}%
\pgfpathlineto{\pgfqpoint{2.018885in}{2.968851in}}%
\pgfpathlineto{\pgfqpoint{2.024029in}{2.968851in}}%
\pgfpathlineto{\pgfqpoint{2.026601in}{2.985307in}}%
\pgfpathlineto{\pgfqpoint{2.031744in}{2.985307in}}%
\pgfpathlineto{\pgfqpoint{2.034316in}{3.001763in}}%
\pgfpathlineto{\pgfqpoint{2.036888in}{3.001763in}}%
\pgfpathlineto{\pgfqpoint{2.039460in}{2.985307in}}%
\pgfpathlineto{\pgfqpoint{2.042032in}{2.985307in}}%
\pgfpathlineto{\pgfqpoint{2.044604in}{3.001763in}}%
\pgfpathlineto{\pgfqpoint{2.057463in}{3.001763in}}%
\pgfpathlineto{\pgfqpoint{2.060035in}{3.018219in}}%
\pgfpathlineto{\pgfqpoint{2.065179in}{3.018219in}}%
\pgfpathlineto{\pgfqpoint{2.067751in}{3.001763in}}%
\pgfpathlineto{\pgfqpoint{2.070322in}{3.001763in}}%
\pgfpathlineto{\pgfqpoint{2.075466in}{3.034675in}}%
\pgfpathlineto{\pgfqpoint{2.078038in}{3.001763in}}%
\pgfpathlineto{\pgfqpoint{2.083182in}{3.001763in}}%
\pgfpathlineto{\pgfqpoint{2.085754in}{3.018219in}}%
\pgfpathlineto{\pgfqpoint{2.088325in}{3.018219in}}%
\pgfpathlineto{\pgfqpoint{2.090897in}{3.001763in}}%
\pgfpathlineto{\pgfqpoint{2.103757in}{3.001763in}}%
\pgfpathlineto{\pgfqpoint{2.106329in}{2.985307in}}%
\pgfpathlineto{\pgfqpoint{2.124332in}{2.985307in}}%
\pgfpathlineto{\pgfqpoint{2.126904in}{2.968851in}}%
\pgfpathlineto{\pgfqpoint{2.129475in}{2.985307in}}%
\pgfpathlineto{\pgfqpoint{2.139763in}{2.985307in}}%
\pgfpathlineto{\pgfqpoint{2.142335in}{3.001763in}}%
\pgfpathlineto{\pgfqpoint{2.147478in}{3.001763in}}%
\pgfpathlineto{\pgfqpoint{2.150050in}{3.018219in}}%
\pgfpathlineto{\pgfqpoint{2.160338in}{3.018219in}}%
\pgfpathlineto{\pgfqpoint{2.162910in}{3.001763in}}%
\pgfpathlineto{\pgfqpoint{2.165482in}{3.001763in}}%
\pgfpathlineto{\pgfqpoint{2.168053in}{3.018219in}}%
\pgfpathlineto{\pgfqpoint{2.178341in}{3.018219in}}%
\pgfpathlineto{\pgfqpoint{2.180913in}{3.034675in}}%
\pgfpathlineto{\pgfqpoint{2.191200in}{3.034675in}}%
\pgfpathlineto{\pgfqpoint{2.193772in}{3.051131in}}%
\pgfpathlineto{\pgfqpoint{2.196344in}{3.051131in}}%
\pgfpathlineto{\pgfqpoint{2.198916in}{3.018219in}}%
\pgfpathlineto{\pgfqpoint{2.204060in}{3.018219in}}%
\pgfpathlineto{\pgfqpoint{2.206631in}{3.034675in}}%
\pgfpathlineto{\pgfqpoint{2.211775in}{3.034675in}}%
\pgfpathlineto{\pgfqpoint{2.214347in}{3.018219in}}%
\pgfpathlineto{\pgfqpoint{2.219491in}{3.018219in}}%
\pgfpathlineto{\pgfqpoint{2.222063in}{3.034675in}}%
\pgfpathlineto{\pgfqpoint{2.224635in}{3.034675in}}%
\pgfpathlineto{\pgfqpoint{2.227206in}{3.018219in}}%
\pgfpathlineto{\pgfqpoint{2.229778in}{3.034675in}}%
\pgfpathlineto{\pgfqpoint{2.232350in}{3.067587in}}%
\pgfpathlineto{\pgfqpoint{2.237494in}{3.067587in}}%
\pgfpathlineto{\pgfqpoint{2.242638in}{3.034675in}}%
\pgfpathlineto{\pgfqpoint{2.250353in}{3.034675in}}%
\pgfpathlineto{\pgfqpoint{2.252925in}{3.051131in}}%
\pgfpathlineto{\pgfqpoint{2.265785in}{3.051131in}}%
\pgfpathlineto{\pgfqpoint{2.268356in}{3.084043in}}%
\pgfpathlineto{\pgfqpoint{2.270928in}{3.067587in}}%
\pgfpathlineto{\pgfqpoint{2.273500in}{3.084043in}}%
\pgfpathlineto{\pgfqpoint{2.276072in}{3.084043in}}%
\pgfpathlineto{\pgfqpoint{2.278644in}{3.100499in}}%
\pgfpathlineto{\pgfqpoint{2.283788in}{3.067587in}}%
\pgfpathlineto{\pgfqpoint{2.288931in}{3.067587in}}%
\pgfpathlineto{\pgfqpoint{2.294075in}{3.100499in}}%
\pgfpathlineto{\pgfqpoint{2.304363in}{3.100499in}}%
\pgfpathlineto{\pgfqpoint{2.309506in}{3.133411in}}%
\pgfpathlineto{\pgfqpoint{2.312078in}{3.133411in}}%
\pgfpathlineto{\pgfqpoint{2.314650in}{3.100499in}}%
\pgfpathlineto{\pgfqpoint{2.317222in}{3.100499in}}%
\pgfpathlineto{\pgfqpoint{2.319794in}{3.133411in}}%
\pgfpathlineto{\pgfqpoint{2.322366in}{3.116955in}}%
\pgfpathlineto{\pgfqpoint{2.324938in}{3.084043in}}%
\pgfpathlineto{\pgfqpoint{2.327509in}{3.067587in}}%
\pgfpathlineto{\pgfqpoint{2.330081in}{3.067587in}}%
\pgfpathlineto{\pgfqpoint{2.332653in}{3.051131in}}%
\pgfpathlineto{\pgfqpoint{2.345512in}{3.051131in}}%
\pgfpathlineto{\pgfqpoint{2.348084in}{3.034675in}}%
\pgfpathlineto{\pgfqpoint{2.355800in}{3.034675in}}%
\pgfpathlineto{\pgfqpoint{2.358372in}{3.018219in}}%
\pgfpathlineto{\pgfqpoint{2.360944in}{3.034675in}}%
\pgfpathlineto{\pgfqpoint{2.363516in}{3.018219in}}%
\pgfpathlineto{\pgfqpoint{2.368659in}{3.018219in}}%
\pgfpathlineto{\pgfqpoint{2.371231in}{3.001763in}}%
\pgfpathlineto{\pgfqpoint{2.381519in}{3.001763in}}%
\pgfpathlineto{\pgfqpoint{2.384091in}{3.018219in}}%
\pgfpathlineto{\pgfqpoint{2.386662in}{3.018219in}}%
\pgfpathlineto{\pgfqpoint{2.391806in}{3.051131in}}%
\pgfpathlineto{\pgfqpoint{2.394378in}{3.051131in}}%
\pgfpathlineto{\pgfqpoint{2.399522in}{3.018219in}}%
\pgfpathlineto{\pgfqpoint{2.404665in}{3.018219in}}%
\pgfpathlineto{\pgfqpoint{2.407237in}{3.001763in}}%
\pgfpathlineto{\pgfqpoint{2.409809in}{3.018219in}}%
\pgfpathlineto{\pgfqpoint{2.412381in}{3.001763in}}%
\pgfpathlineto{\pgfqpoint{2.414953in}{3.018219in}}%
\pgfpathlineto{\pgfqpoint{2.420097in}{2.985307in}}%
\pgfpathlineto{\pgfqpoint{2.422669in}{3.001763in}}%
\pgfpathlineto{\pgfqpoint{2.425240in}{3.034675in}}%
\pgfpathlineto{\pgfqpoint{2.427812in}{3.051131in}}%
\pgfpathlineto{\pgfqpoint{2.430384in}{3.051131in}}%
\pgfpathlineto{\pgfqpoint{2.435528in}{3.084043in}}%
\pgfpathlineto{\pgfqpoint{2.438100in}{3.084043in}}%
\pgfpathlineto{\pgfqpoint{2.443244in}{3.051131in}}%
\pgfpathlineto{\pgfqpoint{2.445815in}{3.051131in}}%
\pgfpathlineto{\pgfqpoint{2.450959in}{3.084043in}}%
\pgfpathlineto{\pgfqpoint{2.456103in}{3.084043in}}%
\pgfpathlineto{\pgfqpoint{2.458675in}{3.100499in}}%
\pgfpathlineto{\pgfqpoint{2.466390in}{3.100499in}}%
\pgfpathlineto{\pgfqpoint{2.468962in}{3.084043in}}%
\pgfpathlineto{\pgfqpoint{2.471534in}{3.116955in}}%
\pgfpathlineto{\pgfqpoint{2.474106in}{3.067587in}}%
\pgfpathlineto{\pgfqpoint{2.476678in}{3.084043in}}%
\pgfpathlineto{\pgfqpoint{2.479250in}{3.067587in}}%
\pgfpathlineto{\pgfqpoint{2.489537in}{3.067587in}}%
\pgfpathlineto{\pgfqpoint{2.492109in}{3.051131in}}%
\pgfpathlineto{\pgfqpoint{2.504968in}{3.051131in}}%
\pgfpathlineto{\pgfqpoint{2.507540in}{3.034675in}}%
\pgfpathlineto{\pgfqpoint{2.515256in}{3.034675in}}%
\pgfpathlineto{\pgfqpoint{2.517828in}{3.067587in}}%
\pgfpathlineto{\pgfqpoint{2.520400in}{3.067587in}}%
\pgfpathlineto{\pgfqpoint{2.522972in}{3.051131in}}%
\pgfpathlineto{\pgfqpoint{2.525543in}{3.067587in}}%
\pgfpathlineto{\pgfqpoint{2.530687in}{3.067587in}}%
\pgfpathlineto{\pgfqpoint{2.535831in}{3.034675in}}%
\pgfpathlineto{\pgfqpoint{2.538403in}{3.034675in}}%
\pgfpathlineto{\pgfqpoint{2.540975in}{3.051131in}}%
\pgfpathlineto{\pgfqpoint{2.546118in}{3.051131in}}%
\pgfpathlineto{\pgfqpoint{2.548690in}{3.034675in}}%
\pgfpathlineto{\pgfqpoint{2.551262in}{3.034675in}}%
\pgfpathlineto{\pgfqpoint{2.553834in}{3.018219in}}%
\pgfpathlineto{\pgfqpoint{2.556406in}{3.018219in}}%
\pgfpathlineto{\pgfqpoint{2.558978in}{3.051131in}}%
\pgfpathlineto{\pgfqpoint{2.564121in}{3.051131in}}%
\pgfpathlineto{\pgfqpoint{2.566693in}{3.067587in}}%
\pgfpathlineto{\pgfqpoint{2.574409in}{3.067587in}}%
\pgfpathlineto{\pgfqpoint{2.576981in}{3.051131in}}%
\pgfpathlineto{\pgfqpoint{2.579553in}{3.051131in}}%
\pgfpathlineto{\pgfqpoint{2.584696in}{3.084043in}}%
\pgfpathlineto{\pgfqpoint{2.589840in}{3.084043in}}%
\pgfpathlineto{\pgfqpoint{2.592412in}{3.116955in}}%
\pgfpathlineto{\pgfqpoint{2.594984in}{3.084043in}}%
\pgfpathlineto{\pgfqpoint{2.605271in}{3.084043in}}%
\pgfpathlineto{\pgfqpoint{2.607843in}{3.133411in}}%
\pgfpathlineto{\pgfqpoint{2.610415in}{3.149867in}}%
\pgfpathlineto{\pgfqpoint{2.612987in}{3.100499in}}%
\pgfpathlineto{\pgfqpoint{2.615559in}{3.084043in}}%
\pgfpathlineto{\pgfqpoint{2.620703in}{3.084043in}}%
\pgfpathlineto{\pgfqpoint{2.628418in}{3.034675in}}%
\pgfpathlineto{\pgfqpoint{2.633562in}{3.034675in}}%
\pgfpathlineto{\pgfqpoint{2.638706in}{3.067587in}}%
\pgfpathlineto{\pgfqpoint{2.641278in}{3.067587in}}%
\pgfpathlineto{\pgfqpoint{2.643849in}{3.034675in}}%
\pgfpathlineto{\pgfqpoint{2.646421in}{3.018219in}}%
\pgfpathlineto{\pgfqpoint{2.646421in}{3.018219in}}%
\pgfusepath{stroke}%
\end{pgfscope}%
\begin{pgfscope}%
\pgfpathrectangle{\pgfqpoint{0.488751in}{1.946106in}}{\pgfqpoint{2.260417in}{1.502439in}}%
\pgfusepath{clip}%
\pgfsetrectcap%
\pgfsetroundjoin%
\pgfsetlinewidth{0.803000pt}%
\definecolor{currentstroke}{rgb}{0.686275,0.352941,0.313725}%
\pgfsetstrokecolor{currentstroke}%
\pgfsetdash{}{0pt}%
\pgfpathmoveto{\pgfqpoint{0.591497in}{2.376432in}}%
\pgfpathlineto{\pgfqpoint{0.599213in}{2.376432in}}%
\pgfpathlineto{\pgfqpoint{0.601785in}{2.359976in}}%
\pgfpathlineto{\pgfqpoint{0.604356in}{2.359976in}}%
\pgfpathlineto{\pgfqpoint{0.606928in}{2.327064in}}%
\pgfpathlineto{\pgfqpoint{0.609500in}{2.343520in}}%
\pgfpathlineto{\pgfqpoint{0.614644in}{2.343520in}}%
\pgfpathlineto{\pgfqpoint{0.622359in}{2.294152in}}%
\pgfpathlineto{\pgfqpoint{0.630075in}{2.294152in}}%
\pgfpathlineto{\pgfqpoint{0.632647in}{2.277696in}}%
\pgfpathlineto{\pgfqpoint{0.640363in}{2.277696in}}%
\pgfpathlineto{\pgfqpoint{0.642934in}{2.261240in}}%
\pgfpathlineto{\pgfqpoint{0.653222in}{2.261240in}}%
\pgfpathlineto{\pgfqpoint{0.655794in}{2.244784in}}%
\pgfpathlineto{\pgfqpoint{0.660938in}{2.244784in}}%
\pgfpathlineto{\pgfqpoint{0.663509in}{2.228328in}}%
\pgfpathlineto{\pgfqpoint{0.694372in}{2.228328in}}%
\pgfpathlineto{\pgfqpoint{0.696944in}{2.244784in}}%
\pgfpathlineto{\pgfqpoint{0.709803in}{2.244784in}}%
\pgfpathlineto{\pgfqpoint{0.712375in}{2.228328in}}%
\pgfpathlineto{\pgfqpoint{0.714947in}{2.228328in}}%
\pgfpathlineto{\pgfqpoint{0.717519in}{2.244784in}}%
\pgfpathlineto{\pgfqpoint{0.722662in}{2.244784in}}%
\pgfpathlineto{\pgfqpoint{0.725234in}{2.261240in}}%
\pgfpathlineto{\pgfqpoint{0.750953in}{2.261240in}}%
\pgfpathlineto{\pgfqpoint{0.756097in}{2.228328in}}%
\pgfpathlineto{\pgfqpoint{0.774100in}{2.228328in}}%
\pgfpathlineto{\pgfqpoint{0.776672in}{2.244784in}}%
\pgfpathlineto{\pgfqpoint{0.835825in}{2.244784in}}%
\pgfpathlineto{\pgfqpoint{0.838397in}{2.261240in}}%
\pgfpathlineto{\pgfqpoint{0.864115in}{2.261240in}}%
\pgfpathlineto{\pgfqpoint{0.866687in}{2.244784in}}%
\pgfpathlineto{\pgfqpoint{0.912981in}{2.244784in}}%
\pgfpathlineto{\pgfqpoint{0.915553in}{2.261240in}}%
\pgfpathlineto{\pgfqpoint{0.918125in}{2.261240in}}%
\pgfpathlineto{\pgfqpoint{0.920696in}{2.244784in}}%
\pgfpathlineto{\pgfqpoint{0.923268in}{2.244784in}}%
\pgfpathlineto{\pgfqpoint{0.925840in}{2.261240in}}%
\pgfpathlineto{\pgfqpoint{0.943843in}{2.261240in}}%
\pgfpathlineto{\pgfqpoint{0.946415in}{2.277696in}}%
\pgfpathlineto{\pgfqpoint{0.961846in}{2.277696in}}%
\pgfpathlineto{\pgfqpoint{0.966990in}{2.310608in}}%
\pgfpathlineto{\pgfqpoint{0.979849in}{2.310608in}}%
\pgfpathlineto{\pgfqpoint{0.982421in}{2.343520in}}%
\pgfpathlineto{\pgfqpoint{0.984993in}{2.343520in}}%
\pgfpathlineto{\pgfqpoint{0.987565in}{2.359976in}}%
\pgfpathlineto{\pgfqpoint{1.008140in}{2.359976in}}%
\pgfpathlineto{\pgfqpoint{1.010712in}{2.376432in}}%
\pgfpathlineto{\pgfqpoint{1.013284in}{2.376432in}}%
\pgfpathlineto{\pgfqpoint{1.015856in}{2.392888in}}%
\pgfpathlineto{\pgfqpoint{1.028715in}{2.392888in}}%
\pgfpathlineto{\pgfqpoint{1.031287in}{2.442256in}}%
\pgfpathlineto{\pgfqpoint{1.033859in}{2.442256in}}%
\pgfpathlineto{\pgfqpoint{1.036431in}{2.458712in}}%
\pgfpathlineto{\pgfqpoint{1.039002in}{2.491625in}}%
\pgfpathlineto{\pgfqpoint{1.041574in}{2.458712in}}%
\pgfpathlineto{\pgfqpoint{1.046718in}{2.458712in}}%
\pgfpathlineto{\pgfqpoint{1.049290in}{2.475169in}}%
\pgfpathlineto{\pgfqpoint{1.057006in}{2.475169in}}%
\pgfpathlineto{\pgfqpoint{1.059577in}{2.491625in}}%
\pgfpathlineto{\pgfqpoint{1.062149in}{2.491625in}}%
\pgfpathlineto{\pgfqpoint{1.064721in}{2.508081in}}%
\pgfpathlineto{\pgfqpoint{1.067293in}{2.508081in}}%
\pgfpathlineto{\pgfqpoint{1.069865in}{2.491625in}}%
\pgfpathlineto{\pgfqpoint{1.080152in}{2.557449in}}%
\pgfpathlineto{\pgfqpoint{1.082724in}{2.540993in}}%
\pgfpathlineto{\pgfqpoint{1.093012in}{2.540993in}}%
\pgfpathlineto{\pgfqpoint{1.095584in}{2.557449in}}%
\pgfpathlineto{\pgfqpoint{1.098155in}{2.524537in}}%
\pgfpathlineto{\pgfqpoint{1.108443in}{2.524537in}}%
\pgfpathlineto{\pgfqpoint{1.111015in}{2.508081in}}%
\pgfpathlineto{\pgfqpoint{1.113587in}{2.524537in}}%
\pgfpathlineto{\pgfqpoint{1.121302in}{2.524537in}}%
\pgfpathlineto{\pgfqpoint{1.123874in}{2.540993in}}%
\pgfpathlineto{\pgfqpoint{1.126446in}{2.540993in}}%
\pgfpathlineto{\pgfqpoint{1.129018in}{2.590361in}}%
\pgfpathlineto{\pgfqpoint{1.131590in}{2.606817in}}%
\pgfpathlineto{\pgfqpoint{1.134162in}{2.540993in}}%
\pgfpathlineto{\pgfqpoint{1.136734in}{2.540993in}}%
\pgfpathlineto{\pgfqpoint{1.139305in}{2.557449in}}%
\pgfpathlineto{\pgfqpoint{1.141877in}{2.590361in}}%
\pgfpathlineto{\pgfqpoint{1.144449in}{2.590361in}}%
\pgfpathlineto{\pgfqpoint{1.147021in}{2.606817in}}%
\pgfpathlineto{\pgfqpoint{1.149593in}{2.639729in}}%
\pgfpathlineto{\pgfqpoint{1.152165in}{2.623273in}}%
\pgfpathlineto{\pgfqpoint{1.162452in}{2.623273in}}%
\pgfpathlineto{\pgfqpoint{1.165024in}{2.639729in}}%
\pgfpathlineto{\pgfqpoint{1.170168in}{2.639729in}}%
\pgfpathlineto{\pgfqpoint{1.175312in}{2.672641in}}%
\pgfpathlineto{\pgfqpoint{1.180455in}{2.672641in}}%
\pgfpathlineto{\pgfqpoint{1.183027in}{2.656185in}}%
\pgfpathlineto{\pgfqpoint{1.185599in}{2.656185in}}%
\pgfpathlineto{\pgfqpoint{1.188171in}{2.672641in}}%
\pgfpathlineto{\pgfqpoint{1.190743in}{2.672641in}}%
\pgfpathlineto{\pgfqpoint{1.193315in}{2.689097in}}%
\pgfpathlineto{\pgfqpoint{1.198458in}{2.689097in}}%
\pgfpathlineto{\pgfqpoint{1.201030in}{2.606817in}}%
\pgfpathlineto{\pgfqpoint{1.208746in}{2.606817in}}%
\pgfpathlineto{\pgfqpoint{1.211318in}{2.639729in}}%
\pgfpathlineto{\pgfqpoint{1.213890in}{2.639729in}}%
\pgfpathlineto{\pgfqpoint{1.216461in}{2.590361in}}%
\pgfpathlineto{\pgfqpoint{1.219033in}{2.606817in}}%
\pgfpathlineto{\pgfqpoint{1.221605in}{2.672641in}}%
\pgfpathlineto{\pgfqpoint{1.224177in}{2.656185in}}%
\pgfpathlineto{\pgfqpoint{1.226749in}{2.590361in}}%
\pgfpathlineto{\pgfqpoint{1.231893in}{2.590361in}}%
\pgfpathlineto{\pgfqpoint{1.234465in}{2.573905in}}%
\pgfpathlineto{\pgfqpoint{1.237036in}{2.524537in}}%
\pgfpathlineto{\pgfqpoint{1.239608in}{2.524537in}}%
\pgfpathlineto{\pgfqpoint{1.242180in}{2.508081in}}%
\pgfpathlineto{\pgfqpoint{1.244752in}{2.475169in}}%
\pgfpathlineto{\pgfqpoint{1.247324in}{2.491625in}}%
\pgfpathlineto{\pgfqpoint{1.249896in}{2.491625in}}%
\pgfpathlineto{\pgfqpoint{1.252468in}{2.524537in}}%
\pgfpathlineto{\pgfqpoint{1.255040in}{2.540993in}}%
\pgfpathlineto{\pgfqpoint{1.257611in}{2.540993in}}%
\pgfpathlineto{\pgfqpoint{1.262755in}{2.458712in}}%
\pgfpathlineto{\pgfqpoint{1.265327in}{2.475169in}}%
\pgfpathlineto{\pgfqpoint{1.267899in}{2.508081in}}%
\pgfpathlineto{\pgfqpoint{1.275615in}{2.508081in}}%
\pgfpathlineto{\pgfqpoint{1.278186in}{2.491625in}}%
\pgfpathlineto{\pgfqpoint{1.288474in}{2.491625in}}%
\pgfpathlineto{\pgfqpoint{1.291046in}{2.524537in}}%
\pgfpathlineto{\pgfqpoint{1.293618in}{2.508081in}}%
\pgfpathlineto{\pgfqpoint{1.298761in}{2.540993in}}%
\pgfpathlineto{\pgfqpoint{1.301333in}{2.508081in}}%
\pgfpathlineto{\pgfqpoint{1.303905in}{2.524537in}}%
\pgfpathlineto{\pgfqpoint{1.309049in}{2.524537in}}%
\pgfpathlineto{\pgfqpoint{1.311621in}{2.491625in}}%
\pgfpathlineto{\pgfqpoint{1.319336in}{2.491625in}}%
\pgfpathlineto{\pgfqpoint{1.321908in}{2.475169in}}%
\pgfpathlineto{\pgfqpoint{1.324480in}{2.524537in}}%
\pgfpathlineto{\pgfqpoint{1.327052in}{2.475169in}}%
\pgfpathlineto{\pgfqpoint{1.329624in}{2.491625in}}%
\pgfpathlineto{\pgfqpoint{1.332196in}{2.475169in}}%
\pgfpathlineto{\pgfqpoint{1.334768in}{2.491625in}}%
\pgfpathlineto{\pgfqpoint{1.337339in}{2.475169in}}%
\pgfpathlineto{\pgfqpoint{1.339911in}{2.475169in}}%
\pgfpathlineto{\pgfqpoint{1.345055in}{2.508081in}}%
\pgfpathlineto{\pgfqpoint{1.352771in}{2.508081in}}%
\pgfpathlineto{\pgfqpoint{1.355342in}{2.524537in}}%
\pgfpathlineto{\pgfqpoint{1.357914in}{2.524537in}}%
\pgfpathlineto{\pgfqpoint{1.365630in}{2.425800in}}%
\pgfpathlineto{\pgfqpoint{1.373346in}{2.425800in}}%
\pgfpathlineto{\pgfqpoint{1.375917in}{2.458712in}}%
\pgfpathlineto{\pgfqpoint{1.378489in}{2.458712in}}%
\pgfpathlineto{\pgfqpoint{1.381061in}{2.442256in}}%
\pgfpathlineto{\pgfqpoint{1.383633in}{2.442256in}}%
\pgfpathlineto{\pgfqpoint{1.386205in}{2.458712in}}%
\pgfpathlineto{\pgfqpoint{1.388777in}{2.491625in}}%
\pgfpathlineto{\pgfqpoint{1.391349in}{2.458712in}}%
\pgfpathlineto{\pgfqpoint{1.401636in}{2.458712in}}%
\pgfpathlineto{\pgfqpoint{1.404208in}{2.475169in}}%
\pgfpathlineto{\pgfqpoint{1.406780in}{2.475169in}}%
\pgfpathlineto{\pgfqpoint{1.409352in}{2.458712in}}%
\pgfpathlineto{\pgfqpoint{1.417067in}{2.458712in}}%
\pgfpathlineto{\pgfqpoint{1.419639in}{2.475169in}}%
\pgfpathlineto{\pgfqpoint{1.422211in}{2.458712in}}%
\pgfpathlineto{\pgfqpoint{1.424783in}{2.491625in}}%
\pgfpathlineto{\pgfqpoint{1.427355in}{2.491625in}}%
\pgfpathlineto{\pgfqpoint{1.429927in}{2.475169in}}%
\pgfpathlineto{\pgfqpoint{1.432499in}{2.524537in}}%
\pgfpathlineto{\pgfqpoint{1.435070in}{2.475169in}}%
\pgfpathlineto{\pgfqpoint{1.437642in}{2.475169in}}%
\pgfpathlineto{\pgfqpoint{1.440214in}{2.491625in}}%
\pgfpathlineto{\pgfqpoint{1.442786in}{2.491625in}}%
\pgfpathlineto{\pgfqpoint{1.445358in}{2.557449in}}%
\pgfpathlineto{\pgfqpoint{1.447930in}{2.524537in}}%
\pgfpathlineto{\pgfqpoint{1.453074in}{2.524537in}}%
\pgfpathlineto{\pgfqpoint{1.455645in}{2.475169in}}%
\pgfpathlineto{\pgfqpoint{1.458217in}{2.475169in}}%
\pgfpathlineto{\pgfqpoint{1.460789in}{2.491625in}}%
\pgfpathlineto{\pgfqpoint{1.463361in}{2.540993in}}%
\pgfpathlineto{\pgfqpoint{1.471077in}{2.590361in}}%
\pgfpathlineto{\pgfqpoint{1.481364in}{2.590361in}}%
\pgfpathlineto{\pgfqpoint{1.483936in}{2.573905in}}%
\pgfpathlineto{\pgfqpoint{1.486508in}{2.590361in}}%
\pgfpathlineto{\pgfqpoint{1.489080in}{2.573905in}}%
\pgfpathlineto{\pgfqpoint{1.491652in}{2.606817in}}%
\pgfpathlineto{\pgfqpoint{1.494223in}{2.590361in}}%
\pgfpathlineto{\pgfqpoint{1.501939in}{2.639729in}}%
\pgfpathlineto{\pgfqpoint{1.509655in}{2.639729in}}%
\pgfpathlineto{\pgfqpoint{1.512227in}{2.590361in}}%
\pgfpathlineto{\pgfqpoint{1.514798in}{2.623273in}}%
\pgfpathlineto{\pgfqpoint{1.519942in}{2.623273in}}%
\pgfpathlineto{\pgfqpoint{1.522514in}{2.606817in}}%
\pgfpathlineto{\pgfqpoint{1.527658in}{2.606817in}}%
\pgfpathlineto{\pgfqpoint{1.530230in}{2.639729in}}%
\pgfpathlineto{\pgfqpoint{1.532802in}{2.606817in}}%
\pgfpathlineto{\pgfqpoint{1.537945in}{2.606817in}}%
\pgfpathlineto{\pgfqpoint{1.540517in}{2.623273in}}%
\pgfpathlineto{\pgfqpoint{1.543089in}{2.623273in}}%
\pgfpathlineto{\pgfqpoint{1.545661in}{2.606817in}}%
\pgfpathlineto{\pgfqpoint{1.548233in}{2.606817in}}%
\pgfpathlineto{\pgfqpoint{1.550805in}{2.590361in}}%
\pgfpathlineto{\pgfqpoint{1.558520in}{2.590361in}}%
\pgfpathlineto{\pgfqpoint{1.561092in}{2.557449in}}%
\pgfpathlineto{\pgfqpoint{1.563664in}{2.573905in}}%
\pgfpathlineto{\pgfqpoint{1.571380in}{2.573905in}}%
\pgfpathlineto{\pgfqpoint{1.573951in}{2.606817in}}%
\pgfpathlineto{\pgfqpoint{1.584239in}{2.606817in}}%
\pgfpathlineto{\pgfqpoint{1.586811in}{2.623273in}}%
\pgfpathlineto{\pgfqpoint{1.591955in}{2.623273in}}%
\pgfpathlineto{\pgfqpoint{1.597098in}{2.590361in}}%
\pgfpathlineto{\pgfqpoint{1.599670in}{2.623273in}}%
\pgfpathlineto{\pgfqpoint{1.602242in}{2.606817in}}%
\pgfpathlineto{\pgfqpoint{1.604814in}{2.623273in}}%
\pgfpathlineto{\pgfqpoint{1.609958in}{2.623273in}}%
\pgfpathlineto{\pgfqpoint{1.612529in}{2.590361in}}%
\pgfpathlineto{\pgfqpoint{1.615101in}{2.606817in}}%
\pgfpathlineto{\pgfqpoint{1.617673in}{2.606817in}}%
\pgfpathlineto{\pgfqpoint{1.620245in}{2.623273in}}%
\pgfpathlineto{\pgfqpoint{1.638248in}{2.623273in}}%
\pgfpathlineto{\pgfqpoint{1.640820in}{2.606817in}}%
\pgfpathlineto{\pgfqpoint{1.648536in}{2.606817in}}%
\pgfpathlineto{\pgfqpoint{1.651108in}{2.623273in}}%
\pgfpathlineto{\pgfqpoint{1.653679in}{2.606817in}}%
\pgfpathlineto{\pgfqpoint{1.656251in}{2.606817in}}%
\pgfpathlineto{\pgfqpoint{1.658823in}{2.573905in}}%
\pgfpathlineto{\pgfqpoint{1.661395in}{2.573905in}}%
\pgfpathlineto{\pgfqpoint{1.663967in}{2.524537in}}%
\pgfpathlineto{\pgfqpoint{1.671683in}{2.524537in}}%
\pgfpathlineto{\pgfqpoint{1.674254in}{2.540993in}}%
\pgfpathlineto{\pgfqpoint{1.676826in}{2.540993in}}%
\pgfpathlineto{\pgfqpoint{1.679398in}{2.557449in}}%
\pgfpathlineto{\pgfqpoint{1.694829in}{2.557449in}}%
\pgfpathlineto{\pgfqpoint{1.697401in}{2.540993in}}%
\pgfpathlineto{\pgfqpoint{1.712832in}{2.540993in}}%
\pgfpathlineto{\pgfqpoint{1.715404in}{2.524537in}}%
\pgfpathlineto{\pgfqpoint{1.717976in}{2.524537in}}%
\pgfpathlineto{\pgfqpoint{1.720548in}{2.508081in}}%
\pgfpathlineto{\pgfqpoint{1.725692in}{2.508081in}}%
\pgfpathlineto{\pgfqpoint{1.728264in}{2.524537in}}%
\pgfpathlineto{\pgfqpoint{1.743695in}{2.524537in}}%
\pgfpathlineto{\pgfqpoint{1.746267in}{2.540993in}}%
\pgfpathlineto{\pgfqpoint{1.748839in}{2.540993in}}%
\pgfpathlineto{\pgfqpoint{1.751410in}{2.557449in}}%
\pgfpathlineto{\pgfqpoint{1.759126in}{2.557449in}}%
\pgfpathlineto{\pgfqpoint{1.761698in}{2.573905in}}%
\pgfpathlineto{\pgfqpoint{1.766842in}{2.573905in}}%
\pgfpathlineto{\pgfqpoint{1.771985in}{2.606817in}}%
\pgfpathlineto{\pgfqpoint{1.774557in}{2.590361in}}%
\pgfpathlineto{\pgfqpoint{1.777129in}{2.590361in}}%
\pgfpathlineto{\pgfqpoint{1.779701in}{2.573905in}}%
\pgfpathlineto{\pgfqpoint{1.787417in}{2.573905in}}%
\pgfpathlineto{\pgfqpoint{1.789989in}{2.590361in}}%
\pgfpathlineto{\pgfqpoint{1.797704in}{2.590361in}}%
\pgfpathlineto{\pgfqpoint{1.800276in}{2.606817in}}%
\pgfpathlineto{\pgfqpoint{1.802848in}{2.590361in}}%
\pgfpathlineto{\pgfqpoint{1.805420in}{2.590361in}}%
\pgfpathlineto{\pgfqpoint{1.810563in}{2.623273in}}%
\pgfpathlineto{\pgfqpoint{1.815707in}{2.623273in}}%
\pgfpathlineto{\pgfqpoint{1.818279in}{2.639729in}}%
\pgfpathlineto{\pgfqpoint{1.820851in}{2.639729in}}%
\pgfpathlineto{\pgfqpoint{1.823423in}{2.623273in}}%
\pgfpathlineto{\pgfqpoint{1.828567in}{2.623273in}}%
\pgfpathlineto{\pgfqpoint{1.831138in}{2.639729in}}%
\pgfpathlineto{\pgfqpoint{1.833710in}{2.623273in}}%
\pgfpathlineto{\pgfqpoint{1.836282in}{2.623273in}}%
\pgfpathlineto{\pgfqpoint{1.838854in}{2.606817in}}%
\pgfpathlineto{\pgfqpoint{1.849142in}{2.606817in}}%
\pgfpathlineto{\pgfqpoint{1.851713in}{2.590361in}}%
\pgfpathlineto{\pgfqpoint{1.877432in}{2.590361in}}%
\pgfpathlineto{\pgfqpoint{1.880004in}{2.606817in}}%
\pgfpathlineto{\pgfqpoint{1.885148in}{2.606817in}}%
\pgfpathlineto{\pgfqpoint{1.887720in}{2.623273in}}%
\pgfpathlineto{\pgfqpoint{1.908295in}{2.623273in}}%
\pgfpathlineto{\pgfqpoint{1.910866in}{2.606817in}}%
\pgfpathlineto{\pgfqpoint{1.926298in}{2.606817in}}%
\pgfpathlineto{\pgfqpoint{1.928870in}{2.623273in}}%
\pgfpathlineto{\pgfqpoint{1.934013in}{2.623273in}}%
\pgfpathlineto{\pgfqpoint{1.936585in}{2.606817in}}%
\pgfpathlineto{\pgfqpoint{1.946873in}{2.606817in}}%
\pgfpathlineto{\pgfqpoint{1.949444in}{2.590361in}}%
\pgfpathlineto{\pgfqpoint{1.959732in}{2.590361in}}%
\pgfpathlineto{\pgfqpoint{1.962304in}{2.606817in}}%
\pgfpathlineto{\pgfqpoint{1.967448in}{2.606817in}}%
\pgfpathlineto{\pgfqpoint{1.970019in}{2.623273in}}%
\pgfpathlineto{\pgfqpoint{1.972591in}{2.623273in}}%
\pgfpathlineto{\pgfqpoint{1.975163in}{2.639729in}}%
\pgfpathlineto{\pgfqpoint{1.977735in}{2.672641in}}%
\pgfpathlineto{\pgfqpoint{1.990594in}{2.672641in}}%
\pgfpathlineto{\pgfqpoint{1.993166in}{2.656185in}}%
\pgfpathlineto{\pgfqpoint{2.018885in}{2.656185in}}%
\pgfpathlineto{\pgfqpoint{2.021457in}{2.639729in}}%
\pgfpathlineto{\pgfqpoint{2.024029in}{2.639729in}}%
\pgfpathlineto{\pgfqpoint{2.034316in}{2.573905in}}%
\pgfpathlineto{\pgfqpoint{2.036888in}{2.590361in}}%
\pgfpathlineto{\pgfqpoint{2.042032in}{2.590361in}}%
\pgfpathlineto{\pgfqpoint{2.044604in}{2.606817in}}%
\pgfpathlineto{\pgfqpoint{2.049747in}{2.606817in}}%
\pgfpathlineto{\pgfqpoint{2.052319in}{2.623273in}}%
\pgfpathlineto{\pgfqpoint{2.060035in}{2.623273in}}%
\pgfpathlineto{\pgfqpoint{2.062607in}{2.639729in}}%
\pgfpathlineto{\pgfqpoint{2.080610in}{2.639729in}}%
\pgfpathlineto{\pgfqpoint{2.083182in}{2.656185in}}%
\pgfpathlineto{\pgfqpoint{2.090897in}{2.656185in}}%
\pgfpathlineto{\pgfqpoint{2.093469in}{2.672641in}}%
\pgfpathlineto{\pgfqpoint{2.096041in}{2.656185in}}%
\pgfpathlineto{\pgfqpoint{2.106329in}{2.656185in}}%
\pgfpathlineto{\pgfqpoint{2.111472in}{2.689097in}}%
\pgfpathlineto{\pgfqpoint{2.114044in}{2.672641in}}%
\pgfpathlineto{\pgfqpoint{2.116616in}{2.672641in}}%
\pgfpathlineto{\pgfqpoint{2.119188in}{2.656185in}}%
\pgfpathlineto{\pgfqpoint{2.121760in}{2.656185in}}%
\pgfpathlineto{\pgfqpoint{2.124332in}{2.639729in}}%
\pgfpathlineto{\pgfqpoint{2.165482in}{2.639729in}}%
\pgfpathlineto{\pgfqpoint{2.168053in}{2.623273in}}%
\pgfpathlineto{\pgfqpoint{2.175769in}{2.623273in}}%
\pgfpathlineto{\pgfqpoint{2.178341in}{2.639729in}}%
\pgfpathlineto{\pgfqpoint{2.180913in}{2.623273in}}%
\pgfpathlineto{\pgfqpoint{2.191200in}{2.623273in}}%
\pgfpathlineto{\pgfqpoint{2.193772in}{2.606817in}}%
\pgfpathlineto{\pgfqpoint{2.196344in}{2.606817in}}%
\pgfpathlineto{\pgfqpoint{2.198916in}{2.623273in}}%
\pgfpathlineto{\pgfqpoint{2.211775in}{2.623273in}}%
\pgfpathlineto{\pgfqpoint{2.214347in}{2.639729in}}%
\pgfpathlineto{\pgfqpoint{2.216919in}{2.623273in}}%
\pgfpathlineto{\pgfqpoint{2.224635in}{2.623273in}}%
\pgfpathlineto{\pgfqpoint{2.227206in}{2.639729in}}%
\pgfpathlineto{\pgfqpoint{2.234922in}{2.639729in}}%
\pgfpathlineto{\pgfqpoint{2.237494in}{2.623273in}}%
\pgfpathlineto{\pgfqpoint{2.245210in}{2.623273in}}%
\pgfpathlineto{\pgfqpoint{2.247781in}{2.639729in}}%
\pgfpathlineto{\pgfqpoint{2.250353in}{2.639729in}}%
\pgfpathlineto{\pgfqpoint{2.252925in}{2.623273in}}%
\pgfpathlineto{\pgfqpoint{2.255497in}{2.623273in}}%
\pgfpathlineto{\pgfqpoint{2.260641in}{2.656185in}}%
\pgfpathlineto{\pgfqpoint{2.263213in}{2.639729in}}%
\pgfpathlineto{\pgfqpoint{2.265785in}{2.639729in}}%
\pgfpathlineto{\pgfqpoint{2.268356in}{2.623273in}}%
\pgfpathlineto{\pgfqpoint{2.270928in}{2.639729in}}%
\pgfpathlineto{\pgfqpoint{2.273500in}{2.639729in}}%
\pgfpathlineto{\pgfqpoint{2.276072in}{2.656185in}}%
\pgfpathlineto{\pgfqpoint{2.281216in}{2.623273in}}%
\pgfpathlineto{\pgfqpoint{2.306934in}{2.623273in}}%
\pgfpathlineto{\pgfqpoint{2.309506in}{2.606817in}}%
\pgfpathlineto{\pgfqpoint{2.312078in}{2.606817in}}%
\pgfpathlineto{\pgfqpoint{2.314650in}{2.573905in}}%
\pgfpathlineto{\pgfqpoint{2.319794in}{2.573905in}}%
\pgfpathlineto{\pgfqpoint{2.322366in}{2.557449in}}%
\pgfpathlineto{\pgfqpoint{2.324938in}{2.557449in}}%
\pgfpathlineto{\pgfqpoint{2.327509in}{2.540993in}}%
\pgfpathlineto{\pgfqpoint{2.330081in}{2.540993in}}%
\pgfpathlineto{\pgfqpoint{2.332653in}{2.557449in}}%
\pgfpathlineto{\pgfqpoint{2.335225in}{2.540993in}}%
\pgfpathlineto{\pgfqpoint{2.345512in}{2.540993in}}%
\pgfpathlineto{\pgfqpoint{2.348084in}{2.557449in}}%
\pgfpathlineto{\pgfqpoint{2.360944in}{2.557449in}}%
\pgfpathlineto{\pgfqpoint{2.366087in}{2.524537in}}%
\pgfpathlineto{\pgfqpoint{2.373803in}{2.524537in}}%
\pgfpathlineto{\pgfqpoint{2.378947in}{2.557449in}}%
\pgfpathlineto{\pgfqpoint{2.381519in}{2.557449in}}%
\pgfpathlineto{\pgfqpoint{2.384091in}{2.540993in}}%
\pgfpathlineto{\pgfqpoint{2.386662in}{2.557449in}}%
\pgfpathlineto{\pgfqpoint{2.394378in}{2.557449in}}%
\pgfpathlineto{\pgfqpoint{2.396950in}{2.573905in}}%
\pgfpathlineto{\pgfqpoint{2.399522in}{2.557449in}}%
\pgfpathlineto{\pgfqpoint{2.402094in}{2.524537in}}%
\pgfpathlineto{\pgfqpoint{2.407237in}{2.524537in}}%
\pgfpathlineto{\pgfqpoint{2.409809in}{2.540993in}}%
\pgfpathlineto{\pgfqpoint{2.414953in}{2.508081in}}%
\pgfpathlineto{\pgfqpoint{2.420097in}{2.540993in}}%
\pgfpathlineto{\pgfqpoint{2.427812in}{2.540993in}}%
\pgfpathlineto{\pgfqpoint{2.432956in}{2.508081in}}%
\pgfpathlineto{\pgfqpoint{2.435528in}{2.524537in}}%
\pgfpathlineto{\pgfqpoint{2.445815in}{2.524537in}}%
\pgfpathlineto{\pgfqpoint{2.448387in}{2.540993in}}%
\pgfpathlineto{\pgfqpoint{2.450959in}{2.540993in}}%
\pgfpathlineto{\pgfqpoint{2.453531in}{2.557449in}}%
\pgfpathlineto{\pgfqpoint{2.466390in}{2.557449in}}%
\pgfpathlineto{\pgfqpoint{2.468962in}{2.573905in}}%
\pgfpathlineto{\pgfqpoint{2.471534in}{2.557449in}}%
\pgfpathlineto{\pgfqpoint{2.476678in}{2.557449in}}%
\pgfpathlineto{\pgfqpoint{2.479250in}{2.540993in}}%
\pgfpathlineto{\pgfqpoint{2.484393in}{2.540993in}}%
\pgfpathlineto{\pgfqpoint{2.486965in}{2.524537in}}%
\pgfpathlineto{\pgfqpoint{2.492109in}{2.524537in}}%
\pgfpathlineto{\pgfqpoint{2.494681in}{2.508081in}}%
\pgfpathlineto{\pgfqpoint{2.499825in}{2.508081in}}%
\pgfpathlineto{\pgfqpoint{2.502397in}{2.491625in}}%
\pgfpathlineto{\pgfqpoint{2.507540in}{2.491625in}}%
\pgfpathlineto{\pgfqpoint{2.510112in}{2.508081in}}%
\pgfpathlineto{\pgfqpoint{2.520400in}{2.508081in}}%
\pgfpathlineto{\pgfqpoint{2.525543in}{2.540993in}}%
\pgfpathlineto{\pgfqpoint{2.528115in}{2.524537in}}%
\pgfpathlineto{\pgfqpoint{2.530687in}{2.524537in}}%
\pgfpathlineto{\pgfqpoint{2.533259in}{2.540993in}}%
\pgfpathlineto{\pgfqpoint{2.535831in}{2.540993in}}%
\pgfpathlineto{\pgfqpoint{2.538403in}{2.557449in}}%
\pgfpathlineto{\pgfqpoint{2.548690in}{2.557449in}}%
\pgfpathlineto{\pgfqpoint{2.551262in}{2.540993in}}%
\pgfpathlineto{\pgfqpoint{2.561550in}{2.540993in}}%
\pgfpathlineto{\pgfqpoint{2.564121in}{2.557449in}}%
\pgfpathlineto{\pgfqpoint{2.569265in}{2.557449in}}%
\pgfpathlineto{\pgfqpoint{2.571837in}{2.540993in}}%
\pgfpathlineto{\pgfqpoint{2.594984in}{2.540993in}}%
\pgfpathlineto{\pgfqpoint{2.597556in}{2.524537in}}%
\pgfpathlineto{\pgfqpoint{2.605271in}{2.524537in}}%
\pgfpathlineto{\pgfqpoint{2.607843in}{2.508081in}}%
\pgfpathlineto{\pgfqpoint{2.623274in}{2.508081in}}%
\pgfpathlineto{\pgfqpoint{2.625846in}{2.475169in}}%
\pgfpathlineto{\pgfqpoint{2.628418in}{2.491625in}}%
\pgfpathlineto{\pgfqpoint{2.641278in}{2.491625in}}%
\pgfpathlineto{\pgfqpoint{2.643849in}{2.508081in}}%
\pgfpathlineto{\pgfqpoint{2.646421in}{2.508081in}}%
\pgfpathlineto{\pgfqpoint{2.646421in}{2.508081in}}%
\pgfusepath{stroke}%
\end{pgfscope}%
\begin{pgfscope}%
\pgfpathrectangle{\pgfqpoint{0.488751in}{1.946106in}}{\pgfqpoint{2.260417in}{1.502439in}}%
\pgfusepath{clip}%
\pgfsetrectcap%
\pgfsetroundjoin%
\pgfsetlinewidth{0.803000pt}%
\definecolor{currentstroke}{rgb}{0.000000,0.356863,0.509804}%
\pgfsetstrokecolor{currentstroke}%
\pgfsetdash{}{0pt}%
\pgfpathmoveto{\pgfqpoint{0.591497in}{2.837202in}}%
\pgfpathlineto{\pgfqpoint{0.596641in}{2.804290in}}%
\pgfpathlineto{\pgfqpoint{0.599213in}{2.853658in}}%
\pgfpathlineto{\pgfqpoint{0.601785in}{2.804290in}}%
\pgfpathlineto{\pgfqpoint{0.604356in}{2.853658in}}%
\pgfpathlineto{\pgfqpoint{0.606928in}{2.870114in}}%
\pgfpathlineto{\pgfqpoint{0.609500in}{2.870114in}}%
\pgfpathlineto{\pgfqpoint{0.612072in}{2.886570in}}%
\pgfpathlineto{\pgfqpoint{0.614644in}{2.935938in}}%
\pgfpathlineto{\pgfqpoint{0.617216in}{2.952395in}}%
\pgfpathlineto{\pgfqpoint{0.619788in}{2.952395in}}%
\pgfpathlineto{\pgfqpoint{0.622359in}{2.985307in}}%
\pgfpathlineto{\pgfqpoint{0.624931in}{3.001763in}}%
\pgfpathlineto{\pgfqpoint{0.627503in}{3.034675in}}%
\pgfpathlineto{\pgfqpoint{0.630075in}{3.051131in}}%
\pgfpathlineto{\pgfqpoint{0.632647in}{2.968851in}}%
\pgfpathlineto{\pgfqpoint{0.635219in}{2.935938in}}%
\pgfpathlineto{\pgfqpoint{0.637791in}{2.985307in}}%
\pgfpathlineto{\pgfqpoint{0.640363in}{2.985307in}}%
\pgfpathlineto{\pgfqpoint{0.642934in}{2.952395in}}%
\pgfpathlineto{\pgfqpoint{0.648078in}{3.018219in}}%
\pgfpathlineto{\pgfqpoint{0.650650in}{2.985307in}}%
\pgfpathlineto{\pgfqpoint{0.653222in}{2.968851in}}%
\pgfpathlineto{\pgfqpoint{0.655794in}{2.935938in}}%
\pgfpathlineto{\pgfqpoint{0.660938in}{2.903026in}}%
\pgfpathlineto{\pgfqpoint{0.663509in}{2.903026in}}%
\pgfpathlineto{\pgfqpoint{0.671225in}{2.853658in}}%
\pgfpathlineto{\pgfqpoint{0.673797in}{2.870114in}}%
\pgfpathlineto{\pgfqpoint{0.676369in}{2.870114in}}%
\pgfpathlineto{\pgfqpoint{0.678941in}{2.886570in}}%
\pgfpathlineto{\pgfqpoint{0.681513in}{2.886570in}}%
\pgfpathlineto{\pgfqpoint{0.684084in}{2.903026in}}%
\pgfpathlineto{\pgfqpoint{0.686656in}{2.886570in}}%
\pgfpathlineto{\pgfqpoint{0.689228in}{2.853658in}}%
\pgfpathlineto{\pgfqpoint{0.696944in}{2.853658in}}%
\pgfpathlineto{\pgfqpoint{0.699516in}{2.886570in}}%
\pgfpathlineto{\pgfqpoint{0.702087in}{2.903026in}}%
\pgfpathlineto{\pgfqpoint{0.704659in}{2.886570in}}%
\pgfpathlineto{\pgfqpoint{0.707231in}{2.952395in}}%
\pgfpathlineto{\pgfqpoint{0.712375in}{2.985307in}}%
\pgfpathlineto{\pgfqpoint{0.714947in}{2.985307in}}%
\pgfpathlineto{\pgfqpoint{0.717519in}{2.968851in}}%
\pgfpathlineto{\pgfqpoint{0.722662in}{2.968851in}}%
\pgfpathlineto{\pgfqpoint{0.725234in}{2.985307in}}%
\pgfpathlineto{\pgfqpoint{0.727806in}{2.952395in}}%
\pgfpathlineto{\pgfqpoint{0.730378in}{2.952395in}}%
\pgfpathlineto{\pgfqpoint{0.732950in}{2.968851in}}%
\pgfpathlineto{\pgfqpoint{0.735522in}{2.968851in}}%
\pgfpathlineto{\pgfqpoint{0.738094in}{2.952395in}}%
\pgfpathlineto{\pgfqpoint{0.740666in}{2.952395in}}%
\pgfpathlineto{\pgfqpoint{0.743237in}{2.968851in}}%
\pgfpathlineto{\pgfqpoint{0.745809in}{2.935938in}}%
\pgfpathlineto{\pgfqpoint{0.748381in}{2.935938in}}%
\pgfpathlineto{\pgfqpoint{0.753525in}{2.820746in}}%
\pgfpathlineto{\pgfqpoint{0.758669in}{2.853658in}}%
\pgfpathlineto{\pgfqpoint{0.761240in}{2.837202in}}%
\pgfpathlineto{\pgfqpoint{0.763812in}{2.870114in}}%
\pgfpathlineto{\pgfqpoint{0.766384in}{2.853658in}}%
\pgfpathlineto{\pgfqpoint{0.768956in}{2.853658in}}%
\pgfpathlineto{\pgfqpoint{0.771528in}{2.837202in}}%
\pgfpathlineto{\pgfqpoint{0.776672in}{2.870114in}}%
\pgfpathlineto{\pgfqpoint{0.779244in}{2.837202in}}%
\pgfpathlineto{\pgfqpoint{0.784387in}{2.804290in}}%
\pgfpathlineto{\pgfqpoint{0.792103in}{2.804290in}}%
\pgfpathlineto{\pgfqpoint{0.794675in}{2.787834in}}%
\pgfpathlineto{\pgfqpoint{0.799819in}{2.820746in}}%
\pgfpathlineto{\pgfqpoint{0.802390in}{2.820746in}}%
\pgfpathlineto{\pgfqpoint{0.807534in}{2.853658in}}%
\pgfpathlineto{\pgfqpoint{0.810106in}{2.787834in}}%
\pgfpathlineto{\pgfqpoint{0.812678in}{2.804290in}}%
\pgfpathlineto{\pgfqpoint{0.817822in}{2.804290in}}%
\pgfpathlineto{\pgfqpoint{0.822965in}{2.837202in}}%
\pgfpathlineto{\pgfqpoint{0.825537in}{2.837202in}}%
\pgfpathlineto{\pgfqpoint{0.830681in}{2.870114in}}%
\pgfpathlineto{\pgfqpoint{0.835825in}{2.837202in}}%
\pgfpathlineto{\pgfqpoint{0.838397in}{2.870114in}}%
\pgfpathlineto{\pgfqpoint{0.843540in}{2.870114in}}%
\pgfpathlineto{\pgfqpoint{0.846112in}{2.837202in}}%
\pgfpathlineto{\pgfqpoint{0.848684in}{2.853658in}}%
\pgfpathlineto{\pgfqpoint{0.851256in}{2.886570in}}%
\pgfpathlineto{\pgfqpoint{0.856400in}{2.919482in}}%
\pgfpathlineto{\pgfqpoint{0.858972in}{2.919482in}}%
\pgfpathlineto{\pgfqpoint{0.864115in}{2.886570in}}%
\pgfpathlineto{\pgfqpoint{0.869259in}{2.886570in}}%
\pgfpathlineto{\pgfqpoint{0.871831in}{2.919482in}}%
\pgfpathlineto{\pgfqpoint{0.874403in}{2.935938in}}%
\pgfpathlineto{\pgfqpoint{0.876975in}{2.919482in}}%
\pgfpathlineto{\pgfqpoint{0.879547in}{2.919482in}}%
\pgfpathlineto{\pgfqpoint{0.882118in}{2.903026in}}%
\pgfpathlineto{\pgfqpoint{0.884690in}{2.919482in}}%
\pgfpathlineto{\pgfqpoint{0.889834in}{2.886570in}}%
\pgfpathlineto{\pgfqpoint{0.894978in}{2.886570in}}%
\pgfpathlineto{\pgfqpoint{0.897550in}{2.903026in}}%
\pgfpathlineto{\pgfqpoint{0.902693in}{2.903026in}}%
\pgfpathlineto{\pgfqpoint{0.905265in}{2.870114in}}%
\pgfpathlineto{\pgfqpoint{0.907837in}{2.853658in}}%
\pgfpathlineto{\pgfqpoint{0.910409in}{2.853658in}}%
\pgfpathlineto{\pgfqpoint{0.912981in}{2.870114in}}%
\pgfpathlineto{\pgfqpoint{0.918125in}{2.870114in}}%
\pgfpathlineto{\pgfqpoint{0.920696in}{2.820746in}}%
\pgfpathlineto{\pgfqpoint{0.923268in}{2.870114in}}%
\pgfpathlineto{\pgfqpoint{0.925840in}{2.886570in}}%
\pgfpathlineto{\pgfqpoint{0.936128in}{2.886570in}}%
\pgfpathlineto{\pgfqpoint{0.938700in}{2.919482in}}%
\pgfpathlineto{\pgfqpoint{0.941271in}{2.919482in}}%
\pgfpathlineto{\pgfqpoint{0.943843in}{2.886570in}}%
\pgfpathlineto{\pgfqpoint{0.946415in}{2.870114in}}%
\pgfpathlineto{\pgfqpoint{0.948987in}{2.870114in}}%
\pgfpathlineto{\pgfqpoint{0.951559in}{2.853658in}}%
\pgfpathlineto{\pgfqpoint{0.954131in}{2.886570in}}%
\pgfpathlineto{\pgfqpoint{0.956703in}{2.903026in}}%
\pgfpathlineto{\pgfqpoint{0.961846in}{2.903026in}}%
\pgfpathlineto{\pgfqpoint{0.964418in}{2.870114in}}%
\pgfpathlineto{\pgfqpoint{0.972134in}{2.919482in}}%
\pgfpathlineto{\pgfqpoint{0.974706in}{2.952395in}}%
\pgfpathlineto{\pgfqpoint{0.977278in}{2.968851in}}%
\pgfpathlineto{\pgfqpoint{0.979849in}{2.968851in}}%
\pgfpathlineto{\pgfqpoint{0.982421in}{2.985307in}}%
\pgfpathlineto{\pgfqpoint{0.984993in}{2.985307in}}%
\pgfpathlineto{\pgfqpoint{0.987565in}{3.001763in}}%
\pgfpathlineto{\pgfqpoint{0.990137in}{3.034675in}}%
\pgfpathlineto{\pgfqpoint{1.002996in}{3.034675in}}%
\pgfpathlineto{\pgfqpoint{1.005568in}{3.018219in}}%
\pgfpathlineto{\pgfqpoint{1.008140in}{2.985307in}}%
\pgfpathlineto{\pgfqpoint{1.010712in}{2.935938in}}%
\pgfpathlineto{\pgfqpoint{1.013284in}{2.935938in}}%
\pgfpathlineto{\pgfqpoint{1.015856in}{2.903026in}}%
\pgfpathlineto{\pgfqpoint{1.018427in}{2.837202in}}%
\pgfpathlineto{\pgfqpoint{1.023571in}{2.837202in}}%
\pgfpathlineto{\pgfqpoint{1.026143in}{2.870114in}}%
\pgfpathlineto{\pgfqpoint{1.028715in}{2.853658in}}%
\pgfpathlineto{\pgfqpoint{1.031287in}{2.870114in}}%
\pgfpathlineto{\pgfqpoint{1.033859in}{2.853658in}}%
\pgfpathlineto{\pgfqpoint{1.036431in}{2.853658in}}%
\pgfpathlineto{\pgfqpoint{1.041574in}{2.820746in}}%
\pgfpathlineto{\pgfqpoint{1.044146in}{2.870114in}}%
\pgfpathlineto{\pgfqpoint{1.046718in}{2.853658in}}%
\pgfpathlineto{\pgfqpoint{1.049290in}{2.853658in}}%
\pgfpathlineto{\pgfqpoint{1.051862in}{2.870114in}}%
\pgfpathlineto{\pgfqpoint{1.054434in}{2.870114in}}%
\pgfpathlineto{\pgfqpoint{1.057006in}{2.853658in}}%
\pgfpathlineto{\pgfqpoint{1.059577in}{2.919482in}}%
\pgfpathlineto{\pgfqpoint{1.062149in}{2.952395in}}%
\pgfpathlineto{\pgfqpoint{1.064721in}{2.952395in}}%
\pgfpathlineto{\pgfqpoint{1.069865in}{3.018219in}}%
\pgfpathlineto{\pgfqpoint{1.072437in}{3.034675in}}%
\pgfpathlineto{\pgfqpoint{1.087868in}{3.034675in}}%
\pgfpathlineto{\pgfqpoint{1.090440in}{3.084043in}}%
\pgfpathlineto{\pgfqpoint{1.093012in}{3.051131in}}%
\pgfpathlineto{\pgfqpoint{1.095584in}{3.067587in}}%
\pgfpathlineto{\pgfqpoint{1.098155in}{3.067587in}}%
\pgfpathlineto{\pgfqpoint{1.103299in}{3.100499in}}%
\pgfpathlineto{\pgfqpoint{1.108443in}{3.100499in}}%
\pgfpathlineto{\pgfqpoint{1.111015in}{3.051131in}}%
\pgfpathlineto{\pgfqpoint{1.118730in}{3.051131in}}%
\pgfpathlineto{\pgfqpoint{1.123874in}{3.018219in}}%
\pgfpathlineto{\pgfqpoint{1.126446in}{3.018219in}}%
\pgfpathlineto{\pgfqpoint{1.129018in}{2.985307in}}%
\pgfpathlineto{\pgfqpoint{1.131590in}{3.001763in}}%
\pgfpathlineto{\pgfqpoint{1.134162in}{2.952395in}}%
\pgfpathlineto{\pgfqpoint{1.136734in}{2.952395in}}%
\pgfpathlineto{\pgfqpoint{1.139305in}{3.001763in}}%
\pgfpathlineto{\pgfqpoint{1.147021in}{3.001763in}}%
\pgfpathlineto{\pgfqpoint{1.149593in}{3.034675in}}%
\pgfpathlineto{\pgfqpoint{1.152165in}{3.034675in}}%
\pgfpathlineto{\pgfqpoint{1.154737in}{3.067587in}}%
\pgfpathlineto{\pgfqpoint{1.157308in}{3.067587in}}%
\pgfpathlineto{\pgfqpoint{1.159880in}{3.084043in}}%
\pgfpathlineto{\pgfqpoint{1.167596in}{3.084043in}}%
\pgfpathlineto{\pgfqpoint{1.170168in}{3.067587in}}%
\pgfpathlineto{\pgfqpoint{1.172740in}{3.067587in}}%
\pgfpathlineto{\pgfqpoint{1.177883in}{3.100499in}}%
\pgfpathlineto{\pgfqpoint{1.183027in}{3.067587in}}%
\pgfpathlineto{\pgfqpoint{1.188171in}{3.067587in}}%
\pgfpathlineto{\pgfqpoint{1.190743in}{3.051131in}}%
\pgfpathlineto{\pgfqpoint{1.198458in}{3.051131in}}%
\pgfpathlineto{\pgfqpoint{1.201030in}{2.952395in}}%
\pgfpathlineto{\pgfqpoint{1.206174in}{2.952395in}}%
\pgfpathlineto{\pgfqpoint{1.208746in}{2.935938in}}%
\pgfpathlineto{\pgfqpoint{1.211318in}{2.968851in}}%
\pgfpathlineto{\pgfqpoint{1.213890in}{2.968851in}}%
\pgfpathlineto{\pgfqpoint{1.216461in}{2.919482in}}%
\pgfpathlineto{\pgfqpoint{1.219033in}{2.919482in}}%
\pgfpathlineto{\pgfqpoint{1.221605in}{2.985307in}}%
\pgfpathlineto{\pgfqpoint{1.224177in}{2.968851in}}%
\pgfpathlineto{\pgfqpoint{1.226749in}{2.903026in}}%
\pgfpathlineto{\pgfqpoint{1.234465in}{2.903026in}}%
\pgfpathlineto{\pgfqpoint{1.237036in}{2.935938in}}%
\pgfpathlineto{\pgfqpoint{1.239608in}{2.935938in}}%
\pgfpathlineto{\pgfqpoint{1.242180in}{2.903026in}}%
\pgfpathlineto{\pgfqpoint{1.244752in}{2.886570in}}%
\pgfpathlineto{\pgfqpoint{1.249896in}{2.886570in}}%
\pgfpathlineto{\pgfqpoint{1.255040in}{2.919482in}}%
\pgfpathlineto{\pgfqpoint{1.257611in}{2.919482in}}%
\pgfpathlineto{\pgfqpoint{1.260183in}{2.985307in}}%
\pgfpathlineto{\pgfqpoint{1.265327in}{2.952395in}}%
\pgfpathlineto{\pgfqpoint{1.267899in}{2.952395in}}%
\pgfpathlineto{\pgfqpoint{1.270471in}{2.935938in}}%
\pgfpathlineto{\pgfqpoint{1.275615in}{2.968851in}}%
\pgfpathlineto{\pgfqpoint{1.278186in}{2.968851in}}%
\pgfpathlineto{\pgfqpoint{1.280758in}{2.985307in}}%
\pgfpathlineto{\pgfqpoint{1.285902in}{2.985307in}}%
\pgfpathlineto{\pgfqpoint{1.288474in}{2.952395in}}%
\pgfpathlineto{\pgfqpoint{1.291046in}{3.001763in}}%
\pgfpathlineto{\pgfqpoint{1.293618in}{2.985307in}}%
\pgfpathlineto{\pgfqpoint{1.298761in}{2.985307in}}%
\pgfpathlineto{\pgfqpoint{1.301333in}{2.952395in}}%
\pgfpathlineto{\pgfqpoint{1.309049in}{2.952395in}}%
\pgfpathlineto{\pgfqpoint{1.311621in}{2.919482in}}%
\pgfpathlineto{\pgfqpoint{1.316764in}{2.919482in}}%
\pgfpathlineto{\pgfqpoint{1.319336in}{2.886570in}}%
\pgfpathlineto{\pgfqpoint{1.321908in}{2.886570in}}%
\pgfpathlineto{\pgfqpoint{1.324480in}{2.820746in}}%
\pgfpathlineto{\pgfqpoint{1.327052in}{2.853658in}}%
\pgfpathlineto{\pgfqpoint{1.329624in}{2.870114in}}%
\pgfpathlineto{\pgfqpoint{1.332196in}{2.870114in}}%
\pgfpathlineto{\pgfqpoint{1.334768in}{2.837202in}}%
\pgfpathlineto{\pgfqpoint{1.337339in}{2.837202in}}%
\pgfpathlineto{\pgfqpoint{1.339911in}{2.903026in}}%
\pgfpathlineto{\pgfqpoint{1.342483in}{2.886570in}}%
\pgfpathlineto{\pgfqpoint{1.345055in}{2.886570in}}%
\pgfpathlineto{\pgfqpoint{1.347627in}{2.870114in}}%
\pgfpathlineto{\pgfqpoint{1.350199in}{2.870114in}}%
\pgfpathlineto{\pgfqpoint{1.352771in}{2.853658in}}%
\pgfpathlineto{\pgfqpoint{1.355342in}{2.886570in}}%
\pgfpathlineto{\pgfqpoint{1.357914in}{2.870114in}}%
\pgfpathlineto{\pgfqpoint{1.365630in}{2.771378in}}%
\pgfpathlineto{\pgfqpoint{1.373346in}{2.771378in}}%
\pgfpathlineto{\pgfqpoint{1.378489in}{2.804290in}}%
\pgfpathlineto{\pgfqpoint{1.383633in}{2.804290in}}%
\pgfpathlineto{\pgfqpoint{1.386205in}{2.787834in}}%
\pgfpathlineto{\pgfqpoint{1.388777in}{2.853658in}}%
\pgfpathlineto{\pgfqpoint{1.391349in}{2.820746in}}%
\pgfpathlineto{\pgfqpoint{1.399064in}{2.820746in}}%
\pgfpathlineto{\pgfqpoint{1.401636in}{2.837202in}}%
\pgfpathlineto{\pgfqpoint{1.406780in}{2.837202in}}%
\pgfpathlineto{\pgfqpoint{1.409352in}{2.853658in}}%
\pgfpathlineto{\pgfqpoint{1.411924in}{2.886570in}}%
\pgfpathlineto{\pgfqpoint{1.417067in}{2.919482in}}%
\pgfpathlineto{\pgfqpoint{1.427355in}{2.919482in}}%
\pgfpathlineto{\pgfqpoint{1.429927in}{2.903026in}}%
\pgfpathlineto{\pgfqpoint{1.435070in}{2.787834in}}%
\pgfpathlineto{\pgfqpoint{1.440214in}{2.820746in}}%
\pgfpathlineto{\pgfqpoint{1.442786in}{2.853658in}}%
\pgfpathlineto{\pgfqpoint{1.445358in}{2.820746in}}%
\pgfpathlineto{\pgfqpoint{1.447930in}{2.804290in}}%
\pgfpathlineto{\pgfqpoint{1.450502in}{2.804290in}}%
\pgfpathlineto{\pgfqpoint{1.453074in}{2.820746in}}%
\pgfpathlineto{\pgfqpoint{1.455645in}{2.771378in}}%
\pgfpathlineto{\pgfqpoint{1.458217in}{2.787834in}}%
\pgfpathlineto{\pgfqpoint{1.460789in}{2.787834in}}%
\pgfpathlineto{\pgfqpoint{1.463361in}{2.754922in}}%
\pgfpathlineto{\pgfqpoint{1.465933in}{2.771378in}}%
\pgfpathlineto{\pgfqpoint{1.468505in}{2.771378in}}%
\pgfpathlineto{\pgfqpoint{1.473649in}{2.804290in}}%
\pgfpathlineto{\pgfqpoint{1.476220in}{2.804290in}}%
\pgfpathlineto{\pgfqpoint{1.478792in}{2.771378in}}%
\pgfpathlineto{\pgfqpoint{1.481364in}{2.771378in}}%
\pgfpathlineto{\pgfqpoint{1.483936in}{2.754922in}}%
\pgfpathlineto{\pgfqpoint{1.489080in}{2.754922in}}%
\pgfpathlineto{\pgfqpoint{1.494223in}{2.689097in}}%
\pgfpathlineto{\pgfqpoint{1.496795in}{2.722010in}}%
\pgfpathlineto{\pgfqpoint{1.499367in}{2.722010in}}%
\pgfpathlineto{\pgfqpoint{1.507083in}{2.771378in}}%
\pgfpathlineto{\pgfqpoint{1.509655in}{2.754922in}}%
\pgfpathlineto{\pgfqpoint{1.512227in}{2.787834in}}%
\pgfpathlineto{\pgfqpoint{1.514798in}{2.804290in}}%
\pgfpathlineto{\pgfqpoint{1.517370in}{2.853658in}}%
\pgfpathlineto{\pgfqpoint{1.519942in}{2.853658in}}%
\pgfpathlineto{\pgfqpoint{1.522514in}{2.820746in}}%
\pgfpathlineto{\pgfqpoint{1.530230in}{2.820746in}}%
\pgfpathlineto{\pgfqpoint{1.532802in}{2.771378in}}%
\pgfpathlineto{\pgfqpoint{1.535373in}{2.787834in}}%
\pgfpathlineto{\pgfqpoint{1.537945in}{2.787834in}}%
\pgfpathlineto{\pgfqpoint{1.540517in}{2.837202in}}%
\pgfpathlineto{\pgfqpoint{1.543089in}{2.820746in}}%
\pgfpathlineto{\pgfqpoint{1.545661in}{2.820746in}}%
\pgfpathlineto{\pgfqpoint{1.548233in}{2.837202in}}%
\pgfpathlineto{\pgfqpoint{1.550805in}{2.804290in}}%
\pgfpathlineto{\pgfqpoint{1.553376in}{2.804290in}}%
\pgfpathlineto{\pgfqpoint{1.555948in}{2.820746in}}%
\pgfpathlineto{\pgfqpoint{1.558520in}{2.820746in}}%
\pgfpathlineto{\pgfqpoint{1.563664in}{2.787834in}}%
\pgfpathlineto{\pgfqpoint{1.571380in}{2.787834in}}%
\pgfpathlineto{\pgfqpoint{1.576523in}{2.722010in}}%
\pgfpathlineto{\pgfqpoint{1.579095in}{2.722010in}}%
\pgfpathlineto{\pgfqpoint{1.581667in}{2.738466in}}%
\pgfpathlineto{\pgfqpoint{1.584239in}{2.722010in}}%
\pgfpathlineto{\pgfqpoint{1.586811in}{2.689097in}}%
\pgfpathlineto{\pgfqpoint{1.591955in}{2.689097in}}%
\pgfpathlineto{\pgfqpoint{1.594526in}{2.672641in}}%
\pgfpathlineto{\pgfqpoint{1.597098in}{2.639729in}}%
\pgfpathlineto{\pgfqpoint{1.604814in}{2.639729in}}%
\pgfpathlineto{\pgfqpoint{1.607386in}{2.623273in}}%
\pgfpathlineto{\pgfqpoint{1.609958in}{2.639729in}}%
\pgfpathlineto{\pgfqpoint{1.612529in}{2.606817in}}%
\pgfpathlineto{\pgfqpoint{1.615101in}{2.623273in}}%
\pgfpathlineto{\pgfqpoint{1.617673in}{2.606817in}}%
\pgfpathlineto{\pgfqpoint{1.620245in}{2.623273in}}%
\pgfpathlineto{\pgfqpoint{1.622817in}{2.623273in}}%
\pgfpathlineto{\pgfqpoint{1.627961in}{2.656185in}}%
\pgfpathlineto{\pgfqpoint{1.635676in}{2.606817in}}%
\pgfpathlineto{\pgfqpoint{1.638248in}{2.606817in}}%
\pgfpathlineto{\pgfqpoint{1.643392in}{2.540993in}}%
\pgfpathlineto{\pgfqpoint{1.645964in}{2.573905in}}%
\pgfpathlineto{\pgfqpoint{1.651108in}{2.573905in}}%
\pgfpathlineto{\pgfqpoint{1.653679in}{2.540993in}}%
\pgfpathlineto{\pgfqpoint{1.658823in}{2.573905in}}%
\pgfpathlineto{\pgfqpoint{1.661395in}{2.606817in}}%
\pgfpathlineto{\pgfqpoint{1.666539in}{2.606817in}}%
\pgfpathlineto{\pgfqpoint{1.669111in}{2.623273in}}%
\pgfpathlineto{\pgfqpoint{1.674254in}{2.623273in}}%
\pgfpathlineto{\pgfqpoint{1.676826in}{2.656185in}}%
\pgfpathlineto{\pgfqpoint{1.679398in}{2.672641in}}%
\pgfpathlineto{\pgfqpoint{1.681970in}{2.672641in}}%
\pgfpathlineto{\pgfqpoint{1.684542in}{2.656185in}}%
\pgfpathlineto{\pgfqpoint{1.687114in}{2.606817in}}%
\pgfpathlineto{\pgfqpoint{1.689686in}{2.623273in}}%
\pgfpathlineto{\pgfqpoint{1.692257in}{2.606817in}}%
\pgfpathlineto{\pgfqpoint{1.694829in}{2.606817in}}%
\pgfpathlineto{\pgfqpoint{1.697401in}{2.623273in}}%
\pgfpathlineto{\pgfqpoint{1.699973in}{2.606817in}}%
\pgfpathlineto{\pgfqpoint{1.702545in}{2.606817in}}%
\pgfpathlineto{\pgfqpoint{1.705117in}{2.590361in}}%
\pgfpathlineto{\pgfqpoint{1.707689in}{2.590361in}}%
\pgfpathlineto{\pgfqpoint{1.712832in}{2.524537in}}%
\pgfpathlineto{\pgfqpoint{1.715404in}{2.524537in}}%
\pgfpathlineto{\pgfqpoint{1.717976in}{2.475169in}}%
\pgfpathlineto{\pgfqpoint{1.720548in}{2.475169in}}%
\pgfpathlineto{\pgfqpoint{1.723120in}{2.491625in}}%
\pgfpathlineto{\pgfqpoint{1.725692in}{2.491625in}}%
\pgfpathlineto{\pgfqpoint{1.728264in}{2.508081in}}%
\pgfpathlineto{\pgfqpoint{1.730836in}{2.491625in}}%
\pgfpathlineto{\pgfqpoint{1.735979in}{2.491625in}}%
\pgfpathlineto{\pgfqpoint{1.738551in}{2.458712in}}%
\pgfpathlineto{\pgfqpoint{1.741123in}{2.475169in}}%
\pgfpathlineto{\pgfqpoint{1.746267in}{2.442256in}}%
\pgfpathlineto{\pgfqpoint{1.751410in}{2.442256in}}%
\pgfpathlineto{\pgfqpoint{1.753982in}{2.425800in}}%
\pgfpathlineto{\pgfqpoint{1.759126in}{2.425800in}}%
\pgfpathlineto{\pgfqpoint{1.761698in}{2.442256in}}%
\pgfpathlineto{\pgfqpoint{1.764270in}{2.442256in}}%
\pgfpathlineto{\pgfqpoint{1.766842in}{2.425800in}}%
\pgfpathlineto{\pgfqpoint{1.774557in}{2.425800in}}%
\pgfpathlineto{\pgfqpoint{1.779701in}{2.392888in}}%
\pgfpathlineto{\pgfqpoint{1.792560in}{2.392888in}}%
\pgfpathlineto{\pgfqpoint{1.795132in}{2.376432in}}%
\pgfpathlineto{\pgfqpoint{1.825995in}{2.376432in}}%
\pgfpathlineto{\pgfqpoint{1.828567in}{2.359976in}}%
\pgfpathlineto{\pgfqpoint{1.877432in}{2.359976in}}%
\pgfpathlineto{\pgfqpoint{1.880004in}{2.376432in}}%
\pgfpathlineto{\pgfqpoint{1.887720in}{2.376432in}}%
\pgfpathlineto{\pgfqpoint{1.890291in}{2.359976in}}%
\pgfpathlineto{\pgfqpoint{1.898007in}{2.359976in}}%
\pgfpathlineto{\pgfqpoint{1.900579in}{2.343520in}}%
\pgfpathlineto{\pgfqpoint{1.903151in}{2.343520in}}%
\pgfpathlineto{\pgfqpoint{1.905723in}{2.310608in}}%
\pgfpathlineto{\pgfqpoint{1.908295in}{2.310608in}}%
\pgfpathlineto{\pgfqpoint{1.910866in}{2.294152in}}%
\pgfpathlineto{\pgfqpoint{1.941729in}{2.294152in}}%
\pgfpathlineto{\pgfqpoint{1.944301in}{2.277696in}}%
\pgfpathlineto{\pgfqpoint{1.972591in}{2.277696in}}%
\pgfpathlineto{\pgfqpoint{1.975163in}{2.261240in}}%
\pgfpathlineto{\pgfqpoint{2.144907in}{2.261240in}}%
\pgfpathlineto{\pgfqpoint{2.147478in}{2.277696in}}%
\pgfpathlineto{\pgfqpoint{2.268356in}{2.277696in}}%
\pgfpathlineto{\pgfqpoint{2.270928in}{2.261240in}}%
\pgfpathlineto{\pgfqpoint{2.294075in}{2.261240in}}%
\pgfpathlineto{\pgfqpoint{2.296647in}{2.244784in}}%
\pgfpathlineto{\pgfqpoint{2.299219in}{2.244784in}}%
\pgfpathlineto{\pgfqpoint{2.301791in}{2.228328in}}%
\pgfpathlineto{\pgfqpoint{2.312078in}{2.228328in}}%
\pgfpathlineto{\pgfqpoint{2.314650in}{2.211871in}}%
\pgfpathlineto{\pgfqpoint{2.317222in}{2.211871in}}%
\pgfpathlineto{\pgfqpoint{2.319794in}{2.195415in}}%
\pgfpathlineto{\pgfqpoint{2.337797in}{2.195415in}}%
\pgfpathlineto{\pgfqpoint{2.340369in}{2.178959in}}%
\pgfpathlineto{\pgfqpoint{2.376375in}{2.178959in}}%
\pgfpathlineto{\pgfqpoint{2.378947in}{2.162503in}}%
\pgfpathlineto{\pgfqpoint{2.402094in}{2.162503in}}%
\pgfpathlineto{\pgfqpoint{2.404665in}{2.146047in}}%
\pgfpathlineto{\pgfqpoint{2.407237in}{2.146047in}}%
\pgfpathlineto{\pgfqpoint{2.409809in}{2.129591in}}%
\pgfpathlineto{\pgfqpoint{2.430384in}{2.129591in}}%
\pgfpathlineto{\pgfqpoint{2.432956in}{2.113135in}}%
\pgfpathlineto{\pgfqpoint{2.448387in}{2.113135in}}%
\pgfpathlineto{\pgfqpoint{2.450959in}{2.096679in}}%
\pgfpathlineto{\pgfqpoint{2.528115in}{2.096679in}}%
\pgfpathlineto{\pgfqpoint{2.530687in}{2.080223in}}%
\pgfpathlineto{\pgfqpoint{2.610415in}{2.080223in}}%
\pgfpathlineto{\pgfqpoint{2.612987in}{2.096679in}}%
\pgfpathlineto{\pgfqpoint{2.618131in}{2.096679in}}%
\pgfpathlineto{\pgfqpoint{2.620703in}{2.113135in}}%
\pgfpathlineto{\pgfqpoint{2.646421in}{2.113135in}}%
\pgfpathlineto{\pgfqpoint{2.646421in}{2.113135in}}%
\pgfusepath{stroke}%
\end{pgfscope}%
\begin{pgfscope}%
\pgfpathrectangle{\pgfqpoint{0.488751in}{1.946106in}}{\pgfqpoint{2.260417in}{1.502439in}}%
\pgfusepath{clip}%
\pgfsetrectcap%
\pgfsetroundjoin%
\pgfsetlinewidth{0.803000pt}%
\definecolor{currentstroke}{rgb}{0.490196,0.588235,0.431373}%
\pgfsetstrokecolor{currentstroke}%
\pgfsetdash{}{0pt}%
\pgfpathmoveto{\pgfqpoint{0.591497in}{2.425800in}}%
\pgfpathlineto{\pgfqpoint{0.594069in}{2.425800in}}%
\pgfpathlineto{\pgfqpoint{0.596641in}{2.409344in}}%
\pgfpathlineto{\pgfqpoint{0.614644in}{2.409344in}}%
\pgfpathlineto{\pgfqpoint{0.619788in}{2.376432in}}%
\pgfpathlineto{\pgfqpoint{0.622359in}{2.376432in}}%
\pgfpathlineto{\pgfqpoint{0.624931in}{2.359976in}}%
\pgfpathlineto{\pgfqpoint{0.630075in}{2.359976in}}%
\pgfpathlineto{\pgfqpoint{0.632647in}{2.327064in}}%
\pgfpathlineto{\pgfqpoint{0.635219in}{2.310608in}}%
\pgfpathlineto{\pgfqpoint{0.750953in}{2.310608in}}%
\pgfpathlineto{\pgfqpoint{0.753525in}{2.294152in}}%
\pgfpathlineto{\pgfqpoint{0.830681in}{2.294152in}}%
\pgfpathlineto{\pgfqpoint{0.833253in}{2.277696in}}%
\pgfpathlineto{\pgfqpoint{0.897550in}{2.277696in}}%
\pgfpathlineto{\pgfqpoint{0.900121in}{2.261240in}}%
\pgfpathlineto{\pgfqpoint{1.198458in}{2.261240in}}%
\pgfpathlineto{\pgfqpoint{1.201030in}{2.244784in}}%
\pgfpathlineto{\pgfqpoint{1.296189in}{2.244784in}}%
\pgfpathlineto{\pgfqpoint{1.298761in}{2.228328in}}%
\pgfpathlineto{\pgfqpoint{1.710261in}{2.228328in}}%
\pgfpathlineto{\pgfqpoint{1.712832in}{2.211871in}}%
\pgfpathlineto{\pgfqpoint{1.867145in}{2.211871in}}%
\pgfpathlineto{\pgfqpoint{1.869717in}{2.195415in}}%
\pgfpathlineto{\pgfqpoint{2.070322in}{2.195415in}}%
\pgfpathlineto{\pgfqpoint{2.072894in}{2.178959in}}%
\pgfpathlineto{\pgfqpoint{2.317222in}{2.178959in}}%
\pgfpathlineto{\pgfqpoint{2.319794in}{2.162503in}}%
\pgfpathlineto{\pgfqpoint{2.430384in}{2.162503in}}%
\pgfpathlineto{\pgfqpoint{2.432956in}{2.146047in}}%
\pgfpathlineto{\pgfqpoint{2.633562in}{2.146047in}}%
\pgfpathlineto{\pgfqpoint{2.636134in}{2.129591in}}%
\pgfpathlineto{\pgfqpoint{2.646421in}{2.129591in}}%
\pgfpathlineto{\pgfqpoint{2.646421in}{2.129591in}}%
\pgfusepath{stroke}%
\end{pgfscope}%
\begin{pgfscope}%
\pgfpathrectangle{\pgfqpoint{0.488751in}{1.946106in}}{\pgfqpoint{2.260417in}{1.502439in}}%
\pgfusepath{clip}%
\pgfsetrectcap%
\pgfsetroundjoin%
\pgfsetlinewidth{0.803000pt}%
\definecolor{currentstroke}{rgb}{0.843137,0.666667,0.313725}%
\pgfsetstrokecolor{currentstroke}%
\pgfsetdash{}{0pt}%
\pgfpathmoveto{\pgfqpoint{0.591497in}{2.376432in}}%
\pgfpathlineto{\pgfqpoint{0.594069in}{2.442256in}}%
\pgfpathlineto{\pgfqpoint{0.596641in}{2.442256in}}%
\pgfpathlineto{\pgfqpoint{0.599213in}{2.409344in}}%
\pgfpathlineto{\pgfqpoint{0.601785in}{2.425800in}}%
\pgfpathlineto{\pgfqpoint{0.604356in}{2.508081in}}%
\pgfpathlineto{\pgfqpoint{0.606928in}{2.540993in}}%
\pgfpathlineto{\pgfqpoint{0.609500in}{2.524537in}}%
\pgfpathlineto{\pgfqpoint{0.612072in}{2.557449in}}%
\pgfpathlineto{\pgfqpoint{0.614644in}{2.606817in}}%
\pgfpathlineto{\pgfqpoint{0.617216in}{2.606817in}}%
\pgfpathlineto{\pgfqpoint{0.619788in}{2.656185in}}%
\pgfpathlineto{\pgfqpoint{0.622359in}{2.672641in}}%
\pgfpathlineto{\pgfqpoint{0.624931in}{2.639729in}}%
\pgfpathlineto{\pgfqpoint{0.627503in}{2.672641in}}%
\pgfpathlineto{\pgfqpoint{0.630075in}{2.590361in}}%
\pgfpathlineto{\pgfqpoint{0.632647in}{2.557449in}}%
\pgfpathlineto{\pgfqpoint{0.635219in}{2.540993in}}%
\pgfpathlineto{\pgfqpoint{0.637791in}{2.590361in}}%
\pgfpathlineto{\pgfqpoint{0.642934in}{2.590361in}}%
\pgfpathlineto{\pgfqpoint{0.645506in}{2.623273in}}%
\pgfpathlineto{\pgfqpoint{0.648078in}{2.606817in}}%
\pgfpathlineto{\pgfqpoint{0.650650in}{2.639729in}}%
\pgfpathlineto{\pgfqpoint{0.653222in}{2.623273in}}%
\pgfpathlineto{\pgfqpoint{0.655794in}{2.590361in}}%
\pgfpathlineto{\pgfqpoint{0.658366in}{2.689097in}}%
\pgfpathlineto{\pgfqpoint{0.660938in}{2.722010in}}%
\pgfpathlineto{\pgfqpoint{0.663509in}{2.705554in}}%
\pgfpathlineto{\pgfqpoint{0.666081in}{2.738466in}}%
\pgfpathlineto{\pgfqpoint{0.668653in}{2.787834in}}%
\pgfpathlineto{\pgfqpoint{0.671225in}{2.771378in}}%
\pgfpathlineto{\pgfqpoint{0.673797in}{2.771378in}}%
\pgfpathlineto{\pgfqpoint{0.676369in}{2.787834in}}%
\pgfpathlineto{\pgfqpoint{0.678941in}{2.771378in}}%
\pgfpathlineto{\pgfqpoint{0.681513in}{2.837202in}}%
\pgfpathlineto{\pgfqpoint{0.684084in}{2.771378in}}%
\pgfpathlineto{\pgfqpoint{0.686656in}{2.820746in}}%
\pgfpathlineto{\pgfqpoint{0.689228in}{2.705554in}}%
\pgfpathlineto{\pgfqpoint{0.691800in}{2.672641in}}%
\pgfpathlineto{\pgfqpoint{0.694372in}{2.656185in}}%
\pgfpathlineto{\pgfqpoint{0.696944in}{2.656185in}}%
\pgfpathlineto{\pgfqpoint{0.699516in}{2.639729in}}%
\pgfpathlineto{\pgfqpoint{0.707231in}{2.639729in}}%
\pgfpathlineto{\pgfqpoint{0.709803in}{2.623273in}}%
\pgfpathlineto{\pgfqpoint{0.725234in}{2.623273in}}%
\pgfpathlineto{\pgfqpoint{0.727806in}{2.639729in}}%
\pgfpathlineto{\pgfqpoint{0.750953in}{2.639729in}}%
\pgfpathlineto{\pgfqpoint{0.753525in}{2.656185in}}%
\pgfpathlineto{\pgfqpoint{0.756097in}{2.656185in}}%
\pgfpathlineto{\pgfqpoint{0.758669in}{2.639729in}}%
\pgfpathlineto{\pgfqpoint{0.781815in}{2.639729in}}%
\pgfpathlineto{\pgfqpoint{0.784387in}{2.623273in}}%
\pgfpathlineto{\pgfqpoint{0.789531in}{2.623273in}}%
\pgfpathlineto{\pgfqpoint{0.792103in}{2.606817in}}%
\pgfpathlineto{\pgfqpoint{0.843540in}{2.606817in}}%
\pgfpathlineto{\pgfqpoint{0.846112in}{2.623273in}}%
\pgfpathlineto{\pgfqpoint{0.848684in}{2.623273in}}%
\pgfpathlineto{\pgfqpoint{0.853828in}{2.656185in}}%
\pgfpathlineto{\pgfqpoint{0.856400in}{2.689097in}}%
\pgfpathlineto{\pgfqpoint{0.861543in}{2.722010in}}%
\pgfpathlineto{\pgfqpoint{0.864115in}{2.689097in}}%
\pgfpathlineto{\pgfqpoint{0.866687in}{2.722010in}}%
\pgfpathlineto{\pgfqpoint{0.869259in}{2.722010in}}%
\pgfpathlineto{\pgfqpoint{0.871831in}{2.705554in}}%
\pgfpathlineto{\pgfqpoint{0.874403in}{2.606817in}}%
\pgfpathlineto{\pgfqpoint{0.879547in}{2.639729in}}%
\pgfpathlineto{\pgfqpoint{0.884690in}{2.639729in}}%
\pgfpathlineto{\pgfqpoint{0.887262in}{2.623273in}}%
\pgfpathlineto{\pgfqpoint{0.889834in}{2.639729in}}%
\pgfpathlineto{\pgfqpoint{0.894978in}{2.639729in}}%
\pgfpathlineto{\pgfqpoint{0.897550in}{2.623273in}}%
\pgfpathlineto{\pgfqpoint{0.915553in}{2.623273in}}%
\pgfpathlineto{\pgfqpoint{0.918125in}{2.639729in}}%
\pgfpathlineto{\pgfqpoint{0.923268in}{2.639729in}}%
\pgfpathlineto{\pgfqpoint{0.928412in}{2.672641in}}%
\pgfpathlineto{\pgfqpoint{0.930984in}{2.672641in}}%
\pgfpathlineto{\pgfqpoint{0.933556in}{2.656185in}}%
\pgfpathlineto{\pgfqpoint{0.936128in}{2.656185in}}%
\pgfpathlineto{\pgfqpoint{0.938700in}{2.639729in}}%
\pgfpathlineto{\pgfqpoint{1.015856in}{2.639729in}}%
\pgfpathlineto{\pgfqpoint{1.018427in}{2.656185in}}%
\pgfpathlineto{\pgfqpoint{1.031287in}{2.656185in}}%
\pgfpathlineto{\pgfqpoint{1.033859in}{2.672641in}}%
\pgfpathlineto{\pgfqpoint{1.036431in}{2.672641in}}%
\pgfpathlineto{\pgfqpoint{1.039002in}{2.705554in}}%
\pgfpathlineto{\pgfqpoint{1.041574in}{2.722010in}}%
\pgfpathlineto{\pgfqpoint{1.044146in}{2.672641in}}%
\pgfpathlineto{\pgfqpoint{1.046718in}{2.705554in}}%
\pgfpathlineto{\pgfqpoint{1.049290in}{2.705554in}}%
\pgfpathlineto{\pgfqpoint{1.051862in}{2.722010in}}%
\pgfpathlineto{\pgfqpoint{1.054434in}{2.705554in}}%
\pgfpathlineto{\pgfqpoint{1.057006in}{2.656185in}}%
\pgfpathlineto{\pgfqpoint{1.064721in}{2.656185in}}%
\pgfpathlineto{\pgfqpoint{1.067293in}{2.689097in}}%
\pgfpathlineto{\pgfqpoint{1.069865in}{2.656185in}}%
\pgfpathlineto{\pgfqpoint{1.085296in}{2.656185in}}%
\pgfpathlineto{\pgfqpoint{1.087868in}{2.672641in}}%
\pgfpathlineto{\pgfqpoint{1.095584in}{2.672641in}}%
\pgfpathlineto{\pgfqpoint{1.098155in}{2.656185in}}%
\pgfpathlineto{\pgfqpoint{1.108443in}{2.656185in}}%
\pgfpathlineto{\pgfqpoint{1.111015in}{2.672641in}}%
\pgfpathlineto{\pgfqpoint{1.118730in}{2.672641in}}%
\pgfpathlineto{\pgfqpoint{1.121302in}{2.639729in}}%
\pgfpathlineto{\pgfqpoint{1.123874in}{2.623273in}}%
\pgfpathlineto{\pgfqpoint{1.126446in}{2.623273in}}%
\pgfpathlineto{\pgfqpoint{1.129018in}{2.639729in}}%
\pgfpathlineto{\pgfqpoint{1.131590in}{2.639729in}}%
\pgfpathlineto{\pgfqpoint{1.134162in}{2.656185in}}%
\pgfpathlineto{\pgfqpoint{1.152165in}{2.656185in}}%
\pgfpathlineto{\pgfqpoint{1.154737in}{2.672641in}}%
\pgfpathlineto{\pgfqpoint{1.162452in}{2.672641in}}%
\pgfpathlineto{\pgfqpoint{1.165024in}{2.722010in}}%
\pgfpathlineto{\pgfqpoint{1.167596in}{2.705554in}}%
\pgfpathlineto{\pgfqpoint{1.170168in}{2.738466in}}%
\pgfpathlineto{\pgfqpoint{1.172740in}{2.705554in}}%
\pgfpathlineto{\pgfqpoint{1.175312in}{2.705554in}}%
\pgfpathlineto{\pgfqpoint{1.180455in}{2.672641in}}%
\pgfpathlineto{\pgfqpoint{1.188171in}{2.672641in}}%
\pgfpathlineto{\pgfqpoint{1.190743in}{2.705554in}}%
\pgfpathlineto{\pgfqpoint{1.198458in}{2.705554in}}%
\pgfpathlineto{\pgfqpoint{1.201030in}{2.738466in}}%
\pgfpathlineto{\pgfqpoint{1.203602in}{2.738466in}}%
\pgfpathlineto{\pgfqpoint{1.206174in}{2.722010in}}%
\pgfpathlineto{\pgfqpoint{1.208746in}{2.738466in}}%
\pgfpathlineto{\pgfqpoint{1.211318in}{2.705554in}}%
\pgfpathlineto{\pgfqpoint{1.213890in}{2.689097in}}%
\pgfpathlineto{\pgfqpoint{1.216461in}{2.738466in}}%
\pgfpathlineto{\pgfqpoint{1.219033in}{2.738466in}}%
\pgfpathlineto{\pgfqpoint{1.221605in}{2.689097in}}%
\pgfpathlineto{\pgfqpoint{1.224177in}{2.689097in}}%
\pgfpathlineto{\pgfqpoint{1.226749in}{2.705554in}}%
\pgfpathlineto{\pgfqpoint{1.242180in}{2.705554in}}%
\pgfpathlineto{\pgfqpoint{1.244752in}{2.738466in}}%
\pgfpathlineto{\pgfqpoint{1.249896in}{2.738466in}}%
\pgfpathlineto{\pgfqpoint{1.252468in}{2.722010in}}%
\pgfpathlineto{\pgfqpoint{1.257611in}{2.722010in}}%
\pgfpathlineto{\pgfqpoint{1.260183in}{2.754922in}}%
\pgfpathlineto{\pgfqpoint{1.262755in}{2.754922in}}%
\pgfpathlineto{\pgfqpoint{1.265327in}{2.804290in}}%
\pgfpathlineto{\pgfqpoint{1.267899in}{2.804290in}}%
\pgfpathlineto{\pgfqpoint{1.270471in}{2.870114in}}%
\pgfpathlineto{\pgfqpoint{1.273043in}{2.837202in}}%
\pgfpathlineto{\pgfqpoint{1.278186in}{2.804290in}}%
\pgfpathlineto{\pgfqpoint{1.280758in}{2.804290in}}%
\pgfpathlineto{\pgfqpoint{1.283330in}{2.787834in}}%
\pgfpathlineto{\pgfqpoint{1.285902in}{2.787834in}}%
\pgfpathlineto{\pgfqpoint{1.288474in}{2.820746in}}%
\pgfpathlineto{\pgfqpoint{1.291046in}{2.787834in}}%
\pgfpathlineto{\pgfqpoint{1.293618in}{2.820746in}}%
\pgfpathlineto{\pgfqpoint{1.301333in}{2.771378in}}%
\pgfpathlineto{\pgfqpoint{1.309049in}{2.771378in}}%
\pgfpathlineto{\pgfqpoint{1.311621in}{2.787834in}}%
\pgfpathlineto{\pgfqpoint{1.319336in}{2.787834in}}%
\pgfpathlineto{\pgfqpoint{1.321908in}{2.771378in}}%
\pgfpathlineto{\pgfqpoint{1.329624in}{2.771378in}}%
\pgfpathlineto{\pgfqpoint{1.332196in}{2.804290in}}%
\pgfpathlineto{\pgfqpoint{1.334768in}{2.804290in}}%
\pgfpathlineto{\pgfqpoint{1.337339in}{2.787834in}}%
\pgfpathlineto{\pgfqpoint{1.339911in}{2.722010in}}%
\pgfpathlineto{\pgfqpoint{1.342483in}{2.705554in}}%
\pgfpathlineto{\pgfqpoint{1.345055in}{2.705554in}}%
\pgfpathlineto{\pgfqpoint{1.347627in}{2.754922in}}%
\pgfpathlineto{\pgfqpoint{1.350199in}{2.738466in}}%
\pgfpathlineto{\pgfqpoint{1.352771in}{2.771378in}}%
\pgfpathlineto{\pgfqpoint{1.355342in}{2.738466in}}%
\pgfpathlineto{\pgfqpoint{1.357914in}{2.722010in}}%
\pgfpathlineto{\pgfqpoint{1.360486in}{2.738466in}}%
\pgfpathlineto{\pgfqpoint{1.373346in}{2.738466in}}%
\pgfpathlineto{\pgfqpoint{1.375917in}{2.722010in}}%
\pgfpathlineto{\pgfqpoint{1.386205in}{2.722010in}}%
\pgfpathlineto{\pgfqpoint{1.388777in}{2.689097in}}%
\pgfpathlineto{\pgfqpoint{1.406780in}{2.689097in}}%
\pgfpathlineto{\pgfqpoint{1.409352in}{2.722010in}}%
\pgfpathlineto{\pgfqpoint{1.417067in}{2.722010in}}%
\pgfpathlineto{\pgfqpoint{1.422211in}{2.689097in}}%
\pgfpathlineto{\pgfqpoint{1.427355in}{2.689097in}}%
\pgfpathlineto{\pgfqpoint{1.429927in}{2.705554in}}%
\pgfpathlineto{\pgfqpoint{1.432499in}{2.705554in}}%
\pgfpathlineto{\pgfqpoint{1.435070in}{2.722010in}}%
\pgfpathlineto{\pgfqpoint{1.437642in}{2.722010in}}%
\pgfpathlineto{\pgfqpoint{1.440214in}{2.705554in}}%
\pgfpathlineto{\pgfqpoint{1.442786in}{2.705554in}}%
\pgfpathlineto{\pgfqpoint{1.445358in}{2.656185in}}%
\pgfpathlineto{\pgfqpoint{1.447930in}{2.639729in}}%
\pgfpathlineto{\pgfqpoint{1.453074in}{2.639729in}}%
\pgfpathlineto{\pgfqpoint{1.455645in}{2.672641in}}%
\pgfpathlineto{\pgfqpoint{1.460789in}{2.672641in}}%
\pgfpathlineto{\pgfqpoint{1.463361in}{2.639729in}}%
\pgfpathlineto{\pgfqpoint{1.468505in}{2.639729in}}%
\pgfpathlineto{\pgfqpoint{1.471077in}{2.623273in}}%
\pgfpathlineto{\pgfqpoint{1.476220in}{2.623273in}}%
\pgfpathlineto{\pgfqpoint{1.478792in}{2.639729in}}%
\pgfpathlineto{\pgfqpoint{1.481364in}{2.639729in}}%
\pgfpathlineto{\pgfqpoint{1.483936in}{2.606817in}}%
\pgfpathlineto{\pgfqpoint{1.486508in}{2.623273in}}%
\pgfpathlineto{\pgfqpoint{1.489080in}{2.623273in}}%
\pgfpathlineto{\pgfqpoint{1.491652in}{2.656185in}}%
\pgfpathlineto{\pgfqpoint{1.494223in}{2.672641in}}%
\pgfpathlineto{\pgfqpoint{1.499367in}{2.672641in}}%
\pgfpathlineto{\pgfqpoint{1.501939in}{2.656185in}}%
\pgfpathlineto{\pgfqpoint{1.504511in}{2.689097in}}%
\pgfpathlineto{\pgfqpoint{1.512227in}{2.689097in}}%
\pgfpathlineto{\pgfqpoint{1.514798in}{2.705554in}}%
\pgfpathlineto{\pgfqpoint{1.517370in}{2.689097in}}%
\pgfpathlineto{\pgfqpoint{1.519942in}{2.705554in}}%
\pgfpathlineto{\pgfqpoint{1.522514in}{2.689097in}}%
\pgfpathlineto{\pgfqpoint{1.525086in}{2.656185in}}%
\pgfpathlineto{\pgfqpoint{1.532802in}{2.656185in}}%
\pgfpathlineto{\pgfqpoint{1.535373in}{2.639729in}}%
\pgfpathlineto{\pgfqpoint{1.543089in}{2.639729in}}%
\pgfpathlineto{\pgfqpoint{1.545661in}{2.672641in}}%
\pgfpathlineto{\pgfqpoint{1.548233in}{2.672641in}}%
\pgfpathlineto{\pgfqpoint{1.550805in}{2.689097in}}%
\pgfpathlineto{\pgfqpoint{1.553376in}{2.689097in}}%
\pgfpathlineto{\pgfqpoint{1.555948in}{2.672641in}}%
\pgfpathlineto{\pgfqpoint{1.566236in}{2.672641in}}%
\pgfpathlineto{\pgfqpoint{1.568808in}{2.689097in}}%
\pgfpathlineto{\pgfqpoint{1.576523in}{2.639729in}}%
\pgfpathlineto{\pgfqpoint{1.579095in}{2.656185in}}%
\pgfpathlineto{\pgfqpoint{1.591955in}{2.656185in}}%
\pgfpathlineto{\pgfqpoint{1.594526in}{2.639729in}}%
\pgfpathlineto{\pgfqpoint{1.604814in}{2.639729in}}%
\pgfpathlineto{\pgfqpoint{1.607386in}{2.656185in}}%
\pgfpathlineto{\pgfqpoint{1.617673in}{2.656185in}}%
\pgfpathlineto{\pgfqpoint{1.620245in}{2.623273in}}%
\pgfpathlineto{\pgfqpoint{1.622817in}{2.623273in}}%
\pgfpathlineto{\pgfqpoint{1.625389in}{2.639729in}}%
\pgfpathlineto{\pgfqpoint{1.630533in}{2.639729in}}%
\pgfpathlineto{\pgfqpoint{1.633104in}{2.656185in}}%
\pgfpathlineto{\pgfqpoint{1.638248in}{2.656185in}}%
\pgfpathlineto{\pgfqpoint{1.640820in}{2.689097in}}%
\pgfpathlineto{\pgfqpoint{1.643392in}{2.689097in}}%
\pgfpathlineto{\pgfqpoint{1.645964in}{2.722010in}}%
\pgfpathlineto{\pgfqpoint{1.648536in}{2.689097in}}%
\pgfpathlineto{\pgfqpoint{1.651108in}{2.672641in}}%
\pgfpathlineto{\pgfqpoint{1.653679in}{2.689097in}}%
\pgfpathlineto{\pgfqpoint{1.656251in}{2.672641in}}%
\pgfpathlineto{\pgfqpoint{1.658823in}{2.672641in}}%
\pgfpathlineto{\pgfqpoint{1.661395in}{2.656185in}}%
\pgfpathlineto{\pgfqpoint{1.663967in}{2.689097in}}%
\pgfpathlineto{\pgfqpoint{1.674254in}{2.689097in}}%
\pgfpathlineto{\pgfqpoint{1.676826in}{2.738466in}}%
\pgfpathlineto{\pgfqpoint{1.679398in}{2.689097in}}%
\pgfpathlineto{\pgfqpoint{1.681970in}{2.689097in}}%
\pgfpathlineto{\pgfqpoint{1.684542in}{2.705554in}}%
\pgfpathlineto{\pgfqpoint{1.687114in}{2.689097in}}%
\pgfpathlineto{\pgfqpoint{1.692257in}{2.689097in}}%
\pgfpathlineto{\pgfqpoint{1.694829in}{2.672641in}}%
\pgfpathlineto{\pgfqpoint{1.699973in}{2.705554in}}%
\pgfpathlineto{\pgfqpoint{1.702545in}{2.705554in}}%
\pgfpathlineto{\pgfqpoint{1.705117in}{2.738466in}}%
\pgfpathlineto{\pgfqpoint{1.707689in}{2.722010in}}%
\pgfpathlineto{\pgfqpoint{1.710261in}{2.738466in}}%
\pgfpathlineto{\pgfqpoint{1.712832in}{2.787834in}}%
\pgfpathlineto{\pgfqpoint{1.715404in}{2.804290in}}%
\pgfpathlineto{\pgfqpoint{1.717976in}{2.886570in}}%
\pgfpathlineto{\pgfqpoint{1.720548in}{2.903026in}}%
\pgfpathlineto{\pgfqpoint{1.723120in}{2.903026in}}%
\pgfpathlineto{\pgfqpoint{1.725692in}{2.935938in}}%
\pgfpathlineto{\pgfqpoint{1.728264in}{2.919482in}}%
\pgfpathlineto{\pgfqpoint{1.733407in}{2.952395in}}%
\pgfpathlineto{\pgfqpoint{1.735979in}{2.935938in}}%
\pgfpathlineto{\pgfqpoint{1.738551in}{2.985307in}}%
\pgfpathlineto{\pgfqpoint{1.746267in}{2.985307in}}%
\pgfpathlineto{\pgfqpoint{1.748839in}{3.001763in}}%
\pgfpathlineto{\pgfqpoint{1.751410in}{3.001763in}}%
\pgfpathlineto{\pgfqpoint{1.753982in}{2.985307in}}%
\pgfpathlineto{\pgfqpoint{1.756554in}{2.985307in}}%
\pgfpathlineto{\pgfqpoint{1.759126in}{2.968851in}}%
\pgfpathlineto{\pgfqpoint{1.761698in}{2.968851in}}%
\pgfpathlineto{\pgfqpoint{1.764270in}{2.952395in}}%
\pgfpathlineto{\pgfqpoint{1.766842in}{2.952395in}}%
\pgfpathlineto{\pgfqpoint{1.769414in}{2.919482in}}%
\pgfpathlineto{\pgfqpoint{1.774557in}{2.886570in}}%
\pgfpathlineto{\pgfqpoint{1.779701in}{2.985307in}}%
\pgfpathlineto{\pgfqpoint{1.782273in}{2.985307in}}%
\pgfpathlineto{\pgfqpoint{1.784845in}{3.001763in}}%
\pgfpathlineto{\pgfqpoint{1.787417in}{2.985307in}}%
\pgfpathlineto{\pgfqpoint{1.792560in}{2.985307in}}%
\pgfpathlineto{\pgfqpoint{1.795132in}{3.001763in}}%
\pgfpathlineto{\pgfqpoint{1.797704in}{3.001763in}}%
\pgfpathlineto{\pgfqpoint{1.800276in}{3.034675in}}%
\pgfpathlineto{\pgfqpoint{1.802848in}{3.051131in}}%
\pgfpathlineto{\pgfqpoint{1.805420in}{3.051131in}}%
\pgfpathlineto{\pgfqpoint{1.807992in}{3.067587in}}%
\pgfpathlineto{\pgfqpoint{1.810563in}{3.051131in}}%
\pgfpathlineto{\pgfqpoint{1.818279in}{3.100499in}}%
\pgfpathlineto{\pgfqpoint{1.820851in}{3.100499in}}%
\pgfpathlineto{\pgfqpoint{1.823423in}{3.133411in}}%
\pgfpathlineto{\pgfqpoint{1.825995in}{3.100499in}}%
\pgfpathlineto{\pgfqpoint{1.828567in}{3.100499in}}%
\pgfpathlineto{\pgfqpoint{1.831138in}{3.084043in}}%
\pgfpathlineto{\pgfqpoint{1.838854in}{3.084043in}}%
\pgfpathlineto{\pgfqpoint{1.841426in}{3.067587in}}%
\pgfpathlineto{\pgfqpoint{1.843998in}{3.084043in}}%
\pgfpathlineto{\pgfqpoint{1.862001in}{3.084043in}}%
\pgfpathlineto{\pgfqpoint{1.864573in}{3.100499in}}%
\pgfpathlineto{\pgfqpoint{1.867145in}{3.084043in}}%
\pgfpathlineto{\pgfqpoint{1.874860in}{3.084043in}}%
\pgfpathlineto{\pgfqpoint{1.877432in}{3.100499in}}%
\pgfpathlineto{\pgfqpoint{1.880004in}{3.084043in}}%
\pgfpathlineto{\pgfqpoint{1.885148in}{3.084043in}}%
\pgfpathlineto{\pgfqpoint{1.887720in}{3.100499in}}%
\pgfpathlineto{\pgfqpoint{1.890291in}{3.100499in}}%
\pgfpathlineto{\pgfqpoint{1.895435in}{3.067587in}}%
\pgfpathlineto{\pgfqpoint{1.900579in}{3.133411in}}%
\pgfpathlineto{\pgfqpoint{1.903151in}{3.100499in}}%
\pgfpathlineto{\pgfqpoint{1.905723in}{3.116955in}}%
\pgfpathlineto{\pgfqpoint{1.908295in}{3.100499in}}%
\pgfpathlineto{\pgfqpoint{1.921154in}{3.100499in}}%
\pgfpathlineto{\pgfqpoint{1.923726in}{3.116955in}}%
\pgfpathlineto{\pgfqpoint{1.926298in}{3.067587in}}%
\pgfpathlineto{\pgfqpoint{1.928870in}{3.067587in}}%
\pgfpathlineto{\pgfqpoint{1.934013in}{3.100499in}}%
\pgfpathlineto{\pgfqpoint{1.936585in}{3.100499in}}%
\pgfpathlineto{\pgfqpoint{1.939157in}{3.116955in}}%
\pgfpathlineto{\pgfqpoint{1.941729in}{3.100499in}}%
\pgfpathlineto{\pgfqpoint{1.952016in}{3.100499in}}%
\pgfpathlineto{\pgfqpoint{1.954588in}{3.084043in}}%
\pgfpathlineto{\pgfqpoint{1.959732in}{3.084043in}}%
\pgfpathlineto{\pgfqpoint{1.962304in}{3.067587in}}%
\pgfpathlineto{\pgfqpoint{1.967448in}{3.100499in}}%
\pgfpathlineto{\pgfqpoint{1.988023in}{3.100499in}}%
\pgfpathlineto{\pgfqpoint{1.990594in}{3.084043in}}%
\pgfpathlineto{\pgfqpoint{1.998310in}{3.084043in}}%
\pgfpathlineto{\pgfqpoint{2.000882in}{3.051131in}}%
\pgfpathlineto{\pgfqpoint{2.003454in}{3.084043in}}%
\pgfpathlineto{\pgfqpoint{2.006026in}{3.100499in}}%
\pgfpathlineto{\pgfqpoint{2.008597in}{3.084043in}}%
\pgfpathlineto{\pgfqpoint{2.011169in}{3.100499in}}%
\pgfpathlineto{\pgfqpoint{2.016313in}{3.100499in}}%
\pgfpathlineto{\pgfqpoint{2.018885in}{3.067587in}}%
\pgfpathlineto{\pgfqpoint{2.021457in}{3.084043in}}%
\pgfpathlineto{\pgfqpoint{2.024029in}{3.084043in}}%
\pgfpathlineto{\pgfqpoint{2.026601in}{3.100499in}}%
\pgfpathlineto{\pgfqpoint{2.042032in}{3.100499in}}%
\pgfpathlineto{\pgfqpoint{2.044604in}{3.084043in}}%
\pgfpathlineto{\pgfqpoint{2.052319in}{3.084043in}}%
\pgfpathlineto{\pgfqpoint{2.054891in}{3.067587in}}%
\pgfpathlineto{\pgfqpoint{2.057463in}{3.067587in}}%
\pgfpathlineto{\pgfqpoint{2.062607in}{3.034675in}}%
\pgfpathlineto{\pgfqpoint{2.065179in}{3.034675in}}%
\pgfpathlineto{\pgfqpoint{2.067751in}{3.018219in}}%
\pgfpathlineto{\pgfqpoint{2.070322in}{3.018219in}}%
\pgfpathlineto{\pgfqpoint{2.078038in}{2.968851in}}%
\pgfpathlineto{\pgfqpoint{2.080610in}{3.018219in}}%
\pgfpathlineto{\pgfqpoint{2.083182in}{3.001763in}}%
\pgfpathlineto{\pgfqpoint{2.085754in}{3.018219in}}%
\pgfpathlineto{\pgfqpoint{2.088325in}{3.018219in}}%
\pgfpathlineto{\pgfqpoint{2.090897in}{3.001763in}}%
\pgfpathlineto{\pgfqpoint{2.093469in}{3.001763in}}%
\pgfpathlineto{\pgfqpoint{2.096041in}{2.985307in}}%
\pgfpathlineto{\pgfqpoint{2.103757in}{2.985307in}}%
\pgfpathlineto{\pgfqpoint{2.106329in}{3.001763in}}%
\pgfpathlineto{\pgfqpoint{2.108900in}{3.001763in}}%
\pgfpathlineto{\pgfqpoint{2.111472in}{2.985307in}}%
\pgfpathlineto{\pgfqpoint{2.114044in}{3.018219in}}%
\pgfpathlineto{\pgfqpoint{2.116616in}{3.018219in}}%
\pgfpathlineto{\pgfqpoint{2.119188in}{3.034675in}}%
\pgfpathlineto{\pgfqpoint{2.121760in}{3.018219in}}%
\pgfpathlineto{\pgfqpoint{2.124332in}{3.034675in}}%
\pgfpathlineto{\pgfqpoint{2.126904in}{3.034675in}}%
\pgfpathlineto{\pgfqpoint{2.129475in}{3.051131in}}%
\pgfpathlineto{\pgfqpoint{2.134619in}{3.051131in}}%
\pgfpathlineto{\pgfqpoint{2.137191in}{3.067587in}}%
\pgfpathlineto{\pgfqpoint{2.139763in}{3.067587in}}%
\pgfpathlineto{\pgfqpoint{2.144907in}{3.100499in}}%
\pgfpathlineto{\pgfqpoint{2.152622in}{3.100499in}}%
\pgfpathlineto{\pgfqpoint{2.155194in}{3.116955in}}%
\pgfpathlineto{\pgfqpoint{2.160338in}{3.084043in}}%
\pgfpathlineto{\pgfqpoint{2.165482in}{3.084043in}}%
\pgfpathlineto{\pgfqpoint{2.168053in}{3.100499in}}%
\pgfpathlineto{\pgfqpoint{2.173197in}{3.100499in}}%
\pgfpathlineto{\pgfqpoint{2.175769in}{3.116955in}}%
\pgfpathlineto{\pgfqpoint{2.178341in}{3.100499in}}%
\pgfpathlineto{\pgfqpoint{2.180913in}{3.116955in}}%
\pgfpathlineto{\pgfqpoint{2.183485in}{3.100499in}}%
\pgfpathlineto{\pgfqpoint{2.191200in}{3.100499in}}%
\pgfpathlineto{\pgfqpoint{2.193772in}{3.116955in}}%
\pgfpathlineto{\pgfqpoint{2.198916in}{3.084043in}}%
\pgfpathlineto{\pgfqpoint{2.201488in}{3.084043in}}%
\pgfpathlineto{\pgfqpoint{2.204060in}{3.100499in}}%
\pgfpathlineto{\pgfqpoint{2.209203in}{3.100499in}}%
\pgfpathlineto{\pgfqpoint{2.216919in}{3.051131in}}%
\pgfpathlineto{\pgfqpoint{2.224635in}{3.100499in}}%
\pgfpathlineto{\pgfqpoint{2.227206in}{3.084043in}}%
\pgfpathlineto{\pgfqpoint{2.232350in}{3.084043in}}%
\pgfpathlineto{\pgfqpoint{2.234922in}{3.067587in}}%
\pgfpathlineto{\pgfqpoint{2.245210in}{3.067587in}}%
\pgfpathlineto{\pgfqpoint{2.247781in}{3.034675in}}%
\pgfpathlineto{\pgfqpoint{2.250353in}{3.034675in}}%
\pgfpathlineto{\pgfqpoint{2.252925in}{3.067587in}}%
\pgfpathlineto{\pgfqpoint{2.258069in}{3.067587in}}%
\pgfpathlineto{\pgfqpoint{2.260641in}{3.084043in}}%
\pgfpathlineto{\pgfqpoint{2.263213in}{3.084043in}}%
\pgfpathlineto{\pgfqpoint{2.265785in}{3.067587in}}%
\pgfpathlineto{\pgfqpoint{2.268356in}{3.067587in}}%
\pgfpathlineto{\pgfqpoint{2.270928in}{3.084043in}}%
\pgfpathlineto{\pgfqpoint{2.273500in}{3.116955in}}%
\pgfpathlineto{\pgfqpoint{2.276072in}{3.100499in}}%
\pgfpathlineto{\pgfqpoint{2.278644in}{3.133411in}}%
\pgfpathlineto{\pgfqpoint{2.281216in}{3.133411in}}%
\pgfpathlineto{\pgfqpoint{2.283788in}{3.116955in}}%
\pgfpathlineto{\pgfqpoint{2.286359in}{3.133411in}}%
\pgfpathlineto{\pgfqpoint{2.291503in}{3.100499in}}%
\pgfpathlineto{\pgfqpoint{2.299219in}{3.100499in}}%
\pgfpathlineto{\pgfqpoint{2.301791in}{3.116955in}}%
\pgfpathlineto{\pgfqpoint{2.312078in}{3.116955in}}%
\pgfpathlineto{\pgfqpoint{2.314650in}{3.100499in}}%
\pgfpathlineto{\pgfqpoint{2.317222in}{3.100499in}}%
\pgfpathlineto{\pgfqpoint{2.319794in}{3.116955in}}%
\pgfpathlineto{\pgfqpoint{2.322366in}{3.100499in}}%
\pgfpathlineto{\pgfqpoint{2.342941in}{3.100499in}}%
\pgfpathlineto{\pgfqpoint{2.345512in}{3.084043in}}%
\pgfpathlineto{\pgfqpoint{2.350656in}{3.084043in}}%
\pgfpathlineto{\pgfqpoint{2.353228in}{3.100499in}}%
\pgfpathlineto{\pgfqpoint{2.360944in}{3.100499in}}%
\pgfpathlineto{\pgfqpoint{2.363516in}{3.084043in}}%
\pgfpathlineto{\pgfqpoint{2.368659in}{3.084043in}}%
\pgfpathlineto{\pgfqpoint{2.371231in}{3.100499in}}%
\pgfpathlineto{\pgfqpoint{2.378947in}{3.100499in}}%
\pgfpathlineto{\pgfqpoint{2.381519in}{3.084043in}}%
\pgfpathlineto{\pgfqpoint{2.386662in}{3.084043in}}%
\pgfpathlineto{\pgfqpoint{2.389234in}{3.018219in}}%
\pgfpathlineto{\pgfqpoint{2.391806in}{3.034675in}}%
\pgfpathlineto{\pgfqpoint{2.394378in}{3.001763in}}%
\pgfpathlineto{\pgfqpoint{2.396950in}{2.952395in}}%
\pgfpathlineto{\pgfqpoint{2.402094in}{2.985307in}}%
\pgfpathlineto{\pgfqpoint{2.404665in}{2.952395in}}%
\pgfpathlineto{\pgfqpoint{2.407237in}{2.935938in}}%
\pgfpathlineto{\pgfqpoint{2.409809in}{2.935938in}}%
\pgfpathlineto{\pgfqpoint{2.412381in}{2.952395in}}%
\pgfpathlineto{\pgfqpoint{2.414953in}{2.985307in}}%
\pgfpathlineto{\pgfqpoint{2.417525in}{2.952395in}}%
\pgfpathlineto{\pgfqpoint{2.420097in}{2.935938in}}%
\pgfpathlineto{\pgfqpoint{2.422669in}{2.952395in}}%
\pgfpathlineto{\pgfqpoint{2.425240in}{2.952395in}}%
\pgfpathlineto{\pgfqpoint{2.427812in}{2.985307in}}%
\pgfpathlineto{\pgfqpoint{2.430384in}{3.001763in}}%
\pgfpathlineto{\pgfqpoint{2.432956in}{3.001763in}}%
\pgfpathlineto{\pgfqpoint{2.435528in}{2.985307in}}%
\pgfpathlineto{\pgfqpoint{2.448387in}{2.985307in}}%
\pgfpathlineto{\pgfqpoint{2.450959in}{3.001763in}}%
\pgfpathlineto{\pgfqpoint{2.466390in}{3.001763in}}%
\pgfpathlineto{\pgfqpoint{2.468962in}{2.968851in}}%
\pgfpathlineto{\pgfqpoint{2.471534in}{3.001763in}}%
\pgfpathlineto{\pgfqpoint{2.474106in}{2.968851in}}%
\pgfpathlineto{\pgfqpoint{2.476678in}{2.952395in}}%
\pgfpathlineto{\pgfqpoint{2.479250in}{2.985307in}}%
\pgfpathlineto{\pgfqpoint{2.481822in}{3.001763in}}%
\pgfpathlineto{\pgfqpoint{2.486965in}{3.001763in}}%
\pgfpathlineto{\pgfqpoint{2.489537in}{3.034675in}}%
\pgfpathlineto{\pgfqpoint{2.492109in}{3.034675in}}%
\pgfpathlineto{\pgfqpoint{2.494681in}{3.051131in}}%
\pgfpathlineto{\pgfqpoint{2.497253in}{3.034675in}}%
\pgfpathlineto{\pgfqpoint{2.502397in}{3.034675in}}%
\pgfpathlineto{\pgfqpoint{2.507540in}{3.067587in}}%
\pgfpathlineto{\pgfqpoint{2.510112in}{3.067587in}}%
\pgfpathlineto{\pgfqpoint{2.512684in}{3.084043in}}%
\pgfpathlineto{\pgfqpoint{2.515256in}{3.116955in}}%
\pgfpathlineto{\pgfqpoint{2.517828in}{3.116955in}}%
\pgfpathlineto{\pgfqpoint{2.520400in}{3.100499in}}%
\pgfpathlineto{\pgfqpoint{2.530687in}{3.100499in}}%
\pgfpathlineto{\pgfqpoint{2.533259in}{3.084043in}}%
\pgfpathlineto{\pgfqpoint{2.535831in}{3.084043in}}%
\pgfpathlineto{\pgfqpoint{2.540975in}{3.116955in}}%
\pgfpathlineto{\pgfqpoint{2.546118in}{3.116955in}}%
\pgfpathlineto{\pgfqpoint{2.548690in}{3.100499in}}%
\pgfpathlineto{\pgfqpoint{2.553834in}{3.100499in}}%
\pgfpathlineto{\pgfqpoint{2.556406in}{3.084043in}}%
\pgfpathlineto{\pgfqpoint{2.558978in}{3.100499in}}%
\pgfpathlineto{\pgfqpoint{2.569265in}{3.100499in}}%
\pgfpathlineto{\pgfqpoint{2.571837in}{3.116955in}}%
\pgfpathlineto{\pgfqpoint{2.574409in}{3.100499in}}%
\pgfpathlineto{\pgfqpoint{2.582125in}{3.100499in}}%
\pgfpathlineto{\pgfqpoint{2.584696in}{3.084043in}}%
\pgfpathlineto{\pgfqpoint{2.587268in}{3.084043in}}%
\pgfpathlineto{\pgfqpoint{2.589840in}{3.067587in}}%
\pgfpathlineto{\pgfqpoint{2.592412in}{3.100499in}}%
\pgfpathlineto{\pgfqpoint{2.594984in}{3.100499in}}%
\pgfpathlineto{\pgfqpoint{2.597556in}{3.133411in}}%
\pgfpathlineto{\pgfqpoint{2.600128in}{3.116955in}}%
\pgfpathlineto{\pgfqpoint{2.602699in}{3.116955in}}%
\pgfpathlineto{\pgfqpoint{2.605271in}{3.084043in}}%
\pgfpathlineto{\pgfqpoint{2.610415in}{3.116955in}}%
\pgfpathlineto{\pgfqpoint{2.612987in}{3.100499in}}%
\pgfpathlineto{\pgfqpoint{2.623274in}{3.100499in}}%
\pgfpathlineto{\pgfqpoint{2.625846in}{3.116955in}}%
\pgfpathlineto{\pgfqpoint{2.628418in}{3.084043in}}%
\pgfpathlineto{\pgfqpoint{2.630990in}{3.100499in}}%
\pgfpathlineto{\pgfqpoint{2.633562in}{3.100499in}}%
\pgfpathlineto{\pgfqpoint{2.636134in}{3.084043in}}%
\pgfpathlineto{\pgfqpoint{2.641278in}{3.084043in}}%
\pgfpathlineto{\pgfqpoint{2.643849in}{3.051131in}}%
\pgfpathlineto{\pgfqpoint{2.646421in}{3.034675in}}%
\pgfpathlineto{\pgfqpoint{2.646421in}{3.034675in}}%
\pgfusepath{stroke}%
\end{pgfscope}%
\begin{pgfscope}%
\pgfpathrectangle{\pgfqpoint{0.488751in}{1.946106in}}{\pgfqpoint{2.260417in}{1.502439in}}%
\pgfusepath{clip}%
\pgfsetbuttcap%
\pgfsetroundjoin%
\pgfsetlinewidth{0.803000pt}%
\definecolor{currentstroke}{rgb}{0.333333,0.333333,0.333333}%
\pgfsetstrokecolor{currentstroke}%
\pgfsetdash{{2.960000pt}{1.280000pt}}{0.000000pt}%
\pgfpathmoveto{\pgfqpoint{0.591497in}{3.001763in}}%
\pgfpathlineto{\pgfqpoint{0.594069in}{3.051131in}}%
\pgfpathlineto{\pgfqpoint{0.596641in}{3.051131in}}%
\pgfpathlineto{\pgfqpoint{0.599213in}{3.034675in}}%
\pgfpathlineto{\pgfqpoint{0.601785in}{3.084043in}}%
\pgfpathlineto{\pgfqpoint{0.604356in}{3.166323in}}%
\pgfpathlineto{\pgfqpoint{0.606928in}{3.116955in}}%
\pgfpathlineto{\pgfqpoint{0.609500in}{3.100499in}}%
\pgfpathlineto{\pgfqpoint{0.612072in}{3.100499in}}%
\pgfpathlineto{\pgfqpoint{0.614644in}{3.166323in}}%
\pgfpathlineto{\pgfqpoint{0.619788in}{3.034675in}}%
\pgfpathlineto{\pgfqpoint{0.622359in}{3.001763in}}%
\pgfpathlineto{\pgfqpoint{0.624931in}{3.001763in}}%
\pgfpathlineto{\pgfqpoint{0.627503in}{2.903026in}}%
\pgfpathlineto{\pgfqpoint{0.630075in}{2.853658in}}%
\pgfpathlineto{\pgfqpoint{0.632647in}{2.853658in}}%
\pgfpathlineto{\pgfqpoint{0.635219in}{2.820746in}}%
\pgfpathlineto{\pgfqpoint{0.637791in}{2.837202in}}%
\pgfpathlineto{\pgfqpoint{0.640363in}{2.804290in}}%
\pgfpathlineto{\pgfqpoint{0.642934in}{2.787834in}}%
\pgfpathlineto{\pgfqpoint{0.648078in}{2.656185in}}%
\pgfpathlineto{\pgfqpoint{0.653222in}{2.623273in}}%
\pgfpathlineto{\pgfqpoint{0.655794in}{2.623273in}}%
\pgfpathlineto{\pgfqpoint{0.658366in}{2.606817in}}%
\pgfpathlineto{\pgfqpoint{0.663509in}{2.606817in}}%
\pgfpathlineto{\pgfqpoint{0.666081in}{2.672641in}}%
\pgfpathlineto{\pgfqpoint{0.668653in}{2.639729in}}%
\pgfpathlineto{\pgfqpoint{0.671225in}{2.656185in}}%
\pgfpathlineto{\pgfqpoint{0.686656in}{2.656185in}}%
\pgfpathlineto{\pgfqpoint{0.689228in}{2.639729in}}%
\pgfpathlineto{\pgfqpoint{0.704659in}{2.639729in}}%
\pgfpathlineto{\pgfqpoint{0.707231in}{2.672641in}}%
\pgfpathlineto{\pgfqpoint{0.709803in}{2.656185in}}%
\pgfpathlineto{\pgfqpoint{0.725234in}{2.656185in}}%
\pgfpathlineto{\pgfqpoint{0.727806in}{2.606817in}}%
\pgfpathlineto{\pgfqpoint{0.732950in}{2.606817in}}%
\pgfpathlineto{\pgfqpoint{0.735522in}{2.590361in}}%
\pgfpathlineto{\pgfqpoint{0.740666in}{2.524537in}}%
\pgfpathlineto{\pgfqpoint{0.743237in}{2.508081in}}%
\pgfpathlineto{\pgfqpoint{0.745809in}{2.540993in}}%
\pgfpathlineto{\pgfqpoint{0.750953in}{2.540993in}}%
\pgfpathlineto{\pgfqpoint{0.753525in}{2.573905in}}%
\pgfpathlineto{\pgfqpoint{0.758669in}{2.491625in}}%
\pgfpathlineto{\pgfqpoint{0.776672in}{2.491625in}}%
\pgfpathlineto{\pgfqpoint{0.779244in}{2.508081in}}%
\pgfpathlineto{\pgfqpoint{0.781815in}{2.491625in}}%
\pgfpathlineto{\pgfqpoint{0.784387in}{2.524537in}}%
\pgfpathlineto{\pgfqpoint{0.786959in}{2.475169in}}%
\pgfpathlineto{\pgfqpoint{0.794675in}{2.425800in}}%
\pgfpathlineto{\pgfqpoint{0.797247in}{2.425800in}}%
\pgfpathlineto{\pgfqpoint{0.799819in}{2.409344in}}%
\pgfpathlineto{\pgfqpoint{0.807534in}{2.409344in}}%
\pgfpathlineto{\pgfqpoint{0.810106in}{2.442256in}}%
\pgfpathlineto{\pgfqpoint{0.815250in}{2.442256in}}%
\pgfpathlineto{\pgfqpoint{0.817822in}{2.458712in}}%
\pgfpathlineto{\pgfqpoint{0.820393in}{2.442256in}}%
\pgfpathlineto{\pgfqpoint{0.833253in}{2.442256in}}%
\pgfpathlineto{\pgfqpoint{0.835825in}{2.475169in}}%
\pgfpathlineto{\pgfqpoint{0.838397in}{2.458712in}}%
\pgfpathlineto{\pgfqpoint{0.840968in}{2.458712in}}%
\pgfpathlineto{\pgfqpoint{0.846112in}{2.524537in}}%
\pgfpathlineto{\pgfqpoint{0.848684in}{2.524537in}}%
\pgfpathlineto{\pgfqpoint{0.851256in}{2.458712in}}%
\pgfpathlineto{\pgfqpoint{0.853828in}{2.442256in}}%
\pgfpathlineto{\pgfqpoint{0.856400in}{2.458712in}}%
\pgfpathlineto{\pgfqpoint{0.869259in}{2.458712in}}%
\pgfpathlineto{\pgfqpoint{0.871831in}{2.475169in}}%
\pgfpathlineto{\pgfqpoint{0.884690in}{2.475169in}}%
\pgfpathlineto{\pgfqpoint{0.889834in}{2.442256in}}%
\pgfpathlineto{\pgfqpoint{0.902693in}{2.442256in}}%
\pgfpathlineto{\pgfqpoint{0.905265in}{2.425800in}}%
\pgfpathlineto{\pgfqpoint{0.915553in}{2.425800in}}%
\pgfpathlineto{\pgfqpoint{0.918125in}{2.442256in}}%
\pgfpathlineto{\pgfqpoint{0.923268in}{2.442256in}}%
\pgfpathlineto{\pgfqpoint{0.925840in}{2.458712in}}%
\pgfpathlineto{\pgfqpoint{0.959274in}{2.458712in}}%
\pgfpathlineto{\pgfqpoint{0.961846in}{2.475169in}}%
\pgfpathlineto{\pgfqpoint{0.964418in}{2.458712in}}%
\pgfpathlineto{\pgfqpoint{0.969562in}{2.458712in}}%
\pgfpathlineto{\pgfqpoint{0.972134in}{2.475169in}}%
\pgfpathlineto{\pgfqpoint{0.987565in}{2.475169in}}%
\pgfpathlineto{\pgfqpoint{0.990137in}{2.508081in}}%
\pgfpathlineto{\pgfqpoint{0.992709in}{2.508081in}}%
\pgfpathlineto{\pgfqpoint{0.997853in}{2.475169in}}%
\pgfpathlineto{\pgfqpoint{1.000424in}{2.475169in}}%
\pgfpathlineto{\pgfqpoint{1.002996in}{2.442256in}}%
\pgfpathlineto{\pgfqpoint{1.005568in}{2.442256in}}%
\pgfpathlineto{\pgfqpoint{1.008140in}{2.392888in}}%
\pgfpathlineto{\pgfqpoint{1.010712in}{2.409344in}}%
\pgfpathlineto{\pgfqpoint{1.013284in}{2.409344in}}%
\pgfpathlineto{\pgfqpoint{1.015856in}{2.392888in}}%
\pgfpathlineto{\pgfqpoint{1.036431in}{2.392888in}}%
\pgfpathlineto{\pgfqpoint{1.039002in}{2.425800in}}%
\pgfpathlineto{\pgfqpoint{1.041574in}{2.392888in}}%
\pgfpathlineto{\pgfqpoint{1.044146in}{2.392888in}}%
\pgfpathlineto{\pgfqpoint{1.046718in}{2.409344in}}%
\pgfpathlineto{\pgfqpoint{1.057006in}{2.409344in}}%
\pgfpathlineto{\pgfqpoint{1.059577in}{2.376432in}}%
\pgfpathlineto{\pgfqpoint{1.067293in}{2.376432in}}%
\pgfpathlineto{\pgfqpoint{1.069865in}{2.392888in}}%
\pgfpathlineto{\pgfqpoint{1.087868in}{2.392888in}}%
\pgfpathlineto{\pgfqpoint{1.090440in}{2.409344in}}%
\pgfpathlineto{\pgfqpoint{1.095584in}{2.409344in}}%
\pgfpathlineto{\pgfqpoint{1.098155in}{2.376432in}}%
\pgfpathlineto{\pgfqpoint{1.126446in}{2.376432in}}%
\pgfpathlineto{\pgfqpoint{1.129018in}{2.392888in}}%
\pgfpathlineto{\pgfqpoint{1.131590in}{2.392888in}}%
\pgfpathlineto{\pgfqpoint{1.134162in}{2.409344in}}%
\pgfpathlineto{\pgfqpoint{1.136734in}{2.409344in}}%
\pgfpathlineto{\pgfqpoint{1.139305in}{2.392888in}}%
\pgfpathlineto{\pgfqpoint{1.147021in}{2.392888in}}%
\pgfpathlineto{\pgfqpoint{1.152165in}{2.458712in}}%
\pgfpathlineto{\pgfqpoint{1.154737in}{2.442256in}}%
\pgfpathlineto{\pgfqpoint{1.157308in}{2.442256in}}%
\pgfpathlineto{\pgfqpoint{1.159880in}{2.425800in}}%
\pgfpathlineto{\pgfqpoint{1.167596in}{2.425800in}}%
\pgfpathlineto{\pgfqpoint{1.170168in}{2.458712in}}%
\pgfpathlineto{\pgfqpoint{1.172740in}{2.425800in}}%
\pgfpathlineto{\pgfqpoint{1.175312in}{2.409344in}}%
\pgfpathlineto{\pgfqpoint{1.177883in}{2.409344in}}%
\pgfpathlineto{\pgfqpoint{1.180455in}{2.376432in}}%
\pgfpathlineto{\pgfqpoint{1.198458in}{2.376432in}}%
\pgfpathlineto{\pgfqpoint{1.201030in}{2.409344in}}%
\pgfpathlineto{\pgfqpoint{1.208746in}{2.409344in}}%
\pgfpathlineto{\pgfqpoint{1.211318in}{2.425800in}}%
\pgfpathlineto{\pgfqpoint{1.216461in}{2.425800in}}%
\pgfpathlineto{\pgfqpoint{1.219033in}{2.409344in}}%
\pgfpathlineto{\pgfqpoint{1.221605in}{2.376432in}}%
\pgfpathlineto{\pgfqpoint{1.224177in}{2.376432in}}%
\pgfpathlineto{\pgfqpoint{1.226749in}{2.327064in}}%
\pgfpathlineto{\pgfqpoint{1.234465in}{2.327064in}}%
\pgfpathlineto{\pgfqpoint{1.237036in}{2.343520in}}%
\pgfpathlineto{\pgfqpoint{1.242180in}{2.343520in}}%
\pgfpathlineto{\pgfqpoint{1.244752in}{2.376432in}}%
\pgfpathlineto{\pgfqpoint{1.257611in}{2.376432in}}%
\pgfpathlineto{\pgfqpoint{1.260183in}{2.343520in}}%
\pgfpathlineto{\pgfqpoint{1.267899in}{2.343520in}}%
\pgfpathlineto{\pgfqpoint{1.273043in}{2.376432in}}%
\pgfpathlineto{\pgfqpoint{1.275615in}{2.359976in}}%
\pgfpathlineto{\pgfqpoint{1.288474in}{2.359976in}}%
\pgfpathlineto{\pgfqpoint{1.291046in}{2.327064in}}%
\pgfpathlineto{\pgfqpoint{1.293618in}{2.343520in}}%
\pgfpathlineto{\pgfqpoint{1.298761in}{2.343520in}}%
\pgfpathlineto{\pgfqpoint{1.303905in}{2.376432in}}%
\pgfpathlineto{\pgfqpoint{1.321908in}{2.376432in}}%
\pgfpathlineto{\pgfqpoint{1.324480in}{2.359976in}}%
\pgfpathlineto{\pgfqpoint{1.329624in}{2.359976in}}%
\pgfpathlineto{\pgfqpoint{1.332196in}{2.376432in}}%
\pgfpathlineto{\pgfqpoint{1.339911in}{2.376432in}}%
\pgfpathlineto{\pgfqpoint{1.345055in}{2.343520in}}%
\pgfpathlineto{\pgfqpoint{1.352771in}{2.343520in}}%
\pgfpathlineto{\pgfqpoint{1.355342in}{2.376432in}}%
\pgfpathlineto{\pgfqpoint{1.357914in}{2.359976in}}%
\pgfpathlineto{\pgfqpoint{1.360486in}{2.359976in}}%
\pgfpathlineto{\pgfqpoint{1.363058in}{2.343520in}}%
\pgfpathlineto{\pgfqpoint{1.404208in}{2.343520in}}%
\pgfpathlineto{\pgfqpoint{1.406780in}{2.327064in}}%
\pgfpathlineto{\pgfqpoint{1.409352in}{2.359976in}}%
\pgfpathlineto{\pgfqpoint{1.432499in}{2.359976in}}%
\pgfpathlineto{\pgfqpoint{1.435070in}{2.392888in}}%
\pgfpathlineto{\pgfqpoint{1.453074in}{2.392888in}}%
\pgfpathlineto{\pgfqpoint{1.455645in}{2.409344in}}%
\pgfpathlineto{\pgfqpoint{1.481364in}{2.409344in}}%
\pgfpathlineto{\pgfqpoint{1.483936in}{2.425800in}}%
\pgfpathlineto{\pgfqpoint{1.489080in}{2.425800in}}%
\pgfpathlineto{\pgfqpoint{1.491652in}{2.442256in}}%
\pgfpathlineto{\pgfqpoint{1.494223in}{2.475169in}}%
\pgfpathlineto{\pgfqpoint{1.496795in}{2.475169in}}%
\pgfpathlineto{\pgfqpoint{1.499367in}{2.491625in}}%
\pgfpathlineto{\pgfqpoint{1.501939in}{2.475169in}}%
\pgfpathlineto{\pgfqpoint{1.509655in}{2.475169in}}%
\pgfpathlineto{\pgfqpoint{1.512227in}{2.442256in}}%
\pgfpathlineto{\pgfqpoint{1.573951in}{2.442256in}}%
\pgfpathlineto{\pgfqpoint{1.576523in}{2.458712in}}%
\pgfpathlineto{\pgfqpoint{1.589383in}{2.458712in}}%
\pgfpathlineto{\pgfqpoint{1.591955in}{2.475169in}}%
\pgfpathlineto{\pgfqpoint{1.594526in}{2.458712in}}%
\pgfpathlineto{\pgfqpoint{1.597098in}{2.491625in}}%
\pgfpathlineto{\pgfqpoint{1.599670in}{2.475169in}}%
\pgfpathlineto{\pgfqpoint{1.602242in}{2.491625in}}%
\pgfpathlineto{\pgfqpoint{1.604814in}{2.475169in}}%
\pgfpathlineto{\pgfqpoint{1.607386in}{2.524537in}}%
\pgfpathlineto{\pgfqpoint{1.609958in}{2.524537in}}%
\pgfpathlineto{\pgfqpoint{1.612529in}{2.540993in}}%
\pgfpathlineto{\pgfqpoint{1.615101in}{2.524537in}}%
\pgfpathlineto{\pgfqpoint{1.617673in}{2.524537in}}%
\pgfpathlineto{\pgfqpoint{1.620245in}{2.475169in}}%
\pgfpathlineto{\pgfqpoint{1.627961in}{2.475169in}}%
\pgfpathlineto{\pgfqpoint{1.630533in}{2.491625in}}%
\pgfpathlineto{\pgfqpoint{1.638248in}{2.491625in}}%
\pgfpathlineto{\pgfqpoint{1.645964in}{2.540993in}}%
\pgfpathlineto{\pgfqpoint{1.648536in}{2.524537in}}%
\pgfpathlineto{\pgfqpoint{1.658823in}{2.524537in}}%
\pgfpathlineto{\pgfqpoint{1.661395in}{2.508081in}}%
\pgfpathlineto{\pgfqpoint{1.663967in}{2.524537in}}%
\pgfpathlineto{\pgfqpoint{1.676826in}{2.524537in}}%
\pgfpathlineto{\pgfqpoint{1.681970in}{2.491625in}}%
\pgfpathlineto{\pgfqpoint{1.687114in}{2.557449in}}%
\pgfpathlineto{\pgfqpoint{1.694829in}{2.557449in}}%
\pgfpathlineto{\pgfqpoint{1.697401in}{2.540993in}}%
\pgfpathlineto{\pgfqpoint{1.699973in}{2.557449in}}%
\pgfpathlineto{\pgfqpoint{1.702545in}{2.557449in}}%
\pgfpathlineto{\pgfqpoint{1.705117in}{2.590361in}}%
\pgfpathlineto{\pgfqpoint{1.707689in}{2.590361in}}%
\pgfpathlineto{\pgfqpoint{1.710261in}{2.623273in}}%
\pgfpathlineto{\pgfqpoint{1.717976in}{2.623273in}}%
\pgfpathlineto{\pgfqpoint{1.720548in}{2.639729in}}%
\pgfpathlineto{\pgfqpoint{1.725692in}{2.639729in}}%
\pgfpathlineto{\pgfqpoint{1.728264in}{2.623273in}}%
\pgfpathlineto{\pgfqpoint{1.730836in}{2.623273in}}%
\pgfpathlineto{\pgfqpoint{1.733407in}{2.656185in}}%
\pgfpathlineto{\pgfqpoint{1.735979in}{2.639729in}}%
\pgfpathlineto{\pgfqpoint{1.746267in}{2.639729in}}%
\pgfpathlineto{\pgfqpoint{1.753982in}{2.689097in}}%
\pgfpathlineto{\pgfqpoint{1.756554in}{2.656185in}}%
\pgfpathlineto{\pgfqpoint{1.759126in}{2.672641in}}%
\pgfpathlineto{\pgfqpoint{1.764270in}{2.672641in}}%
\pgfpathlineto{\pgfqpoint{1.766842in}{2.689097in}}%
\pgfpathlineto{\pgfqpoint{1.771985in}{2.689097in}}%
\pgfpathlineto{\pgfqpoint{1.774557in}{2.672641in}}%
\pgfpathlineto{\pgfqpoint{1.779701in}{2.672641in}}%
\pgfpathlineto{\pgfqpoint{1.782273in}{2.656185in}}%
\pgfpathlineto{\pgfqpoint{1.787417in}{2.689097in}}%
\pgfpathlineto{\pgfqpoint{1.792560in}{2.656185in}}%
\pgfpathlineto{\pgfqpoint{1.795132in}{2.656185in}}%
\pgfpathlineto{\pgfqpoint{1.797704in}{2.672641in}}%
\pgfpathlineto{\pgfqpoint{1.800276in}{2.656185in}}%
\pgfpathlineto{\pgfqpoint{1.802848in}{2.689097in}}%
\pgfpathlineto{\pgfqpoint{1.805420in}{2.705554in}}%
\pgfpathlineto{\pgfqpoint{1.807992in}{2.689097in}}%
\pgfpathlineto{\pgfqpoint{1.810563in}{2.689097in}}%
\pgfpathlineto{\pgfqpoint{1.813135in}{2.672641in}}%
\pgfpathlineto{\pgfqpoint{1.818279in}{2.672641in}}%
\pgfpathlineto{\pgfqpoint{1.820851in}{2.656185in}}%
\pgfpathlineto{\pgfqpoint{1.828567in}{2.656185in}}%
\pgfpathlineto{\pgfqpoint{1.831138in}{2.639729in}}%
\pgfpathlineto{\pgfqpoint{1.836282in}{2.639729in}}%
\pgfpathlineto{\pgfqpoint{1.838854in}{2.656185in}}%
\pgfpathlineto{\pgfqpoint{1.856857in}{2.656185in}}%
\pgfpathlineto{\pgfqpoint{1.862001in}{2.623273in}}%
\pgfpathlineto{\pgfqpoint{1.867145in}{2.623273in}}%
\pgfpathlineto{\pgfqpoint{1.869717in}{2.639729in}}%
\pgfpathlineto{\pgfqpoint{1.872288in}{2.639729in}}%
\pgfpathlineto{\pgfqpoint{1.874860in}{2.623273in}}%
\pgfpathlineto{\pgfqpoint{1.882576in}{2.623273in}}%
\pgfpathlineto{\pgfqpoint{1.885148in}{2.606817in}}%
\pgfpathlineto{\pgfqpoint{1.887720in}{2.623273in}}%
\pgfpathlineto{\pgfqpoint{1.903151in}{2.623273in}}%
\pgfpathlineto{\pgfqpoint{1.905723in}{2.606817in}}%
\pgfpathlineto{\pgfqpoint{1.921154in}{2.606817in}}%
\pgfpathlineto{\pgfqpoint{1.923726in}{2.590361in}}%
\pgfpathlineto{\pgfqpoint{1.926298in}{2.606817in}}%
\pgfpathlineto{\pgfqpoint{1.939157in}{2.606817in}}%
\pgfpathlineto{\pgfqpoint{1.941729in}{2.590361in}}%
\pgfpathlineto{\pgfqpoint{1.944301in}{2.606817in}}%
\pgfpathlineto{\pgfqpoint{1.946873in}{2.639729in}}%
\pgfpathlineto{\pgfqpoint{1.954588in}{2.639729in}}%
\pgfpathlineto{\pgfqpoint{1.957160in}{2.623273in}}%
\pgfpathlineto{\pgfqpoint{1.967448in}{2.623273in}}%
\pgfpathlineto{\pgfqpoint{1.970019in}{2.606817in}}%
\pgfpathlineto{\pgfqpoint{1.977735in}{2.606817in}}%
\pgfpathlineto{\pgfqpoint{1.980307in}{2.590361in}}%
\pgfpathlineto{\pgfqpoint{1.998310in}{2.590361in}}%
\pgfpathlineto{\pgfqpoint{2.000882in}{2.573905in}}%
\pgfpathlineto{\pgfqpoint{2.003454in}{2.573905in}}%
\pgfpathlineto{\pgfqpoint{2.006026in}{2.590361in}}%
\pgfpathlineto{\pgfqpoint{2.011169in}{2.590361in}}%
\pgfpathlineto{\pgfqpoint{2.013741in}{2.606817in}}%
\pgfpathlineto{\pgfqpoint{2.024029in}{2.606817in}}%
\pgfpathlineto{\pgfqpoint{2.029172in}{2.639729in}}%
\pgfpathlineto{\pgfqpoint{2.031744in}{2.639729in}}%
\pgfpathlineto{\pgfqpoint{2.034316in}{2.656185in}}%
\pgfpathlineto{\pgfqpoint{2.039460in}{2.623273in}}%
\pgfpathlineto{\pgfqpoint{2.042032in}{2.623273in}}%
\pgfpathlineto{\pgfqpoint{2.044604in}{2.606817in}}%
\pgfpathlineto{\pgfqpoint{2.054891in}{2.606817in}}%
\pgfpathlineto{\pgfqpoint{2.057463in}{2.590361in}}%
\pgfpathlineto{\pgfqpoint{2.060035in}{2.606817in}}%
\pgfpathlineto{\pgfqpoint{2.065179in}{2.606817in}}%
\pgfpathlineto{\pgfqpoint{2.067751in}{2.590361in}}%
\pgfpathlineto{\pgfqpoint{2.083182in}{2.590361in}}%
\pgfpathlineto{\pgfqpoint{2.085754in}{2.606817in}}%
\pgfpathlineto{\pgfqpoint{2.088325in}{2.639729in}}%
\pgfpathlineto{\pgfqpoint{2.108900in}{2.639729in}}%
\pgfpathlineto{\pgfqpoint{2.111472in}{2.623273in}}%
\pgfpathlineto{\pgfqpoint{2.137191in}{2.623273in}}%
\pgfpathlineto{\pgfqpoint{2.139763in}{2.639729in}}%
\pgfpathlineto{\pgfqpoint{2.142335in}{2.639729in}}%
\pgfpathlineto{\pgfqpoint{2.144907in}{2.623273in}}%
\pgfpathlineto{\pgfqpoint{2.157766in}{2.623273in}}%
\pgfpathlineto{\pgfqpoint{2.162910in}{2.590361in}}%
\pgfpathlineto{\pgfqpoint{2.165482in}{2.590361in}}%
\pgfpathlineto{\pgfqpoint{2.168053in}{2.606817in}}%
\pgfpathlineto{\pgfqpoint{2.175769in}{2.606817in}}%
\pgfpathlineto{\pgfqpoint{2.178341in}{2.590361in}}%
\pgfpathlineto{\pgfqpoint{2.186057in}{2.590361in}}%
\pgfpathlineto{\pgfqpoint{2.188628in}{2.623273in}}%
\pgfpathlineto{\pgfqpoint{2.191200in}{2.623273in}}%
\pgfpathlineto{\pgfqpoint{2.193772in}{2.639729in}}%
\pgfpathlineto{\pgfqpoint{2.196344in}{2.639729in}}%
\pgfpathlineto{\pgfqpoint{2.198916in}{2.606817in}}%
\pgfpathlineto{\pgfqpoint{2.201488in}{2.623273in}}%
\pgfpathlineto{\pgfqpoint{2.209203in}{2.623273in}}%
\pgfpathlineto{\pgfqpoint{2.214347in}{2.590361in}}%
\pgfpathlineto{\pgfqpoint{2.219491in}{2.590361in}}%
\pgfpathlineto{\pgfqpoint{2.222063in}{2.639729in}}%
\pgfpathlineto{\pgfqpoint{2.224635in}{2.656185in}}%
\pgfpathlineto{\pgfqpoint{2.227206in}{2.639729in}}%
\pgfpathlineto{\pgfqpoint{2.229778in}{2.639729in}}%
\pgfpathlineto{\pgfqpoint{2.232350in}{2.672641in}}%
\pgfpathlineto{\pgfqpoint{2.234922in}{2.656185in}}%
\pgfpathlineto{\pgfqpoint{2.237494in}{2.656185in}}%
\pgfpathlineto{\pgfqpoint{2.240066in}{2.623273in}}%
\pgfpathlineto{\pgfqpoint{2.242638in}{2.606817in}}%
\pgfpathlineto{\pgfqpoint{2.245210in}{2.606817in}}%
\pgfpathlineto{\pgfqpoint{2.247781in}{2.590361in}}%
\pgfpathlineto{\pgfqpoint{2.250353in}{2.590361in}}%
\pgfpathlineto{\pgfqpoint{2.252925in}{2.606817in}}%
\pgfpathlineto{\pgfqpoint{2.258069in}{2.606817in}}%
\pgfpathlineto{\pgfqpoint{2.260641in}{2.623273in}}%
\pgfpathlineto{\pgfqpoint{2.268356in}{2.623273in}}%
\pgfpathlineto{\pgfqpoint{2.270928in}{2.606817in}}%
\pgfpathlineto{\pgfqpoint{2.273500in}{2.606817in}}%
\pgfpathlineto{\pgfqpoint{2.276072in}{2.573905in}}%
\pgfpathlineto{\pgfqpoint{2.278644in}{2.573905in}}%
\pgfpathlineto{\pgfqpoint{2.281216in}{2.540993in}}%
\pgfpathlineto{\pgfqpoint{2.288931in}{2.540993in}}%
\pgfpathlineto{\pgfqpoint{2.294075in}{2.573905in}}%
\pgfpathlineto{\pgfqpoint{2.296647in}{2.573905in}}%
\pgfpathlineto{\pgfqpoint{2.299219in}{2.590361in}}%
\pgfpathlineto{\pgfqpoint{2.301791in}{2.573905in}}%
\pgfpathlineto{\pgfqpoint{2.304363in}{2.573905in}}%
\pgfpathlineto{\pgfqpoint{2.306934in}{2.590361in}}%
\pgfpathlineto{\pgfqpoint{2.309506in}{2.623273in}}%
\pgfpathlineto{\pgfqpoint{2.314650in}{2.623273in}}%
\pgfpathlineto{\pgfqpoint{2.317222in}{2.606817in}}%
\pgfpathlineto{\pgfqpoint{2.319794in}{2.623273in}}%
\pgfpathlineto{\pgfqpoint{2.322366in}{2.623273in}}%
\pgfpathlineto{\pgfqpoint{2.324938in}{2.639729in}}%
\pgfpathlineto{\pgfqpoint{2.330081in}{2.606817in}}%
\pgfpathlineto{\pgfqpoint{2.345512in}{2.606817in}}%
\pgfpathlineto{\pgfqpoint{2.348084in}{2.590361in}}%
\pgfpathlineto{\pgfqpoint{2.363516in}{2.590361in}}%
\pgfpathlineto{\pgfqpoint{2.366087in}{2.606817in}}%
\pgfpathlineto{\pgfqpoint{2.381519in}{2.606817in}}%
\pgfpathlineto{\pgfqpoint{2.384091in}{2.623273in}}%
\pgfpathlineto{\pgfqpoint{2.386662in}{2.623273in}}%
\pgfpathlineto{\pgfqpoint{2.389234in}{2.606817in}}%
\pgfpathlineto{\pgfqpoint{2.391806in}{2.623273in}}%
\pgfpathlineto{\pgfqpoint{2.394378in}{2.606817in}}%
\pgfpathlineto{\pgfqpoint{2.412381in}{2.606817in}}%
\pgfpathlineto{\pgfqpoint{2.414953in}{2.623273in}}%
\pgfpathlineto{\pgfqpoint{2.420097in}{2.623273in}}%
\pgfpathlineto{\pgfqpoint{2.422669in}{2.639729in}}%
\pgfpathlineto{\pgfqpoint{2.425240in}{2.639729in}}%
\pgfpathlineto{\pgfqpoint{2.427812in}{2.623273in}}%
\pgfpathlineto{\pgfqpoint{2.430384in}{2.623273in}}%
\pgfpathlineto{\pgfqpoint{2.432956in}{2.639729in}}%
\pgfpathlineto{\pgfqpoint{2.435528in}{2.606817in}}%
\pgfpathlineto{\pgfqpoint{2.438100in}{2.590361in}}%
\pgfpathlineto{\pgfqpoint{2.445815in}{2.590361in}}%
\pgfpathlineto{\pgfqpoint{2.448387in}{2.606817in}}%
\pgfpathlineto{\pgfqpoint{2.466390in}{2.606817in}}%
\pgfpathlineto{\pgfqpoint{2.468962in}{2.590361in}}%
\pgfpathlineto{\pgfqpoint{2.471534in}{2.623273in}}%
\pgfpathlineto{\pgfqpoint{2.474106in}{2.590361in}}%
\pgfpathlineto{\pgfqpoint{2.515256in}{2.590361in}}%
\pgfpathlineto{\pgfqpoint{2.517828in}{2.639729in}}%
\pgfpathlineto{\pgfqpoint{2.525543in}{2.639729in}}%
\pgfpathlineto{\pgfqpoint{2.528115in}{2.656185in}}%
\pgfpathlineto{\pgfqpoint{2.530687in}{2.656185in}}%
\pgfpathlineto{\pgfqpoint{2.535831in}{2.623273in}}%
\pgfpathlineto{\pgfqpoint{2.543546in}{2.623273in}}%
\pgfpathlineto{\pgfqpoint{2.546118in}{2.639729in}}%
\pgfpathlineto{\pgfqpoint{2.548690in}{2.623273in}}%
\pgfpathlineto{\pgfqpoint{2.556406in}{2.623273in}}%
\pgfpathlineto{\pgfqpoint{2.558978in}{2.656185in}}%
\pgfpathlineto{\pgfqpoint{2.561550in}{2.656185in}}%
\pgfpathlineto{\pgfqpoint{2.564121in}{2.639729in}}%
\pgfpathlineto{\pgfqpoint{2.571837in}{2.639729in}}%
\pgfpathlineto{\pgfqpoint{2.574409in}{2.656185in}}%
\pgfpathlineto{\pgfqpoint{2.576981in}{2.639729in}}%
\pgfpathlineto{\pgfqpoint{2.579553in}{2.639729in}}%
\pgfpathlineto{\pgfqpoint{2.582125in}{2.656185in}}%
\pgfpathlineto{\pgfqpoint{2.589840in}{2.656185in}}%
\pgfpathlineto{\pgfqpoint{2.592412in}{2.672641in}}%
\pgfpathlineto{\pgfqpoint{2.594984in}{2.672641in}}%
\pgfpathlineto{\pgfqpoint{2.597556in}{2.656185in}}%
\pgfpathlineto{\pgfqpoint{2.600128in}{2.672641in}}%
\pgfpathlineto{\pgfqpoint{2.605271in}{2.672641in}}%
\pgfpathlineto{\pgfqpoint{2.607843in}{2.689097in}}%
\pgfpathlineto{\pgfqpoint{2.610415in}{2.689097in}}%
\pgfpathlineto{\pgfqpoint{2.612987in}{2.656185in}}%
\pgfpathlineto{\pgfqpoint{2.623274in}{2.656185in}}%
\pgfpathlineto{\pgfqpoint{2.625846in}{2.672641in}}%
\pgfpathlineto{\pgfqpoint{2.628418in}{2.656185in}}%
\pgfpathlineto{\pgfqpoint{2.630990in}{2.656185in}}%
\pgfpathlineto{\pgfqpoint{2.633562in}{2.639729in}}%
\pgfpathlineto{\pgfqpoint{2.641278in}{2.639729in}}%
\pgfpathlineto{\pgfqpoint{2.643849in}{2.606817in}}%
\pgfpathlineto{\pgfqpoint{2.646421in}{2.606817in}}%
\pgfpathlineto{\pgfqpoint{2.646421in}{2.606817in}}%
\pgfusepath{stroke}%
\end{pgfscope}%
\begin{pgfscope}%
\pgfpathrectangle{\pgfqpoint{0.488751in}{1.946106in}}{\pgfqpoint{2.260417in}{1.502439in}}%
\pgfusepath{clip}%
\pgfsetbuttcap%
\pgfsetroundjoin%
\pgfsetlinewidth{0.803000pt}%
\definecolor{currentstroke}{rgb}{0.686275,0.352941,0.313725}%
\pgfsetstrokecolor{currentstroke}%
\pgfsetdash{{2.960000pt}{1.280000pt}}{0.000000pt}%
\pgfpathmoveto{\pgfqpoint{0.591497in}{2.195415in}}%
\pgfpathlineto{\pgfqpoint{0.594069in}{2.228328in}}%
\pgfpathlineto{\pgfqpoint{0.596641in}{2.178959in}}%
\pgfpathlineto{\pgfqpoint{0.599213in}{2.178959in}}%
\pgfpathlineto{\pgfqpoint{0.601785in}{2.162503in}}%
\pgfpathlineto{\pgfqpoint{0.606928in}{2.162503in}}%
\pgfpathlineto{\pgfqpoint{0.609500in}{2.178959in}}%
\pgfpathlineto{\pgfqpoint{0.619788in}{2.178959in}}%
\pgfpathlineto{\pgfqpoint{0.622359in}{2.162503in}}%
\pgfpathlineto{\pgfqpoint{0.627503in}{2.162503in}}%
\pgfpathlineto{\pgfqpoint{0.630075in}{2.129591in}}%
\pgfpathlineto{\pgfqpoint{0.632647in}{2.146047in}}%
\pgfpathlineto{\pgfqpoint{0.642934in}{2.146047in}}%
\pgfpathlineto{\pgfqpoint{0.645506in}{2.162503in}}%
\pgfpathlineto{\pgfqpoint{0.650650in}{2.162503in}}%
\pgfpathlineto{\pgfqpoint{0.653222in}{2.178959in}}%
\pgfpathlineto{\pgfqpoint{0.655794in}{2.129591in}}%
\pgfpathlineto{\pgfqpoint{0.671225in}{2.129591in}}%
\pgfpathlineto{\pgfqpoint{0.673797in}{2.146047in}}%
\pgfpathlineto{\pgfqpoint{0.676369in}{2.129591in}}%
\pgfpathlineto{\pgfqpoint{0.709803in}{2.129591in}}%
\pgfpathlineto{\pgfqpoint{0.712375in}{2.096679in}}%
\pgfpathlineto{\pgfqpoint{0.714947in}{2.096679in}}%
\pgfpathlineto{\pgfqpoint{0.720091in}{2.063767in}}%
\pgfpathlineto{\pgfqpoint{0.722662in}{2.080223in}}%
\pgfpathlineto{\pgfqpoint{0.725234in}{2.080223in}}%
\pgfpathlineto{\pgfqpoint{0.727806in}{2.063767in}}%
\pgfpathlineto{\pgfqpoint{0.730378in}{2.080223in}}%
\pgfpathlineto{\pgfqpoint{0.735522in}{2.080223in}}%
\pgfpathlineto{\pgfqpoint{0.738094in}{2.063767in}}%
\pgfpathlineto{\pgfqpoint{0.740666in}{2.080223in}}%
\pgfpathlineto{\pgfqpoint{0.761240in}{2.080223in}}%
\pgfpathlineto{\pgfqpoint{0.766384in}{2.047311in}}%
\pgfpathlineto{\pgfqpoint{0.771528in}{2.080223in}}%
\pgfpathlineto{\pgfqpoint{0.779244in}{2.080223in}}%
\pgfpathlineto{\pgfqpoint{0.781815in}{2.096679in}}%
\pgfpathlineto{\pgfqpoint{0.784387in}{2.080223in}}%
\pgfpathlineto{\pgfqpoint{0.830681in}{2.080223in}}%
\pgfpathlineto{\pgfqpoint{0.833253in}{2.063767in}}%
\pgfpathlineto{\pgfqpoint{0.843540in}{2.129591in}}%
\pgfpathlineto{\pgfqpoint{0.848684in}{2.129591in}}%
\pgfpathlineto{\pgfqpoint{0.851256in}{2.113135in}}%
\pgfpathlineto{\pgfqpoint{0.858972in}{2.113135in}}%
\pgfpathlineto{\pgfqpoint{0.861543in}{2.096679in}}%
\pgfpathlineto{\pgfqpoint{0.876975in}{2.096679in}}%
\pgfpathlineto{\pgfqpoint{0.879547in}{2.080223in}}%
\pgfpathlineto{\pgfqpoint{0.912981in}{2.080223in}}%
\pgfpathlineto{\pgfqpoint{0.915553in}{2.063767in}}%
\pgfpathlineto{\pgfqpoint{0.918125in}{2.080223in}}%
\pgfpathlineto{\pgfqpoint{0.920696in}{2.080223in}}%
\pgfpathlineto{\pgfqpoint{0.923268in}{2.063767in}}%
\pgfpathlineto{\pgfqpoint{0.925840in}{2.063767in}}%
\pgfpathlineto{\pgfqpoint{0.928412in}{2.080223in}}%
\pgfpathlineto{\pgfqpoint{0.936128in}{2.080223in}}%
\pgfpathlineto{\pgfqpoint{0.938700in}{2.096679in}}%
\pgfpathlineto{\pgfqpoint{0.948987in}{2.096679in}}%
\pgfpathlineto{\pgfqpoint{0.951559in}{2.080223in}}%
\pgfpathlineto{\pgfqpoint{0.974706in}{2.080223in}}%
\pgfpathlineto{\pgfqpoint{0.977278in}{2.096679in}}%
\pgfpathlineto{\pgfqpoint{0.990137in}{2.096679in}}%
\pgfpathlineto{\pgfqpoint{0.992709in}{2.113135in}}%
\pgfpathlineto{\pgfqpoint{1.005568in}{2.113135in}}%
\pgfpathlineto{\pgfqpoint{1.008140in}{2.129591in}}%
\pgfpathlineto{\pgfqpoint{1.013284in}{2.129591in}}%
\pgfpathlineto{\pgfqpoint{1.015856in}{2.113135in}}%
\pgfpathlineto{\pgfqpoint{1.028715in}{2.113135in}}%
\pgfpathlineto{\pgfqpoint{1.031287in}{2.162503in}}%
\pgfpathlineto{\pgfqpoint{1.041574in}{2.228328in}}%
\pgfpathlineto{\pgfqpoint{1.044146in}{2.211871in}}%
\pgfpathlineto{\pgfqpoint{1.057006in}{2.211871in}}%
\pgfpathlineto{\pgfqpoint{1.059577in}{2.178959in}}%
\pgfpathlineto{\pgfqpoint{1.062149in}{2.178959in}}%
\pgfpathlineto{\pgfqpoint{1.064721in}{2.195415in}}%
\pgfpathlineto{\pgfqpoint{1.069865in}{2.195415in}}%
\pgfpathlineto{\pgfqpoint{1.072437in}{2.211871in}}%
\pgfpathlineto{\pgfqpoint{1.075009in}{2.211871in}}%
\pgfpathlineto{\pgfqpoint{1.077581in}{2.228328in}}%
\pgfpathlineto{\pgfqpoint{1.085296in}{2.228328in}}%
\pgfpathlineto{\pgfqpoint{1.087868in}{2.244784in}}%
\pgfpathlineto{\pgfqpoint{1.090440in}{2.228328in}}%
\pgfpathlineto{\pgfqpoint{1.093012in}{2.228328in}}%
\pgfpathlineto{\pgfqpoint{1.103299in}{2.294152in}}%
\pgfpathlineto{\pgfqpoint{1.108443in}{2.294152in}}%
\pgfpathlineto{\pgfqpoint{1.111015in}{2.327064in}}%
\pgfpathlineto{\pgfqpoint{1.113587in}{2.327064in}}%
\pgfpathlineto{\pgfqpoint{1.116159in}{2.294152in}}%
\pgfpathlineto{\pgfqpoint{1.118730in}{2.310608in}}%
\pgfpathlineto{\pgfqpoint{1.123874in}{2.376432in}}%
\pgfpathlineto{\pgfqpoint{1.126446in}{2.392888in}}%
\pgfpathlineto{\pgfqpoint{1.129018in}{2.359976in}}%
\pgfpathlineto{\pgfqpoint{1.141877in}{2.359976in}}%
\pgfpathlineto{\pgfqpoint{1.144449in}{2.376432in}}%
\pgfpathlineto{\pgfqpoint{1.149593in}{2.376432in}}%
\pgfpathlineto{\pgfqpoint{1.154737in}{2.343520in}}%
\pgfpathlineto{\pgfqpoint{1.157308in}{2.343520in}}%
\pgfpathlineto{\pgfqpoint{1.159880in}{2.359976in}}%
\pgfpathlineto{\pgfqpoint{1.162452in}{2.343520in}}%
\pgfpathlineto{\pgfqpoint{1.172740in}{2.343520in}}%
\pgfpathlineto{\pgfqpoint{1.175312in}{2.359976in}}%
\pgfpathlineto{\pgfqpoint{1.177883in}{2.392888in}}%
\pgfpathlineto{\pgfqpoint{1.180455in}{2.376432in}}%
\pgfpathlineto{\pgfqpoint{1.183027in}{2.343520in}}%
\pgfpathlineto{\pgfqpoint{1.185599in}{2.343520in}}%
\pgfpathlineto{\pgfqpoint{1.190743in}{2.310608in}}%
\pgfpathlineto{\pgfqpoint{1.198458in}{2.310608in}}%
\pgfpathlineto{\pgfqpoint{1.201030in}{2.294152in}}%
\pgfpathlineto{\pgfqpoint{1.206174in}{2.294152in}}%
\pgfpathlineto{\pgfqpoint{1.208746in}{2.277696in}}%
\pgfpathlineto{\pgfqpoint{1.211318in}{2.310608in}}%
\pgfpathlineto{\pgfqpoint{1.213890in}{2.310608in}}%
\pgfpathlineto{\pgfqpoint{1.216461in}{2.327064in}}%
\pgfpathlineto{\pgfqpoint{1.219033in}{2.327064in}}%
\pgfpathlineto{\pgfqpoint{1.221605in}{2.409344in}}%
\pgfpathlineto{\pgfqpoint{1.224177in}{2.425800in}}%
\pgfpathlineto{\pgfqpoint{1.226749in}{2.392888in}}%
\pgfpathlineto{\pgfqpoint{1.229321in}{2.392888in}}%
\pgfpathlineto{\pgfqpoint{1.234465in}{2.425800in}}%
\pgfpathlineto{\pgfqpoint{1.237036in}{2.376432in}}%
\pgfpathlineto{\pgfqpoint{1.239608in}{2.376432in}}%
\pgfpathlineto{\pgfqpoint{1.242180in}{2.343520in}}%
\pgfpathlineto{\pgfqpoint{1.244752in}{2.327064in}}%
\pgfpathlineto{\pgfqpoint{1.249896in}{2.327064in}}%
\pgfpathlineto{\pgfqpoint{1.252468in}{2.359976in}}%
\pgfpathlineto{\pgfqpoint{1.255040in}{2.376432in}}%
\pgfpathlineto{\pgfqpoint{1.257611in}{2.376432in}}%
\pgfpathlineto{\pgfqpoint{1.260183in}{2.343520in}}%
\pgfpathlineto{\pgfqpoint{1.265327in}{2.310608in}}%
\pgfpathlineto{\pgfqpoint{1.267899in}{2.343520in}}%
\pgfpathlineto{\pgfqpoint{1.270471in}{2.343520in}}%
\pgfpathlineto{\pgfqpoint{1.275615in}{2.376432in}}%
\pgfpathlineto{\pgfqpoint{1.278186in}{2.359976in}}%
\pgfpathlineto{\pgfqpoint{1.280758in}{2.376432in}}%
\pgfpathlineto{\pgfqpoint{1.283330in}{2.409344in}}%
\pgfpathlineto{\pgfqpoint{1.285902in}{2.409344in}}%
\pgfpathlineto{\pgfqpoint{1.288474in}{2.376432in}}%
\pgfpathlineto{\pgfqpoint{1.291046in}{2.409344in}}%
\pgfpathlineto{\pgfqpoint{1.298761in}{2.409344in}}%
\pgfpathlineto{\pgfqpoint{1.301333in}{2.376432in}}%
\pgfpathlineto{\pgfqpoint{1.309049in}{2.376432in}}%
\pgfpathlineto{\pgfqpoint{1.311621in}{2.343520in}}%
\pgfpathlineto{\pgfqpoint{1.321908in}{2.343520in}}%
\pgfpathlineto{\pgfqpoint{1.324480in}{2.294152in}}%
\pgfpathlineto{\pgfqpoint{1.332196in}{2.294152in}}%
\pgfpathlineto{\pgfqpoint{1.334768in}{2.310608in}}%
\pgfpathlineto{\pgfqpoint{1.350199in}{2.310608in}}%
\pgfpathlineto{\pgfqpoint{1.352771in}{2.294152in}}%
\pgfpathlineto{\pgfqpoint{1.355342in}{2.310608in}}%
\pgfpathlineto{\pgfqpoint{1.357914in}{2.277696in}}%
\pgfpathlineto{\pgfqpoint{1.360486in}{2.261240in}}%
\pgfpathlineto{\pgfqpoint{1.363058in}{2.261240in}}%
\pgfpathlineto{\pgfqpoint{1.365630in}{2.294152in}}%
\pgfpathlineto{\pgfqpoint{1.373346in}{2.294152in}}%
\pgfpathlineto{\pgfqpoint{1.375917in}{2.327064in}}%
\pgfpathlineto{\pgfqpoint{1.378489in}{2.343520in}}%
\pgfpathlineto{\pgfqpoint{1.386205in}{2.343520in}}%
\pgfpathlineto{\pgfqpoint{1.388777in}{2.359976in}}%
\pgfpathlineto{\pgfqpoint{1.391349in}{2.327064in}}%
\pgfpathlineto{\pgfqpoint{1.393921in}{2.343520in}}%
\pgfpathlineto{\pgfqpoint{1.406780in}{2.343520in}}%
\pgfpathlineto{\pgfqpoint{1.409352in}{2.294152in}}%
\pgfpathlineto{\pgfqpoint{1.411924in}{2.327064in}}%
\pgfpathlineto{\pgfqpoint{1.414495in}{2.310608in}}%
\pgfpathlineto{\pgfqpoint{1.419639in}{2.310608in}}%
\pgfpathlineto{\pgfqpoint{1.422211in}{2.277696in}}%
\pgfpathlineto{\pgfqpoint{1.424783in}{2.277696in}}%
\pgfpathlineto{\pgfqpoint{1.427355in}{2.261240in}}%
\pgfpathlineto{\pgfqpoint{1.432499in}{2.327064in}}%
\pgfpathlineto{\pgfqpoint{1.435070in}{2.277696in}}%
\pgfpathlineto{\pgfqpoint{1.442786in}{2.277696in}}%
\pgfpathlineto{\pgfqpoint{1.447930in}{2.195415in}}%
\pgfpathlineto{\pgfqpoint{1.453074in}{2.195415in}}%
\pgfpathlineto{\pgfqpoint{1.455645in}{2.146047in}}%
\pgfpathlineto{\pgfqpoint{1.460789in}{2.146047in}}%
\pgfpathlineto{\pgfqpoint{1.463361in}{2.178959in}}%
\pgfpathlineto{\pgfqpoint{1.473649in}{2.178959in}}%
\pgfpathlineto{\pgfqpoint{1.476220in}{2.162503in}}%
\pgfpathlineto{\pgfqpoint{1.481364in}{2.162503in}}%
\pgfpathlineto{\pgfqpoint{1.483936in}{2.146047in}}%
\pgfpathlineto{\pgfqpoint{1.489080in}{2.178959in}}%
\pgfpathlineto{\pgfqpoint{1.491652in}{2.211871in}}%
\pgfpathlineto{\pgfqpoint{1.496795in}{2.211871in}}%
\pgfpathlineto{\pgfqpoint{1.499367in}{2.195415in}}%
\pgfpathlineto{\pgfqpoint{1.501939in}{2.211871in}}%
\pgfpathlineto{\pgfqpoint{1.504511in}{2.162503in}}%
\pgfpathlineto{\pgfqpoint{1.507083in}{2.162503in}}%
\pgfpathlineto{\pgfqpoint{1.509655in}{2.178959in}}%
\pgfpathlineto{\pgfqpoint{1.512227in}{2.129591in}}%
\pgfpathlineto{\pgfqpoint{1.514798in}{2.162503in}}%
\pgfpathlineto{\pgfqpoint{1.517370in}{2.162503in}}%
\pgfpathlineto{\pgfqpoint{1.519942in}{2.146047in}}%
\pgfpathlineto{\pgfqpoint{1.522514in}{2.113135in}}%
\pgfpathlineto{\pgfqpoint{1.530230in}{2.162503in}}%
\pgfpathlineto{\pgfqpoint{1.532802in}{2.146047in}}%
\pgfpathlineto{\pgfqpoint{1.537945in}{2.146047in}}%
\pgfpathlineto{\pgfqpoint{1.540517in}{2.162503in}}%
\pgfpathlineto{\pgfqpoint{1.543089in}{2.146047in}}%
\pgfpathlineto{\pgfqpoint{1.548233in}{2.146047in}}%
\pgfpathlineto{\pgfqpoint{1.550805in}{2.129591in}}%
\pgfpathlineto{\pgfqpoint{1.558520in}{2.129591in}}%
\pgfpathlineto{\pgfqpoint{1.561092in}{2.113135in}}%
\pgfpathlineto{\pgfqpoint{1.563664in}{2.113135in}}%
\pgfpathlineto{\pgfqpoint{1.566236in}{2.129591in}}%
\pgfpathlineto{\pgfqpoint{1.568808in}{2.113135in}}%
\pgfpathlineto{\pgfqpoint{1.571380in}{2.113135in}}%
\pgfpathlineto{\pgfqpoint{1.573951in}{2.080223in}}%
\pgfpathlineto{\pgfqpoint{1.591955in}{2.080223in}}%
\pgfpathlineto{\pgfqpoint{1.594526in}{2.063767in}}%
\pgfpathlineto{\pgfqpoint{1.597098in}{2.063767in}}%
\pgfpathlineto{\pgfqpoint{1.599670in}{2.080223in}}%
\pgfpathlineto{\pgfqpoint{1.602242in}{2.063767in}}%
\pgfpathlineto{\pgfqpoint{1.604814in}{2.080223in}}%
\pgfpathlineto{\pgfqpoint{1.609958in}{2.080223in}}%
\pgfpathlineto{\pgfqpoint{1.612529in}{2.047311in}}%
\pgfpathlineto{\pgfqpoint{1.615101in}{2.063767in}}%
\pgfpathlineto{\pgfqpoint{1.617673in}{2.063767in}}%
\pgfpathlineto{\pgfqpoint{1.620245in}{2.080223in}}%
\pgfpathlineto{\pgfqpoint{1.625389in}{2.080223in}}%
\pgfpathlineto{\pgfqpoint{1.627961in}{2.096679in}}%
\pgfpathlineto{\pgfqpoint{1.635676in}{2.096679in}}%
\pgfpathlineto{\pgfqpoint{1.638248in}{2.113135in}}%
\pgfpathlineto{\pgfqpoint{1.651108in}{2.113135in}}%
\pgfpathlineto{\pgfqpoint{1.653679in}{2.080223in}}%
\pgfpathlineto{\pgfqpoint{1.656251in}{2.080223in}}%
\pgfpathlineto{\pgfqpoint{1.658823in}{2.063767in}}%
\pgfpathlineto{\pgfqpoint{1.661395in}{2.080223in}}%
\pgfpathlineto{\pgfqpoint{1.671683in}{2.080223in}}%
\pgfpathlineto{\pgfqpoint{1.674254in}{2.096679in}}%
\pgfpathlineto{\pgfqpoint{1.676826in}{2.096679in}}%
\pgfpathlineto{\pgfqpoint{1.681970in}{2.129591in}}%
\pgfpathlineto{\pgfqpoint{1.699973in}{2.129591in}}%
\pgfpathlineto{\pgfqpoint{1.702545in}{2.113135in}}%
\pgfpathlineto{\pgfqpoint{1.715404in}{2.113135in}}%
\pgfpathlineto{\pgfqpoint{1.717976in}{2.096679in}}%
\pgfpathlineto{\pgfqpoint{1.720548in}{2.096679in}}%
\pgfpathlineto{\pgfqpoint{1.723120in}{2.080223in}}%
\pgfpathlineto{\pgfqpoint{1.725692in}{2.080223in}}%
\pgfpathlineto{\pgfqpoint{1.728264in}{2.096679in}}%
\pgfpathlineto{\pgfqpoint{1.741123in}{2.096679in}}%
\pgfpathlineto{\pgfqpoint{1.743695in}{2.080223in}}%
\pgfpathlineto{\pgfqpoint{1.746267in}{2.096679in}}%
\pgfpathlineto{\pgfqpoint{1.766842in}{2.096679in}}%
\pgfpathlineto{\pgfqpoint{1.769414in}{2.113135in}}%
\pgfpathlineto{\pgfqpoint{1.774557in}{2.080223in}}%
\pgfpathlineto{\pgfqpoint{1.777129in}{2.096679in}}%
\pgfpathlineto{\pgfqpoint{1.779701in}{2.080223in}}%
\pgfpathlineto{\pgfqpoint{1.789989in}{2.080223in}}%
\pgfpathlineto{\pgfqpoint{1.792560in}{2.096679in}}%
\pgfpathlineto{\pgfqpoint{1.800276in}{2.096679in}}%
\pgfpathlineto{\pgfqpoint{1.802848in}{2.063767in}}%
\pgfpathlineto{\pgfqpoint{1.805420in}{2.080223in}}%
\pgfpathlineto{\pgfqpoint{1.810563in}{2.080223in}}%
\pgfpathlineto{\pgfqpoint{1.815707in}{2.113135in}}%
\pgfpathlineto{\pgfqpoint{1.823423in}{2.113135in}}%
\pgfpathlineto{\pgfqpoint{1.825995in}{2.129591in}}%
\pgfpathlineto{\pgfqpoint{1.828567in}{2.113135in}}%
\pgfpathlineto{\pgfqpoint{1.831138in}{2.113135in}}%
\pgfpathlineto{\pgfqpoint{1.833710in}{2.096679in}}%
\pgfpathlineto{\pgfqpoint{1.843998in}{2.096679in}}%
\pgfpathlineto{\pgfqpoint{1.846570in}{2.080223in}}%
\pgfpathlineto{\pgfqpoint{1.849142in}{2.080223in}}%
\pgfpathlineto{\pgfqpoint{1.851713in}{2.047311in}}%
\pgfpathlineto{\pgfqpoint{1.854285in}{2.047311in}}%
\pgfpathlineto{\pgfqpoint{1.856857in}{2.080223in}}%
\pgfpathlineto{\pgfqpoint{1.885148in}{2.080223in}}%
\pgfpathlineto{\pgfqpoint{1.887720in}{2.063767in}}%
\pgfpathlineto{\pgfqpoint{1.890291in}{2.080223in}}%
\pgfpathlineto{\pgfqpoint{1.895435in}{2.080223in}}%
\pgfpathlineto{\pgfqpoint{1.898007in}{2.063767in}}%
\pgfpathlineto{\pgfqpoint{1.900579in}{2.080223in}}%
\pgfpathlineto{\pgfqpoint{1.903151in}{2.080223in}}%
\pgfpathlineto{\pgfqpoint{1.908295in}{2.113135in}}%
\pgfpathlineto{\pgfqpoint{1.913438in}{2.113135in}}%
\pgfpathlineto{\pgfqpoint{1.918582in}{2.146047in}}%
\pgfpathlineto{\pgfqpoint{1.921154in}{2.146047in}}%
\pgfpathlineto{\pgfqpoint{1.923726in}{2.162503in}}%
\pgfpathlineto{\pgfqpoint{1.928870in}{2.162503in}}%
\pgfpathlineto{\pgfqpoint{1.931441in}{2.146047in}}%
\pgfpathlineto{\pgfqpoint{1.939157in}{2.146047in}}%
\pgfpathlineto{\pgfqpoint{1.941729in}{2.162503in}}%
\pgfpathlineto{\pgfqpoint{1.944301in}{2.162503in}}%
\pgfpathlineto{\pgfqpoint{1.946873in}{2.178959in}}%
\pgfpathlineto{\pgfqpoint{1.952016in}{2.146047in}}%
\pgfpathlineto{\pgfqpoint{1.954588in}{2.146047in}}%
\pgfpathlineto{\pgfqpoint{1.957160in}{2.129591in}}%
\pgfpathlineto{\pgfqpoint{1.967448in}{2.129591in}}%
\pgfpathlineto{\pgfqpoint{1.970019in}{2.113135in}}%
\pgfpathlineto{\pgfqpoint{1.975163in}{2.146047in}}%
\pgfpathlineto{\pgfqpoint{2.003454in}{2.146047in}}%
\pgfpathlineto{\pgfqpoint{2.006026in}{2.129591in}}%
\pgfpathlineto{\pgfqpoint{2.008597in}{2.129591in}}%
\pgfpathlineto{\pgfqpoint{2.011169in}{2.146047in}}%
\pgfpathlineto{\pgfqpoint{2.024029in}{2.146047in}}%
\pgfpathlineto{\pgfqpoint{2.026601in}{2.129591in}}%
\pgfpathlineto{\pgfqpoint{2.029172in}{2.129591in}}%
\pgfpathlineto{\pgfqpoint{2.031744in}{2.113135in}}%
\pgfpathlineto{\pgfqpoint{2.034316in}{2.129591in}}%
\pgfpathlineto{\pgfqpoint{2.036888in}{2.113135in}}%
\pgfpathlineto{\pgfqpoint{2.039460in}{2.129591in}}%
\pgfpathlineto{\pgfqpoint{2.042032in}{2.129591in}}%
\pgfpathlineto{\pgfqpoint{2.044604in}{2.146047in}}%
\pgfpathlineto{\pgfqpoint{2.085754in}{2.146047in}}%
\pgfpathlineto{\pgfqpoint{2.088325in}{2.162503in}}%
\pgfpathlineto{\pgfqpoint{2.111472in}{2.162503in}}%
\pgfpathlineto{\pgfqpoint{2.114044in}{2.146047in}}%
\pgfpathlineto{\pgfqpoint{2.124332in}{2.146047in}}%
\pgfpathlineto{\pgfqpoint{2.126904in}{2.162503in}}%
\pgfpathlineto{\pgfqpoint{2.142335in}{2.162503in}}%
\pgfpathlineto{\pgfqpoint{2.144907in}{2.146047in}}%
\pgfpathlineto{\pgfqpoint{2.150050in}{2.146047in}}%
\pgfpathlineto{\pgfqpoint{2.152622in}{2.162503in}}%
\pgfpathlineto{\pgfqpoint{2.155194in}{2.146047in}}%
\pgfpathlineto{\pgfqpoint{2.168053in}{2.146047in}}%
\pgfpathlineto{\pgfqpoint{2.170625in}{2.129591in}}%
\pgfpathlineto{\pgfqpoint{2.173197in}{2.146047in}}%
\pgfpathlineto{\pgfqpoint{2.175769in}{2.129591in}}%
\pgfpathlineto{\pgfqpoint{2.201488in}{2.129591in}}%
\pgfpathlineto{\pgfqpoint{2.204060in}{2.113135in}}%
\pgfpathlineto{\pgfqpoint{2.219491in}{2.113135in}}%
\pgfpathlineto{\pgfqpoint{2.222063in}{2.129591in}}%
\pgfpathlineto{\pgfqpoint{2.234922in}{2.129591in}}%
\pgfpathlineto{\pgfqpoint{2.237494in}{2.113135in}}%
\pgfpathlineto{\pgfqpoint{2.245210in}{2.113135in}}%
\pgfpathlineto{\pgfqpoint{2.247781in}{2.129591in}}%
\pgfpathlineto{\pgfqpoint{2.250353in}{2.113135in}}%
\pgfpathlineto{\pgfqpoint{2.255497in}{2.113135in}}%
\pgfpathlineto{\pgfqpoint{2.258069in}{2.129591in}}%
\pgfpathlineto{\pgfqpoint{2.278644in}{2.129591in}}%
\pgfpathlineto{\pgfqpoint{2.281216in}{2.113135in}}%
\pgfpathlineto{\pgfqpoint{2.288931in}{2.113135in}}%
\pgfpathlineto{\pgfqpoint{2.291503in}{2.080223in}}%
\pgfpathlineto{\pgfqpoint{2.294075in}{2.096679in}}%
\pgfpathlineto{\pgfqpoint{2.304363in}{2.096679in}}%
\pgfpathlineto{\pgfqpoint{2.306934in}{2.080223in}}%
\pgfpathlineto{\pgfqpoint{2.314650in}{2.080223in}}%
\pgfpathlineto{\pgfqpoint{2.317222in}{2.096679in}}%
\pgfpathlineto{\pgfqpoint{2.319794in}{2.096679in}}%
\pgfpathlineto{\pgfqpoint{2.322366in}{2.080223in}}%
\pgfpathlineto{\pgfqpoint{2.345512in}{2.080223in}}%
\pgfpathlineto{\pgfqpoint{2.348084in}{2.096679in}}%
\pgfpathlineto{\pgfqpoint{2.353228in}{2.096679in}}%
\pgfpathlineto{\pgfqpoint{2.355800in}{2.080223in}}%
\pgfpathlineto{\pgfqpoint{2.438100in}{2.080223in}}%
\pgfpathlineto{\pgfqpoint{2.440672in}{2.096679in}}%
\pgfpathlineto{\pgfqpoint{2.443244in}{2.096679in}}%
\pgfpathlineto{\pgfqpoint{2.445815in}{2.080223in}}%
\pgfpathlineto{\pgfqpoint{2.450959in}{2.080223in}}%
\pgfpathlineto{\pgfqpoint{2.453531in}{2.096679in}}%
\pgfpathlineto{\pgfqpoint{2.466390in}{2.096679in}}%
\pgfpathlineto{\pgfqpoint{2.468962in}{2.113135in}}%
\pgfpathlineto{\pgfqpoint{2.471534in}{2.113135in}}%
\pgfpathlineto{\pgfqpoint{2.474106in}{2.129591in}}%
\pgfpathlineto{\pgfqpoint{2.476678in}{2.129591in}}%
\pgfpathlineto{\pgfqpoint{2.479250in}{2.113135in}}%
\pgfpathlineto{\pgfqpoint{2.502397in}{2.113135in}}%
\pgfpathlineto{\pgfqpoint{2.504968in}{2.129591in}}%
\pgfpathlineto{\pgfqpoint{2.517828in}{2.129591in}}%
\pgfpathlineto{\pgfqpoint{2.520400in}{2.113135in}}%
\pgfpathlineto{\pgfqpoint{2.522972in}{2.113135in}}%
\pgfpathlineto{\pgfqpoint{2.528115in}{2.080223in}}%
\pgfpathlineto{\pgfqpoint{2.551262in}{2.080223in}}%
\pgfpathlineto{\pgfqpoint{2.553834in}{2.063767in}}%
\pgfpathlineto{\pgfqpoint{2.556406in}{2.080223in}}%
\pgfpathlineto{\pgfqpoint{2.561550in}{2.080223in}}%
\pgfpathlineto{\pgfqpoint{2.564121in}{2.096679in}}%
\pgfpathlineto{\pgfqpoint{2.574409in}{2.096679in}}%
\pgfpathlineto{\pgfqpoint{2.576981in}{2.113135in}}%
\pgfpathlineto{\pgfqpoint{2.579553in}{2.096679in}}%
\pgfpathlineto{\pgfqpoint{2.589840in}{2.096679in}}%
\pgfpathlineto{\pgfqpoint{2.594984in}{2.063767in}}%
\pgfpathlineto{\pgfqpoint{2.597556in}{2.096679in}}%
\pgfpathlineto{\pgfqpoint{2.600128in}{2.096679in}}%
\pgfpathlineto{\pgfqpoint{2.602699in}{2.113135in}}%
\pgfpathlineto{\pgfqpoint{2.605271in}{2.113135in}}%
\pgfpathlineto{\pgfqpoint{2.610415in}{2.080223in}}%
\pgfpathlineto{\pgfqpoint{2.612987in}{2.096679in}}%
\pgfpathlineto{\pgfqpoint{2.615559in}{2.080223in}}%
\pgfpathlineto{\pgfqpoint{2.618131in}{2.096679in}}%
\pgfpathlineto{\pgfqpoint{2.620703in}{2.080223in}}%
\pgfpathlineto{\pgfqpoint{2.646421in}{2.080223in}}%
\pgfpathlineto{\pgfqpoint{2.646421in}{2.080223in}}%
\pgfusepath{stroke}%
\end{pgfscope}%
\begin{pgfscope}%
\pgfpathrectangle{\pgfqpoint{0.488751in}{1.946106in}}{\pgfqpoint{2.260417in}{1.502439in}}%
\pgfusepath{clip}%
\pgfsetbuttcap%
\pgfsetroundjoin%
\pgfsetlinewidth{0.803000pt}%
\definecolor{currentstroke}{rgb}{0.000000,0.356863,0.509804}%
\pgfsetstrokecolor{currentstroke}%
\pgfsetdash{{2.960000pt}{1.280000pt}}{0.000000pt}%
\pgfpathmoveto{\pgfqpoint{0.591497in}{2.458712in}}%
\pgfpathlineto{\pgfqpoint{0.596641in}{2.425800in}}%
\pgfpathlineto{\pgfqpoint{0.599213in}{2.442256in}}%
\pgfpathlineto{\pgfqpoint{0.604356in}{2.508081in}}%
\pgfpathlineto{\pgfqpoint{0.606928in}{2.557449in}}%
\pgfpathlineto{\pgfqpoint{0.609500in}{2.557449in}}%
\pgfpathlineto{\pgfqpoint{0.612072in}{2.524537in}}%
\pgfpathlineto{\pgfqpoint{0.614644in}{2.540993in}}%
\pgfpathlineto{\pgfqpoint{0.617216in}{2.524537in}}%
\pgfpathlineto{\pgfqpoint{0.619788in}{2.524537in}}%
\pgfpathlineto{\pgfqpoint{0.622359in}{2.491625in}}%
\pgfpathlineto{\pgfqpoint{0.627503in}{2.623273in}}%
\pgfpathlineto{\pgfqpoint{0.630075in}{2.656185in}}%
\pgfpathlineto{\pgfqpoint{0.632647in}{2.557449in}}%
\pgfpathlineto{\pgfqpoint{0.637791in}{2.672641in}}%
\pgfpathlineto{\pgfqpoint{0.640363in}{2.656185in}}%
\pgfpathlineto{\pgfqpoint{0.642934in}{2.623273in}}%
\pgfpathlineto{\pgfqpoint{0.645506in}{2.639729in}}%
\pgfpathlineto{\pgfqpoint{0.648078in}{2.606817in}}%
\pgfpathlineto{\pgfqpoint{0.650650in}{2.623273in}}%
\pgfpathlineto{\pgfqpoint{0.653222in}{2.623273in}}%
\pgfpathlineto{\pgfqpoint{0.658366in}{2.656185in}}%
\pgfpathlineto{\pgfqpoint{0.660938in}{2.623273in}}%
\pgfpathlineto{\pgfqpoint{0.663509in}{2.623273in}}%
\pgfpathlineto{\pgfqpoint{0.666081in}{2.639729in}}%
\pgfpathlineto{\pgfqpoint{0.668653in}{2.606817in}}%
\pgfpathlineto{\pgfqpoint{0.671225in}{2.606817in}}%
\pgfpathlineto{\pgfqpoint{0.673797in}{2.573905in}}%
\pgfpathlineto{\pgfqpoint{0.676369in}{2.491625in}}%
\pgfpathlineto{\pgfqpoint{0.678941in}{2.557449in}}%
\pgfpathlineto{\pgfqpoint{0.684084in}{2.557449in}}%
\pgfpathlineto{\pgfqpoint{0.686656in}{2.540993in}}%
\pgfpathlineto{\pgfqpoint{0.689228in}{2.508081in}}%
\pgfpathlineto{\pgfqpoint{0.691800in}{2.573905in}}%
\pgfpathlineto{\pgfqpoint{0.694372in}{2.557449in}}%
\pgfpathlineto{\pgfqpoint{0.696944in}{2.557449in}}%
\pgfpathlineto{\pgfqpoint{0.699516in}{2.590361in}}%
\pgfpathlineto{\pgfqpoint{0.702087in}{2.639729in}}%
\pgfpathlineto{\pgfqpoint{0.704659in}{2.606817in}}%
\pgfpathlineto{\pgfqpoint{0.707231in}{2.606817in}}%
\pgfpathlineto{\pgfqpoint{0.709803in}{2.639729in}}%
\pgfpathlineto{\pgfqpoint{0.712375in}{2.705554in}}%
\pgfpathlineto{\pgfqpoint{0.714947in}{2.722010in}}%
\pgfpathlineto{\pgfqpoint{0.717519in}{2.722010in}}%
\pgfpathlineto{\pgfqpoint{0.720091in}{2.705554in}}%
\pgfpathlineto{\pgfqpoint{0.722662in}{2.738466in}}%
\pgfpathlineto{\pgfqpoint{0.725234in}{2.738466in}}%
\pgfpathlineto{\pgfqpoint{0.727806in}{2.722010in}}%
\pgfpathlineto{\pgfqpoint{0.730378in}{2.722010in}}%
\pgfpathlineto{\pgfqpoint{0.732950in}{2.672641in}}%
\pgfpathlineto{\pgfqpoint{0.735522in}{2.689097in}}%
\pgfpathlineto{\pgfqpoint{0.738094in}{2.656185in}}%
\pgfpathlineto{\pgfqpoint{0.740666in}{2.639729in}}%
\pgfpathlineto{\pgfqpoint{0.743237in}{2.672641in}}%
\pgfpathlineto{\pgfqpoint{0.745809in}{2.656185in}}%
\pgfpathlineto{\pgfqpoint{0.748381in}{2.656185in}}%
\pgfpathlineto{\pgfqpoint{0.750953in}{2.623273in}}%
\pgfpathlineto{\pgfqpoint{0.753525in}{2.656185in}}%
\pgfpathlineto{\pgfqpoint{0.756097in}{2.656185in}}%
\pgfpathlineto{\pgfqpoint{0.758669in}{2.672641in}}%
\pgfpathlineto{\pgfqpoint{0.761240in}{2.639729in}}%
\pgfpathlineto{\pgfqpoint{0.763812in}{2.639729in}}%
\pgfpathlineto{\pgfqpoint{0.766384in}{2.606817in}}%
\pgfpathlineto{\pgfqpoint{0.771528in}{2.606817in}}%
\pgfpathlineto{\pgfqpoint{0.774100in}{2.623273in}}%
\pgfpathlineto{\pgfqpoint{0.779244in}{2.540993in}}%
\pgfpathlineto{\pgfqpoint{0.786959in}{2.491625in}}%
\pgfpathlineto{\pgfqpoint{0.792103in}{2.524537in}}%
\pgfpathlineto{\pgfqpoint{0.794675in}{2.491625in}}%
\pgfpathlineto{\pgfqpoint{0.799819in}{2.491625in}}%
\pgfpathlineto{\pgfqpoint{0.802390in}{2.524537in}}%
\pgfpathlineto{\pgfqpoint{0.804962in}{2.524537in}}%
\pgfpathlineto{\pgfqpoint{0.807534in}{2.508081in}}%
\pgfpathlineto{\pgfqpoint{0.810106in}{2.425800in}}%
\pgfpathlineto{\pgfqpoint{0.812678in}{2.442256in}}%
\pgfpathlineto{\pgfqpoint{0.815250in}{2.425800in}}%
\pgfpathlineto{\pgfqpoint{0.817822in}{2.425800in}}%
\pgfpathlineto{\pgfqpoint{0.820393in}{2.458712in}}%
\pgfpathlineto{\pgfqpoint{0.822965in}{2.458712in}}%
\pgfpathlineto{\pgfqpoint{0.825537in}{2.442256in}}%
\pgfpathlineto{\pgfqpoint{0.828109in}{2.458712in}}%
\pgfpathlineto{\pgfqpoint{0.830681in}{2.458712in}}%
\pgfpathlineto{\pgfqpoint{0.833253in}{2.442256in}}%
\pgfpathlineto{\pgfqpoint{0.835825in}{2.442256in}}%
\pgfpathlineto{\pgfqpoint{0.838397in}{2.458712in}}%
\pgfpathlineto{\pgfqpoint{0.840968in}{2.425800in}}%
\pgfpathlineto{\pgfqpoint{0.843540in}{2.425800in}}%
\pgfpathlineto{\pgfqpoint{0.846112in}{2.392888in}}%
\pgfpathlineto{\pgfqpoint{0.848684in}{2.442256in}}%
\pgfpathlineto{\pgfqpoint{0.851256in}{2.458712in}}%
\pgfpathlineto{\pgfqpoint{0.853828in}{2.458712in}}%
\pgfpathlineto{\pgfqpoint{0.856400in}{2.491625in}}%
\pgfpathlineto{\pgfqpoint{0.858972in}{2.508081in}}%
\pgfpathlineto{\pgfqpoint{0.861543in}{2.475169in}}%
\pgfpathlineto{\pgfqpoint{0.866687in}{2.540993in}}%
\pgfpathlineto{\pgfqpoint{0.869259in}{2.557449in}}%
\pgfpathlineto{\pgfqpoint{0.871831in}{2.590361in}}%
\pgfpathlineto{\pgfqpoint{0.874403in}{2.606817in}}%
\pgfpathlineto{\pgfqpoint{0.879547in}{2.524537in}}%
\pgfpathlineto{\pgfqpoint{0.882118in}{2.557449in}}%
\pgfpathlineto{\pgfqpoint{0.884690in}{2.508081in}}%
\pgfpathlineto{\pgfqpoint{0.887262in}{2.491625in}}%
\pgfpathlineto{\pgfqpoint{0.889834in}{2.491625in}}%
\pgfpathlineto{\pgfqpoint{0.892406in}{2.524537in}}%
\pgfpathlineto{\pgfqpoint{0.894978in}{2.475169in}}%
\pgfpathlineto{\pgfqpoint{0.897550in}{2.491625in}}%
\pgfpathlineto{\pgfqpoint{0.900121in}{2.475169in}}%
\pgfpathlineto{\pgfqpoint{0.902693in}{2.508081in}}%
\pgfpathlineto{\pgfqpoint{0.905265in}{2.491625in}}%
\pgfpathlineto{\pgfqpoint{0.907837in}{2.442256in}}%
\pgfpathlineto{\pgfqpoint{0.910409in}{2.442256in}}%
\pgfpathlineto{\pgfqpoint{0.912981in}{2.491625in}}%
\pgfpathlineto{\pgfqpoint{0.915553in}{2.442256in}}%
\pgfpathlineto{\pgfqpoint{0.918125in}{2.442256in}}%
\pgfpathlineto{\pgfqpoint{0.920696in}{2.392888in}}%
\pgfpathlineto{\pgfqpoint{0.923268in}{2.425800in}}%
\pgfpathlineto{\pgfqpoint{0.925840in}{2.442256in}}%
\pgfpathlineto{\pgfqpoint{0.928412in}{2.475169in}}%
\pgfpathlineto{\pgfqpoint{0.930984in}{2.491625in}}%
\pgfpathlineto{\pgfqpoint{0.933556in}{2.573905in}}%
\pgfpathlineto{\pgfqpoint{0.936128in}{2.590361in}}%
\pgfpathlineto{\pgfqpoint{0.938700in}{2.557449in}}%
\pgfpathlineto{\pgfqpoint{0.941271in}{2.557449in}}%
\pgfpathlineto{\pgfqpoint{0.943843in}{2.540993in}}%
\pgfpathlineto{\pgfqpoint{0.946415in}{2.557449in}}%
\pgfpathlineto{\pgfqpoint{0.948987in}{2.524537in}}%
\pgfpathlineto{\pgfqpoint{0.951559in}{2.524537in}}%
\pgfpathlineto{\pgfqpoint{0.954131in}{2.540993in}}%
\pgfpathlineto{\pgfqpoint{0.956703in}{2.540993in}}%
\pgfpathlineto{\pgfqpoint{0.959274in}{2.524537in}}%
\pgfpathlineto{\pgfqpoint{0.961846in}{2.524537in}}%
\pgfpathlineto{\pgfqpoint{0.964418in}{2.458712in}}%
\pgfpathlineto{\pgfqpoint{0.966990in}{2.475169in}}%
\pgfpathlineto{\pgfqpoint{0.969562in}{2.458712in}}%
\pgfpathlineto{\pgfqpoint{0.972134in}{2.425800in}}%
\pgfpathlineto{\pgfqpoint{0.974706in}{2.442256in}}%
\pgfpathlineto{\pgfqpoint{0.977278in}{2.475169in}}%
\pgfpathlineto{\pgfqpoint{0.979849in}{2.458712in}}%
\pgfpathlineto{\pgfqpoint{0.987565in}{2.508081in}}%
\pgfpathlineto{\pgfqpoint{0.990137in}{2.557449in}}%
\pgfpathlineto{\pgfqpoint{0.992709in}{2.573905in}}%
\pgfpathlineto{\pgfqpoint{0.995281in}{2.557449in}}%
\pgfpathlineto{\pgfqpoint{1.000424in}{2.557449in}}%
\pgfpathlineto{\pgfqpoint{1.002996in}{2.540993in}}%
\pgfpathlineto{\pgfqpoint{1.008140in}{2.540993in}}%
\pgfpathlineto{\pgfqpoint{1.010712in}{2.491625in}}%
\pgfpathlineto{\pgfqpoint{1.015856in}{2.491625in}}%
\pgfpathlineto{\pgfqpoint{1.020999in}{2.409344in}}%
\pgfpathlineto{\pgfqpoint{1.023571in}{2.409344in}}%
\pgfpathlineto{\pgfqpoint{1.026143in}{2.425800in}}%
\pgfpathlineto{\pgfqpoint{1.028715in}{2.458712in}}%
\pgfpathlineto{\pgfqpoint{1.031287in}{2.475169in}}%
\pgfpathlineto{\pgfqpoint{1.033859in}{2.458712in}}%
\pgfpathlineto{\pgfqpoint{1.036431in}{2.475169in}}%
\pgfpathlineto{\pgfqpoint{1.039002in}{2.475169in}}%
\pgfpathlineto{\pgfqpoint{1.049290in}{2.409344in}}%
\pgfpathlineto{\pgfqpoint{1.051862in}{2.376432in}}%
\pgfpathlineto{\pgfqpoint{1.054434in}{2.376432in}}%
\pgfpathlineto{\pgfqpoint{1.057006in}{2.409344in}}%
\pgfpathlineto{\pgfqpoint{1.064721in}{2.458712in}}%
\pgfpathlineto{\pgfqpoint{1.069865in}{2.524537in}}%
\pgfpathlineto{\pgfqpoint{1.075009in}{2.557449in}}%
\pgfpathlineto{\pgfqpoint{1.080152in}{2.557449in}}%
\pgfpathlineto{\pgfqpoint{1.082724in}{2.540993in}}%
\pgfpathlineto{\pgfqpoint{1.085296in}{2.540993in}}%
\pgfpathlineto{\pgfqpoint{1.087868in}{2.508081in}}%
\pgfpathlineto{\pgfqpoint{1.090440in}{2.524537in}}%
\pgfpathlineto{\pgfqpoint{1.093012in}{2.557449in}}%
\pgfpathlineto{\pgfqpoint{1.095584in}{2.573905in}}%
\pgfpathlineto{\pgfqpoint{1.098155in}{2.557449in}}%
\pgfpathlineto{\pgfqpoint{1.100727in}{2.590361in}}%
\pgfpathlineto{\pgfqpoint{1.103299in}{2.573905in}}%
\pgfpathlineto{\pgfqpoint{1.108443in}{2.573905in}}%
\pgfpathlineto{\pgfqpoint{1.111015in}{2.524537in}}%
\pgfpathlineto{\pgfqpoint{1.113587in}{2.508081in}}%
\pgfpathlineto{\pgfqpoint{1.116159in}{2.458712in}}%
\pgfpathlineto{\pgfqpoint{1.118730in}{2.458712in}}%
\pgfpathlineto{\pgfqpoint{1.123874in}{2.425800in}}%
\pgfpathlineto{\pgfqpoint{1.126446in}{2.425800in}}%
\pgfpathlineto{\pgfqpoint{1.129018in}{2.409344in}}%
\pgfpathlineto{\pgfqpoint{1.131590in}{2.409344in}}%
\pgfpathlineto{\pgfqpoint{1.134162in}{2.425800in}}%
\pgfpathlineto{\pgfqpoint{1.136734in}{2.425800in}}%
\pgfpathlineto{\pgfqpoint{1.139305in}{2.458712in}}%
\pgfpathlineto{\pgfqpoint{1.147021in}{2.458712in}}%
\pgfpathlineto{\pgfqpoint{1.149593in}{2.475169in}}%
\pgfpathlineto{\pgfqpoint{1.152165in}{2.508081in}}%
\pgfpathlineto{\pgfqpoint{1.162452in}{2.508081in}}%
\pgfpathlineto{\pgfqpoint{1.165024in}{2.491625in}}%
\pgfpathlineto{\pgfqpoint{1.167596in}{2.508081in}}%
\pgfpathlineto{\pgfqpoint{1.170168in}{2.508081in}}%
\pgfpathlineto{\pgfqpoint{1.172740in}{2.524537in}}%
\pgfpathlineto{\pgfqpoint{1.175312in}{2.508081in}}%
\pgfpathlineto{\pgfqpoint{1.177883in}{2.557449in}}%
\pgfpathlineto{\pgfqpoint{1.183027in}{2.524537in}}%
\pgfpathlineto{\pgfqpoint{1.185599in}{2.540993in}}%
\pgfpathlineto{\pgfqpoint{1.188171in}{2.540993in}}%
\pgfpathlineto{\pgfqpoint{1.190743in}{2.524537in}}%
\pgfpathlineto{\pgfqpoint{1.198458in}{2.524537in}}%
\pgfpathlineto{\pgfqpoint{1.201030in}{2.442256in}}%
\pgfpathlineto{\pgfqpoint{1.206174in}{2.475169in}}%
\pgfpathlineto{\pgfqpoint{1.208746in}{2.458712in}}%
\pgfpathlineto{\pgfqpoint{1.211318in}{2.491625in}}%
\pgfpathlineto{\pgfqpoint{1.216461in}{2.491625in}}%
\pgfpathlineto{\pgfqpoint{1.219033in}{2.442256in}}%
\pgfpathlineto{\pgfqpoint{1.221605in}{2.442256in}}%
\pgfpathlineto{\pgfqpoint{1.224177in}{2.475169in}}%
\pgfpathlineto{\pgfqpoint{1.226749in}{2.425800in}}%
\pgfpathlineto{\pgfqpoint{1.229321in}{2.442256in}}%
\pgfpathlineto{\pgfqpoint{1.231893in}{2.442256in}}%
\pgfpathlineto{\pgfqpoint{1.234465in}{2.458712in}}%
\pgfpathlineto{\pgfqpoint{1.237036in}{2.409344in}}%
\pgfpathlineto{\pgfqpoint{1.239608in}{2.409344in}}%
\pgfpathlineto{\pgfqpoint{1.242180in}{2.359976in}}%
\pgfpathlineto{\pgfqpoint{1.244752in}{2.343520in}}%
\pgfpathlineto{\pgfqpoint{1.247324in}{2.359976in}}%
\pgfpathlineto{\pgfqpoint{1.249896in}{2.359976in}}%
\pgfpathlineto{\pgfqpoint{1.252468in}{2.442256in}}%
\pgfpathlineto{\pgfqpoint{1.257611in}{2.442256in}}%
\pgfpathlineto{\pgfqpoint{1.260183in}{2.491625in}}%
\pgfpathlineto{\pgfqpoint{1.262755in}{2.491625in}}%
\pgfpathlineto{\pgfqpoint{1.265327in}{2.458712in}}%
\pgfpathlineto{\pgfqpoint{1.267899in}{2.442256in}}%
\pgfpathlineto{\pgfqpoint{1.270471in}{2.458712in}}%
\pgfpathlineto{\pgfqpoint{1.273043in}{2.458712in}}%
\pgfpathlineto{\pgfqpoint{1.275615in}{2.491625in}}%
\pgfpathlineto{\pgfqpoint{1.278186in}{2.475169in}}%
\pgfpathlineto{\pgfqpoint{1.285902in}{2.475169in}}%
\pgfpathlineto{\pgfqpoint{1.288474in}{2.442256in}}%
\pgfpathlineto{\pgfqpoint{1.291046in}{2.475169in}}%
\pgfpathlineto{\pgfqpoint{1.293618in}{2.491625in}}%
\pgfpathlineto{\pgfqpoint{1.296189in}{2.475169in}}%
\pgfpathlineto{\pgfqpoint{1.301333in}{2.475169in}}%
\pgfpathlineto{\pgfqpoint{1.303905in}{2.491625in}}%
\pgfpathlineto{\pgfqpoint{1.306477in}{2.491625in}}%
\pgfpathlineto{\pgfqpoint{1.309049in}{2.508081in}}%
\pgfpathlineto{\pgfqpoint{1.314193in}{2.442256in}}%
\pgfpathlineto{\pgfqpoint{1.316764in}{2.425800in}}%
\pgfpathlineto{\pgfqpoint{1.319336in}{2.392888in}}%
\pgfpathlineto{\pgfqpoint{1.321908in}{2.376432in}}%
\pgfpathlineto{\pgfqpoint{1.324480in}{2.327064in}}%
\pgfpathlineto{\pgfqpoint{1.327052in}{2.359976in}}%
\pgfpathlineto{\pgfqpoint{1.329624in}{2.376432in}}%
\pgfpathlineto{\pgfqpoint{1.332196in}{2.376432in}}%
\pgfpathlineto{\pgfqpoint{1.334768in}{2.359976in}}%
\pgfpathlineto{\pgfqpoint{1.337339in}{2.359976in}}%
\pgfpathlineto{\pgfqpoint{1.342483in}{2.392888in}}%
\pgfpathlineto{\pgfqpoint{1.350199in}{2.392888in}}%
\pgfpathlineto{\pgfqpoint{1.355342in}{2.491625in}}%
\pgfpathlineto{\pgfqpoint{1.360486in}{2.425800in}}%
\pgfpathlineto{\pgfqpoint{1.363058in}{2.425800in}}%
\pgfpathlineto{\pgfqpoint{1.365630in}{2.442256in}}%
\pgfpathlineto{\pgfqpoint{1.368202in}{2.409344in}}%
\pgfpathlineto{\pgfqpoint{1.373346in}{2.409344in}}%
\pgfpathlineto{\pgfqpoint{1.375917in}{2.425800in}}%
\pgfpathlineto{\pgfqpoint{1.386205in}{2.425800in}}%
\pgfpathlineto{\pgfqpoint{1.388777in}{2.458712in}}%
\pgfpathlineto{\pgfqpoint{1.391349in}{2.425800in}}%
\pgfpathlineto{\pgfqpoint{1.406780in}{2.327064in}}%
\pgfpathlineto{\pgfqpoint{1.409352in}{2.327064in}}%
\pgfpathlineto{\pgfqpoint{1.411924in}{2.376432in}}%
\pgfpathlineto{\pgfqpoint{1.417067in}{2.409344in}}%
\pgfpathlineto{\pgfqpoint{1.419639in}{2.409344in}}%
\pgfpathlineto{\pgfqpoint{1.422211in}{2.425800in}}%
\pgfpathlineto{\pgfqpoint{1.424783in}{2.458712in}}%
\pgfpathlineto{\pgfqpoint{1.427355in}{2.475169in}}%
\pgfpathlineto{\pgfqpoint{1.432499in}{2.442256in}}%
\pgfpathlineto{\pgfqpoint{1.435070in}{2.475169in}}%
\pgfpathlineto{\pgfqpoint{1.437642in}{2.475169in}}%
\pgfpathlineto{\pgfqpoint{1.440214in}{2.491625in}}%
\pgfpathlineto{\pgfqpoint{1.442786in}{2.491625in}}%
\pgfpathlineto{\pgfqpoint{1.445358in}{2.475169in}}%
\pgfpathlineto{\pgfqpoint{1.447930in}{2.442256in}}%
\pgfpathlineto{\pgfqpoint{1.450502in}{2.442256in}}%
\pgfpathlineto{\pgfqpoint{1.453074in}{2.475169in}}%
\pgfpathlineto{\pgfqpoint{1.455645in}{2.458712in}}%
\pgfpathlineto{\pgfqpoint{1.458217in}{2.491625in}}%
\pgfpathlineto{\pgfqpoint{1.460789in}{2.491625in}}%
\pgfpathlineto{\pgfqpoint{1.463361in}{2.475169in}}%
\pgfpathlineto{\pgfqpoint{1.468505in}{2.475169in}}%
\pgfpathlineto{\pgfqpoint{1.471077in}{2.458712in}}%
\pgfpathlineto{\pgfqpoint{1.476220in}{2.458712in}}%
\pgfpathlineto{\pgfqpoint{1.481364in}{2.425800in}}%
\pgfpathlineto{\pgfqpoint{1.483936in}{2.442256in}}%
\pgfpathlineto{\pgfqpoint{1.486508in}{2.442256in}}%
\pgfpathlineto{\pgfqpoint{1.489080in}{2.458712in}}%
\pgfpathlineto{\pgfqpoint{1.491652in}{2.409344in}}%
\pgfpathlineto{\pgfqpoint{1.494223in}{2.392888in}}%
\pgfpathlineto{\pgfqpoint{1.496795in}{2.392888in}}%
\pgfpathlineto{\pgfqpoint{1.499367in}{2.425800in}}%
\pgfpathlineto{\pgfqpoint{1.501939in}{2.425800in}}%
\pgfpathlineto{\pgfqpoint{1.504511in}{2.409344in}}%
\pgfpathlineto{\pgfqpoint{1.507083in}{2.409344in}}%
\pgfpathlineto{\pgfqpoint{1.509655in}{2.425800in}}%
\pgfpathlineto{\pgfqpoint{1.512227in}{2.409344in}}%
\pgfpathlineto{\pgfqpoint{1.514798in}{2.425800in}}%
\pgfpathlineto{\pgfqpoint{1.517370in}{2.425800in}}%
\pgfpathlineto{\pgfqpoint{1.519942in}{2.442256in}}%
\pgfpathlineto{\pgfqpoint{1.522514in}{2.475169in}}%
\pgfpathlineto{\pgfqpoint{1.525086in}{2.491625in}}%
\pgfpathlineto{\pgfqpoint{1.527658in}{2.491625in}}%
\pgfpathlineto{\pgfqpoint{1.530230in}{2.573905in}}%
\pgfpathlineto{\pgfqpoint{1.532802in}{2.540993in}}%
\pgfpathlineto{\pgfqpoint{1.535373in}{2.557449in}}%
\pgfpathlineto{\pgfqpoint{1.537945in}{2.557449in}}%
\pgfpathlineto{\pgfqpoint{1.540517in}{2.606817in}}%
\pgfpathlineto{\pgfqpoint{1.543089in}{2.623273in}}%
\pgfpathlineto{\pgfqpoint{1.545661in}{2.623273in}}%
\pgfpathlineto{\pgfqpoint{1.548233in}{2.639729in}}%
\pgfpathlineto{\pgfqpoint{1.555948in}{2.639729in}}%
\pgfpathlineto{\pgfqpoint{1.558520in}{2.656185in}}%
\pgfpathlineto{\pgfqpoint{1.563664in}{2.590361in}}%
\pgfpathlineto{\pgfqpoint{1.566236in}{2.590361in}}%
\pgfpathlineto{\pgfqpoint{1.568808in}{2.573905in}}%
\pgfpathlineto{\pgfqpoint{1.571380in}{2.573905in}}%
\pgfpathlineto{\pgfqpoint{1.573951in}{2.557449in}}%
\pgfpathlineto{\pgfqpoint{1.576523in}{2.524537in}}%
\pgfpathlineto{\pgfqpoint{1.579095in}{2.524537in}}%
\pgfpathlineto{\pgfqpoint{1.581667in}{2.540993in}}%
\pgfpathlineto{\pgfqpoint{1.586811in}{2.540993in}}%
\pgfpathlineto{\pgfqpoint{1.589383in}{2.557449in}}%
\pgfpathlineto{\pgfqpoint{1.591955in}{2.540993in}}%
\pgfpathlineto{\pgfqpoint{1.594526in}{2.508081in}}%
\pgfpathlineto{\pgfqpoint{1.597098in}{2.508081in}}%
\pgfpathlineto{\pgfqpoint{1.602242in}{2.475169in}}%
\pgfpathlineto{\pgfqpoint{1.604814in}{2.475169in}}%
\pgfpathlineto{\pgfqpoint{1.609958in}{2.442256in}}%
\pgfpathlineto{\pgfqpoint{1.612529in}{2.409344in}}%
\pgfpathlineto{\pgfqpoint{1.615101in}{2.409344in}}%
\pgfpathlineto{\pgfqpoint{1.617673in}{2.376432in}}%
\pgfpathlineto{\pgfqpoint{1.620245in}{2.392888in}}%
\pgfpathlineto{\pgfqpoint{1.622817in}{2.376432in}}%
\pgfpathlineto{\pgfqpoint{1.625389in}{2.376432in}}%
\pgfpathlineto{\pgfqpoint{1.627961in}{2.392888in}}%
\pgfpathlineto{\pgfqpoint{1.630533in}{2.392888in}}%
\pgfpathlineto{\pgfqpoint{1.635676in}{2.359976in}}%
\pgfpathlineto{\pgfqpoint{1.638248in}{2.359976in}}%
\pgfpathlineto{\pgfqpoint{1.640820in}{2.327064in}}%
\pgfpathlineto{\pgfqpoint{1.643392in}{2.327064in}}%
\pgfpathlineto{\pgfqpoint{1.645964in}{2.310608in}}%
\pgfpathlineto{\pgfqpoint{1.648536in}{2.327064in}}%
\pgfpathlineto{\pgfqpoint{1.651108in}{2.376432in}}%
\pgfpathlineto{\pgfqpoint{1.656251in}{2.376432in}}%
\pgfpathlineto{\pgfqpoint{1.658823in}{2.343520in}}%
\pgfpathlineto{\pgfqpoint{1.661395in}{2.359976in}}%
\pgfpathlineto{\pgfqpoint{1.663967in}{2.392888in}}%
\pgfpathlineto{\pgfqpoint{1.666539in}{2.392888in}}%
\pgfpathlineto{\pgfqpoint{1.669111in}{2.376432in}}%
\pgfpathlineto{\pgfqpoint{1.674254in}{2.376432in}}%
\pgfpathlineto{\pgfqpoint{1.676826in}{2.310608in}}%
\pgfpathlineto{\pgfqpoint{1.679398in}{2.359976in}}%
\pgfpathlineto{\pgfqpoint{1.681970in}{2.376432in}}%
\pgfpathlineto{\pgfqpoint{1.684542in}{2.343520in}}%
\pgfpathlineto{\pgfqpoint{1.687114in}{2.327064in}}%
\pgfpathlineto{\pgfqpoint{1.694829in}{2.327064in}}%
\pgfpathlineto{\pgfqpoint{1.699973in}{2.359976in}}%
\pgfpathlineto{\pgfqpoint{1.702545in}{2.343520in}}%
\pgfpathlineto{\pgfqpoint{1.707689in}{2.343520in}}%
\pgfpathlineto{\pgfqpoint{1.710261in}{2.327064in}}%
\pgfpathlineto{\pgfqpoint{1.723120in}{2.327064in}}%
\pgfpathlineto{\pgfqpoint{1.725692in}{2.310608in}}%
\pgfpathlineto{\pgfqpoint{1.733407in}{2.310608in}}%
\pgfpathlineto{\pgfqpoint{1.735979in}{2.327064in}}%
\pgfpathlineto{\pgfqpoint{1.738551in}{2.310608in}}%
\pgfpathlineto{\pgfqpoint{1.741123in}{2.310608in}}%
\pgfpathlineto{\pgfqpoint{1.743695in}{2.294152in}}%
\pgfpathlineto{\pgfqpoint{1.746267in}{2.261240in}}%
\pgfpathlineto{\pgfqpoint{1.751410in}{2.261240in}}%
\pgfpathlineto{\pgfqpoint{1.753982in}{2.277696in}}%
\pgfpathlineto{\pgfqpoint{1.759126in}{2.277696in}}%
\pgfpathlineto{\pgfqpoint{1.761698in}{2.294152in}}%
\pgfpathlineto{\pgfqpoint{1.766842in}{2.294152in}}%
\pgfpathlineto{\pgfqpoint{1.769414in}{2.310608in}}%
\pgfpathlineto{\pgfqpoint{1.771985in}{2.310608in}}%
\pgfpathlineto{\pgfqpoint{1.774557in}{2.294152in}}%
\pgfpathlineto{\pgfqpoint{1.787417in}{2.294152in}}%
\pgfpathlineto{\pgfqpoint{1.789989in}{2.310608in}}%
\pgfpathlineto{\pgfqpoint{1.805420in}{2.310608in}}%
\pgfpathlineto{\pgfqpoint{1.807992in}{2.327064in}}%
\pgfpathlineto{\pgfqpoint{1.818279in}{2.327064in}}%
\pgfpathlineto{\pgfqpoint{1.820851in}{2.310608in}}%
\pgfpathlineto{\pgfqpoint{1.823423in}{2.277696in}}%
\pgfpathlineto{\pgfqpoint{1.825995in}{2.294152in}}%
\pgfpathlineto{\pgfqpoint{1.833710in}{2.294152in}}%
\pgfpathlineto{\pgfqpoint{1.836282in}{2.277696in}}%
\pgfpathlineto{\pgfqpoint{1.849142in}{2.277696in}}%
\pgfpathlineto{\pgfqpoint{1.851713in}{2.261240in}}%
\pgfpathlineto{\pgfqpoint{1.867145in}{2.261240in}}%
\pgfpathlineto{\pgfqpoint{1.869717in}{2.244784in}}%
\pgfpathlineto{\pgfqpoint{1.877432in}{2.244784in}}%
\pgfpathlineto{\pgfqpoint{1.880004in}{2.228328in}}%
\pgfpathlineto{\pgfqpoint{1.887720in}{2.228328in}}%
\pgfpathlineto{\pgfqpoint{1.890291in}{2.244784in}}%
\pgfpathlineto{\pgfqpoint{1.900579in}{2.244784in}}%
\pgfpathlineto{\pgfqpoint{1.903151in}{2.261240in}}%
\pgfpathlineto{\pgfqpoint{1.910866in}{2.261240in}}%
\pgfpathlineto{\pgfqpoint{1.913438in}{2.244784in}}%
\pgfpathlineto{\pgfqpoint{1.998310in}{2.244784in}}%
\pgfpathlineto{\pgfqpoint{2.000882in}{2.228328in}}%
\pgfpathlineto{\pgfqpoint{2.013741in}{2.228328in}}%
\pgfpathlineto{\pgfqpoint{2.016313in}{2.211871in}}%
\pgfpathlineto{\pgfqpoint{2.047176in}{2.211871in}}%
\pgfpathlineto{\pgfqpoint{2.049747in}{2.228328in}}%
\pgfpathlineto{\pgfqpoint{2.157766in}{2.228328in}}%
\pgfpathlineto{\pgfqpoint{2.160338in}{2.211871in}}%
\pgfpathlineto{\pgfqpoint{2.162910in}{2.211871in}}%
\pgfpathlineto{\pgfqpoint{2.165482in}{2.195415in}}%
\pgfpathlineto{\pgfqpoint{2.299219in}{2.195415in}}%
\pgfpathlineto{\pgfqpoint{2.301791in}{2.178959in}}%
\pgfpathlineto{\pgfqpoint{2.378947in}{2.178959in}}%
\pgfpathlineto{\pgfqpoint{2.381519in}{2.162503in}}%
\pgfpathlineto{\pgfqpoint{2.399522in}{2.162503in}}%
\pgfpathlineto{\pgfqpoint{2.402094in}{2.146047in}}%
\pgfpathlineto{\pgfqpoint{2.432956in}{2.146047in}}%
\pgfpathlineto{\pgfqpoint{2.435528in}{2.162503in}}%
\pgfpathlineto{\pgfqpoint{2.474106in}{2.162503in}}%
\pgfpathlineto{\pgfqpoint{2.476678in}{2.146047in}}%
\pgfpathlineto{\pgfqpoint{2.494681in}{2.146047in}}%
\pgfpathlineto{\pgfqpoint{2.497253in}{2.129591in}}%
\pgfpathlineto{\pgfqpoint{2.540975in}{2.129591in}}%
\pgfpathlineto{\pgfqpoint{2.543546in}{2.113135in}}%
\pgfpathlineto{\pgfqpoint{2.551262in}{2.113135in}}%
\pgfpathlineto{\pgfqpoint{2.553834in}{2.096679in}}%
\pgfpathlineto{\pgfqpoint{2.612987in}{2.096679in}}%
\pgfpathlineto{\pgfqpoint{2.615559in}{2.080223in}}%
\pgfpathlineto{\pgfqpoint{2.646421in}{2.080223in}}%
\pgfpathlineto{\pgfqpoint{2.646421in}{2.080223in}}%
\pgfusepath{stroke}%
\end{pgfscope}%
\begin{pgfscope}%
\pgfpathrectangle{\pgfqpoint{0.488751in}{1.946106in}}{\pgfqpoint{2.260417in}{1.502439in}}%
\pgfusepath{clip}%
\pgfsetbuttcap%
\pgfsetroundjoin%
\pgfsetlinewidth{0.803000pt}%
\definecolor{currentstroke}{rgb}{0.490196,0.588235,0.431373}%
\pgfsetstrokecolor{currentstroke}%
\pgfsetdash{{2.960000pt}{1.280000pt}}{0.000000pt}%
\pgfpathmoveto{\pgfqpoint{0.591497in}{2.359976in}}%
\pgfpathlineto{\pgfqpoint{0.594069in}{2.327064in}}%
\pgfpathlineto{\pgfqpoint{0.599213in}{2.327064in}}%
\pgfpathlineto{\pgfqpoint{0.601785in}{2.310608in}}%
\pgfpathlineto{\pgfqpoint{0.609500in}{2.310608in}}%
\pgfpathlineto{\pgfqpoint{0.612072in}{2.294152in}}%
\pgfpathlineto{\pgfqpoint{0.617216in}{2.294152in}}%
\pgfpathlineto{\pgfqpoint{0.622359in}{2.261240in}}%
\pgfpathlineto{\pgfqpoint{0.630075in}{2.261240in}}%
\pgfpathlineto{\pgfqpoint{0.632647in}{2.228328in}}%
\pgfpathlineto{\pgfqpoint{0.635219in}{2.228328in}}%
\pgfpathlineto{\pgfqpoint{0.637791in}{2.244784in}}%
\pgfpathlineto{\pgfqpoint{0.671225in}{2.244784in}}%
\pgfpathlineto{\pgfqpoint{0.673797in}{2.228328in}}%
\pgfpathlineto{\pgfqpoint{0.686656in}{2.228328in}}%
\pgfpathlineto{\pgfqpoint{0.689228in}{2.211871in}}%
\pgfpathlineto{\pgfqpoint{0.709803in}{2.211871in}}%
\pgfpathlineto{\pgfqpoint{0.712375in}{2.195415in}}%
\pgfpathlineto{\pgfqpoint{0.730378in}{2.195415in}}%
\pgfpathlineto{\pgfqpoint{0.732950in}{2.178959in}}%
\pgfpathlineto{\pgfqpoint{0.918125in}{2.178959in}}%
\pgfpathlineto{\pgfqpoint{0.920696in}{2.162503in}}%
\pgfpathlineto{\pgfqpoint{1.075009in}{2.162503in}}%
\pgfpathlineto{\pgfqpoint{1.077581in}{2.146047in}}%
\pgfpathlineto{\pgfqpoint{1.669111in}{2.146047in}}%
\pgfpathlineto{\pgfqpoint{1.671683in}{2.129591in}}%
\pgfpathlineto{\pgfqpoint{2.646421in}{2.129591in}}%
\pgfpathlineto{\pgfqpoint{2.646421in}{2.129591in}}%
\pgfusepath{stroke}%
\end{pgfscope}%
\begin{pgfscope}%
\pgfpathrectangle{\pgfqpoint{0.488751in}{1.946106in}}{\pgfqpoint{2.260417in}{1.502439in}}%
\pgfusepath{clip}%
\pgfsetbuttcap%
\pgfsetroundjoin%
\pgfsetlinewidth{0.803000pt}%
\definecolor{currentstroke}{rgb}{0.843137,0.666667,0.313725}%
\pgfsetstrokecolor{currentstroke}%
\pgfsetdash{{2.960000pt}{1.280000pt}}{0.000000pt}%
\pgfpathmoveto{\pgfqpoint{0.591497in}{2.853658in}}%
\pgfpathlineto{\pgfqpoint{0.594069in}{2.903026in}}%
\pgfpathlineto{\pgfqpoint{0.596641in}{2.919482in}}%
\pgfpathlineto{\pgfqpoint{0.599213in}{2.886570in}}%
\pgfpathlineto{\pgfqpoint{0.601785in}{2.919482in}}%
\pgfpathlineto{\pgfqpoint{0.606928in}{2.952395in}}%
\pgfpathlineto{\pgfqpoint{0.609500in}{2.919482in}}%
\pgfpathlineto{\pgfqpoint{0.612072in}{2.903026in}}%
\pgfpathlineto{\pgfqpoint{0.617216in}{3.018219in}}%
\pgfpathlineto{\pgfqpoint{0.619788in}{3.001763in}}%
\pgfpathlineto{\pgfqpoint{0.622359in}{3.018219in}}%
\pgfpathlineto{\pgfqpoint{0.624931in}{2.985307in}}%
\pgfpathlineto{\pgfqpoint{0.627503in}{3.034675in}}%
\pgfpathlineto{\pgfqpoint{0.630075in}{2.968851in}}%
\pgfpathlineto{\pgfqpoint{0.632647in}{2.935938in}}%
\pgfpathlineto{\pgfqpoint{0.635219in}{2.919482in}}%
\pgfpathlineto{\pgfqpoint{0.637791in}{2.952395in}}%
\pgfpathlineto{\pgfqpoint{0.640363in}{3.034675in}}%
\pgfpathlineto{\pgfqpoint{0.642934in}{3.018219in}}%
\pgfpathlineto{\pgfqpoint{0.645506in}{3.084043in}}%
\pgfpathlineto{\pgfqpoint{0.648078in}{3.067587in}}%
\pgfpathlineto{\pgfqpoint{0.650650in}{2.985307in}}%
\pgfpathlineto{\pgfqpoint{0.653222in}{2.968851in}}%
\pgfpathlineto{\pgfqpoint{0.658366in}{2.903026in}}%
\pgfpathlineto{\pgfqpoint{0.660938in}{2.853658in}}%
\pgfpathlineto{\pgfqpoint{0.663509in}{2.837202in}}%
\pgfpathlineto{\pgfqpoint{0.666081in}{2.804290in}}%
\pgfpathlineto{\pgfqpoint{0.668653in}{2.853658in}}%
\pgfpathlineto{\pgfqpoint{0.671225in}{2.804290in}}%
\pgfpathlineto{\pgfqpoint{0.673797in}{2.705554in}}%
\pgfpathlineto{\pgfqpoint{0.676369in}{2.754922in}}%
\pgfpathlineto{\pgfqpoint{0.678941in}{2.722010in}}%
\pgfpathlineto{\pgfqpoint{0.681513in}{2.656185in}}%
\pgfpathlineto{\pgfqpoint{0.686656in}{2.689097in}}%
\pgfpathlineto{\pgfqpoint{0.691800in}{2.623273in}}%
\pgfpathlineto{\pgfqpoint{0.694372in}{2.623273in}}%
\pgfpathlineto{\pgfqpoint{0.699516in}{2.557449in}}%
\pgfpathlineto{\pgfqpoint{0.730378in}{2.557449in}}%
\pgfpathlineto{\pgfqpoint{0.732950in}{2.573905in}}%
\pgfpathlineto{\pgfqpoint{0.735522in}{2.557449in}}%
\pgfpathlineto{\pgfqpoint{0.748381in}{2.557449in}}%
\pgfpathlineto{\pgfqpoint{0.750953in}{2.573905in}}%
\pgfpathlineto{\pgfqpoint{0.789531in}{2.573905in}}%
\pgfpathlineto{\pgfqpoint{0.792103in}{2.557449in}}%
\pgfpathlineto{\pgfqpoint{0.817822in}{2.557449in}}%
\pgfpathlineto{\pgfqpoint{0.820393in}{2.573905in}}%
\pgfpathlineto{\pgfqpoint{0.838397in}{2.573905in}}%
\pgfpathlineto{\pgfqpoint{0.843540in}{2.606817in}}%
\pgfpathlineto{\pgfqpoint{0.846112in}{2.606817in}}%
\pgfpathlineto{\pgfqpoint{0.848684in}{2.590361in}}%
\pgfpathlineto{\pgfqpoint{0.853828in}{2.672641in}}%
\pgfpathlineto{\pgfqpoint{0.856400in}{2.705554in}}%
\pgfpathlineto{\pgfqpoint{0.858972in}{2.722010in}}%
\pgfpathlineto{\pgfqpoint{0.861543in}{2.754922in}}%
\pgfpathlineto{\pgfqpoint{0.864115in}{2.771378in}}%
\pgfpathlineto{\pgfqpoint{0.866687in}{2.754922in}}%
\pgfpathlineto{\pgfqpoint{0.869259in}{2.771378in}}%
\pgfpathlineto{\pgfqpoint{0.871831in}{2.771378in}}%
\pgfpathlineto{\pgfqpoint{0.874403in}{2.705554in}}%
\pgfpathlineto{\pgfqpoint{0.876975in}{2.689097in}}%
\pgfpathlineto{\pgfqpoint{0.879547in}{2.656185in}}%
\pgfpathlineto{\pgfqpoint{0.884690in}{2.656185in}}%
\pgfpathlineto{\pgfqpoint{0.887262in}{2.639729in}}%
\pgfpathlineto{\pgfqpoint{0.894978in}{2.639729in}}%
\pgfpathlineto{\pgfqpoint{0.897550in}{2.656185in}}%
\pgfpathlineto{\pgfqpoint{0.915553in}{2.656185in}}%
\pgfpathlineto{\pgfqpoint{0.918125in}{2.672641in}}%
\pgfpathlineto{\pgfqpoint{0.936128in}{2.672641in}}%
\pgfpathlineto{\pgfqpoint{0.938700in}{2.639729in}}%
\pgfpathlineto{\pgfqpoint{0.954131in}{2.639729in}}%
\pgfpathlineto{\pgfqpoint{0.956703in}{2.623273in}}%
\pgfpathlineto{\pgfqpoint{0.959274in}{2.639729in}}%
\pgfpathlineto{\pgfqpoint{1.026143in}{2.639729in}}%
\pgfpathlineto{\pgfqpoint{1.028715in}{2.656185in}}%
\pgfpathlineto{\pgfqpoint{1.031287in}{2.656185in}}%
\pgfpathlineto{\pgfqpoint{1.033859in}{2.672641in}}%
\pgfpathlineto{\pgfqpoint{1.036431in}{2.672641in}}%
\pgfpathlineto{\pgfqpoint{1.039002in}{2.689097in}}%
\pgfpathlineto{\pgfqpoint{1.041574in}{2.722010in}}%
\pgfpathlineto{\pgfqpoint{1.044146in}{2.689097in}}%
\pgfpathlineto{\pgfqpoint{1.046718in}{2.705554in}}%
\pgfpathlineto{\pgfqpoint{1.049290in}{2.689097in}}%
\pgfpathlineto{\pgfqpoint{1.051862in}{2.689097in}}%
\pgfpathlineto{\pgfqpoint{1.054434in}{2.672641in}}%
\pgfpathlineto{\pgfqpoint{1.057006in}{2.672641in}}%
\pgfpathlineto{\pgfqpoint{1.059577in}{2.705554in}}%
\pgfpathlineto{\pgfqpoint{1.064721in}{2.705554in}}%
\pgfpathlineto{\pgfqpoint{1.067293in}{2.722010in}}%
\pgfpathlineto{\pgfqpoint{1.085296in}{2.722010in}}%
\pgfpathlineto{\pgfqpoint{1.087868in}{2.738466in}}%
\pgfpathlineto{\pgfqpoint{1.090440in}{2.738466in}}%
\pgfpathlineto{\pgfqpoint{1.093012in}{2.754922in}}%
\pgfpathlineto{\pgfqpoint{1.095584in}{2.754922in}}%
\pgfpathlineto{\pgfqpoint{1.098155in}{2.705554in}}%
\pgfpathlineto{\pgfqpoint{1.108443in}{2.705554in}}%
\pgfpathlineto{\pgfqpoint{1.111015in}{2.738466in}}%
\pgfpathlineto{\pgfqpoint{1.118730in}{2.738466in}}%
\pgfpathlineto{\pgfqpoint{1.121302in}{2.771378in}}%
\pgfpathlineto{\pgfqpoint{1.123874in}{2.754922in}}%
\pgfpathlineto{\pgfqpoint{1.126446in}{2.754922in}}%
\pgfpathlineto{\pgfqpoint{1.129018in}{2.771378in}}%
\pgfpathlineto{\pgfqpoint{1.131590in}{2.771378in}}%
\pgfpathlineto{\pgfqpoint{1.134162in}{2.754922in}}%
\pgfpathlineto{\pgfqpoint{1.136734in}{2.754922in}}%
\pgfpathlineto{\pgfqpoint{1.139305in}{2.771378in}}%
\pgfpathlineto{\pgfqpoint{1.152165in}{2.771378in}}%
\pgfpathlineto{\pgfqpoint{1.154737in}{2.804290in}}%
\pgfpathlineto{\pgfqpoint{1.157308in}{2.787834in}}%
\pgfpathlineto{\pgfqpoint{1.162452in}{2.787834in}}%
\pgfpathlineto{\pgfqpoint{1.165024in}{2.804290in}}%
\pgfpathlineto{\pgfqpoint{1.167596in}{2.787834in}}%
\pgfpathlineto{\pgfqpoint{1.170168in}{2.820746in}}%
\pgfpathlineto{\pgfqpoint{1.172740in}{2.771378in}}%
\pgfpathlineto{\pgfqpoint{1.175312in}{2.771378in}}%
\pgfpathlineto{\pgfqpoint{1.177883in}{2.754922in}}%
\pgfpathlineto{\pgfqpoint{1.180455in}{2.722010in}}%
\pgfpathlineto{\pgfqpoint{1.183027in}{2.754922in}}%
\pgfpathlineto{\pgfqpoint{1.188171in}{2.754922in}}%
\pgfpathlineto{\pgfqpoint{1.190743in}{2.722010in}}%
\pgfpathlineto{\pgfqpoint{1.198458in}{2.722010in}}%
\pgfpathlineto{\pgfqpoint{1.201030in}{2.754922in}}%
\pgfpathlineto{\pgfqpoint{1.203602in}{2.738466in}}%
\pgfpathlineto{\pgfqpoint{1.206174in}{2.738466in}}%
\pgfpathlineto{\pgfqpoint{1.208746in}{2.754922in}}%
\pgfpathlineto{\pgfqpoint{1.211318in}{2.787834in}}%
\pgfpathlineto{\pgfqpoint{1.216461in}{2.787834in}}%
\pgfpathlineto{\pgfqpoint{1.219033in}{2.804290in}}%
\pgfpathlineto{\pgfqpoint{1.221605in}{2.771378in}}%
\pgfpathlineto{\pgfqpoint{1.226749in}{2.804290in}}%
\pgfpathlineto{\pgfqpoint{1.234465in}{2.804290in}}%
\pgfpathlineto{\pgfqpoint{1.237036in}{2.837202in}}%
\pgfpathlineto{\pgfqpoint{1.239608in}{2.837202in}}%
\pgfpathlineto{\pgfqpoint{1.242180in}{2.870114in}}%
\pgfpathlineto{\pgfqpoint{1.244752in}{2.837202in}}%
\pgfpathlineto{\pgfqpoint{1.249896in}{2.837202in}}%
\pgfpathlineto{\pgfqpoint{1.252468in}{2.853658in}}%
\pgfpathlineto{\pgfqpoint{1.257611in}{2.853658in}}%
\pgfpathlineto{\pgfqpoint{1.262755in}{2.886570in}}%
\pgfpathlineto{\pgfqpoint{1.267899in}{2.853658in}}%
\pgfpathlineto{\pgfqpoint{1.270471in}{2.886570in}}%
\pgfpathlineto{\pgfqpoint{1.273043in}{2.853658in}}%
\pgfpathlineto{\pgfqpoint{1.275615in}{2.837202in}}%
\pgfpathlineto{\pgfqpoint{1.278186in}{2.853658in}}%
\pgfpathlineto{\pgfqpoint{1.283330in}{2.820746in}}%
\pgfpathlineto{\pgfqpoint{1.285902in}{2.820746in}}%
\pgfpathlineto{\pgfqpoint{1.288474in}{2.787834in}}%
\pgfpathlineto{\pgfqpoint{1.291046in}{2.837202in}}%
\pgfpathlineto{\pgfqpoint{1.296189in}{2.804290in}}%
\pgfpathlineto{\pgfqpoint{1.319336in}{2.804290in}}%
\pgfpathlineto{\pgfqpoint{1.321908in}{2.853658in}}%
\pgfpathlineto{\pgfqpoint{1.324480in}{2.870114in}}%
\pgfpathlineto{\pgfqpoint{1.327052in}{2.853658in}}%
\pgfpathlineto{\pgfqpoint{1.329624in}{2.853658in}}%
\pgfpathlineto{\pgfqpoint{1.332196in}{2.870114in}}%
\pgfpathlineto{\pgfqpoint{1.334768in}{2.853658in}}%
\pgfpathlineto{\pgfqpoint{1.339911in}{2.935938in}}%
\pgfpathlineto{\pgfqpoint{1.342483in}{2.919482in}}%
\pgfpathlineto{\pgfqpoint{1.345055in}{2.919482in}}%
\pgfpathlineto{\pgfqpoint{1.347627in}{2.952395in}}%
\pgfpathlineto{\pgfqpoint{1.350199in}{2.952395in}}%
\pgfpathlineto{\pgfqpoint{1.357914in}{2.903026in}}%
\pgfpathlineto{\pgfqpoint{1.360486in}{2.903026in}}%
\pgfpathlineto{\pgfqpoint{1.365630in}{2.870114in}}%
\pgfpathlineto{\pgfqpoint{1.373346in}{2.870114in}}%
\pgfpathlineto{\pgfqpoint{1.375917in}{2.853658in}}%
\pgfpathlineto{\pgfqpoint{1.419639in}{2.853658in}}%
\pgfpathlineto{\pgfqpoint{1.422211in}{2.837202in}}%
\pgfpathlineto{\pgfqpoint{1.427355in}{2.837202in}}%
\pgfpathlineto{\pgfqpoint{1.429927in}{2.853658in}}%
\pgfpathlineto{\pgfqpoint{1.432499in}{2.886570in}}%
\pgfpathlineto{\pgfqpoint{1.435070in}{2.870114in}}%
\pgfpathlineto{\pgfqpoint{1.442786in}{2.870114in}}%
\pgfpathlineto{\pgfqpoint{1.445358in}{2.903026in}}%
\pgfpathlineto{\pgfqpoint{1.453074in}{2.903026in}}%
\pgfpathlineto{\pgfqpoint{1.455645in}{2.935938in}}%
\pgfpathlineto{\pgfqpoint{1.460789in}{2.935938in}}%
\pgfpathlineto{\pgfqpoint{1.463361in}{2.919482in}}%
\pgfpathlineto{\pgfqpoint{1.476220in}{2.919482in}}%
\pgfpathlineto{\pgfqpoint{1.478792in}{2.952395in}}%
\pgfpathlineto{\pgfqpoint{1.481364in}{2.952395in}}%
\pgfpathlineto{\pgfqpoint{1.483936in}{2.985307in}}%
\pgfpathlineto{\pgfqpoint{1.486508in}{2.968851in}}%
\pgfpathlineto{\pgfqpoint{1.489080in}{2.968851in}}%
\pgfpathlineto{\pgfqpoint{1.491652in}{3.001763in}}%
\pgfpathlineto{\pgfqpoint{1.496795in}{2.935938in}}%
\pgfpathlineto{\pgfqpoint{1.499367in}{2.935938in}}%
\pgfpathlineto{\pgfqpoint{1.501939in}{2.952395in}}%
\pgfpathlineto{\pgfqpoint{1.509655in}{2.952395in}}%
\pgfpathlineto{\pgfqpoint{1.512227in}{2.985307in}}%
\pgfpathlineto{\pgfqpoint{1.514798in}{3.001763in}}%
\pgfpathlineto{\pgfqpoint{1.517370in}{2.985307in}}%
\pgfpathlineto{\pgfqpoint{1.522514in}{3.018219in}}%
\pgfpathlineto{\pgfqpoint{1.525086in}{2.968851in}}%
\pgfpathlineto{\pgfqpoint{1.527658in}{2.968851in}}%
\pgfpathlineto{\pgfqpoint{1.530230in}{2.985307in}}%
\pgfpathlineto{\pgfqpoint{1.532802in}{2.985307in}}%
\pgfpathlineto{\pgfqpoint{1.535373in}{2.968851in}}%
\pgfpathlineto{\pgfqpoint{1.537945in}{2.968851in}}%
\pgfpathlineto{\pgfqpoint{1.540517in}{2.985307in}}%
\pgfpathlineto{\pgfqpoint{1.543089in}{2.985307in}}%
\pgfpathlineto{\pgfqpoint{1.545661in}{3.018219in}}%
\pgfpathlineto{\pgfqpoint{1.550805in}{3.018219in}}%
\pgfpathlineto{\pgfqpoint{1.553376in}{3.034675in}}%
\pgfpathlineto{\pgfqpoint{1.561092in}{3.034675in}}%
\pgfpathlineto{\pgfqpoint{1.563664in}{3.001763in}}%
\pgfpathlineto{\pgfqpoint{1.566236in}{3.001763in}}%
\pgfpathlineto{\pgfqpoint{1.568808in}{3.018219in}}%
\pgfpathlineto{\pgfqpoint{1.571380in}{3.001763in}}%
\pgfpathlineto{\pgfqpoint{1.573951in}{3.018219in}}%
\pgfpathlineto{\pgfqpoint{1.576523in}{3.018219in}}%
\pgfpathlineto{\pgfqpoint{1.579095in}{3.034675in}}%
\pgfpathlineto{\pgfqpoint{1.589383in}{3.034675in}}%
\pgfpathlineto{\pgfqpoint{1.591955in}{3.018219in}}%
\pgfpathlineto{\pgfqpoint{1.594526in}{3.051131in}}%
\pgfpathlineto{\pgfqpoint{1.599670in}{3.051131in}}%
\pgfpathlineto{\pgfqpoint{1.602242in}{3.034675in}}%
\pgfpathlineto{\pgfqpoint{1.604814in}{3.034675in}}%
\pgfpathlineto{\pgfqpoint{1.607386in}{3.067587in}}%
\pgfpathlineto{\pgfqpoint{1.609958in}{3.067587in}}%
\pgfpathlineto{\pgfqpoint{1.612529in}{3.100499in}}%
\pgfpathlineto{\pgfqpoint{1.615101in}{3.100499in}}%
\pgfpathlineto{\pgfqpoint{1.617673in}{3.133411in}}%
\pgfpathlineto{\pgfqpoint{1.620245in}{3.100499in}}%
\pgfpathlineto{\pgfqpoint{1.622817in}{3.100499in}}%
\pgfpathlineto{\pgfqpoint{1.625389in}{3.116955in}}%
\pgfpathlineto{\pgfqpoint{1.630533in}{3.116955in}}%
\pgfpathlineto{\pgfqpoint{1.633104in}{3.100499in}}%
\pgfpathlineto{\pgfqpoint{1.635676in}{3.116955in}}%
\pgfpathlineto{\pgfqpoint{1.638248in}{3.100499in}}%
\pgfpathlineto{\pgfqpoint{1.643392in}{3.100499in}}%
\pgfpathlineto{\pgfqpoint{1.645964in}{3.133411in}}%
\pgfpathlineto{\pgfqpoint{1.648536in}{3.100499in}}%
\pgfpathlineto{\pgfqpoint{1.651108in}{3.084043in}}%
\pgfpathlineto{\pgfqpoint{1.653679in}{3.116955in}}%
\pgfpathlineto{\pgfqpoint{1.656251in}{3.100499in}}%
\pgfpathlineto{\pgfqpoint{1.658823in}{3.100499in}}%
\pgfpathlineto{\pgfqpoint{1.669111in}{3.034675in}}%
\pgfpathlineto{\pgfqpoint{1.671683in}{3.034675in}}%
\pgfpathlineto{\pgfqpoint{1.674254in}{3.018219in}}%
\pgfpathlineto{\pgfqpoint{1.676826in}{3.051131in}}%
\pgfpathlineto{\pgfqpoint{1.679398in}{3.001763in}}%
\pgfpathlineto{\pgfqpoint{1.681970in}{3.001763in}}%
\pgfpathlineto{\pgfqpoint{1.684542in}{3.018219in}}%
\pgfpathlineto{\pgfqpoint{1.687114in}{3.001763in}}%
\pgfpathlineto{\pgfqpoint{1.694829in}{3.001763in}}%
\pgfpathlineto{\pgfqpoint{1.697401in}{3.018219in}}%
\pgfpathlineto{\pgfqpoint{1.699973in}{3.051131in}}%
\pgfpathlineto{\pgfqpoint{1.702545in}{3.051131in}}%
\pgfpathlineto{\pgfqpoint{1.705117in}{3.100499in}}%
\pgfpathlineto{\pgfqpoint{1.707689in}{3.084043in}}%
\pgfpathlineto{\pgfqpoint{1.710261in}{3.100499in}}%
\pgfpathlineto{\pgfqpoint{1.712832in}{3.067587in}}%
\pgfpathlineto{\pgfqpoint{1.715404in}{3.067587in}}%
\pgfpathlineto{\pgfqpoint{1.717976in}{3.116955in}}%
\pgfpathlineto{\pgfqpoint{1.720548in}{3.116955in}}%
\pgfpathlineto{\pgfqpoint{1.723120in}{3.133411in}}%
\pgfpathlineto{\pgfqpoint{1.728264in}{3.100499in}}%
\pgfpathlineto{\pgfqpoint{1.733407in}{3.100499in}}%
\pgfpathlineto{\pgfqpoint{1.735979in}{3.084043in}}%
\pgfpathlineto{\pgfqpoint{1.738551in}{3.100499in}}%
\pgfpathlineto{\pgfqpoint{1.748839in}{3.100499in}}%
\pgfpathlineto{\pgfqpoint{1.751410in}{3.084043in}}%
\pgfpathlineto{\pgfqpoint{1.753982in}{3.084043in}}%
\pgfpathlineto{\pgfqpoint{1.756554in}{3.100499in}}%
\pgfpathlineto{\pgfqpoint{1.759126in}{3.100499in}}%
\pgfpathlineto{\pgfqpoint{1.761698in}{3.084043in}}%
\pgfpathlineto{\pgfqpoint{1.764270in}{3.100499in}}%
\pgfpathlineto{\pgfqpoint{1.766842in}{3.100499in}}%
\pgfpathlineto{\pgfqpoint{1.769414in}{3.084043in}}%
\pgfpathlineto{\pgfqpoint{1.771985in}{3.084043in}}%
\pgfpathlineto{\pgfqpoint{1.774557in}{3.051131in}}%
\pgfpathlineto{\pgfqpoint{1.777129in}{3.051131in}}%
\pgfpathlineto{\pgfqpoint{1.784845in}{3.100499in}}%
\pgfpathlineto{\pgfqpoint{1.787417in}{3.084043in}}%
\pgfpathlineto{\pgfqpoint{1.792560in}{3.084043in}}%
\pgfpathlineto{\pgfqpoint{1.795132in}{3.116955in}}%
\pgfpathlineto{\pgfqpoint{1.797704in}{3.116955in}}%
\pgfpathlineto{\pgfqpoint{1.800276in}{3.100499in}}%
\pgfpathlineto{\pgfqpoint{1.802848in}{3.116955in}}%
\pgfpathlineto{\pgfqpoint{1.805420in}{3.100499in}}%
\pgfpathlineto{\pgfqpoint{1.813135in}{3.100499in}}%
\pgfpathlineto{\pgfqpoint{1.815707in}{3.133411in}}%
\pgfpathlineto{\pgfqpoint{1.818279in}{3.133411in}}%
\pgfpathlineto{\pgfqpoint{1.820851in}{3.100499in}}%
\pgfpathlineto{\pgfqpoint{1.825995in}{3.100499in}}%
\pgfpathlineto{\pgfqpoint{1.828567in}{3.116955in}}%
\pgfpathlineto{\pgfqpoint{1.831138in}{3.100499in}}%
\pgfpathlineto{\pgfqpoint{1.833710in}{3.116955in}}%
\pgfpathlineto{\pgfqpoint{1.836282in}{3.100499in}}%
\pgfpathlineto{\pgfqpoint{1.838854in}{3.116955in}}%
\pgfpathlineto{\pgfqpoint{1.841426in}{3.084043in}}%
\pgfpathlineto{\pgfqpoint{1.843998in}{3.116955in}}%
\pgfpathlineto{\pgfqpoint{1.846570in}{3.100499in}}%
\pgfpathlineto{\pgfqpoint{1.851713in}{3.100499in}}%
\pgfpathlineto{\pgfqpoint{1.854285in}{3.116955in}}%
\pgfpathlineto{\pgfqpoint{1.864573in}{3.116955in}}%
\pgfpathlineto{\pgfqpoint{1.867145in}{3.100499in}}%
\pgfpathlineto{\pgfqpoint{1.869717in}{3.100499in}}%
\pgfpathlineto{\pgfqpoint{1.872288in}{3.133411in}}%
\pgfpathlineto{\pgfqpoint{1.880004in}{3.084043in}}%
\pgfpathlineto{\pgfqpoint{1.882576in}{3.100499in}}%
\pgfpathlineto{\pgfqpoint{1.887720in}{3.100499in}}%
\pgfpathlineto{\pgfqpoint{1.890291in}{3.084043in}}%
\pgfpathlineto{\pgfqpoint{1.892863in}{3.116955in}}%
\pgfpathlineto{\pgfqpoint{1.895435in}{3.100499in}}%
\pgfpathlineto{\pgfqpoint{1.898007in}{3.100499in}}%
\pgfpathlineto{\pgfqpoint{1.900579in}{3.084043in}}%
\pgfpathlineto{\pgfqpoint{1.903151in}{3.051131in}}%
\pgfpathlineto{\pgfqpoint{1.908295in}{3.051131in}}%
\pgfpathlineto{\pgfqpoint{1.910866in}{3.067587in}}%
\pgfpathlineto{\pgfqpoint{1.918582in}{3.067587in}}%
\pgfpathlineto{\pgfqpoint{1.921154in}{3.084043in}}%
\pgfpathlineto{\pgfqpoint{1.923726in}{3.116955in}}%
\pgfpathlineto{\pgfqpoint{1.928870in}{3.116955in}}%
\pgfpathlineto{\pgfqpoint{1.931441in}{3.100499in}}%
\pgfpathlineto{\pgfqpoint{1.936585in}{3.100499in}}%
\pgfpathlineto{\pgfqpoint{1.939157in}{3.067587in}}%
\pgfpathlineto{\pgfqpoint{1.946873in}{3.067587in}}%
\pgfpathlineto{\pgfqpoint{1.949444in}{3.100499in}}%
\pgfpathlineto{\pgfqpoint{1.954588in}{3.067587in}}%
\pgfpathlineto{\pgfqpoint{1.959732in}{3.067587in}}%
\pgfpathlineto{\pgfqpoint{1.962304in}{3.084043in}}%
\pgfpathlineto{\pgfqpoint{1.967448in}{3.084043in}}%
\pgfpathlineto{\pgfqpoint{1.970019in}{3.100499in}}%
\pgfpathlineto{\pgfqpoint{1.972591in}{3.100499in}}%
\pgfpathlineto{\pgfqpoint{1.975163in}{3.067587in}}%
\pgfpathlineto{\pgfqpoint{1.988023in}{3.067587in}}%
\pgfpathlineto{\pgfqpoint{1.993166in}{3.034675in}}%
\pgfpathlineto{\pgfqpoint{1.998310in}{3.034675in}}%
\pgfpathlineto{\pgfqpoint{2.000882in}{3.018219in}}%
\pgfpathlineto{\pgfqpoint{2.011169in}{3.018219in}}%
\pgfpathlineto{\pgfqpoint{2.016313in}{3.051131in}}%
\pgfpathlineto{\pgfqpoint{2.018885in}{3.034675in}}%
\pgfpathlineto{\pgfqpoint{2.034316in}{3.034675in}}%
\pgfpathlineto{\pgfqpoint{2.036888in}{3.051131in}}%
\pgfpathlineto{\pgfqpoint{2.039460in}{3.034675in}}%
\pgfpathlineto{\pgfqpoint{2.042032in}{3.051131in}}%
\pgfpathlineto{\pgfqpoint{2.044604in}{3.084043in}}%
\pgfpathlineto{\pgfqpoint{2.052319in}{3.084043in}}%
\pgfpathlineto{\pgfqpoint{2.057463in}{3.051131in}}%
\pgfpathlineto{\pgfqpoint{2.065179in}{3.051131in}}%
\pgfpathlineto{\pgfqpoint{2.067751in}{3.034675in}}%
\pgfpathlineto{\pgfqpoint{2.070322in}{3.051131in}}%
\pgfpathlineto{\pgfqpoint{2.078038in}{3.051131in}}%
\pgfpathlineto{\pgfqpoint{2.080610in}{3.067587in}}%
\pgfpathlineto{\pgfqpoint{2.083182in}{3.067587in}}%
\pgfpathlineto{\pgfqpoint{2.085754in}{3.100499in}}%
\pgfpathlineto{\pgfqpoint{2.088325in}{3.100499in}}%
\pgfpathlineto{\pgfqpoint{2.093469in}{3.067587in}}%
\pgfpathlineto{\pgfqpoint{2.098613in}{3.067587in}}%
\pgfpathlineto{\pgfqpoint{2.101185in}{3.051131in}}%
\pgfpathlineto{\pgfqpoint{2.106329in}{3.051131in}}%
\pgfpathlineto{\pgfqpoint{2.111472in}{3.018219in}}%
\pgfpathlineto{\pgfqpoint{2.114044in}{3.051131in}}%
\pgfpathlineto{\pgfqpoint{2.116616in}{3.034675in}}%
\pgfpathlineto{\pgfqpoint{2.121760in}{3.067587in}}%
\pgfpathlineto{\pgfqpoint{2.124332in}{3.067587in}}%
\pgfpathlineto{\pgfqpoint{2.126904in}{3.051131in}}%
\pgfpathlineto{\pgfqpoint{2.129475in}{3.051131in}}%
\pgfpathlineto{\pgfqpoint{2.132047in}{3.067587in}}%
\pgfpathlineto{\pgfqpoint{2.137191in}{3.067587in}}%
\pgfpathlineto{\pgfqpoint{2.139763in}{3.084043in}}%
\pgfpathlineto{\pgfqpoint{2.142335in}{3.084043in}}%
\pgfpathlineto{\pgfqpoint{2.144907in}{3.067587in}}%
\pgfpathlineto{\pgfqpoint{2.147478in}{3.067587in}}%
\pgfpathlineto{\pgfqpoint{2.152622in}{3.100499in}}%
\pgfpathlineto{\pgfqpoint{2.162910in}{3.100499in}}%
\pgfpathlineto{\pgfqpoint{2.165482in}{3.116955in}}%
\pgfpathlineto{\pgfqpoint{2.168053in}{3.100499in}}%
\pgfpathlineto{\pgfqpoint{2.170625in}{3.100499in}}%
\pgfpathlineto{\pgfqpoint{2.173197in}{3.084043in}}%
\pgfpathlineto{\pgfqpoint{2.180913in}{3.084043in}}%
\pgfpathlineto{\pgfqpoint{2.183485in}{3.100499in}}%
\pgfpathlineto{\pgfqpoint{2.188628in}{3.100499in}}%
\pgfpathlineto{\pgfqpoint{2.191200in}{3.084043in}}%
\pgfpathlineto{\pgfqpoint{2.193772in}{3.084043in}}%
\pgfpathlineto{\pgfqpoint{2.196344in}{3.100499in}}%
\pgfpathlineto{\pgfqpoint{2.198916in}{3.084043in}}%
\pgfpathlineto{\pgfqpoint{2.201488in}{3.084043in}}%
\pgfpathlineto{\pgfqpoint{2.206631in}{3.116955in}}%
\pgfpathlineto{\pgfqpoint{2.209203in}{3.116955in}}%
\pgfpathlineto{\pgfqpoint{2.211775in}{3.133411in}}%
\pgfpathlineto{\pgfqpoint{2.216919in}{3.100499in}}%
\pgfpathlineto{\pgfqpoint{2.219491in}{3.116955in}}%
\pgfpathlineto{\pgfqpoint{2.222063in}{3.100499in}}%
\pgfpathlineto{\pgfqpoint{2.224635in}{3.100499in}}%
\pgfpathlineto{\pgfqpoint{2.227206in}{3.084043in}}%
\pgfpathlineto{\pgfqpoint{2.229778in}{3.084043in}}%
\pgfpathlineto{\pgfqpoint{2.232350in}{3.100499in}}%
\pgfpathlineto{\pgfqpoint{2.240066in}{3.100499in}}%
\pgfpathlineto{\pgfqpoint{2.242638in}{3.084043in}}%
\pgfpathlineto{\pgfqpoint{2.245210in}{3.100499in}}%
\pgfpathlineto{\pgfqpoint{2.247781in}{3.084043in}}%
\pgfpathlineto{\pgfqpoint{2.250353in}{3.084043in}}%
\pgfpathlineto{\pgfqpoint{2.255497in}{3.116955in}}%
\pgfpathlineto{\pgfqpoint{2.258069in}{3.100499in}}%
\pgfpathlineto{\pgfqpoint{2.265785in}{3.100499in}}%
\pgfpathlineto{\pgfqpoint{2.268356in}{3.116955in}}%
\pgfpathlineto{\pgfqpoint{2.270928in}{3.116955in}}%
\pgfpathlineto{\pgfqpoint{2.273500in}{3.100499in}}%
\pgfpathlineto{\pgfqpoint{2.288931in}{3.100499in}}%
\pgfpathlineto{\pgfqpoint{2.291503in}{3.116955in}}%
\pgfpathlineto{\pgfqpoint{2.294075in}{3.116955in}}%
\pgfpathlineto{\pgfqpoint{2.296647in}{3.100499in}}%
\pgfpathlineto{\pgfqpoint{2.299219in}{3.116955in}}%
\pgfpathlineto{\pgfqpoint{2.301791in}{3.116955in}}%
\pgfpathlineto{\pgfqpoint{2.306934in}{3.149867in}}%
\pgfpathlineto{\pgfqpoint{2.309506in}{3.116955in}}%
\pgfpathlineto{\pgfqpoint{2.312078in}{3.116955in}}%
\pgfpathlineto{\pgfqpoint{2.314650in}{3.100499in}}%
\pgfpathlineto{\pgfqpoint{2.324938in}{3.100499in}}%
\pgfpathlineto{\pgfqpoint{2.327509in}{3.084043in}}%
\pgfpathlineto{\pgfqpoint{2.332653in}{3.084043in}}%
\pgfpathlineto{\pgfqpoint{2.335225in}{3.100499in}}%
\pgfpathlineto{\pgfqpoint{2.340369in}{3.100499in}}%
\pgfpathlineto{\pgfqpoint{2.342941in}{3.116955in}}%
\pgfpathlineto{\pgfqpoint{2.345512in}{3.116955in}}%
\pgfpathlineto{\pgfqpoint{2.348084in}{3.133411in}}%
\pgfpathlineto{\pgfqpoint{2.353228in}{3.100499in}}%
\pgfpathlineto{\pgfqpoint{2.355800in}{3.100499in}}%
\pgfpathlineto{\pgfqpoint{2.358372in}{3.084043in}}%
\pgfpathlineto{\pgfqpoint{2.360944in}{3.084043in}}%
\pgfpathlineto{\pgfqpoint{2.363516in}{3.034675in}}%
\pgfpathlineto{\pgfqpoint{2.366087in}{3.051131in}}%
\pgfpathlineto{\pgfqpoint{2.371231in}{3.051131in}}%
\pgfpathlineto{\pgfqpoint{2.373803in}{3.034675in}}%
\pgfpathlineto{\pgfqpoint{2.376375in}{3.051131in}}%
\pgfpathlineto{\pgfqpoint{2.378947in}{3.051131in}}%
\pgfpathlineto{\pgfqpoint{2.381519in}{3.034675in}}%
\pgfpathlineto{\pgfqpoint{2.384091in}{3.051131in}}%
\pgfpathlineto{\pgfqpoint{2.389234in}{3.051131in}}%
\pgfpathlineto{\pgfqpoint{2.391806in}{3.067587in}}%
\pgfpathlineto{\pgfqpoint{2.396950in}{3.001763in}}%
\pgfpathlineto{\pgfqpoint{2.399522in}{2.985307in}}%
\pgfpathlineto{\pgfqpoint{2.402094in}{3.001763in}}%
\pgfpathlineto{\pgfqpoint{2.404665in}{3.001763in}}%
\pgfpathlineto{\pgfqpoint{2.409809in}{2.968851in}}%
\pgfpathlineto{\pgfqpoint{2.412381in}{2.985307in}}%
\pgfpathlineto{\pgfqpoint{2.414953in}{3.018219in}}%
\pgfpathlineto{\pgfqpoint{2.420097in}{2.985307in}}%
\pgfpathlineto{\pgfqpoint{2.425240in}{3.018219in}}%
\pgfpathlineto{\pgfqpoint{2.427812in}{2.985307in}}%
\pgfpathlineto{\pgfqpoint{2.432956in}{3.018219in}}%
\pgfpathlineto{\pgfqpoint{2.435528in}{2.968851in}}%
\pgfpathlineto{\pgfqpoint{2.438100in}{2.968851in}}%
\pgfpathlineto{\pgfqpoint{2.440672in}{2.952395in}}%
\pgfpathlineto{\pgfqpoint{2.453531in}{2.952395in}}%
\pgfpathlineto{\pgfqpoint{2.456103in}{2.935938in}}%
\pgfpathlineto{\pgfqpoint{2.458675in}{2.968851in}}%
\pgfpathlineto{\pgfqpoint{2.463819in}{3.001763in}}%
\pgfpathlineto{\pgfqpoint{2.468962in}{2.968851in}}%
\pgfpathlineto{\pgfqpoint{2.471534in}{3.001763in}}%
\pgfpathlineto{\pgfqpoint{2.474106in}{2.968851in}}%
\pgfpathlineto{\pgfqpoint{2.481822in}{2.968851in}}%
\pgfpathlineto{\pgfqpoint{2.484393in}{2.985307in}}%
\pgfpathlineto{\pgfqpoint{2.486965in}{2.952395in}}%
\pgfpathlineto{\pgfqpoint{2.489537in}{2.985307in}}%
\pgfpathlineto{\pgfqpoint{2.494681in}{3.018219in}}%
\pgfpathlineto{\pgfqpoint{2.499825in}{3.018219in}}%
\pgfpathlineto{\pgfqpoint{2.502397in}{3.034675in}}%
\pgfpathlineto{\pgfqpoint{2.510112in}{3.034675in}}%
\pgfpathlineto{\pgfqpoint{2.512684in}{3.084043in}}%
\pgfpathlineto{\pgfqpoint{2.517828in}{3.116955in}}%
\pgfpathlineto{\pgfqpoint{2.522972in}{3.084043in}}%
\pgfpathlineto{\pgfqpoint{2.525543in}{3.100499in}}%
\pgfpathlineto{\pgfqpoint{2.528115in}{3.149867in}}%
\pgfpathlineto{\pgfqpoint{2.530687in}{3.116955in}}%
\pgfpathlineto{\pgfqpoint{2.535831in}{3.116955in}}%
\pgfpathlineto{\pgfqpoint{2.538403in}{3.100499in}}%
\pgfpathlineto{\pgfqpoint{2.540975in}{3.116955in}}%
\pgfpathlineto{\pgfqpoint{2.543546in}{3.116955in}}%
\pgfpathlineto{\pgfqpoint{2.546118in}{3.133411in}}%
\pgfpathlineto{\pgfqpoint{2.548690in}{3.100499in}}%
\pgfpathlineto{\pgfqpoint{2.551262in}{3.100499in}}%
\pgfpathlineto{\pgfqpoint{2.553834in}{3.133411in}}%
\pgfpathlineto{\pgfqpoint{2.556406in}{3.116955in}}%
\pgfpathlineto{\pgfqpoint{2.558978in}{3.166323in}}%
\pgfpathlineto{\pgfqpoint{2.564121in}{3.166323in}}%
\pgfpathlineto{\pgfqpoint{2.566693in}{3.100499in}}%
\pgfpathlineto{\pgfqpoint{2.576981in}{3.100499in}}%
\pgfpathlineto{\pgfqpoint{2.579553in}{3.084043in}}%
\pgfpathlineto{\pgfqpoint{2.582125in}{3.100499in}}%
\pgfpathlineto{\pgfqpoint{2.589840in}{3.100499in}}%
\pgfpathlineto{\pgfqpoint{2.592412in}{3.051131in}}%
\pgfpathlineto{\pgfqpoint{2.594984in}{3.051131in}}%
\pgfpathlineto{\pgfqpoint{2.597556in}{3.084043in}}%
\pgfpathlineto{\pgfqpoint{2.600128in}{3.100499in}}%
\pgfpathlineto{\pgfqpoint{2.605271in}{3.067587in}}%
\pgfpathlineto{\pgfqpoint{2.610415in}{3.100499in}}%
\pgfpathlineto{\pgfqpoint{2.612987in}{3.084043in}}%
\pgfpathlineto{\pgfqpoint{2.615559in}{3.100499in}}%
\pgfpathlineto{\pgfqpoint{2.623274in}{3.100499in}}%
\pgfpathlineto{\pgfqpoint{2.625846in}{3.116955in}}%
\pgfpathlineto{\pgfqpoint{2.630990in}{3.084043in}}%
\pgfpathlineto{\pgfqpoint{2.643849in}{3.084043in}}%
\pgfpathlineto{\pgfqpoint{2.646421in}{3.133411in}}%
\pgfpathlineto{\pgfqpoint{2.646421in}{3.133411in}}%
\pgfusepath{stroke}%
\end{pgfscope}%
\begin{pgfscope}%
\pgfpathrectangle{\pgfqpoint{0.488751in}{1.946106in}}{\pgfqpoint{2.260417in}{1.502439in}}%
\pgfusepath{clip}%
\pgfsetrectcap%
\pgfsetroundjoin%
\pgfsetlinewidth{0.803000pt}%
\definecolor{currentstroke}{rgb}{0.333333,0.333333,0.333333}%
\pgfsetstrokecolor{currentstroke}%
\pgfsetstrokeopacity{0.270000}%
\pgfsetdash{}{0pt}%
\pgfpathmoveto{\pgfqpoint{0.591497in}{2.705554in}}%
\pgfpathlineto{\pgfqpoint{0.599213in}{2.705554in}}%
\pgfpathlineto{\pgfqpoint{0.601785in}{2.689097in}}%
\pgfpathlineto{\pgfqpoint{0.604356in}{2.689097in}}%
\pgfpathlineto{\pgfqpoint{0.606928in}{2.672641in}}%
\pgfpathlineto{\pgfqpoint{0.635219in}{2.672641in}}%
\pgfpathlineto{\pgfqpoint{0.637791in}{2.689097in}}%
\pgfpathlineto{\pgfqpoint{0.645506in}{2.689097in}}%
\pgfpathlineto{\pgfqpoint{0.648078in}{2.705554in}}%
\pgfpathlineto{\pgfqpoint{0.655794in}{2.705554in}}%
\pgfpathlineto{\pgfqpoint{0.658366in}{2.738466in}}%
\pgfpathlineto{\pgfqpoint{0.663509in}{2.738466in}}%
\pgfpathlineto{\pgfqpoint{0.666081in}{2.754922in}}%
\pgfpathlineto{\pgfqpoint{0.671225in}{2.722010in}}%
\pgfpathlineto{\pgfqpoint{0.673797in}{2.722010in}}%
\pgfpathlineto{\pgfqpoint{0.676369in}{2.738466in}}%
\pgfpathlineto{\pgfqpoint{0.678941in}{2.722010in}}%
\pgfpathlineto{\pgfqpoint{0.681513in}{2.722010in}}%
\pgfpathlineto{\pgfqpoint{0.684084in}{2.738466in}}%
\pgfpathlineto{\pgfqpoint{0.686656in}{2.738466in}}%
\pgfpathlineto{\pgfqpoint{0.689228in}{2.722010in}}%
\pgfpathlineto{\pgfqpoint{0.691800in}{2.722010in}}%
\pgfpathlineto{\pgfqpoint{0.694372in}{2.738466in}}%
\pgfpathlineto{\pgfqpoint{0.696944in}{2.738466in}}%
\pgfpathlineto{\pgfqpoint{0.699516in}{2.754922in}}%
\pgfpathlineto{\pgfqpoint{0.702087in}{2.787834in}}%
\pgfpathlineto{\pgfqpoint{0.707231in}{2.820746in}}%
\pgfpathlineto{\pgfqpoint{0.709803in}{2.853658in}}%
\pgfpathlineto{\pgfqpoint{0.712375in}{2.870114in}}%
\pgfpathlineto{\pgfqpoint{0.725234in}{2.870114in}}%
\pgfpathlineto{\pgfqpoint{0.727806in}{2.853658in}}%
\pgfpathlineto{\pgfqpoint{0.735522in}{2.853658in}}%
\pgfpathlineto{\pgfqpoint{0.738094in}{2.837202in}}%
\pgfpathlineto{\pgfqpoint{0.743237in}{2.837202in}}%
\pgfpathlineto{\pgfqpoint{0.745809in}{2.820746in}}%
\pgfpathlineto{\pgfqpoint{0.748381in}{2.837202in}}%
\pgfpathlineto{\pgfqpoint{0.753525in}{2.804290in}}%
\pgfpathlineto{\pgfqpoint{0.761240in}{2.804290in}}%
\pgfpathlineto{\pgfqpoint{0.763812in}{2.820746in}}%
\pgfpathlineto{\pgfqpoint{0.766384in}{2.804290in}}%
\pgfpathlineto{\pgfqpoint{0.768956in}{2.804290in}}%
\pgfpathlineto{\pgfqpoint{0.771528in}{2.787834in}}%
\pgfpathlineto{\pgfqpoint{0.774100in}{2.787834in}}%
\pgfpathlineto{\pgfqpoint{0.779244in}{2.754922in}}%
\pgfpathlineto{\pgfqpoint{0.781815in}{2.754922in}}%
\pgfpathlineto{\pgfqpoint{0.786959in}{2.787834in}}%
\pgfpathlineto{\pgfqpoint{0.789531in}{2.787834in}}%
\pgfpathlineto{\pgfqpoint{0.792103in}{2.804290in}}%
\pgfpathlineto{\pgfqpoint{0.797247in}{2.804290in}}%
\pgfpathlineto{\pgfqpoint{0.799819in}{2.820746in}}%
\pgfpathlineto{\pgfqpoint{0.802390in}{2.820746in}}%
\pgfpathlineto{\pgfqpoint{0.804962in}{2.804290in}}%
\pgfpathlineto{\pgfqpoint{0.810106in}{2.804290in}}%
\pgfpathlineto{\pgfqpoint{0.812678in}{2.837202in}}%
\pgfpathlineto{\pgfqpoint{0.815250in}{2.837202in}}%
\pgfpathlineto{\pgfqpoint{0.820393in}{2.804290in}}%
\pgfpathlineto{\pgfqpoint{0.825537in}{2.804290in}}%
\pgfpathlineto{\pgfqpoint{0.828109in}{2.820746in}}%
\pgfpathlineto{\pgfqpoint{0.830681in}{2.820746in}}%
\pgfpathlineto{\pgfqpoint{0.833253in}{2.837202in}}%
\pgfpathlineto{\pgfqpoint{0.835825in}{2.837202in}}%
\pgfpathlineto{\pgfqpoint{0.838397in}{2.820746in}}%
\pgfpathlineto{\pgfqpoint{0.840968in}{2.820746in}}%
\pgfpathlineto{\pgfqpoint{0.843540in}{2.837202in}}%
\pgfpathlineto{\pgfqpoint{0.846112in}{2.837202in}}%
\pgfpathlineto{\pgfqpoint{0.848684in}{2.853658in}}%
\pgfpathlineto{\pgfqpoint{0.851256in}{2.853658in}}%
\pgfpathlineto{\pgfqpoint{0.853828in}{2.820746in}}%
\pgfpathlineto{\pgfqpoint{0.856400in}{2.804290in}}%
\pgfpathlineto{\pgfqpoint{0.861543in}{2.804290in}}%
\pgfpathlineto{\pgfqpoint{0.864115in}{2.820746in}}%
\pgfpathlineto{\pgfqpoint{0.869259in}{2.820746in}}%
\pgfpathlineto{\pgfqpoint{0.871831in}{2.853658in}}%
\pgfpathlineto{\pgfqpoint{0.874403in}{2.870114in}}%
\pgfpathlineto{\pgfqpoint{0.876975in}{2.870114in}}%
\pgfpathlineto{\pgfqpoint{0.879547in}{2.903026in}}%
\pgfpathlineto{\pgfqpoint{0.884690in}{2.903026in}}%
\pgfpathlineto{\pgfqpoint{0.887262in}{2.886570in}}%
\pgfpathlineto{\pgfqpoint{0.889834in}{2.886570in}}%
\pgfpathlineto{\pgfqpoint{0.892406in}{2.870114in}}%
\pgfpathlineto{\pgfqpoint{0.900121in}{2.870114in}}%
\pgfpathlineto{\pgfqpoint{0.902693in}{2.903026in}}%
\pgfpathlineto{\pgfqpoint{0.905265in}{2.886570in}}%
\pgfpathlineto{\pgfqpoint{0.910409in}{2.886570in}}%
\pgfpathlineto{\pgfqpoint{0.912981in}{2.919482in}}%
\pgfpathlineto{\pgfqpoint{0.915553in}{2.919482in}}%
\pgfpathlineto{\pgfqpoint{0.918125in}{2.886570in}}%
\pgfpathlineto{\pgfqpoint{0.920696in}{2.886570in}}%
\pgfpathlineto{\pgfqpoint{0.923268in}{2.903026in}}%
\pgfpathlineto{\pgfqpoint{0.925840in}{2.903026in}}%
\pgfpathlineto{\pgfqpoint{0.928412in}{2.886570in}}%
\pgfpathlineto{\pgfqpoint{0.930984in}{2.886570in}}%
\pgfpathlineto{\pgfqpoint{0.933556in}{2.870114in}}%
\pgfpathlineto{\pgfqpoint{0.936128in}{2.870114in}}%
\pgfpathlineto{\pgfqpoint{0.938700in}{2.837202in}}%
\pgfpathlineto{\pgfqpoint{0.941271in}{2.820746in}}%
\pgfpathlineto{\pgfqpoint{0.943843in}{2.837202in}}%
\pgfpathlineto{\pgfqpoint{0.946415in}{2.837202in}}%
\pgfpathlineto{\pgfqpoint{0.948987in}{2.853658in}}%
\pgfpathlineto{\pgfqpoint{0.951559in}{2.853658in}}%
\pgfpathlineto{\pgfqpoint{0.954131in}{2.870114in}}%
\pgfpathlineto{\pgfqpoint{0.956703in}{2.919482in}}%
\pgfpathlineto{\pgfqpoint{0.959274in}{2.903026in}}%
\pgfpathlineto{\pgfqpoint{0.961846in}{2.903026in}}%
\pgfpathlineto{\pgfqpoint{0.964418in}{2.886570in}}%
\pgfpathlineto{\pgfqpoint{0.969562in}{2.886570in}}%
\pgfpathlineto{\pgfqpoint{0.972134in}{2.870114in}}%
\pgfpathlineto{\pgfqpoint{0.992709in}{2.870114in}}%
\pgfpathlineto{\pgfqpoint{1.000424in}{2.919482in}}%
\pgfpathlineto{\pgfqpoint{1.002996in}{2.919482in}}%
\pgfpathlineto{\pgfqpoint{1.010712in}{2.870114in}}%
\pgfpathlineto{\pgfqpoint{1.013284in}{2.870114in}}%
\pgfpathlineto{\pgfqpoint{1.015856in}{2.853658in}}%
\pgfpathlineto{\pgfqpoint{1.018427in}{2.787834in}}%
\pgfpathlineto{\pgfqpoint{1.026143in}{2.787834in}}%
\pgfpathlineto{\pgfqpoint{1.028715in}{2.804290in}}%
\pgfpathlineto{\pgfqpoint{1.031287in}{2.804290in}}%
\pgfpathlineto{\pgfqpoint{1.033859in}{2.787834in}}%
\pgfpathlineto{\pgfqpoint{1.039002in}{2.787834in}}%
\pgfpathlineto{\pgfqpoint{1.041574in}{2.771378in}}%
\pgfpathlineto{\pgfqpoint{1.044146in}{2.738466in}}%
\pgfpathlineto{\pgfqpoint{1.046718in}{2.722010in}}%
\pgfpathlineto{\pgfqpoint{1.049290in}{2.722010in}}%
\pgfpathlineto{\pgfqpoint{1.051862in}{2.672641in}}%
\pgfpathlineto{\pgfqpoint{1.057006in}{2.672641in}}%
\pgfpathlineto{\pgfqpoint{1.059577in}{2.656185in}}%
\pgfpathlineto{\pgfqpoint{1.087868in}{2.656185in}}%
\pgfpathlineto{\pgfqpoint{1.090440in}{2.639729in}}%
\pgfpathlineto{\pgfqpoint{1.121302in}{2.639729in}}%
\pgfpathlineto{\pgfqpoint{1.123874in}{2.656185in}}%
\pgfpathlineto{\pgfqpoint{1.126446in}{2.656185in}}%
\pgfpathlineto{\pgfqpoint{1.129018in}{2.639729in}}%
\pgfpathlineto{\pgfqpoint{1.141877in}{2.639729in}}%
\pgfpathlineto{\pgfqpoint{1.144449in}{2.656185in}}%
\pgfpathlineto{\pgfqpoint{1.147021in}{2.656185in}}%
\pgfpathlineto{\pgfqpoint{1.152165in}{2.689097in}}%
\pgfpathlineto{\pgfqpoint{1.157308in}{2.689097in}}%
\pgfpathlineto{\pgfqpoint{1.165024in}{2.738466in}}%
\pgfpathlineto{\pgfqpoint{1.172740in}{2.738466in}}%
\pgfpathlineto{\pgfqpoint{1.175312in}{2.771378in}}%
\pgfpathlineto{\pgfqpoint{1.177883in}{2.771378in}}%
\pgfpathlineto{\pgfqpoint{1.180455in}{2.804290in}}%
\pgfpathlineto{\pgfqpoint{1.183027in}{2.787834in}}%
\pgfpathlineto{\pgfqpoint{1.195887in}{2.787834in}}%
\pgfpathlineto{\pgfqpoint{1.198458in}{2.804290in}}%
\pgfpathlineto{\pgfqpoint{1.201030in}{2.754922in}}%
\pgfpathlineto{\pgfqpoint{1.206174in}{2.787834in}}%
\pgfpathlineto{\pgfqpoint{1.208746in}{2.787834in}}%
\pgfpathlineto{\pgfqpoint{1.211318in}{2.754922in}}%
\pgfpathlineto{\pgfqpoint{1.213890in}{2.754922in}}%
\pgfpathlineto{\pgfqpoint{1.216461in}{2.738466in}}%
\pgfpathlineto{\pgfqpoint{1.219033in}{2.738466in}}%
\pgfpathlineto{\pgfqpoint{1.221605in}{2.705554in}}%
\pgfpathlineto{\pgfqpoint{1.224177in}{2.689097in}}%
\pgfpathlineto{\pgfqpoint{1.226749in}{2.656185in}}%
\pgfpathlineto{\pgfqpoint{1.237036in}{2.656185in}}%
\pgfpathlineto{\pgfqpoint{1.239608in}{2.672641in}}%
\pgfpathlineto{\pgfqpoint{1.242180in}{2.672641in}}%
\pgfpathlineto{\pgfqpoint{1.244752in}{2.639729in}}%
\pgfpathlineto{\pgfqpoint{1.275615in}{2.639729in}}%
\pgfpathlineto{\pgfqpoint{1.278186in}{2.656185in}}%
\pgfpathlineto{\pgfqpoint{1.296189in}{2.656185in}}%
\pgfpathlineto{\pgfqpoint{1.298761in}{2.639729in}}%
\pgfpathlineto{\pgfqpoint{1.319336in}{2.639729in}}%
\pgfpathlineto{\pgfqpoint{1.321908in}{2.623273in}}%
\pgfpathlineto{\pgfqpoint{1.334768in}{2.623273in}}%
\pgfpathlineto{\pgfqpoint{1.337339in}{2.639729in}}%
\pgfpathlineto{\pgfqpoint{1.347627in}{2.639729in}}%
\pgfpathlineto{\pgfqpoint{1.350199in}{2.656185in}}%
\pgfpathlineto{\pgfqpoint{1.393921in}{2.656185in}}%
\pgfpathlineto{\pgfqpoint{1.396492in}{2.672641in}}%
\pgfpathlineto{\pgfqpoint{1.422211in}{2.672641in}}%
\pgfpathlineto{\pgfqpoint{1.424783in}{2.689097in}}%
\pgfpathlineto{\pgfqpoint{1.432499in}{2.689097in}}%
\pgfpathlineto{\pgfqpoint{1.435070in}{2.672641in}}%
\pgfpathlineto{\pgfqpoint{1.458217in}{2.672641in}}%
\pgfpathlineto{\pgfqpoint{1.460789in}{2.656185in}}%
\pgfpathlineto{\pgfqpoint{1.548233in}{2.656185in}}%
\pgfpathlineto{\pgfqpoint{1.550805in}{2.639729in}}%
\pgfpathlineto{\pgfqpoint{1.581667in}{2.639729in}}%
\pgfpathlineto{\pgfqpoint{1.584239in}{2.623273in}}%
\pgfpathlineto{\pgfqpoint{1.597098in}{2.623273in}}%
\pgfpathlineto{\pgfqpoint{1.599670in}{2.606817in}}%
\pgfpathlineto{\pgfqpoint{1.663967in}{2.606817in}}%
\pgfpathlineto{\pgfqpoint{1.666539in}{2.590361in}}%
\pgfpathlineto{\pgfqpoint{2.646421in}{2.590361in}}%
\pgfpathlineto{\pgfqpoint{2.646421in}{2.590361in}}%
\pgfusepath{stroke}%
\end{pgfscope}%
\begin{pgfscope}%
\pgfpathrectangle{\pgfqpoint{0.488751in}{1.946106in}}{\pgfqpoint{2.260417in}{1.502439in}}%
\pgfusepath{clip}%
\pgfsetrectcap%
\pgfsetroundjoin%
\pgfsetlinewidth{0.803000pt}%
\definecolor{currentstroke}{rgb}{0.686275,0.352941,0.313725}%
\pgfsetstrokecolor{currentstroke}%
\pgfsetstrokeopacity{0.270000}%
\pgfsetdash{}{0pt}%
\pgfpathmoveto{\pgfqpoint{0.591497in}{2.870114in}}%
\pgfpathlineto{\pgfqpoint{0.596641in}{2.952395in}}%
\pgfpathlineto{\pgfqpoint{0.599213in}{2.968851in}}%
\pgfpathlineto{\pgfqpoint{0.609500in}{2.903026in}}%
\pgfpathlineto{\pgfqpoint{0.612072in}{2.919482in}}%
\pgfpathlineto{\pgfqpoint{0.614644in}{2.853658in}}%
\pgfpathlineto{\pgfqpoint{0.617216in}{2.853658in}}%
\pgfpathlineto{\pgfqpoint{0.619788in}{2.837202in}}%
\pgfpathlineto{\pgfqpoint{0.622359in}{2.804290in}}%
\pgfpathlineto{\pgfqpoint{0.627503in}{2.837202in}}%
\pgfpathlineto{\pgfqpoint{0.630075in}{2.787834in}}%
\pgfpathlineto{\pgfqpoint{0.632647in}{2.787834in}}%
\pgfpathlineto{\pgfqpoint{0.635219in}{2.771378in}}%
\pgfpathlineto{\pgfqpoint{0.637791in}{2.738466in}}%
\pgfpathlineto{\pgfqpoint{0.648078in}{2.738466in}}%
\pgfpathlineto{\pgfqpoint{0.650650in}{2.705554in}}%
\pgfpathlineto{\pgfqpoint{0.660938in}{2.705554in}}%
\pgfpathlineto{\pgfqpoint{0.663509in}{2.722010in}}%
\pgfpathlineto{\pgfqpoint{0.676369in}{2.722010in}}%
\pgfpathlineto{\pgfqpoint{0.678941in}{2.738466in}}%
\pgfpathlineto{\pgfqpoint{0.684084in}{2.738466in}}%
\pgfpathlineto{\pgfqpoint{0.686656in}{2.771378in}}%
\pgfpathlineto{\pgfqpoint{0.689228in}{2.771378in}}%
\pgfpathlineto{\pgfqpoint{0.691800in}{2.738466in}}%
\pgfpathlineto{\pgfqpoint{0.714947in}{2.738466in}}%
\pgfpathlineto{\pgfqpoint{0.717519in}{2.722010in}}%
\pgfpathlineto{\pgfqpoint{0.725234in}{2.722010in}}%
\pgfpathlineto{\pgfqpoint{0.727806in}{2.738466in}}%
\pgfpathlineto{\pgfqpoint{0.738094in}{2.738466in}}%
\pgfpathlineto{\pgfqpoint{0.740666in}{2.754922in}}%
\pgfpathlineto{\pgfqpoint{0.743237in}{2.738466in}}%
\pgfpathlineto{\pgfqpoint{0.761240in}{2.738466in}}%
\pgfpathlineto{\pgfqpoint{0.763812in}{2.722010in}}%
\pgfpathlineto{\pgfqpoint{0.766384in}{2.754922in}}%
\pgfpathlineto{\pgfqpoint{0.768956in}{2.754922in}}%
\pgfpathlineto{\pgfqpoint{0.771528in}{2.771378in}}%
\pgfpathlineto{\pgfqpoint{0.774100in}{2.754922in}}%
\pgfpathlineto{\pgfqpoint{0.776672in}{2.754922in}}%
\pgfpathlineto{\pgfqpoint{0.781815in}{2.787834in}}%
\pgfpathlineto{\pgfqpoint{0.786959in}{2.787834in}}%
\pgfpathlineto{\pgfqpoint{0.789531in}{2.804290in}}%
\pgfpathlineto{\pgfqpoint{0.792103in}{2.804290in}}%
\pgfpathlineto{\pgfqpoint{0.794675in}{2.787834in}}%
\pgfpathlineto{\pgfqpoint{0.797247in}{2.787834in}}%
\pgfpathlineto{\pgfqpoint{0.799819in}{2.771378in}}%
\pgfpathlineto{\pgfqpoint{0.802390in}{2.771378in}}%
\pgfpathlineto{\pgfqpoint{0.804962in}{2.787834in}}%
\pgfpathlineto{\pgfqpoint{0.822965in}{2.787834in}}%
\pgfpathlineto{\pgfqpoint{0.828109in}{2.754922in}}%
\pgfpathlineto{\pgfqpoint{0.830681in}{2.754922in}}%
\pgfpathlineto{\pgfqpoint{0.833253in}{2.771378in}}%
\pgfpathlineto{\pgfqpoint{0.851256in}{2.771378in}}%
\pgfpathlineto{\pgfqpoint{0.856400in}{2.804290in}}%
\pgfpathlineto{\pgfqpoint{0.864115in}{2.804290in}}%
\pgfpathlineto{\pgfqpoint{0.866687in}{2.820746in}}%
\pgfpathlineto{\pgfqpoint{0.869259in}{2.820746in}}%
\pgfpathlineto{\pgfqpoint{0.871831in}{2.837202in}}%
\pgfpathlineto{\pgfqpoint{0.874403in}{2.804290in}}%
\pgfpathlineto{\pgfqpoint{0.882118in}{2.804290in}}%
\pgfpathlineto{\pgfqpoint{0.884690in}{2.820746in}}%
\pgfpathlineto{\pgfqpoint{0.887262in}{2.820746in}}%
\pgfpathlineto{\pgfqpoint{0.889834in}{2.804290in}}%
\pgfpathlineto{\pgfqpoint{0.897550in}{2.804290in}}%
\pgfpathlineto{\pgfqpoint{0.900121in}{2.820746in}}%
\pgfpathlineto{\pgfqpoint{0.905265in}{2.787834in}}%
\pgfpathlineto{\pgfqpoint{0.920696in}{2.787834in}}%
\pgfpathlineto{\pgfqpoint{0.923268in}{2.804290in}}%
\pgfpathlineto{\pgfqpoint{0.925840in}{2.837202in}}%
\pgfpathlineto{\pgfqpoint{0.930984in}{2.837202in}}%
\pgfpathlineto{\pgfqpoint{0.933556in}{2.853658in}}%
\pgfpathlineto{\pgfqpoint{0.936128in}{2.837202in}}%
\pgfpathlineto{\pgfqpoint{0.938700in}{2.837202in}}%
\pgfpathlineto{\pgfqpoint{0.941271in}{2.820746in}}%
\pgfpathlineto{\pgfqpoint{0.954131in}{2.820746in}}%
\pgfpathlineto{\pgfqpoint{0.956703in}{2.804290in}}%
\pgfpathlineto{\pgfqpoint{0.959274in}{2.804290in}}%
\pgfpathlineto{\pgfqpoint{0.961846in}{2.837202in}}%
\pgfpathlineto{\pgfqpoint{0.964418in}{2.853658in}}%
\pgfpathlineto{\pgfqpoint{0.990137in}{2.853658in}}%
\pgfpathlineto{\pgfqpoint{0.992709in}{2.820746in}}%
\pgfpathlineto{\pgfqpoint{0.997853in}{2.820746in}}%
\pgfpathlineto{\pgfqpoint{1.002996in}{2.787834in}}%
\pgfpathlineto{\pgfqpoint{1.008140in}{2.787834in}}%
\pgfpathlineto{\pgfqpoint{1.010712in}{2.804290in}}%
\pgfpathlineto{\pgfqpoint{1.013284in}{2.787834in}}%
\pgfpathlineto{\pgfqpoint{1.026143in}{2.787834in}}%
\pgfpathlineto{\pgfqpoint{1.028715in}{2.771378in}}%
\pgfpathlineto{\pgfqpoint{1.039002in}{2.771378in}}%
\pgfpathlineto{\pgfqpoint{1.041574in}{2.787834in}}%
\pgfpathlineto{\pgfqpoint{1.044146in}{2.771378in}}%
\pgfpathlineto{\pgfqpoint{1.057006in}{2.771378in}}%
\pgfpathlineto{\pgfqpoint{1.059577in}{2.787834in}}%
\pgfpathlineto{\pgfqpoint{1.069865in}{2.787834in}}%
\pgfpathlineto{\pgfqpoint{1.072437in}{2.771378in}}%
\pgfpathlineto{\pgfqpoint{1.075009in}{2.787834in}}%
\pgfpathlineto{\pgfqpoint{1.077581in}{2.787834in}}%
\pgfpathlineto{\pgfqpoint{1.080152in}{2.771378in}}%
\pgfpathlineto{\pgfqpoint{1.085296in}{2.771378in}}%
\pgfpathlineto{\pgfqpoint{1.090440in}{2.804290in}}%
\pgfpathlineto{\pgfqpoint{1.118730in}{2.804290in}}%
\pgfpathlineto{\pgfqpoint{1.121302in}{2.820746in}}%
\pgfpathlineto{\pgfqpoint{1.123874in}{2.804290in}}%
\pgfpathlineto{\pgfqpoint{1.126446in}{2.804290in}}%
\pgfpathlineto{\pgfqpoint{1.129018in}{2.771378in}}%
\pgfpathlineto{\pgfqpoint{1.134162in}{2.771378in}}%
\pgfpathlineto{\pgfqpoint{1.136734in}{2.754922in}}%
\pgfpathlineto{\pgfqpoint{1.139305in}{2.771378in}}%
\pgfpathlineto{\pgfqpoint{1.152165in}{2.771378in}}%
\pgfpathlineto{\pgfqpoint{1.154737in}{2.754922in}}%
\pgfpathlineto{\pgfqpoint{1.172740in}{2.754922in}}%
\pgfpathlineto{\pgfqpoint{1.175312in}{2.738466in}}%
\pgfpathlineto{\pgfqpoint{1.188171in}{2.738466in}}%
\pgfpathlineto{\pgfqpoint{1.190743in}{2.754922in}}%
\pgfpathlineto{\pgfqpoint{1.198458in}{2.754922in}}%
\pgfpathlineto{\pgfqpoint{1.201030in}{2.820746in}}%
\pgfpathlineto{\pgfqpoint{1.203602in}{2.804290in}}%
\pgfpathlineto{\pgfqpoint{1.211318in}{2.804290in}}%
\pgfpathlineto{\pgfqpoint{1.213890in}{2.787834in}}%
\pgfpathlineto{\pgfqpoint{1.216461in}{2.787834in}}%
\pgfpathlineto{\pgfqpoint{1.219033in}{2.771378in}}%
\pgfpathlineto{\pgfqpoint{1.224177in}{2.771378in}}%
\pgfpathlineto{\pgfqpoint{1.226749in}{2.787834in}}%
\pgfpathlineto{\pgfqpoint{1.231893in}{2.787834in}}%
\pgfpathlineto{\pgfqpoint{1.234465in}{2.754922in}}%
\pgfpathlineto{\pgfqpoint{1.237036in}{2.787834in}}%
\pgfpathlineto{\pgfqpoint{1.242180in}{2.787834in}}%
\pgfpathlineto{\pgfqpoint{1.244752in}{2.754922in}}%
\pgfpathlineto{\pgfqpoint{1.257611in}{2.754922in}}%
\pgfpathlineto{\pgfqpoint{1.260183in}{2.738466in}}%
\pgfpathlineto{\pgfqpoint{1.270471in}{2.738466in}}%
\pgfpathlineto{\pgfqpoint{1.273043in}{2.754922in}}%
\pgfpathlineto{\pgfqpoint{1.301333in}{2.754922in}}%
\pgfpathlineto{\pgfqpoint{1.303905in}{2.771378in}}%
\pgfpathlineto{\pgfqpoint{1.306477in}{2.754922in}}%
\pgfpathlineto{\pgfqpoint{1.334768in}{2.754922in}}%
\pgfpathlineto{\pgfqpoint{1.337339in}{2.771378in}}%
\pgfpathlineto{\pgfqpoint{1.339911in}{2.771378in}}%
\pgfpathlineto{\pgfqpoint{1.342483in}{2.754922in}}%
\pgfpathlineto{\pgfqpoint{1.378489in}{2.754922in}}%
\pgfpathlineto{\pgfqpoint{1.381061in}{2.771378in}}%
\pgfpathlineto{\pgfqpoint{1.404208in}{2.771378in}}%
\pgfpathlineto{\pgfqpoint{1.406780in}{2.787834in}}%
\pgfpathlineto{\pgfqpoint{1.422211in}{2.787834in}}%
\pgfpathlineto{\pgfqpoint{1.424783in}{2.771378in}}%
\pgfpathlineto{\pgfqpoint{1.432499in}{2.771378in}}%
\pgfpathlineto{\pgfqpoint{1.435070in}{2.787834in}}%
\pgfpathlineto{\pgfqpoint{1.440214in}{2.754922in}}%
\pgfpathlineto{\pgfqpoint{1.450502in}{2.754922in}}%
\pgfpathlineto{\pgfqpoint{1.453074in}{2.738466in}}%
\pgfpathlineto{\pgfqpoint{1.455645in}{2.754922in}}%
\pgfpathlineto{\pgfqpoint{1.458217in}{2.754922in}}%
\pgfpathlineto{\pgfqpoint{1.460789in}{2.738466in}}%
\pgfpathlineto{\pgfqpoint{1.486508in}{2.738466in}}%
\pgfpathlineto{\pgfqpoint{1.489080in}{2.754922in}}%
\pgfpathlineto{\pgfqpoint{1.491652in}{2.754922in}}%
\pgfpathlineto{\pgfqpoint{1.494223in}{2.771378in}}%
\pgfpathlineto{\pgfqpoint{1.496795in}{2.771378in}}%
\pgfpathlineto{\pgfqpoint{1.499367in}{2.804290in}}%
\pgfpathlineto{\pgfqpoint{1.509655in}{2.804290in}}%
\pgfpathlineto{\pgfqpoint{1.512227in}{2.820746in}}%
\pgfpathlineto{\pgfqpoint{1.514798in}{2.853658in}}%
\pgfpathlineto{\pgfqpoint{1.522514in}{2.804290in}}%
\pgfpathlineto{\pgfqpoint{1.532802in}{2.804290in}}%
\pgfpathlineto{\pgfqpoint{1.535373in}{2.771378in}}%
\pgfpathlineto{\pgfqpoint{1.545661in}{2.771378in}}%
\pgfpathlineto{\pgfqpoint{1.548233in}{2.754922in}}%
\pgfpathlineto{\pgfqpoint{1.550805in}{2.754922in}}%
\pgfpathlineto{\pgfqpoint{1.553376in}{2.738466in}}%
\pgfpathlineto{\pgfqpoint{1.555948in}{2.738466in}}%
\pgfpathlineto{\pgfqpoint{1.558520in}{2.754922in}}%
\pgfpathlineto{\pgfqpoint{1.571380in}{2.754922in}}%
\pgfpathlineto{\pgfqpoint{1.573951in}{2.787834in}}%
\pgfpathlineto{\pgfqpoint{1.581667in}{2.787834in}}%
\pgfpathlineto{\pgfqpoint{1.584239in}{2.837202in}}%
\pgfpathlineto{\pgfqpoint{1.586811in}{2.853658in}}%
\pgfpathlineto{\pgfqpoint{1.589383in}{2.820746in}}%
\pgfpathlineto{\pgfqpoint{1.594526in}{2.853658in}}%
\pgfpathlineto{\pgfqpoint{1.597098in}{2.886570in}}%
\pgfpathlineto{\pgfqpoint{1.599670in}{2.837202in}}%
\pgfpathlineto{\pgfqpoint{1.607386in}{2.837202in}}%
\pgfpathlineto{\pgfqpoint{1.609958in}{2.820746in}}%
\pgfpathlineto{\pgfqpoint{1.612529in}{2.837202in}}%
\pgfpathlineto{\pgfqpoint{1.615101in}{2.820746in}}%
\pgfpathlineto{\pgfqpoint{1.620245in}{2.820746in}}%
\pgfpathlineto{\pgfqpoint{1.622817in}{2.804290in}}%
\pgfpathlineto{\pgfqpoint{1.625389in}{2.804290in}}%
\pgfpathlineto{\pgfqpoint{1.627961in}{2.787834in}}%
\pgfpathlineto{\pgfqpoint{1.633104in}{2.787834in}}%
\pgfpathlineto{\pgfqpoint{1.635676in}{2.804290in}}%
\pgfpathlineto{\pgfqpoint{1.638248in}{2.804290in}}%
\pgfpathlineto{\pgfqpoint{1.640820in}{2.787834in}}%
\pgfpathlineto{\pgfqpoint{1.643392in}{2.804290in}}%
\pgfpathlineto{\pgfqpoint{1.645964in}{2.837202in}}%
\pgfpathlineto{\pgfqpoint{1.648536in}{2.837202in}}%
\pgfpathlineto{\pgfqpoint{1.651108in}{2.820746in}}%
\pgfpathlineto{\pgfqpoint{1.653679in}{2.837202in}}%
\pgfpathlineto{\pgfqpoint{1.656251in}{2.837202in}}%
\pgfpathlineto{\pgfqpoint{1.658823in}{2.804290in}}%
\pgfpathlineto{\pgfqpoint{1.661395in}{2.820746in}}%
\pgfpathlineto{\pgfqpoint{1.663967in}{2.804290in}}%
\pgfpathlineto{\pgfqpoint{1.666539in}{2.820746in}}%
\pgfpathlineto{\pgfqpoint{1.671683in}{2.820746in}}%
\pgfpathlineto{\pgfqpoint{1.674254in}{2.804290in}}%
\pgfpathlineto{\pgfqpoint{1.676826in}{2.820746in}}%
\pgfpathlineto{\pgfqpoint{1.681970in}{2.787834in}}%
\pgfpathlineto{\pgfqpoint{1.684542in}{2.787834in}}%
\pgfpathlineto{\pgfqpoint{1.687114in}{2.820746in}}%
\pgfpathlineto{\pgfqpoint{1.689686in}{2.804290in}}%
\pgfpathlineto{\pgfqpoint{1.697401in}{2.804290in}}%
\pgfpathlineto{\pgfqpoint{1.702545in}{2.837202in}}%
\pgfpathlineto{\pgfqpoint{1.705117in}{2.870114in}}%
\pgfpathlineto{\pgfqpoint{1.707689in}{2.853658in}}%
\pgfpathlineto{\pgfqpoint{1.710261in}{2.870114in}}%
\pgfpathlineto{\pgfqpoint{1.712832in}{2.903026in}}%
\pgfpathlineto{\pgfqpoint{1.715404in}{2.903026in}}%
\pgfpathlineto{\pgfqpoint{1.720548in}{2.935938in}}%
\pgfpathlineto{\pgfqpoint{1.723120in}{2.935938in}}%
\pgfpathlineto{\pgfqpoint{1.725692in}{2.952395in}}%
\pgfpathlineto{\pgfqpoint{1.730836in}{2.919482in}}%
\pgfpathlineto{\pgfqpoint{1.733407in}{2.919482in}}%
\pgfpathlineto{\pgfqpoint{1.735979in}{2.903026in}}%
\pgfpathlineto{\pgfqpoint{1.738551in}{2.935938in}}%
\pgfpathlineto{\pgfqpoint{1.741123in}{2.935938in}}%
\pgfpathlineto{\pgfqpoint{1.746267in}{2.903026in}}%
\pgfpathlineto{\pgfqpoint{1.748839in}{2.919482in}}%
\pgfpathlineto{\pgfqpoint{1.753982in}{2.919482in}}%
\pgfpathlineto{\pgfqpoint{1.756554in}{2.903026in}}%
\pgfpathlineto{\pgfqpoint{1.766842in}{2.903026in}}%
\pgfpathlineto{\pgfqpoint{1.769414in}{2.886570in}}%
\pgfpathlineto{\pgfqpoint{1.771985in}{2.903026in}}%
\pgfpathlineto{\pgfqpoint{1.777129in}{2.903026in}}%
\pgfpathlineto{\pgfqpoint{1.782273in}{2.870114in}}%
\pgfpathlineto{\pgfqpoint{1.787417in}{2.870114in}}%
\pgfpathlineto{\pgfqpoint{1.789989in}{2.886570in}}%
\pgfpathlineto{\pgfqpoint{1.800276in}{2.886570in}}%
\pgfpathlineto{\pgfqpoint{1.802848in}{2.935938in}}%
\pgfpathlineto{\pgfqpoint{1.805420in}{2.935938in}}%
\pgfpathlineto{\pgfqpoint{1.807992in}{2.952395in}}%
\pgfpathlineto{\pgfqpoint{1.823423in}{2.952395in}}%
\pgfpathlineto{\pgfqpoint{1.825995in}{2.935938in}}%
\pgfpathlineto{\pgfqpoint{1.828567in}{2.935938in}}%
\pgfpathlineto{\pgfqpoint{1.831138in}{2.919482in}}%
\pgfpathlineto{\pgfqpoint{1.833710in}{2.935938in}}%
\pgfpathlineto{\pgfqpoint{1.838854in}{2.935938in}}%
\pgfpathlineto{\pgfqpoint{1.841426in}{2.903026in}}%
\pgfpathlineto{\pgfqpoint{1.843998in}{2.903026in}}%
\pgfpathlineto{\pgfqpoint{1.846570in}{2.886570in}}%
\pgfpathlineto{\pgfqpoint{1.849142in}{2.903026in}}%
\pgfpathlineto{\pgfqpoint{1.854285in}{2.903026in}}%
\pgfpathlineto{\pgfqpoint{1.856857in}{2.919482in}}%
\pgfpathlineto{\pgfqpoint{1.859429in}{2.903026in}}%
\pgfpathlineto{\pgfqpoint{1.864573in}{2.903026in}}%
\pgfpathlineto{\pgfqpoint{1.867145in}{2.886570in}}%
\pgfpathlineto{\pgfqpoint{1.926298in}{2.886570in}}%
\pgfpathlineto{\pgfqpoint{1.928870in}{2.903026in}}%
\pgfpathlineto{\pgfqpoint{1.931441in}{2.886570in}}%
\pgfpathlineto{\pgfqpoint{1.939157in}{2.886570in}}%
\pgfpathlineto{\pgfqpoint{1.941729in}{2.870114in}}%
\pgfpathlineto{\pgfqpoint{1.946873in}{2.870114in}}%
\pgfpathlineto{\pgfqpoint{1.949444in}{2.886570in}}%
\pgfpathlineto{\pgfqpoint{1.959732in}{2.886570in}}%
\pgfpathlineto{\pgfqpoint{1.962304in}{2.870114in}}%
\pgfpathlineto{\pgfqpoint{1.967448in}{2.870114in}}%
\pgfpathlineto{\pgfqpoint{1.970019in}{2.886570in}}%
\pgfpathlineto{\pgfqpoint{1.977735in}{2.886570in}}%
\pgfpathlineto{\pgfqpoint{1.982879in}{2.853658in}}%
\pgfpathlineto{\pgfqpoint{1.995738in}{2.853658in}}%
\pgfpathlineto{\pgfqpoint{2.000882in}{2.820746in}}%
\pgfpathlineto{\pgfqpoint{2.016313in}{2.820746in}}%
\pgfpathlineto{\pgfqpoint{2.018885in}{2.804290in}}%
\pgfpathlineto{\pgfqpoint{2.021457in}{2.820746in}}%
\pgfpathlineto{\pgfqpoint{2.024029in}{2.820746in}}%
\pgfpathlineto{\pgfqpoint{2.026601in}{2.837202in}}%
\pgfpathlineto{\pgfqpoint{2.031744in}{2.837202in}}%
\pgfpathlineto{\pgfqpoint{2.034316in}{2.853658in}}%
\pgfpathlineto{\pgfqpoint{2.036888in}{2.853658in}}%
\pgfpathlineto{\pgfqpoint{2.039460in}{2.837202in}}%
\pgfpathlineto{\pgfqpoint{2.049747in}{2.837202in}}%
\pgfpathlineto{\pgfqpoint{2.052319in}{2.820746in}}%
\pgfpathlineto{\pgfqpoint{2.083182in}{2.820746in}}%
\pgfpathlineto{\pgfqpoint{2.085754in}{2.837202in}}%
\pgfpathlineto{\pgfqpoint{2.093469in}{2.837202in}}%
\pgfpathlineto{\pgfqpoint{2.096041in}{2.820746in}}%
\pgfpathlineto{\pgfqpoint{2.108900in}{2.820746in}}%
\pgfpathlineto{\pgfqpoint{2.111472in}{2.804290in}}%
\pgfpathlineto{\pgfqpoint{2.119188in}{2.804290in}}%
\pgfpathlineto{\pgfqpoint{2.121760in}{2.787834in}}%
\pgfpathlineto{\pgfqpoint{2.124332in}{2.787834in}}%
\pgfpathlineto{\pgfqpoint{2.126904in}{2.771378in}}%
\pgfpathlineto{\pgfqpoint{2.129475in}{2.804290in}}%
\pgfpathlineto{\pgfqpoint{2.134619in}{2.804290in}}%
\pgfpathlineto{\pgfqpoint{2.137191in}{2.787834in}}%
\pgfpathlineto{\pgfqpoint{2.147478in}{2.787834in}}%
\pgfpathlineto{\pgfqpoint{2.152622in}{2.820746in}}%
\pgfpathlineto{\pgfqpoint{2.155194in}{2.820746in}}%
\pgfpathlineto{\pgfqpoint{2.157766in}{2.804290in}}%
\pgfpathlineto{\pgfqpoint{2.160338in}{2.804290in}}%
\pgfpathlineto{\pgfqpoint{2.162910in}{2.787834in}}%
\pgfpathlineto{\pgfqpoint{2.165482in}{2.787834in}}%
\pgfpathlineto{\pgfqpoint{2.168053in}{2.804290in}}%
\pgfpathlineto{\pgfqpoint{2.186057in}{2.804290in}}%
\pgfpathlineto{\pgfqpoint{2.188628in}{2.820746in}}%
\pgfpathlineto{\pgfqpoint{2.196344in}{2.820746in}}%
\pgfpathlineto{\pgfqpoint{2.198916in}{2.787834in}}%
\pgfpathlineto{\pgfqpoint{2.201488in}{2.820746in}}%
\pgfpathlineto{\pgfqpoint{2.204060in}{2.820746in}}%
\pgfpathlineto{\pgfqpoint{2.206631in}{2.837202in}}%
\pgfpathlineto{\pgfqpoint{2.219491in}{2.837202in}}%
\pgfpathlineto{\pgfqpoint{2.222063in}{2.820746in}}%
\pgfpathlineto{\pgfqpoint{2.227206in}{2.820746in}}%
\pgfpathlineto{\pgfqpoint{2.229778in}{2.837202in}}%
\pgfpathlineto{\pgfqpoint{2.232350in}{2.820746in}}%
\pgfpathlineto{\pgfqpoint{2.237494in}{2.820746in}}%
\pgfpathlineto{\pgfqpoint{2.240066in}{2.804290in}}%
\pgfpathlineto{\pgfqpoint{2.242638in}{2.804290in}}%
\pgfpathlineto{\pgfqpoint{2.245210in}{2.787834in}}%
\pgfpathlineto{\pgfqpoint{2.252925in}{2.787834in}}%
\pgfpathlineto{\pgfqpoint{2.258069in}{2.754922in}}%
\pgfpathlineto{\pgfqpoint{2.299219in}{2.754922in}}%
\pgfpathlineto{\pgfqpoint{2.301791in}{2.771378in}}%
\pgfpathlineto{\pgfqpoint{2.324938in}{2.771378in}}%
\pgfpathlineto{\pgfqpoint{2.327509in}{2.754922in}}%
\pgfpathlineto{\pgfqpoint{2.358372in}{2.754922in}}%
\pgfpathlineto{\pgfqpoint{2.360944in}{2.771378in}}%
\pgfpathlineto{\pgfqpoint{2.368659in}{2.771378in}}%
\pgfpathlineto{\pgfqpoint{2.371231in}{2.787834in}}%
\pgfpathlineto{\pgfqpoint{2.386662in}{2.787834in}}%
\pgfpathlineto{\pgfqpoint{2.391806in}{2.820746in}}%
\pgfpathlineto{\pgfqpoint{2.399522in}{2.820746in}}%
\pgfpathlineto{\pgfqpoint{2.402094in}{2.853658in}}%
\pgfpathlineto{\pgfqpoint{2.404665in}{2.870114in}}%
\pgfpathlineto{\pgfqpoint{2.407237in}{2.870114in}}%
\pgfpathlineto{\pgfqpoint{2.409809in}{2.886570in}}%
\pgfpathlineto{\pgfqpoint{2.417525in}{2.886570in}}%
\pgfpathlineto{\pgfqpoint{2.420097in}{2.870114in}}%
\pgfpathlineto{\pgfqpoint{2.422669in}{2.870114in}}%
\pgfpathlineto{\pgfqpoint{2.425240in}{2.903026in}}%
\pgfpathlineto{\pgfqpoint{2.445815in}{2.903026in}}%
\pgfpathlineto{\pgfqpoint{2.448387in}{2.919482in}}%
\pgfpathlineto{\pgfqpoint{2.468962in}{2.919482in}}%
\pgfpathlineto{\pgfqpoint{2.471534in}{2.935938in}}%
\pgfpathlineto{\pgfqpoint{2.474106in}{2.903026in}}%
\pgfpathlineto{\pgfqpoint{2.489537in}{2.903026in}}%
\pgfpathlineto{\pgfqpoint{2.492109in}{2.886570in}}%
\pgfpathlineto{\pgfqpoint{2.502397in}{2.886570in}}%
\pgfpathlineto{\pgfqpoint{2.504968in}{2.903026in}}%
\pgfpathlineto{\pgfqpoint{2.507540in}{2.903026in}}%
\pgfpathlineto{\pgfqpoint{2.510112in}{2.919482in}}%
\pgfpathlineto{\pgfqpoint{2.515256in}{2.919482in}}%
\pgfpathlineto{\pgfqpoint{2.517828in}{2.903026in}}%
\pgfpathlineto{\pgfqpoint{2.528115in}{2.903026in}}%
\pgfpathlineto{\pgfqpoint{2.530687in}{2.919482in}}%
\pgfpathlineto{\pgfqpoint{2.535831in}{2.886570in}}%
\pgfpathlineto{\pgfqpoint{2.546118in}{2.886570in}}%
\pgfpathlineto{\pgfqpoint{2.548690in}{2.870114in}}%
\pgfpathlineto{\pgfqpoint{2.558978in}{2.870114in}}%
\pgfpathlineto{\pgfqpoint{2.561550in}{2.886570in}}%
\pgfpathlineto{\pgfqpoint{2.566693in}{2.886570in}}%
\pgfpathlineto{\pgfqpoint{2.569265in}{2.903026in}}%
\pgfpathlineto{\pgfqpoint{2.571837in}{2.886570in}}%
\pgfpathlineto{\pgfqpoint{2.582125in}{2.886570in}}%
\pgfpathlineto{\pgfqpoint{2.584696in}{2.903026in}}%
\pgfpathlineto{\pgfqpoint{2.589840in}{2.903026in}}%
\pgfpathlineto{\pgfqpoint{2.592412in}{2.919482in}}%
\pgfpathlineto{\pgfqpoint{2.594984in}{2.903026in}}%
\pgfpathlineto{\pgfqpoint{2.607843in}{2.903026in}}%
\pgfpathlineto{\pgfqpoint{2.610415in}{2.919482in}}%
\pgfpathlineto{\pgfqpoint{2.612987in}{2.903026in}}%
\pgfpathlineto{\pgfqpoint{2.615559in}{2.919482in}}%
\pgfpathlineto{\pgfqpoint{2.618131in}{2.886570in}}%
\pgfpathlineto{\pgfqpoint{2.623274in}{2.886570in}}%
\pgfpathlineto{\pgfqpoint{2.625846in}{2.870114in}}%
\pgfpathlineto{\pgfqpoint{2.646421in}{2.870114in}}%
\pgfpathlineto{\pgfqpoint{2.646421in}{2.870114in}}%
\pgfusepath{stroke}%
\end{pgfscope}%
\begin{pgfscope}%
\pgfpathrectangle{\pgfqpoint{0.488751in}{1.946106in}}{\pgfqpoint{2.260417in}{1.502439in}}%
\pgfusepath{clip}%
\pgfsetrectcap%
\pgfsetroundjoin%
\pgfsetlinewidth{0.803000pt}%
\definecolor{currentstroke}{rgb}{0.000000,0.356863,0.509804}%
\pgfsetstrokecolor{currentstroke}%
\pgfsetstrokeopacity{0.270000}%
\pgfsetdash{}{0pt}%
\pgfpathmoveto{\pgfqpoint{0.591497in}{2.310608in}}%
\pgfpathlineto{\pgfqpoint{0.594069in}{2.343520in}}%
\pgfpathlineto{\pgfqpoint{0.596641in}{2.392888in}}%
\pgfpathlineto{\pgfqpoint{0.599213in}{2.392888in}}%
\pgfpathlineto{\pgfqpoint{0.604356in}{2.491625in}}%
\pgfpathlineto{\pgfqpoint{0.606928in}{2.573905in}}%
\pgfpathlineto{\pgfqpoint{0.612072in}{2.606817in}}%
\pgfpathlineto{\pgfqpoint{0.614644in}{2.656185in}}%
\pgfpathlineto{\pgfqpoint{0.617216in}{2.672641in}}%
\pgfpathlineto{\pgfqpoint{0.619788in}{2.672641in}}%
\pgfpathlineto{\pgfqpoint{0.627503in}{2.853658in}}%
\pgfpathlineto{\pgfqpoint{0.630075in}{2.837202in}}%
\pgfpathlineto{\pgfqpoint{0.632647in}{2.853658in}}%
\pgfpathlineto{\pgfqpoint{0.635219in}{2.919482in}}%
\pgfpathlineto{\pgfqpoint{0.637791in}{2.935938in}}%
\pgfpathlineto{\pgfqpoint{0.640363in}{2.935938in}}%
\pgfpathlineto{\pgfqpoint{0.642934in}{2.952395in}}%
\pgfpathlineto{\pgfqpoint{0.645506in}{2.952395in}}%
\pgfpathlineto{\pgfqpoint{0.648078in}{2.968851in}}%
\pgfpathlineto{\pgfqpoint{0.650650in}{3.001763in}}%
\pgfpathlineto{\pgfqpoint{0.653222in}{2.985307in}}%
\pgfpathlineto{\pgfqpoint{0.663509in}{3.051131in}}%
\pgfpathlineto{\pgfqpoint{0.666081in}{3.051131in}}%
\pgfpathlineto{\pgfqpoint{0.668653in}{3.067587in}}%
\pgfpathlineto{\pgfqpoint{0.684084in}{3.067587in}}%
\pgfpathlineto{\pgfqpoint{0.686656in}{3.100499in}}%
\pgfpathlineto{\pgfqpoint{0.689228in}{3.116955in}}%
\pgfpathlineto{\pgfqpoint{0.691800in}{3.100499in}}%
\pgfpathlineto{\pgfqpoint{0.694372in}{3.116955in}}%
\pgfpathlineto{\pgfqpoint{0.696944in}{3.100499in}}%
\pgfpathlineto{\pgfqpoint{0.707231in}{3.100499in}}%
\pgfpathlineto{\pgfqpoint{0.709803in}{3.084043in}}%
\pgfpathlineto{\pgfqpoint{0.722662in}{3.084043in}}%
\pgfpathlineto{\pgfqpoint{0.725234in}{3.067587in}}%
\pgfpathlineto{\pgfqpoint{0.727806in}{3.100499in}}%
\pgfpathlineto{\pgfqpoint{0.735522in}{3.100499in}}%
\pgfpathlineto{\pgfqpoint{0.740666in}{3.067587in}}%
\pgfpathlineto{\pgfqpoint{0.743237in}{3.067587in}}%
\pgfpathlineto{\pgfqpoint{0.745809in}{3.084043in}}%
\pgfpathlineto{\pgfqpoint{0.748381in}{3.067587in}}%
\pgfpathlineto{\pgfqpoint{0.750953in}{3.084043in}}%
\pgfpathlineto{\pgfqpoint{0.753525in}{3.133411in}}%
\pgfpathlineto{\pgfqpoint{0.756097in}{3.100499in}}%
\pgfpathlineto{\pgfqpoint{0.763812in}{3.100499in}}%
\pgfpathlineto{\pgfqpoint{0.766384in}{3.133411in}}%
\pgfpathlineto{\pgfqpoint{0.771528in}{3.133411in}}%
\pgfpathlineto{\pgfqpoint{0.774100in}{3.100499in}}%
\pgfpathlineto{\pgfqpoint{0.776672in}{3.100499in}}%
\pgfpathlineto{\pgfqpoint{0.779244in}{3.116955in}}%
\pgfpathlineto{\pgfqpoint{0.781815in}{3.100499in}}%
\pgfpathlineto{\pgfqpoint{0.784387in}{3.116955in}}%
\pgfpathlineto{\pgfqpoint{0.789531in}{3.116955in}}%
\pgfpathlineto{\pgfqpoint{0.792103in}{3.084043in}}%
\pgfpathlineto{\pgfqpoint{0.799819in}{3.084043in}}%
\pgfpathlineto{\pgfqpoint{0.804962in}{3.116955in}}%
\pgfpathlineto{\pgfqpoint{0.807534in}{3.100499in}}%
\pgfpathlineto{\pgfqpoint{0.810106in}{3.116955in}}%
\pgfpathlineto{\pgfqpoint{0.812678in}{3.100499in}}%
\pgfpathlineto{\pgfqpoint{0.815250in}{3.100499in}}%
\pgfpathlineto{\pgfqpoint{0.817822in}{3.084043in}}%
\pgfpathlineto{\pgfqpoint{0.820393in}{3.100499in}}%
\pgfpathlineto{\pgfqpoint{0.822965in}{3.100499in}}%
\pgfpathlineto{\pgfqpoint{0.825537in}{3.084043in}}%
\pgfpathlineto{\pgfqpoint{0.835825in}{3.084043in}}%
\pgfpathlineto{\pgfqpoint{0.838397in}{3.100499in}}%
\pgfpathlineto{\pgfqpoint{0.840968in}{3.100499in}}%
\pgfpathlineto{\pgfqpoint{0.843540in}{3.116955in}}%
\pgfpathlineto{\pgfqpoint{0.846112in}{3.116955in}}%
\pgfpathlineto{\pgfqpoint{0.848684in}{3.100499in}}%
\pgfpathlineto{\pgfqpoint{0.851256in}{3.100499in}}%
\pgfpathlineto{\pgfqpoint{0.856400in}{3.133411in}}%
\pgfpathlineto{\pgfqpoint{0.858972in}{3.116955in}}%
\pgfpathlineto{\pgfqpoint{0.861543in}{3.116955in}}%
\pgfpathlineto{\pgfqpoint{0.864115in}{3.100499in}}%
\pgfpathlineto{\pgfqpoint{0.871831in}{3.100499in}}%
\pgfpathlineto{\pgfqpoint{0.874403in}{3.084043in}}%
\pgfpathlineto{\pgfqpoint{0.879547in}{3.084043in}}%
\pgfpathlineto{\pgfqpoint{0.882118in}{3.100499in}}%
\pgfpathlineto{\pgfqpoint{0.884690in}{3.100499in}}%
\pgfpathlineto{\pgfqpoint{0.887262in}{3.116955in}}%
\pgfpathlineto{\pgfqpoint{0.889834in}{3.100499in}}%
\pgfpathlineto{\pgfqpoint{0.894978in}{3.100499in}}%
\pgfpathlineto{\pgfqpoint{0.897550in}{3.084043in}}%
\pgfpathlineto{\pgfqpoint{0.900121in}{3.100499in}}%
\pgfpathlineto{\pgfqpoint{0.905265in}{3.100499in}}%
\pgfpathlineto{\pgfqpoint{0.907837in}{3.116955in}}%
\pgfpathlineto{\pgfqpoint{0.912981in}{3.116955in}}%
\pgfpathlineto{\pgfqpoint{0.915553in}{3.100499in}}%
\pgfpathlineto{\pgfqpoint{0.918125in}{3.116955in}}%
\pgfpathlineto{\pgfqpoint{0.923268in}{3.116955in}}%
\pgfpathlineto{\pgfqpoint{0.925840in}{3.084043in}}%
\pgfpathlineto{\pgfqpoint{0.928412in}{3.100499in}}%
\pgfpathlineto{\pgfqpoint{0.930984in}{3.100499in}}%
\pgfpathlineto{\pgfqpoint{0.933556in}{3.182779in}}%
\pgfpathlineto{\pgfqpoint{0.936128in}{3.116955in}}%
\pgfpathlineto{\pgfqpoint{0.941271in}{3.149867in}}%
\pgfpathlineto{\pgfqpoint{0.946415in}{3.084043in}}%
\pgfpathlineto{\pgfqpoint{0.948987in}{3.084043in}}%
\pgfpathlineto{\pgfqpoint{0.951559in}{3.100499in}}%
\pgfpathlineto{\pgfqpoint{0.954131in}{3.133411in}}%
\pgfpathlineto{\pgfqpoint{0.956703in}{3.100499in}}%
\pgfpathlineto{\pgfqpoint{0.959274in}{3.100499in}}%
\pgfpathlineto{\pgfqpoint{0.961846in}{3.116955in}}%
\pgfpathlineto{\pgfqpoint{0.966990in}{3.116955in}}%
\pgfpathlineto{\pgfqpoint{0.972134in}{3.084043in}}%
\pgfpathlineto{\pgfqpoint{0.987565in}{3.084043in}}%
\pgfpathlineto{\pgfqpoint{0.990137in}{3.100499in}}%
\pgfpathlineto{\pgfqpoint{0.995281in}{3.100499in}}%
\pgfpathlineto{\pgfqpoint{0.997853in}{3.084043in}}%
\pgfpathlineto{\pgfqpoint{1.000424in}{3.084043in}}%
\pgfpathlineto{\pgfqpoint{1.002996in}{3.100499in}}%
\pgfpathlineto{\pgfqpoint{1.005568in}{3.100499in}}%
\pgfpathlineto{\pgfqpoint{1.010712in}{3.133411in}}%
\pgfpathlineto{\pgfqpoint{1.013284in}{3.116955in}}%
\pgfpathlineto{\pgfqpoint{1.018427in}{3.182779in}}%
\pgfpathlineto{\pgfqpoint{1.020999in}{3.149867in}}%
\pgfpathlineto{\pgfqpoint{1.023571in}{3.149867in}}%
\pgfpathlineto{\pgfqpoint{1.026143in}{3.116955in}}%
\pgfpathlineto{\pgfqpoint{1.028715in}{3.133411in}}%
\pgfpathlineto{\pgfqpoint{1.036431in}{3.084043in}}%
\pgfpathlineto{\pgfqpoint{1.041574in}{3.116955in}}%
\pgfpathlineto{\pgfqpoint{1.044146in}{3.149867in}}%
\pgfpathlineto{\pgfqpoint{1.046718in}{3.116955in}}%
\pgfpathlineto{\pgfqpoint{1.049290in}{3.100499in}}%
\pgfpathlineto{\pgfqpoint{1.051862in}{3.100499in}}%
\pgfpathlineto{\pgfqpoint{1.054434in}{3.116955in}}%
\pgfpathlineto{\pgfqpoint{1.057006in}{3.100499in}}%
\pgfpathlineto{\pgfqpoint{1.059577in}{3.133411in}}%
\pgfpathlineto{\pgfqpoint{1.064721in}{3.100499in}}%
\pgfpathlineto{\pgfqpoint{1.067293in}{3.116955in}}%
\pgfpathlineto{\pgfqpoint{1.075009in}{3.116955in}}%
\pgfpathlineto{\pgfqpoint{1.077581in}{3.100499in}}%
\pgfpathlineto{\pgfqpoint{1.085296in}{3.100499in}}%
\pgfpathlineto{\pgfqpoint{1.087868in}{3.116955in}}%
\pgfpathlineto{\pgfqpoint{1.093012in}{3.182779in}}%
\pgfpathlineto{\pgfqpoint{1.095584in}{3.182779in}}%
\pgfpathlineto{\pgfqpoint{1.098155in}{3.166323in}}%
\pgfpathlineto{\pgfqpoint{1.100727in}{3.133411in}}%
\pgfpathlineto{\pgfqpoint{1.103299in}{3.116955in}}%
\pgfpathlineto{\pgfqpoint{1.113587in}{3.116955in}}%
\pgfpathlineto{\pgfqpoint{1.116159in}{3.100499in}}%
\pgfpathlineto{\pgfqpoint{1.118730in}{3.100499in}}%
\pgfpathlineto{\pgfqpoint{1.123874in}{3.133411in}}%
\pgfpathlineto{\pgfqpoint{1.126446in}{3.100499in}}%
\pgfpathlineto{\pgfqpoint{1.129018in}{3.133411in}}%
\pgfpathlineto{\pgfqpoint{1.131590in}{3.100499in}}%
\pgfpathlineto{\pgfqpoint{1.136734in}{3.100499in}}%
\pgfpathlineto{\pgfqpoint{1.139305in}{3.116955in}}%
\pgfpathlineto{\pgfqpoint{1.141877in}{3.084043in}}%
\pgfpathlineto{\pgfqpoint{1.154737in}{3.084043in}}%
\pgfpathlineto{\pgfqpoint{1.157308in}{3.100499in}}%
\pgfpathlineto{\pgfqpoint{1.162452in}{3.100499in}}%
\pgfpathlineto{\pgfqpoint{1.165024in}{3.084043in}}%
\pgfpathlineto{\pgfqpoint{1.170168in}{3.084043in}}%
\pgfpathlineto{\pgfqpoint{1.172740in}{3.100499in}}%
\pgfpathlineto{\pgfqpoint{1.177883in}{3.067587in}}%
\pgfpathlineto{\pgfqpoint{1.180455in}{3.067587in}}%
\pgfpathlineto{\pgfqpoint{1.183027in}{3.084043in}}%
\pgfpathlineto{\pgfqpoint{1.188171in}{3.084043in}}%
\pgfpathlineto{\pgfqpoint{1.190743in}{3.100499in}}%
\pgfpathlineto{\pgfqpoint{1.198458in}{3.100499in}}%
\pgfpathlineto{\pgfqpoint{1.201030in}{3.166323in}}%
\pgfpathlineto{\pgfqpoint{1.203602in}{3.133411in}}%
\pgfpathlineto{\pgfqpoint{1.208746in}{3.100499in}}%
\pgfpathlineto{\pgfqpoint{1.211318in}{3.116955in}}%
\pgfpathlineto{\pgfqpoint{1.213890in}{3.100499in}}%
\pgfpathlineto{\pgfqpoint{1.216461in}{3.133411in}}%
\pgfpathlineto{\pgfqpoint{1.219033in}{3.100499in}}%
\pgfpathlineto{\pgfqpoint{1.221605in}{3.133411in}}%
\pgfpathlineto{\pgfqpoint{1.224177in}{3.100499in}}%
\pgfpathlineto{\pgfqpoint{1.226749in}{3.149867in}}%
\pgfpathlineto{\pgfqpoint{1.231893in}{3.116955in}}%
\pgfpathlineto{\pgfqpoint{1.234465in}{3.116955in}}%
\pgfpathlineto{\pgfqpoint{1.237036in}{3.149867in}}%
\pgfpathlineto{\pgfqpoint{1.239608in}{3.100499in}}%
\pgfpathlineto{\pgfqpoint{1.242180in}{3.100499in}}%
\pgfpathlineto{\pgfqpoint{1.244752in}{3.084043in}}%
\pgfpathlineto{\pgfqpoint{1.247324in}{3.084043in}}%
\pgfpathlineto{\pgfqpoint{1.249896in}{3.067587in}}%
\pgfpathlineto{\pgfqpoint{1.252468in}{3.116955in}}%
\pgfpathlineto{\pgfqpoint{1.255040in}{3.084043in}}%
\pgfpathlineto{\pgfqpoint{1.257611in}{3.084043in}}%
\pgfpathlineto{\pgfqpoint{1.260183in}{3.116955in}}%
\pgfpathlineto{\pgfqpoint{1.262755in}{3.100499in}}%
\pgfpathlineto{\pgfqpoint{1.267899in}{3.133411in}}%
\pgfpathlineto{\pgfqpoint{1.273043in}{3.133411in}}%
\pgfpathlineto{\pgfqpoint{1.275615in}{3.100499in}}%
\pgfpathlineto{\pgfqpoint{1.280758in}{3.100499in}}%
\pgfpathlineto{\pgfqpoint{1.283330in}{3.084043in}}%
\pgfpathlineto{\pgfqpoint{1.288474in}{3.084043in}}%
\pgfpathlineto{\pgfqpoint{1.291046in}{3.100499in}}%
\pgfpathlineto{\pgfqpoint{1.303905in}{3.100499in}}%
\pgfpathlineto{\pgfqpoint{1.306477in}{3.084043in}}%
\pgfpathlineto{\pgfqpoint{1.309049in}{3.084043in}}%
\pgfpathlineto{\pgfqpoint{1.311621in}{3.100499in}}%
\pgfpathlineto{\pgfqpoint{1.314193in}{3.084043in}}%
\pgfpathlineto{\pgfqpoint{1.319336in}{3.084043in}}%
\pgfpathlineto{\pgfqpoint{1.321908in}{3.100499in}}%
\pgfpathlineto{\pgfqpoint{1.324480in}{3.133411in}}%
\pgfpathlineto{\pgfqpoint{1.327052in}{3.133411in}}%
\pgfpathlineto{\pgfqpoint{1.329624in}{3.100499in}}%
\pgfpathlineto{\pgfqpoint{1.332196in}{3.100499in}}%
\pgfpathlineto{\pgfqpoint{1.334768in}{3.116955in}}%
\pgfpathlineto{\pgfqpoint{1.337339in}{3.100499in}}%
\pgfpathlineto{\pgfqpoint{1.339911in}{3.100499in}}%
\pgfpathlineto{\pgfqpoint{1.345055in}{3.067587in}}%
\pgfpathlineto{\pgfqpoint{1.352771in}{3.067587in}}%
\pgfpathlineto{\pgfqpoint{1.355342in}{3.100499in}}%
\pgfpathlineto{\pgfqpoint{1.357914in}{3.116955in}}%
\pgfpathlineto{\pgfqpoint{1.360486in}{3.100499in}}%
\pgfpathlineto{\pgfqpoint{1.363058in}{3.116955in}}%
\pgfpathlineto{\pgfqpoint{1.365630in}{3.100499in}}%
\pgfpathlineto{\pgfqpoint{1.386205in}{3.100499in}}%
\pgfpathlineto{\pgfqpoint{1.391349in}{3.133411in}}%
\pgfpathlineto{\pgfqpoint{1.393921in}{3.133411in}}%
\pgfpathlineto{\pgfqpoint{1.396492in}{3.116955in}}%
\pgfpathlineto{\pgfqpoint{1.399064in}{3.084043in}}%
\pgfpathlineto{\pgfqpoint{1.401636in}{3.116955in}}%
\pgfpathlineto{\pgfqpoint{1.404208in}{3.100499in}}%
\pgfpathlineto{\pgfqpoint{1.406780in}{3.100499in}}%
\pgfpathlineto{\pgfqpoint{1.409352in}{3.133411in}}%
\pgfpathlineto{\pgfqpoint{1.414495in}{3.100499in}}%
\pgfpathlineto{\pgfqpoint{1.419639in}{3.100499in}}%
\pgfpathlineto{\pgfqpoint{1.422211in}{3.116955in}}%
\pgfpathlineto{\pgfqpoint{1.424783in}{3.084043in}}%
\pgfpathlineto{\pgfqpoint{1.427355in}{3.100499in}}%
\pgfpathlineto{\pgfqpoint{1.429927in}{3.100499in}}%
\pgfpathlineto{\pgfqpoint{1.432499in}{3.149867in}}%
\pgfpathlineto{\pgfqpoint{1.435070in}{3.149867in}}%
\pgfpathlineto{\pgfqpoint{1.440214in}{3.084043in}}%
\pgfpathlineto{\pgfqpoint{1.442786in}{3.084043in}}%
\pgfpathlineto{\pgfqpoint{1.445358in}{3.100499in}}%
\pgfpathlineto{\pgfqpoint{1.447930in}{3.084043in}}%
\pgfpathlineto{\pgfqpoint{1.453074in}{3.084043in}}%
\pgfpathlineto{\pgfqpoint{1.455645in}{3.100499in}}%
\pgfpathlineto{\pgfqpoint{1.460789in}{3.100499in}}%
\pgfpathlineto{\pgfqpoint{1.463361in}{3.116955in}}%
\pgfpathlineto{\pgfqpoint{1.465933in}{3.100499in}}%
\pgfpathlineto{\pgfqpoint{1.499367in}{3.100499in}}%
\pgfpathlineto{\pgfqpoint{1.504511in}{3.133411in}}%
\pgfpathlineto{\pgfqpoint{1.509655in}{3.100499in}}%
\pgfpathlineto{\pgfqpoint{1.514798in}{3.133411in}}%
\pgfpathlineto{\pgfqpoint{1.517370in}{3.116955in}}%
\pgfpathlineto{\pgfqpoint{1.525086in}{3.116955in}}%
\pgfpathlineto{\pgfqpoint{1.527658in}{3.084043in}}%
\pgfpathlineto{\pgfqpoint{1.532802in}{3.149867in}}%
\pgfpathlineto{\pgfqpoint{1.537945in}{3.116955in}}%
\pgfpathlineto{\pgfqpoint{1.543089in}{3.116955in}}%
\pgfpathlineto{\pgfqpoint{1.545661in}{3.100499in}}%
\pgfpathlineto{\pgfqpoint{1.548233in}{3.067587in}}%
\pgfpathlineto{\pgfqpoint{1.550805in}{3.084043in}}%
\pgfpathlineto{\pgfqpoint{1.553376in}{3.084043in}}%
\pgfpathlineto{\pgfqpoint{1.558520in}{3.051131in}}%
\pgfpathlineto{\pgfqpoint{1.561092in}{3.067587in}}%
\pgfpathlineto{\pgfqpoint{1.568808in}{3.067587in}}%
\pgfpathlineto{\pgfqpoint{1.571380in}{3.051131in}}%
\pgfpathlineto{\pgfqpoint{1.573951in}{3.084043in}}%
\pgfpathlineto{\pgfqpoint{1.576523in}{3.100499in}}%
\pgfpathlineto{\pgfqpoint{1.579095in}{3.100499in}}%
\pgfpathlineto{\pgfqpoint{1.581667in}{3.084043in}}%
\pgfpathlineto{\pgfqpoint{1.586811in}{3.116955in}}%
\pgfpathlineto{\pgfqpoint{1.589383in}{3.100499in}}%
\pgfpathlineto{\pgfqpoint{1.594526in}{3.100499in}}%
\pgfpathlineto{\pgfqpoint{1.597098in}{3.116955in}}%
\pgfpathlineto{\pgfqpoint{1.599670in}{3.100499in}}%
\pgfpathlineto{\pgfqpoint{1.645964in}{3.100499in}}%
\pgfpathlineto{\pgfqpoint{1.648536in}{3.084043in}}%
\pgfpathlineto{\pgfqpoint{1.651108in}{3.100499in}}%
\pgfpathlineto{\pgfqpoint{1.656251in}{3.100499in}}%
\pgfpathlineto{\pgfqpoint{1.658823in}{3.116955in}}%
\pgfpathlineto{\pgfqpoint{1.661395in}{3.067587in}}%
\pgfpathlineto{\pgfqpoint{1.663967in}{3.067587in}}%
\pgfpathlineto{\pgfqpoint{1.669111in}{3.100499in}}%
\pgfpathlineto{\pgfqpoint{1.707689in}{3.100499in}}%
\pgfpathlineto{\pgfqpoint{1.710261in}{3.116955in}}%
\pgfpathlineto{\pgfqpoint{1.712832in}{3.100499in}}%
\pgfpathlineto{\pgfqpoint{1.787417in}{3.100499in}}%
\pgfpathlineto{\pgfqpoint{1.789989in}{3.116955in}}%
\pgfpathlineto{\pgfqpoint{1.792560in}{3.100499in}}%
\pgfpathlineto{\pgfqpoint{1.797704in}{3.100499in}}%
\pgfpathlineto{\pgfqpoint{1.800276in}{3.084043in}}%
\pgfpathlineto{\pgfqpoint{1.820851in}{3.084043in}}%
\pgfpathlineto{\pgfqpoint{1.823423in}{3.100499in}}%
\pgfpathlineto{\pgfqpoint{1.838854in}{3.100499in}}%
\pgfpathlineto{\pgfqpoint{1.841426in}{3.084043in}}%
\pgfpathlineto{\pgfqpoint{1.843998in}{3.084043in}}%
\pgfpathlineto{\pgfqpoint{1.846570in}{3.067587in}}%
\pgfpathlineto{\pgfqpoint{1.862001in}{3.067587in}}%
\pgfpathlineto{\pgfqpoint{1.864573in}{3.084043in}}%
\pgfpathlineto{\pgfqpoint{1.872288in}{3.084043in}}%
\pgfpathlineto{\pgfqpoint{1.874860in}{3.100499in}}%
\pgfpathlineto{\pgfqpoint{1.887720in}{3.100499in}}%
\pgfpathlineto{\pgfqpoint{1.890291in}{3.116955in}}%
\pgfpathlineto{\pgfqpoint{1.892863in}{3.084043in}}%
\pgfpathlineto{\pgfqpoint{1.895435in}{3.067587in}}%
\pgfpathlineto{\pgfqpoint{1.900579in}{3.067587in}}%
\pgfpathlineto{\pgfqpoint{1.903151in}{3.051131in}}%
\pgfpathlineto{\pgfqpoint{1.928870in}{3.051131in}}%
\pgfpathlineto{\pgfqpoint{1.931441in}{3.034675in}}%
\pgfpathlineto{\pgfqpoint{1.934013in}{3.034675in}}%
\pgfpathlineto{\pgfqpoint{1.936585in}{3.051131in}}%
\pgfpathlineto{\pgfqpoint{1.939157in}{3.051131in}}%
\pgfpathlineto{\pgfqpoint{1.941729in}{3.034675in}}%
\pgfpathlineto{\pgfqpoint{1.944301in}{3.051131in}}%
\pgfpathlineto{\pgfqpoint{1.970019in}{3.051131in}}%
\pgfpathlineto{\pgfqpoint{1.972591in}{3.034675in}}%
\pgfpathlineto{\pgfqpoint{1.975163in}{3.051131in}}%
\pgfpathlineto{\pgfqpoint{1.977735in}{3.051131in}}%
\pgfpathlineto{\pgfqpoint{1.980307in}{3.034675in}}%
\pgfpathlineto{\pgfqpoint{1.985451in}{3.034675in}}%
\pgfpathlineto{\pgfqpoint{1.990594in}{3.001763in}}%
\pgfpathlineto{\pgfqpoint{1.998310in}{3.001763in}}%
\pgfpathlineto{\pgfqpoint{2.000882in}{2.985307in}}%
\pgfpathlineto{\pgfqpoint{2.008597in}{2.985307in}}%
\pgfpathlineto{\pgfqpoint{2.011169in}{2.968851in}}%
\pgfpathlineto{\pgfqpoint{2.024029in}{2.968851in}}%
\pgfpathlineto{\pgfqpoint{2.026601in}{2.985307in}}%
\pgfpathlineto{\pgfqpoint{2.034316in}{2.985307in}}%
\pgfpathlineto{\pgfqpoint{2.036888in}{2.968851in}}%
\pgfpathlineto{\pgfqpoint{2.039460in}{2.968851in}}%
\pgfpathlineto{\pgfqpoint{2.044604in}{3.001763in}}%
\pgfpathlineto{\pgfqpoint{2.054891in}{3.001763in}}%
\pgfpathlineto{\pgfqpoint{2.057463in}{2.985307in}}%
\pgfpathlineto{\pgfqpoint{2.083182in}{2.985307in}}%
\pgfpathlineto{\pgfqpoint{2.085754in}{3.001763in}}%
\pgfpathlineto{\pgfqpoint{2.108900in}{3.001763in}}%
\pgfpathlineto{\pgfqpoint{2.111472in}{2.985307in}}%
\pgfpathlineto{\pgfqpoint{2.114044in}{3.001763in}}%
\pgfpathlineto{\pgfqpoint{2.186057in}{3.001763in}}%
\pgfpathlineto{\pgfqpoint{2.188628in}{3.018219in}}%
\pgfpathlineto{\pgfqpoint{2.193772in}{3.018219in}}%
\pgfpathlineto{\pgfqpoint{2.196344in}{3.001763in}}%
\pgfpathlineto{\pgfqpoint{2.198916in}{3.018219in}}%
\pgfpathlineto{\pgfqpoint{2.214347in}{3.018219in}}%
\pgfpathlineto{\pgfqpoint{2.222063in}{3.067587in}}%
\pgfpathlineto{\pgfqpoint{2.224635in}{3.067587in}}%
\pgfpathlineto{\pgfqpoint{2.227206in}{3.051131in}}%
\pgfpathlineto{\pgfqpoint{2.229778in}{3.051131in}}%
\pgfpathlineto{\pgfqpoint{2.232350in}{3.034675in}}%
\pgfpathlineto{\pgfqpoint{2.245210in}{3.034675in}}%
\pgfpathlineto{\pgfqpoint{2.247781in}{3.018219in}}%
\pgfpathlineto{\pgfqpoint{2.250353in}{3.018219in}}%
\pgfpathlineto{\pgfqpoint{2.252925in}{3.034675in}}%
\pgfpathlineto{\pgfqpoint{2.268356in}{3.034675in}}%
\pgfpathlineto{\pgfqpoint{2.270928in}{3.018219in}}%
\pgfpathlineto{\pgfqpoint{2.273500in}{3.018219in}}%
\pgfpathlineto{\pgfqpoint{2.276072in}{3.001763in}}%
\pgfpathlineto{\pgfqpoint{2.299219in}{3.001763in}}%
\pgfpathlineto{\pgfqpoint{2.301791in}{3.018219in}}%
\pgfpathlineto{\pgfqpoint{2.306934in}{3.018219in}}%
\pgfpathlineto{\pgfqpoint{2.312078in}{3.051131in}}%
\pgfpathlineto{\pgfqpoint{2.317222in}{3.051131in}}%
\pgfpathlineto{\pgfqpoint{2.319794in}{3.034675in}}%
\pgfpathlineto{\pgfqpoint{2.322366in}{3.034675in}}%
\pgfpathlineto{\pgfqpoint{2.324938in}{3.051131in}}%
\pgfpathlineto{\pgfqpoint{2.335225in}{3.051131in}}%
\pgfpathlineto{\pgfqpoint{2.337797in}{3.034675in}}%
\pgfpathlineto{\pgfqpoint{2.345512in}{3.034675in}}%
\pgfpathlineto{\pgfqpoint{2.348084in}{3.051131in}}%
\pgfpathlineto{\pgfqpoint{2.358372in}{3.051131in}}%
\pgfpathlineto{\pgfqpoint{2.360944in}{3.067587in}}%
\pgfpathlineto{\pgfqpoint{2.363516in}{3.051131in}}%
\pgfpathlineto{\pgfqpoint{2.381519in}{3.051131in}}%
\pgfpathlineto{\pgfqpoint{2.384091in}{3.067587in}}%
\pgfpathlineto{\pgfqpoint{2.386662in}{3.067587in}}%
\pgfpathlineto{\pgfqpoint{2.389234in}{3.084043in}}%
\pgfpathlineto{\pgfqpoint{2.402094in}{3.084043in}}%
\pgfpathlineto{\pgfqpoint{2.404665in}{3.116955in}}%
\pgfpathlineto{\pgfqpoint{2.407237in}{3.100499in}}%
\pgfpathlineto{\pgfqpoint{2.412381in}{3.100499in}}%
\pgfpathlineto{\pgfqpoint{2.414953in}{3.116955in}}%
\pgfpathlineto{\pgfqpoint{2.417525in}{3.100499in}}%
\pgfpathlineto{\pgfqpoint{2.430384in}{3.100499in}}%
\pgfpathlineto{\pgfqpoint{2.432956in}{3.133411in}}%
\pgfpathlineto{\pgfqpoint{2.435528in}{3.116955in}}%
\pgfpathlineto{\pgfqpoint{2.443244in}{3.116955in}}%
\pgfpathlineto{\pgfqpoint{2.445815in}{3.100499in}}%
\pgfpathlineto{\pgfqpoint{2.448387in}{3.100499in}}%
\pgfpathlineto{\pgfqpoint{2.450959in}{3.116955in}}%
\pgfpathlineto{\pgfqpoint{2.453531in}{3.100499in}}%
\pgfpathlineto{\pgfqpoint{2.468962in}{3.100499in}}%
\pgfpathlineto{\pgfqpoint{2.471534in}{3.116955in}}%
\pgfpathlineto{\pgfqpoint{2.474106in}{3.100499in}}%
\pgfpathlineto{\pgfqpoint{2.499825in}{3.100499in}}%
\pgfpathlineto{\pgfqpoint{2.502397in}{3.116955in}}%
\pgfpathlineto{\pgfqpoint{2.504968in}{3.100499in}}%
\pgfpathlineto{\pgfqpoint{2.520400in}{3.100499in}}%
\pgfpathlineto{\pgfqpoint{2.522972in}{3.084043in}}%
\pgfpathlineto{\pgfqpoint{2.528115in}{3.084043in}}%
\pgfpathlineto{\pgfqpoint{2.530687in}{3.100499in}}%
\pgfpathlineto{\pgfqpoint{2.556406in}{3.100499in}}%
\pgfpathlineto{\pgfqpoint{2.558978in}{3.116955in}}%
\pgfpathlineto{\pgfqpoint{2.561550in}{3.100499in}}%
\pgfpathlineto{\pgfqpoint{2.566693in}{3.100499in}}%
\pgfpathlineto{\pgfqpoint{2.569265in}{3.084043in}}%
\pgfpathlineto{\pgfqpoint{2.571837in}{3.084043in}}%
\pgfpathlineto{\pgfqpoint{2.574409in}{3.100499in}}%
\pgfpathlineto{\pgfqpoint{2.576981in}{3.084043in}}%
\pgfpathlineto{\pgfqpoint{2.589840in}{3.084043in}}%
\pgfpathlineto{\pgfqpoint{2.592412in}{3.100499in}}%
\pgfpathlineto{\pgfqpoint{2.607843in}{3.100499in}}%
\pgfpathlineto{\pgfqpoint{2.610415in}{3.116955in}}%
\pgfpathlineto{\pgfqpoint{2.612987in}{3.100499in}}%
\pgfpathlineto{\pgfqpoint{2.630990in}{3.100499in}}%
\pgfpathlineto{\pgfqpoint{2.633562in}{3.116955in}}%
\pgfpathlineto{\pgfqpoint{2.638706in}{3.116955in}}%
\pgfpathlineto{\pgfqpoint{2.643849in}{3.084043in}}%
\pgfpathlineto{\pgfqpoint{2.646421in}{3.100499in}}%
\pgfpathlineto{\pgfqpoint{2.646421in}{3.100499in}}%
\pgfusepath{stroke}%
\end{pgfscope}%
\begin{pgfscope}%
\pgfpathrectangle{\pgfqpoint{0.488751in}{1.946106in}}{\pgfqpoint{2.260417in}{1.502439in}}%
\pgfusepath{clip}%
\pgfsetrectcap%
\pgfsetroundjoin%
\pgfsetlinewidth{0.803000pt}%
\definecolor{currentstroke}{rgb}{0.490196,0.588235,0.431373}%
\pgfsetstrokecolor{currentstroke}%
\pgfsetstrokeopacity{0.270000}%
\pgfsetdash{}{0pt}%
\pgfpathmoveto{\pgfqpoint{0.591497in}{2.837202in}}%
\pgfpathlineto{\pgfqpoint{0.596641in}{2.804290in}}%
\pgfpathlineto{\pgfqpoint{0.599213in}{2.804290in}}%
\pgfpathlineto{\pgfqpoint{0.604356in}{2.705554in}}%
\pgfpathlineto{\pgfqpoint{0.606928in}{2.705554in}}%
\pgfpathlineto{\pgfqpoint{0.609500in}{2.689097in}}%
\pgfpathlineto{\pgfqpoint{0.612072in}{2.689097in}}%
\pgfpathlineto{\pgfqpoint{0.614644in}{2.656185in}}%
\pgfpathlineto{\pgfqpoint{0.617216in}{2.573905in}}%
\pgfpathlineto{\pgfqpoint{0.619788in}{2.573905in}}%
\pgfpathlineto{\pgfqpoint{0.622359in}{2.639729in}}%
\pgfpathlineto{\pgfqpoint{0.624931in}{2.590361in}}%
\pgfpathlineto{\pgfqpoint{0.627503in}{2.623273in}}%
\pgfpathlineto{\pgfqpoint{0.630075in}{2.639729in}}%
\pgfpathlineto{\pgfqpoint{0.632647in}{2.623273in}}%
\pgfpathlineto{\pgfqpoint{0.635219in}{2.590361in}}%
\pgfpathlineto{\pgfqpoint{0.637791in}{2.573905in}}%
\pgfpathlineto{\pgfqpoint{0.640363in}{2.606817in}}%
\pgfpathlineto{\pgfqpoint{0.642934in}{2.606817in}}%
\pgfpathlineto{\pgfqpoint{0.645506in}{2.639729in}}%
\pgfpathlineto{\pgfqpoint{0.655794in}{2.573905in}}%
\pgfpathlineto{\pgfqpoint{0.658366in}{2.573905in}}%
\pgfpathlineto{\pgfqpoint{0.660938in}{2.557449in}}%
\pgfpathlineto{\pgfqpoint{0.663509in}{2.557449in}}%
\pgfpathlineto{\pgfqpoint{0.666081in}{2.639729in}}%
\pgfpathlineto{\pgfqpoint{0.668653in}{2.656185in}}%
\pgfpathlineto{\pgfqpoint{0.671225in}{2.689097in}}%
\pgfpathlineto{\pgfqpoint{0.673797in}{2.754922in}}%
\pgfpathlineto{\pgfqpoint{0.676369in}{2.771378in}}%
\pgfpathlineto{\pgfqpoint{0.678941in}{2.754922in}}%
\pgfpathlineto{\pgfqpoint{0.684084in}{2.787834in}}%
\pgfpathlineto{\pgfqpoint{0.686656in}{2.787834in}}%
\pgfpathlineto{\pgfqpoint{0.689228in}{2.804290in}}%
\pgfpathlineto{\pgfqpoint{0.691800in}{2.787834in}}%
\pgfpathlineto{\pgfqpoint{0.704659in}{2.870114in}}%
\pgfpathlineto{\pgfqpoint{0.707231in}{2.903026in}}%
\pgfpathlineto{\pgfqpoint{0.709803in}{2.886570in}}%
\pgfpathlineto{\pgfqpoint{0.712375in}{2.903026in}}%
\pgfpathlineto{\pgfqpoint{0.714947in}{2.935938in}}%
\pgfpathlineto{\pgfqpoint{0.722662in}{2.985307in}}%
\pgfpathlineto{\pgfqpoint{0.725234in}{2.985307in}}%
\pgfpathlineto{\pgfqpoint{0.727806in}{2.935938in}}%
\pgfpathlineto{\pgfqpoint{0.732950in}{2.935938in}}%
\pgfpathlineto{\pgfqpoint{0.735522in}{2.952395in}}%
\pgfpathlineto{\pgfqpoint{0.738094in}{2.952395in}}%
\pgfpathlineto{\pgfqpoint{0.740666in}{2.935938in}}%
\pgfpathlineto{\pgfqpoint{0.743237in}{2.952395in}}%
\pgfpathlineto{\pgfqpoint{0.745809in}{2.935938in}}%
\pgfpathlineto{\pgfqpoint{0.748381in}{2.935938in}}%
\pgfpathlineto{\pgfqpoint{0.753525in}{2.903026in}}%
\pgfpathlineto{\pgfqpoint{0.756097in}{2.903026in}}%
\pgfpathlineto{\pgfqpoint{0.761240in}{2.935938in}}%
\pgfpathlineto{\pgfqpoint{0.779244in}{2.935938in}}%
\pgfpathlineto{\pgfqpoint{0.784387in}{2.903026in}}%
\pgfpathlineto{\pgfqpoint{0.797247in}{2.903026in}}%
\pgfpathlineto{\pgfqpoint{0.799819in}{2.935938in}}%
\pgfpathlineto{\pgfqpoint{0.802390in}{2.952395in}}%
\pgfpathlineto{\pgfqpoint{0.804962in}{2.935938in}}%
\pgfpathlineto{\pgfqpoint{0.807534in}{2.952395in}}%
\pgfpathlineto{\pgfqpoint{0.810106in}{2.919482in}}%
\pgfpathlineto{\pgfqpoint{0.812678in}{2.935938in}}%
\pgfpathlineto{\pgfqpoint{0.815250in}{2.935938in}}%
\pgfpathlineto{\pgfqpoint{0.820393in}{2.870114in}}%
\pgfpathlineto{\pgfqpoint{0.822965in}{2.886570in}}%
\pgfpathlineto{\pgfqpoint{0.833253in}{2.886570in}}%
\pgfpathlineto{\pgfqpoint{0.838397in}{2.919482in}}%
\pgfpathlineto{\pgfqpoint{0.840968in}{2.919482in}}%
\pgfpathlineto{\pgfqpoint{0.843540in}{2.903026in}}%
\pgfpathlineto{\pgfqpoint{0.848684in}{2.903026in}}%
\pgfpathlineto{\pgfqpoint{0.851256in}{2.886570in}}%
\pgfpathlineto{\pgfqpoint{0.856400in}{2.820746in}}%
\pgfpathlineto{\pgfqpoint{0.858972in}{2.820746in}}%
\pgfpathlineto{\pgfqpoint{0.861543in}{2.804290in}}%
\pgfpathlineto{\pgfqpoint{0.866687in}{2.804290in}}%
\pgfpathlineto{\pgfqpoint{0.869259in}{2.787834in}}%
\pgfpathlineto{\pgfqpoint{0.871831in}{2.787834in}}%
\pgfpathlineto{\pgfqpoint{0.874403in}{2.853658in}}%
\pgfpathlineto{\pgfqpoint{0.876975in}{2.837202in}}%
\pgfpathlineto{\pgfqpoint{0.882118in}{2.837202in}}%
\pgfpathlineto{\pgfqpoint{0.884690in}{2.853658in}}%
\pgfpathlineto{\pgfqpoint{0.889834in}{2.853658in}}%
\pgfpathlineto{\pgfqpoint{0.892406in}{2.870114in}}%
\pgfpathlineto{\pgfqpoint{0.900121in}{2.870114in}}%
\pgfpathlineto{\pgfqpoint{0.902693in}{2.886570in}}%
\pgfpathlineto{\pgfqpoint{0.905265in}{2.886570in}}%
\pgfpathlineto{\pgfqpoint{0.907837in}{2.870114in}}%
\pgfpathlineto{\pgfqpoint{0.910409in}{2.886570in}}%
\pgfpathlineto{\pgfqpoint{0.912981in}{2.886570in}}%
\pgfpathlineto{\pgfqpoint{0.915553in}{2.903026in}}%
\pgfpathlineto{\pgfqpoint{0.918125in}{2.853658in}}%
\pgfpathlineto{\pgfqpoint{0.925840in}{2.804290in}}%
\pgfpathlineto{\pgfqpoint{0.928412in}{2.837202in}}%
\pgfpathlineto{\pgfqpoint{0.930984in}{2.837202in}}%
\pgfpathlineto{\pgfqpoint{0.933556in}{2.804290in}}%
\pgfpathlineto{\pgfqpoint{0.936128in}{2.820746in}}%
\pgfpathlineto{\pgfqpoint{0.938700in}{2.787834in}}%
\pgfpathlineto{\pgfqpoint{0.941271in}{2.804290in}}%
\pgfpathlineto{\pgfqpoint{0.943843in}{2.771378in}}%
\pgfpathlineto{\pgfqpoint{0.959274in}{2.771378in}}%
\pgfpathlineto{\pgfqpoint{0.961846in}{2.787834in}}%
\pgfpathlineto{\pgfqpoint{0.964418in}{2.771378in}}%
\pgfpathlineto{\pgfqpoint{0.982421in}{2.771378in}}%
\pgfpathlineto{\pgfqpoint{0.984993in}{2.787834in}}%
\pgfpathlineto{\pgfqpoint{0.987565in}{2.787834in}}%
\pgfpathlineto{\pgfqpoint{0.990137in}{2.804290in}}%
\pgfpathlineto{\pgfqpoint{1.000424in}{2.804290in}}%
\pgfpathlineto{\pgfqpoint{1.002996in}{2.771378in}}%
\pgfpathlineto{\pgfqpoint{1.008140in}{2.771378in}}%
\pgfpathlineto{\pgfqpoint{1.010712in}{2.754922in}}%
\pgfpathlineto{\pgfqpoint{1.126446in}{2.754922in}}%
\pgfpathlineto{\pgfqpoint{1.129018in}{2.738466in}}%
\pgfpathlineto{\pgfqpoint{1.180455in}{2.738466in}}%
\pgfpathlineto{\pgfqpoint{1.183027in}{2.722010in}}%
\pgfpathlineto{\pgfqpoint{1.257611in}{2.722010in}}%
\pgfpathlineto{\pgfqpoint{1.260183in}{2.705554in}}%
\pgfpathlineto{\pgfqpoint{1.460789in}{2.705554in}}%
\pgfpathlineto{\pgfqpoint{1.463361in}{2.689097in}}%
\pgfpathlineto{\pgfqpoint{1.710261in}{2.689097in}}%
\pgfpathlineto{\pgfqpoint{1.712832in}{2.672641in}}%
\pgfpathlineto{\pgfqpoint{1.728264in}{2.672641in}}%
\pgfpathlineto{\pgfqpoint{1.730836in}{2.689097in}}%
\pgfpathlineto{\pgfqpoint{1.774557in}{2.689097in}}%
\pgfpathlineto{\pgfqpoint{1.777129in}{2.672641in}}%
\pgfpathlineto{\pgfqpoint{1.838854in}{2.672641in}}%
\pgfpathlineto{\pgfqpoint{1.841426in}{2.689097in}}%
\pgfpathlineto{\pgfqpoint{2.306934in}{2.689097in}}%
\pgfpathlineto{\pgfqpoint{2.309506in}{2.672641in}}%
\pgfpathlineto{\pgfqpoint{2.625846in}{2.672641in}}%
\pgfpathlineto{\pgfqpoint{2.628418in}{2.689097in}}%
\pgfpathlineto{\pgfqpoint{2.630990in}{2.689097in}}%
\pgfpathlineto{\pgfqpoint{2.633562in}{2.705554in}}%
\pgfpathlineto{\pgfqpoint{2.646421in}{2.705554in}}%
\pgfpathlineto{\pgfqpoint{2.646421in}{2.705554in}}%
\pgfusepath{stroke}%
\end{pgfscope}%
\begin{pgfscope}%
\pgfpathrectangle{\pgfqpoint{0.488751in}{1.946106in}}{\pgfqpoint{2.260417in}{1.502439in}}%
\pgfusepath{clip}%
\pgfsetrectcap%
\pgfsetroundjoin%
\pgfsetlinewidth{0.803000pt}%
\definecolor{currentstroke}{rgb}{0.843137,0.666667,0.313725}%
\pgfsetstrokecolor{currentstroke}%
\pgfsetstrokeopacity{0.270000}%
\pgfsetdash{}{0pt}%
\pgfpathmoveto{\pgfqpoint{0.591497in}{2.146047in}}%
\pgfpathlineto{\pgfqpoint{0.599213in}{2.014399in}}%
\pgfpathlineto{\pgfqpoint{0.601785in}{2.030855in}}%
\pgfpathlineto{\pgfqpoint{0.604356in}{2.080223in}}%
\pgfpathlineto{\pgfqpoint{0.606928in}{2.096679in}}%
\pgfpathlineto{\pgfqpoint{0.609500in}{2.096679in}}%
\pgfpathlineto{\pgfqpoint{0.614644in}{2.162503in}}%
\pgfpathlineto{\pgfqpoint{0.617216in}{2.129591in}}%
\pgfpathlineto{\pgfqpoint{0.619788in}{2.113135in}}%
\pgfpathlineto{\pgfqpoint{0.624931in}{2.178959in}}%
\pgfpathlineto{\pgfqpoint{0.630075in}{2.294152in}}%
\pgfpathlineto{\pgfqpoint{0.632647in}{2.261240in}}%
\pgfpathlineto{\pgfqpoint{0.635219in}{2.294152in}}%
\pgfpathlineto{\pgfqpoint{0.640363in}{2.228328in}}%
\pgfpathlineto{\pgfqpoint{0.642934in}{2.228328in}}%
\pgfpathlineto{\pgfqpoint{0.645506in}{2.327064in}}%
\pgfpathlineto{\pgfqpoint{0.648078in}{2.359976in}}%
\pgfpathlineto{\pgfqpoint{0.650650in}{2.409344in}}%
\pgfpathlineto{\pgfqpoint{0.653222in}{2.409344in}}%
\pgfpathlineto{\pgfqpoint{0.655794in}{2.392888in}}%
\pgfpathlineto{\pgfqpoint{0.658366in}{2.425800in}}%
\pgfpathlineto{\pgfqpoint{0.660938in}{2.409344in}}%
\pgfpathlineto{\pgfqpoint{0.666081in}{2.491625in}}%
\pgfpathlineto{\pgfqpoint{0.673797in}{2.442256in}}%
\pgfpathlineto{\pgfqpoint{0.676369in}{2.458712in}}%
\pgfpathlineto{\pgfqpoint{0.678941in}{2.442256in}}%
\pgfpathlineto{\pgfqpoint{0.684084in}{2.442256in}}%
\pgfpathlineto{\pgfqpoint{0.686656in}{2.425800in}}%
\pgfpathlineto{\pgfqpoint{0.689228in}{2.376432in}}%
\pgfpathlineto{\pgfqpoint{0.691800in}{2.359976in}}%
\pgfpathlineto{\pgfqpoint{0.694372in}{2.392888in}}%
\pgfpathlineto{\pgfqpoint{0.696944in}{2.376432in}}%
\pgfpathlineto{\pgfqpoint{0.699516in}{2.392888in}}%
\pgfpathlineto{\pgfqpoint{0.702087in}{2.392888in}}%
\pgfpathlineto{\pgfqpoint{0.704659in}{2.409344in}}%
\pgfpathlineto{\pgfqpoint{0.707231in}{2.458712in}}%
\pgfpathlineto{\pgfqpoint{0.709803in}{2.475169in}}%
\pgfpathlineto{\pgfqpoint{0.712375in}{2.508081in}}%
\pgfpathlineto{\pgfqpoint{0.714947in}{2.524537in}}%
\pgfpathlineto{\pgfqpoint{0.717519in}{2.524537in}}%
\pgfpathlineto{\pgfqpoint{0.720091in}{2.540993in}}%
\pgfpathlineto{\pgfqpoint{0.722662in}{2.573905in}}%
\pgfpathlineto{\pgfqpoint{0.725234in}{2.590361in}}%
\pgfpathlineto{\pgfqpoint{0.727806in}{2.557449in}}%
\pgfpathlineto{\pgfqpoint{0.730378in}{2.557449in}}%
\pgfpathlineto{\pgfqpoint{0.735522in}{2.590361in}}%
\pgfpathlineto{\pgfqpoint{0.738094in}{2.573905in}}%
\pgfpathlineto{\pgfqpoint{0.740666in}{2.540993in}}%
\pgfpathlineto{\pgfqpoint{0.743237in}{2.540993in}}%
\pgfpathlineto{\pgfqpoint{0.745809in}{2.508081in}}%
\pgfpathlineto{\pgfqpoint{0.748381in}{2.524537in}}%
\pgfpathlineto{\pgfqpoint{0.750953in}{2.491625in}}%
\pgfpathlineto{\pgfqpoint{0.753525in}{2.425800in}}%
\pgfpathlineto{\pgfqpoint{0.758669in}{2.425800in}}%
\pgfpathlineto{\pgfqpoint{0.761240in}{2.442256in}}%
\pgfpathlineto{\pgfqpoint{0.771528in}{2.442256in}}%
\pgfpathlineto{\pgfqpoint{0.774100in}{2.458712in}}%
\pgfpathlineto{\pgfqpoint{0.776672in}{2.491625in}}%
\pgfpathlineto{\pgfqpoint{0.779244in}{2.491625in}}%
\pgfpathlineto{\pgfqpoint{0.781815in}{2.508081in}}%
\pgfpathlineto{\pgfqpoint{0.786959in}{2.508081in}}%
\pgfpathlineto{\pgfqpoint{0.789531in}{2.491625in}}%
\pgfpathlineto{\pgfqpoint{0.792103in}{2.508081in}}%
\pgfpathlineto{\pgfqpoint{0.799819in}{2.508081in}}%
\pgfpathlineto{\pgfqpoint{0.802390in}{2.491625in}}%
\pgfpathlineto{\pgfqpoint{0.804962in}{2.458712in}}%
\pgfpathlineto{\pgfqpoint{0.807534in}{2.458712in}}%
\pgfpathlineto{\pgfqpoint{0.810106in}{2.392888in}}%
\pgfpathlineto{\pgfqpoint{0.812678in}{2.409344in}}%
\pgfpathlineto{\pgfqpoint{0.815250in}{2.409344in}}%
\pgfpathlineto{\pgfqpoint{0.817822in}{2.442256in}}%
\pgfpathlineto{\pgfqpoint{0.820393in}{2.409344in}}%
\pgfpathlineto{\pgfqpoint{0.825537in}{2.409344in}}%
\pgfpathlineto{\pgfqpoint{0.828109in}{2.442256in}}%
\pgfpathlineto{\pgfqpoint{0.830681in}{2.442256in}}%
\pgfpathlineto{\pgfqpoint{0.835825in}{2.409344in}}%
\pgfpathlineto{\pgfqpoint{0.840968in}{2.442256in}}%
\pgfpathlineto{\pgfqpoint{0.846112in}{2.409344in}}%
\pgfpathlineto{\pgfqpoint{0.848684in}{2.409344in}}%
\pgfpathlineto{\pgfqpoint{0.851256in}{2.327064in}}%
\pgfpathlineto{\pgfqpoint{0.856400in}{2.425800in}}%
\pgfpathlineto{\pgfqpoint{0.869259in}{2.425800in}}%
\pgfpathlineto{\pgfqpoint{0.871831in}{2.442256in}}%
\pgfpathlineto{\pgfqpoint{0.874403in}{2.475169in}}%
\pgfpathlineto{\pgfqpoint{0.879547in}{2.475169in}}%
\pgfpathlineto{\pgfqpoint{0.882118in}{2.458712in}}%
\pgfpathlineto{\pgfqpoint{0.884690in}{2.458712in}}%
\pgfpathlineto{\pgfqpoint{0.887262in}{2.425800in}}%
\pgfpathlineto{\pgfqpoint{0.892406in}{2.392888in}}%
\pgfpathlineto{\pgfqpoint{0.897550in}{2.425800in}}%
\pgfpathlineto{\pgfqpoint{0.902693in}{2.425800in}}%
\pgfpathlineto{\pgfqpoint{0.907837in}{2.359976in}}%
\pgfpathlineto{\pgfqpoint{0.910409in}{2.343520in}}%
\pgfpathlineto{\pgfqpoint{0.918125in}{2.392888in}}%
\pgfpathlineto{\pgfqpoint{0.920696in}{2.359976in}}%
\pgfpathlineto{\pgfqpoint{0.923268in}{2.392888in}}%
\pgfpathlineto{\pgfqpoint{0.925840in}{2.409344in}}%
\pgfpathlineto{\pgfqpoint{0.930984in}{2.409344in}}%
\pgfpathlineto{\pgfqpoint{0.933556in}{2.508081in}}%
\pgfpathlineto{\pgfqpoint{0.936128in}{2.508081in}}%
\pgfpathlineto{\pgfqpoint{0.938700in}{2.540993in}}%
\pgfpathlineto{\pgfqpoint{0.941271in}{2.524537in}}%
\pgfpathlineto{\pgfqpoint{0.943843in}{2.524537in}}%
\pgfpathlineto{\pgfqpoint{0.946415in}{2.557449in}}%
\pgfpathlineto{\pgfqpoint{0.951559in}{2.590361in}}%
\pgfpathlineto{\pgfqpoint{0.954131in}{2.557449in}}%
\pgfpathlineto{\pgfqpoint{0.956703in}{2.557449in}}%
\pgfpathlineto{\pgfqpoint{0.959274in}{2.540993in}}%
\pgfpathlineto{\pgfqpoint{0.961846in}{2.508081in}}%
\pgfpathlineto{\pgfqpoint{0.964418in}{2.491625in}}%
\pgfpathlineto{\pgfqpoint{0.977278in}{2.491625in}}%
\pgfpathlineto{\pgfqpoint{0.982421in}{2.524537in}}%
\pgfpathlineto{\pgfqpoint{0.987565in}{2.524537in}}%
\pgfpathlineto{\pgfqpoint{0.990137in}{2.573905in}}%
\pgfpathlineto{\pgfqpoint{0.992709in}{2.573905in}}%
\pgfpathlineto{\pgfqpoint{0.995281in}{2.590361in}}%
\pgfpathlineto{\pgfqpoint{1.000424in}{2.590361in}}%
\pgfpathlineto{\pgfqpoint{1.002996in}{2.573905in}}%
\pgfpathlineto{\pgfqpoint{1.005568in}{2.573905in}}%
\pgfpathlineto{\pgfqpoint{1.010712in}{2.508081in}}%
\pgfpathlineto{\pgfqpoint{1.013284in}{2.508081in}}%
\pgfpathlineto{\pgfqpoint{1.020999in}{2.376432in}}%
\pgfpathlineto{\pgfqpoint{1.023571in}{2.376432in}}%
\pgfpathlineto{\pgfqpoint{1.026143in}{2.392888in}}%
\pgfpathlineto{\pgfqpoint{1.028715in}{2.359976in}}%
\pgfpathlineto{\pgfqpoint{1.031287in}{2.359976in}}%
\pgfpathlineto{\pgfqpoint{1.033859in}{2.310608in}}%
\pgfpathlineto{\pgfqpoint{1.036431in}{2.310608in}}%
\pgfpathlineto{\pgfqpoint{1.039002in}{2.294152in}}%
\pgfpathlineto{\pgfqpoint{1.041574in}{2.310608in}}%
\pgfpathlineto{\pgfqpoint{1.044146in}{2.343520in}}%
\pgfpathlineto{\pgfqpoint{1.049290in}{2.343520in}}%
\pgfpathlineto{\pgfqpoint{1.051862in}{2.376432in}}%
\pgfpathlineto{\pgfqpoint{1.057006in}{2.376432in}}%
\pgfpathlineto{\pgfqpoint{1.059577in}{2.425800in}}%
\pgfpathlineto{\pgfqpoint{1.062149in}{2.392888in}}%
\pgfpathlineto{\pgfqpoint{1.067293in}{2.425800in}}%
\pgfpathlineto{\pgfqpoint{1.069865in}{2.475169in}}%
\pgfpathlineto{\pgfqpoint{1.085296in}{2.475169in}}%
\pgfpathlineto{\pgfqpoint{1.087868in}{2.458712in}}%
\pgfpathlineto{\pgfqpoint{1.095584in}{2.508081in}}%
\pgfpathlineto{\pgfqpoint{1.098155in}{2.491625in}}%
\pgfpathlineto{\pgfqpoint{1.100727in}{2.491625in}}%
\pgfpathlineto{\pgfqpoint{1.103299in}{2.508081in}}%
\pgfpathlineto{\pgfqpoint{1.105871in}{2.508081in}}%
\pgfpathlineto{\pgfqpoint{1.108443in}{2.491625in}}%
\pgfpathlineto{\pgfqpoint{1.111015in}{2.458712in}}%
\pgfpathlineto{\pgfqpoint{1.113587in}{2.475169in}}%
\pgfpathlineto{\pgfqpoint{1.118730in}{2.475169in}}%
\pgfpathlineto{\pgfqpoint{1.123874in}{2.409344in}}%
\pgfpathlineto{\pgfqpoint{1.126446in}{2.392888in}}%
\pgfpathlineto{\pgfqpoint{1.129018in}{2.442256in}}%
\pgfpathlineto{\pgfqpoint{1.131590in}{2.458712in}}%
\pgfpathlineto{\pgfqpoint{1.134162in}{2.442256in}}%
\pgfpathlineto{\pgfqpoint{1.136734in}{2.442256in}}%
\pgfpathlineto{\pgfqpoint{1.139305in}{2.475169in}}%
\pgfpathlineto{\pgfqpoint{1.141877in}{2.475169in}}%
\pgfpathlineto{\pgfqpoint{1.144449in}{2.491625in}}%
\pgfpathlineto{\pgfqpoint{1.147021in}{2.491625in}}%
\pgfpathlineto{\pgfqpoint{1.149593in}{2.475169in}}%
\pgfpathlineto{\pgfqpoint{1.154737in}{2.508081in}}%
\pgfpathlineto{\pgfqpoint{1.157308in}{2.508081in}}%
\pgfpathlineto{\pgfqpoint{1.159880in}{2.491625in}}%
\pgfpathlineto{\pgfqpoint{1.167596in}{2.491625in}}%
\pgfpathlineto{\pgfqpoint{1.172740in}{2.524537in}}%
\pgfpathlineto{\pgfqpoint{1.177883in}{2.524537in}}%
\pgfpathlineto{\pgfqpoint{1.183027in}{2.491625in}}%
\pgfpathlineto{\pgfqpoint{1.185599in}{2.491625in}}%
\pgfpathlineto{\pgfqpoint{1.188171in}{2.475169in}}%
\pgfpathlineto{\pgfqpoint{1.198458in}{2.475169in}}%
\pgfpathlineto{\pgfqpoint{1.201030in}{2.458712in}}%
\pgfpathlineto{\pgfqpoint{1.203602in}{2.475169in}}%
\pgfpathlineto{\pgfqpoint{1.208746in}{2.475169in}}%
\pgfpathlineto{\pgfqpoint{1.211318in}{2.508081in}}%
\pgfpathlineto{\pgfqpoint{1.213890in}{2.524537in}}%
\pgfpathlineto{\pgfqpoint{1.216461in}{2.491625in}}%
\pgfpathlineto{\pgfqpoint{1.219033in}{2.491625in}}%
\pgfpathlineto{\pgfqpoint{1.221605in}{2.540993in}}%
\pgfpathlineto{\pgfqpoint{1.224177in}{2.508081in}}%
\pgfpathlineto{\pgfqpoint{1.226749in}{2.508081in}}%
\pgfpathlineto{\pgfqpoint{1.229321in}{2.540993in}}%
\pgfpathlineto{\pgfqpoint{1.234465in}{2.573905in}}%
\pgfpathlineto{\pgfqpoint{1.237036in}{2.639729in}}%
\pgfpathlineto{\pgfqpoint{1.239608in}{2.656185in}}%
\pgfpathlineto{\pgfqpoint{1.242180in}{2.639729in}}%
\pgfpathlineto{\pgfqpoint{1.244752in}{2.606817in}}%
\pgfpathlineto{\pgfqpoint{1.247324in}{2.623273in}}%
\pgfpathlineto{\pgfqpoint{1.249896in}{2.656185in}}%
\pgfpathlineto{\pgfqpoint{1.252468in}{2.606817in}}%
\pgfpathlineto{\pgfqpoint{1.257611in}{2.606817in}}%
\pgfpathlineto{\pgfqpoint{1.260183in}{2.656185in}}%
\pgfpathlineto{\pgfqpoint{1.262755in}{2.672641in}}%
\pgfpathlineto{\pgfqpoint{1.265327in}{2.639729in}}%
\pgfpathlineto{\pgfqpoint{1.267899in}{2.639729in}}%
\pgfpathlineto{\pgfqpoint{1.270471in}{2.623273in}}%
\pgfpathlineto{\pgfqpoint{1.285902in}{2.623273in}}%
\pgfpathlineto{\pgfqpoint{1.291046in}{2.705554in}}%
\pgfpathlineto{\pgfqpoint{1.296189in}{2.705554in}}%
\pgfpathlineto{\pgfqpoint{1.298761in}{2.722010in}}%
\pgfpathlineto{\pgfqpoint{1.301333in}{2.754922in}}%
\pgfpathlineto{\pgfqpoint{1.303905in}{2.754922in}}%
\pgfpathlineto{\pgfqpoint{1.306477in}{2.738466in}}%
\pgfpathlineto{\pgfqpoint{1.309049in}{2.754922in}}%
\pgfpathlineto{\pgfqpoint{1.311621in}{2.722010in}}%
\pgfpathlineto{\pgfqpoint{1.314193in}{2.754922in}}%
\pgfpathlineto{\pgfqpoint{1.321908in}{2.754922in}}%
\pgfpathlineto{\pgfqpoint{1.327052in}{2.722010in}}%
\pgfpathlineto{\pgfqpoint{1.329624in}{2.738466in}}%
\pgfpathlineto{\pgfqpoint{1.332196in}{2.738466in}}%
\pgfpathlineto{\pgfqpoint{1.334768in}{2.705554in}}%
\pgfpathlineto{\pgfqpoint{1.337339in}{2.722010in}}%
\pgfpathlineto{\pgfqpoint{1.339911in}{2.754922in}}%
\pgfpathlineto{\pgfqpoint{1.345055in}{2.754922in}}%
\pgfpathlineto{\pgfqpoint{1.347627in}{2.771378in}}%
\pgfpathlineto{\pgfqpoint{1.352771in}{2.771378in}}%
\pgfpathlineto{\pgfqpoint{1.355342in}{2.738466in}}%
\pgfpathlineto{\pgfqpoint{1.357914in}{2.722010in}}%
\pgfpathlineto{\pgfqpoint{1.360486in}{2.689097in}}%
\pgfpathlineto{\pgfqpoint{1.363058in}{2.689097in}}%
\pgfpathlineto{\pgfqpoint{1.365630in}{2.672641in}}%
\pgfpathlineto{\pgfqpoint{1.373346in}{2.672641in}}%
\pgfpathlineto{\pgfqpoint{1.375917in}{2.722010in}}%
\pgfpathlineto{\pgfqpoint{1.381061in}{2.722010in}}%
\pgfpathlineto{\pgfqpoint{1.383633in}{2.738466in}}%
\pgfpathlineto{\pgfqpoint{1.386205in}{2.738466in}}%
\pgfpathlineto{\pgfqpoint{1.391349in}{2.705554in}}%
\pgfpathlineto{\pgfqpoint{1.396492in}{2.705554in}}%
\pgfpathlineto{\pgfqpoint{1.399064in}{2.722010in}}%
\pgfpathlineto{\pgfqpoint{1.401636in}{2.754922in}}%
\pgfpathlineto{\pgfqpoint{1.404208in}{2.754922in}}%
\pgfpathlineto{\pgfqpoint{1.406780in}{2.771378in}}%
\pgfpathlineto{\pgfqpoint{1.409352in}{2.722010in}}%
\pgfpathlineto{\pgfqpoint{1.419639in}{2.722010in}}%
\pgfpathlineto{\pgfqpoint{1.422211in}{2.705554in}}%
\pgfpathlineto{\pgfqpoint{1.427355in}{2.705554in}}%
\pgfpathlineto{\pgfqpoint{1.429927in}{2.672641in}}%
\pgfpathlineto{\pgfqpoint{1.432499in}{2.606817in}}%
\pgfpathlineto{\pgfqpoint{1.435070in}{2.606817in}}%
\pgfpathlineto{\pgfqpoint{1.437642in}{2.623273in}}%
\pgfpathlineto{\pgfqpoint{1.440214in}{2.623273in}}%
\pgfpathlineto{\pgfqpoint{1.442786in}{2.639729in}}%
\pgfpathlineto{\pgfqpoint{1.445358in}{2.606817in}}%
\pgfpathlineto{\pgfqpoint{1.447930in}{2.590361in}}%
\pgfpathlineto{\pgfqpoint{1.450502in}{2.590361in}}%
\pgfpathlineto{\pgfqpoint{1.453074in}{2.623273in}}%
\pgfpathlineto{\pgfqpoint{1.455645in}{2.606817in}}%
\pgfpathlineto{\pgfqpoint{1.460789in}{2.639729in}}%
\pgfpathlineto{\pgfqpoint{1.468505in}{2.639729in}}%
\pgfpathlineto{\pgfqpoint{1.471077in}{2.623273in}}%
\pgfpathlineto{\pgfqpoint{1.473649in}{2.639729in}}%
\pgfpathlineto{\pgfqpoint{1.476220in}{2.639729in}}%
\pgfpathlineto{\pgfqpoint{1.478792in}{2.623273in}}%
\pgfpathlineto{\pgfqpoint{1.486508in}{2.672641in}}%
\pgfpathlineto{\pgfqpoint{1.489080in}{2.672641in}}%
\pgfpathlineto{\pgfqpoint{1.494223in}{2.606817in}}%
\pgfpathlineto{\pgfqpoint{1.499367in}{2.639729in}}%
\pgfpathlineto{\pgfqpoint{1.504511in}{2.573905in}}%
\pgfpathlineto{\pgfqpoint{1.507083in}{2.557449in}}%
\pgfpathlineto{\pgfqpoint{1.509655in}{2.573905in}}%
\pgfpathlineto{\pgfqpoint{1.512227in}{2.623273in}}%
\pgfpathlineto{\pgfqpoint{1.517370in}{2.656185in}}%
\pgfpathlineto{\pgfqpoint{1.519942in}{2.689097in}}%
\pgfpathlineto{\pgfqpoint{1.522514in}{2.656185in}}%
\pgfpathlineto{\pgfqpoint{1.525086in}{2.639729in}}%
\pgfpathlineto{\pgfqpoint{1.527658in}{2.639729in}}%
\pgfpathlineto{\pgfqpoint{1.530230in}{2.722010in}}%
\pgfpathlineto{\pgfqpoint{1.532802in}{2.689097in}}%
\pgfpathlineto{\pgfqpoint{1.535373in}{2.705554in}}%
\pgfpathlineto{\pgfqpoint{1.537945in}{2.705554in}}%
\pgfpathlineto{\pgfqpoint{1.540517in}{2.656185in}}%
\pgfpathlineto{\pgfqpoint{1.543089in}{2.689097in}}%
\pgfpathlineto{\pgfqpoint{1.548233in}{2.689097in}}%
\pgfpathlineto{\pgfqpoint{1.550805in}{2.656185in}}%
\pgfpathlineto{\pgfqpoint{1.553376in}{2.639729in}}%
\pgfpathlineto{\pgfqpoint{1.558520in}{2.639729in}}%
\pgfpathlineto{\pgfqpoint{1.561092in}{2.606817in}}%
\pgfpathlineto{\pgfqpoint{1.563664in}{2.590361in}}%
\pgfpathlineto{\pgfqpoint{1.566236in}{2.590361in}}%
\pgfpathlineto{\pgfqpoint{1.568808in}{2.573905in}}%
\pgfpathlineto{\pgfqpoint{1.571380in}{2.590361in}}%
\pgfpathlineto{\pgfqpoint{1.576523in}{2.491625in}}%
\pgfpathlineto{\pgfqpoint{1.579095in}{2.508081in}}%
\pgfpathlineto{\pgfqpoint{1.581667in}{2.491625in}}%
\pgfpathlineto{\pgfqpoint{1.584239in}{2.458712in}}%
\pgfpathlineto{\pgfqpoint{1.586811in}{2.442256in}}%
\pgfpathlineto{\pgfqpoint{1.589383in}{2.442256in}}%
\pgfpathlineto{\pgfqpoint{1.594526in}{2.359976in}}%
\pgfpathlineto{\pgfqpoint{1.597098in}{2.392888in}}%
\pgfpathlineto{\pgfqpoint{1.599670in}{2.392888in}}%
\pgfpathlineto{\pgfqpoint{1.602242in}{2.425800in}}%
\pgfpathlineto{\pgfqpoint{1.607386in}{2.425800in}}%
\pgfpathlineto{\pgfqpoint{1.612529in}{2.392888in}}%
\pgfpathlineto{\pgfqpoint{1.615101in}{2.409344in}}%
\pgfpathlineto{\pgfqpoint{1.617673in}{2.376432in}}%
\pgfpathlineto{\pgfqpoint{1.622817in}{2.442256in}}%
\pgfpathlineto{\pgfqpoint{1.625389in}{2.442256in}}%
\pgfpathlineto{\pgfqpoint{1.627961in}{2.458712in}}%
\pgfpathlineto{\pgfqpoint{1.630533in}{2.425800in}}%
\pgfpathlineto{\pgfqpoint{1.633104in}{2.425800in}}%
\pgfpathlineto{\pgfqpoint{1.638248in}{2.392888in}}%
\pgfpathlineto{\pgfqpoint{1.643392in}{2.327064in}}%
\pgfpathlineto{\pgfqpoint{1.645964in}{2.310608in}}%
\pgfpathlineto{\pgfqpoint{1.648536in}{2.327064in}}%
\pgfpathlineto{\pgfqpoint{1.651108in}{2.327064in}}%
\pgfpathlineto{\pgfqpoint{1.658823in}{2.376432in}}%
\pgfpathlineto{\pgfqpoint{1.663967in}{2.376432in}}%
\pgfpathlineto{\pgfqpoint{1.671683in}{2.327064in}}%
\pgfpathlineto{\pgfqpoint{1.674254in}{2.327064in}}%
\pgfpathlineto{\pgfqpoint{1.679398in}{2.392888in}}%
\pgfpathlineto{\pgfqpoint{1.681970in}{2.392888in}}%
\pgfpathlineto{\pgfqpoint{1.684542in}{2.343520in}}%
\pgfpathlineto{\pgfqpoint{1.687114in}{2.359976in}}%
\pgfpathlineto{\pgfqpoint{1.689686in}{2.359976in}}%
\pgfpathlineto{\pgfqpoint{1.692257in}{2.392888in}}%
\pgfpathlineto{\pgfqpoint{1.697401in}{2.392888in}}%
\pgfpathlineto{\pgfqpoint{1.699973in}{2.359976in}}%
\pgfpathlineto{\pgfqpoint{1.702545in}{2.359976in}}%
\pgfpathlineto{\pgfqpoint{1.705117in}{2.294152in}}%
\pgfpathlineto{\pgfqpoint{1.707689in}{2.310608in}}%
\pgfpathlineto{\pgfqpoint{1.712832in}{2.211871in}}%
\pgfpathlineto{\pgfqpoint{1.715404in}{2.211871in}}%
\pgfpathlineto{\pgfqpoint{1.717976in}{2.195415in}}%
\pgfpathlineto{\pgfqpoint{1.720548in}{2.195415in}}%
\pgfpathlineto{\pgfqpoint{1.723120in}{2.178959in}}%
\pgfpathlineto{\pgfqpoint{1.728264in}{2.178959in}}%
\pgfpathlineto{\pgfqpoint{1.733407in}{2.146047in}}%
\pgfpathlineto{\pgfqpoint{1.735979in}{2.146047in}}%
\pgfpathlineto{\pgfqpoint{1.738551in}{2.113135in}}%
\pgfpathlineto{\pgfqpoint{1.771985in}{2.113135in}}%
\pgfpathlineto{\pgfqpoint{1.774557in}{2.129591in}}%
\pgfpathlineto{\pgfqpoint{1.787417in}{2.129591in}}%
\pgfpathlineto{\pgfqpoint{1.789989in}{2.113135in}}%
\pgfpathlineto{\pgfqpoint{1.820851in}{2.113135in}}%
\pgfpathlineto{\pgfqpoint{1.823423in}{2.096679in}}%
\pgfpathlineto{\pgfqpoint{1.849142in}{2.096679in}}%
\pgfpathlineto{\pgfqpoint{1.851713in}{2.080223in}}%
\pgfpathlineto{\pgfqpoint{1.903151in}{2.080223in}}%
\pgfpathlineto{\pgfqpoint{1.905723in}{2.063767in}}%
\pgfpathlineto{\pgfqpoint{1.908295in}{2.063767in}}%
\pgfpathlineto{\pgfqpoint{1.910866in}{2.080223in}}%
\pgfpathlineto{\pgfqpoint{1.941729in}{2.080223in}}%
\pgfpathlineto{\pgfqpoint{1.944301in}{2.096679in}}%
\pgfpathlineto{\pgfqpoint{1.959732in}{2.096679in}}%
\pgfpathlineto{\pgfqpoint{1.962304in}{2.113135in}}%
\pgfpathlineto{\pgfqpoint{1.967448in}{2.113135in}}%
\pgfpathlineto{\pgfqpoint{1.970019in}{2.096679in}}%
\pgfpathlineto{\pgfqpoint{2.039460in}{2.096679in}}%
\pgfpathlineto{\pgfqpoint{2.042032in}{2.063767in}}%
\pgfpathlineto{\pgfqpoint{2.047176in}{2.063767in}}%
\pgfpathlineto{\pgfqpoint{2.049747in}{2.080223in}}%
\pgfpathlineto{\pgfqpoint{2.196344in}{2.080223in}}%
\pgfpathlineto{\pgfqpoint{2.198916in}{2.063767in}}%
\pgfpathlineto{\pgfqpoint{2.201488in}{2.080223in}}%
\pgfpathlineto{\pgfqpoint{2.312078in}{2.080223in}}%
\pgfpathlineto{\pgfqpoint{2.314650in}{2.063767in}}%
\pgfpathlineto{\pgfqpoint{2.317222in}{2.080223in}}%
\pgfpathlineto{\pgfqpoint{2.378947in}{2.080223in}}%
\pgfpathlineto{\pgfqpoint{2.381519in}{2.063767in}}%
\pgfpathlineto{\pgfqpoint{2.384091in}{2.063767in}}%
\pgfpathlineto{\pgfqpoint{2.386662in}{2.080223in}}%
\pgfpathlineto{\pgfqpoint{2.430384in}{2.080223in}}%
\pgfpathlineto{\pgfqpoint{2.432956in}{2.063767in}}%
\pgfpathlineto{\pgfqpoint{2.435528in}{2.080223in}}%
\pgfpathlineto{\pgfqpoint{2.438100in}{2.080223in}}%
\pgfpathlineto{\pgfqpoint{2.440672in}{2.063767in}}%
\pgfpathlineto{\pgfqpoint{2.443244in}{2.063767in}}%
\pgfpathlineto{\pgfqpoint{2.445815in}{2.080223in}}%
\pgfpathlineto{\pgfqpoint{2.448387in}{2.063767in}}%
\pgfpathlineto{\pgfqpoint{2.450959in}{2.063767in}}%
\pgfpathlineto{\pgfqpoint{2.453531in}{2.080223in}}%
\pgfpathlineto{\pgfqpoint{2.461247in}{2.080223in}}%
\pgfpathlineto{\pgfqpoint{2.463819in}{2.063767in}}%
\pgfpathlineto{\pgfqpoint{2.466390in}{2.063767in}}%
\pgfpathlineto{\pgfqpoint{2.468962in}{2.080223in}}%
\pgfpathlineto{\pgfqpoint{2.504968in}{2.080223in}}%
\pgfpathlineto{\pgfqpoint{2.507540in}{2.096679in}}%
\pgfpathlineto{\pgfqpoint{2.510112in}{2.096679in}}%
\pgfpathlineto{\pgfqpoint{2.512684in}{2.080223in}}%
\pgfpathlineto{\pgfqpoint{2.556406in}{2.080223in}}%
\pgfpathlineto{\pgfqpoint{2.558978in}{2.063767in}}%
\pgfpathlineto{\pgfqpoint{2.561550in}{2.080223in}}%
\pgfpathlineto{\pgfqpoint{2.589840in}{2.080223in}}%
\pgfpathlineto{\pgfqpoint{2.592412in}{2.063767in}}%
\pgfpathlineto{\pgfqpoint{2.594984in}{2.080223in}}%
\pgfpathlineto{\pgfqpoint{2.600128in}{2.080223in}}%
\pgfpathlineto{\pgfqpoint{2.602699in}{2.063767in}}%
\pgfpathlineto{\pgfqpoint{2.605271in}{2.080223in}}%
\pgfpathlineto{\pgfqpoint{2.607843in}{2.080223in}}%
\pgfpathlineto{\pgfqpoint{2.610415in}{2.063767in}}%
\pgfpathlineto{\pgfqpoint{2.612987in}{2.080223in}}%
\pgfpathlineto{\pgfqpoint{2.646421in}{2.080223in}}%
\pgfpathlineto{\pgfqpoint{2.646421in}{2.080223in}}%
\pgfusepath{stroke}%
\end{pgfscope}%
\begin{pgfscope}%
\pgfpathrectangle{\pgfqpoint{0.488751in}{1.946106in}}{\pgfqpoint{2.260417in}{1.502439in}}%
\pgfusepath{clip}%
\pgfsetrectcap%
\pgfsetroundjoin%
\pgfsetlinewidth{0.803000pt}%
\definecolor{currentstroke}{rgb}{0.333333,0.333333,0.333333}%
\pgfsetstrokecolor{currentstroke}%
\pgfsetstrokeopacity{0.270000}%
\pgfsetdash{}{0pt}%
\pgfpathmoveto{\pgfqpoint{0.591497in}{2.327064in}}%
\pgfpathlineto{\pgfqpoint{0.594069in}{2.343520in}}%
\pgfpathlineto{\pgfqpoint{0.599213in}{2.343520in}}%
\pgfpathlineto{\pgfqpoint{0.601785in}{2.359976in}}%
\pgfpathlineto{\pgfqpoint{0.604356in}{2.343520in}}%
\pgfpathlineto{\pgfqpoint{0.606928in}{2.343520in}}%
\pgfpathlineto{\pgfqpoint{0.609500in}{2.327064in}}%
\pgfpathlineto{\pgfqpoint{0.612072in}{2.327064in}}%
\pgfpathlineto{\pgfqpoint{0.617216in}{2.392888in}}%
\pgfpathlineto{\pgfqpoint{0.619788in}{2.409344in}}%
\pgfpathlineto{\pgfqpoint{0.622359in}{2.409344in}}%
\pgfpathlineto{\pgfqpoint{0.627503in}{2.491625in}}%
\pgfpathlineto{\pgfqpoint{0.630075in}{2.475169in}}%
\pgfpathlineto{\pgfqpoint{0.632647in}{2.508081in}}%
\pgfpathlineto{\pgfqpoint{0.635219in}{2.491625in}}%
\pgfpathlineto{\pgfqpoint{0.640363in}{2.491625in}}%
\pgfpathlineto{\pgfqpoint{0.642934in}{2.508081in}}%
\pgfpathlineto{\pgfqpoint{0.645506in}{2.491625in}}%
\pgfpathlineto{\pgfqpoint{0.648078in}{2.508081in}}%
\pgfpathlineto{\pgfqpoint{0.653222in}{2.475169in}}%
\pgfpathlineto{\pgfqpoint{0.655794in}{2.475169in}}%
\pgfpathlineto{\pgfqpoint{0.660938in}{2.508081in}}%
\pgfpathlineto{\pgfqpoint{0.663509in}{2.491625in}}%
\pgfpathlineto{\pgfqpoint{0.668653in}{2.491625in}}%
\pgfpathlineto{\pgfqpoint{0.673797in}{2.458712in}}%
\pgfpathlineto{\pgfqpoint{0.686656in}{2.458712in}}%
\pgfpathlineto{\pgfqpoint{0.691800in}{2.392888in}}%
\pgfpathlineto{\pgfqpoint{0.694372in}{2.376432in}}%
\pgfpathlineto{\pgfqpoint{0.696944in}{2.376432in}}%
\pgfpathlineto{\pgfqpoint{0.702087in}{2.310608in}}%
\pgfpathlineto{\pgfqpoint{0.704659in}{2.294152in}}%
\pgfpathlineto{\pgfqpoint{0.707231in}{2.261240in}}%
\pgfpathlineto{\pgfqpoint{0.709803in}{2.244784in}}%
\pgfpathlineto{\pgfqpoint{0.712375in}{2.195415in}}%
\pgfpathlineto{\pgfqpoint{0.714947in}{2.195415in}}%
\pgfpathlineto{\pgfqpoint{0.717519in}{2.178959in}}%
\pgfpathlineto{\pgfqpoint{0.720091in}{2.178959in}}%
\pgfpathlineto{\pgfqpoint{0.722662in}{2.146047in}}%
\pgfpathlineto{\pgfqpoint{0.732950in}{2.146047in}}%
\pgfpathlineto{\pgfqpoint{0.735522in}{2.129591in}}%
\pgfpathlineto{\pgfqpoint{0.740666in}{2.129591in}}%
\pgfpathlineto{\pgfqpoint{0.743237in}{2.113135in}}%
\pgfpathlineto{\pgfqpoint{0.758669in}{2.113135in}}%
\pgfpathlineto{\pgfqpoint{0.761240in}{2.129591in}}%
\pgfpathlineto{\pgfqpoint{0.763812in}{2.129591in}}%
\pgfpathlineto{\pgfqpoint{0.766384in}{2.113135in}}%
\pgfpathlineto{\pgfqpoint{0.768956in}{2.113135in}}%
\pgfpathlineto{\pgfqpoint{0.771528in}{2.096679in}}%
\pgfpathlineto{\pgfqpoint{0.774100in}{2.096679in}}%
\pgfpathlineto{\pgfqpoint{0.776672in}{2.080223in}}%
\pgfpathlineto{\pgfqpoint{0.786959in}{2.080223in}}%
\pgfpathlineto{\pgfqpoint{0.789531in}{2.096679in}}%
\pgfpathlineto{\pgfqpoint{0.820393in}{2.096679in}}%
\pgfpathlineto{\pgfqpoint{0.825537in}{2.063767in}}%
\pgfpathlineto{\pgfqpoint{0.828109in}{2.080223in}}%
\pgfpathlineto{\pgfqpoint{0.869259in}{2.080223in}}%
\pgfpathlineto{\pgfqpoint{0.871831in}{2.096679in}}%
\pgfpathlineto{\pgfqpoint{0.874403in}{2.096679in}}%
\pgfpathlineto{\pgfqpoint{0.876975in}{2.113135in}}%
\pgfpathlineto{\pgfqpoint{0.884690in}{2.113135in}}%
\pgfpathlineto{\pgfqpoint{0.887262in}{2.096679in}}%
\pgfpathlineto{\pgfqpoint{0.889834in}{2.096679in}}%
\pgfpathlineto{\pgfqpoint{0.892406in}{2.080223in}}%
\pgfpathlineto{\pgfqpoint{0.894978in}{2.080223in}}%
\pgfpathlineto{\pgfqpoint{0.897550in}{2.063767in}}%
\pgfpathlineto{\pgfqpoint{0.900121in}{2.063767in}}%
\pgfpathlineto{\pgfqpoint{0.905265in}{2.096679in}}%
\pgfpathlineto{\pgfqpoint{0.907837in}{2.129591in}}%
\pgfpathlineto{\pgfqpoint{0.915553in}{2.129591in}}%
\pgfpathlineto{\pgfqpoint{0.918125in}{2.162503in}}%
\pgfpathlineto{\pgfqpoint{0.920696in}{2.162503in}}%
\pgfpathlineto{\pgfqpoint{0.923268in}{2.178959in}}%
\pgfpathlineto{\pgfqpoint{0.928412in}{2.178959in}}%
\pgfpathlineto{\pgfqpoint{0.930984in}{2.162503in}}%
\pgfpathlineto{\pgfqpoint{0.933556in}{2.195415in}}%
\pgfpathlineto{\pgfqpoint{0.966990in}{2.195415in}}%
\pgfpathlineto{\pgfqpoint{0.969562in}{2.211871in}}%
\pgfpathlineto{\pgfqpoint{1.010712in}{2.211871in}}%
\pgfpathlineto{\pgfqpoint{1.015856in}{2.178959in}}%
\pgfpathlineto{\pgfqpoint{1.018427in}{2.211871in}}%
\pgfpathlineto{\pgfqpoint{1.023571in}{2.211871in}}%
\pgfpathlineto{\pgfqpoint{1.028715in}{2.178959in}}%
\pgfpathlineto{\pgfqpoint{1.049290in}{2.178959in}}%
\pgfpathlineto{\pgfqpoint{1.051862in}{2.162503in}}%
\pgfpathlineto{\pgfqpoint{1.054434in}{2.162503in}}%
\pgfpathlineto{\pgfqpoint{1.059577in}{2.129591in}}%
\pgfpathlineto{\pgfqpoint{1.067293in}{2.129591in}}%
\pgfpathlineto{\pgfqpoint{1.069865in}{2.146047in}}%
\pgfpathlineto{\pgfqpoint{1.090440in}{2.146047in}}%
\pgfpathlineto{\pgfqpoint{1.093012in}{2.195415in}}%
\pgfpathlineto{\pgfqpoint{1.098155in}{2.129591in}}%
\pgfpathlineto{\pgfqpoint{1.100727in}{2.113135in}}%
\pgfpathlineto{\pgfqpoint{1.103299in}{2.113135in}}%
\pgfpathlineto{\pgfqpoint{1.105871in}{2.096679in}}%
\pgfpathlineto{\pgfqpoint{1.108443in}{2.096679in}}%
\pgfpathlineto{\pgfqpoint{1.111015in}{2.113135in}}%
\pgfpathlineto{\pgfqpoint{1.116159in}{2.113135in}}%
\pgfpathlineto{\pgfqpoint{1.118730in}{2.096679in}}%
\pgfpathlineto{\pgfqpoint{1.121302in}{2.096679in}}%
\pgfpathlineto{\pgfqpoint{1.123874in}{2.080223in}}%
\pgfpathlineto{\pgfqpoint{1.126446in}{2.096679in}}%
\pgfpathlineto{\pgfqpoint{1.129018in}{2.096679in}}%
\pgfpathlineto{\pgfqpoint{1.131590in}{2.080223in}}%
\pgfpathlineto{\pgfqpoint{1.134162in}{2.113135in}}%
\pgfpathlineto{\pgfqpoint{1.136734in}{2.113135in}}%
\pgfpathlineto{\pgfqpoint{1.139305in}{2.129591in}}%
\pgfpathlineto{\pgfqpoint{1.141877in}{2.096679in}}%
\pgfpathlineto{\pgfqpoint{1.144449in}{2.113135in}}%
\pgfpathlineto{\pgfqpoint{1.147021in}{2.113135in}}%
\pgfpathlineto{\pgfqpoint{1.149593in}{2.096679in}}%
\pgfpathlineto{\pgfqpoint{1.157308in}{2.096679in}}%
\pgfpathlineto{\pgfqpoint{1.159880in}{2.080223in}}%
\pgfpathlineto{\pgfqpoint{1.175312in}{2.080223in}}%
\pgfpathlineto{\pgfqpoint{1.177883in}{2.063767in}}%
\pgfpathlineto{\pgfqpoint{1.180455in}{2.080223in}}%
\pgfpathlineto{\pgfqpoint{1.188171in}{2.080223in}}%
\pgfpathlineto{\pgfqpoint{1.190743in}{2.096679in}}%
\pgfpathlineto{\pgfqpoint{1.193315in}{2.096679in}}%
\pgfpathlineto{\pgfqpoint{1.195887in}{2.080223in}}%
\pgfpathlineto{\pgfqpoint{1.198458in}{2.080223in}}%
\pgfpathlineto{\pgfqpoint{1.201030in}{2.113135in}}%
\pgfpathlineto{\pgfqpoint{1.203602in}{2.096679in}}%
\pgfpathlineto{\pgfqpoint{1.213890in}{2.096679in}}%
\pgfpathlineto{\pgfqpoint{1.216461in}{2.129591in}}%
\pgfpathlineto{\pgfqpoint{1.219033in}{2.113135in}}%
\pgfpathlineto{\pgfqpoint{1.221605in}{2.113135in}}%
\pgfpathlineto{\pgfqpoint{1.224177in}{2.129591in}}%
\pgfpathlineto{\pgfqpoint{1.226749in}{2.113135in}}%
\pgfpathlineto{\pgfqpoint{1.229321in}{2.113135in}}%
\pgfpathlineto{\pgfqpoint{1.231893in}{2.096679in}}%
\pgfpathlineto{\pgfqpoint{1.237036in}{2.096679in}}%
\pgfpathlineto{\pgfqpoint{1.239608in}{2.063767in}}%
\pgfpathlineto{\pgfqpoint{1.242180in}{2.080223in}}%
\pgfpathlineto{\pgfqpoint{1.255040in}{2.080223in}}%
\pgfpathlineto{\pgfqpoint{1.257611in}{2.096679in}}%
\pgfpathlineto{\pgfqpoint{1.260183in}{2.096679in}}%
\pgfpathlineto{\pgfqpoint{1.265327in}{2.129591in}}%
\pgfpathlineto{\pgfqpoint{1.270471in}{2.096679in}}%
\pgfpathlineto{\pgfqpoint{1.275615in}{2.096679in}}%
\pgfpathlineto{\pgfqpoint{1.278186in}{2.129591in}}%
\pgfpathlineto{\pgfqpoint{1.280758in}{2.113135in}}%
\pgfpathlineto{\pgfqpoint{1.285902in}{2.113135in}}%
\pgfpathlineto{\pgfqpoint{1.288474in}{2.096679in}}%
\pgfpathlineto{\pgfqpoint{1.298761in}{2.096679in}}%
\pgfpathlineto{\pgfqpoint{1.301333in}{2.113135in}}%
\pgfpathlineto{\pgfqpoint{1.306477in}{2.080223in}}%
\pgfpathlineto{\pgfqpoint{1.311621in}{2.080223in}}%
\pgfpathlineto{\pgfqpoint{1.314193in}{2.063767in}}%
\pgfpathlineto{\pgfqpoint{1.316764in}{2.080223in}}%
\pgfpathlineto{\pgfqpoint{1.332196in}{2.080223in}}%
\pgfpathlineto{\pgfqpoint{1.334768in}{2.096679in}}%
\pgfpathlineto{\pgfqpoint{1.337339in}{2.096679in}}%
\pgfpathlineto{\pgfqpoint{1.339911in}{2.080223in}}%
\pgfpathlineto{\pgfqpoint{1.363058in}{2.080223in}}%
\pgfpathlineto{\pgfqpoint{1.365630in}{2.096679in}}%
\pgfpathlineto{\pgfqpoint{1.381061in}{2.096679in}}%
\pgfpathlineto{\pgfqpoint{1.383633in}{2.080223in}}%
\pgfpathlineto{\pgfqpoint{1.409352in}{2.080223in}}%
\pgfpathlineto{\pgfqpoint{1.411924in}{2.063767in}}%
\pgfpathlineto{\pgfqpoint{1.414495in}{2.080223in}}%
\pgfpathlineto{\pgfqpoint{1.417067in}{2.063767in}}%
\pgfpathlineto{\pgfqpoint{1.419639in}{2.080223in}}%
\pgfpathlineto{\pgfqpoint{1.432499in}{2.080223in}}%
\pgfpathlineto{\pgfqpoint{1.435070in}{2.113135in}}%
\pgfpathlineto{\pgfqpoint{1.437642in}{2.113135in}}%
\pgfpathlineto{\pgfqpoint{1.442786in}{2.080223in}}%
\pgfpathlineto{\pgfqpoint{1.445358in}{2.047311in}}%
\pgfpathlineto{\pgfqpoint{1.447930in}{2.080223in}}%
\pgfpathlineto{\pgfqpoint{1.455645in}{2.080223in}}%
\pgfpathlineto{\pgfqpoint{1.458217in}{2.063767in}}%
\pgfpathlineto{\pgfqpoint{1.460789in}{2.080223in}}%
\pgfpathlineto{\pgfqpoint{1.463361in}{2.063767in}}%
\pgfpathlineto{\pgfqpoint{1.465933in}{2.080223in}}%
\pgfpathlineto{\pgfqpoint{1.486508in}{2.080223in}}%
\pgfpathlineto{\pgfqpoint{1.489080in}{2.047311in}}%
\pgfpathlineto{\pgfqpoint{1.491652in}{2.080223in}}%
\pgfpathlineto{\pgfqpoint{1.494223in}{2.080223in}}%
\pgfpathlineto{\pgfqpoint{1.496795in}{2.047311in}}%
\pgfpathlineto{\pgfqpoint{1.499367in}{2.047311in}}%
\pgfpathlineto{\pgfqpoint{1.501939in}{2.080223in}}%
\pgfpathlineto{\pgfqpoint{1.504511in}{2.096679in}}%
\pgfpathlineto{\pgfqpoint{1.509655in}{2.096679in}}%
\pgfpathlineto{\pgfqpoint{1.514798in}{2.063767in}}%
\pgfpathlineto{\pgfqpoint{1.517370in}{2.080223in}}%
\pgfpathlineto{\pgfqpoint{1.519942in}{2.063767in}}%
\pgfpathlineto{\pgfqpoint{1.522514in}{2.080223in}}%
\pgfpathlineto{\pgfqpoint{1.525086in}{2.080223in}}%
\pgfpathlineto{\pgfqpoint{1.527658in}{2.063767in}}%
\pgfpathlineto{\pgfqpoint{1.535373in}{2.063767in}}%
\pgfpathlineto{\pgfqpoint{1.537945in}{2.080223in}}%
\pgfpathlineto{\pgfqpoint{1.543089in}{2.080223in}}%
\pgfpathlineto{\pgfqpoint{1.545661in}{2.063767in}}%
\pgfpathlineto{\pgfqpoint{1.548233in}{2.080223in}}%
\pgfpathlineto{\pgfqpoint{1.561092in}{2.080223in}}%
\pgfpathlineto{\pgfqpoint{1.563664in}{2.063767in}}%
\pgfpathlineto{\pgfqpoint{1.566236in}{2.080223in}}%
\pgfpathlineto{\pgfqpoint{1.576523in}{2.080223in}}%
\pgfpathlineto{\pgfqpoint{1.579095in}{2.063767in}}%
\pgfpathlineto{\pgfqpoint{1.581667in}{2.080223in}}%
\pgfpathlineto{\pgfqpoint{1.586811in}{2.080223in}}%
\pgfpathlineto{\pgfqpoint{1.589383in}{2.063767in}}%
\pgfpathlineto{\pgfqpoint{1.599670in}{2.063767in}}%
\pgfpathlineto{\pgfqpoint{1.602242in}{2.047311in}}%
\pgfpathlineto{\pgfqpoint{1.604814in}{2.080223in}}%
\pgfpathlineto{\pgfqpoint{1.640820in}{2.080223in}}%
\pgfpathlineto{\pgfqpoint{1.643392in}{2.096679in}}%
\pgfpathlineto{\pgfqpoint{1.661395in}{2.096679in}}%
\pgfpathlineto{\pgfqpoint{1.663967in}{2.113135in}}%
\pgfpathlineto{\pgfqpoint{1.676826in}{2.113135in}}%
\pgfpathlineto{\pgfqpoint{1.679398in}{2.096679in}}%
\pgfpathlineto{\pgfqpoint{1.715404in}{2.096679in}}%
\pgfpathlineto{\pgfqpoint{1.717976in}{2.113135in}}%
\pgfpathlineto{\pgfqpoint{1.723120in}{2.113135in}}%
\pgfpathlineto{\pgfqpoint{1.725692in}{2.129591in}}%
\pgfpathlineto{\pgfqpoint{1.751410in}{2.129591in}}%
\pgfpathlineto{\pgfqpoint{1.753982in}{2.113135in}}%
\pgfpathlineto{\pgfqpoint{1.756554in}{2.113135in}}%
\pgfpathlineto{\pgfqpoint{1.759126in}{2.129591in}}%
\pgfpathlineto{\pgfqpoint{1.769414in}{2.129591in}}%
\pgfpathlineto{\pgfqpoint{1.774557in}{2.162503in}}%
\pgfpathlineto{\pgfqpoint{1.787417in}{2.162503in}}%
\pgfpathlineto{\pgfqpoint{1.792560in}{2.129591in}}%
\pgfpathlineto{\pgfqpoint{1.795132in}{2.129591in}}%
\pgfpathlineto{\pgfqpoint{1.797704in}{2.146047in}}%
\pgfpathlineto{\pgfqpoint{1.825995in}{2.146047in}}%
\pgfpathlineto{\pgfqpoint{1.828567in}{2.162503in}}%
\pgfpathlineto{\pgfqpoint{1.831138in}{2.146047in}}%
\pgfpathlineto{\pgfqpoint{1.833710in}{2.162503in}}%
\pgfpathlineto{\pgfqpoint{1.836282in}{2.162503in}}%
\pgfpathlineto{\pgfqpoint{1.838854in}{2.195415in}}%
\pgfpathlineto{\pgfqpoint{1.841426in}{2.162503in}}%
\pgfpathlineto{\pgfqpoint{1.851713in}{2.162503in}}%
\pgfpathlineto{\pgfqpoint{1.854285in}{2.178959in}}%
\pgfpathlineto{\pgfqpoint{1.859429in}{2.178959in}}%
\pgfpathlineto{\pgfqpoint{1.862001in}{2.162503in}}%
\pgfpathlineto{\pgfqpoint{1.864573in}{2.178959in}}%
\pgfpathlineto{\pgfqpoint{1.885148in}{2.178959in}}%
\pgfpathlineto{\pgfqpoint{1.887720in}{2.195415in}}%
\pgfpathlineto{\pgfqpoint{1.895435in}{2.195415in}}%
\pgfpathlineto{\pgfqpoint{1.898007in}{2.211871in}}%
\pgfpathlineto{\pgfqpoint{1.900579in}{2.195415in}}%
\pgfpathlineto{\pgfqpoint{1.923726in}{2.195415in}}%
\pgfpathlineto{\pgfqpoint{1.926298in}{2.211871in}}%
\pgfpathlineto{\pgfqpoint{1.928870in}{2.211871in}}%
\pgfpathlineto{\pgfqpoint{1.934013in}{2.178959in}}%
\pgfpathlineto{\pgfqpoint{1.941729in}{2.178959in}}%
\pgfpathlineto{\pgfqpoint{1.946873in}{2.146047in}}%
\pgfpathlineto{\pgfqpoint{1.959732in}{2.146047in}}%
\pgfpathlineto{\pgfqpoint{1.962304in}{2.129591in}}%
\pgfpathlineto{\pgfqpoint{1.977735in}{2.129591in}}%
\pgfpathlineto{\pgfqpoint{1.980307in}{2.113135in}}%
\pgfpathlineto{\pgfqpoint{1.982879in}{2.113135in}}%
\pgfpathlineto{\pgfqpoint{1.985451in}{2.096679in}}%
\pgfpathlineto{\pgfqpoint{2.042032in}{2.096679in}}%
\pgfpathlineto{\pgfqpoint{2.044604in}{2.113135in}}%
\pgfpathlineto{\pgfqpoint{2.052319in}{2.113135in}}%
\pgfpathlineto{\pgfqpoint{2.054891in}{2.096679in}}%
\pgfpathlineto{\pgfqpoint{2.065179in}{2.096679in}}%
\pgfpathlineto{\pgfqpoint{2.067751in}{2.080223in}}%
\pgfpathlineto{\pgfqpoint{2.088325in}{2.080223in}}%
\pgfpathlineto{\pgfqpoint{2.090897in}{2.063767in}}%
\pgfpathlineto{\pgfqpoint{2.093469in}{2.080223in}}%
\pgfpathlineto{\pgfqpoint{2.103757in}{2.080223in}}%
\pgfpathlineto{\pgfqpoint{2.106329in}{2.096679in}}%
\pgfpathlineto{\pgfqpoint{2.108900in}{2.080223in}}%
\pgfpathlineto{\pgfqpoint{2.111472in}{2.080223in}}%
\pgfpathlineto{\pgfqpoint{2.114044in}{2.096679in}}%
\pgfpathlineto{\pgfqpoint{2.116616in}{2.096679in}}%
\pgfpathlineto{\pgfqpoint{2.119188in}{2.113135in}}%
\pgfpathlineto{\pgfqpoint{2.129475in}{2.113135in}}%
\pgfpathlineto{\pgfqpoint{2.132047in}{2.129591in}}%
\pgfpathlineto{\pgfqpoint{2.155194in}{2.129591in}}%
\pgfpathlineto{\pgfqpoint{2.157766in}{2.146047in}}%
\pgfpathlineto{\pgfqpoint{2.193772in}{2.146047in}}%
\pgfpathlineto{\pgfqpoint{2.196344in}{2.129591in}}%
\pgfpathlineto{\pgfqpoint{2.219491in}{2.129591in}}%
\pgfpathlineto{\pgfqpoint{2.222063in}{2.146047in}}%
\pgfpathlineto{\pgfqpoint{2.250353in}{2.146047in}}%
\pgfpathlineto{\pgfqpoint{2.252925in}{2.162503in}}%
\pgfpathlineto{\pgfqpoint{2.304363in}{2.162503in}}%
\pgfpathlineto{\pgfqpoint{2.306934in}{2.178959in}}%
\pgfpathlineto{\pgfqpoint{2.312078in}{2.178959in}}%
\pgfpathlineto{\pgfqpoint{2.314650in}{2.195415in}}%
\pgfpathlineto{\pgfqpoint{2.317222in}{2.178959in}}%
\pgfpathlineto{\pgfqpoint{2.330081in}{2.178959in}}%
\pgfpathlineto{\pgfqpoint{2.332653in}{2.195415in}}%
\pgfpathlineto{\pgfqpoint{2.340369in}{2.195415in}}%
\pgfpathlineto{\pgfqpoint{2.342941in}{2.211871in}}%
\pgfpathlineto{\pgfqpoint{2.360944in}{2.211871in}}%
\pgfpathlineto{\pgfqpoint{2.363516in}{2.195415in}}%
\pgfpathlineto{\pgfqpoint{2.371231in}{2.195415in}}%
\pgfpathlineto{\pgfqpoint{2.373803in}{2.178959in}}%
\pgfpathlineto{\pgfqpoint{2.376375in}{2.195415in}}%
\pgfpathlineto{\pgfqpoint{2.412381in}{2.195415in}}%
\pgfpathlineto{\pgfqpoint{2.414953in}{2.211871in}}%
\pgfpathlineto{\pgfqpoint{2.432956in}{2.211871in}}%
\pgfpathlineto{\pgfqpoint{2.438100in}{2.178959in}}%
\pgfpathlineto{\pgfqpoint{2.602699in}{2.178959in}}%
\pgfpathlineto{\pgfqpoint{2.605271in}{2.162503in}}%
\pgfpathlineto{\pgfqpoint{2.615559in}{2.162503in}}%
\pgfpathlineto{\pgfqpoint{2.618131in}{2.178959in}}%
\pgfpathlineto{\pgfqpoint{2.620703in}{2.178959in}}%
\pgfpathlineto{\pgfqpoint{2.623274in}{2.162503in}}%
\pgfpathlineto{\pgfqpoint{2.625846in}{2.162503in}}%
\pgfpathlineto{\pgfqpoint{2.628418in}{2.146047in}}%
\pgfpathlineto{\pgfqpoint{2.646421in}{2.146047in}}%
\pgfpathlineto{\pgfqpoint{2.646421in}{2.146047in}}%
\pgfusepath{stroke}%
\end{pgfscope}%
\begin{pgfscope}%
\pgfpathrectangle{\pgfqpoint{0.488751in}{1.946106in}}{\pgfqpoint{2.260417in}{1.502439in}}%
\pgfusepath{clip}%
\pgfsetbuttcap%
\pgfsetroundjoin%
\pgfsetlinewidth{0.803000pt}%
\definecolor{currentstroke}{rgb}{0.686275,0.352941,0.313725}%
\pgfsetstrokecolor{currentstroke}%
\pgfsetstrokeopacity{0.270000}%
\pgfsetdash{{2.960000pt}{1.280000pt}}{0.000000pt}%
\pgfpathmoveto{\pgfqpoint{0.591497in}{2.475169in}}%
\pgfpathlineto{\pgfqpoint{0.594069in}{2.458712in}}%
\pgfpathlineto{\pgfqpoint{0.606928in}{2.458712in}}%
\pgfpathlineto{\pgfqpoint{0.609500in}{2.442256in}}%
\pgfpathlineto{\pgfqpoint{0.632647in}{2.442256in}}%
\pgfpathlineto{\pgfqpoint{0.635219in}{2.425800in}}%
\pgfpathlineto{\pgfqpoint{0.642934in}{2.425800in}}%
\pgfpathlineto{\pgfqpoint{0.645506in}{2.442256in}}%
\pgfpathlineto{\pgfqpoint{0.660938in}{2.442256in}}%
\pgfpathlineto{\pgfqpoint{0.663509in}{2.458712in}}%
\pgfpathlineto{\pgfqpoint{0.666081in}{2.491625in}}%
\pgfpathlineto{\pgfqpoint{0.671225in}{2.491625in}}%
\pgfpathlineto{\pgfqpoint{0.673797in}{2.508081in}}%
\pgfpathlineto{\pgfqpoint{0.676369in}{2.508081in}}%
\pgfpathlineto{\pgfqpoint{0.678941in}{2.524537in}}%
\pgfpathlineto{\pgfqpoint{0.686656in}{2.524537in}}%
\pgfpathlineto{\pgfqpoint{0.689228in}{2.508081in}}%
\pgfpathlineto{\pgfqpoint{0.691800in}{2.524537in}}%
\pgfpathlineto{\pgfqpoint{0.696944in}{2.524537in}}%
\pgfpathlineto{\pgfqpoint{0.704659in}{2.573905in}}%
\pgfpathlineto{\pgfqpoint{0.707231in}{2.606817in}}%
\pgfpathlineto{\pgfqpoint{0.709803in}{2.623273in}}%
\pgfpathlineto{\pgfqpoint{0.712375in}{2.656185in}}%
\pgfpathlineto{\pgfqpoint{0.717519in}{2.689097in}}%
\pgfpathlineto{\pgfqpoint{0.720091in}{2.689097in}}%
\pgfpathlineto{\pgfqpoint{0.722662in}{2.705554in}}%
\pgfpathlineto{\pgfqpoint{0.738094in}{2.705554in}}%
\pgfpathlineto{\pgfqpoint{0.740666in}{2.722010in}}%
\pgfpathlineto{\pgfqpoint{0.748381in}{2.722010in}}%
\pgfpathlineto{\pgfqpoint{0.750953in}{2.705554in}}%
\pgfpathlineto{\pgfqpoint{0.753525in}{2.672641in}}%
\pgfpathlineto{\pgfqpoint{0.756097in}{2.689097in}}%
\pgfpathlineto{\pgfqpoint{0.763812in}{2.689097in}}%
\pgfpathlineto{\pgfqpoint{0.766384in}{2.656185in}}%
\pgfpathlineto{\pgfqpoint{0.771528in}{2.656185in}}%
\pgfpathlineto{\pgfqpoint{0.774100in}{2.672641in}}%
\pgfpathlineto{\pgfqpoint{0.776672in}{2.672641in}}%
\pgfpathlineto{\pgfqpoint{0.779244in}{2.656185in}}%
\pgfpathlineto{\pgfqpoint{0.784387in}{2.656185in}}%
\pgfpathlineto{\pgfqpoint{0.786959in}{2.672641in}}%
\pgfpathlineto{\pgfqpoint{0.802390in}{2.672641in}}%
\pgfpathlineto{\pgfqpoint{0.804962in}{2.639729in}}%
\pgfpathlineto{\pgfqpoint{0.815250in}{2.639729in}}%
\pgfpathlineto{\pgfqpoint{0.817822in}{2.606817in}}%
\pgfpathlineto{\pgfqpoint{0.820393in}{2.590361in}}%
\pgfpathlineto{\pgfqpoint{0.822965in}{2.590361in}}%
\pgfpathlineto{\pgfqpoint{0.825537in}{2.573905in}}%
\pgfpathlineto{\pgfqpoint{0.828109in}{2.590361in}}%
\pgfpathlineto{\pgfqpoint{0.833253in}{2.557449in}}%
\pgfpathlineto{\pgfqpoint{0.838397in}{2.557449in}}%
\pgfpathlineto{\pgfqpoint{0.840968in}{2.573905in}}%
\pgfpathlineto{\pgfqpoint{0.843540in}{2.573905in}}%
\pgfpathlineto{\pgfqpoint{0.851256in}{2.623273in}}%
\pgfpathlineto{\pgfqpoint{0.856400in}{2.557449in}}%
\pgfpathlineto{\pgfqpoint{0.858972in}{2.540993in}}%
\pgfpathlineto{\pgfqpoint{0.864115in}{2.540993in}}%
\pgfpathlineto{\pgfqpoint{0.866687in}{2.508081in}}%
\pgfpathlineto{\pgfqpoint{0.869259in}{2.491625in}}%
\pgfpathlineto{\pgfqpoint{0.871831in}{2.491625in}}%
\pgfpathlineto{\pgfqpoint{0.874403in}{2.508081in}}%
\pgfpathlineto{\pgfqpoint{0.876975in}{2.475169in}}%
\pgfpathlineto{\pgfqpoint{0.882118in}{2.475169in}}%
\pgfpathlineto{\pgfqpoint{0.884690in}{2.458712in}}%
\pgfpathlineto{\pgfqpoint{0.887262in}{2.475169in}}%
\pgfpathlineto{\pgfqpoint{0.894978in}{2.475169in}}%
\pgfpathlineto{\pgfqpoint{0.897550in}{2.524537in}}%
\pgfpathlineto{\pgfqpoint{0.900121in}{2.524537in}}%
\pgfpathlineto{\pgfqpoint{0.902693in}{2.491625in}}%
\pgfpathlineto{\pgfqpoint{0.905265in}{2.475169in}}%
\pgfpathlineto{\pgfqpoint{0.912981in}{2.475169in}}%
\pgfpathlineto{\pgfqpoint{0.915553in}{2.458712in}}%
\pgfpathlineto{\pgfqpoint{0.918125in}{2.425800in}}%
\pgfpathlineto{\pgfqpoint{0.923268in}{2.392888in}}%
\pgfpathlineto{\pgfqpoint{0.925840in}{2.425800in}}%
\pgfpathlineto{\pgfqpoint{0.928412in}{2.392888in}}%
\pgfpathlineto{\pgfqpoint{0.930984in}{2.425800in}}%
\pgfpathlineto{\pgfqpoint{0.933556in}{2.409344in}}%
\pgfpathlineto{\pgfqpoint{0.936128in}{2.409344in}}%
\pgfpathlineto{\pgfqpoint{0.938700in}{2.392888in}}%
\pgfpathlineto{\pgfqpoint{0.941271in}{2.409344in}}%
\pgfpathlineto{\pgfqpoint{0.946415in}{2.409344in}}%
\pgfpathlineto{\pgfqpoint{0.951559in}{2.376432in}}%
\pgfpathlineto{\pgfqpoint{0.954131in}{2.409344in}}%
\pgfpathlineto{\pgfqpoint{0.956703in}{2.425800in}}%
\pgfpathlineto{\pgfqpoint{0.959274in}{2.425800in}}%
\pgfpathlineto{\pgfqpoint{0.961846in}{2.442256in}}%
\pgfpathlineto{\pgfqpoint{0.964418in}{2.409344in}}%
\pgfpathlineto{\pgfqpoint{0.966990in}{2.409344in}}%
\pgfpathlineto{\pgfqpoint{0.969562in}{2.392888in}}%
\pgfpathlineto{\pgfqpoint{0.977278in}{2.442256in}}%
\pgfpathlineto{\pgfqpoint{0.979849in}{2.425800in}}%
\pgfpathlineto{\pgfqpoint{0.982421in}{2.442256in}}%
\pgfpathlineto{\pgfqpoint{0.987565in}{2.442256in}}%
\pgfpathlineto{\pgfqpoint{0.990137in}{2.425800in}}%
\pgfpathlineto{\pgfqpoint{0.997853in}{2.425800in}}%
\pgfpathlineto{\pgfqpoint{1.000424in}{2.442256in}}%
\pgfpathlineto{\pgfqpoint{1.008140in}{2.442256in}}%
\pgfpathlineto{\pgfqpoint{1.010712in}{2.425800in}}%
\pgfpathlineto{\pgfqpoint{1.013284in}{2.425800in}}%
\pgfpathlineto{\pgfqpoint{1.015856in}{2.409344in}}%
\pgfpathlineto{\pgfqpoint{1.018427in}{2.343520in}}%
\pgfpathlineto{\pgfqpoint{1.023571in}{2.343520in}}%
\pgfpathlineto{\pgfqpoint{1.026143in}{2.376432in}}%
\pgfpathlineto{\pgfqpoint{1.028715in}{2.392888in}}%
\pgfpathlineto{\pgfqpoint{1.031287in}{2.425800in}}%
\pgfpathlineto{\pgfqpoint{1.033859in}{2.409344in}}%
\pgfpathlineto{\pgfqpoint{1.036431in}{2.409344in}}%
\pgfpathlineto{\pgfqpoint{1.039002in}{2.425800in}}%
\pgfpathlineto{\pgfqpoint{1.041574in}{2.458712in}}%
\pgfpathlineto{\pgfqpoint{1.046718in}{2.425800in}}%
\pgfpathlineto{\pgfqpoint{1.049290in}{2.425800in}}%
\pgfpathlineto{\pgfqpoint{1.051862in}{2.392888in}}%
\pgfpathlineto{\pgfqpoint{1.057006in}{2.392888in}}%
\pgfpathlineto{\pgfqpoint{1.059577in}{2.376432in}}%
\pgfpathlineto{\pgfqpoint{1.095584in}{2.376432in}}%
\pgfpathlineto{\pgfqpoint{1.100727in}{2.409344in}}%
\pgfpathlineto{\pgfqpoint{1.141877in}{2.409344in}}%
\pgfpathlineto{\pgfqpoint{1.144449in}{2.425800in}}%
\pgfpathlineto{\pgfqpoint{1.147021in}{2.425800in}}%
\pgfpathlineto{\pgfqpoint{1.152165in}{2.458712in}}%
\pgfpathlineto{\pgfqpoint{1.157308in}{2.458712in}}%
\pgfpathlineto{\pgfqpoint{1.159880in}{2.491625in}}%
\pgfpathlineto{\pgfqpoint{1.170168in}{2.491625in}}%
\pgfpathlineto{\pgfqpoint{1.172740in}{2.524537in}}%
\pgfpathlineto{\pgfqpoint{1.175312in}{2.508081in}}%
\pgfpathlineto{\pgfqpoint{1.177883in}{2.540993in}}%
\pgfpathlineto{\pgfqpoint{1.180455in}{2.540993in}}%
\pgfpathlineto{\pgfqpoint{1.183027in}{2.524537in}}%
\pgfpathlineto{\pgfqpoint{1.185599in}{2.540993in}}%
\pgfpathlineto{\pgfqpoint{1.195887in}{2.540993in}}%
\pgfpathlineto{\pgfqpoint{1.198458in}{2.557449in}}%
\pgfpathlineto{\pgfqpoint{1.201030in}{2.508081in}}%
\pgfpathlineto{\pgfqpoint{1.203602in}{2.540993in}}%
\pgfpathlineto{\pgfqpoint{1.206174in}{2.540993in}}%
\pgfpathlineto{\pgfqpoint{1.211318in}{2.475169in}}%
\pgfpathlineto{\pgfqpoint{1.213890in}{2.475169in}}%
\pgfpathlineto{\pgfqpoint{1.216461in}{2.442256in}}%
\pgfpathlineto{\pgfqpoint{1.219033in}{2.458712in}}%
\pgfpathlineto{\pgfqpoint{1.221605in}{2.425800in}}%
\pgfpathlineto{\pgfqpoint{1.226749in}{2.392888in}}%
\pgfpathlineto{\pgfqpoint{1.239608in}{2.392888in}}%
\pgfpathlineto{\pgfqpoint{1.242180in}{2.409344in}}%
\pgfpathlineto{\pgfqpoint{1.244752in}{2.392888in}}%
\pgfpathlineto{\pgfqpoint{1.291046in}{2.392888in}}%
\pgfpathlineto{\pgfqpoint{1.293618in}{2.376432in}}%
\pgfpathlineto{\pgfqpoint{1.298761in}{2.376432in}}%
\pgfpathlineto{\pgfqpoint{1.301333in}{2.392888in}}%
\pgfpathlineto{\pgfqpoint{1.409352in}{2.392888in}}%
\pgfpathlineto{\pgfqpoint{1.411924in}{2.409344in}}%
\pgfpathlineto{\pgfqpoint{1.432499in}{2.409344in}}%
\pgfpathlineto{\pgfqpoint{1.435070in}{2.392888in}}%
\pgfpathlineto{\pgfqpoint{1.571380in}{2.392888in}}%
\pgfpathlineto{\pgfqpoint{1.573951in}{2.376432in}}%
\pgfpathlineto{\pgfqpoint{1.676826in}{2.376432in}}%
\pgfpathlineto{\pgfqpoint{1.679398in}{2.392888in}}%
\pgfpathlineto{\pgfqpoint{2.646421in}{2.392888in}}%
\pgfpathlineto{\pgfqpoint{2.646421in}{2.392888in}}%
\pgfusepath{stroke}%
\end{pgfscope}%
\begin{pgfscope}%
\pgfpathrectangle{\pgfqpoint{0.488751in}{1.946106in}}{\pgfqpoint{2.260417in}{1.502439in}}%
\pgfusepath{clip}%
\pgfsetbuttcap%
\pgfsetroundjoin%
\pgfsetlinewidth{0.803000pt}%
\definecolor{currentstroke}{rgb}{0.000000,0.356863,0.509804}%
\pgfsetstrokecolor{currentstroke}%
\pgfsetstrokeopacity{0.270000}%
\pgfsetdash{{2.960000pt}{1.280000pt}}{0.000000pt}%
\pgfpathmoveto{\pgfqpoint{0.591497in}{2.392888in}}%
\pgfpathlineto{\pgfqpoint{0.594069in}{2.392888in}}%
\pgfpathlineto{\pgfqpoint{0.596641in}{2.458712in}}%
\pgfpathlineto{\pgfqpoint{0.601785in}{2.458712in}}%
\pgfpathlineto{\pgfqpoint{0.604356in}{2.392888in}}%
\pgfpathlineto{\pgfqpoint{0.606928in}{2.359976in}}%
\pgfpathlineto{\pgfqpoint{0.609500in}{2.343520in}}%
\pgfpathlineto{\pgfqpoint{0.612072in}{2.343520in}}%
\pgfpathlineto{\pgfqpoint{0.614644in}{2.376432in}}%
\pgfpathlineto{\pgfqpoint{0.617216in}{2.392888in}}%
\pgfpathlineto{\pgfqpoint{0.619788in}{2.376432in}}%
\pgfpathlineto{\pgfqpoint{0.622359in}{2.392888in}}%
\pgfpathlineto{\pgfqpoint{0.624931in}{2.376432in}}%
\pgfpathlineto{\pgfqpoint{0.627503in}{2.409344in}}%
\pgfpathlineto{\pgfqpoint{0.630075in}{2.376432in}}%
\pgfpathlineto{\pgfqpoint{0.632647in}{2.392888in}}%
\pgfpathlineto{\pgfqpoint{0.635219in}{2.376432in}}%
\pgfpathlineto{\pgfqpoint{0.637791in}{2.327064in}}%
\pgfpathlineto{\pgfqpoint{0.658366in}{2.327064in}}%
\pgfpathlineto{\pgfqpoint{0.660938in}{2.343520in}}%
\pgfpathlineto{\pgfqpoint{0.666081in}{2.310608in}}%
\pgfpathlineto{\pgfqpoint{0.684084in}{2.310608in}}%
\pgfpathlineto{\pgfqpoint{0.686656in}{2.327064in}}%
\pgfpathlineto{\pgfqpoint{0.704659in}{2.327064in}}%
\pgfpathlineto{\pgfqpoint{0.707231in}{2.310608in}}%
\pgfpathlineto{\pgfqpoint{0.714947in}{2.310608in}}%
\pgfpathlineto{\pgfqpoint{0.717519in}{2.327064in}}%
\pgfpathlineto{\pgfqpoint{0.727806in}{2.327064in}}%
\pgfpathlineto{\pgfqpoint{0.730378in}{2.343520in}}%
\pgfpathlineto{\pgfqpoint{0.738094in}{2.343520in}}%
\pgfpathlineto{\pgfqpoint{0.740666in}{2.359976in}}%
\pgfpathlineto{\pgfqpoint{0.745809in}{2.327064in}}%
\pgfpathlineto{\pgfqpoint{0.753525in}{2.327064in}}%
\pgfpathlineto{\pgfqpoint{0.756097in}{2.310608in}}%
\pgfpathlineto{\pgfqpoint{0.766384in}{2.310608in}}%
\pgfpathlineto{\pgfqpoint{0.771528in}{2.277696in}}%
\pgfpathlineto{\pgfqpoint{0.774100in}{2.294152in}}%
\pgfpathlineto{\pgfqpoint{0.779244in}{2.294152in}}%
\pgfpathlineto{\pgfqpoint{0.781815in}{2.277696in}}%
\pgfpathlineto{\pgfqpoint{0.792103in}{2.277696in}}%
\pgfpathlineto{\pgfqpoint{0.794675in}{2.261240in}}%
\pgfpathlineto{\pgfqpoint{0.810106in}{2.261240in}}%
\pgfpathlineto{\pgfqpoint{0.812678in}{2.244784in}}%
\pgfpathlineto{\pgfqpoint{0.817822in}{2.244784in}}%
\pgfpathlineto{\pgfqpoint{0.820393in}{2.228328in}}%
\pgfpathlineto{\pgfqpoint{0.830681in}{2.228328in}}%
\pgfpathlineto{\pgfqpoint{0.833253in}{2.244784in}}%
\pgfpathlineto{\pgfqpoint{0.835825in}{2.244784in}}%
\pgfpathlineto{\pgfqpoint{0.838397in}{2.261240in}}%
\pgfpathlineto{\pgfqpoint{0.840968in}{2.261240in}}%
\pgfpathlineto{\pgfqpoint{0.843540in}{2.244784in}}%
\pgfpathlineto{\pgfqpoint{0.848684in}{2.244784in}}%
\pgfpathlineto{\pgfqpoint{0.851256in}{2.228328in}}%
\pgfpathlineto{\pgfqpoint{0.853828in}{2.228328in}}%
\pgfpathlineto{\pgfqpoint{0.856400in}{2.211871in}}%
\pgfpathlineto{\pgfqpoint{0.869259in}{2.211871in}}%
\pgfpathlineto{\pgfqpoint{0.871831in}{2.228328in}}%
\pgfpathlineto{\pgfqpoint{0.874403in}{2.195415in}}%
\pgfpathlineto{\pgfqpoint{0.876975in}{2.178959in}}%
\pgfpathlineto{\pgfqpoint{0.897550in}{2.178959in}}%
\pgfpathlineto{\pgfqpoint{0.900121in}{2.211871in}}%
\pgfpathlineto{\pgfqpoint{0.902693in}{2.228328in}}%
\pgfpathlineto{\pgfqpoint{0.910409in}{2.228328in}}%
\pgfpathlineto{\pgfqpoint{0.912981in}{2.211871in}}%
\pgfpathlineto{\pgfqpoint{0.918125in}{2.211871in}}%
\pgfpathlineto{\pgfqpoint{0.920696in}{2.228328in}}%
\pgfpathlineto{\pgfqpoint{0.923268in}{2.228328in}}%
\pgfpathlineto{\pgfqpoint{0.925840in}{2.211871in}}%
\pgfpathlineto{\pgfqpoint{0.948987in}{2.211871in}}%
\pgfpathlineto{\pgfqpoint{0.951559in}{2.228328in}}%
\pgfpathlineto{\pgfqpoint{0.959274in}{2.228328in}}%
\pgfpathlineto{\pgfqpoint{0.961846in}{2.244784in}}%
\pgfpathlineto{\pgfqpoint{0.966990in}{2.244784in}}%
\pgfpathlineto{\pgfqpoint{0.972134in}{2.211871in}}%
\pgfpathlineto{\pgfqpoint{0.987565in}{2.211871in}}%
\pgfpathlineto{\pgfqpoint{0.990137in}{2.228328in}}%
\pgfpathlineto{\pgfqpoint{0.997853in}{2.228328in}}%
\pgfpathlineto{\pgfqpoint{1.000424in}{2.211871in}}%
\pgfpathlineto{\pgfqpoint{1.026143in}{2.211871in}}%
\pgfpathlineto{\pgfqpoint{1.028715in}{2.228328in}}%
\pgfpathlineto{\pgfqpoint{1.039002in}{2.228328in}}%
\pgfpathlineto{\pgfqpoint{1.041574in}{2.261240in}}%
\pgfpathlineto{\pgfqpoint{1.044146in}{2.261240in}}%
\pgfpathlineto{\pgfqpoint{1.046718in}{2.277696in}}%
\pgfpathlineto{\pgfqpoint{1.057006in}{2.277696in}}%
\pgfpathlineto{\pgfqpoint{1.062149in}{2.310608in}}%
\pgfpathlineto{\pgfqpoint{1.067293in}{2.310608in}}%
\pgfpathlineto{\pgfqpoint{1.069865in}{2.327064in}}%
\pgfpathlineto{\pgfqpoint{1.085296in}{2.327064in}}%
\pgfpathlineto{\pgfqpoint{1.087868in}{2.310608in}}%
\pgfpathlineto{\pgfqpoint{1.090440in}{2.343520in}}%
\pgfpathlineto{\pgfqpoint{1.095584in}{2.310608in}}%
\pgfpathlineto{\pgfqpoint{1.098155in}{2.327064in}}%
\pgfpathlineto{\pgfqpoint{1.111015in}{2.327064in}}%
\pgfpathlineto{\pgfqpoint{1.113587in}{2.310608in}}%
\pgfpathlineto{\pgfqpoint{1.118730in}{2.310608in}}%
\pgfpathlineto{\pgfqpoint{1.121302in}{2.327064in}}%
\pgfpathlineto{\pgfqpoint{1.123874in}{2.327064in}}%
\pgfpathlineto{\pgfqpoint{1.126446in}{2.343520in}}%
\pgfpathlineto{\pgfqpoint{1.129018in}{2.294152in}}%
\pgfpathlineto{\pgfqpoint{1.141877in}{2.294152in}}%
\pgfpathlineto{\pgfqpoint{1.144449in}{2.277696in}}%
\pgfpathlineto{\pgfqpoint{1.162452in}{2.277696in}}%
\pgfpathlineto{\pgfqpoint{1.165024in}{2.261240in}}%
\pgfpathlineto{\pgfqpoint{1.172740in}{2.261240in}}%
\pgfpathlineto{\pgfqpoint{1.177883in}{2.228328in}}%
\pgfpathlineto{\pgfqpoint{1.190743in}{2.228328in}}%
\pgfpathlineto{\pgfqpoint{1.193315in}{2.211871in}}%
\pgfpathlineto{\pgfqpoint{1.208746in}{2.211871in}}%
\pgfpathlineto{\pgfqpoint{1.211318in}{2.228328in}}%
\pgfpathlineto{\pgfqpoint{1.224177in}{2.228328in}}%
\pgfpathlineto{\pgfqpoint{1.226749in}{2.211871in}}%
\pgfpathlineto{\pgfqpoint{1.231893in}{2.211871in}}%
\pgfpathlineto{\pgfqpoint{1.234465in}{2.228328in}}%
\pgfpathlineto{\pgfqpoint{1.239608in}{2.228328in}}%
\pgfpathlineto{\pgfqpoint{1.242180in}{2.211871in}}%
\pgfpathlineto{\pgfqpoint{1.244752in}{2.228328in}}%
\pgfpathlineto{\pgfqpoint{1.257611in}{2.228328in}}%
\pgfpathlineto{\pgfqpoint{1.260183in}{2.244784in}}%
\pgfpathlineto{\pgfqpoint{1.275615in}{2.244784in}}%
\pgfpathlineto{\pgfqpoint{1.278186in}{2.228328in}}%
\pgfpathlineto{\pgfqpoint{1.280758in}{2.228328in}}%
\pgfpathlineto{\pgfqpoint{1.283330in}{2.211871in}}%
\pgfpathlineto{\pgfqpoint{1.301333in}{2.211871in}}%
\pgfpathlineto{\pgfqpoint{1.303905in}{2.228328in}}%
\pgfpathlineto{\pgfqpoint{1.309049in}{2.228328in}}%
\pgfpathlineto{\pgfqpoint{1.311621in}{2.211871in}}%
\pgfpathlineto{\pgfqpoint{1.329624in}{2.211871in}}%
\pgfpathlineto{\pgfqpoint{1.332196in}{2.228328in}}%
\pgfpathlineto{\pgfqpoint{1.347627in}{2.228328in}}%
\pgfpathlineto{\pgfqpoint{1.350199in}{2.211871in}}%
\pgfpathlineto{\pgfqpoint{1.393921in}{2.211871in}}%
\pgfpathlineto{\pgfqpoint{1.396492in}{2.195415in}}%
\pgfpathlineto{\pgfqpoint{1.432499in}{2.195415in}}%
\pgfpathlineto{\pgfqpoint{1.435070in}{2.228328in}}%
\pgfpathlineto{\pgfqpoint{1.437642in}{2.228328in}}%
\pgfpathlineto{\pgfqpoint{1.440214in}{2.211871in}}%
\pgfpathlineto{\pgfqpoint{1.445358in}{2.211871in}}%
\pgfpathlineto{\pgfqpoint{1.447930in}{2.228328in}}%
\pgfpathlineto{\pgfqpoint{1.491652in}{2.228328in}}%
\pgfpathlineto{\pgfqpoint{1.494223in}{2.261240in}}%
\pgfpathlineto{\pgfqpoint{1.499367in}{2.261240in}}%
\pgfpathlineto{\pgfqpoint{1.501939in}{2.277696in}}%
\pgfpathlineto{\pgfqpoint{1.514798in}{2.277696in}}%
\pgfpathlineto{\pgfqpoint{1.517370in}{2.244784in}}%
\pgfpathlineto{\pgfqpoint{1.530230in}{2.244784in}}%
\pgfpathlineto{\pgfqpoint{1.532802in}{2.277696in}}%
\pgfpathlineto{\pgfqpoint{1.535373in}{2.277696in}}%
\pgfpathlineto{\pgfqpoint{1.537945in}{2.261240in}}%
\pgfpathlineto{\pgfqpoint{1.548233in}{2.261240in}}%
\pgfpathlineto{\pgfqpoint{1.550805in}{2.310608in}}%
\pgfpathlineto{\pgfqpoint{1.553376in}{2.310608in}}%
\pgfpathlineto{\pgfqpoint{1.555948in}{2.327064in}}%
\pgfpathlineto{\pgfqpoint{1.563664in}{2.327064in}}%
\pgfpathlineto{\pgfqpoint{1.566236in}{2.343520in}}%
\pgfpathlineto{\pgfqpoint{1.568808in}{2.343520in}}%
\pgfpathlineto{\pgfqpoint{1.571380in}{2.327064in}}%
\pgfpathlineto{\pgfqpoint{1.573951in}{2.327064in}}%
\pgfpathlineto{\pgfqpoint{1.576523in}{2.343520in}}%
\pgfpathlineto{\pgfqpoint{1.591955in}{2.343520in}}%
\pgfpathlineto{\pgfqpoint{1.597098in}{2.376432in}}%
\pgfpathlineto{\pgfqpoint{1.599670in}{2.327064in}}%
\pgfpathlineto{\pgfqpoint{1.602242in}{2.310608in}}%
\pgfpathlineto{\pgfqpoint{1.617673in}{2.310608in}}%
\pgfpathlineto{\pgfqpoint{1.620245in}{2.294152in}}%
\pgfpathlineto{\pgfqpoint{1.638248in}{2.294152in}}%
\pgfpathlineto{\pgfqpoint{1.640820in}{2.277696in}}%
\pgfpathlineto{\pgfqpoint{1.651108in}{2.277696in}}%
\pgfpathlineto{\pgfqpoint{1.653679in}{2.294152in}}%
\pgfpathlineto{\pgfqpoint{1.656251in}{2.277696in}}%
\pgfpathlineto{\pgfqpoint{1.658823in}{2.310608in}}%
\pgfpathlineto{\pgfqpoint{1.661395in}{2.294152in}}%
\pgfpathlineto{\pgfqpoint{1.663967in}{2.327064in}}%
\pgfpathlineto{\pgfqpoint{1.669111in}{2.327064in}}%
\pgfpathlineto{\pgfqpoint{1.671683in}{2.343520in}}%
\pgfpathlineto{\pgfqpoint{1.674254in}{2.343520in}}%
\pgfpathlineto{\pgfqpoint{1.676826in}{2.359976in}}%
\pgfpathlineto{\pgfqpoint{1.681970in}{2.327064in}}%
\pgfpathlineto{\pgfqpoint{1.684542in}{2.343520in}}%
\pgfpathlineto{\pgfqpoint{1.687114in}{2.376432in}}%
\pgfpathlineto{\pgfqpoint{1.692257in}{2.343520in}}%
\pgfpathlineto{\pgfqpoint{1.694829in}{2.343520in}}%
\pgfpathlineto{\pgfqpoint{1.697401in}{2.327064in}}%
\pgfpathlineto{\pgfqpoint{1.712832in}{2.327064in}}%
\pgfpathlineto{\pgfqpoint{1.715404in}{2.310608in}}%
\pgfpathlineto{\pgfqpoint{1.717976in}{2.310608in}}%
\pgfpathlineto{\pgfqpoint{1.723120in}{2.277696in}}%
\pgfpathlineto{\pgfqpoint{1.733407in}{2.277696in}}%
\pgfpathlineto{\pgfqpoint{1.735979in}{2.261240in}}%
\pgfpathlineto{\pgfqpoint{1.738551in}{2.294152in}}%
\pgfpathlineto{\pgfqpoint{1.741123in}{2.294152in}}%
\pgfpathlineto{\pgfqpoint{1.743695in}{2.310608in}}%
\pgfpathlineto{\pgfqpoint{1.751410in}{2.310608in}}%
\pgfpathlineto{\pgfqpoint{1.753982in}{2.294152in}}%
\pgfpathlineto{\pgfqpoint{1.756554in}{2.261240in}}%
\pgfpathlineto{\pgfqpoint{1.759126in}{2.261240in}}%
\pgfpathlineto{\pgfqpoint{1.761698in}{2.244784in}}%
\pgfpathlineto{\pgfqpoint{1.774557in}{2.244784in}}%
\pgfpathlineto{\pgfqpoint{1.777129in}{2.211871in}}%
\pgfpathlineto{\pgfqpoint{1.782273in}{2.211871in}}%
\pgfpathlineto{\pgfqpoint{1.784845in}{2.228328in}}%
\pgfpathlineto{\pgfqpoint{1.795132in}{2.228328in}}%
\pgfpathlineto{\pgfqpoint{1.797704in}{2.211871in}}%
\pgfpathlineto{\pgfqpoint{1.805420in}{2.211871in}}%
\pgfpathlineto{\pgfqpoint{1.807992in}{2.195415in}}%
\pgfpathlineto{\pgfqpoint{1.810563in}{2.195415in}}%
\pgfpathlineto{\pgfqpoint{1.813135in}{2.178959in}}%
\pgfpathlineto{\pgfqpoint{1.818279in}{2.178959in}}%
\pgfpathlineto{\pgfqpoint{1.820851in}{2.162503in}}%
\pgfpathlineto{\pgfqpoint{1.825995in}{2.162503in}}%
\pgfpathlineto{\pgfqpoint{1.828567in}{2.178959in}}%
\pgfpathlineto{\pgfqpoint{1.836282in}{2.178959in}}%
\pgfpathlineto{\pgfqpoint{1.838854in}{2.195415in}}%
\pgfpathlineto{\pgfqpoint{1.843998in}{2.195415in}}%
\pgfpathlineto{\pgfqpoint{1.846570in}{2.178959in}}%
\pgfpathlineto{\pgfqpoint{1.856857in}{2.178959in}}%
\pgfpathlineto{\pgfqpoint{1.859429in}{2.162503in}}%
\pgfpathlineto{\pgfqpoint{1.862001in}{2.162503in}}%
\pgfpathlineto{\pgfqpoint{1.864573in}{2.178959in}}%
\pgfpathlineto{\pgfqpoint{1.882576in}{2.178959in}}%
\pgfpathlineto{\pgfqpoint{1.887720in}{2.211871in}}%
\pgfpathlineto{\pgfqpoint{1.892863in}{2.211871in}}%
\pgfpathlineto{\pgfqpoint{1.895435in}{2.195415in}}%
\pgfpathlineto{\pgfqpoint{1.916010in}{2.195415in}}%
\pgfpathlineto{\pgfqpoint{1.923726in}{2.146047in}}%
\pgfpathlineto{\pgfqpoint{1.952016in}{2.146047in}}%
\pgfpathlineto{\pgfqpoint{1.954588in}{2.162503in}}%
\pgfpathlineto{\pgfqpoint{1.959732in}{2.162503in}}%
\pgfpathlineto{\pgfqpoint{1.964876in}{2.195415in}}%
\pgfpathlineto{\pgfqpoint{1.970019in}{2.195415in}}%
\pgfpathlineto{\pgfqpoint{1.972591in}{2.178959in}}%
\pgfpathlineto{\pgfqpoint{1.977735in}{2.178959in}}%
\pgfpathlineto{\pgfqpoint{1.980307in}{2.162503in}}%
\pgfpathlineto{\pgfqpoint{1.998310in}{2.162503in}}%
\pgfpathlineto{\pgfqpoint{2.000882in}{2.146047in}}%
\pgfpathlineto{\pgfqpoint{2.024029in}{2.146047in}}%
\pgfpathlineto{\pgfqpoint{2.026601in}{2.162503in}}%
\pgfpathlineto{\pgfqpoint{2.031744in}{2.162503in}}%
\pgfpathlineto{\pgfqpoint{2.034316in}{2.146047in}}%
\pgfpathlineto{\pgfqpoint{2.065179in}{2.146047in}}%
\pgfpathlineto{\pgfqpoint{2.067751in}{2.129591in}}%
\pgfpathlineto{\pgfqpoint{2.078038in}{2.129591in}}%
\pgfpathlineto{\pgfqpoint{2.080610in}{2.162503in}}%
\pgfpathlineto{\pgfqpoint{2.083182in}{2.162503in}}%
\pgfpathlineto{\pgfqpoint{2.085754in}{2.178959in}}%
\pgfpathlineto{\pgfqpoint{2.103757in}{2.178959in}}%
\pgfpathlineto{\pgfqpoint{2.106329in}{2.195415in}}%
\pgfpathlineto{\pgfqpoint{2.114044in}{2.195415in}}%
\pgfpathlineto{\pgfqpoint{2.119188in}{2.228328in}}%
\pgfpathlineto{\pgfqpoint{2.121760in}{2.211871in}}%
\pgfpathlineto{\pgfqpoint{2.129475in}{2.211871in}}%
\pgfpathlineto{\pgfqpoint{2.134619in}{2.244784in}}%
\pgfpathlineto{\pgfqpoint{2.139763in}{2.244784in}}%
\pgfpathlineto{\pgfqpoint{2.142335in}{2.261240in}}%
\pgfpathlineto{\pgfqpoint{2.144907in}{2.244784in}}%
\pgfpathlineto{\pgfqpoint{2.150050in}{2.244784in}}%
\pgfpathlineto{\pgfqpoint{2.152622in}{2.228328in}}%
\pgfpathlineto{\pgfqpoint{2.160338in}{2.228328in}}%
\pgfpathlineto{\pgfqpoint{2.162910in}{2.211871in}}%
\pgfpathlineto{\pgfqpoint{2.173197in}{2.211871in}}%
\pgfpathlineto{\pgfqpoint{2.175769in}{2.228328in}}%
\pgfpathlineto{\pgfqpoint{2.183485in}{2.228328in}}%
\pgfpathlineto{\pgfqpoint{2.186057in}{2.211871in}}%
\pgfpathlineto{\pgfqpoint{2.196344in}{2.211871in}}%
\pgfpathlineto{\pgfqpoint{2.201488in}{2.178959in}}%
\pgfpathlineto{\pgfqpoint{2.219491in}{2.178959in}}%
\pgfpathlineto{\pgfqpoint{2.224635in}{2.146047in}}%
\pgfpathlineto{\pgfqpoint{2.229778in}{2.146047in}}%
\pgfpathlineto{\pgfqpoint{2.232350in}{2.162503in}}%
\pgfpathlineto{\pgfqpoint{2.234922in}{2.129591in}}%
\pgfpathlineto{\pgfqpoint{2.288931in}{2.129591in}}%
\pgfpathlineto{\pgfqpoint{2.291503in}{2.146047in}}%
\pgfpathlineto{\pgfqpoint{2.332653in}{2.146047in}}%
\pgfpathlineto{\pgfqpoint{2.335225in}{2.162503in}}%
\pgfpathlineto{\pgfqpoint{2.371231in}{2.162503in}}%
\pgfpathlineto{\pgfqpoint{2.373803in}{2.178959in}}%
\pgfpathlineto{\pgfqpoint{2.391806in}{2.178959in}}%
\pgfpathlineto{\pgfqpoint{2.394378in}{2.162503in}}%
\pgfpathlineto{\pgfqpoint{2.404665in}{2.162503in}}%
\pgfpathlineto{\pgfqpoint{2.407237in}{2.178959in}}%
\pgfpathlineto{\pgfqpoint{2.409809in}{2.178959in}}%
\pgfpathlineto{\pgfqpoint{2.412381in}{2.162503in}}%
\pgfpathlineto{\pgfqpoint{2.414953in}{2.178959in}}%
\pgfpathlineto{\pgfqpoint{2.417525in}{2.178959in}}%
\pgfpathlineto{\pgfqpoint{2.420097in}{2.162503in}}%
\pgfpathlineto{\pgfqpoint{2.425240in}{2.162503in}}%
\pgfpathlineto{\pgfqpoint{2.427812in}{2.146047in}}%
\pgfpathlineto{\pgfqpoint{2.432956in}{2.146047in}}%
\pgfpathlineto{\pgfqpoint{2.435528in}{2.113135in}}%
\pgfpathlineto{\pgfqpoint{2.438100in}{2.113135in}}%
\pgfpathlineto{\pgfqpoint{2.440672in}{2.096679in}}%
\pgfpathlineto{\pgfqpoint{2.445815in}{2.096679in}}%
\pgfpathlineto{\pgfqpoint{2.448387in}{2.080223in}}%
\pgfpathlineto{\pgfqpoint{2.468962in}{2.080223in}}%
\pgfpathlineto{\pgfqpoint{2.471534in}{2.113135in}}%
\pgfpathlineto{\pgfqpoint{2.474106in}{2.080223in}}%
\pgfpathlineto{\pgfqpoint{2.486965in}{2.080223in}}%
\pgfpathlineto{\pgfqpoint{2.489537in}{2.096679in}}%
\pgfpathlineto{\pgfqpoint{2.492109in}{2.096679in}}%
\pgfpathlineto{\pgfqpoint{2.494681in}{2.113135in}}%
\pgfpathlineto{\pgfqpoint{2.510112in}{2.113135in}}%
\pgfpathlineto{\pgfqpoint{2.512684in}{2.096679in}}%
\pgfpathlineto{\pgfqpoint{2.515256in}{2.096679in}}%
\pgfpathlineto{\pgfqpoint{2.517828in}{2.080223in}}%
\pgfpathlineto{\pgfqpoint{2.528115in}{2.080223in}}%
\pgfpathlineto{\pgfqpoint{2.530687in}{2.063767in}}%
\pgfpathlineto{\pgfqpoint{2.533259in}{2.063767in}}%
\pgfpathlineto{\pgfqpoint{2.535831in}{2.080223in}}%
\pgfpathlineto{\pgfqpoint{2.546118in}{2.080223in}}%
\pgfpathlineto{\pgfqpoint{2.548690in}{2.063767in}}%
\pgfpathlineto{\pgfqpoint{2.551262in}{2.080223in}}%
\pgfpathlineto{\pgfqpoint{2.561550in}{2.080223in}}%
\pgfpathlineto{\pgfqpoint{2.564121in}{2.063767in}}%
\pgfpathlineto{\pgfqpoint{2.566693in}{2.080223in}}%
\pgfpathlineto{\pgfqpoint{2.582125in}{2.080223in}}%
\pgfpathlineto{\pgfqpoint{2.584696in}{2.063767in}}%
\pgfpathlineto{\pgfqpoint{2.587268in}{2.080223in}}%
\pgfpathlineto{\pgfqpoint{2.592412in}{2.080223in}}%
\pgfpathlineto{\pgfqpoint{2.594984in}{2.063767in}}%
\pgfpathlineto{\pgfqpoint{2.597556in}{2.080223in}}%
\pgfpathlineto{\pgfqpoint{2.612987in}{2.080223in}}%
\pgfpathlineto{\pgfqpoint{2.615559in}{2.063767in}}%
\pgfpathlineto{\pgfqpoint{2.618131in}{2.063767in}}%
\pgfpathlineto{\pgfqpoint{2.620703in}{2.080223in}}%
\pgfpathlineto{\pgfqpoint{2.625846in}{2.080223in}}%
\pgfpathlineto{\pgfqpoint{2.628418in}{2.063767in}}%
\pgfpathlineto{\pgfqpoint{2.630990in}{2.080223in}}%
\pgfpathlineto{\pgfqpoint{2.646421in}{2.080223in}}%
\pgfpathlineto{\pgfqpoint{2.646421in}{2.080223in}}%
\pgfusepath{stroke}%
\end{pgfscope}%
\begin{pgfscope}%
\pgfpathrectangle{\pgfqpoint{0.488751in}{1.946106in}}{\pgfqpoint{2.260417in}{1.502439in}}%
\pgfusepath{clip}%
\pgfsetbuttcap%
\pgfsetroundjoin%
\pgfsetlinewidth{0.803000pt}%
\definecolor{currentstroke}{rgb}{0.490196,0.588235,0.431373}%
\pgfsetstrokecolor{currentstroke}%
\pgfsetstrokeopacity{0.270000}%
\pgfsetdash{{2.960000pt}{1.280000pt}}{0.000000pt}%
\pgfpathmoveto{\pgfqpoint{0.591497in}{3.051131in}}%
\pgfpathlineto{\pgfqpoint{0.594069in}{3.116955in}}%
\pgfpathlineto{\pgfqpoint{0.596641in}{3.149867in}}%
\pgfpathlineto{\pgfqpoint{0.599213in}{3.133411in}}%
\pgfpathlineto{\pgfqpoint{0.601785in}{3.182779in}}%
\pgfpathlineto{\pgfqpoint{0.604356in}{3.199236in}}%
\pgfpathlineto{\pgfqpoint{0.606928in}{3.149867in}}%
\pgfpathlineto{\pgfqpoint{0.609500in}{3.133411in}}%
\pgfpathlineto{\pgfqpoint{0.612072in}{3.133411in}}%
\pgfpathlineto{\pgfqpoint{0.617216in}{3.215692in}}%
\pgfpathlineto{\pgfqpoint{0.624931in}{3.314428in}}%
\pgfpathlineto{\pgfqpoint{0.627503in}{3.380252in}}%
\pgfpathlineto{\pgfqpoint{0.630075in}{3.363796in}}%
\pgfpathlineto{\pgfqpoint{0.632647in}{3.380252in}}%
\pgfpathlineto{\pgfqpoint{0.637791in}{3.347340in}}%
\pgfpathlineto{\pgfqpoint{0.640363in}{3.363796in}}%
\pgfpathlineto{\pgfqpoint{0.648078in}{3.363796in}}%
\pgfpathlineto{\pgfqpoint{0.650650in}{3.330884in}}%
\pgfpathlineto{\pgfqpoint{0.653222in}{3.314428in}}%
\pgfpathlineto{\pgfqpoint{0.658366in}{3.314428in}}%
\pgfpathlineto{\pgfqpoint{0.660938in}{3.330884in}}%
\pgfpathlineto{\pgfqpoint{0.663509in}{3.314428in}}%
\pgfpathlineto{\pgfqpoint{0.676369in}{3.314428in}}%
\pgfpathlineto{\pgfqpoint{0.678941in}{3.297972in}}%
\pgfpathlineto{\pgfqpoint{0.681513in}{3.314428in}}%
\pgfpathlineto{\pgfqpoint{0.684084in}{3.265060in}}%
\pgfpathlineto{\pgfqpoint{0.686656in}{3.248604in}}%
\pgfpathlineto{\pgfqpoint{0.689228in}{3.215692in}}%
\pgfpathlineto{\pgfqpoint{0.691800in}{3.199236in}}%
\pgfpathlineto{\pgfqpoint{0.694372in}{3.166323in}}%
\pgfpathlineto{\pgfqpoint{0.696944in}{3.149867in}}%
\pgfpathlineto{\pgfqpoint{0.699516in}{3.100499in}}%
\pgfpathlineto{\pgfqpoint{0.702087in}{3.100499in}}%
\pgfpathlineto{\pgfqpoint{0.704659in}{3.067587in}}%
\pgfpathlineto{\pgfqpoint{0.709803in}{3.034675in}}%
\pgfpathlineto{\pgfqpoint{0.712375in}{3.051131in}}%
\pgfpathlineto{\pgfqpoint{0.714947in}{3.051131in}}%
\pgfpathlineto{\pgfqpoint{0.717519in}{3.034675in}}%
\pgfpathlineto{\pgfqpoint{0.720091in}{3.051131in}}%
\pgfpathlineto{\pgfqpoint{0.722662in}{3.034675in}}%
\pgfpathlineto{\pgfqpoint{0.725234in}{3.034675in}}%
\pgfpathlineto{\pgfqpoint{0.727806in}{3.067587in}}%
\pgfpathlineto{\pgfqpoint{0.730378in}{3.067587in}}%
\pgfpathlineto{\pgfqpoint{0.732950in}{3.084043in}}%
\pgfpathlineto{\pgfqpoint{0.735522in}{3.084043in}}%
\pgfpathlineto{\pgfqpoint{0.738094in}{3.100499in}}%
\pgfpathlineto{\pgfqpoint{0.740666in}{3.100499in}}%
\pgfpathlineto{\pgfqpoint{0.743237in}{3.084043in}}%
\pgfpathlineto{\pgfqpoint{0.745809in}{3.100499in}}%
\pgfpathlineto{\pgfqpoint{0.758669in}{3.100499in}}%
\pgfpathlineto{\pgfqpoint{0.761240in}{3.116955in}}%
\pgfpathlineto{\pgfqpoint{0.763812in}{3.116955in}}%
\pgfpathlineto{\pgfqpoint{0.766384in}{3.100499in}}%
\pgfpathlineto{\pgfqpoint{0.768956in}{3.100499in}}%
\pgfpathlineto{\pgfqpoint{0.771528in}{3.084043in}}%
\pgfpathlineto{\pgfqpoint{0.774100in}{3.084043in}}%
\pgfpathlineto{\pgfqpoint{0.776672in}{3.100499in}}%
\pgfpathlineto{\pgfqpoint{0.779244in}{3.133411in}}%
\pgfpathlineto{\pgfqpoint{0.781815in}{3.116955in}}%
\pgfpathlineto{\pgfqpoint{0.784387in}{3.133411in}}%
\pgfpathlineto{\pgfqpoint{0.786959in}{3.133411in}}%
\pgfpathlineto{\pgfqpoint{0.789531in}{3.100499in}}%
\pgfpathlineto{\pgfqpoint{0.794675in}{3.100499in}}%
\pgfpathlineto{\pgfqpoint{0.799819in}{3.067587in}}%
\pgfpathlineto{\pgfqpoint{0.812678in}{3.149867in}}%
\pgfpathlineto{\pgfqpoint{0.815250in}{3.133411in}}%
\pgfpathlineto{\pgfqpoint{0.817822in}{3.149867in}}%
\pgfpathlineto{\pgfqpoint{0.820393in}{3.116955in}}%
\pgfpathlineto{\pgfqpoint{0.825537in}{3.084043in}}%
\pgfpathlineto{\pgfqpoint{0.828109in}{3.051131in}}%
\pgfpathlineto{\pgfqpoint{0.830681in}{3.034675in}}%
\pgfpathlineto{\pgfqpoint{0.835825in}{3.067587in}}%
\pgfpathlineto{\pgfqpoint{0.840968in}{3.067587in}}%
\pgfpathlineto{\pgfqpoint{0.843540in}{3.051131in}}%
\pgfpathlineto{\pgfqpoint{0.846112in}{3.051131in}}%
\pgfpathlineto{\pgfqpoint{0.848684in}{3.018219in}}%
\pgfpathlineto{\pgfqpoint{0.861543in}{3.100499in}}%
\pgfpathlineto{\pgfqpoint{0.864115in}{3.100499in}}%
\pgfpathlineto{\pgfqpoint{0.866687in}{3.116955in}}%
\pgfpathlineto{\pgfqpoint{0.871831in}{3.116955in}}%
\pgfpathlineto{\pgfqpoint{0.874403in}{3.100499in}}%
\pgfpathlineto{\pgfqpoint{0.876975in}{3.116955in}}%
\pgfpathlineto{\pgfqpoint{0.879547in}{3.116955in}}%
\pgfpathlineto{\pgfqpoint{0.882118in}{3.100499in}}%
\pgfpathlineto{\pgfqpoint{0.887262in}{3.100499in}}%
\pgfpathlineto{\pgfqpoint{0.889834in}{3.116955in}}%
\pgfpathlineto{\pgfqpoint{0.892406in}{3.100499in}}%
\pgfpathlineto{\pgfqpoint{0.894978in}{3.116955in}}%
\pgfpathlineto{\pgfqpoint{0.900121in}{3.116955in}}%
\pgfpathlineto{\pgfqpoint{0.905265in}{3.084043in}}%
\pgfpathlineto{\pgfqpoint{0.907837in}{3.100499in}}%
\pgfpathlineto{\pgfqpoint{0.912981in}{3.100499in}}%
\pgfpathlineto{\pgfqpoint{0.915553in}{3.116955in}}%
\pgfpathlineto{\pgfqpoint{0.923268in}{3.116955in}}%
\pgfpathlineto{\pgfqpoint{0.925840in}{3.100499in}}%
\pgfpathlineto{\pgfqpoint{0.928412in}{3.133411in}}%
\pgfpathlineto{\pgfqpoint{0.930984in}{3.116955in}}%
\pgfpathlineto{\pgfqpoint{0.933556in}{3.166323in}}%
\pgfpathlineto{\pgfqpoint{0.936128in}{3.149867in}}%
\pgfpathlineto{\pgfqpoint{0.938700in}{3.149867in}}%
\pgfpathlineto{\pgfqpoint{0.941271in}{3.133411in}}%
\pgfpathlineto{\pgfqpoint{0.943843in}{3.133411in}}%
\pgfpathlineto{\pgfqpoint{0.946415in}{3.084043in}}%
\pgfpathlineto{\pgfqpoint{0.948987in}{3.084043in}}%
\pgfpathlineto{\pgfqpoint{0.954131in}{3.116955in}}%
\pgfpathlineto{\pgfqpoint{0.956703in}{3.100499in}}%
\pgfpathlineto{\pgfqpoint{0.961846in}{3.100499in}}%
\pgfpathlineto{\pgfqpoint{0.964418in}{3.116955in}}%
\pgfpathlineto{\pgfqpoint{0.966990in}{3.084043in}}%
\pgfpathlineto{\pgfqpoint{0.969562in}{3.084043in}}%
\pgfpathlineto{\pgfqpoint{0.972134in}{3.100499in}}%
\pgfpathlineto{\pgfqpoint{0.974706in}{3.100499in}}%
\pgfpathlineto{\pgfqpoint{0.977278in}{3.084043in}}%
\pgfpathlineto{\pgfqpoint{0.979849in}{3.100499in}}%
\pgfpathlineto{\pgfqpoint{0.987565in}{3.100499in}}%
\pgfpathlineto{\pgfqpoint{0.990137in}{3.116955in}}%
\pgfpathlineto{\pgfqpoint{0.992709in}{3.100499in}}%
\pgfpathlineto{\pgfqpoint{1.002996in}{3.100499in}}%
\pgfpathlineto{\pgfqpoint{1.005568in}{3.116955in}}%
\pgfpathlineto{\pgfqpoint{1.008140in}{3.116955in}}%
\pgfpathlineto{\pgfqpoint{1.010712in}{3.084043in}}%
\pgfpathlineto{\pgfqpoint{1.013284in}{3.084043in}}%
\pgfpathlineto{\pgfqpoint{1.015856in}{3.100499in}}%
\pgfpathlineto{\pgfqpoint{1.018427in}{3.133411in}}%
\pgfpathlineto{\pgfqpoint{1.020999in}{3.116955in}}%
\pgfpathlineto{\pgfqpoint{1.023571in}{3.084043in}}%
\pgfpathlineto{\pgfqpoint{1.026143in}{3.084043in}}%
\pgfpathlineto{\pgfqpoint{1.028715in}{3.100499in}}%
\pgfpathlineto{\pgfqpoint{1.031287in}{3.100499in}}%
\pgfpathlineto{\pgfqpoint{1.033859in}{3.116955in}}%
\pgfpathlineto{\pgfqpoint{1.039002in}{3.116955in}}%
\pgfpathlineto{\pgfqpoint{1.041574in}{3.084043in}}%
\pgfpathlineto{\pgfqpoint{1.044146in}{3.100499in}}%
\pgfpathlineto{\pgfqpoint{1.049290in}{3.100499in}}%
\pgfpathlineto{\pgfqpoint{1.057006in}{3.149867in}}%
\pgfpathlineto{\pgfqpoint{1.059577in}{3.149867in}}%
\pgfpathlineto{\pgfqpoint{1.062149in}{3.116955in}}%
\pgfpathlineto{\pgfqpoint{1.064721in}{3.100499in}}%
\pgfpathlineto{\pgfqpoint{1.067293in}{3.133411in}}%
\pgfpathlineto{\pgfqpoint{1.069865in}{3.149867in}}%
\pgfpathlineto{\pgfqpoint{1.072437in}{3.149867in}}%
\pgfpathlineto{\pgfqpoint{1.080152in}{3.100499in}}%
\pgfpathlineto{\pgfqpoint{1.082724in}{3.116955in}}%
\pgfpathlineto{\pgfqpoint{1.085296in}{3.100499in}}%
\pgfpathlineto{\pgfqpoint{1.087868in}{3.100499in}}%
\pgfpathlineto{\pgfqpoint{1.090440in}{3.133411in}}%
\pgfpathlineto{\pgfqpoint{1.093012in}{3.149867in}}%
\pgfpathlineto{\pgfqpoint{1.095584in}{3.116955in}}%
\pgfpathlineto{\pgfqpoint{1.098155in}{3.100499in}}%
\pgfpathlineto{\pgfqpoint{1.108443in}{3.100499in}}%
\pgfpathlineto{\pgfqpoint{1.111015in}{3.116955in}}%
\pgfpathlineto{\pgfqpoint{1.113587in}{3.100499in}}%
\pgfpathlineto{\pgfqpoint{1.116159in}{3.116955in}}%
\pgfpathlineto{\pgfqpoint{1.118730in}{3.116955in}}%
\pgfpathlineto{\pgfqpoint{1.121302in}{3.100499in}}%
\pgfpathlineto{\pgfqpoint{1.126446in}{3.100499in}}%
\pgfpathlineto{\pgfqpoint{1.129018in}{3.116955in}}%
\pgfpathlineto{\pgfqpoint{1.131590in}{3.100499in}}%
\pgfpathlineto{\pgfqpoint{1.134162in}{3.149867in}}%
\pgfpathlineto{\pgfqpoint{1.136734in}{3.149867in}}%
\pgfpathlineto{\pgfqpoint{1.139305in}{3.166323in}}%
\pgfpathlineto{\pgfqpoint{1.144449in}{3.133411in}}%
\pgfpathlineto{\pgfqpoint{1.147021in}{3.133411in}}%
\pgfpathlineto{\pgfqpoint{1.149593in}{3.100499in}}%
\pgfpathlineto{\pgfqpoint{1.157308in}{3.100499in}}%
\pgfpathlineto{\pgfqpoint{1.159880in}{3.084043in}}%
\pgfpathlineto{\pgfqpoint{1.167596in}{3.084043in}}%
\pgfpathlineto{\pgfqpoint{1.170168in}{3.067587in}}%
\pgfpathlineto{\pgfqpoint{1.172740in}{3.067587in}}%
\pgfpathlineto{\pgfqpoint{1.175312in}{3.051131in}}%
\pgfpathlineto{\pgfqpoint{1.177883in}{3.051131in}}%
\pgfpathlineto{\pgfqpoint{1.180455in}{3.067587in}}%
\pgfpathlineto{\pgfqpoint{1.183027in}{3.067587in}}%
\pgfpathlineto{\pgfqpoint{1.185599in}{3.084043in}}%
\pgfpathlineto{\pgfqpoint{1.198458in}{3.084043in}}%
\pgfpathlineto{\pgfqpoint{1.201030in}{3.133411in}}%
\pgfpathlineto{\pgfqpoint{1.203602in}{3.100499in}}%
\pgfpathlineto{\pgfqpoint{1.213890in}{3.100499in}}%
\pgfpathlineto{\pgfqpoint{1.216461in}{3.116955in}}%
\pgfpathlineto{\pgfqpoint{1.219033in}{3.100499in}}%
\pgfpathlineto{\pgfqpoint{1.221605in}{3.166323in}}%
\pgfpathlineto{\pgfqpoint{1.226749in}{3.166323in}}%
\pgfpathlineto{\pgfqpoint{1.229321in}{3.149867in}}%
\pgfpathlineto{\pgfqpoint{1.231893in}{3.116955in}}%
\pgfpathlineto{\pgfqpoint{1.234465in}{3.116955in}}%
\pgfpathlineto{\pgfqpoint{1.237036in}{3.133411in}}%
\pgfpathlineto{\pgfqpoint{1.239608in}{3.084043in}}%
\pgfpathlineto{\pgfqpoint{1.242180in}{3.084043in}}%
\pgfpathlineto{\pgfqpoint{1.244752in}{3.100499in}}%
\pgfpathlineto{\pgfqpoint{1.249896in}{3.100499in}}%
\pgfpathlineto{\pgfqpoint{1.252468in}{3.133411in}}%
\pgfpathlineto{\pgfqpoint{1.255040in}{3.133411in}}%
\pgfpathlineto{\pgfqpoint{1.260183in}{3.100499in}}%
\pgfpathlineto{\pgfqpoint{1.265327in}{3.133411in}}%
\pgfpathlineto{\pgfqpoint{1.267899in}{3.116955in}}%
\pgfpathlineto{\pgfqpoint{1.270471in}{3.116955in}}%
\pgfpathlineto{\pgfqpoint{1.273043in}{3.100499in}}%
\pgfpathlineto{\pgfqpoint{1.278186in}{3.100499in}}%
\pgfpathlineto{\pgfqpoint{1.285902in}{3.051131in}}%
\pgfpathlineto{\pgfqpoint{1.288474in}{3.067587in}}%
\pgfpathlineto{\pgfqpoint{1.291046in}{3.100499in}}%
\pgfpathlineto{\pgfqpoint{1.293618in}{3.116955in}}%
\pgfpathlineto{\pgfqpoint{1.296189in}{3.100499in}}%
\pgfpathlineto{\pgfqpoint{1.298761in}{3.100499in}}%
\pgfpathlineto{\pgfqpoint{1.301333in}{3.116955in}}%
\pgfpathlineto{\pgfqpoint{1.303905in}{3.084043in}}%
\pgfpathlineto{\pgfqpoint{1.309049in}{3.084043in}}%
\pgfpathlineto{\pgfqpoint{1.311621in}{3.100499in}}%
\pgfpathlineto{\pgfqpoint{1.321908in}{3.100499in}}%
\pgfpathlineto{\pgfqpoint{1.324480in}{3.084043in}}%
\pgfpathlineto{\pgfqpoint{1.327052in}{3.100499in}}%
\pgfpathlineto{\pgfqpoint{1.337339in}{3.100499in}}%
\pgfpathlineto{\pgfqpoint{1.339911in}{3.116955in}}%
\pgfpathlineto{\pgfqpoint{1.342483in}{3.100499in}}%
\pgfpathlineto{\pgfqpoint{1.345055in}{3.100499in}}%
\pgfpathlineto{\pgfqpoint{1.347627in}{3.084043in}}%
\pgfpathlineto{\pgfqpoint{1.350199in}{3.084043in}}%
\pgfpathlineto{\pgfqpoint{1.355342in}{3.116955in}}%
\pgfpathlineto{\pgfqpoint{1.357914in}{3.100499in}}%
\pgfpathlineto{\pgfqpoint{1.360486in}{3.116955in}}%
\pgfpathlineto{\pgfqpoint{1.363058in}{3.100499in}}%
\pgfpathlineto{\pgfqpoint{1.365630in}{3.116955in}}%
\pgfpathlineto{\pgfqpoint{1.370774in}{3.116955in}}%
\pgfpathlineto{\pgfqpoint{1.373346in}{3.100499in}}%
\pgfpathlineto{\pgfqpoint{1.375917in}{3.116955in}}%
\pgfpathlineto{\pgfqpoint{1.378489in}{3.084043in}}%
\pgfpathlineto{\pgfqpoint{1.381061in}{3.100499in}}%
\pgfpathlineto{\pgfqpoint{1.386205in}{3.100499in}}%
\pgfpathlineto{\pgfqpoint{1.388777in}{3.116955in}}%
\pgfpathlineto{\pgfqpoint{1.391349in}{3.100499in}}%
\pgfpathlineto{\pgfqpoint{1.393921in}{3.100499in}}%
\pgfpathlineto{\pgfqpoint{1.396492in}{3.084043in}}%
\pgfpathlineto{\pgfqpoint{1.399064in}{3.084043in}}%
\pgfpathlineto{\pgfqpoint{1.401636in}{3.100499in}}%
\pgfpathlineto{\pgfqpoint{1.404208in}{3.084043in}}%
\pgfpathlineto{\pgfqpoint{1.414495in}{3.084043in}}%
\pgfpathlineto{\pgfqpoint{1.417067in}{3.067587in}}%
\pgfpathlineto{\pgfqpoint{1.422211in}{3.067587in}}%
\pgfpathlineto{\pgfqpoint{1.424783in}{3.051131in}}%
\pgfpathlineto{\pgfqpoint{1.429927in}{3.051131in}}%
\pgfpathlineto{\pgfqpoint{1.432499in}{3.100499in}}%
\pgfpathlineto{\pgfqpoint{1.447930in}{3.100499in}}%
\pgfpathlineto{\pgfqpoint{1.450502in}{3.084043in}}%
\pgfpathlineto{\pgfqpoint{1.453074in}{3.084043in}}%
\pgfpathlineto{\pgfqpoint{1.455645in}{3.100499in}}%
\pgfpathlineto{\pgfqpoint{1.468505in}{3.100499in}}%
\pgfpathlineto{\pgfqpoint{1.471077in}{3.116955in}}%
\pgfpathlineto{\pgfqpoint{1.473649in}{3.100499in}}%
\pgfpathlineto{\pgfqpoint{1.481364in}{3.100499in}}%
\pgfpathlineto{\pgfqpoint{1.483936in}{3.116955in}}%
\pgfpathlineto{\pgfqpoint{1.486508in}{3.116955in}}%
\pgfpathlineto{\pgfqpoint{1.489080in}{3.100499in}}%
\pgfpathlineto{\pgfqpoint{1.496795in}{3.100499in}}%
\pgfpathlineto{\pgfqpoint{1.499367in}{3.084043in}}%
\pgfpathlineto{\pgfqpoint{1.501939in}{3.084043in}}%
\pgfpathlineto{\pgfqpoint{1.504511in}{3.100499in}}%
\pgfpathlineto{\pgfqpoint{1.509655in}{3.100499in}}%
\pgfpathlineto{\pgfqpoint{1.512227in}{3.084043in}}%
\pgfpathlineto{\pgfqpoint{1.514798in}{3.084043in}}%
\pgfpathlineto{\pgfqpoint{1.517370in}{3.067587in}}%
\pgfpathlineto{\pgfqpoint{1.519942in}{3.067587in}}%
\pgfpathlineto{\pgfqpoint{1.522514in}{3.084043in}}%
\pgfpathlineto{\pgfqpoint{1.525086in}{3.067587in}}%
\pgfpathlineto{\pgfqpoint{1.527658in}{3.067587in}}%
\pgfpathlineto{\pgfqpoint{1.532802in}{3.133411in}}%
\pgfpathlineto{\pgfqpoint{1.537945in}{3.100499in}}%
\pgfpathlineto{\pgfqpoint{1.543089in}{3.100499in}}%
\pgfpathlineto{\pgfqpoint{1.548233in}{3.067587in}}%
\pgfpathlineto{\pgfqpoint{1.555948in}{3.067587in}}%
\pgfpathlineto{\pgfqpoint{1.558520in}{3.051131in}}%
\pgfpathlineto{\pgfqpoint{1.561092in}{3.067587in}}%
\pgfpathlineto{\pgfqpoint{1.563664in}{3.067587in}}%
\pgfpathlineto{\pgfqpoint{1.566236in}{3.084043in}}%
\pgfpathlineto{\pgfqpoint{1.573951in}{3.084043in}}%
\pgfpathlineto{\pgfqpoint{1.576523in}{3.100499in}}%
\pgfpathlineto{\pgfqpoint{1.591955in}{3.100499in}}%
\pgfpathlineto{\pgfqpoint{1.594526in}{3.116955in}}%
\pgfpathlineto{\pgfqpoint{1.597098in}{3.100499in}}%
\pgfpathlineto{\pgfqpoint{1.615101in}{3.100499in}}%
\pgfpathlineto{\pgfqpoint{1.617673in}{3.116955in}}%
\pgfpathlineto{\pgfqpoint{1.620245in}{3.100499in}}%
\pgfpathlineto{\pgfqpoint{1.643392in}{3.100499in}}%
\pgfpathlineto{\pgfqpoint{1.645964in}{3.116955in}}%
\pgfpathlineto{\pgfqpoint{1.648536in}{3.084043in}}%
\pgfpathlineto{\pgfqpoint{1.656251in}{3.084043in}}%
\pgfpathlineto{\pgfqpoint{1.658823in}{3.100499in}}%
\pgfpathlineto{\pgfqpoint{1.661395in}{3.084043in}}%
\pgfpathlineto{\pgfqpoint{1.674254in}{3.084043in}}%
\pgfpathlineto{\pgfqpoint{1.676826in}{3.116955in}}%
\pgfpathlineto{\pgfqpoint{1.679398in}{3.100499in}}%
\pgfpathlineto{\pgfqpoint{1.681970in}{3.100499in}}%
\pgfpathlineto{\pgfqpoint{1.684542in}{3.116955in}}%
\pgfpathlineto{\pgfqpoint{1.687114in}{3.116955in}}%
\pgfpathlineto{\pgfqpoint{1.689686in}{3.100499in}}%
\pgfpathlineto{\pgfqpoint{1.702545in}{3.100499in}}%
\pgfpathlineto{\pgfqpoint{1.705117in}{3.133411in}}%
\pgfpathlineto{\pgfqpoint{1.707689in}{3.100499in}}%
\pgfpathlineto{\pgfqpoint{1.771985in}{3.100499in}}%
\pgfpathlineto{\pgfqpoint{1.774557in}{3.084043in}}%
\pgfpathlineto{\pgfqpoint{1.777129in}{3.084043in}}%
\pgfpathlineto{\pgfqpoint{1.779701in}{3.100499in}}%
\pgfpathlineto{\pgfqpoint{1.795132in}{3.100499in}}%
\pgfpathlineto{\pgfqpoint{1.797704in}{3.084043in}}%
\pgfpathlineto{\pgfqpoint{1.800276in}{3.084043in}}%
\pgfpathlineto{\pgfqpoint{1.802848in}{3.100499in}}%
\pgfpathlineto{\pgfqpoint{1.820851in}{3.100499in}}%
\pgfpathlineto{\pgfqpoint{1.823423in}{3.116955in}}%
\pgfpathlineto{\pgfqpoint{1.825995in}{3.084043in}}%
\pgfpathlineto{\pgfqpoint{1.838854in}{3.084043in}}%
\pgfpathlineto{\pgfqpoint{1.841426in}{3.100499in}}%
\pgfpathlineto{\pgfqpoint{1.843998in}{3.100499in}}%
\pgfpathlineto{\pgfqpoint{1.846570in}{3.116955in}}%
\pgfpathlineto{\pgfqpoint{1.849142in}{3.100499in}}%
\pgfpathlineto{\pgfqpoint{1.851713in}{3.133411in}}%
\pgfpathlineto{\pgfqpoint{1.854285in}{3.133411in}}%
\pgfpathlineto{\pgfqpoint{1.856857in}{3.100499in}}%
\pgfpathlineto{\pgfqpoint{1.885148in}{3.100499in}}%
\pgfpathlineto{\pgfqpoint{1.887720in}{3.116955in}}%
\pgfpathlineto{\pgfqpoint{1.890291in}{3.100499in}}%
\pgfpathlineto{\pgfqpoint{1.910866in}{3.100499in}}%
\pgfpathlineto{\pgfqpoint{1.913438in}{3.116955in}}%
\pgfpathlineto{\pgfqpoint{1.916010in}{3.100499in}}%
\pgfpathlineto{\pgfqpoint{1.949444in}{3.100499in}}%
\pgfpathlineto{\pgfqpoint{1.952016in}{3.084043in}}%
\pgfpathlineto{\pgfqpoint{1.957160in}{3.084043in}}%
\pgfpathlineto{\pgfqpoint{1.959732in}{3.067587in}}%
\pgfpathlineto{\pgfqpoint{1.970019in}{3.067587in}}%
\pgfpathlineto{\pgfqpoint{1.972591in}{3.051131in}}%
\pgfpathlineto{\pgfqpoint{1.998310in}{3.051131in}}%
\pgfpathlineto{\pgfqpoint{2.000882in}{3.034675in}}%
\pgfpathlineto{\pgfqpoint{2.021457in}{3.034675in}}%
\pgfpathlineto{\pgfqpoint{2.024029in}{3.051131in}}%
\pgfpathlineto{\pgfqpoint{2.031744in}{3.051131in}}%
\pgfpathlineto{\pgfqpoint{2.034316in}{3.034675in}}%
\pgfpathlineto{\pgfqpoint{2.042032in}{3.034675in}}%
\pgfpathlineto{\pgfqpoint{2.044604in}{3.051131in}}%
\pgfpathlineto{\pgfqpoint{2.078038in}{3.051131in}}%
\pgfpathlineto{\pgfqpoint{2.080610in}{3.067587in}}%
\pgfpathlineto{\pgfqpoint{2.129475in}{3.067587in}}%
\pgfpathlineto{\pgfqpoint{2.132047in}{3.084043in}}%
\pgfpathlineto{\pgfqpoint{2.162910in}{3.084043in}}%
\pgfpathlineto{\pgfqpoint{2.168053in}{3.116955in}}%
\pgfpathlineto{\pgfqpoint{2.170625in}{3.100499in}}%
\pgfpathlineto{\pgfqpoint{2.188628in}{3.100499in}}%
\pgfpathlineto{\pgfqpoint{2.191200in}{3.084043in}}%
\pgfpathlineto{\pgfqpoint{2.193772in}{3.100499in}}%
\pgfpathlineto{\pgfqpoint{2.196344in}{3.100499in}}%
\pgfpathlineto{\pgfqpoint{2.198916in}{3.084043in}}%
\pgfpathlineto{\pgfqpoint{2.201488in}{3.100499in}}%
\pgfpathlineto{\pgfqpoint{2.204060in}{3.100499in}}%
\pgfpathlineto{\pgfqpoint{2.206631in}{3.116955in}}%
\pgfpathlineto{\pgfqpoint{2.211775in}{3.116955in}}%
\pgfpathlineto{\pgfqpoint{2.214347in}{3.100499in}}%
\pgfpathlineto{\pgfqpoint{2.224635in}{3.100499in}}%
\pgfpathlineto{\pgfqpoint{2.229778in}{3.133411in}}%
\pgfpathlineto{\pgfqpoint{2.232350in}{3.116955in}}%
\pgfpathlineto{\pgfqpoint{2.234922in}{3.116955in}}%
\pgfpathlineto{\pgfqpoint{2.240066in}{3.084043in}}%
\pgfpathlineto{\pgfqpoint{2.242638in}{3.100499in}}%
\pgfpathlineto{\pgfqpoint{2.268356in}{3.100499in}}%
\pgfpathlineto{\pgfqpoint{2.270928in}{3.084043in}}%
\pgfpathlineto{\pgfqpoint{2.273500in}{3.084043in}}%
\pgfpathlineto{\pgfqpoint{2.276072in}{3.067587in}}%
\pgfpathlineto{\pgfqpoint{2.278644in}{3.067587in}}%
\pgfpathlineto{\pgfqpoint{2.281216in}{3.051131in}}%
\pgfpathlineto{\pgfqpoint{2.299219in}{3.051131in}}%
\pgfpathlineto{\pgfqpoint{2.301791in}{3.067587in}}%
\pgfpathlineto{\pgfqpoint{2.304363in}{3.067587in}}%
\pgfpathlineto{\pgfqpoint{2.306934in}{3.051131in}}%
\pgfpathlineto{\pgfqpoint{2.394378in}{3.051131in}}%
\pgfpathlineto{\pgfqpoint{2.396950in}{3.034675in}}%
\pgfpathlineto{\pgfqpoint{2.402094in}{3.034675in}}%
\pgfpathlineto{\pgfqpoint{2.404665in}{3.018219in}}%
\pgfpathlineto{\pgfqpoint{2.412381in}{3.018219in}}%
\pgfpathlineto{\pgfqpoint{2.414953in}{3.034675in}}%
\pgfpathlineto{\pgfqpoint{2.432956in}{3.034675in}}%
\pgfpathlineto{\pgfqpoint{2.438100in}{3.001763in}}%
\pgfpathlineto{\pgfqpoint{2.450959in}{3.001763in}}%
\pgfpathlineto{\pgfqpoint{2.453531in}{3.018219in}}%
\pgfpathlineto{\pgfqpoint{2.463819in}{3.018219in}}%
\pgfpathlineto{\pgfqpoint{2.466390in}{3.001763in}}%
\pgfpathlineto{\pgfqpoint{2.468962in}{3.001763in}}%
\pgfpathlineto{\pgfqpoint{2.471534in}{3.018219in}}%
\pgfpathlineto{\pgfqpoint{2.510112in}{3.018219in}}%
\pgfpathlineto{\pgfqpoint{2.512684in}{3.034675in}}%
\pgfpathlineto{\pgfqpoint{2.515256in}{3.034675in}}%
\pgfpathlineto{\pgfqpoint{2.517828in}{3.067587in}}%
\pgfpathlineto{\pgfqpoint{2.522972in}{3.067587in}}%
\pgfpathlineto{\pgfqpoint{2.525543in}{3.100499in}}%
\pgfpathlineto{\pgfqpoint{2.528115in}{3.116955in}}%
\pgfpathlineto{\pgfqpoint{2.530687in}{3.100499in}}%
\pgfpathlineto{\pgfqpoint{2.533259in}{3.100499in}}%
\pgfpathlineto{\pgfqpoint{2.535831in}{3.116955in}}%
\pgfpathlineto{\pgfqpoint{2.538403in}{3.100499in}}%
\pgfpathlineto{\pgfqpoint{2.556406in}{3.100499in}}%
\pgfpathlineto{\pgfqpoint{2.558978in}{3.133411in}}%
\pgfpathlineto{\pgfqpoint{2.561550in}{3.116955in}}%
\pgfpathlineto{\pgfqpoint{2.564121in}{3.133411in}}%
\pgfpathlineto{\pgfqpoint{2.566693in}{3.100499in}}%
\pgfpathlineto{\pgfqpoint{2.576981in}{3.100499in}}%
\pgfpathlineto{\pgfqpoint{2.579553in}{3.116955in}}%
\pgfpathlineto{\pgfqpoint{2.582125in}{3.100499in}}%
\pgfpathlineto{\pgfqpoint{2.592412in}{3.100499in}}%
\pgfpathlineto{\pgfqpoint{2.594984in}{3.116955in}}%
\pgfpathlineto{\pgfqpoint{2.597556in}{3.100499in}}%
\pgfpathlineto{\pgfqpoint{2.602699in}{3.100499in}}%
\pgfpathlineto{\pgfqpoint{2.605271in}{3.084043in}}%
\pgfpathlineto{\pgfqpoint{2.636134in}{3.084043in}}%
\pgfpathlineto{\pgfqpoint{2.638706in}{3.133411in}}%
\pgfpathlineto{\pgfqpoint{2.646421in}{3.133411in}}%
\pgfpathlineto{\pgfqpoint{2.646421in}{3.133411in}}%
\pgfusepath{stroke}%
\end{pgfscope}%
\begin{pgfscope}%
\pgfpathrectangle{\pgfqpoint{0.488751in}{1.946106in}}{\pgfqpoint{2.260417in}{1.502439in}}%
\pgfusepath{clip}%
\pgfsetbuttcap%
\pgfsetroundjoin%
\pgfsetlinewidth{0.803000pt}%
\definecolor{currentstroke}{rgb}{0.843137,0.666667,0.313725}%
\pgfsetstrokecolor{currentstroke}%
\pgfsetstrokeopacity{0.270000}%
\pgfsetdash{{2.960000pt}{1.280000pt}}{0.000000pt}%
\pgfpathmoveto{\pgfqpoint{0.591497in}{2.985307in}}%
\pgfpathlineto{\pgfqpoint{0.596641in}{3.018219in}}%
\pgfpathlineto{\pgfqpoint{0.599213in}{3.018219in}}%
\pgfpathlineto{\pgfqpoint{0.601785in}{3.001763in}}%
\pgfpathlineto{\pgfqpoint{0.604356in}{2.968851in}}%
\pgfpathlineto{\pgfqpoint{0.606928in}{2.985307in}}%
\pgfpathlineto{\pgfqpoint{0.609500in}{2.985307in}}%
\pgfpathlineto{\pgfqpoint{0.612072in}{3.001763in}}%
\pgfpathlineto{\pgfqpoint{0.614644in}{3.001763in}}%
\pgfpathlineto{\pgfqpoint{0.617216in}{2.985307in}}%
\pgfpathlineto{\pgfqpoint{0.619788in}{3.001763in}}%
\pgfpathlineto{\pgfqpoint{0.622359in}{3.034675in}}%
\pgfpathlineto{\pgfqpoint{0.624931in}{2.968851in}}%
\pgfpathlineto{\pgfqpoint{0.627503in}{2.935938in}}%
\pgfpathlineto{\pgfqpoint{0.630075in}{2.952395in}}%
\pgfpathlineto{\pgfqpoint{0.637791in}{2.952395in}}%
\pgfpathlineto{\pgfqpoint{0.640363in}{2.985307in}}%
\pgfpathlineto{\pgfqpoint{0.642934in}{2.968851in}}%
\pgfpathlineto{\pgfqpoint{0.645506in}{2.968851in}}%
\pgfpathlineto{\pgfqpoint{0.648078in}{2.903026in}}%
\pgfpathlineto{\pgfqpoint{0.653222in}{2.903026in}}%
\pgfpathlineto{\pgfqpoint{0.655794in}{2.919482in}}%
\pgfpathlineto{\pgfqpoint{0.658366in}{2.919482in}}%
\pgfpathlineto{\pgfqpoint{0.660938in}{2.903026in}}%
\pgfpathlineto{\pgfqpoint{0.663509in}{2.903026in}}%
\pgfpathlineto{\pgfqpoint{0.666081in}{2.968851in}}%
\pgfpathlineto{\pgfqpoint{0.668653in}{2.935938in}}%
\pgfpathlineto{\pgfqpoint{0.671225in}{2.935938in}}%
\pgfpathlineto{\pgfqpoint{0.673797in}{2.985307in}}%
\pgfpathlineto{\pgfqpoint{0.676369in}{3.001763in}}%
\pgfpathlineto{\pgfqpoint{0.678941in}{3.001763in}}%
\pgfpathlineto{\pgfqpoint{0.681513in}{2.985307in}}%
\pgfpathlineto{\pgfqpoint{0.684084in}{2.985307in}}%
\pgfpathlineto{\pgfqpoint{0.686656in}{3.001763in}}%
\pgfpathlineto{\pgfqpoint{0.691800in}{2.968851in}}%
\pgfpathlineto{\pgfqpoint{0.696944in}{3.001763in}}%
\pgfpathlineto{\pgfqpoint{0.699516in}{2.985307in}}%
\pgfpathlineto{\pgfqpoint{0.702087in}{3.018219in}}%
\pgfpathlineto{\pgfqpoint{0.704659in}{2.985307in}}%
\pgfpathlineto{\pgfqpoint{0.707231in}{3.001763in}}%
\pgfpathlineto{\pgfqpoint{0.712375in}{3.001763in}}%
\pgfpathlineto{\pgfqpoint{0.717519in}{3.034675in}}%
\pgfpathlineto{\pgfqpoint{0.722662in}{3.034675in}}%
\pgfpathlineto{\pgfqpoint{0.725234in}{3.051131in}}%
\pgfpathlineto{\pgfqpoint{0.727806in}{3.051131in}}%
\pgfpathlineto{\pgfqpoint{0.732950in}{3.018219in}}%
\pgfpathlineto{\pgfqpoint{0.735522in}{3.034675in}}%
\pgfpathlineto{\pgfqpoint{0.738094in}{3.018219in}}%
\pgfpathlineto{\pgfqpoint{0.743237in}{3.051131in}}%
\pgfpathlineto{\pgfqpoint{0.745809in}{3.034675in}}%
\pgfpathlineto{\pgfqpoint{0.748381in}{3.034675in}}%
\pgfpathlineto{\pgfqpoint{0.750953in}{3.018219in}}%
\pgfpathlineto{\pgfqpoint{0.753525in}{2.985307in}}%
\pgfpathlineto{\pgfqpoint{0.761240in}{3.034675in}}%
\pgfpathlineto{\pgfqpoint{0.763812in}{3.034675in}}%
\pgfpathlineto{\pgfqpoint{0.766384in}{3.018219in}}%
\pgfpathlineto{\pgfqpoint{0.768956in}{3.051131in}}%
\pgfpathlineto{\pgfqpoint{0.789531in}{3.051131in}}%
\pgfpathlineto{\pgfqpoint{0.792103in}{3.067587in}}%
\pgfpathlineto{\pgfqpoint{0.794675in}{3.067587in}}%
\pgfpathlineto{\pgfqpoint{0.799819in}{3.100499in}}%
\pgfpathlineto{\pgfqpoint{0.804962in}{3.100499in}}%
\pgfpathlineto{\pgfqpoint{0.810106in}{3.067587in}}%
\pgfpathlineto{\pgfqpoint{0.815250in}{3.067587in}}%
\pgfpathlineto{\pgfqpoint{0.820393in}{3.034675in}}%
\pgfpathlineto{\pgfqpoint{0.825537in}{3.067587in}}%
\pgfpathlineto{\pgfqpoint{0.830681in}{3.067587in}}%
\pgfpathlineto{\pgfqpoint{0.833253in}{3.051131in}}%
\pgfpathlineto{\pgfqpoint{0.838397in}{3.051131in}}%
\pgfpathlineto{\pgfqpoint{0.840968in}{3.018219in}}%
\pgfpathlineto{\pgfqpoint{0.843540in}{3.018219in}}%
\pgfpathlineto{\pgfqpoint{0.846112in}{3.051131in}}%
\pgfpathlineto{\pgfqpoint{0.848684in}{3.067587in}}%
\pgfpathlineto{\pgfqpoint{0.851256in}{3.051131in}}%
\pgfpathlineto{\pgfqpoint{0.853828in}{3.018219in}}%
\pgfpathlineto{\pgfqpoint{0.856400in}{3.001763in}}%
\pgfpathlineto{\pgfqpoint{0.858972in}{2.968851in}}%
\pgfpathlineto{\pgfqpoint{0.861543in}{2.985307in}}%
\pgfpathlineto{\pgfqpoint{0.864115in}{2.968851in}}%
\pgfpathlineto{\pgfqpoint{0.866687in}{2.968851in}}%
\pgfpathlineto{\pgfqpoint{0.871831in}{2.935938in}}%
\pgfpathlineto{\pgfqpoint{0.874403in}{2.968851in}}%
\pgfpathlineto{\pgfqpoint{0.879547in}{2.968851in}}%
\pgfpathlineto{\pgfqpoint{0.887262in}{2.919482in}}%
\pgfpathlineto{\pgfqpoint{0.889834in}{2.935938in}}%
\pgfpathlineto{\pgfqpoint{0.892406in}{2.935938in}}%
\pgfpathlineto{\pgfqpoint{0.894978in}{2.919482in}}%
\pgfpathlineto{\pgfqpoint{0.900121in}{2.952395in}}%
\pgfpathlineto{\pgfqpoint{0.905265in}{2.919482in}}%
\pgfpathlineto{\pgfqpoint{0.907837in}{2.870114in}}%
\pgfpathlineto{\pgfqpoint{0.910409in}{2.853658in}}%
\pgfpathlineto{\pgfqpoint{0.912981in}{2.870114in}}%
\pgfpathlineto{\pgfqpoint{0.915553in}{2.853658in}}%
\pgfpathlineto{\pgfqpoint{0.923268in}{2.754922in}}%
\pgfpathlineto{\pgfqpoint{0.925840in}{2.754922in}}%
\pgfpathlineto{\pgfqpoint{0.928412in}{2.787834in}}%
\pgfpathlineto{\pgfqpoint{0.930984in}{2.771378in}}%
\pgfpathlineto{\pgfqpoint{0.933556in}{2.738466in}}%
\pgfpathlineto{\pgfqpoint{0.936128in}{2.738466in}}%
\pgfpathlineto{\pgfqpoint{0.938700in}{2.722010in}}%
\pgfpathlineto{\pgfqpoint{0.941271in}{2.722010in}}%
\pgfpathlineto{\pgfqpoint{0.943843in}{2.705554in}}%
\pgfpathlineto{\pgfqpoint{0.959274in}{2.705554in}}%
\pgfpathlineto{\pgfqpoint{0.961846in}{2.672641in}}%
\pgfpathlineto{\pgfqpoint{0.969562in}{2.672641in}}%
\pgfpathlineto{\pgfqpoint{0.972134in}{2.656185in}}%
\pgfpathlineto{\pgfqpoint{0.987565in}{2.656185in}}%
\pgfpathlineto{\pgfqpoint{0.990137in}{2.672641in}}%
\pgfpathlineto{\pgfqpoint{1.000424in}{2.672641in}}%
\pgfpathlineto{\pgfqpoint{1.002996in}{2.656185in}}%
\pgfpathlineto{\pgfqpoint{1.008140in}{2.656185in}}%
\pgfpathlineto{\pgfqpoint{1.010712in}{2.639729in}}%
\pgfpathlineto{\pgfqpoint{1.051862in}{2.639729in}}%
\pgfpathlineto{\pgfqpoint{1.054434in}{2.623273in}}%
\pgfpathlineto{\pgfqpoint{1.087868in}{2.623273in}}%
\pgfpathlineto{\pgfqpoint{1.090440in}{2.606817in}}%
\pgfpathlineto{\pgfqpoint{1.100727in}{2.606817in}}%
\pgfpathlineto{\pgfqpoint{1.103299in}{2.623273in}}%
\pgfpathlineto{\pgfqpoint{1.766842in}{2.623273in}}%
\pgfpathlineto{\pgfqpoint{1.769414in}{2.639729in}}%
\pgfpathlineto{\pgfqpoint{1.885148in}{2.639729in}}%
\pgfpathlineto{\pgfqpoint{1.887720in}{2.623273in}}%
\pgfpathlineto{\pgfqpoint{1.946873in}{2.623273in}}%
\pgfpathlineto{\pgfqpoint{1.949444in}{2.606817in}}%
\pgfpathlineto{\pgfqpoint{2.378947in}{2.606817in}}%
\pgfpathlineto{\pgfqpoint{2.381519in}{2.623273in}}%
\pgfpathlineto{\pgfqpoint{2.646421in}{2.623273in}}%
\pgfpathlineto{\pgfqpoint{2.646421in}{2.623273in}}%
\pgfusepath{stroke}%
\end{pgfscope}%
\begin{pgfscope}%
\pgfpathrectangle{\pgfqpoint{0.488751in}{1.946106in}}{\pgfqpoint{2.260417in}{1.502439in}}%
\pgfusepath{clip}%
\pgfsetbuttcap%
\pgfsetroundjoin%
\pgfsetlinewidth{0.803000pt}%
\definecolor{currentstroke}{rgb}{0.333333,0.333333,0.333333}%
\pgfsetstrokecolor{currentstroke}%
\pgfsetstrokeopacity{0.270000}%
\pgfsetdash{{2.960000pt}{1.280000pt}}{0.000000pt}%
\pgfpathmoveto{\pgfqpoint{0.591497in}{2.754922in}}%
\pgfpathlineto{\pgfqpoint{0.594069in}{2.754922in}}%
\pgfpathlineto{\pgfqpoint{0.596641in}{2.672641in}}%
\pgfpathlineto{\pgfqpoint{0.599213in}{2.689097in}}%
\pgfpathlineto{\pgfqpoint{0.601785in}{2.656185in}}%
\pgfpathlineto{\pgfqpoint{0.604356in}{2.656185in}}%
\pgfpathlineto{\pgfqpoint{0.609500in}{2.689097in}}%
\pgfpathlineto{\pgfqpoint{0.612072in}{2.689097in}}%
\pgfpathlineto{\pgfqpoint{0.614644in}{2.672641in}}%
\pgfpathlineto{\pgfqpoint{0.617216in}{2.672641in}}%
\pgfpathlineto{\pgfqpoint{0.622359in}{2.557449in}}%
\pgfpathlineto{\pgfqpoint{0.624931in}{2.540993in}}%
\pgfpathlineto{\pgfqpoint{0.627503in}{2.557449in}}%
\pgfpathlineto{\pgfqpoint{0.630075in}{2.623273in}}%
\pgfpathlineto{\pgfqpoint{0.632647in}{2.557449in}}%
\pgfpathlineto{\pgfqpoint{0.635219in}{2.540993in}}%
\pgfpathlineto{\pgfqpoint{0.637791in}{2.508081in}}%
\pgfpathlineto{\pgfqpoint{0.640363in}{2.606817in}}%
\pgfpathlineto{\pgfqpoint{0.642934in}{2.606817in}}%
\pgfpathlineto{\pgfqpoint{0.645506in}{2.639729in}}%
\pgfpathlineto{\pgfqpoint{0.648078in}{2.705554in}}%
\pgfpathlineto{\pgfqpoint{0.650650in}{2.722010in}}%
\pgfpathlineto{\pgfqpoint{0.653222in}{2.722010in}}%
\pgfpathlineto{\pgfqpoint{0.658366in}{2.804290in}}%
\pgfpathlineto{\pgfqpoint{0.660938in}{2.804290in}}%
\pgfpathlineto{\pgfqpoint{0.663509in}{2.837202in}}%
\pgfpathlineto{\pgfqpoint{0.666081in}{2.837202in}}%
\pgfpathlineto{\pgfqpoint{0.668653in}{2.804290in}}%
\pgfpathlineto{\pgfqpoint{0.673797in}{2.837202in}}%
\pgfpathlineto{\pgfqpoint{0.676369in}{2.820746in}}%
\pgfpathlineto{\pgfqpoint{0.678941in}{2.837202in}}%
\pgfpathlineto{\pgfqpoint{0.684084in}{2.837202in}}%
\pgfpathlineto{\pgfqpoint{0.686656in}{2.853658in}}%
\pgfpathlineto{\pgfqpoint{0.689228in}{2.787834in}}%
\pgfpathlineto{\pgfqpoint{0.696944in}{2.837202in}}%
\pgfpathlineto{\pgfqpoint{0.699516in}{2.837202in}}%
\pgfpathlineto{\pgfqpoint{0.702087in}{2.853658in}}%
\pgfpathlineto{\pgfqpoint{0.707231in}{2.919482in}}%
\pgfpathlineto{\pgfqpoint{0.709803in}{2.919482in}}%
\pgfpathlineto{\pgfqpoint{0.712375in}{2.968851in}}%
\pgfpathlineto{\pgfqpoint{0.714947in}{2.968851in}}%
\pgfpathlineto{\pgfqpoint{0.717519in}{2.952395in}}%
\pgfpathlineto{\pgfqpoint{0.720091in}{2.952395in}}%
\pgfpathlineto{\pgfqpoint{0.722662in}{2.985307in}}%
\pgfpathlineto{\pgfqpoint{0.725234in}{3.001763in}}%
\pgfpathlineto{\pgfqpoint{0.730378in}{3.001763in}}%
\pgfpathlineto{\pgfqpoint{0.732950in}{2.968851in}}%
\pgfpathlineto{\pgfqpoint{0.735522in}{2.985307in}}%
\pgfpathlineto{\pgfqpoint{0.738094in}{2.985307in}}%
\pgfpathlineto{\pgfqpoint{0.740666in}{2.968851in}}%
\pgfpathlineto{\pgfqpoint{0.743237in}{2.968851in}}%
\pgfpathlineto{\pgfqpoint{0.745809in}{2.952395in}}%
\pgfpathlineto{\pgfqpoint{0.748381in}{2.952395in}}%
\pgfpathlineto{\pgfqpoint{0.750953in}{2.919482in}}%
\pgfpathlineto{\pgfqpoint{0.753525in}{2.935938in}}%
\pgfpathlineto{\pgfqpoint{0.758669in}{3.001763in}}%
\pgfpathlineto{\pgfqpoint{0.761240in}{3.001763in}}%
\pgfpathlineto{\pgfqpoint{0.768956in}{3.051131in}}%
\pgfpathlineto{\pgfqpoint{0.771528in}{3.100499in}}%
\pgfpathlineto{\pgfqpoint{0.774100in}{3.116955in}}%
\pgfpathlineto{\pgfqpoint{0.776672in}{3.116955in}}%
\pgfpathlineto{\pgfqpoint{0.779244in}{3.084043in}}%
\pgfpathlineto{\pgfqpoint{0.786959in}{3.034675in}}%
\pgfpathlineto{\pgfqpoint{0.789531in}{3.034675in}}%
\pgfpathlineto{\pgfqpoint{0.792103in}{3.067587in}}%
\pgfpathlineto{\pgfqpoint{0.794675in}{3.051131in}}%
\pgfpathlineto{\pgfqpoint{0.799819in}{3.051131in}}%
\pgfpathlineto{\pgfqpoint{0.807534in}{3.100499in}}%
\pgfpathlineto{\pgfqpoint{0.810106in}{3.067587in}}%
\pgfpathlineto{\pgfqpoint{0.812678in}{3.067587in}}%
\pgfpathlineto{\pgfqpoint{0.815250in}{3.084043in}}%
\pgfpathlineto{\pgfqpoint{0.817822in}{3.084043in}}%
\pgfpathlineto{\pgfqpoint{0.820393in}{3.034675in}}%
\pgfpathlineto{\pgfqpoint{0.822965in}{3.034675in}}%
\pgfpathlineto{\pgfqpoint{0.825537in}{3.018219in}}%
\pgfpathlineto{\pgfqpoint{0.828109in}{3.018219in}}%
\pgfpathlineto{\pgfqpoint{0.830681in}{3.034675in}}%
\pgfpathlineto{\pgfqpoint{0.833253in}{3.034675in}}%
\pgfpathlineto{\pgfqpoint{0.835825in}{3.018219in}}%
\pgfpathlineto{\pgfqpoint{0.843540in}{3.018219in}}%
\pgfpathlineto{\pgfqpoint{0.846112in}{3.001763in}}%
\pgfpathlineto{\pgfqpoint{0.848684in}{3.001763in}}%
\pgfpathlineto{\pgfqpoint{0.851256in}{3.018219in}}%
\pgfpathlineto{\pgfqpoint{0.853828in}{2.985307in}}%
\pgfpathlineto{\pgfqpoint{0.858972in}{2.952395in}}%
\pgfpathlineto{\pgfqpoint{0.864115in}{2.952395in}}%
\pgfpathlineto{\pgfqpoint{0.866687in}{2.919482in}}%
\pgfpathlineto{\pgfqpoint{0.869259in}{2.903026in}}%
\pgfpathlineto{\pgfqpoint{0.871831in}{2.903026in}}%
\pgfpathlineto{\pgfqpoint{0.874403in}{2.952395in}}%
\pgfpathlineto{\pgfqpoint{0.882118in}{2.853658in}}%
\pgfpathlineto{\pgfqpoint{0.884690in}{2.837202in}}%
\pgfpathlineto{\pgfqpoint{0.887262in}{2.804290in}}%
\pgfpathlineto{\pgfqpoint{0.889834in}{2.804290in}}%
\pgfpathlineto{\pgfqpoint{0.894978in}{2.771378in}}%
\pgfpathlineto{\pgfqpoint{0.897550in}{2.804290in}}%
\pgfpathlineto{\pgfqpoint{0.902693in}{2.771378in}}%
\pgfpathlineto{\pgfqpoint{0.907837in}{2.837202in}}%
\pgfpathlineto{\pgfqpoint{0.912981in}{2.837202in}}%
\pgfpathlineto{\pgfqpoint{0.915553in}{2.853658in}}%
\pgfpathlineto{\pgfqpoint{0.918125in}{2.935938in}}%
\pgfpathlineto{\pgfqpoint{0.923268in}{2.853658in}}%
\pgfpathlineto{\pgfqpoint{0.925840in}{2.870114in}}%
\pgfpathlineto{\pgfqpoint{0.930984in}{2.870114in}}%
\pgfpathlineto{\pgfqpoint{0.933556in}{2.935938in}}%
\pgfpathlineto{\pgfqpoint{0.936128in}{2.935938in}}%
\pgfpathlineto{\pgfqpoint{0.938700in}{2.968851in}}%
\pgfpathlineto{\pgfqpoint{0.943843in}{2.968851in}}%
\pgfpathlineto{\pgfqpoint{0.948987in}{3.001763in}}%
\pgfpathlineto{\pgfqpoint{0.951559in}{3.001763in}}%
\pgfpathlineto{\pgfqpoint{0.954131in}{2.968851in}}%
\pgfpathlineto{\pgfqpoint{0.956703in}{2.985307in}}%
\pgfpathlineto{\pgfqpoint{0.959274in}{2.968851in}}%
\pgfpathlineto{\pgfqpoint{0.961846in}{2.985307in}}%
\pgfpathlineto{\pgfqpoint{0.964418in}{2.935938in}}%
\pgfpathlineto{\pgfqpoint{0.969562in}{2.968851in}}%
\pgfpathlineto{\pgfqpoint{0.972134in}{2.935938in}}%
\pgfpathlineto{\pgfqpoint{0.974706in}{2.952395in}}%
\pgfpathlineto{\pgfqpoint{0.977278in}{2.952395in}}%
\pgfpathlineto{\pgfqpoint{0.979849in}{2.968851in}}%
\pgfpathlineto{\pgfqpoint{0.984993in}{3.034675in}}%
\pgfpathlineto{\pgfqpoint{0.987565in}{3.051131in}}%
\pgfpathlineto{\pgfqpoint{0.990137in}{3.100499in}}%
\pgfpathlineto{\pgfqpoint{0.992709in}{3.100499in}}%
\pgfpathlineto{\pgfqpoint{0.995281in}{3.116955in}}%
\pgfpathlineto{\pgfqpoint{0.997853in}{3.100499in}}%
\pgfpathlineto{\pgfqpoint{1.002996in}{3.100499in}}%
\pgfpathlineto{\pgfqpoint{1.005568in}{3.116955in}}%
\pgfpathlineto{\pgfqpoint{1.008140in}{3.067587in}}%
\pgfpathlineto{\pgfqpoint{1.010712in}{3.051131in}}%
\pgfpathlineto{\pgfqpoint{1.013284in}{3.067587in}}%
\pgfpathlineto{\pgfqpoint{1.015856in}{3.034675in}}%
\pgfpathlineto{\pgfqpoint{1.018427in}{2.968851in}}%
\pgfpathlineto{\pgfqpoint{1.020999in}{2.968851in}}%
\pgfpathlineto{\pgfqpoint{1.023571in}{2.952395in}}%
\pgfpathlineto{\pgfqpoint{1.026143in}{2.968851in}}%
\pgfpathlineto{\pgfqpoint{1.028715in}{2.935938in}}%
\pgfpathlineto{\pgfqpoint{1.031287in}{2.952395in}}%
\pgfpathlineto{\pgfqpoint{1.033859in}{2.919482in}}%
\pgfpathlineto{\pgfqpoint{1.036431in}{2.935938in}}%
\pgfpathlineto{\pgfqpoint{1.039002in}{2.919482in}}%
\pgfpathlineto{\pgfqpoint{1.041574in}{2.952395in}}%
\pgfpathlineto{\pgfqpoint{1.044146in}{3.001763in}}%
\pgfpathlineto{\pgfqpoint{1.049290in}{3.001763in}}%
\pgfpathlineto{\pgfqpoint{1.051862in}{3.018219in}}%
\pgfpathlineto{\pgfqpoint{1.054434in}{3.018219in}}%
\pgfpathlineto{\pgfqpoint{1.059577in}{2.985307in}}%
\pgfpathlineto{\pgfqpoint{1.062149in}{3.018219in}}%
\pgfpathlineto{\pgfqpoint{1.064721in}{3.018219in}}%
\pgfpathlineto{\pgfqpoint{1.067293in}{3.034675in}}%
\pgfpathlineto{\pgfqpoint{1.069865in}{3.067587in}}%
\pgfpathlineto{\pgfqpoint{1.072437in}{3.084043in}}%
\pgfpathlineto{\pgfqpoint{1.075009in}{3.084043in}}%
\pgfpathlineto{\pgfqpoint{1.077581in}{3.100499in}}%
\pgfpathlineto{\pgfqpoint{1.085296in}{3.100499in}}%
\pgfpathlineto{\pgfqpoint{1.087868in}{3.084043in}}%
\pgfpathlineto{\pgfqpoint{1.090440in}{3.100499in}}%
\pgfpathlineto{\pgfqpoint{1.095584in}{3.100499in}}%
\pgfpathlineto{\pgfqpoint{1.098155in}{3.116955in}}%
\pgfpathlineto{\pgfqpoint{1.103299in}{3.116955in}}%
\pgfpathlineto{\pgfqpoint{1.105871in}{3.100499in}}%
\pgfpathlineto{\pgfqpoint{1.108443in}{3.100499in}}%
\pgfpathlineto{\pgfqpoint{1.111015in}{3.133411in}}%
\pgfpathlineto{\pgfqpoint{1.113587in}{3.100499in}}%
\pgfpathlineto{\pgfqpoint{1.116159in}{3.084043in}}%
\pgfpathlineto{\pgfqpoint{1.118730in}{3.084043in}}%
\pgfpathlineto{\pgfqpoint{1.123874in}{3.018219in}}%
\pgfpathlineto{\pgfqpoint{1.126446in}{3.034675in}}%
\pgfpathlineto{\pgfqpoint{1.129018in}{3.001763in}}%
\pgfpathlineto{\pgfqpoint{1.131590in}{3.001763in}}%
\pgfpathlineto{\pgfqpoint{1.134162in}{2.935938in}}%
\pgfpathlineto{\pgfqpoint{1.136734in}{2.935938in}}%
\pgfpathlineto{\pgfqpoint{1.139305in}{2.968851in}}%
\pgfpathlineto{\pgfqpoint{1.144449in}{3.001763in}}%
\pgfpathlineto{\pgfqpoint{1.157308in}{3.001763in}}%
\pgfpathlineto{\pgfqpoint{1.159880in}{3.018219in}}%
\pgfpathlineto{\pgfqpoint{1.162452in}{3.018219in}}%
\pgfpathlineto{\pgfqpoint{1.165024in}{3.001763in}}%
\pgfpathlineto{\pgfqpoint{1.170168in}{3.001763in}}%
\pgfpathlineto{\pgfqpoint{1.175312in}{3.067587in}}%
\pgfpathlineto{\pgfqpoint{1.177883in}{3.084043in}}%
\pgfpathlineto{\pgfqpoint{1.180455in}{3.084043in}}%
\pgfpathlineto{\pgfqpoint{1.185599in}{3.051131in}}%
\pgfpathlineto{\pgfqpoint{1.188171in}{3.067587in}}%
\pgfpathlineto{\pgfqpoint{1.190743in}{3.051131in}}%
\pgfpathlineto{\pgfqpoint{1.193315in}{3.051131in}}%
\pgfpathlineto{\pgfqpoint{1.195887in}{3.067587in}}%
\pgfpathlineto{\pgfqpoint{1.198458in}{3.067587in}}%
\pgfpathlineto{\pgfqpoint{1.201030in}{3.084043in}}%
\pgfpathlineto{\pgfqpoint{1.206174in}{3.084043in}}%
\pgfpathlineto{\pgfqpoint{1.208746in}{3.067587in}}%
\pgfpathlineto{\pgfqpoint{1.211318in}{3.034675in}}%
\pgfpathlineto{\pgfqpoint{1.213890in}{3.034675in}}%
\pgfpathlineto{\pgfqpoint{1.216461in}{2.985307in}}%
\pgfpathlineto{\pgfqpoint{1.219033in}{2.968851in}}%
\pgfpathlineto{\pgfqpoint{1.221605in}{3.034675in}}%
\pgfpathlineto{\pgfqpoint{1.226749in}{2.935938in}}%
\pgfpathlineto{\pgfqpoint{1.231893in}{2.935938in}}%
\pgfpathlineto{\pgfqpoint{1.234465in}{2.952395in}}%
\pgfpathlineto{\pgfqpoint{1.237036in}{2.903026in}}%
\pgfpathlineto{\pgfqpoint{1.239608in}{2.919482in}}%
\pgfpathlineto{\pgfqpoint{1.242180in}{2.886570in}}%
\pgfpathlineto{\pgfqpoint{1.244752in}{2.870114in}}%
\pgfpathlineto{\pgfqpoint{1.247324in}{2.886570in}}%
\pgfpathlineto{\pgfqpoint{1.249896in}{2.886570in}}%
\pgfpathlineto{\pgfqpoint{1.252468in}{2.870114in}}%
\pgfpathlineto{\pgfqpoint{1.257611in}{2.870114in}}%
\pgfpathlineto{\pgfqpoint{1.260183in}{2.903026in}}%
\pgfpathlineto{\pgfqpoint{1.273043in}{2.820746in}}%
\pgfpathlineto{\pgfqpoint{1.275615in}{2.820746in}}%
\pgfpathlineto{\pgfqpoint{1.278186in}{2.804290in}}%
\pgfpathlineto{\pgfqpoint{1.283330in}{2.837202in}}%
\pgfpathlineto{\pgfqpoint{1.285902in}{2.837202in}}%
\pgfpathlineto{\pgfqpoint{1.291046in}{2.919482in}}%
\pgfpathlineto{\pgfqpoint{1.298761in}{2.919482in}}%
\pgfpathlineto{\pgfqpoint{1.301333in}{2.886570in}}%
\pgfpathlineto{\pgfqpoint{1.306477in}{2.886570in}}%
\pgfpathlineto{\pgfqpoint{1.309049in}{2.903026in}}%
\pgfpathlineto{\pgfqpoint{1.311621in}{2.870114in}}%
\pgfpathlineto{\pgfqpoint{1.314193in}{2.853658in}}%
\pgfpathlineto{\pgfqpoint{1.316764in}{2.870114in}}%
\pgfpathlineto{\pgfqpoint{1.321908in}{2.870114in}}%
\pgfpathlineto{\pgfqpoint{1.324480in}{2.853658in}}%
\pgfpathlineto{\pgfqpoint{1.327052in}{2.820746in}}%
\pgfpathlineto{\pgfqpoint{1.329624in}{2.837202in}}%
\pgfpathlineto{\pgfqpoint{1.332196in}{2.837202in}}%
\pgfpathlineto{\pgfqpoint{1.334768in}{2.820746in}}%
\pgfpathlineto{\pgfqpoint{1.337339in}{2.820746in}}%
\pgfpathlineto{\pgfqpoint{1.339911in}{2.870114in}}%
\pgfpathlineto{\pgfqpoint{1.342483in}{2.870114in}}%
\pgfpathlineto{\pgfqpoint{1.350199in}{2.919482in}}%
\pgfpathlineto{\pgfqpoint{1.352771in}{2.903026in}}%
\pgfpathlineto{\pgfqpoint{1.360486in}{2.804290in}}%
\pgfpathlineto{\pgfqpoint{1.363058in}{2.787834in}}%
\pgfpathlineto{\pgfqpoint{1.365630in}{2.754922in}}%
\pgfpathlineto{\pgfqpoint{1.370774in}{2.754922in}}%
\pgfpathlineto{\pgfqpoint{1.373346in}{2.771378in}}%
\pgfpathlineto{\pgfqpoint{1.375917in}{2.820746in}}%
\pgfpathlineto{\pgfqpoint{1.378489in}{2.837202in}}%
\pgfpathlineto{\pgfqpoint{1.381061in}{2.837202in}}%
\pgfpathlineto{\pgfqpoint{1.383633in}{2.870114in}}%
\pgfpathlineto{\pgfqpoint{1.386205in}{2.870114in}}%
\pgfpathlineto{\pgfqpoint{1.388777in}{2.837202in}}%
\pgfpathlineto{\pgfqpoint{1.391349in}{2.820746in}}%
\pgfpathlineto{\pgfqpoint{1.393921in}{2.820746in}}%
\pgfpathlineto{\pgfqpoint{1.396492in}{2.837202in}}%
\pgfpathlineto{\pgfqpoint{1.399064in}{2.837202in}}%
\pgfpathlineto{\pgfqpoint{1.401636in}{2.870114in}}%
\pgfpathlineto{\pgfqpoint{1.404208in}{2.886570in}}%
\pgfpathlineto{\pgfqpoint{1.406780in}{2.886570in}}%
\pgfpathlineto{\pgfqpoint{1.409352in}{2.853658in}}%
\pgfpathlineto{\pgfqpoint{1.411924in}{2.853658in}}%
\pgfpathlineto{\pgfqpoint{1.414495in}{2.870114in}}%
\pgfpathlineto{\pgfqpoint{1.419639in}{2.870114in}}%
\pgfpathlineto{\pgfqpoint{1.422211in}{2.837202in}}%
\pgfpathlineto{\pgfqpoint{1.424783in}{2.853658in}}%
\pgfpathlineto{\pgfqpoint{1.429927in}{2.853658in}}%
\pgfpathlineto{\pgfqpoint{1.432499in}{2.870114in}}%
\pgfpathlineto{\pgfqpoint{1.435070in}{2.820746in}}%
\pgfpathlineto{\pgfqpoint{1.440214in}{2.853658in}}%
\pgfpathlineto{\pgfqpoint{1.442786in}{2.853658in}}%
\pgfpathlineto{\pgfqpoint{1.445358in}{2.870114in}}%
\pgfpathlineto{\pgfqpoint{1.453074in}{2.870114in}}%
\pgfpathlineto{\pgfqpoint{1.455645in}{2.853658in}}%
\pgfpathlineto{\pgfqpoint{1.460789in}{2.886570in}}%
\pgfpathlineto{\pgfqpoint{1.463361in}{2.853658in}}%
\pgfpathlineto{\pgfqpoint{1.465933in}{2.853658in}}%
\pgfpathlineto{\pgfqpoint{1.468505in}{2.837202in}}%
\pgfpathlineto{\pgfqpoint{1.473649in}{2.837202in}}%
\pgfpathlineto{\pgfqpoint{1.476220in}{2.853658in}}%
\pgfpathlineto{\pgfqpoint{1.481364in}{2.853658in}}%
\pgfpathlineto{\pgfqpoint{1.483936in}{2.870114in}}%
\pgfpathlineto{\pgfqpoint{1.486508in}{2.870114in}}%
\pgfpathlineto{\pgfqpoint{1.489080in}{2.903026in}}%
\pgfpathlineto{\pgfqpoint{1.491652in}{2.903026in}}%
\pgfpathlineto{\pgfqpoint{1.494223in}{2.886570in}}%
\pgfpathlineto{\pgfqpoint{1.496795in}{2.903026in}}%
\pgfpathlineto{\pgfqpoint{1.499367in}{2.886570in}}%
\pgfpathlineto{\pgfqpoint{1.501939in}{2.853658in}}%
\pgfpathlineto{\pgfqpoint{1.504511in}{2.787834in}}%
\pgfpathlineto{\pgfqpoint{1.507083in}{2.804290in}}%
\pgfpathlineto{\pgfqpoint{1.509655in}{2.771378in}}%
\pgfpathlineto{\pgfqpoint{1.512227in}{2.722010in}}%
\pgfpathlineto{\pgfqpoint{1.514798in}{2.722010in}}%
\pgfpathlineto{\pgfqpoint{1.517370in}{2.738466in}}%
\pgfpathlineto{\pgfqpoint{1.519942in}{2.722010in}}%
\pgfpathlineto{\pgfqpoint{1.522514in}{2.738466in}}%
\pgfpathlineto{\pgfqpoint{1.525086in}{2.722010in}}%
\pgfpathlineto{\pgfqpoint{1.527658in}{2.754922in}}%
\pgfpathlineto{\pgfqpoint{1.530230in}{2.837202in}}%
\pgfpathlineto{\pgfqpoint{1.532802in}{2.804290in}}%
\pgfpathlineto{\pgfqpoint{1.535373in}{2.804290in}}%
\pgfpathlineto{\pgfqpoint{1.537945in}{2.820746in}}%
\pgfpathlineto{\pgfqpoint{1.540517in}{2.870114in}}%
\pgfpathlineto{\pgfqpoint{1.543089in}{2.853658in}}%
\pgfpathlineto{\pgfqpoint{1.548233in}{2.853658in}}%
\pgfpathlineto{\pgfqpoint{1.550805in}{2.837202in}}%
\pgfpathlineto{\pgfqpoint{1.555948in}{2.837202in}}%
\pgfpathlineto{\pgfqpoint{1.558520in}{2.853658in}}%
\pgfpathlineto{\pgfqpoint{1.563664in}{2.787834in}}%
\pgfpathlineto{\pgfqpoint{1.571380in}{2.787834in}}%
\pgfpathlineto{\pgfqpoint{1.573951in}{2.771378in}}%
\pgfpathlineto{\pgfqpoint{1.576523in}{2.738466in}}%
\pgfpathlineto{\pgfqpoint{1.579095in}{2.722010in}}%
\pgfpathlineto{\pgfqpoint{1.581667in}{2.722010in}}%
\pgfpathlineto{\pgfqpoint{1.586811in}{2.689097in}}%
\pgfpathlineto{\pgfqpoint{1.589383in}{2.722010in}}%
\pgfpathlineto{\pgfqpoint{1.591955in}{2.738466in}}%
\pgfpathlineto{\pgfqpoint{1.594526in}{2.722010in}}%
\pgfpathlineto{\pgfqpoint{1.597098in}{2.689097in}}%
\pgfpathlineto{\pgfqpoint{1.602242in}{2.754922in}}%
\pgfpathlineto{\pgfqpoint{1.604814in}{2.754922in}}%
\pgfpathlineto{\pgfqpoint{1.607386in}{2.738466in}}%
\pgfpathlineto{\pgfqpoint{1.609958in}{2.754922in}}%
\pgfpathlineto{\pgfqpoint{1.612529in}{2.722010in}}%
\pgfpathlineto{\pgfqpoint{1.615101in}{2.705554in}}%
\pgfpathlineto{\pgfqpoint{1.617673in}{2.705554in}}%
\pgfpathlineto{\pgfqpoint{1.625389in}{2.754922in}}%
\pgfpathlineto{\pgfqpoint{1.635676in}{2.754922in}}%
\pgfpathlineto{\pgfqpoint{1.640820in}{2.722010in}}%
\pgfpathlineto{\pgfqpoint{1.643392in}{2.722010in}}%
\pgfpathlineto{\pgfqpoint{1.648536in}{2.689097in}}%
\pgfpathlineto{\pgfqpoint{1.651108in}{2.689097in}}%
\pgfpathlineto{\pgfqpoint{1.653679in}{2.705554in}}%
\pgfpathlineto{\pgfqpoint{1.656251in}{2.705554in}}%
\pgfpathlineto{\pgfqpoint{1.658823in}{2.672641in}}%
\pgfpathlineto{\pgfqpoint{1.661395in}{2.672641in}}%
\pgfpathlineto{\pgfqpoint{1.663967in}{2.705554in}}%
\pgfpathlineto{\pgfqpoint{1.666539in}{2.705554in}}%
\pgfpathlineto{\pgfqpoint{1.669111in}{2.689097in}}%
\pgfpathlineto{\pgfqpoint{1.671683in}{2.689097in}}%
\pgfpathlineto{\pgfqpoint{1.674254in}{2.705554in}}%
\pgfpathlineto{\pgfqpoint{1.676826in}{2.656185in}}%
\pgfpathlineto{\pgfqpoint{1.679398in}{2.722010in}}%
\pgfpathlineto{\pgfqpoint{1.681970in}{2.705554in}}%
\pgfpathlineto{\pgfqpoint{1.684542in}{2.672641in}}%
\pgfpathlineto{\pgfqpoint{1.687114in}{2.672641in}}%
\pgfpathlineto{\pgfqpoint{1.692257in}{2.705554in}}%
\pgfpathlineto{\pgfqpoint{1.697401in}{2.705554in}}%
\pgfpathlineto{\pgfqpoint{1.699973in}{2.672641in}}%
\pgfpathlineto{\pgfqpoint{1.702545in}{2.672641in}}%
\pgfpathlineto{\pgfqpoint{1.705117in}{2.623273in}}%
\pgfpathlineto{\pgfqpoint{1.707689in}{2.639729in}}%
\pgfpathlineto{\pgfqpoint{1.712832in}{2.540993in}}%
\pgfpathlineto{\pgfqpoint{1.715404in}{2.540993in}}%
\pgfpathlineto{\pgfqpoint{1.717976in}{2.524537in}}%
\pgfpathlineto{\pgfqpoint{1.725692in}{2.524537in}}%
\pgfpathlineto{\pgfqpoint{1.728264in}{2.557449in}}%
\pgfpathlineto{\pgfqpoint{1.730836in}{2.557449in}}%
\pgfpathlineto{\pgfqpoint{1.733407in}{2.524537in}}%
\pgfpathlineto{\pgfqpoint{1.735979in}{2.508081in}}%
\pgfpathlineto{\pgfqpoint{1.746267in}{2.508081in}}%
\pgfpathlineto{\pgfqpoint{1.748839in}{2.491625in}}%
\pgfpathlineto{\pgfqpoint{1.753982in}{2.491625in}}%
\pgfpathlineto{\pgfqpoint{1.756554in}{2.508081in}}%
\pgfpathlineto{\pgfqpoint{1.766842in}{2.508081in}}%
\pgfpathlineto{\pgfqpoint{1.769414in}{2.524537in}}%
\pgfpathlineto{\pgfqpoint{1.771985in}{2.508081in}}%
\pgfpathlineto{\pgfqpoint{1.774557in}{2.508081in}}%
\pgfpathlineto{\pgfqpoint{1.777129in}{2.491625in}}%
\pgfpathlineto{\pgfqpoint{1.779701in}{2.491625in}}%
\pgfpathlineto{\pgfqpoint{1.782273in}{2.475169in}}%
\pgfpathlineto{\pgfqpoint{1.784845in}{2.475169in}}%
\pgfpathlineto{\pgfqpoint{1.787417in}{2.458712in}}%
\pgfpathlineto{\pgfqpoint{1.795132in}{2.458712in}}%
\pgfpathlineto{\pgfqpoint{1.800276in}{2.425800in}}%
\pgfpathlineto{\pgfqpoint{1.859429in}{2.425800in}}%
\pgfpathlineto{\pgfqpoint{1.862001in}{2.409344in}}%
\pgfpathlineto{\pgfqpoint{1.864573in}{2.425800in}}%
\pgfpathlineto{\pgfqpoint{1.877432in}{2.425800in}}%
\pgfpathlineto{\pgfqpoint{1.880004in}{2.409344in}}%
\pgfpathlineto{\pgfqpoint{1.885148in}{2.409344in}}%
\pgfpathlineto{\pgfqpoint{1.892863in}{2.359976in}}%
\pgfpathlineto{\pgfqpoint{1.910866in}{2.359976in}}%
\pgfpathlineto{\pgfqpoint{1.913438in}{2.392888in}}%
\pgfpathlineto{\pgfqpoint{1.946873in}{2.392888in}}%
\pgfpathlineto{\pgfqpoint{1.949444in}{2.376432in}}%
\pgfpathlineto{\pgfqpoint{2.006026in}{2.376432in}}%
\pgfpathlineto{\pgfqpoint{2.008597in}{2.392888in}}%
\pgfpathlineto{\pgfqpoint{2.047176in}{2.392888in}}%
\pgfpathlineto{\pgfqpoint{2.049747in}{2.425800in}}%
\pgfpathlineto{\pgfqpoint{2.057463in}{2.425800in}}%
\pgfpathlineto{\pgfqpoint{2.060035in}{2.409344in}}%
\pgfpathlineto{\pgfqpoint{2.070322in}{2.409344in}}%
\pgfpathlineto{\pgfqpoint{2.072894in}{2.425800in}}%
\pgfpathlineto{\pgfqpoint{2.085754in}{2.425800in}}%
\pgfpathlineto{\pgfqpoint{2.088325in}{2.442256in}}%
\pgfpathlineto{\pgfqpoint{2.090897in}{2.442256in}}%
\pgfpathlineto{\pgfqpoint{2.093469in}{2.458712in}}%
\pgfpathlineto{\pgfqpoint{2.142335in}{2.458712in}}%
\pgfpathlineto{\pgfqpoint{2.144907in}{2.442256in}}%
\pgfpathlineto{\pgfqpoint{2.162910in}{2.442256in}}%
\pgfpathlineto{\pgfqpoint{2.165482in}{2.425800in}}%
\pgfpathlineto{\pgfqpoint{2.337797in}{2.425800in}}%
\pgfpathlineto{\pgfqpoint{2.340369in}{2.442256in}}%
\pgfpathlineto{\pgfqpoint{2.348084in}{2.442256in}}%
\pgfpathlineto{\pgfqpoint{2.350656in}{2.425800in}}%
\pgfpathlineto{\pgfqpoint{2.355800in}{2.425800in}}%
\pgfpathlineto{\pgfqpoint{2.358372in}{2.442256in}}%
\pgfpathlineto{\pgfqpoint{2.371231in}{2.442256in}}%
\pgfpathlineto{\pgfqpoint{2.373803in}{2.458712in}}%
\pgfpathlineto{\pgfqpoint{2.381519in}{2.458712in}}%
\pgfpathlineto{\pgfqpoint{2.384091in}{2.442256in}}%
\pgfpathlineto{\pgfqpoint{2.386662in}{2.458712in}}%
\pgfpathlineto{\pgfqpoint{2.414953in}{2.458712in}}%
\pgfpathlineto{\pgfqpoint{2.417525in}{2.475169in}}%
\pgfpathlineto{\pgfqpoint{2.438100in}{2.475169in}}%
\pgfpathlineto{\pgfqpoint{2.440672in}{2.458712in}}%
\pgfpathlineto{\pgfqpoint{2.450959in}{2.458712in}}%
\pgfpathlineto{\pgfqpoint{2.453531in}{2.442256in}}%
\pgfpathlineto{\pgfqpoint{2.458675in}{2.442256in}}%
\pgfpathlineto{\pgfqpoint{2.461247in}{2.425800in}}%
\pgfpathlineto{\pgfqpoint{2.474106in}{2.425800in}}%
\pgfpathlineto{\pgfqpoint{2.476678in}{2.409344in}}%
\pgfpathlineto{\pgfqpoint{2.479250in}{2.376432in}}%
\pgfpathlineto{\pgfqpoint{2.486965in}{2.376432in}}%
\pgfpathlineto{\pgfqpoint{2.489537in}{2.359976in}}%
\pgfpathlineto{\pgfqpoint{2.494681in}{2.359976in}}%
\pgfpathlineto{\pgfqpoint{2.497253in}{2.343520in}}%
\pgfpathlineto{\pgfqpoint{2.515256in}{2.343520in}}%
\pgfpathlineto{\pgfqpoint{2.517828in}{2.327064in}}%
\pgfpathlineto{\pgfqpoint{2.520400in}{2.294152in}}%
\pgfpathlineto{\pgfqpoint{2.522972in}{2.294152in}}%
\pgfpathlineto{\pgfqpoint{2.528115in}{2.359976in}}%
\pgfpathlineto{\pgfqpoint{2.543546in}{2.359976in}}%
\pgfpathlineto{\pgfqpoint{2.546118in}{2.310608in}}%
\pgfpathlineto{\pgfqpoint{2.558978in}{2.310608in}}%
\pgfpathlineto{\pgfqpoint{2.561550in}{2.294152in}}%
\pgfpathlineto{\pgfqpoint{2.574409in}{2.294152in}}%
\pgfpathlineto{\pgfqpoint{2.579553in}{2.261240in}}%
\pgfpathlineto{\pgfqpoint{2.589840in}{2.261240in}}%
\pgfpathlineto{\pgfqpoint{2.592412in}{2.244784in}}%
\pgfpathlineto{\pgfqpoint{2.612987in}{2.244784in}}%
\pgfpathlineto{\pgfqpoint{2.615559in}{2.228328in}}%
\pgfpathlineto{\pgfqpoint{2.636134in}{2.228328in}}%
\pgfpathlineto{\pgfqpoint{2.638706in}{2.195415in}}%
\pgfpathlineto{\pgfqpoint{2.641278in}{2.195415in}}%
\pgfpathlineto{\pgfqpoint{2.643849in}{2.129591in}}%
\pgfpathlineto{\pgfqpoint{2.646421in}{2.129591in}}%
\pgfpathlineto{\pgfqpoint{2.646421in}{2.129591in}}%
\pgfusepath{stroke}%
\end{pgfscope}%
\begin{pgfscope}%
\pgfpathrectangle{\pgfqpoint{0.488751in}{1.946106in}}{\pgfqpoint{2.260417in}{1.502439in}}%
\pgfusepath{clip}%
\pgfsetbuttcap%
\pgfsetroundjoin%
\pgfsetlinewidth{0.803000pt}%
\definecolor{currentstroke}{rgb}{0.686275,0.352941,0.313725}%
\pgfsetstrokecolor{currentstroke}%
\pgfsetstrokeopacity{0.270000}%
\pgfsetdash{{2.960000pt}{1.280000pt}}{0.000000pt}%
\pgfpathmoveto{\pgfqpoint{0.591497in}{2.985307in}}%
\pgfpathlineto{\pgfqpoint{0.594069in}{3.034675in}}%
\pgfpathlineto{\pgfqpoint{0.596641in}{3.051131in}}%
\pgfpathlineto{\pgfqpoint{0.599213in}{3.034675in}}%
\pgfpathlineto{\pgfqpoint{0.604356in}{3.100499in}}%
\pgfpathlineto{\pgfqpoint{0.612072in}{3.100499in}}%
\pgfpathlineto{\pgfqpoint{0.614644in}{3.149867in}}%
\pgfpathlineto{\pgfqpoint{0.617216in}{3.166323in}}%
\pgfpathlineto{\pgfqpoint{0.619788in}{3.166323in}}%
\pgfpathlineto{\pgfqpoint{0.624931in}{3.232148in}}%
\pgfpathlineto{\pgfqpoint{0.627503in}{3.281516in}}%
\pgfpathlineto{\pgfqpoint{0.630075in}{3.232148in}}%
\pgfpathlineto{\pgfqpoint{0.632647in}{3.248604in}}%
\pgfpathlineto{\pgfqpoint{0.635219in}{3.232148in}}%
\pgfpathlineto{\pgfqpoint{0.637791in}{3.166323in}}%
\pgfpathlineto{\pgfqpoint{0.640363in}{3.133411in}}%
\pgfpathlineto{\pgfqpoint{0.648078in}{3.133411in}}%
\pgfpathlineto{\pgfqpoint{0.650650in}{3.100499in}}%
\pgfpathlineto{\pgfqpoint{0.653222in}{3.100499in}}%
\pgfpathlineto{\pgfqpoint{0.655794in}{3.116955in}}%
\pgfpathlineto{\pgfqpoint{0.658366in}{3.100499in}}%
\pgfpathlineto{\pgfqpoint{0.660938in}{3.100499in}}%
\pgfpathlineto{\pgfqpoint{0.663509in}{3.084043in}}%
\pgfpathlineto{\pgfqpoint{0.666081in}{3.100499in}}%
\pgfpathlineto{\pgfqpoint{0.668653in}{3.100499in}}%
\pgfpathlineto{\pgfqpoint{0.671225in}{3.067587in}}%
\pgfpathlineto{\pgfqpoint{0.673797in}{3.067587in}}%
\pgfpathlineto{\pgfqpoint{0.676369in}{3.100499in}}%
\pgfpathlineto{\pgfqpoint{0.681513in}{3.100499in}}%
\pgfpathlineto{\pgfqpoint{0.684084in}{3.067587in}}%
\pgfpathlineto{\pgfqpoint{0.686656in}{3.067587in}}%
\pgfpathlineto{\pgfqpoint{0.689228in}{3.084043in}}%
\pgfpathlineto{\pgfqpoint{0.691800in}{3.051131in}}%
\pgfpathlineto{\pgfqpoint{0.694372in}{3.051131in}}%
\pgfpathlineto{\pgfqpoint{0.702087in}{3.001763in}}%
\pgfpathlineto{\pgfqpoint{0.704659in}{3.001763in}}%
\pgfpathlineto{\pgfqpoint{0.707231in}{2.985307in}}%
\pgfpathlineto{\pgfqpoint{0.709803in}{2.935938in}}%
\pgfpathlineto{\pgfqpoint{0.712375in}{2.919482in}}%
\pgfpathlineto{\pgfqpoint{0.714947in}{2.886570in}}%
\pgfpathlineto{\pgfqpoint{0.717519in}{2.886570in}}%
\pgfpathlineto{\pgfqpoint{0.722662in}{2.919482in}}%
\pgfpathlineto{\pgfqpoint{0.725234in}{2.919482in}}%
\pgfpathlineto{\pgfqpoint{0.727806in}{2.935938in}}%
\pgfpathlineto{\pgfqpoint{0.730378in}{2.935938in}}%
\pgfpathlineto{\pgfqpoint{0.732950in}{2.952395in}}%
\pgfpathlineto{\pgfqpoint{0.735522in}{2.952395in}}%
\pgfpathlineto{\pgfqpoint{0.738094in}{2.968851in}}%
\pgfpathlineto{\pgfqpoint{0.740666in}{2.952395in}}%
\pgfpathlineto{\pgfqpoint{0.745809in}{2.952395in}}%
\pgfpathlineto{\pgfqpoint{0.748381in}{2.935938in}}%
\pgfpathlineto{\pgfqpoint{0.753525in}{2.935938in}}%
\pgfpathlineto{\pgfqpoint{0.756097in}{2.952395in}}%
\pgfpathlineto{\pgfqpoint{0.761240in}{2.952395in}}%
\pgfpathlineto{\pgfqpoint{0.763812in}{2.968851in}}%
\pgfpathlineto{\pgfqpoint{0.776672in}{2.968851in}}%
\pgfpathlineto{\pgfqpoint{0.781815in}{3.001763in}}%
\pgfpathlineto{\pgfqpoint{0.789531in}{3.001763in}}%
\pgfpathlineto{\pgfqpoint{0.792103in}{2.985307in}}%
\pgfpathlineto{\pgfqpoint{0.810106in}{2.985307in}}%
\pgfpathlineto{\pgfqpoint{0.812678in}{2.968851in}}%
\pgfpathlineto{\pgfqpoint{0.815250in}{2.985307in}}%
\pgfpathlineto{\pgfqpoint{0.822965in}{2.985307in}}%
\pgfpathlineto{\pgfqpoint{0.825537in}{3.001763in}}%
\pgfpathlineto{\pgfqpoint{0.828109in}{3.001763in}}%
\pgfpathlineto{\pgfqpoint{0.830681in}{2.985307in}}%
\pgfpathlineto{\pgfqpoint{0.838397in}{2.985307in}}%
\pgfpathlineto{\pgfqpoint{0.840968in}{3.001763in}}%
\pgfpathlineto{\pgfqpoint{0.846112in}{3.001763in}}%
\pgfpathlineto{\pgfqpoint{0.848684in}{2.985307in}}%
\pgfpathlineto{\pgfqpoint{0.851256in}{3.001763in}}%
\pgfpathlineto{\pgfqpoint{0.856400in}{3.001763in}}%
\pgfpathlineto{\pgfqpoint{0.858972in}{3.018219in}}%
\pgfpathlineto{\pgfqpoint{0.866687in}{3.018219in}}%
\pgfpathlineto{\pgfqpoint{0.871831in}{3.051131in}}%
\pgfpathlineto{\pgfqpoint{0.874403in}{3.034675in}}%
\pgfpathlineto{\pgfqpoint{0.876975in}{3.067587in}}%
\pgfpathlineto{\pgfqpoint{0.882118in}{3.067587in}}%
\pgfpathlineto{\pgfqpoint{0.884690in}{3.084043in}}%
\pgfpathlineto{\pgfqpoint{0.887262in}{3.084043in}}%
\pgfpathlineto{\pgfqpoint{0.889834in}{3.067587in}}%
\pgfpathlineto{\pgfqpoint{0.894978in}{3.067587in}}%
\pgfpathlineto{\pgfqpoint{0.897550in}{3.051131in}}%
\pgfpathlineto{\pgfqpoint{0.900121in}{3.051131in}}%
\pgfpathlineto{\pgfqpoint{0.902693in}{3.067587in}}%
\pgfpathlineto{\pgfqpoint{0.905265in}{3.067587in}}%
\pgfpathlineto{\pgfqpoint{0.907837in}{3.084043in}}%
\pgfpathlineto{\pgfqpoint{0.910409in}{3.067587in}}%
\pgfpathlineto{\pgfqpoint{0.915553in}{3.067587in}}%
\pgfpathlineto{\pgfqpoint{0.918125in}{3.018219in}}%
\pgfpathlineto{\pgfqpoint{0.923268in}{3.018219in}}%
\pgfpathlineto{\pgfqpoint{0.925840in}{2.985307in}}%
\pgfpathlineto{\pgfqpoint{0.930984in}{2.985307in}}%
\pgfpathlineto{\pgfqpoint{0.933556in}{2.952395in}}%
\pgfpathlineto{\pgfqpoint{0.936128in}{2.952395in}}%
\pgfpathlineto{\pgfqpoint{0.938700in}{2.935938in}}%
\pgfpathlineto{\pgfqpoint{0.943843in}{2.935938in}}%
\pgfpathlineto{\pgfqpoint{0.946415in}{2.919482in}}%
\pgfpathlineto{\pgfqpoint{0.954131in}{2.919482in}}%
\pgfpathlineto{\pgfqpoint{0.956703in}{2.903026in}}%
\pgfpathlineto{\pgfqpoint{0.974706in}{2.903026in}}%
\pgfpathlineto{\pgfqpoint{0.977278in}{2.886570in}}%
\pgfpathlineto{\pgfqpoint{0.979849in}{2.886570in}}%
\pgfpathlineto{\pgfqpoint{0.982421in}{2.870114in}}%
\pgfpathlineto{\pgfqpoint{0.987565in}{2.870114in}}%
\pgfpathlineto{\pgfqpoint{0.990137in}{2.886570in}}%
\pgfpathlineto{\pgfqpoint{0.992709in}{2.886570in}}%
\pgfpathlineto{\pgfqpoint{0.995281in}{2.903026in}}%
\pgfpathlineto{\pgfqpoint{1.002996in}{2.903026in}}%
\pgfpathlineto{\pgfqpoint{1.005568in}{2.919482in}}%
\pgfpathlineto{\pgfqpoint{1.008140in}{2.919482in}}%
\pgfpathlineto{\pgfqpoint{1.010712in}{2.935938in}}%
\pgfpathlineto{\pgfqpoint{1.015856in}{2.935938in}}%
\pgfpathlineto{\pgfqpoint{1.018427in}{2.968851in}}%
\pgfpathlineto{\pgfqpoint{1.023571in}{2.968851in}}%
\pgfpathlineto{\pgfqpoint{1.026143in}{2.952395in}}%
\pgfpathlineto{\pgfqpoint{1.028715in}{2.968851in}}%
\pgfpathlineto{\pgfqpoint{1.031287in}{2.952395in}}%
\pgfpathlineto{\pgfqpoint{1.036431in}{2.952395in}}%
\pgfpathlineto{\pgfqpoint{1.041574in}{2.919482in}}%
\pgfpathlineto{\pgfqpoint{1.054434in}{2.919482in}}%
\pgfpathlineto{\pgfqpoint{1.057006in}{2.952395in}}%
\pgfpathlineto{\pgfqpoint{1.059577in}{2.952395in}}%
\pgfpathlineto{\pgfqpoint{1.062149in}{2.968851in}}%
\pgfpathlineto{\pgfqpoint{1.067293in}{2.968851in}}%
\pgfpathlineto{\pgfqpoint{1.069865in}{2.985307in}}%
\pgfpathlineto{\pgfqpoint{1.075009in}{2.985307in}}%
\pgfpathlineto{\pgfqpoint{1.080152in}{2.952395in}}%
\pgfpathlineto{\pgfqpoint{1.082724in}{2.952395in}}%
\pgfpathlineto{\pgfqpoint{1.087868in}{2.985307in}}%
\pgfpathlineto{\pgfqpoint{1.090440in}{2.985307in}}%
\pgfpathlineto{\pgfqpoint{1.093012in}{3.051131in}}%
\pgfpathlineto{\pgfqpoint{1.098155in}{3.018219in}}%
\pgfpathlineto{\pgfqpoint{1.100727in}{2.985307in}}%
\pgfpathlineto{\pgfqpoint{1.108443in}{2.985307in}}%
\pgfpathlineto{\pgfqpoint{1.111015in}{3.018219in}}%
\pgfpathlineto{\pgfqpoint{1.113587in}{3.018219in}}%
\pgfpathlineto{\pgfqpoint{1.116159in}{3.034675in}}%
\pgfpathlineto{\pgfqpoint{1.118730in}{3.034675in}}%
\pgfpathlineto{\pgfqpoint{1.121302in}{3.051131in}}%
\pgfpathlineto{\pgfqpoint{1.131590in}{3.051131in}}%
\pgfpathlineto{\pgfqpoint{1.134162in}{3.034675in}}%
\pgfpathlineto{\pgfqpoint{1.139305in}{3.067587in}}%
\pgfpathlineto{\pgfqpoint{1.141877in}{3.067587in}}%
\pgfpathlineto{\pgfqpoint{1.144449in}{3.034675in}}%
\pgfpathlineto{\pgfqpoint{1.147021in}{3.034675in}}%
\pgfpathlineto{\pgfqpoint{1.149593in}{3.018219in}}%
\pgfpathlineto{\pgfqpoint{1.152165in}{2.985307in}}%
\pgfpathlineto{\pgfqpoint{1.157308in}{2.985307in}}%
\pgfpathlineto{\pgfqpoint{1.159880in}{2.968851in}}%
\pgfpathlineto{\pgfqpoint{1.170168in}{2.968851in}}%
\pgfpathlineto{\pgfqpoint{1.177883in}{2.919482in}}%
\pgfpathlineto{\pgfqpoint{1.188171in}{2.919482in}}%
\pgfpathlineto{\pgfqpoint{1.190743in}{2.935938in}}%
\pgfpathlineto{\pgfqpoint{1.193315in}{2.935938in}}%
\pgfpathlineto{\pgfqpoint{1.195887in}{2.919482in}}%
\pgfpathlineto{\pgfqpoint{1.198458in}{2.919482in}}%
\pgfpathlineto{\pgfqpoint{1.201030in}{2.952395in}}%
\pgfpathlineto{\pgfqpoint{1.203602in}{2.919482in}}%
\pgfpathlineto{\pgfqpoint{1.206174in}{2.903026in}}%
\pgfpathlineto{\pgfqpoint{1.213890in}{2.903026in}}%
\pgfpathlineto{\pgfqpoint{1.216461in}{2.952395in}}%
\pgfpathlineto{\pgfqpoint{1.219033in}{2.968851in}}%
\pgfpathlineto{\pgfqpoint{1.221605in}{2.952395in}}%
\pgfpathlineto{\pgfqpoint{1.226749in}{2.952395in}}%
\pgfpathlineto{\pgfqpoint{1.229321in}{2.935938in}}%
\pgfpathlineto{\pgfqpoint{1.234465in}{2.935938in}}%
\pgfpathlineto{\pgfqpoint{1.237036in}{2.919482in}}%
\pgfpathlineto{\pgfqpoint{1.247324in}{2.919482in}}%
\pgfpathlineto{\pgfqpoint{1.249896in}{2.903026in}}%
\pgfpathlineto{\pgfqpoint{1.252468in}{2.919482in}}%
\pgfpathlineto{\pgfqpoint{1.255040in}{2.903026in}}%
\pgfpathlineto{\pgfqpoint{1.260183in}{2.903026in}}%
\pgfpathlineto{\pgfqpoint{1.262755in}{2.935938in}}%
\pgfpathlineto{\pgfqpoint{1.273043in}{2.935938in}}%
\pgfpathlineto{\pgfqpoint{1.275615in}{2.919482in}}%
\pgfpathlineto{\pgfqpoint{1.278186in}{2.952395in}}%
\pgfpathlineto{\pgfqpoint{1.283330in}{2.952395in}}%
\pgfpathlineto{\pgfqpoint{1.285902in}{2.968851in}}%
\pgfpathlineto{\pgfqpoint{1.288474in}{2.968851in}}%
\pgfpathlineto{\pgfqpoint{1.291046in}{2.952395in}}%
\pgfpathlineto{\pgfqpoint{1.306477in}{2.952395in}}%
\pgfpathlineto{\pgfqpoint{1.309049in}{2.935938in}}%
\pgfpathlineto{\pgfqpoint{1.337339in}{2.935938in}}%
\pgfpathlineto{\pgfqpoint{1.339911in}{2.919482in}}%
\pgfpathlineto{\pgfqpoint{1.342483in}{2.919482in}}%
\pgfpathlineto{\pgfqpoint{1.345055in}{2.903026in}}%
\pgfpathlineto{\pgfqpoint{1.347627in}{2.903026in}}%
\pgfpathlineto{\pgfqpoint{1.350199in}{2.886570in}}%
\pgfpathlineto{\pgfqpoint{1.352771in}{2.919482in}}%
\pgfpathlineto{\pgfqpoint{1.363058in}{2.919482in}}%
\pgfpathlineto{\pgfqpoint{1.365630in}{2.952395in}}%
\pgfpathlineto{\pgfqpoint{1.368202in}{2.952395in}}%
\pgfpathlineto{\pgfqpoint{1.370774in}{2.968851in}}%
\pgfpathlineto{\pgfqpoint{1.383633in}{2.968851in}}%
\pgfpathlineto{\pgfqpoint{1.386205in}{2.952395in}}%
\pgfpathlineto{\pgfqpoint{1.409352in}{2.952395in}}%
\pgfpathlineto{\pgfqpoint{1.411924in}{2.935938in}}%
\pgfpathlineto{\pgfqpoint{1.417067in}{2.935938in}}%
\pgfpathlineto{\pgfqpoint{1.419639in}{2.919482in}}%
\pgfpathlineto{\pgfqpoint{1.422211in}{2.935938in}}%
\pgfpathlineto{\pgfqpoint{1.429927in}{2.935938in}}%
\pgfpathlineto{\pgfqpoint{1.435070in}{2.968851in}}%
\pgfpathlineto{\pgfqpoint{1.437642in}{2.952395in}}%
\pgfpathlineto{\pgfqpoint{1.442786in}{2.952395in}}%
\pgfpathlineto{\pgfqpoint{1.445358in}{2.935938in}}%
\pgfpathlineto{\pgfqpoint{1.447930in}{2.952395in}}%
\pgfpathlineto{\pgfqpoint{1.453074in}{2.952395in}}%
\pgfpathlineto{\pgfqpoint{1.455645in}{2.968851in}}%
\pgfpathlineto{\pgfqpoint{1.460789in}{2.968851in}}%
\pgfpathlineto{\pgfqpoint{1.463361in}{2.985307in}}%
\pgfpathlineto{\pgfqpoint{1.476220in}{2.985307in}}%
\pgfpathlineto{\pgfqpoint{1.478792in}{3.001763in}}%
\pgfpathlineto{\pgfqpoint{1.483936in}{3.001763in}}%
\pgfpathlineto{\pgfqpoint{1.486508in}{2.985307in}}%
\pgfpathlineto{\pgfqpoint{1.499367in}{2.985307in}}%
\pgfpathlineto{\pgfqpoint{1.501939in}{3.018219in}}%
\pgfpathlineto{\pgfqpoint{1.504511in}{3.034675in}}%
\pgfpathlineto{\pgfqpoint{1.509655in}{3.034675in}}%
\pgfpathlineto{\pgfqpoint{1.517370in}{2.985307in}}%
\pgfpathlineto{\pgfqpoint{1.519942in}{2.985307in}}%
\pgfpathlineto{\pgfqpoint{1.522514in}{3.001763in}}%
\pgfpathlineto{\pgfqpoint{1.525086in}{3.001763in}}%
\pgfpathlineto{\pgfqpoint{1.530230in}{3.034675in}}%
\pgfpathlineto{\pgfqpoint{1.532802in}{3.034675in}}%
\pgfpathlineto{\pgfqpoint{1.535373in}{3.018219in}}%
\pgfpathlineto{\pgfqpoint{1.537945in}{3.018219in}}%
\pgfpathlineto{\pgfqpoint{1.545661in}{3.067587in}}%
\pgfpathlineto{\pgfqpoint{1.555948in}{3.067587in}}%
\pgfpathlineto{\pgfqpoint{1.558520in}{3.051131in}}%
\pgfpathlineto{\pgfqpoint{1.561092in}{3.051131in}}%
\pgfpathlineto{\pgfqpoint{1.566236in}{3.018219in}}%
\pgfpathlineto{\pgfqpoint{1.571380in}{3.018219in}}%
\pgfpathlineto{\pgfqpoint{1.576523in}{2.985307in}}%
\pgfpathlineto{\pgfqpoint{1.579095in}{2.985307in}}%
\pgfpathlineto{\pgfqpoint{1.581667in}{2.968851in}}%
\pgfpathlineto{\pgfqpoint{1.591955in}{2.968851in}}%
\pgfpathlineto{\pgfqpoint{1.594526in}{3.001763in}}%
\pgfpathlineto{\pgfqpoint{1.599670in}{3.001763in}}%
\pgfpathlineto{\pgfqpoint{1.604814in}{2.968851in}}%
\pgfpathlineto{\pgfqpoint{1.607386in}{2.968851in}}%
\pgfpathlineto{\pgfqpoint{1.609958in}{2.952395in}}%
\pgfpathlineto{\pgfqpoint{1.612529in}{2.952395in}}%
\pgfpathlineto{\pgfqpoint{1.615101in}{2.935938in}}%
\pgfpathlineto{\pgfqpoint{1.617673in}{2.935938in}}%
\pgfpathlineto{\pgfqpoint{1.620245in}{2.919482in}}%
\pgfpathlineto{\pgfqpoint{1.622817in}{2.886570in}}%
\pgfpathlineto{\pgfqpoint{1.625389in}{2.886570in}}%
\pgfpathlineto{\pgfqpoint{1.627961in}{2.870114in}}%
\pgfpathlineto{\pgfqpoint{1.653679in}{2.870114in}}%
\pgfpathlineto{\pgfqpoint{1.656251in}{2.853658in}}%
\pgfpathlineto{\pgfqpoint{1.707689in}{2.853658in}}%
\pgfpathlineto{\pgfqpoint{1.712832in}{2.886570in}}%
\pgfpathlineto{\pgfqpoint{1.720548in}{2.886570in}}%
\pgfpathlineto{\pgfqpoint{1.723120in}{2.903026in}}%
\pgfpathlineto{\pgfqpoint{1.725692in}{2.935938in}}%
\pgfpathlineto{\pgfqpoint{1.730836in}{2.935938in}}%
\pgfpathlineto{\pgfqpoint{1.733407in}{2.919482in}}%
\pgfpathlineto{\pgfqpoint{1.748839in}{2.919482in}}%
\pgfpathlineto{\pgfqpoint{1.751410in}{2.935938in}}%
\pgfpathlineto{\pgfqpoint{1.753982in}{2.919482in}}%
\pgfpathlineto{\pgfqpoint{1.759126in}{2.919482in}}%
\pgfpathlineto{\pgfqpoint{1.761698in}{2.935938in}}%
\pgfpathlineto{\pgfqpoint{1.769414in}{2.935938in}}%
\pgfpathlineto{\pgfqpoint{1.771985in}{2.952395in}}%
\pgfpathlineto{\pgfqpoint{1.774557in}{2.952395in}}%
\pgfpathlineto{\pgfqpoint{1.777129in}{2.968851in}}%
\pgfpathlineto{\pgfqpoint{1.779701in}{2.968851in}}%
\pgfpathlineto{\pgfqpoint{1.782273in}{2.952395in}}%
\pgfpathlineto{\pgfqpoint{1.797704in}{2.952395in}}%
\pgfpathlineto{\pgfqpoint{1.800276in}{2.935938in}}%
\pgfpathlineto{\pgfqpoint{1.810563in}{2.935938in}}%
\pgfpathlineto{\pgfqpoint{1.813135in}{2.919482in}}%
\pgfpathlineto{\pgfqpoint{1.815707in}{2.919482in}}%
\pgfpathlineto{\pgfqpoint{1.818279in}{2.903026in}}%
\pgfpathlineto{\pgfqpoint{1.823423in}{2.903026in}}%
\pgfpathlineto{\pgfqpoint{1.825995in}{2.886570in}}%
\pgfpathlineto{\pgfqpoint{1.828567in}{2.886570in}}%
\pgfpathlineto{\pgfqpoint{1.831138in}{2.903026in}}%
\pgfpathlineto{\pgfqpoint{1.841426in}{2.903026in}}%
\pgfpathlineto{\pgfqpoint{1.843998in}{2.919482in}}%
\pgfpathlineto{\pgfqpoint{1.859429in}{2.919482in}}%
\pgfpathlineto{\pgfqpoint{1.862001in}{2.903026in}}%
\pgfpathlineto{\pgfqpoint{1.864573in}{2.903026in}}%
\pgfpathlineto{\pgfqpoint{1.867145in}{2.919482in}}%
\pgfpathlineto{\pgfqpoint{1.877432in}{2.919482in}}%
\pgfpathlineto{\pgfqpoint{1.880004in}{2.903026in}}%
\pgfpathlineto{\pgfqpoint{1.887720in}{2.903026in}}%
\pgfpathlineto{\pgfqpoint{1.890291in}{2.919482in}}%
\pgfpathlineto{\pgfqpoint{1.892863in}{2.919482in}}%
\pgfpathlineto{\pgfqpoint{1.895435in}{2.903026in}}%
\pgfpathlineto{\pgfqpoint{1.905723in}{2.903026in}}%
\pgfpathlineto{\pgfqpoint{1.908295in}{2.886570in}}%
\pgfpathlineto{\pgfqpoint{1.916010in}{2.886570in}}%
\pgfpathlineto{\pgfqpoint{1.918582in}{2.870114in}}%
\pgfpathlineto{\pgfqpoint{1.921154in}{2.886570in}}%
\pgfpathlineto{\pgfqpoint{1.928870in}{2.886570in}}%
\pgfpathlineto{\pgfqpoint{1.931441in}{2.870114in}}%
\pgfpathlineto{\pgfqpoint{1.934013in}{2.886570in}}%
\pgfpathlineto{\pgfqpoint{1.949444in}{2.886570in}}%
\pgfpathlineto{\pgfqpoint{1.952016in}{2.903026in}}%
\pgfpathlineto{\pgfqpoint{1.954588in}{2.903026in}}%
\pgfpathlineto{\pgfqpoint{1.957160in}{2.886570in}}%
\pgfpathlineto{\pgfqpoint{1.962304in}{2.886570in}}%
\pgfpathlineto{\pgfqpoint{1.964876in}{2.903026in}}%
\pgfpathlineto{\pgfqpoint{1.967448in}{2.903026in}}%
\pgfpathlineto{\pgfqpoint{1.970019in}{2.919482in}}%
\pgfpathlineto{\pgfqpoint{1.993166in}{2.919482in}}%
\pgfpathlineto{\pgfqpoint{1.995738in}{2.903026in}}%
\pgfpathlineto{\pgfqpoint{1.998310in}{2.903026in}}%
\pgfpathlineto{\pgfqpoint{2.000882in}{2.886570in}}%
\pgfpathlineto{\pgfqpoint{2.011169in}{2.886570in}}%
\pgfpathlineto{\pgfqpoint{2.013741in}{2.903026in}}%
\pgfpathlineto{\pgfqpoint{2.016313in}{2.903026in}}%
\pgfpathlineto{\pgfqpoint{2.018885in}{2.870114in}}%
\pgfpathlineto{\pgfqpoint{2.042032in}{2.870114in}}%
\pgfpathlineto{\pgfqpoint{2.044604in}{2.886570in}}%
\pgfpathlineto{\pgfqpoint{2.057463in}{2.886570in}}%
\pgfpathlineto{\pgfqpoint{2.060035in}{2.903026in}}%
\pgfpathlineto{\pgfqpoint{2.072894in}{2.903026in}}%
\pgfpathlineto{\pgfqpoint{2.075466in}{2.886570in}}%
\pgfpathlineto{\pgfqpoint{2.080610in}{2.886570in}}%
\pgfpathlineto{\pgfqpoint{2.083182in}{2.870114in}}%
\pgfpathlineto{\pgfqpoint{2.108900in}{2.870114in}}%
\pgfpathlineto{\pgfqpoint{2.111472in}{2.853658in}}%
\pgfpathlineto{\pgfqpoint{2.124332in}{2.853658in}}%
\pgfpathlineto{\pgfqpoint{2.126904in}{2.837202in}}%
\pgfpathlineto{\pgfqpoint{2.134619in}{2.837202in}}%
\pgfpathlineto{\pgfqpoint{2.137191in}{2.820746in}}%
\pgfpathlineto{\pgfqpoint{2.178341in}{2.820746in}}%
\pgfpathlineto{\pgfqpoint{2.180913in}{2.837202in}}%
\pgfpathlineto{\pgfqpoint{2.183485in}{2.820746in}}%
\pgfpathlineto{\pgfqpoint{2.191200in}{2.820746in}}%
\pgfpathlineto{\pgfqpoint{2.193772in}{2.837202in}}%
\pgfpathlineto{\pgfqpoint{2.229778in}{2.837202in}}%
\pgfpathlineto{\pgfqpoint{2.232350in}{2.853658in}}%
\pgfpathlineto{\pgfqpoint{2.234922in}{2.837202in}}%
\pgfpathlineto{\pgfqpoint{2.242638in}{2.837202in}}%
\pgfpathlineto{\pgfqpoint{2.245210in}{2.853658in}}%
\pgfpathlineto{\pgfqpoint{2.252925in}{2.853658in}}%
\pgfpathlineto{\pgfqpoint{2.255497in}{2.837202in}}%
\pgfpathlineto{\pgfqpoint{2.263213in}{2.837202in}}%
\pgfpathlineto{\pgfqpoint{2.265785in}{2.820746in}}%
\pgfpathlineto{\pgfqpoint{2.268356in}{2.820746in}}%
\pgfpathlineto{\pgfqpoint{2.270928in}{2.804290in}}%
\pgfpathlineto{\pgfqpoint{2.278644in}{2.804290in}}%
\pgfpathlineto{\pgfqpoint{2.281216in}{2.787834in}}%
\pgfpathlineto{\pgfqpoint{2.283788in}{2.787834in}}%
\pgfpathlineto{\pgfqpoint{2.286359in}{2.804290in}}%
\pgfpathlineto{\pgfqpoint{2.304363in}{2.804290in}}%
\pgfpathlineto{\pgfqpoint{2.312078in}{2.853658in}}%
\pgfpathlineto{\pgfqpoint{2.314650in}{2.853658in}}%
\pgfpathlineto{\pgfqpoint{2.317222in}{2.820746in}}%
\pgfpathlineto{\pgfqpoint{2.319794in}{2.837202in}}%
\pgfpathlineto{\pgfqpoint{2.342941in}{2.837202in}}%
\pgfpathlineto{\pgfqpoint{2.345512in}{2.853658in}}%
\pgfpathlineto{\pgfqpoint{2.368659in}{2.853658in}}%
\pgfpathlineto{\pgfqpoint{2.371231in}{2.870114in}}%
\pgfpathlineto{\pgfqpoint{2.394378in}{2.870114in}}%
\pgfpathlineto{\pgfqpoint{2.396950in}{2.853658in}}%
\pgfpathlineto{\pgfqpoint{2.402094in}{2.853658in}}%
\pgfpathlineto{\pgfqpoint{2.404665in}{2.837202in}}%
\pgfpathlineto{\pgfqpoint{2.425240in}{2.837202in}}%
\pgfpathlineto{\pgfqpoint{2.427812in}{2.853658in}}%
\pgfpathlineto{\pgfqpoint{2.432956in}{2.853658in}}%
\pgfpathlineto{\pgfqpoint{2.438100in}{2.820746in}}%
\pgfpathlineto{\pgfqpoint{2.440672in}{2.820746in}}%
\pgfpathlineto{\pgfqpoint{2.443244in}{2.804290in}}%
\pgfpathlineto{\pgfqpoint{2.456103in}{2.804290in}}%
\pgfpathlineto{\pgfqpoint{2.458675in}{2.787834in}}%
\pgfpathlineto{\pgfqpoint{2.471534in}{2.787834in}}%
\pgfpathlineto{\pgfqpoint{2.474106in}{2.771378in}}%
\pgfpathlineto{\pgfqpoint{2.517828in}{2.771378in}}%
\pgfpathlineto{\pgfqpoint{2.520400in}{2.787834in}}%
\pgfpathlineto{\pgfqpoint{2.543546in}{2.787834in}}%
\pgfpathlineto{\pgfqpoint{2.546118in}{2.804290in}}%
\pgfpathlineto{\pgfqpoint{2.576981in}{2.804290in}}%
\pgfpathlineto{\pgfqpoint{2.579553in}{2.820746in}}%
\pgfpathlineto{\pgfqpoint{2.592412in}{2.820746in}}%
\pgfpathlineto{\pgfqpoint{2.594984in}{2.804290in}}%
\pgfpathlineto{\pgfqpoint{2.612987in}{2.804290in}}%
\pgfpathlineto{\pgfqpoint{2.618131in}{2.837202in}}%
\pgfpathlineto{\pgfqpoint{2.623274in}{2.837202in}}%
\pgfpathlineto{\pgfqpoint{2.625846in}{2.870114in}}%
\pgfpathlineto{\pgfqpoint{2.643849in}{2.870114in}}%
\pgfpathlineto{\pgfqpoint{2.646421in}{2.886570in}}%
\pgfpathlineto{\pgfqpoint{2.646421in}{2.886570in}}%
\pgfusepath{stroke}%
\end{pgfscope}%
\begin{pgfscope}%
\pgfsetrectcap%
\pgfsetmiterjoin%
\pgfsetlinewidth{0.501875pt}%
\definecolor{currentstroke}{rgb}{0.317647,0.317647,0.317647}%
\pgfsetstrokecolor{currentstroke}%
\pgfsetdash{}{0pt}%
\pgfpathmoveto{\pgfqpoint{0.488751in}{1.946106in}}%
\pgfpathlineto{\pgfqpoint{0.488751in}{3.448545in}}%
\pgfusepath{stroke}%
\end{pgfscope}%
\begin{pgfscope}%
\pgfsetrectcap%
\pgfsetmiterjoin%
\pgfsetlinewidth{0.501875pt}%
\definecolor{currentstroke}{rgb}{0.317647,0.317647,0.317647}%
\pgfsetstrokecolor{currentstroke}%
\pgfsetdash{}{0pt}%
\pgfpathmoveto{\pgfqpoint{0.488751in}{1.946106in}}%
\pgfpathlineto{\pgfqpoint{2.749168in}{1.946106in}}%
\pgfusepath{stroke}%
\end{pgfscope}%
\begin{pgfscope}%
\pgfsetrectcap%
\pgfsetroundjoin%
\pgfsetlinewidth{0.803000pt}%
\definecolor{currentstroke}{rgb}{0.333333,0.333333,0.333333}%
\pgfsetstrokecolor{currentstroke}%
\pgfsetdash{}{0pt}%
\pgfpathmoveto{\pgfqpoint{2.703959in}{3.262426in}}%
\pgfpathlineto{\pgfqpoint{2.770626in}{3.262426in}}%
\pgfusepath{stroke}%
\end{pgfscope}%
\begin{pgfscope}%
\definecolor{textcolor}{rgb}{0.000000,0.000000,0.000000}%
\pgfsetstrokecolor{textcolor}%
\pgfsetfillcolor{textcolor}%
\pgftext[x=2.812293in,y=3.233259in,left,base]{\color{textcolor}\rmfamily\fontsize{6.000000}{7.200000}\selectfont \(\displaystyle w_{00}\)}%
\end{pgfscope}%
\begin{pgfscope}%
\pgfsetrectcap%
\pgfsetroundjoin%
\pgfsetlinewidth{0.803000pt}%
\definecolor{currentstroke}{rgb}{0.686275,0.352941,0.313725}%
\pgfsetstrokecolor{currentstroke}%
\pgfsetdash{}{0pt}%
\pgfpathmoveto{\pgfqpoint{2.703959in}{3.154593in}}%
\pgfpathlineto{\pgfqpoint{2.770626in}{3.154593in}}%
\pgfusepath{stroke}%
\end{pgfscope}%
\begin{pgfscope}%
\definecolor{textcolor}{rgb}{0.000000,0.000000,0.000000}%
\pgfsetstrokecolor{textcolor}%
\pgfsetfillcolor{textcolor}%
\pgftext[x=2.812293in,y=3.125426in,left,base]{\color{textcolor}\rmfamily\fontsize{6.000000}{7.200000}\selectfont \(\displaystyle w_{01}\)}%
\end{pgfscope}%
\begin{pgfscope}%
\pgfsetrectcap%
\pgfsetroundjoin%
\pgfsetlinewidth{0.803000pt}%
\definecolor{currentstroke}{rgb}{0.000000,0.356863,0.509804}%
\pgfsetstrokecolor{currentstroke}%
\pgfsetdash{}{0pt}%
\pgfpathmoveto{\pgfqpoint{2.703959in}{3.046759in}}%
\pgfpathlineto{\pgfqpoint{2.770626in}{3.046759in}}%
\pgfusepath{stroke}%
\end{pgfscope}%
\begin{pgfscope}%
\definecolor{textcolor}{rgb}{0.000000,0.000000,0.000000}%
\pgfsetstrokecolor{textcolor}%
\pgfsetfillcolor{textcolor}%
\pgftext[x=2.812293in,y=3.017593in,left,base]{\color{textcolor}\rmfamily\fontsize{6.000000}{7.200000}\selectfont \(\displaystyle w_{02}\)}%
\end{pgfscope}%
\begin{pgfscope}%
\pgfsetrectcap%
\pgfsetroundjoin%
\pgfsetlinewidth{0.803000pt}%
\definecolor{currentstroke}{rgb}{0.490196,0.588235,0.431373}%
\pgfsetstrokecolor{currentstroke}%
\pgfsetdash{}{0pt}%
\pgfpathmoveto{\pgfqpoint{2.703959in}{2.938926in}}%
\pgfpathlineto{\pgfqpoint{2.770626in}{2.938926in}}%
\pgfusepath{stroke}%
\end{pgfscope}%
\begin{pgfscope}%
\definecolor{textcolor}{rgb}{0.000000,0.000000,0.000000}%
\pgfsetstrokecolor{textcolor}%
\pgfsetfillcolor{textcolor}%
\pgftext[x=2.812293in,y=2.909759in,left,base]{\color{textcolor}\rmfamily\fontsize{6.000000}{7.200000}\selectfont \(\displaystyle w_{03}\)}%
\end{pgfscope}%
\begin{pgfscope}%
\pgfsetrectcap%
\pgfsetroundjoin%
\pgfsetlinewidth{0.803000pt}%
\definecolor{currentstroke}{rgb}{0.843137,0.666667,0.313725}%
\pgfsetstrokecolor{currentstroke}%
\pgfsetdash{}{0pt}%
\pgfpathmoveto{\pgfqpoint{2.703959in}{2.831093in}}%
\pgfpathlineto{\pgfqpoint{2.770626in}{2.831093in}}%
\pgfusepath{stroke}%
\end{pgfscope}%
\begin{pgfscope}%
\definecolor{textcolor}{rgb}{0.000000,0.000000,0.000000}%
\pgfsetstrokecolor{textcolor}%
\pgfsetfillcolor{textcolor}%
\pgftext[x=2.812293in,y=2.801926in,left,base]{\color{textcolor}\rmfamily\fontsize{6.000000}{7.200000}\selectfont \(\displaystyle w_{04}\)}%
\end{pgfscope}%
\begin{pgfscope}%
\pgfsetbuttcap%
\pgfsetroundjoin%
\pgfsetlinewidth{0.803000pt}%
\definecolor{currentstroke}{rgb}{0.333333,0.333333,0.333333}%
\pgfsetstrokecolor{currentstroke}%
\pgfsetdash{{2.960000pt}{1.280000pt}}{0.000000pt}%
\pgfpathmoveto{\pgfqpoint{2.703959in}{2.723260in}}%
\pgfpathlineto{\pgfqpoint{2.770626in}{2.723260in}}%
\pgfusepath{stroke}%
\end{pgfscope}%
\begin{pgfscope}%
\definecolor{textcolor}{rgb}{0.000000,0.000000,0.000000}%
\pgfsetstrokecolor{textcolor}%
\pgfsetfillcolor{textcolor}%
\pgftext[x=2.812293in,y=2.694093in,left,base]{\color{textcolor}\rmfamily\fontsize{6.000000}{7.200000}\selectfont \(\displaystyle w_{10}\)}%
\end{pgfscope}%
\begin{pgfscope}%
\pgfsetbuttcap%
\pgfsetroundjoin%
\pgfsetlinewidth{0.803000pt}%
\definecolor{currentstroke}{rgb}{0.686275,0.352941,0.313725}%
\pgfsetstrokecolor{currentstroke}%
\pgfsetdash{{2.960000pt}{1.280000pt}}{0.000000pt}%
\pgfpathmoveto{\pgfqpoint{2.703959in}{2.615427in}}%
\pgfpathlineto{\pgfqpoint{2.770626in}{2.615427in}}%
\pgfusepath{stroke}%
\end{pgfscope}%
\begin{pgfscope}%
\definecolor{textcolor}{rgb}{0.000000,0.000000,0.000000}%
\pgfsetstrokecolor{textcolor}%
\pgfsetfillcolor{textcolor}%
\pgftext[x=2.812293in,y=2.586260in,left,base]{\color{textcolor}\rmfamily\fontsize{6.000000}{7.200000}\selectfont \(\displaystyle w_{11}\)}%
\end{pgfscope}%
\begin{pgfscope}%
\pgfsetbuttcap%
\pgfsetroundjoin%
\pgfsetlinewidth{0.803000pt}%
\definecolor{currentstroke}{rgb}{0.000000,0.356863,0.509804}%
\pgfsetstrokecolor{currentstroke}%
\pgfsetdash{{2.960000pt}{1.280000pt}}{0.000000pt}%
\pgfpathmoveto{\pgfqpoint{2.703959in}{2.507593in}}%
\pgfpathlineto{\pgfqpoint{2.770626in}{2.507593in}}%
\pgfusepath{stroke}%
\end{pgfscope}%
\begin{pgfscope}%
\definecolor{textcolor}{rgb}{0.000000,0.000000,0.000000}%
\pgfsetstrokecolor{textcolor}%
\pgfsetfillcolor{textcolor}%
\pgftext[x=2.812293in,y=2.478427in,left,base]{\color{textcolor}\rmfamily\fontsize{6.000000}{7.200000}\selectfont \(\displaystyle w_{12}\)}%
\end{pgfscope}%
\begin{pgfscope}%
\pgfsetbuttcap%
\pgfsetroundjoin%
\pgfsetlinewidth{0.803000pt}%
\definecolor{currentstroke}{rgb}{0.490196,0.588235,0.431373}%
\pgfsetstrokecolor{currentstroke}%
\pgfsetdash{{2.960000pt}{1.280000pt}}{0.000000pt}%
\pgfpathmoveto{\pgfqpoint{2.703959in}{2.399760in}}%
\pgfpathlineto{\pgfqpoint{2.770626in}{2.399760in}}%
\pgfusepath{stroke}%
\end{pgfscope}%
\begin{pgfscope}%
\definecolor{textcolor}{rgb}{0.000000,0.000000,0.000000}%
\pgfsetstrokecolor{textcolor}%
\pgfsetfillcolor{textcolor}%
\pgftext[x=2.812293in,y=2.370594in,left,base]{\color{textcolor}\rmfamily\fontsize{6.000000}{7.200000}\selectfont \(\displaystyle w_{13}\)}%
\end{pgfscope}%
\begin{pgfscope}%
\pgfsetbuttcap%
\pgfsetroundjoin%
\pgfsetlinewidth{0.803000pt}%
\definecolor{currentstroke}{rgb}{0.843137,0.666667,0.313725}%
\pgfsetstrokecolor{currentstroke}%
\pgfsetdash{{2.960000pt}{1.280000pt}}{0.000000pt}%
\pgfpathmoveto{\pgfqpoint{2.703959in}{2.291927in}}%
\pgfpathlineto{\pgfqpoint{2.770626in}{2.291927in}}%
\pgfusepath{stroke}%
\end{pgfscope}%
\begin{pgfscope}%
\definecolor{textcolor}{rgb}{0.000000,0.000000,0.000000}%
\pgfsetstrokecolor{textcolor}%
\pgfsetfillcolor{textcolor}%
\pgftext[x=2.812293in,y=2.262760in,left,base]{\color{textcolor}\rmfamily\fontsize{6.000000}{7.200000}\selectfont \(\displaystyle w_{14}\)}%
\end{pgfscope}%
\begin{pgfscope}%
\pgfsetbuttcap%
\pgfsetmiterjoin%
\pgfsetlinewidth{0.000000pt}%
\definecolor{currentstroke}{rgb}{0.000000,0.000000,0.000000}%
\pgfsetstrokecolor{currentstroke}%
\pgfsetstrokeopacity{0.000000}%
\pgfsetdash{}{0pt}%
\pgfpathmoveto{\pgfqpoint{3.653334in}{1.946106in}}%
\pgfpathlineto{\pgfqpoint{5.913751in}{1.946106in}}%
\pgfpathlineto{\pgfqpoint{5.913751in}{3.448545in}}%
\pgfpathlineto{\pgfqpoint{3.653334in}{3.448545in}}%
\pgfpathclose%
\pgfusepath{}%
\end{pgfscope}%
\begin{pgfscope}%
\pgfsetbuttcap%
\pgfsetroundjoin%
\definecolor{currentfill}{rgb}{0.317647,0.317647,0.317647}%
\pgfsetfillcolor{currentfill}%
\pgfsetlinewidth{0.501875pt}%
\definecolor{currentstroke}{rgb}{0.317647,0.317647,0.317647}%
\pgfsetstrokecolor{currentstroke}%
\pgfsetdash{}{0pt}%
\pgfsys@defobject{currentmarker}{\pgfqpoint{0.000000in}{-0.020833in}}{\pgfqpoint{0.000000in}{0.000000in}}{%
\pgfpathmoveto{\pgfqpoint{0.000000in}{0.000000in}}%
\pgfpathlineto{\pgfqpoint{0.000000in}{-0.020833in}}%
\pgfusepath{stroke,fill}%
}%
\begin{pgfscope}%
\pgfsys@transformshift{3.756080in}{1.946106in}%
\pgfsys@useobject{currentmarker}{}%
\end{pgfscope}%
\end{pgfscope}%
\begin{pgfscope}%
\pgfsetbuttcap%
\pgfsetroundjoin%
\definecolor{currentfill}{rgb}{0.317647,0.317647,0.317647}%
\pgfsetfillcolor{currentfill}%
\pgfsetlinewidth{0.501875pt}%
\definecolor{currentstroke}{rgb}{0.317647,0.317647,0.317647}%
\pgfsetstrokecolor{currentstroke}%
\pgfsetdash{}{0pt}%
\pgfsys@defobject{currentmarker}{\pgfqpoint{0.000000in}{-0.020833in}}{\pgfqpoint{0.000000in}{0.000000in}}{%
\pgfpathmoveto{\pgfqpoint{0.000000in}{0.000000in}}%
\pgfpathlineto{\pgfqpoint{0.000000in}{-0.020833in}}%
\pgfusepath{stroke,fill}%
}%
\begin{pgfscope}%
\pgfsys@transformshift{4.270454in}{1.946106in}%
\pgfsys@useobject{currentmarker}{}%
\end{pgfscope}%
\end{pgfscope}%
\begin{pgfscope}%
\pgfsetbuttcap%
\pgfsetroundjoin%
\definecolor{currentfill}{rgb}{0.317647,0.317647,0.317647}%
\pgfsetfillcolor{currentfill}%
\pgfsetlinewidth{0.501875pt}%
\definecolor{currentstroke}{rgb}{0.317647,0.317647,0.317647}%
\pgfsetstrokecolor{currentstroke}%
\pgfsetdash{}{0pt}%
\pgfsys@defobject{currentmarker}{\pgfqpoint{0.000000in}{-0.020833in}}{\pgfqpoint{0.000000in}{0.000000in}}{%
\pgfpathmoveto{\pgfqpoint{0.000000in}{0.000000in}}%
\pgfpathlineto{\pgfqpoint{0.000000in}{-0.020833in}}%
\pgfusepath{stroke,fill}%
}%
\begin{pgfscope}%
\pgfsys@transformshift{4.784828in}{1.946106in}%
\pgfsys@useobject{currentmarker}{}%
\end{pgfscope}%
\end{pgfscope}%
\begin{pgfscope}%
\pgfsetbuttcap%
\pgfsetroundjoin%
\definecolor{currentfill}{rgb}{0.317647,0.317647,0.317647}%
\pgfsetfillcolor{currentfill}%
\pgfsetlinewidth{0.501875pt}%
\definecolor{currentstroke}{rgb}{0.317647,0.317647,0.317647}%
\pgfsetstrokecolor{currentstroke}%
\pgfsetdash{}{0pt}%
\pgfsys@defobject{currentmarker}{\pgfqpoint{0.000000in}{-0.020833in}}{\pgfqpoint{0.000000in}{0.000000in}}{%
\pgfpathmoveto{\pgfqpoint{0.000000in}{0.000000in}}%
\pgfpathlineto{\pgfqpoint{0.000000in}{-0.020833in}}%
\pgfusepath{stroke,fill}%
}%
\begin{pgfscope}%
\pgfsys@transformshift{5.299202in}{1.946106in}%
\pgfsys@useobject{currentmarker}{}%
\end{pgfscope}%
\end{pgfscope}%
\begin{pgfscope}%
\pgfsetbuttcap%
\pgfsetroundjoin%
\definecolor{currentfill}{rgb}{0.317647,0.317647,0.317647}%
\pgfsetfillcolor{currentfill}%
\pgfsetlinewidth{0.501875pt}%
\definecolor{currentstroke}{rgb}{0.317647,0.317647,0.317647}%
\pgfsetstrokecolor{currentstroke}%
\pgfsetdash{}{0pt}%
\pgfsys@defobject{currentmarker}{\pgfqpoint{0.000000in}{-0.020833in}}{\pgfqpoint{0.000000in}{0.000000in}}{%
\pgfpathmoveto{\pgfqpoint{0.000000in}{0.000000in}}%
\pgfpathlineto{\pgfqpoint{0.000000in}{-0.020833in}}%
\pgfusepath{stroke,fill}%
}%
\begin{pgfscope}%
\pgfsys@transformshift{5.813576in}{1.946106in}%
\pgfsys@useobject{currentmarker}{}%
\end{pgfscope}%
\end{pgfscope}%
\begin{pgfscope}%
\pgfsetbuttcap%
\pgfsetroundjoin%
\definecolor{currentfill}{rgb}{0.317647,0.317647,0.317647}%
\pgfsetfillcolor{currentfill}%
\pgfsetlinewidth{0.501875pt}%
\definecolor{currentstroke}{rgb}{0.317647,0.317647,0.317647}%
\pgfsetstrokecolor{currentstroke}%
\pgfsetdash{}{0pt}%
\pgfsys@defobject{currentmarker}{\pgfqpoint{-0.020833in}{0.000000in}}{\pgfqpoint{0.000000in}{0.000000in}}{%
\pgfpathmoveto{\pgfqpoint{0.000000in}{0.000000in}}%
\pgfpathlineto{\pgfqpoint{-0.020833in}{0.000000in}}%
\pgfusepath{stroke,fill}%
}%
\begin{pgfscope}%
\pgfsys@transformshift{3.653334in}{2.116328in}%
\pgfsys@useobject{currentmarker}{}%
\end{pgfscope}%
\end{pgfscope}%
\begin{pgfscope}%
\definecolor{textcolor}{rgb}{0.317647,0.317647,0.317647}%
\pgfsetstrokecolor{textcolor}%
\pgfsetfillcolor{textcolor}%
\pgftext[x=3.452523in,y=2.084211in,left,base]{\color{textcolor}\rmfamily\fontsize{6.664000}{7.996800}\selectfont \(\displaystyle 270\)}%
\end{pgfscope}%
\begin{pgfscope}%
\pgfsetbuttcap%
\pgfsetroundjoin%
\definecolor{currentfill}{rgb}{0.317647,0.317647,0.317647}%
\pgfsetfillcolor{currentfill}%
\pgfsetlinewidth{0.501875pt}%
\definecolor{currentstroke}{rgb}{0.317647,0.317647,0.317647}%
\pgfsetstrokecolor{currentstroke}%
\pgfsetdash{}{0pt}%
\pgfsys@defobject{currentmarker}{\pgfqpoint{-0.020833in}{0.000000in}}{\pgfqpoint{0.000000in}{0.000000in}}{%
\pgfpathmoveto{\pgfqpoint{0.000000in}{0.000000in}}%
\pgfpathlineto{\pgfqpoint{-0.020833in}{0.000000in}}%
\pgfusepath{stroke,fill}%
}%
\begin{pgfscope}%
\pgfsys@transformshift{3.653334in}{2.320187in}%
\pgfsys@useobject{currentmarker}{}%
\end{pgfscope}%
\end{pgfscope}%
\begin{pgfscope}%
\definecolor{textcolor}{rgb}{0.317647,0.317647,0.317647}%
\pgfsetstrokecolor{textcolor}%
\pgfsetfillcolor{textcolor}%
\pgftext[x=3.452523in,y=2.288070in,left,base]{\color{textcolor}\rmfamily\fontsize{6.664000}{7.996800}\selectfont \(\displaystyle 280\)}%
\end{pgfscope}%
\begin{pgfscope}%
\pgfsetbuttcap%
\pgfsetroundjoin%
\definecolor{currentfill}{rgb}{0.317647,0.317647,0.317647}%
\pgfsetfillcolor{currentfill}%
\pgfsetlinewidth{0.501875pt}%
\definecolor{currentstroke}{rgb}{0.317647,0.317647,0.317647}%
\pgfsetstrokecolor{currentstroke}%
\pgfsetdash{}{0pt}%
\pgfsys@defobject{currentmarker}{\pgfqpoint{-0.020833in}{0.000000in}}{\pgfqpoint{0.000000in}{0.000000in}}{%
\pgfpathmoveto{\pgfqpoint{0.000000in}{0.000000in}}%
\pgfpathlineto{\pgfqpoint{-0.020833in}{0.000000in}}%
\pgfusepath{stroke,fill}%
}%
\begin{pgfscope}%
\pgfsys@transformshift{3.653334in}{2.524046in}%
\pgfsys@useobject{currentmarker}{}%
\end{pgfscope}%
\end{pgfscope}%
\begin{pgfscope}%
\definecolor{textcolor}{rgb}{0.317647,0.317647,0.317647}%
\pgfsetstrokecolor{textcolor}%
\pgfsetfillcolor{textcolor}%
\pgftext[x=3.452523in,y=2.491929in,left,base]{\color{textcolor}\rmfamily\fontsize{6.664000}{7.996800}\selectfont \(\displaystyle 290\)}%
\end{pgfscope}%
\begin{pgfscope}%
\pgfsetbuttcap%
\pgfsetroundjoin%
\definecolor{currentfill}{rgb}{0.317647,0.317647,0.317647}%
\pgfsetfillcolor{currentfill}%
\pgfsetlinewidth{0.501875pt}%
\definecolor{currentstroke}{rgb}{0.317647,0.317647,0.317647}%
\pgfsetstrokecolor{currentstroke}%
\pgfsetdash{}{0pt}%
\pgfsys@defobject{currentmarker}{\pgfqpoint{-0.020833in}{0.000000in}}{\pgfqpoint{0.000000in}{0.000000in}}{%
\pgfpathmoveto{\pgfqpoint{0.000000in}{0.000000in}}%
\pgfpathlineto{\pgfqpoint{-0.020833in}{0.000000in}}%
\pgfusepath{stroke,fill}%
}%
\begin{pgfscope}%
\pgfsys@transformshift{3.653334in}{2.727904in}%
\pgfsys@useobject{currentmarker}{}%
\end{pgfscope}%
\end{pgfscope}%
\begin{pgfscope}%
\definecolor{textcolor}{rgb}{0.317647,0.317647,0.317647}%
\pgfsetstrokecolor{textcolor}%
\pgfsetfillcolor{textcolor}%
\pgftext[x=3.452523in,y=2.695787in,left,base]{\color{textcolor}\rmfamily\fontsize{6.664000}{7.996800}\selectfont \(\displaystyle 300\)}%
\end{pgfscope}%
\begin{pgfscope}%
\pgfsetbuttcap%
\pgfsetroundjoin%
\definecolor{currentfill}{rgb}{0.317647,0.317647,0.317647}%
\pgfsetfillcolor{currentfill}%
\pgfsetlinewidth{0.501875pt}%
\definecolor{currentstroke}{rgb}{0.317647,0.317647,0.317647}%
\pgfsetstrokecolor{currentstroke}%
\pgfsetdash{}{0pt}%
\pgfsys@defobject{currentmarker}{\pgfqpoint{-0.020833in}{0.000000in}}{\pgfqpoint{0.000000in}{0.000000in}}{%
\pgfpathmoveto{\pgfqpoint{0.000000in}{0.000000in}}%
\pgfpathlineto{\pgfqpoint{-0.020833in}{0.000000in}}%
\pgfusepath{stroke,fill}%
}%
\begin{pgfscope}%
\pgfsys@transformshift{3.653334in}{2.931763in}%
\pgfsys@useobject{currentmarker}{}%
\end{pgfscope}%
\end{pgfscope}%
\begin{pgfscope}%
\definecolor{textcolor}{rgb}{0.317647,0.317647,0.317647}%
\pgfsetstrokecolor{textcolor}%
\pgfsetfillcolor{textcolor}%
\pgftext[x=3.452523in,y=2.899646in,left,base]{\color{textcolor}\rmfamily\fontsize{6.664000}{7.996800}\selectfont \(\displaystyle 310\)}%
\end{pgfscope}%
\begin{pgfscope}%
\pgfsetbuttcap%
\pgfsetroundjoin%
\definecolor{currentfill}{rgb}{0.317647,0.317647,0.317647}%
\pgfsetfillcolor{currentfill}%
\pgfsetlinewidth{0.501875pt}%
\definecolor{currentstroke}{rgb}{0.317647,0.317647,0.317647}%
\pgfsetstrokecolor{currentstroke}%
\pgfsetdash{}{0pt}%
\pgfsys@defobject{currentmarker}{\pgfqpoint{-0.020833in}{0.000000in}}{\pgfqpoint{0.000000in}{0.000000in}}{%
\pgfpathmoveto{\pgfqpoint{0.000000in}{0.000000in}}%
\pgfpathlineto{\pgfqpoint{-0.020833in}{0.000000in}}%
\pgfusepath{stroke,fill}%
}%
\begin{pgfscope}%
\pgfsys@transformshift{3.653334in}{3.135622in}%
\pgfsys@useobject{currentmarker}{}%
\end{pgfscope}%
\end{pgfscope}%
\begin{pgfscope}%
\definecolor{textcolor}{rgb}{0.317647,0.317647,0.317647}%
\pgfsetstrokecolor{textcolor}%
\pgfsetfillcolor{textcolor}%
\pgftext[x=3.452523in,y=3.103505in,left,base]{\color{textcolor}\rmfamily\fontsize{6.664000}{7.996800}\selectfont \(\displaystyle 320\)}%
\end{pgfscope}%
\begin{pgfscope}%
\pgfsetbuttcap%
\pgfsetroundjoin%
\definecolor{currentfill}{rgb}{0.317647,0.317647,0.317647}%
\pgfsetfillcolor{currentfill}%
\pgfsetlinewidth{0.501875pt}%
\definecolor{currentstroke}{rgb}{0.317647,0.317647,0.317647}%
\pgfsetstrokecolor{currentstroke}%
\pgfsetdash{}{0pt}%
\pgfsys@defobject{currentmarker}{\pgfqpoint{-0.020833in}{0.000000in}}{\pgfqpoint{0.000000in}{0.000000in}}{%
\pgfpathmoveto{\pgfqpoint{0.000000in}{0.000000in}}%
\pgfpathlineto{\pgfqpoint{-0.020833in}{0.000000in}}%
\pgfusepath{stroke,fill}%
}%
\begin{pgfscope}%
\pgfsys@transformshift{3.653334in}{3.339481in}%
\pgfsys@useobject{currentmarker}{}%
\end{pgfscope}%
\end{pgfscope}%
\begin{pgfscope}%
\definecolor{textcolor}{rgb}{0.317647,0.317647,0.317647}%
\pgfsetstrokecolor{textcolor}%
\pgfsetfillcolor{textcolor}%
\pgftext[x=3.452523in,y=3.307364in,left,base]{\color{textcolor}\rmfamily\fontsize{6.664000}{7.996800}\selectfont \(\displaystyle 330\)}%
\end{pgfscope}%
\begin{pgfscope}%
\definecolor{textcolor}{rgb}{0.317647,0.317647,0.317647}%
\pgfsetstrokecolor{textcolor}%
\pgfsetfillcolor{textcolor}%
\pgftext[x=3.396968in,y=2.697325in,,bottom,rotate=90.000000]{\color{textcolor}\rmfamily\fontsize{6.664000}{7.996800}\selectfont \(\displaystyle \vartheta \propto -b^{(\mathrm{h})} \quad (\si{\milli \V})\)}%
\end{pgfscope}%
\begin{pgfscope}%
\pgfpathrectangle{\pgfqpoint{3.653334in}{1.946106in}}{\pgfqpoint{2.260417in}{1.502439in}}%
\pgfusepath{clip}%
\pgfsetrectcap%
\pgfsetroundjoin%
\pgfsetlinewidth{0.803000pt}%
\definecolor{currentstroke}{rgb}{0.333333,0.333333,0.333333}%
\pgfsetstrokecolor{currentstroke}%
\pgfsetdash{}{0pt}%
\pgfpathmoveto{\pgfqpoint{3.756080in}{3.156008in}}%
\pgfpathlineto{\pgfqpoint{3.758652in}{3.176394in}}%
\pgfpathlineto{\pgfqpoint{3.774083in}{3.176394in}}%
\pgfpathlineto{\pgfqpoint{3.776655in}{3.196779in}}%
\pgfpathlineto{\pgfqpoint{3.781799in}{3.196779in}}%
\pgfpathlineto{\pgfqpoint{3.784371in}{3.217165in}}%
\pgfpathlineto{\pgfqpoint{3.789515in}{3.217165in}}%
\pgfpathlineto{\pgfqpoint{3.792087in}{3.237551in}}%
\pgfpathlineto{\pgfqpoint{3.794658in}{3.217165in}}%
\pgfpathlineto{\pgfqpoint{3.799802in}{3.217165in}}%
\pgfpathlineto{\pgfqpoint{3.802374in}{3.176394in}}%
\pgfpathlineto{\pgfqpoint{3.804946in}{3.176394in}}%
\pgfpathlineto{\pgfqpoint{3.807518in}{3.156008in}}%
\pgfpathlineto{\pgfqpoint{3.812662in}{3.156008in}}%
\pgfpathlineto{\pgfqpoint{3.820377in}{3.094850in}}%
\pgfpathlineto{\pgfqpoint{3.941255in}{3.094850in}}%
\pgfpathlineto{\pgfqpoint{3.943827in}{3.115236in}}%
\pgfpathlineto{\pgfqpoint{3.951543in}{3.115236in}}%
\pgfpathlineto{\pgfqpoint{3.954114in}{3.135622in}}%
\pgfpathlineto{\pgfqpoint{3.956686in}{3.115236in}}%
\pgfpathlineto{\pgfqpoint{3.984977in}{3.115236in}}%
\pgfpathlineto{\pgfqpoint{3.987549in}{3.094850in}}%
\pgfpathlineto{\pgfqpoint{4.013267in}{3.094850in}}%
\pgfpathlineto{\pgfqpoint{4.015839in}{3.115236in}}%
\pgfpathlineto{\pgfqpoint{4.046702in}{3.115236in}}%
\pgfpathlineto{\pgfqpoint{4.049274in}{3.094850in}}%
\pgfpathlineto{\pgfqpoint{4.067277in}{3.094850in}}%
\pgfpathlineto{\pgfqpoint{4.069849in}{3.115236in}}%
\pgfpathlineto{\pgfqpoint{4.159864in}{3.115236in}}%
\pgfpathlineto{\pgfqpoint{4.162436in}{3.094850in}}%
\pgfpathlineto{\pgfqpoint{4.221589in}{3.094850in}}%
\pgfpathlineto{\pgfqpoint{4.224161in}{3.115236in}}%
\pgfpathlineto{\pgfqpoint{4.316748in}{3.115236in}}%
\pgfpathlineto{\pgfqpoint{4.319320in}{3.094850in}}%
\pgfpathlineto{\pgfqpoint{4.373329in}{3.094850in}}%
\pgfpathlineto{\pgfqpoint{4.375901in}{3.115236in}}%
\pgfpathlineto{\pgfqpoint{4.504495in}{3.115236in}}%
\pgfpathlineto{\pgfqpoint{4.507066in}{3.094850in}}%
\pgfpathlineto{\pgfqpoint{4.522498in}{3.094850in}}%
\pgfpathlineto{\pgfqpoint{4.525070in}{3.115236in}}%
\pgfpathlineto{\pgfqpoint{4.563648in}{3.115236in}}%
\pgfpathlineto{\pgfqpoint{4.566219in}{3.135622in}}%
\pgfpathlineto{\pgfqpoint{4.676810in}{3.135622in}}%
\pgfpathlineto{\pgfqpoint{4.679382in}{3.156008in}}%
\pgfpathlineto{\pgfqpoint{4.805403in}{3.156008in}}%
\pgfpathlineto{\pgfqpoint{4.807975in}{3.176394in}}%
\pgfpathlineto{\pgfqpoint{4.815691in}{3.176394in}}%
\pgfpathlineto{\pgfqpoint{4.818263in}{3.196779in}}%
\pgfpathlineto{\pgfqpoint{4.828550in}{3.196779in}}%
\pgfpathlineto{\pgfqpoint{4.831122in}{3.217165in}}%
\pgfpathlineto{\pgfqpoint{4.836266in}{3.217165in}}%
\pgfpathlineto{\pgfqpoint{4.838838in}{3.237551in}}%
\pgfpathlineto{\pgfqpoint{4.887703in}{3.237551in}}%
\pgfpathlineto{\pgfqpoint{4.890275in}{3.257937in}}%
\pgfpathlineto{\pgfqpoint{4.892847in}{3.237551in}}%
\pgfpathlineto{\pgfqpoint{4.931425in}{3.237551in}}%
\pgfpathlineto{\pgfqpoint{4.933997in}{3.217165in}}%
\pgfpathlineto{\pgfqpoint{5.114028in}{3.217165in}}%
\pgfpathlineto{\pgfqpoint{5.116600in}{3.196779in}}%
\pgfpathlineto{\pgfqpoint{5.144890in}{3.196779in}}%
\pgfpathlineto{\pgfqpoint{5.147462in}{3.176394in}}%
\pgfpathlineto{\pgfqpoint{5.222046in}{3.176394in}}%
\pgfpathlineto{\pgfqpoint{5.224618in}{3.196779in}}%
\pgfpathlineto{\pgfqpoint{5.368643in}{3.196779in}}%
\pgfpathlineto{\pgfqpoint{5.371215in}{3.217165in}}%
\pgfpathlineto{\pgfqpoint{5.386646in}{3.217165in}}%
\pgfpathlineto{\pgfqpoint{5.389218in}{3.237551in}}%
\pgfpathlineto{\pgfqpoint{5.394362in}{3.237551in}}%
\pgfpathlineto{\pgfqpoint{5.399505in}{3.196779in}}%
\pgfpathlineto{\pgfqpoint{5.448371in}{3.196779in}}%
\pgfpathlineto{\pgfqpoint{5.450943in}{3.217165in}}%
\pgfpathlineto{\pgfqpoint{5.566677in}{3.217165in}}%
\pgfpathlineto{\pgfqpoint{5.569249in}{3.237551in}}%
\pgfpathlineto{\pgfqpoint{5.584680in}{3.237551in}}%
\pgfpathlineto{\pgfqpoint{5.587252in}{3.257937in}}%
\pgfpathlineto{\pgfqpoint{5.628402in}{3.257937in}}%
\pgfpathlineto{\pgfqpoint{5.630974in}{3.237551in}}%
\pgfpathlineto{\pgfqpoint{5.643833in}{3.237551in}}%
\pgfpathlineto{\pgfqpoint{5.646405in}{3.217165in}}%
\pgfpathlineto{\pgfqpoint{5.733849in}{3.217165in}}%
\pgfpathlineto{\pgfqpoint{5.736420in}{3.237551in}}%
\pgfpathlineto{\pgfqpoint{5.754423in}{3.237551in}}%
\pgfpathlineto{\pgfqpoint{5.756995in}{3.257937in}}%
\pgfpathlineto{\pgfqpoint{5.811005in}{3.257937in}}%
\pgfpathlineto{\pgfqpoint{5.811005in}{3.257937in}}%
\pgfusepath{stroke}%
\end{pgfscope}%
\begin{pgfscope}%
\pgfpathrectangle{\pgfqpoint{3.653334in}{1.946106in}}{\pgfqpoint{2.260417in}{1.502439in}}%
\pgfusepath{clip}%
\pgfsetrectcap%
\pgfsetroundjoin%
\pgfsetlinewidth{0.803000pt}%
\definecolor{currentstroke}{rgb}{0.686275,0.352941,0.313725}%
\pgfsetstrokecolor{currentstroke}%
\pgfsetdash{}{0pt}%
\pgfpathmoveto{\pgfqpoint{3.756080in}{2.360959in}}%
\pgfpathlineto{\pgfqpoint{3.786943in}{2.360959in}}%
\pgfpathlineto{\pgfqpoint{3.789515in}{2.340573in}}%
\pgfpathlineto{\pgfqpoint{3.794658in}{2.340573in}}%
\pgfpathlineto{\pgfqpoint{3.797230in}{2.320187in}}%
\pgfpathlineto{\pgfqpoint{3.799802in}{2.320187in}}%
\pgfpathlineto{\pgfqpoint{3.802374in}{2.340573in}}%
\pgfpathlineto{\pgfqpoint{3.874386in}{2.340573in}}%
\pgfpathlineto{\pgfqpoint{3.876958in}{2.320187in}}%
\pgfpathlineto{\pgfqpoint{3.894961in}{2.320187in}}%
\pgfpathlineto{\pgfqpoint{3.897533in}{2.299801in}}%
\pgfpathlineto{\pgfqpoint{3.925824in}{2.299801in}}%
\pgfpathlineto{\pgfqpoint{3.928396in}{2.279415in}}%
\pgfpathlineto{\pgfqpoint{4.023555in}{2.279415in}}%
\pgfpathlineto{\pgfqpoint{4.026127in}{2.259029in}}%
\pgfpathlineto{\pgfqpoint{4.056989in}{2.259029in}}%
\pgfpathlineto{\pgfqpoint{4.059561in}{2.238643in}}%
\pgfpathlineto{\pgfqpoint{4.062133in}{2.238643in}}%
\pgfpathlineto{\pgfqpoint{4.064705in}{2.218257in}}%
\pgfpathlineto{\pgfqpoint{4.149577in}{2.218257in}}%
\pgfpathlineto{\pgfqpoint{4.152148in}{2.238643in}}%
\pgfpathlineto{\pgfqpoint{4.167580in}{2.238643in}}%
\pgfpathlineto{\pgfqpoint{4.170151in}{2.259029in}}%
\pgfpathlineto{\pgfqpoint{4.262739in}{2.259029in}}%
\pgfpathlineto{\pgfqpoint{4.265311in}{2.238643in}}%
\pgfpathlineto{\pgfqpoint{4.285886in}{2.238643in}}%
\pgfpathlineto{\pgfqpoint{4.288457in}{2.218257in}}%
\pgfpathlineto{\pgfqpoint{4.296173in}{2.218257in}}%
\pgfpathlineto{\pgfqpoint{4.298745in}{2.197872in}}%
\pgfpathlineto{\pgfqpoint{4.306461in}{2.197872in}}%
\pgfpathlineto{\pgfqpoint{4.309032in}{2.177486in}}%
\pgfpathlineto{\pgfqpoint{4.337323in}{2.177486in}}%
\pgfpathlineto{\pgfqpoint{4.339895in}{2.197872in}}%
\pgfpathlineto{\pgfqpoint{4.363042in}{2.197872in}}%
\pgfpathlineto{\pgfqpoint{4.365614in}{2.177486in}}%
\pgfpathlineto{\pgfqpoint{4.373329in}{2.177486in}}%
\pgfpathlineto{\pgfqpoint{4.375901in}{2.157100in}}%
\pgfpathlineto{\pgfqpoint{4.383617in}{2.157100in}}%
\pgfpathlineto{\pgfqpoint{4.386189in}{2.136714in}}%
\pgfpathlineto{\pgfqpoint{4.414479in}{2.136714in}}%
\pgfpathlineto{\pgfqpoint{4.417051in}{2.116328in}}%
\pgfpathlineto{\pgfqpoint{4.427338in}{2.116328in}}%
\pgfpathlineto{\pgfqpoint{4.429910in}{2.095942in}}%
\pgfpathlineto{\pgfqpoint{4.447913in}{2.095942in}}%
\pgfpathlineto{\pgfqpoint{4.450485in}{2.075556in}}%
\pgfpathlineto{\pgfqpoint{4.453057in}{2.075556in}}%
\pgfpathlineto{\pgfqpoint{4.455629in}{2.055170in}}%
\pgfpathlineto{\pgfqpoint{4.486491in}{2.055170in}}%
\pgfpathlineto{\pgfqpoint{4.489063in}{2.034785in}}%
\pgfpathlineto{\pgfqpoint{4.514782in}{2.034785in}}%
\pgfpathlineto{\pgfqpoint{4.517354in}{2.014399in}}%
\pgfpathlineto{\pgfqpoint{4.584223in}{2.014399in}}%
\pgfpathlineto{\pgfqpoint{4.586794in}{2.034785in}}%
\pgfpathlineto{\pgfqpoint{4.589366in}{2.034785in}}%
\pgfpathlineto{\pgfqpoint{4.591938in}{2.055170in}}%
\pgfpathlineto{\pgfqpoint{4.607369in}{2.055170in}}%
\pgfpathlineto{\pgfqpoint{4.609941in}{2.034785in}}%
\pgfpathlineto{\pgfqpoint{4.674238in}{2.034785in}}%
\pgfpathlineto{\pgfqpoint{4.676810in}{2.055170in}}%
\pgfpathlineto{\pgfqpoint{4.699957in}{2.055170in}}%
\pgfpathlineto{\pgfqpoint{4.702529in}{2.034785in}}%
\pgfpathlineto{\pgfqpoint{4.774541in}{2.034785in}}%
\pgfpathlineto{\pgfqpoint{4.777113in}{2.014399in}}%
\pgfpathlineto{\pgfqpoint{4.779685in}{2.034785in}}%
\pgfpathlineto{\pgfqpoint{4.952000in}{2.034785in}}%
\pgfpathlineto{\pgfqpoint{4.954572in}{2.055170in}}%
\pgfpathlineto{\pgfqpoint{4.972575in}{2.055170in}}%
\pgfpathlineto{\pgfqpoint{4.975147in}{2.075556in}}%
\pgfpathlineto{\pgfqpoint{4.980291in}{2.075556in}}%
\pgfpathlineto{\pgfqpoint{4.982862in}{2.095942in}}%
\pgfpathlineto{\pgfqpoint{4.995722in}{2.095942in}}%
\pgfpathlineto{\pgfqpoint{4.998294in}{2.075556in}}%
\pgfpathlineto{\pgfqpoint{5.201471in}{2.075556in}}%
\pgfpathlineto{\pgfqpoint{5.204043in}{2.095942in}}%
\pgfpathlineto{\pgfqpoint{5.330065in}{2.095942in}}%
\pgfpathlineto{\pgfqpoint{5.332637in}{2.075556in}}%
\pgfpathlineto{\pgfqpoint{5.389218in}{2.075556in}}%
\pgfpathlineto{\pgfqpoint{5.391790in}{2.095942in}}%
\pgfpathlineto{\pgfqpoint{5.417508in}{2.095942in}}%
\pgfpathlineto{\pgfqpoint{5.420080in}{2.075556in}}%
\pgfpathlineto{\pgfqpoint{5.499808in}{2.075556in}}%
\pgfpathlineto{\pgfqpoint{5.502380in}{2.095942in}}%
\pgfpathlineto{\pgfqpoint{5.561533in}{2.095942in}}%
\pgfpathlineto{\pgfqpoint{5.564105in}{2.075556in}}%
\pgfpathlineto{\pgfqpoint{5.582108in}{2.075556in}}%
\pgfpathlineto{\pgfqpoint{5.584680in}{2.055170in}}%
\pgfpathlineto{\pgfqpoint{5.654121in}{2.055170in}}%
\pgfpathlineto{\pgfqpoint{5.656692in}{2.075556in}}%
\pgfpathlineto{\pgfqpoint{5.659264in}{2.075556in}}%
\pgfpathlineto{\pgfqpoint{5.661836in}{2.055170in}}%
\pgfpathlineto{\pgfqpoint{5.702986in}{2.055170in}}%
\pgfpathlineto{\pgfqpoint{5.705558in}{2.034785in}}%
\pgfpathlineto{\pgfqpoint{5.754423in}{2.034785in}}%
\pgfpathlineto{\pgfqpoint{5.756995in}{2.014399in}}%
\pgfpathlineto{\pgfqpoint{5.782714in}{2.014399in}}%
\pgfpathlineto{\pgfqpoint{5.785286in}{2.034785in}}%
\pgfpathlineto{\pgfqpoint{5.790430in}{2.034785in}}%
\pgfpathlineto{\pgfqpoint{5.793002in}{2.055170in}}%
\pgfpathlineto{\pgfqpoint{5.805861in}{2.055170in}}%
\pgfpathlineto{\pgfqpoint{5.808433in}{2.075556in}}%
\pgfpathlineto{\pgfqpoint{5.811005in}{2.075556in}}%
\pgfpathlineto{\pgfqpoint{5.811005in}{2.075556in}}%
\pgfusepath{stroke}%
\end{pgfscope}%
\begin{pgfscope}%
\pgfpathrectangle{\pgfqpoint{3.653334in}{1.946106in}}{\pgfqpoint{2.260417in}{1.502439in}}%
\pgfusepath{clip}%
\pgfsetrectcap%
\pgfsetroundjoin%
\pgfsetlinewidth{0.803000pt}%
\definecolor{currentstroke}{rgb}{0.000000,0.356863,0.509804}%
\pgfsetstrokecolor{currentstroke}%
\pgfsetdash{}{0pt}%
\pgfpathmoveto{\pgfqpoint{3.756080in}{2.870605in}}%
\pgfpathlineto{\pgfqpoint{3.758652in}{2.850220in}}%
\pgfpathlineto{\pgfqpoint{3.784371in}{2.850220in}}%
\pgfpathlineto{\pgfqpoint{3.786943in}{2.809448in}}%
\pgfpathlineto{\pgfqpoint{3.789515in}{2.789062in}}%
\pgfpathlineto{\pgfqpoint{3.794658in}{2.789062in}}%
\pgfpathlineto{\pgfqpoint{3.797230in}{2.727904in}}%
\pgfpathlineto{\pgfqpoint{3.799802in}{2.707518in}}%
\pgfpathlineto{\pgfqpoint{3.815233in}{2.707518in}}%
\pgfpathlineto{\pgfqpoint{3.820377in}{2.666747in}}%
\pgfpathlineto{\pgfqpoint{3.822949in}{2.687133in}}%
\pgfpathlineto{\pgfqpoint{3.828093in}{2.646361in}}%
\pgfpathlineto{\pgfqpoint{3.866671in}{2.646361in}}%
\pgfpathlineto{\pgfqpoint{3.869243in}{2.666747in}}%
\pgfpathlineto{\pgfqpoint{3.892389in}{2.666747in}}%
\pgfpathlineto{\pgfqpoint{3.894961in}{2.646361in}}%
\pgfpathlineto{\pgfqpoint{3.923252in}{2.646361in}}%
\pgfpathlineto{\pgfqpoint{3.925824in}{2.625975in}}%
\pgfpathlineto{\pgfqpoint{3.936111in}{2.625975in}}%
\pgfpathlineto{\pgfqpoint{3.938683in}{2.605589in}}%
\pgfpathlineto{\pgfqpoint{3.941255in}{2.605589in}}%
\pgfpathlineto{\pgfqpoint{3.943827in}{2.585203in}}%
\pgfpathlineto{\pgfqpoint{3.961830in}{2.585203in}}%
\pgfpathlineto{\pgfqpoint{3.964402in}{2.564817in}}%
\pgfpathlineto{\pgfqpoint{3.997836in}{2.564817in}}%
\pgfpathlineto{\pgfqpoint{4.000408in}{2.544431in}}%
\pgfpathlineto{\pgfqpoint{4.013267in}{2.544431in}}%
\pgfpathlineto{\pgfqpoint{4.015839in}{2.524046in}}%
\pgfpathlineto{\pgfqpoint{4.051845in}{2.524046in}}%
\pgfpathlineto{\pgfqpoint{4.054417in}{2.544431in}}%
\pgfpathlineto{\pgfqpoint{4.077564in}{2.544431in}}%
\pgfpathlineto{\pgfqpoint{4.080136in}{2.564817in}}%
\pgfpathlineto{\pgfqpoint{4.082708in}{2.524046in}}%
\pgfpathlineto{\pgfqpoint{4.090423in}{2.524046in}}%
\pgfpathlineto{\pgfqpoint{4.092995in}{2.544431in}}%
\pgfpathlineto{\pgfqpoint{4.095567in}{2.524046in}}%
\pgfpathlineto{\pgfqpoint{4.103283in}{2.524046in}}%
\pgfpathlineto{\pgfqpoint{4.105855in}{2.503660in}}%
\pgfpathlineto{\pgfqpoint{4.108427in}{2.503660in}}%
\pgfpathlineto{\pgfqpoint{4.110998in}{2.524046in}}%
\pgfpathlineto{\pgfqpoint{4.116142in}{2.524046in}}%
\pgfpathlineto{\pgfqpoint{4.118714in}{2.503660in}}%
\pgfpathlineto{\pgfqpoint{4.139289in}{2.503660in}}%
\pgfpathlineto{\pgfqpoint{4.141861in}{2.483274in}}%
\pgfpathlineto{\pgfqpoint{4.170151in}{2.483274in}}%
\pgfpathlineto{\pgfqpoint{4.172723in}{2.503660in}}%
\pgfpathlineto{\pgfqpoint{4.221589in}{2.503660in}}%
\pgfpathlineto{\pgfqpoint{4.224161in}{2.483274in}}%
\pgfpathlineto{\pgfqpoint{4.231876in}{2.483274in}}%
\pgfpathlineto{\pgfqpoint{4.234448in}{2.462888in}}%
\pgfpathlineto{\pgfqpoint{4.244736in}{2.462888in}}%
\pgfpathlineto{\pgfqpoint{4.247308in}{2.442502in}}%
\pgfpathlineto{\pgfqpoint{4.337323in}{2.442502in}}%
\pgfpathlineto{\pgfqpoint{4.339895in}{2.462888in}}%
\pgfpathlineto{\pgfqpoint{4.345039in}{2.462888in}}%
\pgfpathlineto{\pgfqpoint{4.347611in}{2.442502in}}%
\pgfpathlineto{\pgfqpoint{4.381045in}{2.442502in}}%
\pgfpathlineto{\pgfqpoint{4.383617in}{2.422116in}}%
\pgfpathlineto{\pgfqpoint{4.388760in}{2.422116in}}%
\pgfpathlineto{\pgfqpoint{4.391332in}{2.401730in}}%
\pgfpathlineto{\pgfqpoint{4.406764in}{2.401730in}}%
\pgfpathlineto{\pgfqpoint{4.409335in}{2.381344in}}%
\pgfpathlineto{\pgfqpoint{4.411907in}{2.401730in}}%
\pgfpathlineto{\pgfqpoint{4.537929in}{2.401730in}}%
\pgfpathlineto{\pgfqpoint{4.540501in}{2.381344in}}%
\pgfpathlineto{\pgfqpoint{4.545645in}{2.381344in}}%
\pgfpathlineto{\pgfqpoint{4.548216in}{2.360959in}}%
\pgfpathlineto{\pgfqpoint{4.563648in}{2.360959in}}%
\pgfpathlineto{\pgfqpoint{4.566219in}{2.340573in}}%
\pgfpathlineto{\pgfqpoint{4.581651in}{2.340573in}}%
\pgfpathlineto{\pgfqpoint{4.584223in}{2.360959in}}%
\pgfpathlineto{\pgfqpoint{4.599654in}{2.360959in}}%
\pgfpathlineto{\pgfqpoint{4.602226in}{2.381344in}}%
\pgfpathlineto{\pgfqpoint{4.607369in}{2.381344in}}%
\pgfpathlineto{\pgfqpoint{4.609941in}{2.401730in}}%
\pgfpathlineto{\pgfqpoint{4.620229in}{2.401730in}}%
\pgfpathlineto{\pgfqpoint{4.622801in}{2.422116in}}%
\pgfpathlineto{\pgfqpoint{4.648519in}{2.422116in}}%
\pgfpathlineto{\pgfqpoint{4.651091in}{2.401730in}}%
\pgfpathlineto{\pgfqpoint{4.666522in}{2.401730in}}%
\pgfpathlineto{\pgfqpoint{4.669094in}{2.381344in}}%
\pgfpathlineto{\pgfqpoint{4.674238in}{2.381344in}}%
\pgfpathlineto{\pgfqpoint{4.676810in}{2.360959in}}%
\pgfpathlineto{\pgfqpoint{4.705100in}{2.360959in}}%
\pgfpathlineto{\pgfqpoint{4.707672in}{2.340573in}}%
\pgfpathlineto{\pgfqpoint{4.841410in}{2.340573in}}%
\pgfpathlineto{\pgfqpoint{4.843981in}{2.360959in}}%
\pgfpathlineto{\pgfqpoint{4.846553in}{2.360959in}}%
\pgfpathlineto{\pgfqpoint{4.851697in}{2.320187in}}%
\pgfpathlineto{\pgfqpoint{4.861985in}{2.320187in}}%
\pgfpathlineto{\pgfqpoint{4.864556in}{2.340573in}}%
\pgfpathlineto{\pgfqpoint{5.157750in}{2.340573in}}%
\pgfpathlineto{\pgfqpoint{5.160321in}{2.360959in}}%
\pgfpathlineto{\pgfqpoint{5.162893in}{2.340573in}}%
\pgfpathlineto{\pgfqpoint{5.337781in}{2.340573in}}%
\pgfpathlineto{\pgfqpoint{5.340352in}{2.320187in}}%
\pgfpathlineto{\pgfqpoint{5.811005in}{2.320187in}}%
\pgfpathlineto{\pgfqpoint{5.811005in}{2.320187in}}%
\pgfusepath{stroke}%
\end{pgfscope}%
\begin{pgfscope}%
\pgfpathrectangle{\pgfqpoint{3.653334in}{1.946106in}}{\pgfqpoint{2.260417in}{1.502439in}}%
\pgfusepath{clip}%
\pgfsetrectcap%
\pgfsetroundjoin%
\pgfsetlinewidth{0.803000pt}%
\definecolor{currentstroke}{rgb}{0.490196,0.588235,0.431373}%
\pgfsetstrokecolor{currentstroke}%
\pgfsetdash{}{0pt}%
\pgfpathmoveto{\pgfqpoint{3.756080in}{2.503660in}}%
\pgfpathlineto{\pgfqpoint{3.794658in}{2.503660in}}%
\pgfpathlineto{\pgfqpoint{3.797230in}{2.483274in}}%
\pgfpathlineto{\pgfqpoint{3.804946in}{2.483274in}}%
\pgfpathlineto{\pgfqpoint{3.807518in}{2.462888in}}%
\pgfpathlineto{\pgfqpoint{3.938683in}{2.462888in}}%
\pgfpathlineto{\pgfqpoint{3.941255in}{2.442502in}}%
\pgfpathlineto{\pgfqpoint{4.077564in}{2.442502in}}%
\pgfpathlineto{\pgfqpoint{4.080136in}{2.422116in}}%
\pgfpathlineto{\pgfqpoint{4.337323in}{2.422116in}}%
\pgfpathlineto{\pgfqpoint{4.339895in}{2.401730in}}%
\pgfpathlineto{\pgfqpoint{5.075450in}{2.401730in}}%
\pgfpathlineto{\pgfqpoint{5.078022in}{2.381344in}}%
\pgfpathlineto{\pgfqpoint{5.811005in}{2.381344in}}%
\pgfpathlineto{\pgfqpoint{5.811005in}{2.381344in}}%
\pgfusepath{stroke}%
\end{pgfscope}%
\begin{pgfscope}%
\pgfpathrectangle{\pgfqpoint{3.653334in}{1.946106in}}{\pgfqpoint{2.260417in}{1.502439in}}%
\pgfusepath{clip}%
\pgfsetrectcap%
\pgfsetroundjoin%
\pgfsetlinewidth{0.803000pt}%
\definecolor{currentstroke}{rgb}{0.843137,0.666667,0.313725}%
\pgfsetstrokecolor{currentstroke}%
\pgfsetdash{}{0pt}%
\pgfpathmoveto{\pgfqpoint{3.756080in}{2.748290in}}%
\pgfpathlineto{\pgfqpoint{3.758652in}{2.809448in}}%
\pgfpathlineto{\pgfqpoint{3.761224in}{2.829834in}}%
\pgfpathlineto{\pgfqpoint{3.774083in}{2.829834in}}%
\pgfpathlineto{\pgfqpoint{3.776655in}{2.850220in}}%
\pgfpathlineto{\pgfqpoint{3.779227in}{2.890991in}}%
\pgfpathlineto{\pgfqpoint{3.781799in}{2.890991in}}%
\pgfpathlineto{\pgfqpoint{3.784371in}{2.911377in}}%
\pgfpathlineto{\pgfqpoint{3.792087in}{2.911377in}}%
\pgfpathlineto{\pgfqpoint{3.794658in}{2.890991in}}%
\pgfpathlineto{\pgfqpoint{3.797230in}{2.931763in}}%
\pgfpathlineto{\pgfqpoint{3.799802in}{2.911377in}}%
\pgfpathlineto{\pgfqpoint{3.802374in}{2.829834in}}%
\pgfpathlineto{\pgfqpoint{3.807518in}{2.829834in}}%
\pgfpathlineto{\pgfqpoint{3.810090in}{2.850220in}}%
\pgfpathlineto{\pgfqpoint{3.812662in}{2.850220in}}%
\pgfpathlineto{\pgfqpoint{3.815233in}{2.829834in}}%
\pgfpathlineto{\pgfqpoint{3.817805in}{2.850220in}}%
\pgfpathlineto{\pgfqpoint{3.820377in}{2.850220in}}%
\pgfpathlineto{\pgfqpoint{3.822949in}{2.809448in}}%
\pgfpathlineto{\pgfqpoint{3.825521in}{2.809448in}}%
\pgfpathlineto{\pgfqpoint{3.828093in}{2.829834in}}%
\pgfpathlineto{\pgfqpoint{3.851240in}{2.829834in}}%
\pgfpathlineto{\pgfqpoint{3.856383in}{2.789062in}}%
\pgfpathlineto{\pgfqpoint{4.013267in}{2.789062in}}%
\pgfpathlineto{\pgfqpoint{4.023555in}{2.870605in}}%
\pgfpathlineto{\pgfqpoint{4.036414in}{2.870605in}}%
\pgfpathlineto{\pgfqpoint{4.038986in}{2.850220in}}%
\pgfpathlineto{\pgfqpoint{4.090423in}{2.850220in}}%
\pgfpathlineto{\pgfqpoint{4.092995in}{2.870605in}}%
\pgfpathlineto{\pgfqpoint{4.095567in}{2.870605in}}%
\pgfpathlineto{\pgfqpoint{4.098139in}{2.890991in}}%
\pgfpathlineto{\pgfqpoint{4.221589in}{2.890991in}}%
\pgfpathlineto{\pgfqpoint{4.226733in}{2.931763in}}%
\pgfpathlineto{\pgfqpoint{4.231876in}{2.931763in}}%
\pgfpathlineto{\pgfqpoint{4.234448in}{2.952149in}}%
\pgfpathlineto{\pgfqpoint{4.252451in}{2.952149in}}%
\pgfpathlineto{\pgfqpoint{4.255023in}{2.972535in}}%
\pgfpathlineto{\pgfqpoint{4.296173in}{2.972535in}}%
\pgfpathlineto{\pgfqpoint{4.298745in}{2.992921in}}%
\pgfpathlineto{\pgfqpoint{4.342467in}{2.992921in}}%
\pgfpathlineto{\pgfqpoint{4.345039in}{3.013307in}}%
\pgfpathlineto{\pgfqpoint{4.373329in}{3.013307in}}%
\pgfpathlineto{\pgfqpoint{4.375901in}{3.033692in}}%
\pgfpathlineto{\pgfqpoint{4.378473in}{3.033692in}}%
\pgfpathlineto{\pgfqpoint{4.381045in}{3.074464in}}%
\pgfpathlineto{\pgfqpoint{4.383617in}{3.074464in}}%
\pgfpathlineto{\pgfqpoint{4.386189in}{3.115236in}}%
\pgfpathlineto{\pgfqpoint{4.399048in}{3.115236in}}%
\pgfpathlineto{\pgfqpoint{4.401620in}{3.135622in}}%
\pgfpathlineto{\pgfqpoint{4.414479in}{3.135622in}}%
\pgfpathlineto{\pgfqpoint{4.417051in}{3.156008in}}%
\pgfpathlineto{\pgfqpoint{4.422195in}{3.156008in}}%
\pgfpathlineto{\pgfqpoint{4.427338in}{3.196779in}}%
\pgfpathlineto{\pgfqpoint{4.445342in}{3.196779in}}%
\pgfpathlineto{\pgfqpoint{4.447913in}{3.176394in}}%
\pgfpathlineto{\pgfqpoint{4.496779in}{3.176394in}}%
\pgfpathlineto{\pgfqpoint{4.499351in}{3.196779in}}%
\pgfpathlineto{\pgfqpoint{4.501923in}{3.196779in}}%
\pgfpathlineto{\pgfqpoint{4.504495in}{3.176394in}}%
\pgfpathlineto{\pgfqpoint{4.537929in}{3.176394in}}%
\pgfpathlineto{\pgfqpoint{4.540501in}{3.196779in}}%
\pgfpathlineto{\pgfqpoint{4.591938in}{3.196779in}}%
\pgfpathlineto{\pgfqpoint{4.594510in}{3.217165in}}%
\pgfpathlineto{\pgfqpoint{4.645947in}{3.217165in}}%
\pgfpathlineto{\pgfqpoint{4.648519in}{3.237551in}}%
\pgfpathlineto{\pgfqpoint{4.674238in}{3.237551in}}%
\pgfpathlineto{\pgfqpoint{4.676810in}{3.257937in}}%
\pgfpathlineto{\pgfqpoint{4.756538in}{3.257937in}}%
\pgfpathlineto{\pgfqpoint{4.759110in}{3.278323in}}%
\pgfpathlineto{\pgfqpoint{4.795116in}{3.278323in}}%
\pgfpathlineto{\pgfqpoint{4.797688in}{3.257937in}}%
\pgfpathlineto{\pgfqpoint{4.885131in}{3.257937in}}%
\pgfpathlineto{\pgfqpoint{4.887703in}{3.278323in}}%
\pgfpathlineto{\pgfqpoint{4.941713in}{3.278323in}}%
\pgfpathlineto{\pgfqpoint{4.944284in}{3.298709in}}%
\pgfpathlineto{\pgfqpoint{4.993150in}{3.298709in}}%
\pgfpathlineto{\pgfqpoint{4.995722in}{3.278323in}}%
\pgfpathlineto{\pgfqpoint{5.042015in}{3.278323in}}%
\pgfpathlineto{\pgfqpoint{5.044587in}{3.257937in}}%
\pgfpathlineto{\pgfqpoint{5.054875in}{3.257937in}}%
\pgfpathlineto{\pgfqpoint{5.057447in}{3.237551in}}%
\pgfpathlineto{\pgfqpoint{5.088309in}{3.237551in}}%
\pgfpathlineto{\pgfqpoint{5.090881in}{3.217165in}}%
\pgfpathlineto{\pgfqpoint{5.103740in}{3.217165in}}%
\pgfpathlineto{\pgfqpoint{5.106312in}{3.196779in}}%
\pgfpathlineto{\pgfqpoint{5.144890in}{3.196779in}}%
\pgfpathlineto{\pgfqpoint{5.147462in}{3.217165in}}%
\pgfpathlineto{\pgfqpoint{5.281199in}{3.217165in}}%
\pgfpathlineto{\pgfqpoint{5.283771in}{3.237551in}}%
\pgfpathlineto{\pgfqpoint{5.345496in}{3.237551in}}%
\pgfpathlineto{\pgfqpoint{5.348068in}{3.257937in}}%
\pgfpathlineto{\pgfqpoint{5.427796in}{3.257937in}}%
\pgfpathlineto{\pgfqpoint{5.430368in}{3.237551in}}%
\pgfpathlineto{\pgfqpoint{5.443227in}{3.237551in}}%
\pgfpathlineto{\pgfqpoint{5.445799in}{3.217165in}}%
\pgfpathlineto{\pgfqpoint{5.463802in}{3.217165in}}%
\pgfpathlineto{\pgfqpoint{5.466374in}{3.237551in}}%
\pgfpathlineto{\pgfqpoint{5.497236in}{3.237551in}}%
\pgfpathlineto{\pgfqpoint{5.499808in}{3.257937in}}%
\pgfpathlineto{\pgfqpoint{5.533243in}{3.257937in}}%
\pgfpathlineto{\pgfqpoint{5.535815in}{3.278323in}}%
\pgfpathlineto{\pgfqpoint{5.538386in}{3.278323in}}%
\pgfpathlineto{\pgfqpoint{5.540958in}{3.298709in}}%
\pgfpathlineto{\pgfqpoint{5.558961in}{3.298709in}}%
\pgfpathlineto{\pgfqpoint{5.561533in}{3.278323in}}%
\pgfpathlineto{\pgfqpoint{5.602683in}{3.278323in}}%
\pgfpathlineto{\pgfqpoint{5.605255in}{3.257937in}}%
\pgfpathlineto{\pgfqpoint{5.618114in}{3.257937in}}%
\pgfpathlineto{\pgfqpoint{5.620686in}{3.237551in}}%
\pgfpathlineto{\pgfqpoint{5.625830in}{3.237551in}}%
\pgfpathlineto{\pgfqpoint{5.628402in}{3.257937in}}%
\pgfpathlineto{\pgfqpoint{5.636117in}{3.257937in}}%
\pgfpathlineto{\pgfqpoint{5.638689in}{3.278323in}}%
\pgfpathlineto{\pgfqpoint{5.651549in}{3.278323in}}%
\pgfpathlineto{\pgfqpoint{5.654121in}{3.298709in}}%
\pgfpathlineto{\pgfqpoint{5.702986in}{3.298709in}}%
\pgfpathlineto{\pgfqpoint{5.705558in}{3.319095in}}%
\pgfpathlineto{\pgfqpoint{5.731277in}{3.319095in}}%
\pgfpathlineto{\pgfqpoint{5.733849in}{3.339481in}}%
\pgfpathlineto{\pgfqpoint{5.738992in}{3.339481in}}%
\pgfpathlineto{\pgfqpoint{5.741564in}{3.359866in}}%
\pgfpathlineto{\pgfqpoint{5.780142in}{3.359866in}}%
\pgfpathlineto{\pgfqpoint{5.782714in}{3.380252in}}%
\pgfpathlineto{\pgfqpoint{5.793002in}{3.380252in}}%
\pgfpathlineto{\pgfqpoint{5.795573in}{3.359866in}}%
\pgfpathlineto{\pgfqpoint{5.811005in}{3.359866in}}%
\pgfpathlineto{\pgfqpoint{5.811005in}{3.359866in}}%
\pgfusepath{stroke}%
\end{pgfscope}%
\begin{pgfscope}%
\pgfpathrectangle{\pgfqpoint{3.653334in}{1.946106in}}{\pgfqpoint{2.260417in}{1.502439in}}%
\pgfusepath{clip}%
\pgfsetrectcap%
\pgfsetroundjoin%
\pgfsetlinewidth{0.803000pt}%
\definecolor{currentstroke}{rgb}{0.333333,0.333333,0.333333}%
\pgfsetstrokecolor{currentstroke}%
\pgfsetstrokeopacity{0.270000}%
\pgfsetdash{}{0pt}%
\pgfpathmoveto{\pgfqpoint{3.756080in}{2.789062in}}%
\pgfpathlineto{\pgfqpoint{3.835808in}{2.789062in}}%
\pgfpathlineto{\pgfqpoint{3.838380in}{2.809448in}}%
\pgfpathlineto{\pgfqpoint{3.882102in}{2.809448in}}%
\pgfpathlineto{\pgfqpoint{3.884674in}{2.789062in}}%
\pgfpathlineto{\pgfqpoint{3.894961in}{2.789062in}}%
\pgfpathlineto{\pgfqpoint{3.897533in}{2.809448in}}%
\pgfpathlineto{\pgfqpoint{3.900105in}{2.789062in}}%
\pgfpathlineto{\pgfqpoint{3.902677in}{2.789062in}}%
\pgfpathlineto{\pgfqpoint{3.905249in}{2.768676in}}%
\pgfpathlineto{\pgfqpoint{3.966974in}{2.768676in}}%
\pgfpathlineto{\pgfqpoint{3.969546in}{2.748290in}}%
\pgfpathlineto{\pgfqpoint{4.077564in}{2.748290in}}%
\pgfpathlineto{\pgfqpoint{4.080136in}{2.707518in}}%
\pgfpathlineto{\pgfqpoint{4.126430in}{2.707518in}}%
\pgfpathlineto{\pgfqpoint{4.129002in}{2.687133in}}%
\pgfpathlineto{\pgfqpoint{4.136717in}{2.687133in}}%
\pgfpathlineto{\pgfqpoint{4.139289in}{2.707518in}}%
\pgfpathlineto{\pgfqpoint{4.141861in}{2.687133in}}%
\pgfpathlineto{\pgfqpoint{4.314176in}{2.687133in}}%
\pgfpathlineto{\pgfqpoint{4.316748in}{2.707518in}}%
\pgfpathlineto{\pgfqpoint{4.365614in}{2.707518in}}%
\pgfpathlineto{\pgfqpoint{4.368185in}{2.727904in}}%
\pgfpathlineto{\pgfqpoint{4.396476in}{2.727904in}}%
\pgfpathlineto{\pgfqpoint{4.399048in}{2.707518in}}%
\pgfpathlineto{\pgfqpoint{4.584223in}{2.707518in}}%
\pgfpathlineto{\pgfqpoint{4.589366in}{2.748290in}}%
\pgfpathlineto{\pgfqpoint{5.811005in}{2.748290in}}%
\pgfpathlineto{\pgfqpoint{5.811005in}{2.748290in}}%
\pgfusepath{stroke}%
\end{pgfscope}%
\begin{pgfscope}%
\pgfpathrectangle{\pgfqpoint{3.653334in}{1.946106in}}{\pgfqpoint{2.260417in}{1.502439in}}%
\pgfusepath{clip}%
\pgfsetrectcap%
\pgfsetroundjoin%
\pgfsetlinewidth{0.803000pt}%
\definecolor{currentstroke}{rgb}{0.686275,0.352941,0.313725}%
\pgfsetstrokecolor{currentstroke}%
\pgfsetstrokeopacity{0.270000}%
\pgfsetdash{}{0pt}%
\pgfpathmoveto{\pgfqpoint{3.756080in}{2.666747in}}%
\pgfpathlineto{\pgfqpoint{3.761224in}{2.707518in}}%
\pgfpathlineto{\pgfqpoint{3.768940in}{2.707518in}}%
\pgfpathlineto{\pgfqpoint{3.771512in}{2.687133in}}%
\pgfpathlineto{\pgfqpoint{3.776655in}{2.687133in}}%
\pgfpathlineto{\pgfqpoint{3.779227in}{2.707518in}}%
\pgfpathlineto{\pgfqpoint{3.786943in}{2.707518in}}%
\pgfpathlineto{\pgfqpoint{3.789515in}{2.727904in}}%
\pgfpathlineto{\pgfqpoint{3.794658in}{2.727904in}}%
\pgfpathlineto{\pgfqpoint{3.797230in}{2.768676in}}%
\pgfpathlineto{\pgfqpoint{3.799802in}{2.768676in}}%
\pgfpathlineto{\pgfqpoint{3.802374in}{2.748290in}}%
\pgfpathlineto{\pgfqpoint{3.812662in}{2.748290in}}%
\pgfpathlineto{\pgfqpoint{3.815233in}{2.768676in}}%
\pgfpathlineto{\pgfqpoint{3.840952in}{2.768676in}}%
\pgfpathlineto{\pgfqpoint{3.843524in}{2.748290in}}%
\pgfpathlineto{\pgfqpoint{3.879530in}{2.748290in}}%
\pgfpathlineto{\pgfqpoint{3.882102in}{2.768676in}}%
\pgfpathlineto{\pgfqpoint{3.905249in}{2.768676in}}%
\pgfpathlineto{\pgfqpoint{3.907821in}{2.748290in}}%
\pgfpathlineto{\pgfqpoint{3.954114in}{2.748290in}}%
\pgfpathlineto{\pgfqpoint{3.956686in}{2.727904in}}%
\pgfpathlineto{\pgfqpoint{4.026127in}{2.727904in}}%
\pgfpathlineto{\pgfqpoint{4.028699in}{2.707518in}}%
\pgfpathlineto{\pgfqpoint{4.051845in}{2.707518in}}%
\pgfpathlineto{\pgfqpoint{4.054417in}{2.727904in}}%
\pgfpathlineto{\pgfqpoint{4.085280in}{2.727904in}}%
\pgfpathlineto{\pgfqpoint{4.090423in}{2.768676in}}%
\pgfpathlineto{\pgfqpoint{4.134145in}{2.768676in}}%
\pgfpathlineto{\pgfqpoint{4.136717in}{2.789062in}}%
\pgfpathlineto{\pgfqpoint{4.298745in}{2.789062in}}%
\pgfpathlineto{\pgfqpoint{4.301317in}{2.768676in}}%
\pgfpathlineto{\pgfqpoint{4.324464in}{2.768676in}}%
\pgfpathlineto{\pgfqpoint{4.327036in}{2.748290in}}%
\pgfpathlineto{\pgfqpoint{4.396476in}{2.748290in}}%
\pgfpathlineto{\pgfqpoint{4.399048in}{2.768676in}}%
\pgfpathlineto{\pgfqpoint{4.522498in}{2.768676in}}%
\pgfpathlineto{\pgfqpoint{4.525070in}{2.748290in}}%
\pgfpathlineto{\pgfqpoint{4.674238in}{2.748290in}}%
\pgfpathlineto{\pgfqpoint{4.676810in}{2.727904in}}%
\pgfpathlineto{\pgfqpoint{4.741107in}{2.727904in}}%
\pgfpathlineto{\pgfqpoint{4.743679in}{2.748290in}}%
\pgfpathlineto{\pgfqpoint{4.761682in}{2.748290in}}%
\pgfpathlineto{\pgfqpoint{4.764253in}{2.727904in}}%
\pgfpathlineto{\pgfqpoint{4.913422in}{2.727904in}}%
\pgfpathlineto{\pgfqpoint{4.915994in}{2.707518in}}%
\pgfpathlineto{\pgfqpoint{4.936569in}{2.707518in}}%
\pgfpathlineto{\pgfqpoint{4.939141in}{2.727904in}}%
\pgfpathlineto{\pgfqpoint{5.080594in}{2.727904in}}%
\pgfpathlineto{\pgfqpoint{5.083165in}{2.707518in}}%
\pgfpathlineto{\pgfqpoint{5.157750in}{2.707518in}}%
\pgfpathlineto{\pgfqpoint{5.160321in}{2.687133in}}%
\pgfpathlineto{\pgfqpoint{5.350640in}{2.687133in}}%
\pgfpathlineto{\pgfqpoint{5.353212in}{2.707518in}}%
\pgfpathlineto{\pgfqpoint{5.355784in}{2.687133in}}%
\pgfpathlineto{\pgfqpoint{5.381502in}{2.687133in}}%
\pgfpathlineto{\pgfqpoint{5.386646in}{2.646361in}}%
\pgfpathlineto{\pgfqpoint{5.453515in}{2.646361in}}%
\pgfpathlineto{\pgfqpoint{5.456087in}{2.666747in}}%
\pgfpathlineto{\pgfqpoint{5.597539in}{2.666747in}}%
\pgfpathlineto{\pgfqpoint{5.600111in}{2.646361in}}%
\pgfpathlineto{\pgfqpoint{5.636117in}{2.646361in}}%
\pgfpathlineto{\pgfqpoint{5.638689in}{2.625975in}}%
\pgfpathlineto{\pgfqpoint{5.811005in}{2.625975in}}%
\pgfpathlineto{\pgfqpoint{5.811005in}{2.625975in}}%
\pgfusepath{stroke}%
\end{pgfscope}%
\begin{pgfscope}%
\pgfpathrectangle{\pgfqpoint{3.653334in}{1.946106in}}{\pgfqpoint{2.260417in}{1.502439in}}%
\pgfusepath{clip}%
\pgfsetrectcap%
\pgfsetroundjoin%
\pgfsetlinewidth{0.803000pt}%
\definecolor{currentstroke}{rgb}{0.000000,0.356863,0.509804}%
\pgfsetstrokecolor{currentstroke}%
\pgfsetstrokeopacity{0.270000}%
\pgfsetdash{}{0pt}%
\pgfpathmoveto{\pgfqpoint{3.756080in}{2.748290in}}%
\pgfpathlineto{\pgfqpoint{3.758652in}{2.768676in}}%
\pgfpathlineto{\pgfqpoint{3.781799in}{2.768676in}}%
\pgfpathlineto{\pgfqpoint{3.786943in}{2.809448in}}%
\pgfpathlineto{\pgfqpoint{3.794658in}{2.809448in}}%
\pgfpathlineto{\pgfqpoint{3.799802in}{2.850220in}}%
\pgfpathlineto{\pgfqpoint{3.848668in}{2.850220in}}%
\pgfpathlineto{\pgfqpoint{3.851240in}{2.870605in}}%
\pgfpathlineto{\pgfqpoint{3.856383in}{2.870605in}}%
\pgfpathlineto{\pgfqpoint{3.858955in}{2.890991in}}%
\pgfpathlineto{\pgfqpoint{3.861527in}{2.890991in}}%
\pgfpathlineto{\pgfqpoint{3.864099in}{2.911377in}}%
\pgfpathlineto{\pgfqpoint{3.879530in}{2.911377in}}%
\pgfpathlineto{\pgfqpoint{3.882102in}{2.931763in}}%
\pgfpathlineto{\pgfqpoint{3.897533in}{2.931763in}}%
\pgfpathlineto{\pgfqpoint{3.900105in}{2.952149in}}%
\pgfpathlineto{\pgfqpoint{3.928396in}{2.952149in}}%
\pgfpathlineto{\pgfqpoint{3.930968in}{2.972535in}}%
\pgfpathlineto{\pgfqpoint{3.938683in}{2.972535in}}%
\pgfpathlineto{\pgfqpoint{3.941255in}{2.992921in}}%
\pgfpathlineto{\pgfqpoint{3.969546in}{2.992921in}}%
\pgfpathlineto{\pgfqpoint{3.972117in}{3.013307in}}%
\pgfpathlineto{\pgfqpoint{3.992692in}{3.013307in}}%
\pgfpathlineto{\pgfqpoint{3.995264in}{2.992921in}}%
\pgfpathlineto{\pgfqpoint{4.062133in}{2.992921in}}%
\pgfpathlineto{\pgfqpoint{4.064705in}{3.013307in}}%
\pgfpathlineto{\pgfqpoint{4.069849in}{3.013307in}}%
\pgfpathlineto{\pgfqpoint{4.072420in}{3.033692in}}%
\pgfpathlineto{\pgfqpoint{4.082708in}{3.033692in}}%
\pgfpathlineto{\pgfqpoint{4.087852in}{3.074464in}}%
\pgfpathlineto{\pgfqpoint{4.100711in}{3.074464in}}%
\pgfpathlineto{\pgfqpoint{4.103283in}{3.094850in}}%
\pgfpathlineto{\pgfqpoint{4.147005in}{3.094850in}}%
\pgfpathlineto{\pgfqpoint{4.149577in}{3.074464in}}%
\pgfpathlineto{\pgfqpoint{4.188155in}{3.074464in}}%
\pgfpathlineto{\pgfqpoint{4.190726in}{3.054078in}}%
\pgfpathlineto{\pgfqpoint{4.216445in}{3.054078in}}%
\pgfpathlineto{\pgfqpoint{4.219017in}{3.074464in}}%
\pgfpathlineto{\pgfqpoint{4.229304in}{3.074464in}}%
\pgfpathlineto{\pgfqpoint{4.231876in}{3.094850in}}%
\pgfpathlineto{\pgfqpoint{4.468488in}{3.094850in}}%
\pgfpathlineto{\pgfqpoint{4.471060in}{3.074464in}}%
\pgfpathlineto{\pgfqpoint{4.478776in}{3.074464in}}%
\pgfpathlineto{\pgfqpoint{4.481348in}{3.054078in}}%
\pgfpathlineto{\pgfqpoint{4.522498in}{3.054078in}}%
\pgfpathlineto{\pgfqpoint{4.525070in}{3.074464in}}%
\pgfpathlineto{\pgfqpoint{4.661379in}{3.074464in}}%
\pgfpathlineto{\pgfqpoint{4.663951in}{3.094850in}}%
\pgfpathlineto{\pgfqpoint{4.843981in}{3.094850in}}%
\pgfpathlineto{\pgfqpoint{4.846553in}{3.115236in}}%
\pgfpathlineto{\pgfqpoint{4.887703in}{3.115236in}}%
\pgfpathlineto{\pgfqpoint{4.890275in}{3.135622in}}%
\pgfpathlineto{\pgfqpoint{4.900563in}{3.135622in}}%
\pgfpathlineto{\pgfqpoint{4.903134in}{3.156008in}}%
\pgfpathlineto{\pgfqpoint{4.915994in}{3.156008in}}%
\pgfpathlineto{\pgfqpoint{4.918566in}{3.176394in}}%
\pgfpathlineto{\pgfqpoint{4.936569in}{3.176394in}}%
\pgfpathlineto{\pgfqpoint{4.939141in}{3.196779in}}%
\pgfpathlineto{\pgfqpoint{5.093453in}{3.196779in}}%
\pgfpathlineto{\pgfqpoint{5.096025in}{3.176394in}}%
\pgfpathlineto{\pgfqpoint{5.425224in}{3.176394in}}%
\pgfpathlineto{\pgfqpoint{5.427796in}{3.196779in}}%
\pgfpathlineto{\pgfqpoint{5.633546in}{3.196779in}}%
\pgfpathlineto{\pgfqpoint{5.636117in}{3.217165in}}%
\pgfpathlineto{\pgfqpoint{5.674696in}{3.217165in}}%
\pgfpathlineto{\pgfqpoint{5.677267in}{3.196779in}}%
\pgfpathlineto{\pgfqpoint{5.811005in}{3.196779in}}%
\pgfpathlineto{\pgfqpoint{5.811005in}{3.196779in}}%
\pgfusepath{stroke}%
\end{pgfscope}%
\begin{pgfscope}%
\pgfpathrectangle{\pgfqpoint{3.653334in}{1.946106in}}{\pgfqpoint{2.260417in}{1.502439in}}%
\pgfusepath{clip}%
\pgfsetrectcap%
\pgfsetroundjoin%
\pgfsetlinewidth{0.803000pt}%
\definecolor{currentstroke}{rgb}{0.490196,0.588235,0.431373}%
\pgfsetstrokecolor{currentstroke}%
\pgfsetstrokeopacity{0.270000}%
\pgfsetdash{}{0pt}%
\pgfpathmoveto{\pgfqpoint{3.756080in}{3.135622in}}%
\pgfpathlineto{\pgfqpoint{3.758652in}{3.135622in}}%
\pgfpathlineto{\pgfqpoint{3.761224in}{3.115236in}}%
\pgfpathlineto{\pgfqpoint{3.766368in}{3.115236in}}%
\pgfpathlineto{\pgfqpoint{3.768940in}{3.094850in}}%
\pgfpathlineto{\pgfqpoint{3.776655in}{3.094850in}}%
\pgfpathlineto{\pgfqpoint{3.779227in}{3.074464in}}%
\pgfpathlineto{\pgfqpoint{3.794658in}{3.074464in}}%
\pgfpathlineto{\pgfqpoint{3.799802in}{3.033692in}}%
\pgfpathlineto{\pgfqpoint{3.804946in}{3.033692in}}%
\pgfpathlineto{\pgfqpoint{3.807518in}{3.013307in}}%
\pgfpathlineto{\pgfqpoint{3.828093in}{3.013307in}}%
\pgfpathlineto{\pgfqpoint{3.830665in}{3.033692in}}%
\pgfpathlineto{\pgfqpoint{3.851240in}{3.033692in}}%
\pgfpathlineto{\pgfqpoint{3.853811in}{3.013307in}}%
\pgfpathlineto{\pgfqpoint{3.856383in}{3.013307in}}%
\pgfpathlineto{\pgfqpoint{3.858955in}{3.033692in}}%
\pgfpathlineto{\pgfqpoint{3.897533in}{3.033692in}}%
\pgfpathlineto{\pgfqpoint{3.900105in}{3.054078in}}%
\pgfpathlineto{\pgfqpoint{3.928396in}{3.054078in}}%
\pgfpathlineto{\pgfqpoint{3.930968in}{3.033692in}}%
\pgfpathlineto{\pgfqpoint{4.010696in}{3.033692in}}%
\pgfpathlineto{\pgfqpoint{4.013267in}{3.054078in}}%
\pgfpathlineto{\pgfqpoint{4.067277in}{3.054078in}}%
\pgfpathlineto{\pgfqpoint{4.069849in}{3.074464in}}%
\pgfpathlineto{\pgfqpoint{4.105855in}{3.074464in}}%
\pgfpathlineto{\pgfqpoint{4.108427in}{3.054078in}}%
\pgfpathlineto{\pgfqpoint{5.811005in}{3.054078in}}%
\pgfpathlineto{\pgfqpoint{5.811005in}{3.054078in}}%
\pgfusepath{stroke}%
\end{pgfscope}%
\begin{pgfscope}%
\pgfpathrectangle{\pgfqpoint{3.653334in}{1.946106in}}{\pgfqpoint{2.260417in}{1.502439in}}%
\pgfusepath{clip}%
\pgfsetrectcap%
\pgfsetroundjoin%
\pgfsetlinewidth{0.803000pt}%
\definecolor{currentstroke}{rgb}{0.843137,0.666667,0.313725}%
\pgfsetstrokecolor{currentstroke}%
\pgfsetstrokeopacity{0.270000}%
\pgfsetdash{}{0pt}%
\pgfpathmoveto{\pgfqpoint{3.756080in}{2.625975in}}%
\pgfpathlineto{\pgfqpoint{3.761224in}{2.544431in}}%
\pgfpathlineto{\pgfqpoint{3.766368in}{2.544431in}}%
\pgfpathlineto{\pgfqpoint{3.768940in}{2.524046in}}%
\pgfpathlineto{\pgfqpoint{3.776655in}{2.524046in}}%
\pgfpathlineto{\pgfqpoint{3.779227in}{2.503660in}}%
\pgfpathlineto{\pgfqpoint{3.786943in}{2.503660in}}%
\pgfpathlineto{\pgfqpoint{3.789515in}{2.483274in}}%
\pgfpathlineto{\pgfqpoint{3.797230in}{2.483274in}}%
\pgfpathlineto{\pgfqpoint{3.799802in}{2.462888in}}%
\pgfpathlineto{\pgfqpoint{3.802374in}{2.483274in}}%
\pgfpathlineto{\pgfqpoint{3.817805in}{2.483274in}}%
\pgfpathlineto{\pgfqpoint{3.820377in}{2.462888in}}%
\pgfpathlineto{\pgfqpoint{3.822949in}{2.483274in}}%
\pgfpathlineto{\pgfqpoint{3.828093in}{2.483274in}}%
\pgfpathlineto{\pgfqpoint{3.835808in}{2.422116in}}%
\pgfpathlineto{\pgfqpoint{3.838380in}{2.422116in}}%
\pgfpathlineto{\pgfqpoint{3.843524in}{2.381344in}}%
\pgfpathlineto{\pgfqpoint{3.848668in}{2.381344in}}%
\pgfpathlineto{\pgfqpoint{3.851240in}{2.401730in}}%
\pgfpathlineto{\pgfqpoint{3.861527in}{2.401730in}}%
\pgfpathlineto{\pgfqpoint{3.864099in}{2.422116in}}%
\pgfpathlineto{\pgfqpoint{3.874386in}{2.422116in}}%
\pgfpathlineto{\pgfqpoint{3.876958in}{2.442502in}}%
\pgfpathlineto{\pgfqpoint{3.889818in}{2.442502in}}%
\pgfpathlineto{\pgfqpoint{3.892389in}{2.422116in}}%
\pgfpathlineto{\pgfqpoint{3.902677in}{2.422116in}}%
\pgfpathlineto{\pgfqpoint{3.905249in}{2.401730in}}%
\pgfpathlineto{\pgfqpoint{3.920680in}{2.401730in}}%
\pgfpathlineto{\pgfqpoint{3.923252in}{2.422116in}}%
\pgfpathlineto{\pgfqpoint{3.930968in}{2.422116in}}%
\pgfpathlineto{\pgfqpoint{3.933539in}{2.442502in}}%
\pgfpathlineto{\pgfqpoint{3.943827in}{2.442502in}}%
\pgfpathlineto{\pgfqpoint{3.946399in}{2.422116in}}%
\pgfpathlineto{\pgfqpoint{4.015839in}{2.422116in}}%
\pgfpathlineto{\pgfqpoint{4.018411in}{2.401730in}}%
\pgfpathlineto{\pgfqpoint{4.036414in}{2.401730in}}%
\pgfpathlineto{\pgfqpoint{4.038986in}{2.422116in}}%
\pgfpathlineto{\pgfqpoint{4.041558in}{2.422116in}}%
\pgfpathlineto{\pgfqpoint{4.044130in}{2.401730in}}%
\pgfpathlineto{\pgfqpoint{4.046702in}{2.401730in}}%
\pgfpathlineto{\pgfqpoint{4.049274in}{2.381344in}}%
\pgfpathlineto{\pgfqpoint{4.059561in}{2.381344in}}%
\pgfpathlineto{\pgfqpoint{4.062133in}{2.401730in}}%
\pgfpathlineto{\pgfqpoint{4.095567in}{2.401730in}}%
\pgfpathlineto{\pgfqpoint{4.098139in}{2.360959in}}%
\pgfpathlineto{\pgfqpoint{4.103283in}{2.360959in}}%
\pgfpathlineto{\pgfqpoint{4.105855in}{2.381344in}}%
\pgfpathlineto{\pgfqpoint{4.188155in}{2.381344in}}%
\pgfpathlineto{\pgfqpoint{4.190726in}{2.401730in}}%
\pgfpathlineto{\pgfqpoint{4.203586in}{2.401730in}}%
\pgfpathlineto{\pgfqpoint{4.206158in}{2.381344in}}%
\pgfpathlineto{\pgfqpoint{4.252451in}{2.381344in}}%
\pgfpathlineto{\pgfqpoint{4.255023in}{2.360959in}}%
\pgfpathlineto{\pgfqpoint{4.365614in}{2.360959in}}%
\pgfpathlineto{\pgfqpoint{4.368185in}{2.381344in}}%
\pgfpathlineto{\pgfqpoint{4.386189in}{2.381344in}}%
\pgfpathlineto{\pgfqpoint{4.388760in}{2.401730in}}%
\pgfpathlineto{\pgfqpoint{4.414479in}{2.401730in}}%
\pgfpathlineto{\pgfqpoint{4.417051in}{2.381344in}}%
\pgfpathlineto{\pgfqpoint{4.537929in}{2.381344in}}%
\pgfpathlineto{\pgfqpoint{4.540501in}{2.360959in}}%
\pgfpathlineto{\pgfqpoint{4.573935in}{2.360959in}}%
\pgfpathlineto{\pgfqpoint{4.576507in}{2.381344in}}%
\pgfpathlineto{\pgfqpoint{4.589366in}{2.381344in}}%
\pgfpathlineto{\pgfqpoint{4.591938in}{2.401730in}}%
\pgfpathlineto{\pgfqpoint{4.620229in}{2.401730in}}%
\pgfpathlineto{\pgfqpoint{4.622801in}{2.422116in}}%
\pgfpathlineto{\pgfqpoint{4.679382in}{2.422116in}}%
\pgfpathlineto{\pgfqpoint{4.681954in}{2.442502in}}%
\pgfpathlineto{\pgfqpoint{4.697385in}{2.442502in}}%
\pgfpathlineto{\pgfqpoint{4.699957in}{2.462888in}}%
\pgfpathlineto{\pgfqpoint{4.753966in}{2.462888in}}%
\pgfpathlineto{\pgfqpoint{4.756538in}{2.442502in}}%
\pgfpathlineto{\pgfqpoint{4.787400in}{2.442502in}}%
\pgfpathlineto{\pgfqpoint{4.789972in}{2.462888in}}%
\pgfpathlineto{\pgfqpoint{4.802832in}{2.462888in}}%
\pgfpathlineto{\pgfqpoint{4.807975in}{2.422116in}}%
\pgfpathlineto{\pgfqpoint{4.818263in}{2.422116in}}%
\pgfpathlineto{\pgfqpoint{4.820835in}{2.442502in}}%
\pgfpathlineto{\pgfqpoint{4.874844in}{2.442502in}}%
\pgfpathlineto{\pgfqpoint{4.877416in}{2.422116in}}%
\pgfpathlineto{\pgfqpoint{5.240049in}{2.422116in}}%
\pgfpathlineto{\pgfqpoint{5.242621in}{2.401730in}}%
\pgfpathlineto{\pgfqpoint{5.252909in}{2.401730in}}%
\pgfpathlineto{\pgfqpoint{5.255481in}{2.422116in}}%
\pgfpathlineto{\pgfqpoint{5.543530in}{2.422116in}}%
\pgfpathlineto{\pgfqpoint{5.546102in}{2.442502in}}%
\pgfpathlineto{\pgfqpoint{5.659264in}{2.442502in}}%
\pgfpathlineto{\pgfqpoint{5.661836in}{2.422116in}}%
\pgfpathlineto{\pgfqpoint{5.687555in}{2.422116in}}%
\pgfpathlineto{\pgfqpoint{5.690127in}{2.401730in}}%
\pgfpathlineto{\pgfqpoint{5.723561in}{2.401730in}}%
\pgfpathlineto{\pgfqpoint{5.726133in}{2.381344in}}%
\pgfpathlineto{\pgfqpoint{5.811005in}{2.381344in}}%
\pgfpathlineto{\pgfqpoint{5.811005in}{2.381344in}}%
\pgfusepath{stroke}%
\end{pgfscope}%
\begin{pgfscope}%
\pgfpathrectangle{\pgfqpoint{3.653334in}{1.946106in}}{\pgfqpoint{2.260417in}{1.502439in}}%
\pgfusepath{clip}%
\pgfsetrectcap%
\pgfsetroundjoin%
\pgfsetlinewidth{0.803000pt}%
\definecolor{currentstroke}{rgb}{0.333333,0.333333,0.333333}%
\pgfsetstrokecolor{currentstroke}%
\pgfsetstrokeopacity{0.270000}%
\pgfsetdash{}{0pt}%
\pgfpathmoveto{\pgfqpoint{3.756080in}{2.768676in}}%
\pgfpathlineto{\pgfqpoint{3.758652in}{2.789062in}}%
\pgfpathlineto{\pgfqpoint{3.776655in}{2.789062in}}%
\pgfpathlineto{\pgfqpoint{3.779227in}{2.809448in}}%
\pgfpathlineto{\pgfqpoint{3.810090in}{2.809448in}}%
\pgfpathlineto{\pgfqpoint{3.812662in}{2.829834in}}%
\pgfpathlineto{\pgfqpoint{3.861527in}{2.829834in}}%
\pgfpathlineto{\pgfqpoint{3.864099in}{2.809448in}}%
\pgfpathlineto{\pgfqpoint{3.874386in}{2.809448in}}%
\pgfpathlineto{\pgfqpoint{3.876958in}{2.789062in}}%
\pgfpathlineto{\pgfqpoint{3.894961in}{2.789062in}}%
\pgfpathlineto{\pgfqpoint{3.897533in}{2.809448in}}%
\pgfpathlineto{\pgfqpoint{3.928396in}{2.809448in}}%
\pgfpathlineto{\pgfqpoint{3.930968in}{2.829834in}}%
\pgfpathlineto{\pgfqpoint{3.974689in}{2.829834in}}%
\pgfpathlineto{\pgfqpoint{3.977261in}{2.809448in}}%
\pgfpathlineto{\pgfqpoint{3.995264in}{2.809448in}}%
\pgfpathlineto{\pgfqpoint{3.997836in}{2.829834in}}%
\pgfpathlineto{\pgfqpoint{4.098139in}{2.829834in}}%
\pgfpathlineto{\pgfqpoint{4.100711in}{2.809448in}}%
\pgfpathlineto{\pgfqpoint{4.183011in}{2.809448in}}%
\pgfpathlineto{\pgfqpoint{4.185583in}{2.789062in}}%
\pgfpathlineto{\pgfqpoint{4.188155in}{2.789062in}}%
\pgfpathlineto{\pgfqpoint{4.190726in}{2.768676in}}%
\pgfpathlineto{\pgfqpoint{4.275598in}{2.768676in}}%
\pgfpathlineto{\pgfqpoint{4.278170in}{2.748290in}}%
\pgfpathlineto{\pgfqpoint{4.303889in}{2.748290in}}%
\pgfpathlineto{\pgfqpoint{4.309032in}{2.707518in}}%
\pgfpathlineto{\pgfqpoint{4.321892in}{2.707518in}}%
\pgfpathlineto{\pgfqpoint{4.324464in}{2.687133in}}%
\pgfpathlineto{\pgfqpoint{4.345039in}{2.687133in}}%
\pgfpathlineto{\pgfqpoint{4.347611in}{2.707518in}}%
\pgfpathlineto{\pgfqpoint{4.555932in}{2.707518in}}%
\pgfpathlineto{\pgfqpoint{4.558504in}{2.727904in}}%
\pgfpathlineto{\pgfqpoint{4.728247in}{2.727904in}}%
\pgfpathlineto{\pgfqpoint{4.730819in}{2.707518in}}%
\pgfpathlineto{\pgfqpoint{4.761682in}{2.707518in}}%
\pgfpathlineto{\pgfqpoint{4.764253in}{2.687133in}}%
\pgfpathlineto{\pgfqpoint{4.782257in}{2.687133in}}%
\pgfpathlineto{\pgfqpoint{4.784828in}{2.666747in}}%
\pgfpathlineto{\pgfqpoint{5.026584in}{2.666747in}}%
\pgfpathlineto{\pgfqpoint{5.029156in}{2.687133in}}%
\pgfpathlineto{\pgfqpoint{5.222046in}{2.687133in}}%
\pgfpathlineto{\pgfqpoint{5.224618in}{2.666747in}}%
\pgfpathlineto{\pgfqpoint{5.440655in}{2.666747in}}%
\pgfpathlineto{\pgfqpoint{5.443227in}{2.687133in}}%
\pgfpathlineto{\pgfqpoint{5.615542in}{2.687133in}}%
\pgfpathlineto{\pgfqpoint{5.618114in}{2.707518in}}%
\pgfpathlineto{\pgfqpoint{5.811005in}{2.707518in}}%
\pgfpathlineto{\pgfqpoint{5.811005in}{2.707518in}}%
\pgfusepath{stroke}%
\end{pgfscope}%
\begin{pgfscope}%
\pgfsetrectcap%
\pgfsetmiterjoin%
\pgfsetlinewidth{0.501875pt}%
\definecolor{currentstroke}{rgb}{0.317647,0.317647,0.317647}%
\pgfsetstrokecolor{currentstroke}%
\pgfsetdash{}{0pt}%
\pgfpathmoveto{\pgfqpoint{3.653334in}{1.946106in}}%
\pgfpathlineto{\pgfqpoint{3.653334in}{3.448545in}}%
\pgfusepath{stroke}%
\end{pgfscope}%
\begin{pgfscope}%
\pgfsetrectcap%
\pgfsetmiterjoin%
\pgfsetlinewidth{0.501875pt}%
\definecolor{currentstroke}{rgb}{0.317647,0.317647,0.317647}%
\pgfsetstrokecolor{currentstroke}%
\pgfsetdash{}{0pt}%
\pgfpathmoveto{\pgfqpoint{3.653334in}{1.946106in}}%
\pgfpathlineto{\pgfqpoint{5.913751in}{1.946106in}}%
\pgfusepath{stroke}%
\end{pgfscope}%
\begin{pgfscope}%
\pgfsetrectcap%
\pgfsetroundjoin%
\pgfsetlinewidth{0.803000pt}%
\definecolor{currentstroke}{rgb}{0.333333,0.333333,0.333333}%
\pgfsetstrokecolor{currentstroke}%
\pgfsetdash{}{0pt}%
\pgfpathmoveto{\pgfqpoint{5.868543in}{3.324235in}}%
\pgfpathlineto{\pgfqpoint{5.935209in}{3.324235in}}%
\pgfusepath{stroke}%
\end{pgfscope}%
\begin{pgfscope}%
\definecolor{textcolor}{rgb}{0.000000,0.000000,0.000000}%
\pgfsetstrokecolor{textcolor}%
\pgfsetfillcolor{textcolor}%
\pgftext[x=5.976876in,y=3.295069in,left,base]{\color{textcolor}\rmfamily\fontsize{6.000000}{7.200000}\selectfont \(\displaystyle b_0\)}%
\end{pgfscope}%
\begin{pgfscope}%
\pgfsetrectcap%
\pgfsetroundjoin%
\pgfsetlinewidth{0.803000pt}%
\definecolor{currentstroke}{rgb}{0.686275,0.352941,0.313725}%
\pgfsetstrokecolor{currentstroke}%
\pgfsetdash{}{0pt}%
\pgfpathmoveto{\pgfqpoint{5.868543in}{3.216402in}}%
\pgfpathlineto{\pgfqpoint{5.935209in}{3.216402in}}%
\pgfusepath{stroke}%
\end{pgfscope}%
\begin{pgfscope}%
\definecolor{textcolor}{rgb}{0.000000,0.000000,0.000000}%
\pgfsetstrokecolor{textcolor}%
\pgfsetfillcolor{textcolor}%
\pgftext[x=5.976876in,y=3.187236in,left,base]{\color{textcolor}\rmfamily\fontsize{6.000000}{7.200000}\selectfont \(\displaystyle b_1\)}%
\end{pgfscope}%
\begin{pgfscope}%
\pgfsetrectcap%
\pgfsetroundjoin%
\pgfsetlinewidth{0.803000pt}%
\definecolor{currentstroke}{rgb}{0.000000,0.356863,0.509804}%
\pgfsetstrokecolor{currentstroke}%
\pgfsetdash{}{0pt}%
\pgfpathmoveto{\pgfqpoint{5.868543in}{3.108569in}}%
\pgfpathlineto{\pgfqpoint{5.935209in}{3.108569in}}%
\pgfusepath{stroke}%
\end{pgfscope}%
\begin{pgfscope}%
\definecolor{textcolor}{rgb}{0.000000,0.000000,0.000000}%
\pgfsetstrokecolor{textcolor}%
\pgfsetfillcolor{textcolor}%
\pgftext[x=5.976876in,y=3.079402in,left,base]{\color{textcolor}\rmfamily\fontsize{6.000000}{7.200000}\selectfont \(\displaystyle b_2\)}%
\end{pgfscope}%
\begin{pgfscope}%
\pgfsetrectcap%
\pgfsetroundjoin%
\pgfsetlinewidth{0.803000pt}%
\definecolor{currentstroke}{rgb}{0.490196,0.588235,0.431373}%
\pgfsetstrokecolor{currentstroke}%
\pgfsetdash{}{0pt}%
\pgfpathmoveto{\pgfqpoint{5.868543in}{3.000736in}}%
\pgfpathlineto{\pgfqpoint{5.935209in}{3.000736in}}%
\pgfusepath{stroke}%
\end{pgfscope}%
\begin{pgfscope}%
\definecolor{textcolor}{rgb}{0.000000,0.000000,0.000000}%
\pgfsetstrokecolor{textcolor}%
\pgfsetfillcolor{textcolor}%
\pgftext[x=5.976876in,y=2.971569in,left,base]{\color{textcolor}\rmfamily\fontsize{6.000000}{7.200000}\selectfont \(\displaystyle b_3\)}%
\end{pgfscope}%
\begin{pgfscope}%
\pgfsetrectcap%
\pgfsetroundjoin%
\pgfsetlinewidth{0.803000pt}%
\definecolor{currentstroke}{rgb}{0.843137,0.666667,0.313725}%
\pgfsetstrokecolor{currentstroke}%
\pgfsetdash{}{0pt}%
\pgfpathmoveto{\pgfqpoint{5.868543in}{2.892903in}}%
\pgfpathlineto{\pgfqpoint{5.935209in}{2.892903in}}%
\pgfusepath{stroke}%
\end{pgfscope}%
\begin{pgfscope}%
\definecolor{textcolor}{rgb}{0.000000,0.000000,0.000000}%
\pgfsetstrokecolor{textcolor}%
\pgfsetfillcolor{textcolor}%
\pgftext[x=5.976876in,y=2.863736in,left,base]{\color{textcolor}\rmfamily\fontsize{6.000000}{7.200000}\selectfont \(\displaystyle b_4\)}%
\end{pgfscope}%
\begin{pgfscope}%
\pgfsetbuttcap%
\pgfsetmiterjoin%
\pgfsetlinewidth{0.000000pt}%
\definecolor{currentstroke}{rgb}{0.000000,0.000000,0.000000}%
\pgfsetstrokecolor{currentstroke}%
\pgfsetstrokeopacity{0.000000}%
\pgfsetdash{}{0pt}%
\pgfpathmoveto{\pgfqpoint{0.488751in}{0.368545in}}%
\pgfpathlineto{\pgfqpoint{2.749168in}{0.368545in}}%
\pgfpathlineto{\pgfqpoint{2.749168in}{1.870984in}}%
\pgfpathlineto{\pgfqpoint{0.488751in}{1.870984in}}%
\pgfpathclose%
\pgfusepath{}%
\end{pgfscope}%
\begin{pgfscope}%
\pgfsetbuttcap%
\pgfsetroundjoin%
\definecolor{currentfill}{rgb}{0.317647,0.317647,0.317647}%
\pgfsetfillcolor{currentfill}%
\pgfsetlinewidth{0.501875pt}%
\definecolor{currentstroke}{rgb}{0.317647,0.317647,0.317647}%
\pgfsetstrokecolor{currentstroke}%
\pgfsetdash{}{0pt}%
\pgfsys@defobject{currentmarker}{\pgfqpoint{0.000000in}{-0.020833in}}{\pgfqpoint{0.000000in}{0.000000in}}{%
\pgfpathmoveto{\pgfqpoint{0.000000in}{0.000000in}}%
\pgfpathlineto{\pgfqpoint{0.000000in}{-0.020833in}}%
\pgfusepath{stroke,fill}%
}%
\begin{pgfscope}%
\pgfsys@transformshift{0.591497in}{0.368545in}%
\pgfsys@useobject{currentmarker}{}%
\end{pgfscope}%
\end{pgfscope}%
\begin{pgfscope}%
\definecolor{textcolor}{rgb}{0.317647,0.317647,0.317647}%
\pgfsetstrokecolor{textcolor}%
\pgfsetfillcolor{textcolor}%
\pgftext[x=0.591497in,y=0.319934in,,top]{\color{textcolor}\rmfamily\fontsize{6.664000}{7.996800}\selectfont \(\displaystyle 0\)}%
\end{pgfscope}%
\begin{pgfscope}%
\pgfsetbuttcap%
\pgfsetroundjoin%
\definecolor{currentfill}{rgb}{0.317647,0.317647,0.317647}%
\pgfsetfillcolor{currentfill}%
\pgfsetlinewidth{0.501875pt}%
\definecolor{currentstroke}{rgb}{0.317647,0.317647,0.317647}%
\pgfsetstrokecolor{currentstroke}%
\pgfsetdash{}{0pt}%
\pgfsys@defobject{currentmarker}{\pgfqpoint{0.000000in}{-0.020833in}}{\pgfqpoint{0.000000in}{0.000000in}}{%
\pgfpathmoveto{\pgfqpoint{0.000000in}{0.000000in}}%
\pgfpathlineto{\pgfqpoint{0.000000in}{-0.020833in}}%
\pgfusepath{stroke,fill}%
}%
\begin{pgfscope}%
\pgfsys@transformshift{1.105871in}{0.368545in}%
\pgfsys@useobject{currentmarker}{}%
\end{pgfscope}%
\end{pgfscope}%
\begin{pgfscope}%
\definecolor{textcolor}{rgb}{0.317647,0.317647,0.317647}%
\pgfsetstrokecolor{textcolor}%
\pgfsetfillcolor{textcolor}%
\pgftext[x=1.105871in,y=0.319934in,,top]{\color{textcolor}\rmfamily\fontsize{6.664000}{7.996800}\selectfont \(\displaystyle 1000\)}%
\end{pgfscope}%
\begin{pgfscope}%
\pgfsetbuttcap%
\pgfsetroundjoin%
\definecolor{currentfill}{rgb}{0.317647,0.317647,0.317647}%
\pgfsetfillcolor{currentfill}%
\pgfsetlinewidth{0.501875pt}%
\definecolor{currentstroke}{rgb}{0.317647,0.317647,0.317647}%
\pgfsetstrokecolor{currentstroke}%
\pgfsetdash{}{0pt}%
\pgfsys@defobject{currentmarker}{\pgfqpoint{0.000000in}{-0.020833in}}{\pgfqpoint{0.000000in}{0.000000in}}{%
\pgfpathmoveto{\pgfqpoint{0.000000in}{0.000000in}}%
\pgfpathlineto{\pgfqpoint{0.000000in}{-0.020833in}}%
\pgfusepath{stroke,fill}%
}%
\begin{pgfscope}%
\pgfsys@transformshift{1.620245in}{0.368545in}%
\pgfsys@useobject{currentmarker}{}%
\end{pgfscope}%
\end{pgfscope}%
\begin{pgfscope}%
\definecolor{textcolor}{rgb}{0.317647,0.317647,0.317647}%
\pgfsetstrokecolor{textcolor}%
\pgfsetfillcolor{textcolor}%
\pgftext[x=1.620245in,y=0.319934in,,top]{\color{textcolor}\rmfamily\fontsize{6.664000}{7.996800}\selectfont \(\displaystyle 2000\)}%
\end{pgfscope}%
\begin{pgfscope}%
\pgfsetbuttcap%
\pgfsetroundjoin%
\definecolor{currentfill}{rgb}{0.317647,0.317647,0.317647}%
\pgfsetfillcolor{currentfill}%
\pgfsetlinewidth{0.501875pt}%
\definecolor{currentstroke}{rgb}{0.317647,0.317647,0.317647}%
\pgfsetstrokecolor{currentstroke}%
\pgfsetdash{}{0pt}%
\pgfsys@defobject{currentmarker}{\pgfqpoint{0.000000in}{-0.020833in}}{\pgfqpoint{0.000000in}{0.000000in}}{%
\pgfpathmoveto{\pgfqpoint{0.000000in}{0.000000in}}%
\pgfpathlineto{\pgfqpoint{0.000000in}{-0.020833in}}%
\pgfusepath{stroke,fill}%
}%
\begin{pgfscope}%
\pgfsys@transformshift{2.134619in}{0.368545in}%
\pgfsys@useobject{currentmarker}{}%
\end{pgfscope}%
\end{pgfscope}%
\begin{pgfscope}%
\definecolor{textcolor}{rgb}{0.317647,0.317647,0.317647}%
\pgfsetstrokecolor{textcolor}%
\pgfsetfillcolor{textcolor}%
\pgftext[x=2.134619in,y=0.319934in,,top]{\color{textcolor}\rmfamily\fontsize{6.664000}{7.996800}\selectfont \(\displaystyle 3000\)}%
\end{pgfscope}%
\begin{pgfscope}%
\pgfsetbuttcap%
\pgfsetroundjoin%
\definecolor{currentfill}{rgb}{0.317647,0.317647,0.317647}%
\pgfsetfillcolor{currentfill}%
\pgfsetlinewidth{0.501875pt}%
\definecolor{currentstroke}{rgb}{0.317647,0.317647,0.317647}%
\pgfsetstrokecolor{currentstroke}%
\pgfsetdash{}{0pt}%
\pgfsys@defobject{currentmarker}{\pgfqpoint{0.000000in}{-0.020833in}}{\pgfqpoint{0.000000in}{0.000000in}}{%
\pgfpathmoveto{\pgfqpoint{0.000000in}{0.000000in}}%
\pgfpathlineto{\pgfqpoint{0.000000in}{-0.020833in}}%
\pgfusepath{stroke,fill}%
}%
\begin{pgfscope}%
\pgfsys@transformshift{2.648993in}{0.368545in}%
\pgfsys@useobject{currentmarker}{}%
\end{pgfscope}%
\end{pgfscope}%
\begin{pgfscope}%
\definecolor{textcolor}{rgb}{0.317647,0.317647,0.317647}%
\pgfsetstrokecolor{textcolor}%
\pgfsetfillcolor{textcolor}%
\pgftext[x=2.648993in,y=0.319934in,,top]{\color{textcolor}\rmfamily\fontsize{6.664000}{7.996800}\selectfont \(\displaystyle 4000\)}%
\end{pgfscope}%
\begin{pgfscope}%
\definecolor{textcolor}{rgb}{0.317647,0.317647,0.317647}%
\pgfsetstrokecolor{textcolor}%
\pgfsetfillcolor{textcolor}%
\pgftext[x=1.618959in,y=0.182189in,,top]{\color{textcolor}\rmfamily\fontsize{6.664000}{7.996800}\selectfont Iteration}%
\end{pgfscope}%
\begin{pgfscope}%
\pgfsetbuttcap%
\pgfsetroundjoin%
\definecolor{currentfill}{rgb}{0.317647,0.317647,0.317647}%
\pgfsetfillcolor{currentfill}%
\pgfsetlinewidth{0.501875pt}%
\definecolor{currentstroke}{rgb}{0.317647,0.317647,0.317647}%
\pgfsetstrokecolor{currentstroke}%
\pgfsetdash{}{0pt}%
\pgfsys@defobject{currentmarker}{\pgfqpoint{-0.020833in}{0.000000in}}{\pgfqpoint{0.000000in}{0.000000in}}{%
\pgfpathmoveto{\pgfqpoint{0.000000in}{0.000000in}}%
\pgfpathlineto{\pgfqpoint{-0.020833in}{0.000000in}}%
\pgfusepath{stroke,fill}%
}%
\begin{pgfscope}%
\pgfsys@transformshift{0.488751in}{0.683679in}%
\pgfsys@useobject{currentmarker}{}%
\end{pgfscope}%
\end{pgfscope}%
\begin{pgfscope}%
\definecolor{textcolor}{rgb}{0.317647,0.317647,0.317647}%
\pgfsetstrokecolor{textcolor}%
\pgfsetfillcolor{textcolor}%
\pgftext[x=0.256497in,y=0.651562in,left,base]{\color{textcolor}\rmfamily\fontsize{6.664000}{7.996800}\selectfont \(\displaystyle -20\)}%
\end{pgfscope}%
\begin{pgfscope}%
\pgfsetbuttcap%
\pgfsetroundjoin%
\definecolor{currentfill}{rgb}{0.317647,0.317647,0.317647}%
\pgfsetfillcolor{currentfill}%
\pgfsetlinewidth{0.501875pt}%
\definecolor{currentstroke}{rgb}{0.317647,0.317647,0.317647}%
\pgfsetstrokecolor{currentstroke}%
\pgfsetdash{}{0pt}%
\pgfsys@defobject{currentmarker}{\pgfqpoint{-0.020833in}{0.000000in}}{\pgfqpoint{0.000000in}{0.000000in}}{%
\pgfpathmoveto{\pgfqpoint{0.000000in}{0.000000in}}%
\pgfpathlineto{\pgfqpoint{-0.020833in}{0.000000in}}%
\pgfusepath{stroke,fill}%
}%
\begin{pgfscope}%
\pgfsys@transformshift{0.488751in}{1.012800in}%
\pgfsys@useobject{currentmarker}{}%
\end{pgfscope}%
\end{pgfscope}%
\begin{pgfscope}%
\definecolor{textcolor}{rgb}{0.317647,0.317647,0.317647}%
\pgfsetstrokecolor{textcolor}%
\pgfsetfillcolor{textcolor}%
\pgftext[x=0.398666in,y=0.980683in,left,base]{\color{textcolor}\rmfamily\fontsize{6.664000}{7.996800}\selectfont \(\displaystyle 0\)}%
\end{pgfscope}%
\begin{pgfscope}%
\pgfsetbuttcap%
\pgfsetroundjoin%
\definecolor{currentfill}{rgb}{0.317647,0.317647,0.317647}%
\pgfsetfillcolor{currentfill}%
\pgfsetlinewidth{0.501875pt}%
\definecolor{currentstroke}{rgb}{0.317647,0.317647,0.317647}%
\pgfsetstrokecolor{currentstroke}%
\pgfsetdash{}{0pt}%
\pgfsys@defobject{currentmarker}{\pgfqpoint{-0.020833in}{0.000000in}}{\pgfqpoint{0.000000in}{0.000000in}}{%
\pgfpathmoveto{\pgfqpoint{0.000000in}{0.000000in}}%
\pgfpathlineto{\pgfqpoint{-0.020833in}{0.000000in}}%
\pgfusepath{stroke,fill}%
}%
\begin{pgfscope}%
\pgfsys@transformshift{0.488751in}{1.341921in}%
\pgfsys@useobject{currentmarker}{}%
\end{pgfscope}%
\end{pgfscope}%
\begin{pgfscope}%
\definecolor{textcolor}{rgb}{0.317647,0.317647,0.317647}%
\pgfsetstrokecolor{textcolor}%
\pgfsetfillcolor{textcolor}%
\pgftext[x=0.343303in,y=1.309805in,left,base]{\color{textcolor}\rmfamily\fontsize{6.664000}{7.996800}\selectfont \(\displaystyle 20\)}%
\end{pgfscope}%
\begin{pgfscope}%
\pgfsetbuttcap%
\pgfsetroundjoin%
\definecolor{currentfill}{rgb}{0.317647,0.317647,0.317647}%
\pgfsetfillcolor{currentfill}%
\pgfsetlinewidth{0.501875pt}%
\definecolor{currentstroke}{rgb}{0.317647,0.317647,0.317647}%
\pgfsetstrokecolor{currentstroke}%
\pgfsetdash{}{0pt}%
\pgfsys@defobject{currentmarker}{\pgfqpoint{-0.020833in}{0.000000in}}{\pgfqpoint{0.000000in}{0.000000in}}{%
\pgfpathmoveto{\pgfqpoint{0.000000in}{0.000000in}}%
\pgfpathlineto{\pgfqpoint{-0.020833in}{0.000000in}}%
\pgfusepath{stroke,fill}%
}%
\begin{pgfscope}%
\pgfsys@transformshift{0.488751in}{1.671043in}%
\pgfsys@useobject{currentmarker}{}%
\end{pgfscope}%
\end{pgfscope}%
\begin{pgfscope}%
\definecolor{textcolor}{rgb}{0.317647,0.317647,0.317647}%
\pgfsetstrokecolor{textcolor}%
\pgfsetfillcolor{textcolor}%
\pgftext[x=0.343303in,y=1.638926in,left,base]{\color{textcolor}\rmfamily\fontsize{6.664000}{7.996800}\selectfont \(\displaystyle 40\)}%
\end{pgfscope}%
\begin{pgfscope}%
\definecolor{textcolor}{rgb}{0.317647,0.317647,0.317647}%
\pgfsetstrokecolor{textcolor}%
\pgfsetfillcolor{textcolor}%
\pgftext[x=0.200942in,y=1.119765in,,bottom,rotate=90.000000]{\color{textcolor}\rmfamily\fontsize{6.664000}{7.996800}\selectfont \(\displaystyle W^{(\mathrm{o})}\)}%
\end{pgfscope}%
\begin{pgfscope}%
\pgfpathrectangle{\pgfqpoint{0.488751in}{0.368545in}}{\pgfqpoint{2.260417in}{1.502439in}}%
\pgfusepath{clip}%
\pgfsetrectcap%
\pgfsetroundjoin%
\pgfsetlinewidth{0.803000pt}%
\definecolor{currentstroke}{rgb}{0.333333,0.333333,0.333333}%
\pgfsetstrokecolor{currentstroke}%
\pgfsetdash{}{0pt}%
\pgfpathmoveto{\pgfqpoint{0.591497in}{1.078624in}}%
\pgfpathlineto{\pgfqpoint{0.594069in}{1.095080in}}%
\pgfpathlineto{\pgfqpoint{0.601785in}{1.095080in}}%
\pgfpathlineto{\pgfqpoint{0.604356in}{1.111536in}}%
\pgfpathlineto{\pgfqpoint{0.617216in}{1.111536in}}%
\pgfpathlineto{\pgfqpoint{0.619788in}{1.127993in}}%
\pgfpathlineto{\pgfqpoint{0.622359in}{1.127993in}}%
\pgfpathlineto{\pgfqpoint{0.627503in}{1.160905in}}%
\pgfpathlineto{\pgfqpoint{0.630075in}{1.144449in}}%
\pgfpathlineto{\pgfqpoint{0.632647in}{1.144449in}}%
\pgfpathlineto{\pgfqpoint{0.635219in}{1.160905in}}%
\pgfpathlineto{\pgfqpoint{0.645506in}{1.160905in}}%
\pgfpathlineto{\pgfqpoint{0.648078in}{1.177361in}}%
\pgfpathlineto{\pgfqpoint{0.725234in}{1.177361in}}%
\pgfpathlineto{\pgfqpoint{0.727806in}{1.193817in}}%
\pgfpathlineto{\pgfqpoint{0.776672in}{1.193817in}}%
\pgfpathlineto{\pgfqpoint{0.779244in}{1.210273in}}%
\pgfpathlineto{\pgfqpoint{0.784387in}{1.210273in}}%
\pgfpathlineto{\pgfqpoint{0.789531in}{1.243185in}}%
\pgfpathlineto{\pgfqpoint{0.843540in}{1.243185in}}%
\pgfpathlineto{\pgfqpoint{0.846112in}{1.259641in}}%
\pgfpathlineto{\pgfqpoint{0.889834in}{1.259641in}}%
\pgfpathlineto{\pgfqpoint{0.892406in}{1.276097in}}%
\pgfpathlineto{\pgfqpoint{1.013284in}{1.276097in}}%
\pgfpathlineto{\pgfqpoint{1.015856in}{1.292553in}}%
\pgfpathlineto{\pgfqpoint{1.057006in}{1.292553in}}%
\pgfpathlineto{\pgfqpoint{1.059577in}{1.309009in}}%
\pgfpathlineto{\pgfqpoint{1.085296in}{1.309009in}}%
\pgfpathlineto{\pgfqpoint{1.087868in}{1.325465in}}%
\pgfpathlineto{\pgfqpoint{1.129018in}{1.325465in}}%
\pgfpathlineto{\pgfqpoint{1.131590in}{1.341921in}}%
\pgfpathlineto{\pgfqpoint{1.152165in}{1.341921in}}%
\pgfpathlineto{\pgfqpoint{1.154737in}{1.325465in}}%
\pgfpathlineto{\pgfqpoint{1.162452in}{1.325465in}}%
\pgfpathlineto{\pgfqpoint{1.165024in}{1.341921in}}%
\pgfpathlineto{\pgfqpoint{1.177883in}{1.341921in}}%
\pgfpathlineto{\pgfqpoint{1.180455in}{1.325465in}}%
\pgfpathlineto{\pgfqpoint{1.257611in}{1.325465in}}%
\pgfpathlineto{\pgfqpoint{1.260183in}{1.309009in}}%
\pgfpathlineto{\pgfqpoint{1.496795in}{1.309009in}}%
\pgfpathlineto{\pgfqpoint{1.499367in}{1.325465in}}%
\pgfpathlineto{\pgfqpoint{1.656251in}{1.325465in}}%
\pgfpathlineto{\pgfqpoint{1.658823in}{1.341921in}}%
\pgfpathlineto{\pgfqpoint{1.661395in}{1.341921in}}%
\pgfpathlineto{\pgfqpoint{1.663967in}{1.358378in}}%
\pgfpathlineto{\pgfqpoint{1.674254in}{1.358378in}}%
\pgfpathlineto{\pgfqpoint{1.676826in}{1.374834in}}%
\pgfpathlineto{\pgfqpoint{1.684542in}{1.374834in}}%
\pgfpathlineto{\pgfqpoint{1.689686in}{1.407746in}}%
\pgfpathlineto{\pgfqpoint{1.807992in}{1.407746in}}%
\pgfpathlineto{\pgfqpoint{1.810563in}{1.424202in}}%
\pgfpathlineto{\pgfqpoint{1.838854in}{1.424202in}}%
\pgfpathlineto{\pgfqpoint{1.841426in}{1.407746in}}%
\pgfpathlineto{\pgfqpoint{1.854285in}{1.407746in}}%
\pgfpathlineto{\pgfqpoint{1.856857in}{1.424202in}}%
\pgfpathlineto{\pgfqpoint{2.013741in}{1.424202in}}%
\pgfpathlineto{\pgfqpoint{2.016313in}{1.407746in}}%
\pgfpathlineto{\pgfqpoint{2.026601in}{1.407746in}}%
\pgfpathlineto{\pgfqpoint{2.029172in}{1.424202in}}%
\pgfpathlineto{\pgfqpoint{2.119188in}{1.424202in}}%
\pgfpathlineto{\pgfqpoint{2.121760in}{1.407746in}}%
\pgfpathlineto{\pgfqpoint{2.214347in}{1.407746in}}%
\pgfpathlineto{\pgfqpoint{2.216919in}{1.424202in}}%
\pgfpathlineto{\pgfqpoint{2.278644in}{1.424202in}}%
\pgfpathlineto{\pgfqpoint{2.281216in}{1.407746in}}%
\pgfpathlineto{\pgfqpoint{2.366087in}{1.407746in}}%
\pgfpathlineto{\pgfqpoint{2.368659in}{1.391290in}}%
\pgfpathlineto{\pgfqpoint{2.561550in}{1.391290in}}%
\pgfpathlineto{\pgfqpoint{2.564121in}{1.374834in}}%
\pgfpathlineto{\pgfqpoint{2.646421in}{1.374834in}}%
\pgfpathlineto{\pgfqpoint{2.646421in}{1.374834in}}%
\pgfusepath{stroke}%
\end{pgfscope}%
\begin{pgfscope}%
\pgfpathrectangle{\pgfqpoint{0.488751in}{0.368545in}}{\pgfqpoint{2.260417in}{1.502439in}}%
\pgfusepath{clip}%
\pgfsetrectcap%
\pgfsetroundjoin%
\pgfsetlinewidth{0.803000pt}%
\definecolor{currentstroke}{rgb}{0.686275,0.352941,0.313725}%
\pgfsetstrokecolor{currentstroke}%
\pgfsetdash{}{0pt}%
\pgfpathmoveto{\pgfqpoint{0.591497in}{0.683679in}}%
\pgfpathlineto{\pgfqpoint{0.601785in}{0.749503in}}%
\pgfpathlineto{\pgfqpoint{0.609500in}{0.749503in}}%
\pgfpathlineto{\pgfqpoint{0.612072in}{0.765959in}}%
\pgfpathlineto{\pgfqpoint{0.617216in}{0.765959in}}%
\pgfpathlineto{\pgfqpoint{0.619788in}{0.749503in}}%
\pgfpathlineto{\pgfqpoint{0.630075in}{0.749503in}}%
\pgfpathlineto{\pgfqpoint{0.632647in}{0.765959in}}%
\pgfpathlineto{\pgfqpoint{0.653222in}{0.765959in}}%
\pgfpathlineto{\pgfqpoint{0.655794in}{0.782415in}}%
\pgfpathlineto{\pgfqpoint{0.660938in}{0.782415in}}%
\pgfpathlineto{\pgfqpoint{0.663509in}{0.798871in}}%
\pgfpathlineto{\pgfqpoint{0.668653in}{0.798871in}}%
\pgfpathlineto{\pgfqpoint{0.671225in}{0.815327in}}%
\pgfpathlineto{\pgfqpoint{0.684084in}{0.815327in}}%
\pgfpathlineto{\pgfqpoint{0.686656in}{0.831783in}}%
\pgfpathlineto{\pgfqpoint{0.714947in}{0.831783in}}%
\pgfpathlineto{\pgfqpoint{0.717519in}{0.848239in}}%
\pgfpathlineto{\pgfqpoint{0.730378in}{0.848239in}}%
\pgfpathlineto{\pgfqpoint{0.732950in}{0.864695in}}%
\pgfpathlineto{\pgfqpoint{0.735522in}{0.864695in}}%
\pgfpathlineto{\pgfqpoint{0.738094in}{0.881152in}}%
\pgfpathlineto{\pgfqpoint{0.758669in}{0.881152in}}%
\pgfpathlineto{\pgfqpoint{0.761240in}{0.897608in}}%
\pgfpathlineto{\pgfqpoint{0.768956in}{0.897608in}}%
\pgfpathlineto{\pgfqpoint{0.771528in}{0.914064in}}%
\pgfpathlineto{\pgfqpoint{0.776672in}{0.914064in}}%
\pgfpathlineto{\pgfqpoint{0.779244in}{0.946976in}}%
\pgfpathlineto{\pgfqpoint{0.781815in}{0.963432in}}%
\pgfpathlineto{\pgfqpoint{0.792103in}{0.963432in}}%
\pgfpathlineto{\pgfqpoint{0.794675in}{0.979888in}}%
\pgfpathlineto{\pgfqpoint{0.802390in}{0.979888in}}%
\pgfpathlineto{\pgfqpoint{0.807534in}{1.012800in}}%
\pgfpathlineto{\pgfqpoint{0.840968in}{1.012800in}}%
\pgfpathlineto{\pgfqpoint{0.843540in}{0.996344in}}%
\pgfpathlineto{\pgfqpoint{0.858972in}{0.996344in}}%
\pgfpathlineto{\pgfqpoint{0.861543in}{1.012800in}}%
\pgfpathlineto{\pgfqpoint{0.912981in}{1.012800in}}%
\pgfpathlineto{\pgfqpoint{0.915553in}{1.029256in}}%
\pgfpathlineto{\pgfqpoint{0.930984in}{1.029256in}}%
\pgfpathlineto{\pgfqpoint{0.933556in}{1.045712in}}%
\pgfpathlineto{\pgfqpoint{0.946415in}{1.045712in}}%
\pgfpathlineto{\pgfqpoint{0.948987in}{1.062168in}}%
\pgfpathlineto{\pgfqpoint{0.959274in}{1.062168in}}%
\pgfpathlineto{\pgfqpoint{0.961846in}{1.045712in}}%
\pgfpathlineto{\pgfqpoint{0.966990in}{1.078624in}}%
\pgfpathlineto{\pgfqpoint{1.010712in}{1.078624in}}%
\pgfpathlineto{\pgfqpoint{1.013284in}{1.095080in}}%
\pgfpathlineto{\pgfqpoint{1.033859in}{1.095080in}}%
\pgfpathlineto{\pgfqpoint{1.036431in}{1.078624in}}%
\pgfpathlineto{\pgfqpoint{1.054434in}{1.078624in}}%
\pgfpathlineto{\pgfqpoint{1.057006in}{1.062168in}}%
\pgfpathlineto{\pgfqpoint{1.072437in}{1.062168in}}%
\pgfpathlineto{\pgfqpoint{1.075009in}{1.045712in}}%
\pgfpathlineto{\pgfqpoint{1.077581in}{1.045712in}}%
\pgfpathlineto{\pgfqpoint{1.080152in}{1.029256in}}%
\pgfpathlineto{\pgfqpoint{1.087868in}{1.029256in}}%
\pgfpathlineto{\pgfqpoint{1.090440in}{1.045712in}}%
\pgfpathlineto{\pgfqpoint{1.093012in}{1.045712in}}%
\pgfpathlineto{\pgfqpoint{1.098155in}{1.012800in}}%
\pgfpathlineto{\pgfqpoint{1.111015in}{1.012800in}}%
\pgfpathlineto{\pgfqpoint{1.113587in}{0.996344in}}%
\pgfpathlineto{\pgfqpoint{1.123874in}{0.996344in}}%
\pgfpathlineto{\pgfqpoint{1.129018in}{0.963432in}}%
\pgfpathlineto{\pgfqpoint{1.131590in}{0.963432in}}%
\pgfpathlineto{\pgfqpoint{1.134162in}{0.996344in}}%
\pgfpathlineto{\pgfqpoint{1.141877in}{0.996344in}}%
\pgfpathlineto{\pgfqpoint{1.144449in}{1.029256in}}%
\pgfpathlineto{\pgfqpoint{1.152165in}{1.029256in}}%
\pgfpathlineto{\pgfqpoint{1.154737in}{1.045712in}}%
\pgfpathlineto{\pgfqpoint{1.157308in}{1.045712in}}%
\pgfpathlineto{\pgfqpoint{1.159880in}{1.012800in}}%
\pgfpathlineto{\pgfqpoint{1.172740in}{1.012800in}}%
\pgfpathlineto{\pgfqpoint{1.175312in}{1.029256in}}%
\pgfpathlineto{\pgfqpoint{1.198458in}{1.029256in}}%
\pgfpathlineto{\pgfqpoint{1.201030in}{1.045712in}}%
\pgfpathlineto{\pgfqpoint{1.203602in}{1.029256in}}%
\pgfpathlineto{\pgfqpoint{1.206174in}{1.029256in}}%
\pgfpathlineto{\pgfqpoint{1.208746in}{1.012800in}}%
\pgfpathlineto{\pgfqpoint{1.219033in}{1.012800in}}%
\pgfpathlineto{\pgfqpoint{1.221605in}{1.029256in}}%
\pgfpathlineto{\pgfqpoint{1.229321in}{1.029256in}}%
\pgfpathlineto{\pgfqpoint{1.234465in}{0.996344in}}%
\pgfpathlineto{\pgfqpoint{1.237036in}{1.012800in}}%
\pgfpathlineto{\pgfqpoint{1.239608in}{1.012800in}}%
\pgfpathlineto{\pgfqpoint{1.242180in}{0.996344in}}%
\pgfpathlineto{\pgfqpoint{1.257611in}{0.996344in}}%
\pgfpathlineto{\pgfqpoint{1.260183in}{0.979888in}}%
\pgfpathlineto{\pgfqpoint{1.262755in}{0.996344in}}%
\pgfpathlineto{\pgfqpoint{1.285902in}{0.996344in}}%
\pgfpathlineto{\pgfqpoint{1.288474in}{1.012800in}}%
\pgfpathlineto{\pgfqpoint{1.306477in}{1.012800in}}%
\pgfpathlineto{\pgfqpoint{1.309049in}{0.979888in}}%
\pgfpathlineto{\pgfqpoint{1.311621in}{0.963432in}}%
\pgfpathlineto{\pgfqpoint{1.329624in}{0.963432in}}%
\pgfpathlineto{\pgfqpoint{1.332196in}{0.979888in}}%
\pgfpathlineto{\pgfqpoint{1.347627in}{0.979888in}}%
\pgfpathlineto{\pgfqpoint{1.350199in}{0.963432in}}%
\pgfpathlineto{\pgfqpoint{1.357914in}{0.963432in}}%
\pgfpathlineto{\pgfqpoint{1.360486in}{0.946976in}}%
\pgfpathlineto{\pgfqpoint{1.363058in}{0.914064in}}%
\pgfpathlineto{\pgfqpoint{1.365630in}{0.897608in}}%
\pgfpathlineto{\pgfqpoint{1.375917in}{0.897608in}}%
\pgfpathlineto{\pgfqpoint{1.378489in}{0.881152in}}%
\pgfpathlineto{\pgfqpoint{1.388777in}{0.881152in}}%
\pgfpathlineto{\pgfqpoint{1.391349in}{0.864695in}}%
\pgfpathlineto{\pgfqpoint{1.401636in}{0.864695in}}%
\pgfpathlineto{\pgfqpoint{1.404208in}{0.897608in}}%
\pgfpathlineto{\pgfqpoint{1.406780in}{0.897608in}}%
\pgfpathlineto{\pgfqpoint{1.409352in}{0.930520in}}%
\pgfpathlineto{\pgfqpoint{1.424783in}{0.930520in}}%
\pgfpathlineto{\pgfqpoint{1.427355in}{0.914064in}}%
\pgfpathlineto{\pgfqpoint{1.429927in}{0.914064in}}%
\pgfpathlineto{\pgfqpoint{1.432499in}{0.930520in}}%
\pgfpathlineto{\pgfqpoint{1.437642in}{0.930520in}}%
\pgfpathlineto{\pgfqpoint{1.440214in}{0.914064in}}%
\pgfpathlineto{\pgfqpoint{1.442786in}{0.914064in}}%
\pgfpathlineto{\pgfqpoint{1.445358in}{0.897608in}}%
\pgfpathlineto{\pgfqpoint{1.460789in}{0.897608in}}%
\pgfpathlineto{\pgfqpoint{1.463361in}{0.864695in}}%
\pgfpathlineto{\pgfqpoint{1.465933in}{0.864695in}}%
\pgfpathlineto{\pgfqpoint{1.468505in}{0.848239in}}%
\pgfpathlineto{\pgfqpoint{1.499367in}{0.848239in}}%
\pgfpathlineto{\pgfqpoint{1.504511in}{0.881152in}}%
\pgfpathlineto{\pgfqpoint{1.509655in}{0.881152in}}%
\pgfpathlineto{\pgfqpoint{1.514798in}{0.848239in}}%
\pgfpathlineto{\pgfqpoint{1.522514in}{0.848239in}}%
\pgfpathlineto{\pgfqpoint{1.527658in}{0.815327in}}%
\pgfpathlineto{\pgfqpoint{1.540517in}{0.815327in}}%
\pgfpathlineto{\pgfqpoint{1.543089in}{0.798871in}}%
\pgfpathlineto{\pgfqpoint{1.548233in}{0.798871in}}%
\pgfpathlineto{\pgfqpoint{1.550805in}{0.815327in}}%
\pgfpathlineto{\pgfqpoint{1.561092in}{0.815327in}}%
\pgfpathlineto{\pgfqpoint{1.563664in}{0.782415in}}%
\pgfpathlineto{\pgfqpoint{1.568808in}{0.782415in}}%
\pgfpathlineto{\pgfqpoint{1.571380in}{0.765959in}}%
\pgfpathlineto{\pgfqpoint{1.591955in}{0.765959in}}%
\pgfpathlineto{\pgfqpoint{1.594526in}{0.782415in}}%
\pgfpathlineto{\pgfqpoint{1.597098in}{0.782415in}}%
\pgfpathlineto{\pgfqpoint{1.599670in}{0.765959in}}%
\pgfpathlineto{\pgfqpoint{1.638248in}{0.765959in}}%
\pgfpathlineto{\pgfqpoint{1.640820in}{0.749503in}}%
\pgfpathlineto{\pgfqpoint{1.658823in}{0.749503in}}%
\pgfpathlineto{\pgfqpoint{1.661395in}{0.733047in}}%
\pgfpathlineto{\pgfqpoint{1.679398in}{0.733047in}}%
\pgfpathlineto{\pgfqpoint{1.681970in}{0.716591in}}%
\pgfpathlineto{\pgfqpoint{1.715404in}{0.716591in}}%
\pgfpathlineto{\pgfqpoint{1.717976in}{0.733047in}}%
\pgfpathlineto{\pgfqpoint{1.720548in}{0.733047in}}%
\pgfpathlineto{\pgfqpoint{1.723120in}{0.716591in}}%
\pgfpathlineto{\pgfqpoint{1.728264in}{0.716591in}}%
\pgfpathlineto{\pgfqpoint{1.730836in}{0.733047in}}%
\pgfpathlineto{\pgfqpoint{1.764270in}{0.733047in}}%
\pgfpathlineto{\pgfqpoint{1.766842in}{0.749503in}}%
\pgfpathlineto{\pgfqpoint{1.774557in}{0.749503in}}%
\pgfpathlineto{\pgfqpoint{1.777129in}{0.733047in}}%
\pgfpathlineto{\pgfqpoint{1.784845in}{0.733047in}}%
\pgfpathlineto{\pgfqpoint{1.787417in}{0.749503in}}%
\pgfpathlineto{\pgfqpoint{1.789989in}{0.749503in}}%
\pgfpathlineto{\pgfqpoint{1.792560in}{0.733047in}}%
\pgfpathlineto{\pgfqpoint{1.797704in}{0.733047in}}%
\pgfpathlineto{\pgfqpoint{1.800276in}{0.716591in}}%
\pgfpathlineto{\pgfqpoint{1.818279in}{0.716591in}}%
\pgfpathlineto{\pgfqpoint{1.820851in}{0.700135in}}%
\pgfpathlineto{\pgfqpoint{1.831138in}{0.700135in}}%
\pgfpathlineto{\pgfqpoint{1.833710in}{0.683679in}}%
\pgfpathlineto{\pgfqpoint{1.838854in}{0.683679in}}%
\pgfpathlineto{\pgfqpoint{1.841426in}{0.667223in}}%
\pgfpathlineto{\pgfqpoint{1.849142in}{0.667223in}}%
\pgfpathlineto{\pgfqpoint{1.851713in}{0.683679in}}%
\pgfpathlineto{\pgfqpoint{1.856857in}{0.683679in}}%
\pgfpathlineto{\pgfqpoint{1.862001in}{0.716591in}}%
\pgfpathlineto{\pgfqpoint{1.872288in}{0.716591in}}%
\pgfpathlineto{\pgfqpoint{1.874860in}{0.700135in}}%
\pgfpathlineto{\pgfqpoint{1.892863in}{0.700135in}}%
\pgfpathlineto{\pgfqpoint{1.895435in}{0.683679in}}%
\pgfpathlineto{\pgfqpoint{1.898007in}{0.683679in}}%
\pgfpathlineto{\pgfqpoint{1.900579in}{0.700135in}}%
\pgfpathlineto{\pgfqpoint{1.903151in}{0.700135in}}%
\pgfpathlineto{\pgfqpoint{1.905723in}{0.683679in}}%
\pgfpathlineto{\pgfqpoint{1.910866in}{0.683679in}}%
\pgfpathlineto{\pgfqpoint{1.913438in}{0.700135in}}%
\pgfpathlineto{\pgfqpoint{1.926298in}{0.700135in}}%
\pgfpathlineto{\pgfqpoint{1.928870in}{0.683679in}}%
\pgfpathlineto{\pgfqpoint{1.931441in}{0.650767in}}%
\pgfpathlineto{\pgfqpoint{1.949444in}{0.650767in}}%
\pgfpathlineto{\pgfqpoint{1.952016in}{0.667223in}}%
\pgfpathlineto{\pgfqpoint{1.957160in}{0.667223in}}%
\pgfpathlineto{\pgfqpoint{1.959732in}{0.650767in}}%
\pgfpathlineto{\pgfqpoint{1.972591in}{0.650767in}}%
\pgfpathlineto{\pgfqpoint{1.982879in}{0.716591in}}%
\pgfpathlineto{\pgfqpoint{2.018885in}{0.716591in}}%
\pgfpathlineto{\pgfqpoint{2.021457in}{0.733047in}}%
\pgfpathlineto{\pgfqpoint{2.036888in}{0.733047in}}%
\pgfpathlineto{\pgfqpoint{2.039460in}{0.716591in}}%
\pgfpathlineto{\pgfqpoint{2.049747in}{0.716591in}}%
\pgfpathlineto{\pgfqpoint{2.052319in}{0.700135in}}%
\pgfpathlineto{\pgfqpoint{2.108900in}{0.700135in}}%
\pgfpathlineto{\pgfqpoint{2.111472in}{0.683679in}}%
\pgfpathlineto{\pgfqpoint{2.132047in}{0.683679in}}%
\pgfpathlineto{\pgfqpoint{2.134619in}{0.667223in}}%
\pgfpathlineto{\pgfqpoint{2.155194in}{0.667223in}}%
\pgfpathlineto{\pgfqpoint{2.157766in}{0.683679in}}%
\pgfpathlineto{\pgfqpoint{2.191200in}{0.683679in}}%
\pgfpathlineto{\pgfqpoint{2.193772in}{0.716591in}}%
\pgfpathlineto{\pgfqpoint{2.224635in}{0.716591in}}%
\pgfpathlineto{\pgfqpoint{2.227206in}{0.700135in}}%
\pgfpathlineto{\pgfqpoint{2.229778in}{0.716591in}}%
\pgfpathlineto{\pgfqpoint{2.237494in}{0.716591in}}%
\pgfpathlineto{\pgfqpoint{2.240066in}{0.700135in}}%
\pgfpathlineto{\pgfqpoint{2.265785in}{0.700135in}}%
\pgfpathlineto{\pgfqpoint{2.268356in}{0.683679in}}%
\pgfpathlineto{\pgfqpoint{2.345512in}{0.683679in}}%
\pgfpathlineto{\pgfqpoint{2.348084in}{0.667223in}}%
\pgfpathlineto{\pgfqpoint{2.353228in}{0.667223in}}%
\pgfpathlineto{\pgfqpoint{2.355800in}{0.683679in}}%
\pgfpathlineto{\pgfqpoint{2.389234in}{0.683679in}}%
\pgfpathlineto{\pgfqpoint{2.391806in}{0.667223in}}%
\pgfpathlineto{\pgfqpoint{2.394378in}{0.667223in}}%
\pgfpathlineto{\pgfqpoint{2.396950in}{0.650767in}}%
\pgfpathlineto{\pgfqpoint{2.404665in}{0.650767in}}%
\pgfpathlineto{\pgfqpoint{2.407237in}{0.634311in}}%
\pgfpathlineto{\pgfqpoint{2.456103in}{0.634311in}}%
\pgfpathlineto{\pgfqpoint{2.458675in}{0.650767in}}%
\pgfpathlineto{\pgfqpoint{2.486965in}{0.650767in}}%
\pgfpathlineto{\pgfqpoint{2.489537in}{0.667223in}}%
\pgfpathlineto{\pgfqpoint{2.494681in}{0.667223in}}%
\pgfpathlineto{\pgfqpoint{2.497253in}{0.683679in}}%
\pgfpathlineto{\pgfqpoint{2.543546in}{0.683679in}}%
\pgfpathlineto{\pgfqpoint{2.546118in}{0.700135in}}%
\pgfpathlineto{\pgfqpoint{2.584696in}{0.700135in}}%
\pgfpathlineto{\pgfqpoint{2.587268in}{0.683679in}}%
\pgfpathlineto{\pgfqpoint{2.646421in}{0.683679in}}%
\pgfpathlineto{\pgfqpoint{2.646421in}{0.683679in}}%
\pgfusepath{stroke}%
\end{pgfscope}%
\begin{pgfscope}%
\pgfpathrectangle{\pgfqpoint{0.488751in}{0.368545in}}{\pgfqpoint{2.260417in}{1.502439in}}%
\pgfusepath{clip}%
\pgfsetrectcap%
\pgfsetroundjoin%
\pgfsetlinewidth{0.803000pt}%
\definecolor{currentstroke}{rgb}{0.000000,0.356863,0.509804}%
\pgfsetstrokecolor{currentstroke}%
\pgfsetdash{}{0pt}%
\pgfpathmoveto{\pgfqpoint{0.591497in}{0.700135in}}%
\pgfpathlineto{\pgfqpoint{0.596641in}{0.700135in}}%
\pgfpathlineto{\pgfqpoint{0.601785in}{0.733047in}}%
\pgfpathlineto{\pgfqpoint{0.604356in}{0.733047in}}%
\pgfpathlineto{\pgfqpoint{0.606928in}{0.782415in}}%
\pgfpathlineto{\pgfqpoint{0.612072in}{0.782415in}}%
\pgfpathlineto{\pgfqpoint{0.617216in}{0.815327in}}%
\pgfpathlineto{\pgfqpoint{0.619788in}{0.815327in}}%
\pgfpathlineto{\pgfqpoint{0.622359in}{0.848239in}}%
\pgfpathlineto{\pgfqpoint{0.627503in}{0.881152in}}%
\pgfpathlineto{\pgfqpoint{0.632647in}{0.881152in}}%
\pgfpathlineto{\pgfqpoint{0.635219in}{0.914064in}}%
\pgfpathlineto{\pgfqpoint{0.640363in}{0.881152in}}%
\pgfpathlineto{\pgfqpoint{0.645506in}{0.881152in}}%
\pgfpathlineto{\pgfqpoint{0.648078in}{0.897608in}}%
\pgfpathlineto{\pgfqpoint{0.650650in}{0.881152in}}%
\pgfpathlineto{\pgfqpoint{0.655794in}{0.881152in}}%
\pgfpathlineto{\pgfqpoint{0.658366in}{0.864695in}}%
\pgfpathlineto{\pgfqpoint{0.663509in}{0.864695in}}%
\pgfpathlineto{\pgfqpoint{0.668653in}{0.897608in}}%
\pgfpathlineto{\pgfqpoint{0.671225in}{0.881152in}}%
\pgfpathlineto{\pgfqpoint{0.689228in}{0.881152in}}%
\pgfpathlineto{\pgfqpoint{0.691800in}{0.864695in}}%
\pgfpathlineto{\pgfqpoint{0.699516in}{0.864695in}}%
\pgfpathlineto{\pgfqpoint{0.702087in}{0.848239in}}%
\pgfpathlineto{\pgfqpoint{0.704659in}{0.848239in}}%
\pgfpathlineto{\pgfqpoint{0.707231in}{0.831783in}}%
\pgfpathlineto{\pgfqpoint{0.709803in}{0.831783in}}%
\pgfpathlineto{\pgfqpoint{0.712375in}{0.815327in}}%
\pgfpathlineto{\pgfqpoint{0.720091in}{0.815327in}}%
\pgfpathlineto{\pgfqpoint{0.722662in}{0.831783in}}%
\pgfpathlineto{\pgfqpoint{0.727806in}{0.798871in}}%
\pgfpathlineto{\pgfqpoint{0.730378in}{0.798871in}}%
\pgfpathlineto{\pgfqpoint{0.732950in}{0.815327in}}%
\pgfpathlineto{\pgfqpoint{0.735522in}{0.815327in}}%
\pgfpathlineto{\pgfqpoint{0.738094in}{0.798871in}}%
\pgfpathlineto{\pgfqpoint{0.750953in}{0.798871in}}%
\pgfpathlineto{\pgfqpoint{0.753525in}{0.831783in}}%
\pgfpathlineto{\pgfqpoint{0.756097in}{0.848239in}}%
\pgfpathlineto{\pgfqpoint{0.761240in}{0.848239in}}%
\pgfpathlineto{\pgfqpoint{0.763812in}{0.831783in}}%
\pgfpathlineto{\pgfqpoint{0.771528in}{0.831783in}}%
\pgfpathlineto{\pgfqpoint{0.774100in}{0.848239in}}%
\pgfpathlineto{\pgfqpoint{0.776672in}{0.831783in}}%
\pgfpathlineto{\pgfqpoint{0.779244in}{0.864695in}}%
\pgfpathlineto{\pgfqpoint{0.781815in}{0.831783in}}%
\pgfpathlineto{\pgfqpoint{0.789531in}{0.881152in}}%
\pgfpathlineto{\pgfqpoint{0.797247in}{0.881152in}}%
\pgfpathlineto{\pgfqpoint{0.799819in}{0.864695in}}%
\pgfpathlineto{\pgfqpoint{0.802390in}{0.881152in}}%
\pgfpathlineto{\pgfqpoint{0.804962in}{0.881152in}}%
\pgfpathlineto{\pgfqpoint{0.807534in}{0.897608in}}%
\pgfpathlineto{\pgfqpoint{0.833253in}{0.897608in}}%
\pgfpathlineto{\pgfqpoint{0.835825in}{0.914064in}}%
\pgfpathlineto{\pgfqpoint{0.838397in}{0.897608in}}%
\pgfpathlineto{\pgfqpoint{0.843540in}{0.897608in}}%
\pgfpathlineto{\pgfqpoint{0.846112in}{0.881152in}}%
\pgfpathlineto{\pgfqpoint{0.858972in}{0.881152in}}%
\pgfpathlineto{\pgfqpoint{0.861543in}{0.897608in}}%
\pgfpathlineto{\pgfqpoint{0.866687in}{0.897608in}}%
\pgfpathlineto{\pgfqpoint{0.869259in}{0.914064in}}%
\pgfpathlineto{\pgfqpoint{0.871831in}{0.914064in}}%
\pgfpathlineto{\pgfqpoint{0.879547in}{0.864695in}}%
\pgfpathlineto{\pgfqpoint{0.882118in}{0.881152in}}%
\pgfpathlineto{\pgfqpoint{0.884690in}{0.881152in}}%
\pgfpathlineto{\pgfqpoint{0.887262in}{0.897608in}}%
\pgfpathlineto{\pgfqpoint{0.892406in}{0.897608in}}%
\pgfpathlineto{\pgfqpoint{0.894978in}{0.914064in}}%
\pgfpathlineto{\pgfqpoint{0.902693in}{0.914064in}}%
\pgfpathlineto{\pgfqpoint{0.905265in}{0.881152in}}%
\pgfpathlineto{\pgfqpoint{0.915553in}{0.881152in}}%
\pgfpathlineto{\pgfqpoint{0.918125in}{0.897608in}}%
\pgfpathlineto{\pgfqpoint{0.930984in}{0.897608in}}%
\pgfpathlineto{\pgfqpoint{0.933556in}{0.930520in}}%
\pgfpathlineto{\pgfqpoint{0.943843in}{0.930520in}}%
\pgfpathlineto{\pgfqpoint{0.946415in}{0.914064in}}%
\pgfpathlineto{\pgfqpoint{0.954131in}{0.914064in}}%
\pgfpathlineto{\pgfqpoint{0.956703in}{0.897608in}}%
\pgfpathlineto{\pgfqpoint{0.959274in}{0.897608in}}%
\pgfpathlineto{\pgfqpoint{0.961846in}{0.914064in}}%
\pgfpathlineto{\pgfqpoint{0.964418in}{0.914064in}}%
\pgfpathlineto{\pgfqpoint{0.966990in}{0.881152in}}%
\pgfpathlineto{\pgfqpoint{0.969562in}{0.864695in}}%
\pgfpathlineto{\pgfqpoint{0.972134in}{0.864695in}}%
\pgfpathlineto{\pgfqpoint{0.974706in}{0.848239in}}%
\pgfpathlineto{\pgfqpoint{0.977278in}{0.864695in}}%
\pgfpathlineto{\pgfqpoint{0.979849in}{0.897608in}}%
\pgfpathlineto{\pgfqpoint{0.987565in}{0.897608in}}%
\pgfpathlineto{\pgfqpoint{0.990137in}{0.881152in}}%
\pgfpathlineto{\pgfqpoint{0.992709in}{0.881152in}}%
\pgfpathlineto{\pgfqpoint{0.995281in}{0.864695in}}%
\pgfpathlineto{\pgfqpoint{0.997853in}{0.881152in}}%
\pgfpathlineto{\pgfqpoint{1.005568in}{0.881152in}}%
\pgfpathlineto{\pgfqpoint{1.010712in}{0.848239in}}%
\pgfpathlineto{\pgfqpoint{1.013284in}{0.848239in}}%
\pgfpathlineto{\pgfqpoint{1.018427in}{0.881152in}}%
\pgfpathlineto{\pgfqpoint{1.028715in}{0.881152in}}%
\pgfpathlineto{\pgfqpoint{1.033859in}{0.848239in}}%
\pgfpathlineto{\pgfqpoint{1.039002in}{0.848239in}}%
\pgfpathlineto{\pgfqpoint{1.041574in}{0.864695in}}%
\pgfpathlineto{\pgfqpoint{1.051862in}{0.864695in}}%
\pgfpathlineto{\pgfqpoint{1.054434in}{0.848239in}}%
\pgfpathlineto{\pgfqpoint{1.077581in}{0.848239in}}%
\pgfpathlineto{\pgfqpoint{1.080152in}{0.864695in}}%
\pgfpathlineto{\pgfqpoint{1.087868in}{0.864695in}}%
\pgfpathlineto{\pgfqpoint{1.093012in}{0.897608in}}%
\pgfpathlineto{\pgfqpoint{1.095584in}{0.897608in}}%
\pgfpathlineto{\pgfqpoint{1.098155in}{0.914064in}}%
\pgfpathlineto{\pgfqpoint{1.100727in}{0.897608in}}%
\pgfpathlineto{\pgfqpoint{1.103299in}{0.897608in}}%
\pgfpathlineto{\pgfqpoint{1.105871in}{0.881152in}}%
\pgfpathlineto{\pgfqpoint{1.126446in}{0.881152in}}%
\pgfpathlineto{\pgfqpoint{1.129018in}{0.864695in}}%
\pgfpathlineto{\pgfqpoint{1.131590in}{0.864695in}}%
\pgfpathlineto{\pgfqpoint{1.134162in}{0.881152in}}%
\pgfpathlineto{\pgfqpoint{1.141877in}{0.881152in}}%
\pgfpathlineto{\pgfqpoint{1.144449in}{0.864695in}}%
\pgfpathlineto{\pgfqpoint{1.147021in}{0.864695in}}%
\pgfpathlineto{\pgfqpoint{1.152165in}{0.897608in}}%
\pgfpathlineto{\pgfqpoint{1.154737in}{0.881152in}}%
\pgfpathlineto{\pgfqpoint{1.157308in}{0.881152in}}%
\pgfpathlineto{\pgfqpoint{1.159880in}{0.864695in}}%
\pgfpathlineto{\pgfqpoint{1.195887in}{0.864695in}}%
\pgfpathlineto{\pgfqpoint{1.198458in}{0.848239in}}%
\pgfpathlineto{\pgfqpoint{1.201030in}{0.864695in}}%
\pgfpathlineto{\pgfqpoint{1.203602in}{0.864695in}}%
\pgfpathlineto{\pgfqpoint{1.206174in}{0.848239in}}%
\pgfpathlineto{\pgfqpoint{1.219033in}{0.848239in}}%
\pgfpathlineto{\pgfqpoint{1.221605in}{0.864695in}}%
\pgfpathlineto{\pgfqpoint{1.224177in}{0.864695in}}%
\pgfpathlineto{\pgfqpoint{1.226749in}{0.881152in}}%
\pgfpathlineto{\pgfqpoint{1.229321in}{0.864695in}}%
\pgfpathlineto{\pgfqpoint{1.231893in}{0.864695in}}%
\pgfpathlineto{\pgfqpoint{1.234465in}{0.848239in}}%
\pgfpathlineto{\pgfqpoint{1.237036in}{0.864695in}}%
\pgfpathlineto{\pgfqpoint{1.242180in}{0.864695in}}%
\pgfpathlineto{\pgfqpoint{1.244752in}{0.881152in}}%
\pgfpathlineto{\pgfqpoint{1.265327in}{0.881152in}}%
\pgfpathlineto{\pgfqpoint{1.267899in}{0.864695in}}%
\pgfpathlineto{\pgfqpoint{1.270471in}{0.864695in}}%
\pgfpathlineto{\pgfqpoint{1.273043in}{0.848239in}}%
\pgfpathlineto{\pgfqpoint{1.278186in}{0.848239in}}%
\pgfpathlineto{\pgfqpoint{1.280758in}{0.831783in}}%
\pgfpathlineto{\pgfqpoint{1.288474in}{0.831783in}}%
\pgfpathlineto{\pgfqpoint{1.291046in}{0.848239in}}%
\pgfpathlineto{\pgfqpoint{1.303905in}{0.848239in}}%
\pgfpathlineto{\pgfqpoint{1.306477in}{0.831783in}}%
\pgfpathlineto{\pgfqpoint{1.309049in}{0.831783in}}%
\pgfpathlineto{\pgfqpoint{1.311621in}{0.848239in}}%
\pgfpathlineto{\pgfqpoint{1.321908in}{0.848239in}}%
\pgfpathlineto{\pgfqpoint{1.324480in}{0.864695in}}%
\pgfpathlineto{\pgfqpoint{1.327052in}{0.848239in}}%
\pgfpathlineto{\pgfqpoint{1.332196in}{0.848239in}}%
\pgfpathlineto{\pgfqpoint{1.334768in}{0.864695in}}%
\pgfpathlineto{\pgfqpoint{1.337339in}{0.864695in}}%
\pgfpathlineto{\pgfqpoint{1.339911in}{0.848239in}}%
\pgfpathlineto{\pgfqpoint{1.342483in}{0.848239in}}%
\pgfpathlineto{\pgfqpoint{1.345055in}{0.831783in}}%
\pgfpathlineto{\pgfqpoint{1.347627in}{0.848239in}}%
\pgfpathlineto{\pgfqpoint{1.352771in}{0.848239in}}%
\pgfpathlineto{\pgfqpoint{1.355342in}{0.864695in}}%
\pgfpathlineto{\pgfqpoint{1.357914in}{0.848239in}}%
\pgfpathlineto{\pgfqpoint{1.360486in}{0.864695in}}%
\pgfpathlineto{\pgfqpoint{1.381061in}{0.864695in}}%
\pgfpathlineto{\pgfqpoint{1.383633in}{0.881152in}}%
\pgfpathlineto{\pgfqpoint{1.386205in}{0.881152in}}%
\pgfpathlineto{\pgfqpoint{1.388777in}{0.864695in}}%
\pgfpathlineto{\pgfqpoint{1.393921in}{0.864695in}}%
\pgfpathlineto{\pgfqpoint{1.399064in}{0.831783in}}%
\pgfpathlineto{\pgfqpoint{1.401636in}{0.848239in}}%
\pgfpathlineto{\pgfqpoint{1.406780in}{0.848239in}}%
\pgfpathlineto{\pgfqpoint{1.409352in}{0.881152in}}%
\pgfpathlineto{\pgfqpoint{1.432499in}{0.881152in}}%
\pgfpathlineto{\pgfqpoint{1.435070in}{0.897608in}}%
\pgfpathlineto{\pgfqpoint{1.442786in}{0.897608in}}%
\pgfpathlineto{\pgfqpoint{1.445358in}{0.914064in}}%
\pgfpathlineto{\pgfqpoint{1.455645in}{0.914064in}}%
\pgfpathlineto{\pgfqpoint{1.458217in}{0.897608in}}%
\pgfpathlineto{\pgfqpoint{1.460789in}{0.914064in}}%
\pgfpathlineto{\pgfqpoint{1.465933in}{0.914064in}}%
\pgfpathlineto{\pgfqpoint{1.468505in}{0.930520in}}%
\pgfpathlineto{\pgfqpoint{1.471077in}{0.914064in}}%
\pgfpathlineto{\pgfqpoint{1.473649in}{0.930520in}}%
\pgfpathlineto{\pgfqpoint{1.476220in}{0.930520in}}%
\pgfpathlineto{\pgfqpoint{1.478792in}{0.946976in}}%
\pgfpathlineto{\pgfqpoint{1.481364in}{0.946976in}}%
\pgfpathlineto{\pgfqpoint{1.483936in}{0.963432in}}%
\pgfpathlineto{\pgfqpoint{1.489080in}{0.963432in}}%
\pgfpathlineto{\pgfqpoint{1.491652in}{0.946976in}}%
\pgfpathlineto{\pgfqpoint{1.494223in}{0.963432in}}%
\pgfpathlineto{\pgfqpoint{1.496795in}{0.963432in}}%
\pgfpathlineto{\pgfqpoint{1.499367in}{0.946976in}}%
\pgfpathlineto{\pgfqpoint{1.509655in}{0.946976in}}%
\pgfpathlineto{\pgfqpoint{1.512227in}{0.963432in}}%
\pgfpathlineto{\pgfqpoint{1.525086in}{0.963432in}}%
\pgfpathlineto{\pgfqpoint{1.527658in}{0.946976in}}%
\pgfpathlineto{\pgfqpoint{1.530230in}{0.963432in}}%
\pgfpathlineto{\pgfqpoint{1.532802in}{0.963432in}}%
\pgfpathlineto{\pgfqpoint{1.535373in}{0.946976in}}%
\pgfpathlineto{\pgfqpoint{1.543089in}{0.946976in}}%
\pgfpathlineto{\pgfqpoint{1.545661in}{0.930520in}}%
\pgfpathlineto{\pgfqpoint{1.563664in}{0.930520in}}%
\pgfpathlineto{\pgfqpoint{1.566236in}{0.946976in}}%
\pgfpathlineto{\pgfqpoint{1.586811in}{0.946976in}}%
\pgfpathlineto{\pgfqpoint{1.589383in}{0.930520in}}%
\pgfpathlineto{\pgfqpoint{1.597098in}{0.930520in}}%
\pgfpathlineto{\pgfqpoint{1.602242in}{0.897608in}}%
\pgfpathlineto{\pgfqpoint{1.604814in}{0.914064in}}%
\pgfpathlineto{\pgfqpoint{1.617673in}{0.914064in}}%
\pgfpathlineto{\pgfqpoint{1.620245in}{0.930520in}}%
\pgfpathlineto{\pgfqpoint{1.627961in}{0.930520in}}%
\pgfpathlineto{\pgfqpoint{1.630533in}{0.946976in}}%
\pgfpathlineto{\pgfqpoint{1.669111in}{0.946976in}}%
\pgfpathlineto{\pgfqpoint{1.671683in}{0.963432in}}%
\pgfpathlineto{\pgfqpoint{1.676826in}{0.963432in}}%
\pgfpathlineto{\pgfqpoint{1.679398in}{0.946976in}}%
\pgfpathlineto{\pgfqpoint{1.699973in}{0.946976in}}%
\pgfpathlineto{\pgfqpoint{1.702545in}{0.963432in}}%
\pgfpathlineto{\pgfqpoint{1.710261in}{0.963432in}}%
\pgfpathlineto{\pgfqpoint{1.712832in}{0.979888in}}%
\pgfpathlineto{\pgfqpoint{1.733407in}{0.979888in}}%
\pgfpathlineto{\pgfqpoint{1.735979in}{0.963432in}}%
\pgfpathlineto{\pgfqpoint{1.741123in}{0.963432in}}%
\pgfpathlineto{\pgfqpoint{1.743695in}{0.979888in}}%
\pgfpathlineto{\pgfqpoint{1.777129in}{0.979888in}}%
\pgfpathlineto{\pgfqpoint{1.779701in}{0.996344in}}%
\pgfpathlineto{\pgfqpoint{1.946873in}{0.996344in}}%
\pgfpathlineto{\pgfqpoint{1.949444in}{1.012800in}}%
\pgfpathlineto{\pgfqpoint{2.155194in}{1.012800in}}%
\pgfpathlineto{\pgfqpoint{2.157766in}{1.029256in}}%
\pgfpathlineto{\pgfqpoint{2.646421in}{1.029256in}}%
\pgfpathlineto{\pgfqpoint{2.646421in}{1.029256in}}%
\pgfusepath{stroke}%
\end{pgfscope}%
\begin{pgfscope}%
\pgfpathrectangle{\pgfqpoint{0.488751in}{0.368545in}}{\pgfqpoint{2.260417in}{1.502439in}}%
\pgfusepath{clip}%
\pgfsetrectcap%
\pgfsetroundjoin%
\pgfsetlinewidth{0.803000pt}%
\definecolor{currentstroke}{rgb}{0.490196,0.588235,0.431373}%
\pgfsetstrokecolor{currentstroke}%
\pgfsetdash{}{0pt}%
\pgfpathmoveto{\pgfqpoint{0.591497in}{0.831783in}}%
\pgfpathlineto{\pgfqpoint{0.596641in}{0.831783in}}%
\pgfpathlineto{\pgfqpoint{0.599213in}{0.848239in}}%
\pgfpathlineto{\pgfqpoint{0.609500in}{0.848239in}}%
\pgfpathlineto{\pgfqpoint{0.612072in}{0.864695in}}%
\pgfpathlineto{\pgfqpoint{0.617216in}{0.864695in}}%
\pgfpathlineto{\pgfqpoint{0.619788in}{0.881152in}}%
\pgfpathlineto{\pgfqpoint{0.622359in}{0.881152in}}%
\pgfpathlineto{\pgfqpoint{0.624931in}{0.897608in}}%
\pgfpathlineto{\pgfqpoint{0.630075in}{0.897608in}}%
\pgfpathlineto{\pgfqpoint{0.632647in}{0.914064in}}%
\pgfpathlineto{\pgfqpoint{0.648078in}{0.914064in}}%
\pgfpathlineto{\pgfqpoint{0.650650in}{0.930520in}}%
\pgfpathlineto{\pgfqpoint{0.653222in}{0.930520in}}%
\pgfpathlineto{\pgfqpoint{0.655794in}{0.946976in}}%
\pgfpathlineto{\pgfqpoint{0.668653in}{0.946976in}}%
\pgfpathlineto{\pgfqpoint{0.671225in}{0.963432in}}%
\pgfpathlineto{\pgfqpoint{0.681513in}{0.963432in}}%
\pgfpathlineto{\pgfqpoint{0.686656in}{0.996344in}}%
\pgfpathlineto{\pgfqpoint{0.691800in}{0.996344in}}%
\pgfpathlineto{\pgfqpoint{0.694372in}{1.012800in}}%
\pgfpathlineto{\pgfqpoint{0.714947in}{1.012800in}}%
\pgfpathlineto{\pgfqpoint{0.717519in}{1.029256in}}%
\pgfpathlineto{\pgfqpoint{0.725234in}{1.029256in}}%
\pgfpathlineto{\pgfqpoint{0.727806in}{1.045712in}}%
\pgfpathlineto{\pgfqpoint{0.730378in}{1.045712in}}%
\pgfpathlineto{\pgfqpoint{0.732950in}{1.062168in}}%
\pgfpathlineto{\pgfqpoint{0.768956in}{1.062168in}}%
\pgfpathlineto{\pgfqpoint{0.771528in}{1.078624in}}%
\pgfpathlineto{\pgfqpoint{0.776672in}{1.078624in}}%
\pgfpathlineto{\pgfqpoint{0.784387in}{1.127993in}}%
\pgfpathlineto{\pgfqpoint{0.858972in}{1.127993in}}%
\pgfpathlineto{\pgfqpoint{0.861543in}{1.144449in}}%
\pgfpathlineto{\pgfqpoint{0.892406in}{1.144449in}}%
\pgfpathlineto{\pgfqpoint{0.894978in}{1.160905in}}%
\pgfpathlineto{\pgfqpoint{0.920696in}{1.160905in}}%
\pgfpathlineto{\pgfqpoint{0.923268in}{1.177361in}}%
\pgfpathlineto{\pgfqpoint{0.930984in}{1.177361in}}%
\pgfpathlineto{\pgfqpoint{0.933556in}{1.193817in}}%
\pgfpathlineto{\pgfqpoint{0.938700in}{1.193817in}}%
\pgfpathlineto{\pgfqpoint{0.941271in}{1.210273in}}%
\pgfpathlineto{\pgfqpoint{0.946415in}{1.210273in}}%
\pgfpathlineto{\pgfqpoint{0.948987in}{1.226729in}}%
\pgfpathlineto{\pgfqpoint{0.964418in}{1.226729in}}%
\pgfpathlineto{\pgfqpoint{0.966990in}{1.243185in}}%
\pgfpathlineto{\pgfqpoint{1.002996in}{1.243185in}}%
\pgfpathlineto{\pgfqpoint{1.005568in}{1.259641in}}%
\pgfpathlineto{\pgfqpoint{1.093012in}{1.259641in}}%
\pgfpathlineto{\pgfqpoint{1.098155in}{1.292553in}}%
\pgfpathlineto{\pgfqpoint{1.154737in}{1.292553in}}%
\pgfpathlineto{\pgfqpoint{1.157308in}{1.309009in}}%
\pgfpathlineto{\pgfqpoint{1.478792in}{1.309009in}}%
\pgfpathlineto{\pgfqpoint{1.481364in}{1.325465in}}%
\pgfpathlineto{\pgfqpoint{1.725692in}{1.325465in}}%
\pgfpathlineto{\pgfqpoint{1.728264in}{1.341921in}}%
\pgfpathlineto{\pgfqpoint{1.841426in}{1.341921in}}%
\pgfpathlineto{\pgfqpoint{1.843998in}{1.358378in}}%
\pgfpathlineto{\pgfqpoint{2.615559in}{1.358378in}}%
\pgfpathlineto{\pgfqpoint{2.618131in}{1.374834in}}%
\pgfpathlineto{\pgfqpoint{2.646421in}{1.374834in}}%
\pgfpathlineto{\pgfqpoint{2.646421in}{1.374834in}}%
\pgfusepath{stroke}%
\end{pgfscope}%
\begin{pgfscope}%
\pgfpathrectangle{\pgfqpoint{0.488751in}{0.368545in}}{\pgfqpoint{2.260417in}{1.502439in}}%
\pgfusepath{clip}%
\pgfsetrectcap%
\pgfsetroundjoin%
\pgfsetlinewidth{0.803000pt}%
\definecolor{currentstroke}{rgb}{0.843137,0.666667,0.313725}%
\pgfsetstrokecolor{currentstroke}%
\pgfsetdash{}{0pt}%
\pgfpathmoveto{\pgfqpoint{0.591497in}{0.667223in}}%
\pgfpathlineto{\pgfqpoint{0.594069in}{0.667223in}}%
\pgfpathlineto{\pgfqpoint{0.596641in}{0.683679in}}%
\pgfpathlineto{\pgfqpoint{0.601785in}{0.683679in}}%
\pgfpathlineto{\pgfqpoint{0.604356in}{0.700135in}}%
\pgfpathlineto{\pgfqpoint{0.609500in}{0.700135in}}%
\pgfpathlineto{\pgfqpoint{0.614644in}{0.733047in}}%
\pgfpathlineto{\pgfqpoint{0.624931in}{0.733047in}}%
\pgfpathlineto{\pgfqpoint{0.627503in}{0.749503in}}%
\pgfpathlineto{\pgfqpoint{0.630075in}{0.749503in}}%
\pgfpathlineto{\pgfqpoint{0.635219in}{0.782415in}}%
\pgfpathlineto{\pgfqpoint{0.637791in}{0.765959in}}%
\pgfpathlineto{\pgfqpoint{0.640363in}{0.782415in}}%
\pgfpathlineto{\pgfqpoint{0.653222in}{0.782415in}}%
\pgfpathlineto{\pgfqpoint{0.655794in}{0.798871in}}%
\pgfpathlineto{\pgfqpoint{0.658366in}{0.798871in}}%
\pgfpathlineto{\pgfqpoint{0.660938in}{0.815327in}}%
\pgfpathlineto{\pgfqpoint{0.663509in}{0.815327in}}%
\pgfpathlineto{\pgfqpoint{0.666081in}{0.831783in}}%
\pgfpathlineto{\pgfqpoint{0.671225in}{0.831783in}}%
\pgfpathlineto{\pgfqpoint{0.676369in}{0.798871in}}%
\pgfpathlineto{\pgfqpoint{0.678941in}{0.798871in}}%
\pgfpathlineto{\pgfqpoint{0.681513in}{0.815327in}}%
\pgfpathlineto{\pgfqpoint{0.684084in}{0.798871in}}%
\pgfpathlineto{\pgfqpoint{0.686656in}{0.798871in}}%
\pgfpathlineto{\pgfqpoint{0.689228in}{0.815327in}}%
\pgfpathlineto{\pgfqpoint{0.856400in}{0.815327in}}%
\pgfpathlineto{\pgfqpoint{0.858972in}{0.831783in}}%
\pgfpathlineto{\pgfqpoint{0.864115in}{0.831783in}}%
\pgfpathlineto{\pgfqpoint{0.866687in}{0.815327in}}%
\pgfpathlineto{\pgfqpoint{1.041574in}{0.815327in}}%
\pgfpathlineto{\pgfqpoint{1.044146in}{0.831783in}}%
\pgfpathlineto{\pgfqpoint{1.051862in}{0.831783in}}%
\pgfpathlineto{\pgfqpoint{1.054434in}{0.848239in}}%
\pgfpathlineto{\pgfqpoint{1.057006in}{0.848239in}}%
\pgfpathlineto{\pgfqpoint{1.059577in}{0.864695in}}%
\pgfpathlineto{\pgfqpoint{1.087868in}{0.864695in}}%
\pgfpathlineto{\pgfqpoint{1.090440in}{0.881152in}}%
\pgfpathlineto{\pgfqpoint{1.198458in}{0.881152in}}%
\pgfpathlineto{\pgfqpoint{1.201030in}{0.897608in}}%
\pgfpathlineto{\pgfqpoint{1.219033in}{0.897608in}}%
\pgfpathlineto{\pgfqpoint{1.221605in}{0.914064in}}%
\pgfpathlineto{\pgfqpoint{1.350199in}{0.914064in}}%
\pgfpathlineto{\pgfqpoint{1.352771in}{0.930520in}}%
\pgfpathlineto{\pgfqpoint{1.355342in}{0.930520in}}%
\pgfpathlineto{\pgfqpoint{1.357914in}{0.946976in}}%
\pgfpathlineto{\pgfqpoint{1.386205in}{0.946976in}}%
\pgfpathlineto{\pgfqpoint{1.388777in}{0.963432in}}%
\pgfpathlineto{\pgfqpoint{1.481364in}{0.963432in}}%
\pgfpathlineto{\pgfqpoint{1.483936in}{0.979888in}}%
\pgfpathlineto{\pgfqpoint{1.519942in}{0.979888in}}%
\pgfpathlineto{\pgfqpoint{1.522514in}{0.996344in}}%
\pgfpathlineto{\pgfqpoint{1.530230in}{0.996344in}}%
\pgfpathlineto{\pgfqpoint{1.532802in}{1.012800in}}%
\pgfpathlineto{\pgfqpoint{1.543089in}{1.012800in}}%
\pgfpathlineto{\pgfqpoint{1.545661in}{1.029256in}}%
\pgfpathlineto{\pgfqpoint{1.571380in}{1.029256in}}%
\pgfpathlineto{\pgfqpoint{1.573951in}{1.045712in}}%
\pgfpathlineto{\pgfqpoint{1.684542in}{1.045712in}}%
\pgfpathlineto{\pgfqpoint{1.687114in}{1.062168in}}%
\pgfpathlineto{\pgfqpoint{1.715404in}{1.062168in}}%
\pgfpathlineto{\pgfqpoint{1.717976in}{1.095080in}}%
\pgfpathlineto{\pgfqpoint{1.723120in}{1.095080in}}%
\pgfpathlineto{\pgfqpoint{1.725692in}{1.111536in}}%
\pgfpathlineto{\pgfqpoint{1.728264in}{1.095080in}}%
\pgfpathlineto{\pgfqpoint{1.733407in}{1.127993in}}%
\pgfpathlineto{\pgfqpoint{1.741123in}{1.127993in}}%
\pgfpathlineto{\pgfqpoint{1.743695in}{1.144449in}}%
\pgfpathlineto{\pgfqpoint{1.746267in}{1.127993in}}%
\pgfpathlineto{\pgfqpoint{1.748839in}{1.144449in}}%
\pgfpathlineto{\pgfqpoint{1.751410in}{1.144449in}}%
\pgfpathlineto{\pgfqpoint{1.753982in}{1.160905in}}%
\pgfpathlineto{\pgfqpoint{1.774557in}{1.160905in}}%
\pgfpathlineto{\pgfqpoint{1.777129in}{1.177361in}}%
\pgfpathlineto{\pgfqpoint{1.813135in}{1.177361in}}%
\pgfpathlineto{\pgfqpoint{1.815707in}{1.193817in}}%
\pgfpathlineto{\pgfqpoint{1.818279in}{1.193817in}}%
\pgfpathlineto{\pgfqpoint{1.820851in}{1.210273in}}%
\pgfpathlineto{\pgfqpoint{1.825995in}{1.210273in}}%
\pgfpathlineto{\pgfqpoint{1.828567in}{1.226729in}}%
\pgfpathlineto{\pgfqpoint{1.838854in}{1.226729in}}%
\pgfpathlineto{\pgfqpoint{1.841426in}{1.210273in}}%
\pgfpathlineto{\pgfqpoint{1.856857in}{1.210273in}}%
\pgfpathlineto{\pgfqpoint{1.859429in}{1.193817in}}%
\pgfpathlineto{\pgfqpoint{1.862001in}{1.210273in}}%
\pgfpathlineto{\pgfqpoint{1.874860in}{1.210273in}}%
\pgfpathlineto{\pgfqpoint{1.877432in}{1.226729in}}%
\pgfpathlineto{\pgfqpoint{1.921154in}{1.226729in}}%
\pgfpathlineto{\pgfqpoint{1.923726in}{1.210273in}}%
\pgfpathlineto{\pgfqpoint{1.957160in}{1.210273in}}%
\pgfpathlineto{\pgfqpoint{1.959732in}{1.193817in}}%
\pgfpathlineto{\pgfqpoint{1.980307in}{1.193817in}}%
\pgfpathlineto{\pgfqpoint{1.982879in}{1.210273in}}%
\pgfpathlineto{\pgfqpoint{1.998310in}{1.210273in}}%
\pgfpathlineto{\pgfqpoint{2.000882in}{1.193817in}}%
\pgfpathlineto{\pgfqpoint{2.013741in}{1.193817in}}%
\pgfpathlineto{\pgfqpoint{2.016313in}{1.177361in}}%
\pgfpathlineto{\pgfqpoint{2.031744in}{1.177361in}}%
\pgfpathlineto{\pgfqpoint{2.034316in}{1.160905in}}%
\pgfpathlineto{\pgfqpoint{2.039460in}{1.160905in}}%
\pgfpathlineto{\pgfqpoint{2.042032in}{1.177361in}}%
\pgfpathlineto{\pgfqpoint{2.083182in}{1.177361in}}%
\pgfpathlineto{\pgfqpoint{2.085754in}{1.193817in}}%
\pgfpathlineto{\pgfqpoint{2.093469in}{1.193817in}}%
\pgfpathlineto{\pgfqpoint{2.096041in}{1.210273in}}%
\pgfpathlineto{\pgfqpoint{2.121760in}{1.210273in}}%
\pgfpathlineto{\pgfqpoint{2.124332in}{1.226729in}}%
\pgfpathlineto{\pgfqpoint{2.142335in}{1.226729in}}%
\pgfpathlineto{\pgfqpoint{2.144907in}{1.243185in}}%
\pgfpathlineto{\pgfqpoint{2.211775in}{1.243185in}}%
\pgfpathlineto{\pgfqpoint{2.214347in}{1.226729in}}%
\pgfpathlineto{\pgfqpoint{2.222063in}{1.226729in}}%
\pgfpathlineto{\pgfqpoint{2.224635in}{1.210273in}}%
\pgfpathlineto{\pgfqpoint{2.229778in}{1.210273in}}%
\pgfpathlineto{\pgfqpoint{2.232350in}{1.226729in}}%
\pgfpathlineto{\pgfqpoint{2.242638in}{1.226729in}}%
\pgfpathlineto{\pgfqpoint{2.245210in}{1.243185in}}%
\pgfpathlineto{\pgfqpoint{2.247781in}{1.226729in}}%
\pgfpathlineto{\pgfqpoint{2.283788in}{1.226729in}}%
\pgfpathlineto{\pgfqpoint{2.286359in}{1.243185in}}%
\pgfpathlineto{\pgfqpoint{2.330081in}{1.243185in}}%
\pgfpathlineto{\pgfqpoint{2.332653in}{1.259641in}}%
\pgfpathlineto{\pgfqpoint{2.335225in}{1.259641in}}%
\pgfpathlineto{\pgfqpoint{2.337797in}{1.243185in}}%
\pgfpathlineto{\pgfqpoint{2.345512in}{1.243185in}}%
\pgfpathlineto{\pgfqpoint{2.348084in}{1.226729in}}%
\pgfpathlineto{\pgfqpoint{2.407237in}{1.226729in}}%
\pgfpathlineto{\pgfqpoint{2.409809in}{1.243185in}}%
\pgfpathlineto{\pgfqpoint{2.425240in}{1.243185in}}%
\pgfpathlineto{\pgfqpoint{2.427812in}{1.226729in}}%
\pgfpathlineto{\pgfqpoint{2.463819in}{1.226729in}}%
\pgfpathlineto{\pgfqpoint{2.468962in}{1.193817in}}%
\pgfpathlineto{\pgfqpoint{2.471534in}{1.210273in}}%
\pgfpathlineto{\pgfqpoint{2.538403in}{1.210273in}}%
\pgfpathlineto{\pgfqpoint{2.540975in}{1.226729in}}%
\pgfpathlineto{\pgfqpoint{2.646421in}{1.226729in}}%
\pgfpathlineto{\pgfqpoint{2.646421in}{1.226729in}}%
\pgfusepath{stroke}%
\end{pgfscope}%
\begin{pgfscope}%
\pgfpathrectangle{\pgfqpoint{0.488751in}{0.368545in}}{\pgfqpoint{2.260417in}{1.502439in}}%
\pgfusepath{clip}%
\pgfsetrectcap%
\pgfsetroundjoin%
\pgfsetlinewidth{0.803000pt}%
\definecolor{currentstroke}{rgb}{0.333333,0.333333,0.333333}%
\pgfsetstrokecolor{currentstroke}%
\pgfsetstrokeopacity{0.270000}%
\pgfsetdash{}{0pt}%
\pgfpathmoveto{\pgfqpoint{0.591497in}{1.374834in}}%
\pgfpathlineto{\pgfqpoint{0.612072in}{1.374834in}}%
\pgfpathlineto{\pgfqpoint{0.614644in}{1.391290in}}%
\pgfpathlineto{\pgfqpoint{0.676369in}{1.391290in}}%
\pgfpathlineto{\pgfqpoint{0.678941in}{1.407746in}}%
\pgfpathlineto{\pgfqpoint{0.684084in}{1.407746in}}%
\pgfpathlineto{\pgfqpoint{0.686656in}{1.424202in}}%
\pgfpathlineto{\pgfqpoint{0.694372in}{1.424202in}}%
\pgfpathlineto{\pgfqpoint{0.699516in}{1.391290in}}%
\pgfpathlineto{\pgfqpoint{0.702087in}{1.391290in}}%
\pgfpathlineto{\pgfqpoint{0.704659in}{1.374834in}}%
\pgfpathlineto{\pgfqpoint{0.707231in}{1.391290in}}%
\pgfpathlineto{\pgfqpoint{0.714947in}{1.391290in}}%
\pgfpathlineto{\pgfqpoint{0.722662in}{1.341921in}}%
\pgfpathlineto{\pgfqpoint{0.725234in}{1.341921in}}%
\pgfpathlineto{\pgfqpoint{0.727806in}{1.358378in}}%
\pgfpathlineto{\pgfqpoint{0.730378in}{1.358378in}}%
\pgfpathlineto{\pgfqpoint{0.732950in}{1.341921in}}%
\pgfpathlineto{\pgfqpoint{0.738094in}{1.341921in}}%
\pgfpathlineto{\pgfqpoint{0.740666in}{1.325465in}}%
\pgfpathlineto{\pgfqpoint{0.743237in}{1.325465in}}%
\pgfpathlineto{\pgfqpoint{0.745809in}{1.309009in}}%
\pgfpathlineto{\pgfqpoint{0.750953in}{1.309009in}}%
\pgfpathlineto{\pgfqpoint{0.753525in}{1.341921in}}%
\pgfpathlineto{\pgfqpoint{0.761240in}{1.341921in}}%
\pgfpathlineto{\pgfqpoint{0.763812in}{1.325465in}}%
\pgfpathlineto{\pgfqpoint{0.768956in}{1.325465in}}%
\pgfpathlineto{\pgfqpoint{0.771528in}{1.309009in}}%
\pgfpathlineto{\pgfqpoint{0.776672in}{1.309009in}}%
\pgfpathlineto{\pgfqpoint{0.779244in}{1.292553in}}%
\pgfpathlineto{\pgfqpoint{0.781815in}{1.309009in}}%
\pgfpathlineto{\pgfqpoint{0.794675in}{1.309009in}}%
\pgfpathlineto{\pgfqpoint{0.797247in}{1.325465in}}%
\pgfpathlineto{\pgfqpoint{0.799819in}{1.309009in}}%
\pgfpathlineto{\pgfqpoint{0.822965in}{1.309009in}}%
\pgfpathlineto{\pgfqpoint{0.825537in}{1.292553in}}%
\pgfpathlineto{\pgfqpoint{0.833253in}{1.292553in}}%
\pgfpathlineto{\pgfqpoint{0.835825in}{1.309009in}}%
\pgfpathlineto{\pgfqpoint{0.838397in}{1.309009in}}%
\pgfpathlineto{\pgfqpoint{0.840968in}{1.325465in}}%
\pgfpathlineto{\pgfqpoint{0.843540in}{1.325465in}}%
\pgfpathlineto{\pgfqpoint{0.846112in}{1.341921in}}%
\pgfpathlineto{\pgfqpoint{0.864115in}{1.341921in}}%
\pgfpathlineto{\pgfqpoint{0.866687in}{1.358378in}}%
\pgfpathlineto{\pgfqpoint{0.871831in}{1.358378in}}%
\pgfpathlineto{\pgfqpoint{0.874403in}{1.341921in}}%
\pgfpathlineto{\pgfqpoint{0.876975in}{1.341921in}}%
\pgfpathlineto{\pgfqpoint{0.879547in}{1.325465in}}%
\pgfpathlineto{\pgfqpoint{0.907837in}{1.325465in}}%
\pgfpathlineto{\pgfqpoint{0.915553in}{1.276097in}}%
\pgfpathlineto{\pgfqpoint{0.923268in}{1.276097in}}%
\pgfpathlineto{\pgfqpoint{0.925840in}{1.259641in}}%
\pgfpathlineto{\pgfqpoint{0.928412in}{1.276097in}}%
\pgfpathlineto{\pgfqpoint{0.930984in}{1.276097in}}%
\pgfpathlineto{\pgfqpoint{0.933556in}{1.292553in}}%
\pgfpathlineto{\pgfqpoint{0.936128in}{1.292553in}}%
\pgfpathlineto{\pgfqpoint{0.938700in}{1.276097in}}%
\pgfpathlineto{\pgfqpoint{0.943843in}{1.276097in}}%
\pgfpathlineto{\pgfqpoint{0.946415in}{1.259641in}}%
\pgfpathlineto{\pgfqpoint{0.948987in}{1.259641in}}%
\pgfpathlineto{\pgfqpoint{0.951559in}{1.243185in}}%
\pgfpathlineto{\pgfqpoint{0.954131in}{1.243185in}}%
\pgfpathlineto{\pgfqpoint{0.956703in}{1.210273in}}%
\pgfpathlineto{\pgfqpoint{0.959274in}{1.210273in}}%
\pgfpathlineto{\pgfqpoint{0.961846in}{1.177361in}}%
\pgfpathlineto{\pgfqpoint{0.974706in}{1.177361in}}%
\pgfpathlineto{\pgfqpoint{0.977278in}{1.193817in}}%
\pgfpathlineto{\pgfqpoint{0.982421in}{1.193817in}}%
\pgfpathlineto{\pgfqpoint{0.984993in}{1.177361in}}%
\pgfpathlineto{\pgfqpoint{0.987565in}{1.177361in}}%
\pgfpathlineto{\pgfqpoint{0.990137in}{1.210273in}}%
\pgfpathlineto{\pgfqpoint{1.005568in}{1.210273in}}%
\pgfpathlineto{\pgfqpoint{1.010712in}{1.177361in}}%
\pgfpathlineto{\pgfqpoint{1.015856in}{1.177361in}}%
\pgfpathlineto{\pgfqpoint{1.018427in}{1.193817in}}%
\pgfpathlineto{\pgfqpoint{1.041574in}{1.193817in}}%
\pgfpathlineto{\pgfqpoint{1.044146in}{1.210273in}}%
\pgfpathlineto{\pgfqpoint{1.072437in}{1.210273in}}%
\pgfpathlineto{\pgfqpoint{1.075009in}{1.226729in}}%
\pgfpathlineto{\pgfqpoint{1.126446in}{1.226729in}}%
\pgfpathlineto{\pgfqpoint{1.129018in}{1.243185in}}%
\pgfpathlineto{\pgfqpoint{1.152165in}{1.243185in}}%
\pgfpathlineto{\pgfqpoint{1.154737in}{1.226729in}}%
\pgfpathlineto{\pgfqpoint{1.170168in}{1.226729in}}%
\pgfpathlineto{\pgfqpoint{1.172740in}{1.243185in}}%
\pgfpathlineto{\pgfqpoint{1.175312in}{1.243185in}}%
\pgfpathlineto{\pgfqpoint{1.177883in}{1.210273in}}%
\pgfpathlineto{\pgfqpoint{1.180455in}{1.193817in}}%
\pgfpathlineto{\pgfqpoint{1.183027in}{1.193817in}}%
\pgfpathlineto{\pgfqpoint{1.185599in}{1.177361in}}%
\pgfpathlineto{\pgfqpoint{1.188171in}{1.177361in}}%
\pgfpathlineto{\pgfqpoint{1.190743in}{1.160905in}}%
\pgfpathlineto{\pgfqpoint{1.206174in}{1.160905in}}%
\pgfpathlineto{\pgfqpoint{1.208746in}{1.177361in}}%
\pgfpathlineto{\pgfqpoint{1.242180in}{1.177361in}}%
\pgfpathlineto{\pgfqpoint{1.244752in}{1.193817in}}%
\pgfpathlineto{\pgfqpoint{1.291046in}{1.193817in}}%
\pgfpathlineto{\pgfqpoint{1.293618in}{1.210273in}}%
\pgfpathlineto{\pgfqpoint{2.646421in}{1.210273in}}%
\pgfpathlineto{\pgfqpoint{2.646421in}{1.210273in}}%
\pgfusepath{stroke}%
\end{pgfscope}%
\begin{pgfscope}%
\pgfpathrectangle{\pgfqpoint{0.488751in}{0.368545in}}{\pgfqpoint{2.260417in}{1.502439in}}%
\pgfusepath{clip}%
\pgfsetrectcap%
\pgfsetroundjoin%
\pgfsetlinewidth{0.803000pt}%
\definecolor{currentstroke}{rgb}{0.686275,0.352941,0.313725}%
\pgfsetstrokecolor{currentstroke}%
\pgfsetstrokeopacity{0.270000}%
\pgfsetdash{}{0pt}%
\pgfpathmoveto{\pgfqpoint{0.591497in}{1.243185in}}%
\pgfpathlineto{\pgfqpoint{0.596641in}{1.309009in}}%
\pgfpathlineto{\pgfqpoint{0.601785in}{1.341921in}}%
\pgfpathlineto{\pgfqpoint{0.604356in}{1.341921in}}%
\pgfpathlineto{\pgfqpoint{0.606928in}{1.358378in}}%
\pgfpathlineto{\pgfqpoint{0.622359in}{1.358378in}}%
\pgfpathlineto{\pgfqpoint{0.624931in}{1.374834in}}%
\pgfpathlineto{\pgfqpoint{0.810106in}{1.374834in}}%
\pgfpathlineto{\pgfqpoint{0.812678in}{1.391290in}}%
\pgfpathlineto{\pgfqpoint{0.941271in}{1.391290in}}%
\pgfpathlineto{\pgfqpoint{0.943843in}{1.407746in}}%
\pgfpathlineto{\pgfqpoint{1.126446in}{1.407746in}}%
\pgfpathlineto{\pgfqpoint{1.129018in}{1.424202in}}%
\pgfpathlineto{\pgfqpoint{1.334768in}{1.424202in}}%
\pgfpathlineto{\pgfqpoint{1.337339in}{1.440658in}}%
\pgfpathlineto{\pgfqpoint{1.563664in}{1.440658in}}%
\pgfpathlineto{\pgfqpoint{1.566236in}{1.407746in}}%
\pgfpathlineto{\pgfqpoint{1.663967in}{1.407746in}}%
\pgfpathlineto{\pgfqpoint{1.666539in}{1.424202in}}%
\pgfpathlineto{\pgfqpoint{1.689686in}{1.424202in}}%
\pgfpathlineto{\pgfqpoint{1.692257in}{1.440658in}}%
\pgfpathlineto{\pgfqpoint{1.916010in}{1.440658in}}%
\pgfpathlineto{\pgfqpoint{1.918582in}{1.424202in}}%
\pgfpathlineto{\pgfqpoint{2.000882in}{1.424202in}}%
\pgfpathlineto{\pgfqpoint{2.003454in}{1.440658in}}%
\pgfpathlineto{\pgfqpoint{2.134619in}{1.440658in}}%
\pgfpathlineto{\pgfqpoint{2.137191in}{1.424202in}}%
\pgfpathlineto{\pgfqpoint{2.168053in}{1.424202in}}%
\pgfpathlineto{\pgfqpoint{2.170625in}{1.407746in}}%
\pgfpathlineto{\pgfqpoint{2.193772in}{1.407746in}}%
\pgfpathlineto{\pgfqpoint{2.196344in}{1.391290in}}%
\pgfpathlineto{\pgfqpoint{2.589840in}{1.391290in}}%
\pgfpathlineto{\pgfqpoint{2.592412in}{1.407746in}}%
\pgfpathlineto{\pgfqpoint{2.630990in}{1.407746in}}%
\pgfpathlineto{\pgfqpoint{2.633562in}{1.391290in}}%
\pgfpathlineto{\pgfqpoint{2.646421in}{1.391290in}}%
\pgfpathlineto{\pgfqpoint{2.646421in}{1.391290in}}%
\pgfusepath{stroke}%
\end{pgfscope}%
\begin{pgfscope}%
\pgfpathrectangle{\pgfqpoint{0.488751in}{0.368545in}}{\pgfqpoint{2.260417in}{1.502439in}}%
\pgfusepath{clip}%
\pgfsetrectcap%
\pgfsetroundjoin%
\pgfsetlinewidth{0.803000pt}%
\definecolor{currentstroke}{rgb}{0.000000,0.356863,0.509804}%
\pgfsetstrokecolor{currentstroke}%
\pgfsetstrokeopacity{0.270000}%
\pgfsetdash{}{0pt}%
\pgfpathmoveto{\pgfqpoint{0.591497in}{1.029256in}}%
\pgfpathlineto{\pgfqpoint{0.594069in}{1.062168in}}%
\pgfpathlineto{\pgfqpoint{0.596641in}{1.078624in}}%
\pgfpathlineto{\pgfqpoint{0.601785in}{1.078624in}}%
\pgfpathlineto{\pgfqpoint{0.604356in}{1.127993in}}%
\pgfpathlineto{\pgfqpoint{0.612072in}{1.127993in}}%
\pgfpathlineto{\pgfqpoint{0.614644in}{1.111536in}}%
\pgfpathlineto{\pgfqpoint{0.617216in}{1.127993in}}%
\pgfpathlineto{\pgfqpoint{0.622359in}{1.193817in}}%
\pgfpathlineto{\pgfqpoint{0.624931in}{1.177361in}}%
\pgfpathlineto{\pgfqpoint{0.627503in}{1.177361in}}%
\pgfpathlineto{\pgfqpoint{0.630075in}{1.193817in}}%
\pgfpathlineto{\pgfqpoint{0.632647in}{1.226729in}}%
\pgfpathlineto{\pgfqpoint{0.635219in}{1.243185in}}%
\pgfpathlineto{\pgfqpoint{0.637791in}{1.210273in}}%
\pgfpathlineto{\pgfqpoint{0.642934in}{1.243185in}}%
\pgfpathlineto{\pgfqpoint{0.645506in}{1.243185in}}%
\pgfpathlineto{\pgfqpoint{0.650650in}{1.309009in}}%
\pgfpathlineto{\pgfqpoint{0.653222in}{1.309009in}}%
\pgfpathlineto{\pgfqpoint{0.655794in}{1.325465in}}%
\pgfpathlineto{\pgfqpoint{0.660938in}{1.325465in}}%
\pgfpathlineto{\pgfqpoint{0.666081in}{1.358378in}}%
\pgfpathlineto{\pgfqpoint{0.668653in}{1.358378in}}%
\pgfpathlineto{\pgfqpoint{0.671225in}{1.341921in}}%
\pgfpathlineto{\pgfqpoint{0.673797in}{1.341921in}}%
\pgfpathlineto{\pgfqpoint{0.676369in}{1.325465in}}%
\pgfpathlineto{\pgfqpoint{0.678941in}{1.325465in}}%
\pgfpathlineto{\pgfqpoint{0.681513in}{1.309009in}}%
\pgfpathlineto{\pgfqpoint{0.686656in}{1.309009in}}%
\pgfpathlineto{\pgfqpoint{0.691800in}{1.276097in}}%
\pgfpathlineto{\pgfqpoint{0.694372in}{1.276097in}}%
\pgfpathlineto{\pgfqpoint{0.696944in}{1.259641in}}%
\pgfpathlineto{\pgfqpoint{0.704659in}{1.259641in}}%
\pgfpathlineto{\pgfqpoint{0.709803in}{1.292553in}}%
\pgfpathlineto{\pgfqpoint{0.720091in}{1.292553in}}%
\pgfpathlineto{\pgfqpoint{0.722662in}{1.276097in}}%
\pgfpathlineto{\pgfqpoint{0.725234in}{1.276097in}}%
\pgfpathlineto{\pgfqpoint{0.727806in}{1.292553in}}%
\pgfpathlineto{\pgfqpoint{0.732950in}{1.292553in}}%
\pgfpathlineto{\pgfqpoint{0.735522in}{1.276097in}}%
\pgfpathlineto{\pgfqpoint{0.740666in}{1.276097in}}%
\pgfpathlineto{\pgfqpoint{0.745809in}{1.243185in}}%
\pgfpathlineto{\pgfqpoint{0.753525in}{1.243185in}}%
\pgfpathlineto{\pgfqpoint{0.758669in}{1.210273in}}%
\pgfpathlineto{\pgfqpoint{0.763812in}{1.210273in}}%
\pgfpathlineto{\pgfqpoint{0.766384in}{1.226729in}}%
\pgfpathlineto{\pgfqpoint{0.768956in}{1.226729in}}%
\pgfpathlineto{\pgfqpoint{0.771528in}{1.243185in}}%
\pgfpathlineto{\pgfqpoint{0.776672in}{1.210273in}}%
\pgfpathlineto{\pgfqpoint{0.779244in}{1.226729in}}%
\pgfpathlineto{\pgfqpoint{0.781815in}{1.226729in}}%
\pgfpathlineto{\pgfqpoint{0.784387in}{1.243185in}}%
\pgfpathlineto{\pgfqpoint{0.786959in}{1.226729in}}%
\pgfpathlineto{\pgfqpoint{0.789531in}{1.226729in}}%
\pgfpathlineto{\pgfqpoint{0.792103in}{1.210273in}}%
\pgfpathlineto{\pgfqpoint{0.799819in}{1.210273in}}%
\pgfpathlineto{\pgfqpoint{0.802390in}{1.193817in}}%
\pgfpathlineto{\pgfqpoint{0.810106in}{1.193817in}}%
\pgfpathlineto{\pgfqpoint{0.812678in}{1.177361in}}%
\pgfpathlineto{\pgfqpoint{0.822965in}{1.177361in}}%
\pgfpathlineto{\pgfqpoint{0.825537in}{1.160905in}}%
\pgfpathlineto{\pgfqpoint{0.833253in}{1.160905in}}%
\pgfpathlineto{\pgfqpoint{0.835825in}{1.144449in}}%
\pgfpathlineto{\pgfqpoint{0.840968in}{1.144449in}}%
\pgfpathlineto{\pgfqpoint{0.843540in}{1.160905in}}%
\pgfpathlineto{\pgfqpoint{0.846112in}{1.160905in}}%
\pgfpathlineto{\pgfqpoint{0.848684in}{1.144449in}}%
\pgfpathlineto{\pgfqpoint{0.869259in}{1.144449in}}%
\pgfpathlineto{\pgfqpoint{0.871831in}{1.127993in}}%
\pgfpathlineto{\pgfqpoint{0.874403in}{1.127993in}}%
\pgfpathlineto{\pgfqpoint{0.876975in}{1.144449in}}%
\pgfpathlineto{\pgfqpoint{0.879547in}{1.127993in}}%
\pgfpathlineto{\pgfqpoint{0.882118in}{1.127993in}}%
\pgfpathlineto{\pgfqpoint{0.884690in}{1.144449in}}%
\pgfpathlineto{\pgfqpoint{0.892406in}{1.144449in}}%
\pgfpathlineto{\pgfqpoint{0.894978in}{1.160905in}}%
\pgfpathlineto{\pgfqpoint{0.897550in}{1.144449in}}%
\pgfpathlineto{\pgfqpoint{0.900121in}{1.144449in}}%
\pgfpathlineto{\pgfqpoint{0.905265in}{1.111536in}}%
\pgfpathlineto{\pgfqpoint{0.907837in}{1.127993in}}%
\pgfpathlineto{\pgfqpoint{0.918125in}{1.127993in}}%
\pgfpathlineto{\pgfqpoint{0.920696in}{1.160905in}}%
\pgfpathlineto{\pgfqpoint{0.923268in}{1.177361in}}%
\pgfpathlineto{\pgfqpoint{0.925840in}{1.160905in}}%
\pgfpathlineto{\pgfqpoint{0.928412in}{1.177361in}}%
\pgfpathlineto{\pgfqpoint{0.930984in}{1.160905in}}%
\pgfpathlineto{\pgfqpoint{0.933556in}{1.177361in}}%
\pgfpathlineto{\pgfqpoint{0.936128in}{1.177361in}}%
\pgfpathlineto{\pgfqpoint{0.938700in}{1.160905in}}%
\pgfpathlineto{\pgfqpoint{0.946415in}{1.160905in}}%
\pgfpathlineto{\pgfqpoint{0.948987in}{1.144449in}}%
\pgfpathlineto{\pgfqpoint{0.956703in}{1.144449in}}%
\pgfpathlineto{\pgfqpoint{0.964418in}{1.193817in}}%
\pgfpathlineto{\pgfqpoint{0.974706in}{1.193817in}}%
\pgfpathlineto{\pgfqpoint{0.977278in}{1.177361in}}%
\pgfpathlineto{\pgfqpoint{0.982421in}{1.177361in}}%
\pgfpathlineto{\pgfqpoint{0.987565in}{1.144449in}}%
\pgfpathlineto{\pgfqpoint{0.990137in}{1.144449in}}%
\pgfpathlineto{\pgfqpoint{0.997853in}{1.095080in}}%
\pgfpathlineto{\pgfqpoint{1.000424in}{1.095080in}}%
\pgfpathlineto{\pgfqpoint{1.005568in}{1.127993in}}%
\pgfpathlineto{\pgfqpoint{1.008140in}{1.095080in}}%
\pgfpathlineto{\pgfqpoint{1.010712in}{1.078624in}}%
\pgfpathlineto{\pgfqpoint{1.013284in}{1.078624in}}%
\pgfpathlineto{\pgfqpoint{1.018427in}{1.144449in}}%
\pgfpathlineto{\pgfqpoint{1.028715in}{1.144449in}}%
\pgfpathlineto{\pgfqpoint{1.031287in}{1.127993in}}%
\pgfpathlineto{\pgfqpoint{1.039002in}{1.127993in}}%
\pgfpathlineto{\pgfqpoint{1.041574in}{1.144449in}}%
\pgfpathlineto{\pgfqpoint{1.051862in}{1.144449in}}%
\pgfpathlineto{\pgfqpoint{1.054434in}{1.127993in}}%
\pgfpathlineto{\pgfqpoint{1.072437in}{1.127993in}}%
\pgfpathlineto{\pgfqpoint{1.075009in}{1.111536in}}%
\pgfpathlineto{\pgfqpoint{1.077581in}{1.127993in}}%
\pgfpathlineto{\pgfqpoint{1.085296in}{1.127993in}}%
\pgfpathlineto{\pgfqpoint{1.090440in}{1.160905in}}%
\pgfpathlineto{\pgfqpoint{1.098155in}{1.160905in}}%
\pgfpathlineto{\pgfqpoint{1.100727in}{1.127993in}}%
\pgfpathlineto{\pgfqpoint{1.108443in}{1.127993in}}%
\pgfpathlineto{\pgfqpoint{1.111015in}{1.144449in}}%
\pgfpathlineto{\pgfqpoint{1.118730in}{1.144449in}}%
\pgfpathlineto{\pgfqpoint{1.123874in}{1.111536in}}%
\pgfpathlineto{\pgfqpoint{1.131590in}{1.111536in}}%
\pgfpathlineto{\pgfqpoint{1.134162in}{1.095080in}}%
\pgfpathlineto{\pgfqpoint{1.139305in}{1.095080in}}%
\pgfpathlineto{\pgfqpoint{1.144449in}{1.127993in}}%
\pgfpathlineto{\pgfqpoint{1.170168in}{1.127993in}}%
\pgfpathlineto{\pgfqpoint{1.177883in}{1.078624in}}%
\pgfpathlineto{\pgfqpoint{1.188171in}{1.078624in}}%
\pgfpathlineto{\pgfqpoint{1.190743in}{1.095080in}}%
\pgfpathlineto{\pgfqpoint{1.193315in}{1.078624in}}%
\pgfpathlineto{\pgfqpoint{1.198458in}{1.078624in}}%
\pgfpathlineto{\pgfqpoint{1.201030in}{1.095080in}}%
\pgfpathlineto{\pgfqpoint{1.203602in}{1.062168in}}%
\pgfpathlineto{\pgfqpoint{1.208746in}{1.062168in}}%
\pgfpathlineto{\pgfqpoint{1.211318in}{1.078624in}}%
\pgfpathlineto{\pgfqpoint{1.213890in}{1.078624in}}%
\pgfpathlineto{\pgfqpoint{1.224177in}{1.144449in}}%
\pgfpathlineto{\pgfqpoint{1.231893in}{1.144449in}}%
\pgfpathlineto{\pgfqpoint{1.234465in}{1.127993in}}%
\pgfpathlineto{\pgfqpoint{1.244752in}{1.127993in}}%
\pgfpathlineto{\pgfqpoint{1.247324in}{1.111536in}}%
\pgfpathlineto{\pgfqpoint{1.257611in}{1.111536in}}%
\pgfpathlineto{\pgfqpoint{1.260183in}{1.095080in}}%
\pgfpathlineto{\pgfqpoint{1.262755in}{1.111536in}}%
\pgfpathlineto{\pgfqpoint{1.265327in}{1.111536in}}%
\pgfpathlineto{\pgfqpoint{1.267899in}{1.127993in}}%
\pgfpathlineto{\pgfqpoint{1.270471in}{1.111536in}}%
\pgfpathlineto{\pgfqpoint{1.280758in}{1.111536in}}%
\pgfpathlineto{\pgfqpoint{1.283330in}{1.095080in}}%
\pgfpathlineto{\pgfqpoint{1.285902in}{1.095080in}}%
\pgfpathlineto{\pgfqpoint{1.288474in}{1.078624in}}%
\pgfpathlineto{\pgfqpoint{1.291046in}{1.095080in}}%
\pgfpathlineto{\pgfqpoint{1.298761in}{1.045712in}}%
\pgfpathlineto{\pgfqpoint{1.301333in}{1.062168in}}%
\pgfpathlineto{\pgfqpoint{1.306477in}{1.062168in}}%
\pgfpathlineto{\pgfqpoint{1.309049in}{1.045712in}}%
\pgfpathlineto{\pgfqpoint{1.314193in}{1.045712in}}%
\pgfpathlineto{\pgfqpoint{1.316764in}{1.029256in}}%
\pgfpathlineto{\pgfqpoint{1.319336in}{1.029256in}}%
\pgfpathlineto{\pgfqpoint{1.321908in}{1.045712in}}%
\pgfpathlineto{\pgfqpoint{1.324480in}{1.078624in}}%
\pgfpathlineto{\pgfqpoint{1.327052in}{1.078624in}}%
\pgfpathlineto{\pgfqpoint{1.329624in}{1.062168in}}%
\pgfpathlineto{\pgfqpoint{1.332196in}{1.062168in}}%
\pgfpathlineto{\pgfqpoint{1.334768in}{1.045712in}}%
\pgfpathlineto{\pgfqpoint{1.337339in}{1.062168in}}%
\pgfpathlineto{\pgfqpoint{1.339911in}{1.062168in}}%
\pgfpathlineto{\pgfqpoint{1.342483in}{1.045712in}}%
\pgfpathlineto{\pgfqpoint{1.355342in}{1.045712in}}%
\pgfpathlineto{\pgfqpoint{1.357914in}{1.012800in}}%
\pgfpathlineto{\pgfqpoint{1.360486in}{1.029256in}}%
\pgfpathlineto{\pgfqpoint{1.363058in}{1.029256in}}%
\pgfpathlineto{\pgfqpoint{1.365630in}{1.045712in}}%
\pgfpathlineto{\pgfqpoint{1.386205in}{1.045712in}}%
\pgfpathlineto{\pgfqpoint{1.388777in}{1.062168in}}%
\pgfpathlineto{\pgfqpoint{1.391349in}{1.045712in}}%
\pgfpathlineto{\pgfqpoint{1.406780in}{1.045712in}}%
\pgfpathlineto{\pgfqpoint{1.409352in}{1.062168in}}%
\pgfpathlineto{\pgfqpoint{1.432499in}{1.062168in}}%
\pgfpathlineto{\pgfqpoint{1.437642in}{1.095080in}}%
\pgfpathlineto{\pgfqpoint{1.445358in}{1.095080in}}%
\pgfpathlineto{\pgfqpoint{1.447930in}{1.111536in}}%
\pgfpathlineto{\pgfqpoint{1.453074in}{1.111536in}}%
\pgfpathlineto{\pgfqpoint{1.458217in}{1.078624in}}%
\pgfpathlineto{\pgfqpoint{1.460789in}{1.078624in}}%
\pgfpathlineto{\pgfqpoint{1.463361in}{1.062168in}}%
\pgfpathlineto{\pgfqpoint{1.509655in}{1.062168in}}%
\pgfpathlineto{\pgfqpoint{1.512227in}{1.078624in}}%
\pgfpathlineto{\pgfqpoint{1.514798in}{1.078624in}}%
\pgfpathlineto{\pgfqpoint{1.517370in}{1.062168in}}%
\pgfpathlineto{\pgfqpoint{1.522514in}{1.062168in}}%
\pgfpathlineto{\pgfqpoint{1.527658in}{1.029256in}}%
\pgfpathlineto{\pgfqpoint{1.530230in}{1.045712in}}%
\pgfpathlineto{\pgfqpoint{1.537945in}{1.045712in}}%
\pgfpathlineto{\pgfqpoint{1.540517in}{1.062168in}}%
\pgfpathlineto{\pgfqpoint{1.543089in}{1.062168in}}%
\pgfpathlineto{\pgfqpoint{1.545661in}{1.045712in}}%
\pgfpathlineto{\pgfqpoint{1.579095in}{1.045712in}}%
\pgfpathlineto{\pgfqpoint{1.581667in}{1.029256in}}%
\pgfpathlineto{\pgfqpoint{1.586811in}{1.029256in}}%
\pgfpathlineto{\pgfqpoint{1.589383in}{1.045712in}}%
\pgfpathlineto{\pgfqpoint{1.599670in}{1.045712in}}%
\pgfpathlineto{\pgfqpoint{1.602242in}{1.029256in}}%
\pgfpathlineto{\pgfqpoint{1.604814in}{1.029256in}}%
\pgfpathlineto{\pgfqpoint{1.607386in}{1.045712in}}%
\pgfpathlineto{\pgfqpoint{1.609958in}{1.012800in}}%
\pgfpathlineto{\pgfqpoint{1.612529in}{1.012800in}}%
\pgfpathlineto{\pgfqpoint{1.615101in}{1.029256in}}%
\pgfpathlineto{\pgfqpoint{1.622817in}{1.029256in}}%
\pgfpathlineto{\pgfqpoint{1.625389in}{1.045712in}}%
\pgfpathlineto{\pgfqpoint{1.666539in}{1.045712in}}%
\pgfpathlineto{\pgfqpoint{1.669111in}{1.062168in}}%
\pgfpathlineto{\pgfqpoint{1.676826in}{1.062168in}}%
\pgfpathlineto{\pgfqpoint{1.679398in}{1.045712in}}%
\pgfpathlineto{\pgfqpoint{1.689686in}{1.045712in}}%
\pgfpathlineto{\pgfqpoint{1.692257in}{1.029256in}}%
\pgfpathlineto{\pgfqpoint{1.702545in}{1.029256in}}%
\pgfpathlineto{\pgfqpoint{1.705117in}{1.045712in}}%
\pgfpathlineto{\pgfqpoint{1.717976in}{1.045712in}}%
\pgfpathlineto{\pgfqpoint{1.720548in}{1.062168in}}%
\pgfpathlineto{\pgfqpoint{1.723120in}{1.045712in}}%
\pgfpathlineto{\pgfqpoint{1.761698in}{1.045712in}}%
\pgfpathlineto{\pgfqpoint{1.766842in}{1.078624in}}%
\pgfpathlineto{\pgfqpoint{1.800276in}{1.078624in}}%
\pgfpathlineto{\pgfqpoint{1.805420in}{1.111536in}}%
\pgfpathlineto{\pgfqpoint{1.807992in}{1.111536in}}%
\pgfpathlineto{\pgfqpoint{1.810563in}{1.095080in}}%
\pgfpathlineto{\pgfqpoint{1.820851in}{1.095080in}}%
\pgfpathlineto{\pgfqpoint{1.823423in}{1.111536in}}%
\pgfpathlineto{\pgfqpoint{1.825995in}{1.095080in}}%
\pgfpathlineto{\pgfqpoint{1.828567in}{1.111536in}}%
\pgfpathlineto{\pgfqpoint{1.851713in}{1.111536in}}%
\pgfpathlineto{\pgfqpoint{1.854285in}{1.127993in}}%
\pgfpathlineto{\pgfqpoint{1.880004in}{1.127993in}}%
\pgfpathlineto{\pgfqpoint{1.882576in}{1.144449in}}%
\pgfpathlineto{\pgfqpoint{1.895435in}{1.144449in}}%
\pgfpathlineto{\pgfqpoint{1.898007in}{1.160905in}}%
\pgfpathlineto{\pgfqpoint{1.908295in}{1.160905in}}%
\pgfpathlineto{\pgfqpoint{1.910866in}{1.177361in}}%
\pgfpathlineto{\pgfqpoint{1.926298in}{1.177361in}}%
\pgfpathlineto{\pgfqpoint{1.928870in}{1.160905in}}%
\pgfpathlineto{\pgfqpoint{1.931441in}{1.160905in}}%
\pgfpathlineto{\pgfqpoint{1.934013in}{1.144449in}}%
\pgfpathlineto{\pgfqpoint{1.952016in}{1.144449in}}%
\pgfpathlineto{\pgfqpoint{1.954588in}{1.160905in}}%
\pgfpathlineto{\pgfqpoint{1.964876in}{1.160905in}}%
\pgfpathlineto{\pgfqpoint{1.967448in}{1.144449in}}%
\pgfpathlineto{\pgfqpoint{1.977735in}{1.144449in}}%
\pgfpathlineto{\pgfqpoint{1.980307in}{1.127993in}}%
\pgfpathlineto{\pgfqpoint{1.982879in}{1.144449in}}%
\pgfpathlineto{\pgfqpoint{1.988023in}{1.144449in}}%
\pgfpathlineto{\pgfqpoint{1.990594in}{1.127993in}}%
\pgfpathlineto{\pgfqpoint{1.995738in}{1.127993in}}%
\pgfpathlineto{\pgfqpoint{1.998310in}{1.144449in}}%
\pgfpathlineto{\pgfqpoint{2.000882in}{1.144449in}}%
\pgfpathlineto{\pgfqpoint{2.003454in}{1.160905in}}%
\pgfpathlineto{\pgfqpoint{2.044604in}{1.160905in}}%
\pgfpathlineto{\pgfqpoint{2.047176in}{1.177361in}}%
\pgfpathlineto{\pgfqpoint{2.072894in}{1.177361in}}%
\pgfpathlineto{\pgfqpoint{2.075466in}{1.160905in}}%
\pgfpathlineto{\pgfqpoint{2.098613in}{1.160905in}}%
\pgfpathlineto{\pgfqpoint{2.101185in}{1.144449in}}%
\pgfpathlineto{\pgfqpoint{2.114044in}{1.144449in}}%
\pgfpathlineto{\pgfqpoint{2.116616in}{1.127993in}}%
\pgfpathlineto{\pgfqpoint{2.121760in}{1.127993in}}%
\pgfpathlineto{\pgfqpoint{2.124332in}{1.144449in}}%
\pgfpathlineto{\pgfqpoint{2.126904in}{1.127993in}}%
\pgfpathlineto{\pgfqpoint{2.157766in}{1.127993in}}%
\pgfpathlineto{\pgfqpoint{2.160338in}{1.111536in}}%
\pgfpathlineto{\pgfqpoint{2.170625in}{1.111536in}}%
\pgfpathlineto{\pgfqpoint{2.173197in}{1.095080in}}%
\pgfpathlineto{\pgfqpoint{2.175769in}{1.095080in}}%
\pgfpathlineto{\pgfqpoint{2.178341in}{1.078624in}}%
\pgfpathlineto{\pgfqpoint{2.180913in}{1.078624in}}%
\pgfpathlineto{\pgfqpoint{2.183485in}{1.062168in}}%
\pgfpathlineto{\pgfqpoint{2.201488in}{1.062168in}}%
\pgfpathlineto{\pgfqpoint{2.206631in}{1.095080in}}%
\pgfpathlineto{\pgfqpoint{2.209203in}{1.095080in}}%
\pgfpathlineto{\pgfqpoint{2.211775in}{1.078624in}}%
\pgfpathlineto{\pgfqpoint{2.219491in}{1.078624in}}%
\pgfpathlineto{\pgfqpoint{2.222063in}{1.062168in}}%
\pgfpathlineto{\pgfqpoint{2.224635in}{1.078624in}}%
\pgfpathlineto{\pgfqpoint{2.258069in}{1.078624in}}%
\pgfpathlineto{\pgfqpoint{2.260641in}{1.095080in}}%
\pgfpathlineto{\pgfqpoint{2.268356in}{1.095080in}}%
\pgfpathlineto{\pgfqpoint{2.270928in}{1.078624in}}%
\pgfpathlineto{\pgfqpoint{2.296647in}{1.078624in}}%
\pgfpathlineto{\pgfqpoint{2.299219in}{1.062168in}}%
\pgfpathlineto{\pgfqpoint{2.301791in}{1.062168in}}%
\pgfpathlineto{\pgfqpoint{2.304363in}{1.045712in}}%
\pgfpathlineto{\pgfqpoint{2.317222in}{1.045712in}}%
\pgfpathlineto{\pgfqpoint{2.319794in}{1.062168in}}%
\pgfpathlineto{\pgfqpoint{2.322366in}{1.062168in}}%
\pgfpathlineto{\pgfqpoint{2.324938in}{1.078624in}}%
\pgfpathlineto{\pgfqpoint{2.332653in}{1.078624in}}%
\pgfpathlineto{\pgfqpoint{2.335225in}{1.095080in}}%
\pgfpathlineto{\pgfqpoint{2.353228in}{1.095080in}}%
\pgfpathlineto{\pgfqpoint{2.355800in}{1.111536in}}%
\pgfpathlineto{\pgfqpoint{2.381519in}{1.111536in}}%
\pgfpathlineto{\pgfqpoint{2.384091in}{1.127993in}}%
\pgfpathlineto{\pgfqpoint{2.407237in}{1.127993in}}%
\pgfpathlineto{\pgfqpoint{2.409809in}{1.111536in}}%
\pgfpathlineto{\pgfqpoint{2.412381in}{1.111536in}}%
\pgfpathlineto{\pgfqpoint{2.414953in}{1.127993in}}%
\pgfpathlineto{\pgfqpoint{2.463819in}{1.127993in}}%
\pgfpathlineto{\pgfqpoint{2.466390in}{1.111536in}}%
\pgfpathlineto{\pgfqpoint{2.530687in}{1.111536in}}%
\pgfpathlineto{\pgfqpoint{2.533259in}{1.078624in}}%
\pgfpathlineto{\pgfqpoint{2.538403in}{1.078624in}}%
\pgfpathlineto{\pgfqpoint{2.540975in}{1.095080in}}%
\pgfpathlineto{\pgfqpoint{2.546118in}{1.095080in}}%
\pgfpathlineto{\pgfqpoint{2.548690in}{1.078624in}}%
\pgfpathlineto{\pgfqpoint{2.551262in}{1.078624in}}%
\pgfpathlineto{\pgfqpoint{2.553834in}{1.062168in}}%
\pgfpathlineto{\pgfqpoint{2.602699in}{1.062168in}}%
\pgfpathlineto{\pgfqpoint{2.605271in}{1.045712in}}%
\pgfpathlineto{\pgfqpoint{2.615559in}{1.045712in}}%
\pgfpathlineto{\pgfqpoint{2.618131in}{1.062168in}}%
\pgfpathlineto{\pgfqpoint{2.623274in}{1.062168in}}%
\pgfpathlineto{\pgfqpoint{2.625846in}{1.078624in}}%
\pgfpathlineto{\pgfqpoint{2.633562in}{1.078624in}}%
\pgfpathlineto{\pgfqpoint{2.636134in}{1.095080in}}%
\pgfpathlineto{\pgfqpoint{2.641278in}{1.095080in}}%
\pgfpathlineto{\pgfqpoint{2.643849in}{1.111536in}}%
\pgfpathlineto{\pgfqpoint{2.646421in}{1.111536in}}%
\pgfpathlineto{\pgfqpoint{2.646421in}{1.111536in}}%
\pgfusepath{stroke}%
\end{pgfscope}%
\begin{pgfscope}%
\pgfpathrectangle{\pgfqpoint{0.488751in}{0.368545in}}{\pgfqpoint{2.260417in}{1.502439in}}%
\pgfusepath{clip}%
\pgfsetrectcap%
\pgfsetroundjoin%
\pgfsetlinewidth{0.803000pt}%
\definecolor{currentstroke}{rgb}{0.490196,0.588235,0.431373}%
\pgfsetstrokecolor{currentstroke}%
\pgfsetstrokeopacity{0.270000}%
\pgfsetdash{}{0pt}%
\pgfpathmoveto{\pgfqpoint{0.591497in}{0.650767in}}%
\pgfpathlineto{\pgfqpoint{0.596641in}{0.683679in}}%
\pgfpathlineto{\pgfqpoint{0.599213in}{0.683679in}}%
\pgfpathlineto{\pgfqpoint{0.604356in}{0.716591in}}%
\pgfpathlineto{\pgfqpoint{0.606928in}{0.700135in}}%
\pgfpathlineto{\pgfqpoint{0.612072in}{0.700135in}}%
\pgfpathlineto{\pgfqpoint{0.614644in}{0.716591in}}%
\pgfpathlineto{\pgfqpoint{0.617216in}{0.749503in}}%
\pgfpathlineto{\pgfqpoint{0.619788in}{0.749503in}}%
\pgfpathlineto{\pgfqpoint{0.624931in}{0.782415in}}%
\pgfpathlineto{\pgfqpoint{0.627503in}{0.815327in}}%
\pgfpathlineto{\pgfqpoint{0.632647in}{0.815327in}}%
\pgfpathlineto{\pgfqpoint{0.635219in}{0.831783in}}%
\pgfpathlineto{\pgfqpoint{0.637791in}{0.831783in}}%
\pgfpathlineto{\pgfqpoint{0.640363in}{0.848239in}}%
\pgfpathlineto{\pgfqpoint{0.648078in}{0.848239in}}%
\pgfpathlineto{\pgfqpoint{0.650650in}{0.864695in}}%
\pgfpathlineto{\pgfqpoint{0.658366in}{0.864695in}}%
\pgfpathlineto{\pgfqpoint{0.660938in}{0.881152in}}%
\pgfpathlineto{\pgfqpoint{0.666081in}{0.881152in}}%
\pgfpathlineto{\pgfqpoint{0.668653in}{0.897608in}}%
\pgfpathlineto{\pgfqpoint{0.681513in}{0.897608in}}%
\pgfpathlineto{\pgfqpoint{0.684084in}{0.914064in}}%
\pgfpathlineto{\pgfqpoint{0.704659in}{0.914064in}}%
\pgfpathlineto{\pgfqpoint{0.707231in}{0.881152in}}%
\pgfpathlineto{\pgfqpoint{0.712375in}{0.881152in}}%
\pgfpathlineto{\pgfqpoint{0.714947in}{0.864695in}}%
\pgfpathlineto{\pgfqpoint{0.717519in}{0.864695in}}%
\pgfpathlineto{\pgfqpoint{0.720091in}{0.848239in}}%
\pgfpathlineto{\pgfqpoint{0.722662in}{0.864695in}}%
\pgfpathlineto{\pgfqpoint{0.725234in}{0.864695in}}%
\pgfpathlineto{\pgfqpoint{0.727806in}{0.881152in}}%
\pgfpathlineto{\pgfqpoint{0.732950in}{0.881152in}}%
\pgfpathlineto{\pgfqpoint{0.735522in}{0.864695in}}%
\pgfpathlineto{\pgfqpoint{0.748381in}{0.864695in}}%
\pgfpathlineto{\pgfqpoint{0.750953in}{0.881152in}}%
\pgfpathlineto{\pgfqpoint{0.781815in}{0.881152in}}%
\pgfpathlineto{\pgfqpoint{0.784387in}{0.897608in}}%
\pgfpathlineto{\pgfqpoint{0.789531in}{0.897608in}}%
\pgfpathlineto{\pgfqpoint{0.792103in}{0.881152in}}%
\pgfpathlineto{\pgfqpoint{0.794675in}{0.897608in}}%
\pgfpathlineto{\pgfqpoint{0.804962in}{0.897608in}}%
\pgfpathlineto{\pgfqpoint{0.807534in}{0.914064in}}%
\pgfpathlineto{\pgfqpoint{0.812678in}{0.914064in}}%
\pgfpathlineto{\pgfqpoint{0.815250in}{0.897608in}}%
\pgfpathlineto{\pgfqpoint{0.817822in}{0.914064in}}%
\pgfpathlineto{\pgfqpoint{0.822965in}{0.914064in}}%
\pgfpathlineto{\pgfqpoint{0.825537in}{0.946976in}}%
\pgfpathlineto{\pgfqpoint{0.828109in}{0.930520in}}%
\pgfpathlineto{\pgfqpoint{0.830681in}{0.946976in}}%
\pgfpathlineto{\pgfqpoint{0.833253in}{0.930520in}}%
\pgfpathlineto{\pgfqpoint{0.846112in}{0.930520in}}%
\pgfpathlineto{\pgfqpoint{0.848684in}{0.914064in}}%
\pgfpathlineto{\pgfqpoint{0.851256in}{0.914064in}}%
\pgfpathlineto{\pgfqpoint{0.856400in}{0.946976in}}%
\pgfpathlineto{\pgfqpoint{0.858972in}{0.946976in}}%
\pgfpathlineto{\pgfqpoint{0.861543in}{0.963432in}}%
\pgfpathlineto{\pgfqpoint{0.866687in}{0.963432in}}%
\pgfpathlineto{\pgfqpoint{0.869259in}{0.946976in}}%
\pgfpathlineto{\pgfqpoint{0.874403in}{0.946976in}}%
\pgfpathlineto{\pgfqpoint{0.876975in}{0.963432in}}%
\pgfpathlineto{\pgfqpoint{0.879547in}{0.946976in}}%
\pgfpathlineto{\pgfqpoint{0.884690in}{0.946976in}}%
\pgfpathlineto{\pgfqpoint{0.887262in}{0.930520in}}%
\pgfpathlineto{\pgfqpoint{0.889834in}{0.897608in}}%
\pgfpathlineto{\pgfqpoint{0.892406in}{0.897608in}}%
\pgfpathlineto{\pgfqpoint{0.894978in}{0.914064in}}%
\pgfpathlineto{\pgfqpoint{0.897550in}{0.897608in}}%
\pgfpathlineto{\pgfqpoint{0.902693in}{0.897608in}}%
\pgfpathlineto{\pgfqpoint{0.905265in}{0.881152in}}%
\pgfpathlineto{\pgfqpoint{0.907837in}{0.897608in}}%
\pgfpathlineto{\pgfqpoint{0.915553in}{0.897608in}}%
\pgfpathlineto{\pgfqpoint{0.918125in}{0.914064in}}%
\pgfpathlineto{\pgfqpoint{0.925840in}{0.914064in}}%
\pgfpathlineto{\pgfqpoint{0.928412in}{0.930520in}}%
\pgfpathlineto{\pgfqpoint{0.930984in}{0.930520in}}%
\pgfpathlineto{\pgfqpoint{0.933556in}{0.963432in}}%
\pgfpathlineto{\pgfqpoint{2.646421in}{0.963432in}}%
\pgfpathlineto{\pgfqpoint{2.646421in}{0.963432in}}%
\pgfusepath{stroke}%
\end{pgfscope}%
\begin{pgfscope}%
\pgfpathrectangle{\pgfqpoint{0.488751in}{0.368545in}}{\pgfqpoint{2.260417in}{1.502439in}}%
\pgfusepath{clip}%
\pgfsetrectcap%
\pgfsetroundjoin%
\pgfsetlinewidth{0.803000pt}%
\definecolor{currentstroke}{rgb}{0.843137,0.666667,0.313725}%
\pgfsetstrokecolor{currentstroke}%
\pgfsetstrokeopacity{0.270000}%
\pgfsetdash{}{0pt}%
\pgfpathmoveto{\pgfqpoint{0.591497in}{0.897608in}}%
\pgfpathlineto{\pgfqpoint{0.596641in}{0.930520in}}%
\pgfpathlineto{\pgfqpoint{0.599213in}{0.930520in}}%
\pgfpathlineto{\pgfqpoint{0.604356in}{0.963432in}}%
\pgfpathlineto{\pgfqpoint{0.609500in}{0.963432in}}%
\pgfpathlineto{\pgfqpoint{0.612072in}{0.979888in}}%
\pgfpathlineto{\pgfqpoint{0.617216in}{0.979888in}}%
\pgfpathlineto{\pgfqpoint{0.619788in}{0.996344in}}%
\pgfpathlineto{\pgfqpoint{0.630075in}{0.996344in}}%
\pgfpathlineto{\pgfqpoint{0.632647in}{1.012800in}}%
\pgfpathlineto{\pgfqpoint{0.635219in}{1.012800in}}%
\pgfpathlineto{\pgfqpoint{0.637791in}{0.996344in}}%
\pgfpathlineto{\pgfqpoint{0.653222in}{0.996344in}}%
\pgfpathlineto{\pgfqpoint{0.655794in}{1.012800in}}%
\pgfpathlineto{\pgfqpoint{0.658366in}{1.012800in}}%
\pgfpathlineto{\pgfqpoint{0.660938in}{1.045712in}}%
\pgfpathlineto{\pgfqpoint{0.663509in}{1.045712in}}%
\pgfpathlineto{\pgfqpoint{0.666081in}{1.029256in}}%
\pgfpathlineto{\pgfqpoint{0.668653in}{1.029256in}}%
\pgfpathlineto{\pgfqpoint{0.671225in}{1.045712in}}%
\pgfpathlineto{\pgfqpoint{0.673797in}{1.029256in}}%
\pgfpathlineto{\pgfqpoint{0.678941in}{1.029256in}}%
\pgfpathlineto{\pgfqpoint{0.681513in}{1.012800in}}%
\pgfpathlineto{\pgfqpoint{0.686656in}{1.012800in}}%
\pgfpathlineto{\pgfqpoint{0.689228in}{1.029256in}}%
\pgfpathlineto{\pgfqpoint{0.691800in}{1.012800in}}%
\pgfpathlineto{\pgfqpoint{0.699516in}{1.012800in}}%
\pgfpathlineto{\pgfqpoint{0.702087in}{0.996344in}}%
\pgfpathlineto{\pgfqpoint{0.712375in}{0.996344in}}%
\pgfpathlineto{\pgfqpoint{0.714947in}{0.979888in}}%
\pgfpathlineto{\pgfqpoint{0.720091in}{1.012800in}}%
\pgfpathlineto{\pgfqpoint{0.725234in}{0.979888in}}%
\pgfpathlineto{\pgfqpoint{0.738094in}{0.979888in}}%
\pgfpathlineto{\pgfqpoint{0.740666in}{1.012800in}}%
\pgfpathlineto{\pgfqpoint{0.743237in}{1.012800in}}%
\pgfpathlineto{\pgfqpoint{0.745809in}{1.029256in}}%
\pgfpathlineto{\pgfqpoint{0.756097in}{1.029256in}}%
\pgfpathlineto{\pgfqpoint{0.758669in}{1.012800in}}%
\pgfpathlineto{\pgfqpoint{0.771528in}{1.012800in}}%
\pgfpathlineto{\pgfqpoint{0.774100in}{1.029256in}}%
\pgfpathlineto{\pgfqpoint{0.779244in}{0.996344in}}%
\pgfpathlineto{\pgfqpoint{0.781815in}{1.012800in}}%
\pgfpathlineto{\pgfqpoint{0.789531in}{1.012800in}}%
\pgfpathlineto{\pgfqpoint{0.792103in}{0.996344in}}%
\pgfpathlineto{\pgfqpoint{0.794675in}{0.996344in}}%
\pgfpathlineto{\pgfqpoint{0.797247in}{1.012800in}}%
\pgfpathlineto{\pgfqpoint{0.802390in}{0.979888in}}%
\pgfpathlineto{\pgfqpoint{0.807534in}{1.012800in}}%
\pgfpathlineto{\pgfqpoint{0.815250in}{1.012800in}}%
\pgfpathlineto{\pgfqpoint{0.820393in}{1.045712in}}%
\pgfpathlineto{\pgfqpoint{0.822965in}{1.045712in}}%
\pgfpathlineto{\pgfqpoint{0.825537in}{1.062168in}}%
\pgfpathlineto{\pgfqpoint{0.828109in}{1.062168in}}%
\pgfpathlineto{\pgfqpoint{0.830681in}{1.045712in}}%
\pgfpathlineto{\pgfqpoint{0.833253in}{1.045712in}}%
\pgfpathlineto{\pgfqpoint{0.835825in}{1.078624in}}%
\pgfpathlineto{\pgfqpoint{0.838397in}{1.062168in}}%
\pgfpathlineto{\pgfqpoint{0.840968in}{1.062168in}}%
\pgfpathlineto{\pgfqpoint{0.843540in}{1.045712in}}%
\pgfpathlineto{\pgfqpoint{0.846112in}{1.045712in}}%
\pgfpathlineto{\pgfqpoint{0.848684in}{1.012800in}}%
\pgfpathlineto{\pgfqpoint{0.853828in}{1.012800in}}%
\pgfpathlineto{\pgfqpoint{0.856400in}{1.029256in}}%
\pgfpathlineto{\pgfqpoint{0.866687in}{1.029256in}}%
\pgfpathlineto{\pgfqpoint{0.869259in}{1.045712in}}%
\pgfpathlineto{\pgfqpoint{0.871831in}{1.045712in}}%
\pgfpathlineto{\pgfqpoint{0.876975in}{1.012800in}}%
\pgfpathlineto{\pgfqpoint{0.882118in}{1.012800in}}%
\pgfpathlineto{\pgfqpoint{0.887262in}{1.045712in}}%
\pgfpathlineto{\pgfqpoint{0.897550in}{1.045712in}}%
\pgfpathlineto{\pgfqpoint{0.900121in}{1.062168in}}%
\pgfpathlineto{\pgfqpoint{0.902693in}{1.062168in}}%
\pgfpathlineto{\pgfqpoint{0.907837in}{1.095080in}}%
\pgfpathlineto{\pgfqpoint{0.910409in}{1.095080in}}%
\pgfpathlineto{\pgfqpoint{0.912981in}{1.078624in}}%
\pgfpathlineto{\pgfqpoint{0.915553in}{1.078624in}}%
\pgfpathlineto{\pgfqpoint{0.918125in}{1.095080in}}%
\pgfpathlineto{\pgfqpoint{0.920696in}{1.095080in}}%
\pgfpathlineto{\pgfqpoint{0.923268in}{1.111536in}}%
\pgfpathlineto{\pgfqpoint{0.925840in}{1.095080in}}%
\pgfpathlineto{\pgfqpoint{0.928412in}{1.095080in}}%
\pgfpathlineto{\pgfqpoint{0.933556in}{1.127993in}}%
\pgfpathlineto{\pgfqpoint{0.938700in}{1.127993in}}%
\pgfpathlineto{\pgfqpoint{0.941271in}{1.144449in}}%
\pgfpathlineto{\pgfqpoint{0.951559in}{1.144449in}}%
\pgfpathlineto{\pgfqpoint{0.954131in}{1.127993in}}%
\pgfpathlineto{\pgfqpoint{0.959274in}{1.127993in}}%
\pgfpathlineto{\pgfqpoint{0.961846in}{1.111536in}}%
\pgfpathlineto{\pgfqpoint{0.964418in}{1.127993in}}%
\pgfpathlineto{\pgfqpoint{0.969562in}{1.127993in}}%
\pgfpathlineto{\pgfqpoint{0.972134in}{1.144449in}}%
\pgfpathlineto{\pgfqpoint{0.974706in}{1.144449in}}%
\pgfpathlineto{\pgfqpoint{0.977278in}{1.127993in}}%
\pgfpathlineto{\pgfqpoint{0.984993in}{1.127993in}}%
\pgfpathlineto{\pgfqpoint{0.987565in}{1.111536in}}%
\pgfpathlineto{\pgfqpoint{0.992709in}{1.111536in}}%
\pgfpathlineto{\pgfqpoint{0.997853in}{1.144449in}}%
\pgfpathlineto{\pgfqpoint{1.000424in}{1.144449in}}%
\pgfpathlineto{\pgfqpoint{1.002996in}{1.160905in}}%
\pgfpathlineto{\pgfqpoint{1.005568in}{1.160905in}}%
\pgfpathlineto{\pgfqpoint{1.008140in}{1.177361in}}%
\pgfpathlineto{\pgfqpoint{1.026143in}{1.177361in}}%
\pgfpathlineto{\pgfqpoint{1.028715in}{1.160905in}}%
\pgfpathlineto{\pgfqpoint{1.036431in}{1.160905in}}%
\pgfpathlineto{\pgfqpoint{1.039002in}{1.177361in}}%
\pgfpathlineto{\pgfqpoint{1.054434in}{1.177361in}}%
\pgfpathlineto{\pgfqpoint{1.057006in}{1.193817in}}%
\pgfpathlineto{\pgfqpoint{1.062149in}{1.193817in}}%
\pgfpathlineto{\pgfqpoint{1.064721in}{1.177361in}}%
\pgfpathlineto{\pgfqpoint{1.067293in}{1.210273in}}%
\pgfpathlineto{\pgfqpoint{1.085296in}{1.210273in}}%
\pgfpathlineto{\pgfqpoint{1.087868in}{1.193817in}}%
\pgfpathlineto{\pgfqpoint{1.090440in}{1.193817in}}%
\pgfpathlineto{\pgfqpoint{1.093012in}{1.210273in}}%
\pgfpathlineto{\pgfqpoint{1.095584in}{1.210273in}}%
\pgfpathlineto{\pgfqpoint{1.098155in}{1.193817in}}%
\pgfpathlineto{\pgfqpoint{1.111015in}{1.193817in}}%
\pgfpathlineto{\pgfqpoint{1.113587in}{1.177361in}}%
\pgfpathlineto{\pgfqpoint{1.121302in}{1.226729in}}%
\pgfpathlineto{\pgfqpoint{1.126446in}{1.226729in}}%
\pgfpathlineto{\pgfqpoint{1.129018in}{1.243185in}}%
\pgfpathlineto{\pgfqpoint{1.147021in}{1.243185in}}%
\pgfpathlineto{\pgfqpoint{1.149593in}{1.259641in}}%
\pgfpathlineto{\pgfqpoint{1.157308in}{1.210273in}}%
\pgfpathlineto{\pgfqpoint{1.170168in}{1.210273in}}%
\pgfpathlineto{\pgfqpoint{1.172740in}{1.226729in}}%
\pgfpathlineto{\pgfqpoint{1.175312in}{1.226729in}}%
\pgfpathlineto{\pgfqpoint{1.177883in}{1.210273in}}%
\pgfpathlineto{\pgfqpoint{1.180455in}{1.210273in}}%
\pgfpathlineto{\pgfqpoint{1.183027in}{1.226729in}}%
\pgfpathlineto{\pgfqpoint{1.198458in}{1.226729in}}%
\pgfpathlineto{\pgfqpoint{1.201030in}{1.259641in}}%
\pgfpathlineto{\pgfqpoint{1.203602in}{1.259641in}}%
\pgfpathlineto{\pgfqpoint{1.206174in}{1.243185in}}%
\pgfpathlineto{\pgfqpoint{1.208746in}{1.259641in}}%
\pgfpathlineto{\pgfqpoint{1.213890in}{1.259641in}}%
\pgfpathlineto{\pgfqpoint{1.216461in}{1.243185in}}%
\pgfpathlineto{\pgfqpoint{1.221605in}{1.276097in}}%
\pgfpathlineto{\pgfqpoint{1.224177in}{1.276097in}}%
\pgfpathlineto{\pgfqpoint{1.226749in}{1.325465in}}%
\pgfpathlineto{\pgfqpoint{1.229321in}{1.309009in}}%
\pgfpathlineto{\pgfqpoint{1.231893in}{1.309009in}}%
\pgfpathlineto{\pgfqpoint{1.237036in}{1.341921in}}%
\pgfpathlineto{\pgfqpoint{1.239608in}{1.341921in}}%
\pgfpathlineto{\pgfqpoint{1.242180in}{1.325465in}}%
\pgfpathlineto{\pgfqpoint{1.260183in}{1.325465in}}%
\pgfpathlineto{\pgfqpoint{1.265327in}{1.358378in}}%
\pgfpathlineto{\pgfqpoint{1.270471in}{1.358378in}}%
\pgfpathlineto{\pgfqpoint{1.275615in}{1.325465in}}%
\pgfpathlineto{\pgfqpoint{1.288474in}{1.325465in}}%
\pgfpathlineto{\pgfqpoint{1.291046in}{1.341921in}}%
\pgfpathlineto{\pgfqpoint{1.309049in}{1.341921in}}%
\pgfpathlineto{\pgfqpoint{1.311621in}{1.358378in}}%
\pgfpathlineto{\pgfqpoint{1.321908in}{1.358378in}}%
\pgfpathlineto{\pgfqpoint{1.324480in}{1.374834in}}%
\pgfpathlineto{\pgfqpoint{1.342483in}{1.374834in}}%
\pgfpathlineto{\pgfqpoint{1.345055in}{1.358378in}}%
\pgfpathlineto{\pgfqpoint{1.350199in}{1.358378in}}%
\pgfpathlineto{\pgfqpoint{1.352771in}{1.374834in}}%
\pgfpathlineto{\pgfqpoint{1.357914in}{1.374834in}}%
\pgfpathlineto{\pgfqpoint{1.360486in}{1.391290in}}%
\pgfpathlineto{\pgfqpoint{1.373346in}{1.391290in}}%
\pgfpathlineto{\pgfqpoint{1.375917in}{1.407746in}}%
\pgfpathlineto{\pgfqpoint{1.391349in}{1.407746in}}%
\pgfpathlineto{\pgfqpoint{1.393921in}{1.424202in}}%
\pgfpathlineto{\pgfqpoint{1.406780in}{1.424202in}}%
\pgfpathlineto{\pgfqpoint{1.409352in}{1.457114in}}%
\pgfpathlineto{\pgfqpoint{1.414495in}{1.457114in}}%
\pgfpathlineto{\pgfqpoint{1.417067in}{1.440658in}}%
\pgfpathlineto{\pgfqpoint{1.419639in}{1.440658in}}%
\pgfpathlineto{\pgfqpoint{1.424783in}{1.407746in}}%
\pgfpathlineto{\pgfqpoint{1.427355in}{1.407746in}}%
\pgfpathlineto{\pgfqpoint{1.429927in}{1.391290in}}%
\pgfpathlineto{\pgfqpoint{1.445358in}{1.391290in}}%
\pgfpathlineto{\pgfqpoint{1.447930in}{1.374834in}}%
\pgfpathlineto{\pgfqpoint{1.455645in}{1.374834in}}%
\pgfpathlineto{\pgfqpoint{1.458217in}{1.358378in}}%
\pgfpathlineto{\pgfqpoint{1.465933in}{1.358378in}}%
\pgfpathlineto{\pgfqpoint{1.468505in}{1.374834in}}%
\pgfpathlineto{\pgfqpoint{1.494223in}{1.374834in}}%
\pgfpathlineto{\pgfqpoint{1.496795in}{1.341921in}}%
\pgfpathlineto{\pgfqpoint{1.499367in}{1.341921in}}%
\pgfpathlineto{\pgfqpoint{1.504511in}{1.374834in}}%
\pgfpathlineto{\pgfqpoint{1.509655in}{1.374834in}}%
\pgfpathlineto{\pgfqpoint{1.512227in}{1.407746in}}%
\pgfpathlineto{\pgfqpoint{1.517370in}{1.374834in}}%
\pgfpathlineto{\pgfqpoint{1.530230in}{1.374834in}}%
\pgfpathlineto{\pgfqpoint{1.532802in}{1.391290in}}%
\pgfpathlineto{\pgfqpoint{1.561092in}{1.391290in}}%
\pgfpathlineto{\pgfqpoint{1.563664in}{1.407746in}}%
\pgfpathlineto{\pgfqpoint{1.568808in}{1.407746in}}%
\pgfpathlineto{\pgfqpoint{1.571380in}{1.391290in}}%
\pgfpathlineto{\pgfqpoint{1.573951in}{1.391290in}}%
\pgfpathlineto{\pgfqpoint{1.576523in}{1.374834in}}%
\pgfpathlineto{\pgfqpoint{1.584239in}{1.374834in}}%
\pgfpathlineto{\pgfqpoint{1.586811in}{1.391290in}}%
\pgfpathlineto{\pgfqpoint{1.615101in}{1.391290in}}%
\pgfpathlineto{\pgfqpoint{1.617673in}{1.407746in}}%
\pgfpathlineto{\pgfqpoint{1.643392in}{1.407746in}}%
\pgfpathlineto{\pgfqpoint{1.645964in}{1.424202in}}%
\pgfpathlineto{\pgfqpoint{2.404665in}{1.424202in}}%
\pgfpathlineto{\pgfqpoint{2.407237in}{1.407746in}}%
\pgfpathlineto{\pgfqpoint{2.607843in}{1.407746in}}%
\pgfpathlineto{\pgfqpoint{2.610415in}{1.424202in}}%
\pgfpathlineto{\pgfqpoint{2.646421in}{1.424202in}}%
\pgfpathlineto{\pgfqpoint{2.646421in}{1.424202in}}%
\pgfusepath{stroke}%
\end{pgfscope}%
\begin{pgfscope}%
\pgfpathrectangle{\pgfqpoint{0.488751in}{0.368545in}}{\pgfqpoint{2.260417in}{1.502439in}}%
\pgfusepath{clip}%
\pgfsetrectcap%
\pgfsetroundjoin%
\pgfsetlinewidth{0.803000pt}%
\definecolor{currentstroke}{rgb}{0.333333,0.333333,0.333333}%
\pgfsetstrokecolor{currentstroke}%
\pgfsetstrokeopacity{0.270000}%
\pgfsetdash{}{0pt}%
\pgfpathmoveto{\pgfqpoint{0.591497in}{1.127993in}}%
\pgfpathlineto{\pgfqpoint{0.601785in}{1.127993in}}%
\pgfpathlineto{\pgfqpoint{0.604356in}{1.144449in}}%
\pgfpathlineto{\pgfqpoint{0.606928in}{1.127993in}}%
\pgfpathlineto{\pgfqpoint{0.609500in}{1.127993in}}%
\pgfpathlineto{\pgfqpoint{0.612072in}{1.144449in}}%
\pgfpathlineto{\pgfqpoint{0.617216in}{1.144449in}}%
\pgfpathlineto{\pgfqpoint{0.619788in}{1.177361in}}%
\pgfpathlineto{\pgfqpoint{0.622359in}{1.177361in}}%
\pgfpathlineto{\pgfqpoint{0.624931in}{1.210273in}}%
\pgfpathlineto{\pgfqpoint{0.627503in}{1.226729in}}%
\pgfpathlineto{\pgfqpoint{0.630075in}{1.226729in}}%
\pgfpathlineto{\pgfqpoint{0.632647in}{1.276097in}}%
\pgfpathlineto{\pgfqpoint{0.635219in}{1.276097in}}%
\pgfpathlineto{\pgfqpoint{0.637791in}{1.259641in}}%
\pgfpathlineto{\pgfqpoint{0.642934in}{1.292553in}}%
\pgfpathlineto{\pgfqpoint{0.645506in}{1.276097in}}%
\pgfpathlineto{\pgfqpoint{0.648078in}{1.309009in}}%
\pgfpathlineto{\pgfqpoint{0.650650in}{1.292553in}}%
\pgfpathlineto{\pgfqpoint{0.655794in}{1.292553in}}%
\pgfpathlineto{\pgfqpoint{0.658366in}{1.276097in}}%
\pgfpathlineto{\pgfqpoint{0.663509in}{1.309009in}}%
\pgfpathlineto{\pgfqpoint{0.666081in}{1.309009in}}%
\pgfpathlineto{\pgfqpoint{0.668653in}{1.292553in}}%
\pgfpathlineto{\pgfqpoint{0.671225in}{1.309009in}}%
\pgfpathlineto{\pgfqpoint{0.681513in}{1.309009in}}%
\pgfpathlineto{\pgfqpoint{0.684084in}{1.292553in}}%
\pgfpathlineto{\pgfqpoint{0.694372in}{1.292553in}}%
\pgfpathlineto{\pgfqpoint{0.696944in}{1.276097in}}%
\pgfpathlineto{\pgfqpoint{0.699516in}{1.292553in}}%
\pgfpathlineto{\pgfqpoint{0.702087in}{1.292553in}}%
\pgfpathlineto{\pgfqpoint{0.704659in}{1.309009in}}%
\pgfpathlineto{\pgfqpoint{0.712375in}{1.309009in}}%
\pgfpathlineto{\pgfqpoint{0.714947in}{1.292553in}}%
\pgfpathlineto{\pgfqpoint{0.730378in}{1.292553in}}%
\pgfpathlineto{\pgfqpoint{0.732950in}{1.309009in}}%
\pgfpathlineto{\pgfqpoint{0.756097in}{1.309009in}}%
\pgfpathlineto{\pgfqpoint{0.758669in}{1.325465in}}%
\pgfpathlineto{\pgfqpoint{0.815250in}{1.325465in}}%
\pgfpathlineto{\pgfqpoint{0.820393in}{1.358378in}}%
\pgfpathlineto{\pgfqpoint{0.830681in}{1.358378in}}%
\pgfpathlineto{\pgfqpoint{0.833253in}{1.374834in}}%
\pgfpathlineto{\pgfqpoint{0.869259in}{1.374834in}}%
\pgfpathlineto{\pgfqpoint{0.871831in}{1.391290in}}%
\pgfpathlineto{\pgfqpoint{0.882118in}{1.391290in}}%
\pgfpathlineto{\pgfqpoint{0.884690in}{1.407746in}}%
\pgfpathlineto{\pgfqpoint{0.892406in}{1.407746in}}%
\pgfpathlineto{\pgfqpoint{0.894978in}{1.391290in}}%
\pgfpathlineto{\pgfqpoint{0.902693in}{1.391290in}}%
\pgfpathlineto{\pgfqpoint{0.905265in}{1.374834in}}%
\pgfpathlineto{\pgfqpoint{0.923268in}{1.374834in}}%
\pgfpathlineto{\pgfqpoint{0.925840in}{1.358378in}}%
\pgfpathlineto{\pgfqpoint{0.964418in}{1.358378in}}%
\pgfpathlineto{\pgfqpoint{0.966990in}{1.341921in}}%
\pgfpathlineto{\pgfqpoint{1.018427in}{1.341921in}}%
\pgfpathlineto{\pgfqpoint{1.020999in}{1.358378in}}%
\pgfpathlineto{\pgfqpoint{1.026143in}{1.358378in}}%
\pgfpathlineto{\pgfqpoint{1.031287in}{1.325465in}}%
\pgfpathlineto{\pgfqpoint{1.036431in}{1.325465in}}%
\pgfpathlineto{\pgfqpoint{1.039002in}{1.341921in}}%
\pgfpathlineto{\pgfqpoint{1.049290in}{1.341921in}}%
\pgfpathlineto{\pgfqpoint{1.051862in}{1.358378in}}%
\pgfpathlineto{\pgfqpoint{1.054434in}{1.358378in}}%
\pgfpathlineto{\pgfqpoint{1.057006in}{1.374834in}}%
\pgfpathlineto{\pgfqpoint{1.069865in}{1.374834in}}%
\pgfpathlineto{\pgfqpoint{1.072437in}{1.358378in}}%
\pgfpathlineto{\pgfqpoint{1.095584in}{1.358378in}}%
\pgfpathlineto{\pgfqpoint{1.100727in}{1.325465in}}%
\pgfpathlineto{\pgfqpoint{1.131590in}{1.325465in}}%
\pgfpathlineto{\pgfqpoint{1.134162in}{1.341921in}}%
\pgfpathlineto{\pgfqpoint{1.175312in}{1.341921in}}%
\pgfpathlineto{\pgfqpoint{1.177883in}{1.309009in}}%
\pgfpathlineto{\pgfqpoint{1.239608in}{1.309009in}}%
\pgfpathlineto{\pgfqpoint{1.244752in}{1.341921in}}%
\pgfpathlineto{\pgfqpoint{1.273043in}{1.341921in}}%
\pgfpathlineto{\pgfqpoint{1.275615in}{1.325465in}}%
\pgfpathlineto{\pgfqpoint{1.293618in}{1.325465in}}%
\pgfpathlineto{\pgfqpoint{1.296189in}{1.309009in}}%
\pgfpathlineto{\pgfqpoint{1.301333in}{1.309009in}}%
\pgfpathlineto{\pgfqpoint{1.303905in}{1.292553in}}%
\pgfpathlineto{\pgfqpoint{1.375917in}{1.292553in}}%
\pgfpathlineto{\pgfqpoint{1.378489in}{1.276097in}}%
\pgfpathlineto{\pgfqpoint{1.435070in}{1.276097in}}%
\pgfpathlineto{\pgfqpoint{1.437642in}{1.292553in}}%
\pgfpathlineto{\pgfqpoint{1.463361in}{1.292553in}}%
\pgfpathlineto{\pgfqpoint{1.468505in}{1.259641in}}%
\pgfpathlineto{\pgfqpoint{1.501939in}{1.259641in}}%
\pgfpathlineto{\pgfqpoint{1.504511in}{1.276097in}}%
\pgfpathlineto{\pgfqpoint{1.555948in}{1.276097in}}%
\pgfpathlineto{\pgfqpoint{1.558520in}{1.259641in}}%
\pgfpathlineto{\pgfqpoint{1.576523in}{1.259641in}}%
\pgfpathlineto{\pgfqpoint{1.579095in}{1.276097in}}%
\pgfpathlineto{\pgfqpoint{1.591955in}{1.276097in}}%
\pgfpathlineto{\pgfqpoint{1.594526in}{1.292553in}}%
\pgfpathlineto{\pgfqpoint{1.751410in}{1.292553in}}%
\pgfpathlineto{\pgfqpoint{1.753982in}{1.276097in}}%
\pgfpathlineto{\pgfqpoint{1.784845in}{1.276097in}}%
\pgfpathlineto{\pgfqpoint{1.787417in}{1.292553in}}%
\pgfpathlineto{\pgfqpoint{1.921154in}{1.292553in}}%
\pgfpathlineto{\pgfqpoint{1.923726in}{1.309009in}}%
\pgfpathlineto{\pgfqpoint{2.034316in}{1.309009in}}%
\pgfpathlineto{\pgfqpoint{2.036888in}{1.292553in}}%
\pgfpathlineto{\pgfqpoint{2.193772in}{1.292553in}}%
\pgfpathlineto{\pgfqpoint{2.196344in}{1.276097in}}%
\pgfpathlineto{\pgfqpoint{2.360944in}{1.276097in}}%
\pgfpathlineto{\pgfqpoint{2.363516in}{1.259641in}}%
\pgfpathlineto{\pgfqpoint{2.476678in}{1.259641in}}%
\pgfpathlineto{\pgfqpoint{2.479250in}{1.276097in}}%
\pgfpathlineto{\pgfqpoint{2.558978in}{1.276097in}}%
\pgfpathlineto{\pgfqpoint{2.561550in}{1.292553in}}%
\pgfpathlineto{\pgfqpoint{2.597556in}{1.292553in}}%
\pgfpathlineto{\pgfqpoint{2.600128in}{1.309009in}}%
\pgfpathlineto{\pgfqpoint{2.620703in}{1.309009in}}%
\pgfpathlineto{\pgfqpoint{2.623274in}{1.325465in}}%
\pgfpathlineto{\pgfqpoint{2.636134in}{1.325465in}}%
\pgfpathlineto{\pgfqpoint{2.638706in}{1.341921in}}%
\pgfpathlineto{\pgfqpoint{2.646421in}{1.341921in}}%
\pgfpathlineto{\pgfqpoint{2.646421in}{1.341921in}}%
\pgfusepath{stroke}%
\end{pgfscope}%
\begin{pgfscope}%
\pgfsetrectcap%
\pgfsetmiterjoin%
\pgfsetlinewidth{0.501875pt}%
\definecolor{currentstroke}{rgb}{0.317647,0.317647,0.317647}%
\pgfsetstrokecolor{currentstroke}%
\pgfsetdash{}{0pt}%
\pgfpathmoveto{\pgfqpoint{0.488751in}{0.368545in}}%
\pgfpathlineto{\pgfqpoint{0.488751in}{1.870984in}}%
\pgfusepath{stroke}%
\end{pgfscope}%
\begin{pgfscope}%
\pgfsetrectcap%
\pgfsetmiterjoin%
\pgfsetlinewidth{0.501875pt}%
\definecolor{currentstroke}{rgb}{0.317647,0.317647,0.317647}%
\pgfsetstrokecolor{currentstroke}%
\pgfsetdash{}{0pt}%
\pgfpathmoveto{\pgfqpoint{0.488751in}{0.368545in}}%
\pgfpathlineto{\pgfqpoint{2.749168in}{0.368545in}}%
\pgfusepath{stroke}%
\end{pgfscope}%
\begin{pgfscope}%
\pgfsetrectcap%
\pgfsetroundjoin%
\pgfsetlinewidth{0.803000pt}%
\definecolor{currentstroke}{rgb}{0.333333,0.333333,0.333333}%
\pgfsetstrokecolor{currentstroke}%
\pgfsetdash{}{0pt}%
\pgfpathmoveto{\pgfqpoint{2.703959in}{1.746674in}}%
\pgfpathlineto{\pgfqpoint{2.770626in}{1.746674in}}%
\pgfusepath{stroke}%
\end{pgfscope}%
\begin{pgfscope}%
\definecolor{textcolor}{rgb}{0.000000,0.000000,0.000000}%
\pgfsetstrokecolor{textcolor}%
\pgfsetfillcolor{textcolor}%
\pgftext[x=2.812293in,y=1.717508in,left,base]{\color{textcolor}\rmfamily\fontsize{6.000000}{7.200000}\selectfont \(\displaystyle w_{00}\)}%
\end{pgfscope}%
\begin{pgfscope}%
\pgfsetrectcap%
\pgfsetroundjoin%
\pgfsetlinewidth{0.803000pt}%
\definecolor{currentstroke}{rgb}{0.686275,0.352941,0.313725}%
\pgfsetstrokecolor{currentstroke}%
\pgfsetdash{}{0pt}%
\pgfpathmoveto{\pgfqpoint{2.703959in}{1.638841in}}%
\pgfpathlineto{\pgfqpoint{2.770626in}{1.638841in}}%
\pgfusepath{stroke}%
\end{pgfscope}%
\begin{pgfscope}%
\definecolor{textcolor}{rgb}{0.000000,0.000000,0.000000}%
\pgfsetstrokecolor{textcolor}%
\pgfsetfillcolor{textcolor}%
\pgftext[x=2.812293in,y=1.609675in,left,base]{\color{textcolor}\rmfamily\fontsize{6.000000}{7.200000}\selectfont \(\displaystyle w_{10}\)}%
\end{pgfscope}%
\begin{pgfscope}%
\pgfsetrectcap%
\pgfsetroundjoin%
\pgfsetlinewidth{0.803000pt}%
\definecolor{currentstroke}{rgb}{0.000000,0.356863,0.509804}%
\pgfsetstrokecolor{currentstroke}%
\pgfsetdash{}{0pt}%
\pgfpathmoveto{\pgfqpoint{2.703959in}{1.531008in}}%
\pgfpathlineto{\pgfqpoint{2.770626in}{1.531008in}}%
\pgfusepath{stroke}%
\end{pgfscope}%
\begin{pgfscope}%
\definecolor{textcolor}{rgb}{0.000000,0.000000,0.000000}%
\pgfsetstrokecolor{textcolor}%
\pgfsetfillcolor{textcolor}%
\pgftext[x=2.812293in,y=1.501841in,left,base]{\color{textcolor}\rmfamily\fontsize{6.000000}{7.200000}\selectfont \(\displaystyle w_{20}\)}%
\end{pgfscope}%
\begin{pgfscope}%
\pgfsetrectcap%
\pgfsetroundjoin%
\pgfsetlinewidth{0.803000pt}%
\definecolor{currentstroke}{rgb}{0.490196,0.588235,0.431373}%
\pgfsetstrokecolor{currentstroke}%
\pgfsetdash{}{0pt}%
\pgfpathmoveto{\pgfqpoint{2.703959in}{1.423175in}}%
\pgfpathlineto{\pgfqpoint{2.770626in}{1.423175in}}%
\pgfusepath{stroke}%
\end{pgfscope}%
\begin{pgfscope}%
\definecolor{textcolor}{rgb}{0.000000,0.000000,0.000000}%
\pgfsetstrokecolor{textcolor}%
\pgfsetfillcolor{textcolor}%
\pgftext[x=2.812293in,y=1.394008in,left,base]{\color{textcolor}\rmfamily\fontsize{6.000000}{7.200000}\selectfont \(\displaystyle w_{30}\)}%
\end{pgfscope}%
\begin{pgfscope}%
\pgfsetrectcap%
\pgfsetroundjoin%
\pgfsetlinewidth{0.803000pt}%
\definecolor{currentstroke}{rgb}{0.843137,0.666667,0.313725}%
\pgfsetstrokecolor{currentstroke}%
\pgfsetdash{}{0pt}%
\pgfpathmoveto{\pgfqpoint{2.703959in}{1.315342in}}%
\pgfpathlineto{\pgfqpoint{2.770626in}{1.315342in}}%
\pgfusepath{stroke}%
\end{pgfscope}%
\begin{pgfscope}%
\definecolor{textcolor}{rgb}{0.000000,0.000000,0.000000}%
\pgfsetstrokecolor{textcolor}%
\pgfsetfillcolor{textcolor}%
\pgftext[x=2.812293in,y=1.286175in,left,base]{\color{textcolor}\rmfamily\fontsize{6.000000}{7.200000}\selectfont \(\displaystyle w_{40}\)}%
\end{pgfscope}%
\begin{pgfscope}%
\pgfsetbuttcap%
\pgfsetmiterjoin%
\pgfsetlinewidth{0.000000pt}%
\definecolor{currentstroke}{rgb}{0.000000,0.000000,0.000000}%
\pgfsetstrokecolor{currentstroke}%
\pgfsetstrokeopacity{0.000000}%
\pgfsetdash{}{0pt}%
\pgfpathmoveto{\pgfqpoint{3.653334in}{0.368545in}}%
\pgfpathlineto{\pgfqpoint{5.913751in}{0.368545in}}%
\pgfpathlineto{\pgfqpoint{5.913751in}{1.870984in}}%
\pgfpathlineto{\pgfqpoint{3.653334in}{1.870984in}}%
\pgfpathclose%
\pgfusepath{}%
\end{pgfscope}%
\begin{pgfscope}%
\pgfsetbuttcap%
\pgfsetroundjoin%
\definecolor{currentfill}{rgb}{0.317647,0.317647,0.317647}%
\pgfsetfillcolor{currentfill}%
\pgfsetlinewidth{0.501875pt}%
\definecolor{currentstroke}{rgb}{0.317647,0.317647,0.317647}%
\pgfsetstrokecolor{currentstroke}%
\pgfsetdash{}{0pt}%
\pgfsys@defobject{currentmarker}{\pgfqpoint{0.000000in}{-0.020833in}}{\pgfqpoint{0.000000in}{0.000000in}}{%
\pgfpathmoveto{\pgfqpoint{0.000000in}{0.000000in}}%
\pgfpathlineto{\pgfqpoint{0.000000in}{-0.020833in}}%
\pgfusepath{stroke,fill}%
}%
\begin{pgfscope}%
\pgfsys@transformshift{3.756080in}{0.368545in}%
\pgfsys@useobject{currentmarker}{}%
\end{pgfscope}%
\end{pgfscope}%
\begin{pgfscope}%
\definecolor{textcolor}{rgb}{0.317647,0.317647,0.317647}%
\pgfsetstrokecolor{textcolor}%
\pgfsetfillcolor{textcolor}%
\pgftext[x=3.756080in,y=0.319934in,,top]{\color{textcolor}\rmfamily\fontsize{6.664000}{7.996800}\selectfont \(\displaystyle 0\)}%
\end{pgfscope}%
\begin{pgfscope}%
\pgfsetbuttcap%
\pgfsetroundjoin%
\definecolor{currentfill}{rgb}{0.317647,0.317647,0.317647}%
\pgfsetfillcolor{currentfill}%
\pgfsetlinewidth{0.501875pt}%
\definecolor{currentstroke}{rgb}{0.317647,0.317647,0.317647}%
\pgfsetstrokecolor{currentstroke}%
\pgfsetdash{}{0pt}%
\pgfsys@defobject{currentmarker}{\pgfqpoint{0.000000in}{-0.020833in}}{\pgfqpoint{0.000000in}{0.000000in}}{%
\pgfpathmoveto{\pgfqpoint{0.000000in}{0.000000in}}%
\pgfpathlineto{\pgfqpoint{0.000000in}{-0.020833in}}%
\pgfusepath{stroke,fill}%
}%
\begin{pgfscope}%
\pgfsys@transformshift{4.270454in}{0.368545in}%
\pgfsys@useobject{currentmarker}{}%
\end{pgfscope}%
\end{pgfscope}%
\begin{pgfscope}%
\definecolor{textcolor}{rgb}{0.317647,0.317647,0.317647}%
\pgfsetstrokecolor{textcolor}%
\pgfsetfillcolor{textcolor}%
\pgftext[x=4.270454in,y=0.319934in,,top]{\color{textcolor}\rmfamily\fontsize{6.664000}{7.996800}\selectfont \(\displaystyle 1000\)}%
\end{pgfscope}%
\begin{pgfscope}%
\pgfsetbuttcap%
\pgfsetroundjoin%
\definecolor{currentfill}{rgb}{0.317647,0.317647,0.317647}%
\pgfsetfillcolor{currentfill}%
\pgfsetlinewidth{0.501875pt}%
\definecolor{currentstroke}{rgb}{0.317647,0.317647,0.317647}%
\pgfsetstrokecolor{currentstroke}%
\pgfsetdash{}{0pt}%
\pgfsys@defobject{currentmarker}{\pgfqpoint{0.000000in}{-0.020833in}}{\pgfqpoint{0.000000in}{0.000000in}}{%
\pgfpathmoveto{\pgfqpoint{0.000000in}{0.000000in}}%
\pgfpathlineto{\pgfqpoint{0.000000in}{-0.020833in}}%
\pgfusepath{stroke,fill}%
}%
\begin{pgfscope}%
\pgfsys@transformshift{4.784828in}{0.368545in}%
\pgfsys@useobject{currentmarker}{}%
\end{pgfscope}%
\end{pgfscope}%
\begin{pgfscope}%
\definecolor{textcolor}{rgb}{0.317647,0.317647,0.317647}%
\pgfsetstrokecolor{textcolor}%
\pgfsetfillcolor{textcolor}%
\pgftext[x=4.784828in,y=0.319934in,,top]{\color{textcolor}\rmfamily\fontsize{6.664000}{7.996800}\selectfont \(\displaystyle 2000\)}%
\end{pgfscope}%
\begin{pgfscope}%
\pgfsetbuttcap%
\pgfsetroundjoin%
\definecolor{currentfill}{rgb}{0.317647,0.317647,0.317647}%
\pgfsetfillcolor{currentfill}%
\pgfsetlinewidth{0.501875pt}%
\definecolor{currentstroke}{rgb}{0.317647,0.317647,0.317647}%
\pgfsetstrokecolor{currentstroke}%
\pgfsetdash{}{0pt}%
\pgfsys@defobject{currentmarker}{\pgfqpoint{0.000000in}{-0.020833in}}{\pgfqpoint{0.000000in}{0.000000in}}{%
\pgfpathmoveto{\pgfqpoint{0.000000in}{0.000000in}}%
\pgfpathlineto{\pgfqpoint{0.000000in}{-0.020833in}}%
\pgfusepath{stroke,fill}%
}%
\begin{pgfscope}%
\pgfsys@transformshift{5.299202in}{0.368545in}%
\pgfsys@useobject{currentmarker}{}%
\end{pgfscope}%
\end{pgfscope}%
\begin{pgfscope}%
\definecolor{textcolor}{rgb}{0.317647,0.317647,0.317647}%
\pgfsetstrokecolor{textcolor}%
\pgfsetfillcolor{textcolor}%
\pgftext[x=5.299202in,y=0.319934in,,top]{\color{textcolor}\rmfamily\fontsize{6.664000}{7.996800}\selectfont \(\displaystyle 3000\)}%
\end{pgfscope}%
\begin{pgfscope}%
\pgfsetbuttcap%
\pgfsetroundjoin%
\definecolor{currentfill}{rgb}{0.317647,0.317647,0.317647}%
\pgfsetfillcolor{currentfill}%
\pgfsetlinewidth{0.501875pt}%
\definecolor{currentstroke}{rgb}{0.317647,0.317647,0.317647}%
\pgfsetstrokecolor{currentstroke}%
\pgfsetdash{}{0pt}%
\pgfsys@defobject{currentmarker}{\pgfqpoint{0.000000in}{-0.020833in}}{\pgfqpoint{0.000000in}{0.000000in}}{%
\pgfpathmoveto{\pgfqpoint{0.000000in}{0.000000in}}%
\pgfpathlineto{\pgfqpoint{0.000000in}{-0.020833in}}%
\pgfusepath{stroke,fill}%
}%
\begin{pgfscope}%
\pgfsys@transformshift{5.813576in}{0.368545in}%
\pgfsys@useobject{currentmarker}{}%
\end{pgfscope}%
\end{pgfscope}%
\begin{pgfscope}%
\definecolor{textcolor}{rgb}{0.317647,0.317647,0.317647}%
\pgfsetstrokecolor{textcolor}%
\pgfsetfillcolor{textcolor}%
\pgftext[x=5.813576in,y=0.319934in,,top]{\color{textcolor}\rmfamily\fontsize{6.664000}{7.996800}\selectfont \(\displaystyle 4000\)}%
\end{pgfscope}%
\begin{pgfscope}%
\definecolor{textcolor}{rgb}{0.317647,0.317647,0.317647}%
\pgfsetstrokecolor{textcolor}%
\pgfsetfillcolor{textcolor}%
\pgftext[x=4.783543in,y=0.182189in,,top]{\color{textcolor}\rmfamily\fontsize{6.664000}{7.996800}\selectfont Iteration}%
\end{pgfscope}%
\begin{pgfscope}%
\pgfsetbuttcap%
\pgfsetroundjoin%
\definecolor{currentfill}{rgb}{0.317647,0.317647,0.317647}%
\pgfsetfillcolor{currentfill}%
\pgfsetlinewidth{0.501875pt}%
\definecolor{currentstroke}{rgb}{0.317647,0.317647,0.317647}%
\pgfsetstrokecolor{currentstroke}%
\pgfsetdash{}{0pt}%
\pgfsys@defobject{currentmarker}{\pgfqpoint{-0.020833in}{0.000000in}}{\pgfqpoint{0.000000in}{0.000000in}}{%
\pgfpathmoveto{\pgfqpoint{0.000000in}{0.000000in}}%
\pgfpathlineto{\pgfqpoint{-0.020833in}{0.000000in}}%
\pgfusepath{stroke,fill}%
}%
\begin{pgfscope}%
\pgfsys@transformshift{3.653334in}{0.538767in}%
\pgfsys@useobject{currentmarker}{}%
\end{pgfscope}%
\end{pgfscope}%
\begin{pgfscope}%
\definecolor{textcolor}{rgb}{0.317647,0.317647,0.317647}%
\pgfsetstrokecolor{textcolor}%
\pgfsetfillcolor{textcolor}%
\pgftext[x=3.452523in,y=0.506650in,left,base]{\color{textcolor}\rmfamily\fontsize{6.664000}{7.996800}\selectfont \(\displaystyle 270\)}%
\end{pgfscope}%
\begin{pgfscope}%
\pgfsetbuttcap%
\pgfsetroundjoin%
\definecolor{currentfill}{rgb}{0.317647,0.317647,0.317647}%
\pgfsetfillcolor{currentfill}%
\pgfsetlinewidth{0.501875pt}%
\definecolor{currentstroke}{rgb}{0.317647,0.317647,0.317647}%
\pgfsetstrokecolor{currentstroke}%
\pgfsetdash{}{0pt}%
\pgfsys@defobject{currentmarker}{\pgfqpoint{-0.020833in}{0.000000in}}{\pgfqpoint{0.000000in}{0.000000in}}{%
\pgfpathmoveto{\pgfqpoint{0.000000in}{0.000000in}}%
\pgfpathlineto{\pgfqpoint{-0.020833in}{0.000000in}}%
\pgfusepath{stroke,fill}%
}%
\begin{pgfscope}%
\pgfsys@transformshift{3.653334in}{0.742626in}%
\pgfsys@useobject{currentmarker}{}%
\end{pgfscope}%
\end{pgfscope}%
\begin{pgfscope}%
\definecolor{textcolor}{rgb}{0.317647,0.317647,0.317647}%
\pgfsetstrokecolor{textcolor}%
\pgfsetfillcolor{textcolor}%
\pgftext[x=3.452523in,y=0.710509in,left,base]{\color{textcolor}\rmfamily\fontsize{6.664000}{7.996800}\selectfont \(\displaystyle 280\)}%
\end{pgfscope}%
\begin{pgfscope}%
\pgfsetbuttcap%
\pgfsetroundjoin%
\definecolor{currentfill}{rgb}{0.317647,0.317647,0.317647}%
\pgfsetfillcolor{currentfill}%
\pgfsetlinewidth{0.501875pt}%
\definecolor{currentstroke}{rgb}{0.317647,0.317647,0.317647}%
\pgfsetstrokecolor{currentstroke}%
\pgfsetdash{}{0pt}%
\pgfsys@defobject{currentmarker}{\pgfqpoint{-0.020833in}{0.000000in}}{\pgfqpoint{0.000000in}{0.000000in}}{%
\pgfpathmoveto{\pgfqpoint{0.000000in}{0.000000in}}%
\pgfpathlineto{\pgfqpoint{-0.020833in}{0.000000in}}%
\pgfusepath{stroke,fill}%
}%
\begin{pgfscope}%
\pgfsys@transformshift{3.653334in}{0.946485in}%
\pgfsys@useobject{currentmarker}{}%
\end{pgfscope}%
\end{pgfscope}%
\begin{pgfscope}%
\definecolor{textcolor}{rgb}{0.317647,0.317647,0.317647}%
\pgfsetstrokecolor{textcolor}%
\pgfsetfillcolor{textcolor}%
\pgftext[x=3.452523in,y=0.914368in,left,base]{\color{textcolor}\rmfamily\fontsize{6.664000}{7.996800}\selectfont \(\displaystyle 290\)}%
\end{pgfscope}%
\begin{pgfscope}%
\pgfsetbuttcap%
\pgfsetroundjoin%
\definecolor{currentfill}{rgb}{0.317647,0.317647,0.317647}%
\pgfsetfillcolor{currentfill}%
\pgfsetlinewidth{0.501875pt}%
\definecolor{currentstroke}{rgb}{0.317647,0.317647,0.317647}%
\pgfsetstrokecolor{currentstroke}%
\pgfsetdash{}{0pt}%
\pgfsys@defobject{currentmarker}{\pgfqpoint{-0.020833in}{0.000000in}}{\pgfqpoint{0.000000in}{0.000000in}}{%
\pgfpathmoveto{\pgfqpoint{0.000000in}{0.000000in}}%
\pgfpathlineto{\pgfqpoint{-0.020833in}{0.000000in}}%
\pgfusepath{stroke,fill}%
}%
\begin{pgfscope}%
\pgfsys@transformshift{3.653334in}{1.150343in}%
\pgfsys@useobject{currentmarker}{}%
\end{pgfscope}%
\end{pgfscope}%
\begin{pgfscope}%
\definecolor{textcolor}{rgb}{0.317647,0.317647,0.317647}%
\pgfsetstrokecolor{textcolor}%
\pgfsetfillcolor{textcolor}%
\pgftext[x=3.452523in,y=1.118227in,left,base]{\color{textcolor}\rmfamily\fontsize{6.664000}{7.996800}\selectfont \(\displaystyle 300\)}%
\end{pgfscope}%
\begin{pgfscope}%
\pgfsetbuttcap%
\pgfsetroundjoin%
\definecolor{currentfill}{rgb}{0.317647,0.317647,0.317647}%
\pgfsetfillcolor{currentfill}%
\pgfsetlinewidth{0.501875pt}%
\definecolor{currentstroke}{rgb}{0.317647,0.317647,0.317647}%
\pgfsetstrokecolor{currentstroke}%
\pgfsetdash{}{0pt}%
\pgfsys@defobject{currentmarker}{\pgfqpoint{-0.020833in}{0.000000in}}{\pgfqpoint{0.000000in}{0.000000in}}{%
\pgfpathmoveto{\pgfqpoint{0.000000in}{0.000000in}}%
\pgfpathlineto{\pgfqpoint{-0.020833in}{0.000000in}}%
\pgfusepath{stroke,fill}%
}%
\begin{pgfscope}%
\pgfsys@transformshift{3.653334in}{1.354202in}%
\pgfsys@useobject{currentmarker}{}%
\end{pgfscope}%
\end{pgfscope}%
\begin{pgfscope}%
\definecolor{textcolor}{rgb}{0.317647,0.317647,0.317647}%
\pgfsetstrokecolor{textcolor}%
\pgfsetfillcolor{textcolor}%
\pgftext[x=3.452523in,y=1.322085in,left,base]{\color{textcolor}\rmfamily\fontsize{6.664000}{7.996800}\selectfont \(\displaystyle 310\)}%
\end{pgfscope}%
\begin{pgfscope}%
\pgfsetbuttcap%
\pgfsetroundjoin%
\definecolor{currentfill}{rgb}{0.317647,0.317647,0.317647}%
\pgfsetfillcolor{currentfill}%
\pgfsetlinewidth{0.501875pt}%
\definecolor{currentstroke}{rgb}{0.317647,0.317647,0.317647}%
\pgfsetstrokecolor{currentstroke}%
\pgfsetdash{}{0pt}%
\pgfsys@defobject{currentmarker}{\pgfqpoint{-0.020833in}{0.000000in}}{\pgfqpoint{0.000000in}{0.000000in}}{%
\pgfpathmoveto{\pgfqpoint{0.000000in}{0.000000in}}%
\pgfpathlineto{\pgfqpoint{-0.020833in}{0.000000in}}%
\pgfusepath{stroke,fill}%
}%
\begin{pgfscope}%
\pgfsys@transformshift{3.653334in}{1.558061in}%
\pgfsys@useobject{currentmarker}{}%
\end{pgfscope}%
\end{pgfscope}%
\begin{pgfscope}%
\definecolor{textcolor}{rgb}{0.317647,0.317647,0.317647}%
\pgfsetstrokecolor{textcolor}%
\pgfsetfillcolor{textcolor}%
\pgftext[x=3.452523in,y=1.525944in,left,base]{\color{textcolor}\rmfamily\fontsize{6.664000}{7.996800}\selectfont \(\displaystyle 320\)}%
\end{pgfscope}%
\begin{pgfscope}%
\pgfsetbuttcap%
\pgfsetroundjoin%
\definecolor{currentfill}{rgb}{0.317647,0.317647,0.317647}%
\pgfsetfillcolor{currentfill}%
\pgfsetlinewidth{0.501875pt}%
\definecolor{currentstroke}{rgb}{0.317647,0.317647,0.317647}%
\pgfsetstrokecolor{currentstroke}%
\pgfsetdash{}{0pt}%
\pgfsys@defobject{currentmarker}{\pgfqpoint{-0.020833in}{0.000000in}}{\pgfqpoint{0.000000in}{0.000000in}}{%
\pgfpathmoveto{\pgfqpoint{0.000000in}{0.000000in}}%
\pgfpathlineto{\pgfqpoint{-0.020833in}{0.000000in}}%
\pgfusepath{stroke,fill}%
}%
\begin{pgfscope}%
\pgfsys@transformshift{3.653334in}{1.761920in}%
\pgfsys@useobject{currentmarker}{}%
\end{pgfscope}%
\end{pgfscope}%
\begin{pgfscope}%
\definecolor{textcolor}{rgb}{0.317647,0.317647,0.317647}%
\pgfsetstrokecolor{textcolor}%
\pgfsetfillcolor{textcolor}%
\pgftext[x=3.452523in,y=1.729803in,left,base]{\color{textcolor}\rmfamily\fontsize{6.664000}{7.996800}\selectfont \(\displaystyle 330\)}%
\end{pgfscope}%
\begin{pgfscope}%
\definecolor{textcolor}{rgb}{0.317647,0.317647,0.317647}%
\pgfsetstrokecolor{textcolor}%
\pgfsetfillcolor{textcolor}%
\pgftext[x=3.396968in,y=1.119765in,,bottom,rotate=90.000000]{\color{textcolor}\rmfamily\fontsize{6.664000}{7.996800}\selectfont \(\displaystyle \vartheta \propto -b^{(\mathrm{o})} \quad (\si{\milli \V})\)}%
\end{pgfscope}%
\begin{pgfscope}%
\pgfpathrectangle{\pgfqpoint{3.653334in}{0.368545in}}{\pgfqpoint{2.260417in}{1.502439in}}%
\pgfusepath{clip}%
\pgfsetrectcap%
\pgfsetroundjoin%
\pgfsetlinewidth{0.803000pt}%
\definecolor{currentstroke}{rgb}{0.333333,0.333333,0.333333}%
\pgfsetstrokecolor{currentstroke}%
\pgfsetdash{}{0pt}%
\pgfpathmoveto{\pgfqpoint{3.756080in}{1.150343in}}%
\pgfpathlineto{\pgfqpoint{3.812662in}{1.150343in}}%
\pgfpathlineto{\pgfqpoint{3.815233in}{1.170729in}}%
\pgfpathlineto{\pgfqpoint{3.825521in}{1.170729in}}%
\pgfpathlineto{\pgfqpoint{3.828093in}{1.191115in}}%
\pgfpathlineto{\pgfqpoint{3.928396in}{1.191115in}}%
\pgfpathlineto{\pgfqpoint{3.930968in}{1.170729in}}%
\pgfpathlineto{\pgfqpoint{3.946399in}{1.170729in}}%
\pgfpathlineto{\pgfqpoint{3.948971in}{1.191115in}}%
\pgfpathlineto{\pgfqpoint{4.028699in}{1.191115in}}%
\pgfpathlineto{\pgfqpoint{4.031270in}{1.211501in}}%
\pgfpathlineto{\pgfqpoint{4.080136in}{1.211501in}}%
\pgfpathlineto{\pgfqpoint{4.082708in}{1.231887in}}%
\pgfpathlineto{\pgfqpoint{4.116142in}{1.231887in}}%
\pgfpathlineto{\pgfqpoint{4.118714in}{1.211501in}}%
\pgfpathlineto{\pgfqpoint{4.185583in}{1.211501in}}%
\pgfpathlineto{\pgfqpoint{4.188155in}{1.191115in}}%
\pgfpathlineto{\pgfqpoint{4.273026in}{1.191115in}}%
\pgfpathlineto{\pgfqpoint{4.275598in}{1.211501in}}%
\pgfpathlineto{\pgfqpoint{4.411907in}{1.211501in}}%
\pgfpathlineto{\pgfqpoint{4.414479in}{1.191115in}}%
\pgfpathlineto{\pgfqpoint{4.746250in}{1.191115in}}%
\pgfpathlineto{\pgfqpoint{4.748822in}{1.211501in}}%
\pgfpathlineto{\pgfqpoint{4.908278in}{1.211501in}}%
\pgfpathlineto{\pgfqpoint{4.910850in}{1.191115in}}%
\pgfpathlineto{\pgfqpoint{5.090881in}{1.191115in}}%
\pgfpathlineto{\pgfqpoint{5.093453in}{1.170729in}}%
\pgfpathlineto{\pgfqpoint{5.142318in}{1.170729in}}%
\pgfpathlineto{\pgfqpoint{5.144890in}{1.150343in}}%
\pgfpathlineto{\pgfqpoint{5.811005in}{1.150343in}}%
\pgfpathlineto{\pgfqpoint{5.811005in}{1.150343in}}%
\pgfusepath{stroke}%
\end{pgfscope}%
\begin{pgfscope}%
\pgfsetrectcap%
\pgfsetmiterjoin%
\pgfsetlinewidth{0.501875pt}%
\definecolor{currentstroke}{rgb}{0.317647,0.317647,0.317647}%
\pgfsetstrokecolor{currentstroke}%
\pgfsetdash{}{0pt}%
\pgfpathmoveto{\pgfqpoint{3.653334in}{0.368545in}}%
\pgfpathlineto{\pgfqpoint{3.653334in}{1.870984in}}%
\pgfusepath{stroke}%
\end{pgfscope}%
\begin{pgfscope}%
\pgfsetrectcap%
\pgfsetmiterjoin%
\pgfsetlinewidth{0.501875pt}%
\definecolor{currentstroke}{rgb}{0.317647,0.317647,0.317647}%
\pgfsetstrokecolor{currentstroke}%
\pgfsetdash{}{0pt}%
\pgfpathmoveto{\pgfqpoint{3.653334in}{0.368545in}}%
\pgfpathlineto{\pgfqpoint{5.913751in}{0.368545in}}%
\pgfusepath{stroke}%
\end{pgfscope}%
\begin{pgfscope}%
\pgfsetrectcap%
\pgfsetroundjoin%
\pgfsetlinewidth{0.803000pt}%
\definecolor{currentstroke}{rgb}{0.333333,0.333333,0.333333}%
\pgfsetstrokecolor{currentstroke}%
\pgfsetdash{}{0pt}%
\pgfpathmoveto{\pgfqpoint{5.868543in}{1.766073in}}%
\pgfpathlineto{\pgfqpoint{5.935209in}{1.766073in}}%
\pgfusepath{stroke}%
\end{pgfscope}%
\begin{pgfscope}%
\definecolor{textcolor}{rgb}{0.000000,0.000000,0.000000}%
\pgfsetstrokecolor{textcolor}%
\pgfsetfillcolor{textcolor}%
\pgftext[x=5.976876in,y=1.736907in,left,base]{\color{textcolor}\rmfamily\fontsize{6.000000}{7.200000}\selectfont \(\displaystyle b_0\)}%
\end{pgfscope}%
\end{pgfpicture}%
\makeatother%
\endgroup%

%	\caption{Describe what happens in the picture...}
%	\label{network_monitoring}
%\end{figure}