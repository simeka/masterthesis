\chapter{Surrograte Gradient Descent (SuperSpike) on \acrshort{bss2}}

The second experiment in this thesis is done on the most recent tape-out of the \gls{bss2} platform the \gls{hx}. In contrast to the first experiment the neural coding is changed from rate based to temporal based. In particular, the access to the evolution of the membrane potential allows the implementation of the SuperSpike learning rule from \cref{superspike}, a unique approach to training \glspl{snn} on the basis of individual spikes. With the increased temporal resolution, the overall measurement period can be reduced to the length of several spike duration equivalents, making SuperSpike a candidate to process streaming spike data \emph{online}.

To challenge the surrogate gradient approach a constructed task that is equivalent to solving the XOR operator is taken from its original publication by \cite{zenke2018superspike}.

\section{Task}

A total of 96 input units, each firing once at a fixed random spike time, is split into four overlapping collections of different size. As in the exclusive-or, the four different input patterns are assigned to two target classes which are represented as two distinct neurons in the output layer. Despite the multiple input sources, the task is by construction only two-dimensional. The first input spike train $S_1$ overlaps with all other patterns. The units involved in $S_1$ can therefore be interpreted as bias units, since they will fire regardless of the active pattern. In analogy to the xor operator yielding $1 \veebar 0 = 0 \veebar 1 = 1$ for a single active input, the second and third spike train don't overlap except for the bias units. The union of $S_2$ and $S_3$ yields the final pattern, equaling to two active inputs $1 \veebar 1 = 0$. With respect to the xor operator, $S_1$ and $S_4$ yield class $0$ whereas the remaining two are part of class $1$. 

\begin{figure}
	\begin{subfigure}{0.5\textwidth}
		\caption{}
		%% Creator: Matplotlib, PGF backend
%%
%% To include the figure in your LaTeX document, write
%%   \input{<filename>.pgf}
%%
%% Make sure the required packages are loaded in your preamble
%%   \usepackage{pgf}
%%
%% Figures using additional raster images can only be included by \input if
%% they are in the same directory as the main LaTeX file. For loading figures
%% from other directories you can use the `import` package
%%   \usepackage{import}
%% and then include the figures with
%%   \import{<path to file>}{<filename>.pgf}
%%
%% Matplotlib used the following preamble
%%   \usepackage{amsmath} \usepackage{pifont} \usepackage{xcolor} \definecolor{green}{HTML}{467821} \definecolor{red}{HTML}{CF4457} \usepackage[detect-all]{siunitx}
%%   \usepackage{fontspec}
%%
\begingroup%
\makeatletter%
\begin{pgfpicture}%
\pgfpathrectangle{\pgfpointorigin}{\pgfqpoint{2.863556in}{2.825695in}}%
\pgfusepath{use as bounding box, clip}%
\begin{pgfscope}%
\pgfsetbuttcap%
\pgfsetmiterjoin%
\pgfsetlinewidth{0.000000pt}%
\definecolor{currentstroke}{rgb}{0.000000,0.000000,0.000000}%
\pgfsetstrokecolor{currentstroke}%
\pgfsetstrokeopacity{0.000000}%
\pgfsetdash{}{0pt}%
\pgfpathmoveto{\pgfqpoint{0.000000in}{0.000000in}}%
\pgfpathlineto{\pgfqpoint{2.863556in}{0.000000in}}%
\pgfpathlineto{\pgfqpoint{2.863556in}{2.825695in}}%
\pgfpathlineto{\pgfqpoint{0.000000in}{2.825695in}}%
\pgfpathclose%
\pgfusepath{}%
\end{pgfscope}%
\begin{pgfscope}%
\pgfsetbuttcap%
\pgfsetmiterjoin%
\pgfsetlinewidth{0.000000pt}%
\definecolor{currentstroke}{rgb}{0.000000,0.000000,0.000000}%
\pgfsetstrokecolor{currentstroke}%
\pgfsetstrokeopacity{0.000000}%
\pgfsetdash{}{0pt}%
\pgfpathmoveto{\pgfqpoint{0.383193in}{0.383578in}}%
\pgfpathlineto{\pgfqpoint{2.708193in}{0.383578in}}%
\pgfpathlineto{\pgfqpoint{2.708193in}{2.693578in}}%
\pgfpathlineto{\pgfqpoint{0.383193in}{2.693578in}}%
\pgfpathclose%
\pgfusepath{}%
\end{pgfscope}%
\begin{pgfscope}%
\pgfsetbuttcap%
\pgfsetroundjoin%
\definecolor{currentfill}{rgb}{0.317647,0.317647,0.317647}%
\pgfsetfillcolor{currentfill}%
\pgfsetlinewidth{0.501875pt}%
\definecolor{currentstroke}{rgb}{0.317647,0.317647,0.317647}%
\pgfsetstrokecolor{currentstroke}%
\pgfsetdash{}{0pt}%
\pgfsys@defobject{currentmarker}{\pgfqpoint{0.000000in}{-0.020833in}}{\pgfqpoint{0.000000in}{0.000000in}}{%
\pgfpathmoveto{\pgfqpoint{0.000000in}{0.000000in}}%
\pgfpathlineto{\pgfqpoint{0.000000in}{-0.020833in}}%
\pgfusepath{stroke,fill}%
}%
\begin{pgfscope}%
\pgfsys@transformshift{0.770693in}{0.383578in}%
\pgfsys@useobject{currentmarker}{}%
\end{pgfscope}%
\end{pgfscope}%
\begin{pgfscope}%
\definecolor{textcolor}{rgb}{0.317647,0.317647,0.317647}%
\pgfsetstrokecolor{textcolor}%
\pgfsetfillcolor{textcolor}%
\pgftext[x=0.770693in,y=0.334967in,,top]{\color{textcolor}\rmfamily\fontsize{6.664000}{7.996800}\selectfont \(\displaystyle 16\)}%
\end{pgfscope}%
\begin{pgfscope}%
\pgfsetbuttcap%
\pgfsetroundjoin%
\definecolor{currentfill}{rgb}{0.317647,0.317647,0.317647}%
\pgfsetfillcolor{currentfill}%
\pgfsetlinewidth{0.501875pt}%
\definecolor{currentstroke}{rgb}{0.317647,0.317647,0.317647}%
\pgfsetstrokecolor{currentstroke}%
\pgfsetdash{}{0pt}%
\pgfsys@defobject{currentmarker}{\pgfqpoint{0.000000in}{-0.020833in}}{\pgfqpoint{0.000000in}{0.000000in}}{%
\pgfpathmoveto{\pgfqpoint{0.000000in}{0.000000in}}%
\pgfpathlineto{\pgfqpoint{0.000000in}{-0.020833in}}%
\pgfusepath{stroke,fill}%
}%
\begin{pgfscope}%
\pgfsys@transformshift{1.255068in}{0.383578in}%
\pgfsys@useobject{currentmarker}{}%
\end{pgfscope}%
\end{pgfscope}%
\begin{pgfscope}%
\definecolor{textcolor}{rgb}{0.317647,0.317647,0.317647}%
\pgfsetstrokecolor{textcolor}%
\pgfsetfillcolor{textcolor}%
\pgftext[x=1.255068in,y=0.334967in,,top]{\color{textcolor}\rmfamily\fontsize{6.664000}{7.996800}\selectfont \(\displaystyle 18\)}%
\end{pgfscope}%
\begin{pgfscope}%
\pgfsetbuttcap%
\pgfsetroundjoin%
\definecolor{currentfill}{rgb}{0.317647,0.317647,0.317647}%
\pgfsetfillcolor{currentfill}%
\pgfsetlinewidth{0.501875pt}%
\definecolor{currentstroke}{rgb}{0.317647,0.317647,0.317647}%
\pgfsetstrokecolor{currentstroke}%
\pgfsetdash{}{0pt}%
\pgfsys@defobject{currentmarker}{\pgfqpoint{0.000000in}{-0.020833in}}{\pgfqpoint{0.000000in}{0.000000in}}{%
\pgfpathmoveto{\pgfqpoint{0.000000in}{0.000000in}}%
\pgfpathlineto{\pgfqpoint{0.000000in}{-0.020833in}}%
\pgfusepath{stroke,fill}%
}%
\begin{pgfscope}%
\pgfsys@transformshift{1.739443in}{0.383578in}%
\pgfsys@useobject{currentmarker}{}%
\end{pgfscope}%
\end{pgfscope}%
\begin{pgfscope}%
\definecolor{textcolor}{rgb}{0.317647,0.317647,0.317647}%
\pgfsetstrokecolor{textcolor}%
\pgfsetfillcolor{textcolor}%
\pgftext[x=1.739443in,y=0.334967in,,top]{\color{textcolor}\rmfamily\fontsize{6.664000}{7.996800}\selectfont \(\displaystyle 20\)}%
\end{pgfscope}%
\begin{pgfscope}%
\pgfsetbuttcap%
\pgfsetroundjoin%
\definecolor{currentfill}{rgb}{0.317647,0.317647,0.317647}%
\pgfsetfillcolor{currentfill}%
\pgfsetlinewidth{0.501875pt}%
\definecolor{currentstroke}{rgb}{0.317647,0.317647,0.317647}%
\pgfsetstrokecolor{currentstroke}%
\pgfsetdash{}{0pt}%
\pgfsys@defobject{currentmarker}{\pgfqpoint{0.000000in}{-0.020833in}}{\pgfqpoint{0.000000in}{0.000000in}}{%
\pgfpathmoveto{\pgfqpoint{0.000000in}{0.000000in}}%
\pgfpathlineto{\pgfqpoint{0.000000in}{-0.020833in}}%
\pgfusepath{stroke,fill}%
}%
\begin{pgfscope}%
\pgfsys@transformshift{2.223818in}{0.383578in}%
\pgfsys@useobject{currentmarker}{}%
\end{pgfscope}%
\end{pgfscope}%
\begin{pgfscope}%
\definecolor{textcolor}{rgb}{0.317647,0.317647,0.317647}%
\pgfsetstrokecolor{textcolor}%
\pgfsetfillcolor{textcolor}%
\pgftext[x=2.223818in,y=0.334967in,,top]{\color{textcolor}\rmfamily\fontsize{6.664000}{7.996800}\selectfont \(\displaystyle 22\)}%
\end{pgfscope}%
\begin{pgfscope}%
\pgfsetbuttcap%
\pgfsetroundjoin%
\definecolor{currentfill}{rgb}{0.317647,0.317647,0.317647}%
\pgfsetfillcolor{currentfill}%
\pgfsetlinewidth{0.501875pt}%
\definecolor{currentstroke}{rgb}{0.317647,0.317647,0.317647}%
\pgfsetstrokecolor{currentstroke}%
\pgfsetdash{}{0pt}%
\pgfsys@defobject{currentmarker}{\pgfqpoint{0.000000in}{-0.020833in}}{\pgfqpoint{0.000000in}{0.000000in}}{%
\pgfpathmoveto{\pgfqpoint{0.000000in}{0.000000in}}%
\pgfpathlineto{\pgfqpoint{0.000000in}{-0.020833in}}%
\pgfusepath{stroke,fill}%
}%
\begin{pgfscope}%
\pgfsys@transformshift{2.708193in}{0.383578in}%
\pgfsys@useobject{currentmarker}{}%
\end{pgfscope}%
\end{pgfscope}%
\begin{pgfscope}%
\definecolor{textcolor}{rgb}{0.317647,0.317647,0.317647}%
\pgfsetstrokecolor{textcolor}%
\pgfsetfillcolor{textcolor}%
\pgftext[x=2.708193in,y=0.334967in,,top]{\color{textcolor}\rmfamily\fontsize{6.664000}{7.996800}\selectfont \(\displaystyle 24\)}%
\end{pgfscope}%
\begin{pgfscope}%
\definecolor{textcolor}{rgb}{0.317647,0.317647,0.317647}%
\pgfsetstrokecolor{textcolor}%
\pgfsetfillcolor{textcolor}%
\pgftext[x=1.545693in,y=0.197222in,,top]{\color{textcolor}\rmfamily\fontsize{6.664000}{7.996800}\selectfont spike time \(\displaystyle (\si{\micro \s})\)}%
\end{pgfscope}%
\begin{pgfscope}%
\pgfsetbuttcap%
\pgfsetroundjoin%
\definecolor{currentfill}{rgb}{0.317647,0.317647,0.317647}%
\pgfsetfillcolor{currentfill}%
\pgfsetlinewidth{0.501875pt}%
\definecolor{currentstroke}{rgb}{0.317647,0.317647,0.317647}%
\pgfsetstrokecolor{currentstroke}%
\pgfsetdash{}{0pt}%
\pgfsys@defobject{currentmarker}{\pgfqpoint{-0.020833in}{0.000000in}}{\pgfqpoint{0.000000in}{0.000000in}}{%
\pgfpathmoveto{\pgfqpoint{0.000000in}{0.000000in}}%
\pgfpathlineto{\pgfqpoint{-0.020833in}{0.000000in}}%
\pgfusepath{stroke,fill}%
}%
\begin{pgfscope}%
\pgfsys@transformshift{0.383193in}{0.383578in}%
\pgfsys@useobject{currentmarker}{}%
\end{pgfscope}%
\end{pgfscope}%
\begin{pgfscope}%
\definecolor{textcolor}{rgb}{0.317647,0.317647,0.317647}%
\pgfsetstrokecolor{textcolor}%
\pgfsetfillcolor{textcolor}%
\pgftext[x=0.237745in,y=0.351461in,left,base]{\color{textcolor}\rmfamily\fontsize{6.664000}{7.996800}\selectfont \(\displaystyle 60\)}%
\end{pgfscope}%
\begin{pgfscope}%
\pgfsetbuttcap%
\pgfsetroundjoin%
\definecolor{currentfill}{rgb}{0.317647,0.317647,0.317647}%
\pgfsetfillcolor{currentfill}%
\pgfsetlinewidth{0.501875pt}%
\definecolor{currentstroke}{rgb}{0.317647,0.317647,0.317647}%
\pgfsetstrokecolor{currentstroke}%
\pgfsetdash{}{0pt}%
\pgfsys@defobject{currentmarker}{\pgfqpoint{-0.020833in}{0.000000in}}{\pgfqpoint{0.000000in}{0.000000in}}{%
\pgfpathmoveto{\pgfqpoint{0.000000in}{0.000000in}}%
\pgfpathlineto{\pgfqpoint{-0.020833in}{0.000000in}}%
\pgfusepath{stroke,fill}%
}%
\begin{pgfscope}%
\pgfsys@transformshift{0.383193in}{0.768578in}%
\pgfsys@useobject{currentmarker}{}%
\end{pgfscope}%
\end{pgfscope}%
\begin{pgfscope}%
\definecolor{textcolor}{rgb}{0.317647,0.317647,0.317647}%
\pgfsetstrokecolor{textcolor}%
\pgfsetfillcolor{textcolor}%
\pgftext[x=0.237745in,y=0.736461in,left,base]{\color{textcolor}\rmfamily\fontsize{6.664000}{7.996800}\selectfont \(\displaystyle 62\)}%
\end{pgfscope}%
\begin{pgfscope}%
\pgfsetbuttcap%
\pgfsetroundjoin%
\definecolor{currentfill}{rgb}{0.317647,0.317647,0.317647}%
\pgfsetfillcolor{currentfill}%
\pgfsetlinewidth{0.501875pt}%
\definecolor{currentstroke}{rgb}{0.317647,0.317647,0.317647}%
\pgfsetstrokecolor{currentstroke}%
\pgfsetdash{}{0pt}%
\pgfsys@defobject{currentmarker}{\pgfqpoint{-0.020833in}{0.000000in}}{\pgfqpoint{0.000000in}{0.000000in}}{%
\pgfpathmoveto{\pgfqpoint{0.000000in}{0.000000in}}%
\pgfpathlineto{\pgfqpoint{-0.020833in}{0.000000in}}%
\pgfusepath{stroke,fill}%
}%
\begin{pgfscope}%
\pgfsys@transformshift{0.383193in}{1.153578in}%
\pgfsys@useobject{currentmarker}{}%
\end{pgfscope}%
\end{pgfscope}%
\begin{pgfscope}%
\definecolor{textcolor}{rgb}{0.317647,0.317647,0.317647}%
\pgfsetstrokecolor{textcolor}%
\pgfsetfillcolor{textcolor}%
\pgftext[x=0.237745in,y=1.121461in,left,base]{\color{textcolor}\rmfamily\fontsize{6.664000}{7.996800}\selectfont \(\displaystyle 64\)}%
\end{pgfscope}%
\begin{pgfscope}%
\pgfsetbuttcap%
\pgfsetroundjoin%
\definecolor{currentfill}{rgb}{0.317647,0.317647,0.317647}%
\pgfsetfillcolor{currentfill}%
\pgfsetlinewidth{0.501875pt}%
\definecolor{currentstroke}{rgb}{0.317647,0.317647,0.317647}%
\pgfsetstrokecolor{currentstroke}%
\pgfsetdash{}{0pt}%
\pgfsys@defobject{currentmarker}{\pgfqpoint{-0.020833in}{0.000000in}}{\pgfqpoint{0.000000in}{0.000000in}}{%
\pgfpathmoveto{\pgfqpoint{0.000000in}{0.000000in}}%
\pgfpathlineto{\pgfqpoint{-0.020833in}{0.000000in}}%
\pgfusepath{stroke,fill}%
}%
\begin{pgfscope}%
\pgfsys@transformshift{0.383193in}{1.538578in}%
\pgfsys@useobject{currentmarker}{}%
\end{pgfscope}%
\end{pgfscope}%
\begin{pgfscope}%
\definecolor{textcolor}{rgb}{0.317647,0.317647,0.317647}%
\pgfsetstrokecolor{textcolor}%
\pgfsetfillcolor{textcolor}%
\pgftext[x=0.237745in,y=1.506461in,left,base]{\color{textcolor}\rmfamily\fontsize{6.664000}{7.996800}\selectfont \(\displaystyle 66\)}%
\end{pgfscope}%
\begin{pgfscope}%
\pgfsetbuttcap%
\pgfsetroundjoin%
\definecolor{currentfill}{rgb}{0.317647,0.317647,0.317647}%
\pgfsetfillcolor{currentfill}%
\pgfsetlinewidth{0.501875pt}%
\definecolor{currentstroke}{rgb}{0.317647,0.317647,0.317647}%
\pgfsetstrokecolor{currentstroke}%
\pgfsetdash{}{0pt}%
\pgfsys@defobject{currentmarker}{\pgfqpoint{-0.020833in}{0.000000in}}{\pgfqpoint{0.000000in}{0.000000in}}{%
\pgfpathmoveto{\pgfqpoint{0.000000in}{0.000000in}}%
\pgfpathlineto{\pgfqpoint{-0.020833in}{0.000000in}}%
\pgfusepath{stroke,fill}%
}%
\begin{pgfscope}%
\pgfsys@transformshift{0.383193in}{1.923578in}%
\pgfsys@useobject{currentmarker}{}%
\end{pgfscope}%
\end{pgfscope}%
\begin{pgfscope}%
\definecolor{textcolor}{rgb}{0.317647,0.317647,0.317647}%
\pgfsetstrokecolor{textcolor}%
\pgfsetfillcolor{textcolor}%
\pgftext[x=0.237745in,y=1.891461in,left,base]{\color{textcolor}\rmfamily\fontsize{6.664000}{7.996800}\selectfont \(\displaystyle 68\)}%
\end{pgfscope}%
\begin{pgfscope}%
\pgfsetbuttcap%
\pgfsetroundjoin%
\definecolor{currentfill}{rgb}{0.317647,0.317647,0.317647}%
\pgfsetfillcolor{currentfill}%
\pgfsetlinewidth{0.501875pt}%
\definecolor{currentstroke}{rgb}{0.317647,0.317647,0.317647}%
\pgfsetstrokecolor{currentstroke}%
\pgfsetdash{}{0pt}%
\pgfsys@defobject{currentmarker}{\pgfqpoint{-0.020833in}{0.000000in}}{\pgfqpoint{0.000000in}{0.000000in}}{%
\pgfpathmoveto{\pgfqpoint{0.000000in}{0.000000in}}%
\pgfpathlineto{\pgfqpoint{-0.020833in}{0.000000in}}%
\pgfusepath{stroke,fill}%
}%
\begin{pgfscope}%
\pgfsys@transformshift{0.383193in}{2.308578in}%
\pgfsys@useobject{currentmarker}{}%
\end{pgfscope}%
\end{pgfscope}%
\begin{pgfscope}%
\definecolor{textcolor}{rgb}{0.317647,0.317647,0.317647}%
\pgfsetstrokecolor{textcolor}%
\pgfsetfillcolor{textcolor}%
\pgftext[x=0.237745in,y=2.276461in,left,base]{\color{textcolor}\rmfamily\fontsize{6.664000}{7.996800}\selectfont \(\displaystyle 70\)}%
\end{pgfscope}%
\begin{pgfscope}%
\pgfsetbuttcap%
\pgfsetroundjoin%
\definecolor{currentfill}{rgb}{0.317647,0.317647,0.317647}%
\pgfsetfillcolor{currentfill}%
\pgfsetlinewidth{0.501875pt}%
\definecolor{currentstroke}{rgb}{0.317647,0.317647,0.317647}%
\pgfsetstrokecolor{currentstroke}%
\pgfsetdash{}{0pt}%
\pgfsys@defobject{currentmarker}{\pgfqpoint{-0.020833in}{0.000000in}}{\pgfqpoint{0.000000in}{0.000000in}}{%
\pgfpathmoveto{\pgfqpoint{0.000000in}{0.000000in}}%
\pgfpathlineto{\pgfqpoint{-0.020833in}{0.000000in}}%
\pgfusepath{stroke,fill}%
}%
\begin{pgfscope}%
\pgfsys@transformshift{0.383193in}{2.693578in}%
\pgfsys@useobject{currentmarker}{}%
\end{pgfscope}%
\end{pgfscope}%
\begin{pgfscope}%
\definecolor{textcolor}{rgb}{0.317647,0.317647,0.317647}%
\pgfsetstrokecolor{textcolor}%
\pgfsetfillcolor{textcolor}%
\pgftext[x=0.237745in,y=2.661461in,left,base]{\color{textcolor}\rmfamily\fontsize{6.664000}{7.996800}\selectfont \(\displaystyle 72\)}%
\end{pgfscope}%
\begin{pgfscope}%
\definecolor{textcolor}{rgb}{0.317647,0.317647,0.317647}%
\pgfsetstrokecolor{textcolor}%
\pgfsetfillcolor{textcolor}%
\pgftext[x=0.182189in,y=1.538578in,,bottom,rotate=90.000000]{\color{textcolor}\rmfamily\fontsize{6.664000}{7.996800}\selectfont input unit}%
\end{pgfscope}%
\begin{pgfscope}%
\pgfpathrectangle{\pgfqpoint{0.383193in}{0.383578in}}{\pgfqpoint{2.325000in}{2.310000in}}%
\pgfusepath{clip}%
\pgfsetbuttcap%
\pgfsetroundjoin%
\pgfsetlinewidth{1.405250pt}%
\definecolor{currentstroke}{rgb}{0.333333,0.333333,0.333333}%
\pgfsetstrokecolor{currentstroke}%
\pgfsetdash{}{0pt}%
\pgfpathmoveto{\pgfqpoint{-2.009620in}{-9.877950in}}%
\pgfpathcurveto{\pgfqpoint{-1.993965in}{-9.877950in}}{\pgfqpoint{-1.978950in}{-9.871730in}}{\pgfqpoint{-1.967881in}{-9.860661in}}%
\pgfpathcurveto{\pgfqpoint{-1.956811in}{-9.849592in}}{\pgfqpoint{-1.950592in}{-9.834576in}}{\pgfqpoint{-1.950592in}{-9.818922in}}%
\pgfpathcurveto{\pgfqpoint{-1.950592in}{-9.803268in}}{\pgfqpoint{-1.956811in}{-9.788252in}}{\pgfqpoint{-1.967881in}{-9.777183in}}%
\pgfpathcurveto{\pgfqpoint{-1.978950in}{-9.766114in}}{\pgfqpoint{-1.993965in}{-9.759894in}}{\pgfqpoint{-2.009620in}{-9.759894in}}%
\pgfpathcurveto{\pgfqpoint{-2.025274in}{-9.759894in}}{\pgfqpoint{-2.040289in}{-9.766114in}}{\pgfqpoint{-2.051359in}{-9.777183in}}%
\pgfpathcurveto{\pgfqpoint{-2.062428in}{-9.788252in}}{\pgfqpoint{-2.068647in}{-9.803268in}}{\pgfqpoint{-2.068647in}{-9.818922in}}%
\pgfpathcurveto{\pgfqpoint{-2.068647in}{-9.834576in}}{\pgfqpoint{-2.062428in}{-9.849592in}}{\pgfqpoint{-2.051359in}{-9.860661in}}%
\pgfpathcurveto{\pgfqpoint{-2.040289in}{-9.871730in}}{\pgfqpoint{-2.025274in}{-9.877950in}}{\pgfqpoint{-2.009620in}{-9.877950in}}%
\pgfpathclose%
\pgfusepath{stroke}%
\end{pgfscope}%
\begin{pgfscope}%
\pgfpathrectangle{\pgfqpoint{0.383193in}{0.383578in}}{\pgfqpoint{2.325000in}{2.310000in}}%
\pgfusepath{clip}%
\pgfsetbuttcap%
\pgfsetroundjoin%
\pgfsetlinewidth{1.405250pt}%
\definecolor{currentstroke}{rgb}{0.333333,0.333333,0.333333}%
\pgfsetstrokecolor{currentstroke}%
\pgfsetdash{}{0pt}%
\pgfpathmoveto{\pgfqpoint{0.846255in}{-7.952950in}}%
\pgfpathcurveto{\pgfqpoint{0.861910in}{-7.952950in}}{\pgfqpoint{0.876925in}{-7.946730in}}{\pgfqpoint{0.887994in}{-7.935661in}}%
\pgfpathcurveto{\pgfqpoint{0.899064in}{-7.924592in}}{\pgfqpoint{0.905283in}{-7.909576in}}{\pgfqpoint{0.905283in}{-7.893922in}}%
\pgfpathcurveto{\pgfqpoint{0.905283in}{-7.878268in}}{\pgfqpoint{0.899064in}{-7.863252in}}{\pgfqpoint{0.887994in}{-7.852183in}}%
\pgfpathcurveto{\pgfqpoint{0.876925in}{-7.841114in}}{\pgfqpoint{0.861910in}{-7.834894in}}{\pgfqpoint{0.846255in}{-7.834894in}}%
\pgfpathcurveto{\pgfqpoint{0.830601in}{-7.834894in}}{\pgfqpoint{0.815586in}{-7.841114in}}{\pgfqpoint{0.804516in}{-7.852183in}}%
\pgfpathcurveto{\pgfqpoint{0.793447in}{-7.863252in}}{\pgfqpoint{0.787227in}{-7.878268in}}{\pgfqpoint{0.787227in}{-7.893922in}}%
\pgfpathcurveto{\pgfqpoint{0.787227in}{-7.909576in}}{\pgfqpoint{0.793447in}{-7.924592in}}{\pgfqpoint{0.804516in}{-7.935661in}}%
\pgfpathcurveto{\pgfqpoint{0.815586in}{-7.946730in}}{\pgfqpoint{0.830601in}{-7.952950in}}{\pgfqpoint{0.846255in}{-7.952950in}}%
\pgfpathclose%
\pgfusepath{stroke}%
\end{pgfscope}%
\begin{pgfscope}%
\pgfpathrectangle{\pgfqpoint{0.383193in}{0.383578in}}{\pgfqpoint{2.325000in}{2.310000in}}%
\pgfusepath{clip}%
\pgfsetbuttcap%
\pgfsetroundjoin%
\pgfsetlinewidth{1.405250pt}%
\definecolor{currentstroke}{rgb}{0.333333,0.333333,0.333333}%
\pgfsetstrokecolor{currentstroke}%
\pgfsetdash{}{0pt}%
\pgfpathmoveto{\pgfqpoint{6.183099in}{-7.760450in}}%
\pgfpathcurveto{\pgfqpoint{6.198753in}{-7.760450in}}{\pgfqpoint{6.213769in}{-7.754230in}}{\pgfqpoint{6.224838in}{-7.743161in}}%
\pgfpathcurveto{\pgfqpoint{6.235907in}{-7.732092in}}{\pgfqpoint{6.242127in}{-7.717076in}}{\pgfqpoint{6.242127in}{-7.701422in}}%
\pgfpathcurveto{\pgfqpoint{6.242127in}{-7.685768in}}{\pgfqpoint{6.235907in}{-7.670752in}}{\pgfqpoint{6.224838in}{-7.659683in}}%
\pgfpathcurveto{\pgfqpoint{6.213769in}{-7.648614in}}{\pgfqpoint{6.198753in}{-7.642394in}}{\pgfqpoint{6.183099in}{-7.642394in}}%
\pgfpathcurveto{\pgfqpoint{6.167445in}{-7.642394in}}{\pgfqpoint{6.152429in}{-7.648614in}}{\pgfqpoint{6.141360in}{-7.659683in}}%
\pgfpathcurveto{\pgfqpoint{6.130291in}{-7.670752in}}{\pgfqpoint{6.124071in}{-7.685768in}}{\pgfqpoint{6.124071in}{-7.701422in}}%
\pgfpathcurveto{\pgfqpoint{6.124071in}{-7.717076in}}{\pgfqpoint{6.130291in}{-7.732092in}}{\pgfqpoint{6.141360in}{-7.743161in}}%
\pgfpathcurveto{\pgfqpoint{6.152429in}{-7.754230in}}{\pgfqpoint{6.167445in}{-7.760450in}}{\pgfqpoint{6.183099in}{-7.760450in}}%
\pgfpathclose%
\pgfusepath{stroke}%
\end{pgfscope}%
\begin{pgfscope}%
\pgfpathrectangle{\pgfqpoint{0.383193in}{0.383578in}}{\pgfqpoint{2.325000in}{2.310000in}}%
\pgfusepath{clip}%
\pgfsetbuttcap%
\pgfsetroundjoin%
\pgfsetlinewidth{1.405250pt}%
\definecolor{currentstroke}{rgb}{0.333333,0.333333,0.333333}%
\pgfsetstrokecolor{currentstroke}%
\pgfsetdash{}{0pt}%
\pgfpathmoveto{\pgfqpoint{3.967568in}{-6.605450in}}%
\pgfpathcurveto{\pgfqpoint{3.983222in}{-6.605450in}}{\pgfqpoint{3.998237in}{-6.599230in}}{\pgfqpoint{4.009307in}{-6.588161in}}%
\pgfpathcurveto{\pgfqpoint{4.020376in}{-6.577092in}}{\pgfqpoint{4.026596in}{-6.562076in}}{\pgfqpoint{4.026596in}{-6.546422in}}%
\pgfpathcurveto{\pgfqpoint{4.026596in}{-6.530768in}}{\pgfqpoint{4.020376in}{-6.515752in}}{\pgfqpoint{4.009307in}{-6.504683in}}%
\pgfpathcurveto{\pgfqpoint{3.998237in}{-6.493614in}}{\pgfqpoint{3.983222in}{-6.487394in}}{\pgfqpoint{3.967568in}{-6.487394in}}%
\pgfpathcurveto{\pgfqpoint{3.951913in}{-6.487394in}}{\pgfqpoint{3.936898in}{-6.493614in}}{\pgfqpoint{3.925829in}{-6.504683in}}%
\pgfpathcurveto{\pgfqpoint{3.914760in}{-6.515752in}}{\pgfqpoint{3.908540in}{-6.530768in}}{\pgfqpoint{3.908540in}{-6.546422in}}%
\pgfpathcurveto{\pgfqpoint{3.908540in}{-6.562076in}}{\pgfqpoint{3.914760in}{-6.577092in}}{\pgfqpoint{3.925829in}{-6.588161in}}%
\pgfpathcurveto{\pgfqpoint{3.936898in}{-6.599230in}}{\pgfqpoint{3.951913in}{-6.605450in}}{\pgfqpoint{3.967568in}{-6.605450in}}%
\pgfpathclose%
\pgfusepath{stroke}%
\end{pgfscope}%
\begin{pgfscope}%
\pgfpathrectangle{\pgfqpoint{0.383193in}{0.383578in}}{\pgfqpoint{2.325000in}{2.310000in}}%
\pgfusepath{clip}%
\pgfsetbuttcap%
\pgfsetroundjoin%
\pgfsetlinewidth{1.405250pt}%
\definecolor{currentstroke}{rgb}{0.333333,0.333333,0.333333}%
\pgfsetstrokecolor{currentstroke}%
\pgfsetdash{}{0pt}%
\pgfpathmoveto{\pgfqpoint{0.274693in}{-6.412950in}}%
\pgfpathcurveto{\pgfqpoint{0.290347in}{-6.412950in}}{\pgfqpoint{0.305362in}{-6.406730in}}{\pgfqpoint{0.316432in}{-6.395661in}}%
\pgfpathcurveto{\pgfqpoint{0.327501in}{-6.384592in}}{\pgfqpoint{0.333721in}{-6.369576in}}{\pgfqpoint{0.333721in}{-6.353922in}}%
\pgfpathcurveto{\pgfqpoint{0.333721in}{-6.338268in}}{\pgfqpoint{0.327501in}{-6.323252in}}{\pgfqpoint{0.316432in}{-6.312183in}}%
\pgfpathcurveto{\pgfqpoint{0.305362in}{-6.301114in}}{\pgfqpoint{0.290347in}{-6.294894in}}{\pgfqpoint{0.274693in}{-6.294894in}}%
\pgfpathcurveto{\pgfqpoint{0.259038in}{-6.294894in}}{\pgfqpoint{0.244023in}{-6.301114in}}{\pgfqpoint{0.232954in}{-6.312183in}}%
\pgfpathcurveto{\pgfqpoint{0.221885in}{-6.323252in}}{\pgfqpoint{0.215665in}{-6.338268in}}{\pgfqpoint{0.215665in}{-6.353922in}}%
\pgfpathcurveto{\pgfqpoint{0.215665in}{-6.369576in}}{\pgfqpoint{0.221885in}{-6.384592in}}{\pgfqpoint{0.232954in}{-6.395661in}}%
\pgfpathcurveto{\pgfqpoint{0.244023in}{-6.406730in}}{\pgfqpoint{0.259038in}{-6.412950in}}{\pgfqpoint{0.274693in}{-6.412950in}}%
\pgfpathclose%
\pgfusepath{stroke}%
\end{pgfscope}%
\begin{pgfscope}%
\pgfpathrectangle{\pgfqpoint{0.383193in}{0.383578in}}{\pgfqpoint{2.325000in}{2.310000in}}%
\pgfusepath{clip}%
\pgfsetbuttcap%
\pgfsetroundjoin%
\pgfsetlinewidth{1.405250pt}%
\definecolor{currentstroke}{rgb}{0.333333,0.333333,0.333333}%
\pgfsetstrokecolor{currentstroke}%
\pgfsetdash{}{0pt}%
\pgfpathmoveto{\pgfqpoint{-0.031432in}{-4.872950in}}%
\pgfpathcurveto{\pgfqpoint{-0.015778in}{-4.872950in}}{\pgfqpoint{-0.000763in}{-4.866730in}}{\pgfqpoint{0.010307in}{-4.855661in}}%
\pgfpathcurveto{\pgfqpoint{0.021376in}{-4.844592in}}{\pgfqpoint{0.027596in}{-4.829576in}}{\pgfqpoint{0.027596in}{-4.813922in}}%
\pgfpathcurveto{\pgfqpoint{0.027596in}{-4.798268in}}{\pgfqpoint{0.021376in}{-4.783252in}}{\pgfqpoint{0.010307in}{-4.772183in}}%
\pgfpathcurveto{\pgfqpoint{-0.000763in}{-4.761114in}}{\pgfqpoint{-0.015778in}{-4.754894in}}{\pgfqpoint{-0.031432in}{-4.754894in}}%
\pgfpathcurveto{\pgfqpoint{-0.047087in}{-4.754894in}}{\pgfqpoint{-0.062102in}{-4.761114in}}{\pgfqpoint{-0.073171in}{-4.772183in}}%
\pgfpathcurveto{\pgfqpoint{-0.084240in}{-4.783252in}}{\pgfqpoint{-0.090460in}{-4.798268in}}{\pgfqpoint{-0.090460in}{-4.813922in}}%
\pgfpathcurveto{\pgfqpoint{-0.090460in}{-4.829576in}}{\pgfqpoint{-0.084240in}{-4.844592in}}{\pgfqpoint{-0.073171in}{-4.855661in}}%
\pgfpathcurveto{\pgfqpoint{-0.062102in}{-4.866730in}}{\pgfqpoint{-0.047087in}{-4.872950in}}{\pgfqpoint{-0.031432in}{-4.872950in}}%
\pgfpathclose%
\pgfusepath{stroke}%
\end{pgfscope}%
\begin{pgfscope}%
\pgfpathrectangle{\pgfqpoint{0.383193in}{0.383578in}}{\pgfqpoint{2.325000in}{2.310000in}}%
\pgfusepath{clip}%
\pgfsetbuttcap%
\pgfsetroundjoin%
\pgfsetlinewidth{1.405250pt}%
\definecolor{currentstroke}{rgb}{0.333333,0.333333,0.333333}%
\pgfsetstrokecolor{currentstroke}%
\pgfsetdash{}{0pt}%
\pgfpathmoveto{\pgfqpoint{-0.682432in}{-4.102950in}}%
\pgfpathcurveto{\pgfqpoint{-0.666778in}{-4.102950in}}{\pgfqpoint{-0.651763in}{-4.096730in}}{\pgfqpoint{-0.640693in}{-4.085661in}}%
\pgfpathcurveto{\pgfqpoint{-0.629624in}{-4.074592in}}{\pgfqpoint{-0.623404in}{-4.059576in}}{\pgfqpoint{-0.623404in}{-4.043922in}}%
\pgfpathcurveto{\pgfqpoint{-0.623404in}{-4.028268in}}{\pgfqpoint{-0.629624in}{-4.013252in}}{\pgfqpoint{-0.640693in}{-4.002183in}}%
\pgfpathcurveto{\pgfqpoint{-0.651763in}{-3.991114in}}{\pgfqpoint{-0.666778in}{-3.984894in}}{\pgfqpoint{-0.682432in}{-3.984894in}}%
\pgfpathcurveto{\pgfqpoint{-0.698087in}{-3.984894in}}{\pgfqpoint{-0.713102in}{-3.991114in}}{\pgfqpoint{-0.724171in}{-4.002183in}}%
\pgfpathcurveto{\pgfqpoint{-0.735240in}{-4.013252in}}{\pgfqpoint{-0.741460in}{-4.028268in}}{\pgfqpoint{-0.741460in}{-4.043922in}}%
\pgfpathcurveto{\pgfqpoint{-0.741460in}{-4.059576in}}{\pgfqpoint{-0.735240in}{-4.074592in}}{\pgfqpoint{-0.724171in}{-4.085661in}}%
\pgfpathcurveto{\pgfqpoint{-0.713102in}{-4.096730in}}{\pgfqpoint{-0.698087in}{-4.102950in}}{\pgfqpoint{-0.682432in}{-4.102950in}}%
\pgfpathclose%
\pgfusepath{stroke}%
\end{pgfscope}%
\begin{pgfscope}%
\pgfpathrectangle{\pgfqpoint{0.383193in}{0.383578in}}{\pgfqpoint{2.325000in}{2.310000in}}%
\pgfusepath{clip}%
\pgfsetbuttcap%
\pgfsetroundjoin%
\pgfsetlinewidth{1.405250pt}%
\definecolor{currentstroke}{rgb}{0.333333,0.333333,0.333333}%
\pgfsetstrokecolor{currentstroke}%
\pgfsetdash{}{0pt}%
\pgfpathmoveto{\pgfqpoint{5.039005in}{-3.717950in}}%
\pgfpathcurveto{\pgfqpoint{5.054660in}{-3.717950in}}{\pgfqpoint{5.069675in}{-3.711730in}}{\pgfqpoint{5.080744in}{-3.700661in}}%
\pgfpathcurveto{\pgfqpoint{5.091814in}{-3.689592in}}{\pgfqpoint{5.098033in}{-3.674576in}}{\pgfqpoint{5.098033in}{-3.658922in}}%
\pgfpathcurveto{\pgfqpoint{5.098033in}{-3.643268in}}{\pgfqpoint{5.091814in}{-3.628252in}}{\pgfqpoint{5.080744in}{-3.617183in}}%
\pgfpathcurveto{\pgfqpoint{5.069675in}{-3.606114in}}{\pgfqpoint{5.054660in}{-3.599894in}}{\pgfqpoint{5.039005in}{-3.599894in}}%
\pgfpathcurveto{\pgfqpoint{5.023351in}{-3.599894in}}{\pgfqpoint{5.008336in}{-3.606114in}}{\pgfqpoint{4.997266in}{-3.617183in}}%
\pgfpathcurveto{\pgfqpoint{4.986197in}{-3.628252in}}{\pgfqpoint{4.979978in}{-3.643268in}}{\pgfqpoint{4.979978in}{-3.658922in}}%
\pgfpathcurveto{\pgfqpoint{4.979978in}{-3.674576in}}{\pgfqpoint{4.986197in}{-3.689592in}}{\pgfqpoint{4.997266in}{-3.700661in}}%
\pgfpathcurveto{\pgfqpoint{5.008336in}{-3.711730in}}{\pgfqpoint{5.023351in}{-3.717950in}}{\pgfqpoint{5.039005in}{-3.717950in}}%
\pgfpathclose%
\pgfusepath{stroke}%
\end{pgfscope}%
\begin{pgfscope}%
\pgfpathrectangle{\pgfqpoint{0.383193in}{0.383578in}}{\pgfqpoint{2.325000in}{2.310000in}}%
\pgfusepath{clip}%
\pgfsetbuttcap%
\pgfsetroundjoin%
\pgfsetlinewidth{1.405250pt}%
\definecolor{currentstroke}{rgb}{0.333333,0.333333,0.333333}%
\pgfsetstrokecolor{currentstroke}%
\pgfsetdash{}{0pt}%
\pgfpathmoveto{\pgfqpoint{-0.427651in}{-2.755450in}}%
\pgfpathcurveto{\pgfqpoint{-0.411997in}{-2.755450in}}{\pgfqpoint{-0.396981in}{-2.749230in}}{\pgfqpoint{-0.385912in}{-2.738161in}}%
\pgfpathcurveto{\pgfqpoint{-0.374843in}{-2.727092in}}{\pgfqpoint{-0.368623in}{-2.712076in}}{\pgfqpoint{-0.368623in}{-2.696422in}}%
\pgfpathcurveto{\pgfqpoint{-0.368623in}{-2.680768in}}{\pgfqpoint{-0.374843in}{-2.665752in}}{\pgfqpoint{-0.385912in}{-2.654683in}}%
\pgfpathcurveto{\pgfqpoint{-0.396981in}{-2.643614in}}{\pgfqpoint{-0.411997in}{-2.637394in}}{\pgfqpoint{-0.427651in}{-2.637394in}}%
\pgfpathcurveto{\pgfqpoint{-0.443305in}{-2.637394in}}{\pgfqpoint{-0.458321in}{-2.643614in}}{\pgfqpoint{-0.469390in}{-2.654683in}}%
\pgfpathcurveto{\pgfqpoint{-0.480459in}{-2.665752in}}{\pgfqpoint{-0.486679in}{-2.680768in}}{\pgfqpoint{-0.486679in}{-2.696422in}}%
\pgfpathcurveto{\pgfqpoint{-0.486679in}{-2.712076in}}{\pgfqpoint{-0.480459in}{-2.727092in}}{\pgfqpoint{-0.469390in}{-2.738161in}}%
\pgfpathcurveto{\pgfqpoint{-0.458321in}{-2.749230in}}{\pgfqpoint{-0.443305in}{-2.755450in}}{\pgfqpoint{-0.427651in}{-2.755450in}}%
\pgfpathclose%
\pgfusepath{stroke}%
\end{pgfscope}%
\begin{pgfscope}%
\pgfpathrectangle{\pgfqpoint{0.383193in}{0.383578in}}{\pgfqpoint{2.325000in}{2.310000in}}%
\pgfusepath{clip}%
\pgfsetbuttcap%
\pgfsetroundjoin%
\pgfsetlinewidth{1.405250pt}%
\definecolor{currentstroke}{rgb}{0.333333,0.333333,0.333333}%
\pgfsetstrokecolor{currentstroke}%
\pgfsetdash{}{0pt}%
\pgfpathmoveto{\pgfqpoint{4.335693in}{-0.637950in}}%
\pgfpathcurveto{\pgfqpoint{4.351347in}{-0.637950in}}{\pgfqpoint{4.366362in}{-0.631730in}}{\pgfqpoint{4.377432in}{-0.620661in}}%
\pgfpathcurveto{\pgfqpoint{4.388501in}{-0.609592in}}{\pgfqpoint{4.394721in}{-0.594576in}}{\pgfqpoint{4.394721in}{-0.578922in}}%
\pgfpathcurveto{\pgfqpoint{4.394721in}{-0.563268in}}{\pgfqpoint{4.388501in}{-0.548252in}}{\pgfqpoint{4.377432in}{-0.537183in}}%
\pgfpathcurveto{\pgfqpoint{4.366362in}{-0.526114in}}{\pgfqpoint{4.351347in}{-0.519894in}}{\pgfqpoint{4.335693in}{-0.519894in}}%
\pgfpathcurveto{\pgfqpoint{4.320038in}{-0.519894in}}{\pgfqpoint{4.305023in}{-0.526114in}}{\pgfqpoint{4.293954in}{-0.537183in}}%
\pgfpathcurveto{\pgfqpoint{4.282885in}{-0.548252in}}{\pgfqpoint{4.276665in}{-0.563268in}}{\pgfqpoint{4.276665in}{-0.578922in}}%
\pgfpathcurveto{\pgfqpoint{4.276665in}{-0.594576in}}{\pgfqpoint{4.282885in}{-0.609592in}}{\pgfqpoint{4.293954in}{-0.620661in}}%
\pgfpathcurveto{\pgfqpoint{4.305023in}{-0.631730in}}{\pgfqpoint{4.320038in}{-0.637950in}}{\pgfqpoint{4.335693in}{-0.637950in}}%
\pgfpathclose%
\pgfusepath{stroke}%
\end{pgfscope}%
\begin{pgfscope}%
\pgfpathrectangle{\pgfqpoint{0.383193in}{0.383578in}}{\pgfqpoint{2.325000in}{2.310000in}}%
\pgfusepath{clip}%
\pgfsetbuttcap%
\pgfsetroundjoin%
\pgfsetlinewidth{1.405250pt}%
\definecolor{currentstroke}{rgb}{0.333333,0.333333,0.333333}%
\pgfsetstrokecolor{currentstroke}%
\pgfsetdash{}{0pt}%
\pgfpathmoveto{\pgfqpoint{-0.466401in}{0.132050in}}%
\pgfpathcurveto{\pgfqpoint{-0.450747in}{0.132050in}}{\pgfqpoint{-0.435731in}{0.138270in}}{\pgfqpoint{-0.424662in}{0.149339in}}%
\pgfpathcurveto{\pgfqpoint{-0.413593in}{0.160408in}}{\pgfqpoint{-0.407373in}{0.175424in}}{\pgfqpoint{-0.407373in}{0.191078in}}%
\pgfpathcurveto{\pgfqpoint{-0.407373in}{0.206732in}}{\pgfqpoint{-0.413593in}{0.221748in}}{\pgfqpoint{-0.424662in}{0.232817in}}%
\pgfpathcurveto{\pgfqpoint{-0.435731in}{0.243886in}}{\pgfqpoint{-0.450747in}{0.250106in}}{\pgfqpoint{-0.466401in}{0.250106in}}%
\pgfpathcurveto{\pgfqpoint{-0.482055in}{0.250106in}}{\pgfqpoint{-0.497071in}{0.243886in}}{\pgfqpoint{-0.508140in}{0.232817in}}%
\pgfpathcurveto{\pgfqpoint{-0.519209in}{0.221748in}}{\pgfqpoint{-0.525429in}{0.206732in}}{\pgfqpoint{-0.525429in}{0.191078in}}%
\pgfpathcurveto{\pgfqpoint{-0.525429in}{0.175424in}}{\pgfqpoint{-0.519209in}{0.160408in}}{\pgfqpoint{-0.508140in}{0.149339in}}%
\pgfpathcurveto{\pgfqpoint{-0.497071in}{0.138270in}}{\pgfqpoint{-0.482055in}{0.132050in}}{\pgfqpoint{-0.466401in}{0.132050in}}%
\pgfpathclose%
\pgfusepath{stroke}%
\end{pgfscope}%
\begin{pgfscope}%
\pgfpathrectangle{\pgfqpoint{0.383193in}{0.383578in}}{\pgfqpoint{2.325000in}{2.310000in}}%
\pgfusepath{clip}%
\pgfsetbuttcap%
\pgfsetroundjoin%
\pgfsetlinewidth{1.405250pt}%
\definecolor{currentstroke}{rgb}{0.333333,0.333333,0.333333}%
\pgfsetstrokecolor{currentstroke}%
\pgfsetdash{}{0pt}%
\pgfpathmoveto{\pgfqpoint{-2.015432in}{0.324550in}}%
\pgfpathcurveto{\pgfqpoint{-1.999778in}{0.324550in}}{\pgfqpoint{-1.984763in}{0.330770in}}{\pgfqpoint{-1.973693in}{0.341839in}}%
\pgfpathcurveto{\pgfqpoint{-1.962624in}{0.352908in}}{\pgfqpoint{-1.956404in}{0.367924in}}{\pgfqpoint{-1.956404in}{0.383578in}}%
\pgfpathcurveto{\pgfqpoint{-1.956404in}{0.399232in}}{\pgfqpoint{-1.962624in}{0.414248in}}{\pgfqpoint{-1.973693in}{0.425317in}}%
\pgfpathcurveto{\pgfqpoint{-1.984763in}{0.436386in}}{\pgfqpoint{-1.999778in}{0.442606in}}{\pgfqpoint{-2.015432in}{0.442606in}}%
\pgfpathcurveto{\pgfqpoint{-2.031087in}{0.442606in}}{\pgfqpoint{-2.046102in}{0.436386in}}{\pgfqpoint{-2.057171in}{0.425317in}}%
\pgfpathcurveto{\pgfqpoint{-2.068240in}{0.414248in}}{\pgfqpoint{-2.074460in}{0.399232in}}{\pgfqpoint{-2.074460in}{0.383578in}}%
\pgfpathcurveto{\pgfqpoint{-2.074460in}{0.367924in}}{\pgfqpoint{-2.068240in}{0.352908in}}{\pgfqpoint{-2.057171in}{0.341839in}}%
\pgfpathcurveto{\pgfqpoint{-2.046102in}{0.330770in}}{\pgfqpoint{-2.031087in}{0.324550in}}{\pgfqpoint{-2.015432in}{0.324550in}}%
\pgfpathclose%
\pgfusepath{stroke}%
\end{pgfscope}%
\begin{pgfscope}%
\pgfpathrectangle{\pgfqpoint{0.383193in}{0.383578in}}{\pgfqpoint{2.325000in}{2.310000in}}%
\pgfusepath{clip}%
\pgfsetbuttcap%
\pgfsetroundjoin%
\pgfsetlinewidth{1.405250pt}%
\definecolor{currentstroke}{rgb}{0.333333,0.333333,0.333333}%
\pgfsetstrokecolor{currentstroke}%
\pgfsetdash{}{0pt}%
\pgfpathmoveto{\pgfqpoint{1.708443in}{0.709550in}}%
\pgfpathcurveto{\pgfqpoint{1.724097in}{0.709550in}}{\pgfqpoint{1.739112in}{0.715770in}}{\pgfqpoint{1.750182in}{0.726839in}}%
\pgfpathcurveto{\pgfqpoint{1.761251in}{0.737908in}}{\pgfqpoint{1.767471in}{0.752924in}}{\pgfqpoint{1.767471in}{0.768578in}}%
\pgfpathcurveto{\pgfqpoint{1.767471in}{0.784232in}}{\pgfqpoint{1.761251in}{0.799248in}}{\pgfqpoint{1.750182in}{0.810317in}}%
\pgfpathcurveto{\pgfqpoint{1.739112in}{0.821386in}}{\pgfqpoint{1.724097in}{0.827606in}}{\pgfqpoint{1.708443in}{0.827606in}}%
\pgfpathcurveto{\pgfqpoint{1.692788in}{0.827606in}}{\pgfqpoint{1.677773in}{0.821386in}}{\pgfqpoint{1.666704in}{0.810317in}}%
\pgfpathcurveto{\pgfqpoint{1.655635in}{0.799248in}}{\pgfqpoint{1.649415in}{0.784232in}}{\pgfqpoint{1.649415in}{0.768578in}}%
\pgfpathcurveto{\pgfqpoint{1.649415in}{0.752924in}}{\pgfqpoint{1.655635in}{0.737908in}}{\pgfqpoint{1.666704in}{0.726839in}}%
\pgfpathcurveto{\pgfqpoint{1.677773in}{0.715770in}}{\pgfqpoint{1.692788in}{0.709550in}}{\pgfqpoint{1.708443in}{0.709550in}}%
\pgfpathclose%
\pgfusepath{stroke}%
\end{pgfscope}%
\begin{pgfscope}%
\pgfpathrectangle{\pgfqpoint{0.383193in}{0.383578in}}{\pgfqpoint{2.325000in}{2.310000in}}%
\pgfusepath{clip}%
\pgfsetbuttcap%
\pgfsetroundjoin%
\pgfsetlinewidth{1.405250pt}%
\definecolor{currentstroke}{rgb}{0.333333,0.333333,0.333333}%
\pgfsetstrokecolor{currentstroke}%
\pgfsetdash{}{0pt}%
\pgfpathmoveto{\pgfqpoint{2.163755in}{0.902050in}}%
\pgfpathcurveto{\pgfqpoint{2.179410in}{0.902050in}}{\pgfqpoint{2.194425in}{0.908270in}}{\pgfqpoint{2.205494in}{0.919339in}}%
\pgfpathcurveto{\pgfqpoint{2.216564in}{0.930408in}}{\pgfqpoint{2.222783in}{0.945424in}}{\pgfqpoint{2.222783in}{0.961078in}}%
\pgfpathcurveto{\pgfqpoint{2.222783in}{0.976732in}}{\pgfqpoint{2.216564in}{0.991748in}}{\pgfqpoint{2.205494in}{1.002817in}}%
\pgfpathcurveto{\pgfqpoint{2.194425in}{1.013886in}}{\pgfqpoint{2.179410in}{1.020106in}}{\pgfqpoint{2.163755in}{1.020106in}}%
\pgfpathcurveto{\pgfqpoint{2.148101in}{1.020106in}}{\pgfqpoint{2.133086in}{1.013886in}}{\pgfqpoint{2.122016in}{1.002817in}}%
\pgfpathcurveto{\pgfqpoint{2.110947in}{0.991748in}}{\pgfqpoint{2.104728in}{0.976732in}}{\pgfqpoint{2.104728in}{0.961078in}}%
\pgfpathcurveto{\pgfqpoint{2.104728in}{0.945424in}}{\pgfqpoint{2.110947in}{0.930408in}}{\pgfqpoint{2.122016in}{0.919339in}}%
\pgfpathcurveto{\pgfqpoint{2.133086in}{0.908270in}}{\pgfqpoint{2.148101in}{0.902050in}}{\pgfqpoint{2.163755in}{0.902050in}}%
\pgfpathclose%
\pgfusepath{stroke}%
\end{pgfscope}%
\begin{pgfscope}%
\pgfpathrectangle{\pgfqpoint{0.383193in}{0.383578in}}{\pgfqpoint{2.325000in}{2.310000in}}%
\pgfusepath{clip}%
\pgfsetbuttcap%
\pgfsetroundjoin%
\pgfsetlinewidth{1.405250pt}%
\definecolor{currentstroke}{rgb}{0.333333,0.333333,0.333333}%
\pgfsetstrokecolor{currentstroke}%
\pgfsetdash{}{0pt}%
\pgfpathmoveto{\pgfqpoint{-1.236557in}{2.057050in}}%
\pgfpathcurveto{\pgfqpoint{-1.220903in}{2.057050in}}{\pgfqpoint{-1.205888in}{2.063270in}}{\pgfqpoint{-1.194818in}{2.074339in}}%
\pgfpathcurveto{\pgfqpoint{-1.183749in}{2.085408in}}{\pgfqpoint{-1.177529in}{2.100424in}}{\pgfqpoint{-1.177529in}{2.116078in}}%
\pgfpathcurveto{\pgfqpoint{-1.177529in}{2.131732in}}{\pgfqpoint{-1.183749in}{2.146748in}}{\pgfqpoint{-1.194818in}{2.157817in}}%
\pgfpathcurveto{\pgfqpoint{-1.205888in}{2.168886in}}{\pgfqpoint{-1.220903in}{2.175106in}}{\pgfqpoint{-1.236557in}{2.175106in}}%
\pgfpathcurveto{\pgfqpoint{-1.252212in}{2.175106in}}{\pgfqpoint{-1.267227in}{2.168886in}}{\pgfqpoint{-1.278296in}{2.157817in}}%
\pgfpathcurveto{\pgfqpoint{-1.289365in}{2.146748in}}{\pgfqpoint{-1.295585in}{2.131732in}}{\pgfqpoint{-1.295585in}{2.116078in}}%
\pgfpathcurveto{\pgfqpoint{-1.295585in}{2.100424in}}{\pgfqpoint{-1.289365in}{2.085408in}}{\pgfqpoint{-1.278296in}{2.074339in}}%
\pgfpathcurveto{\pgfqpoint{-1.267227in}{2.063270in}}{\pgfqpoint{-1.252212in}{2.057050in}}{\pgfqpoint{-1.236557in}{2.057050in}}%
\pgfpathclose%
\pgfusepath{stroke}%
\end{pgfscope}%
\begin{pgfscope}%
\pgfpathrectangle{\pgfqpoint{0.383193in}{0.383578in}}{\pgfqpoint{2.325000in}{2.310000in}}%
\pgfusepath{clip}%
\pgfsetbuttcap%
\pgfsetroundjoin%
\pgfsetlinewidth{1.405250pt}%
\definecolor{currentstroke}{rgb}{0.333333,0.333333,0.333333}%
\pgfsetstrokecolor{currentstroke}%
\pgfsetdash{}{0pt}%
\pgfpathmoveto{\pgfqpoint{-0.811276in}{2.827050in}}%
\pgfpathcurveto{\pgfqpoint{-0.795622in}{2.827050in}}{\pgfqpoint{-0.780606in}{2.833270in}}{\pgfqpoint{-0.769537in}{2.844339in}}%
\pgfpathcurveto{\pgfqpoint{-0.758468in}{2.855408in}}{\pgfqpoint{-0.752248in}{2.870424in}}{\pgfqpoint{-0.752248in}{2.886078in}}%
\pgfpathcurveto{\pgfqpoint{-0.752248in}{2.901732in}}{\pgfqpoint{-0.758468in}{2.916748in}}{\pgfqpoint{-0.769537in}{2.927817in}}%
\pgfpathcurveto{\pgfqpoint{-0.780606in}{2.938886in}}{\pgfqpoint{-0.795622in}{2.945106in}}{\pgfqpoint{-0.811276in}{2.945106in}}%
\pgfpathcurveto{\pgfqpoint{-0.826930in}{2.945106in}}{\pgfqpoint{-0.841946in}{2.938886in}}{\pgfqpoint{-0.853015in}{2.927817in}}%
\pgfpathcurveto{\pgfqpoint{-0.864084in}{2.916748in}}{\pgfqpoint{-0.870304in}{2.901732in}}{\pgfqpoint{-0.870304in}{2.886078in}}%
\pgfpathcurveto{\pgfqpoint{-0.870304in}{2.870424in}}{\pgfqpoint{-0.864084in}{2.855408in}}{\pgfqpoint{-0.853015in}{2.844339in}}%
\pgfpathcurveto{\pgfqpoint{-0.841946in}{2.833270in}}{\pgfqpoint{-0.826930in}{2.827050in}}{\pgfqpoint{-0.811276in}{2.827050in}}%
\pgfpathclose%
\pgfusepath{stroke}%
\end{pgfscope}%
\begin{pgfscope}%
\pgfpathrectangle{\pgfqpoint{0.383193in}{0.383578in}}{\pgfqpoint{2.325000in}{2.310000in}}%
\pgfusepath{clip}%
\pgfsetbuttcap%
\pgfsetroundjoin%
\pgfsetlinewidth{1.405250pt}%
\definecolor{currentstroke}{rgb}{0.333333,0.333333,0.333333}%
\pgfsetstrokecolor{currentstroke}%
\pgfsetdash{}{0pt}%
\pgfpathmoveto{\pgfqpoint{5.498193in}{3.597050in}}%
\pgfpathcurveto{\pgfqpoint{5.513847in}{3.597050in}}{\pgfqpoint{5.528862in}{3.603270in}}{\pgfqpoint{5.539932in}{3.614339in}}%
\pgfpathcurveto{\pgfqpoint{5.551001in}{3.625408in}}{\pgfqpoint{5.557221in}{3.640424in}}{\pgfqpoint{5.557221in}{3.656078in}}%
\pgfpathcurveto{\pgfqpoint{5.557221in}{3.671732in}}{\pgfqpoint{5.551001in}{3.686748in}}{\pgfqpoint{5.539932in}{3.697817in}}%
\pgfpathcurveto{\pgfqpoint{5.528862in}{3.708886in}}{\pgfqpoint{5.513847in}{3.715106in}}{\pgfqpoint{5.498193in}{3.715106in}}%
\pgfpathcurveto{\pgfqpoint{5.482538in}{3.715106in}}{\pgfqpoint{5.467523in}{3.708886in}}{\pgfqpoint{5.456454in}{3.697817in}}%
\pgfpathcurveto{\pgfqpoint{5.445385in}{3.686748in}}{\pgfqpoint{5.439165in}{3.671732in}}{\pgfqpoint{5.439165in}{3.656078in}}%
\pgfpathcurveto{\pgfqpoint{5.439165in}{3.640424in}}{\pgfqpoint{5.445385in}{3.625408in}}{\pgfqpoint{5.456454in}{3.614339in}}%
\pgfpathcurveto{\pgfqpoint{5.467523in}{3.603270in}}{\pgfqpoint{5.482538in}{3.597050in}}{\pgfqpoint{5.498193in}{3.597050in}}%
\pgfpathclose%
\pgfusepath{stroke}%
\end{pgfscope}%
\begin{pgfscope}%
\pgfpathrectangle{\pgfqpoint{0.383193in}{0.383578in}}{\pgfqpoint{2.325000in}{2.310000in}}%
\pgfusepath{clip}%
\pgfsetbuttcap%
\pgfsetroundjoin%
\pgfsetlinewidth{1.405250pt}%
\definecolor{currentstroke}{rgb}{0.333333,0.333333,0.333333}%
\pgfsetstrokecolor{currentstroke}%
\pgfsetdash{}{0pt}%
\pgfpathmoveto{\pgfqpoint{1.132037in}{4.367050in}}%
\pgfpathcurveto{\pgfqpoint{1.147691in}{4.367050in}}{\pgfqpoint{1.162706in}{4.373270in}}{\pgfqpoint{1.173775in}{4.384339in}}%
\pgfpathcurveto{\pgfqpoint{1.184845in}{4.395408in}}{\pgfqpoint{1.191064in}{4.410424in}}{\pgfqpoint{1.191064in}{4.426078in}}%
\pgfpathcurveto{\pgfqpoint{1.191064in}{4.441732in}}{\pgfqpoint{1.184845in}{4.456748in}}{\pgfqpoint{1.173775in}{4.467817in}}%
\pgfpathcurveto{\pgfqpoint{1.162706in}{4.478886in}}{\pgfqpoint{1.147691in}{4.485106in}}{\pgfqpoint{1.132037in}{4.485106in}}%
\pgfpathcurveto{\pgfqpoint{1.116382in}{4.485106in}}{\pgfqpoint{1.101367in}{4.478886in}}{\pgfqpoint{1.090298in}{4.467817in}}%
\pgfpathcurveto{\pgfqpoint{1.079228in}{4.456748in}}{\pgfqpoint{1.073009in}{4.441732in}}{\pgfqpoint{1.073009in}{4.426078in}}%
\pgfpathcurveto{\pgfqpoint{1.073009in}{4.410424in}}{\pgfqpoint{1.079228in}{4.395408in}}{\pgfqpoint{1.090298in}{4.384339in}}%
\pgfpathcurveto{\pgfqpoint{1.101367in}{4.373270in}}{\pgfqpoint{1.116382in}{4.367050in}}{\pgfqpoint{1.132037in}{4.367050in}}%
\pgfpathclose%
\pgfusepath{stroke}%
\end{pgfscope}%
\begin{pgfscope}%
\pgfpathrectangle{\pgfqpoint{0.383193in}{0.383578in}}{\pgfqpoint{2.325000in}{2.310000in}}%
\pgfusepath{clip}%
\pgfsetbuttcap%
\pgfsetroundjoin%
\pgfsetlinewidth{1.405250pt}%
\definecolor{currentstroke}{rgb}{0.333333,0.333333,0.333333}%
\pgfsetstrokecolor{currentstroke}%
\pgfsetdash{}{0pt}%
\pgfpathmoveto{\pgfqpoint{2.032005in}{4.944550in}}%
\pgfpathcurveto{\pgfqpoint{2.047660in}{4.944550in}}{\pgfqpoint{2.062675in}{4.950770in}}{\pgfqpoint{2.073744in}{4.961839in}}%
\pgfpathcurveto{\pgfqpoint{2.084814in}{4.972908in}}{\pgfqpoint{2.091033in}{4.987924in}}{\pgfqpoint{2.091033in}{5.003578in}}%
\pgfpathcurveto{\pgfqpoint{2.091033in}{5.019232in}}{\pgfqpoint{2.084814in}{5.034248in}}{\pgfqpoint{2.073744in}{5.045317in}}%
\pgfpathcurveto{\pgfqpoint{2.062675in}{5.056386in}}{\pgfqpoint{2.047660in}{5.062606in}}{\pgfqpoint{2.032005in}{5.062606in}}%
\pgfpathcurveto{\pgfqpoint{2.016351in}{5.062606in}}{\pgfqpoint{2.001336in}{5.056386in}}{\pgfqpoint{1.990266in}{5.045317in}}%
\pgfpathcurveto{\pgfqpoint{1.979197in}{5.034248in}}{\pgfqpoint{1.972978in}{5.019232in}}{\pgfqpoint{1.972978in}{5.003578in}}%
\pgfpathcurveto{\pgfqpoint{1.972978in}{4.987924in}}{\pgfqpoint{1.979197in}{4.972908in}}{\pgfqpoint{1.990266in}{4.961839in}}%
\pgfpathcurveto{\pgfqpoint{2.001336in}{4.950770in}}{\pgfqpoint{2.016351in}{4.944550in}}{\pgfqpoint{2.032005in}{4.944550in}}%
\pgfpathclose%
\pgfusepath{stroke}%
\end{pgfscope}%
\begin{pgfscope}%
\pgfpathrectangle{\pgfqpoint{0.383193in}{0.383578in}}{\pgfqpoint{2.325000in}{2.310000in}}%
\pgfusepath{clip}%
\pgfsetbuttcap%
\pgfsetroundjoin%
\pgfsetlinewidth{1.405250pt}%
\definecolor{currentstroke}{rgb}{0.333333,0.333333,0.333333}%
\pgfsetstrokecolor{currentstroke}%
\pgfsetdash{}{0pt}%
\pgfpathmoveto{\pgfqpoint{3.946255in}{5.137050in}}%
\pgfpathcurveto{\pgfqpoint{3.961910in}{5.137050in}}{\pgfqpoint{3.976925in}{5.143270in}}{\pgfqpoint{3.987994in}{5.154339in}}%
\pgfpathcurveto{\pgfqpoint{3.999064in}{5.165408in}}{\pgfqpoint{4.005283in}{5.180424in}}{\pgfqpoint{4.005283in}{5.196078in}}%
\pgfpathcurveto{\pgfqpoint{4.005283in}{5.211732in}}{\pgfqpoint{3.999064in}{5.226748in}}{\pgfqpoint{3.987994in}{5.237817in}}%
\pgfpathcurveto{\pgfqpoint{3.976925in}{5.248886in}}{\pgfqpoint{3.961910in}{5.255106in}}{\pgfqpoint{3.946255in}{5.255106in}}%
\pgfpathcurveto{\pgfqpoint{3.930601in}{5.255106in}}{\pgfqpoint{3.915586in}{5.248886in}}{\pgfqpoint{3.904516in}{5.237817in}}%
\pgfpathcurveto{\pgfqpoint{3.893447in}{5.226748in}}{\pgfqpoint{3.887227in}{5.211732in}}{\pgfqpoint{3.887227in}{5.196078in}}%
\pgfpathcurveto{\pgfqpoint{3.887227in}{5.180424in}}{\pgfqpoint{3.893447in}{5.165408in}}{\pgfqpoint{3.904516in}{5.154339in}}%
\pgfpathcurveto{\pgfqpoint{3.915586in}{5.143270in}}{\pgfqpoint{3.930601in}{5.137050in}}{\pgfqpoint{3.946255in}{5.137050in}}%
\pgfpathclose%
\pgfusepath{stroke}%
\end{pgfscope}%
\begin{pgfscope}%
\pgfpathrectangle{\pgfqpoint{0.383193in}{0.383578in}}{\pgfqpoint{2.325000in}{2.310000in}}%
\pgfusepath{clip}%
\pgfsetbuttcap%
\pgfsetroundjoin%
\pgfsetlinewidth{1.405250pt}%
\definecolor{currentstroke}{rgb}{0.686275,0.352941,0.313725}%
\pgfsetstrokecolor{currentstroke}%
\pgfsetdash{}{0pt}%
\pgfpathmoveto{\pgfqpoint{-1.909838in}{-11.208089in}}%
\pgfpathcurveto{\pgfqpoint{-1.898788in}{-11.208089in}}{\pgfqpoint{-1.888189in}{-11.203698in}}{\pgfqpoint{-1.880376in}{-11.195885in}}%
\pgfpathcurveto{\pgfqpoint{-1.872562in}{-11.188071in}}{\pgfqpoint{-1.868172in}{-11.177472in}}{\pgfqpoint{-1.868172in}{-11.166422in}}%
\pgfpathcurveto{\pgfqpoint{-1.868172in}{-11.155372in}}{\pgfqpoint{-1.872562in}{-11.144773in}}{\pgfqpoint{-1.880376in}{-11.136959in}}%
\pgfpathcurveto{\pgfqpoint{-1.888189in}{-11.129146in}}{\pgfqpoint{-1.898788in}{-11.124755in}}{\pgfqpoint{-1.909838in}{-11.124755in}}%
\pgfpathcurveto{\pgfqpoint{-1.920889in}{-11.124755in}}{\pgfqpoint{-1.931488in}{-11.129146in}}{\pgfqpoint{-1.939301in}{-11.136959in}}%
\pgfpathcurveto{\pgfqpoint{-1.947115in}{-11.144773in}}{\pgfqpoint{-1.951505in}{-11.155372in}}{\pgfqpoint{-1.951505in}{-11.166422in}}%
\pgfpathcurveto{\pgfqpoint{-1.951505in}{-11.177472in}}{\pgfqpoint{-1.947115in}{-11.188071in}}{\pgfqpoint{-1.939301in}{-11.195885in}}%
\pgfpathcurveto{\pgfqpoint{-1.931488in}{-11.203698in}}{\pgfqpoint{-1.920889in}{-11.208089in}}{\pgfqpoint{-1.909838in}{-11.208089in}}%
\pgfpathclose%
\pgfusepath{stroke}%
\end{pgfscope}%
\begin{pgfscope}%
\pgfpathrectangle{\pgfqpoint{0.383193in}{0.383578in}}{\pgfqpoint{2.325000in}{2.310000in}}%
\pgfusepath{clip}%
\pgfsetbuttcap%
\pgfsetroundjoin%
\pgfsetlinewidth{1.405250pt}%
\definecolor{currentstroke}{rgb}{0.686275,0.352941,0.313725}%
\pgfsetstrokecolor{currentstroke}%
\pgfsetdash{}{0pt}%
\pgfpathmoveto{\pgfqpoint{-2.009620in}{-9.860589in}}%
\pgfpathcurveto{\pgfqpoint{-1.998570in}{-9.860589in}}{\pgfqpoint{-1.987971in}{-9.856198in}}{\pgfqpoint{-1.980157in}{-9.848385in}}%
\pgfpathcurveto{\pgfqpoint{-1.972343in}{-9.840571in}}{\pgfqpoint{-1.967953in}{-9.829972in}}{\pgfqpoint{-1.967953in}{-9.818922in}}%
\pgfpathcurveto{\pgfqpoint{-1.967953in}{-9.807872in}}{\pgfqpoint{-1.972343in}{-9.797273in}}{\pgfqpoint{-1.980157in}{-9.789459in}}%
\pgfpathcurveto{\pgfqpoint{-1.987971in}{-9.781646in}}{\pgfqpoint{-1.998570in}{-9.777255in}}{\pgfqpoint{-2.009620in}{-9.777255in}}%
\pgfpathcurveto{\pgfqpoint{-2.020670in}{-9.777255in}}{\pgfqpoint{-2.031269in}{-9.781646in}}{\pgfqpoint{-2.039083in}{-9.789459in}}%
\pgfpathcurveto{\pgfqpoint{-2.046896in}{-9.797273in}}{\pgfqpoint{-2.051286in}{-9.807872in}}{\pgfqpoint{-2.051286in}{-9.818922in}}%
\pgfpathcurveto{\pgfqpoint{-2.051286in}{-9.829972in}}{\pgfqpoint{-2.046896in}{-9.840571in}}{\pgfqpoint{-2.039083in}{-9.848385in}}%
\pgfpathcurveto{\pgfqpoint{-2.031269in}{-9.856198in}}{\pgfqpoint{-2.020670in}{-9.860589in}}{\pgfqpoint{-2.009620in}{-9.860589in}}%
\pgfpathclose%
\pgfusepath{stroke}%
\end{pgfscope}%
\begin{pgfscope}%
\pgfpathrectangle{\pgfqpoint{0.383193in}{0.383578in}}{\pgfqpoint{2.325000in}{2.310000in}}%
\pgfusepath{clip}%
\pgfsetbuttcap%
\pgfsetroundjoin%
\pgfsetlinewidth{1.405250pt}%
\definecolor{currentstroke}{rgb}{0.686275,0.352941,0.313725}%
\pgfsetstrokecolor{currentstroke}%
\pgfsetdash{}{0pt}%
\pgfpathmoveto{\pgfqpoint{5.416818in}{-9.475589in}}%
\pgfpathcurveto{\pgfqpoint{5.427868in}{-9.475589in}}{\pgfqpoint{5.438467in}{-9.471198in}}{\pgfqpoint{5.446281in}{-9.463385in}}%
\pgfpathcurveto{\pgfqpoint{5.454094in}{-9.455571in}}{\pgfqpoint{5.458484in}{-9.444972in}}{\pgfqpoint{5.458484in}{-9.433922in}}%
\pgfpathcurveto{\pgfqpoint{5.458484in}{-9.422872in}}{\pgfqpoint{5.454094in}{-9.412273in}}{\pgfqpoint{5.446281in}{-9.404459in}}%
\pgfpathcurveto{\pgfqpoint{5.438467in}{-9.396646in}}{\pgfqpoint{5.427868in}{-9.392255in}}{\pgfqpoint{5.416818in}{-9.392255in}}%
\pgfpathcurveto{\pgfqpoint{5.405768in}{-9.392255in}}{\pgfqpoint{5.395169in}{-9.396646in}}{\pgfqpoint{5.387355in}{-9.404459in}}%
\pgfpathcurveto{\pgfqpoint{5.379541in}{-9.412273in}}{\pgfqpoint{5.375151in}{-9.422872in}}{\pgfqpoint{5.375151in}{-9.433922in}}%
\pgfpathcurveto{\pgfqpoint{5.375151in}{-9.444972in}}{\pgfqpoint{5.379541in}{-9.455571in}}{\pgfqpoint{5.387355in}{-9.463385in}}%
\pgfpathcurveto{\pgfqpoint{5.395169in}{-9.471198in}}{\pgfqpoint{5.405768in}{-9.475589in}}{\pgfqpoint{5.416818in}{-9.475589in}}%
\pgfpathclose%
\pgfusepath{stroke}%
\end{pgfscope}%
\begin{pgfscope}%
\pgfpathrectangle{\pgfqpoint{0.383193in}{0.383578in}}{\pgfqpoint{2.325000in}{2.310000in}}%
\pgfusepath{clip}%
\pgfsetbuttcap%
\pgfsetroundjoin%
\pgfsetlinewidth{1.405250pt}%
\definecolor{currentstroke}{rgb}{0.686275,0.352941,0.313725}%
\pgfsetstrokecolor{currentstroke}%
\pgfsetdash{}{0pt}%
\pgfpathmoveto{\pgfqpoint{1.163037in}{-9.283089in}}%
\pgfpathcurveto{\pgfqpoint{1.174087in}{-9.283089in}}{\pgfqpoint{1.184686in}{-9.278698in}}{\pgfqpoint{1.192499in}{-9.270885in}}%
\pgfpathcurveto{\pgfqpoint{1.200313in}{-9.263071in}}{\pgfqpoint{1.204703in}{-9.252472in}}{\pgfqpoint{1.204703in}{-9.241422in}}%
\pgfpathcurveto{\pgfqpoint{1.204703in}{-9.230372in}}{\pgfqpoint{1.200313in}{-9.219773in}}{\pgfqpoint{1.192499in}{-9.211959in}}%
\pgfpathcurveto{\pgfqpoint{1.184686in}{-9.204146in}}{\pgfqpoint{1.174087in}{-9.199755in}}{\pgfqpoint{1.163037in}{-9.199755in}}%
\pgfpathcurveto{\pgfqpoint{1.151986in}{-9.199755in}}{\pgfqpoint{1.141387in}{-9.204146in}}{\pgfqpoint{1.133574in}{-9.211959in}}%
\pgfpathcurveto{\pgfqpoint{1.125760in}{-9.219773in}}{\pgfqpoint{1.121370in}{-9.230372in}}{\pgfqpoint{1.121370in}{-9.241422in}}%
\pgfpathcurveto{\pgfqpoint{1.121370in}{-9.252472in}}{\pgfqpoint{1.125760in}{-9.263071in}}{\pgfqpoint{1.133574in}{-9.270885in}}%
\pgfpathcurveto{\pgfqpoint{1.141387in}{-9.278698in}}{\pgfqpoint{1.151986in}{-9.283089in}}{\pgfqpoint{1.163037in}{-9.283089in}}%
\pgfpathclose%
\pgfusepath{stroke}%
\end{pgfscope}%
\begin{pgfscope}%
\pgfpathrectangle{\pgfqpoint{0.383193in}{0.383578in}}{\pgfqpoint{2.325000in}{2.310000in}}%
\pgfusepath{clip}%
\pgfsetbuttcap%
\pgfsetroundjoin%
\pgfsetlinewidth{1.405250pt}%
\definecolor{currentstroke}{rgb}{0.686275,0.352941,0.313725}%
\pgfsetstrokecolor{currentstroke}%
\pgfsetdash{}{0pt}%
\pgfpathmoveto{\pgfqpoint{6.130787in}{-8.320589in}}%
\pgfpathcurveto{\pgfqpoint{6.141837in}{-8.320589in}}{\pgfqpoint{6.152436in}{-8.316198in}}{\pgfqpoint{6.160249in}{-8.308385in}}%
\pgfpathcurveto{\pgfqpoint{6.168063in}{-8.300571in}}{\pgfqpoint{6.172453in}{-8.289972in}}{\pgfqpoint{6.172453in}{-8.278922in}}%
\pgfpathcurveto{\pgfqpoint{6.172453in}{-8.267872in}}{\pgfqpoint{6.168063in}{-8.257273in}}{\pgfqpoint{6.160249in}{-8.249459in}}%
\pgfpathcurveto{\pgfqpoint{6.152436in}{-8.241646in}}{\pgfqpoint{6.141837in}{-8.237255in}}{\pgfqpoint{6.130787in}{-8.237255in}}%
\pgfpathcurveto{\pgfqpoint{6.119736in}{-8.237255in}}{\pgfqpoint{6.109137in}{-8.241646in}}{\pgfqpoint{6.101324in}{-8.249459in}}%
\pgfpathcurveto{\pgfqpoint{6.093510in}{-8.257273in}}{\pgfqpoint{6.089120in}{-8.267872in}}{\pgfqpoint{6.089120in}{-8.278922in}}%
\pgfpathcurveto{\pgfqpoint{6.089120in}{-8.289972in}}{\pgfqpoint{6.093510in}{-8.300571in}}{\pgfqpoint{6.101324in}{-8.308385in}}%
\pgfpathcurveto{\pgfqpoint{6.109137in}{-8.316198in}}{\pgfqpoint{6.119736in}{-8.320589in}}{\pgfqpoint{6.130787in}{-8.320589in}}%
\pgfpathclose%
\pgfusepath{stroke}%
\end{pgfscope}%
\begin{pgfscope}%
\pgfpathrectangle{\pgfqpoint{0.383193in}{0.383578in}}{\pgfqpoint{2.325000in}{2.310000in}}%
\pgfusepath{clip}%
\pgfsetbuttcap%
\pgfsetroundjoin%
\pgfsetlinewidth{1.405250pt}%
\definecolor{currentstroke}{rgb}{0.686275,0.352941,0.313725}%
\pgfsetstrokecolor{currentstroke}%
\pgfsetdash{}{0pt}%
\pgfpathmoveto{\pgfqpoint{0.846255in}{-7.935589in}}%
\pgfpathcurveto{\pgfqpoint{0.857305in}{-7.935589in}}{\pgfqpoint{0.867904in}{-7.931198in}}{\pgfqpoint{0.875718in}{-7.923385in}}%
\pgfpathcurveto{\pgfqpoint{0.883532in}{-7.915571in}}{\pgfqpoint{0.887922in}{-7.904972in}}{\pgfqpoint{0.887922in}{-7.893922in}}%
\pgfpathcurveto{\pgfqpoint{0.887922in}{-7.882872in}}{\pgfqpoint{0.883532in}{-7.872273in}}{\pgfqpoint{0.875718in}{-7.864459in}}%
\pgfpathcurveto{\pgfqpoint{0.867904in}{-7.856646in}}{\pgfqpoint{0.857305in}{-7.852255in}}{\pgfqpoint{0.846255in}{-7.852255in}}%
\pgfpathcurveto{\pgfqpoint{0.835205in}{-7.852255in}}{\pgfqpoint{0.824606in}{-7.856646in}}{\pgfqpoint{0.816792in}{-7.864459in}}%
\pgfpathcurveto{\pgfqpoint{0.808979in}{-7.872273in}}{\pgfqpoint{0.804589in}{-7.882872in}}{\pgfqpoint{0.804589in}{-7.893922in}}%
\pgfpathcurveto{\pgfqpoint{0.804589in}{-7.904972in}}{\pgfqpoint{0.808979in}{-7.915571in}}{\pgfqpoint{0.816792in}{-7.923385in}}%
\pgfpathcurveto{\pgfqpoint{0.824606in}{-7.931198in}}{\pgfqpoint{0.835205in}{-7.935589in}}{\pgfqpoint{0.846255in}{-7.935589in}}%
\pgfpathclose%
\pgfusepath{stroke}%
\end{pgfscope}%
\begin{pgfscope}%
\pgfpathrectangle{\pgfqpoint{0.383193in}{0.383578in}}{\pgfqpoint{2.325000in}{2.310000in}}%
\pgfusepath{clip}%
\pgfsetbuttcap%
\pgfsetroundjoin%
\pgfsetlinewidth{1.405250pt}%
\definecolor{currentstroke}{rgb}{0.686275,0.352941,0.313725}%
\pgfsetstrokecolor{currentstroke}%
\pgfsetdash{}{0pt}%
\pgfpathmoveto{\pgfqpoint{6.183099in}{-7.743089in}}%
\pgfpathcurveto{\pgfqpoint{6.194149in}{-7.743089in}}{\pgfqpoint{6.204748in}{-7.738698in}}{\pgfqpoint{6.212562in}{-7.730885in}}%
\pgfpathcurveto{\pgfqpoint{6.220375in}{-7.723071in}}{\pgfqpoint{6.224766in}{-7.712472in}}{\pgfqpoint{6.224766in}{-7.701422in}}%
\pgfpathcurveto{\pgfqpoint{6.224766in}{-7.690372in}}{\pgfqpoint{6.220375in}{-7.679773in}}{\pgfqpoint{6.212562in}{-7.671959in}}%
\pgfpathcurveto{\pgfqpoint{6.204748in}{-7.664146in}}{\pgfqpoint{6.194149in}{-7.659755in}}{\pgfqpoint{6.183099in}{-7.659755in}}%
\pgfpathcurveto{\pgfqpoint{6.172049in}{-7.659755in}}{\pgfqpoint{6.161450in}{-7.664146in}}{\pgfqpoint{6.153636in}{-7.671959in}}%
\pgfpathcurveto{\pgfqpoint{6.145823in}{-7.679773in}}{\pgfqpoint{6.141432in}{-7.690372in}}{\pgfqpoint{6.141432in}{-7.701422in}}%
\pgfpathcurveto{\pgfqpoint{6.141432in}{-7.712472in}}{\pgfqpoint{6.145823in}{-7.723071in}}{\pgfqpoint{6.153636in}{-7.730885in}}%
\pgfpathcurveto{\pgfqpoint{6.161450in}{-7.738698in}}{\pgfqpoint{6.172049in}{-7.743089in}}{\pgfqpoint{6.183099in}{-7.743089in}}%
\pgfpathclose%
\pgfusepath{stroke}%
\end{pgfscope}%
\begin{pgfscope}%
\pgfpathrectangle{\pgfqpoint{0.383193in}{0.383578in}}{\pgfqpoint{2.325000in}{2.310000in}}%
\pgfusepath{clip}%
\pgfsetbuttcap%
\pgfsetroundjoin%
\pgfsetlinewidth{1.405250pt}%
\definecolor{currentstroke}{rgb}{0.686275,0.352941,0.313725}%
\pgfsetstrokecolor{currentstroke}%
\pgfsetdash{}{0pt}%
\pgfpathmoveto{\pgfqpoint{-2.006713in}{-7.358089in}}%
\pgfpathcurveto{\pgfqpoint{-1.995663in}{-7.358089in}}{\pgfqpoint{-1.985064in}{-7.353698in}}{\pgfqpoint{-1.977251in}{-7.345885in}}%
\pgfpathcurveto{\pgfqpoint{-1.969437in}{-7.338071in}}{\pgfqpoint{-1.965047in}{-7.327472in}}{\pgfqpoint{-1.965047in}{-7.316422in}}%
\pgfpathcurveto{\pgfqpoint{-1.965047in}{-7.305372in}}{\pgfqpoint{-1.969437in}{-7.294773in}}{\pgfqpoint{-1.977251in}{-7.286959in}}%
\pgfpathcurveto{\pgfqpoint{-1.985064in}{-7.279146in}}{\pgfqpoint{-1.995663in}{-7.274755in}}{\pgfqpoint{-2.006713in}{-7.274755in}}%
\pgfpathcurveto{\pgfqpoint{-2.017764in}{-7.274755in}}{\pgfqpoint{-2.028363in}{-7.279146in}}{\pgfqpoint{-2.036176in}{-7.286959in}}%
\pgfpathcurveto{\pgfqpoint{-2.043990in}{-7.294773in}}{\pgfqpoint{-2.048380in}{-7.305372in}}{\pgfqpoint{-2.048380in}{-7.316422in}}%
\pgfpathcurveto{\pgfqpoint{-2.048380in}{-7.327472in}}{\pgfqpoint{-2.043990in}{-7.338071in}}{\pgfqpoint{-2.036176in}{-7.345885in}}%
\pgfpathcurveto{\pgfqpoint{-2.028363in}{-7.353698in}}{\pgfqpoint{-2.017764in}{-7.358089in}}{\pgfqpoint{-2.006713in}{-7.358089in}}%
\pgfpathclose%
\pgfusepath{stroke}%
\end{pgfscope}%
\begin{pgfscope}%
\pgfpathrectangle{\pgfqpoint{0.383193in}{0.383578in}}{\pgfqpoint{2.325000in}{2.310000in}}%
\pgfusepath{clip}%
\pgfsetbuttcap%
\pgfsetroundjoin%
\pgfsetlinewidth{1.405250pt}%
\definecolor{currentstroke}{rgb}{0.686275,0.352941,0.313725}%
\pgfsetstrokecolor{currentstroke}%
\pgfsetdash{}{0pt}%
\pgfpathmoveto{\pgfqpoint{-1.430307in}{-7.165589in}}%
\pgfpathcurveto{\pgfqpoint{-1.419257in}{-7.165589in}}{\pgfqpoint{-1.408658in}{-7.161198in}}{\pgfqpoint{-1.400844in}{-7.153385in}}%
\pgfpathcurveto{\pgfqpoint{-1.393031in}{-7.145571in}}{\pgfqpoint{-1.388641in}{-7.134972in}}{\pgfqpoint{-1.388641in}{-7.123922in}}%
\pgfpathcurveto{\pgfqpoint{-1.388641in}{-7.112872in}}{\pgfqpoint{-1.393031in}{-7.102273in}}{\pgfqpoint{-1.400844in}{-7.094459in}}%
\pgfpathcurveto{\pgfqpoint{-1.408658in}{-7.086646in}}{\pgfqpoint{-1.419257in}{-7.082255in}}{\pgfqpoint{-1.430307in}{-7.082255in}}%
\pgfpathcurveto{\pgfqpoint{-1.441357in}{-7.082255in}}{\pgfqpoint{-1.451956in}{-7.086646in}}{\pgfqpoint{-1.459770in}{-7.094459in}}%
\pgfpathcurveto{\pgfqpoint{-1.467584in}{-7.102273in}}{\pgfqpoint{-1.471974in}{-7.112872in}}{\pgfqpoint{-1.471974in}{-7.123922in}}%
\pgfpathcurveto{\pgfqpoint{-1.471974in}{-7.134972in}}{\pgfqpoint{-1.467584in}{-7.145571in}}{\pgfqpoint{-1.459770in}{-7.153385in}}%
\pgfpathcurveto{\pgfqpoint{-1.451956in}{-7.161198in}}{\pgfqpoint{-1.441357in}{-7.165589in}}{\pgfqpoint{-1.430307in}{-7.165589in}}%
\pgfpathclose%
\pgfusepath{stroke}%
\end{pgfscope}%
\begin{pgfscope}%
\pgfpathrectangle{\pgfqpoint{0.383193in}{0.383578in}}{\pgfqpoint{2.325000in}{2.310000in}}%
\pgfusepath{clip}%
\pgfsetbuttcap%
\pgfsetroundjoin%
\pgfsetlinewidth{1.405250pt}%
\definecolor{currentstroke}{rgb}{0.686275,0.352941,0.313725}%
\pgfsetstrokecolor{currentstroke}%
\pgfsetdash{}{0pt}%
\pgfpathmoveto{\pgfqpoint{3.967568in}{-6.588089in}}%
\pgfpathcurveto{\pgfqpoint{3.978618in}{-6.588089in}}{\pgfqpoint{3.989217in}{-6.583698in}}{\pgfqpoint{3.997031in}{-6.575885in}}%
\pgfpathcurveto{\pgfqpoint{4.004844in}{-6.568071in}}{\pgfqpoint{4.009234in}{-6.557472in}}{\pgfqpoint{4.009234in}{-6.546422in}}%
\pgfpathcurveto{\pgfqpoint{4.009234in}{-6.535372in}}{\pgfqpoint{4.004844in}{-6.524773in}}{\pgfqpoint{3.997031in}{-6.516959in}}%
\pgfpathcurveto{\pgfqpoint{3.989217in}{-6.509146in}}{\pgfqpoint{3.978618in}{-6.504755in}}{\pgfqpoint{3.967568in}{-6.504755in}}%
\pgfpathcurveto{\pgfqpoint{3.956518in}{-6.504755in}}{\pgfqpoint{3.945919in}{-6.509146in}}{\pgfqpoint{3.938105in}{-6.516959in}}%
\pgfpathcurveto{\pgfqpoint{3.930291in}{-6.524773in}}{\pgfqpoint{3.925901in}{-6.535372in}}{\pgfqpoint{3.925901in}{-6.546422in}}%
\pgfpathcurveto{\pgfqpoint{3.925901in}{-6.557472in}}{\pgfqpoint{3.930291in}{-6.568071in}}{\pgfqpoint{3.938105in}{-6.575885in}}%
\pgfpathcurveto{\pgfqpoint{3.945919in}{-6.583698in}}{\pgfqpoint{3.956518in}{-6.588089in}}{\pgfqpoint{3.967568in}{-6.588089in}}%
\pgfpathclose%
\pgfusepath{stroke}%
\end{pgfscope}%
\begin{pgfscope}%
\pgfpathrectangle{\pgfqpoint{0.383193in}{0.383578in}}{\pgfqpoint{2.325000in}{2.310000in}}%
\pgfusepath{clip}%
\pgfsetbuttcap%
\pgfsetroundjoin%
\pgfsetlinewidth{1.405250pt}%
\definecolor{currentstroke}{rgb}{0.686275,0.352941,0.313725}%
\pgfsetstrokecolor{currentstroke}%
\pgfsetdash{}{0pt}%
\pgfpathmoveto{\pgfqpoint{0.274693in}{-6.395589in}}%
\pgfpathcurveto{\pgfqpoint{0.285743in}{-6.395589in}}{\pgfqpoint{0.296342in}{-6.391198in}}{\pgfqpoint{0.304156in}{-6.383385in}}%
\pgfpathcurveto{\pgfqpoint{0.311969in}{-6.375571in}}{\pgfqpoint{0.316359in}{-6.364972in}}{\pgfqpoint{0.316359in}{-6.353922in}}%
\pgfpathcurveto{\pgfqpoint{0.316359in}{-6.342872in}}{\pgfqpoint{0.311969in}{-6.332273in}}{\pgfqpoint{0.304156in}{-6.324459in}}%
\pgfpathcurveto{\pgfqpoint{0.296342in}{-6.316646in}}{\pgfqpoint{0.285743in}{-6.312255in}}{\pgfqpoint{0.274693in}{-6.312255in}}%
\pgfpathcurveto{\pgfqpoint{0.263643in}{-6.312255in}}{\pgfqpoint{0.253044in}{-6.316646in}}{\pgfqpoint{0.245230in}{-6.324459in}}%
\pgfpathcurveto{\pgfqpoint{0.237416in}{-6.332273in}}{\pgfqpoint{0.233026in}{-6.342872in}}{\pgfqpoint{0.233026in}{-6.353922in}}%
\pgfpathcurveto{\pgfqpoint{0.233026in}{-6.364972in}}{\pgfqpoint{0.237416in}{-6.375571in}}{\pgfqpoint{0.245230in}{-6.383385in}}%
\pgfpathcurveto{\pgfqpoint{0.253044in}{-6.391198in}}{\pgfqpoint{0.263643in}{-6.395589in}}{\pgfqpoint{0.274693in}{-6.395589in}}%
\pgfpathclose%
\pgfusepath{stroke}%
\end{pgfscope}%
\begin{pgfscope}%
\pgfpathrectangle{\pgfqpoint{0.383193in}{0.383578in}}{\pgfqpoint{2.325000in}{2.310000in}}%
\pgfusepath{clip}%
\pgfsetbuttcap%
\pgfsetroundjoin%
\pgfsetlinewidth{1.405250pt}%
\definecolor{currentstroke}{rgb}{0.686275,0.352941,0.313725}%
\pgfsetstrokecolor{currentstroke}%
\pgfsetdash{}{0pt}%
\pgfpathmoveto{\pgfqpoint{0.443255in}{-6.010589in}}%
\pgfpathcurveto{\pgfqpoint{0.454305in}{-6.010589in}}{\pgfqpoint{0.464904in}{-6.006198in}}{\pgfqpoint{0.472718in}{-5.998385in}}%
\pgfpathcurveto{\pgfqpoint{0.480532in}{-5.990571in}}{\pgfqpoint{0.484922in}{-5.979972in}}{\pgfqpoint{0.484922in}{-5.968922in}}%
\pgfpathcurveto{\pgfqpoint{0.484922in}{-5.957872in}}{\pgfqpoint{0.480532in}{-5.947273in}}{\pgfqpoint{0.472718in}{-5.939459in}}%
\pgfpathcurveto{\pgfqpoint{0.464904in}{-5.931646in}}{\pgfqpoint{0.454305in}{-5.927255in}}{\pgfqpoint{0.443255in}{-5.927255in}}%
\pgfpathcurveto{\pgfqpoint{0.432205in}{-5.927255in}}{\pgfqpoint{0.421606in}{-5.931646in}}{\pgfqpoint{0.413792in}{-5.939459in}}%
\pgfpathcurveto{\pgfqpoint{0.405979in}{-5.947273in}}{\pgfqpoint{0.401589in}{-5.957872in}}{\pgfqpoint{0.401589in}{-5.968922in}}%
\pgfpathcurveto{\pgfqpoint{0.401589in}{-5.979972in}}{\pgfqpoint{0.405979in}{-5.990571in}}{\pgfqpoint{0.413792in}{-5.998385in}}%
\pgfpathcurveto{\pgfqpoint{0.421606in}{-6.006198in}}{\pgfqpoint{0.432205in}{-6.010589in}}{\pgfqpoint{0.443255in}{-6.010589in}}%
\pgfpathclose%
\pgfusepath{stroke}%
\end{pgfscope}%
\begin{pgfscope}%
\pgfpathrectangle{\pgfqpoint{0.383193in}{0.383578in}}{\pgfqpoint{2.325000in}{2.310000in}}%
\pgfusepath{clip}%
\pgfsetbuttcap%
\pgfsetroundjoin%
\pgfsetlinewidth{1.405250pt}%
\definecolor{currentstroke}{rgb}{0.686275,0.352941,0.313725}%
\pgfsetstrokecolor{currentstroke}%
\pgfsetdash{}{0pt}%
\pgfpathmoveto{\pgfqpoint{5.633818in}{-5.240589in}}%
\pgfpathcurveto{\pgfqpoint{5.644868in}{-5.240589in}}{\pgfqpoint{5.655467in}{-5.236198in}}{\pgfqpoint{5.663281in}{-5.228385in}}%
\pgfpathcurveto{\pgfqpoint{5.671094in}{-5.220571in}}{\pgfqpoint{5.675484in}{-5.209972in}}{\pgfqpoint{5.675484in}{-5.198922in}}%
\pgfpathcurveto{\pgfqpoint{5.675484in}{-5.187872in}}{\pgfqpoint{5.671094in}{-5.177273in}}{\pgfqpoint{5.663281in}{-5.169459in}}%
\pgfpathcurveto{\pgfqpoint{5.655467in}{-5.161646in}}{\pgfqpoint{5.644868in}{-5.157255in}}{\pgfqpoint{5.633818in}{-5.157255in}}%
\pgfpathcurveto{\pgfqpoint{5.622768in}{-5.157255in}}{\pgfqpoint{5.612169in}{-5.161646in}}{\pgfqpoint{5.604355in}{-5.169459in}}%
\pgfpathcurveto{\pgfqpoint{5.596541in}{-5.177273in}}{\pgfqpoint{5.592151in}{-5.187872in}}{\pgfqpoint{5.592151in}{-5.198922in}}%
\pgfpathcurveto{\pgfqpoint{5.592151in}{-5.209972in}}{\pgfqpoint{5.596541in}{-5.220571in}}{\pgfqpoint{5.604355in}{-5.228385in}}%
\pgfpathcurveto{\pgfqpoint{5.612169in}{-5.236198in}}{\pgfqpoint{5.622768in}{-5.240589in}}{\pgfqpoint{5.633818in}{-5.240589in}}%
\pgfpathclose%
\pgfusepath{stroke}%
\end{pgfscope}%
\begin{pgfscope}%
\pgfpathrectangle{\pgfqpoint{0.383193in}{0.383578in}}{\pgfqpoint{2.325000in}{2.310000in}}%
\pgfusepath{clip}%
\pgfsetbuttcap%
\pgfsetroundjoin%
\pgfsetlinewidth{1.405250pt}%
\definecolor{currentstroke}{rgb}{0.686275,0.352941,0.313725}%
\pgfsetstrokecolor{currentstroke}%
\pgfsetdash{}{0pt}%
\pgfpathmoveto{\pgfqpoint{-0.031432in}{-4.855589in}}%
\pgfpathcurveto{\pgfqpoint{-0.020382in}{-4.855589in}}{\pgfqpoint{-0.009783in}{-4.851198in}}{\pgfqpoint{-0.001969in}{-4.843385in}}%
\pgfpathcurveto{\pgfqpoint{0.005844in}{-4.835571in}}{\pgfqpoint{0.010234in}{-4.824972in}}{\pgfqpoint{0.010234in}{-4.813922in}}%
\pgfpathcurveto{\pgfqpoint{0.010234in}{-4.802872in}}{\pgfqpoint{0.005844in}{-4.792273in}}{\pgfqpoint{-0.001969in}{-4.784459in}}%
\pgfpathcurveto{\pgfqpoint{-0.009783in}{-4.776646in}}{\pgfqpoint{-0.020382in}{-4.772255in}}{\pgfqpoint{-0.031432in}{-4.772255in}}%
\pgfpathcurveto{\pgfqpoint{-0.042482in}{-4.772255in}}{\pgfqpoint{-0.053081in}{-4.776646in}}{\pgfqpoint{-0.060895in}{-4.784459in}}%
\pgfpathcurveto{\pgfqpoint{-0.068709in}{-4.792273in}}{\pgfqpoint{-0.073099in}{-4.802872in}}{\pgfqpoint{-0.073099in}{-4.813922in}}%
\pgfpathcurveto{\pgfqpoint{-0.073099in}{-4.824972in}}{\pgfqpoint{-0.068709in}{-4.835571in}}{\pgfqpoint{-0.060895in}{-4.843385in}}%
\pgfpathcurveto{\pgfqpoint{-0.053081in}{-4.851198in}}{\pgfqpoint{-0.042482in}{-4.855589in}}{\pgfqpoint{-0.031432in}{-4.855589in}}%
\pgfpathclose%
\pgfusepath{stroke}%
\end{pgfscope}%
\begin{pgfscope}%
\pgfpathrectangle{\pgfqpoint{0.383193in}{0.383578in}}{\pgfqpoint{2.325000in}{2.310000in}}%
\pgfusepath{clip}%
\pgfsetbuttcap%
\pgfsetroundjoin%
\pgfsetlinewidth{1.405250pt}%
\definecolor{currentstroke}{rgb}{0.686275,0.352941,0.313725}%
\pgfsetstrokecolor{currentstroke}%
\pgfsetdash{}{0pt}%
\pgfpathmoveto{\pgfqpoint{-0.682432in}{-4.085589in}}%
\pgfpathcurveto{\pgfqpoint{-0.671382in}{-4.085589in}}{\pgfqpoint{-0.660783in}{-4.081198in}}{\pgfqpoint{-0.652969in}{-4.073385in}}%
\pgfpathcurveto{\pgfqpoint{-0.645156in}{-4.065571in}}{\pgfqpoint{-0.640766in}{-4.054972in}}{\pgfqpoint{-0.640766in}{-4.043922in}}%
\pgfpathcurveto{\pgfqpoint{-0.640766in}{-4.032872in}}{\pgfqpoint{-0.645156in}{-4.022273in}}{\pgfqpoint{-0.652969in}{-4.014459in}}%
\pgfpathcurveto{\pgfqpoint{-0.660783in}{-4.006646in}}{\pgfqpoint{-0.671382in}{-4.002255in}}{\pgfqpoint{-0.682432in}{-4.002255in}}%
\pgfpathcurveto{\pgfqpoint{-0.693482in}{-4.002255in}}{\pgfqpoint{-0.704081in}{-4.006646in}}{\pgfqpoint{-0.711895in}{-4.014459in}}%
\pgfpathcurveto{\pgfqpoint{-0.719709in}{-4.022273in}}{\pgfqpoint{-0.724099in}{-4.032872in}}{\pgfqpoint{-0.724099in}{-4.043922in}}%
\pgfpathcurveto{\pgfqpoint{-0.724099in}{-4.054972in}}{\pgfqpoint{-0.719709in}{-4.065571in}}{\pgfqpoint{-0.711895in}{-4.073385in}}%
\pgfpathcurveto{\pgfqpoint{-0.704081in}{-4.081198in}}{\pgfqpoint{-0.693482in}{-4.085589in}}{\pgfqpoint{-0.682432in}{-4.085589in}}%
\pgfpathclose%
\pgfusepath{stroke}%
\end{pgfscope}%
\begin{pgfscope}%
\pgfpathrectangle{\pgfqpoint{0.383193in}{0.383578in}}{\pgfqpoint{2.325000in}{2.310000in}}%
\pgfusepath{clip}%
\pgfsetbuttcap%
\pgfsetroundjoin%
\pgfsetlinewidth{1.405250pt}%
\definecolor{currentstroke}{rgb}{0.686275,0.352941,0.313725}%
\pgfsetstrokecolor{currentstroke}%
\pgfsetdash{}{0pt}%
\pgfpathmoveto{\pgfqpoint{-2.666432in}{-3.893089in}}%
\pgfpathcurveto{\pgfqpoint{-2.655382in}{-3.893089in}}{\pgfqpoint{-2.644783in}{-3.888698in}}{\pgfqpoint{-2.636969in}{-3.880885in}}%
\pgfpathcurveto{\pgfqpoint{-2.629156in}{-3.873071in}}{\pgfqpoint{-2.624766in}{-3.862472in}}{\pgfqpoint{-2.624766in}{-3.851422in}}%
\pgfpathcurveto{\pgfqpoint{-2.624766in}{-3.840372in}}{\pgfqpoint{-2.629156in}{-3.829773in}}{\pgfqpoint{-2.636969in}{-3.821959in}}%
\pgfpathcurveto{\pgfqpoint{-2.644783in}{-3.814146in}}{\pgfqpoint{-2.655382in}{-3.809755in}}{\pgfqpoint{-2.666432in}{-3.809755in}}%
\pgfpathcurveto{\pgfqpoint{-2.677482in}{-3.809755in}}{\pgfqpoint{-2.688081in}{-3.814146in}}{\pgfqpoint{-2.695895in}{-3.821959in}}%
\pgfpathcurveto{\pgfqpoint{-2.703709in}{-3.829773in}}{\pgfqpoint{-2.708099in}{-3.840372in}}{\pgfqpoint{-2.708099in}{-3.851422in}}%
\pgfpathcurveto{\pgfqpoint{-2.708099in}{-3.862472in}}{\pgfqpoint{-2.703709in}{-3.873071in}}{\pgfqpoint{-2.695895in}{-3.880885in}}%
\pgfpathcurveto{\pgfqpoint{-2.688081in}{-3.888698in}}{\pgfqpoint{-2.677482in}{-3.893089in}}{\pgfqpoint{-2.666432in}{-3.893089in}}%
\pgfpathclose%
\pgfusepath{stroke}%
\end{pgfscope}%
\begin{pgfscope}%
\pgfpathrectangle{\pgfqpoint{0.383193in}{0.383578in}}{\pgfqpoint{2.325000in}{2.310000in}}%
\pgfusepath{clip}%
\pgfsetbuttcap%
\pgfsetroundjoin%
\pgfsetlinewidth{1.405250pt}%
\definecolor{currentstroke}{rgb}{0.686275,0.352941,0.313725}%
\pgfsetstrokecolor{currentstroke}%
\pgfsetdash{}{0pt}%
\pgfpathmoveto{\pgfqpoint{5.039005in}{-3.700589in}}%
\pgfpathcurveto{\pgfqpoint{5.050055in}{-3.700589in}}{\pgfqpoint{5.060654in}{-3.696198in}}{\pgfqpoint{5.068468in}{-3.688385in}}%
\pgfpathcurveto{\pgfqpoint{5.076282in}{-3.680571in}}{\pgfqpoint{5.080672in}{-3.669972in}}{\pgfqpoint{5.080672in}{-3.658922in}}%
\pgfpathcurveto{\pgfqpoint{5.080672in}{-3.647872in}}{\pgfqpoint{5.076282in}{-3.637273in}}{\pgfqpoint{5.068468in}{-3.629459in}}%
\pgfpathcurveto{\pgfqpoint{5.060654in}{-3.621646in}}{\pgfqpoint{5.050055in}{-3.617255in}}{\pgfqpoint{5.039005in}{-3.617255in}}%
\pgfpathcurveto{\pgfqpoint{5.027955in}{-3.617255in}}{\pgfqpoint{5.017356in}{-3.621646in}}{\pgfqpoint{5.009542in}{-3.629459in}}%
\pgfpathcurveto{\pgfqpoint{5.001729in}{-3.637273in}}{\pgfqpoint{4.997339in}{-3.647872in}}{\pgfqpoint{4.997339in}{-3.658922in}}%
\pgfpathcurveto{\pgfqpoint{4.997339in}{-3.669972in}}{\pgfqpoint{5.001729in}{-3.680571in}}{\pgfqpoint{5.009542in}{-3.688385in}}%
\pgfpathcurveto{\pgfqpoint{5.017356in}{-3.696198in}}{\pgfqpoint{5.027955in}{-3.700589in}}{\pgfqpoint{5.039005in}{-3.700589in}}%
\pgfpathclose%
\pgfusepath{stroke}%
\end{pgfscope}%
\begin{pgfscope}%
\pgfpathrectangle{\pgfqpoint{0.383193in}{0.383578in}}{\pgfqpoint{2.325000in}{2.310000in}}%
\pgfusepath{clip}%
\pgfsetbuttcap%
\pgfsetroundjoin%
\pgfsetlinewidth{1.405250pt}%
\definecolor{currentstroke}{rgb}{0.686275,0.352941,0.313725}%
\pgfsetstrokecolor{currentstroke}%
\pgfsetdash{}{0pt}%
\pgfpathmoveto{\pgfqpoint{-0.427651in}{-2.738089in}}%
\pgfpathcurveto{\pgfqpoint{-0.416601in}{-2.738089in}}{\pgfqpoint{-0.406002in}{-2.733698in}}{\pgfqpoint{-0.398188in}{-2.725885in}}%
\pgfpathcurveto{\pgfqpoint{-0.390375in}{-2.718071in}}{\pgfqpoint{-0.385984in}{-2.707472in}}{\pgfqpoint{-0.385984in}{-2.696422in}}%
\pgfpathcurveto{\pgfqpoint{-0.385984in}{-2.685372in}}{\pgfqpoint{-0.390375in}{-2.674773in}}{\pgfqpoint{-0.398188in}{-2.666959in}}%
\pgfpathcurveto{\pgfqpoint{-0.406002in}{-2.659146in}}{\pgfqpoint{-0.416601in}{-2.654755in}}{\pgfqpoint{-0.427651in}{-2.654755in}}%
\pgfpathcurveto{\pgfqpoint{-0.438701in}{-2.654755in}}{\pgfqpoint{-0.449300in}{-2.659146in}}{\pgfqpoint{-0.457114in}{-2.666959in}}%
\pgfpathcurveto{\pgfqpoint{-0.464927in}{-2.674773in}}{\pgfqpoint{-0.469318in}{-2.685372in}}{\pgfqpoint{-0.469318in}{-2.696422in}}%
\pgfpathcurveto{\pgfqpoint{-0.469318in}{-2.707472in}}{\pgfqpoint{-0.464927in}{-2.718071in}}{\pgfqpoint{-0.457114in}{-2.725885in}}%
\pgfpathcurveto{\pgfqpoint{-0.449300in}{-2.733698in}}{\pgfqpoint{-0.438701in}{-2.738089in}}{\pgfqpoint{-0.427651in}{-2.738089in}}%
\pgfpathclose%
\pgfusepath{stroke}%
\end{pgfscope}%
\begin{pgfscope}%
\pgfpathrectangle{\pgfqpoint{0.383193in}{0.383578in}}{\pgfqpoint{2.325000in}{2.310000in}}%
\pgfusepath{clip}%
\pgfsetbuttcap%
\pgfsetroundjoin%
\pgfsetlinewidth{1.405250pt}%
\definecolor{currentstroke}{rgb}{0.686275,0.352941,0.313725}%
\pgfsetstrokecolor{currentstroke}%
\pgfsetdash{}{0pt}%
\pgfpathmoveto{\pgfqpoint{-2.637370in}{-2.353089in}}%
\pgfpathcurveto{\pgfqpoint{-2.626320in}{-2.353089in}}{\pgfqpoint{-2.615721in}{-2.348698in}}{\pgfqpoint{-2.607907in}{-2.340885in}}%
\pgfpathcurveto{\pgfqpoint{-2.600093in}{-2.333071in}}{\pgfqpoint{-2.595703in}{-2.322472in}}{\pgfqpoint{-2.595703in}{-2.311422in}}%
\pgfpathcurveto{\pgfqpoint{-2.595703in}{-2.300372in}}{\pgfqpoint{-2.600093in}{-2.289773in}}{\pgfqpoint{-2.607907in}{-2.281959in}}%
\pgfpathcurveto{\pgfqpoint{-2.615721in}{-2.274146in}}{\pgfqpoint{-2.626320in}{-2.269755in}}{\pgfqpoint{-2.637370in}{-2.269755in}}%
\pgfpathcurveto{\pgfqpoint{-2.648420in}{-2.269755in}}{\pgfqpoint{-2.659019in}{-2.274146in}}{\pgfqpoint{-2.666833in}{-2.281959in}}%
\pgfpathcurveto{\pgfqpoint{-2.674646in}{-2.289773in}}{\pgfqpoint{-2.679036in}{-2.300372in}}{\pgfqpoint{-2.679036in}{-2.311422in}}%
\pgfpathcurveto{\pgfqpoint{-2.679036in}{-2.322472in}}{\pgfqpoint{-2.674646in}{-2.333071in}}{\pgfqpoint{-2.666833in}{-2.340885in}}%
\pgfpathcurveto{\pgfqpoint{-2.659019in}{-2.348698in}}{\pgfqpoint{-2.648420in}{-2.353089in}}{\pgfqpoint{-2.637370in}{-2.353089in}}%
\pgfpathclose%
\pgfusepath{stroke}%
\end{pgfscope}%
\begin{pgfscope}%
\pgfpathrectangle{\pgfqpoint{0.383193in}{0.383578in}}{\pgfqpoint{2.325000in}{2.310000in}}%
\pgfusepath{clip}%
\pgfsetbuttcap%
\pgfsetroundjoin%
\pgfsetlinewidth{1.405250pt}%
\definecolor{currentstroke}{rgb}{0.686275,0.352941,0.313725}%
\pgfsetstrokecolor{currentstroke}%
\pgfsetdash{}{0pt}%
\pgfpathmoveto{\pgfqpoint{-1.202651in}{-1.968089in}}%
\pgfpathcurveto{\pgfqpoint{-1.191601in}{-1.968089in}}{\pgfqpoint{-1.181002in}{-1.963698in}}{\pgfqpoint{-1.173188in}{-1.955885in}}%
\pgfpathcurveto{\pgfqpoint{-1.165375in}{-1.948071in}}{\pgfqpoint{-1.160984in}{-1.937472in}}{\pgfqpoint{-1.160984in}{-1.926422in}}%
\pgfpathcurveto{\pgfqpoint{-1.160984in}{-1.915372in}}{\pgfqpoint{-1.165375in}{-1.904773in}}{\pgfqpoint{-1.173188in}{-1.896959in}}%
\pgfpathcurveto{\pgfqpoint{-1.181002in}{-1.889146in}}{\pgfqpoint{-1.191601in}{-1.884755in}}{\pgfqpoint{-1.202651in}{-1.884755in}}%
\pgfpathcurveto{\pgfqpoint{-1.213701in}{-1.884755in}}{\pgfqpoint{-1.224300in}{-1.889146in}}{\pgfqpoint{-1.232114in}{-1.896959in}}%
\pgfpathcurveto{\pgfqpoint{-1.239927in}{-1.904773in}}{\pgfqpoint{-1.244318in}{-1.915372in}}{\pgfqpoint{-1.244318in}{-1.926422in}}%
\pgfpathcurveto{\pgfqpoint{-1.244318in}{-1.937472in}}{\pgfqpoint{-1.239927in}{-1.948071in}}{\pgfqpoint{-1.232114in}{-1.955885in}}%
\pgfpathcurveto{\pgfqpoint{-1.224300in}{-1.963698in}}{\pgfqpoint{-1.213701in}{-1.968089in}}{\pgfqpoint{-1.202651in}{-1.968089in}}%
\pgfpathclose%
\pgfusepath{stroke}%
\end{pgfscope}%
\begin{pgfscope}%
\pgfpathrectangle{\pgfqpoint{0.383193in}{0.383578in}}{\pgfqpoint{2.325000in}{2.310000in}}%
\pgfusepath{clip}%
\pgfsetbuttcap%
\pgfsetroundjoin%
\pgfsetlinewidth{1.405250pt}%
\definecolor{currentstroke}{rgb}{0.686275,0.352941,0.313725}%
\pgfsetstrokecolor{currentstroke}%
\pgfsetdash{}{0pt}%
\pgfpathmoveto{\pgfqpoint{4.335693in}{-0.620589in}}%
\pgfpathcurveto{\pgfqpoint{4.346743in}{-0.620589in}}{\pgfqpoint{4.357342in}{-0.616198in}}{\pgfqpoint{4.365156in}{-0.608385in}}%
\pgfpathcurveto{\pgfqpoint{4.372969in}{-0.600571in}}{\pgfqpoint{4.377359in}{-0.589972in}}{\pgfqpoint{4.377359in}{-0.578922in}}%
\pgfpathcurveto{\pgfqpoint{4.377359in}{-0.567872in}}{\pgfqpoint{4.372969in}{-0.557273in}}{\pgfqpoint{4.365156in}{-0.549459in}}%
\pgfpathcurveto{\pgfqpoint{4.357342in}{-0.541646in}}{\pgfqpoint{4.346743in}{-0.537255in}}{\pgfqpoint{4.335693in}{-0.537255in}}%
\pgfpathcurveto{\pgfqpoint{4.324643in}{-0.537255in}}{\pgfqpoint{4.314044in}{-0.541646in}}{\pgfqpoint{4.306230in}{-0.549459in}}%
\pgfpathcurveto{\pgfqpoint{4.298416in}{-0.557273in}}{\pgfqpoint{4.294026in}{-0.567872in}}{\pgfqpoint{4.294026in}{-0.578922in}}%
\pgfpathcurveto{\pgfqpoint{4.294026in}{-0.589972in}}{\pgfqpoint{4.298416in}{-0.600571in}}{\pgfqpoint{4.306230in}{-0.608385in}}%
\pgfpathcurveto{\pgfqpoint{4.314044in}{-0.616198in}}{\pgfqpoint{4.324643in}{-0.620589in}}{\pgfqpoint{4.335693in}{-0.620589in}}%
\pgfpathclose%
\pgfusepath{stroke}%
\end{pgfscope}%
\begin{pgfscope}%
\pgfpathrectangle{\pgfqpoint{0.383193in}{0.383578in}}{\pgfqpoint{2.325000in}{2.310000in}}%
\pgfusepath{clip}%
\pgfsetbuttcap%
\pgfsetroundjoin%
\pgfsetlinewidth{1.405250pt}%
\definecolor{currentstroke}{rgb}{0.686275,0.352941,0.313725}%
\pgfsetstrokecolor{currentstroke}%
\pgfsetdash{}{0pt}%
\pgfpathmoveto{\pgfqpoint{-0.466401in}{0.149411in}}%
\pgfpathcurveto{\pgfqpoint{-0.455351in}{0.149411in}}{\pgfqpoint{-0.444752in}{0.153802in}}{\pgfqpoint{-0.436938in}{0.161615in}}%
\pgfpathcurveto{\pgfqpoint{-0.429125in}{0.169429in}}{\pgfqpoint{-0.424734in}{0.180028in}}{\pgfqpoint{-0.424734in}{0.191078in}}%
\pgfpathcurveto{\pgfqpoint{-0.424734in}{0.202128in}}{\pgfqpoint{-0.429125in}{0.212727in}}{\pgfqpoint{-0.436938in}{0.220541in}}%
\pgfpathcurveto{\pgfqpoint{-0.444752in}{0.228354in}}{\pgfqpoint{-0.455351in}{0.232745in}}{\pgfqpoint{-0.466401in}{0.232745in}}%
\pgfpathcurveto{\pgfqpoint{-0.477451in}{0.232745in}}{\pgfqpoint{-0.488050in}{0.228354in}}{\pgfqpoint{-0.495864in}{0.220541in}}%
\pgfpathcurveto{\pgfqpoint{-0.503677in}{0.212727in}}{\pgfqpoint{-0.508068in}{0.202128in}}{\pgfqpoint{-0.508068in}{0.191078in}}%
\pgfpathcurveto{\pgfqpoint{-0.508068in}{0.180028in}}{\pgfqpoint{-0.503677in}{0.169429in}}{\pgfqpoint{-0.495864in}{0.161615in}}%
\pgfpathcurveto{\pgfqpoint{-0.488050in}{0.153802in}}{\pgfqpoint{-0.477451in}{0.149411in}}{\pgfqpoint{-0.466401in}{0.149411in}}%
\pgfpathclose%
\pgfusepath{stroke}%
\end{pgfscope}%
\begin{pgfscope}%
\pgfpathrectangle{\pgfqpoint{0.383193in}{0.383578in}}{\pgfqpoint{2.325000in}{2.310000in}}%
\pgfusepath{clip}%
\pgfsetbuttcap%
\pgfsetroundjoin%
\pgfsetlinewidth{1.405250pt}%
\definecolor{currentstroke}{rgb}{0.686275,0.352941,0.313725}%
\pgfsetstrokecolor{currentstroke}%
\pgfsetdash{}{0pt}%
\pgfpathmoveto{\pgfqpoint{-2.015432in}{0.341911in}}%
\pgfpathcurveto{\pgfqpoint{-2.004382in}{0.341911in}}{\pgfqpoint{-1.993783in}{0.346302in}}{\pgfqpoint{-1.985969in}{0.354115in}}%
\pgfpathcurveto{\pgfqpoint{-1.978156in}{0.361929in}}{\pgfqpoint{-1.973766in}{0.372528in}}{\pgfqpoint{-1.973766in}{0.383578in}}%
\pgfpathcurveto{\pgfqpoint{-1.973766in}{0.394628in}}{\pgfqpoint{-1.978156in}{0.405227in}}{\pgfqpoint{-1.985969in}{0.413041in}}%
\pgfpathcurveto{\pgfqpoint{-1.993783in}{0.420854in}}{\pgfqpoint{-2.004382in}{0.425245in}}{\pgfqpoint{-2.015432in}{0.425245in}}%
\pgfpathcurveto{\pgfqpoint{-2.026482in}{0.425245in}}{\pgfqpoint{-2.037081in}{0.420854in}}{\pgfqpoint{-2.044895in}{0.413041in}}%
\pgfpathcurveto{\pgfqpoint{-2.052709in}{0.405227in}}{\pgfqpoint{-2.057099in}{0.394628in}}{\pgfqpoint{-2.057099in}{0.383578in}}%
\pgfpathcurveto{\pgfqpoint{-2.057099in}{0.372528in}}{\pgfqpoint{-2.052709in}{0.361929in}}{\pgfqpoint{-2.044895in}{0.354115in}}%
\pgfpathcurveto{\pgfqpoint{-2.037081in}{0.346302in}}{\pgfqpoint{-2.026482in}{0.341911in}}{\pgfqpoint{-2.015432in}{0.341911in}}%
\pgfpathclose%
\pgfusepath{stroke}%
\end{pgfscope}%
\begin{pgfscope}%
\pgfpathrectangle{\pgfqpoint{0.383193in}{0.383578in}}{\pgfqpoint{2.325000in}{2.310000in}}%
\pgfusepath{clip}%
\pgfsetbuttcap%
\pgfsetroundjoin%
\pgfsetlinewidth{1.405250pt}%
\definecolor{currentstroke}{rgb}{0.686275,0.352941,0.313725}%
\pgfsetstrokecolor{currentstroke}%
\pgfsetdash{}{0pt}%
\pgfpathmoveto{\pgfqpoint{1.708443in}{0.726911in}}%
\pgfpathcurveto{\pgfqpoint{1.719493in}{0.726911in}}{\pgfqpoint{1.730092in}{0.731302in}}{\pgfqpoint{1.737906in}{0.739115in}}%
\pgfpathcurveto{\pgfqpoint{1.745719in}{0.746929in}}{\pgfqpoint{1.750109in}{0.757528in}}{\pgfqpoint{1.750109in}{0.768578in}}%
\pgfpathcurveto{\pgfqpoint{1.750109in}{0.779628in}}{\pgfqpoint{1.745719in}{0.790227in}}{\pgfqpoint{1.737906in}{0.798041in}}%
\pgfpathcurveto{\pgfqpoint{1.730092in}{0.805854in}}{\pgfqpoint{1.719493in}{0.810245in}}{\pgfqpoint{1.708443in}{0.810245in}}%
\pgfpathcurveto{\pgfqpoint{1.697393in}{0.810245in}}{\pgfqpoint{1.686794in}{0.805854in}}{\pgfqpoint{1.678980in}{0.798041in}}%
\pgfpathcurveto{\pgfqpoint{1.671166in}{0.790227in}}{\pgfqpoint{1.666776in}{0.779628in}}{\pgfqpoint{1.666776in}{0.768578in}}%
\pgfpathcurveto{\pgfqpoint{1.666776in}{0.757528in}}{\pgfqpoint{1.671166in}{0.746929in}}{\pgfqpoint{1.678980in}{0.739115in}}%
\pgfpathcurveto{\pgfqpoint{1.686794in}{0.731302in}}{\pgfqpoint{1.697393in}{0.726911in}}{\pgfqpoint{1.708443in}{0.726911in}}%
\pgfpathclose%
\pgfusepath{stroke}%
\end{pgfscope}%
\begin{pgfscope}%
\pgfpathrectangle{\pgfqpoint{0.383193in}{0.383578in}}{\pgfqpoint{2.325000in}{2.310000in}}%
\pgfusepath{clip}%
\pgfsetbuttcap%
\pgfsetroundjoin%
\pgfsetlinewidth{1.405250pt}%
\definecolor{currentstroke}{rgb}{0.686275,0.352941,0.313725}%
\pgfsetstrokecolor{currentstroke}%
\pgfsetdash{}{0pt}%
\pgfpathmoveto{\pgfqpoint{2.163755in}{0.919411in}}%
\pgfpathcurveto{\pgfqpoint{2.174805in}{0.919411in}}{\pgfqpoint{2.185404in}{0.923802in}}{\pgfqpoint{2.193218in}{0.931615in}}%
\pgfpathcurveto{\pgfqpoint{2.201032in}{0.939429in}}{\pgfqpoint{2.205422in}{0.950028in}}{\pgfqpoint{2.205422in}{0.961078in}}%
\pgfpathcurveto{\pgfqpoint{2.205422in}{0.972128in}}{\pgfqpoint{2.201032in}{0.982727in}}{\pgfqpoint{2.193218in}{0.990541in}}%
\pgfpathcurveto{\pgfqpoint{2.185404in}{0.998354in}}{\pgfqpoint{2.174805in}{1.002745in}}{\pgfqpoint{2.163755in}{1.002745in}}%
\pgfpathcurveto{\pgfqpoint{2.152705in}{1.002745in}}{\pgfqpoint{2.142106in}{0.998354in}}{\pgfqpoint{2.134292in}{0.990541in}}%
\pgfpathcurveto{\pgfqpoint{2.126479in}{0.982727in}}{\pgfqpoint{2.122089in}{0.972128in}}{\pgfqpoint{2.122089in}{0.961078in}}%
\pgfpathcurveto{\pgfqpoint{2.122089in}{0.950028in}}{\pgfqpoint{2.126479in}{0.939429in}}{\pgfqpoint{2.134292in}{0.931615in}}%
\pgfpathcurveto{\pgfqpoint{2.142106in}{0.923802in}}{\pgfqpoint{2.152705in}{0.919411in}}{\pgfqpoint{2.163755in}{0.919411in}}%
\pgfpathclose%
\pgfusepath{stroke}%
\end{pgfscope}%
\begin{pgfscope}%
\pgfpathrectangle{\pgfqpoint{0.383193in}{0.383578in}}{\pgfqpoint{2.325000in}{2.310000in}}%
\pgfusepath{clip}%
\pgfsetbuttcap%
\pgfsetroundjoin%
\pgfsetlinewidth{1.405250pt}%
\definecolor{currentstroke}{rgb}{0.686275,0.352941,0.313725}%
\pgfsetstrokecolor{currentstroke}%
\pgfsetdash{}{0pt}%
\pgfpathmoveto{\pgfqpoint{1.176599in}{1.496911in}}%
\pgfpathcurveto{\pgfqpoint{1.187649in}{1.496911in}}{\pgfqpoint{1.198248in}{1.501302in}}{\pgfqpoint{1.206062in}{1.509115in}}%
\pgfpathcurveto{\pgfqpoint{1.213875in}{1.516929in}}{\pgfqpoint{1.218266in}{1.527528in}}{\pgfqpoint{1.218266in}{1.538578in}}%
\pgfpathcurveto{\pgfqpoint{1.218266in}{1.549628in}}{\pgfqpoint{1.213875in}{1.560227in}}{\pgfqpoint{1.206062in}{1.568041in}}%
\pgfpathcurveto{\pgfqpoint{1.198248in}{1.575854in}}{\pgfqpoint{1.187649in}{1.580245in}}{\pgfqpoint{1.176599in}{1.580245in}}%
\pgfpathcurveto{\pgfqpoint{1.165549in}{1.580245in}}{\pgfqpoint{1.154950in}{1.575854in}}{\pgfqpoint{1.147136in}{1.568041in}}%
\pgfpathcurveto{\pgfqpoint{1.139323in}{1.560227in}}{\pgfqpoint{1.134932in}{1.549628in}}{\pgfqpoint{1.134932in}{1.538578in}}%
\pgfpathcurveto{\pgfqpoint{1.134932in}{1.527528in}}{\pgfqpoint{1.139323in}{1.516929in}}{\pgfqpoint{1.147136in}{1.509115in}}%
\pgfpathcurveto{\pgfqpoint{1.154950in}{1.501302in}}{\pgfqpoint{1.165549in}{1.496911in}}{\pgfqpoint{1.176599in}{1.496911in}}%
\pgfpathclose%
\pgfusepath{stroke}%
\end{pgfscope}%
\begin{pgfscope}%
\pgfpathrectangle{\pgfqpoint{0.383193in}{0.383578in}}{\pgfqpoint{2.325000in}{2.310000in}}%
\pgfusepath{clip}%
\pgfsetbuttcap%
\pgfsetroundjoin%
\pgfsetlinewidth{1.405250pt}%
\definecolor{currentstroke}{rgb}{0.686275,0.352941,0.313725}%
\pgfsetstrokecolor{currentstroke}%
\pgfsetdash{}{0pt}%
\pgfpathmoveto{\pgfqpoint{-1.236557in}{2.074411in}}%
\pgfpathcurveto{\pgfqpoint{-1.225507in}{2.074411in}}{\pgfqpoint{-1.214908in}{2.078802in}}{\pgfqpoint{-1.207094in}{2.086615in}}%
\pgfpathcurveto{\pgfqpoint{-1.199281in}{2.094429in}}{\pgfqpoint{-1.194891in}{2.105028in}}{\pgfqpoint{-1.194891in}{2.116078in}}%
\pgfpathcurveto{\pgfqpoint{-1.194891in}{2.127128in}}{\pgfqpoint{-1.199281in}{2.137727in}}{\pgfqpoint{-1.207094in}{2.145541in}}%
\pgfpathcurveto{\pgfqpoint{-1.214908in}{2.153354in}}{\pgfqpoint{-1.225507in}{2.157745in}}{\pgfqpoint{-1.236557in}{2.157745in}}%
\pgfpathcurveto{\pgfqpoint{-1.247607in}{2.157745in}}{\pgfqpoint{-1.258206in}{2.153354in}}{\pgfqpoint{-1.266020in}{2.145541in}}%
\pgfpathcurveto{\pgfqpoint{-1.273834in}{2.137727in}}{\pgfqpoint{-1.278224in}{2.127128in}}{\pgfqpoint{-1.278224in}{2.116078in}}%
\pgfpathcurveto{\pgfqpoint{-1.278224in}{2.105028in}}{\pgfqpoint{-1.273834in}{2.094429in}}{\pgfqpoint{-1.266020in}{2.086615in}}%
\pgfpathcurveto{\pgfqpoint{-1.258206in}{2.078802in}}{\pgfqpoint{-1.247607in}{2.074411in}}{\pgfqpoint{-1.236557in}{2.074411in}}%
\pgfpathclose%
\pgfusepath{stroke}%
\end{pgfscope}%
\begin{pgfscope}%
\pgfpathrectangle{\pgfqpoint{0.383193in}{0.383578in}}{\pgfqpoint{2.325000in}{2.310000in}}%
\pgfusepath{clip}%
\pgfsetbuttcap%
\pgfsetroundjoin%
\pgfsetlinewidth{1.405250pt}%
\definecolor{currentstroke}{rgb}{0.686275,0.352941,0.313725}%
\pgfsetstrokecolor{currentstroke}%
\pgfsetdash{}{0pt}%
\pgfpathmoveto{\pgfqpoint{1.324818in}{2.266911in}}%
\pgfpathcurveto{\pgfqpoint{1.335868in}{2.266911in}}{\pgfqpoint{1.346467in}{2.271302in}}{\pgfqpoint{1.354281in}{2.279115in}}%
\pgfpathcurveto{\pgfqpoint{1.362094in}{2.286929in}}{\pgfqpoint{1.366484in}{2.297528in}}{\pgfqpoint{1.366484in}{2.308578in}}%
\pgfpathcurveto{\pgfqpoint{1.366484in}{2.319628in}}{\pgfqpoint{1.362094in}{2.330227in}}{\pgfqpoint{1.354281in}{2.338041in}}%
\pgfpathcurveto{\pgfqpoint{1.346467in}{2.345854in}}{\pgfqpoint{1.335868in}{2.350245in}}{\pgfqpoint{1.324818in}{2.350245in}}%
\pgfpathcurveto{\pgfqpoint{1.313768in}{2.350245in}}{\pgfqpoint{1.303169in}{2.345854in}}{\pgfqpoint{1.295355in}{2.338041in}}%
\pgfpathcurveto{\pgfqpoint{1.287541in}{2.330227in}}{\pgfqpoint{1.283151in}{2.319628in}}{\pgfqpoint{1.283151in}{2.308578in}}%
\pgfpathcurveto{\pgfqpoint{1.283151in}{2.297528in}}{\pgfqpoint{1.287541in}{2.286929in}}{\pgfqpoint{1.295355in}{2.279115in}}%
\pgfpathcurveto{\pgfqpoint{1.303169in}{2.271302in}}{\pgfqpoint{1.313768in}{2.266911in}}{\pgfqpoint{1.324818in}{2.266911in}}%
\pgfpathclose%
\pgfusepath{stroke}%
\end{pgfscope}%
\begin{pgfscope}%
\pgfpathrectangle{\pgfqpoint{0.383193in}{0.383578in}}{\pgfqpoint{2.325000in}{2.310000in}}%
\pgfusepath{clip}%
\pgfsetbuttcap%
\pgfsetroundjoin%
\pgfsetlinewidth{1.405250pt}%
\definecolor{currentstroke}{rgb}{0.686275,0.352941,0.313725}%
\pgfsetstrokecolor{currentstroke}%
\pgfsetdash{}{0pt}%
\pgfpathmoveto{\pgfqpoint{-0.811276in}{2.844411in}}%
\pgfpathcurveto{\pgfqpoint{-0.800226in}{2.844411in}}{\pgfqpoint{-0.789627in}{2.848802in}}{\pgfqpoint{-0.781813in}{2.856615in}}%
\pgfpathcurveto{\pgfqpoint{-0.774000in}{2.864429in}}{\pgfqpoint{-0.769609in}{2.875028in}}{\pgfqpoint{-0.769609in}{2.886078in}}%
\pgfpathcurveto{\pgfqpoint{-0.769609in}{2.897128in}}{\pgfqpoint{-0.774000in}{2.907727in}}{\pgfqpoint{-0.781813in}{2.915541in}}%
\pgfpathcurveto{\pgfqpoint{-0.789627in}{2.923354in}}{\pgfqpoint{-0.800226in}{2.927745in}}{\pgfqpoint{-0.811276in}{2.927745in}}%
\pgfpathcurveto{\pgfqpoint{-0.822326in}{2.927745in}}{\pgfqpoint{-0.832925in}{2.923354in}}{\pgfqpoint{-0.840739in}{2.915541in}}%
\pgfpathcurveto{\pgfqpoint{-0.848552in}{2.907727in}}{\pgfqpoint{-0.852943in}{2.897128in}}{\pgfqpoint{-0.852943in}{2.886078in}}%
\pgfpathcurveto{\pgfqpoint{-0.852943in}{2.875028in}}{\pgfqpoint{-0.848552in}{2.864429in}}{\pgfqpoint{-0.840739in}{2.856615in}}%
\pgfpathcurveto{\pgfqpoint{-0.832925in}{2.848802in}}{\pgfqpoint{-0.822326in}{2.844411in}}{\pgfqpoint{-0.811276in}{2.844411in}}%
\pgfpathclose%
\pgfusepath{stroke}%
\end{pgfscope}%
\begin{pgfscope}%
\pgfpathrectangle{\pgfqpoint{0.383193in}{0.383578in}}{\pgfqpoint{2.325000in}{2.310000in}}%
\pgfusepath{clip}%
\pgfsetbuttcap%
\pgfsetroundjoin%
\pgfsetlinewidth{1.405250pt}%
\definecolor{currentstroke}{rgb}{0.686275,0.352941,0.313725}%
\pgfsetstrokecolor{currentstroke}%
\pgfsetdash{}{0pt}%
\pgfpathmoveto{\pgfqpoint{3.182880in}{3.036911in}}%
\pgfpathcurveto{\pgfqpoint{3.193930in}{3.036911in}}{\pgfqpoint{3.204529in}{3.041302in}}{\pgfqpoint{3.212343in}{3.049115in}}%
\pgfpathcurveto{\pgfqpoint{3.220157in}{3.056929in}}{\pgfqpoint{3.224547in}{3.067528in}}{\pgfqpoint{3.224547in}{3.078578in}}%
\pgfpathcurveto{\pgfqpoint{3.224547in}{3.089628in}}{\pgfqpoint{3.220157in}{3.100227in}}{\pgfqpoint{3.212343in}{3.108041in}}%
\pgfpathcurveto{\pgfqpoint{3.204529in}{3.115854in}}{\pgfqpoint{3.193930in}{3.120245in}}{\pgfqpoint{3.182880in}{3.120245in}}%
\pgfpathcurveto{\pgfqpoint{3.171830in}{3.120245in}}{\pgfqpoint{3.161231in}{3.115854in}}{\pgfqpoint{3.153417in}{3.108041in}}%
\pgfpathcurveto{\pgfqpoint{3.145604in}{3.100227in}}{\pgfqpoint{3.141214in}{3.089628in}}{\pgfqpoint{3.141214in}{3.078578in}}%
\pgfpathcurveto{\pgfqpoint{3.141214in}{3.067528in}}{\pgfqpoint{3.145604in}{3.056929in}}{\pgfqpoint{3.153417in}{3.049115in}}%
\pgfpathcurveto{\pgfqpoint{3.161231in}{3.041302in}}{\pgfqpoint{3.171830in}{3.036911in}}{\pgfqpoint{3.182880in}{3.036911in}}%
\pgfpathclose%
\pgfusepath{stroke}%
\end{pgfscope}%
\begin{pgfscope}%
\pgfpathrectangle{\pgfqpoint{0.383193in}{0.383578in}}{\pgfqpoint{2.325000in}{2.310000in}}%
\pgfusepath{clip}%
\pgfsetbuttcap%
\pgfsetroundjoin%
\pgfsetlinewidth{1.405250pt}%
\definecolor{currentstroke}{rgb}{0.686275,0.352941,0.313725}%
\pgfsetstrokecolor{currentstroke}%
\pgfsetdash{}{0pt}%
\pgfpathmoveto{\pgfqpoint{5.498193in}{3.614411in}}%
\pgfpathcurveto{\pgfqpoint{5.509243in}{3.614411in}}{\pgfqpoint{5.519842in}{3.618802in}}{\pgfqpoint{5.527656in}{3.626615in}}%
\pgfpathcurveto{\pgfqpoint{5.535469in}{3.634429in}}{\pgfqpoint{5.539859in}{3.645028in}}{\pgfqpoint{5.539859in}{3.656078in}}%
\pgfpathcurveto{\pgfqpoint{5.539859in}{3.667128in}}{\pgfqpoint{5.535469in}{3.677727in}}{\pgfqpoint{5.527656in}{3.685541in}}%
\pgfpathcurveto{\pgfqpoint{5.519842in}{3.693354in}}{\pgfqpoint{5.509243in}{3.697745in}}{\pgfqpoint{5.498193in}{3.697745in}}%
\pgfpathcurveto{\pgfqpoint{5.487143in}{3.697745in}}{\pgfqpoint{5.476544in}{3.693354in}}{\pgfqpoint{5.468730in}{3.685541in}}%
\pgfpathcurveto{\pgfqpoint{5.460916in}{3.677727in}}{\pgfqpoint{5.456526in}{3.667128in}}{\pgfqpoint{5.456526in}{3.656078in}}%
\pgfpathcurveto{\pgfqpoint{5.456526in}{3.645028in}}{\pgfqpoint{5.460916in}{3.634429in}}{\pgfqpoint{5.468730in}{3.626615in}}%
\pgfpathcurveto{\pgfqpoint{5.476544in}{3.618802in}}{\pgfqpoint{5.487143in}{3.614411in}}{\pgfqpoint{5.498193in}{3.614411in}}%
\pgfpathclose%
\pgfusepath{stroke}%
\end{pgfscope}%
\begin{pgfscope}%
\pgfpathrectangle{\pgfqpoint{0.383193in}{0.383578in}}{\pgfqpoint{2.325000in}{2.310000in}}%
\pgfusepath{clip}%
\pgfsetbuttcap%
\pgfsetroundjoin%
\pgfsetlinewidth{1.405250pt}%
\definecolor{currentstroke}{rgb}{0.686275,0.352941,0.313725}%
\pgfsetstrokecolor{currentstroke}%
\pgfsetdash{}{0pt}%
\pgfpathmoveto{\pgfqpoint{2.175380in}{4.191911in}}%
\pgfpathcurveto{\pgfqpoint{2.186430in}{4.191911in}}{\pgfqpoint{2.197029in}{4.196302in}}{\pgfqpoint{2.204843in}{4.204115in}}%
\pgfpathcurveto{\pgfqpoint{2.212657in}{4.211929in}}{\pgfqpoint{2.217047in}{4.222528in}}{\pgfqpoint{2.217047in}{4.233578in}}%
\pgfpathcurveto{\pgfqpoint{2.217047in}{4.244628in}}{\pgfqpoint{2.212657in}{4.255227in}}{\pgfqpoint{2.204843in}{4.263041in}}%
\pgfpathcurveto{\pgfqpoint{2.197029in}{4.270854in}}{\pgfqpoint{2.186430in}{4.275245in}}{\pgfqpoint{2.175380in}{4.275245in}}%
\pgfpathcurveto{\pgfqpoint{2.164330in}{4.275245in}}{\pgfqpoint{2.153731in}{4.270854in}}{\pgfqpoint{2.145917in}{4.263041in}}%
\pgfpathcurveto{\pgfqpoint{2.138104in}{4.255227in}}{\pgfqpoint{2.133714in}{4.244628in}}{\pgfqpoint{2.133714in}{4.233578in}}%
\pgfpathcurveto{\pgfqpoint{2.133714in}{4.222528in}}{\pgfqpoint{2.138104in}{4.211929in}}{\pgfqpoint{2.145917in}{4.204115in}}%
\pgfpathcurveto{\pgfqpoint{2.153731in}{4.196302in}}{\pgfqpoint{2.164330in}{4.191911in}}{\pgfqpoint{2.175380in}{4.191911in}}%
\pgfpathclose%
\pgfusepath{stroke}%
\end{pgfscope}%
\begin{pgfscope}%
\pgfpathrectangle{\pgfqpoint{0.383193in}{0.383578in}}{\pgfqpoint{2.325000in}{2.310000in}}%
\pgfusepath{clip}%
\pgfsetbuttcap%
\pgfsetroundjoin%
\pgfsetlinewidth{1.405250pt}%
\definecolor{currentstroke}{rgb}{0.686275,0.352941,0.313725}%
\pgfsetstrokecolor{currentstroke}%
\pgfsetdash{}{0pt}%
\pgfpathmoveto{\pgfqpoint{1.132037in}{4.384411in}}%
\pgfpathcurveto{\pgfqpoint{1.143087in}{4.384411in}}{\pgfqpoint{1.153686in}{4.388802in}}{\pgfqpoint{1.161499in}{4.396615in}}%
\pgfpathcurveto{\pgfqpoint{1.169313in}{4.404429in}}{\pgfqpoint{1.173703in}{4.415028in}}{\pgfqpoint{1.173703in}{4.426078in}}%
\pgfpathcurveto{\pgfqpoint{1.173703in}{4.437128in}}{\pgfqpoint{1.169313in}{4.447727in}}{\pgfqpoint{1.161499in}{4.455541in}}%
\pgfpathcurveto{\pgfqpoint{1.153686in}{4.463354in}}{\pgfqpoint{1.143087in}{4.467745in}}{\pgfqpoint{1.132037in}{4.467745in}}%
\pgfpathcurveto{\pgfqpoint{1.120986in}{4.467745in}}{\pgfqpoint{1.110387in}{4.463354in}}{\pgfqpoint{1.102574in}{4.455541in}}%
\pgfpathcurveto{\pgfqpoint{1.094760in}{4.447727in}}{\pgfqpoint{1.090370in}{4.437128in}}{\pgfqpoint{1.090370in}{4.426078in}}%
\pgfpathcurveto{\pgfqpoint{1.090370in}{4.415028in}}{\pgfqpoint{1.094760in}{4.404429in}}{\pgfqpoint{1.102574in}{4.396615in}}%
\pgfpathcurveto{\pgfqpoint{1.110387in}{4.388802in}}{\pgfqpoint{1.120986in}{4.384411in}}{\pgfqpoint{1.132037in}{4.384411in}}%
\pgfpathclose%
\pgfusepath{stroke}%
\end{pgfscope}%
\begin{pgfscope}%
\pgfpathrectangle{\pgfqpoint{0.383193in}{0.383578in}}{\pgfqpoint{2.325000in}{2.310000in}}%
\pgfusepath{clip}%
\pgfsetbuttcap%
\pgfsetroundjoin%
\pgfsetlinewidth{1.405250pt}%
\definecolor{currentstroke}{rgb}{0.686275,0.352941,0.313725}%
\pgfsetstrokecolor{currentstroke}%
\pgfsetdash{}{0pt}%
\pgfpathmoveto{\pgfqpoint{2.053318in}{4.769411in}}%
\pgfpathcurveto{\pgfqpoint{2.064368in}{4.769411in}}{\pgfqpoint{2.074967in}{4.773802in}}{\pgfqpoint{2.082781in}{4.781615in}}%
\pgfpathcurveto{\pgfqpoint{2.090594in}{4.789429in}}{\pgfqpoint{2.094984in}{4.800028in}}{\pgfqpoint{2.094984in}{4.811078in}}%
\pgfpathcurveto{\pgfqpoint{2.094984in}{4.822128in}}{\pgfqpoint{2.090594in}{4.832727in}}{\pgfqpoint{2.082781in}{4.840541in}}%
\pgfpathcurveto{\pgfqpoint{2.074967in}{4.848354in}}{\pgfqpoint{2.064368in}{4.852745in}}{\pgfqpoint{2.053318in}{4.852745in}}%
\pgfpathcurveto{\pgfqpoint{2.042268in}{4.852745in}}{\pgfqpoint{2.031669in}{4.848354in}}{\pgfqpoint{2.023855in}{4.840541in}}%
\pgfpathcurveto{\pgfqpoint{2.016041in}{4.832727in}}{\pgfqpoint{2.011651in}{4.822128in}}{\pgfqpoint{2.011651in}{4.811078in}}%
\pgfpathcurveto{\pgfqpoint{2.011651in}{4.800028in}}{\pgfqpoint{2.016041in}{4.789429in}}{\pgfqpoint{2.023855in}{4.781615in}}%
\pgfpathcurveto{\pgfqpoint{2.031669in}{4.773802in}}{\pgfqpoint{2.042268in}{4.769411in}}{\pgfqpoint{2.053318in}{4.769411in}}%
\pgfpathclose%
\pgfusepath{stroke}%
\end{pgfscope}%
\begin{pgfscope}%
\pgfpathrectangle{\pgfqpoint{0.383193in}{0.383578in}}{\pgfqpoint{2.325000in}{2.310000in}}%
\pgfusepath{clip}%
\pgfsetbuttcap%
\pgfsetroundjoin%
\pgfsetlinewidth{1.405250pt}%
\definecolor{currentstroke}{rgb}{0.686275,0.352941,0.313725}%
\pgfsetstrokecolor{currentstroke}%
\pgfsetdash{}{0pt}%
\pgfpathmoveto{\pgfqpoint{2.032005in}{4.961911in}}%
\pgfpathcurveto{\pgfqpoint{2.043055in}{4.961911in}}{\pgfqpoint{2.053654in}{4.966302in}}{\pgfqpoint{2.061468in}{4.974115in}}%
\pgfpathcurveto{\pgfqpoint{2.069282in}{4.981929in}}{\pgfqpoint{2.073672in}{4.992528in}}{\pgfqpoint{2.073672in}{5.003578in}}%
\pgfpathcurveto{\pgfqpoint{2.073672in}{5.014628in}}{\pgfqpoint{2.069282in}{5.025227in}}{\pgfqpoint{2.061468in}{5.033041in}}%
\pgfpathcurveto{\pgfqpoint{2.053654in}{5.040854in}}{\pgfqpoint{2.043055in}{5.045245in}}{\pgfqpoint{2.032005in}{5.045245in}}%
\pgfpathcurveto{\pgfqpoint{2.020955in}{5.045245in}}{\pgfqpoint{2.010356in}{5.040854in}}{\pgfqpoint{2.002542in}{5.033041in}}%
\pgfpathcurveto{\pgfqpoint{1.994729in}{5.025227in}}{\pgfqpoint{1.990339in}{5.014628in}}{\pgfqpoint{1.990339in}{5.003578in}}%
\pgfpathcurveto{\pgfqpoint{1.990339in}{4.992528in}}{\pgfqpoint{1.994729in}{4.981929in}}{\pgfqpoint{2.002542in}{4.974115in}}%
\pgfpathcurveto{\pgfqpoint{2.010356in}{4.966302in}}{\pgfqpoint{2.020955in}{4.961911in}}{\pgfqpoint{2.032005in}{4.961911in}}%
\pgfpathclose%
\pgfusepath{stroke}%
\end{pgfscope}%
\begin{pgfscope}%
\pgfpathrectangle{\pgfqpoint{0.383193in}{0.383578in}}{\pgfqpoint{2.325000in}{2.310000in}}%
\pgfusepath{clip}%
\pgfsetbuttcap%
\pgfsetroundjoin%
\pgfsetlinewidth{1.405250pt}%
\definecolor{currentstroke}{rgb}{0.686275,0.352941,0.313725}%
\pgfsetstrokecolor{currentstroke}%
\pgfsetdash{}{0pt}%
\pgfpathmoveto{\pgfqpoint{3.946255in}{5.154411in}}%
\pgfpathcurveto{\pgfqpoint{3.957305in}{5.154411in}}{\pgfqpoint{3.967904in}{5.158802in}}{\pgfqpoint{3.975718in}{5.166615in}}%
\pgfpathcurveto{\pgfqpoint{3.983532in}{5.174429in}}{\pgfqpoint{3.987922in}{5.185028in}}{\pgfqpoint{3.987922in}{5.196078in}}%
\pgfpathcurveto{\pgfqpoint{3.987922in}{5.207128in}}{\pgfqpoint{3.983532in}{5.217727in}}{\pgfqpoint{3.975718in}{5.225541in}}%
\pgfpathcurveto{\pgfqpoint{3.967904in}{5.233354in}}{\pgfqpoint{3.957305in}{5.237745in}}{\pgfqpoint{3.946255in}{5.237745in}}%
\pgfpathcurveto{\pgfqpoint{3.935205in}{5.237745in}}{\pgfqpoint{3.924606in}{5.233354in}}{\pgfqpoint{3.916792in}{5.225541in}}%
\pgfpathcurveto{\pgfqpoint{3.908979in}{5.217727in}}{\pgfqpoint{3.904589in}{5.207128in}}{\pgfqpoint{3.904589in}{5.196078in}}%
\pgfpathcurveto{\pgfqpoint{3.904589in}{5.185028in}}{\pgfqpoint{3.908979in}{5.174429in}}{\pgfqpoint{3.916792in}{5.166615in}}%
\pgfpathcurveto{\pgfqpoint{3.924606in}{5.158802in}}{\pgfqpoint{3.935205in}{5.154411in}}{\pgfqpoint{3.946255in}{5.154411in}}%
\pgfpathclose%
\pgfusepath{stroke}%
\end{pgfscope}%
\begin{pgfscope}%
\pgfpathrectangle{\pgfqpoint{0.383193in}{0.383578in}}{\pgfqpoint{2.325000in}{2.310000in}}%
\pgfusepath{clip}%
\pgfsetbuttcap%
\pgfsetroundjoin%
\pgfsetlinewidth{1.405250pt}%
\definecolor{currentstroke}{rgb}{0.686275,0.352941,0.313725}%
\pgfsetstrokecolor{currentstroke}%
\pgfsetdash{}{0pt}%
\pgfpathmoveto{\pgfqpoint{6.067818in}{5.539411in}}%
\pgfpathcurveto{\pgfqpoint{6.078868in}{5.539411in}}{\pgfqpoint{6.089467in}{5.543802in}}{\pgfqpoint{6.097281in}{5.551615in}}%
\pgfpathcurveto{\pgfqpoint{6.105094in}{5.559429in}}{\pgfqpoint{6.109484in}{5.570028in}}{\pgfqpoint{6.109484in}{5.581078in}}%
\pgfpathcurveto{\pgfqpoint{6.109484in}{5.592128in}}{\pgfqpoint{6.105094in}{5.602727in}}{\pgfqpoint{6.097281in}{5.610541in}}%
\pgfpathcurveto{\pgfqpoint{6.089467in}{5.618354in}}{\pgfqpoint{6.078868in}{5.622745in}}{\pgfqpoint{6.067818in}{5.622745in}}%
\pgfpathcurveto{\pgfqpoint{6.056768in}{5.622745in}}{\pgfqpoint{6.046169in}{5.618354in}}{\pgfqpoint{6.038355in}{5.610541in}}%
\pgfpathcurveto{\pgfqpoint{6.030541in}{5.602727in}}{\pgfqpoint{6.026151in}{5.592128in}}{\pgfqpoint{6.026151in}{5.581078in}}%
\pgfpathcurveto{\pgfqpoint{6.026151in}{5.570028in}}{\pgfqpoint{6.030541in}{5.559429in}}{\pgfqpoint{6.038355in}{5.551615in}}%
\pgfpathcurveto{\pgfqpoint{6.046169in}{5.543802in}}{\pgfqpoint{6.056768in}{5.539411in}}{\pgfqpoint{6.067818in}{5.539411in}}%
\pgfpathclose%
\pgfusepath{stroke}%
\end{pgfscope}%
\begin{pgfscope}%
\pgfpathrectangle{\pgfqpoint{0.383193in}{0.383578in}}{\pgfqpoint{2.325000in}{2.310000in}}%
\pgfusepath{clip}%
\pgfsetbuttcap%
\pgfsetroundjoin%
\pgfsetlinewidth{1.405250pt}%
\definecolor{currentstroke}{rgb}{0.686275,0.352941,0.313725}%
\pgfsetstrokecolor{currentstroke}%
\pgfsetdash{}{0pt}%
\pgfpathmoveto{\pgfqpoint{-1.450651in}{5.731911in}}%
\pgfpathcurveto{\pgfqpoint{-1.439601in}{5.731911in}}{\pgfqpoint{-1.429002in}{5.736302in}}{\pgfqpoint{-1.421188in}{5.744115in}}%
\pgfpathcurveto{\pgfqpoint{-1.413375in}{5.751929in}}{\pgfqpoint{-1.408984in}{5.762528in}}{\pgfqpoint{-1.408984in}{5.773578in}}%
\pgfpathcurveto{\pgfqpoint{-1.408984in}{5.784628in}}{\pgfqpoint{-1.413375in}{5.795227in}}{\pgfqpoint{-1.421188in}{5.803041in}}%
\pgfpathcurveto{\pgfqpoint{-1.429002in}{5.810854in}}{\pgfqpoint{-1.439601in}{5.815245in}}{\pgfqpoint{-1.450651in}{5.815245in}}%
\pgfpathcurveto{\pgfqpoint{-1.461701in}{5.815245in}}{\pgfqpoint{-1.472300in}{5.810854in}}{\pgfqpoint{-1.480114in}{5.803041in}}%
\pgfpathcurveto{\pgfqpoint{-1.487927in}{5.795227in}}{\pgfqpoint{-1.492318in}{5.784628in}}{\pgfqpoint{-1.492318in}{5.773578in}}%
\pgfpathcurveto{\pgfqpoint{-1.492318in}{5.762528in}}{\pgfqpoint{-1.487927in}{5.751929in}}{\pgfqpoint{-1.480114in}{5.744115in}}%
\pgfpathcurveto{\pgfqpoint{-1.472300in}{5.736302in}}{\pgfqpoint{-1.461701in}{5.731911in}}{\pgfqpoint{-1.450651in}{5.731911in}}%
\pgfpathclose%
\pgfusepath{stroke}%
\end{pgfscope}%
\begin{pgfscope}%
\pgfpathrectangle{\pgfqpoint{0.383193in}{0.383578in}}{\pgfqpoint{2.325000in}{2.310000in}}%
\pgfusepath{clip}%
\pgfsetbuttcap%
\pgfsetroundjoin%
\pgfsetlinewidth{1.405250pt}%
\definecolor{currentstroke}{rgb}{0.686275,0.352941,0.313725}%
\pgfsetstrokecolor{currentstroke}%
\pgfsetdash{}{0pt}%
\pgfpathmoveto{\pgfqpoint{0.430662in}{6.501911in}}%
\pgfpathcurveto{\pgfqpoint{0.441712in}{6.501911in}}{\pgfqpoint{0.452311in}{6.506302in}}{\pgfqpoint{0.460124in}{6.514115in}}%
\pgfpathcurveto{\pgfqpoint{0.467938in}{6.521929in}}{\pgfqpoint{0.472328in}{6.532528in}}{\pgfqpoint{0.472328in}{6.543578in}}%
\pgfpathcurveto{\pgfqpoint{0.472328in}{6.554628in}}{\pgfqpoint{0.467938in}{6.565227in}}{\pgfqpoint{0.460124in}{6.573041in}}%
\pgfpathcurveto{\pgfqpoint{0.452311in}{6.580854in}}{\pgfqpoint{0.441712in}{6.585245in}}{\pgfqpoint{0.430662in}{6.585245in}}%
\pgfpathcurveto{\pgfqpoint{0.419611in}{6.585245in}}{\pgfqpoint{0.409012in}{6.580854in}}{\pgfqpoint{0.401199in}{6.573041in}}%
\pgfpathcurveto{\pgfqpoint{0.393385in}{6.565227in}}{\pgfqpoint{0.388995in}{6.554628in}}{\pgfqpoint{0.388995in}{6.543578in}}%
\pgfpathcurveto{\pgfqpoint{0.388995in}{6.532528in}}{\pgfqpoint{0.393385in}{6.521929in}}{\pgfqpoint{0.401199in}{6.514115in}}%
\pgfpathcurveto{\pgfqpoint{0.409012in}{6.506302in}}{\pgfqpoint{0.419611in}{6.501911in}}{\pgfqpoint{0.430662in}{6.501911in}}%
\pgfpathclose%
\pgfusepath{stroke}%
\end{pgfscope}%
\begin{pgfscope}%
\pgfpathrectangle{\pgfqpoint{0.383193in}{0.383578in}}{\pgfqpoint{2.325000in}{2.310000in}}%
\pgfusepath{clip}%
\pgfsetbuttcap%
\pgfsetroundjoin%
\pgfsetlinewidth{1.405250pt}%
\definecolor{currentstroke}{rgb}{0.686275,0.352941,0.313725}%
\pgfsetstrokecolor{currentstroke}%
\pgfsetdash{}{0pt}%
\pgfpathmoveto{\pgfqpoint{-2.335120in}{6.694411in}}%
\pgfpathcurveto{\pgfqpoint{-2.324070in}{6.694411in}}{\pgfqpoint{-2.313471in}{6.698802in}}{\pgfqpoint{-2.305657in}{6.706615in}}%
\pgfpathcurveto{\pgfqpoint{-2.297843in}{6.714429in}}{\pgfqpoint{-2.293453in}{6.725028in}}{\pgfqpoint{-2.293453in}{6.736078in}}%
\pgfpathcurveto{\pgfqpoint{-2.293453in}{6.747128in}}{\pgfqpoint{-2.297843in}{6.757727in}}{\pgfqpoint{-2.305657in}{6.765541in}}%
\pgfpathcurveto{\pgfqpoint{-2.313471in}{6.773354in}}{\pgfqpoint{-2.324070in}{6.777745in}}{\pgfqpoint{-2.335120in}{6.777745in}}%
\pgfpathcurveto{\pgfqpoint{-2.346170in}{6.777745in}}{\pgfqpoint{-2.356769in}{6.773354in}}{\pgfqpoint{-2.364583in}{6.765541in}}%
\pgfpathcurveto{\pgfqpoint{-2.372396in}{6.757727in}}{\pgfqpoint{-2.376786in}{6.747128in}}{\pgfqpoint{-2.376786in}{6.736078in}}%
\pgfpathcurveto{\pgfqpoint{-2.376786in}{6.725028in}}{\pgfqpoint{-2.372396in}{6.714429in}}{\pgfqpoint{-2.364583in}{6.706615in}}%
\pgfpathcurveto{\pgfqpoint{-2.356769in}{6.698802in}}{\pgfqpoint{-2.346170in}{6.694411in}}{\pgfqpoint{-2.335120in}{6.694411in}}%
\pgfpathclose%
\pgfusepath{stroke}%
\end{pgfscope}%
\begin{pgfscope}%
\pgfpathrectangle{\pgfqpoint{0.383193in}{0.383578in}}{\pgfqpoint{2.325000in}{2.310000in}}%
\pgfusepath{clip}%
\pgfsetbuttcap%
\pgfsetroundjoin%
\pgfsetlinewidth{1.405250pt}%
\definecolor{currentstroke}{rgb}{0.000000,0.356863,0.509804}%
\pgfsetstrokecolor{currentstroke}%
\pgfsetdash{}{0pt}%
\pgfpathmoveto{\pgfqpoint{0.737755in}{-10.035728in}}%
\pgfpathcurveto{\pgfqpoint{0.744201in}{-10.035728in}}{\pgfqpoint{0.750384in}{-10.033167in}}{\pgfqpoint{0.754942in}{-10.028609in}}%
\pgfpathcurveto{\pgfqpoint{0.759500in}{-10.024051in}}{\pgfqpoint{0.762061in}{-10.017868in}}{\pgfqpoint{0.762061in}{-10.011422in}}%
\pgfpathcurveto{\pgfqpoint{0.762061in}{-10.004976in}}{\pgfqpoint{0.759500in}{-9.998793in}}{\pgfqpoint{0.754942in}{-9.994235in}}%
\pgfpathcurveto{\pgfqpoint{0.750384in}{-9.989677in}}{\pgfqpoint{0.744201in}{-9.987116in}}{\pgfqpoint{0.737755in}{-9.987116in}}%
\pgfpathcurveto{\pgfqpoint{0.731309in}{-9.987116in}}{\pgfqpoint{0.725127in}{-9.989677in}}{\pgfqpoint{0.720569in}{-9.994235in}}%
\pgfpathcurveto{\pgfqpoint{0.716011in}{-9.998793in}}{\pgfqpoint{0.713450in}{-10.004976in}}{\pgfqpoint{0.713450in}{-10.011422in}}%
\pgfpathcurveto{\pgfqpoint{0.713450in}{-10.017868in}}{\pgfqpoint{0.716011in}{-10.024051in}}{\pgfqpoint{0.720569in}{-10.028609in}}%
\pgfpathcurveto{\pgfqpoint{0.725127in}{-10.033167in}}{\pgfqpoint{0.731309in}{-10.035728in}}{\pgfqpoint{0.737755in}{-10.035728in}}%
\pgfpathclose%
\pgfusepath{stroke}%
\end{pgfscope}%
\begin{pgfscope}%
\pgfpathrectangle{\pgfqpoint{0.383193in}{0.383578in}}{\pgfqpoint{2.325000in}{2.310000in}}%
\pgfusepath{clip}%
\pgfsetbuttcap%
\pgfsetroundjoin%
\pgfsetlinewidth{1.405250pt}%
\definecolor{currentstroke}{rgb}{0.000000,0.356863,0.509804}%
\pgfsetstrokecolor{currentstroke}%
\pgfsetdash{}{0pt}%
\pgfpathmoveto{\pgfqpoint{-2.009620in}{-9.843227in}}%
\pgfpathcurveto{\pgfqpoint{-2.003174in}{-9.843227in}}{\pgfqpoint{-1.996991in}{-9.840667in}}{\pgfqpoint{-1.992433in}{-9.836109in}}%
\pgfpathcurveto{\pgfqpoint{-1.987875in}{-9.831551in}}{\pgfqpoint{-1.985314in}{-9.825368in}}{\pgfqpoint{-1.985314in}{-9.818922in}}%
\pgfpathcurveto{\pgfqpoint{-1.985314in}{-9.812476in}}{\pgfqpoint{-1.987875in}{-9.806293in}}{\pgfqpoint{-1.992433in}{-9.801735in}}%
\pgfpathcurveto{\pgfqpoint{-1.996991in}{-9.797177in}}{\pgfqpoint{-2.003174in}{-9.794616in}}{\pgfqpoint{-2.009620in}{-9.794616in}}%
\pgfpathcurveto{\pgfqpoint{-2.016066in}{-9.794616in}}{\pgfqpoint{-2.022248in}{-9.797177in}}{\pgfqpoint{-2.026806in}{-9.801735in}}%
\pgfpathcurveto{\pgfqpoint{-2.031364in}{-9.806293in}}{\pgfqpoint{-2.033925in}{-9.812476in}}{\pgfqpoint{-2.033925in}{-9.818922in}}%
\pgfpathcurveto{\pgfqpoint{-2.033925in}{-9.825368in}}{\pgfqpoint{-2.031364in}{-9.831551in}}{\pgfqpoint{-2.026806in}{-9.836109in}}%
\pgfpathcurveto{\pgfqpoint{-2.022248in}{-9.840667in}}{\pgfqpoint{-2.016066in}{-9.843227in}}{\pgfqpoint{-2.009620in}{-9.843227in}}%
\pgfpathclose%
\pgfusepath{stroke}%
\end{pgfscope}%
\begin{pgfscope}%
\pgfpathrectangle{\pgfqpoint{0.383193in}{0.383578in}}{\pgfqpoint{2.325000in}{2.310000in}}%
\pgfusepath{clip}%
\pgfsetbuttcap%
\pgfsetroundjoin%
\pgfsetlinewidth{1.405250pt}%
\definecolor{currentstroke}{rgb}{0.000000,0.356863,0.509804}%
\pgfsetstrokecolor{currentstroke}%
\pgfsetdash{}{0pt}%
\pgfpathmoveto{\pgfqpoint{-1.340213in}{-8.495727in}}%
\pgfpathcurveto{\pgfqpoint{-1.333768in}{-8.495727in}}{\pgfqpoint{-1.327585in}{-8.493167in}}{\pgfqpoint{-1.323027in}{-8.488609in}}%
\pgfpathcurveto{\pgfqpoint{-1.318469in}{-8.484051in}}{\pgfqpoint{-1.315908in}{-8.477868in}}{\pgfqpoint{-1.315908in}{-8.471422in}}%
\pgfpathcurveto{\pgfqpoint{-1.315908in}{-8.464976in}}{\pgfqpoint{-1.318469in}{-8.458793in}}{\pgfqpoint{-1.323027in}{-8.454235in}}%
\pgfpathcurveto{\pgfqpoint{-1.327585in}{-8.449677in}}{\pgfqpoint{-1.333768in}{-8.447116in}}{\pgfqpoint{-1.340213in}{-8.447116in}}%
\pgfpathcurveto{\pgfqpoint{-1.346659in}{-8.447116in}}{\pgfqpoint{-1.352842in}{-8.449677in}}{\pgfqpoint{-1.357400in}{-8.454235in}}%
\pgfpathcurveto{\pgfqpoint{-1.361958in}{-8.458793in}}{\pgfqpoint{-1.364519in}{-8.464976in}}{\pgfqpoint{-1.364519in}{-8.471422in}}%
\pgfpathcurveto{\pgfqpoint{-1.364519in}{-8.477868in}}{\pgfqpoint{-1.361958in}{-8.484051in}}{\pgfqpoint{-1.357400in}{-8.488609in}}%
\pgfpathcurveto{\pgfqpoint{-1.352842in}{-8.493167in}}{\pgfqpoint{-1.346659in}{-8.495727in}}{\pgfqpoint{-1.340213in}{-8.495727in}}%
\pgfpathclose%
\pgfusepath{stroke}%
\end{pgfscope}%
\begin{pgfscope}%
\pgfpathrectangle{\pgfqpoint{0.383193in}{0.383578in}}{\pgfqpoint{2.325000in}{2.310000in}}%
\pgfusepath{clip}%
\pgfsetbuttcap%
\pgfsetroundjoin%
\pgfsetlinewidth{1.405250pt}%
\definecolor{currentstroke}{rgb}{0.000000,0.356863,0.509804}%
\pgfsetstrokecolor{currentstroke}%
\pgfsetdash{}{0pt}%
\pgfpathmoveto{\pgfqpoint{0.846255in}{-7.918227in}}%
\pgfpathcurveto{\pgfqpoint{0.852701in}{-7.918227in}}{\pgfqpoint{0.858884in}{-7.915667in}}{\pgfqpoint{0.863442in}{-7.911109in}}%
\pgfpathcurveto{\pgfqpoint{0.868000in}{-7.906551in}}{\pgfqpoint{0.870561in}{-7.900368in}}{\pgfqpoint{0.870561in}{-7.893922in}}%
\pgfpathcurveto{\pgfqpoint{0.870561in}{-7.887476in}}{\pgfqpoint{0.868000in}{-7.881293in}}{\pgfqpoint{0.863442in}{-7.876735in}}%
\pgfpathcurveto{\pgfqpoint{0.858884in}{-7.872177in}}{\pgfqpoint{0.852701in}{-7.869616in}}{\pgfqpoint{0.846255in}{-7.869616in}}%
\pgfpathcurveto{\pgfqpoint{0.839809in}{-7.869616in}}{\pgfqpoint{0.833627in}{-7.872177in}}{\pgfqpoint{0.829069in}{-7.876735in}}%
\pgfpathcurveto{\pgfqpoint{0.824511in}{-7.881293in}}{\pgfqpoint{0.821950in}{-7.887476in}}{\pgfqpoint{0.821950in}{-7.893922in}}%
\pgfpathcurveto{\pgfqpoint{0.821950in}{-7.900368in}}{\pgfqpoint{0.824511in}{-7.906551in}}{\pgfqpoint{0.829069in}{-7.911109in}}%
\pgfpathcurveto{\pgfqpoint{0.833627in}{-7.915667in}}{\pgfqpoint{0.839809in}{-7.918227in}}{\pgfqpoint{0.846255in}{-7.918227in}}%
\pgfpathclose%
\pgfusepath{stroke}%
\end{pgfscope}%
\begin{pgfscope}%
\pgfpathrectangle{\pgfqpoint{0.383193in}{0.383578in}}{\pgfqpoint{2.325000in}{2.310000in}}%
\pgfusepath{clip}%
\pgfsetbuttcap%
\pgfsetroundjoin%
\pgfsetlinewidth{1.405250pt}%
\definecolor{currentstroke}{rgb}{0.000000,0.356863,0.509804}%
\pgfsetstrokecolor{currentstroke}%
\pgfsetdash{}{0pt}%
\pgfpathmoveto{\pgfqpoint{6.183099in}{-7.725727in}}%
\pgfpathcurveto{\pgfqpoint{6.189545in}{-7.725727in}}{\pgfqpoint{6.195728in}{-7.723167in}}{\pgfqpoint{6.200286in}{-7.718609in}}%
\pgfpathcurveto{\pgfqpoint{6.204844in}{-7.714051in}}{\pgfqpoint{6.207405in}{-7.707868in}}{\pgfqpoint{6.207405in}{-7.701422in}}%
\pgfpathcurveto{\pgfqpoint{6.207405in}{-7.694976in}}{\pgfqpoint{6.204844in}{-7.688793in}}{\pgfqpoint{6.200286in}{-7.684235in}}%
\pgfpathcurveto{\pgfqpoint{6.195728in}{-7.679677in}}{\pgfqpoint{6.189545in}{-7.677116in}}{\pgfqpoint{6.183099in}{-7.677116in}}%
\pgfpathcurveto{\pgfqpoint{6.176653in}{-7.677116in}}{\pgfqpoint{6.170470in}{-7.679677in}}{\pgfqpoint{6.165912in}{-7.684235in}}%
\pgfpathcurveto{\pgfqpoint{6.161354in}{-7.688793in}}{\pgfqpoint{6.158793in}{-7.694976in}}{\pgfqpoint{6.158793in}{-7.701422in}}%
\pgfpathcurveto{\pgfqpoint{6.158793in}{-7.707868in}}{\pgfqpoint{6.161354in}{-7.714051in}}{\pgfqpoint{6.165912in}{-7.718609in}}%
\pgfpathcurveto{\pgfqpoint{6.170470in}{-7.723167in}}{\pgfqpoint{6.176653in}{-7.725727in}}{\pgfqpoint{6.183099in}{-7.725727in}}%
\pgfpathclose%
\pgfusepath{stroke}%
\end{pgfscope}%
\begin{pgfscope}%
\pgfpathrectangle{\pgfqpoint{0.383193in}{0.383578in}}{\pgfqpoint{2.325000in}{2.310000in}}%
\pgfusepath{clip}%
\pgfsetbuttcap%
\pgfsetroundjoin%
\pgfsetlinewidth{1.405250pt}%
\definecolor{currentstroke}{rgb}{0.000000,0.356863,0.509804}%
\pgfsetstrokecolor{currentstroke}%
\pgfsetdash{}{0pt}%
\pgfpathmoveto{\pgfqpoint{3.967568in}{-6.570727in}}%
\pgfpathcurveto{\pgfqpoint{3.974014in}{-6.570727in}}{\pgfqpoint{3.980196in}{-6.568167in}}{\pgfqpoint{3.984754in}{-6.563609in}}%
\pgfpathcurveto{\pgfqpoint{3.989312in}{-6.559051in}}{\pgfqpoint{3.991873in}{-6.552868in}}{\pgfqpoint{3.991873in}{-6.546422in}}%
\pgfpathcurveto{\pgfqpoint{3.991873in}{-6.539976in}}{\pgfqpoint{3.989312in}{-6.533793in}}{\pgfqpoint{3.984754in}{-6.529235in}}%
\pgfpathcurveto{\pgfqpoint{3.980196in}{-6.524677in}}{\pgfqpoint{3.974014in}{-6.522116in}}{\pgfqpoint{3.967568in}{-6.522116in}}%
\pgfpathcurveto{\pgfqpoint{3.961122in}{-6.522116in}}{\pgfqpoint{3.954939in}{-6.524677in}}{\pgfqpoint{3.950381in}{-6.529235in}}%
\pgfpathcurveto{\pgfqpoint{3.945823in}{-6.533793in}}{\pgfqpoint{3.943262in}{-6.539976in}}{\pgfqpoint{3.943262in}{-6.546422in}}%
\pgfpathcurveto{\pgfqpoint{3.943262in}{-6.552868in}}{\pgfqpoint{3.945823in}{-6.559051in}}{\pgfqpoint{3.950381in}{-6.563609in}}%
\pgfpathcurveto{\pgfqpoint{3.954939in}{-6.568167in}}{\pgfqpoint{3.961122in}{-6.570727in}}{\pgfqpoint{3.967568in}{-6.570727in}}%
\pgfpathclose%
\pgfusepath{stroke}%
\end{pgfscope}%
\begin{pgfscope}%
\pgfpathrectangle{\pgfqpoint{0.383193in}{0.383578in}}{\pgfqpoint{2.325000in}{2.310000in}}%
\pgfusepath{clip}%
\pgfsetbuttcap%
\pgfsetroundjoin%
\pgfsetlinewidth{1.405250pt}%
\definecolor{currentstroke}{rgb}{0.000000,0.356863,0.509804}%
\pgfsetstrokecolor{currentstroke}%
\pgfsetdash{}{0pt}%
\pgfpathmoveto{\pgfqpoint{0.274693in}{-6.378227in}}%
\pgfpathcurveto{\pgfqpoint{0.281139in}{-6.378227in}}{\pgfqpoint{0.287321in}{-6.375667in}}{\pgfqpoint{0.291879in}{-6.371109in}}%
\pgfpathcurveto{\pgfqpoint{0.296437in}{-6.366551in}}{\pgfqpoint{0.298998in}{-6.360368in}}{\pgfqpoint{0.298998in}{-6.353922in}}%
\pgfpathcurveto{\pgfqpoint{0.298998in}{-6.347476in}}{\pgfqpoint{0.296437in}{-6.341293in}}{\pgfqpoint{0.291879in}{-6.336735in}}%
\pgfpathcurveto{\pgfqpoint{0.287321in}{-6.332177in}}{\pgfqpoint{0.281139in}{-6.329616in}}{\pgfqpoint{0.274693in}{-6.329616in}}%
\pgfpathcurveto{\pgfqpoint{0.268247in}{-6.329616in}}{\pgfqpoint{0.262064in}{-6.332177in}}{\pgfqpoint{0.257506in}{-6.336735in}}%
\pgfpathcurveto{\pgfqpoint{0.252948in}{-6.341293in}}{\pgfqpoint{0.250387in}{-6.347476in}}{\pgfqpoint{0.250387in}{-6.353922in}}%
\pgfpathcurveto{\pgfqpoint{0.250387in}{-6.360368in}}{\pgfqpoint{0.252948in}{-6.366551in}}{\pgfqpoint{0.257506in}{-6.371109in}}%
\pgfpathcurveto{\pgfqpoint{0.262064in}{-6.375667in}}{\pgfqpoint{0.268247in}{-6.378227in}}{\pgfqpoint{0.274693in}{-6.378227in}}%
\pgfpathclose%
\pgfusepath{stroke}%
\end{pgfscope}%
\begin{pgfscope}%
\pgfpathrectangle{\pgfqpoint{0.383193in}{0.383578in}}{\pgfqpoint{2.325000in}{2.310000in}}%
\pgfusepath{clip}%
\pgfsetbuttcap%
\pgfsetroundjoin%
\pgfsetlinewidth{1.405250pt}%
\definecolor{currentstroke}{rgb}{0.000000,0.356863,0.509804}%
\pgfsetstrokecolor{currentstroke}%
\pgfsetdash{}{0pt}%
\pgfpathmoveto{\pgfqpoint{1.420724in}{-5.800727in}}%
\pgfpathcurveto{\pgfqpoint{1.427170in}{-5.800727in}}{\pgfqpoint{1.433353in}{-5.798167in}}{\pgfqpoint{1.437911in}{-5.793609in}}%
\pgfpathcurveto{\pgfqpoint{1.442469in}{-5.789051in}}{\pgfqpoint{1.445030in}{-5.782868in}}{\pgfqpoint{1.445030in}{-5.776422in}}%
\pgfpathcurveto{\pgfqpoint{1.445030in}{-5.769976in}}{\pgfqpoint{1.442469in}{-5.763793in}}{\pgfqpoint{1.437911in}{-5.759235in}}%
\pgfpathcurveto{\pgfqpoint{1.433353in}{-5.754677in}}{\pgfqpoint{1.427170in}{-5.752116in}}{\pgfqpoint{1.420724in}{-5.752116in}}%
\pgfpathcurveto{\pgfqpoint{1.414278in}{-5.752116in}}{\pgfqpoint{1.408095in}{-5.754677in}}{\pgfqpoint{1.403537in}{-5.759235in}}%
\pgfpathcurveto{\pgfqpoint{1.398979in}{-5.763793in}}{\pgfqpoint{1.396418in}{-5.769976in}}{\pgfqpoint{1.396418in}{-5.776422in}}%
\pgfpathcurveto{\pgfqpoint{1.396418in}{-5.782868in}}{\pgfqpoint{1.398979in}{-5.789051in}}{\pgfqpoint{1.403537in}{-5.793609in}}%
\pgfpathcurveto{\pgfqpoint{1.408095in}{-5.798167in}}{\pgfqpoint{1.414278in}{-5.800727in}}{\pgfqpoint{1.420724in}{-5.800727in}}%
\pgfpathclose%
\pgfusepath{stroke}%
\end{pgfscope}%
\begin{pgfscope}%
\pgfpathrectangle{\pgfqpoint{0.383193in}{0.383578in}}{\pgfqpoint{2.325000in}{2.310000in}}%
\pgfusepath{clip}%
\pgfsetbuttcap%
\pgfsetroundjoin%
\pgfsetlinewidth{1.405250pt}%
\definecolor{currentstroke}{rgb}{0.000000,0.356863,0.509804}%
\pgfsetstrokecolor{currentstroke}%
\pgfsetdash{}{0pt}%
\pgfpathmoveto{\pgfqpoint{-0.031432in}{-4.838227in}}%
\pgfpathcurveto{\pgfqpoint{-0.024986in}{-4.838227in}}{\pgfqpoint{-0.018804in}{-4.835667in}}{\pgfqpoint{-0.014246in}{-4.831109in}}%
\pgfpathcurveto{\pgfqpoint{-0.009688in}{-4.826551in}}{\pgfqpoint{-0.007127in}{-4.820368in}}{\pgfqpoint{-0.007127in}{-4.813922in}}%
\pgfpathcurveto{\pgfqpoint{-0.007127in}{-4.807476in}}{\pgfqpoint{-0.009688in}{-4.801293in}}{\pgfqpoint{-0.014246in}{-4.796735in}}%
\pgfpathcurveto{\pgfqpoint{-0.018804in}{-4.792177in}}{\pgfqpoint{-0.024986in}{-4.789616in}}{\pgfqpoint{-0.031432in}{-4.789616in}}%
\pgfpathcurveto{\pgfqpoint{-0.037878in}{-4.789616in}}{\pgfqpoint{-0.044061in}{-4.792177in}}{\pgfqpoint{-0.048619in}{-4.796735in}}%
\pgfpathcurveto{\pgfqpoint{-0.053177in}{-4.801293in}}{\pgfqpoint{-0.055738in}{-4.807476in}}{\pgfqpoint{-0.055738in}{-4.813922in}}%
\pgfpathcurveto{\pgfqpoint{-0.055738in}{-4.820368in}}{\pgfqpoint{-0.053177in}{-4.826551in}}{\pgfqpoint{-0.048619in}{-4.831109in}}%
\pgfpathcurveto{\pgfqpoint{-0.044061in}{-4.835667in}}{\pgfqpoint{-0.037878in}{-4.838227in}}{\pgfqpoint{-0.031432in}{-4.838227in}}%
\pgfpathclose%
\pgfusepath{stroke}%
\end{pgfscope}%
\begin{pgfscope}%
\pgfpathrectangle{\pgfqpoint{0.383193in}{0.383578in}}{\pgfqpoint{2.325000in}{2.310000in}}%
\pgfusepath{clip}%
\pgfsetbuttcap%
\pgfsetroundjoin%
\pgfsetlinewidth{1.405250pt}%
\definecolor{currentstroke}{rgb}{0.000000,0.356863,0.509804}%
\pgfsetstrokecolor{currentstroke}%
\pgfsetdash{}{0pt}%
\pgfpathmoveto{\pgfqpoint{6.403005in}{-4.645727in}}%
\pgfpathcurveto{\pgfqpoint{6.409451in}{-4.645727in}}{\pgfqpoint{6.415634in}{-4.643167in}}{\pgfqpoint{6.420192in}{-4.638609in}}%
\pgfpathcurveto{\pgfqpoint{6.424750in}{-4.634051in}}{\pgfqpoint{6.427311in}{-4.627868in}}{\pgfqpoint{6.427311in}{-4.621422in}}%
\pgfpathcurveto{\pgfqpoint{6.427311in}{-4.614976in}}{\pgfqpoint{6.424750in}{-4.608793in}}{\pgfqpoint{6.420192in}{-4.604235in}}%
\pgfpathcurveto{\pgfqpoint{6.415634in}{-4.599677in}}{\pgfqpoint{6.409451in}{-4.597116in}}{\pgfqpoint{6.403005in}{-4.597116in}}%
\pgfpathcurveto{\pgfqpoint{6.396559in}{-4.597116in}}{\pgfqpoint{6.390377in}{-4.599677in}}{\pgfqpoint{6.385819in}{-4.604235in}}%
\pgfpathcurveto{\pgfqpoint{6.381261in}{-4.608793in}}{\pgfqpoint{6.378700in}{-4.614976in}}{\pgfqpoint{6.378700in}{-4.621422in}}%
\pgfpathcurveto{\pgfqpoint{6.378700in}{-4.627868in}}{\pgfqpoint{6.381261in}{-4.634051in}}{\pgfqpoint{6.385819in}{-4.638609in}}%
\pgfpathcurveto{\pgfqpoint{6.390377in}{-4.643167in}}{\pgfqpoint{6.396559in}{-4.645727in}}{\pgfqpoint{6.403005in}{-4.645727in}}%
\pgfpathclose%
\pgfusepath{stroke}%
\end{pgfscope}%
\begin{pgfscope}%
\pgfpathrectangle{\pgfqpoint{0.383193in}{0.383578in}}{\pgfqpoint{2.325000in}{2.310000in}}%
\pgfusepath{clip}%
\pgfsetbuttcap%
\pgfsetroundjoin%
\pgfsetlinewidth{1.405250pt}%
\definecolor{currentstroke}{rgb}{0.000000,0.356863,0.509804}%
\pgfsetstrokecolor{currentstroke}%
\pgfsetdash{}{0pt}%
\pgfpathmoveto{\pgfqpoint{-0.345307in}{-4.453227in}}%
\pgfpathcurveto{\pgfqpoint{-0.338861in}{-4.453227in}}{\pgfqpoint{-0.332679in}{-4.450667in}}{\pgfqpoint{-0.328121in}{-4.446109in}}%
\pgfpathcurveto{\pgfqpoint{-0.323563in}{-4.441551in}}{\pgfqpoint{-0.321002in}{-4.435368in}}{\pgfqpoint{-0.321002in}{-4.428922in}}%
\pgfpathcurveto{\pgfqpoint{-0.321002in}{-4.422476in}}{\pgfqpoint{-0.323563in}{-4.416293in}}{\pgfqpoint{-0.328121in}{-4.411735in}}%
\pgfpathcurveto{\pgfqpoint{-0.332679in}{-4.407177in}}{\pgfqpoint{-0.338861in}{-4.404616in}}{\pgfqpoint{-0.345307in}{-4.404616in}}%
\pgfpathcurveto{\pgfqpoint{-0.351753in}{-4.404616in}}{\pgfqpoint{-0.357936in}{-4.407177in}}{\pgfqpoint{-0.362494in}{-4.411735in}}%
\pgfpathcurveto{\pgfqpoint{-0.367052in}{-4.416293in}}{\pgfqpoint{-0.369613in}{-4.422476in}}{\pgfqpoint{-0.369613in}{-4.428922in}}%
\pgfpathcurveto{\pgfqpoint{-0.369613in}{-4.435368in}}{\pgfqpoint{-0.367052in}{-4.441551in}}{\pgfqpoint{-0.362494in}{-4.446109in}}%
\pgfpathcurveto{\pgfqpoint{-0.357936in}{-4.450667in}}{\pgfqpoint{-0.351753in}{-4.453227in}}{\pgfqpoint{-0.345307in}{-4.453227in}}%
\pgfpathclose%
\pgfusepath{stroke}%
\end{pgfscope}%
\begin{pgfscope}%
\pgfpathrectangle{\pgfqpoint{0.383193in}{0.383578in}}{\pgfqpoint{2.325000in}{2.310000in}}%
\pgfusepath{clip}%
\pgfsetbuttcap%
\pgfsetroundjoin%
\pgfsetlinewidth{1.405250pt}%
\definecolor{currentstroke}{rgb}{0.000000,0.356863,0.509804}%
\pgfsetstrokecolor{currentstroke}%
\pgfsetdash{}{0pt}%
\pgfpathmoveto{\pgfqpoint{4.741599in}{-4.260727in}}%
\pgfpathcurveto{\pgfqpoint{4.748045in}{-4.260727in}}{\pgfqpoint{4.754228in}{-4.258167in}}{\pgfqpoint{4.758786in}{-4.253609in}}%
\pgfpathcurveto{\pgfqpoint{4.763344in}{-4.249051in}}{\pgfqpoint{4.765905in}{-4.242868in}}{\pgfqpoint{4.765905in}{-4.236422in}}%
\pgfpathcurveto{\pgfqpoint{4.765905in}{-4.229976in}}{\pgfqpoint{4.763344in}{-4.223793in}}{\pgfqpoint{4.758786in}{-4.219235in}}%
\pgfpathcurveto{\pgfqpoint{4.754228in}{-4.214677in}}{\pgfqpoint{4.748045in}{-4.212116in}}{\pgfqpoint{4.741599in}{-4.212116in}}%
\pgfpathcurveto{\pgfqpoint{4.735153in}{-4.212116in}}{\pgfqpoint{4.728970in}{-4.214677in}}{\pgfqpoint{4.724412in}{-4.219235in}}%
\pgfpathcurveto{\pgfqpoint{4.719854in}{-4.223793in}}{\pgfqpoint{4.717293in}{-4.229976in}}{\pgfqpoint{4.717293in}{-4.236422in}}%
\pgfpathcurveto{\pgfqpoint{4.717293in}{-4.242868in}}{\pgfqpoint{4.719854in}{-4.249051in}}{\pgfqpoint{4.724412in}{-4.253609in}}%
\pgfpathcurveto{\pgfqpoint{4.728970in}{-4.258167in}}{\pgfqpoint{4.735153in}{-4.260727in}}{\pgfqpoint{4.741599in}{-4.260727in}}%
\pgfpathclose%
\pgfusepath{stroke}%
\end{pgfscope}%
\begin{pgfscope}%
\pgfpathrectangle{\pgfqpoint{0.383193in}{0.383578in}}{\pgfqpoint{2.325000in}{2.310000in}}%
\pgfusepath{clip}%
\pgfsetbuttcap%
\pgfsetroundjoin%
\pgfsetlinewidth{1.405250pt}%
\definecolor{currentstroke}{rgb}{0.000000,0.356863,0.509804}%
\pgfsetstrokecolor{currentstroke}%
\pgfsetdash{}{0pt}%
\pgfpathmoveto{\pgfqpoint{-0.682432in}{-4.068227in}}%
\pgfpathcurveto{\pgfqpoint{-0.675986in}{-4.068227in}}{\pgfqpoint{-0.669804in}{-4.065667in}}{\pgfqpoint{-0.665246in}{-4.061109in}}%
\pgfpathcurveto{\pgfqpoint{-0.660688in}{-4.056551in}}{\pgfqpoint{-0.658127in}{-4.050368in}}{\pgfqpoint{-0.658127in}{-4.043922in}}%
\pgfpathcurveto{\pgfqpoint{-0.658127in}{-4.037476in}}{\pgfqpoint{-0.660688in}{-4.031293in}}{\pgfqpoint{-0.665246in}{-4.026735in}}%
\pgfpathcurveto{\pgfqpoint{-0.669804in}{-4.022177in}}{\pgfqpoint{-0.675986in}{-4.019616in}}{\pgfqpoint{-0.682432in}{-4.019616in}}%
\pgfpathcurveto{\pgfqpoint{-0.688878in}{-4.019616in}}{\pgfqpoint{-0.695061in}{-4.022177in}}{\pgfqpoint{-0.699619in}{-4.026735in}}%
\pgfpathcurveto{\pgfqpoint{-0.704177in}{-4.031293in}}{\pgfqpoint{-0.706738in}{-4.037476in}}{\pgfqpoint{-0.706738in}{-4.043922in}}%
\pgfpathcurveto{\pgfqpoint{-0.706738in}{-4.050368in}}{\pgfqpoint{-0.704177in}{-4.056551in}}{\pgfqpoint{-0.699619in}{-4.061109in}}%
\pgfpathcurveto{\pgfqpoint{-0.695061in}{-4.065667in}}{\pgfqpoint{-0.688878in}{-4.068227in}}{\pgfqpoint{-0.682432in}{-4.068227in}}%
\pgfpathclose%
\pgfusepath{stroke}%
\end{pgfscope}%
\begin{pgfscope}%
\pgfpathrectangle{\pgfqpoint{0.383193in}{0.383578in}}{\pgfqpoint{2.325000in}{2.310000in}}%
\pgfusepath{clip}%
\pgfsetbuttcap%
\pgfsetroundjoin%
\pgfsetlinewidth{1.405250pt}%
\definecolor{currentstroke}{rgb}{0.000000,0.356863,0.509804}%
\pgfsetstrokecolor{currentstroke}%
\pgfsetdash{}{0pt}%
\pgfpathmoveto{\pgfqpoint{5.039005in}{-3.683227in}}%
\pgfpathcurveto{\pgfqpoint{5.045451in}{-3.683227in}}{\pgfqpoint{5.051634in}{-3.680667in}}{\pgfqpoint{5.056192in}{-3.676109in}}%
\pgfpathcurveto{\pgfqpoint{5.060750in}{-3.671551in}}{\pgfqpoint{5.063311in}{-3.665368in}}{\pgfqpoint{5.063311in}{-3.658922in}}%
\pgfpathcurveto{\pgfqpoint{5.063311in}{-3.652476in}}{\pgfqpoint{5.060750in}{-3.646293in}}{\pgfqpoint{5.056192in}{-3.641735in}}%
\pgfpathcurveto{\pgfqpoint{5.051634in}{-3.637177in}}{\pgfqpoint{5.045451in}{-3.634616in}}{\pgfqpoint{5.039005in}{-3.634616in}}%
\pgfpathcurveto{\pgfqpoint{5.032559in}{-3.634616in}}{\pgfqpoint{5.026377in}{-3.637177in}}{\pgfqpoint{5.021819in}{-3.641735in}}%
\pgfpathcurveto{\pgfqpoint{5.017261in}{-3.646293in}}{\pgfqpoint{5.014700in}{-3.652476in}}{\pgfqpoint{5.014700in}{-3.658922in}}%
\pgfpathcurveto{\pgfqpoint{5.014700in}{-3.665368in}}{\pgfqpoint{5.017261in}{-3.671551in}}{\pgfqpoint{5.021819in}{-3.676109in}}%
\pgfpathcurveto{\pgfqpoint{5.026377in}{-3.680667in}}{\pgfqpoint{5.032559in}{-3.683227in}}{\pgfqpoint{5.039005in}{-3.683227in}}%
\pgfpathclose%
\pgfusepath{stroke}%
\end{pgfscope}%
\begin{pgfscope}%
\pgfpathrectangle{\pgfqpoint{0.383193in}{0.383578in}}{\pgfqpoint{2.325000in}{2.310000in}}%
\pgfusepath{clip}%
\pgfsetbuttcap%
\pgfsetroundjoin%
\pgfsetlinewidth{1.405250pt}%
\definecolor{currentstroke}{rgb}{0.000000,0.356863,0.509804}%
\pgfsetstrokecolor{currentstroke}%
\pgfsetdash{}{0pt}%
\pgfpathmoveto{\pgfqpoint{-0.427651in}{-2.720727in}}%
\pgfpathcurveto{\pgfqpoint{-0.421205in}{-2.720727in}}{\pgfqpoint{-0.415022in}{-2.718167in}}{\pgfqpoint{-0.410464in}{-2.713609in}}%
\pgfpathcurveto{\pgfqpoint{-0.405906in}{-2.709051in}}{\pgfqpoint{-0.403345in}{-2.702868in}}{\pgfqpoint{-0.403345in}{-2.696422in}}%
\pgfpathcurveto{\pgfqpoint{-0.403345in}{-2.689976in}}{\pgfqpoint{-0.405906in}{-2.683793in}}{\pgfqpoint{-0.410464in}{-2.679235in}}%
\pgfpathcurveto{\pgfqpoint{-0.415022in}{-2.674677in}}{\pgfqpoint{-0.421205in}{-2.672116in}}{\pgfqpoint{-0.427651in}{-2.672116in}}%
\pgfpathcurveto{\pgfqpoint{-0.434097in}{-2.672116in}}{\pgfqpoint{-0.440280in}{-2.674677in}}{\pgfqpoint{-0.444838in}{-2.679235in}}%
\pgfpathcurveto{\pgfqpoint{-0.449396in}{-2.683793in}}{\pgfqpoint{-0.451957in}{-2.689976in}}{\pgfqpoint{-0.451957in}{-2.696422in}}%
\pgfpathcurveto{\pgfqpoint{-0.451957in}{-2.702868in}}{\pgfqpoint{-0.449396in}{-2.709051in}}{\pgfqpoint{-0.444838in}{-2.713609in}}%
\pgfpathcurveto{\pgfqpoint{-0.440280in}{-2.718167in}}{\pgfqpoint{-0.434097in}{-2.720727in}}{\pgfqpoint{-0.427651in}{-2.720727in}}%
\pgfpathclose%
\pgfusepath{stroke}%
\end{pgfscope}%
\begin{pgfscope}%
\pgfpathrectangle{\pgfqpoint{0.383193in}{0.383578in}}{\pgfqpoint{2.325000in}{2.310000in}}%
\pgfusepath{clip}%
\pgfsetbuttcap%
\pgfsetroundjoin%
\pgfsetlinewidth{1.405250pt}%
\definecolor{currentstroke}{rgb}{0.000000,0.356863,0.509804}%
\pgfsetstrokecolor{currentstroke}%
\pgfsetdash{}{0pt}%
\pgfpathmoveto{\pgfqpoint{0.609880in}{-2.528227in}}%
\pgfpathcurveto{\pgfqpoint{0.616326in}{-2.528227in}}{\pgfqpoint{0.622509in}{-2.525667in}}{\pgfqpoint{0.627067in}{-2.521109in}}%
\pgfpathcurveto{\pgfqpoint{0.631625in}{-2.516551in}}{\pgfqpoint{0.634186in}{-2.510368in}}{\pgfqpoint{0.634186in}{-2.503922in}}%
\pgfpathcurveto{\pgfqpoint{0.634186in}{-2.497476in}}{\pgfqpoint{0.631625in}{-2.491293in}}{\pgfqpoint{0.627067in}{-2.486735in}}%
\pgfpathcurveto{\pgfqpoint{0.622509in}{-2.482177in}}{\pgfqpoint{0.616326in}{-2.479616in}}{\pgfqpoint{0.609880in}{-2.479616in}}%
\pgfpathcurveto{\pgfqpoint{0.603434in}{-2.479616in}}{\pgfqpoint{0.597252in}{-2.482177in}}{\pgfqpoint{0.592694in}{-2.486735in}}%
\pgfpathcurveto{\pgfqpoint{0.588136in}{-2.491293in}}{\pgfqpoint{0.585575in}{-2.497476in}}{\pgfqpoint{0.585575in}{-2.503922in}}%
\pgfpathcurveto{\pgfqpoint{0.585575in}{-2.510368in}}{\pgfqpoint{0.588136in}{-2.516551in}}{\pgfqpoint{0.592694in}{-2.521109in}}%
\pgfpathcurveto{\pgfqpoint{0.597252in}{-2.525667in}}{\pgfqpoint{0.603434in}{-2.528227in}}{\pgfqpoint{0.609880in}{-2.528227in}}%
\pgfpathclose%
\pgfusepath{stroke}%
\end{pgfscope}%
\begin{pgfscope}%
\pgfpathrectangle{\pgfqpoint{0.383193in}{0.383578in}}{\pgfqpoint{2.325000in}{2.310000in}}%
\pgfusepath{clip}%
\pgfsetbuttcap%
\pgfsetroundjoin%
\pgfsetlinewidth{1.405250pt}%
\definecolor{currentstroke}{rgb}{0.000000,0.356863,0.509804}%
\pgfsetstrokecolor{currentstroke}%
\pgfsetdash{}{0pt}%
\pgfpathmoveto{\pgfqpoint{0.832693in}{-0.988227in}}%
\pgfpathcurveto{\pgfqpoint{0.839139in}{-0.988227in}}{\pgfqpoint{0.845321in}{-0.985667in}}{\pgfqpoint{0.849879in}{-0.981109in}}%
\pgfpathcurveto{\pgfqpoint{0.854437in}{-0.976551in}}{\pgfqpoint{0.856998in}{-0.970368in}}{\pgfqpoint{0.856998in}{-0.963922in}}%
\pgfpathcurveto{\pgfqpoint{0.856998in}{-0.957476in}}{\pgfqpoint{0.854437in}{-0.951293in}}{\pgfqpoint{0.849879in}{-0.946735in}}%
\pgfpathcurveto{\pgfqpoint{0.845321in}{-0.942177in}}{\pgfqpoint{0.839139in}{-0.939616in}}{\pgfqpoint{0.832693in}{-0.939616in}}%
\pgfpathcurveto{\pgfqpoint{0.826247in}{-0.939616in}}{\pgfqpoint{0.820064in}{-0.942177in}}{\pgfqpoint{0.815506in}{-0.946735in}}%
\pgfpathcurveto{\pgfqpoint{0.810948in}{-0.951293in}}{\pgfqpoint{0.808387in}{-0.957476in}}{\pgfqpoint{0.808387in}{-0.963922in}}%
\pgfpathcurveto{\pgfqpoint{0.808387in}{-0.970368in}}{\pgfqpoint{0.810948in}{-0.976551in}}{\pgfqpoint{0.815506in}{-0.981109in}}%
\pgfpathcurveto{\pgfqpoint{0.820064in}{-0.985667in}}{\pgfqpoint{0.826247in}{-0.988227in}}{\pgfqpoint{0.832693in}{-0.988227in}}%
\pgfpathclose%
\pgfusepath{stroke}%
\end{pgfscope}%
\begin{pgfscope}%
\pgfpathrectangle{\pgfqpoint{0.383193in}{0.383578in}}{\pgfqpoint{2.325000in}{2.310000in}}%
\pgfusepath{clip}%
\pgfsetbuttcap%
\pgfsetroundjoin%
\pgfsetlinewidth{1.405250pt}%
\definecolor{currentstroke}{rgb}{0.000000,0.356863,0.509804}%
\pgfsetstrokecolor{currentstroke}%
\pgfsetdash{}{0pt}%
\pgfpathmoveto{\pgfqpoint{5.149443in}{-0.795727in}}%
\pgfpathcurveto{\pgfqpoint{5.155889in}{-0.795727in}}{\pgfqpoint{5.162071in}{-0.793167in}}{\pgfqpoint{5.166629in}{-0.788609in}}%
\pgfpathcurveto{\pgfqpoint{5.171187in}{-0.784051in}}{\pgfqpoint{5.173748in}{-0.777868in}}{\pgfqpoint{5.173748in}{-0.771422in}}%
\pgfpathcurveto{\pgfqpoint{5.173748in}{-0.764976in}}{\pgfqpoint{5.171187in}{-0.758793in}}{\pgfqpoint{5.166629in}{-0.754235in}}%
\pgfpathcurveto{\pgfqpoint{5.162071in}{-0.749677in}}{\pgfqpoint{5.155889in}{-0.747116in}}{\pgfqpoint{5.149443in}{-0.747116in}}%
\pgfpathcurveto{\pgfqpoint{5.142997in}{-0.747116in}}{\pgfqpoint{5.136814in}{-0.749677in}}{\pgfqpoint{5.132256in}{-0.754235in}}%
\pgfpathcurveto{\pgfqpoint{5.127698in}{-0.758793in}}{\pgfqpoint{5.125137in}{-0.764976in}}{\pgfqpoint{5.125137in}{-0.771422in}}%
\pgfpathcurveto{\pgfqpoint{5.125137in}{-0.777868in}}{\pgfqpoint{5.127698in}{-0.784051in}}{\pgfqpoint{5.132256in}{-0.788609in}}%
\pgfpathcurveto{\pgfqpoint{5.136814in}{-0.793167in}}{\pgfqpoint{5.142997in}{-0.795727in}}{\pgfqpoint{5.149443in}{-0.795727in}}%
\pgfpathclose%
\pgfusepath{stroke}%
\end{pgfscope}%
\begin{pgfscope}%
\pgfpathrectangle{\pgfqpoint{0.383193in}{0.383578in}}{\pgfqpoint{2.325000in}{2.310000in}}%
\pgfusepath{clip}%
\pgfsetbuttcap%
\pgfsetroundjoin%
\pgfsetlinewidth{1.405250pt}%
\definecolor{currentstroke}{rgb}{0.000000,0.356863,0.509804}%
\pgfsetstrokecolor{currentstroke}%
\pgfsetdash{}{0pt}%
\pgfpathmoveto{\pgfqpoint{4.335693in}{-0.603227in}}%
\pgfpathcurveto{\pgfqpoint{4.342139in}{-0.603227in}}{\pgfqpoint{4.348321in}{-0.600667in}}{\pgfqpoint{4.352879in}{-0.596109in}}%
\pgfpathcurveto{\pgfqpoint{4.357437in}{-0.591551in}}{\pgfqpoint{4.359998in}{-0.585368in}}{\pgfqpoint{4.359998in}{-0.578922in}}%
\pgfpathcurveto{\pgfqpoint{4.359998in}{-0.572476in}}{\pgfqpoint{4.357437in}{-0.566293in}}{\pgfqpoint{4.352879in}{-0.561735in}}%
\pgfpathcurveto{\pgfqpoint{4.348321in}{-0.557177in}}{\pgfqpoint{4.342139in}{-0.554616in}}{\pgfqpoint{4.335693in}{-0.554616in}}%
\pgfpathcurveto{\pgfqpoint{4.329247in}{-0.554616in}}{\pgfqpoint{4.323064in}{-0.557177in}}{\pgfqpoint{4.318506in}{-0.561735in}}%
\pgfpathcurveto{\pgfqpoint{4.313948in}{-0.566293in}}{\pgfqpoint{4.311387in}{-0.572476in}}{\pgfqpoint{4.311387in}{-0.578922in}}%
\pgfpathcurveto{\pgfqpoint{4.311387in}{-0.585368in}}{\pgfqpoint{4.313948in}{-0.591551in}}{\pgfqpoint{4.318506in}{-0.596109in}}%
\pgfpathcurveto{\pgfqpoint{4.323064in}{-0.600667in}}{\pgfqpoint{4.329247in}{-0.603227in}}{\pgfqpoint{4.335693in}{-0.603227in}}%
\pgfpathclose%
\pgfusepath{stroke}%
\end{pgfscope}%
\begin{pgfscope}%
\pgfpathrectangle{\pgfqpoint{0.383193in}{0.383578in}}{\pgfqpoint{2.325000in}{2.310000in}}%
\pgfusepath{clip}%
\pgfsetbuttcap%
\pgfsetroundjoin%
\pgfsetlinewidth{1.405250pt}%
\definecolor{currentstroke}{rgb}{0.000000,0.356863,0.509804}%
\pgfsetstrokecolor{currentstroke}%
\pgfsetdash{}{0pt}%
\pgfpathmoveto{\pgfqpoint{-1.232682in}{-0.410727in}}%
\pgfpathcurveto{\pgfqpoint{-1.226236in}{-0.410727in}}{\pgfqpoint{-1.220054in}{-0.408167in}}{\pgfqpoint{-1.215496in}{-0.403609in}}%
\pgfpathcurveto{\pgfqpoint{-1.210938in}{-0.399051in}}{\pgfqpoint{-1.208377in}{-0.392868in}}{\pgfqpoint{-1.208377in}{-0.386422in}}%
\pgfpathcurveto{\pgfqpoint{-1.208377in}{-0.379976in}}{\pgfqpoint{-1.210938in}{-0.373793in}}{\pgfqpoint{-1.215496in}{-0.369235in}}%
\pgfpathcurveto{\pgfqpoint{-1.220054in}{-0.364677in}}{\pgfqpoint{-1.226236in}{-0.362116in}}{\pgfqpoint{-1.232682in}{-0.362116in}}%
\pgfpathcurveto{\pgfqpoint{-1.239128in}{-0.362116in}}{\pgfqpoint{-1.245311in}{-0.364677in}}{\pgfqpoint{-1.249869in}{-0.369235in}}%
\pgfpathcurveto{\pgfqpoint{-1.254427in}{-0.373793in}}{\pgfqpoint{-1.256988in}{-0.379976in}}{\pgfqpoint{-1.256988in}{-0.386422in}}%
\pgfpathcurveto{\pgfqpoint{-1.256988in}{-0.392868in}}{\pgfqpoint{-1.254427in}{-0.399051in}}{\pgfqpoint{-1.249869in}{-0.403609in}}%
\pgfpathcurveto{\pgfqpoint{-1.245311in}{-0.408167in}}{\pgfqpoint{-1.239128in}{-0.410727in}}{\pgfqpoint{-1.232682in}{-0.410727in}}%
\pgfpathclose%
\pgfusepath{stroke}%
\end{pgfscope}%
\begin{pgfscope}%
\pgfpathrectangle{\pgfqpoint{0.383193in}{0.383578in}}{\pgfqpoint{2.325000in}{2.310000in}}%
\pgfusepath{clip}%
\pgfsetbuttcap%
\pgfsetroundjoin%
\pgfsetlinewidth{1.405250pt}%
\definecolor{currentstroke}{rgb}{0.000000,0.356863,0.509804}%
\pgfsetstrokecolor{currentstroke}%
\pgfsetdash{}{0pt}%
\pgfpathmoveto{\pgfqpoint{-0.466401in}{0.166773in}}%
\pgfpathcurveto{\pgfqpoint{-0.459955in}{0.166773in}}{\pgfqpoint{-0.453772in}{0.169333in}}{\pgfqpoint{-0.449214in}{0.173891in}}%
\pgfpathcurveto{\pgfqpoint{-0.444656in}{0.178449in}}{\pgfqpoint{-0.442095in}{0.184632in}}{\pgfqpoint{-0.442095in}{0.191078in}}%
\pgfpathcurveto{\pgfqpoint{-0.442095in}{0.197524in}}{\pgfqpoint{-0.444656in}{0.203707in}}{\pgfqpoint{-0.449214in}{0.208265in}}%
\pgfpathcurveto{\pgfqpoint{-0.453772in}{0.212823in}}{\pgfqpoint{-0.459955in}{0.215384in}}{\pgfqpoint{-0.466401in}{0.215384in}}%
\pgfpathcurveto{\pgfqpoint{-0.472847in}{0.215384in}}{\pgfqpoint{-0.479030in}{0.212823in}}{\pgfqpoint{-0.483588in}{0.208265in}}%
\pgfpathcurveto{\pgfqpoint{-0.488146in}{0.203707in}}{\pgfqpoint{-0.490707in}{0.197524in}}{\pgfqpoint{-0.490707in}{0.191078in}}%
\pgfpathcurveto{\pgfqpoint{-0.490707in}{0.184632in}}{\pgfqpoint{-0.488146in}{0.178449in}}{\pgfqpoint{-0.483588in}{0.173891in}}%
\pgfpathcurveto{\pgfqpoint{-0.479030in}{0.169333in}}{\pgfqpoint{-0.472847in}{0.166773in}}{\pgfqpoint{-0.466401in}{0.166773in}}%
\pgfpathclose%
\pgfusepath{stroke}%
\end{pgfscope}%
\begin{pgfscope}%
\pgfpathrectangle{\pgfqpoint{0.383193in}{0.383578in}}{\pgfqpoint{2.325000in}{2.310000in}}%
\pgfusepath{clip}%
\pgfsetbuttcap%
\pgfsetroundjoin%
\pgfsetlinewidth{1.405250pt}%
\definecolor{currentstroke}{rgb}{0.000000,0.356863,0.509804}%
\pgfsetstrokecolor{currentstroke}%
\pgfsetdash{}{0pt}%
\pgfpathmoveto{\pgfqpoint{-2.015432in}{0.359273in}}%
\pgfpathcurveto{\pgfqpoint{-2.008986in}{0.359273in}}{\pgfqpoint{-2.002804in}{0.361833in}}{\pgfqpoint{-1.998246in}{0.366391in}}%
\pgfpathcurveto{\pgfqpoint{-1.993688in}{0.370949in}}{\pgfqpoint{-1.991127in}{0.377132in}}{\pgfqpoint{-1.991127in}{0.383578in}}%
\pgfpathcurveto{\pgfqpoint{-1.991127in}{0.390024in}}{\pgfqpoint{-1.993688in}{0.396207in}}{\pgfqpoint{-1.998246in}{0.400765in}}%
\pgfpathcurveto{\pgfqpoint{-2.002804in}{0.405323in}}{\pgfqpoint{-2.008986in}{0.407884in}}{\pgfqpoint{-2.015432in}{0.407884in}}%
\pgfpathcurveto{\pgfqpoint{-2.021878in}{0.407884in}}{\pgfqpoint{-2.028061in}{0.405323in}}{\pgfqpoint{-2.032619in}{0.400765in}}%
\pgfpathcurveto{\pgfqpoint{-2.037177in}{0.396207in}}{\pgfqpoint{-2.039738in}{0.390024in}}{\pgfqpoint{-2.039738in}{0.383578in}}%
\pgfpathcurveto{\pgfqpoint{-2.039738in}{0.377132in}}{\pgfqpoint{-2.037177in}{0.370949in}}{\pgfqpoint{-2.032619in}{0.366391in}}%
\pgfpathcurveto{\pgfqpoint{-2.028061in}{0.361833in}}{\pgfqpoint{-2.021878in}{0.359273in}}{\pgfqpoint{-2.015432in}{0.359273in}}%
\pgfpathclose%
\pgfusepath{stroke}%
\end{pgfscope}%
\begin{pgfscope}%
\pgfpathrectangle{\pgfqpoint{0.383193in}{0.383578in}}{\pgfqpoint{2.325000in}{2.310000in}}%
\pgfusepath{clip}%
\pgfsetbuttcap%
\pgfsetroundjoin%
\pgfsetlinewidth{1.405250pt}%
\definecolor{currentstroke}{rgb}{0.000000,0.356863,0.509804}%
\pgfsetstrokecolor{currentstroke}%
\pgfsetdash{}{0pt}%
\pgfpathmoveto{\pgfqpoint{0.651537in}{0.551773in}}%
\pgfpathcurveto{\pgfqpoint{0.657982in}{0.551773in}}{\pgfqpoint{0.664165in}{0.554333in}}{\pgfqpoint{0.668723in}{0.558891in}}%
\pgfpathcurveto{\pgfqpoint{0.673281in}{0.563449in}}{\pgfqpoint{0.675842in}{0.569632in}}{\pgfqpoint{0.675842in}{0.576078in}}%
\pgfpathcurveto{\pgfqpoint{0.675842in}{0.582524in}}{\pgfqpoint{0.673281in}{0.588707in}}{\pgfqpoint{0.668723in}{0.593265in}}%
\pgfpathcurveto{\pgfqpoint{0.664165in}{0.597823in}}{\pgfqpoint{0.657982in}{0.600384in}}{\pgfqpoint{0.651537in}{0.600384in}}%
\pgfpathcurveto{\pgfqpoint{0.645091in}{0.600384in}}{\pgfqpoint{0.638908in}{0.597823in}}{\pgfqpoint{0.634350in}{0.593265in}}%
\pgfpathcurveto{\pgfqpoint{0.629792in}{0.588707in}}{\pgfqpoint{0.627231in}{0.582524in}}{\pgfqpoint{0.627231in}{0.576078in}}%
\pgfpathcurveto{\pgfqpoint{0.627231in}{0.569632in}}{\pgfqpoint{0.629792in}{0.563449in}}{\pgfqpoint{0.634350in}{0.558891in}}%
\pgfpathcurveto{\pgfqpoint{0.638908in}{0.554333in}}{\pgfqpoint{0.645091in}{0.551773in}}{\pgfqpoint{0.651537in}{0.551773in}}%
\pgfpathclose%
\pgfusepath{stroke}%
\end{pgfscope}%
\begin{pgfscope}%
\pgfpathrectangle{\pgfqpoint{0.383193in}{0.383578in}}{\pgfqpoint{2.325000in}{2.310000in}}%
\pgfusepath{clip}%
\pgfsetbuttcap%
\pgfsetroundjoin%
\pgfsetlinewidth{1.405250pt}%
\definecolor{currentstroke}{rgb}{0.000000,0.356863,0.509804}%
\pgfsetstrokecolor{currentstroke}%
\pgfsetdash{}{0pt}%
\pgfpathmoveto{\pgfqpoint{1.708443in}{0.744273in}}%
\pgfpathcurveto{\pgfqpoint{1.714889in}{0.744273in}}{\pgfqpoint{1.721071in}{0.746833in}}{\pgfqpoint{1.725629in}{0.751391in}}%
\pgfpathcurveto{\pgfqpoint{1.730187in}{0.755949in}}{\pgfqpoint{1.732748in}{0.762132in}}{\pgfqpoint{1.732748in}{0.768578in}}%
\pgfpathcurveto{\pgfqpoint{1.732748in}{0.775024in}}{\pgfqpoint{1.730187in}{0.781207in}}{\pgfqpoint{1.725629in}{0.785765in}}%
\pgfpathcurveto{\pgfqpoint{1.721071in}{0.790323in}}{\pgfqpoint{1.714889in}{0.792884in}}{\pgfqpoint{1.708443in}{0.792884in}}%
\pgfpathcurveto{\pgfqpoint{1.701997in}{0.792884in}}{\pgfqpoint{1.695814in}{0.790323in}}{\pgfqpoint{1.691256in}{0.785765in}}%
\pgfpathcurveto{\pgfqpoint{1.686698in}{0.781207in}}{\pgfqpoint{1.684137in}{0.775024in}}{\pgfqpoint{1.684137in}{0.768578in}}%
\pgfpathcurveto{\pgfqpoint{1.684137in}{0.762132in}}{\pgfqpoint{1.686698in}{0.755949in}}{\pgfqpoint{1.691256in}{0.751391in}}%
\pgfpathcurveto{\pgfqpoint{1.695814in}{0.746833in}}{\pgfqpoint{1.701997in}{0.744273in}}{\pgfqpoint{1.708443in}{0.744273in}}%
\pgfpathclose%
\pgfusepath{stroke}%
\end{pgfscope}%
\begin{pgfscope}%
\pgfpathrectangle{\pgfqpoint{0.383193in}{0.383578in}}{\pgfqpoint{2.325000in}{2.310000in}}%
\pgfusepath{clip}%
\pgfsetbuttcap%
\pgfsetroundjoin%
\pgfsetlinewidth{1.405250pt}%
\definecolor{currentstroke}{rgb}{0.000000,0.356863,0.509804}%
\pgfsetstrokecolor{currentstroke}%
\pgfsetdash{}{0pt}%
\pgfpathmoveto{\pgfqpoint{2.163755in}{0.936773in}}%
\pgfpathcurveto{\pgfqpoint{2.170201in}{0.936773in}}{\pgfqpoint{2.176384in}{0.939333in}}{\pgfqpoint{2.180942in}{0.943891in}}%
\pgfpathcurveto{\pgfqpoint{2.185500in}{0.948449in}}{\pgfqpoint{2.188061in}{0.954632in}}{\pgfqpoint{2.188061in}{0.961078in}}%
\pgfpathcurveto{\pgfqpoint{2.188061in}{0.967524in}}{\pgfqpoint{2.185500in}{0.973707in}}{\pgfqpoint{2.180942in}{0.978265in}}%
\pgfpathcurveto{\pgfqpoint{2.176384in}{0.982823in}}{\pgfqpoint{2.170201in}{0.985384in}}{\pgfqpoint{2.163755in}{0.985384in}}%
\pgfpathcurveto{\pgfqpoint{2.157309in}{0.985384in}}{\pgfqpoint{2.151127in}{0.982823in}}{\pgfqpoint{2.146569in}{0.978265in}}%
\pgfpathcurveto{\pgfqpoint{2.142011in}{0.973707in}}{\pgfqpoint{2.139450in}{0.967524in}}{\pgfqpoint{2.139450in}{0.961078in}}%
\pgfpathcurveto{\pgfqpoint{2.139450in}{0.954632in}}{\pgfqpoint{2.142011in}{0.948449in}}{\pgfqpoint{2.146569in}{0.943891in}}%
\pgfpathcurveto{\pgfqpoint{2.151127in}{0.939333in}}{\pgfqpoint{2.157309in}{0.936773in}}{\pgfqpoint{2.163755in}{0.936773in}}%
\pgfpathclose%
\pgfusepath{stroke}%
\end{pgfscope}%
\begin{pgfscope}%
\pgfpathrectangle{\pgfqpoint{0.383193in}{0.383578in}}{\pgfqpoint{2.325000in}{2.310000in}}%
\pgfusepath{clip}%
\pgfsetbuttcap%
\pgfsetroundjoin%
\pgfsetlinewidth{1.405250pt}%
\definecolor{currentstroke}{rgb}{0.000000,0.356863,0.509804}%
\pgfsetstrokecolor{currentstroke}%
\pgfsetdash{}{0pt}%
\pgfpathmoveto{\pgfqpoint{5.842099in}{1.129273in}}%
\pgfpathcurveto{\pgfqpoint{5.848545in}{1.129273in}}{\pgfqpoint{5.854728in}{1.131833in}}{\pgfqpoint{5.859286in}{1.136391in}}%
\pgfpathcurveto{\pgfqpoint{5.863844in}{1.140949in}}{\pgfqpoint{5.866405in}{1.147132in}}{\pgfqpoint{5.866405in}{1.153578in}}%
\pgfpathcurveto{\pgfqpoint{5.866405in}{1.160024in}}{\pgfqpoint{5.863844in}{1.166207in}}{\pgfqpoint{5.859286in}{1.170765in}}%
\pgfpathcurveto{\pgfqpoint{5.854728in}{1.175323in}}{\pgfqpoint{5.848545in}{1.177884in}}{\pgfqpoint{5.842099in}{1.177884in}}%
\pgfpathcurveto{\pgfqpoint{5.835653in}{1.177884in}}{\pgfqpoint{5.829470in}{1.175323in}}{\pgfqpoint{5.824912in}{1.170765in}}%
\pgfpathcurveto{\pgfqpoint{5.820354in}{1.166207in}}{\pgfqpoint{5.817793in}{1.160024in}}{\pgfqpoint{5.817793in}{1.153578in}}%
\pgfpathcurveto{\pgfqpoint{5.817793in}{1.147132in}}{\pgfqpoint{5.820354in}{1.140949in}}{\pgfqpoint{5.824912in}{1.136391in}}%
\pgfpathcurveto{\pgfqpoint{5.829470in}{1.131833in}}{\pgfqpoint{5.835653in}{1.129273in}}{\pgfqpoint{5.842099in}{1.129273in}}%
\pgfpathclose%
\pgfusepath{stroke}%
\end{pgfscope}%
\begin{pgfscope}%
\pgfpathrectangle{\pgfqpoint{0.383193in}{0.383578in}}{\pgfqpoint{2.325000in}{2.310000in}}%
\pgfusepath{clip}%
\pgfsetbuttcap%
\pgfsetroundjoin%
\pgfsetlinewidth{1.405250pt}%
\definecolor{currentstroke}{rgb}{0.000000,0.356863,0.509804}%
\pgfsetstrokecolor{currentstroke}%
\pgfsetdash{}{0pt}%
\pgfpathmoveto{\pgfqpoint{4.095443in}{1.706773in}}%
\pgfpathcurveto{\pgfqpoint{4.101889in}{1.706773in}}{\pgfqpoint{4.108071in}{1.709333in}}{\pgfqpoint{4.112629in}{1.713891in}}%
\pgfpathcurveto{\pgfqpoint{4.117187in}{1.718449in}}{\pgfqpoint{4.119748in}{1.724632in}}{\pgfqpoint{4.119748in}{1.731078in}}%
\pgfpathcurveto{\pgfqpoint{4.119748in}{1.737524in}}{\pgfqpoint{4.117187in}{1.743707in}}{\pgfqpoint{4.112629in}{1.748265in}}%
\pgfpathcurveto{\pgfqpoint{4.108071in}{1.752823in}}{\pgfqpoint{4.101889in}{1.755384in}}{\pgfqpoint{4.095443in}{1.755384in}}%
\pgfpathcurveto{\pgfqpoint{4.088997in}{1.755384in}}{\pgfqpoint{4.082814in}{1.752823in}}{\pgfqpoint{4.078256in}{1.748265in}}%
\pgfpathcurveto{\pgfqpoint{4.073698in}{1.743707in}}{\pgfqpoint{4.071137in}{1.737524in}}{\pgfqpoint{4.071137in}{1.731078in}}%
\pgfpathcurveto{\pgfqpoint{4.071137in}{1.724632in}}{\pgfqpoint{4.073698in}{1.718449in}}{\pgfqpoint{4.078256in}{1.713891in}}%
\pgfpathcurveto{\pgfqpoint{4.082814in}{1.709333in}}{\pgfqpoint{4.088997in}{1.706773in}}{\pgfqpoint{4.095443in}{1.706773in}}%
\pgfpathclose%
\pgfusepath{stroke}%
\end{pgfscope}%
\begin{pgfscope}%
\pgfpathrectangle{\pgfqpoint{0.383193in}{0.383578in}}{\pgfqpoint{2.325000in}{2.310000in}}%
\pgfusepath{clip}%
\pgfsetbuttcap%
\pgfsetroundjoin%
\pgfsetlinewidth{1.405250pt}%
\definecolor{currentstroke}{rgb}{0.000000,0.356863,0.509804}%
\pgfsetstrokecolor{currentstroke}%
\pgfsetdash{}{0pt}%
\pgfpathmoveto{\pgfqpoint{-1.236557in}{2.091773in}}%
\pgfpathcurveto{\pgfqpoint{-1.230111in}{2.091773in}}{\pgfqpoint{-1.223929in}{2.094333in}}{\pgfqpoint{-1.219371in}{2.098891in}}%
\pgfpathcurveto{\pgfqpoint{-1.214813in}{2.103449in}}{\pgfqpoint{-1.212252in}{2.109632in}}{\pgfqpoint{-1.212252in}{2.116078in}}%
\pgfpathcurveto{\pgfqpoint{-1.212252in}{2.122524in}}{\pgfqpoint{-1.214813in}{2.128707in}}{\pgfqpoint{-1.219371in}{2.133265in}}%
\pgfpathcurveto{\pgfqpoint{-1.223929in}{2.137823in}}{\pgfqpoint{-1.230111in}{2.140384in}}{\pgfqpoint{-1.236557in}{2.140384in}}%
\pgfpathcurveto{\pgfqpoint{-1.243003in}{2.140384in}}{\pgfqpoint{-1.249186in}{2.137823in}}{\pgfqpoint{-1.253744in}{2.133265in}}%
\pgfpathcurveto{\pgfqpoint{-1.258302in}{2.128707in}}{\pgfqpoint{-1.260863in}{2.122524in}}{\pgfqpoint{-1.260863in}{2.116078in}}%
\pgfpathcurveto{\pgfqpoint{-1.260863in}{2.109632in}}{\pgfqpoint{-1.258302in}{2.103449in}}{\pgfqpoint{-1.253744in}{2.098891in}}%
\pgfpathcurveto{\pgfqpoint{-1.249186in}{2.094333in}}{\pgfqpoint{-1.243003in}{2.091773in}}{\pgfqpoint{-1.236557in}{2.091773in}}%
\pgfpathclose%
\pgfusepath{stroke}%
\end{pgfscope}%
\begin{pgfscope}%
\pgfpathrectangle{\pgfqpoint{0.383193in}{0.383578in}}{\pgfqpoint{2.325000in}{2.310000in}}%
\pgfusepath{clip}%
\pgfsetbuttcap%
\pgfsetroundjoin%
\pgfsetlinewidth{1.405250pt}%
\definecolor{currentstroke}{rgb}{0.000000,0.356863,0.509804}%
\pgfsetstrokecolor{currentstroke}%
\pgfsetdash{}{0pt}%
\pgfpathmoveto{\pgfqpoint{2.123068in}{2.476773in}}%
\pgfpathcurveto{\pgfqpoint{2.129514in}{2.476773in}}{\pgfqpoint{2.135696in}{2.479333in}}{\pgfqpoint{2.140254in}{2.483891in}}%
\pgfpathcurveto{\pgfqpoint{2.144812in}{2.488449in}}{\pgfqpoint{2.147373in}{2.494632in}}{\pgfqpoint{2.147373in}{2.501078in}}%
\pgfpathcurveto{\pgfqpoint{2.147373in}{2.507524in}}{\pgfqpoint{2.144812in}{2.513707in}}{\pgfqpoint{2.140254in}{2.518265in}}%
\pgfpathcurveto{\pgfqpoint{2.135696in}{2.522823in}}{\pgfqpoint{2.129514in}{2.525384in}}{\pgfqpoint{2.123068in}{2.525384in}}%
\pgfpathcurveto{\pgfqpoint{2.116622in}{2.525384in}}{\pgfqpoint{2.110439in}{2.522823in}}{\pgfqpoint{2.105881in}{2.518265in}}%
\pgfpathcurveto{\pgfqpoint{2.101323in}{2.513707in}}{\pgfqpoint{2.098762in}{2.507524in}}{\pgfqpoint{2.098762in}{2.501078in}}%
\pgfpathcurveto{\pgfqpoint{2.098762in}{2.494632in}}{\pgfqpoint{2.101323in}{2.488449in}}{\pgfqpoint{2.105881in}{2.483891in}}%
\pgfpathcurveto{\pgfqpoint{2.110439in}{2.479333in}}{\pgfqpoint{2.116622in}{2.476773in}}{\pgfqpoint{2.123068in}{2.476773in}}%
\pgfpathclose%
\pgfusepath{stroke}%
\end{pgfscope}%
\begin{pgfscope}%
\pgfpathrectangle{\pgfqpoint{0.383193in}{0.383578in}}{\pgfqpoint{2.325000in}{2.310000in}}%
\pgfusepath{clip}%
\pgfsetbuttcap%
\pgfsetroundjoin%
\pgfsetlinewidth{1.405250pt}%
\definecolor{currentstroke}{rgb}{0.000000,0.356863,0.509804}%
\pgfsetstrokecolor{currentstroke}%
\pgfsetdash{}{0pt}%
\pgfpathmoveto{\pgfqpoint{-0.811276in}{2.861773in}}%
\pgfpathcurveto{\pgfqpoint{-0.804830in}{2.861773in}}{\pgfqpoint{-0.798647in}{2.864333in}}{\pgfqpoint{-0.794089in}{2.868891in}}%
\pgfpathcurveto{\pgfqpoint{-0.789531in}{2.873449in}}{\pgfqpoint{-0.786970in}{2.879632in}}{\pgfqpoint{-0.786970in}{2.886078in}}%
\pgfpathcurveto{\pgfqpoint{-0.786970in}{2.892524in}}{\pgfqpoint{-0.789531in}{2.898707in}}{\pgfqpoint{-0.794089in}{2.903265in}}%
\pgfpathcurveto{\pgfqpoint{-0.798647in}{2.907823in}}{\pgfqpoint{-0.804830in}{2.910384in}}{\pgfqpoint{-0.811276in}{2.910384in}}%
\pgfpathcurveto{\pgfqpoint{-0.817722in}{2.910384in}}{\pgfqpoint{-0.823905in}{2.907823in}}{\pgfqpoint{-0.828463in}{2.903265in}}%
\pgfpathcurveto{\pgfqpoint{-0.833021in}{2.898707in}}{\pgfqpoint{-0.835582in}{2.892524in}}{\pgfqpoint{-0.835582in}{2.886078in}}%
\pgfpathcurveto{\pgfqpoint{-0.835582in}{2.879632in}}{\pgfqpoint{-0.833021in}{2.873449in}}{\pgfqpoint{-0.828463in}{2.868891in}}%
\pgfpathcurveto{\pgfqpoint{-0.823905in}{2.864333in}}{\pgfqpoint{-0.817722in}{2.861773in}}{\pgfqpoint{-0.811276in}{2.861773in}}%
\pgfpathclose%
\pgfusepath{stroke}%
\end{pgfscope}%
\begin{pgfscope}%
\pgfpathrectangle{\pgfqpoint{0.383193in}{0.383578in}}{\pgfqpoint{2.325000in}{2.310000in}}%
\pgfusepath{clip}%
\pgfsetbuttcap%
\pgfsetroundjoin%
\pgfsetlinewidth{1.405250pt}%
\definecolor{currentstroke}{rgb}{0.000000,0.356863,0.509804}%
\pgfsetstrokecolor{currentstroke}%
\pgfsetdash{}{0pt}%
\pgfpathmoveto{\pgfqpoint{-1.118370in}{3.246773in}}%
\pgfpathcurveto{\pgfqpoint{-1.111924in}{3.246773in}}{\pgfqpoint{-1.105741in}{3.249333in}}{\pgfqpoint{-1.101183in}{3.253891in}}%
\pgfpathcurveto{\pgfqpoint{-1.096625in}{3.258449in}}{\pgfqpoint{-1.094064in}{3.264632in}}{\pgfqpoint{-1.094064in}{3.271078in}}%
\pgfpathcurveto{\pgfqpoint{-1.094064in}{3.277524in}}{\pgfqpoint{-1.096625in}{3.283707in}}{\pgfqpoint{-1.101183in}{3.288265in}}%
\pgfpathcurveto{\pgfqpoint{-1.105741in}{3.292823in}}{\pgfqpoint{-1.111924in}{3.295384in}}{\pgfqpoint{-1.118370in}{3.295384in}}%
\pgfpathcurveto{\pgfqpoint{-1.124816in}{3.295384in}}{\pgfqpoint{-1.130998in}{3.292823in}}{\pgfqpoint{-1.135556in}{3.288265in}}%
\pgfpathcurveto{\pgfqpoint{-1.140114in}{3.283707in}}{\pgfqpoint{-1.142675in}{3.277524in}}{\pgfqpoint{-1.142675in}{3.271078in}}%
\pgfpathcurveto{\pgfqpoint{-1.142675in}{3.264632in}}{\pgfqpoint{-1.140114in}{3.258449in}}{\pgfqpoint{-1.135556in}{3.253891in}}%
\pgfpathcurveto{\pgfqpoint{-1.130998in}{3.249333in}}{\pgfqpoint{-1.124816in}{3.246773in}}{\pgfqpoint{-1.118370in}{3.246773in}}%
\pgfpathclose%
\pgfusepath{stroke}%
\end{pgfscope}%
\begin{pgfscope}%
\pgfpathrectangle{\pgfqpoint{0.383193in}{0.383578in}}{\pgfqpoint{2.325000in}{2.310000in}}%
\pgfusepath{clip}%
\pgfsetbuttcap%
\pgfsetroundjoin%
\pgfsetlinewidth{1.405250pt}%
\definecolor{currentstroke}{rgb}{0.000000,0.356863,0.509804}%
\pgfsetstrokecolor{currentstroke}%
\pgfsetdash{}{0pt}%
\pgfpathmoveto{\pgfqpoint{4.317287in}{3.439273in}}%
\pgfpathcurveto{\pgfqpoint{4.323732in}{3.439273in}}{\pgfqpoint{4.329915in}{3.441833in}}{\pgfqpoint{4.334473in}{3.446391in}}%
\pgfpathcurveto{\pgfqpoint{4.339031in}{3.450949in}}{\pgfqpoint{4.341592in}{3.457132in}}{\pgfqpoint{4.341592in}{3.463578in}}%
\pgfpathcurveto{\pgfqpoint{4.341592in}{3.470024in}}{\pgfqpoint{4.339031in}{3.476207in}}{\pgfqpoint{4.334473in}{3.480765in}}%
\pgfpathcurveto{\pgfqpoint{4.329915in}{3.485323in}}{\pgfqpoint{4.323732in}{3.487884in}}{\pgfqpoint{4.317287in}{3.487884in}}%
\pgfpathcurveto{\pgfqpoint{4.310841in}{3.487884in}}{\pgfqpoint{4.304658in}{3.485323in}}{\pgfqpoint{4.300100in}{3.480765in}}%
\pgfpathcurveto{\pgfqpoint{4.295542in}{3.476207in}}{\pgfqpoint{4.292981in}{3.470024in}}{\pgfqpoint{4.292981in}{3.463578in}}%
\pgfpathcurveto{\pgfqpoint{4.292981in}{3.457132in}}{\pgfqpoint{4.295542in}{3.450949in}}{\pgfqpoint{4.300100in}{3.446391in}}%
\pgfpathcurveto{\pgfqpoint{4.304658in}{3.441833in}}{\pgfqpoint{4.310841in}{3.439273in}}{\pgfqpoint{4.317287in}{3.439273in}}%
\pgfpathclose%
\pgfusepath{stroke}%
\end{pgfscope}%
\begin{pgfscope}%
\pgfpathrectangle{\pgfqpoint{0.383193in}{0.383578in}}{\pgfqpoint{2.325000in}{2.310000in}}%
\pgfusepath{clip}%
\pgfsetbuttcap%
\pgfsetroundjoin%
\pgfsetlinewidth{1.405250pt}%
\definecolor{currentstroke}{rgb}{0.000000,0.356863,0.509804}%
\pgfsetstrokecolor{currentstroke}%
\pgfsetdash{}{0pt}%
\pgfpathmoveto{\pgfqpoint{5.498193in}{3.631773in}}%
\pgfpathcurveto{\pgfqpoint{5.504639in}{3.631773in}}{\pgfqpoint{5.510821in}{3.634333in}}{\pgfqpoint{5.515379in}{3.638891in}}%
\pgfpathcurveto{\pgfqpoint{5.519937in}{3.643449in}}{\pgfqpoint{5.522498in}{3.649632in}}{\pgfqpoint{5.522498in}{3.656078in}}%
\pgfpathcurveto{\pgfqpoint{5.522498in}{3.662524in}}{\pgfqpoint{5.519937in}{3.668707in}}{\pgfqpoint{5.515379in}{3.673265in}}%
\pgfpathcurveto{\pgfqpoint{5.510821in}{3.677823in}}{\pgfqpoint{5.504639in}{3.680384in}}{\pgfqpoint{5.498193in}{3.680384in}}%
\pgfpathcurveto{\pgfqpoint{5.491747in}{3.680384in}}{\pgfqpoint{5.485564in}{3.677823in}}{\pgfqpoint{5.481006in}{3.673265in}}%
\pgfpathcurveto{\pgfqpoint{5.476448in}{3.668707in}}{\pgfqpoint{5.473887in}{3.662524in}}{\pgfqpoint{5.473887in}{3.656078in}}%
\pgfpathcurveto{\pgfqpoint{5.473887in}{3.649632in}}{\pgfqpoint{5.476448in}{3.643449in}}{\pgfqpoint{5.481006in}{3.638891in}}%
\pgfpathcurveto{\pgfqpoint{5.485564in}{3.634333in}}{\pgfqpoint{5.491747in}{3.631773in}}{\pgfqpoint{5.498193in}{3.631773in}}%
\pgfpathclose%
\pgfusepath{stroke}%
\end{pgfscope}%
\begin{pgfscope}%
\pgfpathrectangle{\pgfqpoint{0.383193in}{0.383578in}}{\pgfqpoint{2.325000in}{2.310000in}}%
\pgfusepath{clip}%
\pgfsetbuttcap%
\pgfsetroundjoin%
\pgfsetlinewidth{1.405250pt}%
\definecolor{currentstroke}{rgb}{0.000000,0.356863,0.509804}%
\pgfsetstrokecolor{currentstroke}%
\pgfsetdash{}{0pt}%
\pgfpathmoveto{\pgfqpoint{1.132037in}{4.401773in}}%
\pgfpathcurveto{\pgfqpoint{1.138482in}{4.401773in}}{\pgfqpoint{1.144665in}{4.404333in}}{\pgfqpoint{1.149223in}{4.408891in}}%
\pgfpathcurveto{\pgfqpoint{1.153781in}{4.413449in}}{\pgfqpoint{1.156342in}{4.419632in}}{\pgfqpoint{1.156342in}{4.426078in}}%
\pgfpathcurveto{\pgfqpoint{1.156342in}{4.432524in}}{\pgfqpoint{1.153781in}{4.438707in}}{\pgfqpoint{1.149223in}{4.443265in}}%
\pgfpathcurveto{\pgfqpoint{1.144665in}{4.447823in}}{\pgfqpoint{1.138482in}{4.450384in}}{\pgfqpoint{1.132037in}{4.450384in}}%
\pgfpathcurveto{\pgfqpoint{1.125591in}{4.450384in}}{\pgfqpoint{1.119408in}{4.447823in}}{\pgfqpoint{1.114850in}{4.443265in}}%
\pgfpathcurveto{\pgfqpoint{1.110292in}{4.438707in}}{\pgfqpoint{1.107731in}{4.432524in}}{\pgfqpoint{1.107731in}{4.426078in}}%
\pgfpathcurveto{\pgfqpoint{1.107731in}{4.419632in}}{\pgfqpoint{1.110292in}{4.413449in}}{\pgfqpoint{1.114850in}{4.408891in}}%
\pgfpathcurveto{\pgfqpoint{1.119408in}{4.404333in}}{\pgfqpoint{1.125591in}{4.401773in}}{\pgfqpoint{1.132037in}{4.401773in}}%
\pgfpathclose%
\pgfusepath{stroke}%
\end{pgfscope}%
\begin{pgfscope}%
\pgfpathrectangle{\pgfqpoint{0.383193in}{0.383578in}}{\pgfqpoint{2.325000in}{2.310000in}}%
\pgfusepath{clip}%
\pgfsetbuttcap%
\pgfsetroundjoin%
\pgfsetlinewidth{1.405250pt}%
\definecolor{currentstroke}{rgb}{0.000000,0.356863,0.509804}%
\pgfsetstrokecolor{currentstroke}%
\pgfsetdash{}{0pt}%
\pgfpathmoveto{\pgfqpoint{1.758818in}{4.594273in}}%
\pgfpathcurveto{\pgfqpoint{1.765264in}{4.594273in}}{\pgfqpoint{1.771446in}{4.596833in}}{\pgfqpoint{1.776004in}{4.601391in}}%
\pgfpathcurveto{\pgfqpoint{1.780562in}{4.605949in}}{\pgfqpoint{1.783123in}{4.612132in}}{\pgfqpoint{1.783123in}{4.618578in}}%
\pgfpathcurveto{\pgfqpoint{1.783123in}{4.625024in}}{\pgfqpoint{1.780562in}{4.631207in}}{\pgfqpoint{1.776004in}{4.635765in}}%
\pgfpathcurveto{\pgfqpoint{1.771446in}{4.640323in}}{\pgfqpoint{1.765264in}{4.642884in}}{\pgfqpoint{1.758818in}{4.642884in}}%
\pgfpathcurveto{\pgfqpoint{1.752372in}{4.642884in}}{\pgfqpoint{1.746189in}{4.640323in}}{\pgfqpoint{1.741631in}{4.635765in}}%
\pgfpathcurveto{\pgfqpoint{1.737073in}{4.631207in}}{\pgfqpoint{1.734512in}{4.625024in}}{\pgfqpoint{1.734512in}{4.618578in}}%
\pgfpathcurveto{\pgfqpoint{1.734512in}{4.612132in}}{\pgfqpoint{1.737073in}{4.605949in}}{\pgfqpoint{1.741631in}{4.601391in}}%
\pgfpathcurveto{\pgfqpoint{1.746189in}{4.596833in}}{\pgfqpoint{1.752372in}{4.594273in}}{\pgfqpoint{1.758818in}{4.594273in}}%
\pgfpathclose%
\pgfusepath{stroke}%
\end{pgfscope}%
\begin{pgfscope}%
\pgfpathrectangle{\pgfqpoint{0.383193in}{0.383578in}}{\pgfqpoint{2.325000in}{2.310000in}}%
\pgfusepath{clip}%
\pgfsetbuttcap%
\pgfsetroundjoin%
\pgfsetlinewidth{1.405250pt}%
\definecolor{currentstroke}{rgb}{0.000000,0.356863,0.509804}%
\pgfsetstrokecolor{currentstroke}%
\pgfsetdash{}{0pt}%
\pgfpathmoveto{\pgfqpoint{2.032005in}{4.979273in}}%
\pgfpathcurveto{\pgfqpoint{2.038451in}{4.979273in}}{\pgfqpoint{2.044634in}{4.981833in}}{\pgfqpoint{2.049192in}{4.986391in}}%
\pgfpathcurveto{\pgfqpoint{2.053750in}{4.990949in}}{\pgfqpoint{2.056311in}{4.997132in}}{\pgfqpoint{2.056311in}{5.003578in}}%
\pgfpathcurveto{\pgfqpoint{2.056311in}{5.010024in}}{\pgfqpoint{2.053750in}{5.016207in}}{\pgfqpoint{2.049192in}{5.020765in}}%
\pgfpathcurveto{\pgfqpoint{2.044634in}{5.025323in}}{\pgfqpoint{2.038451in}{5.027884in}}{\pgfqpoint{2.032005in}{5.027884in}}%
\pgfpathcurveto{\pgfqpoint{2.025559in}{5.027884in}}{\pgfqpoint{2.019377in}{5.025323in}}{\pgfqpoint{2.014819in}{5.020765in}}%
\pgfpathcurveto{\pgfqpoint{2.010261in}{5.016207in}}{\pgfqpoint{2.007700in}{5.010024in}}{\pgfqpoint{2.007700in}{5.003578in}}%
\pgfpathcurveto{\pgfqpoint{2.007700in}{4.997132in}}{\pgfqpoint{2.010261in}{4.990949in}}{\pgfqpoint{2.014819in}{4.986391in}}%
\pgfpathcurveto{\pgfqpoint{2.019377in}{4.981833in}}{\pgfqpoint{2.025559in}{4.979273in}}{\pgfqpoint{2.032005in}{4.979273in}}%
\pgfpathclose%
\pgfusepath{stroke}%
\end{pgfscope}%
\begin{pgfscope}%
\pgfpathrectangle{\pgfqpoint{0.383193in}{0.383578in}}{\pgfqpoint{2.325000in}{2.310000in}}%
\pgfusepath{clip}%
\pgfsetbuttcap%
\pgfsetroundjoin%
\pgfsetlinewidth{1.405250pt}%
\definecolor{currentstroke}{rgb}{0.000000,0.356863,0.509804}%
\pgfsetstrokecolor{currentstroke}%
\pgfsetdash{}{0pt}%
\pgfpathmoveto{\pgfqpoint{3.946255in}{5.171773in}}%
\pgfpathcurveto{\pgfqpoint{3.952701in}{5.171773in}}{\pgfqpoint{3.958884in}{5.174333in}}{\pgfqpoint{3.963442in}{5.178891in}}%
\pgfpathcurveto{\pgfqpoint{3.968000in}{5.183449in}}{\pgfqpoint{3.970561in}{5.189632in}}{\pgfqpoint{3.970561in}{5.196078in}}%
\pgfpathcurveto{\pgfqpoint{3.970561in}{5.202524in}}{\pgfqpoint{3.968000in}{5.208707in}}{\pgfqpoint{3.963442in}{5.213265in}}%
\pgfpathcurveto{\pgfqpoint{3.958884in}{5.217823in}}{\pgfqpoint{3.952701in}{5.220384in}}{\pgfqpoint{3.946255in}{5.220384in}}%
\pgfpathcurveto{\pgfqpoint{3.939809in}{5.220384in}}{\pgfqpoint{3.933627in}{5.217823in}}{\pgfqpoint{3.929069in}{5.213265in}}%
\pgfpathcurveto{\pgfqpoint{3.924511in}{5.208707in}}{\pgfqpoint{3.921950in}{5.202524in}}{\pgfqpoint{3.921950in}{5.196078in}}%
\pgfpathcurveto{\pgfqpoint{3.921950in}{5.189632in}}{\pgfqpoint{3.924511in}{5.183449in}}{\pgfqpoint{3.929069in}{5.178891in}}%
\pgfpathcurveto{\pgfqpoint{3.933627in}{5.174333in}}{\pgfqpoint{3.939809in}{5.171773in}}{\pgfqpoint{3.946255in}{5.171773in}}%
\pgfpathclose%
\pgfusepath{stroke}%
\end{pgfscope}%
\begin{pgfscope}%
\pgfpathrectangle{\pgfqpoint{0.383193in}{0.383578in}}{\pgfqpoint{2.325000in}{2.310000in}}%
\pgfusepath{clip}%
\pgfsetbuttcap%
\pgfsetroundjoin%
\pgfsetlinewidth{1.405250pt}%
\definecolor{currentstroke}{rgb}{0.000000,0.356863,0.509804}%
\pgfsetstrokecolor{currentstroke}%
\pgfsetdash{}{0pt}%
\pgfpathmoveto{\pgfqpoint{-0.937213in}{5.941773in}}%
\pgfpathcurveto{\pgfqpoint{-0.930768in}{5.941773in}}{\pgfqpoint{-0.924585in}{5.944333in}}{\pgfqpoint{-0.920027in}{5.948891in}}%
\pgfpathcurveto{\pgfqpoint{-0.915469in}{5.953449in}}{\pgfqpoint{-0.912908in}{5.959632in}}{\pgfqpoint{-0.912908in}{5.966078in}}%
\pgfpathcurveto{\pgfqpoint{-0.912908in}{5.972524in}}{\pgfqpoint{-0.915469in}{5.978707in}}{\pgfqpoint{-0.920027in}{5.983265in}}%
\pgfpathcurveto{\pgfqpoint{-0.924585in}{5.987823in}}{\pgfqpoint{-0.930768in}{5.990384in}}{\pgfqpoint{-0.937213in}{5.990384in}}%
\pgfpathcurveto{\pgfqpoint{-0.943659in}{5.990384in}}{\pgfqpoint{-0.949842in}{5.987823in}}{\pgfqpoint{-0.954400in}{5.983265in}}%
\pgfpathcurveto{\pgfqpoint{-0.958958in}{5.978707in}}{\pgfqpoint{-0.961519in}{5.972524in}}{\pgfqpoint{-0.961519in}{5.966078in}}%
\pgfpathcurveto{\pgfqpoint{-0.961519in}{5.959632in}}{\pgfqpoint{-0.958958in}{5.953449in}}{\pgfqpoint{-0.954400in}{5.948891in}}%
\pgfpathcurveto{\pgfqpoint{-0.949842in}{5.944333in}}{\pgfqpoint{-0.943659in}{5.941773in}}{\pgfqpoint{-0.937213in}{5.941773in}}%
\pgfpathclose%
\pgfusepath{stroke}%
\end{pgfscope}%
\begin{pgfscope}%
\pgfpathrectangle{\pgfqpoint{0.383193in}{0.383578in}}{\pgfqpoint{2.325000in}{2.310000in}}%
\pgfusepath{clip}%
\pgfsetbuttcap%
\pgfsetroundjoin%
\pgfsetlinewidth{1.405250pt}%
\definecolor{currentstroke}{rgb}{0.000000,0.356863,0.509804}%
\pgfsetstrokecolor{currentstroke}%
\pgfsetdash{}{0pt}%
\pgfpathmoveto{\pgfqpoint{5.489474in}{6.134273in}}%
\pgfpathcurveto{\pgfqpoint{5.495920in}{6.134273in}}{\pgfqpoint{5.502103in}{6.136833in}}{\pgfqpoint{5.506661in}{6.141391in}}%
\pgfpathcurveto{\pgfqpoint{5.511219in}{6.145949in}}{\pgfqpoint{5.513780in}{6.152132in}}{\pgfqpoint{5.513780in}{6.158578in}}%
\pgfpathcurveto{\pgfqpoint{5.513780in}{6.165024in}}{\pgfqpoint{5.511219in}{6.171207in}}{\pgfqpoint{5.506661in}{6.175765in}}%
\pgfpathcurveto{\pgfqpoint{5.502103in}{6.180323in}}{\pgfqpoint{5.495920in}{6.182884in}}{\pgfqpoint{5.489474in}{6.182884in}}%
\pgfpathcurveto{\pgfqpoint{5.483028in}{6.182884in}}{\pgfqpoint{5.476845in}{6.180323in}}{\pgfqpoint{5.472287in}{6.175765in}}%
\pgfpathcurveto{\pgfqpoint{5.467729in}{6.171207in}}{\pgfqpoint{5.465168in}{6.165024in}}{\pgfqpoint{5.465168in}{6.158578in}}%
\pgfpathcurveto{\pgfqpoint{5.465168in}{6.152132in}}{\pgfqpoint{5.467729in}{6.145949in}}{\pgfqpoint{5.472287in}{6.141391in}}%
\pgfpathcurveto{\pgfqpoint{5.476845in}{6.136833in}}{\pgfqpoint{5.483028in}{6.134273in}}{\pgfqpoint{5.489474in}{6.134273in}}%
\pgfpathclose%
\pgfusepath{stroke}%
\end{pgfscope}%
\begin{pgfscope}%
\pgfpathrectangle{\pgfqpoint{0.383193in}{0.383578in}}{\pgfqpoint{2.325000in}{2.310000in}}%
\pgfusepath{clip}%
\pgfsetbuttcap%
\pgfsetroundjoin%
\pgfsetlinewidth{1.405250pt}%
\definecolor{currentstroke}{rgb}{0.000000,0.356863,0.509804}%
\pgfsetstrokecolor{currentstroke}%
\pgfsetdash{}{0pt}%
\pgfpathmoveto{\pgfqpoint{2.969755in}{7.096773in}}%
\pgfpathcurveto{\pgfqpoint{2.976201in}{7.096773in}}{\pgfqpoint{2.982384in}{7.099333in}}{\pgfqpoint{2.986942in}{7.103891in}}%
\pgfpathcurveto{\pgfqpoint{2.991500in}{7.108449in}}{\pgfqpoint{2.994061in}{7.114632in}}{\pgfqpoint{2.994061in}{7.121078in}}%
\pgfpathcurveto{\pgfqpoint{2.994061in}{7.127524in}}{\pgfqpoint{2.991500in}{7.133707in}}{\pgfqpoint{2.986942in}{7.138265in}}%
\pgfpathcurveto{\pgfqpoint{2.982384in}{7.142823in}}{\pgfqpoint{2.976201in}{7.145384in}}{\pgfqpoint{2.969755in}{7.145384in}}%
\pgfpathcurveto{\pgfqpoint{2.963309in}{7.145384in}}{\pgfqpoint{2.957127in}{7.142823in}}{\pgfqpoint{2.952569in}{7.138265in}}%
\pgfpathcurveto{\pgfqpoint{2.948011in}{7.133707in}}{\pgfqpoint{2.945450in}{7.127524in}}{\pgfqpoint{2.945450in}{7.121078in}}%
\pgfpathcurveto{\pgfqpoint{2.945450in}{7.114632in}}{\pgfqpoint{2.948011in}{7.108449in}}{\pgfqpoint{2.952569in}{7.103891in}}%
\pgfpathcurveto{\pgfqpoint{2.957127in}{7.099333in}}{\pgfqpoint{2.963309in}{7.096773in}}{\pgfqpoint{2.969755in}{7.096773in}}%
\pgfpathclose%
\pgfusepath{stroke}%
\end{pgfscope}%
\begin{pgfscope}%
\pgfpathrectangle{\pgfqpoint{0.383193in}{0.383578in}}{\pgfqpoint{2.325000in}{2.310000in}}%
\pgfusepath{clip}%
\pgfsetbuttcap%
\pgfsetroundjoin%
\definecolor{currentfill}{rgb}{0.490196,0.588235,0.431373}%
\pgfsetfillcolor{currentfill}%
\pgfsetlinewidth{2.007500pt}%
\definecolor{currentstroke}{rgb}{0.490196,0.588235,0.431373}%
\pgfsetstrokecolor{currentstroke}%
\pgfsetdash{}{0pt}%
\pgfsys@defobject{currentmarker}{\pgfqpoint{-0.006944in}{-0.006944in}}{\pgfqpoint{0.006944in}{0.006944in}}{%
\pgfpathmoveto{\pgfqpoint{0.000000in}{-0.006944in}}%
\pgfpathcurveto{\pgfqpoint{0.001842in}{-0.006944in}}{\pgfqpoint{0.003608in}{-0.006213in}}{\pgfqpoint{0.004910in}{-0.004910in}}%
\pgfpathcurveto{\pgfqpoint{0.006213in}{-0.003608in}}{\pgfqpoint{0.006944in}{-0.001842in}}{\pgfqpoint{0.006944in}{0.000000in}}%
\pgfpathcurveto{\pgfqpoint{0.006944in}{0.001842in}}{\pgfqpoint{0.006213in}{0.003608in}}{\pgfqpoint{0.004910in}{0.004910in}}%
\pgfpathcurveto{\pgfqpoint{0.003608in}{0.006213in}}{\pgfqpoint{0.001842in}{0.006944in}}{\pgfqpoint{0.000000in}{0.006944in}}%
\pgfpathcurveto{\pgfqpoint{-0.001842in}{0.006944in}}{\pgfqpoint{-0.003608in}{0.006213in}}{\pgfqpoint{-0.004910in}{0.004910in}}%
\pgfpathcurveto{\pgfqpoint{-0.006213in}{0.003608in}}{\pgfqpoint{-0.006944in}{0.001842in}}{\pgfqpoint{-0.006944in}{0.000000in}}%
\pgfpathcurveto{\pgfqpoint{-0.006944in}{-0.001842in}}{\pgfqpoint{-0.006213in}{-0.003608in}}{\pgfqpoint{-0.004910in}{-0.004910in}}%
\pgfpathcurveto{\pgfqpoint{-0.003608in}{-0.006213in}}{\pgfqpoint{-0.001842in}{-0.006944in}}{\pgfqpoint{0.000000in}{-0.006944in}}%
\pgfpathclose%
\pgfusepath{stroke,fill}%
}%
\begin{pgfscope}%
\pgfsys@transformshift{-1.909838in}{-11.166422in}%
\pgfsys@useobject{currentmarker}{}%
\end{pgfscope}%
\begin{pgfscope}%
\pgfsys@transformshift{0.737755in}{-10.011422in}%
\pgfsys@useobject{currentmarker}{}%
\end{pgfscope}%
\begin{pgfscope}%
\pgfsys@transformshift{-2.009620in}{-9.818922in}%
\pgfsys@useobject{currentmarker}{}%
\end{pgfscope}%
\begin{pgfscope}%
\pgfsys@transformshift{5.416818in}{-9.433922in}%
\pgfsys@useobject{currentmarker}{}%
\end{pgfscope}%
\begin{pgfscope}%
\pgfsys@transformshift{1.163037in}{-9.241422in}%
\pgfsys@useobject{currentmarker}{}%
\end{pgfscope}%
\begin{pgfscope}%
\pgfsys@transformshift{-1.340213in}{-8.471422in}%
\pgfsys@useobject{currentmarker}{}%
\end{pgfscope}%
\begin{pgfscope}%
\pgfsys@transformshift{6.130787in}{-8.278922in}%
\pgfsys@useobject{currentmarker}{}%
\end{pgfscope}%
\begin{pgfscope}%
\pgfsys@transformshift{0.846255in}{-7.893922in}%
\pgfsys@useobject{currentmarker}{}%
\end{pgfscope}%
\begin{pgfscope}%
\pgfsys@transformshift{6.183099in}{-7.701422in}%
\pgfsys@useobject{currentmarker}{}%
\end{pgfscope}%
\begin{pgfscope}%
\pgfsys@transformshift{-2.006713in}{-7.316422in}%
\pgfsys@useobject{currentmarker}{}%
\end{pgfscope}%
\begin{pgfscope}%
\pgfsys@transformshift{-1.430307in}{-7.123922in}%
\pgfsys@useobject{currentmarker}{}%
\end{pgfscope}%
\begin{pgfscope}%
\pgfsys@transformshift{3.967568in}{-6.546422in}%
\pgfsys@useobject{currentmarker}{}%
\end{pgfscope}%
\begin{pgfscope}%
\pgfsys@transformshift{0.274693in}{-6.353922in}%
\pgfsys@useobject{currentmarker}{}%
\end{pgfscope}%
\begin{pgfscope}%
\pgfsys@transformshift{0.443255in}{-5.968922in}%
\pgfsys@useobject{currentmarker}{}%
\end{pgfscope}%
\begin{pgfscope}%
\pgfsys@transformshift{1.420724in}{-5.776422in}%
\pgfsys@useobject{currentmarker}{}%
\end{pgfscope}%
\begin{pgfscope}%
\pgfsys@transformshift{5.633818in}{-5.198922in}%
\pgfsys@useobject{currentmarker}{}%
\end{pgfscope}%
\begin{pgfscope}%
\pgfsys@transformshift{-0.031432in}{-4.813922in}%
\pgfsys@useobject{currentmarker}{}%
\end{pgfscope}%
\begin{pgfscope}%
\pgfsys@transformshift{6.403005in}{-4.621422in}%
\pgfsys@useobject{currentmarker}{}%
\end{pgfscope}%
\begin{pgfscope}%
\pgfsys@transformshift{-0.345307in}{-4.428922in}%
\pgfsys@useobject{currentmarker}{}%
\end{pgfscope}%
\begin{pgfscope}%
\pgfsys@transformshift{4.741599in}{-4.236422in}%
\pgfsys@useobject{currentmarker}{}%
\end{pgfscope}%
\begin{pgfscope}%
\pgfsys@transformshift{-0.682432in}{-4.043922in}%
\pgfsys@useobject{currentmarker}{}%
\end{pgfscope}%
\begin{pgfscope}%
\pgfsys@transformshift{-2.666432in}{-3.851422in}%
\pgfsys@useobject{currentmarker}{}%
\end{pgfscope}%
\begin{pgfscope}%
\pgfsys@transformshift{5.039005in}{-3.658922in}%
\pgfsys@useobject{currentmarker}{}%
\end{pgfscope}%
\begin{pgfscope}%
\pgfsys@transformshift{-0.427651in}{-2.696422in}%
\pgfsys@useobject{currentmarker}{}%
\end{pgfscope}%
\begin{pgfscope}%
\pgfsys@transformshift{0.609880in}{-2.503922in}%
\pgfsys@useobject{currentmarker}{}%
\end{pgfscope}%
\begin{pgfscope}%
\pgfsys@transformshift{-2.637370in}{-2.311422in}%
\pgfsys@useobject{currentmarker}{}%
\end{pgfscope}%
\begin{pgfscope}%
\pgfsys@transformshift{-1.202651in}{-1.926422in}%
\pgfsys@useobject{currentmarker}{}%
\end{pgfscope}%
\begin{pgfscope}%
\pgfsys@transformshift{0.832693in}{-0.963922in}%
\pgfsys@useobject{currentmarker}{}%
\end{pgfscope}%
\begin{pgfscope}%
\pgfsys@transformshift{5.149443in}{-0.771422in}%
\pgfsys@useobject{currentmarker}{}%
\end{pgfscope}%
\begin{pgfscope}%
\pgfsys@transformshift{4.335693in}{-0.578922in}%
\pgfsys@useobject{currentmarker}{}%
\end{pgfscope}%
\begin{pgfscope}%
\pgfsys@transformshift{-1.232682in}{-0.386422in}%
\pgfsys@useobject{currentmarker}{}%
\end{pgfscope}%
\begin{pgfscope}%
\pgfsys@transformshift{-0.466401in}{0.191078in}%
\pgfsys@useobject{currentmarker}{}%
\end{pgfscope}%
\begin{pgfscope}%
\pgfsys@transformshift{-2.015432in}{0.383578in}%
\pgfsys@useobject{currentmarker}{}%
\end{pgfscope}%
\begin{pgfscope}%
\pgfsys@transformshift{0.651537in}{0.576078in}%
\pgfsys@useobject{currentmarker}{}%
\end{pgfscope}%
\begin{pgfscope}%
\pgfsys@transformshift{1.708443in}{0.768578in}%
\pgfsys@useobject{currentmarker}{}%
\end{pgfscope}%
\begin{pgfscope}%
\pgfsys@transformshift{2.163755in}{0.961078in}%
\pgfsys@useobject{currentmarker}{}%
\end{pgfscope}%
\begin{pgfscope}%
\pgfsys@transformshift{5.842099in}{1.153578in}%
\pgfsys@useobject{currentmarker}{}%
\end{pgfscope}%
\begin{pgfscope}%
\pgfsys@transformshift{1.176599in}{1.538578in}%
\pgfsys@useobject{currentmarker}{}%
\end{pgfscope}%
\begin{pgfscope}%
\pgfsys@transformshift{4.095443in}{1.731078in}%
\pgfsys@useobject{currentmarker}{}%
\end{pgfscope}%
\begin{pgfscope}%
\pgfsys@transformshift{-1.236557in}{2.116078in}%
\pgfsys@useobject{currentmarker}{}%
\end{pgfscope}%
\begin{pgfscope}%
\pgfsys@transformshift{1.324818in}{2.308578in}%
\pgfsys@useobject{currentmarker}{}%
\end{pgfscope}%
\begin{pgfscope}%
\pgfsys@transformshift{2.123068in}{2.501078in}%
\pgfsys@useobject{currentmarker}{}%
\end{pgfscope}%
\begin{pgfscope}%
\pgfsys@transformshift{-0.811276in}{2.886078in}%
\pgfsys@useobject{currentmarker}{}%
\end{pgfscope}%
\begin{pgfscope}%
\pgfsys@transformshift{3.182880in}{3.078578in}%
\pgfsys@useobject{currentmarker}{}%
\end{pgfscope}%
\begin{pgfscope}%
\pgfsys@transformshift{-1.118370in}{3.271078in}%
\pgfsys@useobject{currentmarker}{}%
\end{pgfscope}%
\begin{pgfscope}%
\pgfsys@transformshift{4.317287in}{3.463578in}%
\pgfsys@useobject{currentmarker}{}%
\end{pgfscope}%
\begin{pgfscope}%
\pgfsys@transformshift{5.498193in}{3.656078in}%
\pgfsys@useobject{currentmarker}{}%
\end{pgfscope}%
\begin{pgfscope}%
\pgfsys@transformshift{2.175380in}{4.233578in}%
\pgfsys@useobject{currentmarker}{}%
\end{pgfscope}%
\begin{pgfscope}%
\pgfsys@transformshift{1.132037in}{4.426078in}%
\pgfsys@useobject{currentmarker}{}%
\end{pgfscope}%
\begin{pgfscope}%
\pgfsys@transformshift{1.758818in}{4.618578in}%
\pgfsys@useobject{currentmarker}{}%
\end{pgfscope}%
\begin{pgfscope}%
\pgfsys@transformshift{2.053318in}{4.811078in}%
\pgfsys@useobject{currentmarker}{}%
\end{pgfscope}%
\begin{pgfscope}%
\pgfsys@transformshift{2.032005in}{5.003578in}%
\pgfsys@useobject{currentmarker}{}%
\end{pgfscope}%
\begin{pgfscope}%
\pgfsys@transformshift{3.946255in}{5.196078in}%
\pgfsys@useobject{currentmarker}{}%
\end{pgfscope}%
\begin{pgfscope}%
\pgfsys@transformshift{6.067818in}{5.581078in}%
\pgfsys@useobject{currentmarker}{}%
\end{pgfscope}%
\begin{pgfscope}%
\pgfsys@transformshift{-1.450651in}{5.773578in}%
\pgfsys@useobject{currentmarker}{}%
\end{pgfscope}%
\begin{pgfscope}%
\pgfsys@transformshift{-0.937213in}{5.966078in}%
\pgfsys@useobject{currentmarker}{}%
\end{pgfscope}%
\begin{pgfscope}%
\pgfsys@transformshift{5.489474in}{6.158578in}%
\pgfsys@useobject{currentmarker}{}%
\end{pgfscope}%
\begin{pgfscope}%
\pgfsys@transformshift{0.430662in}{6.543578in}%
\pgfsys@useobject{currentmarker}{}%
\end{pgfscope}%
\begin{pgfscope}%
\pgfsys@transformshift{-2.335120in}{6.736078in}%
\pgfsys@useobject{currentmarker}{}%
\end{pgfscope}%
\begin{pgfscope}%
\pgfsys@transformshift{2.969755in}{7.121078in}%
\pgfsys@useobject{currentmarker}{}%
\end{pgfscope}%
\end{pgfscope}%
\begin{pgfscope}%
\pgfsetrectcap%
\pgfsetmiterjoin%
\pgfsetlinewidth{0.501875pt}%
\definecolor{currentstroke}{rgb}{0.317647,0.317647,0.317647}%
\pgfsetstrokecolor{currentstroke}%
\pgfsetdash{}{0pt}%
\pgfpathmoveto{\pgfqpoint{0.383193in}{0.383578in}}%
\pgfpathlineto{\pgfqpoint{0.383193in}{2.693578in}}%
\pgfusepath{stroke}%
\end{pgfscope}%
\begin{pgfscope}%
\pgfsetrectcap%
\pgfsetmiterjoin%
\pgfsetlinewidth{0.501875pt}%
\definecolor{currentstroke}{rgb}{0.317647,0.317647,0.317647}%
\pgfsetstrokecolor{currentstroke}%
\pgfsetdash{}{0pt}%
\pgfpathmoveto{\pgfqpoint{0.383193in}{0.383578in}}%
\pgfpathlineto{\pgfqpoint{2.708193in}{0.383578in}}%
\pgfusepath{stroke}%
\end{pgfscope}%
\begin{pgfscope}%
\pgfsetbuttcap%
\pgfsetroundjoin%
\pgfsetlinewidth{1.405250pt}%
\definecolor{currentstroke}{rgb}{0.333333,0.333333,0.333333}%
\pgfsetstrokecolor{currentstroke}%
\pgfsetdash{}{0pt}%
\pgfpathmoveto{\pgfqpoint{2.485665in}{2.566291in}}%
\pgfpathcurveto{\pgfqpoint{2.501320in}{2.566291in}}{\pgfqpoint{2.516335in}{2.572510in}}{\pgfqpoint{2.527404in}{2.583579in}}%
\pgfpathcurveto{\pgfqpoint{2.538474in}{2.594649in}}{\pgfqpoint{2.544693in}{2.609664in}}{\pgfqpoint{2.544693in}{2.625318in}}%
\pgfpathcurveto{\pgfqpoint{2.544693in}{2.640973in}}{\pgfqpoint{2.538474in}{2.655988in}}{\pgfqpoint{2.527404in}{2.667057in}}%
\pgfpathcurveto{\pgfqpoint{2.516335in}{2.678127in}}{\pgfqpoint{2.501320in}{2.684346in}}{\pgfqpoint{2.485665in}{2.684346in}}%
\pgfpathcurveto{\pgfqpoint{2.470011in}{2.684346in}}{\pgfqpoint{2.454996in}{2.678127in}}{\pgfqpoint{2.443926in}{2.667057in}}%
\pgfpathcurveto{\pgfqpoint{2.432857in}{2.655988in}}{\pgfqpoint{2.426638in}{2.640973in}}{\pgfqpoint{2.426638in}{2.625318in}}%
\pgfpathcurveto{\pgfqpoint{2.426638in}{2.609664in}}{\pgfqpoint{2.432857in}{2.594649in}}{\pgfqpoint{2.443926in}{2.583579in}}%
\pgfpathcurveto{\pgfqpoint{2.454996in}{2.572510in}}{\pgfqpoint{2.470011in}{2.566291in}}{\pgfqpoint{2.485665in}{2.566291in}}%
\pgfpathclose%
\pgfusepath{stroke}%
\end{pgfscope}%
\begin{pgfscope}%
\definecolor{textcolor}{rgb}{0.000000,0.000000,0.000000}%
\pgfsetstrokecolor{textcolor}%
\pgfsetfillcolor{textcolor}%
\pgftext[x=2.568965in,y=2.601023in,left,base]{\color{textcolor}\rmfamily\fontsize{6.664000}{7.996800}\selectfont \(\displaystyle p_{1}\)}%
\end{pgfscope}%
\begin{pgfscope}%
\pgfsetbuttcap%
\pgfsetroundjoin%
\pgfsetlinewidth{1.405250pt}%
\definecolor{currentstroke}{rgb}{0.686275,0.352941,0.313725}%
\pgfsetstrokecolor{currentstroke}%
\pgfsetdash{}{0pt}%
\pgfpathmoveto{\pgfqpoint{2.485665in}{2.462936in}}%
\pgfpathcurveto{\pgfqpoint{2.496715in}{2.462936in}}{\pgfqpoint{2.507314in}{2.467327in}}{\pgfqpoint{2.515128in}{2.475140in}}%
\pgfpathcurveto{\pgfqpoint{2.522942in}{2.482954in}}{\pgfqpoint{2.527332in}{2.493553in}}{\pgfqpoint{2.527332in}{2.504603in}}%
\pgfpathcurveto{\pgfqpoint{2.527332in}{2.515653in}}{\pgfqpoint{2.522942in}{2.526252in}}{\pgfqpoint{2.515128in}{2.534066in}}%
\pgfpathcurveto{\pgfqpoint{2.507314in}{2.541879in}}{\pgfqpoint{2.496715in}{2.546270in}}{\pgfqpoint{2.485665in}{2.546270in}}%
\pgfpathcurveto{\pgfqpoint{2.474615in}{2.546270in}}{\pgfqpoint{2.464016in}{2.541879in}}{\pgfqpoint{2.456202in}{2.534066in}}%
\pgfpathcurveto{\pgfqpoint{2.448389in}{2.526252in}}{\pgfqpoint{2.443999in}{2.515653in}}{\pgfqpoint{2.443999in}{2.504603in}}%
\pgfpathcurveto{\pgfqpoint{2.443999in}{2.493553in}}{\pgfqpoint{2.448389in}{2.482954in}}{\pgfqpoint{2.456202in}{2.475140in}}%
\pgfpathcurveto{\pgfqpoint{2.464016in}{2.467327in}}{\pgfqpoint{2.474615in}{2.462936in}}{\pgfqpoint{2.485665in}{2.462936in}}%
\pgfpathclose%
\pgfusepath{stroke}%
\end{pgfscope}%
\begin{pgfscope}%
\definecolor{textcolor}{rgb}{0.000000,0.000000,0.000000}%
\pgfsetstrokecolor{textcolor}%
\pgfsetfillcolor{textcolor}%
\pgftext[x=2.568965in,y=2.480307in,left,base]{\color{textcolor}\rmfamily\fontsize{6.664000}{7.996800}\selectfont \(\displaystyle p_{2}\)}%
\end{pgfscope}%
\begin{pgfscope}%
\pgfsetbuttcap%
\pgfsetroundjoin%
\pgfsetlinewidth{1.405250pt}%
\definecolor{currentstroke}{rgb}{0.000000,0.356863,0.509804}%
\pgfsetstrokecolor{currentstroke}%
\pgfsetdash{}{0pt}%
\pgfpathmoveto{\pgfqpoint{2.485665in}{2.359582in}}%
\pgfpathcurveto{\pgfqpoint{2.492111in}{2.359582in}}{\pgfqpoint{2.498294in}{2.362143in}}{\pgfqpoint{2.502852in}{2.366701in}}%
\pgfpathcurveto{\pgfqpoint{2.507410in}{2.371259in}}{\pgfqpoint{2.509971in}{2.377442in}}{\pgfqpoint{2.509971in}{2.383888in}}%
\pgfpathcurveto{\pgfqpoint{2.509971in}{2.390334in}}{\pgfqpoint{2.507410in}{2.396516in}}{\pgfqpoint{2.502852in}{2.401074in}}%
\pgfpathcurveto{\pgfqpoint{2.498294in}{2.405632in}}{\pgfqpoint{2.492111in}{2.408193in}}{\pgfqpoint{2.485665in}{2.408193in}}%
\pgfpathcurveto{\pgfqpoint{2.479219in}{2.408193in}}{\pgfqpoint{2.473037in}{2.405632in}}{\pgfqpoint{2.468479in}{2.401074in}}%
\pgfpathcurveto{\pgfqpoint{2.463921in}{2.396516in}}{\pgfqpoint{2.461360in}{2.390334in}}{\pgfqpoint{2.461360in}{2.383888in}}%
\pgfpathcurveto{\pgfqpoint{2.461360in}{2.377442in}}{\pgfqpoint{2.463921in}{2.371259in}}{\pgfqpoint{2.468479in}{2.366701in}}%
\pgfpathcurveto{\pgfqpoint{2.473037in}{2.362143in}}{\pgfqpoint{2.479219in}{2.359582in}}{\pgfqpoint{2.485665in}{2.359582in}}%
\pgfpathclose%
\pgfusepath{stroke}%
\end{pgfscope}%
\begin{pgfscope}%
\definecolor{textcolor}{rgb}{0.000000,0.000000,0.000000}%
\pgfsetstrokecolor{textcolor}%
\pgfsetfillcolor{textcolor}%
\pgftext[x=2.568965in,y=2.359592in,left,base]{\color{textcolor}\rmfamily\fontsize{6.664000}{7.996800}\selectfont \(\displaystyle p_{3}\)}%
\end{pgfscope}%
\begin{pgfscope}%
\pgfsetbuttcap%
\pgfsetroundjoin%
\definecolor{currentfill}{rgb}{0.490196,0.588235,0.431373}%
\pgfsetfillcolor{currentfill}%
\pgfsetlinewidth{2.007500pt}%
\definecolor{currentstroke}{rgb}{0.490196,0.588235,0.431373}%
\pgfsetstrokecolor{currentstroke}%
\pgfsetdash{}{0pt}%
\pgfsys@defobject{currentmarker}{\pgfqpoint{-0.006944in}{-0.006944in}}{\pgfqpoint{0.006944in}{0.006944in}}{%
\pgfpathmoveto{\pgfqpoint{0.000000in}{-0.006944in}}%
\pgfpathcurveto{\pgfqpoint{0.001842in}{-0.006944in}}{\pgfqpoint{0.003608in}{-0.006213in}}{\pgfqpoint{0.004910in}{-0.004910in}}%
\pgfpathcurveto{\pgfqpoint{0.006213in}{-0.003608in}}{\pgfqpoint{0.006944in}{-0.001842in}}{\pgfqpoint{0.006944in}{0.000000in}}%
\pgfpathcurveto{\pgfqpoint{0.006944in}{0.001842in}}{\pgfqpoint{0.006213in}{0.003608in}}{\pgfqpoint{0.004910in}{0.004910in}}%
\pgfpathcurveto{\pgfqpoint{0.003608in}{0.006213in}}{\pgfqpoint{0.001842in}{0.006944in}}{\pgfqpoint{0.000000in}{0.006944in}}%
\pgfpathcurveto{\pgfqpoint{-0.001842in}{0.006944in}}{\pgfqpoint{-0.003608in}{0.006213in}}{\pgfqpoint{-0.004910in}{0.004910in}}%
\pgfpathcurveto{\pgfqpoint{-0.006213in}{0.003608in}}{\pgfqpoint{-0.006944in}{0.001842in}}{\pgfqpoint{-0.006944in}{0.000000in}}%
\pgfpathcurveto{\pgfqpoint{-0.006944in}{-0.001842in}}{\pgfqpoint{-0.006213in}{-0.003608in}}{\pgfqpoint{-0.004910in}{-0.004910in}}%
\pgfpathcurveto{\pgfqpoint{-0.003608in}{-0.006213in}}{\pgfqpoint{-0.001842in}{-0.006944in}}{\pgfqpoint{0.000000in}{-0.006944in}}%
\pgfpathclose%
\pgfusepath{stroke,fill}%
}%
\begin{pgfscope}%
\pgfsys@transformshift{2.485665in}{2.263173in}%
\pgfsys@useobject{currentmarker}{}%
\end{pgfscope}%
\end{pgfscope}%
\begin{pgfscope}%
\definecolor{textcolor}{rgb}{0.000000,0.000000,0.000000}%
\pgfsetstrokecolor{textcolor}%
\pgfsetfillcolor{textcolor}%
\pgftext[x=2.568965in,y=2.238877in,left,base]{\color{textcolor}\rmfamily\fontsize{6.664000}{7.996800}\selectfont \(\displaystyle p_{4}\)}%
\end{pgfscope}%
\end{pgfpicture}%
\makeatother%
\endgroup%

		\label{superspiketaskpicturesector}
	\end{subfigure}
	\begin{subfigure}{0.5\textwidth}
		\caption{}
		%% Creator: Matplotlib, PGF backend
%%
%% To include the figure in your LaTeX document, write
%%   \input{<filename>.pgf}
%%
%% Make sure the required packages are loaded in your preamble
%%   \usepackage{pgf}
%%
%% Figures using additional raster images can only be included by \input if
%% they are in the same directory as the main LaTeX file. For loading figures
%% from other directories you can use the `import` package
%%   \usepackage{import}
%% and then include the figures with
%%   \import{<path to file>}{<filename>.pgf}
%%
%% Matplotlib used the following preamble
%%   \usepackage{amsmath} \usepackage{pifont} \usepackage{xcolor} \definecolor{green}{HTML}{467821} \definecolor{red}{HTML}{CF4457} \usepackage[detect-all]{siunitx}
%%   \usepackage{fontspec}
%%
\begingroup%
\makeatletter%
\begin{pgfpicture}%
\pgfpathrectangle{\pgfpointorigin}{\pgfqpoint{2.918919in}{2.820261in}}%
\pgfusepath{use as bounding box, clip}%
\begin{pgfscope}%
\pgfsetbuttcap%
\pgfsetmiterjoin%
\pgfsetlinewidth{0.000000pt}%
\definecolor{currentstroke}{rgb}{0.000000,0.000000,0.000000}%
\pgfsetstrokecolor{currentstroke}%
\pgfsetstrokeopacity{0.000000}%
\pgfsetdash{}{0pt}%
\pgfpathmoveto{\pgfqpoint{0.000000in}{0.000000in}}%
\pgfpathlineto{\pgfqpoint{2.918919in}{0.000000in}}%
\pgfpathlineto{\pgfqpoint{2.918919in}{2.820261in}}%
\pgfpathlineto{\pgfqpoint{0.000000in}{2.820261in}}%
\pgfpathclose%
\pgfusepath{}%
\end{pgfscope}%
\begin{pgfscope}%
\pgfsetbuttcap%
\pgfsetmiterjoin%
\pgfsetlinewidth{0.000000pt}%
\definecolor{currentstroke}{rgb}{0.000000,0.000000,0.000000}%
\pgfsetstrokecolor{currentstroke}%
\pgfsetstrokeopacity{0.000000}%
\pgfsetdash{}{0pt}%
\pgfpathmoveto{\pgfqpoint{0.438556in}{0.383578in}}%
\pgfpathlineto{\pgfqpoint{2.763556in}{0.383578in}}%
\pgfpathlineto{\pgfqpoint{2.763556in}{2.693578in}}%
\pgfpathlineto{\pgfqpoint{0.438556in}{2.693578in}}%
\pgfpathclose%
\pgfusepath{}%
\end{pgfscope}%
\begin{pgfscope}%
\pgfsetbuttcap%
\pgfsetroundjoin%
\definecolor{currentfill}{rgb}{0.317647,0.317647,0.317647}%
\pgfsetfillcolor{currentfill}%
\pgfsetlinewidth{0.501875pt}%
\definecolor{currentstroke}{rgb}{0.317647,0.317647,0.317647}%
\pgfsetstrokecolor{currentstroke}%
\pgfsetdash{}{0pt}%
\pgfsys@defobject{currentmarker}{\pgfqpoint{0.000000in}{-0.020833in}}{\pgfqpoint{0.000000in}{0.000000in}}{%
\pgfpathmoveto{\pgfqpoint{0.000000in}{0.000000in}}%
\pgfpathlineto{\pgfqpoint{0.000000in}{-0.020833in}}%
\pgfusepath{stroke,fill}%
}%
\begin{pgfscope}%
\pgfsys@transformshift{0.438556in}{0.383578in}%
\pgfsys@useobject{currentmarker}{}%
\end{pgfscope}%
\end{pgfscope}%
\begin{pgfscope}%
\definecolor{textcolor}{rgb}{0.317647,0.317647,0.317647}%
\pgfsetstrokecolor{textcolor}%
\pgfsetfillcolor{textcolor}%
\pgftext[x=0.438556in,y=0.334967in,,top]{\color{textcolor}\rmfamily\fontsize{6.664000}{7.996800}\selectfont \(\displaystyle 0\)}%
\end{pgfscope}%
\begin{pgfscope}%
\pgfsetbuttcap%
\pgfsetroundjoin%
\definecolor{currentfill}{rgb}{0.317647,0.317647,0.317647}%
\pgfsetfillcolor{currentfill}%
\pgfsetlinewidth{0.501875pt}%
\definecolor{currentstroke}{rgb}{0.317647,0.317647,0.317647}%
\pgfsetstrokecolor{currentstroke}%
\pgfsetdash{}{0pt}%
\pgfsys@defobject{currentmarker}{\pgfqpoint{0.000000in}{-0.020833in}}{\pgfqpoint{0.000000in}{0.000000in}}{%
\pgfpathmoveto{\pgfqpoint{0.000000in}{0.000000in}}%
\pgfpathlineto{\pgfqpoint{0.000000in}{-0.020833in}}%
\pgfusepath{stroke,fill}%
}%
\begin{pgfscope}%
\pgfsys@transformshift{0.903556in}{0.383578in}%
\pgfsys@useobject{currentmarker}{}%
\end{pgfscope}%
\end{pgfscope}%
\begin{pgfscope}%
\definecolor{textcolor}{rgb}{0.317647,0.317647,0.317647}%
\pgfsetstrokecolor{textcolor}%
\pgfsetfillcolor{textcolor}%
\pgftext[x=0.903556in,y=0.334967in,,top]{\color{textcolor}\rmfamily\fontsize{6.664000}{7.996800}\selectfont \(\displaystyle 10\)}%
\end{pgfscope}%
\begin{pgfscope}%
\pgfsetbuttcap%
\pgfsetroundjoin%
\definecolor{currentfill}{rgb}{0.317647,0.317647,0.317647}%
\pgfsetfillcolor{currentfill}%
\pgfsetlinewidth{0.501875pt}%
\definecolor{currentstroke}{rgb}{0.317647,0.317647,0.317647}%
\pgfsetstrokecolor{currentstroke}%
\pgfsetdash{}{0pt}%
\pgfsys@defobject{currentmarker}{\pgfqpoint{0.000000in}{-0.020833in}}{\pgfqpoint{0.000000in}{0.000000in}}{%
\pgfpathmoveto{\pgfqpoint{0.000000in}{0.000000in}}%
\pgfpathlineto{\pgfqpoint{0.000000in}{-0.020833in}}%
\pgfusepath{stroke,fill}%
}%
\begin{pgfscope}%
\pgfsys@transformshift{1.368556in}{0.383578in}%
\pgfsys@useobject{currentmarker}{}%
\end{pgfscope}%
\end{pgfscope}%
\begin{pgfscope}%
\definecolor{textcolor}{rgb}{0.317647,0.317647,0.317647}%
\pgfsetstrokecolor{textcolor}%
\pgfsetfillcolor{textcolor}%
\pgftext[x=1.368556in,y=0.334967in,,top]{\color{textcolor}\rmfamily\fontsize{6.664000}{7.996800}\selectfont \(\displaystyle 20\)}%
\end{pgfscope}%
\begin{pgfscope}%
\pgfsetbuttcap%
\pgfsetroundjoin%
\definecolor{currentfill}{rgb}{0.317647,0.317647,0.317647}%
\pgfsetfillcolor{currentfill}%
\pgfsetlinewidth{0.501875pt}%
\definecolor{currentstroke}{rgb}{0.317647,0.317647,0.317647}%
\pgfsetstrokecolor{currentstroke}%
\pgfsetdash{}{0pt}%
\pgfsys@defobject{currentmarker}{\pgfqpoint{0.000000in}{-0.020833in}}{\pgfqpoint{0.000000in}{0.000000in}}{%
\pgfpathmoveto{\pgfqpoint{0.000000in}{0.000000in}}%
\pgfpathlineto{\pgfqpoint{0.000000in}{-0.020833in}}%
\pgfusepath{stroke,fill}%
}%
\begin{pgfscope}%
\pgfsys@transformshift{1.833556in}{0.383578in}%
\pgfsys@useobject{currentmarker}{}%
\end{pgfscope}%
\end{pgfscope}%
\begin{pgfscope}%
\definecolor{textcolor}{rgb}{0.317647,0.317647,0.317647}%
\pgfsetstrokecolor{textcolor}%
\pgfsetfillcolor{textcolor}%
\pgftext[x=1.833556in,y=0.334967in,,top]{\color{textcolor}\rmfamily\fontsize{6.664000}{7.996800}\selectfont \(\displaystyle 30\)}%
\end{pgfscope}%
\begin{pgfscope}%
\pgfsetbuttcap%
\pgfsetroundjoin%
\definecolor{currentfill}{rgb}{0.317647,0.317647,0.317647}%
\pgfsetfillcolor{currentfill}%
\pgfsetlinewidth{0.501875pt}%
\definecolor{currentstroke}{rgb}{0.317647,0.317647,0.317647}%
\pgfsetstrokecolor{currentstroke}%
\pgfsetdash{}{0pt}%
\pgfsys@defobject{currentmarker}{\pgfqpoint{0.000000in}{-0.020833in}}{\pgfqpoint{0.000000in}{0.000000in}}{%
\pgfpathmoveto{\pgfqpoint{0.000000in}{0.000000in}}%
\pgfpathlineto{\pgfqpoint{0.000000in}{-0.020833in}}%
\pgfusepath{stroke,fill}%
}%
\begin{pgfscope}%
\pgfsys@transformshift{2.298556in}{0.383578in}%
\pgfsys@useobject{currentmarker}{}%
\end{pgfscope}%
\end{pgfscope}%
\begin{pgfscope}%
\definecolor{textcolor}{rgb}{0.317647,0.317647,0.317647}%
\pgfsetstrokecolor{textcolor}%
\pgfsetfillcolor{textcolor}%
\pgftext[x=2.298556in,y=0.334967in,,top]{\color{textcolor}\rmfamily\fontsize{6.664000}{7.996800}\selectfont \(\displaystyle 40\)}%
\end{pgfscope}%
\begin{pgfscope}%
\pgfsetbuttcap%
\pgfsetroundjoin%
\definecolor{currentfill}{rgb}{0.317647,0.317647,0.317647}%
\pgfsetfillcolor{currentfill}%
\pgfsetlinewidth{0.501875pt}%
\definecolor{currentstroke}{rgb}{0.317647,0.317647,0.317647}%
\pgfsetstrokecolor{currentstroke}%
\pgfsetdash{}{0pt}%
\pgfsys@defobject{currentmarker}{\pgfqpoint{0.000000in}{-0.020833in}}{\pgfqpoint{0.000000in}{0.000000in}}{%
\pgfpathmoveto{\pgfqpoint{0.000000in}{0.000000in}}%
\pgfpathlineto{\pgfqpoint{0.000000in}{-0.020833in}}%
\pgfusepath{stroke,fill}%
}%
\begin{pgfscope}%
\pgfsys@transformshift{2.763556in}{0.383578in}%
\pgfsys@useobject{currentmarker}{}%
\end{pgfscope}%
\end{pgfscope}%
\begin{pgfscope}%
\definecolor{textcolor}{rgb}{0.317647,0.317647,0.317647}%
\pgfsetstrokecolor{textcolor}%
\pgfsetfillcolor{textcolor}%
\pgftext[x=2.763556in,y=0.334967in,,top]{\color{textcolor}\rmfamily\fontsize{6.664000}{7.996800}\selectfont \(\displaystyle 50\)}%
\end{pgfscope}%
\begin{pgfscope}%
\definecolor{textcolor}{rgb}{0.317647,0.317647,0.317647}%
\pgfsetstrokecolor{textcolor}%
\pgfsetfillcolor{textcolor}%
\pgftext[x=1.601056in,y=0.197222in,,top]{\color{textcolor}\rmfamily\fontsize{6.664000}{7.996800}\selectfont spike time \(\displaystyle (\si{\micro \s})\)}%
\end{pgfscope}%
\begin{pgfscope}%
\pgfsetbuttcap%
\pgfsetroundjoin%
\definecolor{currentfill}{rgb}{0.317647,0.317647,0.317647}%
\pgfsetfillcolor{currentfill}%
\pgfsetlinewidth{0.501875pt}%
\definecolor{currentstroke}{rgb}{0.317647,0.317647,0.317647}%
\pgfsetstrokecolor{currentstroke}%
\pgfsetdash{}{0pt}%
\pgfsys@defobject{currentmarker}{\pgfqpoint{-0.020833in}{0.000000in}}{\pgfqpoint{0.000000in}{0.000000in}}{%
\pgfpathmoveto{\pgfqpoint{0.000000in}{0.000000in}}%
\pgfpathlineto{\pgfqpoint{-0.020833in}{0.000000in}}%
\pgfusepath{stroke,fill}%
}%
\begin{pgfscope}%
\pgfsys@transformshift{0.438556in}{0.500768in}%
\pgfsys@useobject{currentmarker}{}%
\end{pgfscope}%
\end{pgfscope}%
\begin{pgfscope}%
\definecolor{textcolor}{rgb}{0.317647,0.317647,0.317647}%
\pgfsetstrokecolor{textcolor}%
\pgfsetfillcolor{textcolor}%
\pgftext[x=0.348471in,y=0.468651in,left,base]{\color{textcolor}\rmfamily\fontsize{6.664000}{7.996800}\selectfont \(\displaystyle 0\)}%
\end{pgfscope}%
\begin{pgfscope}%
\pgfsetbuttcap%
\pgfsetroundjoin%
\definecolor{currentfill}{rgb}{0.317647,0.317647,0.317647}%
\pgfsetfillcolor{currentfill}%
\pgfsetlinewidth{0.501875pt}%
\definecolor{currentstroke}{rgb}{0.317647,0.317647,0.317647}%
\pgfsetstrokecolor{currentstroke}%
\pgfsetdash{}{0pt}%
\pgfsys@defobject{currentmarker}{\pgfqpoint{-0.020833in}{0.000000in}}{\pgfqpoint{0.000000in}{0.000000in}}{%
\pgfpathmoveto{\pgfqpoint{0.000000in}{0.000000in}}%
\pgfpathlineto{\pgfqpoint{-0.020833in}{0.000000in}}%
\pgfusepath{stroke,fill}%
}%
\begin{pgfscope}%
\pgfsys@transformshift{0.438556in}{0.938243in}%
\pgfsys@useobject{currentmarker}{}%
\end{pgfscope}%
\end{pgfscope}%
\begin{pgfscope}%
\definecolor{textcolor}{rgb}{0.317647,0.317647,0.317647}%
\pgfsetstrokecolor{textcolor}%
\pgfsetfillcolor{textcolor}%
\pgftext[x=0.293108in,y=0.906126in,left,base]{\color{textcolor}\rmfamily\fontsize{6.664000}{7.996800}\selectfont \(\displaystyle 20\)}%
\end{pgfscope}%
\begin{pgfscope}%
\pgfsetbuttcap%
\pgfsetroundjoin%
\definecolor{currentfill}{rgb}{0.317647,0.317647,0.317647}%
\pgfsetfillcolor{currentfill}%
\pgfsetlinewidth{0.501875pt}%
\definecolor{currentstroke}{rgb}{0.317647,0.317647,0.317647}%
\pgfsetstrokecolor{currentstroke}%
\pgfsetdash{}{0pt}%
\pgfsys@defobject{currentmarker}{\pgfqpoint{-0.020833in}{0.000000in}}{\pgfqpoint{0.000000in}{0.000000in}}{%
\pgfpathmoveto{\pgfqpoint{0.000000in}{0.000000in}}%
\pgfpathlineto{\pgfqpoint{-0.020833in}{0.000000in}}%
\pgfusepath{stroke,fill}%
}%
\begin{pgfscope}%
\pgfsys@transformshift{0.438556in}{1.375718in}%
\pgfsys@useobject{currentmarker}{}%
\end{pgfscope}%
\end{pgfscope}%
\begin{pgfscope}%
\definecolor{textcolor}{rgb}{0.317647,0.317647,0.317647}%
\pgfsetstrokecolor{textcolor}%
\pgfsetfillcolor{textcolor}%
\pgftext[x=0.293108in,y=1.343601in,left,base]{\color{textcolor}\rmfamily\fontsize{6.664000}{7.996800}\selectfont \(\displaystyle 40\)}%
\end{pgfscope}%
\begin{pgfscope}%
\pgfsetbuttcap%
\pgfsetroundjoin%
\definecolor{currentfill}{rgb}{0.317647,0.317647,0.317647}%
\pgfsetfillcolor{currentfill}%
\pgfsetlinewidth{0.501875pt}%
\definecolor{currentstroke}{rgb}{0.317647,0.317647,0.317647}%
\pgfsetstrokecolor{currentstroke}%
\pgfsetdash{}{0pt}%
\pgfsys@defobject{currentmarker}{\pgfqpoint{-0.020833in}{0.000000in}}{\pgfqpoint{0.000000in}{0.000000in}}{%
\pgfpathmoveto{\pgfqpoint{0.000000in}{0.000000in}}%
\pgfpathlineto{\pgfqpoint{-0.020833in}{0.000000in}}%
\pgfusepath{stroke,fill}%
}%
\begin{pgfscope}%
\pgfsys@transformshift{0.438556in}{1.813193in}%
\pgfsys@useobject{currentmarker}{}%
\end{pgfscope}%
\end{pgfscope}%
\begin{pgfscope}%
\definecolor{textcolor}{rgb}{0.317647,0.317647,0.317647}%
\pgfsetstrokecolor{textcolor}%
\pgfsetfillcolor{textcolor}%
\pgftext[x=0.293108in,y=1.781077in,left,base]{\color{textcolor}\rmfamily\fontsize{6.664000}{7.996800}\selectfont \(\displaystyle 60\)}%
\end{pgfscope}%
\begin{pgfscope}%
\pgfsetbuttcap%
\pgfsetroundjoin%
\definecolor{currentfill}{rgb}{0.317647,0.317647,0.317647}%
\pgfsetfillcolor{currentfill}%
\pgfsetlinewidth{0.501875pt}%
\definecolor{currentstroke}{rgb}{0.317647,0.317647,0.317647}%
\pgfsetstrokecolor{currentstroke}%
\pgfsetdash{}{0pt}%
\pgfsys@defobject{currentmarker}{\pgfqpoint{-0.020833in}{0.000000in}}{\pgfqpoint{0.000000in}{0.000000in}}{%
\pgfpathmoveto{\pgfqpoint{0.000000in}{0.000000in}}%
\pgfpathlineto{\pgfqpoint{-0.020833in}{0.000000in}}%
\pgfusepath{stroke,fill}%
}%
\begin{pgfscope}%
\pgfsys@transformshift{0.438556in}{2.250669in}%
\pgfsys@useobject{currentmarker}{}%
\end{pgfscope}%
\end{pgfscope}%
\begin{pgfscope}%
\definecolor{textcolor}{rgb}{0.317647,0.317647,0.317647}%
\pgfsetstrokecolor{textcolor}%
\pgfsetfillcolor{textcolor}%
\pgftext[x=0.293108in,y=2.218552in,left,base]{\color{textcolor}\rmfamily\fontsize{6.664000}{7.996800}\selectfont \(\displaystyle 80\)}%
\end{pgfscope}%
\begin{pgfscope}%
\pgfsetbuttcap%
\pgfsetroundjoin%
\definecolor{currentfill}{rgb}{0.317647,0.317647,0.317647}%
\pgfsetfillcolor{currentfill}%
\pgfsetlinewidth{0.501875pt}%
\definecolor{currentstroke}{rgb}{0.317647,0.317647,0.317647}%
\pgfsetstrokecolor{currentstroke}%
\pgfsetdash{}{0pt}%
\pgfsys@defobject{currentmarker}{\pgfqpoint{-0.020833in}{0.000000in}}{\pgfqpoint{0.000000in}{0.000000in}}{%
\pgfpathmoveto{\pgfqpoint{0.000000in}{0.000000in}}%
\pgfpathlineto{\pgfqpoint{-0.020833in}{0.000000in}}%
\pgfusepath{stroke,fill}%
}%
\begin{pgfscope}%
\pgfsys@transformshift{0.438556in}{2.688144in}%
\pgfsys@useobject{currentmarker}{}%
\end{pgfscope}%
\end{pgfscope}%
\begin{pgfscope}%
\definecolor{textcolor}{rgb}{0.317647,0.317647,0.317647}%
\pgfsetstrokecolor{textcolor}%
\pgfsetfillcolor{textcolor}%
\pgftext[x=0.237745in,y=2.656027in,left,base]{\color{textcolor}\rmfamily\fontsize{6.664000}{7.996800}\selectfont \(\displaystyle 100\)}%
\end{pgfscope}%
\begin{pgfscope}%
\definecolor{textcolor}{rgb}{0.317647,0.317647,0.317647}%
\pgfsetstrokecolor{textcolor}%
\pgfsetfillcolor{textcolor}%
\pgftext[x=0.182189in,y=1.538578in,,bottom,rotate=90.000000]{\color{textcolor}\rmfamily\fontsize{6.664000}{7.996800}\selectfont input unit}%
\end{pgfscope}%
\begin{pgfscope}%
\pgfpathrectangle{\pgfqpoint{0.438556in}{0.383578in}}{\pgfqpoint{2.325000in}{2.310000in}}%
\pgfusepath{clip}%
\pgfsetbuttcap%
\pgfsetroundjoin%
\pgfsetlinewidth{0.803000pt}%
\definecolor{currentstroke}{rgb}{0.333333,0.333333,0.333333}%
\pgfsetstrokecolor{currentstroke}%
\pgfsetdash{}{0pt}%
\pgfpathmoveto{\pgfqpoint{0.648736in}{0.626106in}}%
\pgfpathcurveto{\pgfqpoint{0.656102in}{0.626106in}}{\pgfqpoint{0.663168in}{0.629033in}}{\pgfqpoint{0.668378in}{0.634242in}}%
\pgfpathcurveto{\pgfqpoint{0.673587in}{0.639451in}}{\pgfqpoint{0.676513in}{0.646517in}}{\pgfqpoint{0.676513in}{0.653884in}}%
\pgfpathcurveto{\pgfqpoint{0.676513in}{0.661251in}}{\pgfqpoint{0.673587in}{0.668317in}}{\pgfqpoint{0.668378in}{0.673526in}}%
\pgfpathcurveto{\pgfqpoint{0.663168in}{0.678735in}}{\pgfqpoint{0.656102in}{0.681662in}}{\pgfqpoint{0.648736in}{0.681662in}}%
\pgfpathcurveto{\pgfqpoint{0.641369in}{0.681662in}}{\pgfqpoint{0.634303in}{0.678735in}}{\pgfqpoint{0.629094in}{0.673526in}}%
\pgfpathcurveto{\pgfqpoint{0.623885in}{0.668317in}}{\pgfqpoint{0.620958in}{0.661251in}}{\pgfqpoint{0.620958in}{0.653884in}}%
\pgfpathcurveto{\pgfqpoint{0.620958in}{0.646517in}}{\pgfqpoint{0.623885in}{0.639451in}}{\pgfqpoint{0.629094in}{0.634242in}}%
\pgfpathcurveto{\pgfqpoint{0.634303in}{0.629033in}}{\pgfqpoint{0.641369in}{0.626106in}}{\pgfqpoint{0.648736in}{0.626106in}}%
\pgfpathclose%
\pgfusepath{stroke}%
\end{pgfscope}%
\begin{pgfscope}%
\pgfpathrectangle{\pgfqpoint{0.438556in}{0.383578in}}{\pgfqpoint{2.325000in}{2.310000in}}%
\pgfusepath{clip}%
\pgfsetbuttcap%
\pgfsetroundjoin%
\pgfsetlinewidth{0.803000pt}%
\definecolor{currentstroke}{rgb}{0.333333,0.333333,0.333333}%
\pgfsetstrokecolor{currentstroke}%
\pgfsetdash{}{0pt}%
\pgfpathmoveto{\pgfqpoint{1.197064in}{0.844844in}}%
\pgfpathcurveto{\pgfqpoint{1.204430in}{0.844844in}}{\pgfqpoint{1.211496in}{0.847771in}}{\pgfqpoint{1.216706in}{0.852980in}}%
\pgfpathcurveto{\pgfqpoint{1.221915in}{0.858189in}}{\pgfqpoint{1.224841in}{0.865255in}}{\pgfqpoint{1.224841in}{0.872622in}}%
\pgfpathcurveto{\pgfqpoint{1.224841in}{0.879988in}}{\pgfqpoint{1.221915in}{0.887054in}}{\pgfqpoint{1.216706in}{0.892263in}}%
\pgfpathcurveto{\pgfqpoint{1.211496in}{0.897472in}}{\pgfqpoint{1.204430in}{0.900399in}}{\pgfqpoint{1.197064in}{0.900399in}}%
\pgfpathcurveto{\pgfqpoint{1.189697in}{0.900399in}}{\pgfqpoint{1.182631in}{0.897472in}}{\pgfqpoint{1.177422in}{0.892263in}}%
\pgfpathcurveto{\pgfqpoint{1.172213in}{0.887054in}}{\pgfqpoint{1.169286in}{0.879988in}}{\pgfqpoint{1.169286in}{0.872622in}}%
\pgfpathcurveto{\pgfqpoint{1.169286in}{0.865255in}}{\pgfqpoint{1.172213in}{0.858189in}}{\pgfqpoint{1.177422in}{0.852980in}}%
\pgfpathcurveto{\pgfqpoint{1.182631in}{0.847771in}}{\pgfqpoint{1.189697in}{0.844844in}}{\pgfqpoint{1.197064in}{0.844844in}}%
\pgfpathclose%
\pgfusepath{stroke}%
\end{pgfscope}%
\begin{pgfscope}%
\pgfpathrectangle{\pgfqpoint{0.438556in}{0.383578in}}{\pgfqpoint{2.325000in}{2.310000in}}%
\pgfusepath{clip}%
\pgfsetbuttcap%
\pgfsetroundjoin%
\pgfsetlinewidth{0.803000pt}%
\definecolor{currentstroke}{rgb}{0.333333,0.333333,0.333333}%
\pgfsetstrokecolor{currentstroke}%
\pgfsetdash{}{0pt}%
\pgfpathmoveto{\pgfqpoint{2.221738in}{0.866718in}}%
\pgfpathcurveto{\pgfqpoint{2.229104in}{0.866718in}}{\pgfqpoint{2.236170in}{0.869644in}}{\pgfqpoint{2.241380in}{0.874853in}}%
\pgfpathcurveto{\pgfqpoint{2.246589in}{0.880063in}}{\pgfqpoint{2.249515in}{0.887129in}}{\pgfqpoint{2.249515in}{0.894495in}}%
\pgfpathcurveto{\pgfqpoint{2.249515in}{0.901862in}}{\pgfqpoint{2.246589in}{0.908928in}}{\pgfqpoint{2.241380in}{0.914137in}}%
\pgfpathcurveto{\pgfqpoint{2.236170in}{0.919346in}}{\pgfqpoint{2.229104in}{0.922273in}}{\pgfqpoint{2.221738in}{0.922273in}}%
\pgfpathcurveto{\pgfqpoint{2.214371in}{0.922273in}}{\pgfqpoint{2.207305in}{0.919346in}}{\pgfqpoint{2.202096in}{0.914137in}}%
\pgfpathcurveto{\pgfqpoint{2.196887in}{0.908928in}}{\pgfqpoint{2.193960in}{0.901862in}}{\pgfqpoint{2.193960in}{0.894495in}}%
\pgfpathcurveto{\pgfqpoint{2.193960in}{0.887129in}}{\pgfqpoint{2.196887in}{0.880063in}}{\pgfqpoint{2.202096in}{0.874853in}}%
\pgfpathcurveto{\pgfqpoint{2.207305in}{0.869644in}}{\pgfqpoint{2.214371in}{0.866718in}}{\pgfqpoint{2.221738in}{0.866718in}}%
\pgfpathclose%
\pgfusepath{stroke}%
\end{pgfscope}%
\begin{pgfscope}%
\pgfpathrectangle{\pgfqpoint{0.438556in}{0.383578in}}{\pgfqpoint{2.325000in}{2.310000in}}%
\pgfusepath{clip}%
\pgfsetbuttcap%
\pgfsetroundjoin%
\pgfsetlinewidth{0.803000pt}%
\definecolor{currentstroke}{rgb}{0.333333,0.333333,0.333333}%
\pgfsetstrokecolor{currentstroke}%
\pgfsetdash{}{0pt}%
\pgfpathmoveto{\pgfqpoint{1.796356in}{0.997960in}}%
\pgfpathcurveto{\pgfqpoint{1.803722in}{0.997960in}}{\pgfqpoint{1.810788in}{1.000887in}}{\pgfqpoint{1.815998in}{1.006096in}}%
\pgfpathcurveto{\pgfqpoint{1.821207in}{1.011305in}}{\pgfqpoint{1.824133in}{1.018371in}}{\pgfqpoint{1.824133in}{1.025738in}}%
\pgfpathcurveto{\pgfqpoint{1.824133in}{1.033105in}}{\pgfqpoint{1.821207in}{1.040171in}}{\pgfqpoint{1.815998in}{1.045380in}}%
\pgfpathcurveto{\pgfqpoint{1.810788in}{1.050589in}}{\pgfqpoint{1.803722in}{1.053516in}}{\pgfqpoint{1.796356in}{1.053516in}}%
\pgfpathcurveto{\pgfqpoint{1.788989in}{1.053516in}}{\pgfqpoint{1.781923in}{1.050589in}}{\pgfqpoint{1.776714in}{1.045380in}}%
\pgfpathcurveto{\pgfqpoint{1.771505in}{1.040171in}}{\pgfqpoint{1.768578in}{1.033105in}}{\pgfqpoint{1.768578in}{1.025738in}}%
\pgfpathcurveto{\pgfqpoint{1.768578in}{1.018371in}}{\pgfqpoint{1.771505in}{1.011305in}}{\pgfqpoint{1.776714in}{1.006096in}}%
\pgfpathcurveto{\pgfqpoint{1.781923in}{1.000887in}}{\pgfqpoint{1.788989in}{0.997960in}}{\pgfqpoint{1.796356in}{0.997960in}}%
\pgfpathclose%
\pgfusepath{stroke}%
\end{pgfscope}%
\begin{pgfscope}%
\pgfpathrectangle{\pgfqpoint{0.438556in}{0.383578in}}{\pgfqpoint{2.325000in}{2.310000in}}%
\pgfusepath{clip}%
\pgfsetbuttcap%
\pgfsetroundjoin%
\pgfsetlinewidth{0.803000pt}%
\definecolor{currentstroke}{rgb}{0.333333,0.333333,0.333333}%
\pgfsetstrokecolor{currentstroke}%
\pgfsetdash{}{0pt}%
\pgfpathmoveto{\pgfqpoint{1.087324in}{1.019834in}}%
\pgfpathcurveto{\pgfqpoint{1.094690in}{1.019834in}}{\pgfqpoint{1.101756in}{1.022761in}}{\pgfqpoint{1.106966in}{1.027970in}}%
\pgfpathcurveto{\pgfqpoint{1.112175in}{1.033179in}}{\pgfqpoint{1.115101in}{1.040245in}}{\pgfqpoint{1.115101in}{1.047612in}}%
\pgfpathcurveto{\pgfqpoint{1.115101in}{1.054978in}}{\pgfqpoint{1.112175in}{1.062044in}}{\pgfqpoint{1.106966in}{1.067253in}}%
\pgfpathcurveto{\pgfqpoint{1.101756in}{1.072463in}}{\pgfqpoint{1.094690in}{1.075389in}}{\pgfqpoint{1.087324in}{1.075389in}}%
\pgfpathcurveto{\pgfqpoint{1.079957in}{1.075389in}}{\pgfqpoint{1.072891in}{1.072463in}}{\pgfqpoint{1.067682in}{1.067253in}}%
\pgfpathcurveto{\pgfqpoint{1.062473in}{1.062044in}}{\pgfqpoint{1.059546in}{1.054978in}}{\pgfqpoint{1.059546in}{1.047612in}}%
\pgfpathcurveto{\pgfqpoint{1.059546in}{1.040245in}}{\pgfqpoint{1.062473in}{1.033179in}}{\pgfqpoint{1.067682in}{1.027970in}}%
\pgfpathcurveto{\pgfqpoint{1.072891in}{1.022761in}}{\pgfqpoint{1.079957in}{1.019834in}}{\pgfqpoint{1.087324in}{1.019834in}}%
\pgfpathclose%
\pgfusepath{stroke}%
\end{pgfscope}%
\begin{pgfscope}%
\pgfpathrectangle{\pgfqpoint{0.438556in}{0.383578in}}{\pgfqpoint{2.325000in}{2.310000in}}%
\pgfusepath{clip}%
\pgfsetbuttcap%
\pgfsetroundjoin%
\pgfsetlinewidth{0.803000pt}%
\definecolor{currentstroke}{rgb}{0.333333,0.333333,0.333333}%
\pgfsetstrokecolor{currentstroke}%
\pgfsetdash{}{0pt}%
\pgfpathmoveto{\pgfqpoint{1.028548in}{1.194824in}}%
\pgfpathcurveto{\pgfqpoint{1.035914in}{1.194824in}}{\pgfqpoint{1.042980in}{1.197751in}}{\pgfqpoint{1.048190in}{1.202960in}}%
\pgfpathcurveto{\pgfqpoint{1.053399in}{1.208169in}}{\pgfqpoint{1.056325in}{1.215235in}}{\pgfqpoint{1.056325in}{1.222602in}}%
\pgfpathcurveto{\pgfqpoint{1.056325in}{1.229968in}}{\pgfqpoint{1.053399in}{1.237035in}}{\pgfqpoint{1.048190in}{1.242244in}}%
\pgfpathcurveto{\pgfqpoint{1.042980in}{1.247453in}}{\pgfqpoint{1.035914in}{1.250380in}}{\pgfqpoint{1.028548in}{1.250380in}}%
\pgfpathcurveto{\pgfqpoint{1.021181in}{1.250380in}}{\pgfqpoint{1.014115in}{1.247453in}}{\pgfqpoint{1.008906in}{1.242244in}}%
\pgfpathcurveto{\pgfqpoint{1.003697in}{1.237035in}}{\pgfqpoint{1.000770in}{1.229968in}}{\pgfqpoint{1.000770in}{1.222602in}}%
\pgfpathcurveto{\pgfqpoint{1.000770in}{1.215235in}}{\pgfqpoint{1.003697in}{1.208169in}}{\pgfqpoint{1.008906in}{1.202960in}}%
\pgfpathcurveto{\pgfqpoint{1.014115in}{1.197751in}}{\pgfqpoint{1.021181in}{1.194824in}}{\pgfqpoint{1.028548in}{1.194824in}}%
\pgfpathclose%
\pgfusepath{stroke}%
\end{pgfscope}%
\begin{pgfscope}%
\pgfpathrectangle{\pgfqpoint{0.438556in}{0.383578in}}{\pgfqpoint{2.325000in}{2.310000in}}%
\pgfusepath{clip}%
\pgfsetbuttcap%
\pgfsetroundjoin%
\pgfsetlinewidth{0.803000pt}%
\definecolor{currentstroke}{rgb}{0.333333,0.333333,0.333333}%
\pgfsetstrokecolor{currentstroke}%
\pgfsetdash{}{0pt}%
\pgfpathmoveto{\pgfqpoint{0.903556in}{1.282319in}}%
\pgfpathcurveto{\pgfqpoint{0.910922in}{1.282319in}}{\pgfqpoint{0.917988in}{1.285246in}}{\pgfqpoint{0.923198in}{1.290455in}}%
\pgfpathcurveto{\pgfqpoint{0.928407in}{1.295664in}}{\pgfqpoint{0.931333in}{1.302730in}}{\pgfqpoint{0.931333in}{1.310097in}}%
\pgfpathcurveto{\pgfqpoint{0.931333in}{1.317464in}}{\pgfqpoint{0.928407in}{1.324530in}}{\pgfqpoint{0.923198in}{1.329739in}}%
\pgfpathcurveto{\pgfqpoint{0.917988in}{1.334948in}}{\pgfqpoint{0.910922in}{1.337875in}}{\pgfqpoint{0.903556in}{1.337875in}}%
\pgfpathcurveto{\pgfqpoint{0.896189in}{1.337875in}}{\pgfqpoint{0.889123in}{1.334948in}}{\pgfqpoint{0.883914in}{1.329739in}}%
\pgfpathcurveto{\pgfqpoint{0.878705in}{1.324530in}}{\pgfqpoint{0.875778in}{1.317464in}}{\pgfqpoint{0.875778in}{1.310097in}}%
\pgfpathcurveto{\pgfqpoint{0.875778in}{1.302730in}}{\pgfqpoint{0.878705in}{1.295664in}}{\pgfqpoint{0.883914in}{1.290455in}}%
\pgfpathcurveto{\pgfqpoint{0.889123in}{1.285246in}}{\pgfqpoint{0.896189in}{1.282319in}}{\pgfqpoint{0.903556in}{1.282319in}}%
\pgfpathclose%
\pgfusepath{stroke}%
\end{pgfscope}%
\begin{pgfscope}%
\pgfpathrectangle{\pgfqpoint{0.438556in}{0.383578in}}{\pgfqpoint{2.325000in}{2.310000in}}%
\pgfusepath{clip}%
\pgfsetbuttcap%
\pgfsetroundjoin%
\pgfsetlinewidth{0.803000pt}%
\definecolor{currentstroke}{rgb}{0.333333,0.333333,0.333333}%
\pgfsetstrokecolor{currentstroke}%
\pgfsetdash{}{0pt}%
\pgfpathmoveto{\pgfqpoint{2.002072in}{1.326067in}}%
\pgfpathcurveto{\pgfqpoint{2.009438in}{1.326067in}}{\pgfqpoint{2.016504in}{1.328993in}}{\pgfqpoint{2.021714in}{1.334202in}}%
\pgfpathcurveto{\pgfqpoint{2.026923in}{1.339412in}}{\pgfqpoint{2.029849in}{1.346478in}}{\pgfqpoint{2.029849in}{1.353844in}}%
\pgfpathcurveto{\pgfqpoint{2.029849in}{1.361211in}}{\pgfqpoint{2.026923in}{1.368277in}}{\pgfqpoint{2.021714in}{1.373486in}}%
\pgfpathcurveto{\pgfqpoint{2.016504in}{1.378695in}}{\pgfqpoint{2.009438in}{1.381622in}}{\pgfqpoint{2.002072in}{1.381622in}}%
\pgfpathcurveto{\pgfqpoint{1.994705in}{1.381622in}}{\pgfqpoint{1.987639in}{1.378695in}}{\pgfqpoint{1.982430in}{1.373486in}}%
\pgfpathcurveto{\pgfqpoint{1.977221in}{1.368277in}}{\pgfqpoint{1.974294in}{1.361211in}}{\pgfqpoint{1.974294in}{1.353844in}}%
\pgfpathcurveto{\pgfqpoint{1.974294in}{1.346478in}}{\pgfqpoint{1.977221in}{1.339412in}}{\pgfqpoint{1.982430in}{1.334202in}}%
\pgfpathcurveto{\pgfqpoint{1.987639in}{1.328993in}}{\pgfqpoint{1.994705in}{1.326067in}}{\pgfqpoint{2.002072in}{1.326067in}}%
\pgfpathclose%
\pgfusepath{stroke}%
\end{pgfscope}%
\begin{pgfscope}%
\pgfpathrectangle{\pgfqpoint{0.438556in}{0.383578in}}{\pgfqpoint{2.325000in}{2.310000in}}%
\pgfusepath{clip}%
\pgfsetbuttcap%
\pgfsetroundjoin%
\pgfsetlinewidth{0.803000pt}%
\definecolor{currentstroke}{rgb}{0.333333,0.333333,0.333333}%
\pgfsetstrokecolor{currentstroke}%
\pgfsetdash{}{0pt}%
\pgfpathmoveto{\pgfqpoint{0.952474in}{1.435435in}}%
\pgfpathcurveto{\pgfqpoint{0.959840in}{1.435435in}}{\pgfqpoint{0.966906in}{1.438362in}}{\pgfqpoint{0.972116in}{1.443571in}}%
\pgfpathcurveto{\pgfqpoint{0.977325in}{1.448780in}}{\pgfqpoint{0.980251in}{1.455846in}}{\pgfqpoint{0.980251in}{1.463213in}}%
\pgfpathcurveto{\pgfqpoint{0.980251in}{1.470580in}}{\pgfqpoint{0.977325in}{1.477646in}}{\pgfqpoint{0.972116in}{1.482855in}}%
\pgfpathcurveto{\pgfqpoint{0.966906in}{1.488064in}}{\pgfqpoint{0.959840in}{1.490991in}}{\pgfqpoint{0.952474in}{1.490991in}}%
\pgfpathcurveto{\pgfqpoint{0.945107in}{1.490991in}}{\pgfqpoint{0.938041in}{1.488064in}}{\pgfqpoint{0.932832in}{1.482855in}}%
\pgfpathcurveto{\pgfqpoint{0.927623in}{1.477646in}}{\pgfqpoint{0.924696in}{1.470580in}}{\pgfqpoint{0.924696in}{1.463213in}}%
\pgfpathcurveto{\pgfqpoint{0.924696in}{1.455846in}}{\pgfqpoint{0.927623in}{1.448780in}}{\pgfqpoint{0.932832in}{1.443571in}}%
\pgfpathcurveto{\pgfqpoint{0.938041in}{1.438362in}}{\pgfqpoint{0.945107in}{1.435435in}}{\pgfqpoint{0.952474in}{1.435435in}}%
\pgfpathclose%
\pgfusepath{stroke}%
\end{pgfscope}%
\begin{pgfscope}%
\pgfpathrectangle{\pgfqpoint{0.438556in}{0.383578in}}{\pgfqpoint{2.325000in}{2.310000in}}%
\pgfusepath{clip}%
\pgfsetbuttcap%
\pgfsetroundjoin%
\pgfsetlinewidth{0.803000pt}%
\definecolor{currentstroke}{rgb}{0.333333,0.333333,0.333333}%
\pgfsetstrokecolor{currentstroke}%
\pgfsetdash{}{0pt}%
\pgfpathmoveto{\pgfqpoint{1.867036in}{1.676047in}}%
\pgfpathcurveto{\pgfqpoint{1.874402in}{1.676047in}}{\pgfqpoint{1.881468in}{1.678974in}}{\pgfqpoint{1.886678in}{1.684183in}}%
\pgfpathcurveto{\pgfqpoint{1.891887in}{1.689392in}}{\pgfqpoint{1.894813in}{1.696458in}}{\pgfqpoint{1.894813in}{1.703825in}}%
\pgfpathcurveto{\pgfqpoint{1.894813in}{1.711191in}}{\pgfqpoint{1.891887in}{1.718257in}}{\pgfqpoint{1.886678in}{1.723466in}}%
\pgfpathcurveto{\pgfqpoint{1.881468in}{1.728675in}}{\pgfqpoint{1.874402in}{1.731602in}}{\pgfqpoint{1.867036in}{1.731602in}}%
\pgfpathcurveto{\pgfqpoint{1.859669in}{1.731602in}}{\pgfqpoint{1.852603in}{1.728675in}}{\pgfqpoint{1.847394in}{1.723466in}}%
\pgfpathcurveto{\pgfqpoint{1.842185in}{1.718257in}}{\pgfqpoint{1.839258in}{1.711191in}}{\pgfqpoint{1.839258in}{1.703825in}}%
\pgfpathcurveto{\pgfqpoint{1.839258in}{1.696458in}}{\pgfqpoint{1.842185in}{1.689392in}}{\pgfqpoint{1.847394in}{1.684183in}}%
\pgfpathcurveto{\pgfqpoint{1.852603in}{1.678974in}}{\pgfqpoint{1.859669in}{1.676047in}}{\pgfqpoint{1.867036in}{1.676047in}}%
\pgfpathclose%
\pgfusepath{stroke}%
\end{pgfscope}%
\begin{pgfscope}%
\pgfpathrectangle{\pgfqpoint{0.438556in}{0.383578in}}{\pgfqpoint{2.325000in}{2.310000in}}%
\pgfusepath{clip}%
\pgfsetbuttcap%
\pgfsetroundjoin%
\pgfsetlinewidth{0.803000pt}%
\definecolor{currentstroke}{rgb}{0.333333,0.333333,0.333333}%
\pgfsetstrokecolor{currentstroke}%
\pgfsetdash{}{0pt}%
\pgfpathmoveto{\pgfqpoint{0.945034in}{1.763542in}}%
\pgfpathcurveto{\pgfqpoint{0.952400in}{1.763542in}}{\pgfqpoint{0.959466in}{1.766469in}}{\pgfqpoint{0.964676in}{1.771678in}}%
\pgfpathcurveto{\pgfqpoint{0.969885in}{1.776887in}}{\pgfqpoint{0.972811in}{1.783953in}}{\pgfqpoint{0.972811in}{1.791320in}}%
\pgfpathcurveto{\pgfqpoint{0.972811in}{1.798686in}}{\pgfqpoint{0.969885in}{1.805752in}}{\pgfqpoint{0.964676in}{1.810961in}}%
\pgfpathcurveto{\pgfqpoint{0.959466in}{1.816170in}}{\pgfqpoint{0.952400in}{1.819097in}}{\pgfqpoint{0.945034in}{1.819097in}}%
\pgfpathcurveto{\pgfqpoint{0.937667in}{1.819097in}}{\pgfqpoint{0.930601in}{1.816170in}}{\pgfqpoint{0.925392in}{1.810961in}}%
\pgfpathcurveto{\pgfqpoint{0.920183in}{1.805752in}}{\pgfqpoint{0.917256in}{1.798686in}}{\pgfqpoint{0.917256in}{1.791320in}}%
\pgfpathcurveto{\pgfqpoint{0.917256in}{1.783953in}}{\pgfqpoint{0.920183in}{1.776887in}}{\pgfqpoint{0.925392in}{1.771678in}}%
\pgfpathcurveto{\pgfqpoint{0.930601in}{1.766469in}}{\pgfqpoint{0.937667in}{1.763542in}}{\pgfqpoint{0.945034in}{1.763542in}}%
\pgfpathclose%
\pgfusepath{stroke}%
\end{pgfscope}%
\begin{pgfscope}%
\pgfpathrectangle{\pgfqpoint{0.438556in}{0.383578in}}{\pgfqpoint{2.325000in}{2.310000in}}%
\pgfusepath{clip}%
\pgfsetbuttcap%
\pgfsetroundjoin%
\pgfsetlinewidth{0.803000pt}%
\definecolor{currentstroke}{rgb}{0.333333,0.333333,0.333333}%
\pgfsetstrokecolor{currentstroke}%
\pgfsetdash{}{0pt}%
\pgfpathmoveto{\pgfqpoint{0.647620in}{1.785416in}}%
\pgfpathcurveto{\pgfqpoint{0.654986in}{1.785416in}}{\pgfqpoint{0.662052in}{1.788342in}}{\pgfqpoint{0.667262in}{1.793551in}}%
\pgfpathcurveto{\pgfqpoint{0.672471in}{1.798761in}}{\pgfqpoint{0.675397in}{1.805827in}}{\pgfqpoint{0.675397in}{1.813193in}}%
\pgfpathcurveto{\pgfqpoint{0.675397in}{1.820560in}}{\pgfqpoint{0.672471in}{1.827626in}}{\pgfqpoint{0.667262in}{1.832835in}}%
\pgfpathcurveto{\pgfqpoint{0.662052in}{1.838044in}}{\pgfqpoint{0.654986in}{1.840971in}}{\pgfqpoint{0.647620in}{1.840971in}}%
\pgfpathcurveto{\pgfqpoint{0.640253in}{1.840971in}}{\pgfqpoint{0.633187in}{1.838044in}}{\pgfqpoint{0.627978in}{1.832835in}}%
\pgfpathcurveto{\pgfqpoint{0.622769in}{1.827626in}}{\pgfqpoint{0.619842in}{1.820560in}}{\pgfqpoint{0.619842in}{1.813193in}}%
\pgfpathcurveto{\pgfqpoint{0.619842in}{1.805827in}}{\pgfqpoint{0.622769in}{1.798761in}}{\pgfqpoint{0.627978in}{1.793551in}}%
\pgfpathcurveto{\pgfqpoint{0.633187in}{1.788342in}}{\pgfqpoint{0.640253in}{1.785416in}}{\pgfqpoint{0.647620in}{1.785416in}}%
\pgfpathclose%
\pgfusepath{stroke}%
\end{pgfscope}%
\begin{pgfscope}%
\pgfpathrectangle{\pgfqpoint{0.438556in}{0.383578in}}{\pgfqpoint{2.325000in}{2.310000in}}%
\pgfusepath{clip}%
\pgfsetbuttcap%
\pgfsetroundjoin%
\pgfsetlinewidth{0.803000pt}%
\definecolor{currentstroke}{rgb}{0.333333,0.333333,0.333333}%
\pgfsetstrokecolor{currentstroke}%
\pgfsetdash{}{0pt}%
\pgfpathmoveto{\pgfqpoint{1.362604in}{1.829163in}}%
\pgfpathcurveto{\pgfqpoint{1.369970in}{1.829163in}}{\pgfqpoint{1.377036in}{1.832090in}}{\pgfqpoint{1.382246in}{1.837299in}}%
\pgfpathcurveto{\pgfqpoint{1.387455in}{1.842508in}}{\pgfqpoint{1.390381in}{1.849574in}}{\pgfqpoint{1.390381in}{1.856941in}}%
\pgfpathcurveto{\pgfqpoint{1.390381in}{1.864308in}}{\pgfqpoint{1.387455in}{1.871374in}}{\pgfqpoint{1.382246in}{1.876583in}}%
\pgfpathcurveto{\pgfqpoint{1.377036in}{1.881792in}}{\pgfqpoint{1.369970in}{1.884719in}}{\pgfqpoint{1.362604in}{1.884719in}}%
\pgfpathcurveto{\pgfqpoint{1.355237in}{1.884719in}}{\pgfqpoint{1.348171in}{1.881792in}}{\pgfqpoint{1.342962in}{1.876583in}}%
\pgfpathcurveto{\pgfqpoint{1.337753in}{1.871374in}}{\pgfqpoint{1.334826in}{1.864308in}}{\pgfqpoint{1.334826in}{1.856941in}}%
\pgfpathcurveto{\pgfqpoint{1.334826in}{1.849574in}}{\pgfqpoint{1.337753in}{1.842508in}}{\pgfqpoint{1.342962in}{1.837299in}}%
\pgfpathcurveto{\pgfqpoint{1.348171in}{1.832090in}}{\pgfqpoint{1.355237in}{1.829163in}}{\pgfqpoint{1.362604in}{1.829163in}}%
\pgfpathclose%
\pgfusepath{stroke}%
\end{pgfscope}%
\begin{pgfscope}%
\pgfpathrectangle{\pgfqpoint{0.438556in}{0.383578in}}{\pgfqpoint{2.325000in}{2.310000in}}%
\pgfusepath{clip}%
\pgfsetbuttcap%
\pgfsetroundjoin%
\pgfsetlinewidth{0.803000pt}%
\definecolor{currentstroke}{rgb}{0.333333,0.333333,0.333333}%
\pgfsetstrokecolor{currentstroke}%
\pgfsetdash{}{0pt}%
\pgfpathmoveto{\pgfqpoint{1.450024in}{1.851037in}}%
\pgfpathcurveto{\pgfqpoint{1.457390in}{1.851037in}}{\pgfqpoint{1.464456in}{1.853964in}}{\pgfqpoint{1.469666in}{1.859173in}}%
\pgfpathcurveto{\pgfqpoint{1.474875in}{1.864382in}}{\pgfqpoint{1.477801in}{1.871448in}}{\pgfqpoint{1.477801in}{1.878815in}}%
\pgfpathcurveto{\pgfqpoint{1.477801in}{1.886181in}}{\pgfqpoint{1.474875in}{1.893247in}}{\pgfqpoint{1.469666in}{1.898456in}}%
\pgfpathcurveto{\pgfqpoint{1.464456in}{1.903666in}}{\pgfqpoint{1.457390in}{1.906592in}}{\pgfqpoint{1.450024in}{1.906592in}}%
\pgfpathcurveto{\pgfqpoint{1.442657in}{1.906592in}}{\pgfqpoint{1.435591in}{1.903666in}}{\pgfqpoint{1.430382in}{1.898456in}}%
\pgfpathcurveto{\pgfqpoint{1.425173in}{1.893247in}}{\pgfqpoint{1.422246in}{1.886181in}}{\pgfqpoint{1.422246in}{1.878815in}}%
\pgfpathcurveto{\pgfqpoint{1.422246in}{1.871448in}}{\pgfqpoint{1.425173in}{1.864382in}}{\pgfqpoint{1.430382in}{1.859173in}}%
\pgfpathcurveto{\pgfqpoint{1.435591in}{1.853964in}}{\pgfqpoint{1.442657in}{1.851037in}}{\pgfqpoint{1.450024in}{1.851037in}}%
\pgfpathclose%
\pgfusepath{stroke}%
\end{pgfscope}%
\begin{pgfscope}%
\pgfpathrectangle{\pgfqpoint{0.438556in}{0.383578in}}{\pgfqpoint{2.325000in}{2.310000in}}%
\pgfusepath{clip}%
\pgfsetbuttcap%
\pgfsetroundjoin%
\pgfsetlinewidth{0.803000pt}%
\definecolor{currentstroke}{rgb}{0.333333,0.333333,0.333333}%
\pgfsetstrokecolor{currentstroke}%
\pgfsetdash{}{0pt}%
\pgfpathmoveto{\pgfqpoint{0.797164in}{1.982279in}}%
\pgfpathcurveto{\pgfqpoint{0.804530in}{1.982279in}}{\pgfqpoint{0.811596in}{1.985206in}}{\pgfqpoint{0.816806in}{1.990415in}}%
\pgfpathcurveto{\pgfqpoint{0.822015in}{1.995624in}}{\pgfqpoint{0.824941in}{2.002690in}}{\pgfqpoint{0.824941in}{2.010057in}}%
\pgfpathcurveto{\pgfqpoint{0.824941in}{2.017424in}}{\pgfqpoint{0.822015in}{2.024490in}}{\pgfqpoint{0.816806in}{2.029699in}}%
\pgfpathcurveto{\pgfqpoint{0.811596in}{2.034908in}}{\pgfqpoint{0.804530in}{2.037835in}}{\pgfqpoint{0.797164in}{2.037835in}}%
\pgfpathcurveto{\pgfqpoint{0.789797in}{2.037835in}}{\pgfqpoint{0.782731in}{2.034908in}}{\pgfqpoint{0.777522in}{2.029699in}}%
\pgfpathcurveto{\pgfqpoint{0.772313in}{2.024490in}}{\pgfqpoint{0.769386in}{2.017424in}}{\pgfqpoint{0.769386in}{2.010057in}}%
\pgfpathcurveto{\pgfqpoint{0.769386in}{2.002690in}}{\pgfqpoint{0.772313in}{1.995624in}}{\pgfqpoint{0.777522in}{1.990415in}}%
\pgfpathcurveto{\pgfqpoint{0.782731in}{1.985206in}}{\pgfqpoint{0.789797in}{1.982279in}}{\pgfqpoint{0.797164in}{1.982279in}}%
\pgfpathclose%
\pgfusepath{stroke}%
\end{pgfscope}%
\begin{pgfscope}%
\pgfpathrectangle{\pgfqpoint{0.438556in}{0.383578in}}{\pgfqpoint{2.325000in}{2.310000in}}%
\pgfusepath{clip}%
\pgfsetbuttcap%
\pgfsetroundjoin%
\pgfsetlinewidth{0.803000pt}%
\definecolor{currentstroke}{rgb}{0.333333,0.333333,0.333333}%
\pgfsetstrokecolor{currentstroke}%
\pgfsetdash{}{0pt}%
\pgfpathmoveto{\pgfqpoint{0.878818in}{2.069774in}}%
\pgfpathcurveto{\pgfqpoint{0.886184in}{2.069774in}}{\pgfqpoint{0.893250in}{2.072701in}}{\pgfqpoint{0.898460in}{2.077910in}}%
\pgfpathcurveto{\pgfqpoint{0.903669in}{2.083119in}}{\pgfqpoint{0.906595in}{2.090185in}}{\pgfqpoint{0.906595in}{2.097552in}}%
\pgfpathcurveto{\pgfqpoint{0.906595in}{2.104919in}}{\pgfqpoint{0.903669in}{2.111985in}}{\pgfqpoint{0.898460in}{2.117194in}}%
\pgfpathcurveto{\pgfqpoint{0.893250in}{2.122403in}}{\pgfqpoint{0.886184in}{2.125330in}}{\pgfqpoint{0.878818in}{2.125330in}}%
\pgfpathcurveto{\pgfqpoint{0.871451in}{2.125330in}}{\pgfqpoint{0.864385in}{2.122403in}}{\pgfqpoint{0.859176in}{2.117194in}}%
\pgfpathcurveto{\pgfqpoint{0.853967in}{2.111985in}}{\pgfqpoint{0.851040in}{2.104919in}}{\pgfqpoint{0.851040in}{2.097552in}}%
\pgfpathcurveto{\pgfqpoint{0.851040in}{2.090185in}}{\pgfqpoint{0.853967in}{2.083119in}}{\pgfqpoint{0.859176in}{2.077910in}}%
\pgfpathcurveto{\pgfqpoint{0.864385in}{2.072701in}}{\pgfqpoint{0.871451in}{2.069774in}}{\pgfqpoint{0.878818in}{2.069774in}}%
\pgfpathclose%
\pgfusepath{stroke}%
\end{pgfscope}%
\begin{pgfscope}%
\pgfpathrectangle{\pgfqpoint{0.438556in}{0.383578in}}{\pgfqpoint{2.325000in}{2.310000in}}%
\pgfusepath{clip}%
\pgfsetbuttcap%
\pgfsetroundjoin%
\pgfsetlinewidth{0.803000pt}%
\definecolor{currentstroke}{rgb}{0.333333,0.333333,0.333333}%
\pgfsetstrokecolor{currentstroke}%
\pgfsetdash{}{0pt}%
\pgfpathmoveto{\pgfqpoint{2.090236in}{2.157270in}}%
\pgfpathcurveto{\pgfqpoint{2.097602in}{2.157270in}}{\pgfqpoint{2.104668in}{2.160196in}}{\pgfqpoint{2.109878in}{2.165405in}}%
\pgfpathcurveto{\pgfqpoint{2.115087in}{2.170615in}}{\pgfqpoint{2.118013in}{2.177681in}}{\pgfqpoint{2.118013in}{2.185047in}}%
\pgfpathcurveto{\pgfqpoint{2.118013in}{2.192414in}}{\pgfqpoint{2.115087in}{2.199480in}}{\pgfqpoint{2.109878in}{2.204689in}}%
\pgfpathcurveto{\pgfqpoint{2.104668in}{2.209898in}}{\pgfqpoint{2.097602in}{2.212825in}}{\pgfqpoint{2.090236in}{2.212825in}}%
\pgfpathcurveto{\pgfqpoint{2.082869in}{2.212825in}}{\pgfqpoint{2.075803in}{2.209898in}}{\pgfqpoint{2.070594in}{2.204689in}}%
\pgfpathcurveto{\pgfqpoint{2.065385in}{2.199480in}}{\pgfqpoint{2.062458in}{2.192414in}}{\pgfqpoint{2.062458in}{2.185047in}}%
\pgfpathcurveto{\pgfqpoint{2.062458in}{2.177681in}}{\pgfqpoint{2.065385in}{2.170615in}}{\pgfqpoint{2.070594in}{2.165405in}}%
\pgfpathcurveto{\pgfqpoint{2.075803in}{2.160196in}}{\pgfqpoint{2.082869in}{2.157270in}}{\pgfqpoint{2.090236in}{2.157270in}}%
\pgfpathclose%
\pgfusepath{stroke}%
\end{pgfscope}%
\begin{pgfscope}%
\pgfpathrectangle{\pgfqpoint{0.438556in}{0.383578in}}{\pgfqpoint{2.325000in}{2.310000in}}%
\pgfusepath{clip}%
\pgfsetbuttcap%
\pgfsetroundjoin%
\pgfsetlinewidth{0.803000pt}%
\definecolor{currentstroke}{rgb}{0.333333,0.333333,0.333333}%
\pgfsetstrokecolor{currentstroke}%
\pgfsetdash{}{0pt}%
\pgfpathmoveto{\pgfqpoint{1.251934in}{2.244765in}}%
\pgfpathcurveto{\pgfqpoint{1.259300in}{2.244765in}}{\pgfqpoint{1.266366in}{2.247691in}}{\pgfqpoint{1.271576in}{2.252900in}}%
\pgfpathcurveto{\pgfqpoint{1.276785in}{2.258110in}}{\pgfqpoint{1.279711in}{2.265176in}}{\pgfqpoint{1.279711in}{2.272542in}}%
\pgfpathcurveto{\pgfqpoint{1.279711in}{2.279909in}}{\pgfqpoint{1.276785in}{2.286975in}}{\pgfqpoint{1.271576in}{2.292184in}}%
\pgfpathcurveto{\pgfqpoint{1.266366in}{2.297393in}}{\pgfqpoint{1.259300in}{2.300320in}}{\pgfqpoint{1.251934in}{2.300320in}}%
\pgfpathcurveto{\pgfqpoint{1.244567in}{2.300320in}}{\pgfqpoint{1.237501in}{2.297393in}}{\pgfqpoint{1.232292in}{2.292184in}}%
\pgfpathcurveto{\pgfqpoint{1.227083in}{2.286975in}}{\pgfqpoint{1.224156in}{2.279909in}}{\pgfqpoint{1.224156in}{2.272542in}}%
\pgfpathcurveto{\pgfqpoint{1.224156in}{2.265176in}}{\pgfqpoint{1.227083in}{2.258110in}}{\pgfqpoint{1.232292in}{2.252900in}}%
\pgfpathcurveto{\pgfqpoint{1.237501in}{2.247691in}}{\pgfqpoint{1.244567in}{2.244765in}}{\pgfqpoint{1.251934in}{2.244765in}}%
\pgfpathclose%
\pgfusepath{stroke}%
\end{pgfscope}%
\begin{pgfscope}%
\pgfpathrectangle{\pgfqpoint{0.438556in}{0.383578in}}{\pgfqpoint{2.325000in}{2.310000in}}%
\pgfusepath{clip}%
\pgfsetbuttcap%
\pgfsetroundjoin%
\pgfsetlinewidth{0.803000pt}%
\definecolor{currentstroke}{rgb}{0.333333,0.333333,0.333333}%
\pgfsetstrokecolor{currentstroke}%
\pgfsetdash{}{0pt}%
\pgfpathmoveto{\pgfqpoint{1.424728in}{2.310386in}}%
\pgfpathcurveto{\pgfqpoint{1.432094in}{2.310386in}}{\pgfqpoint{1.439160in}{2.313313in}}{\pgfqpoint{1.444370in}{2.318522in}}%
\pgfpathcurveto{\pgfqpoint{1.449579in}{2.323731in}}{\pgfqpoint{1.452505in}{2.330797in}}{\pgfqpoint{1.452505in}{2.338164in}}%
\pgfpathcurveto{\pgfqpoint{1.452505in}{2.345530in}}{\pgfqpoint{1.449579in}{2.352596in}}{\pgfqpoint{1.444370in}{2.357805in}}%
\pgfpathcurveto{\pgfqpoint{1.439160in}{2.363015in}}{\pgfqpoint{1.432094in}{2.365941in}}{\pgfqpoint{1.424728in}{2.365941in}}%
\pgfpathcurveto{\pgfqpoint{1.417361in}{2.365941in}}{\pgfqpoint{1.410295in}{2.363015in}}{\pgfqpoint{1.405086in}{2.357805in}}%
\pgfpathcurveto{\pgfqpoint{1.399877in}{2.352596in}}{\pgfqpoint{1.396950in}{2.345530in}}{\pgfqpoint{1.396950in}{2.338164in}}%
\pgfpathcurveto{\pgfqpoint{1.396950in}{2.330797in}}{\pgfqpoint{1.399877in}{2.323731in}}{\pgfqpoint{1.405086in}{2.318522in}}%
\pgfpathcurveto{\pgfqpoint{1.410295in}{2.313313in}}{\pgfqpoint{1.417361in}{2.310386in}}{\pgfqpoint{1.424728in}{2.310386in}}%
\pgfpathclose%
\pgfusepath{stroke}%
\end{pgfscope}%
\begin{pgfscope}%
\pgfpathrectangle{\pgfqpoint{0.438556in}{0.383578in}}{\pgfqpoint{2.325000in}{2.310000in}}%
\pgfusepath{clip}%
\pgfsetbuttcap%
\pgfsetroundjoin%
\pgfsetlinewidth{0.803000pt}%
\definecolor{currentstroke}{rgb}{0.333333,0.333333,0.333333}%
\pgfsetstrokecolor{currentstroke}%
\pgfsetdash{}{0pt}%
\pgfpathmoveto{\pgfqpoint{1.792264in}{2.332260in}}%
\pgfpathcurveto{\pgfqpoint{1.799630in}{2.332260in}}{\pgfqpoint{1.806696in}{2.335186in}}{\pgfqpoint{1.811906in}{2.340396in}}%
\pgfpathcurveto{\pgfqpoint{1.817115in}{2.345605in}}{\pgfqpoint{1.820041in}{2.352671in}}{\pgfqpoint{1.820041in}{2.360037in}}%
\pgfpathcurveto{\pgfqpoint{1.820041in}{2.367404in}}{\pgfqpoint{1.817115in}{2.374470in}}{\pgfqpoint{1.811906in}{2.379679in}}%
\pgfpathcurveto{\pgfqpoint{1.806696in}{2.384888in}}{\pgfqpoint{1.799630in}{2.387815in}}{\pgfqpoint{1.792264in}{2.387815in}}%
\pgfpathcurveto{\pgfqpoint{1.784897in}{2.387815in}}{\pgfqpoint{1.777831in}{2.384888in}}{\pgfqpoint{1.772622in}{2.379679in}}%
\pgfpathcurveto{\pgfqpoint{1.767413in}{2.374470in}}{\pgfqpoint{1.764486in}{2.367404in}}{\pgfqpoint{1.764486in}{2.360037in}}%
\pgfpathcurveto{\pgfqpoint{1.764486in}{2.352671in}}{\pgfqpoint{1.767413in}{2.345605in}}{\pgfqpoint{1.772622in}{2.340396in}}%
\pgfpathcurveto{\pgfqpoint{1.777831in}{2.335186in}}{\pgfqpoint{1.784897in}{2.332260in}}{\pgfqpoint{1.792264in}{2.332260in}}%
\pgfpathclose%
\pgfusepath{stroke}%
\end{pgfscope}%
\begin{pgfscope}%
\pgfpathrectangle{\pgfqpoint{0.438556in}{0.383578in}}{\pgfqpoint{2.325000in}{2.310000in}}%
\pgfusepath{clip}%
\pgfsetbuttcap%
\pgfsetroundjoin%
\pgfsetlinewidth{0.803000pt}%
\definecolor{currentstroke}{rgb}{0.686275,0.352941,0.313725}%
\pgfsetstrokecolor{currentstroke}%
\pgfsetdash{}{0pt}%
\pgfpathmoveto{\pgfqpoint{0.667894in}{0.479934in}}%
\pgfpathcurveto{\pgfqpoint{0.673419in}{0.479934in}}{\pgfqpoint{0.678718in}{0.482129in}}{\pgfqpoint{0.682625in}{0.486036in}}%
\pgfpathcurveto{\pgfqpoint{0.686532in}{0.489943in}}{\pgfqpoint{0.688727in}{0.495243in}}{\pgfqpoint{0.688727in}{0.500768in}}%
\pgfpathcurveto{\pgfqpoint{0.688727in}{0.506293in}}{\pgfqpoint{0.686532in}{0.511592in}}{\pgfqpoint{0.682625in}{0.515499in}}%
\pgfpathcurveto{\pgfqpoint{0.678718in}{0.519406in}}{\pgfqpoint{0.673419in}{0.521601in}}{\pgfqpoint{0.667894in}{0.521601in}}%
\pgfpathcurveto{\pgfqpoint{0.662369in}{0.521601in}}{\pgfqpoint{0.657069in}{0.519406in}}{\pgfqpoint{0.653162in}{0.515499in}}%
\pgfpathcurveto{\pgfqpoint{0.649255in}{0.511592in}}{\pgfqpoint{0.647060in}{0.506293in}}{\pgfqpoint{0.647060in}{0.500768in}}%
\pgfpathcurveto{\pgfqpoint{0.647060in}{0.495243in}}{\pgfqpoint{0.649255in}{0.489943in}}{\pgfqpoint{0.653162in}{0.486036in}}%
\pgfpathcurveto{\pgfqpoint{0.657069in}{0.482129in}}{\pgfqpoint{0.662369in}{0.479934in}}{\pgfqpoint{0.667894in}{0.479934in}}%
\pgfpathclose%
\pgfusepath{stroke}%
\end{pgfscope}%
\begin{pgfscope}%
\pgfpathrectangle{\pgfqpoint{0.438556in}{0.383578in}}{\pgfqpoint{2.325000in}{2.310000in}}%
\pgfusepath{clip}%
\pgfsetbuttcap%
\pgfsetroundjoin%
\pgfsetlinewidth{0.803000pt}%
\definecolor{currentstroke}{rgb}{0.686275,0.352941,0.313725}%
\pgfsetstrokecolor{currentstroke}%
\pgfsetdash{}{0pt}%
\pgfpathmoveto{\pgfqpoint{0.648736in}{0.633051in}}%
\pgfpathcurveto{\pgfqpoint{0.654261in}{0.633051in}}{\pgfqpoint{0.659560in}{0.635246in}}{\pgfqpoint{0.663467in}{0.639153in}}%
\pgfpathcurveto{\pgfqpoint{0.667374in}{0.643059in}}{\pgfqpoint{0.669569in}{0.648359in}}{\pgfqpoint{0.669569in}{0.653884in}}%
\pgfpathcurveto{\pgfqpoint{0.669569in}{0.659409in}}{\pgfqpoint{0.667374in}{0.664708in}}{\pgfqpoint{0.663467in}{0.668615in}}%
\pgfpathcurveto{\pgfqpoint{0.659560in}{0.672522in}}{\pgfqpoint{0.654261in}{0.674717in}}{\pgfqpoint{0.648736in}{0.674717in}}%
\pgfpathcurveto{\pgfqpoint{0.643211in}{0.674717in}}{\pgfqpoint{0.637911in}{0.672522in}}{\pgfqpoint{0.634004in}{0.668615in}}%
\pgfpathcurveto{\pgfqpoint{0.630097in}{0.664708in}}{\pgfqpoint{0.627902in}{0.659409in}}{\pgfqpoint{0.627902in}{0.653884in}}%
\pgfpathcurveto{\pgfqpoint{0.627902in}{0.648359in}}{\pgfqpoint{0.630097in}{0.643059in}}{\pgfqpoint{0.634004in}{0.639153in}}%
\pgfpathcurveto{\pgfqpoint{0.637911in}{0.635246in}}{\pgfqpoint{0.643211in}{0.633051in}}{\pgfqpoint{0.648736in}{0.633051in}}%
\pgfpathclose%
\pgfusepath{stroke}%
\end{pgfscope}%
\begin{pgfscope}%
\pgfpathrectangle{\pgfqpoint{0.438556in}{0.383578in}}{\pgfqpoint{2.325000in}{2.310000in}}%
\pgfusepath{clip}%
\pgfsetbuttcap%
\pgfsetroundjoin%
\pgfsetlinewidth{0.803000pt}%
\definecolor{currentstroke}{rgb}{0.686275,0.352941,0.313725}%
\pgfsetstrokecolor{currentstroke}%
\pgfsetdash{}{0pt}%
\pgfpathmoveto{\pgfqpoint{2.074612in}{0.676798in}}%
\pgfpathcurveto{\pgfqpoint{2.080137in}{0.676798in}}{\pgfqpoint{2.085436in}{0.678993in}}{\pgfqpoint{2.089343in}{0.682900in}}%
\pgfpathcurveto{\pgfqpoint{2.093250in}{0.686807in}}{\pgfqpoint{2.095445in}{0.692106in}}{\pgfqpoint{2.095445in}{0.697631in}}%
\pgfpathcurveto{\pgfqpoint{2.095445in}{0.703156in}}{\pgfqpoint{2.093250in}{0.708456in}}{\pgfqpoint{2.089343in}{0.712363in}}%
\pgfpathcurveto{\pgfqpoint{2.085436in}{0.716270in}}{\pgfqpoint{2.080137in}{0.718465in}}{\pgfqpoint{2.074612in}{0.718465in}}%
\pgfpathcurveto{\pgfqpoint{2.069087in}{0.718465in}}{\pgfqpoint{2.063787in}{0.716270in}}{\pgfqpoint{2.059880in}{0.712363in}}%
\pgfpathcurveto{\pgfqpoint{2.055973in}{0.708456in}}{\pgfqpoint{2.053778in}{0.703156in}}{\pgfqpoint{2.053778in}{0.697631in}}%
\pgfpathcurveto{\pgfqpoint{2.053778in}{0.692106in}}{\pgfqpoint{2.055973in}{0.686807in}}{\pgfqpoint{2.059880in}{0.682900in}}%
\pgfpathcurveto{\pgfqpoint{2.063787in}{0.678993in}}{\pgfqpoint{2.069087in}{0.676798in}}{\pgfqpoint{2.074612in}{0.676798in}}%
\pgfpathclose%
\pgfusepath{stroke}%
\end{pgfscope}%
\begin{pgfscope}%
\pgfpathrectangle{\pgfqpoint{0.438556in}{0.383578in}}{\pgfqpoint{2.325000in}{2.310000in}}%
\pgfusepath{clip}%
\pgfsetbuttcap%
\pgfsetroundjoin%
\pgfsetlinewidth{0.803000pt}%
\definecolor{currentstroke}{rgb}{0.686275,0.352941,0.313725}%
\pgfsetstrokecolor{currentstroke}%
\pgfsetdash{}{0pt}%
\pgfpathmoveto{\pgfqpoint{1.257886in}{0.698672in}}%
\pgfpathcurveto{\pgfqpoint{1.263411in}{0.698672in}}{\pgfqpoint{1.268710in}{0.700867in}}{\pgfqpoint{1.272617in}{0.704774in}}%
\pgfpathcurveto{\pgfqpoint{1.276524in}{0.708681in}}{\pgfqpoint{1.278719in}{0.713980in}}{\pgfqpoint{1.278719in}{0.719505in}}%
\pgfpathcurveto{\pgfqpoint{1.278719in}{0.725030in}}{\pgfqpoint{1.276524in}{0.730330in}}{\pgfqpoint{1.272617in}{0.734237in}}%
\pgfpathcurveto{\pgfqpoint{1.268710in}{0.738143in}}{\pgfqpoint{1.263411in}{0.740339in}}{\pgfqpoint{1.257886in}{0.740339in}}%
\pgfpathcurveto{\pgfqpoint{1.252361in}{0.740339in}}{\pgfqpoint{1.247061in}{0.738143in}}{\pgfqpoint{1.243154in}{0.734237in}}%
\pgfpathcurveto{\pgfqpoint{1.239247in}{0.730330in}}{\pgfqpoint{1.237052in}{0.725030in}}{\pgfqpoint{1.237052in}{0.719505in}}%
\pgfpathcurveto{\pgfqpoint{1.237052in}{0.713980in}}{\pgfqpoint{1.239247in}{0.708681in}}{\pgfqpoint{1.243154in}{0.704774in}}%
\pgfpathcurveto{\pgfqpoint{1.247061in}{0.700867in}}{\pgfqpoint{1.252361in}{0.698672in}}{\pgfqpoint{1.257886in}{0.698672in}}%
\pgfpathclose%
\pgfusepath{stroke}%
\end{pgfscope}%
\begin{pgfscope}%
\pgfpathrectangle{\pgfqpoint{0.438556in}{0.383578in}}{\pgfqpoint{2.325000in}{2.310000in}}%
\pgfusepath{clip}%
\pgfsetbuttcap%
\pgfsetroundjoin%
\pgfsetlinewidth{0.803000pt}%
\definecolor{currentstroke}{rgb}{0.686275,0.352941,0.313725}%
\pgfsetstrokecolor{currentstroke}%
\pgfsetdash{}{0pt}%
\pgfpathmoveto{\pgfqpoint{2.211694in}{0.808041in}}%
\pgfpathcurveto{\pgfqpoint{2.217219in}{0.808041in}}{\pgfqpoint{2.222518in}{0.810236in}}{\pgfqpoint{2.226425in}{0.814143in}}%
\pgfpathcurveto{\pgfqpoint{2.230332in}{0.818049in}}{\pgfqpoint{2.232527in}{0.823349in}}{\pgfqpoint{2.232527in}{0.828874in}}%
\pgfpathcurveto{\pgfqpoint{2.232527in}{0.834399in}}{\pgfqpoint{2.230332in}{0.839699in}}{\pgfqpoint{2.226425in}{0.843605in}}%
\pgfpathcurveto{\pgfqpoint{2.222518in}{0.847512in}}{\pgfqpoint{2.217219in}{0.849707in}}{\pgfqpoint{2.211694in}{0.849707in}}%
\pgfpathcurveto{\pgfqpoint{2.206169in}{0.849707in}}{\pgfqpoint{2.200869in}{0.847512in}}{\pgfqpoint{2.196962in}{0.843605in}}%
\pgfpathcurveto{\pgfqpoint{2.193055in}{0.839699in}}{\pgfqpoint{2.190860in}{0.834399in}}{\pgfqpoint{2.190860in}{0.828874in}}%
\pgfpathcurveto{\pgfqpoint{2.190860in}{0.823349in}}{\pgfqpoint{2.193055in}{0.818049in}}{\pgfqpoint{2.196962in}{0.814143in}}%
\pgfpathcurveto{\pgfqpoint{2.200869in}{0.810236in}}{\pgfqpoint{2.206169in}{0.808041in}}{\pgfqpoint{2.211694in}{0.808041in}}%
\pgfpathclose%
\pgfusepath{stroke}%
\end{pgfscope}%
\begin{pgfscope}%
\pgfpathrectangle{\pgfqpoint{0.438556in}{0.383578in}}{\pgfqpoint{2.325000in}{2.310000in}}%
\pgfusepath{clip}%
\pgfsetbuttcap%
\pgfsetroundjoin%
\pgfsetlinewidth{0.803000pt}%
\definecolor{currentstroke}{rgb}{0.686275,0.352941,0.313725}%
\pgfsetstrokecolor{currentstroke}%
\pgfsetdash{}{0pt}%
\pgfpathmoveto{\pgfqpoint{1.197064in}{0.851788in}}%
\pgfpathcurveto{\pgfqpoint{1.202589in}{0.851788in}}{\pgfqpoint{1.207888in}{0.853983in}}{\pgfqpoint{1.211795in}{0.857890in}}%
\pgfpathcurveto{\pgfqpoint{1.215702in}{0.861797in}}{\pgfqpoint{1.217897in}{0.867096in}}{\pgfqpoint{1.217897in}{0.872622in}}%
\pgfpathcurveto{\pgfqpoint{1.217897in}{0.878147in}}{\pgfqpoint{1.215702in}{0.883446in}}{\pgfqpoint{1.211795in}{0.887353in}}%
\pgfpathcurveto{\pgfqpoint{1.207888in}{0.891260in}}{\pgfqpoint{1.202589in}{0.893455in}}{\pgfqpoint{1.197064in}{0.893455in}}%
\pgfpathcurveto{\pgfqpoint{1.191539in}{0.893455in}}{\pgfqpoint{1.186239in}{0.891260in}}{\pgfqpoint{1.182332in}{0.887353in}}%
\pgfpathcurveto{\pgfqpoint{1.178425in}{0.883446in}}{\pgfqpoint{1.176230in}{0.878147in}}{\pgfqpoint{1.176230in}{0.872622in}}%
\pgfpathcurveto{\pgfqpoint{1.176230in}{0.867096in}}{\pgfqpoint{1.178425in}{0.861797in}}{\pgfqpoint{1.182332in}{0.857890in}}%
\pgfpathcurveto{\pgfqpoint{1.186239in}{0.853983in}}{\pgfqpoint{1.191539in}{0.851788in}}{\pgfqpoint{1.197064in}{0.851788in}}%
\pgfpathclose%
\pgfusepath{stroke}%
\end{pgfscope}%
\begin{pgfscope}%
\pgfpathrectangle{\pgfqpoint{0.438556in}{0.383578in}}{\pgfqpoint{2.325000in}{2.310000in}}%
\pgfusepath{clip}%
\pgfsetbuttcap%
\pgfsetroundjoin%
\pgfsetlinewidth{0.803000pt}%
\definecolor{currentstroke}{rgb}{0.686275,0.352941,0.313725}%
\pgfsetstrokecolor{currentstroke}%
\pgfsetdash{}{0pt}%
\pgfpathmoveto{\pgfqpoint{2.221738in}{0.873662in}}%
\pgfpathcurveto{\pgfqpoint{2.227263in}{0.873662in}}{\pgfqpoint{2.232562in}{0.875857in}}{\pgfqpoint{2.236469in}{0.879764in}}%
\pgfpathcurveto{\pgfqpoint{2.240376in}{0.883671in}}{\pgfqpoint{2.242571in}{0.888970in}}{\pgfqpoint{2.242571in}{0.894495in}}%
\pgfpathcurveto{\pgfqpoint{2.242571in}{0.900020in}}{\pgfqpoint{2.240376in}{0.905320in}}{\pgfqpoint{2.236469in}{0.909227in}}%
\pgfpathcurveto{\pgfqpoint{2.232562in}{0.913133in}}{\pgfqpoint{2.227263in}{0.915329in}}{\pgfqpoint{2.221738in}{0.915329in}}%
\pgfpathcurveto{\pgfqpoint{2.216213in}{0.915329in}}{\pgfqpoint{2.210913in}{0.913133in}}{\pgfqpoint{2.207006in}{0.909227in}}%
\pgfpathcurveto{\pgfqpoint{2.203099in}{0.905320in}}{\pgfqpoint{2.200904in}{0.900020in}}{\pgfqpoint{2.200904in}{0.894495in}}%
\pgfpathcurveto{\pgfqpoint{2.200904in}{0.888970in}}{\pgfqpoint{2.203099in}{0.883671in}}{\pgfqpoint{2.207006in}{0.879764in}}%
\pgfpathcurveto{\pgfqpoint{2.210913in}{0.875857in}}{\pgfqpoint{2.216213in}{0.873662in}}{\pgfqpoint{2.221738in}{0.873662in}}%
\pgfpathclose%
\pgfusepath{stroke}%
\end{pgfscope}%
\begin{pgfscope}%
\pgfpathrectangle{\pgfqpoint{0.438556in}{0.383578in}}{\pgfqpoint{2.325000in}{2.310000in}}%
\pgfusepath{clip}%
\pgfsetbuttcap%
\pgfsetroundjoin%
\pgfsetlinewidth{0.803000pt}%
\definecolor{currentstroke}{rgb}{0.686275,0.352941,0.313725}%
\pgfsetstrokecolor{currentstroke}%
\pgfsetdash{}{0pt}%
\pgfpathmoveto{\pgfqpoint{0.649294in}{0.917409in}}%
\pgfpathcurveto{\pgfqpoint{0.654819in}{0.917409in}}{\pgfqpoint{0.660118in}{0.919605in}}{\pgfqpoint{0.664025in}{0.923511in}}%
\pgfpathcurveto{\pgfqpoint{0.667932in}{0.927418in}}{\pgfqpoint{0.670127in}{0.932718in}}{\pgfqpoint{0.670127in}{0.938243in}}%
\pgfpathcurveto{\pgfqpoint{0.670127in}{0.943768in}}{\pgfqpoint{0.667932in}{0.949067in}}{\pgfqpoint{0.664025in}{0.952974in}}%
\pgfpathcurveto{\pgfqpoint{0.660118in}{0.956881in}}{\pgfqpoint{0.654819in}{0.959076in}}{\pgfqpoint{0.649294in}{0.959076in}}%
\pgfpathcurveto{\pgfqpoint{0.643769in}{0.959076in}}{\pgfqpoint{0.638469in}{0.956881in}}{\pgfqpoint{0.634562in}{0.952974in}}%
\pgfpathcurveto{\pgfqpoint{0.630655in}{0.949067in}}{\pgfqpoint{0.628460in}{0.943768in}}{\pgfqpoint{0.628460in}{0.938243in}}%
\pgfpathcurveto{\pgfqpoint{0.628460in}{0.932718in}}{\pgfqpoint{0.630655in}{0.927418in}}{\pgfqpoint{0.634562in}{0.923511in}}%
\pgfpathcurveto{\pgfqpoint{0.638469in}{0.919605in}}{\pgfqpoint{0.643769in}{0.917409in}}{\pgfqpoint{0.649294in}{0.917409in}}%
\pgfpathclose%
\pgfusepath{stroke}%
\end{pgfscope}%
\begin{pgfscope}%
\pgfpathrectangle{\pgfqpoint{0.438556in}{0.383578in}}{\pgfqpoint{2.325000in}{2.310000in}}%
\pgfusepath{clip}%
\pgfsetbuttcap%
\pgfsetroundjoin%
\pgfsetlinewidth{0.803000pt}%
\definecolor{currentstroke}{rgb}{0.686275,0.352941,0.313725}%
\pgfsetstrokecolor{currentstroke}%
\pgfsetdash{}{0pt}%
\pgfpathmoveto{\pgfqpoint{0.759964in}{0.939283in}}%
\pgfpathcurveto{\pgfqpoint{0.765489in}{0.939283in}}{\pgfqpoint{0.770788in}{0.941478in}}{\pgfqpoint{0.774695in}{0.945385in}}%
\pgfpathcurveto{\pgfqpoint{0.778602in}{0.949292in}}{\pgfqpoint{0.780797in}{0.954592in}}{\pgfqpoint{0.780797in}{0.960117in}}%
\pgfpathcurveto{\pgfqpoint{0.780797in}{0.965642in}}{\pgfqpoint{0.778602in}{0.970941in}}{\pgfqpoint{0.774695in}{0.974848in}}%
\pgfpathcurveto{\pgfqpoint{0.770788in}{0.978755in}}{\pgfqpoint{0.765489in}{0.980950in}}{\pgfqpoint{0.759964in}{0.980950in}}%
\pgfpathcurveto{\pgfqpoint{0.754439in}{0.980950in}}{\pgfqpoint{0.749139in}{0.978755in}}{\pgfqpoint{0.745232in}{0.974848in}}%
\pgfpathcurveto{\pgfqpoint{0.741325in}{0.970941in}}{\pgfqpoint{0.739130in}{0.965642in}}{\pgfqpoint{0.739130in}{0.960117in}}%
\pgfpathcurveto{\pgfqpoint{0.739130in}{0.954592in}}{\pgfqpoint{0.741325in}{0.949292in}}{\pgfqpoint{0.745232in}{0.945385in}}%
\pgfpathcurveto{\pgfqpoint{0.749139in}{0.941478in}}{\pgfqpoint{0.754439in}{0.939283in}}{\pgfqpoint{0.759964in}{0.939283in}}%
\pgfpathclose%
\pgfusepath{stroke}%
\end{pgfscope}%
\begin{pgfscope}%
\pgfpathrectangle{\pgfqpoint{0.438556in}{0.383578in}}{\pgfqpoint{2.325000in}{2.310000in}}%
\pgfusepath{clip}%
\pgfsetbuttcap%
\pgfsetroundjoin%
\pgfsetlinewidth{0.803000pt}%
\definecolor{currentstroke}{rgb}{0.686275,0.352941,0.313725}%
\pgfsetstrokecolor{currentstroke}%
\pgfsetdash{}{0pt}%
\pgfpathmoveto{\pgfqpoint{1.796356in}{1.004905in}}%
\pgfpathcurveto{\pgfqpoint{1.801881in}{1.004905in}}{\pgfqpoint{1.807180in}{1.007100in}}{\pgfqpoint{1.811087in}{1.011006in}}%
\pgfpathcurveto{\pgfqpoint{1.814994in}{1.014913in}}{\pgfqpoint{1.817189in}{1.020213in}}{\pgfqpoint{1.817189in}{1.025738in}}%
\pgfpathcurveto{\pgfqpoint{1.817189in}{1.031263in}}{\pgfqpoint{1.814994in}{1.036562in}}{\pgfqpoint{1.811087in}{1.040469in}}%
\pgfpathcurveto{\pgfqpoint{1.807180in}{1.044376in}}{\pgfqpoint{1.801881in}{1.046571in}}{\pgfqpoint{1.796356in}{1.046571in}}%
\pgfpathcurveto{\pgfqpoint{1.790831in}{1.046571in}}{\pgfqpoint{1.785531in}{1.044376in}}{\pgfqpoint{1.781624in}{1.040469in}}%
\pgfpathcurveto{\pgfqpoint{1.777717in}{1.036562in}}{\pgfqpoint{1.775522in}{1.031263in}}{\pgfqpoint{1.775522in}{1.025738in}}%
\pgfpathcurveto{\pgfqpoint{1.775522in}{1.020213in}}{\pgfqpoint{1.777717in}{1.014913in}}{\pgfqpoint{1.781624in}{1.011006in}}%
\pgfpathcurveto{\pgfqpoint{1.785531in}{1.007100in}}{\pgfqpoint{1.790831in}{1.004905in}}{\pgfqpoint{1.796356in}{1.004905in}}%
\pgfpathclose%
\pgfusepath{stroke}%
\end{pgfscope}%
\begin{pgfscope}%
\pgfpathrectangle{\pgfqpoint{0.438556in}{0.383578in}}{\pgfqpoint{2.325000in}{2.310000in}}%
\pgfusepath{clip}%
\pgfsetbuttcap%
\pgfsetroundjoin%
\pgfsetlinewidth{0.803000pt}%
\definecolor{currentstroke}{rgb}{0.686275,0.352941,0.313725}%
\pgfsetstrokecolor{currentstroke}%
\pgfsetdash{}{0pt}%
\pgfpathmoveto{\pgfqpoint{1.087324in}{1.026778in}}%
\pgfpathcurveto{\pgfqpoint{1.092849in}{1.026778in}}{\pgfqpoint{1.098148in}{1.028973in}}{\pgfqpoint{1.102055in}{1.032880in}}%
\pgfpathcurveto{\pgfqpoint{1.105962in}{1.036787in}}{\pgfqpoint{1.108157in}{1.042087in}}{\pgfqpoint{1.108157in}{1.047612in}}%
\pgfpathcurveto{\pgfqpoint{1.108157in}{1.053137in}}{\pgfqpoint{1.105962in}{1.058436in}}{\pgfqpoint{1.102055in}{1.062343in}}%
\pgfpathcurveto{\pgfqpoint{1.098148in}{1.066250in}}{\pgfqpoint{1.092849in}{1.068445in}}{\pgfqpoint{1.087324in}{1.068445in}}%
\pgfpathcurveto{\pgfqpoint{1.081799in}{1.068445in}}{\pgfqpoint{1.076499in}{1.066250in}}{\pgfqpoint{1.072592in}{1.062343in}}%
\pgfpathcurveto{\pgfqpoint{1.068685in}{1.058436in}}{\pgfqpoint{1.066490in}{1.053137in}}{\pgfqpoint{1.066490in}{1.047612in}}%
\pgfpathcurveto{\pgfqpoint{1.066490in}{1.042087in}}{\pgfqpoint{1.068685in}{1.036787in}}{\pgfqpoint{1.072592in}{1.032880in}}%
\pgfpathcurveto{\pgfqpoint{1.076499in}{1.028973in}}{\pgfqpoint{1.081799in}{1.026778in}}{\pgfqpoint{1.087324in}{1.026778in}}%
\pgfpathclose%
\pgfusepath{stroke}%
\end{pgfscope}%
\begin{pgfscope}%
\pgfpathrectangle{\pgfqpoint{0.438556in}{0.383578in}}{\pgfqpoint{2.325000in}{2.310000in}}%
\pgfusepath{clip}%
\pgfsetbuttcap%
\pgfsetroundjoin%
\pgfsetlinewidth{0.803000pt}%
\definecolor{currentstroke}{rgb}{0.686275,0.352941,0.313725}%
\pgfsetstrokecolor{currentstroke}%
\pgfsetdash{}{0pt}%
\pgfpathmoveto{\pgfqpoint{1.119688in}{1.070526in}}%
\pgfpathcurveto{\pgfqpoint{1.125213in}{1.070526in}}{\pgfqpoint{1.130512in}{1.072721in}}{\pgfqpoint{1.134419in}{1.076628in}}%
\pgfpathcurveto{\pgfqpoint{1.138326in}{1.080535in}}{\pgfqpoint{1.140521in}{1.085834in}}{\pgfqpoint{1.140521in}{1.091359in}}%
\pgfpathcurveto{\pgfqpoint{1.140521in}{1.096884in}}{\pgfqpoint{1.138326in}{1.102184in}}{\pgfqpoint{1.134419in}{1.106091in}}%
\pgfpathcurveto{\pgfqpoint{1.130512in}{1.109997in}}{\pgfqpoint{1.125213in}{1.112192in}}{\pgfqpoint{1.119688in}{1.112192in}}%
\pgfpathcurveto{\pgfqpoint{1.114163in}{1.112192in}}{\pgfqpoint{1.108863in}{1.109997in}}{\pgfqpoint{1.104956in}{1.106091in}}%
\pgfpathcurveto{\pgfqpoint{1.101049in}{1.102184in}}{\pgfqpoint{1.098854in}{1.096884in}}{\pgfqpoint{1.098854in}{1.091359in}}%
\pgfpathcurveto{\pgfqpoint{1.098854in}{1.085834in}}{\pgfqpoint{1.101049in}{1.080535in}}{\pgfqpoint{1.104956in}{1.076628in}}%
\pgfpathcurveto{\pgfqpoint{1.108863in}{1.072721in}}{\pgfqpoint{1.114163in}{1.070526in}}{\pgfqpoint{1.119688in}{1.070526in}}%
\pgfpathclose%
\pgfusepath{stroke}%
\end{pgfscope}%
\begin{pgfscope}%
\pgfpathrectangle{\pgfqpoint{0.438556in}{0.383578in}}{\pgfqpoint{2.325000in}{2.310000in}}%
\pgfusepath{clip}%
\pgfsetbuttcap%
\pgfsetroundjoin%
\pgfsetlinewidth{0.803000pt}%
\definecolor{currentstroke}{rgb}{0.686275,0.352941,0.313725}%
\pgfsetstrokecolor{currentstroke}%
\pgfsetdash{}{0pt}%
\pgfpathmoveto{\pgfqpoint{2.116276in}{1.158021in}}%
\pgfpathcurveto{\pgfqpoint{2.121801in}{1.158021in}}{\pgfqpoint{2.127100in}{1.160216in}}{\pgfqpoint{2.131007in}{1.164123in}}%
\pgfpathcurveto{\pgfqpoint{2.134914in}{1.168030in}}{\pgfqpoint{2.137109in}{1.173329in}}{\pgfqpoint{2.137109in}{1.178854in}}%
\pgfpathcurveto{\pgfqpoint{2.137109in}{1.184379in}}{\pgfqpoint{2.134914in}{1.189679in}}{\pgfqpoint{2.131007in}{1.193586in}}%
\pgfpathcurveto{\pgfqpoint{2.127100in}{1.197492in}}{\pgfqpoint{2.121801in}{1.199688in}}{\pgfqpoint{2.116276in}{1.199688in}}%
\pgfpathcurveto{\pgfqpoint{2.110751in}{1.199688in}}{\pgfqpoint{2.105451in}{1.197492in}}{\pgfqpoint{2.101544in}{1.193586in}}%
\pgfpathcurveto{\pgfqpoint{2.097637in}{1.189679in}}{\pgfqpoint{2.095442in}{1.184379in}}{\pgfqpoint{2.095442in}{1.178854in}}%
\pgfpathcurveto{\pgfqpoint{2.095442in}{1.173329in}}{\pgfqpoint{2.097637in}{1.168030in}}{\pgfqpoint{2.101544in}{1.164123in}}%
\pgfpathcurveto{\pgfqpoint{2.105451in}{1.160216in}}{\pgfqpoint{2.110751in}{1.158021in}}{\pgfqpoint{2.116276in}{1.158021in}}%
\pgfpathclose%
\pgfusepath{stroke}%
\end{pgfscope}%
\begin{pgfscope}%
\pgfpathrectangle{\pgfqpoint{0.438556in}{0.383578in}}{\pgfqpoint{2.325000in}{2.310000in}}%
\pgfusepath{clip}%
\pgfsetbuttcap%
\pgfsetroundjoin%
\pgfsetlinewidth{0.803000pt}%
\definecolor{currentstroke}{rgb}{0.686275,0.352941,0.313725}%
\pgfsetstrokecolor{currentstroke}%
\pgfsetdash{}{0pt}%
\pgfpathmoveto{\pgfqpoint{1.028548in}{1.201768in}}%
\pgfpathcurveto{\pgfqpoint{1.034073in}{1.201768in}}{\pgfqpoint{1.039372in}{1.203964in}}{\pgfqpoint{1.043279in}{1.207870in}}%
\pgfpathcurveto{\pgfqpoint{1.047186in}{1.211777in}}{\pgfqpoint{1.049381in}{1.217077in}}{\pgfqpoint{1.049381in}{1.222602in}}%
\pgfpathcurveto{\pgfqpoint{1.049381in}{1.228127in}}{\pgfqpoint{1.047186in}{1.233426in}}{\pgfqpoint{1.043279in}{1.237333in}}%
\pgfpathcurveto{\pgfqpoint{1.039372in}{1.241240in}}{\pgfqpoint{1.034073in}{1.243435in}}{\pgfqpoint{1.028548in}{1.243435in}}%
\pgfpathcurveto{\pgfqpoint{1.023023in}{1.243435in}}{\pgfqpoint{1.017723in}{1.241240in}}{\pgfqpoint{1.013816in}{1.237333in}}%
\pgfpathcurveto{\pgfqpoint{1.009909in}{1.233426in}}{\pgfqpoint{1.007714in}{1.228127in}}{\pgfqpoint{1.007714in}{1.222602in}}%
\pgfpathcurveto{\pgfqpoint{1.007714in}{1.217077in}}{\pgfqpoint{1.009909in}{1.211777in}}{\pgfqpoint{1.013816in}{1.207870in}}%
\pgfpathcurveto{\pgfqpoint{1.017723in}{1.203964in}}{\pgfqpoint{1.023023in}{1.201768in}}{\pgfqpoint{1.028548in}{1.201768in}}%
\pgfpathclose%
\pgfusepath{stroke}%
\end{pgfscope}%
\begin{pgfscope}%
\pgfpathrectangle{\pgfqpoint{0.438556in}{0.383578in}}{\pgfqpoint{2.325000in}{2.310000in}}%
\pgfusepath{clip}%
\pgfsetbuttcap%
\pgfsetroundjoin%
\pgfsetlinewidth{0.803000pt}%
\definecolor{currentstroke}{rgb}{0.686275,0.352941,0.313725}%
\pgfsetstrokecolor{currentstroke}%
\pgfsetdash{}{0pt}%
\pgfpathmoveto{\pgfqpoint{0.903556in}{1.289263in}}%
\pgfpathcurveto{\pgfqpoint{0.909081in}{1.289263in}}{\pgfqpoint{0.914380in}{1.291459in}}{\pgfqpoint{0.918287in}{1.295365in}}%
\pgfpathcurveto{\pgfqpoint{0.922194in}{1.299272in}}{\pgfqpoint{0.924389in}{1.304572in}}{\pgfqpoint{0.924389in}{1.310097in}}%
\pgfpathcurveto{\pgfqpoint{0.924389in}{1.315622in}}{\pgfqpoint{0.922194in}{1.320921in}}{\pgfqpoint{0.918287in}{1.324828in}}%
\pgfpathcurveto{\pgfqpoint{0.914380in}{1.328735in}}{\pgfqpoint{0.909081in}{1.330930in}}{\pgfqpoint{0.903556in}{1.330930in}}%
\pgfpathcurveto{\pgfqpoint{0.898031in}{1.330930in}}{\pgfqpoint{0.892731in}{1.328735in}}{\pgfqpoint{0.888824in}{1.324828in}}%
\pgfpathcurveto{\pgfqpoint{0.884917in}{1.320921in}}{\pgfqpoint{0.882722in}{1.315622in}}{\pgfqpoint{0.882722in}{1.310097in}}%
\pgfpathcurveto{\pgfqpoint{0.882722in}{1.304572in}}{\pgfqpoint{0.884917in}{1.299272in}}{\pgfqpoint{0.888824in}{1.295365in}}%
\pgfpathcurveto{\pgfqpoint{0.892731in}{1.291459in}}{\pgfqpoint{0.898031in}{1.289263in}}{\pgfqpoint{0.903556in}{1.289263in}}%
\pgfpathclose%
\pgfusepath{stroke}%
\end{pgfscope}%
\begin{pgfscope}%
\pgfpathrectangle{\pgfqpoint{0.438556in}{0.383578in}}{\pgfqpoint{2.325000in}{2.310000in}}%
\pgfusepath{clip}%
\pgfsetbuttcap%
\pgfsetroundjoin%
\pgfsetlinewidth{0.803000pt}%
\definecolor{currentstroke}{rgb}{0.686275,0.352941,0.313725}%
\pgfsetstrokecolor{currentstroke}%
\pgfsetdash{}{0pt}%
\pgfpathmoveto{\pgfqpoint{0.522628in}{1.311137in}}%
\pgfpathcurveto{\pgfqpoint{0.528153in}{1.311137in}}{\pgfqpoint{0.533452in}{1.313332in}}{\pgfqpoint{0.537359in}{1.317239in}}%
\pgfpathcurveto{\pgfqpoint{0.541266in}{1.321146in}}{\pgfqpoint{0.543461in}{1.326445in}}{\pgfqpoint{0.543461in}{1.331971in}}%
\pgfpathcurveto{\pgfqpoint{0.543461in}{1.337496in}}{\pgfqpoint{0.541266in}{1.342795in}}{\pgfqpoint{0.537359in}{1.346702in}}%
\pgfpathcurveto{\pgfqpoint{0.533452in}{1.350609in}}{\pgfqpoint{0.528153in}{1.352804in}}{\pgfqpoint{0.522628in}{1.352804in}}%
\pgfpathcurveto{\pgfqpoint{0.517103in}{1.352804in}}{\pgfqpoint{0.511803in}{1.350609in}}{\pgfqpoint{0.507896in}{1.346702in}}%
\pgfpathcurveto{\pgfqpoint{0.503989in}{1.342795in}}{\pgfqpoint{0.501794in}{1.337496in}}{\pgfqpoint{0.501794in}{1.331971in}}%
\pgfpathcurveto{\pgfqpoint{0.501794in}{1.326445in}}{\pgfqpoint{0.503989in}{1.321146in}}{\pgfqpoint{0.507896in}{1.317239in}}%
\pgfpathcurveto{\pgfqpoint{0.511803in}{1.313332in}}{\pgfqpoint{0.517103in}{1.311137in}}{\pgfqpoint{0.522628in}{1.311137in}}%
\pgfpathclose%
\pgfusepath{stroke}%
\end{pgfscope}%
\begin{pgfscope}%
\pgfpathrectangle{\pgfqpoint{0.438556in}{0.383578in}}{\pgfqpoint{2.325000in}{2.310000in}}%
\pgfusepath{clip}%
\pgfsetbuttcap%
\pgfsetroundjoin%
\pgfsetlinewidth{0.803000pt}%
\definecolor{currentstroke}{rgb}{0.686275,0.352941,0.313725}%
\pgfsetstrokecolor{currentstroke}%
\pgfsetdash{}{0pt}%
\pgfpathmoveto{\pgfqpoint{2.002072in}{1.333011in}}%
\pgfpathcurveto{\pgfqpoint{2.007597in}{1.333011in}}{\pgfqpoint{2.012896in}{1.335206in}}{\pgfqpoint{2.016803in}{1.339113in}}%
\pgfpathcurveto{\pgfqpoint{2.020710in}{1.343020in}}{\pgfqpoint{2.022905in}{1.348319in}}{\pgfqpoint{2.022905in}{1.353844in}}%
\pgfpathcurveto{\pgfqpoint{2.022905in}{1.359369in}}{\pgfqpoint{2.020710in}{1.364669in}}{\pgfqpoint{2.016803in}{1.368576in}}%
\pgfpathcurveto{\pgfqpoint{2.012896in}{1.372483in}}{\pgfqpoint{2.007597in}{1.374678in}}{\pgfqpoint{2.002072in}{1.374678in}}%
\pgfpathcurveto{\pgfqpoint{1.996547in}{1.374678in}}{\pgfqpoint{1.991247in}{1.372483in}}{\pgfqpoint{1.987340in}{1.368576in}}%
\pgfpathcurveto{\pgfqpoint{1.983433in}{1.364669in}}{\pgfqpoint{1.981238in}{1.359369in}}{\pgfqpoint{1.981238in}{1.353844in}}%
\pgfpathcurveto{\pgfqpoint{1.981238in}{1.348319in}}{\pgfqpoint{1.983433in}{1.343020in}}{\pgfqpoint{1.987340in}{1.339113in}}%
\pgfpathcurveto{\pgfqpoint{1.991247in}{1.335206in}}{\pgfqpoint{1.996547in}{1.333011in}}{\pgfqpoint{2.002072in}{1.333011in}}%
\pgfpathclose%
\pgfusepath{stroke}%
\end{pgfscope}%
\begin{pgfscope}%
\pgfpathrectangle{\pgfqpoint{0.438556in}{0.383578in}}{\pgfqpoint{2.325000in}{2.310000in}}%
\pgfusepath{clip}%
\pgfsetbuttcap%
\pgfsetroundjoin%
\pgfsetlinewidth{0.803000pt}%
\definecolor{currentstroke}{rgb}{0.686275,0.352941,0.313725}%
\pgfsetstrokecolor{currentstroke}%
\pgfsetdash{}{0pt}%
\pgfpathmoveto{\pgfqpoint{0.952474in}{1.442380in}}%
\pgfpathcurveto{\pgfqpoint{0.957999in}{1.442380in}}{\pgfqpoint{0.963298in}{1.444575in}}{\pgfqpoint{0.967205in}{1.448482in}}%
\pgfpathcurveto{\pgfqpoint{0.971112in}{1.452389in}}{\pgfqpoint{0.973307in}{1.457688in}}{\pgfqpoint{0.973307in}{1.463213in}}%
\pgfpathcurveto{\pgfqpoint{0.973307in}{1.468738in}}{\pgfqpoint{0.971112in}{1.474038in}}{\pgfqpoint{0.967205in}{1.477945in}}%
\pgfpathcurveto{\pgfqpoint{0.963298in}{1.481851in}}{\pgfqpoint{0.957999in}{1.484046in}}{\pgfqpoint{0.952474in}{1.484046in}}%
\pgfpathcurveto{\pgfqpoint{0.946949in}{1.484046in}}{\pgfqpoint{0.941649in}{1.481851in}}{\pgfqpoint{0.937742in}{1.477945in}}%
\pgfpathcurveto{\pgfqpoint{0.933835in}{1.474038in}}{\pgfqpoint{0.931640in}{1.468738in}}{\pgfqpoint{0.931640in}{1.463213in}}%
\pgfpathcurveto{\pgfqpoint{0.931640in}{1.457688in}}{\pgfqpoint{0.933835in}{1.452389in}}{\pgfqpoint{0.937742in}{1.448482in}}%
\pgfpathcurveto{\pgfqpoint{0.941649in}{1.444575in}}{\pgfqpoint{0.946949in}{1.442380in}}{\pgfqpoint{0.952474in}{1.442380in}}%
\pgfpathclose%
\pgfusepath{stroke}%
\end{pgfscope}%
\begin{pgfscope}%
\pgfpathrectangle{\pgfqpoint{0.438556in}{0.383578in}}{\pgfqpoint{2.325000in}{2.310000in}}%
\pgfusepath{clip}%
\pgfsetbuttcap%
\pgfsetroundjoin%
\pgfsetlinewidth{0.803000pt}%
\definecolor{currentstroke}{rgb}{0.686275,0.352941,0.313725}%
\pgfsetstrokecolor{currentstroke}%
\pgfsetdash{}{0pt}%
\pgfpathmoveto{\pgfqpoint{0.528208in}{1.486127in}}%
\pgfpathcurveto{\pgfqpoint{0.533733in}{1.486127in}}{\pgfqpoint{0.539032in}{1.488322in}}{\pgfqpoint{0.542939in}{1.492229in}}%
\pgfpathcurveto{\pgfqpoint{0.546846in}{1.496136in}}{\pgfqpoint{0.549041in}{1.501436in}}{\pgfqpoint{0.549041in}{1.506961in}}%
\pgfpathcurveto{\pgfqpoint{0.549041in}{1.512486in}}{\pgfqpoint{0.546846in}{1.517785in}}{\pgfqpoint{0.542939in}{1.521692in}}%
\pgfpathcurveto{\pgfqpoint{0.539032in}{1.525599in}}{\pgfqpoint{0.533733in}{1.527794in}}{\pgfqpoint{0.528208in}{1.527794in}}%
\pgfpathcurveto{\pgfqpoint{0.522683in}{1.527794in}}{\pgfqpoint{0.517383in}{1.525599in}}{\pgfqpoint{0.513476in}{1.521692in}}%
\pgfpathcurveto{\pgfqpoint{0.509569in}{1.517785in}}{\pgfqpoint{0.507374in}{1.512486in}}{\pgfqpoint{0.507374in}{1.506961in}}%
\pgfpathcurveto{\pgfqpoint{0.507374in}{1.501436in}}{\pgfqpoint{0.509569in}{1.496136in}}{\pgfqpoint{0.513476in}{1.492229in}}%
\pgfpathcurveto{\pgfqpoint{0.517383in}{1.488322in}}{\pgfqpoint{0.522683in}{1.486127in}}{\pgfqpoint{0.528208in}{1.486127in}}%
\pgfpathclose%
\pgfusepath{stroke}%
\end{pgfscope}%
\begin{pgfscope}%
\pgfpathrectangle{\pgfqpoint{0.438556in}{0.383578in}}{\pgfqpoint{2.325000in}{2.310000in}}%
\pgfusepath{clip}%
\pgfsetbuttcap%
\pgfsetroundjoin%
\pgfsetlinewidth{0.803000pt}%
\definecolor{currentstroke}{rgb}{0.686275,0.352941,0.313725}%
\pgfsetstrokecolor{currentstroke}%
\pgfsetdash{}{0pt}%
\pgfpathmoveto{\pgfqpoint{0.803674in}{1.529875in}}%
\pgfpathcurveto{\pgfqpoint{0.809199in}{1.529875in}}{\pgfqpoint{0.814498in}{1.532070in}}{\pgfqpoint{0.818405in}{1.535977in}}%
\pgfpathcurveto{\pgfqpoint{0.822312in}{1.539884in}}{\pgfqpoint{0.824507in}{1.545183in}}{\pgfqpoint{0.824507in}{1.550708in}}%
\pgfpathcurveto{\pgfqpoint{0.824507in}{1.556233in}}{\pgfqpoint{0.822312in}{1.561533in}}{\pgfqpoint{0.818405in}{1.565440in}}%
\pgfpathcurveto{\pgfqpoint{0.814498in}{1.569346in}}{\pgfqpoint{0.809199in}{1.571542in}}{\pgfqpoint{0.803674in}{1.571542in}}%
\pgfpathcurveto{\pgfqpoint{0.798149in}{1.571542in}}{\pgfqpoint{0.792849in}{1.569346in}}{\pgfqpoint{0.788942in}{1.565440in}}%
\pgfpathcurveto{\pgfqpoint{0.785035in}{1.561533in}}{\pgfqpoint{0.782840in}{1.556233in}}{\pgfqpoint{0.782840in}{1.550708in}}%
\pgfpathcurveto{\pgfqpoint{0.782840in}{1.545183in}}{\pgfqpoint{0.785035in}{1.539884in}}{\pgfqpoint{0.788942in}{1.535977in}}%
\pgfpathcurveto{\pgfqpoint{0.792849in}{1.532070in}}{\pgfqpoint{0.798149in}{1.529875in}}{\pgfqpoint{0.803674in}{1.529875in}}%
\pgfpathclose%
\pgfusepath{stroke}%
\end{pgfscope}%
\begin{pgfscope}%
\pgfpathrectangle{\pgfqpoint{0.438556in}{0.383578in}}{\pgfqpoint{2.325000in}{2.310000in}}%
\pgfusepath{clip}%
\pgfsetbuttcap%
\pgfsetroundjoin%
\pgfsetlinewidth{0.803000pt}%
\definecolor{currentstroke}{rgb}{0.686275,0.352941,0.313725}%
\pgfsetstrokecolor{currentstroke}%
\pgfsetdash{}{0pt}%
\pgfpathmoveto{\pgfqpoint{1.867036in}{1.682991in}}%
\pgfpathcurveto{\pgfqpoint{1.872561in}{1.682991in}}{\pgfqpoint{1.877860in}{1.685186in}}{\pgfqpoint{1.881767in}{1.689093in}}%
\pgfpathcurveto{\pgfqpoint{1.885674in}{1.693000in}}{\pgfqpoint{1.887869in}{1.698299in}}{\pgfqpoint{1.887869in}{1.703825in}}%
\pgfpathcurveto{\pgfqpoint{1.887869in}{1.709350in}}{\pgfqpoint{1.885674in}{1.714649in}}{\pgfqpoint{1.881767in}{1.718556in}}%
\pgfpathcurveto{\pgfqpoint{1.877860in}{1.722463in}}{\pgfqpoint{1.872561in}{1.724658in}}{\pgfqpoint{1.867036in}{1.724658in}}%
\pgfpathcurveto{\pgfqpoint{1.861511in}{1.724658in}}{\pgfqpoint{1.856211in}{1.722463in}}{\pgfqpoint{1.852304in}{1.718556in}}%
\pgfpathcurveto{\pgfqpoint{1.848397in}{1.714649in}}{\pgfqpoint{1.846202in}{1.709350in}}{\pgfqpoint{1.846202in}{1.703825in}}%
\pgfpathcurveto{\pgfqpoint{1.846202in}{1.698299in}}{\pgfqpoint{1.848397in}{1.693000in}}{\pgfqpoint{1.852304in}{1.689093in}}%
\pgfpathcurveto{\pgfqpoint{1.856211in}{1.685186in}}{\pgfqpoint{1.861511in}{1.682991in}}{\pgfqpoint{1.867036in}{1.682991in}}%
\pgfpathclose%
\pgfusepath{stroke}%
\end{pgfscope}%
\begin{pgfscope}%
\pgfpathrectangle{\pgfqpoint{0.438556in}{0.383578in}}{\pgfqpoint{2.325000in}{2.310000in}}%
\pgfusepath{clip}%
\pgfsetbuttcap%
\pgfsetroundjoin%
\pgfsetlinewidth{0.803000pt}%
\definecolor{currentstroke}{rgb}{0.686275,0.352941,0.313725}%
\pgfsetstrokecolor{currentstroke}%
\pgfsetdash{}{0pt}%
\pgfpathmoveto{\pgfqpoint{0.945034in}{1.770486in}}%
\pgfpathcurveto{\pgfqpoint{0.950559in}{1.770486in}}{\pgfqpoint{0.955858in}{1.772681in}}{\pgfqpoint{0.959765in}{1.776588in}}%
\pgfpathcurveto{\pgfqpoint{0.963672in}{1.780495in}}{\pgfqpoint{0.965867in}{1.785794in}}{\pgfqpoint{0.965867in}{1.791320in}}%
\pgfpathcurveto{\pgfqpoint{0.965867in}{1.796845in}}{\pgfqpoint{0.963672in}{1.802144in}}{\pgfqpoint{0.959765in}{1.806051in}}%
\pgfpathcurveto{\pgfqpoint{0.955858in}{1.809958in}}{\pgfqpoint{0.950559in}{1.812153in}}{\pgfqpoint{0.945034in}{1.812153in}}%
\pgfpathcurveto{\pgfqpoint{0.939509in}{1.812153in}}{\pgfqpoint{0.934209in}{1.809958in}}{\pgfqpoint{0.930302in}{1.806051in}}%
\pgfpathcurveto{\pgfqpoint{0.926395in}{1.802144in}}{\pgfqpoint{0.924200in}{1.796845in}}{\pgfqpoint{0.924200in}{1.791320in}}%
\pgfpathcurveto{\pgfqpoint{0.924200in}{1.785794in}}{\pgfqpoint{0.926395in}{1.780495in}}{\pgfqpoint{0.930302in}{1.776588in}}%
\pgfpathcurveto{\pgfqpoint{0.934209in}{1.772681in}}{\pgfqpoint{0.939509in}{1.770486in}}{\pgfqpoint{0.945034in}{1.770486in}}%
\pgfpathclose%
\pgfusepath{stroke}%
\end{pgfscope}%
\begin{pgfscope}%
\pgfpathrectangle{\pgfqpoint{0.438556in}{0.383578in}}{\pgfqpoint{2.325000in}{2.310000in}}%
\pgfusepath{clip}%
\pgfsetbuttcap%
\pgfsetroundjoin%
\pgfsetlinewidth{0.803000pt}%
\definecolor{currentstroke}{rgb}{0.686275,0.352941,0.313725}%
\pgfsetstrokecolor{currentstroke}%
\pgfsetdash{}{0pt}%
\pgfpathmoveto{\pgfqpoint{0.647620in}{1.792360in}}%
\pgfpathcurveto{\pgfqpoint{0.653145in}{1.792360in}}{\pgfqpoint{0.658444in}{1.794555in}}{\pgfqpoint{0.662351in}{1.798462in}}%
\pgfpathcurveto{\pgfqpoint{0.666258in}{1.802369in}}{\pgfqpoint{0.668453in}{1.807668in}}{\pgfqpoint{0.668453in}{1.813193in}}%
\pgfpathcurveto{\pgfqpoint{0.668453in}{1.818718in}}{\pgfqpoint{0.666258in}{1.824018in}}{\pgfqpoint{0.662351in}{1.827925in}}%
\pgfpathcurveto{\pgfqpoint{0.658444in}{1.831832in}}{\pgfqpoint{0.653145in}{1.834027in}}{\pgfqpoint{0.647620in}{1.834027in}}%
\pgfpathcurveto{\pgfqpoint{0.642095in}{1.834027in}}{\pgfqpoint{0.636795in}{1.831832in}}{\pgfqpoint{0.632888in}{1.827925in}}%
\pgfpathcurveto{\pgfqpoint{0.628981in}{1.824018in}}{\pgfqpoint{0.626786in}{1.818718in}}{\pgfqpoint{0.626786in}{1.813193in}}%
\pgfpathcurveto{\pgfqpoint{0.626786in}{1.807668in}}{\pgfqpoint{0.628981in}{1.802369in}}{\pgfqpoint{0.632888in}{1.798462in}}%
\pgfpathcurveto{\pgfqpoint{0.636795in}{1.794555in}}{\pgfqpoint{0.642095in}{1.792360in}}{\pgfqpoint{0.647620in}{1.792360in}}%
\pgfpathclose%
\pgfusepath{stroke}%
\end{pgfscope}%
\begin{pgfscope}%
\pgfpathrectangle{\pgfqpoint{0.438556in}{0.383578in}}{\pgfqpoint{2.325000in}{2.310000in}}%
\pgfusepath{clip}%
\pgfsetbuttcap%
\pgfsetroundjoin%
\pgfsetlinewidth{0.803000pt}%
\definecolor{currentstroke}{rgb}{0.686275,0.352941,0.313725}%
\pgfsetstrokecolor{currentstroke}%
\pgfsetdash{}{0pt}%
\pgfpathmoveto{\pgfqpoint{1.362604in}{1.836108in}}%
\pgfpathcurveto{\pgfqpoint{1.368129in}{1.836108in}}{\pgfqpoint{1.373428in}{1.838303in}}{\pgfqpoint{1.377335in}{1.842209in}}%
\pgfpathcurveto{\pgfqpoint{1.381242in}{1.846116in}}{\pgfqpoint{1.383437in}{1.851416in}}{\pgfqpoint{1.383437in}{1.856941in}}%
\pgfpathcurveto{\pgfqpoint{1.383437in}{1.862466in}}{\pgfqpoint{1.381242in}{1.867765in}}{\pgfqpoint{1.377335in}{1.871672in}}%
\pgfpathcurveto{\pgfqpoint{1.373428in}{1.875579in}}{\pgfqpoint{1.368129in}{1.877774in}}{\pgfqpoint{1.362604in}{1.877774in}}%
\pgfpathcurveto{\pgfqpoint{1.357079in}{1.877774in}}{\pgfqpoint{1.351779in}{1.875579in}}{\pgfqpoint{1.347872in}{1.871672in}}%
\pgfpathcurveto{\pgfqpoint{1.343965in}{1.867765in}}{\pgfqpoint{1.341770in}{1.862466in}}{\pgfqpoint{1.341770in}{1.856941in}}%
\pgfpathcurveto{\pgfqpoint{1.341770in}{1.851416in}}{\pgfqpoint{1.343965in}{1.846116in}}{\pgfqpoint{1.347872in}{1.842209in}}%
\pgfpathcurveto{\pgfqpoint{1.351779in}{1.838303in}}{\pgfqpoint{1.357079in}{1.836108in}}{\pgfqpoint{1.362604in}{1.836108in}}%
\pgfpathclose%
\pgfusepath{stroke}%
\end{pgfscope}%
\begin{pgfscope}%
\pgfpathrectangle{\pgfqpoint{0.438556in}{0.383578in}}{\pgfqpoint{2.325000in}{2.310000in}}%
\pgfusepath{clip}%
\pgfsetbuttcap%
\pgfsetroundjoin%
\pgfsetlinewidth{0.803000pt}%
\definecolor{currentstroke}{rgb}{0.686275,0.352941,0.313725}%
\pgfsetstrokecolor{currentstroke}%
\pgfsetdash{}{0pt}%
\pgfpathmoveto{\pgfqpoint{1.450024in}{1.857981in}}%
\pgfpathcurveto{\pgfqpoint{1.455549in}{1.857981in}}{\pgfqpoint{1.460848in}{1.860176in}}{\pgfqpoint{1.464755in}{1.864083in}}%
\pgfpathcurveto{\pgfqpoint{1.468662in}{1.867990in}}{\pgfqpoint{1.470857in}{1.873290in}}{\pgfqpoint{1.470857in}{1.878815in}}%
\pgfpathcurveto{\pgfqpoint{1.470857in}{1.884340in}}{\pgfqpoint{1.468662in}{1.889639in}}{\pgfqpoint{1.464755in}{1.893546in}}%
\pgfpathcurveto{\pgfqpoint{1.460848in}{1.897453in}}{\pgfqpoint{1.455549in}{1.899648in}}{\pgfqpoint{1.450024in}{1.899648in}}%
\pgfpathcurveto{\pgfqpoint{1.444499in}{1.899648in}}{\pgfqpoint{1.439199in}{1.897453in}}{\pgfqpoint{1.435292in}{1.893546in}}%
\pgfpathcurveto{\pgfqpoint{1.431385in}{1.889639in}}{\pgfqpoint{1.429190in}{1.884340in}}{\pgfqpoint{1.429190in}{1.878815in}}%
\pgfpathcurveto{\pgfqpoint{1.429190in}{1.873290in}}{\pgfqpoint{1.431385in}{1.867990in}}{\pgfqpoint{1.435292in}{1.864083in}}%
\pgfpathcurveto{\pgfqpoint{1.439199in}{1.860176in}}{\pgfqpoint{1.444499in}{1.857981in}}{\pgfqpoint{1.450024in}{1.857981in}}%
\pgfpathclose%
\pgfusepath{stroke}%
\end{pgfscope}%
\begin{pgfscope}%
\pgfpathrectangle{\pgfqpoint{0.438556in}{0.383578in}}{\pgfqpoint{2.325000in}{2.310000in}}%
\pgfusepath{clip}%
\pgfsetbuttcap%
\pgfsetroundjoin%
\pgfsetlinewidth{0.803000pt}%
\definecolor{currentstroke}{rgb}{0.686275,0.352941,0.313725}%
\pgfsetstrokecolor{currentstroke}%
\pgfsetdash{}{0pt}%
\pgfpathmoveto{\pgfqpoint{1.260490in}{1.923603in}}%
\pgfpathcurveto{\pgfqpoint{1.266015in}{1.923603in}}{\pgfqpoint{1.271314in}{1.925798in}}{\pgfqpoint{1.275221in}{1.929705in}}%
\pgfpathcurveto{\pgfqpoint{1.279128in}{1.933611in}}{\pgfqpoint{1.281323in}{1.938911in}}{\pgfqpoint{1.281323in}{1.944436in}}%
\pgfpathcurveto{\pgfqpoint{1.281323in}{1.949961in}}{\pgfqpoint{1.279128in}{1.955260in}}{\pgfqpoint{1.275221in}{1.959167in}}%
\pgfpathcurveto{\pgfqpoint{1.271314in}{1.963074in}}{\pgfqpoint{1.266015in}{1.965269in}}{\pgfqpoint{1.260490in}{1.965269in}}%
\pgfpathcurveto{\pgfqpoint{1.254965in}{1.965269in}}{\pgfqpoint{1.249665in}{1.963074in}}{\pgfqpoint{1.245758in}{1.959167in}}%
\pgfpathcurveto{\pgfqpoint{1.241851in}{1.955260in}}{\pgfqpoint{1.239656in}{1.949961in}}{\pgfqpoint{1.239656in}{1.944436in}}%
\pgfpathcurveto{\pgfqpoint{1.239656in}{1.938911in}}{\pgfqpoint{1.241851in}{1.933611in}}{\pgfqpoint{1.245758in}{1.929705in}}%
\pgfpathcurveto{\pgfqpoint{1.249665in}{1.925798in}}{\pgfqpoint{1.254965in}{1.923603in}}{\pgfqpoint{1.260490in}{1.923603in}}%
\pgfpathclose%
\pgfusepath{stroke}%
\end{pgfscope}%
\begin{pgfscope}%
\pgfpathrectangle{\pgfqpoint{0.438556in}{0.383578in}}{\pgfqpoint{2.325000in}{2.310000in}}%
\pgfusepath{clip}%
\pgfsetbuttcap%
\pgfsetroundjoin%
\pgfsetlinewidth{0.803000pt}%
\definecolor{currentstroke}{rgb}{0.686275,0.352941,0.313725}%
\pgfsetstrokecolor{currentstroke}%
\pgfsetdash{}{0pt}%
\pgfpathmoveto{\pgfqpoint{0.797164in}{1.989224in}}%
\pgfpathcurveto{\pgfqpoint{0.802689in}{1.989224in}}{\pgfqpoint{0.807988in}{1.991419in}}{\pgfqpoint{0.811895in}{1.995326in}}%
\pgfpathcurveto{\pgfqpoint{0.815802in}{1.999233in}}{\pgfqpoint{0.817997in}{2.004532in}}{\pgfqpoint{0.817997in}{2.010057in}}%
\pgfpathcurveto{\pgfqpoint{0.817997in}{2.015582in}}{\pgfqpoint{0.815802in}{2.020882in}}{\pgfqpoint{0.811895in}{2.024789in}}%
\pgfpathcurveto{\pgfqpoint{0.807988in}{2.028695in}}{\pgfqpoint{0.802689in}{2.030891in}}{\pgfqpoint{0.797164in}{2.030891in}}%
\pgfpathcurveto{\pgfqpoint{0.791639in}{2.030891in}}{\pgfqpoint{0.786339in}{2.028695in}}{\pgfqpoint{0.782432in}{2.024789in}}%
\pgfpathcurveto{\pgfqpoint{0.778525in}{2.020882in}}{\pgfqpoint{0.776330in}{2.015582in}}{\pgfqpoint{0.776330in}{2.010057in}}%
\pgfpathcurveto{\pgfqpoint{0.776330in}{2.004532in}}{\pgfqpoint{0.778525in}{1.999233in}}{\pgfqpoint{0.782432in}{1.995326in}}%
\pgfpathcurveto{\pgfqpoint{0.786339in}{1.991419in}}{\pgfqpoint{0.791639in}{1.989224in}}{\pgfqpoint{0.797164in}{1.989224in}}%
\pgfpathclose%
\pgfusepath{stroke}%
\end{pgfscope}%
\begin{pgfscope}%
\pgfpathrectangle{\pgfqpoint{0.438556in}{0.383578in}}{\pgfqpoint{2.325000in}{2.310000in}}%
\pgfusepath{clip}%
\pgfsetbuttcap%
\pgfsetroundjoin%
\pgfsetlinewidth{0.803000pt}%
\definecolor{currentstroke}{rgb}{0.686275,0.352941,0.313725}%
\pgfsetstrokecolor{currentstroke}%
\pgfsetdash{}{0pt}%
\pgfpathmoveto{\pgfqpoint{1.288948in}{2.011098in}}%
\pgfpathcurveto{\pgfqpoint{1.294473in}{2.011098in}}{\pgfqpoint{1.299772in}{2.013293in}}{\pgfqpoint{1.303679in}{2.017200in}}%
\pgfpathcurveto{\pgfqpoint{1.307586in}{2.021106in}}{\pgfqpoint{1.309781in}{2.026406in}}{\pgfqpoint{1.309781in}{2.031931in}}%
\pgfpathcurveto{\pgfqpoint{1.309781in}{2.037456in}}{\pgfqpoint{1.307586in}{2.042756in}}{\pgfqpoint{1.303679in}{2.046662in}}%
\pgfpathcurveto{\pgfqpoint{1.299772in}{2.050569in}}{\pgfqpoint{1.294473in}{2.052764in}}{\pgfqpoint{1.288948in}{2.052764in}}%
\pgfpathcurveto{\pgfqpoint{1.283423in}{2.052764in}}{\pgfqpoint{1.278123in}{2.050569in}}{\pgfqpoint{1.274216in}{2.046662in}}%
\pgfpathcurveto{\pgfqpoint{1.270309in}{2.042756in}}{\pgfqpoint{1.268114in}{2.037456in}}{\pgfqpoint{1.268114in}{2.031931in}}%
\pgfpathcurveto{\pgfqpoint{1.268114in}{2.026406in}}{\pgfqpoint{1.270309in}{2.021106in}}{\pgfqpoint{1.274216in}{2.017200in}}%
\pgfpathcurveto{\pgfqpoint{1.278123in}{2.013293in}}{\pgfqpoint{1.283423in}{2.011098in}}{\pgfqpoint{1.288948in}{2.011098in}}%
\pgfpathclose%
\pgfusepath{stroke}%
\end{pgfscope}%
\begin{pgfscope}%
\pgfpathrectangle{\pgfqpoint{0.438556in}{0.383578in}}{\pgfqpoint{2.325000in}{2.310000in}}%
\pgfusepath{clip}%
\pgfsetbuttcap%
\pgfsetroundjoin%
\pgfsetlinewidth{0.803000pt}%
\definecolor{currentstroke}{rgb}{0.686275,0.352941,0.313725}%
\pgfsetstrokecolor{currentstroke}%
\pgfsetdash{}{0pt}%
\pgfpathmoveto{\pgfqpoint{0.878818in}{2.076719in}}%
\pgfpathcurveto{\pgfqpoint{0.884343in}{2.076719in}}{\pgfqpoint{0.889642in}{2.078914in}}{\pgfqpoint{0.893549in}{2.082821in}}%
\pgfpathcurveto{\pgfqpoint{0.897456in}{2.086728in}}{\pgfqpoint{0.899651in}{2.092027in}}{\pgfqpoint{0.899651in}{2.097552in}}%
\pgfpathcurveto{\pgfqpoint{0.899651in}{2.103077in}}{\pgfqpoint{0.897456in}{2.108377in}}{\pgfqpoint{0.893549in}{2.112284in}}%
\pgfpathcurveto{\pgfqpoint{0.889642in}{2.116190in}}{\pgfqpoint{0.884343in}{2.118386in}}{\pgfqpoint{0.878818in}{2.118386in}}%
\pgfpathcurveto{\pgfqpoint{0.873293in}{2.118386in}}{\pgfqpoint{0.867993in}{2.116190in}}{\pgfqpoint{0.864086in}{2.112284in}}%
\pgfpathcurveto{\pgfqpoint{0.860179in}{2.108377in}}{\pgfqpoint{0.857984in}{2.103077in}}{\pgfqpoint{0.857984in}{2.097552in}}%
\pgfpathcurveto{\pgfqpoint{0.857984in}{2.092027in}}{\pgfqpoint{0.860179in}{2.086728in}}{\pgfqpoint{0.864086in}{2.082821in}}%
\pgfpathcurveto{\pgfqpoint{0.867993in}{2.078914in}}{\pgfqpoint{0.873293in}{2.076719in}}{\pgfqpoint{0.878818in}{2.076719in}}%
\pgfpathclose%
\pgfusepath{stroke}%
\end{pgfscope}%
\begin{pgfscope}%
\pgfpathrectangle{\pgfqpoint{0.438556in}{0.383578in}}{\pgfqpoint{2.325000in}{2.310000in}}%
\pgfusepath{clip}%
\pgfsetbuttcap%
\pgfsetroundjoin%
\pgfsetlinewidth{0.803000pt}%
\definecolor{currentstroke}{rgb}{0.686275,0.352941,0.313725}%
\pgfsetstrokecolor{currentstroke}%
\pgfsetdash{}{0pt}%
\pgfpathmoveto{\pgfqpoint{1.645696in}{2.098593in}}%
\pgfpathcurveto{\pgfqpoint{1.651221in}{2.098593in}}{\pgfqpoint{1.656520in}{2.100788in}}{\pgfqpoint{1.660427in}{2.104695in}}%
\pgfpathcurveto{\pgfqpoint{1.664334in}{2.108601in}}{\pgfqpoint{1.666529in}{2.113901in}}{\pgfqpoint{1.666529in}{2.119426in}}%
\pgfpathcurveto{\pgfqpoint{1.666529in}{2.124951in}}{\pgfqpoint{1.664334in}{2.130251in}}{\pgfqpoint{1.660427in}{2.134157in}}%
\pgfpathcurveto{\pgfqpoint{1.656520in}{2.138064in}}{\pgfqpoint{1.651221in}{2.140259in}}{\pgfqpoint{1.645696in}{2.140259in}}%
\pgfpathcurveto{\pgfqpoint{1.640171in}{2.140259in}}{\pgfqpoint{1.634871in}{2.138064in}}{\pgfqpoint{1.630964in}{2.134157in}}%
\pgfpathcurveto{\pgfqpoint{1.627057in}{2.130251in}}{\pgfqpoint{1.624862in}{2.124951in}}{\pgfqpoint{1.624862in}{2.119426in}}%
\pgfpathcurveto{\pgfqpoint{1.624862in}{2.113901in}}{\pgfqpoint{1.627057in}{2.108601in}}{\pgfqpoint{1.630964in}{2.104695in}}%
\pgfpathcurveto{\pgfqpoint{1.634871in}{2.100788in}}{\pgfqpoint{1.640171in}{2.098593in}}{\pgfqpoint{1.645696in}{2.098593in}}%
\pgfpathclose%
\pgfusepath{stroke}%
\end{pgfscope}%
\begin{pgfscope}%
\pgfpathrectangle{\pgfqpoint{0.438556in}{0.383578in}}{\pgfqpoint{2.325000in}{2.310000in}}%
\pgfusepath{clip}%
\pgfsetbuttcap%
\pgfsetroundjoin%
\pgfsetlinewidth{0.803000pt}%
\definecolor{currentstroke}{rgb}{0.686275,0.352941,0.313725}%
\pgfsetstrokecolor{currentstroke}%
\pgfsetdash{}{0pt}%
\pgfpathmoveto{\pgfqpoint{2.090236in}{2.164214in}}%
\pgfpathcurveto{\pgfqpoint{2.095761in}{2.164214in}}{\pgfqpoint{2.101060in}{2.166409in}}{\pgfqpoint{2.104967in}{2.170316in}}%
\pgfpathcurveto{\pgfqpoint{2.108874in}{2.174223in}}{\pgfqpoint{2.111069in}{2.179522in}}{\pgfqpoint{2.111069in}{2.185047in}}%
\pgfpathcurveto{\pgfqpoint{2.111069in}{2.190572in}}{\pgfqpoint{2.108874in}{2.195872in}}{\pgfqpoint{2.104967in}{2.199779in}}%
\pgfpathcurveto{\pgfqpoint{2.101060in}{2.203685in}}{\pgfqpoint{2.095761in}{2.205881in}}{\pgfqpoint{2.090236in}{2.205881in}}%
\pgfpathcurveto{\pgfqpoint{2.084711in}{2.205881in}}{\pgfqpoint{2.079411in}{2.203685in}}{\pgfqpoint{2.075504in}{2.199779in}}%
\pgfpathcurveto{\pgfqpoint{2.071597in}{2.195872in}}{\pgfqpoint{2.069402in}{2.190572in}}{\pgfqpoint{2.069402in}{2.185047in}}%
\pgfpathcurveto{\pgfqpoint{2.069402in}{2.179522in}}{\pgfqpoint{2.071597in}{2.174223in}}{\pgfqpoint{2.075504in}{2.170316in}}%
\pgfpathcurveto{\pgfqpoint{2.079411in}{2.166409in}}{\pgfqpoint{2.084711in}{2.164214in}}{\pgfqpoint{2.090236in}{2.164214in}}%
\pgfpathclose%
\pgfusepath{stroke}%
\end{pgfscope}%
\begin{pgfscope}%
\pgfpathrectangle{\pgfqpoint{0.438556in}{0.383578in}}{\pgfqpoint{2.325000in}{2.310000in}}%
\pgfusepath{clip}%
\pgfsetbuttcap%
\pgfsetroundjoin%
\pgfsetlinewidth{0.803000pt}%
\definecolor{currentstroke}{rgb}{0.686275,0.352941,0.313725}%
\pgfsetstrokecolor{currentstroke}%
\pgfsetdash{}{0pt}%
\pgfpathmoveto{\pgfqpoint{1.452256in}{2.229835in}}%
\pgfpathcurveto{\pgfqpoint{1.457781in}{2.229835in}}{\pgfqpoint{1.463080in}{2.232030in}}{\pgfqpoint{1.466987in}{2.235937in}}%
\pgfpathcurveto{\pgfqpoint{1.470894in}{2.239844in}}{\pgfqpoint{1.473089in}{2.245144in}}{\pgfqpoint{1.473089in}{2.250669in}}%
\pgfpathcurveto{\pgfqpoint{1.473089in}{2.256194in}}{\pgfqpoint{1.470894in}{2.261493in}}{\pgfqpoint{1.466987in}{2.265400in}}%
\pgfpathcurveto{\pgfqpoint{1.463080in}{2.269307in}}{\pgfqpoint{1.457781in}{2.271502in}}{\pgfqpoint{1.452256in}{2.271502in}}%
\pgfpathcurveto{\pgfqpoint{1.446731in}{2.271502in}}{\pgfqpoint{1.441431in}{2.269307in}}{\pgfqpoint{1.437524in}{2.265400in}}%
\pgfpathcurveto{\pgfqpoint{1.433617in}{2.261493in}}{\pgfqpoint{1.431422in}{2.256194in}}{\pgfqpoint{1.431422in}{2.250669in}}%
\pgfpathcurveto{\pgfqpoint{1.431422in}{2.245144in}}{\pgfqpoint{1.433617in}{2.239844in}}{\pgfqpoint{1.437524in}{2.235937in}}%
\pgfpathcurveto{\pgfqpoint{1.441431in}{2.232030in}}{\pgfqpoint{1.446731in}{2.229835in}}{\pgfqpoint{1.452256in}{2.229835in}}%
\pgfpathclose%
\pgfusepath{stroke}%
\end{pgfscope}%
\begin{pgfscope}%
\pgfpathrectangle{\pgfqpoint{0.438556in}{0.383578in}}{\pgfqpoint{2.325000in}{2.310000in}}%
\pgfusepath{clip}%
\pgfsetbuttcap%
\pgfsetroundjoin%
\pgfsetlinewidth{0.803000pt}%
\definecolor{currentstroke}{rgb}{0.686275,0.352941,0.313725}%
\pgfsetstrokecolor{currentstroke}%
\pgfsetdash{}{0pt}%
\pgfpathmoveto{\pgfqpoint{1.251934in}{2.251709in}}%
\pgfpathcurveto{\pgfqpoint{1.257459in}{2.251709in}}{\pgfqpoint{1.262758in}{2.253904in}}{\pgfqpoint{1.266665in}{2.257811in}}%
\pgfpathcurveto{\pgfqpoint{1.270572in}{2.261718in}}{\pgfqpoint{1.272767in}{2.267017in}}{\pgfqpoint{1.272767in}{2.272542in}}%
\pgfpathcurveto{\pgfqpoint{1.272767in}{2.278067in}}{\pgfqpoint{1.270572in}{2.283367in}}{\pgfqpoint{1.266665in}{2.287274in}}%
\pgfpathcurveto{\pgfqpoint{1.262758in}{2.291181in}}{\pgfqpoint{1.257459in}{2.293376in}}{\pgfqpoint{1.251934in}{2.293376in}}%
\pgfpathcurveto{\pgfqpoint{1.246409in}{2.293376in}}{\pgfqpoint{1.241109in}{2.291181in}}{\pgfqpoint{1.237202in}{2.287274in}}%
\pgfpathcurveto{\pgfqpoint{1.233295in}{2.283367in}}{\pgfqpoint{1.231100in}{2.278067in}}{\pgfqpoint{1.231100in}{2.272542in}}%
\pgfpathcurveto{\pgfqpoint{1.231100in}{2.267017in}}{\pgfqpoint{1.233295in}{2.261718in}}{\pgfqpoint{1.237202in}{2.257811in}}%
\pgfpathcurveto{\pgfqpoint{1.241109in}{2.253904in}}{\pgfqpoint{1.246409in}{2.251709in}}{\pgfqpoint{1.251934in}{2.251709in}}%
\pgfpathclose%
\pgfusepath{stroke}%
\end{pgfscope}%
\begin{pgfscope}%
\pgfpathrectangle{\pgfqpoint{0.438556in}{0.383578in}}{\pgfqpoint{2.325000in}{2.310000in}}%
\pgfusepath{clip}%
\pgfsetbuttcap%
\pgfsetroundjoin%
\pgfsetlinewidth{0.803000pt}%
\definecolor{currentstroke}{rgb}{0.686275,0.352941,0.313725}%
\pgfsetstrokecolor{currentstroke}%
\pgfsetdash{}{0pt}%
\pgfpathmoveto{\pgfqpoint{1.428820in}{2.295457in}}%
\pgfpathcurveto{\pgfqpoint{1.434345in}{2.295457in}}{\pgfqpoint{1.439644in}{2.297652in}}{\pgfqpoint{1.443551in}{2.301558in}}%
\pgfpathcurveto{\pgfqpoint{1.447458in}{2.305465in}}{\pgfqpoint{1.449653in}{2.310765in}}{\pgfqpoint{1.449653in}{2.316290in}}%
\pgfpathcurveto{\pgfqpoint{1.449653in}{2.321815in}}{\pgfqpoint{1.447458in}{2.327114in}}{\pgfqpoint{1.443551in}{2.331021in}}%
\pgfpathcurveto{\pgfqpoint{1.439644in}{2.334928in}}{\pgfqpoint{1.434345in}{2.337123in}}{\pgfqpoint{1.428820in}{2.337123in}}%
\pgfpathcurveto{\pgfqpoint{1.423295in}{2.337123in}}{\pgfqpoint{1.417995in}{2.334928in}}{\pgfqpoint{1.414088in}{2.331021in}}%
\pgfpathcurveto{\pgfqpoint{1.410181in}{2.327114in}}{\pgfqpoint{1.407986in}{2.321815in}}{\pgfqpoint{1.407986in}{2.316290in}}%
\pgfpathcurveto{\pgfqpoint{1.407986in}{2.310765in}}{\pgfqpoint{1.410181in}{2.305465in}}{\pgfqpoint{1.414088in}{2.301558in}}%
\pgfpathcurveto{\pgfqpoint{1.417995in}{2.297652in}}{\pgfqpoint{1.423295in}{2.295457in}}{\pgfqpoint{1.428820in}{2.295457in}}%
\pgfpathclose%
\pgfusepath{stroke}%
\end{pgfscope}%
\begin{pgfscope}%
\pgfpathrectangle{\pgfqpoint{0.438556in}{0.383578in}}{\pgfqpoint{2.325000in}{2.310000in}}%
\pgfusepath{clip}%
\pgfsetbuttcap%
\pgfsetroundjoin%
\pgfsetlinewidth{0.803000pt}%
\definecolor{currentstroke}{rgb}{0.686275,0.352941,0.313725}%
\pgfsetstrokecolor{currentstroke}%
\pgfsetdash{}{0pt}%
\pgfpathmoveto{\pgfqpoint{1.424728in}{2.317330in}}%
\pgfpathcurveto{\pgfqpoint{1.430253in}{2.317330in}}{\pgfqpoint{1.435552in}{2.319525in}}{\pgfqpoint{1.439459in}{2.323432in}}%
\pgfpathcurveto{\pgfqpoint{1.443366in}{2.327339in}}{\pgfqpoint{1.445561in}{2.332639in}}{\pgfqpoint{1.445561in}{2.338164in}}%
\pgfpathcurveto{\pgfqpoint{1.445561in}{2.343689in}}{\pgfqpoint{1.443366in}{2.348988in}}{\pgfqpoint{1.439459in}{2.352895in}}%
\pgfpathcurveto{\pgfqpoint{1.435552in}{2.356802in}}{\pgfqpoint{1.430253in}{2.358997in}}{\pgfqpoint{1.424728in}{2.358997in}}%
\pgfpathcurveto{\pgfqpoint{1.419203in}{2.358997in}}{\pgfqpoint{1.413903in}{2.356802in}}{\pgfqpoint{1.409996in}{2.352895in}}%
\pgfpathcurveto{\pgfqpoint{1.406089in}{2.348988in}}{\pgfqpoint{1.403894in}{2.343689in}}{\pgfqpoint{1.403894in}{2.338164in}}%
\pgfpathcurveto{\pgfqpoint{1.403894in}{2.332639in}}{\pgfqpoint{1.406089in}{2.327339in}}{\pgfqpoint{1.409996in}{2.323432in}}%
\pgfpathcurveto{\pgfqpoint{1.413903in}{2.319525in}}{\pgfqpoint{1.419203in}{2.317330in}}{\pgfqpoint{1.424728in}{2.317330in}}%
\pgfpathclose%
\pgfusepath{stroke}%
\end{pgfscope}%
\begin{pgfscope}%
\pgfpathrectangle{\pgfqpoint{0.438556in}{0.383578in}}{\pgfqpoint{2.325000in}{2.310000in}}%
\pgfusepath{clip}%
\pgfsetbuttcap%
\pgfsetroundjoin%
\pgfsetlinewidth{0.803000pt}%
\definecolor{currentstroke}{rgb}{0.686275,0.352941,0.313725}%
\pgfsetstrokecolor{currentstroke}%
\pgfsetdash{}{0pt}%
\pgfpathmoveto{\pgfqpoint{1.792264in}{2.339204in}}%
\pgfpathcurveto{\pgfqpoint{1.797789in}{2.339204in}}{\pgfqpoint{1.803088in}{2.341399in}}{\pgfqpoint{1.806995in}{2.345306in}}%
\pgfpathcurveto{\pgfqpoint{1.810902in}{2.349213in}}{\pgfqpoint{1.813097in}{2.354512in}}{\pgfqpoint{1.813097in}{2.360037in}}%
\pgfpathcurveto{\pgfqpoint{1.813097in}{2.365562in}}{\pgfqpoint{1.810902in}{2.370862in}}{\pgfqpoint{1.806995in}{2.374769in}}%
\pgfpathcurveto{\pgfqpoint{1.803088in}{2.378676in}}{\pgfqpoint{1.797789in}{2.380871in}}{\pgfqpoint{1.792264in}{2.380871in}}%
\pgfpathcurveto{\pgfqpoint{1.786739in}{2.380871in}}{\pgfqpoint{1.781439in}{2.378676in}}{\pgfqpoint{1.777532in}{2.374769in}}%
\pgfpathcurveto{\pgfqpoint{1.773625in}{2.370862in}}{\pgfqpoint{1.771430in}{2.365562in}}{\pgfqpoint{1.771430in}{2.360037in}}%
\pgfpathcurveto{\pgfqpoint{1.771430in}{2.354512in}}{\pgfqpoint{1.773625in}{2.349213in}}{\pgfqpoint{1.777532in}{2.345306in}}%
\pgfpathcurveto{\pgfqpoint{1.781439in}{2.341399in}}{\pgfqpoint{1.786739in}{2.339204in}}{\pgfqpoint{1.792264in}{2.339204in}}%
\pgfpathclose%
\pgfusepath{stroke}%
\end{pgfscope}%
\begin{pgfscope}%
\pgfpathrectangle{\pgfqpoint{0.438556in}{0.383578in}}{\pgfqpoint{2.325000in}{2.310000in}}%
\pgfusepath{clip}%
\pgfsetbuttcap%
\pgfsetroundjoin%
\pgfsetlinewidth{0.803000pt}%
\definecolor{currentstroke}{rgb}{0.686275,0.352941,0.313725}%
\pgfsetstrokecolor{currentstroke}%
\pgfsetdash{}{0pt}%
\pgfpathmoveto{\pgfqpoint{2.199604in}{2.382952in}}%
\pgfpathcurveto{\pgfqpoint{2.205129in}{2.382952in}}{\pgfqpoint{2.210428in}{2.385147in}}{\pgfqpoint{2.214335in}{2.389054in}}%
\pgfpathcurveto{\pgfqpoint{2.218242in}{2.392960in}}{\pgfqpoint{2.220437in}{2.398260in}}{\pgfqpoint{2.220437in}{2.403785in}}%
\pgfpathcurveto{\pgfqpoint{2.220437in}{2.409310in}}{\pgfqpoint{2.218242in}{2.414609in}}{\pgfqpoint{2.214335in}{2.418516in}}%
\pgfpathcurveto{\pgfqpoint{2.210428in}{2.422423in}}{\pgfqpoint{2.205129in}{2.424618in}}{\pgfqpoint{2.199604in}{2.424618in}}%
\pgfpathcurveto{\pgfqpoint{2.194079in}{2.424618in}}{\pgfqpoint{2.188779in}{2.422423in}}{\pgfqpoint{2.184872in}{2.418516in}}%
\pgfpathcurveto{\pgfqpoint{2.180965in}{2.414609in}}{\pgfqpoint{2.178770in}{2.409310in}}{\pgfqpoint{2.178770in}{2.403785in}}%
\pgfpathcurveto{\pgfqpoint{2.178770in}{2.398260in}}{\pgfqpoint{2.180965in}{2.392960in}}{\pgfqpoint{2.184872in}{2.389054in}}%
\pgfpathcurveto{\pgfqpoint{2.188779in}{2.385147in}}{\pgfqpoint{2.194079in}{2.382952in}}{\pgfqpoint{2.199604in}{2.382952in}}%
\pgfpathclose%
\pgfusepath{stroke}%
\end{pgfscope}%
\begin{pgfscope}%
\pgfpathrectangle{\pgfqpoint{0.438556in}{0.383578in}}{\pgfqpoint{2.325000in}{2.310000in}}%
\pgfusepath{clip}%
\pgfsetbuttcap%
\pgfsetroundjoin%
\pgfsetlinewidth{0.803000pt}%
\definecolor{currentstroke}{rgb}{0.686275,0.352941,0.313725}%
\pgfsetstrokecolor{currentstroke}%
\pgfsetdash{}{0pt}%
\pgfpathmoveto{\pgfqpoint{0.756058in}{2.404825in}}%
\pgfpathcurveto{\pgfqpoint{0.761583in}{2.404825in}}{\pgfqpoint{0.766882in}{2.407020in}}{\pgfqpoint{0.770789in}{2.410927in}}%
\pgfpathcurveto{\pgfqpoint{0.774696in}{2.414834in}}{\pgfqpoint{0.776891in}{2.420134in}}{\pgfqpoint{0.776891in}{2.425659in}}%
\pgfpathcurveto{\pgfqpoint{0.776891in}{2.431184in}}{\pgfqpoint{0.774696in}{2.436483in}}{\pgfqpoint{0.770789in}{2.440390in}}%
\pgfpathcurveto{\pgfqpoint{0.766882in}{2.444297in}}{\pgfqpoint{0.761583in}{2.446492in}}{\pgfqpoint{0.756058in}{2.446492in}}%
\pgfpathcurveto{\pgfqpoint{0.750533in}{2.446492in}}{\pgfqpoint{0.745233in}{2.444297in}}{\pgfqpoint{0.741326in}{2.440390in}}%
\pgfpathcurveto{\pgfqpoint{0.737419in}{2.436483in}}{\pgfqpoint{0.735224in}{2.431184in}}{\pgfqpoint{0.735224in}{2.425659in}}%
\pgfpathcurveto{\pgfqpoint{0.735224in}{2.420134in}}{\pgfqpoint{0.737419in}{2.414834in}}{\pgfqpoint{0.741326in}{2.410927in}}%
\pgfpathcurveto{\pgfqpoint{0.745233in}{2.407020in}}{\pgfqpoint{0.750533in}{2.404825in}}{\pgfqpoint{0.756058in}{2.404825in}}%
\pgfpathclose%
\pgfusepath{stroke}%
\end{pgfscope}%
\begin{pgfscope}%
\pgfpathrectangle{\pgfqpoint{0.438556in}{0.383578in}}{\pgfqpoint{2.325000in}{2.310000in}}%
\pgfusepath{clip}%
\pgfsetbuttcap%
\pgfsetroundjoin%
\pgfsetlinewidth{0.803000pt}%
\definecolor{currentstroke}{rgb}{0.686275,0.352941,0.313725}%
\pgfsetstrokecolor{currentstroke}%
\pgfsetdash{}{0pt}%
\pgfpathmoveto{\pgfqpoint{1.117270in}{2.492320in}}%
\pgfpathcurveto{\pgfqpoint{1.122795in}{2.492320in}}{\pgfqpoint{1.128094in}{2.494516in}}{\pgfqpoint{1.132001in}{2.498422in}}%
\pgfpathcurveto{\pgfqpoint{1.135908in}{2.502329in}}{\pgfqpoint{1.138103in}{2.507629in}}{\pgfqpoint{1.138103in}{2.513154in}}%
\pgfpathcurveto{\pgfqpoint{1.138103in}{2.518679in}}{\pgfqpoint{1.135908in}{2.523978in}}{\pgfqpoint{1.132001in}{2.527885in}}%
\pgfpathcurveto{\pgfqpoint{1.128094in}{2.531792in}}{\pgfqpoint{1.122795in}{2.533987in}}{\pgfqpoint{1.117270in}{2.533987in}}%
\pgfpathcurveto{\pgfqpoint{1.111745in}{2.533987in}}{\pgfqpoint{1.106445in}{2.531792in}}{\pgfqpoint{1.102538in}{2.527885in}}%
\pgfpathcurveto{\pgfqpoint{1.098631in}{2.523978in}}{\pgfqpoint{1.096436in}{2.518679in}}{\pgfqpoint{1.096436in}{2.513154in}}%
\pgfpathcurveto{\pgfqpoint{1.096436in}{2.507629in}}{\pgfqpoint{1.098631in}{2.502329in}}{\pgfqpoint{1.102538in}{2.498422in}}%
\pgfpathcurveto{\pgfqpoint{1.106445in}{2.494516in}}{\pgfqpoint{1.111745in}{2.492320in}}{\pgfqpoint{1.117270in}{2.492320in}}%
\pgfpathclose%
\pgfusepath{stroke}%
\end{pgfscope}%
\begin{pgfscope}%
\pgfpathrectangle{\pgfqpoint{0.438556in}{0.383578in}}{\pgfqpoint{2.325000in}{2.310000in}}%
\pgfusepath{clip}%
\pgfsetbuttcap%
\pgfsetroundjoin%
\pgfsetlinewidth{0.803000pt}%
\definecolor{currentstroke}{rgb}{0.686275,0.352941,0.313725}%
\pgfsetstrokecolor{currentstroke}%
\pgfsetdash{}{0pt}%
\pgfpathmoveto{\pgfqpoint{0.586240in}{2.514194in}}%
\pgfpathcurveto{\pgfqpoint{0.591765in}{2.514194in}}{\pgfqpoint{0.597064in}{2.516389in}}{\pgfqpoint{0.600971in}{2.520296in}}%
\pgfpathcurveto{\pgfqpoint{0.604878in}{2.524203in}}{\pgfqpoint{0.607073in}{2.529502in}}{\pgfqpoint{0.607073in}{2.535027in}}%
\pgfpathcurveto{\pgfqpoint{0.607073in}{2.540553in}}{\pgfqpoint{0.604878in}{2.545852in}}{\pgfqpoint{0.600971in}{2.549759in}}%
\pgfpathcurveto{\pgfqpoint{0.597064in}{2.553666in}}{\pgfqpoint{0.591765in}{2.555861in}}{\pgfqpoint{0.586240in}{2.555861in}}%
\pgfpathcurveto{\pgfqpoint{0.580715in}{2.555861in}}{\pgfqpoint{0.575415in}{2.553666in}}{\pgfqpoint{0.571508in}{2.549759in}}%
\pgfpathcurveto{\pgfqpoint{0.567601in}{2.545852in}}{\pgfqpoint{0.565406in}{2.540553in}}{\pgfqpoint{0.565406in}{2.535027in}}%
\pgfpathcurveto{\pgfqpoint{0.565406in}{2.529502in}}{\pgfqpoint{0.567601in}{2.524203in}}{\pgfqpoint{0.571508in}{2.520296in}}%
\pgfpathcurveto{\pgfqpoint{0.575415in}{2.516389in}}{\pgfqpoint{0.580715in}{2.514194in}}{\pgfqpoint{0.586240in}{2.514194in}}%
\pgfpathclose%
\pgfusepath{stroke}%
\end{pgfscope}%
\begin{pgfscope}%
\pgfpathrectangle{\pgfqpoint{0.438556in}{0.383578in}}{\pgfqpoint{2.325000in}{2.310000in}}%
\pgfusepath{clip}%
\pgfsetbuttcap%
\pgfsetroundjoin%
\pgfsetlinewidth{0.803000pt}%
\definecolor{currentstroke}{rgb}{0.000000,0.356863,0.509804}%
\pgfsetstrokecolor{currentstroke}%
\pgfsetdash{}{0pt}%
\pgfpathmoveto{\pgfqpoint{1.176232in}{0.618121in}}%
\pgfpathcurveto{\pgfqpoint{1.179915in}{0.618121in}}{\pgfqpoint{1.183448in}{0.619585in}}{\pgfqpoint{1.186053in}{0.622189in}}%
\pgfpathcurveto{\pgfqpoint{1.188657in}{0.624794in}}{\pgfqpoint{1.190121in}{0.628327in}}{\pgfqpoint{1.190121in}{0.632010in}}%
\pgfpathcurveto{\pgfqpoint{1.190121in}{0.635694in}}{\pgfqpoint{1.188657in}{0.639227in}}{\pgfqpoint{1.186053in}{0.641831in}}%
\pgfpathcurveto{\pgfqpoint{1.183448in}{0.644436in}}{\pgfqpoint{1.179915in}{0.645899in}}{\pgfqpoint{1.176232in}{0.645899in}}%
\pgfpathcurveto{\pgfqpoint{1.172548in}{0.645899in}}{\pgfqpoint{1.169015in}{0.644436in}}{\pgfqpoint{1.166411in}{0.641831in}}%
\pgfpathcurveto{\pgfqpoint{1.163806in}{0.639227in}}{\pgfqpoint{1.162343in}{0.635694in}}{\pgfqpoint{1.162343in}{0.632010in}}%
\pgfpathcurveto{\pgfqpoint{1.162343in}{0.628327in}}{\pgfqpoint{1.163806in}{0.624794in}}{\pgfqpoint{1.166411in}{0.622189in}}%
\pgfpathcurveto{\pgfqpoint{1.169015in}{0.619585in}}{\pgfqpoint{1.172548in}{0.618121in}}{\pgfqpoint{1.176232in}{0.618121in}}%
\pgfpathclose%
\pgfusepath{stroke}%
\end{pgfscope}%
\begin{pgfscope}%
\pgfpathrectangle{\pgfqpoint{0.438556in}{0.383578in}}{\pgfqpoint{2.325000in}{2.310000in}}%
\pgfusepath{clip}%
\pgfsetbuttcap%
\pgfsetroundjoin%
\pgfsetlinewidth{0.803000pt}%
\definecolor{currentstroke}{rgb}{0.000000,0.356863,0.509804}%
\pgfsetstrokecolor{currentstroke}%
\pgfsetdash{}{0pt}%
\pgfpathmoveto{\pgfqpoint{0.648736in}{0.639995in}}%
\pgfpathcurveto{\pgfqpoint{0.652419in}{0.639995in}}{\pgfqpoint{0.655952in}{0.641458in}}{\pgfqpoint{0.658557in}{0.644063in}}%
\pgfpathcurveto{\pgfqpoint{0.661161in}{0.646668in}}{\pgfqpoint{0.662625in}{0.650201in}}{\pgfqpoint{0.662625in}{0.653884in}}%
\pgfpathcurveto{\pgfqpoint{0.662625in}{0.657567in}}{\pgfqpoint{0.661161in}{0.661100in}}{\pgfqpoint{0.658557in}{0.663705in}}%
\pgfpathcurveto{\pgfqpoint{0.655952in}{0.666309in}}{\pgfqpoint{0.652419in}{0.667773in}}{\pgfqpoint{0.648736in}{0.667773in}}%
\pgfpathcurveto{\pgfqpoint{0.645052in}{0.667773in}}{\pgfqpoint{0.641519in}{0.666309in}}{\pgfqpoint{0.638915in}{0.663705in}}%
\pgfpathcurveto{\pgfqpoint{0.636310in}{0.661100in}}{\pgfqpoint{0.634847in}{0.657567in}}{\pgfqpoint{0.634847in}{0.653884in}}%
\pgfpathcurveto{\pgfqpoint{0.634847in}{0.650201in}}{\pgfqpoint{0.636310in}{0.646668in}}{\pgfqpoint{0.638915in}{0.644063in}}%
\pgfpathcurveto{\pgfqpoint{0.641519in}{0.641458in}}{\pgfqpoint{0.645052in}{0.639995in}}{\pgfqpoint{0.648736in}{0.639995in}}%
\pgfpathclose%
\pgfusepath{stroke}%
\end{pgfscope}%
\begin{pgfscope}%
\pgfpathrectangle{\pgfqpoint{0.438556in}{0.383578in}}{\pgfqpoint{2.325000in}{2.310000in}}%
\pgfusepath{clip}%
\pgfsetbuttcap%
\pgfsetroundjoin%
\pgfsetlinewidth{0.803000pt}%
\definecolor{currentstroke}{rgb}{0.000000,0.356863,0.509804}%
\pgfsetstrokecolor{currentstroke}%
\pgfsetdash{}{0pt}%
\pgfpathmoveto{\pgfqpoint{0.777262in}{0.793111in}}%
\pgfpathcurveto{\pgfqpoint{0.780945in}{0.793111in}}{\pgfqpoint{0.784478in}{0.794575in}}{\pgfqpoint{0.787083in}{0.797179in}}%
\pgfpathcurveto{\pgfqpoint{0.789687in}{0.799784in}}{\pgfqpoint{0.791151in}{0.803317in}}{\pgfqpoint{0.791151in}{0.807000in}}%
\pgfpathcurveto{\pgfqpoint{0.791151in}{0.810684in}}{\pgfqpoint{0.789687in}{0.814217in}}{\pgfqpoint{0.787083in}{0.816821in}}%
\pgfpathcurveto{\pgfqpoint{0.784478in}{0.819426in}}{\pgfqpoint{0.780945in}{0.820889in}}{\pgfqpoint{0.777262in}{0.820889in}}%
\pgfpathcurveto{\pgfqpoint{0.773578in}{0.820889in}}{\pgfqpoint{0.770045in}{0.819426in}}{\pgfqpoint{0.767441in}{0.816821in}}%
\pgfpathcurveto{\pgfqpoint{0.764836in}{0.814217in}}{\pgfqpoint{0.763373in}{0.810684in}}{\pgfqpoint{0.763373in}{0.807000in}}%
\pgfpathcurveto{\pgfqpoint{0.763373in}{0.803317in}}{\pgfqpoint{0.764836in}{0.799784in}}{\pgfqpoint{0.767441in}{0.797179in}}%
\pgfpathcurveto{\pgfqpoint{0.770045in}{0.794575in}}{\pgfqpoint{0.773578in}{0.793111in}}{\pgfqpoint{0.777262in}{0.793111in}}%
\pgfpathclose%
\pgfusepath{stroke}%
\end{pgfscope}%
\begin{pgfscope}%
\pgfpathrectangle{\pgfqpoint{0.438556in}{0.383578in}}{\pgfqpoint{2.325000in}{2.310000in}}%
\pgfusepath{clip}%
\pgfsetbuttcap%
\pgfsetroundjoin%
\pgfsetlinewidth{0.803000pt}%
\definecolor{currentstroke}{rgb}{0.000000,0.356863,0.509804}%
\pgfsetstrokecolor{currentstroke}%
\pgfsetdash{}{0pt}%
\pgfpathmoveto{\pgfqpoint{1.197064in}{0.858733in}}%
\pgfpathcurveto{\pgfqpoint{1.200747in}{0.858733in}}{\pgfqpoint{1.204280in}{0.860196in}}{\pgfqpoint{1.206885in}{0.862801in}}%
\pgfpathcurveto{\pgfqpoint{1.209489in}{0.865405in}}{\pgfqpoint{1.210953in}{0.868938in}}{\pgfqpoint{1.210953in}{0.872622in}}%
\pgfpathcurveto{\pgfqpoint{1.210953in}{0.876305in}}{\pgfqpoint{1.209489in}{0.879838in}}{\pgfqpoint{1.206885in}{0.882442in}}%
\pgfpathcurveto{\pgfqpoint{1.204280in}{0.885047in}}{\pgfqpoint{1.200747in}{0.886510in}}{\pgfqpoint{1.197064in}{0.886510in}}%
\pgfpathcurveto{\pgfqpoint{1.193380in}{0.886510in}}{\pgfqpoint{1.189847in}{0.885047in}}{\pgfqpoint{1.187243in}{0.882442in}}%
\pgfpathcurveto{\pgfqpoint{1.184638in}{0.879838in}}{\pgfqpoint{1.183175in}{0.876305in}}{\pgfqpoint{1.183175in}{0.872622in}}%
\pgfpathcurveto{\pgfqpoint{1.183175in}{0.868938in}}{\pgfqpoint{1.184638in}{0.865405in}}{\pgfqpoint{1.187243in}{0.862801in}}%
\pgfpathcurveto{\pgfqpoint{1.189847in}{0.860196in}}{\pgfqpoint{1.193380in}{0.858733in}}{\pgfqpoint{1.197064in}{0.858733in}}%
\pgfpathclose%
\pgfusepath{stroke}%
\end{pgfscope}%
\begin{pgfscope}%
\pgfpathrectangle{\pgfqpoint{0.438556in}{0.383578in}}{\pgfqpoint{2.325000in}{2.310000in}}%
\pgfusepath{clip}%
\pgfsetbuttcap%
\pgfsetroundjoin%
\pgfsetlinewidth{0.803000pt}%
\definecolor{currentstroke}{rgb}{0.000000,0.356863,0.509804}%
\pgfsetstrokecolor{currentstroke}%
\pgfsetdash{}{0pt}%
\pgfpathmoveto{\pgfqpoint{2.221738in}{0.880606in}}%
\pgfpathcurveto{\pgfqpoint{2.225421in}{0.880606in}}{\pgfqpoint{2.228954in}{0.882070in}}{\pgfqpoint{2.231559in}{0.884674in}}%
\pgfpathcurveto{\pgfqpoint{2.234163in}{0.887279in}}{\pgfqpoint{2.235627in}{0.890812in}}{\pgfqpoint{2.235627in}{0.894495in}}%
\pgfpathcurveto{\pgfqpoint{2.235627in}{0.898179in}}{\pgfqpoint{2.234163in}{0.901712in}}{\pgfqpoint{2.231559in}{0.904316in}}%
\pgfpathcurveto{\pgfqpoint{2.228954in}{0.906921in}}{\pgfqpoint{2.225421in}{0.908384in}}{\pgfqpoint{2.221738in}{0.908384in}}%
\pgfpathcurveto{\pgfqpoint{2.218054in}{0.908384in}}{\pgfqpoint{2.214521in}{0.906921in}}{\pgfqpoint{2.211917in}{0.904316in}}%
\pgfpathcurveto{\pgfqpoint{2.209312in}{0.901712in}}{\pgfqpoint{2.207849in}{0.898179in}}{\pgfqpoint{2.207849in}{0.894495in}}%
\pgfpathcurveto{\pgfqpoint{2.207849in}{0.890812in}}{\pgfqpoint{2.209312in}{0.887279in}}{\pgfqpoint{2.211917in}{0.884674in}}%
\pgfpathcurveto{\pgfqpoint{2.214521in}{0.882070in}}{\pgfqpoint{2.218054in}{0.880606in}}{\pgfqpoint{2.221738in}{0.880606in}}%
\pgfpathclose%
\pgfusepath{stroke}%
\end{pgfscope}%
\begin{pgfscope}%
\pgfpathrectangle{\pgfqpoint{0.438556in}{0.383578in}}{\pgfqpoint{2.325000in}{2.310000in}}%
\pgfusepath{clip}%
\pgfsetbuttcap%
\pgfsetroundjoin%
\pgfsetlinewidth{0.803000pt}%
\definecolor{currentstroke}{rgb}{0.000000,0.356863,0.509804}%
\pgfsetstrokecolor{currentstroke}%
\pgfsetdash{}{0pt}%
\pgfpathmoveto{\pgfqpoint{1.796356in}{1.011849in}}%
\pgfpathcurveto{\pgfqpoint{1.800039in}{1.011849in}}{\pgfqpoint{1.803572in}{1.013312in}}{\pgfqpoint{1.806177in}{1.015917in}}%
\pgfpathcurveto{\pgfqpoint{1.808781in}{1.018521in}}{\pgfqpoint{1.810245in}{1.022054in}}{\pgfqpoint{1.810245in}{1.025738in}}%
\pgfpathcurveto{\pgfqpoint{1.810245in}{1.029421in}}{\pgfqpoint{1.808781in}{1.032954in}}{\pgfqpoint{1.806177in}{1.035559in}}%
\pgfpathcurveto{\pgfqpoint{1.803572in}{1.038163in}}{\pgfqpoint{1.800039in}{1.039627in}}{\pgfqpoint{1.796356in}{1.039627in}}%
\pgfpathcurveto{\pgfqpoint{1.792672in}{1.039627in}}{\pgfqpoint{1.789139in}{1.038163in}}{\pgfqpoint{1.786535in}{1.035559in}}%
\pgfpathcurveto{\pgfqpoint{1.783930in}{1.032954in}}{\pgfqpoint{1.782467in}{1.029421in}}{\pgfqpoint{1.782467in}{1.025738in}}%
\pgfpathcurveto{\pgfqpoint{1.782467in}{1.022054in}}{\pgfqpoint{1.783930in}{1.018521in}}{\pgfqpoint{1.786535in}{1.015917in}}%
\pgfpathcurveto{\pgfqpoint{1.789139in}{1.013312in}}{\pgfqpoint{1.792672in}{1.011849in}}{\pgfqpoint{1.796356in}{1.011849in}}%
\pgfpathclose%
\pgfusepath{stroke}%
\end{pgfscope}%
\begin{pgfscope}%
\pgfpathrectangle{\pgfqpoint{0.438556in}{0.383578in}}{\pgfqpoint{2.325000in}{2.310000in}}%
\pgfusepath{clip}%
\pgfsetbuttcap%
\pgfsetroundjoin%
\pgfsetlinewidth{0.803000pt}%
\definecolor{currentstroke}{rgb}{0.000000,0.356863,0.509804}%
\pgfsetstrokecolor{currentstroke}%
\pgfsetdash{}{0pt}%
\pgfpathmoveto{\pgfqpoint{1.087324in}{1.033723in}}%
\pgfpathcurveto{\pgfqpoint{1.091007in}{1.033723in}}{\pgfqpoint{1.094540in}{1.035186in}}{\pgfqpoint{1.097145in}{1.037791in}}%
\pgfpathcurveto{\pgfqpoint{1.099749in}{1.040395in}}{\pgfqpoint{1.101213in}{1.043928in}}{\pgfqpoint{1.101213in}{1.047612in}}%
\pgfpathcurveto{\pgfqpoint{1.101213in}{1.051295in}}{\pgfqpoint{1.099749in}{1.054828in}}{\pgfqpoint{1.097145in}{1.057433in}}%
\pgfpathcurveto{\pgfqpoint{1.094540in}{1.060037in}}{\pgfqpoint{1.091007in}{1.061501in}}{\pgfqpoint{1.087324in}{1.061501in}}%
\pgfpathcurveto{\pgfqpoint{1.083640in}{1.061501in}}{\pgfqpoint{1.080107in}{1.060037in}}{\pgfqpoint{1.077503in}{1.057433in}}%
\pgfpathcurveto{\pgfqpoint{1.074898in}{1.054828in}}{\pgfqpoint{1.073435in}{1.051295in}}{\pgfqpoint{1.073435in}{1.047612in}}%
\pgfpathcurveto{\pgfqpoint{1.073435in}{1.043928in}}{\pgfqpoint{1.074898in}{1.040395in}}{\pgfqpoint{1.077503in}{1.037791in}}%
\pgfpathcurveto{\pgfqpoint{1.080107in}{1.035186in}}{\pgfqpoint{1.083640in}{1.033723in}}{\pgfqpoint{1.087324in}{1.033723in}}%
\pgfpathclose%
\pgfusepath{stroke}%
\end{pgfscope}%
\begin{pgfscope}%
\pgfpathrectangle{\pgfqpoint{0.438556in}{0.383578in}}{\pgfqpoint{2.325000in}{2.310000in}}%
\pgfusepath{clip}%
\pgfsetbuttcap%
\pgfsetroundjoin%
\pgfsetlinewidth{0.803000pt}%
\definecolor{currentstroke}{rgb}{0.000000,0.356863,0.509804}%
\pgfsetstrokecolor{currentstroke}%
\pgfsetdash{}{0pt}%
\pgfpathmoveto{\pgfqpoint{1.307362in}{1.099344in}}%
\pgfpathcurveto{\pgfqpoint{1.311045in}{1.099344in}}{\pgfqpoint{1.314578in}{1.100807in}}{\pgfqpoint{1.317183in}{1.103412in}}%
\pgfpathcurveto{\pgfqpoint{1.319787in}{1.106017in}}{\pgfqpoint{1.321251in}{1.109550in}}{\pgfqpoint{1.321251in}{1.113233in}}%
\pgfpathcurveto{\pgfqpoint{1.321251in}{1.116916in}}{\pgfqpoint{1.319787in}{1.120449in}}{\pgfqpoint{1.317183in}{1.123054in}}%
\pgfpathcurveto{\pgfqpoint{1.314578in}{1.125658in}}{\pgfqpoint{1.311045in}{1.127122in}}{\pgfqpoint{1.307362in}{1.127122in}}%
\pgfpathcurveto{\pgfqpoint{1.303678in}{1.127122in}}{\pgfqpoint{1.300145in}{1.125658in}}{\pgfqpoint{1.297541in}{1.123054in}}%
\pgfpathcurveto{\pgfqpoint{1.294936in}{1.120449in}}{\pgfqpoint{1.293473in}{1.116916in}}{\pgfqpoint{1.293473in}{1.113233in}}%
\pgfpathcurveto{\pgfqpoint{1.293473in}{1.109550in}}{\pgfqpoint{1.294936in}{1.106017in}}{\pgfqpoint{1.297541in}{1.103412in}}%
\pgfpathcurveto{\pgfqpoint{1.300145in}{1.100807in}}{\pgfqpoint{1.303678in}{1.099344in}}{\pgfqpoint{1.307362in}{1.099344in}}%
\pgfpathclose%
\pgfusepath{stroke}%
\end{pgfscope}%
\begin{pgfscope}%
\pgfpathrectangle{\pgfqpoint{0.438556in}{0.383578in}}{\pgfqpoint{2.325000in}{2.310000in}}%
\pgfusepath{clip}%
\pgfsetbuttcap%
\pgfsetroundjoin%
\pgfsetlinewidth{0.803000pt}%
\definecolor{currentstroke}{rgb}{0.000000,0.356863,0.509804}%
\pgfsetstrokecolor{currentstroke}%
\pgfsetdash{}{0pt}%
\pgfpathmoveto{\pgfqpoint{1.028548in}{1.208713in}}%
\pgfpathcurveto{\pgfqpoint{1.032231in}{1.208713in}}{\pgfqpoint{1.035764in}{1.210176in}}{\pgfqpoint{1.038369in}{1.212781in}}%
\pgfpathcurveto{\pgfqpoint{1.040973in}{1.215385in}}{\pgfqpoint{1.042437in}{1.218918in}}{\pgfqpoint{1.042437in}{1.222602in}}%
\pgfpathcurveto{\pgfqpoint{1.042437in}{1.226285in}}{\pgfqpoint{1.040973in}{1.229818in}}{\pgfqpoint{1.038369in}{1.232423in}}%
\pgfpathcurveto{\pgfqpoint{1.035764in}{1.235027in}}{\pgfqpoint{1.032231in}{1.236491in}}{\pgfqpoint{1.028548in}{1.236491in}}%
\pgfpathcurveto{\pgfqpoint{1.024864in}{1.236491in}}{\pgfqpoint{1.021331in}{1.235027in}}{\pgfqpoint{1.018727in}{1.232423in}}%
\pgfpathcurveto{\pgfqpoint{1.016122in}{1.229818in}}{\pgfqpoint{1.014659in}{1.226285in}}{\pgfqpoint{1.014659in}{1.222602in}}%
\pgfpathcurveto{\pgfqpoint{1.014659in}{1.218918in}}{\pgfqpoint{1.016122in}{1.215385in}}{\pgfqpoint{1.018727in}{1.212781in}}%
\pgfpathcurveto{\pgfqpoint{1.021331in}{1.210176in}}{\pgfqpoint{1.024864in}{1.208713in}}{\pgfqpoint{1.028548in}{1.208713in}}%
\pgfpathclose%
\pgfusepath{stroke}%
\end{pgfscope}%
\begin{pgfscope}%
\pgfpathrectangle{\pgfqpoint{0.438556in}{0.383578in}}{\pgfqpoint{2.325000in}{2.310000in}}%
\pgfusepath{clip}%
\pgfsetbuttcap%
\pgfsetroundjoin%
\pgfsetlinewidth{0.803000pt}%
\definecolor{currentstroke}{rgb}{0.000000,0.356863,0.509804}%
\pgfsetstrokecolor{currentstroke}%
\pgfsetdash{}{0pt}%
\pgfpathmoveto{\pgfqpoint{2.263960in}{1.230587in}}%
\pgfpathcurveto{\pgfqpoint{2.267643in}{1.230587in}}{\pgfqpoint{2.271176in}{1.232050in}}{\pgfqpoint{2.273781in}{1.234655in}}%
\pgfpathcurveto{\pgfqpoint{2.276385in}{1.237259in}}{\pgfqpoint{2.277849in}{1.240792in}}{\pgfqpoint{2.277849in}{1.244475in}}%
\pgfpathcurveto{\pgfqpoint{2.277849in}{1.248159in}}{\pgfqpoint{2.276385in}{1.251692in}}{\pgfqpoint{2.273781in}{1.254296in}}%
\pgfpathcurveto{\pgfqpoint{2.271176in}{1.256901in}}{\pgfqpoint{2.267643in}{1.258364in}}{\pgfqpoint{2.263960in}{1.258364in}}%
\pgfpathcurveto{\pgfqpoint{2.260276in}{1.258364in}}{\pgfqpoint{2.256743in}{1.256901in}}{\pgfqpoint{2.254139in}{1.254296in}}%
\pgfpathcurveto{\pgfqpoint{2.251534in}{1.251692in}}{\pgfqpoint{2.250071in}{1.248159in}}{\pgfqpoint{2.250071in}{1.244475in}}%
\pgfpathcurveto{\pgfqpoint{2.250071in}{1.240792in}}{\pgfqpoint{2.251534in}{1.237259in}}{\pgfqpoint{2.254139in}{1.234655in}}%
\pgfpathcurveto{\pgfqpoint{2.256743in}{1.232050in}}{\pgfqpoint{2.260276in}{1.230587in}}{\pgfqpoint{2.263960in}{1.230587in}}%
\pgfpathclose%
\pgfusepath{stroke}%
\end{pgfscope}%
\begin{pgfscope}%
\pgfpathrectangle{\pgfqpoint{0.438556in}{0.383578in}}{\pgfqpoint{2.325000in}{2.310000in}}%
\pgfusepath{clip}%
\pgfsetbuttcap%
\pgfsetroundjoin%
\pgfsetlinewidth{0.803000pt}%
\definecolor{currentstroke}{rgb}{0.000000,0.356863,0.509804}%
\pgfsetstrokecolor{currentstroke}%
\pgfsetdash{}{0pt}%
\pgfpathmoveto{\pgfqpoint{0.968284in}{1.252460in}}%
\pgfpathcurveto{\pgfqpoint{0.971967in}{1.252460in}}{\pgfqpoint{0.975500in}{1.253924in}}{\pgfqpoint{0.978105in}{1.256528in}}%
\pgfpathcurveto{\pgfqpoint{0.980709in}{1.259133in}}{\pgfqpoint{0.982173in}{1.262666in}}{\pgfqpoint{0.982173in}{1.266349in}}%
\pgfpathcurveto{\pgfqpoint{0.982173in}{1.270033in}}{\pgfqpoint{0.980709in}{1.273566in}}{\pgfqpoint{0.978105in}{1.276170in}}%
\pgfpathcurveto{\pgfqpoint{0.975500in}{1.278775in}}{\pgfqpoint{0.971967in}{1.280238in}}{\pgfqpoint{0.968284in}{1.280238in}}%
\pgfpathcurveto{\pgfqpoint{0.964600in}{1.280238in}}{\pgfqpoint{0.961067in}{1.278775in}}{\pgfqpoint{0.958463in}{1.276170in}}%
\pgfpathcurveto{\pgfqpoint{0.955858in}{1.273566in}}{\pgfqpoint{0.954395in}{1.270033in}}{\pgfqpoint{0.954395in}{1.266349in}}%
\pgfpathcurveto{\pgfqpoint{0.954395in}{1.262666in}}{\pgfqpoint{0.955858in}{1.259133in}}{\pgfqpoint{0.958463in}{1.256528in}}%
\pgfpathcurveto{\pgfqpoint{0.961067in}{1.253924in}}{\pgfqpoint{0.964600in}{1.252460in}}{\pgfqpoint{0.968284in}{1.252460in}}%
\pgfpathclose%
\pgfusepath{stroke}%
\end{pgfscope}%
\begin{pgfscope}%
\pgfpathrectangle{\pgfqpoint{0.438556in}{0.383578in}}{\pgfqpoint{2.325000in}{2.310000in}}%
\pgfusepath{clip}%
\pgfsetbuttcap%
\pgfsetroundjoin%
\pgfsetlinewidth{0.803000pt}%
\definecolor{currentstroke}{rgb}{0.000000,0.356863,0.509804}%
\pgfsetstrokecolor{currentstroke}%
\pgfsetdash{}{0pt}%
\pgfpathmoveto{\pgfqpoint{1.944970in}{1.274334in}}%
\pgfpathcurveto{\pgfqpoint{1.948653in}{1.274334in}}{\pgfqpoint{1.952186in}{1.275798in}}{\pgfqpoint{1.954791in}{1.278402in}}%
\pgfpathcurveto{\pgfqpoint{1.957395in}{1.281007in}}{\pgfqpoint{1.958859in}{1.284540in}}{\pgfqpoint{1.958859in}{1.288223in}}%
\pgfpathcurveto{\pgfqpoint{1.958859in}{1.291906in}}{\pgfqpoint{1.957395in}{1.295439in}}{\pgfqpoint{1.954791in}{1.298044in}}%
\pgfpathcurveto{\pgfqpoint{1.952186in}{1.300648in}}{\pgfqpoint{1.948653in}{1.302112in}}{\pgfqpoint{1.944970in}{1.302112in}}%
\pgfpathcurveto{\pgfqpoint{1.941286in}{1.302112in}}{\pgfqpoint{1.937753in}{1.300648in}}{\pgfqpoint{1.935149in}{1.298044in}}%
\pgfpathcurveto{\pgfqpoint{1.932544in}{1.295439in}}{\pgfqpoint{1.931081in}{1.291906in}}{\pgfqpoint{1.931081in}{1.288223in}}%
\pgfpathcurveto{\pgfqpoint{1.931081in}{1.284540in}}{\pgfqpoint{1.932544in}{1.281007in}}{\pgfqpoint{1.935149in}{1.278402in}}%
\pgfpathcurveto{\pgfqpoint{1.937753in}{1.275798in}}{\pgfqpoint{1.941286in}{1.274334in}}{\pgfqpoint{1.944970in}{1.274334in}}%
\pgfpathclose%
\pgfusepath{stroke}%
\end{pgfscope}%
\begin{pgfscope}%
\pgfpathrectangle{\pgfqpoint{0.438556in}{0.383578in}}{\pgfqpoint{2.325000in}{2.310000in}}%
\pgfusepath{clip}%
\pgfsetbuttcap%
\pgfsetroundjoin%
\pgfsetlinewidth{0.803000pt}%
\definecolor{currentstroke}{rgb}{0.000000,0.356863,0.509804}%
\pgfsetstrokecolor{currentstroke}%
\pgfsetdash{}{0pt}%
\pgfpathmoveto{\pgfqpoint{0.903556in}{1.296208in}}%
\pgfpathcurveto{\pgfqpoint{0.907239in}{1.296208in}}{\pgfqpoint{0.910772in}{1.297671in}}{\pgfqpoint{0.913377in}{1.300276in}}%
\pgfpathcurveto{\pgfqpoint{0.915981in}{1.302880in}}{\pgfqpoint{0.917445in}{1.306413in}}{\pgfqpoint{0.917445in}{1.310097in}}%
\pgfpathcurveto{\pgfqpoint{0.917445in}{1.313780in}}{\pgfqpoint{0.915981in}{1.317313in}}{\pgfqpoint{0.913377in}{1.319918in}}%
\pgfpathcurveto{\pgfqpoint{0.910772in}{1.322522in}}{\pgfqpoint{0.907239in}{1.323986in}}{\pgfqpoint{0.903556in}{1.323986in}}%
\pgfpathcurveto{\pgfqpoint{0.899872in}{1.323986in}}{\pgfqpoint{0.896339in}{1.322522in}}{\pgfqpoint{0.893735in}{1.319918in}}%
\pgfpathcurveto{\pgfqpoint{0.891130in}{1.317313in}}{\pgfqpoint{0.889667in}{1.313780in}}{\pgfqpoint{0.889667in}{1.310097in}}%
\pgfpathcurveto{\pgfqpoint{0.889667in}{1.306413in}}{\pgfqpoint{0.891130in}{1.302880in}}{\pgfqpoint{0.893735in}{1.300276in}}%
\pgfpathcurveto{\pgfqpoint{0.896339in}{1.297671in}}{\pgfqpoint{0.899872in}{1.296208in}}{\pgfqpoint{0.903556in}{1.296208in}}%
\pgfpathclose%
\pgfusepath{stroke}%
\end{pgfscope}%
\begin{pgfscope}%
\pgfpathrectangle{\pgfqpoint{0.438556in}{0.383578in}}{\pgfqpoint{2.325000in}{2.310000in}}%
\pgfusepath{clip}%
\pgfsetbuttcap%
\pgfsetroundjoin%
\pgfsetlinewidth{0.803000pt}%
\definecolor{currentstroke}{rgb}{0.000000,0.356863,0.509804}%
\pgfsetstrokecolor{currentstroke}%
\pgfsetdash{}{0pt}%
\pgfpathmoveto{\pgfqpoint{2.002072in}{1.339955in}}%
\pgfpathcurveto{\pgfqpoint{2.005755in}{1.339955in}}{\pgfqpoint{2.009288in}{1.341419in}}{\pgfqpoint{2.011893in}{1.344023in}}%
\pgfpathcurveto{\pgfqpoint{2.014497in}{1.346628in}}{\pgfqpoint{2.015961in}{1.350161in}}{\pgfqpoint{2.015961in}{1.353844in}}%
\pgfpathcurveto{\pgfqpoint{2.015961in}{1.357528in}}{\pgfqpoint{2.014497in}{1.361061in}}{\pgfqpoint{2.011893in}{1.363665in}}%
\pgfpathcurveto{\pgfqpoint{2.009288in}{1.366270in}}{\pgfqpoint{2.005755in}{1.367733in}}{\pgfqpoint{2.002072in}{1.367733in}}%
\pgfpathcurveto{\pgfqpoint{1.998388in}{1.367733in}}{\pgfqpoint{1.994855in}{1.366270in}}{\pgfqpoint{1.992251in}{1.363665in}}%
\pgfpathcurveto{\pgfqpoint{1.989646in}{1.361061in}}{\pgfqpoint{1.988183in}{1.357528in}}{\pgfqpoint{1.988183in}{1.353844in}}%
\pgfpathcurveto{\pgfqpoint{1.988183in}{1.350161in}}{\pgfqpoint{1.989646in}{1.346628in}}{\pgfqpoint{1.992251in}{1.344023in}}%
\pgfpathcurveto{\pgfqpoint{1.994855in}{1.341419in}}{\pgfqpoint{1.998388in}{1.339955in}}{\pgfqpoint{2.002072in}{1.339955in}}%
\pgfpathclose%
\pgfusepath{stroke}%
\end{pgfscope}%
\begin{pgfscope}%
\pgfpathrectangle{\pgfqpoint{0.438556in}{0.383578in}}{\pgfqpoint{2.325000in}{2.310000in}}%
\pgfusepath{clip}%
\pgfsetbuttcap%
\pgfsetroundjoin%
\pgfsetlinewidth{0.803000pt}%
\definecolor{currentstroke}{rgb}{0.000000,0.356863,0.509804}%
\pgfsetstrokecolor{currentstroke}%
\pgfsetdash{}{0pt}%
\pgfpathmoveto{\pgfqpoint{0.952474in}{1.449324in}}%
\pgfpathcurveto{\pgfqpoint{0.956157in}{1.449324in}}{\pgfqpoint{0.959690in}{1.450788in}}{\pgfqpoint{0.962295in}{1.453392in}}%
\pgfpathcurveto{\pgfqpoint{0.964899in}{1.455997in}}{\pgfqpoint{0.966363in}{1.459530in}}{\pgfqpoint{0.966363in}{1.463213in}}%
\pgfpathcurveto{\pgfqpoint{0.966363in}{1.466896in}}{\pgfqpoint{0.964899in}{1.470430in}}{\pgfqpoint{0.962295in}{1.473034in}}%
\pgfpathcurveto{\pgfqpoint{0.959690in}{1.475639in}}{\pgfqpoint{0.956157in}{1.477102in}}{\pgfqpoint{0.952474in}{1.477102in}}%
\pgfpathcurveto{\pgfqpoint{0.948790in}{1.477102in}}{\pgfqpoint{0.945257in}{1.475639in}}{\pgfqpoint{0.942653in}{1.473034in}}%
\pgfpathcurveto{\pgfqpoint{0.940048in}{1.470430in}}{\pgfqpoint{0.938585in}{1.466896in}}{\pgfqpoint{0.938585in}{1.463213in}}%
\pgfpathcurveto{\pgfqpoint{0.938585in}{1.459530in}}{\pgfqpoint{0.940048in}{1.455997in}}{\pgfqpoint{0.942653in}{1.453392in}}%
\pgfpathcurveto{\pgfqpoint{0.945257in}{1.450788in}}{\pgfqpoint{0.948790in}{1.449324in}}{\pgfqpoint{0.952474in}{1.449324in}}%
\pgfpathclose%
\pgfusepath{stroke}%
\end{pgfscope}%
\begin{pgfscope}%
\pgfpathrectangle{\pgfqpoint{0.438556in}{0.383578in}}{\pgfqpoint{2.325000in}{2.310000in}}%
\pgfusepath{clip}%
\pgfsetbuttcap%
\pgfsetroundjoin%
\pgfsetlinewidth{0.803000pt}%
\definecolor{currentstroke}{rgb}{0.000000,0.356863,0.509804}%
\pgfsetstrokecolor{currentstroke}%
\pgfsetdash{}{0pt}%
\pgfpathmoveto{\pgfqpoint{1.151680in}{1.471198in}}%
\pgfpathcurveto{\pgfqpoint{1.155363in}{1.471198in}}{\pgfqpoint{1.158896in}{1.472661in}}{\pgfqpoint{1.161501in}{1.475266in}}%
\pgfpathcurveto{\pgfqpoint{1.164105in}{1.477870in}}{\pgfqpoint{1.165569in}{1.481404in}}{\pgfqpoint{1.165569in}{1.485087in}}%
\pgfpathcurveto{\pgfqpoint{1.165569in}{1.488770in}}{\pgfqpoint{1.164105in}{1.492303in}}{\pgfqpoint{1.161501in}{1.494908in}}%
\pgfpathcurveto{\pgfqpoint{1.158896in}{1.497512in}}{\pgfqpoint{1.155363in}{1.498976in}}{\pgfqpoint{1.151680in}{1.498976in}}%
\pgfpathcurveto{\pgfqpoint{1.147996in}{1.498976in}}{\pgfqpoint{1.144463in}{1.497512in}}{\pgfqpoint{1.141859in}{1.494908in}}%
\pgfpathcurveto{\pgfqpoint{1.139254in}{1.492303in}}{\pgfqpoint{1.137791in}{1.488770in}}{\pgfqpoint{1.137791in}{1.485087in}}%
\pgfpathcurveto{\pgfqpoint{1.137791in}{1.481404in}}{\pgfqpoint{1.139254in}{1.477870in}}{\pgfqpoint{1.141859in}{1.475266in}}%
\pgfpathcurveto{\pgfqpoint{1.144463in}{1.472661in}}{\pgfqpoint{1.147996in}{1.471198in}}{\pgfqpoint{1.151680in}{1.471198in}}%
\pgfpathclose%
\pgfusepath{stroke}%
\end{pgfscope}%
\begin{pgfscope}%
\pgfpathrectangle{\pgfqpoint{0.438556in}{0.383578in}}{\pgfqpoint{2.325000in}{2.310000in}}%
\pgfusepath{clip}%
\pgfsetbuttcap%
\pgfsetroundjoin%
\pgfsetlinewidth{0.803000pt}%
\definecolor{currentstroke}{rgb}{0.000000,0.356863,0.509804}%
\pgfsetstrokecolor{currentstroke}%
\pgfsetdash{}{0pt}%
\pgfpathmoveto{\pgfqpoint{1.194460in}{1.646188in}}%
\pgfpathcurveto{\pgfqpoint{1.198143in}{1.646188in}}{\pgfqpoint{1.201676in}{1.647652in}}{\pgfqpoint{1.204281in}{1.650256in}}%
\pgfpathcurveto{\pgfqpoint{1.206885in}{1.652861in}}{\pgfqpoint{1.208349in}{1.656394in}}{\pgfqpoint{1.208349in}{1.660077in}}%
\pgfpathcurveto{\pgfqpoint{1.208349in}{1.663760in}}{\pgfqpoint{1.206885in}{1.667293in}}{\pgfqpoint{1.204281in}{1.669898in}}%
\pgfpathcurveto{\pgfqpoint{1.201676in}{1.672502in}}{\pgfqpoint{1.198143in}{1.673966in}}{\pgfqpoint{1.194460in}{1.673966in}}%
\pgfpathcurveto{\pgfqpoint{1.190776in}{1.673966in}}{\pgfqpoint{1.187243in}{1.672502in}}{\pgfqpoint{1.184639in}{1.669898in}}%
\pgfpathcurveto{\pgfqpoint{1.182034in}{1.667293in}}{\pgfqpoint{1.180571in}{1.663760in}}{\pgfqpoint{1.180571in}{1.660077in}}%
\pgfpathcurveto{\pgfqpoint{1.180571in}{1.656394in}}{\pgfqpoint{1.182034in}{1.652861in}}{\pgfqpoint{1.184639in}{1.650256in}}%
\pgfpathcurveto{\pgfqpoint{1.187243in}{1.647652in}}{\pgfqpoint{1.190776in}{1.646188in}}{\pgfqpoint{1.194460in}{1.646188in}}%
\pgfpathclose%
\pgfusepath{stroke}%
\end{pgfscope}%
\begin{pgfscope}%
\pgfpathrectangle{\pgfqpoint{0.438556in}{0.383578in}}{\pgfqpoint{2.325000in}{2.310000in}}%
\pgfusepath{clip}%
\pgfsetbuttcap%
\pgfsetroundjoin%
\pgfsetlinewidth{0.803000pt}%
\definecolor{currentstroke}{rgb}{0.000000,0.356863,0.509804}%
\pgfsetstrokecolor{currentstroke}%
\pgfsetdash{}{0pt}%
\pgfpathmoveto{\pgfqpoint{2.023276in}{1.668062in}}%
\pgfpathcurveto{\pgfqpoint{2.026959in}{1.668062in}}{\pgfqpoint{2.030492in}{1.669525in}}{\pgfqpoint{2.033097in}{1.672130in}}%
\pgfpathcurveto{\pgfqpoint{2.035701in}{1.674734in}}{\pgfqpoint{2.037165in}{1.678267in}}{\pgfqpoint{2.037165in}{1.681951in}}%
\pgfpathcurveto{\pgfqpoint{2.037165in}{1.685634in}}{\pgfqpoint{2.035701in}{1.689167in}}{\pgfqpoint{2.033097in}{1.691772in}}%
\pgfpathcurveto{\pgfqpoint{2.030492in}{1.694376in}}{\pgfqpoint{2.026959in}{1.695840in}}{\pgfqpoint{2.023276in}{1.695840in}}%
\pgfpathcurveto{\pgfqpoint{2.019592in}{1.695840in}}{\pgfqpoint{2.016059in}{1.694376in}}{\pgfqpoint{2.013455in}{1.691772in}}%
\pgfpathcurveto{\pgfqpoint{2.010850in}{1.689167in}}{\pgfqpoint{2.009387in}{1.685634in}}{\pgfqpoint{2.009387in}{1.681951in}}%
\pgfpathcurveto{\pgfqpoint{2.009387in}{1.678267in}}{\pgfqpoint{2.010850in}{1.674734in}}{\pgfqpoint{2.013455in}{1.672130in}}%
\pgfpathcurveto{\pgfqpoint{2.016059in}{1.669525in}}{\pgfqpoint{2.019592in}{1.668062in}}{\pgfqpoint{2.023276in}{1.668062in}}%
\pgfpathclose%
\pgfusepath{stroke}%
\end{pgfscope}%
\begin{pgfscope}%
\pgfpathrectangle{\pgfqpoint{0.438556in}{0.383578in}}{\pgfqpoint{2.325000in}{2.310000in}}%
\pgfusepath{clip}%
\pgfsetbuttcap%
\pgfsetroundjoin%
\pgfsetlinewidth{0.803000pt}%
\definecolor{currentstroke}{rgb}{0.000000,0.356863,0.509804}%
\pgfsetstrokecolor{currentstroke}%
\pgfsetdash{}{0pt}%
\pgfpathmoveto{\pgfqpoint{1.867036in}{1.689936in}}%
\pgfpathcurveto{\pgfqpoint{1.870719in}{1.689936in}}{\pgfqpoint{1.874252in}{1.691399in}}{\pgfqpoint{1.876857in}{1.694004in}}%
\pgfpathcurveto{\pgfqpoint{1.879461in}{1.696608in}}{\pgfqpoint{1.880925in}{1.700141in}}{\pgfqpoint{1.880925in}{1.703825in}}%
\pgfpathcurveto{\pgfqpoint{1.880925in}{1.707508in}}{\pgfqpoint{1.879461in}{1.711041in}}{\pgfqpoint{1.876857in}{1.713645in}}%
\pgfpathcurveto{\pgfqpoint{1.874252in}{1.716250in}}{\pgfqpoint{1.870719in}{1.717713in}}{\pgfqpoint{1.867036in}{1.717713in}}%
\pgfpathcurveto{\pgfqpoint{1.863352in}{1.717713in}}{\pgfqpoint{1.859819in}{1.716250in}}{\pgfqpoint{1.857215in}{1.713645in}}%
\pgfpathcurveto{\pgfqpoint{1.854610in}{1.711041in}}{\pgfqpoint{1.853147in}{1.707508in}}{\pgfqpoint{1.853147in}{1.703825in}}%
\pgfpathcurveto{\pgfqpoint{1.853147in}{1.700141in}}{\pgfqpoint{1.854610in}{1.696608in}}{\pgfqpoint{1.857215in}{1.694004in}}%
\pgfpathcurveto{\pgfqpoint{1.859819in}{1.691399in}}{\pgfqpoint{1.863352in}{1.689936in}}{\pgfqpoint{1.867036in}{1.689936in}}%
\pgfpathclose%
\pgfusepath{stroke}%
\end{pgfscope}%
\begin{pgfscope}%
\pgfpathrectangle{\pgfqpoint{0.438556in}{0.383578in}}{\pgfqpoint{2.325000in}{2.310000in}}%
\pgfusepath{clip}%
\pgfsetbuttcap%
\pgfsetroundjoin%
\pgfsetlinewidth{0.803000pt}%
\definecolor{currentstroke}{rgb}{0.000000,0.356863,0.509804}%
\pgfsetstrokecolor{currentstroke}%
\pgfsetdash{}{0pt}%
\pgfpathmoveto{\pgfqpoint{0.797908in}{1.711809in}}%
\pgfpathcurveto{\pgfqpoint{0.801591in}{1.711809in}}{\pgfqpoint{0.805124in}{1.713273in}}{\pgfqpoint{0.807729in}{1.715877in}}%
\pgfpathcurveto{\pgfqpoint{0.810333in}{1.718482in}}{\pgfqpoint{0.811797in}{1.722015in}}{\pgfqpoint{0.811797in}{1.725698in}}%
\pgfpathcurveto{\pgfqpoint{0.811797in}{1.729382in}}{\pgfqpoint{0.810333in}{1.732915in}}{\pgfqpoint{0.807729in}{1.735519in}}%
\pgfpathcurveto{\pgfqpoint{0.805124in}{1.738124in}}{\pgfqpoint{0.801591in}{1.739587in}}{\pgfqpoint{0.797908in}{1.739587in}}%
\pgfpathcurveto{\pgfqpoint{0.794224in}{1.739587in}}{\pgfqpoint{0.790691in}{1.738124in}}{\pgfqpoint{0.788087in}{1.735519in}}%
\pgfpathcurveto{\pgfqpoint{0.785482in}{1.732915in}}{\pgfqpoint{0.784019in}{1.729382in}}{\pgfqpoint{0.784019in}{1.725698in}}%
\pgfpathcurveto{\pgfqpoint{0.784019in}{1.722015in}}{\pgfqpoint{0.785482in}{1.718482in}}{\pgfqpoint{0.788087in}{1.715877in}}%
\pgfpathcurveto{\pgfqpoint{0.790691in}{1.713273in}}{\pgfqpoint{0.794224in}{1.711809in}}{\pgfqpoint{0.797908in}{1.711809in}}%
\pgfpathclose%
\pgfusepath{stroke}%
\end{pgfscope}%
\begin{pgfscope}%
\pgfpathrectangle{\pgfqpoint{0.438556in}{0.383578in}}{\pgfqpoint{2.325000in}{2.310000in}}%
\pgfusepath{clip}%
\pgfsetbuttcap%
\pgfsetroundjoin%
\pgfsetlinewidth{0.803000pt}%
\definecolor{currentstroke}{rgb}{0.000000,0.356863,0.509804}%
\pgfsetstrokecolor{currentstroke}%
\pgfsetdash{}{0pt}%
\pgfpathmoveto{\pgfqpoint{0.945034in}{1.777431in}}%
\pgfpathcurveto{\pgfqpoint{0.948717in}{1.777431in}}{\pgfqpoint{0.952250in}{1.778894in}}{\pgfqpoint{0.954855in}{1.781499in}}%
\pgfpathcurveto{\pgfqpoint{0.957459in}{1.784103in}}{\pgfqpoint{0.958923in}{1.787636in}}{\pgfqpoint{0.958923in}{1.791320in}}%
\pgfpathcurveto{\pgfqpoint{0.958923in}{1.795003in}}{\pgfqpoint{0.957459in}{1.798536in}}{\pgfqpoint{0.954855in}{1.801140in}}%
\pgfpathcurveto{\pgfqpoint{0.952250in}{1.803745in}}{\pgfqpoint{0.948717in}{1.805208in}}{\pgfqpoint{0.945034in}{1.805208in}}%
\pgfpathcurveto{\pgfqpoint{0.941350in}{1.805208in}}{\pgfqpoint{0.937817in}{1.803745in}}{\pgfqpoint{0.935213in}{1.801140in}}%
\pgfpathcurveto{\pgfqpoint{0.932608in}{1.798536in}}{\pgfqpoint{0.931145in}{1.795003in}}{\pgfqpoint{0.931145in}{1.791320in}}%
\pgfpathcurveto{\pgfqpoint{0.931145in}{1.787636in}}{\pgfqpoint{0.932608in}{1.784103in}}{\pgfqpoint{0.935213in}{1.781499in}}%
\pgfpathcurveto{\pgfqpoint{0.937817in}{1.778894in}}{\pgfqpoint{0.941350in}{1.777431in}}{\pgfqpoint{0.945034in}{1.777431in}}%
\pgfpathclose%
\pgfusepath{stroke}%
\end{pgfscope}%
\begin{pgfscope}%
\pgfpathrectangle{\pgfqpoint{0.438556in}{0.383578in}}{\pgfqpoint{2.325000in}{2.310000in}}%
\pgfusepath{clip}%
\pgfsetbuttcap%
\pgfsetroundjoin%
\pgfsetlinewidth{0.803000pt}%
\definecolor{currentstroke}{rgb}{0.000000,0.356863,0.509804}%
\pgfsetstrokecolor{currentstroke}%
\pgfsetdash{}{0pt}%
\pgfpathmoveto{\pgfqpoint{0.647620in}{1.799304in}}%
\pgfpathcurveto{\pgfqpoint{0.651303in}{1.799304in}}{\pgfqpoint{0.654836in}{1.800768in}}{\pgfqpoint{0.657441in}{1.803372in}}%
\pgfpathcurveto{\pgfqpoint{0.660045in}{1.805977in}}{\pgfqpoint{0.661509in}{1.809510in}}{\pgfqpoint{0.661509in}{1.813193in}}%
\pgfpathcurveto{\pgfqpoint{0.661509in}{1.816877in}}{\pgfqpoint{0.660045in}{1.820410in}}{\pgfqpoint{0.657441in}{1.823014in}}%
\pgfpathcurveto{\pgfqpoint{0.654836in}{1.825619in}}{\pgfqpoint{0.651303in}{1.827082in}}{\pgfqpoint{0.647620in}{1.827082in}}%
\pgfpathcurveto{\pgfqpoint{0.643936in}{1.827082in}}{\pgfqpoint{0.640403in}{1.825619in}}{\pgfqpoint{0.637799in}{1.823014in}}%
\pgfpathcurveto{\pgfqpoint{0.635194in}{1.820410in}}{\pgfqpoint{0.633731in}{1.816877in}}{\pgfqpoint{0.633731in}{1.813193in}}%
\pgfpathcurveto{\pgfqpoint{0.633731in}{1.809510in}}{\pgfqpoint{0.635194in}{1.805977in}}{\pgfqpoint{0.637799in}{1.803372in}}%
\pgfpathcurveto{\pgfqpoint{0.640403in}{1.800768in}}{\pgfqpoint{0.643936in}{1.799304in}}{\pgfqpoint{0.647620in}{1.799304in}}%
\pgfpathclose%
\pgfusepath{stroke}%
\end{pgfscope}%
\begin{pgfscope}%
\pgfpathrectangle{\pgfqpoint{0.438556in}{0.383578in}}{\pgfqpoint{2.325000in}{2.310000in}}%
\pgfusepath{clip}%
\pgfsetbuttcap%
\pgfsetroundjoin%
\pgfsetlinewidth{0.803000pt}%
\definecolor{currentstroke}{rgb}{0.000000,0.356863,0.509804}%
\pgfsetstrokecolor{currentstroke}%
\pgfsetdash{}{0pt}%
\pgfpathmoveto{\pgfqpoint{1.159678in}{1.821178in}}%
\pgfpathcurveto{\pgfqpoint{1.163361in}{1.821178in}}{\pgfqpoint{1.166894in}{1.822642in}}{\pgfqpoint{1.169499in}{1.825246in}}%
\pgfpathcurveto{\pgfqpoint{1.172103in}{1.827851in}}{\pgfqpoint{1.173567in}{1.831384in}}{\pgfqpoint{1.173567in}{1.835067in}}%
\pgfpathcurveto{\pgfqpoint{1.173567in}{1.838750in}}{\pgfqpoint{1.172103in}{1.842283in}}{\pgfqpoint{1.169499in}{1.844888in}}%
\pgfpathcurveto{\pgfqpoint{1.166894in}{1.847493in}}{\pgfqpoint{1.163361in}{1.848956in}}{\pgfqpoint{1.159678in}{1.848956in}}%
\pgfpathcurveto{\pgfqpoint{1.155994in}{1.848956in}}{\pgfqpoint{1.152461in}{1.847493in}}{\pgfqpoint{1.149857in}{1.844888in}}%
\pgfpathcurveto{\pgfqpoint{1.147252in}{1.842283in}}{\pgfqpoint{1.145789in}{1.838750in}}{\pgfqpoint{1.145789in}{1.835067in}}%
\pgfpathcurveto{\pgfqpoint{1.145789in}{1.831384in}}{\pgfqpoint{1.147252in}{1.827851in}}{\pgfqpoint{1.149857in}{1.825246in}}%
\pgfpathcurveto{\pgfqpoint{1.152461in}{1.822642in}}{\pgfqpoint{1.155994in}{1.821178in}}{\pgfqpoint{1.159678in}{1.821178in}}%
\pgfpathclose%
\pgfusepath{stroke}%
\end{pgfscope}%
\begin{pgfscope}%
\pgfpathrectangle{\pgfqpoint{0.438556in}{0.383578in}}{\pgfqpoint{2.325000in}{2.310000in}}%
\pgfusepath{clip}%
\pgfsetbuttcap%
\pgfsetroundjoin%
\pgfsetlinewidth{0.803000pt}%
\definecolor{currentstroke}{rgb}{0.000000,0.356863,0.509804}%
\pgfsetstrokecolor{currentstroke}%
\pgfsetdash{}{0pt}%
\pgfpathmoveto{\pgfqpoint{1.362604in}{1.843052in}}%
\pgfpathcurveto{\pgfqpoint{1.366287in}{1.843052in}}{\pgfqpoint{1.369820in}{1.844515in}}{\pgfqpoint{1.372425in}{1.847120in}}%
\pgfpathcurveto{\pgfqpoint{1.375029in}{1.849724in}}{\pgfqpoint{1.376493in}{1.853257in}}{\pgfqpoint{1.376493in}{1.856941in}}%
\pgfpathcurveto{\pgfqpoint{1.376493in}{1.860624in}}{\pgfqpoint{1.375029in}{1.864157in}}{\pgfqpoint{1.372425in}{1.866762in}}%
\pgfpathcurveto{\pgfqpoint{1.369820in}{1.869366in}}{\pgfqpoint{1.366287in}{1.870830in}}{\pgfqpoint{1.362604in}{1.870830in}}%
\pgfpathcurveto{\pgfqpoint{1.358920in}{1.870830in}}{\pgfqpoint{1.355387in}{1.869366in}}{\pgfqpoint{1.352783in}{1.866762in}}%
\pgfpathcurveto{\pgfqpoint{1.350178in}{1.864157in}}{\pgfqpoint{1.348715in}{1.860624in}}{\pgfqpoint{1.348715in}{1.856941in}}%
\pgfpathcurveto{\pgfqpoint{1.348715in}{1.853257in}}{\pgfqpoint{1.350178in}{1.849724in}}{\pgfqpoint{1.352783in}{1.847120in}}%
\pgfpathcurveto{\pgfqpoint{1.355387in}{1.844515in}}{\pgfqpoint{1.358920in}{1.843052in}}{\pgfqpoint{1.362604in}{1.843052in}}%
\pgfpathclose%
\pgfusepath{stroke}%
\end{pgfscope}%
\begin{pgfscope}%
\pgfpathrectangle{\pgfqpoint{0.438556in}{0.383578in}}{\pgfqpoint{2.325000in}{2.310000in}}%
\pgfusepath{clip}%
\pgfsetbuttcap%
\pgfsetroundjoin%
\pgfsetlinewidth{0.803000pt}%
\definecolor{currentstroke}{rgb}{0.000000,0.356863,0.509804}%
\pgfsetstrokecolor{currentstroke}%
\pgfsetdash{}{0pt}%
\pgfpathmoveto{\pgfqpoint{1.450024in}{1.864926in}}%
\pgfpathcurveto{\pgfqpoint{1.453707in}{1.864926in}}{\pgfqpoint{1.457240in}{1.866389in}}{\pgfqpoint{1.459845in}{1.868994in}}%
\pgfpathcurveto{\pgfqpoint{1.462449in}{1.871598in}}{\pgfqpoint{1.463913in}{1.875131in}}{\pgfqpoint{1.463913in}{1.878815in}}%
\pgfpathcurveto{\pgfqpoint{1.463913in}{1.882498in}}{\pgfqpoint{1.462449in}{1.886031in}}{\pgfqpoint{1.459845in}{1.888636in}}%
\pgfpathcurveto{\pgfqpoint{1.457240in}{1.891240in}}{\pgfqpoint{1.453707in}{1.892704in}}{\pgfqpoint{1.450024in}{1.892704in}}%
\pgfpathcurveto{\pgfqpoint{1.446340in}{1.892704in}}{\pgfqpoint{1.442807in}{1.891240in}}{\pgfqpoint{1.440203in}{1.888636in}}%
\pgfpathcurveto{\pgfqpoint{1.437598in}{1.886031in}}{\pgfqpoint{1.436135in}{1.882498in}}{\pgfqpoint{1.436135in}{1.878815in}}%
\pgfpathcurveto{\pgfqpoint{1.436135in}{1.875131in}}{\pgfqpoint{1.437598in}{1.871598in}}{\pgfqpoint{1.440203in}{1.868994in}}%
\pgfpathcurveto{\pgfqpoint{1.442807in}{1.866389in}}{\pgfqpoint{1.446340in}{1.864926in}}{\pgfqpoint{1.450024in}{1.864926in}}%
\pgfpathclose%
\pgfusepath{stroke}%
\end{pgfscope}%
\begin{pgfscope}%
\pgfpathrectangle{\pgfqpoint{0.438556in}{0.383578in}}{\pgfqpoint{2.325000in}{2.310000in}}%
\pgfusepath{clip}%
\pgfsetbuttcap%
\pgfsetroundjoin%
\pgfsetlinewidth{0.803000pt}%
\definecolor{currentstroke}{rgb}{0.000000,0.356863,0.509804}%
\pgfsetstrokecolor{currentstroke}%
\pgfsetdash{}{0pt}%
\pgfpathmoveto{\pgfqpoint{2.156266in}{1.886799in}}%
\pgfpathcurveto{\pgfqpoint{2.159949in}{1.886799in}}{\pgfqpoint{2.163482in}{1.888263in}}{\pgfqpoint{2.166087in}{1.890867in}}%
\pgfpathcurveto{\pgfqpoint{2.168691in}{1.893472in}}{\pgfqpoint{2.170155in}{1.897005in}}{\pgfqpoint{2.170155in}{1.900688in}}%
\pgfpathcurveto{\pgfqpoint{2.170155in}{1.904372in}}{\pgfqpoint{2.168691in}{1.907905in}}{\pgfqpoint{2.166087in}{1.910509in}}%
\pgfpathcurveto{\pgfqpoint{2.163482in}{1.913114in}}{\pgfqpoint{2.159949in}{1.914577in}}{\pgfqpoint{2.156266in}{1.914577in}}%
\pgfpathcurveto{\pgfqpoint{2.152582in}{1.914577in}}{\pgfqpoint{2.149049in}{1.913114in}}{\pgfqpoint{2.146445in}{1.910509in}}%
\pgfpathcurveto{\pgfqpoint{2.143840in}{1.907905in}}{\pgfqpoint{2.142377in}{1.904372in}}{\pgfqpoint{2.142377in}{1.900688in}}%
\pgfpathcurveto{\pgfqpoint{2.142377in}{1.897005in}}{\pgfqpoint{2.143840in}{1.893472in}}{\pgfqpoint{2.146445in}{1.890867in}}%
\pgfpathcurveto{\pgfqpoint{2.149049in}{1.888263in}}{\pgfqpoint{2.152582in}{1.886799in}}{\pgfqpoint{2.156266in}{1.886799in}}%
\pgfpathclose%
\pgfusepath{stroke}%
\end{pgfscope}%
\begin{pgfscope}%
\pgfpathrectangle{\pgfqpoint{0.438556in}{0.383578in}}{\pgfqpoint{2.325000in}{2.310000in}}%
\pgfusepath{clip}%
\pgfsetbuttcap%
\pgfsetroundjoin%
\pgfsetlinewidth{0.803000pt}%
\definecolor{currentstroke}{rgb}{0.000000,0.356863,0.509804}%
\pgfsetstrokecolor{currentstroke}%
\pgfsetdash{}{0pt}%
\pgfpathmoveto{\pgfqpoint{1.820908in}{1.952421in}}%
\pgfpathcurveto{\pgfqpoint{1.824591in}{1.952421in}}{\pgfqpoint{1.828124in}{1.953884in}}{\pgfqpoint{1.830729in}{1.956489in}}%
\pgfpathcurveto{\pgfqpoint{1.833333in}{1.959093in}}{\pgfqpoint{1.834797in}{1.962626in}}{\pgfqpoint{1.834797in}{1.966310in}}%
\pgfpathcurveto{\pgfqpoint{1.834797in}{1.969993in}}{\pgfqpoint{1.833333in}{1.973526in}}{\pgfqpoint{1.830729in}{1.976131in}}%
\pgfpathcurveto{\pgfqpoint{1.828124in}{1.978735in}}{\pgfqpoint{1.824591in}{1.980199in}}{\pgfqpoint{1.820908in}{1.980199in}}%
\pgfpathcurveto{\pgfqpoint{1.817224in}{1.980199in}}{\pgfqpoint{1.813691in}{1.978735in}}{\pgfqpoint{1.811087in}{1.976131in}}%
\pgfpathcurveto{\pgfqpoint{1.808482in}{1.973526in}}{\pgfqpoint{1.807019in}{1.969993in}}{\pgfqpoint{1.807019in}{1.966310in}}%
\pgfpathcurveto{\pgfqpoint{1.807019in}{1.962626in}}{\pgfqpoint{1.808482in}{1.959093in}}{\pgfqpoint{1.811087in}{1.956489in}}%
\pgfpathcurveto{\pgfqpoint{1.813691in}{1.953884in}}{\pgfqpoint{1.817224in}{1.952421in}}{\pgfqpoint{1.820908in}{1.952421in}}%
\pgfpathclose%
\pgfusepath{stroke}%
\end{pgfscope}%
\begin{pgfscope}%
\pgfpathrectangle{\pgfqpoint{0.438556in}{0.383578in}}{\pgfqpoint{2.325000in}{2.310000in}}%
\pgfusepath{clip}%
\pgfsetbuttcap%
\pgfsetroundjoin%
\pgfsetlinewidth{0.803000pt}%
\definecolor{currentstroke}{rgb}{0.000000,0.356863,0.509804}%
\pgfsetstrokecolor{currentstroke}%
\pgfsetdash{}{0pt}%
\pgfpathmoveto{\pgfqpoint{0.797164in}{1.996168in}}%
\pgfpathcurveto{\pgfqpoint{0.800847in}{1.996168in}}{\pgfqpoint{0.804380in}{1.997632in}}{\pgfqpoint{0.806985in}{2.000236in}}%
\pgfpathcurveto{\pgfqpoint{0.809589in}{2.002841in}}{\pgfqpoint{0.811053in}{2.006374in}}{\pgfqpoint{0.811053in}{2.010057in}}%
\pgfpathcurveto{\pgfqpoint{0.811053in}{2.013741in}}{\pgfqpoint{0.809589in}{2.017274in}}{\pgfqpoint{0.806985in}{2.019878in}}%
\pgfpathcurveto{\pgfqpoint{0.804380in}{2.022483in}}{\pgfqpoint{0.800847in}{2.023946in}}{\pgfqpoint{0.797164in}{2.023946in}}%
\pgfpathcurveto{\pgfqpoint{0.793480in}{2.023946in}}{\pgfqpoint{0.789947in}{2.022483in}}{\pgfqpoint{0.787343in}{2.019878in}}%
\pgfpathcurveto{\pgfqpoint{0.784738in}{2.017274in}}{\pgfqpoint{0.783275in}{2.013741in}}{\pgfqpoint{0.783275in}{2.010057in}}%
\pgfpathcurveto{\pgfqpoint{0.783275in}{2.006374in}}{\pgfqpoint{0.784738in}{2.002841in}}{\pgfqpoint{0.787343in}{2.000236in}}%
\pgfpathcurveto{\pgfqpoint{0.789947in}{1.997632in}}{\pgfqpoint{0.793480in}{1.996168in}}{\pgfqpoint{0.797164in}{1.996168in}}%
\pgfpathclose%
\pgfusepath{stroke}%
\end{pgfscope}%
\begin{pgfscope}%
\pgfpathrectangle{\pgfqpoint{0.438556in}{0.383578in}}{\pgfqpoint{2.325000in}{2.310000in}}%
\pgfusepath{clip}%
\pgfsetbuttcap%
\pgfsetroundjoin%
\pgfsetlinewidth{0.803000pt}%
\definecolor{currentstroke}{rgb}{0.000000,0.356863,0.509804}%
\pgfsetstrokecolor{currentstroke}%
\pgfsetdash{}{0pt}%
\pgfpathmoveto{\pgfqpoint{1.442212in}{2.039916in}}%
\pgfpathcurveto{\pgfqpoint{1.445895in}{2.039916in}}{\pgfqpoint{1.449428in}{2.041379in}}{\pgfqpoint{1.452033in}{2.043984in}}%
\pgfpathcurveto{\pgfqpoint{1.454637in}{2.046588in}}{\pgfqpoint{1.456101in}{2.050121in}}{\pgfqpoint{1.456101in}{2.053805in}}%
\pgfpathcurveto{\pgfqpoint{1.456101in}{2.057488in}}{\pgfqpoint{1.454637in}{2.061021in}}{\pgfqpoint{1.452033in}{2.063626in}}%
\pgfpathcurveto{\pgfqpoint{1.449428in}{2.066230in}}{\pgfqpoint{1.445895in}{2.067694in}}{\pgfqpoint{1.442212in}{2.067694in}}%
\pgfpathcurveto{\pgfqpoint{1.438528in}{2.067694in}}{\pgfqpoint{1.434995in}{2.066230in}}{\pgfqpoint{1.432391in}{2.063626in}}%
\pgfpathcurveto{\pgfqpoint{1.429786in}{2.061021in}}{\pgfqpoint{1.428323in}{2.057488in}}{\pgfqpoint{1.428323in}{2.053805in}}%
\pgfpathcurveto{\pgfqpoint{1.428323in}{2.050121in}}{\pgfqpoint{1.429786in}{2.046588in}}{\pgfqpoint{1.432391in}{2.043984in}}%
\pgfpathcurveto{\pgfqpoint{1.434995in}{2.041379in}}{\pgfqpoint{1.438528in}{2.039916in}}{\pgfqpoint{1.442212in}{2.039916in}}%
\pgfpathclose%
\pgfusepath{stroke}%
\end{pgfscope}%
\begin{pgfscope}%
\pgfpathrectangle{\pgfqpoint{0.438556in}{0.383578in}}{\pgfqpoint{2.325000in}{2.310000in}}%
\pgfusepath{clip}%
\pgfsetbuttcap%
\pgfsetroundjoin%
\pgfsetlinewidth{0.803000pt}%
\definecolor{currentstroke}{rgb}{0.000000,0.356863,0.509804}%
\pgfsetstrokecolor{currentstroke}%
\pgfsetdash{}{0pt}%
\pgfpathmoveto{\pgfqpoint{0.878818in}{2.083663in}}%
\pgfpathcurveto{\pgfqpoint{0.882501in}{2.083663in}}{\pgfqpoint{0.886034in}{2.085127in}}{\pgfqpoint{0.888639in}{2.087731in}}%
\pgfpathcurveto{\pgfqpoint{0.891243in}{2.090336in}}{\pgfqpoint{0.892707in}{2.093869in}}{\pgfqpoint{0.892707in}{2.097552in}}%
\pgfpathcurveto{\pgfqpoint{0.892707in}{2.101236in}}{\pgfqpoint{0.891243in}{2.104769in}}{\pgfqpoint{0.888639in}{2.107373in}}%
\pgfpathcurveto{\pgfqpoint{0.886034in}{2.109978in}}{\pgfqpoint{0.882501in}{2.111441in}}{\pgfqpoint{0.878818in}{2.111441in}}%
\pgfpathcurveto{\pgfqpoint{0.875134in}{2.111441in}}{\pgfqpoint{0.871601in}{2.109978in}}{\pgfqpoint{0.868997in}{2.107373in}}%
\pgfpathcurveto{\pgfqpoint{0.866392in}{2.104769in}}{\pgfqpoint{0.864929in}{2.101236in}}{\pgfqpoint{0.864929in}{2.097552in}}%
\pgfpathcurveto{\pgfqpoint{0.864929in}{2.093869in}}{\pgfqpoint{0.866392in}{2.090336in}}{\pgfqpoint{0.868997in}{2.087731in}}%
\pgfpathcurveto{\pgfqpoint{0.871601in}{2.085127in}}{\pgfqpoint{0.875134in}{2.083663in}}{\pgfqpoint{0.878818in}{2.083663in}}%
\pgfpathclose%
\pgfusepath{stroke}%
\end{pgfscope}%
\begin{pgfscope}%
\pgfpathrectangle{\pgfqpoint{0.438556in}{0.383578in}}{\pgfqpoint{2.325000in}{2.310000in}}%
\pgfusepath{clip}%
\pgfsetbuttcap%
\pgfsetroundjoin%
\pgfsetlinewidth{0.803000pt}%
\definecolor{currentstroke}{rgb}{0.000000,0.356863,0.509804}%
\pgfsetstrokecolor{currentstroke}%
\pgfsetdash{}{0pt}%
\pgfpathmoveto{\pgfqpoint{0.819856in}{2.127411in}}%
\pgfpathcurveto{\pgfqpoint{0.823539in}{2.127411in}}{\pgfqpoint{0.827072in}{2.128874in}}{\pgfqpoint{0.829677in}{2.131479in}}%
\pgfpathcurveto{\pgfqpoint{0.832281in}{2.134083in}}{\pgfqpoint{0.833745in}{2.137616in}}{\pgfqpoint{0.833745in}{2.141300in}}%
\pgfpathcurveto{\pgfqpoint{0.833745in}{2.144983in}}{\pgfqpoint{0.832281in}{2.148516in}}{\pgfqpoint{0.829677in}{2.151121in}}%
\pgfpathcurveto{\pgfqpoint{0.827072in}{2.153725in}}{\pgfqpoint{0.823539in}{2.155189in}}{\pgfqpoint{0.819856in}{2.155189in}}%
\pgfpathcurveto{\pgfqpoint{0.816172in}{2.155189in}}{\pgfqpoint{0.812639in}{2.153725in}}{\pgfqpoint{0.810035in}{2.151121in}}%
\pgfpathcurveto{\pgfqpoint{0.807430in}{2.148516in}}{\pgfqpoint{0.805967in}{2.144983in}}{\pgfqpoint{0.805967in}{2.141300in}}%
\pgfpathcurveto{\pgfqpoint{0.805967in}{2.137616in}}{\pgfqpoint{0.807430in}{2.134083in}}{\pgfqpoint{0.810035in}{2.131479in}}%
\pgfpathcurveto{\pgfqpoint{0.812639in}{2.128874in}}{\pgfqpoint{0.816172in}{2.127411in}}{\pgfqpoint{0.819856in}{2.127411in}}%
\pgfpathclose%
\pgfusepath{stroke}%
\end{pgfscope}%
\begin{pgfscope}%
\pgfpathrectangle{\pgfqpoint{0.438556in}{0.383578in}}{\pgfqpoint{2.325000in}{2.310000in}}%
\pgfusepath{clip}%
\pgfsetbuttcap%
\pgfsetroundjoin%
\pgfsetlinewidth{0.803000pt}%
\definecolor{currentstroke}{rgb}{0.000000,0.356863,0.509804}%
\pgfsetstrokecolor{currentstroke}%
\pgfsetdash{}{0pt}%
\pgfpathmoveto{\pgfqpoint{1.863502in}{2.149285in}}%
\pgfpathcurveto{\pgfqpoint{1.867185in}{2.149285in}}{\pgfqpoint{1.870718in}{2.150748in}}{\pgfqpoint{1.873323in}{2.153353in}}%
\pgfpathcurveto{\pgfqpoint{1.875927in}{2.155957in}}{\pgfqpoint{1.877391in}{2.159490in}}{\pgfqpoint{1.877391in}{2.163174in}}%
\pgfpathcurveto{\pgfqpoint{1.877391in}{2.166857in}}{\pgfqpoint{1.875927in}{2.170390in}}{\pgfqpoint{1.873323in}{2.172994in}}%
\pgfpathcurveto{\pgfqpoint{1.870718in}{2.175599in}}{\pgfqpoint{1.867185in}{2.177062in}}{\pgfqpoint{1.863502in}{2.177062in}}%
\pgfpathcurveto{\pgfqpoint{1.859818in}{2.177062in}}{\pgfqpoint{1.856285in}{2.175599in}}{\pgfqpoint{1.853681in}{2.172994in}}%
\pgfpathcurveto{\pgfqpoint{1.851076in}{2.170390in}}{\pgfqpoint{1.849613in}{2.166857in}}{\pgfqpoint{1.849613in}{2.163174in}}%
\pgfpathcurveto{\pgfqpoint{1.849613in}{2.159490in}}{\pgfqpoint{1.851076in}{2.155957in}}{\pgfqpoint{1.853681in}{2.153353in}}%
\pgfpathcurveto{\pgfqpoint{1.856285in}{2.150748in}}{\pgfqpoint{1.859818in}{2.149285in}}{\pgfqpoint{1.863502in}{2.149285in}}%
\pgfpathclose%
\pgfusepath{stroke}%
\end{pgfscope}%
\begin{pgfscope}%
\pgfpathrectangle{\pgfqpoint{0.438556in}{0.383578in}}{\pgfqpoint{2.325000in}{2.310000in}}%
\pgfusepath{clip}%
\pgfsetbuttcap%
\pgfsetroundjoin%
\pgfsetlinewidth{0.803000pt}%
\definecolor{currentstroke}{rgb}{0.000000,0.356863,0.509804}%
\pgfsetstrokecolor{currentstroke}%
\pgfsetdash{}{0pt}%
\pgfpathmoveto{\pgfqpoint{2.090236in}{2.171158in}}%
\pgfpathcurveto{\pgfqpoint{2.093919in}{2.171158in}}{\pgfqpoint{2.097452in}{2.172622in}}{\pgfqpoint{2.100057in}{2.175226in}}%
\pgfpathcurveto{\pgfqpoint{2.102661in}{2.177831in}}{\pgfqpoint{2.104125in}{2.181364in}}{\pgfqpoint{2.104125in}{2.185047in}}%
\pgfpathcurveto{\pgfqpoint{2.104125in}{2.188731in}}{\pgfqpoint{2.102661in}{2.192264in}}{\pgfqpoint{2.100057in}{2.194868in}}%
\pgfpathcurveto{\pgfqpoint{2.097452in}{2.197473in}}{\pgfqpoint{2.093919in}{2.198936in}}{\pgfqpoint{2.090236in}{2.198936in}}%
\pgfpathcurveto{\pgfqpoint{2.086552in}{2.198936in}}{\pgfqpoint{2.083019in}{2.197473in}}{\pgfqpoint{2.080415in}{2.194868in}}%
\pgfpathcurveto{\pgfqpoint{2.077810in}{2.192264in}}{\pgfqpoint{2.076347in}{2.188731in}}{\pgfqpoint{2.076347in}{2.185047in}}%
\pgfpathcurveto{\pgfqpoint{2.076347in}{2.181364in}}{\pgfqpoint{2.077810in}{2.177831in}}{\pgfqpoint{2.080415in}{2.175226in}}%
\pgfpathcurveto{\pgfqpoint{2.083019in}{2.172622in}}{\pgfqpoint{2.086552in}{2.171158in}}{\pgfqpoint{2.090236in}{2.171158in}}%
\pgfpathclose%
\pgfusepath{stroke}%
\end{pgfscope}%
\begin{pgfscope}%
\pgfpathrectangle{\pgfqpoint{0.438556in}{0.383578in}}{\pgfqpoint{2.325000in}{2.310000in}}%
\pgfusepath{clip}%
\pgfsetbuttcap%
\pgfsetroundjoin%
\pgfsetlinewidth{0.803000pt}%
\definecolor{currentstroke}{rgb}{0.000000,0.356863,0.509804}%
\pgfsetstrokecolor{currentstroke}%
\pgfsetdash{}{0pt}%
\pgfpathmoveto{\pgfqpoint{1.251934in}{2.258653in}}%
\pgfpathcurveto{\pgfqpoint{1.255617in}{2.258653in}}{\pgfqpoint{1.259150in}{2.260117in}}{\pgfqpoint{1.261755in}{2.262721in}}%
\pgfpathcurveto{\pgfqpoint{1.264359in}{2.265326in}}{\pgfqpoint{1.265823in}{2.268859in}}{\pgfqpoint{1.265823in}{2.272542in}}%
\pgfpathcurveto{\pgfqpoint{1.265823in}{2.276226in}}{\pgfqpoint{1.264359in}{2.279759in}}{\pgfqpoint{1.261755in}{2.282363in}}%
\pgfpathcurveto{\pgfqpoint{1.259150in}{2.284968in}}{\pgfqpoint{1.255617in}{2.286431in}}{\pgfqpoint{1.251934in}{2.286431in}}%
\pgfpathcurveto{\pgfqpoint{1.248250in}{2.286431in}}{\pgfqpoint{1.244717in}{2.284968in}}{\pgfqpoint{1.242113in}{2.282363in}}%
\pgfpathcurveto{\pgfqpoint{1.239508in}{2.279759in}}{\pgfqpoint{1.238045in}{2.276226in}}{\pgfqpoint{1.238045in}{2.272542in}}%
\pgfpathcurveto{\pgfqpoint{1.238045in}{2.268859in}}{\pgfqpoint{1.239508in}{2.265326in}}{\pgfqpoint{1.242113in}{2.262721in}}%
\pgfpathcurveto{\pgfqpoint{1.244717in}{2.260117in}}{\pgfqpoint{1.248250in}{2.258653in}}{\pgfqpoint{1.251934in}{2.258653in}}%
\pgfpathclose%
\pgfusepath{stroke}%
\end{pgfscope}%
\begin{pgfscope}%
\pgfpathrectangle{\pgfqpoint{0.438556in}{0.383578in}}{\pgfqpoint{2.325000in}{2.310000in}}%
\pgfusepath{clip}%
\pgfsetbuttcap%
\pgfsetroundjoin%
\pgfsetlinewidth{0.803000pt}%
\definecolor{currentstroke}{rgb}{0.000000,0.356863,0.509804}%
\pgfsetstrokecolor{currentstroke}%
\pgfsetdash{}{0pt}%
\pgfpathmoveto{\pgfqpoint{1.372276in}{2.280527in}}%
\pgfpathcurveto{\pgfqpoint{1.375959in}{2.280527in}}{\pgfqpoint{1.379492in}{2.281991in}}{\pgfqpoint{1.382097in}{2.284595in}}%
\pgfpathcurveto{\pgfqpoint{1.384701in}{2.287200in}}{\pgfqpoint{1.386165in}{2.290733in}}{\pgfqpoint{1.386165in}{2.294416in}}%
\pgfpathcurveto{\pgfqpoint{1.386165in}{2.298099in}}{\pgfqpoint{1.384701in}{2.301632in}}{\pgfqpoint{1.382097in}{2.304237in}}%
\pgfpathcurveto{\pgfqpoint{1.379492in}{2.306842in}}{\pgfqpoint{1.375959in}{2.308305in}}{\pgfqpoint{1.372276in}{2.308305in}}%
\pgfpathcurveto{\pgfqpoint{1.368592in}{2.308305in}}{\pgfqpoint{1.365059in}{2.306842in}}{\pgfqpoint{1.362455in}{2.304237in}}%
\pgfpathcurveto{\pgfqpoint{1.359850in}{2.301632in}}{\pgfqpoint{1.358387in}{2.298099in}}{\pgfqpoint{1.358387in}{2.294416in}}%
\pgfpathcurveto{\pgfqpoint{1.358387in}{2.290733in}}{\pgfqpoint{1.359850in}{2.287200in}}{\pgfqpoint{1.362455in}{2.284595in}}%
\pgfpathcurveto{\pgfqpoint{1.365059in}{2.281991in}}{\pgfqpoint{1.368592in}{2.280527in}}{\pgfqpoint{1.372276in}{2.280527in}}%
\pgfpathclose%
\pgfusepath{stroke}%
\end{pgfscope}%
\begin{pgfscope}%
\pgfpathrectangle{\pgfqpoint{0.438556in}{0.383578in}}{\pgfqpoint{2.325000in}{2.310000in}}%
\pgfusepath{clip}%
\pgfsetbuttcap%
\pgfsetroundjoin%
\pgfsetlinewidth{0.803000pt}%
\definecolor{currentstroke}{rgb}{0.000000,0.356863,0.509804}%
\pgfsetstrokecolor{currentstroke}%
\pgfsetdash{}{0pt}%
\pgfpathmoveto{\pgfqpoint{1.424728in}{2.324275in}}%
\pgfpathcurveto{\pgfqpoint{1.428411in}{2.324275in}}{\pgfqpoint{1.431944in}{2.325738in}}{\pgfqpoint{1.434549in}{2.328343in}}%
\pgfpathcurveto{\pgfqpoint{1.437153in}{2.330947in}}{\pgfqpoint{1.438617in}{2.334480in}}{\pgfqpoint{1.438617in}{2.338164in}}%
\pgfpathcurveto{\pgfqpoint{1.438617in}{2.341847in}}{\pgfqpoint{1.437153in}{2.345380in}}{\pgfqpoint{1.434549in}{2.347985in}}%
\pgfpathcurveto{\pgfqpoint{1.431944in}{2.350589in}}{\pgfqpoint{1.428411in}{2.352053in}}{\pgfqpoint{1.424728in}{2.352053in}}%
\pgfpathcurveto{\pgfqpoint{1.421044in}{2.352053in}}{\pgfqpoint{1.417511in}{2.350589in}}{\pgfqpoint{1.414907in}{2.347985in}}%
\pgfpathcurveto{\pgfqpoint{1.412302in}{2.345380in}}{\pgfqpoint{1.410839in}{2.341847in}}{\pgfqpoint{1.410839in}{2.338164in}}%
\pgfpathcurveto{\pgfqpoint{1.410839in}{2.334480in}}{\pgfqpoint{1.412302in}{2.330947in}}{\pgfqpoint{1.414907in}{2.328343in}}%
\pgfpathcurveto{\pgfqpoint{1.417511in}{2.325738in}}{\pgfqpoint{1.421044in}{2.324275in}}{\pgfqpoint{1.424728in}{2.324275in}}%
\pgfpathclose%
\pgfusepath{stroke}%
\end{pgfscope}%
\begin{pgfscope}%
\pgfpathrectangle{\pgfqpoint{0.438556in}{0.383578in}}{\pgfqpoint{2.325000in}{2.310000in}}%
\pgfusepath{clip}%
\pgfsetbuttcap%
\pgfsetroundjoin%
\pgfsetlinewidth{0.803000pt}%
\definecolor{currentstroke}{rgb}{0.000000,0.356863,0.509804}%
\pgfsetstrokecolor{currentstroke}%
\pgfsetdash{}{0pt}%
\pgfpathmoveto{\pgfqpoint{1.792264in}{2.346148in}}%
\pgfpathcurveto{\pgfqpoint{1.795947in}{2.346148in}}{\pgfqpoint{1.799480in}{2.347612in}}{\pgfqpoint{1.802085in}{2.350216in}}%
\pgfpathcurveto{\pgfqpoint{1.804689in}{2.352821in}}{\pgfqpoint{1.806153in}{2.356354in}}{\pgfqpoint{1.806153in}{2.360037in}}%
\pgfpathcurveto{\pgfqpoint{1.806153in}{2.363721in}}{\pgfqpoint{1.804689in}{2.367254in}}{\pgfqpoint{1.802085in}{2.369858in}}%
\pgfpathcurveto{\pgfqpoint{1.799480in}{2.372463in}}{\pgfqpoint{1.795947in}{2.373926in}}{\pgfqpoint{1.792264in}{2.373926in}}%
\pgfpathcurveto{\pgfqpoint{1.788580in}{2.373926in}}{\pgfqpoint{1.785047in}{2.372463in}}{\pgfqpoint{1.782443in}{2.369858in}}%
\pgfpathcurveto{\pgfqpoint{1.779838in}{2.367254in}}{\pgfqpoint{1.778375in}{2.363721in}}{\pgfqpoint{1.778375in}{2.360037in}}%
\pgfpathcurveto{\pgfqpoint{1.778375in}{2.356354in}}{\pgfqpoint{1.779838in}{2.352821in}}{\pgfqpoint{1.782443in}{2.350216in}}%
\pgfpathcurveto{\pgfqpoint{1.785047in}{2.347612in}}{\pgfqpoint{1.788580in}{2.346148in}}{\pgfqpoint{1.792264in}{2.346148in}}%
\pgfpathclose%
\pgfusepath{stroke}%
\end{pgfscope}%
\begin{pgfscope}%
\pgfpathrectangle{\pgfqpoint{0.438556in}{0.383578in}}{\pgfqpoint{2.325000in}{2.310000in}}%
\pgfusepath{clip}%
\pgfsetbuttcap%
\pgfsetroundjoin%
\pgfsetlinewidth{0.803000pt}%
\definecolor{currentstroke}{rgb}{0.000000,0.356863,0.509804}%
\pgfsetstrokecolor{currentstroke}%
\pgfsetdash{}{0pt}%
\pgfpathmoveto{\pgfqpoint{0.854638in}{2.433644in}}%
\pgfpathcurveto{\pgfqpoint{0.858321in}{2.433644in}}{\pgfqpoint{0.861854in}{2.435107in}}{\pgfqpoint{0.864459in}{2.437712in}}%
\pgfpathcurveto{\pgfqpoint{0.867063in}{2.440316in}}{\pgfqpoint{0.868527in}{2.443849in}}{\pgfqpoint{0.868527in}{2.447532in}}%
\pgfpathcurveto{\pgfqpoint{0.868527in}{2.451216in}}{\pgfqpoint{0.867063in}{2.454749in}}{\pgfqpoint{0.864459in}{2.457353in}}%
\pgfpathcurveto{\pgfqpoint{0.861854in}{2.459958in}}{\pgfqpoint{0.858321in}{2.461421in}}{\pgfqpoint{0.854638in}{2.461421in}}%
\pgfpathcurveto{\pgfqpoint{0.850954in}{2.461421in}}{\pgfqpoint{0.847421in}{2.459958in}}{\pgfqpoint{0.844817in}{2.457353in}}%
\pgfpathcurveto{\pgfqpoint{0.842212in}{2.454749in}}{\pgfqpoint{0.840749in}{2.451216in}}{\pgfqpoint{0.840749in}{2.447532in}}%
\pgfpathcurveto{\pgfqpoint{0.840749in}{2.443849in}}{\pgfqpoint{0.842212in}{2.440316in}}{\pgfqpoint{0.844817in}{2.437712in}}%
\pgfpathcurveto{\pgfqpoint{0.847421in}{2.435107in}}{\pgfqpoint{0.850954in}{2.433644in}}{\pgfqpoint{0.854638in}{2.433644in}}%
\pgfpathclose%
\pgfusepath{stroke}%
\end{pgfscope}%
\begin{pgfscope}%
\pgfpathrectangle{\pgfqpoint{0.438556in}{0.383578in}}{\pgfqpoint{2.325000in}{2.310000in}}%
\pgfusepath{clip}%
\pgfsetbuttcap%
\pgfsetroundjoin%
\pgfsetlinewidth{0.803000pt}%
\definecolor{currentstroke}{rgb}{0.000000,0.356863,0.509804}%
\pgfsetstrokecolor{currentstroke}%
\pgfsetdash{}{0pt}%
\pgfpathmoveto{\pgfqpoint{2.088562in}{2.455517in}}%
\pgfpathcurveto{\pgfqpoint{2.092245in}{2.455517in}}{\pgfqpoint{2.095778in}{2.456981in}}{\pgfqpoint{2.098383in}{2.459585in}}%
\pgfpathcurveto{\pgfqpoint{2.100987in}{2.462190in}}{\pgfqpoint{2.102451in}{2.465723in}}{\pgfqpoint{2.102451in}{2.469406in}}%
\pgfpathcurveto{\pgfqpoint{2.102451in}{2.473090in}}{\pgfqpoint{2.100987in}{2.476623in}}{\pgfqpoint{2.098383in}{2.479227in}}%
\pgfpathcurveto{\pgfqpoint{2.095778in}{2.481832in}}{\pgfqpoint{2.092245in}{2.483295in}}{\pgfqpoint{2.088562in}{2.483295in}}%
\pgfpathcurveto{\pgfqpoint{2.084878in}{2.483295in}}{\pgfqpoint{2.081345in}{2.481832in}}{\pgfqpoint{2.078741in}{2.479227in}}%
\pgfpathcurveto{\pgfqpoint{2.076136in}{2.476623in}}{\pgfqpoint{2.074673in}{2.473090in}}{\pgfqpoint{2.074673in}{2.469406in}}%
\pgfpathcurveto{\pgfqpoint{2.074673in}{2.465723in}}{\pgfqpoint{2.076136in}{2.462190in}}{\pgfqpoint{2.078741in}{2.459585in}}%
\pgfpathcurveto{\pgfqpoint{2.081345in}{2.456981in}}{\pgfqpoint{2.084878in}{2.455517in}}{\pgfqpoint{2.088562in}{2.455517in}}%
\pgfpathclose%
\pgfusepath{stroke}%
\end{pgfscope}%
\begin{pgfscope}%
\pgfpathrectangle{\pgfqpoint{0.438556in}{0.383578in}}{\pgfqpoint{2.325000in}{2.310000in}}%
\pgfusepath{clip}%
\pgfsetbuttcap%
\pgfsetroundjoin%
\pgfsetlinewidth{0.803000pt}%
\definecolor{currentstroke}{rgb}{0.000000,0.356863,0.509804}%
\pgfsetstrokecolor{currentstroke}%
\pgfsetdash{}{0pt}%
\pgfpathmoveto{\pgfqpoint{1.604776in}{2.564886in}}%
\pgfpathcurveto{\pgfqpoint{1.608459in}{2.564886in}}{\pgfqpoint{1.611992in}{2.566350in}}{\pgfqpoint{1.614597in}{2.568954in}}%
\pgfpathcurveto{\pgfqpoint{1.617201in}{2.571559in}}{\pgfqpoint{1.618665in}{2.575092in}}{\pgfqpoint{1.618665in}{2.578775in}}%
\pgfpathcurveto{\pgfqpoint{1.618665in}{2.582458in}}{\pgfqpoint{1.617201in}{2.585991in}}{\pgfqpoint{1.614597in}{2.588596in}}%
\pgfpathcurveto{\pgfqpoint{1.611992in}{2.591200in}}{\pgfqpoint{1.608459in}{2.592664in}}{\pgfqpoint{1.604776in}{2.592664in}}%
\pgfpathcurveto{\pgfqpoint{1.601092in}{2.592664in}}{\pgfqpoint{1.597559in}{2.591200in}}{\pgfqpoint{1.594955in}{2.588596in}}%
\pgfpathcurveto{\pgfqpoint{1.592350in}{2.585991in}}{\pgfqpoint{1.590887in}{2.582458in}}{\pgfqpoint{1.590887in}{2.578775in}}%
\pgfpathcurveto{\pgfqpoint{1.590887in}{2.575092in}}{\pgfqpoint{1.592350in}{2.571559in}}{\pgfqpoint{1.594955in}{2.568954in}}%
\pgfpathcurveto{\pgfqpoint{1.597559in}{2.566350in}}{\pgfqpoint{1.601092in}{2.564886in}}{\pgfqpoint{1.604776in}{2.564886in}}%
\pgfpathclose%
\pgfusepath{stroke}%
\end{pgfscope}%
\begin{pgfscope}%
\pgfpathrectangle{\pgfqpoint{0.438556in}{0.383578in}}{\pgfqpoint{2.325000in}{2.310000in}}%
\pgfusepath{clip}%
\pgfsetbuttcap%
\pgfsetroundjoin%
\definecolor{currentfill}{rgb}{0.490196,0.588235,0.431373}%
\pgfsetfillcolor{currentfill}%
\pgfsetlinewidth{0.803000pt}%
\definecolor{currentstroke}{rgb}{0.490196,0.588235,0.431373}%
\pgfsetstrokecolor{currentstroke}%
\pgfsetdash{}{0pt}%
\pgfsys@defobject{currentmarker}{\pgfqpoint{-0.006944in}{-0.006944in}}{\pgfqpoint{0.006944in}{0.006944in}}{%
\pgfpathmoveto{\pgfqpoint{0.000000in}{-0.006944in}}%
\pgfpathcurveto{\pgfqpoint{0.001842in}{-0.006944in}}{\pgfqpoint{0.003608in}{-0.006213in}}{\pgfqpoint{0.004910in}{-0.004910in}}%
\pgfpathcurveto{\pgfqpoint{0.006213in}{-0.003608in}}{\pgfqpoint{0.006944in}{-0.001842in}}{\pgfqpoint{0.006944in}{0.000000in}}%
\pgfpathcurveto{\pgfqpoint{0.006944in}{0.001842in}}{\pgfqpoint{0.006213in}{0.003608in}}{\pgfqpoint{0.004910in}{0.004910in}}%
\pgfpathcurveto{\pgfqpoint{0.003608in}{0.006213in}}{\pgfqpoint{0.001842in}{0.006944in}}{\pgfqpoint{0.000000in}{0.006944in}}%
\pgfpathcurveto{\pgfqpoint{-0.001842in}{0.006944in}}{\pgfqpoint{-0.003608in}{0.006213in}}{\pgfqpoint{-0.004910in}{0.004910in}}%
\pgfpathcurveto{\pgfqpoint{-0.006213in}{0.003608in}}{\pgfqpoint{-0.006944in}{0.001842in}}{\pgfqpoint{-0.006944in}{0.000000in}}%
\pgfpathcurveto{\pgfqpoint{-0.006944in}{-0.001842in}}{\pgfqpoint{-0.006213in}{-0.003608in}}{\pgfqpoint{-0.004910in}{-0.004910in}}%
\pgfpathcurveto{\pgfqpoint{-0.003608in}{-0.006213in}}{\pgfqpoint{-0.001842in}{-0.006944in}}{\pgfqpoint{0.000000in}{-0.006944in}}%
\pgfpathclose%
\pgfusepath{stroke,fill}%
}%
\begin{pgfscope}%
\pgfsys@transformshift{0.667894in}{0.500768in}%
\pgfsys@useobject{currentmarker}{}%
\end{pgfscope}%
\begin{pgfscope}%
\pgfsys@transformshift{1.176232in}{0.632010in}%
\pgfsys@useobject{currentmarker}{}%
\end{pgfscope}%
\begin{pgfscope}%
\pgfsys@transformshift{0.648736in}{0.653884in}%
\pgfsys@useobject{currentmarker}{}%
\end{pgfscope}%
\begin{pgfscope}%
\pgfsys@transformshift{2.074612in}{0.697631in}%
\pgfsys@useobject{currentmarker}{}%
\end{pgfscope}%
\begin{pgfscope}%
\pgfsys@transformshift{1.257886in}{0.719505in}%
\pgfsys@useobject{currentmarker}{}%
\end{pgfscope}%
\begin{pgfscope}%
\pgfsys@transformshift{0.777262in}{0.807000in}%
\pgfsys@useobject{currentmarker}{}%
\end{pgfscope}%
\begin{pgfscope}%
\pgfsys@transformshift{2.211694in}{0.828874in}%
\pgfsys@useobject{currentmarker}{}%
\end{pgfscope}%
\begin{pgfscope}%
\pgfsys@transformshift{1.197064in}{0.872622in}%
\pgfsys@useobject{currentmarker}{}%
\end{pgfscope}%
\begin{pgfscope}%
\pgfsys@transformshift{2.221738in}{0.894495in}%
\pgfsys@useobject{currentmarker}{}%
\end{pgfscope}%
\begin{pgfscope}%
\pgfsys@transformshift{0.649294in}{0.938243in}%
\pgfsys@useobject{currentmarker}{}%
\end{pgfscope}%
\begin{pgfscope}%
\pgfsys@transformshift{0.759964in}{0.960117in}%
\pgfsys@useobject{currentmarker}{}%
\end{pgfscope}%
\begin{pgfscope}%
\pgfsys@transformshift{1.796356in}{1.025738in}%
\pgfsys@useobject{currentmarker}{}%
\end{pgfscope}%
\begin{pgfscope}%
\pgfsys@transformshift{1.087324in}{1.047612in}%
\pgfsys@useobject{currentmarker}{}%
\end{pgfscope}%
\begin{pgfscope}%
\pgfsys@transformshift{1.119688in}{1.091359in}%
\pgfsys@useobject{currentmarker}{}%
\end{pgfscope}%
\begin{pgfscope}%
\pgfsys@transformshift{1.307362in}{1.113233in}%
\pgfsys@useobject{currentmarker}{}%
\end{pgfscope}%
\begin{pgfscope}%
\pgfsys@transformshift{2.116276in}{1.178854in}%
\pgfsys@useobject{currentmarker}{}%
\end{pgfscope}%
\begin{pgfscope}%
\pgfsys@transformshift{1.028548in}{1.222602in}%
\pgfsys@useobject{currentmarker}{}%
\end{pgfscope}%
\begin{pgfscope}%
\pgfsys@transformshift{2.263960in}{1.244475in}%
\pgfsys@useobject{currentmarker}{}%
\end{pgfscope}%
\begin{pgfscope}%
\pgfsys@transformshift{0.968284in}{1.266349in}%
\pgfsys@useobject{currentmarker}{}%
\end{pgfscope}%
\begin{pgfscope}%
\pgfsys@transformshift{1.944970in}{1.288223in}%
\pgfsys@useobject{currentmarker}{}%
\end{pgfscope}%
\begin{pgfscope}%
\pgfsys@transformshift{0.903556in}{1.310097in}%
\pgfsys@useobject{currentmarker}{}%
\end{pgfscope}%
\begin{pgfscope}%
\pgfsys@transformshift{0.522628in}{1.331971in}%
\pgfsys@useobject{currentmarker}{}%
\end{pgfscope}%
\begin{pgfscope}%
\pgfsys@transformshift{2.002072in}{1.353844in}%
\pgfsys@useobject{currentmarker}{}%
\end{pgfscope}%
\begin{pgfscope}%
\pgfsys@transformshift{0.952474in}{1.463213in}%
\pgfsys@useobject{currentmarker}{}%
\end{pgfscope}%
\begin{pgfscope}%
\pgfsys@transformshift{1.151680in}{1.485087in}%
\pgfsys@useobject{currentmarker}{}%
\end{pgfscope}%
\begin{pgfscope}%
\pgfsys@transformshift{0.528208in}{1.506961in}%
\pgfsys@useobject{currentmarker}{}%
\end{pgfscope}%
\begin{pgfscope}%
\pgfsys@transformshift{0.803674in}{1.550708in}%
\pgfsys@useobject{currentmarker}{}%
\end{pgfscope}%
\begin{pgfscope}%
\pgfsys@transformshift{1.194460in}{1.660077in}%
\pgfsys@useobject{currentmarker}{}%
\end{pgfscope}%
\begin{pgfscope}%
\pgfsys@transformshift{2.023276in}{1.681951in}%
\pgfsys@useobject{currentmarker}{}%
\end{pgfscope}%
\begin{pgfscope}%
\pgfsys@transformshift{1.867036in}{1.703825in}%
\pgfsys@useobject{currentmarker}{}%
\end{pgfscope}%
\begin{pgfscope}%
\pgfsys@transformshift{0.797908in}{1.725698in}%
\pgfsys@useobject{currentmarker}{}%
\end{pgfscope}%
\begin{pgfscope}%
\pgfsys@transformshift{0.945034in}{1.791320in}%
\pgfsys@useobject{currentmarker}{}%
\end{pgfscope}%
\begin{pgfscope}%
\pgfsys@transformshift{0.647620in}{1.813193in}%
\pgfsys@useobject{currentmarker}{}%
\end{pgfscope}%
\begin{pgfscope}%
\pgfsys@transformshift{1.159678in}{1.835067in}%
\pgfsys@useobject{currentmarker}{}%
\end{pgfscope}%
\begin{pgfscope}%
\pgfsys@transformshift{1.362604in}{1.856941in}%
\pgfsys@useobject{currentmarker}{}%
\end{pgfscope}%
\begin{pgfscope}%
\pgfsys@transformshift{1.450024in}{1.878815in}%
\pgfsys@useobject{currentmarker}{}%
\end{pgfscope}%
\begin{pgfscope}%
\pgfsys@transformshift{2.156266in}{1.900688in}%
\pgfsys@useobject{currentmarker}{}%
\end{pgfscope}%
\begin{pgfscope}%
\pgfsys@transformshift{1.260490in}{1.944436in}%
\pgfsys@useobject{currentmarker}{}%
\end{pgfscope}%
\begin{pgfscope}%
\pgfsys@transformshift{1.820908in}{1.966310in}%
\pgfsys@useobject{currentmarker}{}%
\end{pgfscope}%
\begin{pgfscope}%
\pgfsys@transformshift{0.797164in}{2.010057in}%
\pgfsys@useobject{currentmarker}{}%
\end{pgfscope}%
\begin{pgfscope}%
\pgfsys@transformshift{1.288948in}{2.031931in}%
\pgfsys@useobject{currentmarker}{}%
\end{pgfscope}%
\begin{pgfscope}%
\pgfsys@transformshift{1.442212in}{2.053805in}%
\pgfsys@useobject{currentmarker}{}%
\end{pgfscope}%
\begin{pgfscope}%
\pgfsys@transformshift{0.878818in}{2.097552in}%
\pgfsys@useobject{currentmarker}{}%
\end{pgfscope}%
\begin{pgfscope}%
\pgfsys@transformshift{1.645696in}{2.119426in}%
\pgfsys@useobject{currentmarker}{}%
\end{pgfscope}%
\begin{pgfscope}%
\pgfsys@transformshift{0.819856in}{2.141300in}%
\pgfsys@useobject{currentmarker}{}%
\end{pgfscope}%
\begin{pgfscope}%
\pgfsys@transformshift{1.863502in}{2.163174in}%
\pgfsys@useobject{currentmarker}{}%
\end{pgfscope}%
\begin{pgfscope}%
\pgfsys@transformshift{2.090236in}{2.185047in}%
\pgfsys@useobject{currentmarker}{}%
\end{pgfscope}%
\begin{pgfscope}%
\pgfsys@transformshift{1.452256in}{2.250669in}%
\pgfsys@useobject{currentmarker}{}%
\end{pgfscope}%
\begin{pgfscope}%
\pgfsys@transformshift{1.251934in}{2.272542in}%
\pgfsys@useobject{currentmarker}{}%
\end{pgfscope}%
\begin{pgfscope}%
\pgfsys@transformshift{1.372276in}{2.294416in}%
\pgfsys@useobject{currentmarker}{}%
\end{pgfscope}%
\begin{pgfscope}%
\pgfsys@transformshift{1.428820in}{2.316290in}%
\pgfsys@useobject{currentmarker}{}%
\end{pgfscope}%
\begin{pgfscope}%
\pgfsys@transformshift{1.424728in}{2.338164in}%
\pgfsys@useobject{currentmarker}{}%
\end{pgfscope}%
\begin{pgfscope}%
\pgfsys@transformshift{1.792264in}{2.360037in}%
\pgfsys@useobject{currentmarker}{}%
\end{pgfscope}%
\begin{pgfscope}%
\pgfsys@transformshift{2.199604in}{2.403785in}%
\pgfsys@useobject{currentmarker}{}%
\end{pgfscope}%
\begin{pgfscope}%
\pgfsys@transformshift{0.756058in}{2.425659in}%
\pgfsys@useobject{currentmarker}{}%
\end{pgfscope}%
\begin{pgfscope}%
\pgfsys@transformshift{0.854638in}{2.447532in}%
\pgfsys@useobject{currentmarker}{}%
\end{pgfscope}%
\begin{pgfscope}%
\pgfsys@transformshift{2.088562in}{2.469406in}%
\pgfsys@useobject{currentmarker}{}%
\end{pgfscope}%
\begin{pgfscope}%
\pgfsys@transformshift{1.117270in}{2.513154in}%
\pgfsys@useobject{currentmarker}{}%
\end{pgfscope}%
\begin{pgfscope}%
\pgfsys@transformshift{0.586240in}{2.535027in}%
\pgfsys@useobject{currentmarker}{}%
\end{pgfscope}%
\begin{pgfscope}%
\pgfsys@transformshift{1.604776in}{2.578775in}%
\pgfsys@useobject{currentmarker}{}%
\end{pgfscope}%
\end{pgfscope}%
\begin{pgfscope}%
\pgfsetrectcap%
\pgfsetmiterjoin%
\pgfsetlinewidth{0.501875pt}%
\definecolor{currentstroke}{rgb}{0.317647,0.317647,0.317647}%
\pgfsetstrokecolor{currentstroke}%
\pgfsetdash{}{0pt}%
\pgfpathmoveto{\pgfqpoint{0.438556in}{0.383578in}}%
\pgfpathlineto{\pgfqpoint{0.438556in}{2.693578in}}%
\pgfusepath{stroke}%
\end{pgfscope}%
\begin{pgfscope}%
\pgfsetrectcap%
\pgfsetmiterjoin%
\pgfsetlinewidth{0.501875pt}%
\definecolor{currentstroke}{rgb}{0.317647,0.317647,0.317647}%
\pgfsetstrokecolor{currentstroke}%
\pgfsetdash{}{0pt}%
\pgfpathmoveto{\pgfqpoint{0.438556in}{0.383578in}}%
\pgfpathlineto{\pgfqpoint{2.763556in}{0.383578in}}%
\pgfusepath{stroke}%
\end{pgfscope}%
\begin{pgfscope}%
\pgfsetbuttcap%
\pgfsetroundjoin%
\pgfsetlinewidth{0.803000pt}%
\definecolor{currentstroke}{rgb}{0.333333,0.333333,0.333333}%
\pgfsetstrokecolor{currentstroke}%
\pgfsetdash{}{0pt}%
\pgfpathmoveto{\pgfqpoint{2.541028in}{2.597541in}}%
\pgfpathcurveto{\pgfqpoint{2.548395in}{2.597541in}}{\pgfqpoint{2.555461in}{2.600467in}}{\pgfqpoint{2.560670in}{2.605676in}}%
\pgfpathcurveto{\pgfqpoint{2.565879in}{2.610886in}}{\pgfqpoint{2.568806in}{2.617952in}}{\pgfqpoint{2.568806in}{2.625318in}}%
\pgfpathcurveto{\pgfqpoint{2.568806in}{2.632685in}}{\pgfqpoint{2.565879in}{2.639751in}}{\pgfqpoint{2.560670in}{2.644960in}}%
\pgfpathcurveto{\pgfqpoint{2.555461in}{2.650169in}}{\pgfqpoint{2.548395in}{2.653096in}}{\pgfqpoint{2.541028in}{2.653096in}}%
\pgfpathcurveto{\pgfqpoint{2.533661in}{2.653096in}}{\pgfqpoint{2.526595in}{2.650169in}}{\pgfqpoint{2.521386in}{2.644960in}}%
\pgfpathcurveto{\pgfqpoint{2.516177in}{2.639751in}}{\pgfqpoint{2.513250in}{2.632685in}}{\pgfqpoint{2.513250in}{2.625318in}}%
\pgfpathcurveto{\pgfqpoint{2.513250in}{2.617952in}}{\pgfqpoint{2.516177in}{2.610886in}}{\pgfqpoint{2.521386in}{2.605676in}}%
\pgfpathcurveto{\pgfqpoint{2.526595in}{2.600467in}}{\pgfqpoint{2.533661in}{2.597541in}}{\pgfqpoint{2.541028in}{2.597541in}}%
\pgfpathclose%
\pgfusepath{stroke}%
\end{pgfscope}%
\begin{pgfscope}%
\definecolor{textcolor}{rgb}{0.000000,0.000000,0.000000}%
\pgfsetstrokecolor{textcolor}%
\pgfsetfillcolor{textcolor}%
\pgftext[x=2.624328in,y=2.601023in,left,base]{\color{textcolor}\rmfamily\fontsize{6.664000}{7.996800}\selectfont \(\displaystyle p_{1}\)}%
\end{pgfscope}%
\begin{pgfscope}%
\pgfsetbuttcap%
\pgfsetroundjoin%
\pgfsetlinewidth{0.803000pt}%
\definecolor{currentstroke}{rgb}{0.686275,0.352941,0.313725}%
\pgfsetstrokecolor{currentstroke}%
\pgfsetdash{}{0pt}%
\pgfpathmoveto{\pgfqpoint{2.541028in}{2.483770in}}%
\pgfpathcurveto{\pgfqpoint{2.546553in}{2.483770in}}{\pgfqpoint{2.551853in}{2.485965in}}{\pgfqpoint{2.555760in}{2.489872in}}%
\pgfpathcurveto{\pgfqpoint{2.559666in}{2.493778in}}{\pgfqpoint{2.561862in}{2.499078in}}{\pgfqpoint{2.561862in}{2.504603in}}%
\pgfpathcurveto{\pgfqpoint{2.561862in}{2.510128in}}{\pgfqpoint{2.559666in}{2.515428in}}{\pgfqpoint{2.555760in}{2.519334in}}%
\pgfpathcurveto{\pgfqpoint{2.551853in}{2.523241in}}{\pgfqpoint{2.546553in}{2.525436in}}{\pgfqpoint{2.541028in}{2.525436in}}%
\pgfpathcurveto{\pgfqpoint{2.535503in}{2.525436in}}{\pgfqpoint{2.530204in}{2.523241in}}{\pgfqpoint{2.526297in}{2.519334in}}%
\pgfpathcurveto{\pgfqpoint{2.522390in}{2.515428in}}{\pgfqpoint{2.520195in}{2.510128in}}{\pgfqpoint{2.520195in}{2.504603in}}%
\pgfpathcurveto{\pgfqpoint{2.520195in}{2.499078in}}{\pgfqpoint{2.522390in}{2.493778in}}{\pgfqpoint{2.526297in}{2.489872in}}%
\pgfpathcurveto{\pgfqpoint{2.530204in}{2.485965in}}{\pgfqpoint{2.535503in}{2.483770in}}{\pgfqpoint{2.541028in}{2.483770in}}%
\pgfpathclose%
\pgfusepath{stroke}%
\end{pgfscope}%
\begin{pgfscope}%
\definecolor{textcolor}{rgb}{0.000000,0.000000,0.000000}%
\pgfsetstrokecolor{textcolor}%
\pgfsetfillcolor{textcolor}%
\pgftext[x=2.624328in,y=2.480307in,left,base]{\color{textcolor}\rmfamily\fontsize{6.664000}{7.996800}\selectfont \(\displaystyle p_{2}\)}%
\end{pgfscope}%
\begin{pgfscope}%
\pgfsetbuttcap%
\pgfsetroundjoin%
\pgfsetlinewidth{0.803000pt}%
\definecolor{currentstroke}{rgb}{0.000000,0.356863,0.509804}%
\pgfsetstrokecolor{currentstroke}%
\pgfsetdash{}{0pt}%
\pgfpathmoveto{\pgfqpoint{2.541028in}{2.369999in}}%
\pgfpathcurveto{\pgfqpoint{2.544712in}{2.369999in}}{\pgfqpoint{2.548245in}{2.371462in}}{\pgfqpoint{2.550849in}{2.374067in}}%
\pgfpathcurveto{\pgfqpoint{2.553454in}{2.376671in}}{\pgfqpoint{2.554917in}{2.380204in}}{\pgfqpoint{2.554917in}{2.383888in}}%
\pgfpathcurveto{\pgfqpoint{2.554917in}{2.387571in}}{\pgfqpoint{2.553454in}{2.391104in}}{\pgfqpoint{2.550849in}{2.393709in}}%
\pgfpathcurveto{\pgfqpoint{2.548245in}{2.396313in}}{\pgfqpoint{2.544712in}{2.397777in}}{\pgfqpoint{2.541028in}{2.397777in}}%
\pgfpathcurveto{\pgfqpoint{2.537345in}{2.397777in}}{\pgfqpoint{2.533812in}{2.396313in}}{\pgfqpoint{2.531207in}{2.393709in}}%
\pgfpathcurveto{\pgfqpoint{2.528603in}{2.391104in}}{\pgfqpoint{2.527139in}{2.387571in}}{\pgfqpoint{2.527139in}{2.383888in}}%
\pgfpathcurveto{\pgfqpoint{2.527139in}{2.380204in}}{\pgfqpoint{2.528603in}{2.376671in}}{\pgfqpoint{2.531207in}{2.374067in}}%
\pgfpathcurveto{\pgfqpoint{2.533812in}{2.371462in}}{\pgfqpoint{2.537345in}{2.369999in}}{\pgfqpoint{2.541028in}{2.369999in}}%
\pgfpathclose%
\pgfusepath{stroke}%
\end{pgfscope}%
\begin{pgfscope}%
\definecolor{textcolor}{rgb}{0.000000,0.000000,0.000000}%
\pgfsetstrokecolor{textcolor}%
\pgfsetfillcolor{textcolor}%
\pgftext[x=2.624328in,y=2.359592in,left,base]{\color{textcolor}\rmfamily\fontsize{6.664000}{7.996800}\selectfont \(\displaystyle p_{3}\)}%
\end{pgfscope}%
\begin{pgfscope}%
\pgfsetbuttcap%
\pgfsetroundjoin%
\definecolor{currentfill}{rgb}{0.490196,0.588235,0.431373}%
\pgfsetfillcolor{currentfill}%
\pgfsetlinewidth{0.803000pt}%
\definecolor{currentstroke}{rgb}{0.490196,0.588235,0.431373}%
\pgfsetstrokecolor{currentstroke}%
\pgfsetdash{}{0pt}%
\pgfsys@defobject{currentmarker}{\pgfqpoint{-0.006944in}{-0.006944in}}{\pgfqpoint{0.006944in}{0.006944in}}{%
\pgfpathmoveto{\pgfqpoint{0.000000in}{-0.006944in}}%
\pgfpathcurveto{\pgfqpoint{0.001842in}{-0.006944in}}{\pgfqpoint{0.003608in}{-0.006213in}}{\pgfqpoint{0.004910in}{-0.004910in}}%
\pgfpathcurveto{\pgfqpoint{0.006213in}{-0.003608in}}{\pgfqpoint{0.006944in}{-0.001842in}}{\pgfqpoint{0.006944in}{0.000000in}}%
\pgfpathcurveto{\pgfqpoint{0.006944in}{0.001842in}}{\pgfqpoint{0.006213in}{0.003608in}}{\pgfqpoint{0.004910in}{0.004910in}}%
\pgfpathcurveto{\pgfqpoint{0.003608in}{0.006213in}}{\pgfqpoint{0.001842in}{0.006944in}}{\pgfqpoint{0.000000in}{0.006944in}}%
\pgfpathcurveto{\pgfqpoint{-0.001842in}{0.006944in}}{\pgfqpoint{-0.003608in}{0.006213in}}{\pgfqpoint{-0.004910in}{0.004910in}}%
\pgfpathcurveto{\pgfqpoint{-0.006213in}{0.003608in}}{\pgfqpoint{-0.006944in}{0.001842in}}{\pgfqpoint{-0.006944in}{0.000000in}}%
\pgfpathcurveto{\pgfqpoint{-0.006944in}{-0.001842in}}{\pgfqpoint{-0.006213in}{-0.003608in}}{\pgfqpoint{-0.004910in}{-0.004910in}}%
\pgfpathcurveto{\pgfqpoint{-0.003608in}{-0.006213in}}{\pgfqpoint{-0.001842in}{-0.006944in}}{\pgfqpoint{0.000000in}{-0.006944in}}%
\pgfpathclose%
\pgfusepath{stroke,fill}%
}%
\begin{pgfscope}%
\pgfsys@transformshift{2.541028in}{2.263173in}%
\pgfsys@useobject{currentmarker}{}%
\end{pgfscope}%
\end{pgfscope}%
\begin{pgfscope}%
\definecolor{textcolor}{rgb}{0.000000,0.000000,0.000000}%
\pgfsetstrokecolor{textcolor}%
\pgfsetfillcolor{textcolor}%
\pgftext[x=2.624328in,y=2.238877in,left,base]{\color{textcolor}\rmfamily\fontsize{6.664000}{7.996800}\selectfont \(\displaystyle p_{4}\)}%
\end{pgfscope}%
\end{pgfpicture}%
\makeatother%
\endgroup%

		\label{superspiketaskpicture}
	\end{subfigure}
	\caption{(\subref{superspiketaskpicturesector}) A representative sector of the input data from the XOR-related task contains $2 + 4 + 4 + 6 = 16$ spikes from $2\times3$ different units and spike times. The first pattern ($S_1$) overlaps with all others and can be interpreted as a bias. The second and third pattern are disjoint apart from the  in the bias units and with the reference pattern, corresponding to $1 \veebar 0 = 0 \veebar 1 = 1$. The reference spike train is always on, equaling $1 \veebar 1 = 0$. (\subref{superspiketaskpicture}) The total number of input spikes per pattern is given by $20, 40, 40, 60$.
	\label{superspiketaskoverview}}
\end{figure} 

The fixed set of random times is picked from a $\SI{40}{\micro \s}$ time window, while the duration of a single measurement period is set to around $\SI{250}{\micro \s}$. In a slight modification to the derivation of SuperSpike, the error metric is changed from a single target spike $\hat{S}_i$ to a target time window $[\hat{t}_0, \hat{t}_1]$. Depending on the input pattern $p_j$, the error of the output unit $i$ corresponding to the target class $i$ is given by
\begin{equation}
e_i = \begin{cases}
\alpha \ast \left(e_\text{outside, i} + e_\text{inside, i}\right),& \quad \quad \text{if} \quad \text{class}(p_j) == i \\
- \alpha \ast S_i,& \quad \quad \text{else} 
\end{cases}
\end{equation}
In analogy to the von Rossum distance the error outside and inside the window can be written as 
\begin{align}
e_\text{outside}(t) &= - S_i(t) \cdot \left(H(\hat{t}_0 - t) + H(t - \hat{t}_1)\right) \\
e_\text{inside}(t) &= 
\begin{cases}
0 ,&\quad \quad \text{if} \quad \exists \; t^{(s)}_i \in [\hat{t}_0, \hat{t}_1], \\
\epsilon \ast \hat{t} ,& \quad \quad \text{else}.
\end{cases}
\end{align}
Despite the technical notation, the error follows the same principles as the von Rossum distance: If the target neuron spikes within the designated time span, zero error is returned. In case there occurs no spike inside the window, a target spike at time $\hat{t}$ will be added to the error instead. Any spikes outside the window sum with a negative sign. The error for the other output neuron is simple. Since it should remain silent, the error correlates to the negative spiking activity.

As in the previous experiment the accuracy is determined by the fraction of correctly identified inputs $n_\text{true}$ over the total number of inputs $n_\text{inputs}$
\begin{equation}
\text{Accuracy} = \frac{n_\text{true}}{n_\text{points}}.
\end{equation}
For debug reasons the accuracy was also monitored per pattern.

For the training, input batches of size eight are randomly drawn from a uniform distribution, resulting in an overall balanced training data set, but unbalanced batches. The performance of a test data set is evaluated on the fly by measuring the accuracy of all patterns after each parameter update. Depending on the initial conditions and the error propagation method (backpropagation or feedback alignment) the task converged after 500 to 2000 iterations.

\subsection{Experiment Implementation}
The necessity of a hidden layer has already been motivated in \cref{deeplearning} for non-linear tasks such as circles or the exclusive-or. In the following a network with a single hidden layer containing 30 units, 96 input sources and two output neurons which serve as target classes is configured on the \gls{hx}. 

The first tape-out of the new chip revision came with some flaws requiring several software workarounds. To minimize the implementation effort only the forward pass is performed on-chip and restricted to the upper half of the chip, i.e. the first two quadrants as shown in REF TO HX design. The computation required by the backward pass as well as the experiment control is outsourced to the controlling host. However, a key element of SuperSpike, the access to the membrane potential, has been implemented as an on-chip readout routine.

\subsubsection*{\gls{cadc} Readout}
With a few \gls{ppu} instructions for the general purpose part and the vector unit, the readout of the \gls{cadc} is performed. For the purpose of the experiment, the membrane traces are required to be sampled at high rate and to cover as much data as possible. The available memory on the \gls{ppu} limits the readout to 100 samples of 128 different neurons with a temporal resolution of roughly \SI{2.5}{\micro \s}. 

Before usable traces can be recorded with the \gls{cadc}, its characteristic needs to be calibrated. The \gls{cadc} has tunable an analog slope and offset parameter per quadrant. In addition to the per-quadrant settings, each channel has an individual offset parameter, that can be adjusted as well.

In the \cref{cadccalibration} the final state of the characteristics is plotted two quadrants from the chip half in use. The thereby implicated conversion from \gls{dac} lsb to \si{\V} will be used for most of the data shown throughout this chapter.
\begin{figure}
	\begin{subfigure}{0.5\textwidth}
		\caption{}
		\includegraphics[width=\textwidth]{figures/temporary/cadc_pre_calib_hx70.pdf}
		\label{precadccalib}
	\end{subfigure}
	\begin{subfigure}{0.5\textwidth}
		\caption{}
		\includegraphics[width=\textwidth]{figures/temporary/cadc_post_calib_hx70.pdf}
		\label{postcadccalib}
	\end{subfigure}
	\caption[Pre and post calibration state of the \gls{cadc}]{Pre and post calibration state of the \gls{cadc}. (\subref{precadccalib}) The raw cadc data of an controlled voltage ranging from 0 to \SI{1.2}{\V}. (\subref{postcadccalib}) the cadc parameters are manually adjusted, such that they cover a useful dynamic range. The offset per channel can then be easily computed and corrected. The manual method was preferred over an automated fit-routine, since the ramp and slope parameters showed a sensitive cross-dependency in certain areas.}
	\label{cadccalibration}
\end{figure}

For the use-case of the \gls{cadc} readout during the experiment, it is vital that both the measured trace and the recorded spike times by the digital back end are synchronized. The result of a respective offset measurement yields an offset of $\delta t \approx \SI{2.2}{\milli \s}$.

\subsubsection*{Calibration of \gls{lif} Neuron}
The new chip revision did not yet get rid of the imperfection the come with analog neuromorphic hardware. Despite the self-correcting behavior of learning algorithms, the chip requires at least some calibrating before the experiment can be executed.

At the time of the experiment, the development of chip-specific software for the new prototype was in an early stage. Among others, there was a lack of a calibration database and more generally, a lack of available calibrated hardware resources. This was largely due to an at that time slow and static calibration routine, that required the allocation of human and hardware resources for several hours to tune a setup.

The efficient parallelized readout of the \gls{cadc} presented itself to be a viable basis to put a quick alternative calibration routine into action. The main objective of the new routine is to provide a fast calibration that requires little interaction to bring a setup into a usable state with respect to the specific experiment requirements. 

Similar to the circles task, the binary search algorithm is chosen to find the proper \gls{dac} values of the analog parameters. Moreover, only a subset of the available parameters is tuned, namely the potentials of the \gls{lif} model \gls{v_leak}, \gls{v_reset} and \gls{thres}. By design, the potentials of the synaptic input $V_\text{syn, inh}$ and $V_\text{syn, exc}$ have a cross-dependency to the resting potential and thus need to be considered as well. The temporal constants \gls{tau_m} and \gls{tau_syn} are certainly important parameters in the \gls{lif} model, but the induced error by their misalignment is, at least for the trained task, not heavily hampering the performance.


\begin{figure}
	\begin{subfigure}{0.32\textwidth}
		\caption{}
		\centering
		%% Creator: Matplotlib, PGF backend
%%
%% To include the figure in your LaTeX document, write
%%   \input{<filename>.pgf}
%%
%% Make sure the required packages are loaded in your preamble
%%   \usepackage{pgf}
%%
%% Figures using additional raster images can only be included by \input if
%% they are in the same directory as the main LaTeX file. For loading figures
%% from other directories you can use the `import` package
%%   \usepackage{import}
%% and then include the figures with
%%   \import{<path to file>}{<filename>.pgf}
%%
%% Matplotlib used the following preamble
%%   \usepackage{amsmath} \usepackage{pifont} \usepackage{xcolor} \definecolor{green}{HTML}{467821} \definecolor{red}{HTML}{CF4457} \usepackage[detect-all]{siunitx}
%%   \usepackage{fontspec}
%%
\begingroup%
\makeatletter%
\begin{pgfpicture}%
\pgfpathrectangle{\pgfpointorigin}{\pgfqpoint{2.105974in}{2.023578in}}%
\pgfusepath{use as bounding box, clip}%
\begin{pgfscope}%
\pgfsetbuttcap%
\pgfsetmiterjoin%
\pgfsetlinewidth{0.000000pt}%
\definecolor{currentstroke}{rgb}{0.000000,0.000000,0.000000}%
\pgfsetstrokecolor{currentstroke}%
\pgfsetstrokeopacity{0.000000}%
\pgfsetdash{}{0pt}%
\pgfpathmoveto{\pgfqpoint{0.000000in}{0.000000in}}%
\pgfpathlineto{\pgfqpoint{2.105974in}{0.000000in}}%
\pgfpathlineto{\pgfqpoint{2.105974in}{2.023578in}}%
\pgfpathlineto{\pgfqpoint{0.000000in}{2.023578in}}%
\pgfpathclose%
\pgfusepath{}%
\end{pgfscope}%
\begin{pgfscope}%
\pgfsetbuttcap%
\pgfsetmiterjoin%
\pgfsetlinewidth{0.000000pt}%
\definecolor{currentstroke}{rgb}{0.000000,0.000000,0.000000}%
\pgfsetstrokecolor{currentstroke}%
\pgfsetstrokeopacity{0.000000}%
\pgfsetdash{}{0pt}%
\pgfpathmoveto{\pgfqpoint{0.384118in}{0.383578in}}%
\pgfpathlineto{\pgfqpoint{1.934118in}{0.383578in}}%
\pgfpathlineto{\pgfqpoint{1.934118in}{1.923578in}}%
\pgfpathlineto{\pgfqpoint{0.384118in}{1.923578in}}%
\pgfpathclose%
\pgfusepath{}%
\end{pgfscope}%
\begin{pgfscope}%
\pgfsetbuttcap%
\pgfsetroundjoin%
\definecolor{currentfill}{rgb}{0.317647,0.317647,0.317647}%
\pgfsetfillcolor{currentfill}%
\pgfsetlinewidth{0.501875pt}%
\definecolor{currentstroke}{rgb}{0.317647,0.317647,0.317647}%
\pgfsetstrokecolor{currentstroke}%
\pgfsetdash{}{0pt}%
\pgfsys@defobject{currentmarker}{\pgfqpoint{0.000000in}{-0.020833in}}{\pgfqpoint{0.000000in}{0.000000in}}{%
\pgfpathmoveto{\pgfqpoint{0.000000in}{0.000000in}}%
\pgfpathlineto{\pgfqpoint{0.000000in}{-0.020833in}}%
\pgfusepath{stroke,fill}%
}%
\begin{pgfscope}%
\pgfsys@transformshift{0.605547in}{0.383578in}%
\pgfsys@useobject{currentmarker}{}%
\end{pgfscope}%
\end{pgfscope}%
\begin{pgfscope}%
\definecolor{textcolor}{rgb}{0.317647,0.317647,0.317647}%
\pgfsetstrokecolor{textcolor}%
\pgfsetfillcolor{textcolor}%
\pgftext[x=0.605547in,y=0.334967in,,top]{\color{textcolor}\rmfamily\fontsize{6.664000}{7.996800}\selectfont \(\displaystyle 0.4\)}%
\end{pgfscope}%
\begin{pgfscope}%
\pgfsetbuttcap%
\pgfsetroundjoin%
\definecolor{currentfill}{rgb}{0.317647,0.317647,0.317647}%
\pgfsetfillcolor{currentfill}%
\pgfsetlinewidth{0.501875pt}%
\definecolor{currentstroke}{rgb}{0.317647,0.317647,0.317647}%
\pgfsetstrokecolor{currentstroke}%
\pgfsetdash{}{0pt}%
\pgfsys@defobject{currentmarker}{\pgfqpoint{0.000000in}{-0.020833in}}{\pgfqpoint{0.000000in}{0.000000in}}{%
\pgfpathmoveto{\pgfqpoint{0.000000in}{0.000000in}}%
\pgfpathlineto{\pgfqpoint{0.000000in}{-0.020833in}}%
\pgfusepath{stroke,fill}%
}%
\begin{pgfscope}%
\pgfsys@transformshift{1.048404in}{0.383578in}%
\pgfsys@useobject{currentmarker}{}%
\end{pgfscope}%
\end{pgfscope}%
\begin{pgfscope}%
\definecolor{textcolor}{rgb}{0.317647,0.317647,0.317647}%
\pgfsetstrokecolor{textcolor}%
\pgfsetfillcolor{textcolor}%
\pgftext[x=1.048404in,y=0.334967in,,top]{\color{textcolor}\rmfamily\fontsize{6.664000}{7.996800}\selectfont \(\displaystyle 0.5\)}%
\end{pgfscope}%
\begin{pgfscope}%
\pgfsetbuttcap%
\pgfsetroundjoin%
\definecolor{currentfill}{rgb}{0.317647,0.317647,0.317647}%
\pgfsetfillcolor{currentfill}%
\pgfsetlinewidth{0.501875pt}%
\definecolor{currentstroke}{rgb}{0.317647,0.317647,0.317647}%
\pgfsetstrokecolor{currentstroke}%
\pgfsetdash{}{0pt}%
\pgfsys@defobject{currentmarker}{\pgfqpoint{0.000000in}{-0.020833in}}{\pgfqpoint{0.000000in}{0.000000in}}{%
\pgfpathmoveto{\pgfqpoint{0.000000in}{0.000000in}}%
\pgfpathlineto{\pgfqpoint{0.000000in}{-0.020833in}}%
\pgfusepath{stroke,fill}%
}%
\begin{pgfscope}%
\pgfsys@transformshift{1.491261in}{0.383578in}%
\pgfsys@useobject{currentmarker}{}%
\end{pgfscope}%
\end{pgfscope}%
\begin{pgfscope}%
\definecolor{textcolor}{rgb}{0.317647,0.317647,0.317647}%
\pgfsetstrokecolor{textcolor}%
\pgfsetfillcolor{textcolor}%
\pgftext[x=1.491261in,y=0.334967in,,top]{\color{textcolor}\rmfamily\fontsize{6.664000}{7.996800}\selectfont \(\displaystyle 0.6\)}%
\end{pgfscope}%
\begin{pgfscope}%
\pgfsetbuttcap%
\pgfsetroundjoin%
\definecolor{currentfill}{rgb}{0.317647,0.317647,0.317647}%
\pgfsetfillcolor{currentfill}%
\pgfsetlinewidth{0.501875pt}%
\definecolor{currentstroke}{rgb}{0.317647,0.317647,0.317647}%
\pgfsetstrokecolor{currentstroke}%
\pgfsetdash{}{0pt}%
\pgfsys@defobject{currentmarker}{\pgfqpoint{0.000000in}{-0.020833in}}{\pgfqpoint{0.000000in}{0.000000in}}{%
\pgfpathmoveto{\pgfqpoint{0.000000in}{0.000000in}}%
\pgfpathlineto{\pgfqpoint{0.000000in}{-0.020833in}}%
\pgfusepath{stroke,fill}%
}%
\begin{pgfscope}%
\pgfsys@transformshift{1.934118in}{0.383578in}%
\pgfsys@useobject{currentmarker}{}%
\end{pgfscope}%
\end{pgfscope}%
\begin{pgfscope}%
\definecolor{textcolor}{rgb}{0.317647,0.317647,0.317647}%
\pgfsetstrokecolor{textcolor}%
\pgfsetfillcolor{textcolor}%
\pgftext[x=1.934118in,y=0.334967in,,top]{\color{textcolor}\rmfamily\fontsize{6.664000}{7.996800}\selectfont \(\displaystyle 0.7\)}%
\end{pgfscope}%
\begin{pgfscope}%
\definecolor{textcolor}{rgb}{0.317647,0.317647,0.317647}%
\pgfsetstrokecolor{textcolor}%
\pgfsetfillcolor{textcolor}%
\pgftext[x=1.159118in,y=0.197222in,,top]{\color{textcolor}\rmfamily\fontsize{6.664000}{7.996800}\selectfont \(\displaystyle V_\mathrm{leak} \quad (\si{\V})\)}%
\end{pgfscope}%
\begin{pgfscope}%
\pgfsetbuttcap%
\pgfsetroundjoin%
\definecolor{currentfill}{rgb}{0.317647,0.317647,0.317647}%
\pgfsetfillcolor{currentfill}%
\pgfsetlinewidth{0.501875pt}%
\definecolor{currentstroke}{rgb}{0.317647,0.317647,0.317647}%
\pgfsetstrokecolor{currentstroke}%
\pgfsetdash{}{0pt}%
\pgfsys@defobject{currentmarker}{\pgfqpoint{-0.020833in}{0.000000in}}{\pgfqpoint{0.000000in}{0.000000in}}{%
\pgfpathmoveto{\pgfqpoint{0.000000in}{0.000000in}}%
\pgfpathlineto{\pgfqpoint{-0.020833in}{0.000000in}}%
\pgfusepath{stroke,fill}%
}%
\begin{pgfscope}%
\pgfsys@transformshift{0.384118in}{0.383578in}%
\pgfsys@useobject{currentmarker}{}%
\end{pgfscope}%
\end{pgfscope}%
\begin{pgfscope}%
\definecolor{textcolor}{rgb}{0.317647,0.317647,0.317647}%
\pgfsetstrokecolor{textcolor}%
\pgfsetfillcolor{textcolor}%
\pgftext[x=0.294033in,y=0.351461in,left,base]{\color{textcolor}\rmfamily\fontsize{6.664000}{7.996800}\selectfont \(\displaystyle 0\)}%
\end{pgfscope}%
\begin{pgfscope}%
\pgfsetbuttcap%
\pgfsetroundjoin%
\definecolor{currentfill}{rgb}{0.317647,0.317647,0.317647}%
\pgfsetfillcolor{currentfill}%
\pgfsetlinewidth{0.501875pt}%
\definecolor{currentstroke}{rgb}{0.317647,0.317647,0.317647}%
\pgfsetstrokecolor{currentstroke}%
\pgfsetdash{}{0pt}%
\pgfsys@defobject{currentmarker}{\pgfqpoint{-0.020833in}{0.000000in}}{\pgfqpoint{0.000000in}{0.000000in}}{%
\pgfpathmoveto{\pgfqpoint{0.000000in}{0.000000in}}%
\pgfpathlineto{\pgfqpoint{-0.020833in}{0.000000in}}%
\pgfusepath{stroke,fill}%
}%
\begin{pgfscope}%
\pgfsys@transformshift{0.384118in}{0.660102in}%
\pgfsys@useobject{currentmarker}{}%
\end{pgfscope}%
\end{pgfscope}%
\begin{pgfscope}%
\definecolor{textcolor}{rgb}{0.317647,0.317647,0.317647}%
\pgfsetstrokecolor{textcolor}%
\pgfsetfillcolor{textcolor}%
\pgftext[x=0.238670in,y=0.627985in,left,base]{\color{textcolor}\rmfamily\fontsize{6.664000}{7.996800}\selectfont \(\displaystyle 10\)}%
\end{pgfscope}%
\begin{pgfscope}%
\pgfsetbuttcap%
\pgfsetroundjoin%
\definecolor{currentfill}{rgb}{0.317647,0.317647,0.317647}%
\pgfsetfillcolor{currentfill}%
\pgfsetlinewidth{0.501875pt}%
\definecolor{currentstroke}{rgb}{0.317647,0.317647,0.317647}%
\pgfsetstrokecolor{currentstroke}%
\pgfsetdash{}{0pt}%
\pgfsys@defobject{currentmarker}{\pgfqpoint{-0.020833in}{0.000000in}}{\pgfqpoint{0.000000in}{0.000000in}}{%
\pgfpathmoveto{\pgfqpoint{0.000000in}{0.000000in}}%
\pgfpathlineto{\pgfqpoint{-0.020833in}{0.000000in}}%
\pgfusepath{stroke,fill}%
}%
\begin{pgfscope}%
\pgfsys@transformshift{0.384118in}{0.936625in}%
\pgfsys@useobject{currentmarker}{}%
\end{pgfscope}%
\end{pgfscope}%
\begin{pgfscope}%
\definecolor{textcolor}{rgb}{0.317647,0.317647,0.317647}%
\pgfsetstrokecolor{textcolor}%
\pgfsetfillcolor{textcolor}%
\pgftext[x=0.238670in,y=0.904508in,left,base]{\color{textcolor}\rmfamily\fontsize{6.664000}{7.996800}\selectfont \(\displaystyle 20\)}%
\end{pgfscope}%
\begin{pgfscope}%
\pgfsetbuttcap%
\pgfsetroundjoin%
\definecolor{currentfill}{rgb}{0.317647,0.317647,0.317647}%
\pgfsetfillcolor{currentfill}%
\pgfsetlinewidth{0.501875pt}%
\definecolor{currentstroke}{rgb}{0.317647,0.317647,0.317647}%
\pgfsetstrokecolor{currentstroke}%
\pgfsetdash{}{0pt}%
\pgfsys@defobject{currentmarker}{\pgfqpoint{-0.020833in}{0.000000in}}{\pgfqpoint{0.000000in}{0.000000in}}{%
\pgfpathmoveto{\pgfqpoint{0.000000in}{0.000000in}}%
\pgfpathlineto{\pgfqpoint{-0.020833in}{0.000000in}}%
\pgfusepath{stroke,fill}%
}%
\begin{pgfscope}%
\pgfsys@transformshift{0.384118in}{1.213149in}%
\pgfsys@useobject{currentmarker}{}%
\end{pgfscope}%
\end{pgfscope}%
\begin{pgfscope}%
\definecolor{textcolor}{rgb}{0.317647,0.317647,0.317647}%
\pgfsetstrokecolor{textcolor}%
\pgfsetfillcolor{textcolor}%
\pgftext[x=0.238670in,y=1.181032in,left,base]{\color{textcolor}\rmfamily\fontsize{6.664000}{7.996800}\selectfont \(\displaystyle 30\)}%
\end{pgfscope}%
\begin{pgfscope}%
\pgfsetbuttcap%
\pgfsetroundjoin%
\definecolor{currentfill}{rgb}{0.317647,0.317647,0.317647}%
\pgfsetfillcolor{currentfill}%
\pgfsetlinewidth{0.501875pt}%
\definecolor{currentstroke}{rgb}{0.317647,0.317647,0.317647}%
\pgfsetstrokecolor{currentstroke}%
\pgfsetdash{}{0pt}%
\pgfsys@defobject{currentmarker}{\pgfqpoint{-0.020833in}{0.000000in}}{\pgfqpoint{0.000000in}{0.000000in}}{%
\pgfpathmoveto{\pgfqpoint{0.000000in}{0.000000in}}%
\pgfpathlineto{\pgfqpoint{-0.020833in}{0.000000in}}%
\pgfusepath{stroke,fill}%
}%
\begin{pgfscope}%
\pgfsys@transformshift{0.384118in}{1.489672in}%
\pgfsys@useobject{currentmarker}{}%
\end{pgfscope}%
\end{pgfscope}%
\begin{pgfscope}%
\definecolor{textcolor}{rgb}{0.317647,0.317647,0.317647}%
\pgfsetstrokecolor{textcolor}%
\pgfsetfillcolor{textcolor}%
\pgftext[x=0.238670in,y=1.457556in,left,base]{\color{textcolor}\rmfamily\fontsize{6.664000}{7.996800}\selectfont \(\displaystyle 40\)}%
\end{pgfscope}%
\begin{pgfscope}%
\pgfsetbuttcap%
\pgfsetroundjoin%
\definecolor{currentfill}{rgb}{0.317647,0.317647,0.317647}%
\pgfsetfillcolor{currentfill}%
\pgfsetlinewidth{0.501875pt}%
\definecolor{currentstroke}{rgb}{0.317647,0.317647,0.317647}%
\pgfsetstrokecolor{currentstroke}%
\pgfsetdash{}{0pt}%
\pgfsys@defobject{currentmarker}{\pgfqpoint{-0.020833in}{0.000000in}}{\pgfqpoint{0.000000in}{0.000000in}}{%
\pgfpathmoveto{\pgfqpoint{0.000000in}{0.000000in}}%
\pgfpathlineto{\pgfqpoint{-0.020833in}{0.000000in}}%
\pgfusepath{stroke,fill}%
}%
\begin{pgfscope}%
\pgfsys@transformshift{0.384118in}{1.766196in}%
\pgfsys@useobject{currentmarker}{}%
\end{pgfscope}%
\end{pgfscope}%
\begin{pgfscope}%
\definecolor{textcolor}{rgb}{0.317647,0.317647,0.317647}%
\pgfsetstrokecolor{textcolor}%
\pgfsetfillcolor{textcolor}%
\pgftext[x=0.238670in,y=1.734079in,left,base]{\color{textcolor}\rmfamily\fontsize{6.664000}{7.996800}\selectfont \(\displaystyle 50\)}%
\end{pgfscope}%
\begin{pgfscope}%
\definecolor{textcolor}{rgb}{0.317647,0.317647,0.317647}%
\pgfsetstrokecolor{textcolor}%
\pgfsetfillcolor{textcolor}%
\pgftext[x=0.183115in,y=1.153578in,,bottom,rotate=90.000000]{\color{textcolor}\rmfamily\fontsize{6.664000}{7.996800}\selectfont density}%
\end{pgfscope}%
\begin{pgfscope}%
\pgfpathrectangle{\pgfqpoint{0.384118in}{0.383578in}}{\pgfqpoint{1.550000in}{1.540000in}}%
\pgfusepath{clip}%
\pgfsetbuttcap%
\pgfsetmiterjoin%
\definecolor{currentfill}{rgb}{0.333333,0.333333,0.333333}%
\pgfsetfillcolor{currentfill}%
\pgfsetfillopacity{0.400000}%
\pgfsetlinewidth{0.000000pt}%
\definecolor{currentstroke}{rgb}{0.000000,0.000000,0.000000}%
\pgfsetstrokecolor{currentstroke}%
\pgfsetstrokeopacity{0.400000}%
\pgfsetdash{}{0pt}%
\pgfpathmoveto{\pgfqpoint{0.489761in}{0.383578in}}%
\pgfpathlineto{\pgfqpoint{0.676907in}{0.383578in}}%
\pgfpathlineto{\pgfqpoint{0.676907in}{0.460260in}}%
\pgfpathlineto{\pgfqpoint{0.489761in}{0.460260in}}%
\pgfpathclose%
\pgfusepath{fill}%
\end{pgfscope}%
\begin{pgfscope}%
\pgfpathrectangle{\pgfqpoint{0.384118in}{0.383578in}}{\pgfqpoint{1.550000in}{1.540000in}}%
\pgfusepath{clip}%
\pgfsetbuttcap%
\pgfsetmiterjoin%
\definecolor{currentfill}{rgb}{0.333333,0.333333,0.333333}%
\pgfsetfillcolor{currentfill}%
\pgfsetfillopacity{0.400000}%
\pgfsetlinewidth{0.000000pt}%
\definecolor{currentstroke}{rgb}{0.000000,0.000000,0.000000}%
\pgfsetstrokecolor{currentstroke}%
\pgfsetstrokeopacity{0.400000}%
\pgfsetdash{}{0pt}%
\pgfpathmoveto{\pgfqpoint{0.676907in}{0.383578in}}%
\pgfpathlineto{\pgfqpoint{0.864054in}{0.383578in}}%
\pgfpathlineto{\pgfqpoint{0.864054in}{0.490933in}}%
\pgfpathlineto{\pgfqpoint{0.676907in}{0.490933in}}%
\pgfpathclose%
\pgfusepath{fill}%
\end{pgfscope}%
\begin{pgfscope}%
\pgfpathrectangle{\pgfqpoint{0.384118in}{0.383578in}}{\pgfqpoint{1.550000in}{1.540000in}}%
\pgfusepath{clip}%
\pgfsetbuttcap%
\pgfsetmiterjoin%
\definecolor{currentfill}{rgb}{0.333333,0.333333,0.333333}%
\pgfsetfillcolor{currentfill}%
\pgfsetfillopacity{0.400000}%
\pgfsetlinewidth{0.000000pt}%
\definecolor{currentstroke}{rgb}{0.000000,0.000000,0.000000}%
\pgfsetstrokecolor{currentstroke}%
\pgfsetstrokeopacity{0.400000}%
\pgfsetdash{}{0pt}%
\pgfpathmoveto{\pgfqpoint{0.864054in}{0.383578in}}%
\pgfpathlineto{\pgfqpoint{1.051200in}{0.383578in}}%
\pgfpathlineto{\pgfqpoint{1.051200in}{0.490933in}}%
\pgfpathlineto{\pgfqpoint{0.864054in}{0.490933in}}%
\pgfpathclose%
\pgfusepath{fill}%
\end{pgfscope}%
\begin{pgfscope}%
\pgfpathrectangle{\pgfqpoint{0.384118in}{0.383578in}}{\pgfqpoint{1.550000in}{1.540000in}}%
\pgfusepath{clip}%
\pgfsetbuttcap%
\pgfsetmiterjoin%
\definecolor{currentfill}{rgb}{0.333333,0.333333,0.333333}%
\pgfsetfillcolor{currentfill}%
\pgfsetfillopacity{0.400000}%
\pgfsetlinewidth{0.000000pt}%
\definecolor{currentstroke}{rgb}{0.000000,0.000000,0.000000}%
\pgfsetstrokecolor{currentstroke}%
\pgfsetstrokeopacity{0.400000}%
\pgfsetdash{}{0pt}%
\pgfpathmoveto{\pgfqpoint{1.051200in}{0.383578in}}%
\pgfpathlineto{\pgfqpoint{1.238347in}{0.383578in}}%
\pgfpathlineto{\pgfqpoint{1.238347in}{0.536943in}}%
\pgfpathlineto{\pgfqpoint{1.051200in}{0.536943in}}%
\pgfpathclose%
\pgfusepath{fill}%
\end{pgfscope}%
\begin{pgfscope}%
\pgfpathrectangle{\pgfqpoint{0.384118in}{0.383578in}}{\pgfqpoint{1.550000in}{1.540000in}}%
\pgfusepath{clip}%
\pgfsetbuttcap%
\pgfsetmiterjoin%
\definecolor{currentfill}{rgb}{0.333333,0.333333,0.333333}%
\pgfsetfillcolor{currentfill}%
\pgfsetfillopacity{0.400000}%
\pgfsetlinewidth{0.000000pt}%
\definecolor{currentstroke}{rgb}{0.000000,0.000000,0.000000}%
\pgfsetstrokecolor{currentstroke}%
\pgfsetstrokeopacity{0.400000}%
\pgfsetdash{}{0pt}%
\pgfpathmoveto{\pgfqpoint{1.238347in}{0.383578in}}%
\pgfpathlineto{\pgfqpoint{1.425493in}{0.383578in}}%
\pgfpathlineto{\pgfqpoint{1.425493in}{0.475597in}}%
\pgfpathlineto{\pgfqpoint{1.238347in}{0.475597in}}%
\pgfpathclose%
\pgfusepath{fill}%
\end{pgfscope}%
\begin{pgfscope}%
\pgfpathrectangle{\pgfqpoint{0.384118in}{0.383578in}}{\pgfqpoint{1.550000in}{1.540000in}}%
\pgfusepath{clip}%
\pgfsetbuttcap%
\pgfsetmiterjoin%
\definecolor{currentfill}{rgb}{0.333333,0.333333,0.333333}%
\pgfsetfillcolor{currentfill}%
\pgfsetfillopacity{0.400000}%
\pgfsetlinewidth{0.000000pt}%
\definecolor{currentstroke}{rgb}{0.000000,0.000000,0.000000}%
\pgfsetstrokecolor{currentstroke}%
\pgfsetstrokeopacity{0.400000}%
\pgfsetdash{}{0pt}%
\pgfpathmoveto{\pgfqpoint{1.425493in}{0.383578in}}%
\pgfpathlineto{\pgfqpoint{1.612640in}{0.383578in}}%
\pgfpathlineto{\pgfqpoint{1.612640in}{0.465373in}}%
\pgfpathlineto{\pgfqpoint{1.425493in}{0.465373in}}%
\pgfpathclose%
\pgfusepath{fill}%
\end{pgfscope}%
\begin{pgfscope}%
\pgfpathrectangle{\pgfqpoint{0.384118in}{0.383578in}}{\pgfqpoint{1.550000in}{1.540000in}}%
\pgfusepath{clip}%
\pgfsetbuttcap%
\pgfsetmiterjoin%
\definecolor{currentfill}{rgb}{0.333333,0.333333,0.333333}%
\pgfsetfillcolor{currentfill}%
\pgfsetfillopacity{0.400000}%
\pgfsetlinewidth{0.000000pt}%
\definecolor{currentstroke}{rgb}{0.000000,0.000000,0.000000}%
\pgfsetstrokecolor{currentstroke}%
\pgfsetstrokeopacity{0.400000}%
\pgfsetdash{}{0pt}%
\pgfpathmoveto{\pgfqpoint{1.612640in}{0.383578in}}%
\pgfpathlineto{\pgfqpoint{1.799787in}{0.383578in}}%
\pgfpathlineto{\pgfqpoint{1.799787in}{0.419363in}}%
\pgfpathlineto{\pgfqpoint{1.612640in}{0.419363in}}%
\pgfpathclose%
\pgfusepath{fill}%
\end{pgfscope}%
\begin{pgfscope}%
\pgfpathrectangle{\pgfqpoint{0.384118in}{0.383578in}}{\pgfqpoint{1.550000in}{1.540000in}}%
\pgfusepath{clip}%
\pgfsetbuttcap%
\pgfsetmiterjoin%
\definecolor{currentfill}{rgb}{0.686275,0.352941,0.313725}%
\pgfsetfillcolor{currentfill}%
\pgfsetfillopacity{0.400000}%
\pgfsetlinewidth{0.000000pt}%
\definecolor{currentstroke}{rgb}{0.000000,0.000000,0.000000}%
\pgfsetstrokecolor{currentstroke}%
\pgfsetstrokeopacity{0.400000}%
\pgfsetdash{}{0pt}%
\pgfpathmoveto{\pgfqpoint{0.951037in}{0.383578in}}%
\pgfpathlineto{\pgfqpoint{0.972564in}{0.383578in}}%
\pgfpathlineto{\pgfqpoint{0.972564in}{0.472467in}}%
\pgfpathlineto{\pgfqpoint{0.951037in}{0.472467in}}%
\pgfpathclose%
\pgfusepath{fill}%
\end{pgfscope}%
\begin{pgfscope}%
\pgfpathrectangle{\pgfqpoint{0.384118in}{0.383578in}}{\pgfqpoint{1.550000in}{1.540000in}}%
\pgfusepath{clip}%
\pgfsetbuttcap%
\pgfsetmiterjoin%
\definecolor{currentfill}{rgb}{0.686275,0.352941,0.313725}%
\pgfsetfillcolor{currentfill}%
\pgfsetfillopacity{0.400000}%
\pgfsetlinewidth{0.000000pt}%
\definecolor{currentstroke}{rgb}{0.000000,0.000000,0.000000}%
\pgfsetstrokecolor{currentstroke}%
\pgfsetstrokeopacity{0.400000}%
\pgfsetdash{}{0pt}%
\pgfpathmoveto{\pgfqpoint{0.972564in}{0.383578in}}%
\pgfpathlineto{\pgfqpoint{0.994090in}{0.383578in}}%
\pgfpathlineto{\pgfqpoint{0.994090in}{0.516911in}}%
\pgfpathlineto{\pgfqpoint{0.972564in}{0.516911in}}%
\pgfpathclose%
\pgfusepath{fill}%
\end{pgfscope}%
\begin{pgfscope}%
\pgfpathrectangle{\pgfqpoint{0.384118in}{0.383578in}}{\pgfqpoint{1.550000in}{1.540000in}}%
\pgfusepath{clip}%
\pgfsetbuttcap%
\pgfsetmiterjoin%
\definecolor{currentfill}{rgb}{0.686275,0.352941,0.313725}%
\pgfsetfillcolor{currentfill}%
\pgfsetfillopacity{0.400000}%
\pgfsetlinewidth{0.000000pt}%
\definecolor{currentstroke}{rgb}{0.000000,0.000000,0.000000}%
\pgfsetstrokecolor{currentstroke}%
\pgfsetstrokeopacity{0.400000}%
\pgfsetdash{}{0pt}%
\pgfpathmoveto{\pgfqpoint{0.994090in}{0.383578in}}%
\pgfpathlineto{\pgfqpoint{1.015616in}{0.383578in}}%
\pgfpathlineto{\pgfqpoint{1.015616in}{0.961356in}}%
\pgfpathlineto{\pgfqpoint{0.994090in}{0.961356in}}%
\pgfpathclose%
\pgfusepath{fill}%
\end{pgfscope}%
\begin{pgfscope}%
\pgfpathrectangle{\pgfqpoint{0.384118in}{0.383578in}}{\pgfqpoint{1.550000in}{1.540000in}}%
\pgfusepath{clip}%
\pgfsetbuttcap%
\pgfsetmiterjoin%
\definecolor{currentfill}{rgb}{0.686275,0.352941,0.313725}%
\pgfsetfillcolor{currentfill}%
\pgfsetfillopacity{0.400000}%
\pgfsetlinewidth{0.000000pt}%
\definecolor{currentstroke}{rgb}{0.000000,0.000000,0.000000}%
\pgfsetstrokecolor{currentstroke}%
\pgfsetstrokeopacity{0.400000}%
\pgfsetdash{}{0pt}%
\pgfpathmoveto{\pgfqpoint{1.015616in}{0.383578in}}%
\pgfpathlineto{\pgfqpoint{1.037142in}{0.383578in}}%
\pgfpathlineto{\pgfqpoint{1.037142in}{1.228022in}}%
\pgfpathlineto{\pgfqpoint{1.015616in}{1.228022in}}%
\pgfpathclose%
\pgfusepath{fill}%
\end{pgfscope}%
\begin{pgfscope}%
\pgfpathrectangle{\pgfqpoint{0.384118in}{0.383578in}}{\pgfqpoint{1.550000in}{1.540000in}}%
\pgfusepath{clip}%
\pgfsetbuttcap%
\pgfsetmiterjoin%
\definecolor{currentfill}{rgb}{0.686275,0.352941,0.313725}%
\pgfsetfillcolor{currentfill}%
\pgfsetfillopacity{0.400000}%
\pgfsetlinewidth{0.000000pt}%
\definecolor{currentstroke}{rgb}{0.000000,0.000000,0.000000}%
\pgfsetstrokecolor{currentstroke}%
\pgfsetstrokeopacity{0.400000}%
\pgfsetdash{}{0pt}%
\pgfpathmoveto{\pgfqpoint{1.037142in}{0.383578in}}%
\pgfpathlineto{\pgfqpoint{1.058669in}{0.383578in}}%
\pgfpathlineto{\pgfqpoint{1.058669in}{1.361356in}}%
\pgfpathlineto{\pgfqpoint{1.037142in}{1.361356in}}%
\pgfpathclose%
\pgfusepath{fill}%
\end{pgfscope}%
\begin{pgfscope}%
\pgfpathrectangle{\pgfqpoint{0.384118in}{0.383578in}}{\pgfqpoint{1.550000in}{1.540000in}}%
\pgfusepath{clip}%
\pgfsetbuttcap%
\pgfsetmiterjoin%
\definecolor{currentfill}{rgb}{0.686275,0.352941,0.313725}%
\pgfsetfillcolor{currentfill}%
\pgfsetfillopacity{0.400000}%
\pgfsetlinewidth{0.000000pt}%
\definecolor{currentstroke}{rgb}{0.000000,0.000000,0.000000}%
\pgfsetstrokecolor{currentstroke}%
\pgfsetstrokeopacity{0.400000}%
\pgfsetdash{}{0pt}%
\pgfpathmoveto{\pgfqpoint{1.058669in}{0.383578in}}%
\pgfpathlineto{\pgfqpoint{1.080195in}{0.383578in}}%
\pgfpathlineto{\pgfqpoint{1.080195in}{1.850245in}}%
\pgfpathlineto{\pgfqpoint{1.058669in}{1.850245in}}%
\pgfpathclose%
\pgfusepath{fill}%
\end{pgfscope}%
\begin{pgfscope}%
\pgfpathrectangle{\pgfqpoint{0.384118in}{0.383578in}}{\pgfqpoint{1.550000in}{1.540000in}}%
\pgfusepath{clip}%
\pgfsetbuttcap%
\pgfsetmiterjoin%
\definecolor{currentfill}{rgb}{0.686275,0.352941,0.313725}%
\pgfsetfillcolor{currentfill}%
\pgfsetfillopacity{0.400000}%
\pgfsetlinewidth{0.000000pt}%
\definecolor{currentstroke}{rgb}{0.000000,0.000000,0.000000}%
\pgfsetstrokecolor{currentstroke}%
\pgfsetstrokeopacity{0.400000}%
\pgfsetdash{}{0pt}%
\pgfpathmoveto{\pgfqpoint{1.080195in}{0.383578in}}%
\pgfpathlineto{\pgfqpoint{1.101721in}{0.383578in}}%
\pgfpathlineto{\pgfqpoint{1.101721in}{1.539134in}}%
\pgfpathlineto{\pgfqpoint{1.080195in}{1.539134in}}%
\pgfpathclose%
\pgfusepath{fill}%
\end{pgfscope}%
\begin{pgfscope}%
\pgfpathrectangle{\pgfqpoint{0.384118in}{0.383578in}}{\pgfqpoint{1.550000in}{1.540000in}}%
\pgfusepath{clip}%
\pgfsetbuttcap%
\pgfsetmiterjoin%
\definecolor{currentfill}{rgb}{0.686275,0.352941,0.313725}%
\pgfsetfillcolor{currentfill}%
\pgfsetfillopacity{0.400000}%
\pgfsetlinewidth{0.000000pt}%
\definecolor{currentstroke}{rgb}{0.000000,0.000000,0.000000}%
\pgfsetstrokecolor{currentstroke}%
\pgfsetstrokeopacity{0.400000}%
\pgfsetdash{}{0pt}%
\pgfpathmoveto{\pgfqpoint{1.101721in}{0.383578in}}%
\pgfpathlineto{\pgfqpoint{1.123247in}{0.383578in}}%
\pgfpathlineto{\pgfqpoint{1.123247in}{0.561356in}}%
\pgfpathlineto{\pgfqpoint{1.101721in}{0.561356in}}%
\pgfpathclose%
\pgfusepath{fill}%
\end{pgfscope}%
\begin{pgfscope}%
\pgfpathrectangle{\pgfqpoint{0.384118in}{0.383578in}}{\pgfqpoint{1.550000in}{1.540000in}}%
\pgfusepath{clip}%
\pgfsetbuttcap%
\pgfsetmiterjoin%
\definecolor{currentfill}{rgb}{0.686275,0.352941,0.313725}%
\pgfsetfillcolor{currentfill}%
\pgfsetfillopacity{0.400000}%
\pgfsetlinewidth{0.000000pt}%
\definecolor{currentstroke}{rgb}{0.000000,0.000000,0.000000}%
\pgfsetstrokecolor{currentstroke}%
\pgfsetstrokeopacity{0.400000}%
\pgfsetdash{}{0pt}%
\pgfpathmoveto{\pgfqpoint{1.123247in}{0.383578in}}%
\pgfpathlineto{\pgfqpoint{1.144774in}{0.383578in}}%
\pgfpathlineto{\pgfqpoint{1.144774in}{0.472467in}}%
\pgfpathlineto{\pgfqpoint{1.123247in}{0.472467in}}%
\pgfpathclose%
\pgfusepath{fill}%
\end{pgfscope}%
\begin{pgfscope}%
\pgfpathrectangle{\pgfqpoint{0.384118in}{0.383578in}}{\pgfqpoint{1.550000in}{1.540000in}}%
\pgfusepath{clip}%
\pgfsetbuttcap%
\pgfsetmiterjoin%
\definecolor{currentfill}{rgb}{0.686275,0.352941,0.313725}%
\pgfsetfillcolor{currentfill}%
\pgfsetfillopacity{0.400000}%
\pgfsetlinewidth{0.000000pt}%
\definecolor{currentstroke}{rgb}{0.000000,0.000000,0.000000}%
\pgfsetstrokecolor{currentstroke}%
\pgfsetstrokeopacity{0.400000}%
\pgfsetdash{}{0pt}%
\pgfpathmoveto{\pgfqpoint{1.144774in}{0.383578in}}%
\pgfpathlineto{\pgfqpoint{1.166300in}{0.383578in}}%
\pgfpathlineto{\pgfqpoint{1.166300in}{0.472467in}}%
\pgfpathlineto{\pgfqpoint{1.144774in}{0.472467in}}%
\pgfpathclose%
\pgfusepath{fill}%
\end{pgfscope}%
\begin{pgfscope}%
\pgfpathrectangle{\pgfqpoint{0.384118in}{0.383578in}}{\pgfqpoint{1.550000in}{1.540000in}}%
\pgfusepath{clip}%
\pgfsetbuttcap%
\pgfsetmiterjoin%
\definecolor{currentfill}{rgb}{0.686275,0.352941,0.313725}%
\pgfsetfillcolor{currentfill}%
\pgfsetfillopacity{0.400000}%
\pgfsetlinewidth{0.000000pt}%
\definecolor{currentstroke}{rgb}{0.000000,0.000000,0.000000}%
\pgfsetstrokecolor{currentstroke}%
\pgfsetstrokeopacity{0.400000}%
\pgfsetdash{}{0pt}%
\pgfpathmoveto{\pgfqpoint{1.166300in}{0.383578in}}%
\pgfpathlineto{\pgfqpoint{1.187826in}{0.383578in}}%
\pgfpathlineto{\pgfqpoint{1.187826in}{0.383578in}}%
\pgfpathlineto{\pgfqpoint{1.166300in}{0.383578in}}%
\pgfpathclose%
\pgfusepath{fill}%
\end{pgfscope}%
\begin{pgfscope}%
\pgfpathrectangle{\pgfqpoint{0.384118in}{0.383578in}}{\pgfqpoint{1.550000in}{1.540000in}}%
\pgfusepath{clip}%
\pgfsetbuttcap%
\pgfsetmiterjoin%
\definecolor{currentfill}{rgb}{0.686275,0.352941,0.313725}%
\pgfsetfillcolor{currentfill}%
\pgfsetfillopacity{0.400000}%
\pgfsetlinewidth{0.000000pt}%
\definecolor{currentstroke}{rgb}{0.000000,0.000000,0.000000}%
\pgfsetstrokecolor{currentstroke}%
\pgfsetstrokeopacity{0.400000}%
\pgfsetdash{}{0pt}%
\pgfpathmoveto{\pgfqpoint{1.187826in}{0.383578in}}%
\pgfpathlineto{\pgfqpoint{1.209352in}{0.383578in}}%
\pgfpathlineto{\pgfqpoint{1.209352in}{0.472467in}}%
\pgfpathlineto{\pgfqpoint{1.187826in}{0.472467in}}%
\pgfpathclose%
\pgfusepath{fill}%
\end{pgfscope}%
\begin{pgfscope}%
\pgfpathrectangle{\pgfqpoint{0.384118in}{0.383578in}}{\pgfqpoint{1.550000in}{1.540000in}}%
\pgfusepath{clip}%
\pgfsetrectcap%
\pgfsetroundjoin%
\pgfsetlinewidth{0.803000pt}%
\definecolor{currentstroke}{rgb}{0.333333,0.333333,0.333333}%
\pgfsetstrokecolor{currentstroke}%
\pgfsetdash{}{0pt}%
\pgfpathmoveto{\pgfqpoint{0.374118in}{0.392707in}}%
\pgfpathlineto{\pgfqpoint{0.380934in}{0.393541in}}%
\pgfpathlineto{\pgfqpoint{0.401303in}{0.396580in}}%
\pgfpathlineto{\pgfqpoint{0.421672in}{0.400192in}}%
\pgfpathlineto{\pgfqpoint{0.442041in}{0.404377in}}%
\pgfpathlineto{\pgfqpoint{0.462410in}{0.409105in}}%
\pgfpathlineto{\pgfqpoint{0.482779in}{0.414315in}}%
\pgfpathlineto{\pgfqpoint{0.503148in}{0.419918in}}%
\pgfpathlineto{\pgfqpoint{0.523517in}{0.425803in}}%
\pgfpathlineto{\pgfqpoint{0.543886in}{0.431850in}}%
\pgfpathlineto{\pgfqpoint{0.564256in}{0.437936in}}%
\pgfpathlineto{\pgfqpoint{0.584625in}{0.443951in}}%
\pgfpathlineto{\pgfqpoint{0.604994in}{0.449802in}}%
\pgfpathlineto{\pgfqpoint{0.625363in}{0.455424in}}%
\pgfpathlineto{\pgfqpoint{0.645732in}{0.460777in}}%
\pgfpathlineto{\pgfqpoint{0.666101in}{0.465841in}}%
\pgfpathlineto{\pgfqpoint{0.686470in}{0.470612in}}%
\pgfpathlineto{\pgfqpoint{0.706839in}{0.475088in}}%
\pgfpathlineto{\pgfqpoint{0.727208in}{0.479264in}}%
\pgfpathlineto{\pgfqpoint{0.747577in}{0.483128in}}%
\pgfpathlineto{\pgfqpoint{0.767946in}{0.486659in}}%
\pgfpathlineto{\pgfqpoint{0.788315in}{0.489833in}}%
\pgfpathlineto{\pgfqpoint{0.808684in}{0.492631in}}%
\pgfpathlineto{\pgfqpoint{0.829053in}{0.495052in}}%
\pgfpathlineto{\pgfqpoint{0.849422in}{0.497114in}}%
\pgfpathlineto{\pgfqpoint{0.869791in}{0.498865in}}%
\pgfpathlineto{\pgfqpoint{0.890160in}{0.500377in}}%
\pgfpathlineto{\pgfqpoint{0.910529in}{0.501738in}}%
\pgfpathlineto{\pgfqpoint{0.930899in}{0.503040in}}%
\pgfpathlineto{\pgfqpoint{0.951268in}{0.504367in}}%
\pgfpathlineto{\pgfqpoint{0.971637in}{0.505774in}}%
\pgfpathlineto{\pgfqpoint{0.992006in}{0.507283in}}%
\pgfpathlineto{\pgfqpoint{1.012375in}{0.508875in}}%
\pgfpathlineto{\pgfqpoint{1.032744in}{0.510487in}}%
\pgfpathlineto{\pgfqpoint{1.053113in}{0.512024in}}%
\pgfpathlineto{\pgfqpoint{1.073482in}{0.513367in}}%
\pgfpathlineto{\pgfqpoint{1.093851in}{0.514392in}}%
\pgfpathlineto{\pgfqpoint{1.114220in}{0.514987in}}%
\pgfpathlineto{\pgfqpoint{1.134589in}{0.515064in}}%
\pgfpathlineto{\pgfqpoint{1.154958in}{0.514576in}}%
\pgfpathlineto{\pgfqpoint{1.175327in}{0.513522in}}%
\pgfpathlineto{\pgfqpoint{1.195696in}{0.511948in}}%
\pgfpathlineto{\pgfqpoint{1.216065in}{0.509935in}}%
\pgfpathlineto{\pgfqpoint{1.236434in}{0.507590in}}%
\pgfpathlineto{\pgfqpoint{1.256803in}{0.505026in}}%
\pgfpathlineto{\pgfqpoint{1.277172in}{0.502344in}}%
\pgfpathlineto{\pgfqpoint{1.297542in}{0.499616in}}%
\pgfpathlineto{\pgfqpoint{1.317911in}{0.496874in}}%
\pgfpathlineto{\pgfqpoint{1.338280in}{0.494107in}}%
\pgfpathlineto{\pgfqpoint{1.358649in}{0.491264in}}%
\pgfpathlineto{\pgfqpoint{1.379018in}{0.488265in}}%
\pgfpathlineto{\pgfqpoint{1.399387in}{0.485019in}}%
\pgfpathlineto{\pgfqpoint{1.419756in}{0.481440in}}%
\pgfpathlineto{\pgfqpoint{1.440125in}{0.477466in}}%
\pgfpathlineto{\pgfqpoint{1.460494in}{0.473069in}}%
\pgfpathlineto{\pgfqpoint{1.480863in}{0.468266in}}%
\pgfpathlineto{\pgfqpoint{1.501232in}{0.463116in}}%
\pgfpathlineto{\pgfqpoint{1.521601in}{0.457714in}}%
\pgfpathlineto{\pgfqpoint{1.541970in}{0.452180in}}%
\pgfpathlineto{\pgfqpoint{1.562339in}{0.446640in}}%
\pgfpathlineto{\pgfqpoint{1.582708in}{0.441216in}}%
\pgfpathlineto{\pgfqpoint{1.603077in}{0.436012in}}%
\pgfpathlineto{\pgfqpoint{1.623446in}{0.431102in}}%
\pgfpathlineto{\pgfqpoint{1.643815in}{0.426533in}}%
\pgfpathlineto{\pgfqpoint{1.664185in}{0.422316in}}%
\pgfpathlineto{\pgfqpoint{1.684554in}{0.418441in}}%
\pgfpathlineto{\pgfqpoint{1.704923in}{0.414877in}}%
\pgfpathlineto{\pgfqpoint{1.725292in}{0.411586in}}%
\pgfpathlineto{\pgfqpoint{1.745661in}{0.408527in}}%
\pgfpathlineto{\pgfqpoint{1.766030in}{0.405665in}}%
\pgfpathlineto{\pgfqpoint{1.786399in}{0.402975in}}%
\pgfpathlineto{\pgfqpoint{1.806768in}{0.400445in}}%
\pgfpathlineto{\pgfqpoint{1.827137in}{0.398070in}}%
\pgfpathlineto{\pgfqpoint{1.847506in}{0.395860in}}%
\pgfpathlineto{\pgfqpoint{1.867875in}{0.393826in}}%
\pgfpathlineto{\pgfqpoint{1.888244in}{0.391983in}}%
\pgfpathlineto{\pgfqpoint{1.908613in}{0.390344in}}%
\pgfpathlineto{\pgfqpoint{1.928982in}{0.388916in}}%
\pgfpathlineto{\pgfqpoint{1.944118in}{0.388014in}}%
\pgfusepath{stroke}%
\end{pgfscope}%
\begin{pgfscope}%
\pgfpathrectangle{\pgfqpoint{0.384118in}{0.383578in}}{\pgfqpoint{1.550000in}{1.540000in}}%
\pgfusepath{clip}%
\pgfsetrectcap%
\pgfsetroundjoin%
\pgfsetlinewidth{0.803000pt}%
\definecolor{currentstroke}{rgb}{0.686275,0.352941,0.313725}%
\pgfsetstrokecolor{currentstroke}%
\pgfsetdash{}{0pt}%
\pgfpathmoveto{\pgfqpoint{0.905378in}{0.383869in}}%
\pgfpathlineto{\pgfqpoint{0.908909in}{0.384146in}}%
\pgfpathlineto{\pgfqpoint{0.912441in}{0.384630in}}%
\pgfpathlineto{\pgfqpoint{0.915973in}{0.385428in}}%
\pgfpathlineto{\pgfqpoint{0.919504in}{0.386670in}}%
\pgfpathlineto{\pgfqpoint{0.923036in}{0.388492in}}%
\pgfpathlineto{\pgfqpoint{0.926568in}{0.391015in}}%
\pgfpathlineto{\pgfqpoint{0.930099in}{0.394308in}}%
\pgfpathlineto{\pgfqpoint{0.933631in}{0.398368in}}%
\pgfpathlineto{\pgfqpoint{0.937163in}{0.403104in}}%
\pgfpathlineto{\pgfqpoint{0.940694in}{0.408359in}}%
\pgfpathlineto{\pgfqpoint{0.944226in}{0.413953in}}%
\pgfpathlineto{\pgfqpoint{0.947758in}{0.419760in}}%
\pgfpathlineto{\pgfqpoint{0.951289in}{0.425793in}}%
\pgfpathlineto{\pgfqpoint{0.954821in}{0.432272in}}%
\pgfpathlineto{\pgfqpoint{0.958353in}{0.439670in}}%
\pgfpathlineto{\pgfqpoint{0.961884in}{0.448718in}}%
\pgfpathlineto{\pgfqpoint{0.965416in}{0.460372in}}%
\pgfpathlineto{\pgfqpoint{0.968948in}{0.475740in}}%
\pgfpathlineto{\pgfqpoint{0.972479in}{0.495990in}}%
\pgfpathlineto{\pgfqpoint{0.976011in}{0.522223in}}%
\pgfpathlineto{\pgfqpoint{0.979543in}{0.555332in}}%
\pgfpathlineto{\pgfqpoint{0.983074in}{0.595854in}}%
\pgfpathlineto{\pgfqpoint{0.986606in}{0.643828in}}%
\pgfpathlineto{\pgfqpoint{0.990137in}{0.698724in}}%
\pgfpathlineto{\pgfqpoint{0.993669in}{0.759453in}}%
\pgfpathlineto{\pgfqpoint{0.997201in}{0.824490in}}%
\pgfpathlineto{\pgfqpoint{1.000732in}{0.892089in}}%
\pgfpathlineto{\pgfqpoint{1.004264in}{0.960524in}}%
\pgfpathlineto{\pgfqpoint{1.007796in}{1.028290in}}%
\pgfpathlineto{\pgfqpoint{1.011327in}{1.094233in}}%
\pgfpathlineto{\pgfqpoint{1.014859in}{1.157580in}}%
\pgfpathlineto{\pgfqpoint{1.018391in}{1.217950in}}%
\pgfpathlineto{\pgfqpoint{1.021922in}{1.275347in}}%
\pgfpathlineto{\pgfqpoint{1.025454in}{1.330151in}}%
\pgfpathlineto{\pgfqpoint{1.028986in}{1.383050in}}%
\pgfpathlineto{\pgfqpoint{1.032517in}{1.434857in}}%
\pgfpathlineto{\pgfqpoint{1.036049in}{1.486181in}}%
\pgfpathlineto{\pgfqpoint{1.039581in}{1.536994in}}%
\pgfpathlineto{\pgfqpoint{1.043112in}{1.586195in}}%
\pgfpathlineto{\pgfqpoint{1.046644in}{1.631344in}}%
\pgfpathlineto{\pgfqpoint{1.050176in}{1.668721in}}%
\pgfpathlineto{\pgfqpoint{1.053707in}{1.693803in}}%
\pgfpathlineto{\pgfqpoint{1.057239in}{1.702141in}}%
\pgfpathlineto{\pgfqpoint{1.060771in}{1.690433in}}%
\pgfpathlineto{\pgfqpoint{1.064302in}{1.657474in}}%
\pgfpathlineto{\pgfqpoint{1.067834in}{1.604642in}}%
\pgfpathlineto{\pgfqpoint{1.071366in}{1.535720in}}%
\pgfpathlineto{\pgfqpoint{1.074897in}{1.456072in}}%
\pgfpathlineto{\pgfqpoint{1.078429in}{1.371463in}}%
\pgfpathlineto{\pgfqpoint{1.081961in}{1.286896in}}%
\pgfpathlineto{\pgfqpoint{1.085492in}{1.205831in}}%
\pgfpathlineto{\pgfqpoint{1.089024in}{1.129931in}}%
\pgfpathlineto{\pgfqpoint{1.092556in}{1.059344in}}%
\pgfpathlineto{\pgfqpoint{1.096087in}{0.993299in}}%
\pgfpathlineto{\pgfqpoint{1.099619in}{0.930793in}}%
\pgfpathlineto{\pgfqpoint{1.103151in}{0.871131in}}%
\pgfpathlineto{\pgfqpoint{1.106682in}{0.814201in}}%
\pgfpathlineto{\pgfqpoint{1.110214in}{0.760461in}}%
\pgfpathlineto{\pgfqpoint{1.113746in}{0.710719in}}%
\pgfpathlineto{\pgfqpoint{1.117277in}{0.665841in}}%
\pgfpathlineto{\pgfqpoint{1.120809in}{0.626479in}}%
\pgfpathlineto{\pgfqpoint{1.124341in}{0.592924in}}%
\pgfpathlineto{\pgfqpoint{1.127872in}{0.565062in}}%
\pgfpathlineto{\pgfqpoint{1.131404in}{0.542422in}}%
\pgfpathlineto{\pgfqpoint{1.134936in}{0.524263in}}%
\pgfpathlineto{\pgfqpoint{1.138467in}{0.509676in}}%
\pgfpathlineto{\pgfqpoint{1.141999in}{0.497681in}}%
\pgfpathlineto{\pgfqpoint{1.145531in}{0.487326in}}%
\pgfpathlineto{\pgfqpoint{1.149062in}{0.477797in}}%
\pgfpathlineto{\pgfqpoint{1.152594in}{0.468518in}}%
\pgfpathlineto{\pgfqpoint{1.156126in}{0.459224in}}%
\pgfpathlineto{\pgfqpoint{1.159657in}{0.449989in}}%
\pgfpathlineto{\pgfqpoint{1.163189in}{0.441175in}}%
\pgfpathlineto{\pgfqpoint{1.166721in}{0.433315in}}%
\pgfpathlineto{\pgfqpoint{1.170252in}{0.426949in}}%
\pgfpathlineto{\pgfqpoint{1.173784in}{0.422469in}}%
\pgfpathlineto{\pgfqpoint{1.177316in}{0.420003in}}%
\pgfpathlineto{\pgfqpoint{1.180847in}{0.419383in}}%
\pgfpathlineto{\pgfqpoint{1.184379in}{0.420182in}}%
\pgfpathlineto{\pgfqpoint{1.187910in}{0.421815in}}%
\pgfpathlineto{\pgfqpoint{1.191442in}{0.423653in}}%
\pgfpathlineto{\pgfqpoint{1.194974in}{0.425126in}}%
\pgfpathlineto{\pgfqpoint{1.198505in}{0.425792in}}%
\pgfpathlineto{\pgfqpoint{1.202037in}{0.425367in}}%
\pgfpathlineto{\pgfqpoint{1.205569in}{0.423730in}}%
\pgfpathlineto{\pgfqpoint{1.209100in}{0.420919in}}%
\pgfpathlineto{\pgfqpoint{1.212632in}{0.417106in}}%
\pgfpathlineto{\pgfqpoint{1.216164in}{0.412576in}}%
\pgfpathlineto{\pgfqpoint{1.219695in}{0.407679in}}%
\pgfpathlineto{\pgfqpoint{1.223227in}{0.402785in}}%
\pgfpathlineto{\pgfqpoint{1.226759in}{0.398225in}}%
\pgfpathlineto{\pgfqpoint{1.230290in}{0.394247in}}%
\pgfpathlineto{\pgfqpoint{1.233822in}{0.390990in}}%
\pgfpathlineto{\pgfqpoint{1.237354in}{0.388483in}}%
\pgfpathlineto{\pgfqpoint{1.240885in}{0.386667in}}%
\pgfpathlineto{\pgfqpoint{1.244417in}{0.385427in}}%
\pgfpathlineto{\pgfqpoint{1.247949in}{0.384630in}}%
\pgfpathlineto{\pgfqpoint{1.251480in}{0.384146in}}%
\pgfpathlineto{\pgfqpoint{1.255012in}{0.383869in}}%
\pgfusepath{stroke}%
\end{pgfscope}%
\begin{pgfscope}%
\pgfsetrectcap%
\pgfsetmiterjoin%
\pgfsetlinewidth{0.501875pt}%
\definecolor{currentstroke}{rgb}{0.317647,0.317647,0.317647}%
\pgfsetstrokecolor{currentstroke}%
\pgfsetdash{}{0pt}%
\pgfpathmoveto{\pgfqpoint{0.384118in}{0.383578in}}%
\pgfpathlineto{\pgfqpoint{0.384118in}{1.923578in}}%
\pgfusepath{stroke}%
\end{pgfscope}%
\begin{pgfscope}%
\pgfsetrectcap%
\pgfsetmiterjoin%
\pgfsetlinewidth{0.501875pt}%
\definecolor{currentstroke}{rgb}{0.317647,0.317647,0.317647}%
\pgfsetstrokecolor{currentstroke}%
\pgfsetdash{}{0pt}%
\pgfpathmoveto{\pgfqpoint{0.384118in}{0.383578in}}%
\pgfpathlineto{\pgfqpoint{1.934118in}{0.383578in}}%
\pgfusepath{stroke}%
\end{pgfscope}%
\begin{pgfscope}%
\pgfsetbuttcap%
\pgfsetmiterjoin%
\definecolor{currentfill}{rgb}{0.333333,0.333333,0.333333}%
\pgfsetfillcolor{currentfill}%
\pgfsetfillopacity{0.400000}%
\pgfsetlinewidth{0.000000pt}%
\definecolor{currentstroke}{rgb}{0.000000,0.000000,0.000000}%
\pgfsetstrokecolor{currentstroke}%
\pgfsetstrokeopacity{0.400000}%
\pgfsetdash{}{0pt}%
\pgfpathmoveto{\pgfqpoint{1.317945in}{1.831022in}}%
\pgfpathlineto{\pgfqpoint{1.391989in}{1.831022in}}%
\pgfpathlineto{\pgfqpoint{1.391989in}{1.895811in}}%
\pgfpathlineto{\pgfqpoint{1.317945in}{1.895811in}}%
\pgfpathclose%
\pgfusepath{fill}%
\end{pgfscope}%
\begin{pgfscope}%
\definecolor{textcolor}{rgb}{0.000000,0.000000,0.000000}%
\pgfsetstrokecolor{textcolor}%
\pgfsetfillcolor{textcolor}%
\pgftext[x=1.438267in,y=1.831022in,left,base]{\color{textcolor}\rmfamily\fontsize{6.664000}{7.996800}\selectfont pre \(\displaystyle \SI{0.5}{\V}\)}%
\end{pgfscope}%
\begin{pgfscope}%
\pgfsetbuttcap%
\pgfsetmiterjoin%
\definecolor{currentfill}{rgb}{0.686275,0.352941,0.313725}%
\pgfsetfillcolor{currentfill}%
\pgfsetfillopacity{0.400000}%
\pgfsetlinewidth{0.000000pt}%
\definecolor{currentstroke}{rgb}{0.000000,0.000000,0.000000}%
\pgfsetstrokecolor{currentstroke}%
\pgfsetstrokeopacity{0.400000}%
\pgfsetdash{}{0pt}%
\pgfpathmoveto{\pgfqpoint{1.317945in}{1.711256in}}%
\pgfpathlineto{\pgfqpoint{1.391989in}{1.711256in}}%
\pgfpathlineto{\pgfqpoint{1.391989in}{1.776045in}}%
\pgfpathlineto{\pgfqpoint{1.317945in}{1.776045in}}%
\pgfpathclose%
\pgfusepath{fill}%
\end{pgfscope}%
\begin{pgfscope}%
\definecolor{textcolor}{rgb}{0.000000,0.000000,0.000000}%
\pgfsetstrokecolor{textcolor}%
\pgfsetfillcolor{textcolor}%
\pgftext[x=1.438267in,y=1.711256in,left,base]{\color{textcolor}\rmfamily\fontsize{6.664000}{7.996800}\selectfont post \(\displaystyle \SI{0.5}{\V}\)}%
\end{pgfscope}%
\end{pgfpicture}%
\makeatother%
\endgroup%

		\label{hxprepostvleak}
	\end{subfigure}
	\begin{subfigure}{0.32\textwidth}
		\caption{}
		\centering
		%% Creator: Matplotlib, PGF backend
%%
%% To include the figure in your LaTeX document, write
%%   \input{<filename>.pgf}
%%
%% Make sure the required packages are loaded in your preamble
%%   \usepackage{pgf}
%%
%% Figures using additional raster images can only be included by \input if
%% they are in the same directory as the main LaTeX file. For loading figures
%% from other directories you can use the `import` package
%%   \usepackage{import}
%% and then include the figures with
%%   \import{<path to file>}{<filename>.pgf}
%%
%% Matplotlib used the following preamble
%%   \usepackage{amsmath} \usepackage{pifont} \usepackage{xcolor} \definecolor{green}{HTML}{467821} \definecolor{red}{HTML}{CF4457} \usepackage[detect-all]{siunitx}
%%   \usepackage{fontspec}
%%
\begingroup%
\makeatletter%
\begin{pgfpicture}%
\pgfpathrectangle{\pgfpointorigin}{\pgfqpoint{2.189019in}{2.023578in}}%
\pgfusepath{use as bounding box, clip}%
\begin{pgfscope}%
\pgfsetbuttcap%
\pgfsetmiterjoin%
\pgfsetlinewidth{0.000000pt}%
\definecolor{currentstroke}{rgb}{0.000000,0.000000,0.000000}%
\pgfsetstrokecolor{currentstroke}%
\pgfsetstrokeopacity{0.000000}%
\pgfsetdash{}{0pt}%
\pgfpathmoveto{\pgfqpoint{0.000000in}{0.000000in}}%
\pgfpathlineto{\pgfqpoint{2.189019in}{0.000000in}}%
\pgfpathlineto{\pgfqpoint{2.189019in}{2.023578in}}%
\pgfpathlineto{\pgfqpoint{0.000000in}{2.023578in}}%
\pgfpathclose%
\pgfusepath{}%
\end{pgfscope}%
\begin{pgfscope}%
\pgfsetbuttcap%
\pgfsetmiterjoin%
\pgfsetlinewidth{0.000000pt}%
\definecolor{currentstroke}{rgb}{0.000000,0.000000,0.000000}%
\pgfsetstrokecolor{currentstroke}%
\pgfsetstrokeopacity{0.000000}%
\pgfsetdash{}{0pt}%
\pgfpathmoveto{\pgfqpoint{0.439481in}{0.383578in}}%
\pgfpathlineto{\pgfqpoint{1.989481in}{0.383578in}}%
\pgfpathlineto{\pgfqpoint{1.989481in}{1.923578in}}%
\pgfpathlineto{\pgfqpoint{0.439481in}{1.923578in}}%
\pgfpathclose%
\pgfusepath{}%
\end{pgfscope}%
\begin{pgfscope}%
\pgfsetbuttcap%
\pgfsetroundjoin%
\definecolor{currentfill}{rgb}{0.317647,0.317647,0.317647}%
\pgfsetfillcolor{currentfill}%
\pgfsetlinewidth{0.501875pt}%
\definecolor{currentstroke}{rgb}{0.317647,0.317647,0.317647}%
\pgfsetstrokecolor{currentstroke}%
\pgfsetdash{}{0pt}%
\pgfsys@defobject{currentmarker}{\pgfqpoint{0.000000in}{-0.020833in}}{\pgfqpoint{0.000000in}{0.000000in}}{%
\pgfpathmoveto{\pgfqpoint{0.000000in}{0.000000in}}%
\pgfpathlineto{\pgfqpoint{0.000000in}{-0.020833in}}%
\pgfusepath{stroke,fill}%
}%
\begin{pgfscope}%
\pgfsys@transformshift{0.697815in}{0.383578in}%
\pgfsys@useobject{currentmarker}{}%
\end{pgfscope}%
\end{pgfscope}%
\begin{pgfscope}%
\definecolor{textcolor}{rgb}{0.317647,0.317647,0.317647}%
\pgfsetstrokecolor{textcolor}%
\pgfsetfillcolor{textcolor}%
\pgftext[x=0.697815in,y=0.334967in,,top]{\color{textcolor}\rmfamily\fontsize{6.664000}{7.996800}\selectfont \(\displaystyle 0.35\)}%
\end{pgfscope}%
\begin{pgfscope}%
\pgfsetbuttcap%
\pgfsetroundjoin%
\definecolor{currentfill}{rgb}{0.317647,0.317647,0.317647}%
\pgfsetfillcolor{currentfill}%
\pgfsetlinewidth{0.501875pt}%
\definecolor{currentstroke}{rgb}{0.317647,0.317647,0.317647}%
\pgfsetstrokecolor{currentstroke}%
\pgfsetdash{}{0pt}%
\pgfsys@defobject{currentmarker}{\pgfqpoint{0.000000in}{-0.020833in}}{\pgfqpoint{0.000000in}{0.000000in}}{%
\pgfpathmoveto{\pgfqpoint{0.000000in}{0.000000in}}%
\pgfpathlineto{\pgfqpoint{0.000000in}{-0.020833in}}%
\pgfusepath{stroke,fill}%
}%
\begin{pgfscope}%
\pgfsys@transformshift{1.128370in}{0.383578in}%
\pgfsys@useobject{currentmarker}{}%
\end{pgfscope}%
\end{pgfscope}%
\begin{pgfscope}%
\definecolor{textcolor}{rgb}{0.317647,0.317647,0.317647}%
\pgfsetstrokecolor{textcolor}%
\pgfsetfillcolor{textcolor}%
\pgftext[x=1.128370in,y=0.334967in,,top]{\color{textcolor}\rmfamily\fontsize{6.664000}{7.996800}\selectfont \(\displaystyle 0.40\)}%
\end{pgfscope}%
\begin{pgfscope}%
\pgfsetbuttcap%
\pgfsetroundjoin%
\definecolor{currentfill}{rgb}{0.317647,0.317647,0.317647}%
\pgfsetfillcolor{currentfill}%
\pgfsetlinewidth{0.501875pt}%
\definecolor{currentstroke}{rgb}{0.317647,0.317647,0.317647}%
\pgfsetstrokecolor{currentstroke}%
\pgfsetdash{}{0pt}%
\pgfsys@defobject{currentmarker}{\pgfqpoint{0.000000in}{-0.020833in}}{\pgfqpoint{0.000000in}{0.000000in}}{%
\pgfpathmoveto{\pgfqpoint{0.000000in}{0.000000in}}%
\pgfpathlineto{\pgfqpoint{0.000000in}{-0.020833in}}%
\pgfusepath{stroke,fill}%
}%
\begin{pgfscope}%
\pgfsys@transformshift{1.558926in}{0.383578in}%
\pgfsys@useobject{currentmarker}{}%
\end{pgfscope}%
\end{pgfscope}%
\begin{pgfscope}%
\definecolor{textcolor}{rgb}{0.317647,0.317647,0.317647}%
\pgfsetstrokecolor{textcolor}%
\pgfsetfillcolor{textcolor}%
\pgftext[x=1.558926in,y=0.334967in,,top]{\color{textcolor}\rmfamily\fontsize{6.664000}{7.996800}\selectfont \(\displaystyle 0.45\)}%
\end{pgfscope}%
\begin{pgfscope}%
\pgfsetbuttcap%
\pgfsetroundjoin%
\definecolor{currentfill}{rgb}{0.317647,0.317647,0.317647}%
\pgfsetfillcolor{currentfill}%
\pgfsetlinewidth{0.501875pt}%
\definecolor{currentstroke}{rgb}{0.317647,0.317647,0.317647}%
\pgfsetstrokecolor{currentstroke}%
\pgfsetdash{}{0pt}%
\pgfsys@defobject{currentmarker}{\pgfqpoint{0.000000in}{-0.020833in}}{\pgfqpoint{0.000000in}{0.000000in}}{%
\pgfpathmoveto{\pgfqpoint{0.000000in}{0.000000in}}%
\pgfpathlineto{\pgfqpoint{0.000000in}{-0.020833in}}%
\pgfusepath{stroke,fill}%
}%
\begin{pgfscope}%
\pgfsys@transformshift{1.989481in}{0.383578in}%
\pgfsys@useobject{currentmarker}{}%
\end{pgfscope}%
\end{pgfscope}%
\begin{pgfscope}%
\definecolor{textcolor}{rgb}{0.317647,0.317647,0.317647}%
\pgfsetstrokecolor{textcolor}%
\pgfsetfillcolor{textcolor}%
\pgftext[x=1.989481in,y=0.334967in,,top]{\color{textcolor}\rmfamily\fontsize{6.664000}{7.996800}\selectfont \(\displaystyle 0.50\)}%
\end{pgfscope}%
\begin{pgfscope}%
\definecolor{textcolor}{rgb}{0.317647,0.317647,0.317647}%
\pgfsetstrokecolor{textcolor}%
\pgfsetfillcolor{textcolor}%
\pgftext[x=1.214481in,y=0.197222in,,top]{\color{textcolor}\rmfamily\fontsize{6.664000}{7.996800}\selectfont \(\displaystyle V_\mathrm{reset} \; (\si{\V})\)}%
\end{pgfscope}%
\begin{pgfscope}%
\pgfsetbuttcap%
\pgfsetroundjoin%
\definecolor{currentfill}{rgb}{0.317647,0.317647,0.317647}%
\pgfsetfillcolor{currentfill}%
\pgfsetlinewidth{0.501875pt}%
\definecolor{currentstroke}{rgb}{0.317647,0.317647,0.317647}%
\pgfsetstrokecolor{currentstroke}%
\pgfsetdash{}{0pt}%
\pgfsys@defobject{currentmarker}{\pgfqpoint{-0.020833in}{0.000000in}}{\pgfqpoint{0.000000in}{0.000000in}}{%
\pgfpathmoveto{\pgfqpoint{0.000000in}{0.000000in}}%
\pgfpathlineto{\pgfqpoint{-0.020833in}{0.000000in}}%
\pgfusepath{stroke,fill}%
}%
\begin{pgfscope}%
\pgfsys@transformshift{0.439481in}{0.383578in}%
\pgfsys@useobject{currentmarker}{}%
\end{pgfscope}%
\end{pgfscope}%
\begin{pgfscope}%
\definecolor{textcolor}{rgb}{0.317647,0.317647,0.317647}%
\pgfsetstrokecolor{textcolor}%
\pgfsetfillcolor{textcolor}%
\pgftext[x=0.349396in,y=0.351461in,left,base]{\color{textcolor}\rmfamily\fontsize{6.664000}{7.996800}\selectfont \(\displaystyle 0\)}%
\end{pgfscope}%
\begin{pgfscope}%
\pgfsetbuttcap%
\pgfsetroundjoin%
\definecolor{currentfill}{rgb}{0.317647,0.317647,0.317647}%
\pgfsetfillcolor{currentfill}%
\pgfsetlinewidth{0.501875pt}%
\definecolor{currentstroke}{rgb}{0.317647,0.317647,0.317647}%
\pgfsetstrokecolor{currentstroke}%
\pgfsetdash{}{0pt}%
\pgfsys@defobject{currentmarker}{\pgfqpoint{-0.020833in}{0.000000in}}{\pgfqpoint{0.000000in}{0.000000in}}{%
\pgfpathmoveto{\pgfqpoint{0.000000in}{0.000000in}}%
\pgfpathlineto{\pgfqpoint{-0.020833in}{0.000000in}}%
\pgfusepath{stroke,fill}%
}%
\begin{pgfscope}%
\pgfsys@transformshift{0.439481in}{0.579120in}%
\pgfsys@useobject{currentmarker}{}%
\end{pgfscope}%
\end{pgfscope}%
\begin{pgfscope}%
\definecolor{textcolor}{rgb}{0.317647,0.317647,0.317647}%
\pgfsetstrokecolor{textcolor}%
\pgfsetfillcolor{textcolor}%
\pgftext[x=0.294033in,y=0.547003in,left,base]{\color{textcolor}\rmfamily\fontsize{6.664000}{7.996800}\selectfont \(\displaystyle 25\)}%
\end{pgfscope}%
\begin{pgfscope}%
\pgfsetbuttcap%
\pgfsetroundjoin%
\definecolor{currentfill}{rgb}{0.317647,0.317647,0.317647}%
\pgfsetfillcolor{currentfill}%
\pgfsetlinewidth{0.501875pt}%
\definecolor{currentstroke}{rgb}{0.317647,0.317647,0.317647}%
\pgfsetstrokecolor{currentstroke}%
\pgfsetdash{}{0pt}%
\pgfsys@defobject{currentmarker}{\pgfqpoint{-0.020833in}{0.000000in}}{\pgfqpoint{0.000000in}{0.000000in}}{%
\pgfpathmoveto{\pgfqpoint{0.000000in}{0.000000in}}%
\pgfpathlineto{\pgfqpoint{-0.020833in}{0.000000in}}%
\pgfusepath{stroke,fill}%
}%
\begin{pgfscope}%
\pgfsys@transformshift{0.439481in}{0.774661in}%
\pgfsys@useobject{currentmarker}{}%
\end{pgfscope}%
\end{pgfscope}%
\begin{pgfscope}%
\definecolor{textcolor}{rgb}{0.317647,0.317647,0.317647}%
\pgfsetstrokecolor{textcolor}%
\pgfsetfillcolor{textcolor}%
\pgftext[x=0.294033in,y=0.742545in,left,base]{\color{textcolor}\rmfamily\fontsize{6.664000}{7.996800}\selectfont \(\displaystyle 50\)}%
\end{pgfscope}%
\begin{pgfscope}%
\pgfsetbuttcap%
\pgfsetroundjoin%
\definecolor{currentfill}{rgb}{0.317647,0.317647,0.317647}%
\pgfsetfillcolor{currentfill}%
\pgfsetlinewidth{0.501875pt}%
\definecolor{currentstroke}{rgb}{0.317647,0.317647,0.317647}%
\pgfsetstrokecolor{currentstroke}%
\pgfsetdash{}{0pt}%
\pgfsys@defobject{currentmarker}{\pgfqpoint{-0.020833in}{0.000000in}}{\pgfqpoint{0.000000in}{0.000000in}}{%
\pgfpathmoveto{\pgfqpoint{0.000000in}{0.000000in}}%
\pgfpathlineto{\pgfqpoint{-0.020833in}{0.000000in}}%
\pgfusepath{stroke,fill}%
}%
\begin{pgfscope}%
\pgfsys@transformshift{0.439481in}{0.970203in}%
\pgfsys@useobject{currentmarker}{}%
\end{pgfscope}%
\end{pgfscope}%
\begin{pgfscope}%
\definecolor{textcolor}{rgb}{0.317647,0.317647,0.317647}%
\pgfsetstrokecolor{textcolor}%
\pgfsetfillcolor{textcolor}%
\pgftext[x=0.294033in,y=0.938086in,left,base]{\color{textcolor}\rmfamily\fontsize{6.664000}{7.996800}\selectfont \(\displaystyle 75\)}%
\end{pgfscope}%
\begin{pgfscope}%
\pgfsetbuttcap%
\pgfsetroundjoin%
\definecolor{currentfill}{rgb}{0.317647,0.317647,0.317647}%
\pgfsetfillcolor{currentfill}%
\pgfsetlinewidth{0.501875pt}%
\definecolor{currentstroke}{rgb}{0.317647,0.317647,0.317647}%
\pgfsetstrokecolor{currentstroke}%
\pgfsetdash{}{0pt}%
\pgfsys@defobject{currentmarker}{\pgfqpoint{-0.020833in}{0.000000in}}{\pgfqpoint{0.000000in}{0.000000in}}{%
\pgfpathmoveto{\pgfqpoint{0.000000in}{0.000000in}}%
\pgfpathlineto{\pgfqpoint{-0.020833in}{0.000000in}}%
\pgfusepath{stroke,fill}%
}%
\begin{pgfscope}%
\pgfsys@transformshift{0.439481in}{1.165745in}%
\pgfsys@useobject{currentmarker}{}%
\end{pgfscope}%
\end{pgfscope}%
\begin{pgfscope}%
\definecolor{textcolor}{rgb}{0.317647,0.317647,0.317647}%
\pgfsetstrokecolor{textcolor}%
\pgfsetfillcolor{textcolor}%
\pgftext[x=0.238670in,y=1.133628in,left,base]{\color{textcolor}\rmfamily\fontsize{6.664000}{7.996800}\selectfont \(\displaystyle 100\)}%
\end{pgfscope}%
\begin{pgfscope}%
\pgfsetbuttcap%
\pgfsetroundjoin%
\definecolor{currentfill}{rgb}{0.317647,0.317647,0.317647}%
\pgfsetfillcolor{currentfill}%
\pgfsetlinewidth{0.501875pt}%
\definecolor{currentstroke}{rgb}{0.317647,0.317647,0.317647}%
\pgfsetstrokecolor{currentstroke}%
\pgfsetdash{}{0pt}%
\pgfsys@defobject{currentmarker}{\pgfqpoint{-0.020833in}{0.000000in}}{\pgfqpoint{0.000000in}{0.000000in}}{%
\pgfpathmoveto{\pgfqpoint{0.000000in}{0.000000in}}%
\pgfpathlineto{\pgfqpoint{-0.020833in}{0.000000in}}%
\pgfusepath{stroke,fill}%
}%
\begin{pgfscope}%
\pgfsys@transformshift{0.439481in}{1.361286in}%
\pgfsys@useobject{currentmarker}{}%
\end{pgfscope}%
\end{pgfscope}%
\begin{pgfscope}%
\definecolor{textcolor}{rgb}{0.317647,0.317647,0.317647}%
\pgfsetstrokecolor{textcolor}%
\pgfsetfillcolor{textcolor}%
\pgftext[x=0.238670in,y=1.329170in,left,base]{\color{textcolor}\rmfamily\fontsize{6.664000}{7.996800}\selectfont \(\displaystyle 125\)}%
\end{pgfscope}%
\begin{pgfscope}%
\pgfsetbuttcap%
\pgfsetroundjoin%
\definecolor{currentfill}{rgb}{0.317647,0.317647,0.317647}%
\pgfsetfillcolor{currentfill}%
\pgfsetlinewidth{0.501875pt}%
\definecolor{currentstroke}{rgb}{0.317647,0.317647,0.317647}%
\pgfsetstrokecolor{currentstroke}%
\pgfsetdash{}{0pt}%
\pgfsys@defobject{currentmarker}{\pgfqpoint{-0.020833in}{0.000000in}}{\pgfqpoint{0.000000in}{0.000000in}}{%
\pgfpathmoveto{\pgfqpoint{0.000000in}{0.000000in}}%
\pgfpathlineto{\pgfqpoint{-0.020833in}{0.000000in}}%
\pgfusepath{stroke,fill}%
}%
\begin{pgfscope}%
\pgfsys@transformshift{0.439481in}{1.556828in}%
\pgfsys@useobject{currentmarker}{}%
\end{pgfscope}%
\end{pgfscope}%
\begin{pgfscope}%
\definecolor{textcolor}{rgb}{0.317647,0.317647,0.317647}%
\pgfsetstrokecolor{textcolor}%
\pgfsetfillcolor{textcolor}%
\pgftext[x=0.238670in,y=1.524711in,left,base]{\color{textcolor}\rmfamily\fontsize{6.664000}{7.996800}\selectfont \(\displaystyle 150\)}%
\end{pgfscope}%
\begin{pgfscope}%
\pgfsetbuttcap%
\pgfsetroundjoin%
\definecolor{currentfill}{rgb}{0.317647,0.317647,0.317647}%
\pgfsetfillcolor{currentfill}%
\pgfsetlinewidth{0.501875pt}%
\definecolor{currentstroke}{rgb}{0.317647,0.317647,0.317647}%
\pgfsetstrokecolor{currentstroke}%
\pgfsetdash{}{0pt}%
\pgfsys@defobject{currentmarker}{\pgfqpoint{-0.020833in}{0.000000in}}{\pgfqpoint{0.000000in}{0.000000in}}{%
\pgfpathmoveto{\pgfqpoint{0.000000in}{0.000000in}}%
\pgfpathlineto{\pgfqpoint{-0.020833in}{0.000000in}}%
\pgfusepath{stroke,fill}%
}%
\begin{pgfscope}%
\pgfsys@transformshift{0.439481in}{1.752370in}%
\pgfsys@useobject{currentmarker}{}%
\end{pgfscope}%
\end{pgfscope}%
\begin{pgfscope}%
\definecolor{textcolor}{rgb}{0.317647,0.317647,0.317647}%
\pgfsetstrokecolor{textcolor}%
\pgfsetfillcolor{textcolor}%
\pgftext[x=0.238670in,y=1.720253in,left,base]{\color{textcolor}\rmfamily\fontsize{6.664000}{7.996800}\selectfont \(\displaystyle 175\)}%
\end{pgfscope}%
\begin{pgfscope}%
\definecolor{textcolor}{rgb}{0.317647,0.317647,0.317647}%
\pgfsetstrokecolor{textcolor}%
\pgfsetfillcolor{textcolor}%
\pgftext[x=0.183115in,y=1.153578in,,bottom,rotate=90.000000]{\color{textcolor}\rmfamily\fontsize{6.664000}{7.996800}\selectfont density}%
\end{pgfscope}%
\begin{pgfscope}%
\pgfpathrectangle{\pgfqpoint{0.439481in}{0.383578in}}{\pgfqpoint{1.550000in}{1.540000in}}%
\pgfusepath{clip}%
\pgfsetbuttcap%
\pgfsetmiterjoin%
\definecolor{currentfill}{rgb}{0.333333,0.333333,0.333333}%
\pgfsetfillcolor{currentfill}%
\pgfsetfillopacity{0.400000}%
\pgfsetlinewidth{0.000000pt}%
\definecolor{currentstroke}{rgb}{0.000000,0.000000,0.000000}%
\pgfsetstrokecolor{currentstroke}%
\pgfsetstrokeopacity{0.400000}%
\pgfsetdash{}{0pt}%
\pgfpathmoveto{\pgfqpoint{0.293319in}{0.383578in}}%
\pgfpathlineto{\pgfqpoint{0.593790in}{0.383578in}}%
\pgfpathlineto{\pgfqpoint{0.593790in}{0.411598in}}%
\pgfpathlineto{\pgfqpoint{0.293319in}{0.411598in}}%
\pgfpathclose%
\pgfusepath{fill}%
\end{pgfscope}%
\begin{pgfscope}%
\pgfpathrectangle{\pgfqpoint{0.439481in}{0.383578in}}{\pgfqpoint{1.550000in}{1.540000in}}%
\pgfusepath{clip}%
\pgfsetbuttcap%
\pgfsetmiterjoin%
\definecolor{currentfill}{rgb}{0.333333,0.333333,0.333333}%
\pgfsetfillcolor{currentfill}%
\pgfsetfillopacity{0.400000}%
\pgfsetlinewidth{0.000000pt}%
\definecolor{currentstroke}{rgb}{0.000000,0.000000,0.000000}%
\pgfsetstrokecolor{currentstroke}%
\pgfsetstrokeopacity{0.400000}%
\pgfsetdash{}{0pt}%
\pgfpathmoveto{\pgfqpoint{0.593790in}{0.383578in}}%
\pgfpathlineto{\pgfqpoint{0.894261in}{0.383578in}}%
\pgfpathlineto{\pgfqpoint{0.894261in}{0.413349in}}%
\pgfpathlineto{\pgfqpoint{0.593790in}{0.413349in}}%
\pgfpathclose%
\pgfusepath{fill}%
\end{pgfscope}%
\begin{pgfscope}%
\pgfpathrectangle{\pgfqpoint{0.439481in}{0.383578in}}{\pgfqpoint{1.550000in}{1.540000in}}%
\pgfusepath{clip}%
\pgfsetbuttcap%
\pgfsetmiterjoin%
\definecolor{currentfill}{rgb}{0.333333,0.333333,0.333333}%
\pgfsetfillcolor{currentfill}%
\pgfsetfillopacity{0.400000}%
\pgfsetlinewidth{0.000000pt}%
\definecolor{currentstroke}{rgb}{0.000000,0.000000,0.000000}%
\pgfsetstrokecolor{currentstroke}%
\pgfsetstrokeopacity{0.400000}%
\pgfsetdash{}{0pt}%
\pgfpathmoveto{\pgfqpoint{0.894261in}{0.383578in}}%
\pgfpathlineto{\pgfqpoint{1.194731in}{0.383578in}}%
\pgfpathlineto{\pgfqpoint{1.194731in}{0.422105in}}%
\pgfpathlineto{\pgfqpoint{0.894261in}{0.422105in}}%
\pgfpathclose%
\pgfusepath{fill}%
\end{pgfscope}%
\begin{pgfscope}%
\pgfpathrectangle{\pgfqpoint{0.439481in}{0.383578in}}{\pgfqpoint{1.550000in}{1.540000in}}%
\pgfusepath{clip}%
\pgfsetbuttcap%
\pgfsetmiterjoin%
\definecolor{currentfill}{rgb}{0.333333,0.333333,0.333333}%
\pgfsetfillcolor{currentfill}%
\pgfsetfillopacity{0.400000}%
\pgfsetlinewidth{0.000000pt}%
\definecolor{currentstroke}{rgb}{0.000000,0.000000,0.000000}%
\pgfsetstrokecolor{currentstroke}%
\pgfsetstrokeopacity{0.400000}%
\pgfsetdash{}{0pt}%
\pgfpathmoveto{\pgfqpoint{1.194731in}{0.383578in}}%
\pgfpathlineto{\pgfqpoint{1.495202in}{0.383578in}}%
\pgfpathlineto{\pgfqpoint{1.495202in}{0.430862in}}%
\pgfpathlineto{\pgfqpoint{1.194731in}{0.430862in}}%
\pgfpathclose%
\pgfusepath{fill}%
\end{pgfscope}%
\begin{pgfscope}%
\pgfpathrectangle{\pgfqpoint{0.439481in}{0.383578in}}{\pgfqpoint{1.550000in}{1.540000in}}%
\pgfusepath{clip}%
\pgfsetbuttcap%
\pgfsetmiterjoin%
\definecolor{currentfill}{rgb}{0.333333,0.333333,0.333333}%
\pgfsetfillcolor{currentfill}%
\pgfsetfillopacity{0.400000}%
\pgfsetlinewidth{0.000000pt}%
\definecolor{currentstroke}{rgb}{0.000000,0.000000,0.000000}%
\pgfsetstrokecolor{currentstroke}%
\pgfsetstrokeopacity{0.400000}%
\pgfsetdash{}{0pt}%
\pgfpathmoveto{\pgfqpoint{1.495202in}{0.383578in}}%
\pgfpathlineto{\pgfqpoint{1.795672in}{0.383578in}}%
\pgfpathlineto{\pgfqpoint{1.795672in}{0.422105in}}%
\pgfpathlineto{\pgfqpoint{1.495202in}{0.422105in}}%
\pgfpathclose%
\pgfusepath{fill}%
\end{pgfscope}%
\begin{pgfscope}%
\pgfpathrectangle{\pgfqpoint{0.439481in}{0.383578in}}{\pgfqpoint{1.550000in}{1.540000in}}%
\pgfusepath{clip}%
\pgfsetbuttcap%
\pgfsetmiterjoin%
\definecolor{currentfill}{rgb}{0.333333,0.333333,0.333333}%
\pgfsetfillcolor{currentfill}%
\pgfsetfillopacity{0.400000}%
\pgfsetlinewidth{0.000000pt}%
\definecolor{currentstroke}{rgb}{0.000000,0.000000,0.000000}%
\pgfsetstrokecolor{currentstroke}%
\pgfsetstrokeopacity{0.400000}%
\pgfsetdash{}{0pt}%
\pgfpathmoveto{\pgfqpoint{1.795672in}{0.383578in}}%
\pgfpathlineto{\pgfqpoint{2.096143in}{0.383578in}}%
\pgfpathlineto{\pgfqpoint{2.096143in}{0.411598in}}%
\pgfpathlineto{\pgfqpoint{1.795672in}{0.411598in}}%
\pgfpathclose%
\pgfusepath{fill}%
\end{pgfscope}%
\begin{pgfscope}%
\pgfpathrectangle{\pgfqpoint{0.439481in}{0.383578in}}{\pgfqpoint{1.550000in}{1.540000in}}%
\pgfusepath{clip}%
\pgfsetbuttcap%
\pgfsetmiterjoin%
\definecolor{currentfill}{rgb}{0.333333,0.333333,0.333333}%
\pgfsetfillcolor{currentfill}%
\pgfsetfillopacity{0.400000}%
\pgfsetlinewidth{0.000000pt}%
\definecolor{currentstroke}{rgb}{0.000000,0.000000,0.000000}%
\pgfsetstrokecolor{currentstroke}%
\pgfsetstrokeopacity{0.400000}%
\pgfsetdash{}{0pt}%
\pgfpathmoveto{\pgfqpoint{2.096143in}{0.383578in}}%
\pgfpathlineto{\pgfqpoint{2.396614in}{0.383578in}}%
\pgfpathlineto{\pgfqpoint{2.396614in}{0.394086in}}%
\pgfpathlineto{\pgfqpoint{2.096143in}{0.394086in}}%
\pgfpathclose%
\pgfusepath{fill}%
\end{pgfscope}%
\begin{pgfscope}%
\pgfpathrectangle{\pgfqpoint{0.439481in}{0.383578in}}{\pgfqpoint{1.550000in}{1.540000in}}%
\pgfusepath{clip}%
\pgfsetbuttcap%
\pgfsetmiterjoin%
\definecolor{currentfill}{rgb}{0.333333,0.333333,0.333333}%
\pgfsetfillcolor{currentfill}%
\pgfsetfillopacity{0.400000}%
\pgfsetlinewidth{0.000000pt}%
\definecolor{currentstroke}{rgb}{0.000000,0.000000,0.000000}%
\pgfsetstrokecolor{currentstroke}%
\pgfsetstrokeopacity{0.400000}%
\pgfsetdash{}{0pt}%
\pgfpathmoveto{\pgfqpoint{2.396614in}{0.383578in}}%
\pgfpathlineto{\pgfqpoint{2.697084in}{0.383578in}}%
\pgfpathlineto{\pgfqpoint{2.697084in}{0.387081in}}%
\pgfpathlineto{\pgfqpoint{2.396614in}{0.387081in}}%
\pgfpathclose%
\pgfusepath{fill}%
\end{pgfscope}%
\begin{pgfscope}%
\pgfpathrectangle{\pgfqpoint{0.439481in}{0.383578in}}{\pgfqpoint{1.550000in}{1.540000in}}%
\pgfusepath{clip}%
\pgfsetbuttcap%
\pgfsetmiterjoin%
\definecolor{currentfill}{rgb}{0.686275,0.352941,0.313725}%
\pgfsetfillcolor{currentfill}%
\pgfsetfillopacity{0.400000}%
\pgfsetlinewidth{0.000000pt}%
\definecolor{currentstroke}{rgb}{0.000000,0.000000,0.000000}%
\pgfsetstrokecolor{currentstroke}%
\pgfsetstrokeopacity{0.400000}%
\pgfsetdash{}{0pt}%
\pgfpathmoveto{\pgfqpoint{1.082615in}{0.383578in}}%
\pgfpathlineto{\pgfqpoint{1.111317in}{0.383578in}}%
\pgfpathlineto{\pgfqpoint{1.111317in}{0.420245in}}%
\pgfpathlineto{\pgfqpoint{1.082615in}{0.420245in}}%
\pgfpathclose%
\pgfusepath{fill}%
\end{pgfscope}%
\begin{pgfscope}%
\pgfpathrectangle{\pgfqpoint{0.439481in}{0.383578in}}{\pgfqpoint{1.550000in}{1.540000in}}%
\pgfusepath{clip}%
\pgfsetbuttcap%
\pgfsetmiterjoin%
\definecolor{currentfill}{rgb}{0.686275,0.352941,0.313725}%
\pgfsetfillcolor{currentfill}%
\pgfsetfillopacity{0.400000}%
\pgfsetlinewidth{0.000000pt}%
\definecolor{currentstroke}{rgb}{0.000000,0.000000,0.000000}%
\pgfsetstrokecolor{currentstroke}%
\pgfsetstrokeopacity{0.400000}%
\pgfsetdash{}{0pt}%
\pgfpathmoveto{\pgfqpoint{1.111317in}{0.383578in}}%
\pgfpathlineto{\pgfqpoint{1.140019in}{0.383578in}}%
\pgfpathlineto{\pgfqpoint{1.140019in}{0.805245in}}%
\pgfpathlineto{\pgfqpoint{1.111317in}{0.805245in}}%
\pgfpathclose%
\pgfusepath{fill}%
\end{pgfscope}%
\begin{pgfscope}%
\pgfpathrectangle{\pgfqpoint{0.439481in}{0.383578in}}{\pgfqpoint{1.550000in}{1.540000in}}%
\pgfusepath{clip}%
\pgfsetbuttcap%
\pgfsetmiterjoin%
\definecolor{currentfill}{rgb}{0.686275,0.352941,0.313725}%
\pgfsetfillcolor{currentfill}%
\pgfsetfillopacity{0.400000}%
\pgfsetlinewidth{0.000000pt}%
\definecolor{currentstroke}{rgb}{0.000000,0.000000,0.000000}%
\pgfsetstrokecolor{currentstroke}%
\pgfsetstrokeopacity{0.400000}%
\pgfsetdash{}{0pt}%
\pgfpathmoveto{\pgfqpoint{1.140019in}{0.383578in}}%
\pgfpathlineto{\pgfqpoint{1.168720in}{0.383578in}}%
\pgfpathlineto{\pgfqpoint{1.168720in}{1.850245in}}%
\pgfpathlineto{\pgfqpoint{1.140019in}{1.850245in}}%
\pgfpathclose%
\pgfusepath{fill}%
\end{pgfscope}%
\begin{pgfscope}%
\pgfpathrectangle{\pgfqpoint{0.439481in}{0.383578in}}{\pgfqpoint{1.550000in}{1.540000in}}%
\pgfusepath{clip}%
\pgfsetbuttcap%
\pgfsetmiterjoin%
\definecolor{currentfill}{rgb}{0.686275,0.352941,0.313725}%
\pgfsetfillcolor{currentfill}%
\pgfsetfillopacity{0.400000}%
\pgfsetlinewidth{0.000000pt}%
\definecolor{currentstroke}{rgb}{0.000000,0.000000,0.000000}%
\pgfsetstrokecolor{currentstroke}%
\pgfsetstrokeopacity{0.400000}%
\pgfsetdash{}{0pt}%
\pgfpathmoveto{\pgfqpoint{1.168720in}{0.383578in}}%
\pgfpathlineto{\pgfqpoint{1.197422in}{0.383578in}}%
\pgfpathlineto{\pgfqpoint{1.197422in}{0.768578in}}%
\pgfpathlineto{\pgfqpoint{1.168720in}{0.768578in}}%
\pgfpathclose%
\pgfusepath{fill}%
\end{pgfscope}%
\begin{pgfscope}%
\pgfpathrectangle{\pgfqpoint{0.439481in}{0.383578in}}{\pgfqpoint{1.550000in}{1.540000in}}%
\pgfusepath{clip}%
\pgfsetbuttcap%
\pgfsetmiterjoin%
\definecolor{currentfill}{rgb}{0.686275,0.352941,0.313725}%
\pgfsetfillcolor{currentfill}%
\pgfsetfillopacity{0.400000}%
\pgfsetlinewidth{0.000000pt}%
\definecolor{currentstroke}{rgb}{0.000000,0.000000,0.000000}%
\pgfsetstrokecolor{currentstroke}%
\pgfsetstrokeopacity{0.400000}%
\pgfsetdash{}{0pt}%
\pgfpathmoveto{\pgfqpoint{1.197422in}{0.383578in}}%
\pgfpathlineto{\pgfqpoint{1.226124in}{0.383578in}}%
\pgfpathlineto{\pgfqpoint{1.226124in}{0.420245in}}%
\pgfpathlineto{\pgfqpoint{1.197422in}{0.420245in}}%
\pgfpathclose%
\pgfusepath{fill}%
\end{pgfscope}%
\begin{pgfscope}%
\pgfsetrectcap%
\pgfsetmiterjoin%
\pgfsetlinewidth{0.501875pt}%
\definecolor{currentstroke}{rgb}{0.317647,0.317647,0.317647}%
\pgfsetstrokecolor{currentstroke}%
\pgfsetdash{}{0pt}%
\pgfpathmoveto{\pgfqpoint{0.439481in}{0.383578in}}%
\pgfpathlineto{\pgfqpoint{0.439481in}{1.923578in}}%
\pgfusepath{stroke}%
\end{pgfscope}%
\begin{pgfscope}%
\pgfsetrectcap%
\pgfsetmiterjoin%
\pgfsetlinewidth{0.501875pt}%
\definecolor{currentstroke}{rgb}{0.317647,0.317647,0.317647}%
\pgfsetstrokecolor{currentstroke}%
\pgfsetdash{}{0pt}%
\pgfpathmoveto{\pgfqpoint{0.439481in}{0.383578in}}%
\pgfpathlineto{\pgfqpoint{1.989481in}{0.383578in}}%
\pgfusepath{stroke}%
\end{pgfscope}%
\begin{pgfscope}%
\pgfsetbuttcap%
\pgfsetmiterjoin%
\definecolor{currentfill}{rgb}{0.333333,0.333333,0.333333}%
\pgfsetfillcolor{currentfill}%
\pgfsetfillopacity{0.400000}%
\pgfsetlinewidth{0.000000pt}%
\definecolor{currentstroke}{rgb}{0.000000,0.000000,0.000000}%
\pgfsetstrokecolor{currentstroke}%
\pgfsetstrokeopacity{0.400000}%
\pgfsetdash{}{0pt}%
\pgfpathmoveto{\pgfqpoint{1.373308in}{1.831022in}}%
\pgfpathlineto{\pgfqpoint{1.447352in}{1.831022in}}%
\pgfpathlineto{\pgfqpoint{1.447352in}{1.895811in}}%
\pgfpathlineto{\pgfqpoint{1.373308in}{1.895811in}}%
\pgfpathclose%
\pgfusepath{fill}%
\end{pgfscope}%
\begin{pgfscope}%
\definecolor{textcolor}{rgb}{0.000000,0.000000,0.000000}%
\pgfsetstrokecolor{textcolor}%
\pgfsetfillcolor{textcolor}%
\pgftext[x=1.493630in,y=1.831022in,left,base]{\color{textcolor}\rmfamily\fontsize{6.664000}{7.996800}\selectfont pre \(\displaystyle \SI{0.4}{\V}\)}%
\end{pgfscope}%
\begin{pgfscope}%
\pgfsetbuttcap%
\pgfsetmiterjoin%
\definecolor{currentfill}{rgb}{0.686275,0.352941,0.313725}%
\pgfsetfillcolor{currentfill}%
\pgfsetfillopacity{0.400000}%
\pgfsetlinewidth{0.000000pt}%
\definecolor{currentstroke}{rgb}{0.000000,0.000000,0.000000}%
\pgfsetstrokecolor{currentstroke}%
\pgfsetstrokeopacity{0.400000}%
\pgfsetdash{}{0pt}%
\pgfpathmoveto{\pgfqpoint{1.373308in}{1.711256in}}%
\pgfpathlineto{\pgfqpoint{1.447352in}{1.711256in}}%
\pgfpathlineto{\pgfqpoint{1.447352in}{1.776045in}}%
\pgfpathlineto{\pgfqpoint{1.373308in}{1.776045in}}%
\pgfpathclose%
\pgfusepath{fill}%
\end{pgfscope}%
\begin{pgfscope}%
\definecolor{textcolor}{rgb}{0.000000,0.000000,0.000000}%
\pgfsetstrokecolor{textcolor}%
\pgfsetfillcolor{textcolor}%
\pgftext[x=1.493630in,y=1.711256in,left,base]{\color{textcolor}\rmfamily\fontsize{6.664000}{7.996800}\selectfont post \(\displaystyle \SI{0.4}{\V}\)}%
\end{pgfscope}%
\end{pgfpicture}%
\makeatother%
\endgroup%

		\label{hxprepostvreset}
	\end{subfigure}
	\begin{subfigure}{0.32\textwidth}
	\caption{}
	\centering
	%% Creator: Matplotlib, PGF backend
%%
%% To include the figure in your LaTeX document, write
%%   \input{<filename>.pgf}
%%
%% Make sure the required packages are loaded in your preamble
%%   \usepackage{pgf}
%%
%% Figures using additional raster images can only be included by \input if
%% they are in the same directory as the main LaTeX file. For loading figures
%% from other directories you can use the `import` package
%%   \usepackage{import}
%% and then include the figures with
%%   \import{<path to file>}{<filename>.pgf}
%%
%% Matplotlib used the following preamble
%%   \usepackage{amsmath} \usepackage{pifont} \usepackage{xcolor} \definecolor{green}{HTML}{467821} \definecolor{red}{HTML}{CF4457} \usepackage[detect-all]{siunitx}
%%   \usepackage{fontspec}
%%
\begingroup%
\makeatletter%
\begin{pgfpicture}%
\pgfpathrectangle{\pgfpointorigin}{\pgfqpoint{2.161337in}{2.053663in}}%
\pgfusepath{use as bounding box, clip}%
\begin{pgfscope}%
\pgfsetbuttcap%
\pgfsetmiterjoin%
\pgfsetlinewidth{0.000000pt}%
\definecolor{currentstroke}{rgb}{0.000000,0.000000,0.000000}%
\pgfsetstrokecolor{currentstroke}%
\pgfsetstrokeopacity{0.000000}%
\pgfsetdash{}{0pt}%
\pgfpathmoveto{\pgfqpoint{0.000000in}{0.000000in}}%
\pgfpathlineto{\pgfqpoint{2.161337in}{0.000000in}}%
\pgfpathlineto{\pgfqpoint{2.161337in}{2.053663in}}%
\pgfpathlineto{\pgfqpoint{0.000000in}{2.053663in}}%
\pgfpathclose%
\pgfusepath{}%
\end{pgfscope}%
\begin{pgfscope}%
\pgfsetbuttcap%
\pgfsetmiterjoin%
\pgfsetlinewidth{0.000000pt}%
\definecolor{currentstroke}{rgb}{0.000000,0.000000,0.000000}%
\pgfsetstrokecolor{currentstroke}%
\pgfsetstrokeopacity{0.000000}%
\pgfsetdash{}{0pt}%
\pgfpathmoveto{\pgfqpoint{0.439481in}{0.383578in}}%
\pgfpathlineto{\pgfqpoint{1.989481in}{0.383578in}}%
\pgfpathlineto{\pgfqpoint{1.989481in}{1.923578in}}%
\pgfpathlineto{\pgfqpoint{0.439481in}{1.923578in}}%
\pgfpathclose%
\pgfusepath{}%
\end{pgfscope}%
\begin{pgfscope}%
\pgfsetbuttcap%
\pgfsetroundjoin%
\definecolor{currentfill}{rgb}{0.317647,0.317647,0.317647}%
\pgfsetfillcolor{currentfill}%
\pgfsetlinewidth{0.501875pt}%
\definecolor{currentstroke}{rgb}{0.317647,0.317647,0.317647}%
\pgfsetstrokecolor{currentstroke}%
\pgfsetdash{}{0pt}%
\pgfsys@defobject{currentmarker}{\pgfqpoint{0.000000in}{-0.020833in}}{\pgfqpoint{0.000000in}{0.000000in}}{%
\pgfpathmoveto{\pgfqpoint{0.000000in}{0.000000in}}%
\pgfpathlineto{\pgfqpoint{0.000000in}{-0.020833in}}%
\pgfusepath{stroke,fill}%
}%
\begin{pgfscope}%
\pgfsys@transformshift{0.660910in}{0.383578in}%
\pgfsys@useobject{currentmarker}{}%
\end{pgfscope}%
\end{pgfscope}%
\begin{pgfscope}%
\definecolor{textcolor}{rgb}{0.317647,0.317647,0.317647}%
\pgfsetstrokecolor{textcolor}%
\pgfsetfillcolor{textcolor}%
\pgftext[x=0.660910in,y=0.334967in,,top]{\color{textcolor}\rmfamily\fontsize{6.664000}{7.996800}\selectfont \(\displaystyle 0.7\)}%
\end{pgfscope}%
\begin{pgfscope}%
\pgfsetbuttcap%
\pgfsetroundjoin%
\definecolor{currentfill}{rgb}{0.317647,0.317647,0.317647}%
\pgfsetfillcolor{currentfill}%
\pgfsetlinewidth{0.501875pt}%
\definecolor{currentstroke}{rgb}{0.317647,0.317647,0.317647}%
\pgfsetstrokecolor{currentstroke}%
\pgfsetdash{}{0pt}%
\pgfsys@defobject{currentmarker}{\pgfqpoint{0.000000in}{-0.020833in}}{\pgfqpoint{0.000000in}{0.000000in}}{%
\pgfpathmoveto{\pgfqpoint{0.000000in}{0.000000in}}%
\pgfpathlineto{\pgfqpoint{0.000000in}{-0.020833in}}%
\pgfusepath{stroke,fill}%
}%
\begin{pgfscope}%
\pgfsys@transformshift{1.103767in}{0.383578in}%
\pgfsys@useobject{currentmarker}{}%
\end{pgfscope}%
\end{pgfscope}%
\begin{pgfscope}%
\definecolor{textcolor}{rgb}{0.317647,0.317647,0.317647}%
\pgfsetstrokecolor{textcolor}%
\pgfsetfillcolor{textcolor}%
\pgftext[x=1.103767in,y=0.334967in,,top]{\color{textcolor}\rmfamily\fontsize{6.664000}{7.996800}\selectfont \(\displaystyle 0.8\)}%
\end{pgfscope}%
\begin{pgfscope}%
\pgfsetbuttcap%
\pgfsetroundjoin%
\definecolor{currentfill}{rgb}{0.317647,0.317647,0.317647}%
\pgfsetfillcolor{currentfill}%
\pgfsetlinewidth{0.501875pt}%
\definecolor{currentstroke}{rgb}{0.317647,0.317647,0.317647}%
\pgfsetstrokecolor{currentstroke}%
\pgfsetdash{}{0pt}%
\pgfsys@defobject{currentmarker}{\pgfqpoint{0.000000in}{-0.020833in}}{\pgfqpoint{0.000000in}{0.000000in}}{%
\pgfpathmoveto{\pgfqpoint{0.000000in}{0.000000in}}%
\pgfpathlineto{\pgfqpoint{0.000000in}{-0.020833in}}%
\pgfusepath{stroke,fill}%
}%
\begin{pgfscope}%
\pgfsys@transformshift{1.546624in}{0.383578in}%
\pgfsys@useobject{currentmarker}{}%
\end{pgfscope}%
\end{pgfscope}%
\begin{pgfscope}%
\definecolor{textcolor}{rgb}{0.317647,0.317647,0.317647}%
\pgfsetstrokecolor{textcolor}%
\pgfsetfillcolor{textcolor}%
\pgftext[x=1.546624in,y=0.334967in,,top]{\color{textcolor}\rmfamily\fontsize{6.664000}{7.996800}\selectfont \(\displaystyle 0.9\)}%
\end{pgfscope}%
\begin{pgfscope}%
\pgfsetbuttcap%
\pgfsetroundjoin%
\definecolor{currentfill}{rgb}{0.317647,0.317647,0.317647}%
\pgfsetfillcolor{currentfill}%
\pgfsetlinewidth{0.501875pt}%
\definecolor{currentstroke}{rgb}{0.317647,0.317647,0.317647}%
\pgfsetstrokecolor{currentstroke}%
\pgfsetdash{}{0pt}%
\pgfsys@defobject{currentmarker}{\pgfqpoint{0.000000in}{-0.020833in}}{\pgfqpoint{0.000000in}{0.000000in}}{%
\pgfpathmoveto{\pgfqpoint{0.000000in}{0.000000in}}%
\pgfpathlineto{\pgfqpoint{0.000000in}{-0.020833in}}%
\pgfusepath{stroke,fill}%
}%
\begin{pgfscope}%
\pgfsys@transformshift{1.989481in}{0.383578in}%
\pgfsys@useobject{currentmarker}{}%
\end{pgfscope}%
\end{pgfscope}%
\begin{pgfscope}%
\definecolor{textcolor}{rgb}{0.317647,0.317647,0.317647}%
\pgfsetstrokecolor{textcolor}%
\pgfsetfillcolor{textcolor}%
\pgftext[x=1.989481in,y=0.334967in,,top]{\color{textcolor}\rmfamily\fontsize{6.664000}{7.996800}\selectfont \(\displaystyle 1.0\)}%
\end{pgfscope}%
\begin{pgfscope}%
\definecolor{textcolor}{rgb}{0.317647,0.317647,0.317647}%
\pgfsetstrokecolor{textcolor}%
\pgfsetfillcolor{textcolor}%
\pgftext[x=1.214481in,y=0.197222in,,top]{\color{textcolor}\rmfamily\fontsize{6.664000}{7.996800}\selectfont \(\displaystyle \vartheta \; (\si{\V})\)}%
\end{pgfscope}%
\begin{pgfscope}%
\pgfsetbuttcap%
\pgfsetroundjoin%
\definecolor{currentfill}{rgb}{0.317647,0.317647,0.317647}%
\pgfsetfillcolor{currentfill}%
\pgfsetlinewidth{0.501875pt}%
\definecolor{currentstroke}{rgb}{0.317647,0.317647,0.317647}%
\pgfsetstrokecolor{currentstroke}%
\pgfsetdash{}{0pt}%
\pgfsys@defobject{currentmarker}{\pgfqpoint{-0.020833in}{0.000000in}}{\pgfqpoint{0.000000in}{0.000000in}}{%
\pgfpathmoveto{\pgfqpoint{0.000000in}{0.000000in}}%
\pgfpathlineto{\pgfqpoint{-0.020833in}{0.000000in}}%
\pgfusepath{stroke,fill}%
}%
\begin{pgfscope}%
\pgfsys@transformshift{0.439481in}{0.383578in}%
\pgfsys@useobject{currentmarker}{}%
\end{pgfscope}%
\end{pgfscope}%
\begin{pgfscope}%
\definecolor{textcolor}{rgb}{0.317647,0.317647,0.317647}%
\pgfsetstrokecolor{textcolor}%
\pgfsetfillcolor{textcolor}%
\pgftext[x=0.349396in,y=0.351461in,left,base]{\color{textcolor}\rmfamily\fontsize{6.664000}{7.996800}\selectfont \(\displaystyle 0\)}%
\end{pgfscope}%
\begin{pgfscope}%
\pgfsetbuttcap%
\pgfsetroundjoin%
\definecolor{currentfill}{rgb}{0.317647,0.317647,0.317647}%
\pgfsetfillcolor{currentfill}%
\pgfsetlinewidth{0.501875pt}%
\definecolor{currentstroke}{rgb}{0.317647,0.317647,0.317647}%
\pgfsetstrokecolor{currentstroke}%
\pgfsetdash{}{0pt}%
\pgfsys@defobject{currentmarker}{\pgfqpoint{-0.020833in}{0.000000in}}{\pgfqpoint{0.000000in}{0.000000in}}{%
\pgfpathmoveto{\pgfqpoint{0.000000in}{0.000000in}}%
\pgfpathlineto{\pgfqpoint{-0.020833in}{0.000000in}}%
\pgfusepath{stroke,fill}%
}%
\begin{pgfscope}%
\pgfsys@transformshift{0.439481in}{0.691172in}%
\pgfsys@useobject{currentmarker}{}%
\end{pgfscope}%
\end{pgfscope}%
\begin{pgfscope}%
\definecolor{textcolor}{rgb}{0.317647,0.317647,0.317647}%
\pgfsetstrokecolor{textcolor}%
\pgfsetfillcolor{textcolor}%
\pgftext[x=0.294033in,y=0.659055in,left,base]{\color{textcolor}\rmfamily\fontsize{6.664000}{7.996800}\selectfont \(\displaystyle 20\)}%
\end{pgfscope}%
\begin{pgfscope}%
\pgfsetbuttcap%
\pgfsetroundjoin%
\definecolor{currentfill}{rgb}{0.317647,0.317647,0.317647}%
\pgfsetfillcolor{currentfill}%
\pgfsetlinewidth{0.501875pt}%
\definecolor{currentstroke}{rgb}{0.317647,0.317647,0.317647}%
\pgfsetstrokecolor{currentstroke}%
\pgfsetdash{}{0pt}%
\pgfsys@defobject{currentmarker}{\pgfqpoint{-0.020833in}{0.000000in}}{\pgfqpoint{0.000000in}{0.000000in}}{%
\pgfpathmoveto{\pgfqpoint{0.000000in}{0.000000in}}%
\pgfpathlineto{\pgfqpoint{-0.020833in}{0.000000in}}%
\pgfusepath{stroke,fill}%
}%
\begin{pgfscope}%
\pgfsys@transformshift{0.439481in}{0.998765in}%
\pgfsys@useobject{currentmarker}{}%
\end{pgfscope}%
\end{pgfscope}%
\begin{pgfscope}%
\definecolor{textcolor}{rgb}{0.317647,0.317647,0.317647}%
\pgfsetstrokecolor{textcolor}%
\pgfsetfillcolor{textcolor}%
\pgftext[x=0.294033in,y=0.966649in,left,base]{\color{textcolor}\rmfamily\fontsize{6.664000}{7.996800}\selectfont \(\displaystyle 40\)}%
\end{pgfscope}%
\begin{pgfscope}%
\pgfsetbuttcap%
\pgfsetroundjoin%
\definecolor{currentfill}{rgb}{0.317647,0.317647,0.317647}%
\pgfsetfillcolor{currentfill}%
\pgfsetlinewidth{0.501875pt}%
\definecolor{currentstroke}{rgb}{0.317647,0.317647,0.317647}%
\pgfsetstrokecolor{currentstroke}%
\pgfsetdash{}{0pt}%
\pgfsys@defobject{currentmarker}{\pgfqpoint{-0.020833in}{0.000000in}}{\pgfqpoint{0.000000in}{0.000000in}}{%
\pgfpathmoveto{\pgfqpoint{0.000000in}{0.000000in}}%
\pgfpathlineto{\pgfqpoint{-0.020833in}{0.000000in}}%
\pgfusepath{stroke,fill}%
}%
\begin{pgfscope}%
\pgfsys@transformshift{0.439481in}{1.306359in}%
\pgfsys@useobject{currentmarker}{}%
\end{pgfscope}%
\end{pgfscope}%
\begin{pgfscope}%
\definecolor{textcolor}{rgb}{0.317647,0.317647,0.317647}%
\pgfsetstrokecolor{textcolor}%
\pgfsetfillcolor{textcolor}%
\pgftext[x=0.294033in,y=1.274242in,left,base]{\color{textcolor}\rmfamily\fontsize{6.664000}{7.996800}\selectfont \(\displaystyle 60\)}%
\end{pgfscope}%
\begin{pgfscope}%
\pgfsetbuttcap%
\pgfsetroundjoin%
\definecolor{currentfill}{rgb}{0.317647,0.317647,0.317647}%
\pgfsetfillcolor{currentfill}%
\pgfsetlinewidth{0.501875pt}%
\definecolor{currentstroke}{rgb}{0.317647,0.317647,0.317647}%
\pgfsetstrokecolor{currentstroke}%
\pgfsetdash{}{0pt}%
\pgfsys@defobject{currentmarker}{\pgfqpoint{-0.020833in}{0.000000in}}{\pgfqpoint{0.000000in}{0.000000in}}{%
\pgfpathmoveto{\pgfqpoint{0.000000in}{0.000000in}}%
\pgfpathlineto{\pgfqpoint{-0.020833in}{0.000000in}}%
\pgfusepath{stroke,fill}%
}%
\begin{pgfscope}%
\pgfsys@transformshift{0.439481in}{1.613953in}%
\pgfsys@useobject{currentmarker}{}%
\end{pgfscope}%
\end{pgfscope}%
\begin{pgfscope}%
\definecolor{textcolor}{rgb}{0.317647,0.317647,0.317647}%
\pgfsetstrokecolor{textcolor}%
\pgfsetfillcolor{textcolor}%
\pgftext[x=0.294033in,y=1.581836in,left,base]{\color{textcolor}\rmfamily\fontsize{6.664000}{7.996800}\selectfont \(\displaystyle 80\)}%
\end{pgfscope}%
\begin{pgfscope}%
\pgfsetbuttcap%
\pgfsetroundjoin%
\definecolor{currentfill}{rgb}{0.317647,0.317647,0.317647}%
\pgfsetfillcolor{currentfill}%
\pgfsetlinewidth{0.501875pt}%
\definecolor{currentstroke}{rgb}{0.317647,0.317647,0.317647}%
\pgfsetstrokecolor{currentstroke}%
\pgfsetdash{}{0pt}%
\pgfsys@defobject{currentmarker}{\pgfqpoint{-0.020833in}{0.000000in}}{\pgfqpoint{0.000000in}{0.000000in}}{%
\pgfpathmoveto{\pgfqpoint{0.000000in}{0.000000in}}%
\pgfpathlineto{\pgfqpoint{-0.020833in}{0.000000in}}%
\pgfusepath{stroke,fill}%
}%
\begin{pgfscope}%
\pgfsys@transformshift{0.439481in}{1.921546in}%
\pgfsys@useobject{currentmarker}{}%
\end{pgfscope}%
\end{pgfscope}%
\begin{pgfscope}%
\definecolor{textcolor}{rgb}{0.317647,0.317647,0.317647}%
\pgfsetstrokecolor{textcolor}%
\pgfsetfillcolor{textcolor}%
\pgftext[x=0.238670in,y=1.889430in,left,base]{\color{textcolor}\rmfamily\fontsize{6.664000}{7.996800}\selectfont \(\displaystyle 100\)}%
\end{pgfscope}%
\begin{pgfscope}%
\definecolor{textcolor}{rgb}{0.317647,0.317647,0.317647}%
\pgfsetstrokecolor{textcolor}%
\pgfsetfillcolor{textcolor}%
\pgftext[x=0.183115in,y=1.153578in,,bottom,rotate=90.000000]{\color{textcolor}\rmfamily\fontsize{6.664000}{7.996800}\selectfont density}%
\end{pgfscope}%
\begin{pgfscope}%
\pgfpathrectangle{\pgfqpoint{0.439481in}{0.383578in}}{\pgfqpoint{1.550000in}{1.540000in}}%
\pgfusepath{clip}%
\pgfsetbuttcap%
\pgfsetmiterjoin%
\definecolor{currentfill}{rgb}{0.333333,0.333333,0.333333}%
\pgfsetfillcolor{currentfill}%
\pgfsetfillopacity{0.400000}%
\pgfsetlinewidth{0.000000pt}%
\definecolor{currentstroke}{rgb}{0.000000,0.000000,0.000000}%
\pgfsetstrokecolor{currentstroke}%
\pgfsetstrokeopacity{0.400000}%
\pgfsetdash{}{0pt}%
\pgfpathmoveto{\pgfqpoint{0.323616in}{0.383578in}}%
\pgfpathlineto{\pgfqpoint{0.407485in}{0.383578in}}%
\pgfpathlineto{\pgfqpoint{0.407485in}{0.396267in}}%
\pgfpathlineto{\pgfqpoint{0.323616in}{0.396267in}}%
\pgfpathclose%
\pgfusepath{fill}%
\end{pgfscope}%
\begin{pgfscope}%
\pgfpathrectangle{\pgfqpoint{0.439481in}{0.383578in}}{\pgfqpoint{1.550000in}{1.540000in}}%
\pgfusepath{clip}%
\pgfsetbuttcap%
\pgfsetmiterjoin%
\definecolor{currentfill}{rgb}{0.333333,0.333333,0.333333}%
\pgfsetfillcolor{currentfill}%
\pgfsetfillopacity{0.400000}%
\pgfsetlinewidth{0.000000pt}%
\definecolor{currentstroke}{rgb}{0.000000,0.000000,0.000000}%
\pgfsetstrokecolor{currentstroke}%
\pgfsetstrokeopacity{0.400000}%
\pgfsetdash{}{0pt}%
\pgfpathmoveto{\pgfqpoint{0.407485in}{0.383578in}}%
\pgfpathlineto{\pgfqpoint{0.491353in}{0.383578in}}%
\pgfpathlineto{\pgfqpoint{0.491353in}{0.396267in}}%
\pgfpathlineto{\pgfqpoint{0.407485in}{0.396267in}}%
\pgfpathclose%
\pgfusepath{fill}%
\end{pgfscope}%
\begin{pgfscope}%
\pgfpathrectangle{\pgfqpoint{0.439481in}{0.383578in}}{\pgfqpoint{1.550000in}{1.540000in}}%
\pgfusepath{clip}%
\pgfsetbuttcap%
\pgfsetmiterjoin%
\definecolor{currentfill}{rgb}{0.333333,0.333333,0.333333}%
\pgfsetfillcolor{currentfill}%
\pgfsetfillopacity{0.400000}%
\pgfsetlinewidth{0.000000pt}%
\definecolor{currentstroke}{rgb}{0.000000,0.000000,0.000000}%
\pgfsetstrokecolor{currentstroke}%
\pgfsetstrokeopacity{0.400000}%
\pgfsetdash{}{0pt}%
\pgfpathmoveto{\pgfqpoint{0.491353in}{0.383578in}}%
\pgfpathlineto{\pgfqpoint{0.575222in}{0.383578in}}%
\pgfpathlineto{\pgfqpoint{0.575222in}{0.389923in}}%
\pgfpathlineto{\pgfqpoint{0.491353in}{0.389923in}}%
\pgfpathclose%
\pgfusepath{fill}%
\end{pgfscope}%
\begin{pgfscope}%
\pgfpathrectangle{\pgfqpoint{0.439481in}{0.383578in}}{\pgfqpoint{1.550000in}{1.540000in}}%
\pgfusepath{clip}%
\pgfsetbuttcap%
\pgfsetmiterjoin%
\definecolor{currentfill}{rgb}{0.333333,0.333333,0.333333}%
\pgfsetfillcolor{currentfill}%
\pgfsetfillopacity{0.400000}%
\pgfsetlinewidth{0.000000pt}%
\definecolor{currentstroke}{rgb}{0.000000,0.000000,0.000000}%
\pgfsetstrokecolor{currentstroke}%
\pgfsetstrokeopacity{0.400000}%
\pgfsetdash{}{0pt}%
\pgfpathmoveto{\pgfqpoint{0.575222in}{0.383578in}}%
\pgfpathlineto{\pgfqpoint{0.659090in}{0.383578in}}%
\pgfpathlineto{\pgfqpoint{0.659090in}{0.459713in}}%
\pgfpathlineto{\pgfqpoint{0.575222in}{0.459713in}}%
\pgfpathclose%
\pgfusepath{fill}%
\end{pgfscope}%
\begin{pgfscope}%
\pgfpathrectangle{\pgfqpoint{0.439481in}{0.383578in}}{\pgfqpoint{1.550000in}{1.540000in}}%
\pgfusepath{clip}%
\pgfsetbuttcap%
\pgfsetmiterjoin%
\definecolor{currentfill}{rgb}{0.333333,0.333333,0.333333}%
\pgfsetfillcolor{currentfill}%
\pgfsetfillopacity{0.400000}%
\pgfsetlinewidth{0.000000pt}%
\definecolor{currentstroke}{rgb}{0.000000,0.000000,0.000000}%
\pgfsetstrokecolor{currentstroke}%
\pgfsetstrokeopacity{0.400000}%
\pgfsetdash{}{0pt}%
\pgfpathmoveto{\pgfqpoint{0.659090in}{0.383578in}}%
\pgfpathlineto{\pgfqpoint{0.742959in}{0.383578in}}%
\pgfpathlineto{\pgfqpoint{0.742959in}{0.472402in}}%
\pgfpathlineto{\pgfqpoint{0.659090in}{0.472402in}}%
\pgfpathclose%
\pgfusepath{fill}%
\end{pgfscope}%
\begin{pgfscope}%
\pgfpathrectangle{\pgfqpoint{0.439481in}{0.383578in}}{\pgfqpoint{1.550000in}{1.540000in}}%
\pgfusepath{clip}%
\pgfsetbuttcap%
\pgfsetmiterjoin%
\definecolor{currentfill}{rgb}{0.333333,0.333333,0.333333}%
\pgfsetfillcolor{currentfill}%
\pgfsetfillopacity{0.400000}%
\pgfsetlinewidth{0.000000pt}%
\definecolor{currentstroke}{rgb}{0.000000,0.000000,0.000000}%
\pgfsetstrokecolor{currentstroke}%
\pgfsetstrokeopacity{0.400000}%
\pgfsetdash{}{0pt}%
\pgfpathmoveto{\pgfqpoint{0.742959in}{0.383578in}}%
\pgfpathlineto{\pgfqpoint{0.826827in}{0.383578in}}%
\pgfpathlineto{\pgfqpoint{0.826827in}{0.580260in}}%
\pgfpathlineto{\pgfqpoint{0.742959in}{0.580260in}}%
\pgfpathclose%
\pgfusepath{fill}%
\end{pgfscope}%
\begin{pgfscope}%
\pgfpathrectangle{\pgfqpoint{0.439481in}{0.383578in}}{\pgfqpoint{1.550000in}{1.540000in}}%
\pgfusepath{clip}%
\pgfsetbuttcap%
\pgfsetmiterjoin%
\definecolor{currentfill}{rgb}{0.333333,0.333333,0.333333}%
\pgfsetfillcolor{currentfill}%
\pgfsetfillopacity{0.400000}%
\pgfsetlinewidth{0.000000pt}%
\definecolor{currentstroke}{rgb}{0.000000,0.000000,0.000000}%
\pgfsetstrokecolor{currentstroke}%
\pgfsetstrokeopacity{0.400000}%
\pgfsetdash{}{0pt}%
\pgfpathmoveto{\pgfqpoint{0.826827in}{0.383578in}}%
\pgfpathlineto{\pgfqpoint{0.910696in}{0.383578in}}%
\pgfpathlineto{\pgfqpoint{0.910696in}{0.478747in}}%
\pgfpathlineto{\pgfqpoint{0.826827in}{0.478747in}}%
\pgfpathclose%
\pgfusepath{fill}%
\end{pgfscope}%
\begin{pgfscope}%
\pgfpathrectangle{\pgfqpoint{0.439481in}{0.383578in}}{\pgfqpoint{1.550000in}{1.540000in}}%
\pgfusepath{clip}%
\pgfsetbuttcap%
\pgfsetmiterjoin%
\definecolor{currentfill}{rgb}{0.333333,0.333333,0.333333}%
\pgfsetfillcolor{currentfill}%
\pgfsetfillopacity{0.400000}%
\pgfsetlinewidth{0.000000pt}%
\definecolor{currentstroke}{rgb}{0.000000,0.000000,0.000000}%
\pgfsetstrokecolor{currentstroke}%
\pgfsetstrokeopacity{0.400000}%
\pgfsetdash{}{0pt}%
\pgfpathmoveto{\pgfqpoint{0.910696in}{0.383578in}}%
\pgfpathlineto{\pgfqpoint{0.994564in}{0.383578in}}%
\pgfpathlineto{\pgfqpoint{0.994564in}{0.554881in}}%
\pgfpathlineto{\pgfqpoint{0.910696in}{0.554881in}}%
\pgfpathclose%
\pgfusepath{fill}%
\end{pgfscope}%
\begin{pgfscope}%
\pgfpathrectangle{\pgfqpoint{0.439481in}{0.383578in}}{\pgfqpoint{1.550000in}{1.540000in}}%
\pgfusepath{clip}%
\pgfsetbuttcap%
\pgfsetmiterjoin%
\definecolor{currentfill}{rgb}{0.333333,0.333333,0.333333}%
\pgfsetfillcolor{currentfill}%
\pgfsetfillopacity{0.400000}%
\pgfsetlinewidth{0.000000pt}%
\definecolor{currentstroke}{rgb}{0.000000,0.000000,0.000000}%
\pgfsetstrokecolor{currentstroke}%
\pgfsetstrokeopacity{0.400000}%
\pgfsetdash{}{0pt}%
\pgfpathmoveto{\pgfqpoint{0.994564in}{0.383578in}}%
\pgfpathlineto{\pgfqpoint{1.078433in}{0.383578in}}%
\pgfpathlineto{\pgfqpoint{1.078433in}{0.459713in}}%
\pgfpathlineto{\pgfqpoint{0.994564in}{0.459713in}}%
\pgfpathclose%
\pgfusepath{fill}%
\end{pgfscope}%
\begin{pgfscope}%
\pgfpathrectangle{\pgfqpoint{0.439481in}{0.383578in}}{\pgfqpoint{1.550000in}{1.540000in}}%
\pgfusepath{clip}%
\pgfsetbuttcap%
\pgfsetmiterjoin%
\definecolor{currentfill}{rgb}{0.333333,0.333333,0.333333}%
\pgfsetfillcolor{currentfill}%
\pgfsetfillopacity{0.400000}%
\pgfsetlinewidth{0.000000pt}%
\definecolor{currentstroke}{rgb}{0.000000,0.000000,0.000000}%
\pgfsetstrokecolor{currentstroke}%
\pgfsetstrokeopacity{0.400000}%
\pgfsetdash{}{0pt}%
\pgfpathmoveto{\pgfqpoint{1.078433in}{0.383578in}}%
\pgfpathlineto{\pgfqpoint{1.162301in}{0.383578in}}%
\pgfpathlineto{\pgfqpoint{1.162301in}{0.447024in}}%
\pgfpathlineto{\pgfqpoint{1.078433in}{0.447024in}}%
\pgfpathclose%
\pgfusepath{fill}%
\end{pgfscope}%
\begin{pgfscope}%
\pgfpathrectangle{\pgfqpoint{0.439481in}{0.383578in}}{\pgfqpoint{1.550000in}{1.540000in}}%
\pgfusepath{clip}%
\pgfsetbuttcap%
\pgfsetmiterjoin%
\definecolor{currentfill}{rgb}{0.333333,0.333333,0.333333}%
\pgfsetfillcolor{currentfill}%
\pgfsetfillopacity{0.400000}%
\pgfsetlinewidth{0.000000pt}%
\definecolor{currentstroke}{rgb}{0.000000,0.000000,0.000000}%
\pgfsetstrokecolor{currentstroke}%
\pgfsetstrokeopacity{0.400000}%
\pgfsetdash{}{0pt}%
\pgfpathmoveto{\pgfqpoint{1.162301in}{0.383578in}}%
\pgfpathlineto{\pgfqpoint{1.246170in}{0.383578in}}%
\pgfpathlineto{\pgfqpoint{1.246170in}{0.396267in}}%
\pgfpathlineto{\pgfqpoint{1.162301in}{0.396267in}}%
\pgfpathclose%
\pgfusepath{fill}%
\end{pgfscope}%
\begin{pgfscope}%
\pgfpathrectangle{\pgfqpoint{0.439481in}{0.383578in}}{\pgfqpoint{1.550000in}{1.540000in}}%
\pgfusepath{clip}%
\pgfsetbuttcap%
\pgfsetmiterjoin%
\definecolor{currentfill}{rgb}{0.686275,0.352941,0.313725}%
\pgfsetfillcolor{currentfill}%
\pgfsetfillopacity{0.400000}%
\pgfsetlinewidth{0.000000pt}%
\definecolor{currentstroke}{rgb}{0.000000,0.000000,0.000000}%
\pgfsetstrokecolor{currentstroke}%
\pgfsetstrokeopacity{0.400000}%
\pgfsetdash{}{0pt}%
\pgfpathmoveto{\pgfqpoint{0.821795in}{0.383578in}}%
\pgfpathlineto{\pgfqpoint{0.854085in}{0.383578in}}%
\pgfpathlineto{\pgfqpoint{0.854085in}{0.531893in}}%
\pgfpathlineto{\pgfqpoint{0.821795in}{0.531893in}}%
\pgfpathclose%
\pgfusepath{fill}%
\end{pgfscope}%
\begin{pgfscope}%
\pgfpathrectangle{\pgfqpoint{0.439481in}{0.383578in}}{\pgfqpoint{1.550000in}{1.540000in}}%
\pgfusepath{clip}%
\pgfsetbuttcap%
\pgfsetmiterjoin%
\definecolor{currentfill}{rgb}{0.686275,0.352941,0.313725}%
\pgfsetfillcolor{currentfill}%
\pgfsetfillopacity{0.400000}%
\pgfsetlinewidth{0.000000pt}%
\definecolor{currentstroke}{rgb}{0.000000,0.000000,0.000000}%
\pgfsetstrokecolor{currentstroke}%
\pgfsetstrokeopacity{0.400000}%
\pgfsetdash{}{0pt}%
\pgfpathmoveto{\pgfqpoint{0.854085in}{0.383578in}}%
\pgfpathlineto{\pgfqpoint{0.886374in}{0.383578in}}%
\pgfpathlineto{\pgfqpoint{0.886374in}{1.570095in}}%
\pgfpathlineto{\pgfqpoint{0.854085in}{1.570095in}}%
\pgfpathclose%
\pgfusepath{fill}%
\end{pgfscope}%
\begin{pgfscope}%
\pgfpathrectangle{\pgfqpoint{0.439481in}{0.383578in}}{\pgfqpoint{1.550000in}{1.540000in}}%
\pgfusepath{clip}%
\pgfsetbuttcap%
\pgfsetmiterjoin%
\definecolor{currentfill}{rgb}{0.686275,0.352941,0.313725}%
\pgfsetfillcolor{currentfill}%
\pgfsetfillopacity{0.400000}%
\pgfsetlinewidth{0.000000pt}%
\definecolor{currentstroke}{rgb}{0.000000,0.000000,0.000000}%
\pgfsetstrokecolor{currentstroke}%
\pgfsetstrokeopacity{0.400000}%
\pgfsetdash{}{0pt}%
\pgfpathmoveto{\pgfqpoint{0.886374in}{0.383578in}}%
\pgfpathlineto{\pgfqpoint{0.918663in}{0.383578in}}%
\pgfpathlineto{\pgfqpoint{0.918663in}{1.108672in}}%
\pgfpathlineto{\pgfqpoint{0.886374in}{1.108672in}}%
\pgfpathclose%
\pgfusepath{fill}%
\end{pgfscope}%
\begin{pgfscope}%
\pgfpathrectangle{\pgfqpoint{0.439481in}{0.383578in}}{\pgfqpoint{1.550000in}{1.540000in}}%
\pgfusepath{clip}%
\pgfsetbuttcap%
\pgfsetmiterjoin%
\definecolor{currentfill}{rgb}{0.686275,0.352941,0.313725}%
\pgfsetfillcolor{currentfill}%
\pgfsetfillopacity{0.400000}%
\pgfsetlinewidth{0.000000pt}%
\definecolor{currentstroke}{rgb}{0.000000,0.000000,0.000000}%
\pgfsetstrokecolor{currentstroke}%
\pgfsetstrokeopacity{0.400000}%
\pgfsetdash{}{0pt}%
\pgfpathmoveto{\pgfqpoint{0.918663in}{0.383578in}}%
\pgfpathlineto{\pgfqpoint{0.950953in}{0.383578in}}%
\pgfpathlineto{\pgfqpoint{0.950953in}{0.433016in}}%
\pgfpathlineto{\pgfqpoint{0.918663in}{0.433016in}}%
\pgfpathclose%
\pgfusepath{fill}%
\end{pgfscope}%
\begin{pgfscope}%
\pgfpathrectangle{\pgfqpoint{0.439481in}{0.383578in}}{\pgfqpoint{1.550000in}{1.540000in}}%
\pgfusepath{clip}%
\pgfsetbuttcap%
\pgfsetmiterjoin%
\definecolor{currentfill}{rgb}{0.000000,0.356863,0.509804}%
\pgfsetfillcolor{currentfill}%
\pgfsetfillopacity{0.400000}%
\pgfsetlinewidth{0.000000pt}%
\definecolor{currentstroke}{rgb}{0.000000,0.000000,0.000000}%
\pgfsetstrokecolor{currentstroke}%
\pgfsetstrokeopacity{0.400000}%
\pgfsetdash{}{0pt}%
\pgfpathmoveto{\pgfqpoint{0.545029in}{0.383578in}}%
\pgfpathlineto{\pgfqpoint{0.633930in}{0.383578in}}%
\pgfpathlineto{\pgfqpoint{0.633930in}{0.395549in}}%
\pgfpathlineto{\pgfqpoint{0.545029in}{0.395549in}}%
\pgfpathclose%
\pgfusepath{fill}%
\end{pgfscope}%
\begin{pgfscope}%
\pgfpathrectangle{\pgfqpoint{0.439481in}{0.383578in}}{\pgfqpoint{1.550000in}{1.540000in}}%
\pgfusepath{clip}%
\pgfsetbuttcap%
\pgfsetmiterjoin%
\definecolor{currentfill}{rgb}{0.000000,0.356863,0.509804}%
\pgfsetfillcolor{currentfill}%
\pgfsetfillopacity{0.400000}%
\pgfsetlinewidth{0.000000pt}%
\definecolor{currentstroke}{rgb}{0.000000,0.000000,0.000000}%
\pgfsetstrokecolor{currentstroke}%
\pgfsetstrokeopacity{0.400000}%
\pgfsetdash{}{0pt}%
\pgfpathmoveto{\pgfqpoint{0.633930in}{0.383578in}}%
\pgfpathlineto{\pgfqpoint{0.722830in}{0.383578in}}%
\pgfpathlineto{\pgfqpoint{0.722830in}{0.395549in}}%
\pgfpathlineto{\pgfqpoint{0.633930in}{0.395549in}}%
\pgfpathclose%
\pgfusepath{fill}%
\end{pgfscope}%
\begin{pgfscope}%
\pgfpathrectangle{\pgfqpoint{0.439481in}{0.383578in}}{\pgfqpoint{1.550000in}{1.540000in}}%
\pgfusepath{clip}%
\pgfsetbuttcap%
\pgfsetmiterjoin%
\definecolor{currentfill}{rgb}{0.000000,0.356863,0.509804}%
\pgfsetfillcolor{currentfill}%
\pgfsetfillopacity{0.400000}%
\pgfsetlinewidth{0.000000pt}%
\definecolor{currentstroke}{rgb}{0.000000,0.000000,0.000000}%
\pgfsetstrokecolor{currentstroke}%
\pgfsetstrokeopacity{0.400000}%
\pgfsetdash{}{0pt}%
\pgfpathmoveto{\pgfqpoint{0.722830in}{0.383578in}}%
\pgfpathlineto{\pgfqpoint{0.811731in}{0.383578in}}%
\pgfpathlineto{\pgfqpoint{0.811731in}{0.395549in}}%
\pgfpathlineto{\pgfqpoint{0.722830in}{0.395549in}}%
\pgfpathclose%
\pgfusepath{fill}%
\end{pgfscope}%
\begin{pgfscope}%
\pgfpathrectangle{\pgfqpoint{0.439481in}{0.383578in}}{\pgfqpoint{1.550000in}{1.540000in}}%
\pgfusepath{clip}%
\pgfsetbuttcap%
\pgfsetmiterjoin%
\definecolor{currentfill}{rgb}{0.000000,0.356863,0.509804}%
\pgfsetfillcolor{currentfill}%
\pgfsetfillopacity{0.400000}%
\pgfsetlinewidth{0.000000pt}%
\definecolor{currentstroke}{rgb}{0.000000,0.000000,0.000000}%
\pgfsetstrokecolor{currentstroke}%
\pgfsetstrokeopacity{0.400000}%
\pgfsetdash{}{0pt}%
\pgfpathmoveto{\pgfqpoint{0.811731in}{0.383578in}}%
\pgfpathlineto{\pgfqpoint{0.900632in}{0.383578in}}%
\pgfpathlineto{\pgfqpoint{0.900632in}{0.437447in}}%
\pgfpathlineto{\pgfqpoint{0.811731in}{0.437447in}}%
\pgfpathclose%
\pgfusepath{fill}%
\end{pgfscope}%
\begin{pgfscope}%
\pgfpathrectangle{\pgfqpoint{0.439481in}{0.383578in}}{\pgfqpoint{1.550000in}{1.540000in}}%
\pgfusepath{clip}%
\pgfsetbuttcap%
\pgfsetmiterjoin%
\definecolor{currentfill}{rgb}{0.000000,0.356863,0.509804}%
\pgfsetfillcolor{currentfill}%
\pgfsetfillopacity{0.400000}%
\pgfsetlinewidth{0.000000pt}%
\definecolor{currentstroke}{rgb}{0.000000,0.000000,0.000000}%
\pgfsetstrokecolor{currentstroke}%
\pgfsetstrokeopacity{0.400000}%
\pgfsetdash{}{0pt}%
\pgfpathmoveto{\pgfqpoint{0.900632in}{0.383578in}}%
\pgfpathlineto{\pgfqpoint{0.989532in}{0.383578in}}%
\pgfpathlineto{\pgfqpoint{0.989532in}{0.515258in}}%
\pgfpathlineto{\pgfqpoint{0.900632in}{0.515258in}}%
\pgfpathclose%
\pgfusepath{fill}%
\end{pgfscope}%
\begin{pgfscope}%
\pgfpathrectangle{\pgfqpoint{0.439481in}{0.383578in}}{\pgfqpoint{1.550000in}{1.540000in}}%
\pgfusepath{clip}%
\pgfsetbuttcap%
\pgfsetmiterjoin%
\definecolor{currentfill}{rgb}{0.000000,0.356863,0.509804}%
\pgfsetfillcolor{currentfill}%
\pgfsetfillopacity{0.400000}%
\pgfsetlinewidth{0.000000pt}%
\definecolor{currentstroke}{rgb}{0.000000,0.000000,0.000000}%
\pgfsetstrokecolor{currentstroke}%
\pgfsetstrokeopacity{0.400000}%
\pgfsetdash{}{0pt}%
\pgfpathmoveto{\pgfqpoint{0.989532in}{0.383578in}}%
\pgfpathlineto{\pgfqpoint{1.078433in}{0.383578in}}%
\pgfpathlineto{\pgfqpoint{1.078433in}{0.497301in}}%
\pgfpathlineto{\pgfqpoint{0.989532in}{0.497301in}}%
\pgfpathclose%
\pgfusepath{fill}%
\end{pgfscope}%
\begin{pgfscope}%
\pgfpathrectangle{\pgfqpoint{0.439481in}{0.383578in}}{\pgfqpoint{1.550000in}{1.540000in}}%
\pgfusepath{clip}%
\pgfsetbuttcap%
\pgfsetmiterjoin%
\definecolor{currentfill}{rgb}{0.000000,0.356863,0.509804}%
\pgfsetfillcolor{currentfill}%
\pgfsetfillopacity{0.400000}%
\pgfsetlinewidth{0.000000pt}%
\definecolor{currentstroke}{rgb}{0.000000,0.000000,0.000000}%
\pgfsetstrokecolor{currentstroke}%
\pgfsetstrokeopacity{0.400000}%
\pgfsetdash{}{0pt}%
\pgfpathmoveto{\pgfqpoint{1.078433in}{0.383578in}}%
\pgfpathlineto{\pgfqpoint{1.167334in}{0.383578in}}%
\pgfpathlineto{\pgfqpoint{1.167334in}{0.491316in}}%
\pgfpathlineto{\pgfqpoint{1.078433in}{0.491316in}}%
\pgfpathclose%
\pgfusepath{fill}%
\end{pgfscope}%
\begin{pgfscope}%
\pgfpathrectangle{\pgfqpoint{0.439481in}{0.383578in}}{\pgfqpoint{1.550000in}{1.540000in}}%
\pgfusepath{clip}%
\pgfsetbuttcap%
\pgfsetmiterjoin%
\definecolor{currentfill}{rgb}{0.000000,0.356863,0.509804}%
\pgfsetfillcolor{currentfill}%
\pgfsetfillopacity{0.400000}%
\pgfsetlinewidth{0.000000pt}%
\definecolor{currentstroke}{rgb}{0.000000,0.000000,0.000000}%
\pgfsetstrokecolor{currentstroke}%
\pgfsetstrokeopacity{0.400000}%
\pgfsetdash{}{0pt}%
\pgfpathmoveto{\pgfqpoint{1.167334in}{0.383578in}}%
\pgfpathlineto{\pgfqpoint{1.256234in}{0.383578in}}%
\pgfpathlineto{\pgfqpoint{1.256234in}{0.551170in}}%
\pgfpathlineto{\pgfqpoint{1.167334in}{0.551170in}}%
\pgfpathclose%
\pgfusepath{fill}%
\end{pgfscope}%
\begin{pgfscope}%
\pgfpathrectangle{\pgfqpoint{0.439481in}{0.383578in}}{\pgfqpoint{1.550000in}{1.540000in}}%
\pgfusepath{clip}%
\pgfsetbuttcap%
\pgfsetmiterjoin%
\definecolor{currentfill}{rgb}{0.000000,0.356863,0.509804}%
\pgfsetfillcolor{currentfill}%
\pgfsetfillopacity{0.400000}%
\pgfsetlinewidth{0.000000pt}%
\definecolor{currentstroke}{rgb}{0.000000,0.000000,0.000000}%
\pgfsetstrokecolor{currentstroke}%
\pgfsetstrokeopacity{0.400000}%
\pgfsetdash{}{0pt}%
\pgfpathmoveto{\pgfqpoint{1.256234in}{0.383578in}}%
\pgfpathlineto{\pgfqpoint{1.345135in}{0.383578in}}%
\pgfpathlineto{\pgfqpoint{1.345135in}{0.467374in}}%
\pgfpathlineto{\pgfqpoint{1.256234in}{0.467374in}}%
\pgfpathclose%
\pgfusepath{fill}%
\end{pgfscope}%
\begin{pgfscope}%
\pgfpathrectangle{\pgfqpoint{0.439481in}{0.383578in}}{\pgfqpoint{1.550000in}{1.540000in}}%
\pgfusepath{clip}%
\pgfsetbuttcap%
\pgfsetmiterjoin%
\definecolor{currentfill}{rgb}{0.000000,0.356863,0.509804}%
\pgfsetfillcolor{currentfill}%
\pgfsetfillopacity{0.400000}%
\pgfsetlinewidth{0.000000pt}%
\definecolor{currentstroke}{rgb}{0.000000,0.000000,0.000000}%
\pgfsetstrokecolor{currentstroke}%
\pgfsetstrokeopacity{0.400000}%
\pgfsetdash{}{0pt}%
\pgfpathmoveto{\pgfqpoint{1.345135in}{0.383578in}}%
\pgfpathlineto{\pgfqpoint{1.434035in}{0.383578in}}%
\pgfpathlineto{\pgfqpoint{1.434035in}{0.443432in}}%
\pgfpathlineto{\pgfqpoint{1.345135in}{0.443432in}}%
\pgfpathclose%
\pgfusepath{fill}%
\end{pgfscope}%
\begin{pgfscope}%
\pgfpathrectangle{\pgfqpoint{0.439481in}{0.383578in}}{\pgfqpoint{1.550000in}{1.540000in}}%
\pgfusepath{clip}%
\pgfsetbuttcap%
\pgfsetmiterjoin%
\definecolor{currentfill}{rgb}{0.000000,0.356863,0.509804}%
\pgfsetfillcolor{currentfill}%
\pgfsetfillopacity{0.400000}%
\pgfsetlinewidth{0.000000pt}%
\definecolor{currentstroke}{rgb}{0.000000,0.000000,0.000000}%
\pgfsetstrokecolor{currentstroke}%
\pgfsetstrokeopacity{0.400000}%
\pgfsetdash{}{0pt}%
\pgfpathmoveto{\pgfqpoint{1.434035in}{0.383578in}}%
\pgfpathlineto{\pgfqpoint{1.522936in}{0.383578in}}%
\pgfpathlineto{\pgfqpoint{1.522936in}{0.395549in}}%
\pgfpathlineto{\pgfqpoint{1.434035in}{0.395549in}}%
\pgfpathclose%
\pgfusepath{fill}%
\end{pgfscope}%
\begin{pgfscope}%
\pgfpathrectangle{\pgfqpoint{0.439481in}{0.383578in}}{\pgfqpoint{1.550000in}{1.540000in}}%
\pgfusepath{clip}%
\pgfsetbuttcap%
\pgfsetmiterjoin%
\definecolor{currentfill}{rgb}{0.490196,0.588235,0.431373}%
\pgfsetfillcolor{currentfill}%
\pgfsetfillopacity{0.400000}%
\pgfsetlinewidth{0.000000pt}%
\definecolor{currentstroke}{rgb}{0.000000,0.000000,0.000000}%
\pgfsetstrokecolor{currentstroke}%
\pgfsetstrokeopacity{0.400000}%
\pgfsetdash{}{0pt}%
\pgfpathmoveto{\pgfqpoint{1.043208in}{0.383578in}}%
\pgfpathlineto{\pgfqpoint{1.075498in}{0.383578in}}%
\pgfpathlineto{\pgfqpoint{1.075498in}{0.581331in}}%
\pgfpathlineto{\pgfqpoint{1.043208in}{0.581331in}}%
\pgfpathclose%
\pgfusepath{fill}%
\end{pgfscope}%
\begin{pgfscope}%
\pgfpathrectangle{\pgfqpoint{0.439481in}{0.383578in}}{\pgfqpoint{1.550000in}{1.540000in}}%
\pgfusepath{clip}%
\pgfsetbuttcap%
\pgfsetmiterjoin%
\definecolor{currentfill}{rgb}{0.490196,0.588235,0.431373}%
\pgfsetfillcolor{currentfill}%
\pgfsetfillopacity{0.400000}%
\pgfsetlinewidth{0.000000pt}%
\definecolor{currentstroke}{rgb}{0.000000,0.000000,0.000000}%
\pgfsetstrokecolor{currentstroke}%
\pgfsetstrokeopacity{0.400000}%
\pgfsetdash{}{0pt}%
\pgfpathmoveto{\pgfqpoint{1.075498in}{0.383578in}}%
\pgfpathlineto{\pgfqpoint{1.107787in}{0.383578in}}%
\pgfpathlineto{\pgfqpoint{1.107787in}{1.850245in}}%
\pgfpathlineto{\pgfqpoint{1.075498in}{1.850245in}}%
\pgfpathclose%
\pgfusepath{fill}%
\end{pgfscope}%
\begin{pgfscope}%
\pgfpathrectangle{\pgfqpoint{0.439481in}{0.383578in}}{\pgfqpoint{1.550000in}{1.540000in}}%
\pgfusepath{clip}%
\pgfsetbuttcap%
\pgfsetmiterjoin%
\definecolor{currentfill}{rgb}{0.490196,0.588235,0.431373}%
\pgfsetfillcolor{currentfill}%
\pgfsetfillopacity{0.400000}%
\pgfsetlinewidth{0.000000pt}%
\definecolor{currentstroke}{rgb}{0.000000,0.000000,0.000000}%
\pgfsetstrokecolor{currentstroke}%
\pgfsetstrokeopacity{0.400000}%
\pgfsetdash{}{0pt}%
\pgfpathmoveto{\pgfqpoint{1.107787in}{0.383578in}}%
\pgfpathlineto{\pgfqpoint{1.140076in}{0.383578in}}%
\pgfpathlineto{\pgfqpoint{1.140076in}{0.762604in}}%
\pgfpathlineto{\pgfqpoint{1.107787in}{0.762604in}}%
\pgfpathclose%
\pgfusepath{fill}%
\end{pgfscope}%
\begin{pgfscope}%
\pgfpathrectangle{\pgfqpoint{0.439481in}{0.383578in}}{\pgfqpoint{1.550000in}{1.540000in}}%
\pgfusepath{clip}%
\pgfsetbuttcap%
\pgfsetmiterjoin%
\definecolor{currentfill}{rgb}{0.490196,0.588235,0.431373}%
\pgfsetfillcolor{currentfill}%
\pgfsetfillopacity{0.400000}%
\pgfsetlinewidth{0.000000pt}%
\definecolor{currentstroke}{rgb}{0.000000,0.000000,0.000000}%
\pgfsetstrokecolor{currentstroke}%
\pgfsetstrokeopacity{0.400000}%
\pgfsetdash{}{0pt}%
\pgfpathmoveto{\pgfqpoint{1.140076in}{0.383578in}}%
\pgfpathlineto{\pgfqpoint{1.172366in}{0.383578in}}%
\pgfpathlineto{\pgfqpoint{1.172366in}{0.449496in}}%
\pgfpathlineto{\pgfqpoint{1.140076in}{0.449496in}}%
\pgfpathclose%
\pgfusepath{fill}%
\end{pgfscope}%
\begin{pgfscope}%
\pgfsetrectcap%
\pgfsetmiterjoin%
\pgfsetlinewidth{0.501875pt}%
\definecolor{currentstroke}{rgb}{0.317647,0.317647,0.317647}%
\pgfsetstrokecolor{currentstroke}%
\pgfsetdash{}{0pt}%
\pgfpathmoveto{\pgfqpoint{0.439481in}{0.383578in}}%
\pgfpathlineto{\pgfqpoint{0.439481in}{1.923578in}}%
\pgfusepath{stroke}%
\end{pgfscope}%
\begin{pgfscope}%
\pgfsetrectcap%
\pgfsetmiterjoin%
\pgfsetlinewidth{0.501875pt}%
\definecolor{currentstroke}{rgb}{0.317647,0.317647,0.317647}%
\pgfsetstrokecolor{currentstroke}%
\pgfsetdash{}{0pt}%
\pgfpathmoveto{\pgfqpoint{0.439481in}{0.383578in}}%
\pgfpathlineto{\pgfqpoint{1.989481in}{0.383578in}}%
\pgfusepath{stroke}%
\end{pgfscope}%
\begin{pgfscope}%
\pgfsetbuttcap%
\pgfsetmiterjoin%
\definecolor{currentfill}{rgb}{0.333333,0.333333,0.333333}%
\pgfsetfillcolor{currentfill}%
\pgfsetfillopacity{0.400000}%
\pgfsetlinewidth{0.000000pt}%
\definecolor{currentstroke}{rgb}{0.000000,0.000000,0.000000}%
\pgfsetstrokecolor{currentstroke}%
\pgfsetstrokeopacity{0.400000}%
\pgfsetdash{}{0pt}%
\pgfpathmoveto{\pgfqpoint{1.320643in}{1.831022in}}%
\pgfpathlineto{\pgfqpoint{1.394688in}{1.831022in}}%
\pgfpathlineto{\pgfqpoint{1.394688in}{1.895811in}}%
\pgfpathlineto{\pgfqpoint{1.320643in}{1.895811in}}%
\pgfpathclose%
\pgfusepath{fill}%
\end{pgfscope}%
\begin{pgfscope}%
\definecolor{textcolor}{rgb}{0.000000,0.000000,0.000000}%
\pgfsetstrokecolor{textcolor}%
\pgfsetfillcolor{textcolor}%
\pgftext[x=1.440966in,y=1.831022in,left,base]{\color{textcolor}\rmfamily\fontsize{6.664000}{7.996800}\selectfont pre \(\displaystyle \SI{0.75}{\V}\)}%
\end{pgfscope}%
\begin{pgfscope}%
\pgfsetbuttcap%
\pgfsetmiterjoin%
\definecolor{currentfill}{rgb}{0.686275,0.352941,0.313725}%
\pgfsetfillcolor{currentfill}%
\pgfsetfillopacity{0.400000}%
\pgfsetlinewidth{0.000000pt}%
\definecolor{currentstroke}{rgb}{0.000000,0.000000,0.000000}%
\pgfsetstrokecolor{currentstroke}%
\pgfsetstrokeopacity{0.400000}%
\pgfsetdash{}{0pt}%
\pgfpathmoveto{\pgfqpoint{1.320643in}{1.711256in}}%
\pgfpathlineto{\pgfqpoint{1.394688in}{1.711256in}}%
\pgfpathlineto{\pgfqpoint{1.394688in}{1.776045in}}%
\pgfpathlineto{\pgfqpoint{1.320643in}{1.776045in}}%
\pgfpathclose%
\pgfusepath{fill}%
\end{pgfscope}%
\begin{pgfscope}%
\definecolor{textcolor}{rgb}{0.000000,0.000000,0.000000}%
\pgfsetstrokecolor{textcolor}%
\pgfsetfillcolor{textcolor}%
\pgftext[x=1.440966in,y=1.711256in,left,base]{\color{textcolor}\rmfamily\fontsize{6.664000}{7.996800}\selectfont post \(\displaystyle \SI{0.75}{\V}\)}%
\end{pgfscope}%
\begin{pgfscope}%
\pgfsetbuttcap%
\pgfsetmiterjoin%
\definecolor{currentfill}{rgb}{0.000000,0.356863,0.509804}%
\pgfsetfillcolor{currentfill}%
\pgfsetfillopacity{0.400000}%
\pgfsetlinewidth{0.000000pt}%
\definecolor{currentstroke}{rgb}{0.000000,0.000000,0.000000}%
\pgfsetstrokecolor{currentstroke}%
\pgfsetstrokeopacity{0.400000}%
\pgfsetdash{}{0pt}%
\pgfpathmoveto{\pgfqpoint{1.320643in}{1.591489in}}%
\pgfpathlineto{\pgfqpoint{1.394688in}{1.591489in}}%
\pgfpathlineto{\pgfqpoint{1.394688in}{1.656278in}}%
\pgfpathlineto{\pgfqpoint{1.320643in}{1.656278in}}%
\pgfpathclose%
\pgfusepath{fill}%
\end{pgfscope}%
\begin{pgfscope}%
\definecolor{textcolor}{rgb}{0.000000,0.000000,0.000000}%
\pgfsetstrokecolor{textcolor}%
\pgfsetfillcolor{textcolor}%
\pgftext[x=1.440966in,y=1.591489in,left,base]{\color{textcolor}\rmfamily\fontsize{6.664000}{7.996800}\selectfont pre \(\displaystyle \SI{0.8}{\V}\)}%
\end{pgfscope}%
\begin{pgfscope}%
\pgfsetbuttcap%
\pgfsetmiterjoin%
\definecolor{currentfill}{rgb}{0.490196,0.588235,0.431373}%
\pgfsetfillcolor{currentfill}%
\pgfsetfillopacity{0.400000}%
\pgfsetlinewidth{0.000000pt}%
\definecolor{currentstroke}{rgb}{0.000000,0.000000,0.000000}%
\pgfsetstrokecolor{currentstroke}%
\pgfsetstrokeopacity{0.400000}%
\pgfsetdash{}{0pt}%
\pgfpathmoveto{\pgfqpoint{1.320643in}{1.471722in}}%
\pgfpathlineto{\pgfqpoint{1.394688in}{1.471722in}}%
\pgfpathlineto{\pgfqpoint{1.394688in}{1.536511in}}%
\pgfpathlineto{\pgfqpoint{1.320643in}{1.536511in}}%
\pgfpathclose%
\pgfusepath{fill}%
\end{pgfscope}%
\begin{pgfscope}%
\definecolor{textcolor}{rgb}{0.000000,0.000000,0.000000}%
\pgfsetstrokecolor{textcolor}%
\pgfsetfillcolor{textcolor}%
\pgftext[x=1.440966in,y=1.471722in,left,base]{\color{textcolor}\rmfamily\fontsize{6.664000}{7.996800}\selectfont post \(\displaystyle \SI{0.8}{\V}\)}%
\end{pgfscope}%
\end{pgfpicture}%
\makeatother%
\endgroup%

	\label{hxprepostvthreshold}
\end{subfigure}
	\caption[Pre and post calibration state of the analog \gls{lif} parameters.]{Pre and post calibration state of the analog \gls{lif} parameters. (\subref{precadccalib}) The raw cadc data of an controlled voltage ranging from 0 to \SI{1.2}{\V}. (\subref{postcadccalib}) the cadc parameters are manually adjusted, such that they cover a useful dynamic range. The offset per channel can then be easily computed and corrected. The manual method was preferred over an automated fit-routine, since the ramp and slope parameters showed a sensitive cross-dependency in certain areas.}
	\label{cadccalibration2}
\end{figure}


\begin{itemize}
	\item vsyn
	\item cadc
	\item vleak
	\item vreset
	\item vthreshold
\end{itemize}



\subsection{Implementation on BSS-2 Platform}



\begin{itemize}
	\item CADC readout of membrane potential
	\item BLACKBOX
	\item HOST implementation of superspike (short)!
	\item calibration of tausyn, vsyn, cadc, vleak, vreset, vthreshold
	\item optimization: c++ speed up using the haldls frame
	\item Backprop weights
	\item FA weights
	\item future plans for mnist (maybe erst in discussion erwähnen)
\end{itemize}

\begin{figure}
	\centering
	%% Creator: Matplotlib, PGF backend
%%
%% To include the figure in your LaTeX document, write
%%   \input{<filename>.pgf}
%%
%% Make sure the required packages are loaded in your preamble
%%   \usepackage{pgf}
%%
%% Figures using additional raster images can only be included by \input if
%% they are in the same directory as the main LaTeX file. For loading figures
%% from other directories you can use the `import` package
%%   \usepackage{import}
%% and then include the figures with
%%   \import{<path to file>}{<filename>.pgf}
%%
%% Matplotlib used the following preamble
%%   \usepackage{amsmath} \usepackage{pifont} \usepackage{xcolor} \definecolor{green}{HTML}{467821} \definecolor{red}{HTML}{CF4457} \usepackage[detect-all]{siunitx}
%%   \usepackage{fontspec}
%%
\begingroup%
\makeatletter%
\begin{pgfpicture}%
\pgfpathrectangle{\pgfpointorigin}{\pgfqpoint{5.299282in}{2.819907in}}%
\pgfusepath{use as bounding box, clip}%
\begin{pgfscope}%
\pgfsetbuttcap%
\pgfsetmiterjoin%
\pgfsetlinewidth{0.000000pt}%
\definecolor{currentstroke}{rgb}{0.000000,0.000000,0.000000}%
\pgfsetstrokecolor{currentstroke}%
\pgfsetstrokeopacity{0.000000}%
\pgfsetdash{}{0pt}%
\pgfpathmoveto{\pgfqpoint{0.000000in}{0.000000in}}%
\pgfpathlineto{\pgfqpoint{5.299282in}{0.000000in}}%
\pgfpathlineto{\pgfqpoint{5.299282in}{2.819907in}}%
\pgfpathlineto{\pgfqpoint{0.000000in}{2.819907in}}%
\pgfpathclose%
\pgfusepath{}%
\end{pgfscope}%
\begin{pgfscope}%
\pgfsetbuttcap%
\pgfsetmiterjoin%
\pgfsetlinewidth{0.000000pt}%
\definecolor{currentstroke}{rgb}{0.000000,0.000000,0.000000}%
\pgfsetstrokecolor{currentstroke}%
\pgfsetstrokeopacity{0.000000}%
\pgfsetdash{}{0pt}%
\pgfpathmoveto{\pgfqpoint{0.438556in}{0.383578in}}%
\pgfpathlineto{\pgfqpoint{5.088556in}{0.383578in}}%
\pgfpathlineto{\pgfqpoint{5.088556in}{2.693578in}}%
\pgfpathlineto{\pgfqpoint{0.438556in}{2.693578in}}%
\pgfpathclose%
\pgfusepath{}%
\end{pgfscope}%
\begin{pgfscope}%
\pgfsetbuttcap%
\pgfsetroundjoin%
\definecolor{currentfill}{rgb}{0.317647,0.317647,0.317647}%
\pgfsetfillcolor{currentfill}%
\pgfsetlinewidth{0.501875pt}%
\definecolor{currentstroke}{rgb}{0.317647,0.317647,0.317647}%
\pgfsetstrokecolor{currentstroke}%
\pgfsetdash{}{0pt}%
\pgfsys@defobject{currentmarker}{\pgfqpoint{0.000000in}{-0.020833in}}{\pgfqpoint{0.000000in}{0.000000in}}{%
\pgfpathmoveto{\pgfqpoint{0.000000in}{0.000000in}}%
\pgfpathlineto{\pgfqpoint{0.000000in}{-0.020833in}}%
\pgfusepath{stroke,fill}%
}%
\begin{pgfscope}%
\pgfsys@transformshift{0.507275in}{0.383578in}%
\pgfsys@useobject{currentmarker}{}%
\end{pgfscope}%
\end{pgfscope}%
\begin{pgfscope}%
\definecolor{textcolor}{rgb}{0.317647,0.317647,0.317647}%
\pgfsetstrokecolor{textcolor}%
\pgfsetfillcolor{textcolor}%
\pgftext[x=0.507275in,y=0.334967in,,top]{\color{textcolor}\rmfamily\fontsize{6.664000}{7.996800}\selectfont \(\displaystyle 0\)}%
\end{pgfscope}%
\begin{pgfscope}%
\pgfsetbuttcap%
\pgfsetroundjoin%
\definecolor{currentfill}{rgb}{0.317647,0.317647,0.317647}%
\pgfsetfillcolor{currentfill}%
\pgfsetlinewidth{0.501875pt}%
\definecolor{currentstroke}{rgb}{0.317647,0.317647,0.317647}%
\pgfsetstrokecolor{currentstroke}%
\pgfsetdash{}{0pt}%
\pgfsys@defobject{currentmarker}{\pgfqpoint{0.000000in}{-0.020833in}}{\pgfqpoint{0.000000in}{0.000000in}}{%
\pgfpathmoveto{\pgfqpoint{0.000000in}{0.000000in}}%
\pgfpathlineto{\pgfqpoint{0.000000in}{-0.020833in}}%
\pgfusepath{stroke,fill}%
}%
\begin{pgfscope}%
\pgfsys@transformshift{1.079935in}{0.383578in}%
\pgfsys@useobject{currentmarker}{}%
\end{pgfscope}%
\end{pgfscope}%
\begin{pgfscope}%
\definecolor{textcolor}{rgb}{0.317647,0.317647,0.317647}%
\pgfsetstrokecolor{textcolor}%
\pgfsetfillcolor{textcolor}%
\pgftext[x=1.079935in,y=0.334967in,,top]{\color{textcolor}\rmfamily\fontsize{6.664000}{7.996800}\selectfont \(\displaystyle 250\)}%
\end{pgfscope}%
\begin{pgfscope}%
\pgfsetbuttcap%
\pgfsetroundjoin%
\definecolor{currentfill}{rgb}{0.317647,0.317647,0.317647}%
\pgfsetfillcolor{currentfill}%
\pgfsetlinewidth{0.501875pt}%
\definecolor{currentstroke}{rgb}{0.317647,0.317647,0.317647}%
\pgfsetstrokecolor{currentstroke}%
\pgfsetdash{}{0pt}%
\pgfsys@defobject{currentmarker}{\pgfqpoint{0.000000in}{-0.020833in}}{\pgfqpoint{0.000000in}{0.000000in}}{%
\pgfpathmoveto{\pgfqpoint{0.000000in}{0.000000in}}%
\pgfpathlineto{\pgfqpoint{0.000000in}{-0.020833in}}%
\pgfusepath{stroke,fill}%
}%
\begin{pgfscope}%
\pgfsys@transformshift{1.652595in}{0.383578in}%
\pgfsys@useobject{currentmarker}{}%
\end{pgfscope}%
\end{pgfscope}%
\begin{pgfscope}%
\definecolor{textcolor}{rgb}{0.317647,0.317647,0.317647}%
\pgfsetstrokecolor{textcolor}%
\pgfsetfillcolor{textcolor}%
\pgftext[x=1.652595in,y=0.334967in,,top]{\color{textcolor}\rmfamily\fontsize{6.664000}{7.996800}\selectfont \(\displaystyle 500\)}%
\end{pgfscope}%
\begin{pgfscope}%
\pgfsetbuttcap%
\pgfsetroundjoin%
\definecolor{currentfill}{rgb}{0.317647,0.317647,0.317647}%
\pgfsetfillcolor{currentfill}%
\pgfsetlinewidth{0.501875pt}%
\definecolor{currentstroke}{rgb}{0.317647,0.317647,0.317647}%
\pgfsetstrokecolor{currentstroke}%
\pgfsetdash{}{0pt}%
\pgfsys@defobject{currentmarker}{\pgfqpoint{0.000000in}{-0.020833in}}{\pgfqpoint{0.000000in}{0.000000in}}{%
\pgfpathmoveto{\pgfqpoint{0.000000in}{0.000000in}}%
\pgfpathlineto{\pgfqpoint{0.000000in}{-0.020833in}}%
\pgfusepath{stroke,fill}%
}%
\begin{pgfscope}%
\pgfsys@transformshift{2.225255in}{0.383578in}%
\pgfsys@useobject{currentmarker}{}%
\end{pgfscope}%
\end{pgfscope}%
\begin{pgfscope}%
\definecolor{textcolor}{rgb}{0.317647,0.317647,0.317647}%
\pgfsetstrokecolor{textcolor}%
\pgfsetfillcolor{textcolor}%
\pgftext[x=2.225255in,y=0.334967in,,top]{\color{textcolor}\rmfamily\fontsize{6.664000}{7.996800}\selectfont \(\displaystyle 750\)}%
\end{pgfscope}%
\begin{pgfscope}%
\pgfsetbuttcap%
\pgfsetroundjoin%
\definecolor{currentfill}{rgb}{0.317647,0.317647,0.317647}%
\pgfsetfillcolor{currentfill}%
\pgfsetlinewidth{0.501875pt}%
\definecolor{currentstroke}{rgb}{0.317647,0.317647,0.317647}%
\pgfsetstrokecolor{currentstroke}%
\pgfsetdash{}{0pt}%
\pgfsys@defobject{currentmarker}{\pgfqpoint{0.000000in}{-0.020833in}}{\pgfqpoint{0.000000in}{0.000000in}}{%
\pgfpathmoveto{\pgfqpoint{0.000000in}{0.000000in}}%
\pgfpathlineto{\pgfqpoint{0.000000in}{-0.020833in}}%
\pgfusepath{stroke,fill}%
}%
\begin{pgfscope}%
\pgfsys@transformshift{2.797915in}{0.383578in}%
\pgfsys@useobject{currentmarker}{}%
\end{pgfscope}%
\end{pgfscope}%
\begin{pgfscope}%
\definecolor{textcolor}{rgb}{0.317647,0.317647,0.317647}%
\pgfsetstrokecolor{textcolor}%
\pgfsetfillcolor{textcolor}%
\pgftext[x=2.797915in,y=0.334967in,,top]{\color{textcolor}\rmfamily\fontsize{6.664000}{7.996800}\selectfont \(\displaystyle 1000\)}%
\end{pgfscope}%
\begin{pgfscope}%
\pgfsetbuttcap%
\pgfsetroundjoin%
\definecolor{currentfill}{rgb}{0.317647,0.317647,0.317647}%
\pgfsetfillcolor{currentfill}%
\pgfsetlinewidth{0.501875pt}%
\definecolor{currentstroke}{rgb}{0.317647,0.317647,0.317647}%
\pgfsetstrokecolor{currentstroke}%
\pgfsetdash{}{0pt}%
\pgfsys@defobject{currentmarker}{\pgfqpoint{0.000000in}{-0.020833in}}{\pgfqpoint{0.000000in}{0.000000in}}{%
\pgfpathmoveto{\pgfqpoint{0.000000in}{0.000000in}}%
\pgfpathlineto{\pgfqpoint{0.000000in}{-0.020833in}}%
\pgfusepath{stroke,fill}%
}%
\begin{pgfscope}%
\pgfsys@transformshift{3.370575in}{0.383578in}%
\pgfsys@useobject{currentmarker}{}%
\end{pgfscope}%
\end{pgfscope}%
\begin{pgfscope}%
\definecolor{textcolor}{rgb}{0.317647,0.317647,0.317647}%
\pgfsetstrokecolor{textcolor}%
\pgfsetfillcolor{textcolor}%
\pgftext[x=3.370575in,y=0.334967in,,top]{\color{textcolor}\rmfamily\fontsize{6.664000}{7.996800}\selectfont \(\displaystyle 1250\)}%
\end{pgfscope}%
\begin{pgfscope}%
\pgfsetbuttcap%
\pgfsetroundjoin%
\definecolor{currentfill}{rgb}{0.317647,0.317647,0.317647}%
\pgfsetfillcolor{currentfill}%
\pgfsetlinewidth{0.501875pt}%
\definecolor{currentstroke}{rgb}{0.317647,0.317647,0.317647}%
\pgfsetstrokecolor{currentstroke}%
\pgfsetdash{}{0pt}%
\pgfsys@defobject{currentmarker}{\pgfqpoint{0.000000in}{-0.020833in}}{\pgfqpoint{0.000000in}{0.000000in}}{%
\pgfpathmoveto{\pgfqpoint{0.000000in}{0.000000in}}%
\pgfpathlineto{\pgfqpoint{0.000000in}{-0.020833in}}%
\pgfusepath{stroke,fill}%
}%
\begin{pgfscope}%
\pgfsys@transformshift{3.943235in}{0.383578in}%
\pgfsys@useobject{currentmarker}{}%
\end{pgfscope}%
\end{pgfscope}%
\begin{pgfscope}%
\definecolor{textcolor}{rgb}{0.317647,0.317647,0.317647}%
\pgfsetstrokecolor{textcolor}%
\pgfsetfillcolor{textcolor}%
\pgftext[x=3.943235in,y=0.334967in,,top]{\color{textcolor}\rmfamily\fontsize{6.664000}{7.996800}\selectfont \(\displaystyle 1500\)}%
\end{pgfscope}%
\begin{pgfscope}%
\pgfsetbuttcap%
\pgfsetroundjoin%
\definecolor{currentfill}{rgb}{0.317647,0.317647,0.317647}%
\pgfsetfillcolor{currentfill}%
\pgfsetlinewidth{0.501875pt}%
\definecolor{currentstroke}{rgb}{0.317647,0.317647,0.317647}%
\pgfsetstrokecolor{currentstroke}%
\pgfsetdash{}{0pt}%
\pgfsys@defobject{currentmarker}{\pgfqpoint{0.000000in}{-0.020833in}}{\pgfqpoint{0.000000in}{0.000000in}}{%
\pgfpathmoveto{\pgfqpoint{0.000000in}{0.000000in}}%
\pgfpathlineto{\pgfqpoint{0.000000in}{-0.020833in}}%
\pgfusepath{stroke,fill}%
}%
\begin{pgfscope}%
\pgfsys@transformshift{4.515896in}{0.383578in}%
\pgfsys@useobject{currentmarker}{}%
\end{pgfscope}%
\end{pgfscope}%
\begin{pgfscope}%
\definecolor{textcolor}{rgb}{0.317647,0.317647,0.317647}%
\pgfsetstrokecolor{textcolor}%
\pgfsetfillcolor{textcolor}%
\pgftext[x=4.515896in,y=0.334967in,,top]{\color{textcolor}\rmfamily\fontsize{6.664000}{7.996800}\selectfont \(\displaystyle 1750\)}%
\end{pgfscope}%
\begin{pgfscope}%
\pgfsetbuttcap%
\pgfsetroundjoin%
\definecolor{currentfill}{rgb}{0.317647,0.317647,0.317647}%
\pgfsetfillcolor{currentfill}%
\pgfsetlinewidth{0.501875pt}%
\definecolor{currentstroke}{rgb}{0.317647,0.317647,0.317647}%
\pgfsetstrokecolor{currentstroke}%
\pgfsetdash{}{0pt}%
\pgfsys@defobject{currentmarker}{\pgfqpoint{0.000000in}{-0.020833in}}{\pgfqpoint{0.000000in}{0.000000in}}{%
\pgfpathmoveto{\pgfqpoint{0.000000in}{0.000000in}}%
\pgfpathlineto{\pgfqpoint{0.000000in}{-0.020833in}}%
\pgfusepath{stroke,fill}%
}%
\begin{pgfscope}%
\pgfsys@transformshift{5.088556in}{0.383578in}%
\pgfsys@useobject{currentmarker}{}%
\end{pgfscope}%
\end{pgfscope}%
\begin{pgfscope}%
\definecolor{textcolor}{rgb}{0.317647,0.317647,0.317647}%
\pgfsetstrokecolor{textcolor}%
\pgfsetfillcolor{textcolor}%
\pgftext[x=5.088556in,y=0.334967in,,top]{\color{textcolor}\rmfamily\fontsize{6.664000}{7.996800}\selectfont \(\displaystyle 2000\)}%
\end{pgfscope}%
\begin{pgfscope}%
\definecolor{textcolor}{rgb}{0.317647,0.317647,0.317647}%
\pgfsetstrokecolor{textcolor}%
\pgfsetfillcolor{textcolor}%
\pgftext[x=2.763556in,y=0.197222in,,top]{\color{textcolor}\rmfamily\fontsize{6.664000}{7.996800}\selectfont spike time \(\displaystyle (\si{\micro \s})\)}%
\end{pgfscope}%
\begin{pgfscope}%
\pgfsetbuttcap%
\pgfsetroundjoin%
\definecolor{currentfill}{rgb}{0.317647,0.317647,0.317647}%
\pgfsetfillcolor{currentfill}%
\pgfsetlinewidth{0.501875pt}%
\definecolor{currentstroke}{rgb}{0.317647,0.317647,0.317647}%
\pgfsetstrokecolor{currentstroke}%
\pgfsetdash{}{0pt}%
\pgfsys@defobject{currentmarker}{\pgfqpoint{-0.020833in}{0.000000in}}{\pgfqpoint{0.000000in}{0.000000in}}{%
\pgfpathmoveto{\pgfqpoint{0.000000in}{0.000000in}}%
\pgfpathlineto{\pgfqpoint{-0.020833in}{0.000000in}}%
\pgfusepath{stroke,fill}%
}%
\begin{pgfscope}%
\pgfsys@transformshift{0.438556in}{0.503577in}%
\pgfsys@useobject{currentmarker}{}%
\end{pgfscope}%
\end{pgfscope}%
\begin{pgfscope}%
\definecolor{textcolor}{rgb}{0.317647,0.317647,0.317647}%
\pgfsetstrokecolor{textcolor}%
\pgfsetfillcolor{textcolor}%
\pgftext[x=0.348471in,y=0.471460in,left,base]{\color{textcolor}\rmfamily\fontsize{6.664000}{7.996800}\selectfont \(\displaystyle 0\)}%
\end{pgfscope}%
\begin{pgfscope}%
\pgfsetbuttcap%
\pgfsetroundjoin%
\definecolor{currentfill}{rgb}{0.317647,0.317647,0.317647}%
\pgfsetfillcolor{currentfill}%
\pgfsetlinewidth{0.501875pt}%
\definecolor{currentstroke}{rgb}{0.317647,0.317647,0.317647}%
\pgfsetstrokecolor{currentstroke}%
\pgfsetdash{}{0pt}%
\pgfsys@defobject{currentmarker}{\pgfqpoint{-0.020833in}{0.000000in}}{\pgfqpoint{0.000000in}{0.000000in}}{%
\pgfpathmoveto{\pgfqpoint{0.000000in}{0.000000in}}%
\pgfpathlineto{\pgfqpoint{-0.020833in}{0.000000in}}%
\pgfusepath{stroke,fill}%
}%
\begin{pgfscope}%
\pgfsys@transformshift{0.438556in}{0.940419in}%
\pgfsys@useobject{currentmarker}{}%
\end{pgfscope}%
\end{pgfscope}%
\begin{pgfscope}%
\definecolor{textcolor}{rgb}{0.317647,0.317647,0.317647}%
\pgfsetstrokecolor{textcolor}%
\pgfsetfillcolor{textcolor}%
\pgftext[x=0.293108in,y=0.908302in,left,base]{\color{textcolor}\rmfamily\fontsize{6.664000}{7.996800}\selectfont \(\displaystyle 20\)}%
\end{pgfscope}%
\begin{pgfscope}%
\pgfsetbuttcap%
\pgfsetroundjoin%
\definecolor{currentfill}{rgb}{0.317647,0.317647,0.317647}%
\pgfsetfillcolor{currentfill}%
\pgfsetlinewidth{0.501875pt}%
\definecolor{currentstroke}{rgb}{0.317647,0.317647,0.317647}%
\pgfsetstrokecolor{currentstroke}%
\pgfsetdash{}{0pt}%
\pgfsys@defobject{currentmarker}{\pgfqpoint{-0.020833in}{0.000000in}}{\pgfqpoint{0.000000in}{0.000000in}}{%
\pgfpathmoveto{\pgfqpoint{0.000000in}{0.000000in}}%
\pgfpathlineto{\pgfqpoint{-0.020833in}{0.000000in}}%
\pgfusepath{stroke,fill}%
}%
\begin{pgfscope}%
\pgfsys@transformshift{0.438556in}{1.377262in}%
\pgfsys@useobject{currentmarker}{}%
\end{pgfscope}%
\end{pgfscope}%
\begin{pgfscope}%
\definecolor{textcolor}{rgb}{0.317647,0.317647,0.317647}%
\pgfsetstrokecolor{textcolor}%
\pgfsetfillcolor{textcolor}%
\pgftext[x=0.293108in,y=1.345145in,left,base]{\color{textcolor}\rmfamily\fontsize{6.664000}{7.996800}\selectfont \(\displaystyle 40\)}%
\end{pgfscope}%
\begin{pgfscope}%
\pgfsetbuttcap%
\pgfsetroundjoin%
\definecolor{currentfill}{rgb}{0.317647,0.317647,0.317647}%
\pgfsetfillcolor{currentfill}%
\pgfsetlinewidth{0.501875pt}%
\definecolor{currentstroke}{rgb}{0.317647,0.317647,0.317647}%
\pgfsetstrokecolor{currentstroke}%
\pgfsetdash{}{0pt}%
\pgfsys@defobject{currentmarker}{\pgfqpoint{-0.020833in}{0.000000in}}{\pgfqpoint{0.000000in}{0.000000in}}{%
\pgfpathmoveto{\pgfqpoint{0.000000in}{0.000000in}}%
\pgfpathlineto{\pgfqpoint{-0.020833in}{0.000000in}}%
\pgfusepath{stroke,fill}%
}%
\begin{pgfscope}%
\pgfsys@transformshift{0.438556in}{1.814104in}%
\pgfsys@useobject{currentmarker}{}%
\end{pgfscope}%
\end{pgfscope}%
\begin{pgfscope}%
\definecolor{textcolor}{rgb}{0.317647,0.317647,0.317647}%
\pgfsetstrokecolor{textcolor}%
\pgfsetfillcolor{textcolor}%
\pgftext[x=0.293108in,y=1.781988in,left,base]{\color{textcolor}\rmfamily\fontsize{6.664000}{7.996800}\selectfont \(\displaystyle 60\)}%
\end{pgfscope}%
\begin{pgfscope}%
\pgfsetbuttcap%
\pgfsetroundjoin%
\definecolor{currentfill}{rgb}{0.317647,0.317647,0.317647}%
\pgfsetfillcolor{currentfill}%
\pgfsetlinewidth{0.501875pt}%
\definecolor{currentstroke}{rgb}{0.317647,0.317647,0.317647}%
\pgfsetstrokecolor{currentstroke}%
\pgfsetdash{}{0pt}%
\pgfsys@defobject{currentmarker}{\pgfqpoint{-0.020833in}{0.000000in}}{\pgfqpoint{0.000000in}{0.000000in}}{%
\pgfpathmoveto{\pgfqpoint{0.000000in}{0.000000in}}%
\pgfpathlineto{\pgfqpoint{-0.020833in}{0.000000in}}%
\pgfusepath{stroke,fill}%
}%
\begin{pgfscope}%
\pgfsys@transformshift{0.438556in}{2.250947in}%
\pgfsys@useobject{currentmarker}{}%
\end{pgfscope}%
\end{pgfscope}%
\begin{pgfscope}%
\definecolor{textcolor}{rgb}{0.317647,0.317647,0.317647}%
\pgfsetstrokecolor{textcolor}%
\pgfsetfillcolor{textcolor}%
\pgftext[x=0.293108in,y=2.218830in,left,base]{\color{textcolor}\rmfamily\fontsize{6.664000}{7.996800}\selectfont \(\displaystyle 80\)}%
\end{pgfscope}%
\begin{pgfscope}%
\pgfsetbuttcap%
\pgfsetroundjoin%
\definecolor{currentfill}{rgb}{0.317647,0.317647,0.317647}%
\pgfsetfillcolor{currentfill}%
\pgfsetlinewidth{0.501875pt}%
\definecolor{currentstroke}{rgb}{0.317647,0.317647,0.317647}%
\pgfsetstrokecolor{currentstroke}%
\pgfsetdash{}{0pt}%
\pgfsys@defobject{currentmarker}{\pgfqpoint{-0.020833in}{0.000000in}}{\pgfqpoint{0.000000in}{0.000000in}}{%
\pgfpathmoveto{\pgfqpoint{0.000000in}{0.000000in}}%
\pgfpathlineto{\pgfqpoint{-0.020833in}{0.000000in}}%
\pgfusepath{stroke,fill}%
}%
\begin{pgfscope}%
\pgfsys@transformshift{0.438556in}{2.687790in}%
\pgfsys@useobject{currentmarker}{}%
\end{pgfscope}%
\end{pgfscope}%
\begin{pgfscope}%
\definecolor{textcolor}{rgb}{0.317647,0.317647,0.317647}%
\pgfsetstrokecolor{textcolor}%
\pgfsetfillcolor{textcolor}%
\pgftext[x=0.237745in,y=2.655673in,left,base]{\color{textcolor}\rmfamily\fontsize{6.664000}{7.996800}\selectfont \(\displaystyle 100\)}%
\end{pgfscope}%
\begin{pgfscope}%
\definecolor{textcolor}{rgb}{0.317647,0.317647,0.317647}%
\pgfsetstrokecolor{textcolor}%
\pgfsetfillcolor{textcolor}%
\pgftext[x=0.182189in,y=1.538578in,,bottom,rotate=90.000000]{\color{textcolor}\rmfamily\fontsize{6.664000}{7.996800}\selectfont input unit}%
\end{pgfscope}%
\begin{pgfscope}%
\pgfpathrectangle{\pgfqpoint{0.438556in}{0.383578in}}{\pgfqpoint{4.650000in}{2.310000in}}%
\pgfusepath{clip}%
\pgfsetbuttcap%
\pgfsetroundjoin%
\pgfsetlinewidth{0.803000pt}%
\definecolor{currentstroke}{rgb}{0.333333,0.333333,0.333333}%
\pgfsetstrokecolor{currentstroke}%
\pgfsetdash{}{0pt}%
\pgfpathmoveto{\pgfqpoint{0.517629in}{0.628694in}}%
\pgfpathcurveto{\pgfqpoint{0.524995in}{0.628694in}}{\pgfqpoint{0.532061in}{0.631621in}}{\pgfqpoint{0.537270in}{0.636830in}}%
\pgfpathcurveto{\pgfqpoint{0.542480in}{0.642039in}}{\pgfqpoint{0.545406in}{0.649105in}}{\pgfqpoint{0.545406in}{0.656471in}}%
\pgfpathcurveto{\pgfqpoint{0.545406in}{0.663838in}}{\pgfqpoint{0.542480in}{0.670904in}}{\pgfqpoint{0.537270in}{0.676113in}}%
\pgfpathcurveto{\pgfqpoint{0.532061in}{0.681322in}}{\pgfqpoint{0.524995in}{0.684249in}}{\pgfqpoint{0.517629in}{0.684249in}}%
\pgfpathcurveto{\pgfqpoint{0.510262in}{0.684249in}}{\pgfqpoint{0.503196in}{0.681322in}}{\pgfqpoint{0.497987in}{0.676113in}}%
\pgfpathcurveto{\pgfqpoint{0.492778in}{0.670904in}}{\pgfqpoint{0.489851in}{0.663838in}}{\pgfqpoint{0.489851in}{0.656471in}}%
\pgfpathcurveto{\pgfqpoint{0.489851in}{0.649105in}}{\pgfqpoint{0.492778in}{0.642039in}}{\pgfqpoint{0.497987in}{0.636830in}}%
\pgfpathcurveto{\pgfqpoint{0.503196in}{0.631621in}}{\pgfqpoint{0.510262in}{0.628694in}}{\pgfqpoint{0.517629in}{0.628694in}}%
\pgfpathclose%
\pgfusepath{stroke}%
\end{pgfscope}%
\begin{pgfscope}%
\pgfpathrectangle{\pgfqpoint{0.438556in}{0.383578in}}{\pgfqpoint{4.650000in}{2.310000in}}%
\pgfusepath{clip}%
\pgfsetbuttcap%
\pgfsetroundjoin%
\pgfsetlinewidth{0.803000pt}%
\definecolor{currentstroke}{rgb}{0.333333,0.333333,0.333333}%
\pgfsetstrokecolor{currentstroke}%
\pgfsetdash{}{0pt}%
\pgfpathmoveto{\pgfqpoint{0.544640in}{0.847115in}}%
\pgfpathcurveto{\pgfqpoint{0.552007in}{0.847115in}}{\pgfqpoint{0.559073in}{0.850042in}}{\pgfqpoint{0.564282in}{0.855251in}}%
\pgfpathcurveto{\pgfqpoint{0.569491in}{0.860460in}}{\pgfqpoint{0.572418in}{0.867526in}}{\pgfqpoint{0.572418in}{0.874893in}}%
\pgfpathcurveto{\pgfqpoint{0.572418in}{0.882260in}}{\pgfqpoint{0.569491in}{0.889326in}}{\pgfqpoint{0.564282in}{0.894535in}}%
\pgfpathcurveto{\pgfqpoint{0.559073in}{0.899744in}}{\pgfqpoint{0.552007in}{0.902671in}}{\pgfqpoint{0.544640in}{0.902671in}}%
\pgfpathcurveto{\pgfqpoint{0.537273in}{0.902671in}}{\pgfqpoint{0.530207in}{0.899744in}}{\pgfqpoint{0.524998in}{0.894535in}}%
\pgfpathcurveto{\pgfqpoint{0.519789in}{0.889326in}}{\pgfqpoint{0.516862in}{0.882260in}}{\pgfqpoint{0.516862in}{0.874893in}}%
\pgfpathcurveto{\pgfqpoint{0.516862in}{0.867526in}}{\pgfqpoint{0.519789in}{0.860460in}}{\pgfqpoint{0.524998in}{0.855251in}}%
\pgfpathcurveto{\pgfqpoint{0.530207in}{0.850042in}}{\pgfqpoint{0.537273in}{0.847115in}}{\pgfqpoint{0.544640in}{0.847115in}}%
\pgfpathclose%
\pgfusepath{stroke}%
\end{pgfscope}%
\begin{pgfscope}%
\pgfpathrectangle{\pgfqpoint{0.438556in}{0.383578in}}{\pgfqpoint{4.650000in}{2.310000in}}%
\pgfusepath{clip}%
\pgfsetbuttcap%
\pgfsetroundjoin%
\pgfsetlinewidth{0.803000pt}%
\definecolor{currentstroke}{rgb}{0.333333,0.333333,0.333333}%
\pgfsetstrokecolor{currentstroke}%
\pgfsetdash{}{0pt}%
\pgfpathmoveto{\pgfqpoint{0.595116in}{0.868957in}}%
\pgfpathcurveto{\pgfqpoint{0.602483in}{0.868957in}}{\pgfqpoint{0.609549in}{0.871884in}}{\pgfqpoint{0.614758in}{0.877093in}}%
\pgfpathcurveto{\pgfqpoint{0.619967in}{0.882302in}}{\pgfqpoint{0.622894in}{0.889368in}}{\pgfqpoint{0.622894in}{0.896735in}}%
\pgfpathcurveto{\pgfqpoint{0.622894in}{0.904102in}}{\pgfqpoint{0.619967in}{0.911168in}}{\pgfqpoint{0.614758in}{0.916377in}}%
\pgfpathcurveto{\pgfqpoint{0.609549in}{0.921586in}}{\pgfqpoint{0.602483in}{0.924513in}}{\pgfqpoint{0.595116in}{0.924513in}}%
\pgfpathcurveto{\pgfqpoint{0.587750in}{0.924513in}}{\pgfqpoint{0.580684in}{0.921586in}}{\pgfqpoint{0.575475in}{0.916377in}}%
\pgfpathcurveto{\pgfqpoint{0.570265in}{0.911168in}}{\pgfqpoint{0.567339in}{0.904102in}}{\pgfqpoint{0.567339in}{0.896735in}}%
\pgfpathcurveto{\pgfqpoint{0.567339in}{0.889368in}}{\pgfqpoint{0.570265in}{0.882302in}}{\pgfqpoint{0.575475in}{0.877093in}}%
\pgfpathcurveto{\pgfqpoint{0.580684in}{0.871884in}}{\pgfqpoint{0.587750in}{0.868957in}}{\pgfqpoint{0.595116in}{0.868957in}}%
\pgfpathclose%
\pgfusepath{stroke}%
\end{pgfscope}%
\begin{pgfscope}%
\pgfpathrectangle{\pgfqpoint{0.438556in}{0.383578in}}{\pgfqpoint{4.650000in}{2.310000in}}%
\pgfusepath{clip}%
\pgfsetbuttcap%
\pgfsetroundjoin%
\pgfsetlinewidth{0.803000pt}%
\definecolor{currentstroke}{rgb}{0.333333,0.333333,0.333333}%
\pgfsetstrokecolor{currentstroke}%
\pgfsetdash{}{0pt}%
\pgfpathmoveto{\pgfqpoint{0.574162in}{1.000010in}}%
\pgfpathcurveto{\pgfqpoint{0.581528in}{1.000010in}}{\pgfqpoint{0.588594in}{1.002937in}}{\pgfqpoint{0.593803in}{1.008146in}}%
\pgfpathcurveto{\pgfqpoint{0.599013in}{1.013355in}}{\pgfqpoint{0.601939in}{1.020421in}}{\pgfqpoint{0.601939in}{1.027788in}}%
\pgfpathcurveto{\pgfqpoint{0.601939in}{1.035154in}}{\pgfqpoint{0.599013in}{1.042220in}}{\pgfqpoint{0.593803in}{1.047430in}}%
\pgfpathcurveto{\pgfqpoint{0.588594in}{1.052639in}}{\pgfqpoint{0.581528in}{1.055565in}}{\pgfqpoint{0.574162in}{1.055565in}}%
\pgfpathcurveto{\pgfqpoint{0.566795in}{1.055565in}}{\pgfqpoint{0.559729in}{1.052639in}}{\pgfqpoint{0.554520in}{1.047430in}}%
\pgfpathcurveto{\pgfqpoint{0.549311in}{1.042220in}}{\pgfqpoint{0.546384in}{1.035154in}}{\pgfqpoint{0.546384in}{1.027788in}}%
\pgfpathcurveto{\pgfqpoint{0.546384in}{1.020421in}}{\pgfqpoint{0.549311in}{1.013355in}}{\pgfqpoint{0.554520in}{1.008146in}}%
\pgfpathcurveto{\pgfqpoint{0.559729in}{1.002937in}}{\pgfqpoint{0.566795in}{1.000010in}}{\pgfqpoint{0.574162in}{1.000010in}}%
\pgfpathclose%
\pgfusepath{stroke}%
\end{pgfscope}%
\begin{pgfscope}%
\pgfpathrectangle{\pgfqpoint{0.438556in}{0.383578in}}{\pgfqpoint{4.650000in}{2.310000in}}%
\pgfusepath{clip}%
\pgfsetbuttcap%
\pgfsetroundjoin%
\pgfsetlinewidth{0.803000pt}%
\definecolor{currentstroke}{rgb}{0.333333,0.333333,0.333333}%
\pgfsetstrokecolor{currentstroke}%
\pgfsetdash{}{0pt}%
\pgfpathmoveto{\pgfqpoint{0.539234in}{1.021852in}}%
\pgfpathcurveto{\pgfqpoint{0.546601in}{1.021852in}}{\pgfqpoint{0.553667in}{1.024779in}}{\pgfqpoint{0.558876in}{1.029988in}}%
\pgfpathcurveto{\pgfqpoint{0.564085in}{1.035197in}}{\pgfqpoint{0.567012in}{1.042263in}}{\pgfqpoint{0.567012in}{1.049630in}}%
\pgfpathcurveto{\pgfqpoint{0.567012in}{1.056997in}}{\pgfqpoint{0.564085in}{1.064063in}}{\pgfqpoint{0.558876in}{1.069272in}}%
\pgfpathcurveto{\pgfqpoint{0.553667in}{1.074481in}}{\pgfqpoint{0.546601in}{1.077408in}}{\pgfqpoint{0.539234in}{1.077408in}}%
\pgfpathcurveto{\pgfqpoint{0.531867in}{1.077408in}}{\pgfqpoint{0.524801in}{1.074481in}}{\pgfqpoint{0.519592in}{1.069272in}}%
\pgfpathcurveto{\pgfqpoint{0.514383in}{1.064063in}}{\pgfqpoint{0.511456in}{1.056997in}}{\pgfqpoint{0.511456in}{1.049630in}}%
\pgfpathcurveto{\pgfqpoint{0.511456in}{1.042263in}}{\pgfqpoint{0.514383in}{1.035197in}}{\pgfqpoint{0.519592in}{1.029988in}}%
\pgfpathcurveto{\pgfqpoint{0.524801in}{1.024779in}}{\pgfqpoint{0.531867in}{1.021852in}}{\pgfqpoint{0.539234in}{1.021852in}}%
\pgfpathclose%
\pgfusepath{stroke}%
\end{pgfscope}%
\begin{pgfscope}%
\pgfpathrectangle{\pgfqpoint{0.438556in}{0.383578in}}{\pgfqpoint{4.650000in}{2.310000in}}%
\pgfusepath{clip}%
\pgfsetbuttcap%
\pgfsetroundjoin%
\pgfsetlinewidth{0.803000pt}%
\definecolor{currentstroke}{rgb}{0.333333,0.333333,0.333333}%
\pgfsetstrokecolor{currentstroke}%
\pgfsetdash{}{0pt}%
\pgfpathmoveto{\pgfqpoint{0.536339in}{1.196589in}}%
\pgfpathcurveto{\pgfqpoint{0.543705in}{1.196589in}}{\pgfqpoint{0.550771in}{1.199516in}}{\pgfqpoint{0.555980in}{1.204725in}}%
\pgfpathcurveto{\pgfqpoint{0.561189in}{1.209934in}}{\pgfqpoint{0.564116in}{1.217000in}}{\pgfqpoint{0.564116in}{1.224367in}}%
\pgfpathcurveto{\pgfqpoint{0.564116in}{1.231734in}}{\pgfqpoint{0.561189in}{1.238800in}}{\pgfqpoint{0.555980in}{1.244009in}}%
\pgfpathcurveto{\pgfqpoint{0.550771in}{1.249218in}}{\pgfqpoint{0.543705in}{1.252145in}}{\pgfqpoint{0.536339in}{1.252145in}}%
\pgfpathcurveto{\pgfqpoint{0.528972in}{1.252145in}}{\pgfqpoint{0.521906in}{1.249218in}}{\pgfqpoint{0.516697in}{1.244009in}}%
\pgfpathcurveto{\pgfqpoint{0.511488in}{1.238800in}}{\pgfqpoint{0.508561in}{1.231734in}}{\pgfqpoint{0.508561in}{1.224367in}}%
\pgfpathcurveto{\pgfqpoint{0.508561in}{1.217000in}}{\pgfqpoint{0.511488in}{1.209934in}}{\pgfqpoint{0.516697in}{1.204725in}}%
\pgfpathcurveto{\pgfqpoint{0.521906in}{1.199516in}}{\pgfqpoint{0.528972in}{1.196589in}}{\pgfqpoint{0.536339in}{1.196589in}}%
\pgfpathclose%
\pgfusepath{stroke}%
\end{pgfscope}%
\begin{pgfscope}%
\pgfpathrectangle{\pgfqpoint{0.438556in}{0.383578in}}{\pgfqpoint{4.650000in}{2.310000in}}%
\pgfusepath{clip}%
\pgfsetbuttcap%
\pgfsetroundjoin%
\pgfsetlinewidth{0.803000pt}%
\definecolor{currentstroke}{rgb}{0.333333,0.333333,0.333333}%
\pgfsetstrokecolor{currentstroke}%
\pgfsetdash{}{0pt}%
\pgfpathmoveto{\pgfqpoint{0.530181in}{1.283958in}}%
\pgfpathcurveto{\pgfqpoint{0.537548in}{1.283958in}}{\pgfqpoint{0.544614in}{1.286884in}}{\pgfqpoint{0.549823in}{1.292094in}}%
\pgfpathcurveto{\pgfqpoint{0.555032in}{1.297303in}}{\pgfqpoint{0.557959in}{1.304369in}}{\pgfqpoint{0.557959in}{1.311735in}}%
\pgfpathcurveto{\pgfqpoint{0.557959in}{1.319102in}}{\pgfqpoint{0.555032in}{1.326168in}}{\pgfqpoint{0.549823in}{1.331377in}}%
\pgfpathcurveto{\pgfqpoint{0.544614in}{1.336586in}}{\pgfqpoint{0.537548in}{1.339513in}}{\pgfqpoint{0.530181in}{1.339513in}}%
\pgfpathcurveto{\pgfqpoint{0.522815in}{1.339513in}}{\pgfqpoint{0.515749in}{1.336586in}}{\pgfqpoint{0.510539in}{1.331377in}}%
\pgfpathcurveto{\pgfqpoint{0.505330in}{1.326168in}}{\pgfqpoint{0.502404in}{1.319102in}}{\pgfqpoint{0.502404in}{1.311735in}}%
\pgfpathcurveto{\pgfqpoint{0.502404in}{1.304369in}}{\pgfqpoint{0.505330in}{1.297303in}}{\pgfqpoint{0.510539in}{1.292094in}}%
\pgfpathcurveto{\pgfqpoint{0.515749in}{1.286884in}}{\pgfqpoint{0.522815in}{1.283958in}}{\pgfqpoint{0.530181in}{1.283958in}}%
\pgfpathclose%
\pgfusepath{stroke}%
\end{pgfscope}%
\begin{pgfscope}%
\pgfpathrectangle{\pgfqpoint{0.438556in}{0.383578in}}{\pgfqpoint{4.650000in}{2.310000in}}%
\pgfusepath{clip}%
\pgfsetbuttcap%
\pgfsetroundjoin%
\pgfsetlinewidth{0.803000pt}%
\definecolor{currentstroke}{rgb}{0.333333,0.333333,0.333333}%
\pgfsetstrokecolor{currentstroke}%
\pgfsetdash{}{0pt}%
\pgfpathmoveto{\pgfqpoint{0.584295in}{1.327642in}}%
\pgfpathcurveto{\pgfqpoint{0.591662in}{1.327642in}}{\pgfqpoint{0.598728in}{1.330569in}}{\pgfqpoint{0.603937in}{1.335778in}}%
\pgfpathcurveto{\pgfqpoint{0.609146in}{1.340987in}}{\pgfqpoint{0.612073in}{1.348053in}}{\pgfqpoint{0.612073in}{1.355420in}}%
\pgfpathcurveto{\pgfqpoint{0.612073in}{1.362786in}}{\pgfqpoint{0.609146in}{1.369852in}}{\pgfqpoint{0.603937in}{1.375062in}}%
\pgfpathcurveto{\pgfqpoint{0.598728in}{1.380271in}}{\pgfqpoint{0.591662in}{1.383197in}}{\pgfqpoint{0.584295in}{1.383197in}}%
\pgfpathcurveto{\pgfqpoint{0.576929in}{1.383197in}}{\pgfqpoint{0.569863in}{1.380271in}}{\pgfqpoint{0.564654in}{1.375062in}}%
\pgfpathcurveto{\pgfqpoint{0.559444in}{1.369852in}}{\pgfqpoint{0.556518in}{1.362786in}}{\pgfqpoint{0.556518in}{1.355420in}}%
\pgfpathcurveto{\pgfqpoint{0.556518in}{1.348053in}}{\pgfqpoint{0.559444in}{1.340987in}}{\pgfqpoint{0.564654in}{1.335778in}}%
\pgfpathcurveto{\pgfqpoint{0.569863in}{1.330569in}}{\pgfqpoint{0.576929in}{1.327642in}}{\pgfqpoint{0.584295in}{1.327642in}}%
\pgfpathclose%
\pgfusepath{stroke}%
\end{pgfscope}%
\begin{pgfscope}%
\pgfpathrectangle{\pgfqpoint{0.438556in}{0.383578in}}{\pgfqpoint{4.650000in}{2.310000in}}%
\pgfusepath{clip}%
\pgfsetbuttcap%
\pgfsetroundjoin%
\pgfsetlinewidth{0.803000pt}%
\definecolor{currentstroke}{rgb}{0.333333,0.333333,0.333333}%
\pgfsetstrokecolor{currentstroke}%
\pgfsetdash{}{0pt}%
\pgfpathmoveto{\pgfqpoint{0.532591in}{1.436853in}}%
\pgfpathcurveto{\pgfqpoint{0.539958in}{1.436853in}}{\pgfqpoint{0.547024in}{1.439779in}}{\pgfqpoint{0.552233in}{1.444988in}}%
\pgfpathcurveto{\pgfqpoint{0.557442in}{1.450198in}}{\pgfqpoint{0.560369in}{1.457264in}}{\pgfqpoint{0.560369in}{1.464630in}}%
\pgfpathcurveto{\pgfqpoint{0.560369in}{1.471997in}}{\pgfqpoint{0.557442in}{1.479063in}}{\pgfqpoint{0.552233in}{1.484272in}}%
\pgfpathcurveto{\pgfqpoint{0.547024in}{1.489481in}}{\pgfqpoint{0.539958in}{1.492408in}}{\pgfqpoint{0.532591in}{1.492408in}}%
\pgfpathcurveto{\pgfqpoint{0.525224in}{1.492408in}}{\pgfqpoint{0.518158in}{1.489481in}}{\pgfqpoint{0.512949in}{1.484272in}}%
\pgfpathcurveto{\pgfqpoint{0.507740in}{1.479063in}}{\pgfqpoint{0.504813in}{1.471997in}}{\pgfqpoint{0.504813in}{1.464630in}}%
\pgfpathcurveto{\pgfqpoint{0.504813in}{1.457264in}}{\pgfqpoint{0.507740in}{1.450198in}}{\pgfqpoint{0.512949in}{1.444988in}}%
\pgfpathcurveto{\pgfqpoint{0.518158in}{1.439779in}}{\pgfqpoint{0.525224in}{1.436853in}}{\pgfqpoint{0.532591in}{1.436853in}}%
\pgfpathclose%
\pgfusepath{stroke}%
\end{pgfscope}%
\begin{pgfscope}%
\pgfpathrectangle{\pgfqpoint{0.438556in}{0.383578in}}{\pgfqpoint{4.650000in}{2.310000in}}%
\pgfusepath{clip}%
\pgfsetbuttcap%
\pgfsetroundjoin%
\pgfsetlinewidth{0.803000pt}%
\definecolor{currentstroke}{rgb}{0.333333,0.333333,0.333333}%
\pgfsetstrokecolor{currentstroke}%
\pgfsetdash{}{0pt}%
\pgfpathmoveto{\pgfqpoint{0.577643in}{1.677116in}}%
\pgfpathcurveto{\pgfqpoint{0.585010in}{1.677116in}}{\pgfqpoint{0.592076in}{1.680043in}}{\pgfqpoint{0.597285in}{1.685252in}}%
\pgfpathcurveto{\pgfqpoint{0.602494in}{1.690461in}}{\pgfqpoint{0.605421in}{1.697527in}}{\pgfqpoint{0.605421in}{1.704894in}}%
\pgfpathcurveto{\pgfqpoint{0.605421in}{1.712261in}}{\pgfqpoint{0.602494in}{1.719327in}}{\pgfqpoint{0.597285in}{1.724536in}}%
\pgfpathcurveto{\pgfqpoint{0.592076in}{1.729745in}}{\pgfqpoint{0.585010in}{1.732672in}}{\pgfqpoint{0.577643in}{1.732672in}}%
\pgfpathcurveto{\pgfqpoint{0.570277in}{1.732672in}}{\pgfqpoint{0.563211in}{1.729745in}}{\pgfqpoint{0.558002in}{1.724536in}}%
\pgfpathcurveto{\pgfqpoint{0.552792in}{1.719327in}}{\pgfqpoint{0.549866in}{1.712261in}}{\pgfqpoint{0.549866in}{1.704894in}}%
\pgfpathcurveto{\pgfqpoint{0.549866in}{1.697527in}}{\pgfqpoint{0.552792in}{1.690461in}}{\pgfqpoint{0.558002in}{1.685252in}}%
\pgfpathcurveto{\pgfqpoint{0.563211in}{1.680043in}}{\pgfqpoint{0.570277in}{1.677116in}}{\pgfqpoint{0.577643in}{1.677116in}}%
\pgfpathclose%
\pgfusepath{stroke}%
\end{pgfscope}%
\begin{pgfscope}%
\pgfpathrectangle{\pgfqpoint{0.438556in}{0.383578in}}{\pgfqpoint{4.650000in}{2.310000in}}%
\pgfusepath{clip}%
\pgfsetbuttcap%
\pgfsetroundjoin%
\pgfsetlinewidth{0.803000pt}%
\definecolor{currentstroke}{rgb}{0.333333,0.333333,0.333333}%
\pgfsetstrokecolor{currentstroke}%
\pgfsetdash{}{0pt}%
\pgfpathmoveto{\pgfqpoint{0.532225in}{1.764485in}}%
\pgfpathcurveto{\pgfqpoint{0.539591in}{1.764485in}}{\pgfqpoint{0.546657in}{1.767411in}}{\pgfqpoint{0.551866in}{1.772620in}}%
\pgfpathcurveto{\pgfqpoint{0.557075in}{1.777830in}}{\pgfqpoint{0.560002in}{1.784896in}}{\pgfqpoint{0.560002in}{1.792262in}}%
\pgfpathcurveto{\pgfqpoint{0.560002in}{1.799629in}}{\pgfqpoint{0.557075in}{1.806695in}}{\pgfqpoint{0.551866in}{1.811904in}}%
\pgfpathcurveto{\pgfqpoint{0.546657in}{1.817113in}}{\pgfqpoint{0.539591in}{1.820040in}}{\pgfqpoint{0.532225in}{1.820040in}}%
\pgfpathcurveto{\pgfqpoint{0.524858in}{1.820040in}}{\pgfqpoint{0.517792in}{1.817113in}}{\pgfqpoint{0.512583in}{1.811904in}}%
\pgfpathcurveto{\pgfqpoint{0.507374in}{1.806695in}}{\pgfqpoint{0.504447in}{1.799629in}}{\pgfqpoint{0.504447in}{1.792262in}}%
\pgfpathcurveto{\pgfqpoint{0.504447in}{1.784896in}}{\pgfqpoint{0.507374in}{1.777830in}}{\pgfqpoint{0.512583in}{1.772620in}}%
\pgfpathcurveto{\pgfqpoint{0.517792in}{1.767411in}}{\pgfqpoint{0.524858in}{1.764485in}}{\pgfqpoint{0.532225in}{1.764485in}}%
\pgfpathclose%
\pgfusepath{stroke}%
\end{pgfscope}%
\begin{pgfscope}%
\pgfpathrectangle{\pgfqpoint{0.438556in}{0.383578in}}{\pgfqpoint{4.650000in}{2.310000in}}%
\pgfusepath{clip}%
\pgfsetbuttcap%
\pgfsetroundjoin%
\pgfsetlinewidth{0.803000pt}%
\definecolor{currentstroke}{rgb}{0.333333,0.333333,0.333333}%
\pgfsetstrokecolor{currentstroke}%
\pgfsetdash{}{0pt}%
\pgfpathmoveto{\pgfqpoint{0.517574in}{1.786327in}}%
\pgfpathcurveto{\pgfqpoint{0.524940in}{1.786327in}}{\pgfqpoint{0.532006in}{1.789254in}}{\pgfqpoint{0.537215in}{1.794463in}}%
\pgfpathcurveto{\pgfqpoint{0.542425in}{1.799672in}}{\pgfqpoint{0.545351in}{1.806738in}}{\pgfqpoint{0.545351in}{1.814104in}}%
\pgfpathcurveto{\pgfqpoint{0.545351in}{1.821471in}}{\pgfqpoint{0.542425in}{1.828537in}}{\pgfqpoint{0.537215in}{1.833746in}}%
\pgfpathcurveto{\pgfqpoint{0.532006in}{1.838955in}}{\pgfqpoint{0.524940in}{1.841882in}}{\pgfqpoint{0.517574in}{1.841882in}}%
\pgfpathcurveto{\pgfqpoint{0.510207in}{1.841882in}}{\pgfqpoint{0.503141in}{1.838955in}}{\pgfqpoint{0.497932in}{1.833746in}}%
\pgfpathcurveto{\pgfqpoint{0.492723in}{1.828537in}}{\pgfqpoint{0.489796in}{1.821471in}}{\pgfqpoint{0.489796in}{1.814104in}}%
\pgfpathcurveto{\pgfqpoint{0.489796in}{1.806738in}}{\pgfqpoint{0.492723in}{1.799672in}}{\pgfqpoint{0.497932in}{1.794463in}}%
\pgfpathcurveto{\pgfqpoint{0.503141in}{1.789254in}}{\pgfqpoint{0.510207in}{1.786327in}}{\pgfqpoint{0.517574in}{1.786327in}}%
\pgfpathclose%
\pgfusepath{stroke}%
\end{pgfscope}%
\begin{pgfscope}%
\pgfpathrectangle{\pgfqpoint{0.438556in}{0.383578in}}{\pgfqpoint{4.650000in}{2.310000in}}%
\pgfusepath{clip}%
\pgfsetbuttcap%
\pgfsetroundjoin%
\pgfsetlinewidth{0.803000pt}%
\definecolor{currentstroke}{rgb}{0.333333,0.333333,0.333333}%
\pgfsetstrokecolor{currentstroke}%
\pgfsetdash{}{0pt}%
\pgfpathmoveto{\pgfqpoint{0.552795in}{1.830011in}}%
\pgfpathcurveto{\pgfqpoint{0.560161in}{1.830011in}}{\pgfqpoint{0.567227in}{1.832938in}}{\pgfqpoint{0.572436in}{1.838147in}}%
\pgfpathcurveto{\pgfqpoint{0.577645in}{1.843356in}}{\pgfqpoint{0.580572in}{1.850422in}}{\pgfqpoint{0.580572in}{1.857789in}}%
\pgfpathcurveto{\pgfqpoint{0.580572in}{1.865155in}}{\pgfqpoint{0.577645in}{1.872221in}}{\pgfqpoint{0.572436in}{1.877431in}}%
\pgfpathcurveto{\pgfqpoint{0.567227in}{1.882640in}}{\pgfqpoint{0.560161in}{1.885566in}}{\pgfqpoint{0.552795in}{1.885566in}}%
\pgfpathcurveto{\pgfqpoint{0.545428in}{1.885566in}}{\pgfqpoint{0.538362in}{1.882640in}}{\pgfqpoint{0.533153in}{1.877431in}}%
\pgfpathcurveto{\pgfqpoint{0.527944in}{1.872221in}}{\pgfqpoint{0.525017in}{1.865155in}}{\pgfqpoint{0.525017in}{1.857789in}}%
\pgfpathcurveto{\pgfqpoint{0.525017in}{1.850422in}}{\pgfqpoint{0.527944in}{1.843356in}}{\pgfqpoint{0.533153in}{1.838147in}}%
\pgfpathcurveto{\pgfqpoint{0.538362in}{1.832938in}}{\pgfqpoint{0.545428in}{1.830011in}}{\pgfqpoint{0.552795in}{1.830011in}}%
\pgfpathclose%
\pgfusepath{stroke}%
\end{pgfscope}%
\begin{pgfscope}%
\pgfpathrectangle{\pgfqpoint{0.438556in}{0.383578in}}{\pgfqpoint{4.650000in}{2.310000in}}%
\pgfusepath{clip}%
\pgfsetbuttcap%
\pgfsetroundjoin%
\pgfsetlinewidth{0.803000pt}%
\definecolor{currentstroke}{rgb}{0.333333,0.333333,0.333333}%
\pgfsetstrokecolor{currentstroke}%
\pgfsetdash{}{0pt}%
\pgfpathmoveto{\pgfqpoint{0.557101in}{1.851853in}}%
\pgfpathcurveto{\pgfqpoint{0.564468in}{1.851853in}}{\pgfqpoint{0.571534in}{1.854780in}}{\pgfqpoint{0.576743in}{1.859989in}}%
\pgfpathcurveto{\pgfqpoint{0.581952in}{1.865198in}}{\pgfqpoint{0.584879in}{1.872264in}}{\pgfqpoint{0.584879in}{1.879631in}}%
\pgfpathcurveto{\pgfqpoint{0.584879in}{1.886998in}}{\pgfqpoint{0.581952in}{1.894064in}}{\pgfqpoint{0.576743in}{1.899273in}}%
\pgfpathcurveto{\pgfqpoint{0.571534in}{1.904482in}}{\pgfqpoint{0.564468in}{1.907409in}}{\pgfqpoint{0.557101in}{1.907409in}}%
\pgfpathcurveto{\pgfqpoint{0.549734in}{1.907409in}}{\pgfqpoint{0.542668in}{1.904482in}}{\pgfqpoint{0.537459in}{1.899273in}}%
\pgfpathcurveto{\pgfqpoint{0.532250in}{1.894064in}}{\pgfqpoint{0.529323in}{1.886998in}}{\pgfqpoint{0.529323in}{1.879631in}}%
\pgfpathcurveto{\pgfqpoint{0.529323in}{1.872264in}}{\pgfqpoint{0.532250in}{1.865198in}}{\pgfqpoint{0.537459in}{1.859989in}}%
\pgfpathcurveto{\pgfqpoint{0.542668in}{1.854780in}}{\pgfqpoint{0.549734in}{1.851853in}}{\pgfqpoint{0.557101in}{1.851853in}}%
\pgfpathclose%
\pgfusepath{stroke}%
\end{pgfscope}%
\begin{pgfscope}%
\pgfpathrectangle{\pgfqpoint{0.438556in}{0.383578in}}{\pgfqpoint{4.650000in}{2.310000in}}%
\pgfusepath{clip}%
\pgfsetbuttcap%
\pgfsetroundjoin%
\pgfsetlinewidth{0.803000pt}%
\definecolor{currentstroke}{rgb}{0.333333,0.333333,0.333333}%
\pgfsetstrokecolor{currentstroke}%
\pgfsetdash{}{0pt}%
\pgfpathmoveto{\pgfqpoint{0.524940in}{1.982906in}}%
\pgfpathcurveto{\pgfqpoint{0.532307in}{1.982906in}}{\pgfqpoint{0.539373in}{1.985833in}}{\pgfqpoint{0.544582in}{1.991042in}}%
\pgfpathcurveto{\pgfqpoint{0.549791in}{1.996251in}}{\pgfqpoint{0.552718in}{2.003317in}}{\pgfqpoint{0.552718in}{2.010684in}}%
\pgfpathcurveto{\pgfqpoint{0.552718in}{2.018050in}}{\pgfqpoint{0.549791in}{2.025116in}}{\pgfqpoint{0.544582in}{2.030325in}}%
\pgfpathcurveto{\pgfqpoint{0.539373in}{2.035535in}}{\pgfqpoint{0.532307in}{2.038461in}}{\pgfqpoint{0.524940in}{2.038461in}}%
\pgfpathcurveto{\pgfqpoint{0.517574in}{2.038461in}}{\pgfqpoint{0.510508in}{2.035535in}}{\pgfqpoint{0.505298in}{2.030325in}}%
\pgfpathcurveto{\pgfqpoint{0.500089in}{2.025116in}}{\pgfqpoint{0.497163in}{2.018050in}}{\pgfqpoint{0.497163in}{2.010684in}}%
\pgfpathcurveto{\pgfqpoint{0.497163in}{2.003317in}}{\pgfqpoint{0.500089in}{1.996251in}}{\pgfqpoint{0.505298in}{1.991042in}}%
\pgfpathcurveto{\pgfqpoint{0.510508in}{1.985833in}}{\pgfqpoint{0.517574in}{1.982906in}}{\pgfqpoint{0.524940in}{1.982906in}}%
\pgfpathclose%
\pgfusepath{stroke}%
\end{pgfscope}%
\begin{pgfscope}%
\pgfpathrectangle{\pgfqpoint{0.438556in}{0.383578in}}{\pgfqpoint{4.650000in}{2.310000in}}%
\pgfusepath{clip}%
\pgfsetbuttcap%
\pgfsetroundjoin%
\pgfsetlinewidth{0.803000pt}%
\definecolor{currentstroke}{rgb}{0.333333,0.333333,0.333333}%
\pgfsetstrokecolor{currentstroke}%
\pgfsetdash{}{0pt}%
\pgfpathmoveto{\pgfqpoint{0.528963in}{2.070274in}}%
\pgfpathcurveto{\pgfqpoint{0.536329in}{2.070274in}}{\pgfqpoint{0.543395in}{2.073201in}}{\pgfqpoint{0.548605in}{2.078410in}}%
\pgfpathcurveto{\pgfqpoint{0.553814in}{2.083619in}}{\pgfqpoint{0.556740in}{2.090685in}}{\pgfqpoint{0.556740in}{2.098052in}}%
\pgfpathcurveto{\pgfqpoint{0.556740in}{2.105419in}}{\pgfqpoint{0.553814in}{2.112485in}}{\pgfqpoint{0.548605in}{2.117694in}}%
\pgfpathcurveto{\pgfqpoint{0.543395in}{2.122903in}}{\pgfqpoint{0.536329in}{2.125830in}}{\pgfqpoint{0.528963in}{2.125830in}}%
\pgfpathcurveto{\pgfqpoint{0.521596in}{2.125830in}}{\pgfqpoint{0.514530in}{2.122903in}}{\pgfqpoint{0.509321in}{2.117694in}}%
\pgfpathcurveto{\pgfqpoint{0.504112in}{2.112485in}}{\pgfqpoint{0.501185in}{2.105419in}}{\pgfqpoint{0.501185in}{2.098052in}}%
\pgfpathcurveto{\pgfqpoint{0.501185in}{2.090685in}}{\pgfqpoint{0.504112in}{2.083619in}}{\pgfqpoint{0.509321in}{2.078410in}}%
\pgfpathcurveto{\pgfqpoint{0.514530in}{2.073201in}}{\pgfqpoint{0.521596in}{2.070274in}}{\pgfqpoint{0.528963in}{2.070274in}}%
\pgfpathclose%
\pgfusepath{stroke}%
\end{pgfscope}%
\begin{pgfscope}%
\pgfpathrectangle{\pgfqpoint{0.438556in}{0.383578in}}{\pgfqpoint{4.650000in}{2.310000in}}%
\pgfusepath{clip}%
\pgfsetbuttcap%
\pgfsetroundjoin%
\pgfsetlinewidth{0.803000pt}%
\definecolor{currentstroke}{rgb}{0.333333,0.333333,0.333333}%
\pgfsetstrokecolor{currentstroke}%
\pgfsetdash{}{0pt}%
\pgfpathmoveto{\pgfqpoint{0.588638in}{2.157643in}}%
\pgfpathcurveto{\pgfqpoint{0.596005in}{2.157643in}}{\pgfqpoint{0.603071in}{2.160570in}}{\pgfqpoint{0.608280in}{2.165779in}}%
\pgfpathcurveto{\pgfqpoint{0.613489in}{2.170988in}}{\pgfqpoint{0.616416in}{2.178054in}}{\pgfqpoint{0.616416in}{2.185421in}}%
\pgfpathcurveto{\pgfqpoint{0.616416in}{2.192787in}}{\pgfqpoint{0.613489in}{2.199853in}}{\pgfqpoint{0.608280in}{2.205063in}}%
\pgfpathcurveto{\pgfqpoint{0.603071in}{2.210272in}}{\pgfqpoint{0.596005in}{2.213198in}}{\pgfqpoint{0.588638in}{2.213198in}}%
\pgfpathcurveto{\pgfqpoint{0.581272in}{2.213198in}}{\pgfqpoint{0.574206in}{2.210272in}}{\pgfqpoint{0.568997in}{2.205063in}}%
\pgfpathcurveto{\pgfqpoint{0.563788in}{2.199853in}}{\pgfqpoint{0.560861in}{2.192787in}}{\pgfqpoint{0.560861in}{2.185421in}}%
\pgfpathcurveto{\pgfqpoint{0.560861in}{2.178054in}}{\pgfqpoint{0.563788in}{2.170988in}}{\pgfqpoint{0.568997in}{2.165779in}}%
\pgfpathcurveto{\pgfqpoint{0.574206in}{2.160570in}}{\pgfqpoint{0.581272in}{2.157643in}}{\pgfqpoint{0.588638in}{2.157643in}}%
\pgfpathclose%
\pgfusepath{stroke}%
\end{pgfscope}%
\begin{pgfscope}%
\pgfpathrectangle{\pgfqpoint{0.438556in}{0.383578in}}{\pgfqpoint{4.650000in}{2.310000in}}%
\pgfusepath{clip}%
\pgfsetbuttcap%
\pgfsetroundjoin%
\pgfsetlinewidth{0.803000pt}%
\definecolor{currentstroke}{rgb}{0.333333,0.333333,0.333333}%
\pgfsetstrokecolor{currentstroke}%
\pgfsetdash{}{0pt}%
\pgfpathmoveto{\pgfqpoint{0.547343in}{2.245011in}}%
\pgfpathcurveto{\pgfqpoint{0.554710in}{2.245011in}}{\pgfqpoint{0.561776in}{2.247938in}}{\pgfqpoint{0.566985in}{2.253147in}}%
\pgfpathcurveto{\pgfqpoint{0.572194in}{2.258356in}}{\pgfqpoint{0.575121in}{2.265422in}}{\pgfqpoint{0.575121in}{2.272789in}}%
\pgfpathcurveto{\pgfqpoint{0.575121in}{2.280156in}}{\pgfqpoint{0.572194in}{2.287222in}}{\pgfqpoint{0.566985in}{2.292431in}}%
\pgfpathcurveto{\pgfqpoint{0.561776in}{2.297640in}}{\pgfqpoint{0.554710in}{2.300567in}}{\pgfqpoint{0.547343in}{2.300567in}}%
\pgfpathcurveto{\pgfqpoint{0.539976in}{2.300567in}}{\pgfqpoint{0.532910in}{2.297640in}}{\pgfqpoint{0.527701in}{2.292431in}}%
\pgfpathcurveto{\pgfqpoint{0.522492in}{2.287222in}}{\pgfqpoint{0.519565in}{2.280156in}}{\pgfqpoint{0.519565in}{2.272789in}}%
\pgfpathcurveto{\pgfqpoint{0.519565in}{2.265422in}}{\pgfqpoint{0.522492in}{2.258356in}}{\pgfqpoint{0.527701in}{2.253147in}}%
\pgfpathcurveto{\pgfqpoint{0.532910in}{2.247938in}}{\pgfqpoint{0.539976in}{2.245011in}}{\pgfqpoint{0.547343in}{2.245011in}}%
\pgfpathclose%
\pgfusepath{stroke}%
\end{pgfscope}%
\begin{pgfscope}%
\pgfpathrectangle{\pgfqpoint{0.438556in}{0.383578in}}{\pgfqpoint{4.650000in}{2.310000in}}%
\pgfusepath{clip}%
\pgfsetbuttcap%
\pgfsetroundjoin%
\pgfsetlinewidth{0.803000pt}%
\definecolor{currentstroke}{rgb}{0.333333,0.333333,0.333333}%
\pgfsetstrokecolor{currentstroke}%
\pgfsetdash{}{0pt}%
\pgfpathmoveto{\pgfqpoint{0.555855in}{2.310538in}}%
\pgfpathcurveto{\pgfqpoint{0.563222in}{2.310538in}}{\pgfqpoint{0.570288in}{2.313465in}}{\pgfqpoint{0.575497in}{2.318674in}}%
\pgfpathcurveto{\pgfqpoint{0.580706in}{2.323883in}}{\pgfqpoint{0.583633in}{2.330949in}}{\pgfqpoint{0.583633in}{2.338316in}}%
\pgfpathcurveto{\pgfqpoint{0.583633in}{2.345682in}}{\pgfqpoint{0.580706in}{2.352748in}}{\pgfqpoint{0.575497in}{2.357957in}}%
\pgfpathcurveto{\pgfqpoint{0.570288in}{2.363167in}}{\pgfqpoint{0.563222in}{2.366093in}}{\pgfqpoint{0.555855in}{2.366093in}}%
\pgfpathcurveto{\pgfqpoint{0.548488in}{2.366093in}}{\pgfqpoint{0.541422in}{2.363167in}}{\pgfqpoint{0.536213in}{2.357957in}}%
\pgfpathcurveto{\pgfqpoint{0.531004in}{2.352748in}}{\pgfqpoint{0.528077in}{2.345682in}}{\pgfqpoint{0.528077in}{2.338316in}}%
\pgfpathcurveto{\pgfqpoint{0.528077in}{2.330949in}}{\pgfqpoint{0.531004in}{2.323883in}}{\pgfqpoint{0.536213in}{2.318674in}}%
\pgfpathcurveto{\pgfqpoint{0.541422in}{2.313465in}}{\pgfqpoint{0.548488in}{2.310538in}}{\pgfqpoint{0.555855in}{2.310538in}}%
\pgfpathclose%
\pgfusepath{stroke}%
\end{pgfscope}%
\begin{pgfscope}%
\pgfpathrectangle{\pgfqpoint{0.438556in}{0.383578in}}{\pgfqpoint{4.650000in}{2.310000in}}%
\pgfusepath{clip}%
\pgfsetbuttcap%
\pgfsetroundjoin%
\pgfsetlinewidth{0.803000pt}%
\definecolor{currentstroke}{rgb}{0.333333,0.333333,0.333333}%
\pgfsetstrokecolor{currentstroke}%
\pgfsetdash{}{0pt}%
\pgfpathmoveto{\pgfqpoint{0.573960in}{2.332380in}}%
\pgfpathcurveto{\pgfqpoint{0.581327in}{2.332380in}}{\pgfqpoint{0.588393in}{2.335307in}}{\pgfqpoint{0.593602in}{2.340516in}}%
\pgfpathcurveto{\pgfqpoint{0.598811in}{2.345725in}}{\pgfqpoint{0.601738in}{2.352791in}}{\pgfqpoint{0.601738in}{2.360158in}}%
\pgfpathcurveto{\pgfqpoint{0.601738in}{2.367524in}}{\pgfqpoint{0.598811in}{2.374591in}}{\pgfqpoint{0.593602in}{2.379800in}}%
\pgfpathcurveto{\pgfqpoint{0.588393in}{2.385009in}}{\pgfqpoint{0.581327in}{2.387936in}}{\pgfqpoint{0.573960in}{2.387936in}}%
\pgfpathcurveto{\pgfqpoint{0.566593in}{2.387936in}}{\pgfqpoint{0.559527in}{2.385009in}}{\pgfqpoint{0.554318in}{2.379800in}}%
\pgfpathcurveto{\pgfqpoint{0.549109in}{2.374591in}}{\pgfqpoint{0.546182in}{2.367524in}}{\pgfqpoint{0.546182in}{2.360158in}}%
\pgfpathcurveto{\pgfqpoint{0.546182in}{2.352791in}}{\pgfqpoint{0.549109in}{2.345725in}}{\pgfqpoint{0.554318in}{2.340516in}}%
\pgfpathcurveto{\pgfqpoint{0.559527in}{2.335307in}}{\pgfqpoint{0.566593in}{2.332380in}}{\pgfqpoint{0.573960in}{2.332380in}}%
\pgfpathclose%
\pgfusepath{stroke}%
\end{pgfscope}%
\begin{pgfscope}%
\pgfpathrectangle{\pgfqpoint{0.438556in}{0.383578in}}{\pgfqpoint{4.650000in}{2.310000in}}%
\pgfusepath{clip}%
\pgfsetbuttcap%
\pgfsetroundjoin%
\pgfsetlinewidth{0.803000pt}%
\definecolor{currentstroke}{rgb}{0.686275,0.352941,0.313725}%
\pgfsetstrokecolor{currentstroke}%
\pgfsetdash{}{0pt}%
\pgfpathmoveto{\pgfqpoint{1.091232in}{0.482743in}}%
\pgfpathcurveto{\pgfqpoint{1.096758in}{0.482743in}}{\pgfqpoint{1.102057in}{0.484938in}}{\pgfqpoint{1.105964in}{0.488845in}}%
\pgfpathcurveto{\pgfqpoint{1.109871in}{0.492752in}}{\pgfqpoint{1.112066in}{0.498051in}}{\pgfqpoint{1.112066in}{0.503577in}}%
\pgfpathcurveto{\pgfqpoint{1.112066in}{0.509102in}}{\pgfqpoint{1.109871in}{0.514401in}}{\pgfqpoint{1.105964in}{0.518308in}}%
\pgfpathcurveto{\pgfqpoint{1.102057in}{0.522215in}}{\pgfqpoint{1.096758in}{0.524410in}}{\pgfqpoint{1.091232in}{0.524410in}}%
\pgfpathcurveto{\pgfqpoint{1.085707in}{0.524410in}}{\pgfqpoint{1.080408in}{0.522215in}}{\pgfqpoint{1.076501in}{0.518308in}}%
\pgfpathcurveto{\pgfqpoint{1.072594in}{0.514401in}}{\pgfqpoint{1.070399in}{0.509102in}}{\pgfqpoint{1.070399in}{0.503577in}}%
\pgfpathcurveto{\pgfqpoint{1.070399in}{0.498051in}}{\pgfqpoint{1.072594in}{0.492752in}}{\pgfqpoint{1.076501in}{0.488845in}}%
\pgfpathcurveto{\pgfqpoint{1.080408in}{0.484938in}}{\pgfqpoint{1.085707in}{0.482743in}}{\pgfqpoint{1.091232in}{0.482743in}}%
\pgfpathclose%
\pgfusepath{stroke}%
\end{pgfscope}%
\begin{pgfscope}%
\pgfpathrectangle{\pgfqpoint{0.438556in}{0.383578in}}{\pgfqpoint{4.650000in}{2.310000in}}%
\pgfusepath{clip}%
\pgfsetbuttcap%
\pgfsetroundjoin%
\pgfsetlinewidth{0.803000pt}%
\definecolor{currentstroke}{rgb}{0.686275,0.352941,0.313725}%
\pgfsetstrokecolor{currentstroke}%
\pgfsetdash{}{0pt}%
\pgfpathmoveto{\pgfqpoint{1.090289in}{0.635638in}}%
\pgfpathcurveto{\pgfqpoint{1.095814in}{0.635638in}}{\pgfqpoint{1.101113in}{0.637833in}}{\pgfqpoint{1.105020in}{0.641740in}}%
\pgfpathcurveto{\pgfqpoint{1.108927in}{0.645647in}}{\pgfqpoint{1.111122in}{0.650946in}}{\pgfqpoint{1.111122in}{0.656471in}}%
\pgfpathcurveto{\pgfqpoint{1.111122in}{0.661997in}}{\pgfqpoint{1.108927in}{0.667296in}}{\pgfqpoint{1.105020in}{0.671203in}}%
\pgfpathcurveto{\pgfqpoint{1.101113in}{0.675110in}}{\pgfqpoint{1.095814in}{0.677305in}}{\pgfqpoint{1.090289in}{0.677305in}}%
\pgfpathcurveto{\pgfqpoint{1.084764in}{0.677305in}}{\pgfqpoint{1.079464in}{0.675110in}}{\pgfqpoint{1.075557in}{0.671203in}}%
\pgfpathcurveto{\pgfqpoint{1.071650in}{0.667296in}}{\pgfqpoint{1.069455in}{0.661997in}}{\pgfqpoint{1.069455in}{0.656471in}}%
\pgfpathcurveto{\pgfqpoint{1.069455in}{0.650946in}}{\pgfqpoint{1.071650in}{0.645647in}}{\pgfqpoint{1.075557in}{0.641740in}}%
\pgfpathcurveto{\pgfqpoint{1.079464in}{0.637833in}}{\pgfqpoint{1.084764in}{0.635638in}}{\pgfqpoint{1.090289in}{0.635638in}}%
\pgfpathclose%
\pgfusepath{stroke}%
\end{pgfscope}%
\begin{pgfscope}%
\pgfpathrectangle{\pgfqpoint{0.438556in}{0.383578in}}{\pgfqpoint{4.650000in}{2.310000in}}%
\pgfusepath{clip}%
\pgfsetbuttcap%
\pgfsetroundjoin%
\pgfsetlinewidth{0.803000pt}%
\definecolor{currentstroke}{rgb}{0.686275,0.352941,0.313725}%
\pgfsetstrokecolor{currentstroke}%
\pgfsetdash{}{0pt}%
\pgfpathmoveto{\pgfqpoint{1.160529in}{0.679322in}}%
\pgfpathcurveto{\pgfqpoint{1.166054in}{0.679322in}}{\pgfqpoint{1.171353in}{0.681518in}}{\pgfqpoint{1.175260in}{0.685424in}}%
\pgfpathcurveto{\pgfqpoint{1.179167in}{0.689331in}}{\pgfqpoint{1.181362in}{0.694631in}}{\pgfqpoint{1.181362in}{0.700156in}}%
\pgfpathcurveto{\pgfqpoint{1.181362in}{0.705681in}}{\pgfqpoint{1.179167in}{0.710980in}}{\pgfqpoint{1.175260in}{0.714887in}}%
\pgfpathcurveto{\pgfqpoint{1.171353in}{0.718794in}}{\pgfqpoint{1.166054in}{0.720989in}}{\pgfqpoint{1.160529in}{0.720989in}}%
\pgfpathcurveto{\pgfqpoint{1.155004in}{0.720989in}}{\pgfqpoint{1.149704in}{0.718794in}}{\pgfqpoint{1.145798in}{0.714887in}}%
\pgfpathcurveto{\pgfqpoint{1.141891in}{0.710980in}}{\pgfqpoint{1.139696in}{0.705681in}}{\pgfqpoint{1.139696in}{0.700156in}}%
\pgfpathcurveto{\pgfqpoint{1.139696in}{0.694631in}}{\pgfqpoint{1.141891in}{0.689331in}}{\pgfqpoint{1.145798in}{0.685424in}}%
\pgfpathcurveto{\pgfqpoint{1.149704in}{0.681518in}}{\pgfqpoint{1.155004in}{0.679322in}}{\pgfqpoint{1.160529in}{0.679322in}}%
\pgfpathclose%
\pgfusepath{stroke}%
\end{pgfscope}%
\begin{pgfscope}%
\pgfpathrectangle{\pgfqpoint{0.438556in}{0.383578in}}{\pgfqpoint{4.650000in}{2.310000in}}%
\pgfusepath{clip}%
\pgfsetbuttcap%
\pgfsetroundjoin%
\pgfsetlinewidth{0.803000pt}%
\definecolor{currentstroke}{rgb}{0.686275,0.352941,0.313725}%
\pgfsetstrokecolor{currentstroke}%
\pgfsetdash{}{0pt}%
\pgfpathmoveto{\pgfqpoint{1.120296in}{0.701165in}}%
\pgfpathcurveto{\pgfqpoint{1.125821in}{0.701165in}}{\pgfqpoint{1.131121in}{0.703360in}}{\pgfqpoint{1.135027in}{0.707266in}}%
\pgfpathcurveto{\pgfqpoint{1.138934in}{0.711173in}}{\pgfqpoint{1.141129in}{0.716473in}}{\pgfqpoint{1.141129in}{0.721998in}}%
\pgfpathcurveto{\pgfqpoint{1.141129in}{0.727523in}}{\pgfqpoint{1.138934in}{0.732822in}}{\pgfqpoint{1.135027in}{0.736729in}}%
\pgfpathcurveto{\pgfqpoint{1.131121in}{0.740636in}}{\pgfqpoint{1.125821in}{0.742831in}}{\pgfqpoint{1.120296in}{0.742831in}}%
\pgfpathcurveto{\pgfqpoint{1.114771in}{0.742831in}}{\pgfqpoint{1.109472in}{0.740636in}}{\pgfqpoint{1.105565in}{0.736729in}}%
\pgfpathcurveto{\pgfqpoint{1.101658in}{0.732822in}}{\pgfqpoint{1.099463in}{0.727523in}}{\pgfqpoint{1.099463in}{0.721998in}}%
\pgfpathcurveto{\pgfqpoint{1.099463in}{0.716473in}}{\pgfqpoint{1.101658in}{0.711173in}}{\pgfqpoint{1.105565in}{0.707266in}}%
\pgfpathcurveto{\pgfqpoint{1.109472in}{0.703360in}}{\pgfqpoint{1.114771in}{0.701165in}}{\pgfqpoint{1.120296in}{0.701165in}}%
\pgfpathclose%
\pgfusepath{stroke}%
\end{pgfscope}%
\begin{pgfscope}%
\pgfpathrectangle{\pgfqpoint{0.438556in}{0.383578in}}{\pgfqpoint{4.650000in}{2.310000in}}%
\pgfusepath{clip}%
\pgfsetbuttcap%
\pgfsetroundjoin%
\pgfsetlinewidth{0.803000pt}%
\definecolor{currentstroke}{rgb}{0.686275,0.352941,0.313725}%
\pgfsetstrokecolor{currentstroke}%
\pgfsetdash{}{0pt}%
\pgfpathmoveto{\pgfqpoint{1.167282in}{0.810375in}}%
\pgfpathcurveto{\pgfqpoint{1.172807in}{0.810375in}}{\pgfqpoint{1.178106in}{0.812570in}}{\pgfqpoint{1.182013in}{0.816477in}}%
\pgfpathcurveto{\pgfqpoint{1.185920in}{0.820384in}}{\pgfqpoint{1.188115in}{0.825683in}}{\pgfqpoint{1.188115in}{0.831209in}}%
\pgfpathcurveto{\pgfqpoint{1.188115in}{0.836734in}}{\pgfqpoint{1.185920in}{0.842033in}}{\pgfqpoint{1.182013in}{0.845940in}}%
\pgfpathcurveto{\pgfqpoint{1.178106in}{0.849847in}}{\pgfqpoint{1.172807in}{0.852042in}}{\pgfqpoint{1.167282in}{0.852042in}}%
\pgfpathcurveto{\pgfqpoint{1.161757in}{0.852042in}}{\pgfqpoint{1.156457in}{0.849847in}}{\pgfqpoint{1.152550in}{0.845940in}}%
\pgfpathcurveto{\pgfqpoint{1.148644in}{0.842033in}}{\pgfqpoint{1.146448in}{0.836734in}}{\pgfqpoint{1.146448in}{0.831209in}}%
\pgfpathcurveto{\pgfqpoint{1.146448in}{0.825683in}}{\pgfqpoint{1.148644in}{0.820384in}}{\pgfqpoint{1.152550in}{0.816477in}}%
\pgfpathcurveto{\pgfqpoint{1.156457in}{0.812570in}}{\pgfqpoint{1.161757in}{0.810375in}}{\pgfqpoint{1.167282in}{0.810375in}}%
\pgfpathclose%
\pgfusepath{stroke}%
\end{pgfscope}%
\begin{pgfscope}%
\pgfpathrectangle{\pgfqpoint{0.438556in}{0.383578in}}{\pgfqpoint{4.650000in}{2.310000in}}%
\pgfusepath{clip}%
\pgfsetbuttcap%
\pgfsetroundjoin%
\pgfsetlinewidth{0.803000pt}%
\definecolor{currentstroke}{rgb}{0.686275,0.352941,0.313725}%
\pgfsetstrokecolor{currentstroke}%
\pgfsetdash{}{0pt}%
\pgfpathmoveto{\pgfqpoint{1.117300in}{0.854059in}}%
\pgfpathcurveto{\pgfqpoint{1.122825in}{0.854059in}}{\pgfqpoint{1.128125in}{0.856255in}}{\pgfqpoint{1.132031in}{0.860161in}}%
\pgfpathcurveto{\pgfqpoint{1.135938in}{0.864068in}}{\pgfqpoint{1.138133in}{0.869368in}}{\pgfqpoint{1.138133in}{0.874893in}}%
\pgfpathcurveto{\pgfqpoint{1.138133in}{0.880418in}}{\pgfqpoint{1.135938in}{0.885717in}}{\pgfqpoint{1.132031in}{0.889624in}}%
\pgfpathcurveto{\pgfqpoint{1.128125in}{0.893531in}}{\pgfqpoint{1.122825in}{0.895726in}}{\pgfqpoint{1.117300in}{0.895726in}}%
\pgfpathcurveto{\pgfqpoint{1.111775in}{0.895726in}}{\pgfqpoint{1.106475in}{0.893531in}}{\pgfqpoint{1.102569in}{0.889624in}}%
\pgfpathcurveto{\pgfqpoint{1.098662in}{0.885717in}}{\pgfqpoint{1.096467in}{0.880418in}}{\pgfqpoint{1.096467in}{0.874893in}}%
\pgfpathcurveto{\pgfqpoint{1.096467in}{0.869368in}}{\pgfqpoint{1.098662in}{0.864068in}}{\pgfqpoint{1.102569in}{0.860161in}}%
\pgfpathcurveto{\pgfqpoint{1.106475in}{0.856255in}}{\pgfqpoint{1.111775in}{0.854059in}}{\pgfqpoint{1.117300in}{0.854059in}}%
\pgfpathclose%
\pgfusepath{stroke}%
\end{pgfscope}%
\begin{pgfscope}%
\pgfpathrectangle{\pgfqpoint{0.438556in}{0.383578in}}{\pgfqpoint{4.650000in}{2.310000in}}%
\pgfusepath{clip}%
\pgfsetbuttcap%
\pgfsetroundjoin%
\pgfsetlinewidth{0.803000pt}%
\definecolor{currentstroke}{rgb}{0.686275,0.352941,0.313725}%
\pgfsetstrokecolor{currentstroke}%
\pgfsetdash{}{0pt}%
\pgfpathmoveto{\pgfqpoint{1.167776in}{0.875902in}}%
\pgfpathcurveto{\pgfqpoint{1.173302in}{0.875902in}}{\pgfqpoint{1.178601in}{0.878097in}}{\pgfqpoint{1.182508in}{0.882004in}}%
\pgfpathcurveto{\pgfqpoint{1.186415in}{0.885910in}}{\pgfqpoint{1.188610in}{0.891210in}}{\pgfqpoint{1.188610in}{0.896735in}}%
\pgfpathcurveto{\pgfqpoint{1.188610in}{0.902260in}}{\pgfqpoint{1.186415in}{0.907560in}}{\pgfqpoint{1.182508in}{0.911466in}}%
\pgfpathcurveto{\pgfqpoint{1.178601in}{0.915373in}}{\pgfqpoint{1.173302in}{0.917568in}}{\pgfqpoint{1.167776in}{0.917568in}}%
\pgfpathcurveto{\pgfqpoint{1.162251in}{0.917568in}}{\pgfqpoint{1.156952in}{0.915373in}}{\pgfqpoint{1.153045in}{0.911466in}}%
\pgfpathcurveto{\pgfqpoint{1.149138in}{0.907560in}}{\pgfqpoint{1.146943in}{0.902260in}}{\pgfqpoint{1.146943in}{0.896735in}}%
\pgfpathcurveto{\pgfqpoint{1.146943in}{0.891210in}}{\pgfqpoint{1.149138in}{0.885910in}}{\pgfqpoint{1.153045in}{0.882004in}}%
\pgfpathcurveto{\pgfqpoint{1.156952in}{0.878097in}}{\pgfqpoint{1.162251in}{0.875902in}}{\pgfqpoint{1.167776in}{0.875902in}}%
\pgfpathclose%
\pgfusepath{stroke}%
\end{pgfscope}%
\begin{pgfscope}%
\pgfpathrectangle{\pgfqpoint{0.438556in}{0.383578in}}{\pgfqpoint{4.650000in}{2.310000in}}%
\pgfusepath{clip}%
\pgfsetbuttcap%
\pgfsetroundjoin%
\pgfsetlinewidth{0.803000pt}%
\definecolor{currentstroke}{rgb}{0.686275,0.352941,0.313725}%
\pgfsetstrokecolor{currentstroke}%
\pgfsetdash{}{0pt}%
\pgfpathmoveto{\pgfqpoint{1.090316in}{0.919586in}}%
\pgfpathcurveto{\pgfqpoint{1.095841in}{0.919586in}}{\pgfqpoint{1.101141in}{0.921781in}}{\pgfqpoint{1.105048in}{0.925688in}}%
\pgfpathcurveto{\pgfqpoint{1.108954in}{0.929595in}}{\pgfqpoint{1.111150in}{0.934894in}}{\pgfqpoint{1.111150in}{0.940419in}}%
\pgfpathcurveto{\pgfqpoint{1.111150in}{0.945944in}}{\pgfqpoint{1.108954in}{0.951244in}}{\pgfqpoint{1.105048in}{0.955151in}}%
\pgfpathcurveto{\pgfqpoint{1.101141in}{0.959057in}}{\pgfqpoint{1.095841in}{0.961253in}}{\pgfqpoint{1.090316in}{0.961253in}}%
\pgfpathcurveto{\pgfqpoint{1.084791in}{0.961253in}}{\pgfqpoint{1.079492in}{0.959057in}}{\pgfqpoint{1.075585in}{0.955151in}}%
\pgfpathcurveto{\pgfqpoint{1.071678in}{0.951244in}}{\pgfqpoint{1.069483in}{0.945944in}}{\pgfqpoint{1.069483in}{0.940419in}}%
\pgfpathcurveto{\pgfqpoint{1.069483in}{0.934894in}}{\pgfqpoint{1.071678in}{0.929595in}}{\pgfqpoint{1.075585in}{0.925688in}}%
\pgfpathcurveto{\pgfqpoint{1.079492in}{0.921781in}}{\pgfqpoint{1.084791in}{0.919586in}}{\pgfqpoint{1.090316in}{0.919586in}}%
\pgfpathclose%
\pgfusepath{stroke}%
\end{pgfscope}%
\begin{pgfscope}%
\pgfpathrectangle{\pgfqpoint{0.438556in}{0.383578in}}{\pgfqpoint{4.650000in}{2.310000in}}%
\pgfusepath{clip}%
\pgfsetbuttcap%
\pgfsetroundjoin%
\pgfsetlinewidth{0.803000pt}%
\definecolor{currentstroke}{rgb}{0.686275,0.352941,0.313725}%
\pgfsetstrokecolor{currentstroke}%
\pgfsetdash{}{0pt}%
\pgfpathmoveto{\pgfqpoint{1.095768in}{0.941428in}}%
\pgfpathcurveto{\pgfqpoint{1.101293in}{0.941428in}}{\pgfqpoint{1.106592in}{0.943623in}}{\pgfqpoint{1.110499in}{0.947530in}}%
\pgfpathcurveto{\pgfqpoint{1.114406in}{0.951437in}}{\pgfqpoint{1.116601in}{0.956736in}}{\pgfqpoint{1.116601in}{0.962261in}}%
\pgfpathcurveto{\pgfqpoint{1.116601in}{0.967786in}}{\pgfqpoint{1.114406in}{0.973086in}}{\pgfqpoint{1.110499in}{0.976993in}}%
\pgfpathcurveto{\pgfqpoint{1.106592in}{0.980900in}}{\pgfqpoint{1.101293in}{0.983095in}}{\pgfqpoint{1.095768in}{0.983095in}}%
\pgfpathcurveto{\pgfqpoint{1.090243in}{0.983095in}}{\pgfqpoint{1.084943in}{0.980900in}}{\pgfqpoint{1.081037in}{0.976993in}}%
\pgfpathcurveto{\pgfqpoint{1.077130in}{0.973086in}}{\pgfqpoint{1.074935in}{0.967786in}}{\pgfqpoint{1.074935in}{0.962261in}}%
\pgfpathcurveto{\pgfqpoint{1.074935in}{0.956736in}}{\pgfqpoint{1.077130in}{0.951437in}}{\pgfqpoint{1.081037in}{0.947530in}}%
\pgfpathcurveto{\pgfqpoint{1.084943in}{0.943623in}}{\pgfqpoint{1.090243in}{0.941428in}}{\pgfqpoint{1.095768in}{0.941428in}}%
\pgfpathclose%
\pgfusepath{stroke}%
\end{pgfscope}%
\begin{pgfscope}%
\pgfpathrectangle{\pgfqpoint{0.438556in}{0.383578in}}{\pgfqpoint{4.650000in}{2.310000in}}%
\pgfusepath{clip}%
\pgfsetbuttcap%
\pgfsetroundjoin%
\pgfsetlinewidth{0.803000pt}%
\definecolor{currentstroke}{rgb}{0.686275,0.352941,0.313725}%
\pgfsetstrokecolor{currentstroke}%
\pgfsetdash{}{0pt}%
\pgfpathmoveto{\pgfqpoint{1.146822in}{1.006954in}}%
\pgfpathcurveto{\pgfqpoint{1.152347in}{1.006954in}}{\pgfqpoint{1.157646in}{1.009150in}}{\pgfqpoint{1.161553in}{1.013056in}}%
\pgfpathcurveto{\pgfqpoint{1.165460in}{1.016963in}}{\pgfqpoint{1.167655in}{1.022263in}}{\pgfqpoint{1.167655in}{1.027788in}}%
\pgfpathcurveto{\pgfqpoint{1.167655in}{1.033313in}}{\pgfqpoint{1.165460in}{1.038612in}}{\pgfqpoint{1.161553in}{1.042519in}}%
\pgfpathcurveto{\pgfqpoint{1.157646in}{1.046426in}}{\pgfqpoint{1.152347in}{1.048621in}}{\pgfqpoint{1.146822in}{1.048621in}}%
\pgfpathcurveto{\pgfqpoint{1.141297in}{1.048621in}}{\pgfqpoint{1.135997in}{1.046426in}}{\pgfqpoint{1.132090in}{1.042519in}}%
\pgfpathcurveto{\pgfqpoint{1.128184in}{1.038612in}}{\pgfqpoint{1.125988in}{1.033313in}}{\pgfqpoint{1.125988in}{1.027788in}}%
\pgfpathcurveto{\pgfqpoint{1.125988in}{1.022263in}}{\pgfqpoint{1.128184in}{1.016963in}}{\pgfqpoint{1.132090in}{1.013056in}}%
\pgfpathcurveto{\pgfqpoint{1.135997in}{1.009150in}}{\pgfqpoint{1.141297in}{1.006954in}}{\pgfqpoint{1.146822in}{1.006954in}}%
\pgfpathclose%
\pgfusepath{stroke}%
\end{pgfscope}%
\begin{pgfscope}%
\pgfpathrectangle{\pgfqpoint{0.438556in}{0.383578in}}{\pgfqpoint{4.650000in}{2.310000in}}%
\pgfusepath{clip}%
\pgfsetbuttcap%
\pgfsetroundjoin%
\pgfsetlinewidth{0.803000pt}%
\definecolor{currentstroke}{rgb}{0.686275,0.352941,0.313725}%
\pgfsetstrokecolor{currentstroke}%
\pgfsetdash{}{0pt}%
\pgfpathmoveto{\pgfqpoint{1.111894in}{1.028797in}}%
\pgfpathcurveto{\pgfqpoint{1.117419in}{1.028797in}}{\pgfqpoint{1.122719in}{1.030992in}}{\pgfqpoint{1.126625in}{1.034898in}}%
\pgfpathcurveto{\pgfqpoint{1.130532in}{1.038805in}}{\pgfqpoint{1.132727in}{1.044105in}}{\pgfqpoint{1.132727in}{1.049630in}}%
\pgfpathcurveto{\pgfqpoint{1.132727in}{1.055155in}}{\pgfqpoint{1.130532in}{1.060454in}}{\pgfqpoint{1.126625in}{1.064361in}}%
\pgfpathcurveto{\pgfqpoint{1.122719in}{1.068268in}}{\pgfqpoint{1.117419in}{1.070463in}}{\pgfqpoint{1.111894in}{1.070463in}}%
\pgfpathcurveto{\pgfqpoint{1.106369in}{1.070463in}}{\pgfqpoint{1.101069in}{1.068268in}}{\pgfqpoint{1.097163in}{1.064361in}}%
\pgfpathcurveto{\pgfqpoint{1.093256in}{1.060454in}}{\pgfqpoint{1.091061in}{1.055155in}}{\pgfqpoint{1.091061in}{1.049630in}}%
\pgfpathcurveto{\pgfqpoint{1.091061in}{1.044105in}}{\pgfqpoint{1.093256in}{1.038805in}}{\pgfqpoint{1.097163in}{1.034898in}}%
\pgfpathcurveto{\pgfqpoint{1.101069in}{1.030992in}}{\pgfqpoint{1.106369in}{1.028797in}}{\pgfqpoint{1.111894in}{1.028797in}}%
\pgfpathclose%
\pgfusepath{stroke}%
\end{pgfscope}%
\begin{pgfscope}%
\pgfpathrectangle{\pgfqpoint{0.438556in}{0.383578in}}{\pgfqpoint{4.650000in}{2.310000in}}%
\pgfusepath{clip}%
\pgfsetbuttcap%
\pgfsetroundjoin%
\pgfsetlinewidth{0.803000pt}%
\definecolor{currentstroke}{rgb}{0.686275,0.352941,0.313725}%
\pgfsetstrokecolor{currentstroke}%
\pgfsetdash{}{0pt}%
\pgfpathmoveto{\pgfqpoint{1.113488in}{1.072481in}}%
\pgfpathcurveto{\pgfqpoint{1.119013in}{1.072481in}}{\pgfqpoint{1.124313in}{1.074676in}}{\pgfqpoint{1.128220in}{1.078583in}}%
\pgfpathcurveto{\pgfqpoint{1.132127in}{1.082490in}}{\pgfqpoint{1.134322in}{1.087789in}}{\pgfqpoint{1.134322in}{1.093314in}}%
\pgfpathcurveto{\pgfqpoint{1.134322in}{1.098839in}}{\pgfqpoint{1.132127in}{1.104139in}}{\pgfqpoint{1.128220in}{1.108045in}}%
\pgfpathcurveto{\pgfqpoint{1.124313in}{1.111952in}}{\pgfqpoint{1.119013in}{1.114147in}}{\pgfqpoint{1.113488in}{1.114147in}}%
\pgfpathcurveto{\pgfqpoint{1.107963in}{1.114147in}}{\pgfqpoint{1.102664in}{1.111952in}}{\pgfqpoint{1.098757in}{1.108045in}}%
\pgfpathcurveto{\pgfqpoint{1.094850in}{1.104139in}}{\pgfqpoint{1.092655in}{1.098839in}}{\pgfqpoint{1.092655in}{1.093314in}}%
\pgfpathcurveto{\pgfqpoint{1.092655in}{1.087789in}}{\pgfqpoint{1.094850in}{1.082490in}}{\pgfqpoint{1.098757in}{1.078583in}}%
\pgfpathcurveto{\pgfqpoint{1.102664in}{1.074676in}}{\pgfqpoint{1.107963in}{1.072481in}}{\pgfqpoint{1.113488in}{1.072481in}}%
\pgfpathclose%
\pgfusepath{stroke}%
\end{pgfscope}%
\begin{pgfscope}%
\pgfpathrectangle{\pgfqpoint{0.438556in}{0.383578in}}{\pgfqpoint{4.650000in}{2.310000in}}%
\pgfusepath{clip}%
\pgfsetbuttcap%
\pgfsetroundjoin%
\pgfsetlinewidth{0.803000pt}%
\definecolor{currentstroke}{rgb}{0.686275,0.352941,0.313725}%
\pgfsetstrokecolor{currentstroke}%
\pgfsetdash{}{0pt}%
\pgfpathmoveto{\pgfqpoint{1.162581in}{1.159849in}}%
\pgfpathcurveto{\pgfqpoint{1.168106in}{1.159849in}}{\pgfqpoint{1.173406in}{1.162044in}}{\pgfqpoint{1.177313in}{1.165951in}}%
\pgfpathcurveto{\pgfqpoint{1.181220in}{1.169858in}}{\pgfqpoint{1.183415in}{1.175158in}}{\pgfqpoint{1.183415in}{1.180683in}}%
\pgfpathcurveto{\pgfqpoint{1.183415in}{1.186208in}}{\pgfqpoint{1.181220in}{1.191507in}}{\pgfqpoint{1.177313in}{1.195414in}}%
\pgfpathcurveto{\pgfqpoint{1.173406in}{1.199321in}}{\pgfqpoint{1.168106in}{1.201516in}}{\pgfqpoint{1.162581in}{1.201516in}}%
\pgfpathcurveto{\pgfqpoint{1.157056in}{1.201516in}}{\pgfqpoint{1.151757in}{1.199321in}}{\pgfqpoint{1.147850in}{1.195414in}}%
\pgfpathcurveto{\pgfqpoint{1.143943in}{1.191507in}}{\pgfqpoint{1.141748in}{1.186208in}}{\pgfqpoint{1.141748in}{1.180683in}}%
\pgfpathcurveto{\pgfqpoint{1.141748in}{1.175158in}}{\pgfqpoint{1.143943in}{1.169858in}}{\pgfqpoint{1.147850in}{1.165951in}}%
\pgfpathcurveto{\pgfqpoint{1.151757in}{1.162044in}}{\pgfqpoint{1.157056in}{1.159849in}}{\pgfqpoint{1.162581in}{1.159849in}}%
\pgfpathclose%
\pgfusepath{stroke}%
\end{pgfscope}%
\begin{pgfscope}%
\pgfpathrectangle{\pgfqpoint{0.438556in}{0.383578in}}{\pgfqpoint{4.650000in}{2.310000in}}%
\pgfusepath{clip}%
\pgfsetbuttcap%
\pgfsetroundjoin%
\pgfsetlinewidth{0.803000pt}%
\definecolor{currentstroke}{rgb}{0.686275,0.352941,0.313725}%
\pgfsetstrokecolor{currentstroke}%
\pgfsetdash{}{0pt}%
\pgfpathmoveto{\pgfqpoint{1.108999in}{1.203534in}}%
\pgfpathcurveto{\pgfqpoint{1.114524in}{1.203534in}}{\pgfqpoint{1.119823in}{1.205729in}}{\pgfqpoint{1.123730in}{1.209636in}}%
\pgfpathcurveto{\pgfqpoint{1.127637in}{1.213542in}}{\pgfqpoint{1.129832in}{1.218842in}}{\pgfqpoint{1.129832in}{1.224367in}}%
\pgfpathcurveto{\pgfqpoint{1.129832in}{1.229892in}}{\pgfqpoint{1.127637in}{1.235191in}}{\pgfqpoint{1.123730in}{1.239098in}}%
\pgfpathcurveto{\pgfqpoint{1.119823in}{1.243005in}}{\pgfqpoint{1.114524in}{1.245200in}}{\pgfqpoint{1.108999in}{1.245200in}}%
\pgfpathcurveto{\pgfqpoint{1.103474in}{1.245200in}}{\pgfqpoint{1.098174in}{1.243005in}}{\pgfqpoint{1.094267in}{1.239098in}}%
\pgfpathcurveto{\pgfqpoint{1.090360in}{1.235191in}}{\pgfqpoint{1.088165in}{1.229892in}}{\pgfqpoint{1.088165in}{1.224367in}}%
\pgfpathcurveto{\pgfqpoint{1.088165in}{1.218842in}}{\pgfqpoint{1.090360in}{1.213542in}}{\pgfqpoint{1.094267in}{1.209636in}}%
\pgfpathcurveto{\pgfqpoint{1.098174in}{1.205729in}}{\pgfqpoint{1.103474in}{1.203534in}}{\pgfqpoint{1.108999in}{1.203534in}}%
\pgfpathclose%
\pgfusepath{stroke}%
\end{pgfscope}%
\begin{pgfscope}%
\pgfpathrectangle{\pgfqpoint{0.438556in}{0.383578in}}{\pgfqpoint{4.650000in}{2.310000in}}%
\pgfusepath{clip}%
\pgfsetbuttcap%
\pgfsetroundjoin%
\pgfsetlinewidth{0.803000pt}%
\definecolor{currentstroke}{rgb}{0.686275,0.352941,0.313725}%
\pgfsetstrokecolor{currentstroke}%
\pgfsetdash{}{0pt}%
\pgfpathmoveto{\pgfqpoint{1.102841in}{1.290902in}}%
\pgfpathcurveto{\pgfqpoint{1.108366in}{1.290902in}}{\pgfqpoint{1.113666in}{1.293097in}}{\pgfqpoint{1.117573in}{1.297004in}}%
\pgfpathcurveto{\pgfqpoint{1.121480in}{1.300911in}}{\pgfqpoint{1.123675in}{1.306210in}}{\pgfqpoint{1.123675in}{1.311735in}}%
\pgfpathcurveto{\pgfqpoint{1.123675in}{1.317260in}}{\pgfqpoint{1.121480in}{1.322560in}}{\pgfqpoint{1.117573in}{1.326467in}}%
\pgfpathcurveto{\pgfqpoint{1.113666in}{1.330374in}}{\pgfqpoint{1.108366in}{1.332569in}}{\pgfqpoint{1.102841in}{1.332569in}}%
\pgfpathcurveto{\pgfqpoint{1.097316in}{1.332569in}}{\pgfqpoint{1.092017in}{1.330374in}}{\pgfqpoint{1.088110in}{1.326467in}}%
\pgfpathcurveto{\pgfqpoint{1.084203in}{1.322560in}}{\pgfqpoint{1.082008in}{1.317260in}}{\pgfqpoint{1.082008in}{1.311735in}}%
\pgfpathcurveto{\pgfqpoint{1.082008in}{1.306210in}}{\pgfqpoint{1.084203in}{1.300911in}}{\pgfqpoint{1.088110in}{1.297004in}}%
\pgfpathcurveto{\pgfqpoint{1.092017in}{1.293097in}}{\pgfqpoint{1.097316in}{1.290902in}}{\pgfqpoint{1.102841in}{1.290902in}}%
\pgfpathclose%
\pgfusepath{stroke}%
\end{pgfscope}%
\begin{pgfscope}%
\pgfpathrectangle{\pgfqpoint{0.438556in}{0.383578in}}{\pgfqpoint{4.650000in}{2.310000in}}%
\pgfusepath{clip}%
\pgfsetbuttcap%
\pgfsetroundjoin%
\pgfsetlinewidth{0.803000pt}%
\definecolor{currentstroke}{rgb}{0.686275,0.352941,0.313725}%
\pgfsetstrokecolor{currentstroke}%
\pgfsetdash{}{0pt}%
\pgfpathmoveto{\pgfqpoint{1.084076in}{1.312744in}}%
\pgfpathcurveto{\pgfqpoint{1.089602in}{1.312744in}}{\pgfqpoint{1.094901in}{1.314939in}}{\pgfqpoint{1.098808in}{1.318846in}}%
\pgfpathcurveto{\pgfqpoint{1.102715in}{1.322753in}}{\pgfqpoint{1.104910in}{1.328052in}}{\pgfqpoint{1.104910in}{1.333578in}}%
\pgfpathcurveto{\pgfqpoint{1.104910in}{1.339103in}}{\pgfqpoint{1.102715in}{1.344402in}}{\pgfqpoint{1.098808in}{1.348309in}}%
\pgfpathcurveto{\pgfqpoint{1.094901in}{1.352216in}}{\pgfqpoint{1.089602in}{1.354411in}}{\pgfqpoint{1.084076in}{1.354411in}}%
\pgfpathcurveto{\pgfqpoint{1.078551in}{1.354411in}}{\pgfqpoint{1.073252in}{1.352216in}}{\pgfqpoint{1.069345in}{1.348309in}}%
\pgfpathcurveto{\pgfqpoint{1.065438in}{1.344402in}}{\pgfqpoint{1.063243in}{1.339103in}}{\pgfqpoint{1.063243in}{1.333578in}}%
\pgfpathcurveto{\pgfqpoint{1.063243in}{1.328052in}}{\pgfqpoint{1.065438in}{1.322753in}}{\pgfqpoint{1.069345in}{1.318846in}}%
\pgfpathcurveto{\pgfqpoint{1.073252in}{1.314939in}}{\pgfqpoint{1.078551in}{1.312744in}}{\pgfqpoint{1.084076in}{1.312744in}}%
\pgfpathclose%
\pgfusepath{stroke}%
\end{pgfscope}%
\begin{pgfscope}%
\pgfpathrectangle{\pgfqpoint{0.438556in}{0.383578in}}{\pgfqpoint{4.650000in}{2.310000in}}%
\pgfusepath{clip}%
\pgfsetbuttcap%
\pgfsetroundjoin%
\pgfsetlinewidth{0.803000pt}%
\definecolor{currentstroke}{rgb}{0.686275,0.352941,0.313725}%
\pgfsetstrokecolor{currentstroke}%
\pgfsetdash{}{0pt}%
\pgfpathmoveto{\pgfqpoint{1.156955in}{1.334586in}}%
\pgfpathcurveto{\pgfqpoint{1.162481in}{1.334586in}}{\pgfqpoint{1.167780in}{1.336781in}}{\pgfqpoint{1.171687in}{1.340688in}}%
\pgfpathcurveto{\pgfqpoint{1.175594in}{1.344595in}}{\pgfqpoint{1.177789in}{1.349895in}}{\pgfqpoint{1.177789in}{1.355420in}}%
\pgfpathcurveto{\pgfqpoint{1.177789in}{1.360945in}}{\pgfqpoint{1.175594in}{1.366244in}}{\pgfqpoint{1.171687in}{1.370151in}}%
\pgfpathcurveto{\pgfqpoint{1.167780in}{1.374058in}}{\pgfqpoint{1.162481in}{1.376253in}}{\pgfqpoint{1.156955in}{1.376253in}}%
\pgfpathcurveto{\pgfqpoint{1.151430in}{1.376253in}}{\pgfqpoint{1.146131in}{1.374058in}}{\pgfqpoint{1.142224in}{1.370151in}}%
\pgfpathcurveto{\pgfqpoint{1.138317in}{1.366244in}}{\pgfqpoint{1.136122in}{1.360945in}}{\pgfqpoint{1.136122in}{1.355420in}}%
\pgfpathcurveto{\pgfqpoint{1.136122in}{1.349895in}}{\pgfqpoint{1.138317in}{1.344595in}}{\pgfqpoint{1.142224in}{1.340688in}}%
\pgfpathcurveto{\pgfqpoint{1.146131in}{1.336781in}}{\pgfqpoint{1.151430in}{1.334586in}}{\pgfqpoint{1.156955in}{1.334586in}}%
\pgfpathclose%
\pgfusepath{stroke}%
\end{pgfscope}%
\begin{pgfscope}%
\pgfpathrectangle{\pgfqpoint{0.438556in}{0.383578in}}{\pgfqpoint{4.650000in}{2.310000in}}%
\pgfusepath{clip}%
\pgfsetbuttcap%
\pgfsetroundjoin%
\pgfsetlinewidth{0.803000pt}%
\definecolor{currentstroke}{rgb}{0.686275,0.352941,0.313725}%
\pgfsetstrokecolor{currentstroke}%
\pgfsetdash{}{0pt}%
\pgfpathmoveto{\pgfqpoint{1.105251in}{1.443797in}}%
\pgfpathcurveto{\pgfqpoint{1.110776in}{1.443797in}}{\pgfqpoint{1.116076in}{1.445992in}}{\pgfqpoint{1.119983in}{1.449899in}}%
\pgfpathcurveto{\pgfqpoint{1.123889in}{1.453806in}}{\pgfqpoint{1.126084in}{1.459105in}}{\pgfqpoint{1.126084in}{1.464630in}}%
\pgfpathcurveto{\pgfqpoint{1.126084in}{1.470155in}}{\pgfqpoint{1.123889in}{1.475455in}}{\pgfqpoint{1.119983in}{1.479362in}}%
\pgfpathcurveto{\pgfqpoint{1.116076in}{1.483269in}}{\pgfqpoint{1.110776in}{1.485464in}}{\pgfqpoint{1.105251in}{1.485464in}}%
\pgfpathcurveto{\pgfqpoint{1.099726in}{1.485464in}}{\pgfqpoint{1.094427in}{1.483269in}}{\pgfqpoint{1.090520in}{1.479362in}}%
\pgfpathcurveto{\pgfqpoint{1.086613in}{1.475455in}}{\pgfqpoint{1.084418in}{1.470155in}}{\pgfqpoint{1.084418in}{1.464630in}}%
\pgfpathcurveto{\pgfqpoint{1.084418in}{1.459105in}}{\pgfqpoint{1.086613in}{1.453806in}}{\pgfqpoint{1.090520in}{1.449899in}}%
\pgfpathcurveto{\pgfqpoint{1.094427in}{1.445992in}}{\pgfqpoint{1.099726in}{1.443797in}}{\pgfqpoint{1.105251in}{1.443797in}}%
\pgfpathclose%
\pgfusepath{stroke}%
\end{pgfscope}%
\begin{pgfscope}%
\pgfpathrectangle{\pgfqpoint{0.438556in}{0.383578in}}{\pgfqpoint{4.650000in}{2.310000in}}%
\pgfusepath{clip}%
\pgfsetbuttcap%
\pgfsetroundjoin%
\pgfsetlinewidth{0.803000pt}%
\definecolor{currentstroke}{rgb}{0.686275,0.352941,0.313725}%
\pgfsetstrokecolor{currentstroke}%
\pgfsetdash{}{0pt}%
\pgfpathmoveto{\pgfqpoint{1.084351in}{1.487481in}}%
\pgfpathcurveto{\pgfqpoint{1.089876in}{1.487481in}}{\pgfqpoint{1.095176in}{1.489676in}}{\pgfqpoint{1.099083in}{1.493583in}}%
\pgfpathcurveto{\pgfqpoint{1.102990in}{1.497490in}}{\pgfqpoint{1.105185in}{1.502790in}}{\pgfqpoint{1.105185in}{1.508315in}}%
\pgfpathcurveto{\pgfqpoint{1.105185in}{1.513840in}}{\pgfqpoint{1.102990in}{1.519139in}}{\pgfqpoint{1.099083in}{1.523046in}}%
\pgfpathcurveto{\pgfqpoint{1.095176in}{1.526953in}}{\pgfqpoint{1.089876in}{1.529148in}}{\pgfqpoint{1.084351in}{1.529148in}}%
\pgfpathcurveto{\pgfqpoint{1.078826in}{1.529148in}}{\pgfqpoint{1.073527in}{1.526953in}}{\pgfqpoint{1.069620in}{1.523046in}}%
\pgfpathcurveto{\pgfqpoint{1.065713in}{1.519139in}}{\pgfqpoint{1.063518in}{1.513840in}}{\pgfqpoint{1.063518in}{1.508315in}}%
\pgfpathcurveto{\pgfqpoint{1.063518in}{1.502790in}}{\pgfqpoint{1.065713in}{1.497490in}}{\pgfqpoint{1.069620in}{1.493583in}}%
\pgfpathcurveto{\pgfqpoint{1.073527in}{1.489676in}}{\pgfqpoint{1.078826in}{1.487481in}}{\pgfqpoint{1.084351in}{1.487481in}}%
\pgfpathclose%
\pgfusepath{stroke}%
\end{pgfscope}%
\begin{pgfscope}%
\pgfpathrectangle{\pgfqpoint{0.438556in}{0.383578in}}{\pgfqpoint{4.650000in}{2.310000in}}%
\pgfusepath{clip}%
\pgfsetbuttcap%
\pgfsetroundjoin%
\pgfsetlinewidth{0.803000pt}%
\definecolor{currentstroke}{rgb}{0.686275,0.352941,0.313725}%
\pgfsetstrokecolor{currentstroke}%
\pgfsetdash{}{0pt}%
\pgfpathmoveto{\pgfqpoint{1.097921in}{1.531166in}}%
\pgfpathcurveto{\pgfqpoint{1.103446in}{1.531166in}}{\pgfqpoint{1.108746in}{1.533361in}}{\pgfqpoint{1.112653in}{1.537267in}}%
\pgfpathcurveto{\pgfqpoint{1.116559in}{1.541174in}}{\pgfqpoint{1.118754in}{1.546474in}}{\pgfqpoint{1.118754in}{1.551999in}}%
\pgfpathcurveto{\pgfqpoint{1.118754in}{1.557524in}}{\pgfqpoint{1.116559in}{1.562823in}}{\pgfqpoint{1.112653in}{1.566730in}}%
\pgfpathcurveto{\pgfqpoint{1.108746in}{1.570637in}}{\pgfqpoint{1.103446in}{1.572832in}}{\pgfqpoint{1.097921in}{1.572832in}}%
\pgfpathcurveto{\pgfqpoint{1.092396in}{1.572832in}}{\pgfqpoint{1.087097in}{1.570637in}}{\pgfqpoint{1.083190in}{1.566730in}}%
\pgfpathcurveto{\pgfqpoint{1.079283in}{1.562823in}}{\pgfqpoint{1.077088in}{1.557524in}}{\pgfqpoint{1.077088in}{1.551999in}}%
\pgfpathcurveto{\pgfqpoint{1.077088in}{1.546474in}}{\pgfqpoint{1.079283in}{1.541174in}}{\pgfqpoint{1.083190in}{1.537267in}}%
\pgfpathcurveto{\pgfqpoint{1.087097in}{1.533361in}}{\pgfqpoint{1.092396in}{1.531166in}}{\pgfqpoint{1.097921in}{1.531166in}}%
\pgfpathclose%
\pgfusepath{stroke}%
\end{pgfscope}%
\begin{pgfscope}%
\pgfpathrectangle{\pgfqpoint{0.438556in}{0.383578in}}{\pgfqpoint{4.650000in}{2.310000in}}%
\pgfusepath{clip}%
\pgfsetbuttcap%
\pgfsetroundjoin%
\pgfsetlinewidth{0.803000pt}%
\definecolor{currentstroke}{rgb}{0.686275,0.352941,0.313725}%
\pgfsetstrokecolor{currentstroke}%
\pgfsetdash{}{0pt}%
\pgfpathmoveto{\pgfqpoint{1.150303in}{1.684060in}}%
\pgfpathcurveto{\pgfqpoint{1.155829in}{1.684060in}}{\pgfqpoint{1.161128in}{1.686256in}}{\pgfqpoint{1.165035in}{1.690162in}}%
\pgfpathcurveto{\pgfqpoint{1.168942in}{1.694069in}}{\pgfqpoint{1.171137in}{1.699369in}}{\pgfqpoint{1.171137in}{1.704894in}}%
\pgfpathcurveto{\pgfqpoint{1.171137in}{1.710419in}}{\pgfqpoint{1.168942in}{1.715718in}}{\pgfqpoint{1.165035in}{1.719625in}}%
\pgfpathcurveto{\pgfqpoint{1.161128in}{1.723532in}}{\pgfqpoint{1.155829in}{1.725727in}}{\pgfqpoint{1.150303in}{1.725727in}}%
\pgfpathcurveto{\pgfqpoint{1.144778in}{1.725727in}}{\pgfqpoint{1.139479in}{1.723532in}}{\pgfqpoint{1.135572in}{1.719625in}}%
\pgfpathcurveto{\pgfqpoint{1.131665in}{1.715718in}}{\pgfqpoint{1.129470in}{1.710419in}}{\pgfqpoint{1.129470in}{1.704894in}}%
\pgfpathcurveto{\pgfqpoint{1.129470in}{1.699369in}}{\pgfqpoint{1.131665in}{1.694069in}}{\pgfqpoint{1.135572in}{1.690162in}}%
\pgfpathcurveto{\pgfqpoint{1.139479in}{1.686256in}}{\pgfqpoint{1.144778in}{1.684060in}}{\pgfqpoint{1.150303in}{1.684060in}}%
\pgfpathclose%
\pgfusepath{stroke}%
\end{pgfscope}%
\begin{pgfscope}%
\pgfpathrectangle{\pgfqpoint{0.438556in}{0.383578in}}{\pgfqpoint{4.650000in}{2.310000in}}%
\pgfusepath{clip}%
\pgfsetbuttcap%
\pgfsetroundjoin%
\pgfsetlinewidth{0.803000pt}%
\definecolor{currentstroke}{rgb}{0.686275,0.352941,0.313725}%
\pgfsetstrokecolor{currentstroke}%
\pgfsetdash{}{0pt}%
\pgfpathmoveto{\pgfqpoint{1.104885in}{1.771429in}}%
\pgfpathcurveto{\pgfqpoint{1.110410in}{1.771429in}}{\pgfqpoint{1.115709in}{1.773624in}}{\pgfqpoint{1.119616in}{1.777531in}}%
\pgfpathcurveto{\pgfqpoint{1.123523in}{1.781438in}}{\pgfqpoint{1.125718in}{1.786737in}}{\pgfqpoint{1.125718in}{1.792262in}}%
\pgfpathcurveto{\pgfqpoint{1.125718in}{1.797787in}}{\pgfqpoint{1.123523in}{1.803087in}}{\pgfqpoint{1.119616in}{1.806994in}}%
\pgfpathcurveto{\pgfqpoint{1.115709in}{1.810901in}}{\pgfqpoint{1.110410in}{1.813096in}}{\pgfqpoint{1.104885in}{1.813096in}}%
\pgfpathcurveto{\pgfqpoint{1.099360in}{1.813096in}}{\pgfqpoint{1.094060in}{1.810901in}}{\pgfqpoint{1.090153in}{1.806994in}}%
\pgfpathcurveto{\pgfqpoint{1.086246in}{1.803087in}}{\pgfqpoint{1.084051in}{1.797787in}}{\pgfqpoint{1.084051in}{1.792262in}}%
\pgfpathcurveto{\pgfqpoint{1.084051in}{1.786737in}}{\pgfqpoint{1.086246in}{1.781438in}}{\pgfqpoint{1.090153in}{1.777531in}}%
\pgfpathcurveto{\pgfqpoint{1.094060in}{1.773624in}}{\pgfqpoint{1.099360in}{1.771429in}}{\pgfqpoint{1.104885in}{1.771429in}}%
\pgfpathclose%
\pgfusepath{stroke}%
\end{pgfscope}%
\begin{pgfscope}%
\pgfpathrectangle{\pgfqpoint{0.438556in}{0.383578in}}{\pgfqpoint{4.650000in}{2.310000in}}%
\pgfusepath{clip}%
\pgfsetbuttcap%
\pgfsetroundjoin%
\pgfsetlinewidth{0.803000pt}%
\definecolor{currentstroke}{rgb}{0.686275,0.352941,0.313725}%
\pgfsetstrokecolor{currentstroke}%
\pgfsetdash{}{0pt}%
\pgfpathmoveto{\pgfqpoint{1.090234in}{1.793271in}}%
\pgfpathcurveto{\pgfqpoint{1.095759in}{1.793271in}}{\pgfqpoint{1.101058in}{1.795466in}}{\pgfqpoint{1.104965in}{1.799373in}}%
\pgfpathcurveto{\pgfqpoint{1.108872in}{1.803280in}}{\pgfqpoint{1.111067in}{1.808579in}}{\pgfqpoint{1.111067in}{1.814104in}}%
\pgfpathcurveto{\pgfqpoint{1.111067in}{1.819630in}}{\pgfqpoint{1.108872in}{1.824929in}}{\pgfqpoint{1.104965in}{1.828836in}}%
\pgfpathcurveto{\pgfqpoint{1.101058in}{1.832743in}}{\pgfqpoint{1.095759in}{1.834938in}}{\pgfqpoint{1.090234in}{1.834938in}}%
\pgfpathcurveto{\pgfqpoint{1.084709in}{1.834938in}}{\pgfqpoint{1.079409in}{1.832743in}}{\pgfqpoint{1.075502in}{1.828836in}}%
\pgfpathcurveto{\pgfqpoint{1.071596in}{1.824929in}}{\pgfqpoint{1.069400in}{1.819630in}}{\pgfqpoint{1.069400in}{1.814104in}}%
\pgfpathcurveto{\pgfqpoint{1.069400in}{1.808579in}}{\pgfqpoint{1.071596in}{1.803280in}}{\pgfqpoint{1.075502in}{1.799373in}}%
\pgfpathcurveto{\pgfqpoint{1.079409in}{1.795466in}}{\pgfqpoint{1.084709in}{1.793271in}}{\pgfqpoint{1.090234in}{1.793271in}}%
\pgfpathclose%
\pgfusepath{stroke}%
\end{pgfscope}%
\begin{pgfscope}%
\pgfpathrectangle{\pgfqpoint{0.438556in}{0.383578in}}{\pgfqpoint{4.650000in}{2.310000in}}%
\pgfusepath{clip}%
\pgfsetbuttcap%
\pgfsetroundjoin%
\pgfsetlinewidth{0.803000pt}%
\definecolor{currentstroke}{rgb}{0.686275,0.352941,0.313725}%
\pgfsetstrokecolor{currentstroke}%
\pgfsetdash{}{0pt}%
\pgfpathmoveto{\pgfqpoint{1.125455in}{1.836955in}}%
\pgfpathcurveto{\pgfqpoint{1.130980in}{1.836955in}}{\pgfqpoint{1.136279in}{1.839151in}}{\pgfqpoint{1.140186in}{1.843057in}}%
\pgfpathcurveto{\pgfqpoint{1.144093in}{1.846964in}}{\pgfqpoint{1.146288in}{1.852264in}}{\pgfqpoint{1.146288in}{1.857789in}}%
\pgfpathcurveto{\pgfqpoint{1.146288in}{1.863314in}}{\pgfqpoint{1.144093in}{1.868613in}}{\pgfqpoint{1.140186in}{1.872520in}}%
\pgfpathcurveto{\pgfqpoint{1.136279in}{1.876427in}}{\pgfqpoint{1.130980in}{1.878622in}}{\pgfqpoint{1.125455in}{1.878622in}}%
\pgfpathcurveto{\pgfqpoint{1.119930in}{1.878622in}}{\pgfqpoint{1.114630in}{1.876427in}}{\pgfqpoint{1.110723in}{1.872520in}}%
\pgfpathcurveto{\pgfqpoint{1.106816in}{1.868613in}}{\pgfqpoint{1.104621in}{1.863314in}}{\pgfqpoint{1.104621in}{1.857789in}}%
\pgfpathcurveto{\pgfqpoint{1.104621in}{1.852264in}}{\pgfqpoint{1.106816in}{1.846964in}}{\pgfqpoint{1.110723in}{1.843057in}}%
\pgfpathcurveto{\pgfqpoint{1.114630in}{1.839151in}}{\pgfqpoint{1.119930in}{1.836955in}}{\pgfqpoint{1.125455in}{1.836955in}}%
\pgfpathclose%
\pgfusepath{stroke}%
\end{pgfscope}%
\begin{pgfscope}%
\pgfpathrectangle{\pgfqpoint{0.438556in}{0.383578in}}{\pgfqpoint{4.650000in}{2.310000in}}%
\pgfusepath{clip}%
\pgfsetbuttcap%
\pgfsetroundjoin%
\pgfsetlinewidth{0.803000pt}%
\definecolor{currentstroke}{rgb}{0.686275,0.352941,0.313725}%
\pgfsetstrokecolor{currentstroke}%
\pgfsetdash{}{0pt}%
\pgfpathmoveto{\pgfqpoint{1.129761in}{1.858798in}}%
\pgfpathcurveto{\pgfqpoint{1.135286in}{1.858798in}}{\pgfqpoint{1.140586in}{1.860993in}}{\pgfqpoint{1.144492in}{1.864899in}}%
\pgfpathcurveto{\pgfqpoint{1.148399in}{1.868806in}}{\pgfqpoint{1.150594in}{1.874106in}}{\pgfqpoint{1.150594in}{1.879631in}}%
\pgfpathcurveto{\pgfqpoint{1.150594in}{1.885156in}}{\pgfqpoint{1.148399in}{1.890455in}}{\pgfqpoint{1.144492in}{1.894362in}}%
\pgfpathcurveto{\pgfqpoint{1.140586in}{1.898269in}}{\pgfqpoint{1.135286in}{1.900464in}}{\pgfqpoint{1.129761in}{1.900464in}}%
\pgfpathcurveto{\pgfqpoint{1.124236in}{1.900464in}}{\pgfqpoint{1.118936in}{1.898269in}}{\pgfqpoint{1.115030in}{1.894362in}}%
\pgfpathcurveto{\pgfqpoint{1.111123in}{1.890455in}}{\pgfqpoint{1.108928in}{1.885156in}}{\pgfqpoint{1.108928in}{1.879631in}}%
\pgfpathcurveto{\pgfqpoint{1.108928in}{1.874106in}}{\pgfqpoint{1.111123in}{1.868806in}}{\pgfqpoint{1.115030in}{1.864899in}}%
\pgfpathcurveto{\pgfqpoint{1.118936in}{1.860993in}}{\pgfqpoint{1.124236in}{1.858798in}}{\pgfqpoint{1.129761in}{1.858798in}}%
\pgfpathclose%
\pgfusepath{stroke}%
\end{pgfscope}%
\begin{pgfscope}%
\pgfpathrectangle{\pgfqpoint{0.438556in}{0.383578in}}{\pgfqpoint{4.650000in}{2.310000in}}%
\pgfusepath{clip}%
\pgfsetbuttcap%
\pgfsetroundjoin%
\pgfsetlinewidth{0.803000pt}%
\definecolor{currentstroke}{rgb}{0.686275,0.352941,0.313725}%
\pgfsetstrokecolor{currentstroke}%
\pgfsetdash{}{0pt}%
\pgfpathmoveto{\pgfqpoint{1.120424in}{1.924324in}}%
\pgfpathcurveto{\pgfqpoint{1.125949in}{1.924324in}}{\pgfqpoint{1.131249in}{1.926519in}}{\pgfqpoint{1.135156in}{1.930426in}}%
\pgfpathcurveto{\pgfqpoint{1.139063in}{1.934333in}}{\pgfqpoint{1.141258in}{1.939632in}}{\pgfqpoint{1.141258in}{1.945157in}}%
\pgfpathcurveto{\pgfqpoint{1.141258in}{1.950682in}}{\pgfqpoint{1.139063in}{1.955982in}}{\pgfqpoint{1.135156in}{1.959889in}}%
\pgfpathcurveto{\pgfqpoint{1.131249in}{1.963795in}}{\pgfqpoint{1.125949in}{1.965991in}}{\pgfqpoint{1.120424in}{1.965991in}}%
\pgfpathcurveto{\pgfqpoint{1.114899in}{1.965991in}}{\pgfqpoint{1.109600in}{1.963795in}}{\pgfqpoint{1.105693in}{1.959889in}}%
\pgfpathcurveto{\pgfqpoint{1.101786in}{1.955982in}}{\pgfqpoint{1.099591in}{1.950682in}}{\pgfqpoint{1.099591in}{1.945157in}}%
\pgfpathcurveto{\pgfqpoint{1.099591in}{1.939632in}}{\pgfqpoint{1.101786in}{1.934333in}}{\pgfqpoint{1.105693in}{1.930426in}}%
\pgfpathcurveto{\pgfqpoint{1.109600in}{1.926519in}}{\pgfqpoint{1.114899in}{1.924324in}}{\pgfqpoint{1.120424in}{1.924324in}}%
\pgfpathclose%
\pgfusepath{stroke}%
\end{pgfscope}%
\begin{pgfscope}%
\pgfpathrectangle{\pgfqpoint{0.438556in}{0.383578in}}{\pgfqpoint{4.650000in}{2.310000in}}%
\pgfusepath{clip}%
\pgfsetbuttcap%
\pgfsetroundjoin%
\pgfsetlinewidth{0.803000pt}%
\definecolor{currentstroke}{rgb}{0.686275,0.352941,0.313725}%
\pgfsetstrokecolor{currentstroke}%
\pgfsetdash{}{0pt}%
\pgfpathmoveto{\pgfqpoint{1.097600in}{1.989850in}}%
\pgfpathcurveto{\pgfqpoint{1.103125in}{1.989850in}}{\pgfqpoint{1.108425in}{1.992045in}}{\pgfqpoint{1.112332in}{1.995952in}}%
\pgfpathcurveto{\pgfqpoint{1.116239in}{1.999859in}}{\pgfqpoint{1.118434in}{2.005159in}}{\pgfqpoint{1.118434in}{2.010684in}}%
\pgfpathcurveto{\pgfqpoint{1.118434in}{2.016209in}}{\pgfqpoint{1.116239in}{2.021508in}}{\pgfqpoint{1.112332in}{2.025415in}}%
\pgfpathcurveto{\pgfqpoint{1.108425in}{2.029322in}}{\pgfqpoint{1.103125in}{2.031517in}}{\pgfqpoint{1.097600in}{2.031517in}}%
\pgfpathcurveto{\pgfqpoint{1.092075in}{2.031517in}}{\pgfqpoint{1.086776in}{2.029322in}}{\pgfqpoint{1.082869in}{2.025415in}}%
\pgfpathcurveto{\pgfqpoint{1.078962in}{2.021508in}}{\pgfqpoint{1.076767in}{2.016209in}}{\pgfqpoint{1.076767in}{2.010684in}}%
\pgfpathcurveto{\pgfqpoint{1.076767in}{2.005159in}}{\pgfqpoint{1.078962in}{1.999859in}}{\pgfqpoint{1.082869in}{1.995952in}}%
\pgfpathcurveto{\pgfqpoint{1.086776in}{1.992045in}}{\pgfqpoint{1.092075in}{1.989850in}}{\pgfqpoint{1.097600in}{1.989850in}}%
\pgfpathclose%
\pgfusepath{stroke}%
\end{pgfscope}%
\begin{pgfscope}%
\pgfpathrectangle{\pgfqpoint{0.438556in}{0.383578in}}{\pgfqpoint{4.650000in}{2.310000in}}%
\pgfusepath{clip}%
\pgfsetbuttcap%
\pgfsetroundjoin%
\pgfsetlinewidth{0.803000pt}%
\definecolor{currentstroke}{rgb}{0.686275,0.352941,0.313725}%
\pgfsetstrokecolor{currentstroke}%
\pgfsetdash{}{0pt}%
\pgfpathmoveto{\pgfqpoint{1.121826in}{2.011692in}}%
\pgfpathcurveto{\pgfqpoint{1.127351in}{2.011692in}}{\pgfqpoint{1.132651in}{2.013888in}}{\pgfqpoint{1.136558in}{2.017794in}}%
\pgfpathcurveto{\pgfqpoint{1.140464in}{2.021701in}}{\pgfqpoint{1.142660in}{2.027001in}}{\pgfqpoint{1.142660in}{2.032526in}}%
\pgfpathcurveto{\pgfqpoint{1.142660in}{2.038051in}}{\pgfqpoint{1.140464in}{2.043350in}}{\pgfqpoint{1.136558in}{2.047257in}}%
\pgfpathcurveto{\pgfqpoint{1.132651in}{2.051164in}}{\pgfqpoint{1.127351in}{2.053359in}}{\pgfqpoint{1.121826in}{2.053359in}}%
\pgfpathcurveto{\pgfqpoint{1.116301in}{2.053359in}}{\pgfqpoint{1.111002in}{2.051164in}}{\pgfqpoint{1.107095in}{2.047257in}}%
\pgfpathcurveto{\pgfqpoint{1.103188in}{2.043350in}}{\pgfqpoint{1.100993in}{2.038051in}}{\pgfqpoint{1.100993in}{2.032526in}}%
\pgfpathcurveto{\pgfqpoint{1.100993in}{2.027001in}}{\pgfqpoint{1.103188in}{2.021701in}}{\pgfqpoint{1.107095in}{2.017794in}}%
\pgfpathcurveto{\pgfqpoint{1.111002in}{2.013888in}}{\pgfqpoint{1.116301in}{2.011692in}}{\pgfqpoint{1.121826in}{2.011692in}}%
\pgfpathclose%
\pgfusepath{stroke}%
\end{pgfscope}%
\begin{pgfscope}%
\pgfpathrectangle{\pgfqpoint{0.438556in}{0.383578in}}{\pgfqpoint{4.650000in}{2.310000in}}%
\pgfusepath{clip}%
\pgfsetbuttcap%
\pgfsetroundjoin%
\pgfsetlinewidth{0.803000pt}%
\definecolor{currentstroke}{rgb}{0.686275,0.352941,0.313725}%
\pgfsetstrokecolor{currentstroke}%
\pgfsetdash{}{0pt}%
\pgfpathmoveto{\pgfqpoint{1.101623in}{2.077219in}}%
\pgfpathcurveto{\pgfqpoint{1.107148in}{2.077219in}}{\pgfqpoint{1.112447in}{2.079414in}}{\pgfqpoint{1.116354in}{2.083321in}}%
\pgfpathcurveto{\pgfqpoint{1.120261in}{2.087228in}}{\pgfqpoint{1.122456in}{2.092527in}}{\pgfqpoint{1.122456in}{2.098052in}}%
\pgfpathcurveto{\pgfqpoint{1.122456in}{2.103577in}}{\pgfqpoint{1.120261in}{2.108877in}}{\pgfqpoint{1.116354in}{2.112784in}}%
\pgfpathcurveto{\pgfqpoint{1.112447in}{2.116690in}}{\pgfqpoint{1.107148in}{2.118885in}}{\pgfqpoint{1.101623in}{2.118885in}}%
\pgfpathcurveto{\pgfqpoint{1.096098in}{2.118885in}}{\pgfqpoint{1.090798in}{2.116690in}}{\pgfqpoint{1.086891in}{2.112784in}}%
\pgfpathcurveto{\pgfqpoint{1.082985in}{2.108877in}}{\pgfqpoint{1.080789in}{2.103577in}}{\pgfqpoint{1.080789in}{2.098052in}}%
\pgfpathcurveto{\pgfqpoint{1.080789in}{2.092527in}}{\pgfqpoint{1.082985in}{2.087228in}}{\pgfqpoint{1.086891in}{2.083321in}}%
\pgfpathcurveto{\pgfqpoint{1.090798in}{2.079414in}}{\pgfqpoint{1.096098in}{2.077219in}}{\pgfqpoint{1.101623in}{2.077219in}}%
\pgfpathclose%
\pgfusepath{stroke}%
\end{pgfscope}%
\begin{pgfscope}%
\pgfpathrectangle{\pgfqpoint{0.438556in}{0.383578in}}{\pgfqpoint{4.650000in}{2.310000in}}%
\pgfusepath{clip}%
\pgfsetbuttcap%
\pgfsetroundjoin%
\pgfsetlinewidth{0.803000pt}%
\definecolor{currentstroke}{rgb}{0.686275,0.352941,0.313725}%
\pgfsetstrokecolor{currentstroke}%
\pgfsetdash{}{0pt}%
\pgfpathmoveto{\pgfqpoint{1.139400in}{2.099061in}}%
\pgfpathcurveto{\pgfqpoint{1.144925in}{2.099061in}}{\pgfqpoint{1.150225in}{2.101256in}}{\pgfqpoint{1.154131in}{2.105163in}}%
\pgfpathcurveto{\pgfqpoint{1.158038in}{2.109070in}}{\pgfqpoint{1.160233in}{2.114369in}}{\pgfqpoint{1.160233in}{2.119894in}}%
\pgfpathcurveto{\pgfqpoint{1.160233in}{2.125419in}}{\pgfqpoint{1.158038in}{2.130719in}}{\pgfqpoint{1.154131in}{2.134626in}}%
\pgfpathcurveto{\pgfqpoint{1.150225in}{2.138532in}}{\pgfqpoint{1.144925in}{2.140728in}}{\pgfqpoint{1.139400in}{2.140728in}}%
\pgfpathcurveto{\pgfqpoint{1.133875in}{2.140728in}}{\pgfqpoint{1.128575in}{2.138532in}}{\pgfqpoint{1.124669in}{2.134626in}}%
\pgfpathcurveto{\pgfqpoint{1.120762in}{2.130719in}}{\pgfqpoint{1.118567in}{2.125419in}}{\pgfqpoint{1.118567in}{2.119894in}}%
\pgfpathcurveto{\pgfqpoint{1.118567in}{2.114369in}}{\pgfqpoint{1.120762in}{2.109070in}}{\pgfqpoint{1.124669in}{2.105163in}}%
\pgfpathcurveto{\pgfqpoint{1.128575in}{2.101256in}}{\pgfqpoint{1.133875in}{2.099061in}}{\pgfqpoint{1.139400in}{2.099061in}}%
\pgfpathclose%
\pgfusepath{stroke}%
\end{pgfscope}%
\begin{pgfscope}%
\pgfpathrectangle{\pgfqpoint{0.438556in}{0.383578in}}{\pgfqpoint{4.650000in}{2.310000in}}%
\pgfusepath{clip}%
\pgfsetbuttcap%
\pgfsetroundjoin%
\pgfsetlinewidth{0.803000pt}%
\definecolor{currentstroke}{rgb}{0.686275,0.352941,0.313725}%
\pgfsetstrokecolor{currentstroke}%
\pgfsetdash{}{0pt}%
\pgfpathmoveto{\pgfqpoint{1.161299in}{2.164587in}}%
\pgfpathcurveto{\pgfqpoint{1.166824in}{2.164587in}}{\pgfqpoint{1.172123in}{2.166782in}}{\pgfqpoint{1.176030in}{2.170689in}}%
\pgfpathcurveto{\pgfqpoint{1.179937in}{2.174596in}}{\pgfqpoint{1.182132in}{2.179896in}}{\pgfqpoint{1.182132in}{2.185421in}}%
\pgfpathcurveto{\pgfqpoint{1.182132in}{2.190946in}}{\pgfqpoint{1.179937in}{2.196245in}}{\pgfqpoint{1.176030in}{2.200152in}}%
\pgfpathcurveto{\pgfqpoint{1.172123in}{2.204059in}}{\pgfqpoint{1.166824in}{2.206254in}}{\pgfqpoint{1.161299in}{2.206254in}}%
\pgfpathcurveto{\pgfqpoint{1.155773in}{2.206254in}}{\pgfqpoint{1.150474in}{2.204059in}}{\pgfqpoint{1.146567in}{2.200152in}}%
\pgfpathcurveto{\pgfqpoint{1.142660in}{2.196245in}}{\pgfqpoint{1.140465in}{2.190946in}}{\pgfqpoint{1.140465in}{2.185421in}}%
\pgfpathcurveto{\pgfqpoint{1.140465in}{2.179896in}}{\pgfqpoint{1.142660in}{2.174596in}}{\pgfqpoint{1.146567in}{2.170689in}}%
\pgfpathcurveto{\pgfqpoint{1.150474in}{2.166782in}}{\pgfqpoint{1.155773in}{2.164587in}}{\pgfqpoint{1.161299in}{2.164587in}}%
\pgfpathclose%
\pgfusepath{stroke}%
\end{pgfscope}%
\begin{pgfscope}%
\pgfpathrectangle{\pgfqpoint{0.438556in}{0.383578in}}{\pgfqpoint{4.650000in}{2.310000in}}%
\pgfusepath{clip}%
\pgfsetbuttcap%
\pgfsetroundjoin%
\pgfsetlinewidth{0.803000pt}%
\definecolor{currentstroke}{rgb}{0.686275,0.352941,0.313725}%
\pgfsetstrokecolor{currentstroke}%
\pgfsetdash{}{0pt}%
\pgfpathmoveto{\pgfqpoint{1.129871in}{2.230114in}}%
\pgfpathcurveto{\pgfqpoint{1.135396in}{2.230114in}}{\pgfqpoint{1.140696in}{2.232309in}}{\pgfqpoint{1.144602in}{2.236216in}}%
\pgfpathcurveto{\pgfqpoint{1.148509in}{2.240123in}}{\pgfqpoint{1.150704in}{2.245422in}}{\pgfqpoint{1.150704in}{2.250947in}}%
\pgfpathcurveto{\pgfqpoint{1.150704in}{2.256472in}}{\pgfqpoint{1.148509in}{2.261772in}}{\pgfqpoint{1.144602in}{2.265678in}}%
\pgfpathcurveto{\pgfqpoint{1.140696in}{2.269585in}}{\pgfqpoint{1.135396in}{2.271780in}}{\pgfqpoint{1.129871in}{2.271780in}}%
\pgfpathcurveto{\pgfqpoint{1.124346in}{2.271780in}}{\pgfqpoint{1.119046in}{2.269585in}}{\pgfqpoint{1.115140in}{2.265678in}}%
\pgfpathcurveto{\pgfqpoint{1.111233in}{2.261772in}}{\pgfqpoint{1.109038in}{2.256472in}}{\pgfqpoint{1.109038in}{2.250947in}}%
\pgfpathcurveto{\pgfqpoint{1.109038in}{2.245422in}}{\pgfqpoint{1.111233in}{2.240123in}}{\pgfqpoint{1.115140in}{2.236216in}}%
\pgfpathcurveto{\pgfqpoint{1.119046in}{2.232309in}}{\pgfqpoint{1.124346in}{2.230114in}}{\pgfqpoint{1.129871in}{2.230114in}}%
\pgfpathclose%
\pgfusepath{stroke}%
\end{pgfscope}%
\begin{pgfscope}%
\pgfpathrectangle{\pgfqpoint{0.438556in}{0.383578in}}{\pgfqpoint{4.650000in}{2.310000in}}%
\pgfusepath{clip}%
\pgfsetbuttcap%
\pgfsetroundjoin%
\pgfsetlinewidth{0.803000pt}%
\definecolor{currentstroke}{rgb}{0.686275,0.352941,0.313725}%
\pgfsetstrokecolor{currentstroke}%
\pgfsetdash{}{0pt}%
\pgfpathmoveto{\pgfqpoint{1.120003in}{2.251956in}}%
\pgfpathcurveto{\pgfqpoint{1.125528in}{2.251956in}}{\pgfqpoint{1.130827in}{2.254151in}}{\pgfqpoint{1.134734in}{2.258058in}}%
\pgfpathcurveto{\pgfqpoint{1.138641in}{2.261965in}}{\pgfqpoint{1.140836in}{2.267264in}}{\pgfqpoint{1.140836in}{2.272789in}}%
\pgfpathcurveto{\pgfqpoint{1.140836in}{2.278314in}}{\pgfqpoint{1.138641in}{2.283614in}}{\pgfqpoint{1.134734in}{2.287521in}}%
\pgfpathcurveto{\pgfqpoint{1.130827in}{2.291427in}}{\pgfqpoint{1.125528in}{2.293623in}}{\pgfqpoint{1.120003in}{2.293623in}}%
\pgfpathcurveto{\pgfqpoint{1.114478in}{2.293623in}}{\pgfqpoint{1.109178in}{2.291427in}}{\pgfqpoint{1.105271in}{2.287521in}}%
\pgfpathcurveto{\pgfqpoint{1.101365in}{2.283614in}}{\pgfqpoint{1.099170in}{2.278314in}}{\pgfqpoint{1.099170in}{2.272789in}}%
\pgfpathcurveto{\pgfqpoint{1.099170in}{2.267264in}}{\pgfqpoint{1.101365in}{2.261965in}}{\pgfqpoint{1.105271in}{2.258058in}}%
\pgfpathcurveto{\pgfqpoint{1.109178in}{2.254151in}}{\pgfqpoint{1.114478in}{2.251956in}}{\pgfqpoint{1.120003in}{2.251956in}}%
\pgfpathclose%
\pgfusepath{stroke}%
\end{pgfscope}%
\begin{pgfscope}%
\pgfpathrectangle{\pgfqpoint{0.438556in}{0.383578in}}{\pgfqpoint{4.650000in}{2.310000in}}%
\pgfusepath{clip}%
\pgfsetbuttcap%
\pgfsetroundjoin%
\pgfsetlinewidth{0.803000pt}%
\definecolor{currentstroke}{rgb}{0.686275,0.352941,0.313725}%
\pgfsetstrokecolor{currentstroke}%
\pgfsetdash{}{0pt}%
\pgfpathmoveto{\pgfqpoint{1.128716in}{2.295640in}}%
\pgfpathcurveto{\pgfqpoint{1.134242in}{2.295640in}}{\pgfqpoint{1.139541in}{2.297835in}}{\pgfqpoint{1.143448in}{2.301742in}}%
\pgfpathcurveto{\pgfqpoint{1.147355in}{2.305649in}}{\pgfqpoint{1.149550in}{2.310948in}}{\pgfqpoint{1.149550in}{2.316473in}}%
\pgfpathcurveto{\pgfqpoint{1.149550in}{2.321999in}}{\pgfqpoint{1.147355in}{2.327298in}}{\pgfqpoint{1.143448in}{2.331205in}}%
\pgfpathcurveto{\pgfqpoint{1.139541in}{2.335112in}}{\pgfqpoint{1.134242in}{2.337307in}}{\pgfqpoint{1.128716in}{2.337307in}}%
\pgfpathcurveto{\pgfqpoint{1.123191in}{2.337307in}}{\pgfqpoint{1.117892in}{2.335112in}}{\pgfqpoint{1.113985in}{2.331205in}}%
\pgfpathcurveto{\pgfqpoint{1.110078in}{2.327298in}}{\pgfqpoint{1.107883in}{2.321999in}}{\pgfqpoint{1.107883in}{2.316473in}}%
\pgfpathcurveto{\pgfqpoint{1.107883in}{2.310948in}}{\pgfqpoint{1.110078in}{2.305649in}}{\pgfqpoint{1.113985in}{2.301742in}}%
\pgfpathcurveto{\pgfqpoint{1.117892in}{2.297835in}}{\pgfqpoint{1.123191in}{2.295640in}}{\pgfqpoint{1.128716in}{2.295640in}}%
\pgfpathclose%
\pgfusepath{stroke}%
\end{pgfscope}%
\begin{pgfscope}%
\pgfpathrectangle{\pgfqpoint{0.438556in}{0.383578in}}{\pgfqpoint{4.650000in}{2.310000in}}%
\pgfusepath{clip}%
\pgfsetbuttcap%
\pgfsetroundjoin%
\pgfsetlinewidth{0.803000pt}%
\definecolor{currentstroke}{rgb}{0.686275,0.352941,0.313725}%
\pgfsetstrokecolor{currentstroke}%
\pgfsetdash{}{0pt}%
\pgfpathmoveto{\pgfqpoint{1.128515in}{2.317482in}}%
\pgfpathcurveto{\pgfqpoint{1.134040in}{2.317482in}}{\pgfqpoint{1.139339in}{2.319677in}}{\pgfqpoint{1.143246in}{2.323584in}}%
\pgfpathcurveto{\pgfqpoint{1.147153in}{2.327491in}}{\pgfqpoint{1.149348in}{2.332791in}}{\pgfqpoint{1.149348in}{2.338316in}}%
\pgfpathcurveto{\pgfqpoint{1.149348in}{2.343841in}}{\pgfqpoint{1.147153in}{2.349140in}}{\pgfqpoint{1.143246in}{2.353047in}}%
\pgfpathcurveto{\pgfqpoint{1.139339in}{2.356954in}}{\pgfqpoint{1.134040in}{2.359149in}}{\pgfqpoint{1.128515in}{2.359149in}}%
\pgfpathcurveto{\pgfqpoint{1.122990in}{2.359149in}}{\pgfqpoint{1.117690in}{2.356954in}}{\pgfqpoint{1.113784in}{2.353047in}}%
\pgfpathcurveto{\pgfqpoint{1.109877in}{2.349140in}}{\pgfqpoint{1.107682in}{2.343841in}}{\pgfqpoint{1.107682in}{2.338316in}}%
\pgfpathcurveto{\pgfqpoint{1.107682in}{2.332791in}}{\pgfqpoint{1.109877in}{2.327491in}}{\pgfqpoint{1.113784in}{2.323584in}}%
\pgfpathcurveto{\pgfqpoint{1.117690in}{2.319677in}}{\pgfqpoint{1.122990in}{2.317482in}}{\pgfqpoint{1.128515in}{2.317482in}}%
\pgfpathclose%
\pgfusepath{stroke}%
\end{pgfscope}%
\begin{pgfscope}%
\pgfpathrectangle{\pgfqpoint{0.438556in}{0.383578in}}{\pgfqpoint{4.650000in}{2.310000in}}%
\pgfusepath{clip}%
\pgfsetbuttcap%
\pgfsetroundjoin%
\pgfsetlinewidth{0.803000pt}%
\definecolor{currentstroke}{rgb}{0.686275,0.352941,0.313725}%
\pgfsetstrokecolor{currentstroke}%
\pgfsetdash{}{0pt}%
\pgfpathmoveto{\pgfqpoint{1.146620in}{2.339324in}}%
\pgfpathcurveto{\pgfqpoint{1.152145in}{2.339324in}}{\pgfqpoint{1.157445in}{2.341520in}}{\pgfqpoint{1.161352in}{2.345426in}}%
\pgfpathcurveto{\pgfqpoint{1.165258in}{2.349333in}}{\pgfqpoint{1.167453in}{2.354633in}}{\pgfqpoint{1.167453in}{2.360158in}}%
\pgfpathcurveto{\pgfqpoint{1.167453in}{2.365683in}}{\pgfqpoint{1.165258in}{2.370982in}}{\pgfqpoint{1.161352in}{2.374889in}}%
\pgfpathcurveto{\pgfqpoint{1.157445in}{2.378796in}}{\pgfqpoint{1.152145in}{2.380991in}}{\pgfqpoint{1.146620in}{2.380991in}}%
\pgfpathcurveto{\pgfqpoint{1.141095in}{2.380991in}}{\pgfqpoint{1.135796in}{2.378796in}}{\pgfqpoint{1.131889in}{2.374889in}}%
\pgfpathcurveto{\pgfqpoint{1.127982in}{2.370982in}}{\pgfqpoint{1.125787in}{2.365683in}}{\pgfqpoint{1.125787in}{2.360158in}}%
\pgfpathcurveto{\pgfqpoint{1.125787in}{2.354633in}}{\pgfqpoint{1.127982in}{2.349333in}}{\pgfqpoint{1.131889in}{2.345426in}}%
\pgfpathcurveto{\pgfqpoint{1.135796in}{2.341520in}}{\pgfqpoint{1.141095in}{2.339324in}}{\pgfqpoint{1.146620in}{2.339324in}}%
\pgfpathclose%
\pgfusepath{stroke}%
\end{pgfscope}%
\begin{pgfscope}%
\pgfpathrectangle{\pgfqpoint{0.438556in}{0.383578in}}{\pgfqpoint{4.650000in}{2.310000in}}%
\pgfusepath{clip}%
\pgfsetbuttcap%
\pgfsetroundjoin%
\pgfsetlinewidth{0.803000pt}%
\definecolor{currentstroke}{rgb}{0.686275,0.352941,0.313725}%
\pgfsetstrokecolor{currentstroke}%
\pgfsetdash{}{0pt}%
\pgfpathmoveto{\pgfqpoint{1.166686in}{2.383009in}}%
\pgfpathcurveto{\pgfqpoint{1.172211in}{2.383009in}}{\pgfqpoint{1.177511in}{2.385204in}}{\pgfqpoint{1.181418in}{2.389111in}}%
\pgfpathcurveto{\pgfqpoint{1.185324in}{2.393017in}}{\pgfqpoint{1.187519in}{2.398317in}}{\pgfqpoint{1.187519in}{2.403842in}}%
\pgfpathcurveto{\pgfqpoint{1.187519in}{2.409367in}}{\pgfqpoint{1.185324in}{2.414667in}}{\pgfqpoint{1.181418in}{2.418573in}}%
\pgfpathcurveto{\pgfqpoint{1.177511in}{2.422480in}}{\pgfqpoint{1.172211in}{2.424675in}}{\pgfqpoint{1.166686in}{2.424675in}}%
\pgfpathcurveto{\pgfqpoint{1.161161in}{2.424675in}}{\pgfqpoint{1.155862in}{2.422480in}}{\pgfqpoint{1.151955in}{2.418573in}}%
\pgfpathcurveto{\pgfqpoint{1.148048in}{2.414667in}}{\pgfqpoint{1.145853in}{2.409367in}}{\pgfqpoint{1.145853in}{2.403842in}}%
\pgfpathcurveto{\pgfqpoint{1.145853in}{2.398317in}}{\pgfqpoint{1.148048in}{2.393017in}}{\pgfqpoint{1.151955in}{2.389111in}}%
\pgfpathcurveto{\pgfqpoint{1.155862in}{2.385204in}}{\pgfqpoint{1.161161in}{2.383009in}}{\pgfqpoint{1.166686in}{2.383009in}}%
\pgfpathclose%
\pgfusepath{stroke}%
\end{pgfscope}%
\begin{pgfscope}%
\pgfpathrectangle{\pgfqpoint{0.438556in}{0.383578in}}{\pgfqpoint{4.650000in}{2.310000in}}%
\pgfusepath{clip}%
\pgfsetbuttcap%
\pgfsetroundjoin%
\pgfsetlinewidth{0.803000pt}%
\definecolor{currentstroke}{rgb}{0.686275,0.352941,0.313725}%
\pgfsetstrokecolor{currentstroke}%
\pgfsetdash{}{0pt}%
\pgfpathmoveto{\pgfqpoint{1.095575in}{2.404851in}}%
\pgfpathcurveto{\pgfqpoint{1.101101in}{2.404851in}}{\pgfqpoint{1.106400in}{2.407046in}}{\pgfqpoint{1.110307in}{2.410953in}}%
\pgfpathcurveto{\pgfqpoint{1.114214in}{2.414860in}}{\pgfqpoint{1.116409in}{2.420159in}}{\pgfqpoint{1.116409in}{2.425684in}}%
\pgfpathcurveto{\pgfqpoint{1.116409in}{2.431209in}}{\pgfqpoint{1.114214in}{2.436509in}}{\pgfqpoint{1.110307in}{2.440416in}}%
\pgfpathcurveto{\pgfqpoint{1.106400in}{2.444322in}}{\pgfqpoint{1.101101in}{2.446517in}}{\pgfqpoint{1.095575in}{2.446517in}}%
\pgfpathcurveto{\pgfqpoint{1.090050in}{2.446517in}}{\pgfqpoint{1.084751in}{2.444322in}}{\pgfqpoint{1.080844in}{2.440416in}}%
\pgfpathcurveto{\pgfqpoint{1.076937in}{2.436509in}}{\pgfqpoint{1.074742in}{2.431209in}}{\pgfqpoint{1.074742in}{2.425684in}}%
\pgfpathcurveto{\pgfqpoint{1.074742in}{2.420159in}}{\pgfqpoint{1.076937in}{2.414860in}}{\pgfqpoint{1.080844in}{2.410953in}}%
\pgfpathcurveto{\pgfqpoint{1.084751in}{2.407046in}}{\pgfqpoint{1.090050in}{2.404851in}}{\pgfqpoint{1.095575in}{2.404851in}}%
\pgfpathclose%
\pgfusepath{stroke}%
\end{pgfscope}%
\begin{pgfscope}%
\pgfpathrectangle{\pgfqpoint{0.438556in}{0.383578in}}{\pgfqpoint{4.650000in}{2.310000in}}%
\pgfusepath{clip}%
\pgfsetbuttcap%
\pgfsetroundjoin%
\pgfsetlinewidth{0.803000pt}%
\definecolor{currentstroke}{rgb}{0.686275,0.352941,0.313725}%
\pgfsetstrokecolor{currentstroke}%
\pgfsetdash{}{0pt}%
\pgfpathmoveto{\pgfqpoint{1.113369in}{2.492219in}}%
\pgfpathcurveto{\pgfqpoint{1.118894in}{2.492219in}}{\pgfqpoint{1.124194in}{2.494414in}}{\pgfqpoint{1.128101in}{2.498321in}}%
\pgfpathcurveto{\pgfqpoint{1.132007in}{2.502228in}}{\pgfqpoint{1.134203in}{2.507528in}}{\pgfqpoint{1.134203in}{2.513053in}}%
\pgfpathcurveto{\pgfqpoint{1.134203in}{2.518578in}}{\pgfqpoint{1.132007in}{2.523877in}}{\pgfqpoint{1.128101in}{2.527784in}}%
\pgfpathcurveto{\pgfqpoint{1.124194in}{2.531691in}}{\pgfqpoint{1.118894in}{2.533886in}}{\pgfqpoint{1.113369in}{2.533886in}}%
\pgfpathcurveto{\pgfqpoint{1.107844in}{2.533886in}}{\pgfqpoint{1.102545in}{2.531691in}}{\pgfqpoint{1.098638in}{2.527784in}}%
\pgfpathcurveto{\pgfqpoint{1.094731in}{2.523877in}}{\pgfqpoint{1.092536in}{2.518578in}}{\pgfqpoint{1.092536in}{2.513053in}}%
\pgfpathcurveto{\pgfqpoint{1.092536in}{2.507528in}}{\pgfqpoint{1.094731in}{2.502228in}}{\pgfqpoint{1.098638in}{2.498321in}}%
\pgfpathcurveto{\pgfqpoint{1.102545in}{2.494414in}}{\pgfqpoint{1.107844in}{2.492219in}}{\pgfqpoint{1.113369in}{2.492219in}}%
\pgfpathclose%
\pgfusepath{stroke}%
\end{pgfscope}%
\begin{pgfscope}%
\pgfpathrectangle{\pgfqpoint{0.438556in}{0.383578in}}{\pgfqpoint{4.650000in}{2.310000in}}%
\pgfusepath{clip}%
\pgfsetbuttcap%
\pgfsetroundjoin%
\pgfsetlinewidth{0.803000pt}%
\definecolor{currentstroke}{rgb}{0.686275,0.352941,0.313725}%
\pgfsetstrokecolor{currentstroke}%
\pgfsetdash{}{0pt}%
\pgfpathmoveto{\pgfqpoint{1.087210in}{2.514061in}}%
\pgfpathcurveto{\pgfqpoint{1.092735in}{2.514061in}}{\pgfqpoint{1.098035in}{2.516257in}}{\pgfqpoint{1.101941in}{2.520163in}}%
\pgfpathcurveto{\pgfqpoint{1.105848in}{2.524070in}}{\pgfqpoint{1.108043in}{2.529370in}}{\pgfqpoint{1.108043in}{2.534895in}}%
\pgfpathcurveto{\pgfqpoint{1.108043in}{2.540420in}}{\pgfqpoint{1.105848in}{2.545719in}}{\pgfqpoint{1.101941in}{2.549626in}}%
\pgfpathcurveto{\pgfqpoint{1.098035in}{2.553533in}}{\pgfqpoint{1.092735in}{2.555728in}}{\pgfqpoint{1.087210in}{2.555728in}}%
\pgfpathcurveto{\pgfqpoint{1.081685in}{2.555728in}}{\pgfqpoint{1.076385in}{2.553533in}}{\pgfqpoint{1.072479in}{2.549626in}}%
\pgfpathcurveto{\pgfqpoint{1.068572in}{2.545719in}}{\pgfqpoint{1.066377in}{2.540420in}}{\pgfqpoint{1.066377in}{2.534895in}}%
\pgfpathcurveto{\pgfqpoint{1.066377in}{2.529370in}}{\pgfqpoint{1.068572in}{2.524070in}}{\pgfqpoint{1.072479in}{2.520163in}}%
\pgfpathcurveto{\pgfqpoint{1.076385in}{2.516257in}}{\pgfqpoint{1.081685in}{2.514061in}}{\pgfqpoint{1.087210in}{2.514061in}}%
\pgfpathclose%
\pgfusepath{stroke}%
\end{pgfscope}%
\begin{pgfscope}%
\pgfpathrectangle{\pgfqpoint{0.438556in}{0.383578in}}{\pgfqpoint{4.650000in}{2.310000in}}%
\pgfusepath{clip}%
\pgfsetbuttcap%
\pgfsetroundjoin%
\pgfsetlinewidth{0.803000pt}%
\definecolor{currentstroke}{rgb}{0.000000,0.356863,0.509804}%
\pgfsetstrokecolor{currentstroke}%
\pgfsetdash{}{0pt}%
\pgfpathmoveto{\pgfqpoint{1.688934in}{0.620740in}}%
\pgfpathcurveto{\pgfqpoint{1.692617in}{0.620740in}}{\pgfqpoint{1.696150in}{0.622204in}}{\pgfqpoint{1.698755in}{0.624808in}}%
\pgfpathcurveto{\pgfqpoint{1.701359in}{0.627413in}}{\pgfqpoint{1.702823in}{0.630946in}}{\pgfqpoint{1.702823in}{0.634629in}}%
\pgfpathcurveto{\pgfqpoint{1.702823in}{0.638313in}}{\pgfqpoint{1.701359in}{0.641846in}}{\pgfqpoint{1.698755in}{0.644450in}}%
\pgfpathcurveto{\pgfqpoint{1.696150in}{0.647055in}}{\pgfqpoint{1.692617in}{0.648518in}}{\pgfqpoint{1.688934in}{0.648518in}}%
\pgfpathcurveto{\pgfqpoint{1.685250in}{0.648518in}}{\pgfqpoint{1.681717in}{0.647055in}}{\pgfqpoint{1.679113in}{0.644450in}}%
\pgfpathcurveto{\pgfqpoint{1.676508in}{0.641846in}}{\pgfqpoint{1.675045in}{0.638313in}}{\pgfqpoint{1.675045in}{0.634629in}}%
\pgfpathcurveto{\pgfqpoint{1.675045in}{0.630946in}}{\pgfqpoint{1.676508in}{0.627413in}}{\pgfqpoint{1.679113in}{0.624808in}}%
\pgfpathcurveto{\pgfqpoint{1.681717in}{0.622204in}}{\pgfqpoint{1.685250in}{0.620740in}}{\pgfqpoint{1.688934in}{0.620740in}}%
\pgfpathclose%
\pgfusepath{stroke}%
\end{pgfscope}%
\begin{pgfscope}%
\pgfpathrectangle{\pgfqpoint{0.438556in}{0.383578in}}{\pgfqpoint{4.650000in}{2.310000in}}%
\pgfusepath{clip}%
\pgfsetbuttcap%
\pgfsetroundjoin%
\pgfsetlinewidth{0.803000pt}%
\definecolor{currentstroke}{rgb}{0.000000,0.356863,0.509804}%
\pgfsetstrokecolor{currentstroke}%
\pgfsetdash{}{0pt}%
\pgfpathmoveto{\pgfqpoint{1.662949in}{0.642583in}}%
\pgfpathcurveto{\pgfqpoint{1.666632in}{0.642583in}}{\pgfqpoint{1.670165in}{0.644046in}}{\pgfqpoint{1.672770in}{0.646651in}}%
\pgfpathcurveto{\pgfqpoint{1.675374in}{0.649255in}}{\pgfqpoint{1.676838in}{0.652788in}}{\pgfqpoint{1.676838in}{0.656471in}}%
\pgfpathcurveto{\pgfqpoint{1.676838in}{0.660155in}}{\pgfqpoint{1.675374in}{0.663688in}}{\pgfqpoint{1.672770in}{0.666292in}}%
\pgfpathcurveto{\pgfqpoint{1.670165in}{0.668897in}}{\pgfqpoint{1.666632in}{0.670360in}}{\pgfqpoint{1.662949in}{0.670360in}}%
\pgfpathcurveto{\pgfqpoint{1.659265in}{0.670360in}}{\pgfqpoint{1.655732in}{0.668897in}}{\pgfqpoint{1.653128in}{0.666292in}}%
\pgfpathcurveto{\pgfqpoint{1.650523in}{0.663688in}}{\pgfqpoint{1.649060in}{0.660155in}}{\pgfqpoint{1.649060in}{0.656471in}}%
\pgfpathcurveto{\pgfqpoint{1.649060in}{0.652788in}}{\pgfqpoint{1.650523in}{0.649255in}}{\pgfqpoint{1.653128in}{0.646651in}}%
\pgfpathcurveto{\pgfqpoint{1.655732in}{0.644046in}}{\pgfqpoint{1.659265in}{0.642583in}}{\pgfqpoint{1.662949in}{0.642583in}}%
\pgfpathclose%
\pgfusepath{stroke}%
\end{pgfscope}%
\begin{pgfscope}%
\pgfpathrectangle{\pgfqpoint{0.438556in}{0.383578in}}{\pgfqpoint{4.650000in}{2.310000in}}%
\pgfusepath{clip}%
\pgfsetbuttcap%
\pgfsetroundjoin%
\pgfsetlinewidth{0.803000pt}%
\definecolor{currentstroke}{rgb}{0.000000,0.356863,0.509804}%
\pgfsetstrokecolor{currentstroke}%
\pgfsetdash{}{0pt}%
\pgfpathmoveto{\pgfqpoint{1.669280in}{0.795478in}}%
\pgfpathcurveto{\pgfqpoint{1.672964in}{0.795478in}}{\pgfqpoint{1.676497in}{0.796941in}}{\pgfqpoint{1.679101in}{0.799545in}}%
\pgfpathcurveto{\pgfqpoint{1.681706in}{0.802150in}}{\pgfqpoint{1.683169in}{0.805683in}}{\pgfqpoint{1.683169in}{0.809366in}}%
\pgfpathcurveto{\pgfqpoint{1.683169in}{0.813050in}}{\pgfqpoint{1.681706in}{0.816583in}}{\pgfqpoint{1.679101in}{0.819187in}}%
\pgfpathcurveto{\pgfqpoint{1.676497in}{0.821792in}}{\pgfqpoint{1.672964in}{0.823255in}}{\pgfqpoint{1.669280in}{0.823255in}}%
\pgfpathcurveto{\pgfqpoint{1.665597in}{0.823255in}}{\pgfqpoint{1.662064in}{0.821792in}}{\pgfqpoint{1.659459in}{0.819187in}}%
\pgfpathcurveto{\pgfqpoint{1.656855in}{0.816583in}}{\pgfqpoint{1.655391in}{0.813050in}}{\pgfqpoint{1.655391in}{0.809366in}}%
\pgfpathcurveto{\pgfqpoint{1.655391in}{0.805683in}}{\pgfqpoint{1.656855in}{0.802150in}}{\pgfqpoint{1.659459in}{0.799545in}}%
\pgfpathcurveto{\pgfqpoint{1.662064in}{0.796941in}}{\pgfqpoint{1.665597in}{0.795478in}}{\pgfqpoint{1.669280in}{0.795478in}}%
\pgfpathclose%
\pgfusepath{stroke}%
\end{pgfscope}%
\begin{pgfscope}%
\pgfpathrectangle{\pgfqpoint{0.438556in}{0.383578in}}{\pgfqpoint{4.650000in}{2.310000in}}%
\pgfusepath{clip}%
\pgfsetbuttcap%
\pgfsetroundjoin%
\pgfsetlinewidth{0.803000pt}%
\definecolor{currentstroke}{rgb}{0.000000,0.356863,0.509804}%
\pgfsetstrokecolor{currentstroke}%
\pgfsetdash{}{0pt}%
\pgfpathmoveto{\pgfqpoint{1.689960in}{0.861004in}}%
\pgfpathcurveto{\pgfqpoint{1.693643in}{0.861004in}}{\pgfqpoint{1.697176in}{0.862467in}}{\pgfqpoint{1.699781in}{0.865072in}}%
\pgfpathcurveto{\pgfqpoint{1.702385in}{0.867676in}}{\pgfqpoint{1.703849in}{0.871209in}}{\pgfqpoint{1.703849in}{0.874893in}}%
\pgfpathcurveto{\pgfqpoint{1.703849in}{0.878576in}}{\pgfqpoint{1.702385in}{0.882109in}}{\pgfqpoint{1.699781in}{0.884714in}}%
\pgfpathcurveto{\pgfqpoint{1.697176in}{0.887318in}}{\pgfqpoint{1.693643in}{0.888782in}}{\pgfqpoint{1.689960in}{0.888782in}}%
\pgfpathcurveto{\pgfqpoint{1.686277in}{0.888782in}}{\pgfqpoint{1.682744in}{0.887318in}}{\pgfqpoint{1.680139in}{0.884714in}}%
\pgfpathcurveto{\pgfqpoint{1.677535in}{0.882109in}}{\pgfqpoint{1.676071in}{0.878576in}}{\pgfqpoint{1.676071in}{0.874893in}}%
\pgfpathcurveto{\pgfqpoint{1.676071in}{0.871209in}}{\pgfqpoint{1.677535in}{0.867676in}}{\pgfqpoint{1.680139in}{0.865072in}}%
\pgfpathcurveto{\pgfqpoint{1.682744in}{0.862467in}}{\pgfqpoint{1.686277in}{0.861004in}}{\pgfqpoint{1.689960in}{0.861004in}}%
\pgfpathclose%
\pgfusepath{stroke}%
\end{pgfscope}%
\begin{pgfscope}%
\pgfpathrectangle{\pgfqpoint{0.438556in}{0.383578in}}{\pgfqpoint{4.650000in}{2.310000in}}%
\pgfusepath{clip}%
\pgfsetbuttcap%
\pgfsetroundjoin%
\pgfsetlinewidth{0.803000pt}%
\definecolor{currentstroke}{rgb}{0.000000,0.356863,0.509804}%
\pgfsetstrokecolor{currentstroke}%
\pgfsetdash{}{0pt}%
\pgfpathmoveto{\pgfqpoint{1.740437in}{0.882846in}}%
\pgfpathcurveto{\pgfqpoint{1.744120in}{0.882846in}}{\pgfqpoint{1.747653in}{0.884309in}}{\pgfqpoint{1.750258in}{0.886914in}}%
\pgfpathcurveto{\pgfqpoint{1.752862in}{0.889519in}}{\pgfqpoint{1.754325in}{0.893052in}}{\pgfqpoint{1.754325in}{0.896735in}}%
\pgfpathcurveto{\pgfqpoint{1.754325in}{0.900418in}}{\pgfqpoint{1.752862in}{0.903951in}}{\pgfqpoint{1.750258in}{0.906556in}}%
\pgfpathcurveto{\pgfqpoint{1.747653in}{0.909160in}}{\pgfqpoint{1.744120in}{0.910624in}}{\pgfqpoint{1.740437in}{0.910624in}}%
\pgfpathcurveto{\pgfqpoint{1.736753in}{0.910624in}}{\pgfqpoint{1.733220in}{0.909160in}}{\pgfqpoint{1.730616in}{0.906556in}}%
\pgfpathcurveto{\pgfqpoint{1.728011in}{0.903951in}}{\pgfqpoint{1.726548in}{0.900418in}}{\pgfqpoint{1.726548in}{0.896735in}}%
\pgfpathcurveto{\pgfqpoint{1.726548in}{0.893052in}}{\pgfqpoint{1.728011in}{0.889519in}}{\pgfqpoint{1.730616in}{0.886914in}}%
\pgfpathcurveto{\pgfqpoint{1.733220in}{0.884309in}}{\pgfqpoint{1.736753in}{0.882846in}}{\pgfqpoint{1.740437in}{0.882846in}}%
\pgfpathclose%
\pgfusepath{stroke}%
\end{pgfscope}%
\begin{pgfscope}%
\pgfpathrectangle{\pgfqpoint{0.438556in}{0.383578in}}{\pgfqpoint{4.650000in}{2.310000in}}%
\pgfusepath{clip}%
\pgfsetbuttcap%
\pgfsetroundjoin%
\pgfsetlinewidth{0.803000pt}%
\definecolor{currentstroke}{rgb}{0.000000,0.356863,0.509804}%
\pgfsetstrokecolor{currentstroke}%
\pgfsetdash{}{0pt}%
\pgfpathmoveto{\pgfqpoint{1.719482in}{1.013899in}}%
\pgfpathcurveto{\pgfqpoint{1.723165in}{1.013899in}}{\pgfqpoint{1.726698in}{1.015362in}}{\pgfqpoint{1.729303in}{1.017967in}}%
\pgfpathcurveto{\pgfqpoint{1.731907in}{1.020571in}}{\pgfqpoint{1.733371in}{1.024104in}}{\pgfqpoint{1.733371in}{1.027788in}}%
\pgfpathcurveto{\pgfqpoint{1.733371in}{1.031471in}}{\pgfqpoint{1.731907in}{1.035004in}}{\pgfqpoint{1.729303in}{1.037609in}}%
\pgfpathcurveto{\pgfqpoint{1.726698in}{1.040213in}}{\pgfqpoint{1.723165in}{1.041677in}}{\pgfqpoint{1.719482in}{1.041677in}}%
\pgfpathcurveto{\pgfqpoint{1.715798in}{1.041677in}}{\pgfqpoint{1.712265in}{1.040213in}}{\pgfqpoint{1.709661in}{1.037609in}}%
\pgfpathcurveto{\pgfqpoint{1.707056in}{1.035004in}}{\pgfqpoint{1.705593in}{1.031471in}}{\pgfqpoint{1.705593in}{1.027788in}}%
\pgfpathcurveto{\pgfqpoint{1.705593in}{1.024104in}}{\pgfqpoint{1.707056in}{1.020571in}}{\pgfqpoint{1.709661in}{1.017967in}}%
\pgfpathcurveto{\pgfqpoint{1.712265in}{1.015362in}}{\pgfqpoint{1.715798in}{1.013899in}}{\pgfqpoint{1.719482in}{1.013899in}}%
\pgfpathclose%
\pgfusepath{stroke}%
\end{pgfscope}%
\begin{pgfscope}%
\pgfpathrectangle{\pgfqpoint{0.438556in}{0.383578in}}{\pgfqpoint{4.650000in}{2.310000in}}%
\pgfusepath{clip}%
\pgfsetbuttcap%
\pgfsetroundjoin%
\pgfsetlinewidth{0.803000pt}%
\definecolor{currentstroke}{rgb}{0.000000,0.356863,0.509804}%
\pgfsetstrokecolor{currentstroke}%
\pgfsetdash{}{0pt}%
\pgfpathmoveto{\pgfqpoint{1.684554in}{1.035741in}}%
\pgfpathcurveto{\pgfqpoint{1.688237in}{1.035741in}}{\pgfqpoint{1.691771in}{1.037204in}}{\pgfqpoint{1.694375in}{1.039809in}}%
\pgfpathcurveto{\pgfqpoint{1.696980in}{1.042413in}}{\pgfqpoint{1.698443in}{1.045946in}}{\pgfqpoint{1.698443in}{1.049630in}}%
\pgfpathcurveto{\pgfqpoint{1.698443in}{1.053313in}}{\pgfqpoint{1.696980in}{1.056846in}}{\pgfqpoint{1.694375in}{1.059451in}}%
\pgfpathcurveto{\pgfqpoint{1.691771in}{1.062055in}}{\pgfqpoint{1.688237in}{1.063519in}}{\pgfqpoint{1.684554in}{1.063519in}}%
\pgfpathcurveto{\pgfqpoint{1.680871in}{1.063519in}}{\pgfqpoint{1.677338in}{1.062055in}}{\pgfqpoint{1.674733in}{1.059451in}}%
\pgfpathcurveto{\pgfqpoint{1.672129in}{1.056846in}}{\pgfqpoint{1.670665in}{1.053313in}}{\pgfqpoint{1.670665in}{1.049630in}}%
\pgfpathcurveto{\pgfqpoint{1.670665in}{1.045946in}}{\pgfqpoint{1.672129in}{1.042413in}}{\pgfqpoint{1.674733in}{1.039809in}}%
\pgfpathcurveto{\pgfqpoint{1.677338in}{1.037204in}}{\pgfqpoint{1.680871in}{1.035741in}}{\pgfqpoint{1.684554in}{1.035741in}}%
\pgfpathclose%
\pgfusepath{stroke}%
\end{pgfscope}%
\begin{pgfscope}%
\pgfpathrectangle{\pgfqpoint{0.438556in}{0.383578in}}{\pgfqpoint{4.650000in}{2.310000in}}%
\pgfusepath{clip}%
\pgfsetbuttcap%
\pgfsetroundjoin%
\pgfsetlinewidth{0.803000pt}%
\definecolor{currentstroke}{rgb}{0.000000,0.356863,0.509804}%
\pgfsetstrokecolor{currentstroke}%
\pgfsetdash{}{0pt}%
\pgfpathmoveto{\pgfqpoint{1.695393in}{1.101267in}}%
\pgfpathcurveto{\pgfqpoint{1.699077in}{1.101267in}}{\pgfqpoint{1.702610in}{1.102731in}}{\pgfqpoint{1.705214in}{1.105335in}}%
\pgfpathcurveto{\pgfqpoint{1.707819in}{1.107940in}}{\pgfqpoint{1.709282in}{1.111473in}}{\pgfqpoint{1.709282in}{1.115156in}}%
\pgfpathcurveto{\pgfqpoint{1.709282in}{1.118840in}}{\pgfqpoint{1.707819in}{1.122373in}}{\pgfqpoint{1.705214in}{1.124977in}}%
\pgfpathcurveto{\pgfqpoint{1.702610in}{1.127582in}}{\pgfqpoint{1.699077in}{1.129045in}}{\pgfqpoint{1.695393in}{1.129045in}}%
\pgfpathcurveto{\pgfqpoint{1.691710in}{1.129045in}}{\pgfqpoint{1.688177in}{1.127582in}}{\pgfqpoint{1.685573in}{1.124977in}}%
\pgfpathcurveto{\pgfqpoint{1.682968in}{1.122373in}}{\pgfqpoint{1.681505in}{1.118840in}}{\pgfqpoint{1.681505in}{1.115156in}}%
\pgfpathcurveto{\pgfqpoint{1.681505in}{1.111473in}}{\pgfqpoint{1.682968in}{1.107940in}}{\pgfqpoint{1.685573in}{1.105335in}}%
\pgfpathcurveto{\pgfqpoint{1.688177in}{1.102731in}}{\pgfqpoint{1.691710in}{1.101267in}}{\pgfqpoint{1.695393in}{1.101267in}}%
\pgfpathclose%
\pgfusepath{stroke}%
\end{pgfscope}%
\begin{pgfscope}%
\pgfpathrectangle{\pgfqpoint{0.438556in}{0.383578in}}{\pgfqpoint{4.650000in}{2.310000in}}%
\pgfusepath{clip}%
\pgfsetbuttcap%
\pgfsetroundjoin%
\pgfsetlinewidth{0.803000pt}%
\definecolor{currentstroke}{rgb}{0.000000,0.356863,0.509804}%
\pgfsetstrokecolor{currentstroke}%
\pgfsetdash{}{0pt}%
\pgfpathmoveto{\pgfqpoint{1.681659in}{1.210478in}}%
\pgfpathcurveto{\pgfqpoint{1.685342in}{1.210478in}}{\pgfqpoint{1.688875in}{1.211941in}}{\pgfqpoint{1.691480in}{1.214546in}}%
\pgfpathcurveto{\pgfqpoint{1.694084in}{1.217151in}}{\pgfqpoint{1.695548in}{1.220684in}}{\pgfqpoint{1.695548in}{1.224367in}}%
\pgfpathcurveto{\pgfqpoint{1.695548in}{1.228050in}}{\pgfqpoint{1.694084in}{1.231583in}}{\pgfqpoint{1.691480in}{1.234188in}}%
\pgfpathcurveto{\pgfqpoint{1.688875in}{1.236792in}}{\pgfqpoint{1.685342in}{1.238256in}}{\pgfqpoint{1.681659in}{1.238256in}}%
\pgfpathcurveto{\pgfqpoint{1.677975in}{1.238256in}}{\pgfqpoint{1.674442in}{1.236792in}}{\pgfqpoint{1.671838in}{1.234188in}}%
\pgfpathcurveto{\pgfqpoint{1.669233in}{1.231583in}}{\pgfqpoint{1.667770in}{1.228050in}}{\pgfqpoint{1.667770in}{1.224367in}}%
\pgfpathcurveto{\pgfqpoint{1.667770in}{1.220684in}}{\pgfqpoint{1.669233in}{1.217151in}}{\pgfqpoint{1.671838in}{1.214546in}}%
\pgfpathcurveto{\pgfqpoint{1.674442in}{1.211941in}}{\pgfqpoint{1.677975in}{1.210478in}}{\pgfqpoint{1.681659in}{1.210478in}}%
\pgfpathclose%
\pgfusepath{stroke}%
\end{pgfscope}%
\begin{pgfscope}%
\pgfpathrectangle{\pgfqpoint{0.438556in}{0.383578in}}{\pgfqpoint{4.650000in}{2.310000in}}%
\pgfusepath{clip}%
\pgfsetbuttcap%
\pgfsetroundjoin%
\pgfsetlinewidth{0.803000pt}%
\definecolor{currentstroke}{rgb}{0.000000,0.356863,0.509804}%
\pgfsetstrokecolor{currentstroke}%
\pgfsetdash{}{0pt}%
\pgfpathmoveto{\pgfqpoint{1.742516in}{1.232320in}}%
\pgfpathcurveto{\pgfqpoint{1.746200in}{1.232320in}}{\pgfqpoint{1.749733in}{1.233784in}}{\pgfqpoint{1.752337in}{1.236388in}}%
\pgfpathcurveto{\pgfqpoint{1.754942in}{1.238993in}}{\pgfqpoint{1.756405in}{1.242526in}}{\pgfqpoint{1.756405in}{1.246209in}}%
\pgfpathcurveto{\pgfqpoint{1.756405in}{1.249892in}}{\pgfqpoint{1.754942in}{1.253425in}}{\pgfqpoint{1.752337in}{1.256030in}}%
\pgfpathcurveto{\pgfqpoint{1.749733in}{1.258634in}}{\pgfqpoint{1.746200in}{1.260098in}}{\pgfqpoint{1.742516in}{1.260098in}}%
\pgfpathcurveto{\pgfqpoint{1.738833in}{1.260098in}}{\pgfqpoint{1.735300in}{1.258634in}}{\pgfqpoint{1.732696in}{1.256030in}}%
\pgfpathcurveto{\pgfqpoint{1.730091in}{1.253425in}}{\pgfqpoint{1.728628in}{1.249892in}}{\pgfqpoint{1.728628in}{1.246209in}}%
\pgfpathcurveto{\pgfqpoint{1.728628in}{1.242526in}}{\pgfqpoint{1.730091in}{1.238993in}}{\pgfqpoint{1.732696in}{1.236388in}}%
\pgfpathcurveto{\pgfqpoint{1.735300in}{1.233784in}}{\pgfqpoint{1.738833in}{1.232320in}}{\pgfqpoint{1.742516in}{1.232320in}}%
\pgfpathclose%
\pgfusepath{stroke}%
\end{pgfscope}%
\begin{pgfscope}%
\pgfpathrectangle{\pgfqpoint{0.438556in}{0.383578in}}{\pgfqpoint{4.650000in}{2.310000in}}%
\pgfusepath{clip}%
\pgfsetbuttcap%
\pgfsetroundjoin%
\pgfsetlinewidth{0.803000pt}%
\definecolor{currentstroke}{rgb}{0.000000,0.356863,0.509804}%
\pgfsetstrokecolor{currentstroke}%
\pgfsetdash{}{0pt}%
\pgfpathmoveto{\pgfqpoint{1.678690in}{1.254162in}}%
\pgfpathcurveto{\pgfqpoint{1.682373in}{1.254162in}}{\pgfqpoint{1.685906in}{1.255626in}}{\pgfqpoint{1.688511in}{1.258230in}}%
\pgfpathcurveto{\pgfqpoint{1.691116in}{1.260835in}}{\pgfqpoint{1.692579in}{1.264368in}}{\pgfqpoint{1.692579in}{1.268051in}}%
\pgfpathcurveto{\pgfqpoint{1.692579in}{1.271735in}}{\pgfqpoint{1.691116in}{1.275268in}}{\pgfqpoint{1.688511in}{1.277872in}}%
\pgfpathcurveto{\pgfqpoint{1.685906in}{1.280477in}}{\pgfqpoint{1.682373in}{1.281940in}}{\pgfqpoint{1.678690in}{1.281940in}}%
\pgfpathcurveto{\pgfqpoint{1.675007in}{1.281940in}}{\pgfqpoint{1.671474in}{1.280477in}}{\pgfqpoint{1.668869in}{1.277872in}}%
\pgfpathcurveto{\pgfqpoint{1.666265in}{1.275268in}}{\pgfqpoint{1.664801in}{1.271735in}}{\pgfqpoint{1.664801in}{1.268051in}}%
\pgfpathcurveto{\pgfqpoint{1.664801in}{1.264368in}}{\pgfqpoint{1.666265in}{1.260835in}}{\pgfqpoint{1.668869in}{1.258230in}}%
\pgfpathcurveto{\pgfqpoint{1.671474in}{1.255626in}}{\pgfqpoint{1.675007in}{1.254162in}}{\pgfqpoint{1.678690in}{1.254162in}}%
\pgfpathclose%
\pgfusepath{stroke}%
\end{pgfscope}%
\begin{pgfscope}%
\pgfpathrectangle{\pgfqpoint{0.438556in}{0.383578in}}{\pgfqpoint{4.650000in}{2.310000in}}%
\pgfusepath{clip}%
\pgfsetbuttcap%
\pgfsetroundjoin%
\pgfsetlinewidth{0.803000pt}%
\definecolor{currentstroke}{rgb}{0.000000,0.356863,0.509804}%
\pgfsetstrokecolor{currentstroke}%
\pgfsetdash{}{0pt}%
\pgfpathmoveto{\pgfqpoint{1.726803in}{1.276004in}}%
\pgfpathcurveto{\pgfqpoint{1.730486in}{1.276004in}}{\pgfqpoint{1.734019in}{1.277468in}}{\pgfqpoint{1.736624in}{1.280072in}}%
\pgfpathcurveto{\pgfqpoint{1.739228in}{1.282677in}}{\pgfqpoint{1.740692in}{1.286210in}}{\pgfqpoint{1.740692in}{1.289893in}}%
\pgfpathcurveto{\pgfqpoint{1.740692in}{1.293577in}}{\pgfqpoint{1.739228in}{1.297110in}}{\pgfqpoint{1.736624in}{1.299714in}}%
\pgfpathcurveto{\pgfqpoint{1.734019in}{1.302319in}}{\pgfqpoint{1.730486in}{1.303782in}}{\pgfqpoint{1.726803in}{1.303782in}}%
\pgfpathcurveto{\pgfqpoint{1.723119in}{1.303782in}}{\pgfqpoint{1.719586in}{1.302319in}}{\pgfqpoint{1.716982in}{1.299714in}}%
\pgfpathcurveto{\pgfqpoint{1.714377in}{1.297110in}}{\pgfqpoint{1.712914in}{1.293577in}}{\pgfqpoint{1.712914in}{1.289893in}}%
\pgfpathcurveto{\pgfqpoint{1.712914in}{1.286210in}}{\pgfqpoint{1.714377in}{1.282677in}}{\pgfqpoint{1.716982in}{1.280072in}}%
\pgfpathcurveto{\pgfqpoint{1.719586in}{1.277468in}}{\pgfqpoint{1.723119in}{1.276004in}}{\pgfqpoint{1.726803in}{1.276004in}}%
\pgfpathclose%
\pgfusepath{stroke}%
\end{pgfscope}%
\begin{pgfscope}%
\pgfpathrectangle{\pgfqpoint{0.438556in}{0.383578in}}{\pgfqpoint{4.650000in}{2.310000in}}%
\pgfusepath{clip}%
\pgfsetbuttcap%
\pgfsetroundjoin%
\pgfsetlinewidth{0.803000pt}%
\definecolor{currentstroke}{rgb}{0.000000,0.356863,0.509804}%
\pgfsetstrokecolor{currentstroke}%
\pgfsetdash{}{0pt}%
\pgfpathmoveto{\pgfqpoint{1.675502in}{1.297847in}}%
\pgfpathcurveto{\pgfqpoint{1.679185in}{1.297847in}}{\pgfqpoint{1.682718in}{1.299310in}}{\pgfqpoint{1.685322in}{1.301914in}}%
\pgfpathcurveto{\pgfqpoint{1.687927in}{1.304519in}}{\pgfqpoint{1.689390in}{1.308052in}}{\pgfqpoint{1.689390in}{1.311735in}}%
\pgfpathcurveto{\pgfqpoint{1.689390in}{1.315419in}}{\pgfqpoint{1.687927in}{1.318952in}}{\pgfqpoint{1.685322in}{1.321556in}}%
\pgfpathcurveto{\pgfqpoint{1.682718in}{1.324161in}}{\pgfqpoint{1.679185in}{1.325624in}}{\pgfqpoint{1.675502in}{1.325624in}}%
\pgfpathcurveto{\pgfqpoint{1.671818in}{1.325624in}}{\pgfqpoint{1.668285in}{1.324161in}}{\pgfqpoint{1.665681in}{1.321556in}}%
\pgfpathcurveto{\pgfqpoint{1.663076in}{1.318952in}}{\pgfqpoint{1.661613in}{1.315419in}}{\pgfqpoint{1.661613in}{1.311735in}}%
\pgfpathcurveto{\pgfqpoint{1.661613in}{1.308052in}}{\pgfqpoint{1.663076in}{1.304519in}}{\pgfqpoint{1.665681in}{1.301914in}}%
\pgfpathcurveto{\pgfqpoint{1.668285in}{1.299310in}}{\pgfqpoint{1.671818in}{1.297847in}}{\pgfqpoint{1.675502in}{1.297847in}}%
\pgfpathclose%
\pgfusepath{stroke}%
\end{pgfscope}%
\begin{pgfscope}%
\pgfpathrectangle{\pgfqpoint{0.438556in}{0.383578in}}{\pgfqpoint{4.650000in}{2.310000in}}%
\pgfusepath{clip}%
\pgfsetbuttcap%
\pgfsetroundjoin%
\pgfsetlinewidth{0.803000pt}%
\definecolor{currentstroke}{rgb}{0.000000,0.356863,0.509804}%
\pgfsetstrokecolor{currentstroke}%
\pgfsetdash{}{0pt}%
\pgfpathmoveto{\pgfqpoint{1.729616in}{1.341531in}}%
\pgfpathcurveto{\pgfqpoint{1.733299in}{1.341531in}}{\pgfqpoint{1.736832in}{1.342994in}}{\pgfqpoint{1.739437in}{1.345599in}}%
\pgfpathcurveto{\pgfqpoint{1.742041in}{1.348203in}}{\pgfqpoint{1.743504in}{1.351736in}}{\pgfqpoint{1.743504in}{1.355420in}}%
\pgfpathcurveto{\pgfqpoint{1.743504in}{1.359103in}}{\pgfqpoint{1.742041in}{1.362636in}}{\pgfqpoint{1.739437in}{1.365241in}}%
\pgfpathcurveto{\pgfqpoint{1.736832in}{1.367845in}}{\pgfqpoint{1.733299in}{1.369309in}}{\pgfqpoint{1.729616in}{1.369309in}}%
\pgfpathcurveto{\pgfqpoint{1.725932in}{1.369309in}}{\pgfqpoint{1.722399in}{1.367845in}}{\pgfqpoint{1.719795in}{1.365241in}}%
\pgfpathcurveto{\pgfqpoint{1.717190in}{1.362636in}}{\pgfqpoint{1.715727in}{1.359103in}}{\pgfqpoint{1.715727in}{1.355420in}}%
\pgfpathcurveto{\pgfqpoint{1.715727in}{1.351736in}}{\pgfqpoint{1.717190in}{1.348203in}}{\pgfqpoint{1.719795in}{1.345599in}}%
\pgfpathcurveto{\pgfqpoint{1.722399in}{1.342994in}}{\pgfqpoint{1.725932in}{1.341531in}}{\pgfqpoint{1.729616in}{1.341531in}}%
\pgfpathclose%
\pgfusepath{stroke}%
\end{pgfscope}%
\begin{pgfscope}%
\pgfpathrectangle{\pgfqpoint{0.438556in}{0.383578in}}{\pgfqpoint{4.650000in}{2.310000in}}%
\pgfusepath{clip}%
\pgfsetbuttcap%
\pgfsetroundjoin%
\pgfsetlinewidth{0.803000pt}%
\definecolor{currentstroke}{rgb}{0.000000,0.356863,0.509804}%
\pgfsetstrokecolor{currentstroke}%
\pgfsetdash{}{0pt}%
\pgfpathmoveto{\pgfqpoint{1.677911in}{1.450741in}}%
\pgfpathcurveto{\pgfqpoint{1.681595in}{1.450741in}}{\pgfqpoint{1.685128in}{1.452205in}}{\pgfqpoint{1.687732in}{1.454809in}}%
\pgfpathcurveto{\pgfqpoint{1.690337in}{1.457414in}}{\pgfqpoint{1.691800in}{1.460947in}}{\pgfqpoint{1.691800in}{1.464630in}}%
\pgfpathcurveto{\pgfqpoint{1.691800in}{1.468314in}}{\pgfqpoint{1.690337in}{1.471847in}}{\pgfqpoint{1.687732in}{1.474451in}}%
\pgfpathcurveto{\pgfqpoint{1.685128in}{1.477056in}}{\pgfqpoint{1.681595in}{1.478519in}}{\pgfqpoint{1.677911in}{1.478519in}}%
\pgfpathcurveto{\pgfqpoint{1.674228in}{1.478519in}}{\pgfqpoint{1.670695in}{1.477056in}}{\pgfqpoint{1.668090in}{1.474451in}}%
\pgfpathcurveto{\pgfqpoint{1.665486in}{1.471847in}}{\pgfqpoint{1.664022in}{1.468314in}}{\pgfqpoint{1.664022in}{1.464630in}}%
\pgfpathcurveto{\pgfqpoint{1.664022in}{1.460947in}}{\pgfqpoint{1.665486in}{1.457414in}}{\pgfqpoint{1.668090in}{1.454809in}}%
\pgfpathcurveto{\pgfqpoint{1.670695in}{1.452205in}}{\pgfqpoint{1.674228in}{1.450741in}}{\pgfqpoint{1.677911in}{1.450741in}}%
\pgfpathclose%
\pgfusepath{stroke}%
\end{pgfscope}%
\begin{pgfscope}%
\pgfpathrectangle{\pgfqpoint{0.438556in}{0.383578in}}{\pgfqpoint{4.650000in}{2.310000in}}%
\pgfusepath{clip}%
\pgfsetbuttcap%
\pgfsetroundjoin%
\pgfsetlinewidth{0.803000pt}%
\definecolor{currentstroke}{rgb}{0.000000,0.356863,0.509804}%
\pgfsetstrokecolor{currentstroke}%
\pgfsetdash{}{0pt}%
\pgfpathmoveto{\pgfqpoint{1.687724in}{1.472584in}}%
\pgfpathcurveto{\pgfqpoint{1.691408in}{1.472584in}}{\pgfqpoint{1.694941in}{1.474047in}}{\pgfqpoint{1.697545in}{1.476652in}}%
\pgfpathcurveto{\pgfqpoint{1.700150in}{1.479256in}}{\pgfqpoint{1.701613in}{1.482789in}}{\pgfqpoint{1.701613in}{1.486472in}}%
\pgfpathcurveto{\pgfqpoint{1.701613in}{1.490156in}}{\pgfqpoint{1.700150in}{1.493689in}}{\pgfqpoint{1.697545in}{1.496293in}}%
\pgfpathcurveto{\pgfqpoint{1.694941in}{1.498898in}}{\pgfqpoint{1.691408in}{1.500361in}}{\pgfqpoint{1.687724in}{1.500361in}}%
\pgfpathcurveto{\pgfqpoint{1.684041in}{1.500361in}}{\pgfqpoint{1.680508in}{1.498898in}}{\pgfqpoint{1.677903in}{1.496293in}}%
\pgfpathcurveto{\pgfqpoint{1.675299in}{1.493689in}}{\pgfqpoint{1.673835in}{1.490156in}}{\pgfqpoint{1.673835in}{1.486472in}}%
\pgfpathcurveto{\pgfqpoint{1.673835in}{1.482789in}}{\pgfqpoint{1.675299in}{1.479256in}}{\pgfqpoint{1.677903in}{1.476652in}}%
\pgfpathcurveto{\pgfqpoint{1.680508in}{1.474047in}}{\pgfqpoint{1.684041in}{1.472584in}}{\pgfqpoint{1.687724in}{1.472584in}}%
\pgfpathclose%
\pgfusepath{stroke}%
\end{pgfscope}%
\begin{pgfscope}%
\pgfpathrectangle{\pgfqpoint{0.438556in}{0.383578in}}{\pgfqpoint{4.650000in}{2.310000in}}%
\pgfusepath{clip}%
\pgfsetbuttcap%
\pgfsetroundjoin%
\pgfsetlinewidth{0.803000pt}%
\definecolor{currentstroke}{rgb}{0.000000,0.356863,0.509804}%
\pgfsetstrokecolor{currentstroke}%
\pgfsetdash{}{0pt}%
\pgfpathmoveto{\pgfqpoint{1.689832in}{1.647321in}}%
\pgfpathcurveto{\pgfqpoint{1.693515in}{1.647321in}}{\pgfqpoint{1.697048in}{1.648784in}}{\pgfqpoint{1.699653in}{1.651389in}}%
\pgfpathcurveto{\pgfqpoint{1.702257in}{1.653993in}}{\pgfqpoint{1.703721in}{1.657526in}}{\pgfqpoint{1.703721in}{1.661210in}}%
\pgfpathcurveto{\pgfqpoint{1.703721in}{1.664893in}}{\pgfqpoint{1.702257in}{1.668426in}}{\pgfqpoint{1.699653in}{1.671030in}}%
\pgfpathcurveto{\pgfqpoint{1.697048in}{1.673635in}}{\pgfqpoint{1.693515in}{1.675098in}}{\pgfqpoint{1.689832in}{1.675098in}}%
\pgfpathcurveto{\pgfqpoint{1.686148in}{1.675098in}}{\pgfqpoint{1.682615in}{1.673635in}}{\pgfqpoint{1.680011in}{1.671030in}}%
\pgfpathcurveto{\pgfqpoint{1.677406in}{1.668426in}}{\pgfqpoint{1.675943in}{1.664893in}}{\pgfqpoint{1.675943in}{1.661210in}}%
\pgfpathcurveto{\pgfqpoint{1.675943in}{1.657526in}}{\pgfqpoint{1.677406in}{1.653993in}}{\pgfqpoint{1.680011in}{1.651389in}}%
\pgfpathcurveto{\pgfqpoint{1.682615in}{1.648784in}}{\pgfqpoint{1.686148in}{1.647321in}}{\pgfqpoint{1.689832in}{1.647321in}}%
\pgfpathclose%
\pgfusepath{stroke}%
\end{pgfscope}%
\begin{pgfscope}%
\pgfpathrectangle{\pgfqpoint{0.438556in}{0.383578in}}{\pgfqpoint{4.650000in}{2.310000in}}%
\pgfusepath{clip}%
\pgfsetbuttcap%
\pgfsetroundjoin%
\pgfsetlinewidth{0.803000pt}%
\definecolor{currentstroke}{rgb}{0.000000,0.356863,0.509804}%
\pgfsetstrokecolor{currentstroke}%
\pgfsetdash{}{0pt}%
\pgfpathmoveto{\pgfqpoint{1.730660in}{1.669163in}}%
\pgfpathcurveto{\pgfqpoint{1.734344in}{1.669163in}}{\pgfqpoint{1.737877in}{1.670626in}}{\pgfqpoint{1.740481in}{1.673231in}}%
\pgfpathcurveto{\pgfqpoint{1.743086in}{1.675835in}}{\pgfqpoint{1.744549in}{1.679368in}}{\pgfqpoint{1.744549in}{1.683052in}}%
\pgfpathcurveto{\pgfqpoint{1.744549in}{1.686735in}}{\pgfqpoint{1.743086in}{1.690268in}}{\pgfqpoint{1.740481in}{1.692873in}}%
\pgfpathcurveto{\pgfqpoint{1.737877in}{1.695477in}}{\pgfqpoint{1.734344in}{1.696941in}}{\pgfqpoint{1.730660in}{1.696941in}}%
\pgfpathcurveto{\pgfqpoint{1.726977in}{1.696941in}}{\pgfqpoint{1.723444in}{1.695477in}}{\pgfqpoint{1.720839in}{1.692873in}}%
\pgfpathcurveto{\pgfqpoint{1.718235in}{1.690268in}}{\pgfqpoint{1.716771in}{1.686735in}}{\pgfqpoint{1.716771in}{1.683052in}}%
\pgfpathcurveto{\pgfqpoint{1.716771in}{1.679368in}}{\pgfqpoint{1.718235in}{1.675835in}}{\pgfqpoint{1.720839in}{1.673231in}}%
\pgfpathcurveto{\pgfqpoint{1.723444in}{1.670626in}}{\pgfqpoint{1.726977in}{1.669163in}}{\pgfqpoint{1.730660in}{1.669163in}}%
\pgfpathclose%
\pgfusepath{stroke}%
\end{pgfscope}%
\begin{pgfscope}%
\pgfpathrectangle{\pgfqpoint{0.438556in}{0.383578in}}{\pgfqpoint{4.650000in}{2.310000in}}%
\pgfusepath{clip}%
\pgfsetbuttcap%
\pgfsetroundjoin%
\pgfsetlinewidth{0.803000pt}%
\definecolor{currentstroke}{rgb}{0.000000,0.356863,0.509804}%
\pgfsetstrokecolor{currentstroke}%
\pgfsetdash{}{0pt}%
\pgfpathmoveto{\pgfqpoint{1.722964in}{1.691005in}}%
\pgfpathcurveto{\pgfqpoint{1.726647in}{1.691005in}}{\pgfqpoint{1.730180in}{1.692468in}}{\pgfqpoint{1.732785in}{1.695073in}}%
\pgfpathcurveto{\pgfqpoint{1.735389in}{1.697677in}}{\pgfqpoint{1.736852in}{1.701210in}}{\pgfqpoint{1.736852in}{1.704894in}}%
\pgfpathcurveto{\pgfqpoint{1.736852in}{1.708577in}}{\pgfqpoint{1.735389in}{1.712110in}}{\pgfqpoint{1.732785in}{1.714715in}}%
\pgfpathcurveto{\pgfqpoint{1.730180in}{1.717319in}}{\pgfqpoint{1.726647in}{1.718783in}}{\pgfqpoint{1.722964in}{1.718783in}}%
\pgfpathcurveto{\pgfqpoint{1.719280in}{1.718783in}}{\pgfqpoint{1.715747in}{1.717319in}}{\pgfqpoint{1.713143in}{1.714715in}}%
\pgfpathcurveto{\pgfqpoint{1.710538in}{1.712110in}}{\pgfqpoint{1.709075in}{1.708577in}}{\pgfqpoint{1.709075in}{1.704894in}}%
\pgfpathcurveto{\pgfqpoint{1.709075in}{1.701210in}}{\pgfqpoint{1.710538in}{1.697677in}}{\pgfqpoint{1.713143in}{1.695073in}}%
\pgfpathcurveto{\pgfqpoint{1.715747in}{1.692468in}}{\pgfqpoint{1.719280in}{1.691005in}}{\pgfqpoint{1.722964in}{1.691005in}}%
\pgfpathclose%
\pgfusepath{stroke}%
\end{pgfscope}%
\begin{pgfscope}%
\pgfpathrectangle{\pgfqpoint{0.438556in}{0.383578in}}{\pgfqpoint{4.650000in}{2.310000in}}%
\pgfusepath{clip}%
\pgfsetbuttcap%
\pgfsetroundjoin%
\pgfsetlinewidth{0.803000pt}%
\definecolor{currentstroke}{rgb}{0.000000,0.356863,0.509804}%
\pgfsetstrokecolor{currentstroke}%
\pgfsetdash{}{0pt}%
\pgfpathmoveto{\pgfqpoint{1.670297in}{1.712847in}}%
\pgfpathcurveto{\pgfqpoint{1.673981in}{1.712847in}}{\pgfqpoint{1.677514in}{1.714310in}}{\pgfqpoint{1.680118in}{1.716915in}}%
\pgfpathcurveto{\pgfqpoint{1.682723in}{1.719520in}}{\pgfqpoint{1.684186in}{1.723053in}}{\pgfqpoint{1.684186in}{1.726736in}}%
\pgfpathcurveto{\pgfqpoint{1.684186in}{1.730419in}}{\pgfqpoint{1.682723in}{1.733952in}}{\pgfqpoint{1.680118in}{1.736557in}}%
\pgfpathcurveto{\pgfqpoint{1.677514in}{1.739161in}}{\pgfqpoint{1.673981in}{1.740625in}}{\pgfqpoint{1.670297in}{1.740625in}}%
\pgfpathcurveto{\pgfqpoint{1.666614in}{1.740625in}}{\pgfqpoint{1.663081in}{1.739161in}}{\pgfqpoint{1.660476in}{1.736557in}}%
\pgfpathcurveto{\pgfqpoint{1.657872in}{1.733952in}}{\pgfqpoint{1.656408in}{1.730419in}}{\pgfqpoint{1.656408in}{1.726736in}}%
\pgfpathcurveto{\pgfqpoint{1.656408in}{1.723053in}}{\pgfqpoint{1.657872in}{1.719520in}}{\pgfqpoint{1.660476in}{1.716915in}}%
\pgfpathcurveto{\pgfqpoint{1.663081in}{1.714310in}}{\pgfqpoint{1.666614in}{1.712847in}}{\pgfqpoint{1.670297in}{1.712847in}}%
\pgfpathclose%
\pgfusepath{stroke}%
\end{pgfscope}%
\begin{pgfscope}%
\pgfpathrectangle{\pgfqpoint{0.438556in}{0.383578in}}{\pgfqpoint{4.650000in}{2.310000in}}%
\pgfusepath{clip}%
\pgfsetbuttcap%
\pgfsetroundjoin%
\pgfsetlinewidth{0.803000pt}%
\definecolor{currentstroke}{rgb}{0.000000,0.356863,0.509804}%
\pgfsetstrokecolor{currentstroke}%
\pgfsetdash{}{0pt}%
\pgfpathmoveto{\pgfqpoint{1.677545in}{1.778373in}}%
\pgfpathcurveto{\pgfqpoint{1.681228in}{1.778373in}}{\pgfqpoint{1.684761in}{1.779837in}}{\pgfqpoint{1.687366in}{1.782441in}}%
\pgfpathcurveto{\pgfqpoint{1.689970in}{1.785046in}}{\pgfqpoint{1.691434in}{1.788579in}}{\pgfqpoint{1.691434in}{1.792262in}}%
\pgfpathcurveto{\pgfqpoint{1.691434in}{1.795946in}}{\pgfqpoint{1.689970in}{1.799479in}}{\pgfqpoint{1.687366in}{1.802083in}}%
\pgfpathcurveto{\pgfqpoint{1.684761in}{1.804688in}}{\pgfqpoint{1.681228in}{1.806151in}}{\pgfqpoint{1.677545in}{1.806151in}}%
\pgfpathcurveto{\pgfqpoint{1.673861in}{1.806151in}}{\pgfqpoint{1.670328in}{1.804688in}}{\pgfqpoint{1.667724in}{1.802083in}}%
\pgfpathcurveto{\pgfqpoint{1.665119in}{1.799479in}}{\pgfqpoint{1.663656in}{1.795946in}}{\pgfqpoint{1.663656in}{1.792262in}}%
\pgfpathcurveto{\pgfqpoint{1.663656in}{1.788579in}}{\pgfqpoint{1.665119in}{1.785046in}}{\pgfqpoint{1.667724in}{1.782441in}}%
\pgfpathcurveto{\pgfqpoint{1.670328in}{1.779837in}}{\pgfqpoint{1.673861in}{1.778373in}}{\pgfqpoint{1.677545in}{1.778373in}}%
\pgfpathclose%
\pgfusepath{stroke}%
\end{pgfscope}%
\begin{pgfscope}%
\pgfpathrectangle{\pgfqpoint{0.438556in}{0.383578in}}{\pgfqpoint{4.650000in}{2.310000in}}%
\pgfusepath{clip}%
\pgfsetbuttcap%
\pgfsetroundjoin%
\pgfsetlinewidth{0.803000pt}%
\definecolor{currentstroke}{rgb}{0.000000,0.356863,0.509804}%
\pgfsetstrokecolor{currentstroke}%
\pgfsetdash{}{0pt}%
\pgfpathmoveto{\pgfqpoint{1.662894in}{1.800216in}}%
\pgfpathcurveto{\pgfqpoint{1.666577in}{1.800216in}}{\pgfqpoint{1.670110in}{1.801679in}}{\pgfqpoint{1.672715in}{1.804284in}}%
\pgfpathcurveto{\pgfqpoint{1.675319in}{1.806888in}}{\pgfqpoint{1.676783in}{1.810421in}}{\pgfqpoint{1.676783in}{1.814104in}}%
\pgfpathcurveto{\pgfqpoint{1.676783in}{1.817788in}}{\pgfqpoint{1.675319in}{1.821321in}}{\pgfqpoint{1.672715in}{1.823925in}}%
\pgfpathcurveto{\pgfqpoint{1.670110in}{1.826530in}}{\pgfqpoint{1.666577in}{1.827993in}}{\pgfqpoint{1.662894in}{1.827993in}}%
\pgfpathcurveto{\pgfqpoint{1.659210in}{1.827993in}}{\pgfqpoint{1.655677in}{1.826530in}}{\pgfqpoint{1.653073in}{1.823925in}}%
\pgfpathcurveto{\pgfqpoint{1.650468in}{1.821321in}}{\pgfqpoint{1.649005in}{1.817788in}}{\pgfqpoint{1.649005in}{1.814104in}}%
\pgfpathcurveto{\pgfqpoint{1.649005in}{1.810421in}}{\pgfqpoint{1.650468in}{1.806888in}}{\pgfqpoint{1.653073in}{1.804284in}}%
\pgfpathcurveto{\pgfqpoint{1.655677in}{1.801679in}}{\pgfqpoint{1.659210in}{1.800216in}}{\pgfqpoint{1.662894in}{1.800216in}}%
\pgfpathclose%
\pgfusepath{stroke}%
\end{pgfscope}%
\begin{pgfscope}%
\pgfpathrectangle{\pgfqpoint{0.438556in}{0.383578in}}{\pgfqpoint{4.650000in}{2.310000in}}%
\pgfusepath{clip}%
\pgfsetbuttcap%
\pgfsetroundjoin%
\pgfsetlinewidth{0.803000pt}%
\definecolor{currentstroke}{rgb}{0.000000,0.356863,0.509804}%
\pgfsetstrokecolor{currentstroke}%
\pgfsetdash{}{0pt}%
\pgfpathmoveto{\pgfqpoint{1.688118in}{1.822058in}}%
\pgfpathcurveto{\pgfqpoint{1.691802in}{1.822058in}}{\pgfqpoint{1.695335in}{1.823521in}}{\pgfqpoint{1.697939in}{1.826126in}}%
\pgfpathcurveto{\pgfqpoint{1.700544in}{1.828730in}}{\pgfqpoint{1.702007in}{1.832263in}}{\pgfqpoint{1.702007in}{1.835947in}}%
\pgfpathcurveto{\pgfqpoint{1.702007in}{1.839630in}}{\pgfqpoint{1.700544in}{1.843163in}}{\pgfqpoint{1.697939in}{1.845768in}}%
\pgfpathcurveto{\pgfqpoint{1.695335in}{1.848372in}}{\pgfqpoint{1.691802in}{1.849835in}}{\pgfqpoint{1.688118in}{1.849835in}}%
\pgfpathcurveto{\pgfqpoint{1.684435in}{1.849835in}}{\pgfqpoint{1.680902in}{1.848372in}}{\pgfqpoint{1.678297in}{1.845768in}}%
\pgfpathcurveto{\pgfqpoint{1.675693in}{1.843163in}}{\pgfqpoint{1.674229in}{1.839630in}}{\pgfqpoint{1.674229in}{1.835947in}}%
\pgfpathcurveto{\pgfqpoint{1.674229in}{1.832263in}}{\pgfqpoint{1.675693in}{1.828730in}}{\pgfqpoint{1.678297in}{1.826126in}}%
\pgfpathcurveto{\pgfqpoint{1.680902in}{1.823521in}}{\pgfqpoint{1.684435in}{1.822058in}}{\pgfqpoint{1.688118in}{1.822058in}}%
\pgfpathclose%
\pgfusepath{stroke}%
\end{pgfscope}%
\begin{pgfscope}%
\pgfpathrectangle{\pgfqpoint{0.438556in}{0.383578in}}{\pgfqpoint{4.650000in}{2.310000in}}%
\pgfusepath{clip}%
\pgfsetbuttcap%
\pgfsetroundjoin%
\pgfsetlinewidth{0.803000pt}%
\definecolor{currentstroke}{rgb}{0.000000,0.356863,0.509804}%
\pgfsetstrokecolor{currentstroke}%
\pgfsetdash{}{0pt}%
\pgfpathmoveto{\pgfqpoint{1.698115in}{1.843900in}}%
\pgfpathcurveto{\pgfqpoint{1.701798in}{1.843900in}}{\pgfqpoint{1.705331in}{1.845363in}}{\pgfqpoint{1.707936in}{1.847968in}}%
\pgfpathcurveto{\pgfqpoint{1.710540in}{1.850572in}}{\pgfqpoint{1.712004in}{1.854105in}}{\pgfqpoint{1.712004in}{1.857789in}}%
\pgfpathcurveto{\pgfqpoint{1.712004in}{1.861472in}}{\pgfqpoint{1.710540in}{1.865005in}}{\pgfqpoint{1.707936in}{1.867610in}}%
\pgfpathcurveto{\pgfqpoint{1.705331in}{1.870214in}}{\pgfqpoint{1.701798in}{1.871678in}}{\pgfqpoint{1.698115in}{1.871678in}}%
\pgfpathcurveto{\pgfqpoint{1.694431in}{1.871678in}}{\pgfqpoint{1.690898in}{1.870214in}}{\pgfqpoint{1.688294in}{1.867610in}}%
\pgfpathcurveto{\pgfqpoint{1.685689in}{1.865005in}}{\pgfqpoint{1.684226in}{1.861472in}}{\pgfqpoint{1.684226in}{1.857789in}}%
\pgfpathcurveto{\pgfqpoint{1.684226in}{1.854105in}}{\pgfqpoint{1.685689in}{1.850572in}}{\pgfqpoint{1.688294in}{1.847968in}}%
\pgfpathcurveto{\pgfqpoint{1.690898in}{1.845363in}}{\pgfqpoint{1.694431in}{1.843900in}}{\pgfqpoint{1.698115in}{1.843900in}}%
\pgfpathclose%
\pgfusepath{stroke}%
\end{pgfscope}%
\begin{pgfscope}%
\pgfpathrectangle{\pgfqpoint{0.438556in}{0.383578in}}{\pgfqpoint{4.650000in}{2.310000in}}%
\pgfusepath{clip}%
\pgfsetbuttcap%
\pgfsetroundjoin%
\pgfsetlinewidth{0.803000pt}%
\definecolor{currentstroke}{rgb}{0.000000,0.356863,0.509804}%
\pgfsetstrokecolor{currentstroke}%
\pgfsetdash{}{0pt}%
\pgfpathmoveto{\pgfqpoint{1.702421in}{1.865742in}}%
\pgfpathcurveto{\pgfqpoint{1.706104in}{1.865742in}}{\pgfqpoint{1.709638in}{1.867205in}}{\pgfqpoint{1.712242in}{1.869810in}}%
\pgfpathcurveto{\pgfqpoint{1.714847in}{1.872414in}}{\pgfqpoint{1.716310in}{1.875947in}}{\pgfqpoint{1.716310in}{1.879631in}}%
\pgfpathcurveto{\pgfqpoint{1.716310in}{1.883314in}}{\pgfqpoint{1.714847in}{1.886847in}}{\pgfqpoint{1.712242in}{1.889452in}}%
\pgfpathcurveto{\pgfqpoint{1.709638in}{1.892056in}}{\pgfqpoint{1.706104in}{1.893520in}}{\pgfqpoint{1.702421in}{1.893520in}}%
\pgfpathcurveto{\pgfqpoint{1.698738in}{1.893520in}}{\pgfqpoint{1.695205in}{1.892056in}}{\pgfqpoint{1.692600in}{1.889452in}}%
\pgfpathcurveto{\pgfqpoint{1.689996in}{1.886847in}}{\pgfqpoint{1.688532in}{1.883314in}}{\pgfqpoint{1.688532in}{1.879631in}}%
\pgfpathcurveto{\pgfqpoint{1.688532in}{1.875947in}}{\pgfqpoint{1.689996in}{1.872414in}}{\pgfqpoint{1.692600in}{1.869810in}}%
\pgfpathcurveto{\pgfqpoint{1.695205in}{1.867205in}}{\pgfqpoint{1.698738in}{1.865742in}}{\pgfqpoint{1.702421in}{1.865742in}}%
\pgfpathclose%
\pgfusepath{stroke}%
\end{pgfscope}%
\begin{pgfscope}%
\pgfpathrectangle{\pgfqpoint{0.438556in}{0.383578in}}{\pgfqpoint{4.650000in}{2.310000in}}%
\pgfusepath{clip}%
\pgfsetbuttcap%
\pgfsetroundjoin%
\pgfsetlinewidth{0.803000pt}%
\definecolor{currentstroke}{rgb}{0.000000,0.356863,0.509804}%
\pgfsetstrokecolor{currentstroke}%
\pgfsetdash{}{0pt}%
\pgfpathmoveto{\pgfqpoint{1.737211in}{1.887584in}}%
\pgfpathcurveto{\pgfqpoint{1.740895in}{1.887584in}}{\pgfqpoint{1.744428in}{1.889048in}}{\pgfqpoint{1.747032in}{1.891652in}}%
\pgfpathcurveto{\pgfqpoint{1.749637in}{1.894257in}}{\pgfqpoint{1.751100in}{1.897790in}}{\pgfqpoint{1.751100in}{1.901473in}}%
\pgfpathcurveto{\pgfqpoint{1.751100in}{1.905156in}}{\pgfqpoint{1.749637in}{1.908689in}}{\pgfqpoint{1.747032in}{1.911294in}}%
\pgfpathcurveto{\pgfqpoint{1.744428in}{1.913898in}}{\pgfqpoint{1.740895in}{1.915362in}}{\pgfqpoint{1.737211in}{1.915362in}}%
\pgfpathcurveto{\pgfqpoint{1.733528in}{1.915362in}}{\pgfqpoint{1.729995in}{1.913898in}}{\pgfqpoint{1.727390in}{1.911294in}}%
\pgfpathcurveto{\pgfqpoint{1.724786in}{1.908689in}}{\pgfqpoint{1.723322in}{1.905156in}}{\pgfqpoint{1.723322in}{1.901473in}}%
\pgfpathcurveto{\pgfqpoint{1.723322in}{1.897790in}}{\pgfqpoint{1.724786in}{1.894257in}}{\pgfqpoint{1.727390in}{1.891652in}}%
\pgfpathcurveto{\pgfqpoint{1.729995in}{1.889048in}}{\pgfqpoint{1.733528in}{1.887584in}}{\pgfqpoint{1.737211in}{1.887584in}}%
\pgfpathclose%
\pgfusepath{stroke}%
\end{pgfscope}%
\begin{pgfscope}%
\pgfpathrectangle{\pgfqpoint{0.438556in}{0.383578in}}{\pgfqpoint{4.650000in}{2.310000in}}%
\pgfusepath{clip}%
\pgfsetbuttcap%
\pgfsetroundjoin%
\pgfsetlinewidth{0.803000pt}%
\definecolor{currentstroke}{rgb}{0.000000,0.356863,0.509804}%
\pgfsetstrokecolor{currentstroke}%
\pgfsetdash{}{0pt}%
\pgfpathmoveto{\pgfqpoint{1.720691in}{1.953110in}}%
\pgfpathcurveto{\pgfqpoint{1.724375in}{1.953110in}}{\pgfqpoint{1.727908in}{1.954574in}}{\pgfqpoint{1.730512in}{1.957178in}}%
\pgfpathcurveto{\pgfqpoint{1.733117in}{1.959783in}}{\pgfqpoint{1.734580in}{1.963316in}}{\pgfqpoint{1.734580in}{1.966999in}}%
\pgfpathcurveto{\pgfqpoint{1.734580in}{1.970683in}}{\pgfqpoint{1.733117in}{1.974216in}}{\pgfqpoint{1.730512in}{1.976820in}}%
\pgfpathcurveto{\pgfqpoint{1.727908in}{1.979425in}}{\pgfqpoint{1.724375in}{1.980888in}}{\pgfqpoint{1.720691in}{1.980888in}}%
\pgfpathcurveto{\pgfqpoint{1.717008in}{1.980888in}}{\pgfqpoint{1.713475in}{1.979425in}}{\pgfqpoint{1.710870in}{1.976820in}}%
\pgfpathcurveto{\pgfqpoint{1.708266in}{1.974216in}}{\pgfqpoint{1.706802in}{1.970683in}}{\pgfqpoint{1.706802in}{1.966999in}}%
\pgfpathcurveto{\pgfqpoint{1.706802in}{1.963316in}}{\pgfqpoint{1.708266in}{1.959783in}}{\pgfqpoint{1.710870in}{1.957178in}}%
\pgfpathcurveto{\pgfqpoint{1.713475in}{1.954574in}}{\pgfqpoint{1.717008in}{1.953110in}}{\pgfqpoint{1.720691in}{1.953110in}}%
\pgfpathclose%
\pgfusepath{stroke}%
\end{pgfscope}%
\begin{pgfscope}%
\pgfpathrectangle{\pgfqpoint{0.438556in}{0.383578in}}{\pgfqpoint{4.650000in}{2.310000in}}%
\pgfusepath{clip}%
\pgfsetbuttcap%
\pgfsetroundjoin%
\pgfsetlinewidth{0.803000pt}%
\definecolor{currentstroke}{rgb}{0.000000,0.356863,0.509804}%
\pgfsetstrokecolor{currentstroke}%
\pgfsetdash{}{0pt}%
\pgfpathmoveto{\pgfqpoint{1.670261in}{1.996795in}}%
\pgfpathcurveto{\pgfqpoint{1.673944in}{1.996795in}}{\pgfqpoint{1.677477in}{1.998258in}}{\pgfqpoint{1.680081in}{2.000863in}}%
\pgfpathcurveto{\pgfqpoint{1.682686in}{2.003467in}}{\pgfqpoint{1.684149in}{2.007000in}}{\pgfqpoint{1.684149in}{2.010684in}}%
\pgfpathcurveto{\pgfqpoint{1.684149in}{2.014367in}}{\pgfqpoint{1.682686in}{2.017900in}}{\pgfqpoint{1.680081in}{2.020505in}}%
\pgfpathcurveto{\pgfqpoint{1.677477in}{2.023109in}}{\pgfqpoint{1.673944in}{2.024573in}}{\pgfqpoint{1.670261in}{2.024573in}}%
\pgfpathcurveto{\pgfqpoint{1.666577in}{2.024573in}}{\pgfqpoint{1.663044in}{2.023109in}}{\pgfqpoint{1.660440in}{2.020505in}}%
\pgfpathcurveto{\pgfqpoint{1.657835in}{2.017900in}}{\pgfqpoint{1.656372in}{2.014367in}}{\pgfqpoint{1.656372in}{2.010684in}}%
\pgfpathcurveto{\pgfqpoint{1.656372in}{2.007000in}}{\pgfqpoint{1.657835in}{2.003467in}}{\pgfqpoint{1.660440in}{2.000863in}}%
\pgfpathcurveto{\pgfqpoint{1.663044in}{1.998258in}}{\pgfqpoint{1.666577in}{1.996795in}}{\pgfqpoint{1.670261in}{1.996795in}}%
\pgfpathclose%
\pgfusepath{stroke}%
\end{pgfscope}%
\begin{pgfscope}%
\pgfpathrectangle{\pgfqpoint{0.438556in}{0.383578in}}{\pgfqpoint{4.650000in}{2.310000in}}%
\pgfusepath{clip}%
\pgfsetbuttcap%
\pgfsetroundjoin%
\pgfsetlinewidth{0.803000pt}%
\definecolor{currentstroke}{rgb}{0.000000,0.356863,0.509804}%
\pgfsetstrokecolor{currentstroke}%
\pgfsetdash{}{0pt}%
\pgfpathmoveto{\pgfqpoint{1.702036in}{2.040479in}}%
\pgfpathcurveto{\pgfqpoint{1.705720in}{2.040479in}}{\pgfqpoint{1.709253in}{2.041942in}}{\pgfqpoint{1.711857in}{2.044547in}}%
\pgfpathcurveto{\pgfqpoint{1.714462in}{2.047152in}}{\pgfqpoint{1.715925in}{2.050685in}}{\pgfqpoint{1.715925in}{2.054368in}}%
\pgfpathcurveto{\pgfqpoint{1.715925in}{2.058051in}}{\pgfqpoint{1.714462in}{2.061584in}}{\pgfqpoint{1.711857in}{2.064189in}}%
\pgfpathcurveto{\pgfqpoint{1.709253in}{2.066793in}}{\pgfqpoint{1.705720in}{2.068257in}}{\pgfqpoint{1.702036in}{2.068257in}}%
\pgfpathcurveto{\pgfqpoint{1.698353in}{2.068257in}}{\pgfqpoint{1.694820in}{2.066793in}}{\pgfqpoint{1.692215in}{2.064189in}}%
\pgfpathcurveto{\pgfqpoint{1.689611in}{2.061584in}}{\pgfqpoint{1.688147in}{2.058051in}}{\pgfqpoint{1.688147in}{2.054368in}}%
\pgfpathcurveto{\pgfqpoint{1.688147in}{2.050685in}}{\pgfqpoint{1.689611in}{2.047152in}}{\pgfqpoint{1.692215in}{2.044547in}}%
\pgfpathcurveto{\pgfqpoint{1.694820in}{2.041942in}}{\pgfqpoint{1.698353in}{2.040479in}}{\pgfqpoint{1.702036in}{2.040479in}}%
\pgfpathclose%
\pgfusepath{stroke}%
\end{pgfscope}%
\begin{pgfscope}%
\pgfpathrectangle{\pgfqpoint{0.438556in}{0.383578in}}{\pgfqpoint{4.650000in}{2.310000in}}%
\pgfusepath{clip}%
\pgfsetbuttcap%
\pgfsetroundjoin%
\pgfsetlinewidth{0.803000pt}%
\definecolor{currentstroke}{rgb}{0.000000,0.356863,0.509804}%
\pgfsetstrokecolor{currentstroke}%
\pgfsetdash{}{0pt}%
\pgfpathmoveto{\pgfqpoint{1.674283in}{2.084163in}}%
\pgfpathcurveto{\pgfqpoint{1.677966in}{2.084163in}}{\pgfqpoint{1.681499in}{2.085627in}}{\pgfqpoint{1.684104in}{2.088231in}}%
\pgfpathcurveto{\pgfqpoint{1.686708in}{2.090836in}}{\pgfqpoint{1.688172in}{2.094369in}}{\pgfqpoint{1.688172in}{2.098052in}}%
\pgfpathcurveto{\pgfqpoint{1.688172in}{2.101736in}}{\pgfqpoint{1.686708in}{2.105269in}}{\pgfqpoint{1.684104in}{2.107873in}}%
\pgfpathcurveto{\pgfqpoint{1.681499in}{2.110478in}}{\pgfqpoint{1.677966in}{2.111941in}}{\pgfqpoint{1.674283in}{2.111941in}}%
\pgfpathcurveto{\pgfqpoint{1.670600in}{2.111941in}}{\pgfqpoint{1.667066in}{2.110478in}}{\pgfqpoint{1.664462in}{2.107873in}}%
\pgfpathcurveto{\pgfqpoint{1.661857in}{2.105269in}}{\pgfqpoint{1.660394in}{2.101736in}}{\pgfqpoint{1.660394in}{2.098052in}}%
\pgfpathcurveto{\pgfqpoint{1.660394in}{2.094369in}}{\pgfqpoint{1.661857in}{2.090836in}}{\pgfqpoint{1.664462in}{2.088231in}}%
\pgfpathcurveto{\pgfqpoint{1.667066in}{2.085627in}}{\pgfqpoint{1.670600in}{2.084163in}}{\pgfqpoint{1.674283in}{2.084163in}}%
\pgfpathclose%
\pgfusepath{stroke}%
\end{pgfscope}%
\begin{pgfscope}%
\pgfpathrectangle{\pgfqpoint{0.438556in}{0.383578in}}{\pgfqpoint{4.650000in}{2.310000in}}%
\pgfusepath{clip}%
\pgfsetbuttcap%
\pgfsetroundjoin%
\pgfsetlinewidth{0.803000pt}%
\definecolor{currentstroke}{rgb}{0.000000,0.356863,0.509804}%
\pgfsetstrokecolor{currentstroke}%
\pgfsetdash{}{0pt}%
\pgfpathmoveto{\pgfqpoint{1.671378in}{2.127848in}}%
\pgfpathcurveto{\pgfqpoint{1.675062in}{2.127848in}}{\pgfqpoint{1.678595in}{2.129311in}}{\pgfqpoint{1.681199in}{2.131915in}}%
\pgfpathcurveto{\pgfqpoint{1.683804in}{2.134520in}}{\pgfqpoint{1.685267in}{2.138053in}}{\pgfqpoint{1.685267in}{2.141736in}}%
\pgfpathcurveto{\pgfqpoint{1.685267in}{2.145420in}}{\pgfqpoint{1.683804in}{2.148953in}}{\pgfqpoint{1.681199in}{2.151557in}}%
\pgfpathcurveto{\pgfqpoint{1.678595in}{2.154162in}}{\pgfqpoint{1.675062in}{2.155625in}}{\pgfqpoint{1.671378in}{2.155625in}}%
\pgfpathcurveto{\pgfqpoint{1.667695in}{2.155625in}}{\pgfqpoint{1.664162in}{2.154162in}}{\pgfqpoint{1.661557in}{2.151557in}}%
\pgfpathcurveto{\pgfqpoint{1.658953in}{2.148953in}}{\pgfqpoint{1.657489in}{2.145420in}}{\pgfqpoint{1.657489in}{2.141736in}}%
\pgfpathcurveto{\pgfqpoint{1.657489in}{2.138053in}}{\pgfqpoint{1.658953in}{2.134520in}}{\pgfqpoint{1.661557in}{2.131915in}}%
\pgfpathcurveto{\pgfqpoint{1.664162in}{2.129311in}}{\pgfqpoint{1.667695in}{2.127848in}}{\pgfqpoint{1.671378in}{2.127848in}}%
\pgfpathclose%
\pgfusepath{stroke}%
\end{pgfscope}%
\begin{pgfscope}%
\pgfpathrectangle{\pgfqpoint{0.438556in}{0.383578in}}{\pgfqpoint{4.650000in}{2.310000in}}%
\pgfusepath{clip}%
\pgfsetbuttcap%
\pgfsetroundjoin%
\pgfsetlinewidth{0.803000pt}%
\definecolor{currentstroke}{rgb}{0.000000,0.356863,0.509804}%
\pgfsetstrokecolor{currentstroke}%
\pgfsetdash{}{0pt}%
\pgfpathmoveto{\pgfqpoint{1.722789in}{2.149690in}}%
\pgfpathcurveto{\pgfqpoint{1.726473in}{2.149690in}}{\pgfqpoint{1.730006in}{2.151153in}}{\pgfqpoint{1.732610in}{2.153758in}}%
\pgfpathcurveto{\pgfqpoint{1.735215in}{2.156362in}}{\pgfqpoint{1.736678in}{2.159895in}}{\pgfqpoint{1.736678in}{2.163579in}}%
\pgfpathcurveto{\pgfqpoint{1.736678in}{2.167262in}}{\pgfqpoint{1.735215in}{2.170795in}}{\pgfqpoint{1.732610in}{2.173399in}}%
\pgfpathcurveto{\pgfqpoint{1.730006in}{2.176004in}}{\pgfqpoint{1.726473in}{2.177467in}}{\pgfqpoint{1.722789in}{2.177467in}}%
\pgfpathcurveto{\pgfqpoint{1.719106in}{2.177467in}}{\pgfqpoint{1.715573in}{2.176004in}}{\pgfqpoint{1.712969in}{2.173399in}}%
\pgfpathcurveto{\pgfqpoint{1.710364in}{2.170795in}}{\pgfqpoint{1.708901in}{2.167262in}}{\pgfqpoint{1.708901in}{2.163579in}}%
\pgfpathcurveto{\pgfqpoint{1.708901in}{2.159895in}}{\pgfqpoint{1.710364in}{2.156362in}}{\pgfqpoint{1.712969in}{2.153758in}}%
\pgfpathcurveto{\pgfqpoint{1.715573in}{2.151153in}}{\pgfqpoint{1.719106in}{2.149690in}}{\pgfqpoint{1.722789in}{2.149690in}}%
\pgfpathclose%
\pgfusepath{stroke}%
\end{pgfscope}%
\begin{pgfscope}%
\pgfpathrectangle{\pgfqpoint{0.438556in}{0.383578in}}{\pgfqpoint{4.650000in}{2.310000in}}%
\pgfusepath{clip}%
\pgfsetbuttcap%
\pgfsetroundjoin%
\pgfsetlinewidth{0.803000pt}%
\definecolor{currentstroke}{rgb}{0.000000,0.356863,0.509804}%
\pgfsetstrokecolor{currentstroke}%
\pgfsetdash{}{0pt}%
\pgfpathmoveto{\pgfqpoint{1.733959in}{2.171532in}}%
\pgfpathcurveto{\pgfqpoint{1.737642in}{2.171532in}}{\pgfqpoint{1.741175in}{2.172995in}}{\pgfqpoint{1.743780in}{2.175600in}}%
\pgfpathcurveto{\pgfqpoint{1.746384in}{2.178204in}}{\pgfqpoint{1.747848in}{2.181737in}}{\pgfqpoint{1.747848in}{2.185421in}}%
\pgfpathcurveto{\pgfqpoint{1.747848in}{2.189104in}}{\pgfqpoint{1.746384in}{2.192637in}}{\pgfqpoint{1.743780in}{2.195242in}}%
\pgfpathcurveto{\pgfqpoint{1.741175in}{2.197846in}}{\pgfqpoint{1.737642in}{2.199310in}}{\pgfqpoint{1.733959in}{2.199310in}}%
\pgfpathcurveto{\pgfqpoint{1.730275in}{2.199310in}}{\pgfqpoint{1.726742in}{2.197846in}}{\pgfqpoint{1.724138in}{2.195242in}}%
\pgfpathcurveto{\pgfqpoint{1.721533in}{2.192637in}}{\pgfqpoint{1.720070in}{2.189104in}}{\pgfqpoint{1.720070in}{2.185421in}}%
\pgfpathcurveto{\pgfqpoint{1.720070in}{2.181737in}}{\pgfqpoint{1.721533in}{2.178204in}}{\pgfqpoint{1.724138in}{2.175600in}}%
\pgfpathcurveto{\pgfqpoint{1.726742in}{2.172995in}}{\pgfqpoint{1.730275in}{2.171532in}}{\pgfqpoint{1.733959in}{2.171532in}}%
\pgfpathclose%
\pgfusepath{stroke}%
\end{pgfscope}%
\begin{pgfscope}%
\pgfpathrectangle{\pgfqpoint{0.438556in}{0.383578in}}{\pgfqpoint{4.650000in}{2.310000in}}%
\pgfusepath{clip}%
\pgfsetbuttcap%
\pgfsetroundjoin%
\pgfsetlinewidth{0.803000pt}%
\definecolor{currentstroke}{rgb}{0.000000,0.356863,0.509804}%
\pgfsetstrokecolor{currentstroke}%
\pgfsetdash{}{0pt}%
\pgfpathmoveto{\pgfqpoint{1.692663in}{2.258900in}}%
\pgfpathcurveto{\pgfqpoint{1.696346in}{2.258900in}}{\pgfqpoint{1.699879in}{2.260364in}}{\pgfqpoint{1.702484in}{2.262968in}}%
\pgfpathcurveto{\pgfqpoint{1.705088in}{2.265573in}}{\pgfqpoint{1.706552in}{2.269106in}}{\pgfqpoint{1.706552in}{2.272789in}}%
\pgfpathcurveto{\pgfqpoint{1.706552in}{2.276473in}}{\pgfqpoint{1.705088in}{2.280006in}}{\pgfqpoint{1.702484in}{2.282610in}}%
\pgfpathcurveto{\pgfqpoint{1.699879in}{2.285215in}}{\pgfqpoint{1.696346in}{2.286678in}}{\pgfqpoint{1.692663in}{2.286678in}}%
\pgfpathcurveto{\pgfqpoint{1.688980in}{2.286678in}}{\pgfqpoint{1.685447in}{2.285215in}}{\pgfqpoint{1.682842in}{2.282610in}}%
\pgfpathcurveto{\pgfqpoint{1.680238in}{2.280006in}}{\pgfqpoint{1.678774in}{2.276473in}}{\pgfqpoint{1.678774in}{2.272789in}}%
\pgfpathcurveto{\pgfqpoint{1.678774in}{2.269106in}}{\pgfqpoint{1.680238in}{2.265573in}}{\pgfqpoint{1.682842in}{2.262968in}}%
\pgfpathcurveto{\pgfqpoint{1.685447in}{2.260364in}}{\pgfqpoint{1.688980in}{2.258900in}}{\pgfqpoint{1.692663in}{2.258900in}}%
\pgfpathclose%
\pgfusepath{stroke}%
\end{pgfscope}%
\begin{pgfscope}%
\pgfpathrectangle{\pgfqpoint{0.438556in}{0.383578in}}{\pgfqpoint{4.650000in}{2.310000in}}%
\pgfusepath{clip}%
\pgfsetbuttcap%
\pgfsetroundjoin%
\pgfsetlinewidth{0.803000pt}%
\definecolor{currentstroke}{rgb}{0.000000,0.356863,0.509804}%
\pgfsetstrokecolor{currentstroke}%
\pgfsetdash{}{0pt}%
\pgfpathmoveto{\pgfqpoint{1.698591in}{2.280742in}}%
\pgfpathcurveto{\pgfqpoint{1.702275in}{2.280742in}}{\pgfqpoint{1.705808in}{2.282206in}}{\pgfqpoint{1.708412in}{2.284810in}}%
\pgfpathcurveto{\pgfqpoint{1.711017in}{2.287415in}}{\pgfqpoint{1.712480in}{2.290948in}}{\pgfqpoint{1.712480in}{2.294631in}}%
\pgfpathcurveto{\pgfqpoint{1.712480in}{2.298315in}}{\pgfqpoint{1.711017in}{2.301848in}}{\pgfqpoint{1.708412in}{2.304452in}}%
\pgfpathcurveto{\pgfqpoint{1.705808in}{2.307057in}}{\pgfqpoint{1.702275in}{2.308520in}}{\pgfqpoint{1.698591in}{2.308520in}}%
\pgfpathcurveto{\pgfqpoint{1.694908in}{2.308520in}}{\pgfqpoint{1.691375in}{2.307057in}}{\pgfqpoint{1.688770in}{2.304452in}}%
\pgfpathcurveto{\pgfqpoint{1.686166in}{2.301848in}}{\pgfqpoint{1.684702in}{2.298315in}}{\pgfqpoint{1.684702in}{2.294631in}}%
\pgfpathcurveto{\pgfqpoint{1.684702in}{2.290948in}}{\pgfqpoint{1.686166in}{2.287415in}}{\pgfqpoint{1.688770in}{2.284810in}}%
\pgfpathcurveto{\pgfqpoint{1.691375in}{2.282206in}}{\pgfqpoint{1.694908in}{2.280742in}}{\pgfqpoint{1.698591in}{2.280742in}}%
\pgfpathclose%
\pgfusepath{stroke}%
\end{pgfscope}%
\begin{pgfscope}%
\pgfpathrectangle{\pgfqpoint{0.438556in}{0.383578in}}{\pgfqpoint{4.650000in}{2.310000in}}%
\pgfusepath{clip}%
\pgfsetbuttcap%
\pgfsetroundjoin%
\pgfsetlinewidth{0.803000pt}%
\definecolor{currentstroke}{rgb}{0.000000,0.356863,0.509804}%
\pgfsetstrokecolor{currentstroke}%
\pgfsetdash{}{0pt}%
\pgfpathmoveto{\pgfqpoint{1.701175in}{2.324427in}}%
\pgfpathcurveto{\pgfqpoint{1.704858in}{2.324427in}}{\pgfqpoint{1.708391in}{2.325890in}}{\pgfqpoint{1.710996in}{2.328495in}}%
\pgfpathcurveto{\pgfqpoint{1.713600in}{2.331099in}}{\pgfqpoint{1.715064in}{2.334632in}}{\pgfqpoint{1.715064in}{2.338316in}}%
\pgfpathcurveto{\pgfqpoint{1.715064in}{2.341999in}}{\pgfqpoint{1.713600in}{2.345532in}}{\pgfqpoint{1.710996in}{2.348137in}}%
\pgfpathcurveto{\pgfqpoint{1.708391in}{2.350741in}}{\pgfqpoint{1.704858in}{2.352204in}}{\pgfqpoint{1.701175in}{2.352204in}}%
\pgfpathcurveto{\pgfqpoint{1.697492in}{2.352204in}}{\pgfqpoint{1.693959in}{2.350741in}}{\pgfqpoint{1.691354in}{2.348137in}}%
\pgfpathcurveto{\pgfqpoint{1.688750in}{2.345532in}}{\pgfqpoint{1.687286in}{2.341999in}}{\pgfqpoint{1.687286in}{2.338316in}}%
\pgfpathcurveto{\pgfqpoint{1.687286in}{2.334632in}}{\pgfqpoint{1.688750in}{2.331099in}}{\pgfqpoint{1.691354in}{2.328495in}}%
\pgfpathcurveto{\pgfqpoint{1.693959in}{2.325890in}}{\pgfqpoint{1.697492in}{2.324427in}}{\pgfqpoint{1.701175in}{2.324427in}}%
\pgfpathclose%
\pgfusepath{stroke}%
\end{pgfscope}%
\begin{pgfscope}%
\pgfpathrectangle{\pgfqpoint{0.438556in}{0.383578in}}{\pgfqpoint{4.650000in}{2.310000in}}%
\pgfusepath{clip}%
\pgfsetbuttcap%
\pgfsetroundjoin%
\pgfsetlinewidth{0.803000pt}%
\definecolor{currentstroke}{rgb}{0.000000,0.356863,0.509804}%
\pgfsetstrokecolor{currentstroke}%
\pgfsetdash{}{0pt}%
\pgfpathmoveto{\pgfqpoint{1.719280in}{2.346269in}}%
\pgfpathcurveto{\pgfqpoint{1.722964in}{2.346269in}}{\pgfqpoint{1.726497in}{2.347732in}}{\pgfqpoint{1.729101in}{2.350337in}}%
\pgfpathcurveto{\pgfqpoint{1.731706in}{2.352941in}}{\pgfqpoint{1.733169in}{2.356474in}}{\pgfqpoint{1.733169in}{2.360158in}}%
\pgfpathcurveto{\pgfqpoint{1.733169in}{2.363841in}}{\pgfqpoint{1.731706in}{2.367374in}}{\pgfqpoint{1.729101in}{2.369979in}}%
\pgfpathcurveto{\pgfqpoint{1.726497in}{2.372583in}}{\pgfqpoint{1.722964in}{2.374047in}}{\pgfqpoint{1.719280in}{2.374047in}}%
\pgfpathcurveto{\pgfqpoint{1.715597in}{2.374047in}}{\pgfqpoint{1.712064in}{2.372583in}}{\pgfqpoint{1.709459in}{2.369979in}}%
\pgfpathcurveto{\pgfqpoint{1.706855in}{2.367374in}}{\pgfqpoint{1.705391in}{2.363841in}}{\pgfqpoint{1.705391in}{2.360158in}}%
\pgfpathcurveto{\pgfqpoint{1.705391in}{2.356474in}}{\pgfqpoint{1.706855in}{2.352941in}}{\pgfqpoint{1.709459in}{2.350337in}}%
\pgfpathcurveto{\pgfqpoint{1.712064in}{2.347732in}}{\pgfqpoint{1.715597in}{2.346269in}}{\pgfqpoint{1.719280in}{2.346269in}}%
\pgfpathclose%
\pgfusepath{stroke}%
\end{pgfscope}%
\begin{pgfscope}%
\pgfpathrectangle{\pgfqpoint{0.438556in}{0.383578in}}{\pgfqpoint{4.650000in}{2.310000in}}%
\pgfusepath{clip}%
\pgfsetbuttcap%
\pgfsetroundjoin%
\pgfsetlinewidth{0.803000pt}%
\definecolor{currentstroke}{rgb}{0.000000,0.356863,0.509804}%
\pgfsetstrokecolor{currentstroke}%
\pgfsetdash{}{0pt}%
\pgfpathmoveto{\pgfqpoint{1.673092in}{2.433637in}}%
\pgfpathcurveto{\pgfqpoint{1.676775in}{2.433637in}}{\pgfqpoint{1.680308in}{2.435101in}}{\pgfqpoint{1.682913in}{2.437705in}}%
\pgfpathcurveto{\pgfqpoint{1.685517in}{2.440310in}}{\pgfqpoint{1.686981in}{2.443843in}}{\pgfqpoint{1.686981in}{2.447526in}}%
\pgfpathcurveto{\pgfqpoint{1.686981in}{2.451210in}}{\pgfqpoint{1.685517in}{2.454743in}}{\pgfqpoint{1.682913in}{2.457347in}}%
\pgfpathcurveto{\pgfqpoint{1.680308in}{2.459952in}}{\pgfqpoint{1.676775in}{2.461415in}}{\pgfqpoint{1.673092in}{2.461415in}}%
\pgfpathcurveto{\pgfqpoint{1.669408in}{2.461415in}}{\pgfqpoint{1.665875in}{2.459952in}}{\pgfqpoint{1.663271in}{2.457347in}}%
\pgfpathcurveto{\pgfqpoint{1.660666in}{2.454743in}}{\pgfqpoint{1.659203in}{2.451210in}}{\pgfqpoint{1.659203in}{2.447526in}}%
\pgfpathcurveto{\pgfqpoint{1.659203in}{2.443843in}}{\pgfqpoint{1.660666in}{2.440310in}}{\pgfqpoint{1.663271in}{2.437705in}}%
\pgfpathcurveto{\pgfqpoint{1.665875in}{2.435101in}}{\pgfqpoint{1.669408in}{2.433637in}}{\pgfqpoint{1.673092in}{2.433637in}}%
\pgfpathclose%
\pgfusepath{stroke}%
\end{pgfscope}%
\begin{pgfscope}%
\pgfpathrectangle{\pgfqpoint{0.438556in}{0.383578in}}{\pgfqpoint{4.650000in}{2.310000in}}%
\pgfusepath{clip}%
\pgfsetbuttcap%
\pgfsetroundjoin%
\pgfsetlinewidth{0.803000pt}%
\definecolor{currentstroke}{rgb}{0.000000,0.356863,0.509804}%
\pgfsetstrokecolor{currentstroke}%
\pgfsetdash{}{0pt}%
\pgfpathmoveto{\pgfqpoint{1.733876in}{2.455480in}}%
\pgfpathcurveto{\pgfqpoint{1.737560in}{2.455480in}}{\pgfqpoint{1.741093in}{2.456943in}}{\pgfqpoint{1.743697in}{2.459547in}}%
\pgfpathcurveto{\pgfqpoint{1.746302in}{2.462152in}}{\pgfqpoint{1.747765in}{2.465685in}}{\pgfqpoint{1.747765in}{2.469368in}}%
\pgfpathcurveto{\pgfqpoint{1.747765in}{2.473052in}}{\pgfqpoint{1.746302in}{2.476585in}}{\pgfqpoint{1.743697in}{2.479189in}}%
\pgfpathcurveto{\pgfqpoint{1.741093in}{2.481794in}}{\pgfqpoint{1.737560in}{2.483257in}}{\pgfqpoint{1.733876in}{2.483257in}}%
\pgfpathcurveto{\pgfqpoint{1.730193in}{2.483257in}}{\pgfqpoint{1.726660in}{2.481794in}}{\pgfqpoint{1.724055in}{2.479189in}}%
\pgfpathcurveto{\pgfqpoint{1.721451in}{2.476585in}}{\pgfqpoint{1.719987in}{2.473052in}}{\pgfqpoint{1.719987in}{2.469368in}}%
\pgfpathcurveto{\pgfqpoint{1.719987in}{2.465685in}}{\pgfqpoint{1.721451in}{2.462152in}}{\pgfqpoint{1.724055in}{2.459547in}}%
\pgfpathcurveto{\pgfqpoint{1.726660in}{2.456943in}}{\pgfqpoint{1.730193in}{2.455480in}}{\pgfqpoint{1.733876in}{2.455480in}}%
\pgfpathclose%
\pgfusepath{stroke}%
\end{pgfscope}%
\begin{pgfscope}%
\pgfpathrectangle{\pgfqpoint{0.438556in}{0.383578in}}{\pgfqpoint{4.650000in}{2.310000in}}%
\pgfusepath{clip}%
\pgfsetbuttcap%
\pgfsetroundjoin%
\pgfsetlinewidth{0.803000pt}%
\definecolor{currentstroke}{rgb}{0.000000,0.356863,0.509804}%
\pgfsetstrokecolor{currentstroke}%
\pgfsetdash{}{0pt}%
\pgfpathmoveto{\pgfqpoint{1.710044in}{2.564690in}}%
\pgfpathcurveto{\pgfqpoint{1.713728in}{2.564690in}}{\pgfqpoint{1.717261in}{2.566154in}}{\pgfqpoint{1.719865in}{2.568758in}}%
\pgfpathcurveto{\pgfqpoint{1.722470in}{2.571363in}}{\pgfqpoint{1.723933in}{2.574896in}}{\pgfqpoint{1.723933in}{2.578579in}}%
\pgfpathcurveto{\pgfqpoint{1.723933in}{2.582262in}}{\pgfqpoint{1.722470in}{2.585795in}}{\pgfqpoint{1.719865in}{2.588400in}}%
\pgfpathcurveto{\pgfqpoint{1.717261in}{2.591005in}}{\pgfqpoint{1.713728in}{2.592468in}}{\pgfqpoint{1.710044in}{2.592468in}}%
\pgfpathcurveto{\pgfqpoint{1.706361in}{2.592468in}}{\pgfqpoint{1.702828in}{2.591005in}}{\pgfqpoint{1.700223in}{2.588400in}}%
\pgfpathcurveto{\pgfqpoint{1.697619in}{2.585795in}}{\pgfqpoint{1.696155in}{2.582262in}}{\pgfqpoint{1.696155in}{2.578579in}}%
\pgfpathcurveto{\pgfqpoint{1.696155in}{2.574896in}}{\pgfqpoint{1.697619in}{2.571363in}}{\pgfqpoint{1.700223in}{2.568758in}}%
\pgfpathcurveto{\pgfqpoint{1.702828in}{2.566154in}}{\pgfqpoint{1.706361in}{2.564690in}}{\pgfqpoint{1.710044in}{2.564690in}}%
\pgfpathclose%
\pgfusepath{stroke}%
\end{pgfscope}%
\begin{pgfscope}%
\pgfpathrectangle{\pgfqpoint{0.438556in}{0.383578in}}{\pgfqpoint{4.650000in}{2.310000in}}%
\pgfusepath{clip}%
\pgfsetbuttcap%
\pgfsetroundjoin%
\definecolor{currentfill}{rgb}{0.490196,0.588235,0.431373}%
\pgfsetfillcolor{currentfill}%
\pgfsetlinewidth{0.803000pt}%
\definecolor{currentstroke}{rgb}{0.490196,0.588235,0.431373}%
\pgfsetstrokecolor{currentstroke}%
\pgfsetdash{}{0pt}%
\pgfsys@defobject{currentmarker}{\pgfqpoint{-0.006944in}{-0.006944in}}{\pgfqpoint{0.006944in}{0.006944in}}{%
\pgfpathmoveto{\pgfqpoint{0.000000in}{-0.006944in}}%
\pgfpathcurveto{\pgfqpoint{0.001842in}{-0.006944in}}{\pgfqpoint{0.003608in}{-0.006213in}}{\pgfqpoint{0.004910in}{-0.004910in}}%
\pgfpathcurveto{\pgfqpoint{0.006213in}{-0.003608in}}{\pgfqpoint{0.006944in}{-0.001842in}}{\pgfqpoint{0.006944in}{0.000000in}}%
\pgfpathcurveto{\pgfqpoint{0.006944in}{0.001842in}}{\pgfqpoint{0.006213in}{0.003608in}}{\pgfqpoint{0.004910in}{0.004910in}}%
\pgfpathcurveto{\pgfqpoint{0.003608in}{0.006213in}}{\pgfqpoint{0.001842in}{0.006944in}}{\pgfqpoint{0.000000in}{0.006944in}}%
\pgfpathcurveto{\pgfqpoint{-0.001842in}{0.006944in}}{\pgfqpoint{-0.003608in}{0.006213in}}{\pgfqpoint{-0.004910in}{0.004910in}}%
\pgfpathcurveto{\pgfqpoint{-0.006213in}{0.003608in}}{\pgfqpoint{-0.006944in}{0.001842in}}{\pgfqpoint{-0.006944in}{0.000000in}}%
\pgfpathcurveto{\pgfqpoint{-0.006944in}{-0.001842in}}{\pgfqpoint{-0.006213in}{-0.003608in}}{\pgfqpoint{-0.004910in}{-0.004910in}}%
\pgfpathcurveto{\pgfqpoint{-0.003608in}{-0.006213in}}{\pgfqpoint{-0.001842in}{-0.006944in}}{\pgfqpoint{0.000000in}{-0.006944in}}%
\pgfpathclose%
\pgfusepath{stroke,fill}%
}%
\begin{pgfscope}%
\pgfsys@transformshift{2.236553in}{0.503577in}%
\pgfsys@useobject{currentmarker}{}%
\end{pgfscope}%
\begin{pgfscope}%
\pgfsys@transformshift{2.261594in}{0.634629in}%
\pgfsys@useobject{currentmarker}{}%
\end{pgfscope}%
\begin{pgfscope}%
\pgfsys@transformshift{2.235609in}{0.656471in}%
\pgfsys@useobject{currentmarker}{}%
\end{pgfscope}%
\begin{pgfscope}%
\pgfsys@transformshift{2.305849in}{0.700156in}%
\pgfsys@useobject{currentmarker}{}%
\end{pgfscope}%
\begin{pgfscope}%
\pgfsys@transformshift{2.265616in}{0.721998in}%
\pgfsys@useobject{currentmarker}{}%
\end{pgfscope}%
\begin{pgfscope}%
\pgfsys@transformshift{2.241940in}{0.809366in}%
\pgfsys@useobject{currentmarker}{}%
\end{pgfscope}%
\begin{pgfscope}%
\pgfsys@transformshift{2.312602in}{0.831209in}%
\pgfsys@useobject{currentmarker}{}%
\end{pgfscope}%
\begin{pgfscope}%
\pgfsys@transformshift{2.262620in}{0.874893in}%
\pgfsys@useobject{currentmarker}{}%
\end{pgfscope}%
\begin{pgfscope}%
\pgfsys@transformshift{2.313097in}{0.896735in}%
\pgfsys@useobject{currentmarker}{}%
\end{pgfscope}%
\begin{pgfscope}%
\pgfsys@transformshift{2.235636in}{0.940419in}%
\pgfsys@useobject{currentmarker}{}%
\end{pgfscope}%
\begin{pgfscope}%
\pgfsys@transformshift{2.241088in}{0.962261in}%
\pgfsys@useobject{currentmarker}{}%
\end{pgfscope}%
\begin{pgfscope}%
\pgfsys@transformshift{2.292142in}{1.027788in}%
\pgfsys@useobject{currentmarker}{}%
\end{pgfscope}%
\begin{pgfscope}%
\pgfsys@transformshift{2.257214in}{1.049630in}%
\pgfsys@useobject{currentmarker}{}%
\end{pgfscope}%
\begin{pgfscope}%
\pgfsys@transformshift{2.258809in}{1.093314in}%
\pgfsys@useobject{currentmarker}{}%
\end{pgfscope}%
\begin{pgfscope}%
\pgfsys@transformshift{2.268054in}{1.115156in}%
\pgfsys@useobject{currentmarker}{}%
\end{pgfscope}%
\begin{pgfscope}%
\pgfsys@transformshift{2.307902in}{1.180683in}%
\pgfsys@useobject{currentmarker}{}%
\end{pgfscope}%
\begin{pgfscope}%
\pgfsys@transformshift{2.254319in}{1.224367in}%
\pgfsys@useobject{currentmarker}{}%
\end{pgfscope}%
\begin{pgfscope}%
\pgfsys@transformshift{2.315177in}{1.246209in}%
\pgfsys@useobject{currentmarker}{}%
\end{pgfscope}%
\begin{pgfscope}%
\pgfsys@transformshift{2.251350in}{1.268051in}%
\pgfsys@useobject{currentmarker}{}%
\end{pgfscope}%
\begin{pgfscope}%
\pgfsys@transformshift{2.299463in}{1.289893in}%
\pgfsys@useobject{currentmarker}{}%
\end{pgfscope}%
\begin{pgfscope}%
\pgfsys@transformshift{2.248162in}{1.311735in}%
\pgfsys@useobject{currentmarker}{}%
\end{pgfscope}%
\begin{pgfscope}%
\pgfsys@transformshift{2.229397in}{1.333578in}%
\pgfsys@useobject{currentmarker}{}%
\end{pgfscope}%
\begin{pgfscope}%
\pgfsys@transformshift{2.302276in}{1.355420in}%
\pgfsys@useobject{currentmarker}{}%
\end{pgfscope}%
\begin{pgfscope}%
\pgfsys@transformshift{2.250571in}{1.464630in}%
\pgfsys@useobject{currentmarker}{}%
\end{pgfscope}%
\begin{pgfscope}%
\pgfsys@transformshift{2.260384in}{1.486472in}%
\pgfsys@useobject{currentmarker}{}%
\end{pgfscope}%
\begin{pgfscope}%
\pgfsys@transformshift{2.229672in}{1.508315in}%
\pgfsys@useobject{currentmarker}{}%
\end{pgfscope}%
\begin{pgfscope}%
\pgfsys@transformshift{2.243241in}{1.551999in}%
\pgfsys@useobject{currentmarker}{}%
\end{pgfscope}%
\begin{pgfscope}%
\pgfsys@transformshift{2.262492in}{1.661210in}%
\pgfsys@useobject{currentmarker}{}%
\end{pgfscope}%
\begin{pgfscope}%
\pgfsys@transformshift{2.303320in}{1.683052in}%
\pgfsys@useobject{currentmarker}{}%
\end{pgfscope}%
\begin{pgfscope}%
\pgfsys@transformshift{2.295624in}{1.704894in}%
\pgfsys@useobject{currentmarker}{}%
\end{pgfscope}%
\begin{pgfscope}%
\pgfsys@transformshift{2.242957in}{1.726736in}%
\pgfsys@useobject{currentmarker}{}%
\end{pgfscope}%
\begin{pgfscope}%
\pgfsys@transformshift{2.250205in}{1.792262in}%
\pgfsys@useobject{currentmarker}{}%
\end{pgfscope}%
\begin{pgfscope}%
\pgfsys@transformshift{2.235554in}{1.814104in}%
\pgfsys@useobject{currentmarker}{}%
\end{pgfscope}%
\begin{pgfscope}%
\pgfsys@transformshift{2.260778in}{1.835947in}%
\pgfsys@useobject{currentmarker}{}%
\end{pgfscope}%
\begin{pgfscope}%
\pgfsys@transformshift{2.270775in}{1.857789in}%
\pgfsys@useobject{currentmarker}{}%
\end{pgfscope}%
\begin{pgfscope}%
\pgfsys@transformshift{2.275081in}{1.879631in}%
\pgfsys@useobject{currentmarker}{}%
\end{pgfscope}%
\begin{pgfscope}%
\pgfsys@transformshift{2.309871in}{1.901473in}%
\pgfsys@useobject{currentmarker}{}%
\end{pgfscope}%
\begin{pgfscope}%
\pgfsys@transformshift{2.265745in}{1.945157in}%
\pgfsys@useobject{currentmarker}{}%
\end{pgfscope}%
\begin{pgfscope}%
\pgfsys@transformshift{2.293351in}{1.966999in}%
\pgfsys@useobject{currentmarker}{}%
\end{pgfscope}%
\begin{pgfscope}%
\pgfsys@transformshift{2.242921in}{2.010684in}%
\pgfsys@useobject{currentmarker}{}%
\end{pgfscope}%
\begin{pgfscope}%
\pgfsys@transformshift{2.267146in}{2.032526in}%
\pgfsys@useobject{currentmarker}{}%
\end{pgfscope}%
\begin{pgfscope}%
\pgfsys@transformshift{2.274696in}{2.054368in}%
\pgfsys@useobject{currentmarker}{}%
\end{pgfscope}%
\begin{pgfscope}%
\pgfsys@transformshift{2.246943in}{2.098052in}%
\pgfsys@useobject{currentmarker}{}%
\end{pgfscope}%
\begin{pgfscope}%
\pgfsys@transformshift{2.284720in}{2.119894in}%
\pgfsys@useobject{currentmarker}{}%
\end{pgfscope}%
\begin{pgfscope}%
\pgfsys@transformshift{2.244038in}{2.141736in}%
\pgfsys@useobject{currentmarker}{}%
\end{pgfscope}%
\begin{pgfscope}%
\pgfsys@transformshift{2.295450in}{2.163579in}%
\pgfsys@useobject{currentmarker}{}%
\end{pgfscope}%
\begin{pgfscope}%
\pgfsys@transformshift{2.306619in}{2.185421in}%
\pgfsys@useobject{currentmarker}{}%
\end{pgfscope}%
\begin{pgfscope}%
\pgfsys@transformshift{2.275191in}{2.250947in}%
\pgfsys@useobject{currentmarker}{}%
\end{pgfscope}%
\begin{pgfscope}%
\pgfsys@transformshift{2.265323in}{2.272789in}%
\pgfsys@useobject{currentmarker}{}%
\end{pgfscope}%
\begin{pgfscope}%
\pgfsys@transformshift{2.271251in}{2.294631in}%
\pgfsys@useobject{currentmarker}{}%
\end{pgfscope}%
\begin{pgfscope}%
\pgfsys@transformshift{2.274037in}{2.316473in}%
\pgfsys@useobject{currentmarker}{}%
\end{pgfscope}%
\begin{pgfscope}%
\pgfsys@transformshift{2.273835in}{2.338316in}%
\pgfsys@useobject{currentmarker}{}%
\end{pgfscope}%
\begin{pgfscope}%
\pgfsys@transformshift{2.291940in}{2.360158in}%
\pgfsys@useobject{currentmarker}{}%
\end{pgfscope}%
\begin{pgfscope}%
\pgfsys@transformshift{2.312006in}{2.403842in}%
\pgfsys@useobject{currentmarker}{}%
\end{pgfscope}%
\begin{pgfscope}%
\pgfsys@transformshift{2.240896in}{2.425684in}%
\pgfsys@useobject{currentmarker}{}%
\end{pgfscope}%
\begin{pgfscope}%
\pgfsys@transformshift{2.245752in}{2.447526in}%
\pgfsys@useobject{currentmarker}{}%
\end{pgfscope}%
\begin{pgfscope}%
\pgfsys@transformshift{2.306536in}{2.469368in}%
\pgfsys@useobject{currentmarker}{}%
\end{pgfscope}%
\begin{pgfscope}%
\pgfsys@transformshift{2.258689in}{2.513053in}%
\pgfsys@useobject{currentmarker}{}%
\end{pgfscope}%
\begin{pgfscope}%
\pgfsys@transformshift{2.232530in}{2.534895in}%
\pgfsys@useobject{currentmarker}{}%
\end{pgfscope}%
\begin{pgfscope}%
\pgfsys@transformshift{2.282704in}{2.578579in}%
\pgfsys@useobject{currentmarker}{}%
\end{pgfscope}%
\end{pgfscope}%
\begin{pgfscope}%
\pgfpathrectangle{\pgfqpoint{0.438556in}{0.383578in}}{\pgfqpoint{4.650000in}{2.310000in}}%
\pgfusepath{clip}%
\pgfsetbuttcap%
\pgfsetroundjoin%
\pgfsetlinewidth{0.803000pt}%
\definecolor{currentstroke}{rgb}{0.686275,0.352941,0.313725}%
\pgfsetstrokecolor{currentstroke}%
\pgfsetdash{}{0pt}%
\pgfpathmoveto{\pgfqpoint{2.809213in}{0.482743in}}%
\pgfpathcurveto{\pgfqpoint{2.814738in}{0.482743in}}{\pgfqpoint{2.820037in}{0.484938in}}{\pgfqpoint{2.823944in}{0.488845in}}%
\pgfpathcurveto{\pgfqpoint{2.827851in}{0.492752in}}{\pgfqpoint{2.830046in}{0.498051in}}{\pgfqpoint{2.830046in}{0.503577in}}%
\pgfpathcurveto{\pgfqpoint{2.830046in}{0.509102in}}{\pgfqpoint{2.827851in}{0.514401in}}{\pgfqpoint{2.823944in}{0.518308in}}%
\pgfpathcurveto{\pgfqpoint{2.820037in}{0.522215in}}{\pgfqpoint{2.814738in}{0.524410in}}{\pgfqpoint{2.809213in}{0.524410in}}%
\pgfpathcurveto{\pgfqpoint{2.803688in}{0.524410in}}{\pgfqpoint{2.798388in}{0.522215in}}{\pgfqpoint{2.794481in}{0.518308in}}%
\pgfpathcurveto{\pgfqpoint{2.790575in}{0.514401in}}{\pgfqpoint{2.788379in}{0.509102in}}{\pgfqpoint{2.788379in}{0.503577in}}%
\pgfpathcurveto{\pgfqpoint{2.788379in}{0.498051in}}{\pgfqpoint{2.790575in}{0.492752in}}{\pgfqpoint{2.794481in}{0.488845in}}%
\pgfpathcurveto{\pgfqpoint{2.798388in}{0.484938in}}{\pgfqpoint{2.803688in}{0.482743in}}{\pgfqpoint{2.809213in}{0.482743in}}%
\pgfpathclose%
\pgfusepath{stroke}%
\end{pgfscope}%
\begin{pgfscope}%
\pgfpathrectangle{\pgfqpoint{0.438556in}{0.383578in}}{\pgfqpoint{4.650000in}{2.310000in}}%
\pgfusepath{clip}%
\pgfsetbuttcap%
\pgfsetroundjoin%
\pgfsetlinewidth{0.803000pt}%
\definecolor{currentstroke}{rgb}{0.686275,0.352941,0.313725}%
\pgfsetstrokecolor{currentstroke}%
\pgfsetdash{}{0pt}%
\pgfpathmoveto{\pgfqpoint{2.808269in}{0.635638in}}%
\pgfpathcurveto{\pgfqpoint{2.813794in}{0.635638in}}{\pgfqpoint{2.819094in}{0.637833in}}{\pgfqpoint{2.823000in}{0.641740in}}%
\pgfpathcurveto{\pgfqpoint{2.826907in}{0.645647in}}{\pgfqpoint{2.829102in}{0.650946in}}{\pgfqpoint{2.829102in}{0.656471in}}%
\pgfpathcurveto{\pgfqpoint{2.829102in}{0.661997in}}{\pgfqpoint{2.826907in}{0.667296in}}{\pgfqpoint{2.823000in}{0.671203in}}%
\pgfpathcurveto{\pgfqpoint{2.819094in}{0.675110in}}{\pgfqpoint{2.813794in}{0.677305in}}{\pgfqpoint{2.808269in}{0.677305in}}%
\pgfpathcurveto{\pgfqpoint{2.802744in}{0.677305in}}{\pgfqpoint{2.797444in}{0.675110in}}{\pgfqpoint{2.793538in}{0.671203in}}%
\pgfpathcurveto{\pgfqpoint{2.789631in}{0.667296in}}{\pgfqpoint{2.787436in}{0.661997in}}{\pgfqpoint{2.787436in}{0.656471in}}%
\pgfpathcurveto{\pgfqpoint{2.787436in}{0.650946in}}{\pgfqpoint{2.789631in}{0.645647in}}{\pgfqpoint{2.793538in}{0.641740in}}%
\pgfpathcurveto{\pgfqpoint{2.797444in}{0.637833in}}{\pgfqpoint{2.802744in}{0.635638in}}{\pgfqpoint{2.808269in}{0.635638in}}%
\pgfpathclose%
\pgfusepath{stroke}%
\end{pgfscope}%
\begin{pgfscope}%
\pgfpathrectangle{\pgfqpoint{0.438556in}{0.383578in}}{\pgfqpoint{4.650000in}{2.310000in}}%
\pgfusepath{clip}%
\pgfsetbuttcap%
\pgfsetroundjoin%
\pgfsetlinewidth{0.803000pt}%
\definecolor{currentstroke}{rgb}{0.686275,0.352941,0.313725}%
\pgfsetstrokecolor{currentstroke}%
\pgfsetdash{}{0pt}%
\pgfpathmoveto{\pgfqpoint{2.878509in}{0.679322in}}%
\pgfpathcurveto{\pgfqpoint{2.884034in}{0.679322in}}{\pgfqpoint{2.889334in}{0.681518in}}{\pgfqpoint{2.893241in}{0.685424in}}%
\pgfpathcurveto{\pgfqpoint{2.897147in}{0.689331in}}{\pgfqpoint{2.899343in}{0.694631in}}{\pgfqpoint{2.899343in}{0.700156in}}%
\pgfpathcurveto{\pgfqpoint{2.899343in}{0.705681in}}{\pgfqpoint{2.897147in}{0.710980in}}{\pgfqpoint{2.893241in}{0.714887in}}%
\pgfpathcurveto{\pgfqpoint{2.889334in}{0.718794in}}{\pgfqpoint{2.884034in}{0.720989in}}{\pgfqpoint{2.878509in}{0.720989in}}%
\pgfpathcurveto{\pgfqpoint{2.872984in}{0.720989in}}{\pgfqpoint{2.867685in}{0.718794in}}{\pgfqpoint{2.863778in}{0.714887in}}%
\pgfpathcurveto{\pgfqpoint{2.859871in}{0.710980in}}{\pgfqpoint{2.857676in}{0.705681in}}{\pgfqpoint{2.857676in}{0.700156in}}%
\pgfpathcurveto{\pgfqpoint{2.857676in}{0.694631in}}{\pgfqpoint{2.859871in}{0.689331in}}{\pgfqpoint{2.863778in}{0.685424in}}%
\pgfpathcurveto{\pgfqpoint{2.867685in}{0.681518in}}{\pgfqpoint{2.872984in}{0.679322in}}{\pgfqpoint{2.878509in}{0.679322in}}%
\pgfpathclose%
\pgfusepath{stroke}%
\end{pgfscope}%
\begin{pgfscope}%
\pgfpathrectangle{\pgfqpoint{0.438556in}{0.383578in}}{\pgfqpoint{4.650000in}{2.310000in}}%
\pgfusepath{clip}%
\pgfsetbuttcap%
\pgfsetroundjoin%
\pgfsetlinewidth{0.803000pt}%
\definecolor{currentstroke}{rgb}{0.686275,0.352941,0.313725}%
\pgfsetstrokecolor{currentstroke}%
\pgfsetdash{}{0pt}%
\pgfpathmoveto{\pgfqpoint{2.838276in}{0.701165in}}%
\pgfpathcurveto{\pgfqpoint{2.843801in}{0.701165in}}{\pgfqpoint{2.849101in}{0.703360in}}{\pgfqpoint{2.853008in}{0.707266in}}%
\pgfpathcurveto{\pgfqpoint{2.856915in}{0.711173in}}{\pgfqpoint{2.859110in}{0.716473in}}{\pgfqpoint{2.859110in}{0.721998in}}%
\pgfpathcurveto{\pgfqpoint{2.859110in}{0.727523in}}{\pgfqpoint{2.856915in}{0.732822in}}{\pgfqpoint{2.853008in}{0.736729in}}%
\pgfpathcurveto{\pgfqpoint{2.849101in}{0.740636in}}{\pgfqpoint{2.843801in}{0.742831in}}{\pgfqpoint{2.838276in}{0.742831in}}%
\pgfpathcurveto{\pgfqpoint{2.832751in}{0.742831in}}{\pgfqpoint{2.827452in}{0.740636in}}{\pgfqpoint{2.823545in}{0.736729in}}%
\pgfpathcurveto{\pgfqpoint{2.819638in}{0.732822in}}{\pgfqpoint{2.817443in}{0.727523in}}{\pgfqpoint{2.817443in}{0.721998in}}%
\pgfpathcurveto{\pgfqpoint{2.817443in}{0.716473in}}{\pgfqpoint{2.819638in}{0.711173in}}{\pgfqpoint{2.823545in}{0.707266in}}%
\pgfpathcurveto{\pgfqpoint{2.827452in}{0.703360in}}{\pgfqpoint{2.832751in}{0.701165in}}{\pgfqpoint{2.838276in}{0.701165in}}%
\pgfpathclose%
\pgfusepath{stroke}%
\end{pgfscope}%
\begin{pgfscope}%
\pgfpathrectangle{\pgfqpoint{0.438556in}{0.383578in}}{\pgfqpoint{4.650000in}{2.310000in}}%
\pgfusepath{clip}%
\pgfsetbuttcap%
\pgfsetroundjoin%
\pgfsetlinewidth{0.803000pt}%
\definecolor{currentstroke}{rgb}{0.686275,0.352941,0.313725}%
\pgfsetstrokecolor{currentstroke}%
\pgfsetdash{}{0pt}%
\pgfpathmoveto{\pgfqpoint{2.885262in}{0.810375in}}%
\pgfpathcurveto{\pgfqpoint{2.890787in}{0.810375in}}{\pgfqpoint{2.896087in}{0.812570in}}{\pgfqpoint{2.899993in}{0.816477in}}%
\pgfpathcurveto{\pgfqpoint{2.903900in}{0.820384in}}{\pgfqpoint{2.906095in}{0.825683in}}{\pgfqpoint{2.906095in}{0.831209in}}%
\pgfpathcurveto{\pgfqpoint{2.906095in}{0.836734in}}{\pgfqpoint{2.903900in}{0.842033in}}{\pgfqpoint{2.899993in}{0.845940in}}%
\pgfpathcurveto{\pgfqpoint{2.896087in}{0.849847in}}{\pgfqpoint{2.890787in}{0.852042in}}{\pgfqpoint{2.885262in}{0.852042in}}%
\pgfpathcurveto{\pgfqpoint{2.879737in}{0.852042in}}{\pgfqpoint{2.874437in}{0.849847in}}{\pgfqpoint{2.870531in}{0.845940in}}%
\pgfpathcurveto{\pgfqpoint{2.866624in}{0.842033in}}{\pgfqpoint{2.864429in}{0.836734in}}{\pgfqpoint{2.864429in}{0.831209in}}%
\pgfpathcurveto{\pgfqpoint{2.864429in}{0.825683in}}{\pgfqpoint{2.866624in}{0.820384in}}{\pgfqpoint{2.870531in}{0.816477in}}%
\pgfpathcurveto{\pgfqpoint{2.874437in}{0.812570in}}{\pgfqpoint{2.879737in}{0.810375in}}{\pgfqpoint{2.885262in}{0.810375in}}%
\pgfpathclose%
\pgfusepath{stroke}%
\end{pgfscope}%
\begin{pgfscope}%
\pgfpathrectangle{\pgfqpoint{0.438556in}{0.383578in}}{\pgfqpoint{4.650000in}{2.310000in}}%
\pgfusepath{clip}%
\pgfsetbuttcap%
\pgfsetroundjoin%
\pgfsetlinewidth{0.803000pt}%
\definecolor{currentstroke}{rgb}{0.686275,0.352941,0.313725}%
\pgfsetstrokecolor{currentstroke}%
\pgfsetdash{}{0pt}%
\pgfpathmoveto{\pgfqpoint{2.835280in}{0.854059in}}%
\pgfpathcurveto{\pgfqpoint{2.840805in}{0.854059in}}{\pgfqpoint{2.846105in}{0.856255in}}{\pgfqpoint{2.850012in}{0.860161in}}%
\pgfpathcurveto{\pgfqpoint{2.853918in}{0.864068in}}{\pgfqpoint{2.856114in}{0.869368in}}{\pgfqpoint{2.856114in}{0.874893in}}%
\pgfpathcurveto{\pgfqpoint{2.856114in}{0.880418in}}{\pgfqpoint{2.853918in}{0.885717in}}{\pgfqpoint{2.850012in}{0.889624in}}%
\pgfpathcurveto{\pgfqpoint{2.846105in}{0.893531in}}{\pgfqpoint{2.840805in}{0.895726in}}{\pgfqpoint{2.835280in}{0.895726in}}%
\pgfpathcurveto{\pgfqpoint{2.829755in}{0.895726in}}{\pgfqpoint{2.824456in}{0.893531in}}{\pgfqpoint{2.820549in}{0.889624in}}%
\pgfpathcurveto{\pgfqpoint{2.816642in}{0.885717in}}{\pgfqpoint{2.814447in}{0.880418in}}{\pgfqpoint{2.814447in}{0.874893in}}%
\pgfpathcurveto{\pgfqpoint{2.814447in}{0.869368in}}{\pgfqpoint{2.816642in}{0.864068in}}{\pgfqpoint{2.820549in}{0.860161in}}%
\pgfpathcurveto{\pgfqpoint{2.824456in}{0.856255in}}{\pgfqpoint{2.829755in}{0.854059in}}{\pgfqpoint{2.835280in}{0.854059in}}%
\pgfpathclose%
\pgfusepath{stroke}%
\end{pgfscope}%
\begin{pgfscope}%
\pgfpathrectangle{\pgfqpoint{0.438556in}{0.383578in}}{\pgfqpoint{4.650000in}{2.310000in}}%
\pgfusepath{clip}%
\pgfsetbuttcap%
\pgfsetroundjoin%
\pgfsetlinewidth{0.803000pt}%
\definecolor{currentstroke}{rgb}{0.686275,0.352941,0.313725}%
\pgfsetstrokecolor{currentstroke}%
\pgfsetdash{}{0pt}%
\pgfpathmoveto{\pgfqpoint{2.885757in}{0.875902in}}%
\pgfpathcurveto{\pgfqpoint{2.891282in}{0.875902in}}{\pgfqpoint{2.896581in}{0.878097in}}{\pgfqpoint{2.900488in}{0.882004in}}%
\pgfpathcurveto{\pgfqpoint{2.904395in}{0.885910in}}{\pgfqpoint{2.906590in}{0.891210in}}{\pgfqpoint{2.906590in}{0.896735in}}%
\pgfpathcurveto{\pgfqpoint{2.906590in}{0.902260in}}{\pgfqpoint{2.904395in}{0.907560in}}{\pgfqpoint{2.900488in}{0.911466in}}%
\pgfpathcurveto{\pgfqpoint{2.896581in}{0.915373in}}{\pgfqpoint{2.891282in}{0.917568in}}{\pgfqpoint{2.885757in}{0.917568in}}%
\pgfpathcurveto{\pgfqpoint{2.880232in}{0.917568in}}{\pgfqpoint{2.874932in}{0.915373in}}{\pgfqpoint{2.871025in}{0.911466in}}%
\pgfpathcurveto{\pgfqpoint{2.867119in}{0.907560in}}{\pgfqpoint{2.864923in}{0.902260in}}{\pgfqpoint{2.864923in}{0.896735in}}%
\pgfpathcurveto{\pgfqpoint{2.864923in}{0.891210in}}{\pgfqpoint{2.867119in}{0.885910in}}{\pgfqpoint{2.871025in}{0.882004in}}%
\pgfpathcurveto{\pgfqpoint{2.874932in}{0.878097in}}{\pgfqpoint{2.880232in}{0.875902in}}{\pgfqpoint{2.885757in}{0.875902in}}%
\pgfpathclose%
\pgfusepath{stroke}%
\end{pgfscope}%
\begin{pgfscope}%
\pgfpathrectangle{\pgfqpoint{0.438556in}{0.383578in}}{\pgfqpoint{4.650000in}{2.310000in}}%
\pgfusepath{clip}%
\pgfsetbuttcap%
\pgfsetroundjoin%
\pgfsetlinewidth{0.803000pt}%
\definecolor{currentstroke}{rgb}{0.686275,0.352941,0.313725}%
\pgfsetstrokecolor{currentstroke}%
\pgfsetdash{}{0pt}%
\pgfpathmoveto{\pgfqpoint{2.808296in}{0.919586in}}%
\pgfpathcurveto{\pgfqpoint{2.813822in}{0.919586in}}{\pgfqpoint{2.819121in}{0.921781in}}{\pgfqpoint{2.823028in}{0.925688in}}%
\pgfpathcurveto{\pgfqpoint{2.826935in}{0.929595in}}{\pgfqpoint{2.829130in}{0.934894in}}{\pgfqpoint{2.829130in}{0.940419in}}%
\pgfpathcurveto{\pgfqpoint{2.829130in}{0.945944in}}{\pgfqpoint{2.826935in}{0.951244in}}{\pgfqpoint{2.823028in}{0.955151in}}%
\pgfpathcurveto{\pgfqpoint{2.819121in}{0.959057in}}{\pgfqpoint{2.813822in}{0.961253in}}{\pgfqpoint{2.808296in}{0.961253in}}%
\pgfpathcurveto{\pgfqpoint{2.802771in}{0.961253in}}{\pgfqpoint{2.797472in}{0.959057in}}{\pgfqpoint{2.793565in}{0.955151in}}%
\pgfpathcurveto{\pgfqpoint{2.789658in}{0.951244in}}{\pgfqpoint{2.787463in}{0.945944in}}{\pgfqpoint{2.787463in}{0.940419in}}%
\pgfpathcurveto{\pgfqpoint{2.787463in}{0.934894in}}{\pgfqpoint{2.789658in}{0.929595in}}{\pgfqpoint{2.793565in}{0.925688in}}%
\pgfpathcurveto{\pgfqpoint{2.797472in}{0.921781in}}{\pgfqpoint{2.802771in}{0.919586in}}{\pgfqpoint{2.808296in}{0.919586in}}%
\pgfpathclose%
\pgfusepath{stroke}%
\end{pgfscope}%
\begin{pgfscope}%
\pgfpathrectangle{\pgfqpoint{0.438556in}{0.383578in}}{\pgfqpoint{4.650000in}{2.310000in}}%
\pgfusepath{clip}%
\pgfsetbuttcap%
\pgfsetroundjoin%
\pgfsetlinewidth{0.803000pt}%
\definecolor{currentstroke}{rgb}{0.686275,0.352941,0.313725}%
\pgfsetstrokecolor{currentstroke}%
\pgfsetdash{}{0pt}%
\pgfpathmoveto{\pgfqpoint{2.813748in}{0.941428in}}%
\pgfpathcurveto{\pgfqpoint{2.819273in}{0.941428in}}{\pgfqpoint{2.824573in}{0.943623in}}{\pgfqpoint{2.828480in}{0.947530in}}%
\pgfpathcurveto{\pgfqpoint{2.832386in}{0.951437in}}{\pgfqpoint{2.834582in}{0.956736in}}{\pgfqpoint{2.834582in}{0.962261in}}%
\pgfpathcurveto{\pgfqpoint{2.834582in}{0.967786in}}{\pgfqpoint{2.832386in}{0.973086in}}{\pgfqpoint{2.828480in}{0.976993in}}%
\pgfpathcurveto{\pgfqpoint{2.824573in}{0.980900in}}{\pgfqpoint{2.819273in}{0.983095in}}{\pgfqpoint{2.813748in}{0.983095in}}%
\pgfpathcurveto{\pgfqpoint{2.808223in}{0.983095in}}{\pgfqpoint{2.802924in}{0.980900in}}{\pgfqpoint{2.799017in}{0.976993in}}%
\pgfpathcurveto{\pgfqpoint{2.795110in}{0.973086in}}{\pgfqpoint{2.792915in}{0.967786in}}{\pgfqpoint{2.792915in}{0.962261in}}%
\pgfpathcurveto{\pgfqpoint{2.792915in}{0.956736in}}{\pgfqpoint{2.795110in}{0.951437in}}{\pgfqpoint{2.799017in}{0.947530in}}%
\pgfpathcurveto{\pgfqpoint{2.802924in}{0.943623in}}{\pgfqpoint{2.808223in}{0.941428in}}{\pgfqpoint{2.813748in}{0.941428in}}%
\pgfpathclose%
\pgfusepath{stroke}%
\end{pgfscope}%
\begin{pgfscope}%
\pgfpathrectangle{\pgfqpoint{0.438556in}{0.383578in}}{\pgfqpoint{4.650000in}{2.310000in}}%
\pgfusepath{clip}%
\pgfsetbuttcap%
\pgfsetroundjoin%
\pgfsetlinewidth{0.803000pt}%
\definecolor{currentstroke}{rgb}{0.686275,0.352941,0.313725}%
\pgfsetstrokecolor{currentstroke}%
\pgfsetdash{}{0pt}%
\pgfpathmoveto{\pgfqpoint{2.864802in}{1.006954in}}%
\pgfpathcurveto{\pgfqpoint{2.870327in}{1.006954in}}{\pgfqpoint{2.875627in}{1.009150in}}{\pgfqpoint{2.879533in}{1.013056in}}%
\pgfpathcurveto{\pgfqpoint{2.883440in}{1.016963in}}{\pgfqpoint{2.885635in}{1.022263in}}{\pgfqpoint{2.885635in}{1.027788in}}%
\pgfpathcurveto{\pgfqpoint{2.885635in}{1.033313in}}{\pgfqpoint{2.883440in}{1.038612in}}{\pgfqpoint{2.879533in}{1.042519in}}%
\pgfpathcurveto{\pgfqpoint{2.875627in}{1.046426in}}{\pgfqpoint{2.870327in}{1.048621in}}{\pgfqpoint{2.864802in}{1.048621in}}%
\pgfpathcurveto{\pgfqpoint{2.859277in}{1.048621in}}{\pgfqpoint{2.853977in}{1.046426in}}{\pgfqpoint{2.850071in}{1.042519in}}%
\pgfpathcurveto{\pgfqpoint{2.846164in}{1.038612in}}{\pgfqpoint{2.843969in}{1.033313in}}{\pgfqpoint{2.843969in}{1.027788in}}%
\pgfpathcurveto{\pgfqpoint{2.843969in}{1.022263in}}{\pgfqpoint{2.846164in}{1.016963in}}{\pgfqpoint{2.850071in}{1.013056in}}%
\pgfpathcurveto{\pgfqpoint{2.853977in}{1.009150in}}{\pgfqpoint{2.859277in}{1.006954in}}{\pgfqpoint{2.864802in}{1.006954in}}%
\pgfpathclose%
\pgfusepath{stroke}%
\end{pgfscope}%
\begin{pgfscope}%
\pgfpathrectangle{\pgfqpoint{0.438556in}{0.383578in}}{\pgfqpoint{4.650000in}{2.310000in}}%
\pgfusepath{clip}%
\pgfsetbuttcap%
\pgfsetroundjoin%
\pgfsetlinewidth{0.803000pt}%
\definecolor{currentstroke}{rgb}{0.686275,0.352941,0.313725}%
\pgfsetstrokecolor{currentstroke}%
\pgfsetdash{}{0pt}%
\pgfpathmoveto{\pgfqpoint{2.829874in}{1.028797in}}%
\pgfpathcurveto{\pgfqpoint{2.835399in}{1.028797in}}{\pgfqpoint{2.840699in}{1.030992in}}{\pgfqpoint{2.844606in}{1.034898in}}%
\pgfpathcurveto{\pgfqpoint{2.848513in}{1.038805in}}{\pgfqpoint{2.850708in}{1.044105in}}{\pgfqpoint{2.850708in}{1.049630in}}%
\pgfpathcurveto{\pgfqpoint{2.850708in}{1.055155in}}{\pgfqpoint{2.848513in}{1.060454in}}{\pgfqpoint{2.844606in}{1.064361in}}%
\pgfpathcurveto{\pgfqpoint{2.840699in}{1.068268in}}{\pgfqpoint{2.835399in}{1.070463in}}{\pgfqpoint{2.829874in}{1.070463in}}%
\pgfpathcurveto{\pgfqpoint{2.824349in}{1.070463in}}{\pgfqpoint{2.819050in}{1.068268in}}{\pgfqpoint{2.815143in}{1.064361in}}%
\pgfpathcurveto{\pgfqpoint{2.811236in}{1.060454in}}{\pgfqpoint{2.809041in}{1.055155in}}{\pgfqpoint{2.809041in}{1.049630in}}%
\pgfpathcurveto{\pgfqpoint{2.809041in}{1.044105in}}{\pgfqpoint{2.811236in}{1.038805in}}{\pgfqpoint{2.815143in}{1.034898in}}%
\pgfpathcurveto{\pgfqpoint{2.819050in}{1.030992in}}{\pgfqpoint{2.824349in}{1.028797in}}{\pgfqpoint{2.829874in}{1.028797in}}%
\pgfpathclose%
\pgfusepath{stroke}%
\end{pgfscope}%
\begin{pgfscope}%
\pgfpathrectangle{\pgfqpoint{0.438556in}{0.383578in}}{\pgfqpoint{4.650000in}{2.310000in}}%
\pgfusepath{clip}%
\pgfsetbuttcap%
\pgfsetroundjoin%
\pgfsetlinewidth{0.803000pt}%
\definecolor{currentstroke}{rgb}{0.686275,0.352941,0.313725}%
\pgfsetstrokecolor{currentstroke}%
\pgfsetdash{}{0pt}%
\pgfpathmoveto{\pgfqpoint{2.831469in}{1.072481in}}%
\pgfpathcurveto{\pgfqpoint{2.836994in}{1.072481in}}{\pgfqpoint{2.842293in}{1.074676in}}{\pgfqpoint{2.846200in}{1.078583in}}%
\pgfpathcurveto{\pgfqpoint{2.850107in}{1.082490in}}{\pgfqpoint{2.852302in}{1.087789in}}{\pgfqpoint{2.852302in}{1.093314in}}%
\pgfpathcurveto{\pgfqpoint{2.852302in}{1.098839in}}{\pgfqpoint{2.850107in}{1.104139in}}{\pgfqpoint{2.846200in}{1.108045in}}%
\pgfpathcurveto{\pgfqpoint{2.842293in}{1.111952in}}{\pgfqpoint{2.836994in}{1.114147in}}{\pgfqpoint{2.831469in}{1.114147in}}%
\pgfpathcurveto{\pgfqpoint{2.825944in}{1.114147in}}{\pgfqpoint{2.820644in}{1.111952in}}{\pgfqpoint{2.816737in}{1.108045in}}%
\pgfpathcurveto{\pgfqpoint{2.812830in}{1.104139in}}{\pgfqpoint{2.810635in}{1.098839in}}{\pgfqpoint{2.810635in}{1.093314in}}%
\pgfpathcurveto{\pgfqpoint{2.810635in}{1.087789in}}{\pgfqpoint{2.812830in}{1.082490in}}{\pgfqpoint{2.816737in}{1.078583in}}%
\pgfpathcurveto{\pgfqpoint{2.820644in}{1.074676in}}{\pgfqpoint{2.825944in}{1.072481in}}{\pgfqpoint{2.831469in}{1.072481in}}%
\pgfpathclose%
\pgfusepath{stroke}%
\end{pgfscope}%
\begin{pgfscope}%
\pgfpathrectangle{\pgfqpoint{0.438556in}{0.383578in}}{\pgfqpoint{4.650000in}{2.310000in}}%
\pgfusepath{clip}%
\pgfsetbuttcap%
\pgfsetroundjoin%
\pgfsetlinewidth{0.803000pt}%
\definecolor{currentstroke}{rgb}{0.686275,0.352941,0.313725}%
\pgfsetstrokecolor{currentstroke}%
\pgfsetdash{}{0pt}%
\pgfpathmoveto{\pgfqpoint{2.880562in}{1.159849in}}%
\pgfpathcurveto{\pgfqpoint{2.886087in}{1.159849in}}{\pgfqpoint{2.891386in}{1.162044in}}{\pgfqpoint{2.895293in}{1.165951in}}%
\pgfpathcurveto{\pgfqpoint{2.899200in}{1.169858in}}{\pgfqpoint{2.901395in}{1.175158in}}{\pgfqpoint{2.901395in}{1.180683in}}%
\pgfpathcurveto{\pgfqpoint{2.901395in}{1.186208in}}{\pgfqpoint{2.899200in}{1.191507in}}{\pgfqpoint{2.895293in}{1.195414in}}%
\pgfpathcurveto{\pgfqpoint{2.891386in}{1.199321in}}{\pgfqpoint{2.886087in}{1.201516in}}{\pgfqpoint{2.880562in}{1.201516in}}%
\pgfpathcurveto{\pgfqpoint{2.875037in}{1.201516in}}{\pgfqpoint{2.869737in}{1.199321in}}{\pgfqpoint{2.865830in}{1.195414in}}%
\pgfpathcurveto{\pgfqpoint{2.861923in}{1.191507in}}{\pgfqpoint{2.859728in}{1.186208in}}{\pgfqpoint{2.859728in}{1.180683in}}%
\pgfpathcurveto{\pgfqpoint{2.859728in}{1.175158in}}{\pgfqpoint{2.861923in}{1.169858in}}{\pgfqpoint{2.865830in}{1.165951in}}%
\pgfpathcurveto{\pgfqpoint{2.869737in}{1.162044in}}{\pgfqpoint{2.875037in}{1.159849in}}{\pgfqpoint{2.880562in}{1.159849in}}%
\pgfpathclose%
\pgfusepath{stroke}%
\end{pgfscope}%
\begin{pgfscope}%
\pgfpathrectangle{\pgfqpoint{0.438556in}{0.383578in}}{\pgfqpoint{4.650000in}{2.310000in}}%
\pgfusepath{clip}%
\pgfsetbuttcap%
\pgfsetroundjoin%
\pgfsetlinewidth{0.803000pt}%
\definecolor{currentstroke}{rgb}{0.686275,0.352941,0.313725}%
\pgfsetstrokecolor{currentstroke}%
\pgfsetdash{}{0pt}%
\pgfpathmoveto{\pgfqpoint{2.826979in}{1.203534in}}%
\pgfpathcurveto{\pgfqpoint{2.832504in}{1.203534in}}{\pgfqpoint{2.837804in}{1.205729in}}{\pgfqpoint{2.841710in}{1.209636in}}%
\pgfpathcurveto{\pgfqpoint{2.845617in}{1.213542in}}{\pgfqpoint{2.847812in}{1.218842in}}{\pgfqpoint{2.847812in}{1.224367in}}%
\pgfpathcurveto{\pgfqpoint{2.847812in}{1.229892in}}{\pgfqpoint{2.845617in}{1.235191in}}{\pgfqpoint{2.841710in}{1.239098in}}%
\pgfpathcurveto{\pgfqpoint{2.837804in}{1.243005in}}{\pgfqpoint{2.832504in}{1.245200in}}{\pgfqpoint{2.826979in}{1.245200in}}%
\pgfpathcurveto{\pgfqpoint{2.821454in}{1.245200in}}{\pgfqpoint{2.816154in}{1.243005in}}{\pgfqpoint{2.812248in}{1.239098in}}%
\pgfpathcurveto{\pgfqpoint{2.808341in}{1.235191in}}{\pgfqpoint{2.806146in}{1.229892in}}{\pgfqpoint{2.806146in}{1.224367in}}%
\pgfpathcurveto{\pgfqpoint{2.806146in}{1.218842in}}{\pgfqpoint{2.808341in}{1.213542in}}{\pgfqpoint{2.812248in}{1.209636in}}%
\pgfpathcurveto{\pgfqpoint{2.816154in}{1.205729in}}{\pgfqpoint{2.821454in}{1.203534in}}{\pgfqpoint{2.826979in}{1.203534in}}%
\pgfpathclose%
\pgfusepath{stroke}%
\end{pgfscope}%
\begin{pgfscope}%
\pgfpathrectangle{\pgfqpoint{0.438556in}{0.383578in}}{\pgfqpoint{4.650000in}{2.310000in}}%
\pgfusepath{clip}%
\pgfsetbuttcap%
\pgfsetroundjoin%
\pgfsetlinewidth{0.803000pt}%
\definecolor{currentstroke}{rgb}{0.686275,0.352941,0.313725}%
\pgfsetstrokecolor{currentstroke}%
\pgfsetdash{}{0pt}%
\pgfpathmoveto{\pgfqpoint{2.820822in}{1.290902in}}%
\pgfpathcurveto{\pgfqpoint{2.826347in}{1.290902in}}{\pgfqpoint{2.831646in}{1.293097in}}{\pgfqpoint{2.835553in}{1.297004in}}%
\pgfpathcurveto{\pgfqpoint{2.839460in}{1.300911in}}{\pgfqpoint{2.841655in}{1.306210in}}{\pgfqpoint{2.841655in}{1.311735in}}%
\pgfpathcurveto{\pgfqpoint{2.841655in}{1.317260in}}{\pgfqpoint{2.839460in}{1.322560in}}{\pgfqpoint{2.835553in}{1.326467in}}%
\pgfpathcurveto{\pgfqpoint{2.831646in}{1.330374in}}{\pgfqpoint{2.826347in}{1.332569in}}{\pgfqpoint{2.820822in}{1.332569in}}%
\pgfpathcurveto{\pgfqpoint{2.815297in}{1.332569in}}{\pgfqpoint{2.809997in}{1.330374in}}{\pgfqpoint{2.806090in}{1.326467in}}%
\pgfpathcurveto{\pgfqpoint{2.802184in}{1.322560in}}{\pgfqpoint{2.799988in}{1.317260in}}{\pgfqpoint{2.799988in}{1.311735in}}%
\pgfpathcurveto{\pgfqpoint{2.799988in}{1.306210in}}{\pgfqpoint{2.802184in}{1.300911in}}{\pgfqpoint{2.806090in}{1.297004in}}%
\pgfpathcurveto{\pgfqpoint{2.809997in}{1.293097in}}{\pgfqpoint{2.815297in}{1.290902in}}{\pgfqpoint{2.820822in}{1.290902in}}%
\pgfpathclose%
\pgfusepath{stroke}%
\end{pgfscope}%
\begin{pgfscope}%
\pgfpathrectangle{\pgfqpoint{0.438556in}{0.383578in}}{\pgfqpoint{4.650000in}{2.310000in}}%
\pgfusepath{clip}%
\pgfsetbuttcap%
\pgfsetroundjoin%
\pgfsetlinewidth{0.803000pt}%
\definecolor{currentstroke}{rgb}{0.686275,0.352941,0.313725}%
\pgfsetstrokecolor{currentstroke}%
\pgfsetdash{}{0pt}%
\pgfpathmoveto{\pgfqpoint{2.802057in}{1.312744in}}%
\pgfpathcurveto{\pgfqpoint{2.807582in}{1.312744in}}{\pgfqpoint{2.812881in}{1.314939in}}{\pgfqpoint{2.816788in}{1.318846in}}%
\pgfpathcurveto{\pgfqpoint{2.820695in}{1.322753in}}{\pgfqpoint{2.822890in}{1.328052in}}{\pgfqpoint{2.822890in}{1.333578in}}%
\pgfpathcurveto{\pgfqpoint{2.822890in}{1.339103in}}{\pgfqpoint{2.820695in}{1.344402in}}{\pgfqpoint{2.816788in}{1.348309in}}%
\pgfpathcurveto{\pgfqpoint{2.812881in}{1.352216in}}{\pgfqpoint{2.807582in}{1.354411in}}{\pgfqpoint{2.802057in}{1.354411in}}%
\pgfpathcurveto{\pgfqpoint{2.796532in}{1.354411in}}{\pgfqpoint{2.791232in}{1.352216in}}{\pgfqpoint{2.787325in}{1.348309in}}%
\pgfpathcurveto{\pgfqpoint{2.783419in}{1.344402in}}{\pgfqpoint{2.781223in}{1.339103in}}{\pgfqpoint{2.781223in}{1.333578in}}%
\pgfpathcurveto{\pgfqpoint{2.781223in}{1.328052in}}{\pgfqpoint{2.783419in}{1.322753in}}{\pgfqpoint{2.787325in}{1.318846in}}%
\pgfpathcurveto{\pgfqpoint{2.791232in}{1.314939in}}{\pgfqpoint{2.796532in}{1.312744in}}{\pgfqpoint{2.802057in}{1.312744in}}%
\pgfpathclose%
\pgfusepath{stroke}%
\end{pgfscope}%
\begin{pgfscope}%
\pgfpathrectangle{\pgfqpoint{0.438556in}{0.383578in}}{\pgfqpoint{4.650000in}{2.310000in}}%
\pgfusepath{clip}%
\pgfsetbuttcap%
\pgfsetroundjoin%
\pgfsetlinewidth{0.803000pt}%
\definecolor{currentstroke}{rgb}{0.686275,0.352941,0.313725}%
\pgfsetstrokecolor{currentstroke}%
\pgfsetdash{}{0pt}%
\pgfpathmoveto{\pgfqpoint{2.874936in}{1.334586in}}%
\pgfpathcurveto{\pgfqpoint{2.880461in}{1.334586in}}{\pgfqpoint{2.885760in}{1.336781in}}{\pgfqpoint{2.889667in}{1.340688in}}%
\pgfpathcurveto{\pgfqpoint{2.893574in}{1.344595in}}{\pgfqpoint{2.895769in}{1.349895in}}{\pgfqpoint{2.895769in}{1.355420in}}%
\pgfpathcurveto{\pgfqpoint{2.895769in}{1.360945in}}{\pgfqpoint{2.893574in}{1.366244in}}{\pgfqpoint{2.889667in}{1.370151in}}%
\pgfpathcurveto{\pgfqpoint{2.885760in}{1.374058in}}{\pgfqpoint{2.880461in}{1.376253in}}{\pgfqpoint{2.874936in}{1.376253in}}%
\pgfpathcurveto{\pgfqpoint{2.869411in}{1.376253in}}{\pgfqpoint{2.864111in}{1.374058in}}{\pgfqpoint{2.860204in}{1.370151in}}%
\pgfpathcurveto{\pgfqpoint{2.856298in}{1.366244in}}{\pgfqpoint{2.854102in}{1.360945in}}{\pgfqpoint{2.854102in}{1.355420in}}%
\pgfpathcurveto{\pgfqpoint{2.854102in}{1.349895in}}{\pgfqpoint{2.856298in}{1.344595in}}{\pgfqpoint{2.860204in}{1.340688in}}%
\pgfpathcurveto{\pgfqpoint{2.864111in}{1.336781in}}{\pgfqpoint{2.869411in}{1.334586in}}{\pgfqpoint{2.874936in}{1.334586in}}%
\pgfpathclose%
\pgfusepath{stroke}%
\end{pgfscope}%
\begin{pgfscope}%
\pgfpathrectangle{\pgfqpoint{0.438556in}{0.383578in}}{\pgfqpoint{4.650000in}{2.310000in}}%
\pgfusepath{clip}%
\pgfsetbuttcap%
\pgfsetroundjoin%
\pgfsetlinewidth{0.803000pt}%
\definecolor{currentstroke}{rgb}{0.686275,0.352941,0.313725}%
\pgfsetstrokecolor{currentstroke}%
\pgfsetdash{}{0pt}%
\pgfpathmoveto{\pgfqpoint{2.823231in}{1.443797in}}%
\pgfpathcurveto{\pgfqpoint{2.828757in}{1.443797in}}{\pgfqpoint{2.834056in}{1.445992in}}{\pgfqpoint{2.837963in}{1.449899in}}%
\pgfpathcurveto{\pgfqpoint{2.841870in}{1.453806in}}{\pgfqpoint{2.844065in}{1.459105in}}{\pgfqpoint{2.844065in}{1.464630in}}%
\pgfpathcurveto{\pgfqpoint{2.844065in}{1.470155in}}{\pgfqpoint{2.841870in}{1.475455in}}{\pgfqpoint{2.837963in}{1.479362in}}%
\pgfpathcurveto{\pgfqpoint{2.834056in}{1.483269in}}{\pgfqpoint{2.828757in}{1.485464in}}{\pgfqpoint{2.823231in}{1.485464in}}%
\pgfpathcurveto{\pgfqpoint{2.817706in}{1.485464in}}{\pgfqpoint{2.812407in}{1.483269in}}{\pgfqpoint{2.808500in}{1.479362in}}%
\pgfpathcurveto{\pgfqpoint{2.804593in}{1.475455in}}{\pgfqpoint{2.802398in}{1.470155in}}{\pgfqpoint{2.802398in}{1.464630in}}%
\pgfpathcurveto{\pgfqpoint{2.802398in}{1.459105in}}{\pgfqpoint{2.804593in}{1.453806in}}{\pgfqpoint{2.808500in}{1.449899in}}%
\pgfpathcurveto{\pgfqpoint{2.812407in}{1.445992in}}{\pgfqpoint{2.817706in}{1.443797in}}{\pgfqpoint{2.823231in}{1.443797in}}%
\pgfpathclose%
\pgfusepath{stroke}%
\end{pgfscope}%
\begin{pgfscope}%
\pgfpathrectangle{\pgfqpoint{0.438556in}{0.383578in}}{\pgfqpoint{4.650000in}{2.310000in}}%
\pgfusepath{clip}%
\pgfsetbuttcap%
\pgfsetroundjoin%
\pgfsetlinewidth{0.803000pt}%
\definecolor{currentstroke}{rgb}{0.686275,0.352941,0.313725}%
\pgfsetstrokecolor{currentstroke}%
\pgfsetdash{}{0pt}%
\pgfpathmoveto{\pgfqpoint{2.802332in}{1.487481in}}%
\pgfpathcurveto{\pgfqpoint{2.807857in}{1.487481in}}{\pgfqpoint{2.813156in}{1.489676in}}{\pgfqpoint{2.817063in}{1.493583in}}%
\pgfpathcurveto{\pgfqpoint{2.820970in}{1.497490in}}{\pgfqpoint{2.823165in}{1.502790in}}{\pgfqpoint{2.823165in}{1.508315in}}%
\pgfpathcurveto{\pgfqpoint{2.823165in}{1.513840in}}{\pgfqpoint{2.820970in}{1.519139in}}{\pgfqpoint{2.817063in}{1.523046in}}%
\pgfpathcurveto{\pgfqpoint{2.813156in}{1.526953in}}{\pgfqpoint{2.807857in}{1.529148in}}{\pgfqpoint{2.802332in}{1.529148in}}%
\pgfpathcurveto{\pgfqpoint{2.796807in}{1.529148in}}{\pgfqpoint{2.791507in}{1.526953in}}{\pgfqpoint{2.787600in}{1.523046in}}%
\pgfpathcurveto{\pgfqpoint{2.783693in}{1.519139in}}{\pgfqpoint{2.781498in}{1.513840in}}{\pgfqpoint{2.781498in}{1.508315in}}%
\pgfpathcurveto{\pgfqpoint{2.781498in}{1.502790in}}{\pgfqpoint{2.783693in}{1.497490in}}{\pgfqpoint{2.787600in}{1.493583in}}%
\pgfpathcurveto{\pgfqpoint{2.791507in}{1.489676in}}{\pgfqpoint{2.796807in}{1.487481in}}{\pgfqpoint{2.802332in}{1.487481in}}%
\pgfpathclose%
\pgfusepath{stroke}%
\end{pgfscope}%
\begin{pgfscope}%
\pgfpathrectangle{\pgfqpoint{0.438556in}{0.383578in}}{\pgfqpoint{4.650000in}{2.310000in}}%
\pgfusepath{clip}%
\pgfsetbuttcap%
\pgfsetroundjoin%
\pgfsetlinewidth{0.803000pt}%
\definecolor{currentstroke}{rgb}{0.686275,0.352941,0.313725}%
\pgfsetstrokecolor{currentstroke}%
\pgfsetdash{}{0pt}%
\pgfpathmoveto{\pgfqpoint{2.815901in}{1.531166in}}%
\pgfpathcurveto{\pgfqpoint{2.821426in}{1.531166in}}{\pgfqpoint{2.826726in}{1.533361in}}{\pgfqpoint{2.830633in}{1.537267in}}%
\pgfpathcurveto{\pgfqpoint{2.834540in}{1.541174in}}{\pgfqpoint{2.836735in}{1.546474in}}{\pgfqpoint{2.836735in}{1.551999in}}%
\pgfpathcurveto{\pgfqpoint{2.836735in}{1.557524in}}{\pgfqpoint{2.834540in}{1.562823in}}{\pgfqpoint{2.830633in}{1.566730in}}%
\pgfpathcurveto{\pgfqpoint{2.826726in}{1.570637in}}{\pgfqpoint{2.821426in}{1.572832in}}{\pgfqpoint{2.815901in}{1.572832in}}%
\pgfpathcurveto{\pgfqpoint{2.810376in}{1.572832in}}{\pgfqpoint{2.805077in}{1.570637in}}{\pgfqpoint{2.801170in}{1.566730in}}%
\pgfpathcurveto{\pgfqpoint{2.797263in}{1.562823in}}{\pgfqpoint{2.795068in}{1.557524in}}{\pgfqpoint{2.795068in}{1.551999in}}%
\pgfpathcurveto{\pgfqpoint{2.795068in}{1.546474in}}{\pgfqpoint{2.797263in}{1.541174in}}{\pgfqpoint{2.801170in}{1.537267in}}%
\pgfpathcurveto{\pgfqpoint{2.805077in}{1.533361in}}{\pgfqpoint{2.810376in}{1.531166in}}{\pgfqpoint{2.815901in}{1.531166in}}%
\pgfpathclose%
\pgfusepath{stroke}%
\end{pgfscope}%
\begin{pgfscope}%
\pgfpathrectangle{\pgfqpoint{0.438556in}{0.383578in}}{\pgfqpoint{4.650000in}{2.310000in}}%
\pgfusepath{clip}%
\pgfsetbuttcap%
\pgfsetroundjoin%
\pgfsetlinewidth{0.803000pt}%
\definecolor{currentstroke}{rgb}{0.686275,0.352941,0.313725}%
\pgfsetstrokecolor{currentstroke}%
\pgfsetdash{}{0pt}%
\pgfpathmoveto{\pgfqpoint{2.868284in}{1.684060in}}%
\pgfpathcurveto{\pgfqpoint{2.873809in}{1.684060in}}{\pgfqpoint{2.879108in}{1.686256in}}{\pgfqpoint{2.883015in}{1.690162in}}%
\pgfpathcurveto{\pgfqpoint{2.886922in}{1.694069in}}{\pgfqpoint{2.889117in}{1.699369in}}{\pgfqpoint{2.889117in}{1.704894in}}%
\pgfpathcurveto{\pgfqpoint{2.889117in}{1.710419in}}{\pgfqpoint{2.886922in}{1.715718in}}{\pgfqpoint{2.883015in}{1.719625in}}%
\pgfpathcurveto{\pgfqpoint{2.879108in}{1.723532in}}{\pgfqpoint{2.873809in}{1.725727in}}{\pgfqpoint{2.868284in}{1.725727in}}%
\pgfpathcurveto{\pgfqpoint{2.862759in}{1.725727in}}{\pgfqpoint{2.857459in}{1.723532in}}{\pgfqpoint{2.853552in}{1.719625in}}%
\pgfpathcurveto{\pgfqpoint{2.849646in}{1.715718in}}{\pgfqpoint{2.847450in}{1.710419in}}{\pgfqpoint{2.847450in}{1.704894in}}%
\pgfpathcurveto{\pgfqpoint{2.847450in}{1.699369in}}{\pgfqpoint{2.849646in}{1.694069in}}{\pgfqpoint{2.853552in}{1.690162in}}%
\pgfpathcurveto{\pgfqpoint{2.857459in}{1.686256in}}{\pgfqpoint{2.862759in}{1.684060in}}{\pgfqpoint{2.868284in}{1.684060in}}%
\pgfpathclose%
\pgfusepath{stroke}%
\end{pgfscope}%
\begin{pgfscope}%
\pgfpathrectangle{\pgfqpoint{0.438556in}{0.383578in}}{\pgfqpoint{4.650000in}{2.310000in}}%
\pgfusepath{clip}%
\pgfsetbuttcap%
\pgfsetroundjoin%
\pgfsetlinewidth{0.803000pt}%
\definecolor{currentstroke}{rgb}{0.686275,0.352941,0.313725}%
\pgfsetstrokecolor{currentstroke}%
\pgfsetdash{}{0pt}%
\pgfpathmoveto{\pgfqpoint{2.822865in}{1.771429in}}%
\pgfpathcurveto{\pgfqpoint{2.828390in}{1.771429in}}{\pgfqpoint{2.833690in}{1.773624in}}{\pgfqpoint{2.837596in}{1.777531in}}%
\pgfpathcurveto{\pgfqpoint{2.841503in}{1.781438in}}{\pgfqpoint{2.843698in}{1.786737in}}{\pgfqpoint{2.843698in}{1.792262in}}%
\pgfpathcurveto{\pgfqpoint{2.843698in}{1.797787in}}{\pgfqpoint{2.841503in}{1.803087in}}{\pgfqpoint{2.837596in}{1.806994in}}%
\pgfpathcurveto{\pgfqpoint{2.833690in}{1.810901in}}{\pgfqpoint{2.828390in}{1.813096in}}{\pgfqpoint{2.822865in}{1.813096in}}%
\pgfpathcurveto{\pgfqpoint{2.817340in}{1.813096in}}{\pgfqpoint{2.812040in}{1.810901in}}{\pgfqpoint{2.808134in}{1.806994in}}%
\pgfpathcurveto{\pgfqpoint{2.804227in}{1.803087in}}{\pgfqpoint{2.802032in}{1.797787in}}{\pgfqpoint{2.802032in}{1.792262in}}%
\pgfpathcurveto{\pgfqpoint{2.802032in}{1.786737in}}{\pgfqpoint{2.804227in}{1.781438in}}{\pgfqpoint{2.808134in}{1.777531in}}%
\pgfpathcurveto{\pgfqpoint{2.812040in}{1.773624in}}{\pgfqpoint{2.817340in}{1.771429in}}{\pgfqpoint{2.822865in}{1.771429in}}%
\pgfpathclose%
\pgfusepath{stroke}%
\end{pgfscope}%
\begin{pgfscope}%
\pgfpathrectangle{\pgfqpoint{0.438556in}{0.383578in}}{\pgfqpoint{4.650000in}{2.310000in}}%
\pgfusepath{clip}%
\pgfsetbuttcap%
\pgfsetroundjoin%
\pgfsetlinewidth{0.803000pt}%
\definecolor{currentstroke}{rgb}{0.686275,0.352941,0.313725}%
\pgfsetstrokecolor{currentstroke}%
\pgfsetdash{}{0pt}%
\pgfpathmoveto{\pgfqpoint{2.808214in}{1.793271in}}%
\pgfpathcurveto{\pgfqpoint{2.813739in}{1.793271in}}{\pgfqpoint{2.819039in}{1.795466in}}{\pgfqpoint{2.822945in}{1.799373in}}%
\pgfpathcurveto{\pgfqpoint{2.826852in}{1.803280in}}{\pgfqpoint{2.829047in}{1.808579in}}{\pgfqpoint{2.829047in}{1.814104in}}%
\pgfpathcurveto{\pgfqpoint{2.829047in}{1.819630in}}{\pgfqpoint{2.826852in}{1.824929in}}{\pgfqpoint{2.822945in}{1.828836in}}%
\pgfpathcurveto{\pgfqpoint{2.819039in}{1.832743in}}{\pgfqpoint{2.813739in}{1.834938in}}{\pgfqpoint{2.808214in}{1.834938in}}%
\pgfpathcurveto{\pgfqpoint{2.802689in}{1.834938in}}{\pgfqpoint{2.797389in}{1.832743in}}{\pgfqpoint{2.793483in}{1.828836in}}%
\pgfpathcurveto{\pgfqpoint{2.789576in}{1.824929in}}{\pgfqpoint{2.787381in}{1.819630in}}{\pgfqpoint{2.787381in}{1.814104in}}%
\pgfpathcurveto{\pgfqpoint{2.787381in}{1.808579in}}{\pgfqpoint{2.789576in}{1.803280in}}{\pgfqpoint{2.793483in}{1.799373in}}%
\pgfpathcurveto{\pgfqpoint{2.797389in}{1.795466in}}{\pgfqpoint{2.802689in}{1.793271in}}{\pgfqpoint{2.808214in}{1.793271in}}%
\pgfpathclose%
\pgfusepath{stroke}%
\end{pgfscope}%
\begin{pgfscope}%
\pgfpathrectangle{\pgfqpoint{0.438556in}{0.383578in}}{\pgfqpoint{4.650000in}{2.310000in}}%
\pgfusepath{clip}%
\pgfsetbuttcap%
\pgfsetroundjoin%
\pgfsetlinewidth{0.803000pt}%
\definecolor{currentstroke}{rgb}{0.686275,0.352941,0.313725}%
\pgfsetstrokecolor{currentstroke}%
\pgfsetdash{}{0pt}%
\pgfpathmoveto{\pgfqpoint{2.843435in}{1.836955in}}%
\pgfpathcurveto{\pgfqpoint{2.848960in}{1.836955in}}{\pgfqpoint{2.854259in}{1.839151in}}{\pgfqpoint{2.858166in}{1.843057in}}%
\pgfpathcurveto{\pgfqpoint{2.862073in}{1.846964in}}{\pgfqpoint{2.864268in}{1.852264in}}{\pgfqpoint{2.864268in}{1.857789in}}%
\pgfpathcurveto{\pgfqpoint{2.864268in}{1.863314in}}{\pgfqpoint{2.862073in}{1.868613in}}{\pgfqpoint{2.858166in}{1.872520in}}%
\pgfpathcurveto{\pgfqpoint{2.854259in}{1.876427in}}{\pgfqpoint{2.848960in}{1.878622in}}{\pgfqpoint{2.843435in}{1.878622in}}%
\pgfpathcurveto{\pgfqpoint{2.837910in}{1.878622in}}{\pgfqpoint{2.832610in}{1.876427in}}{\pgfqpoint{2.828704in}{1.872520in}}%
\pgfpathcurveto{\pgfqpoint{2.824797in}{1.868613in}}{\pgfqpoint{2.822602in}{1.863314in}}{\pgfqpoint{2.822602in}{1.857789in}}%
\pgfpathcurveto{\pgfqpoint{2.822602in}{1.852264in}}{\pgfqpoint{2.824797in}{1.846964in}}{\pgfqpoint{2.828704in}{1.843057in}}%
\pgfpathcurveto{\pgfqpoint{2.832610in}{1.839151in}}{\pgfqpoint{2.837910in}{1.836955in}}{\pgfqpoint{2.843435in}{1.836955in}}%
\pgfpathclose%
\pgfusepath{stroke}%
\end{pgfscope}%
\begin{pgfscope}%
\pgfpathrectangle{\pgfqpoint{0.438556in}{0.383578in}}{\pgfqpoint{4.650000in}{2.310000in}}%
\pgfusepath{clip}%
\pgfsetbuttcap%
\pgfsetroundjoin%
\pgfsetlinewidth{0.803000pt}%
\definecolor{currentstroke}{rgb}{0.686275,0.352941,0.313725}%
\pgfsetstrokecolor{currentstroke}%
\pgfsetdash{}{0pt}%
\pgfpathmoveto{\pgfqpoint{2.847741in}{1.858798in}}%
\pgfpathcurveto{\pgfqpoint{2.853266in}{1.858798in}}{\pgfqpoint{2.858566in}{1.860993in}}{\pgfqpoint{2.862473in}{1.864899in}}%
\pgfpathcurveto{\pgfqpoint{2.866380in}{1.868806in}}{\pgfqpoint{2.868575in}{1.874106in}}{\pgfqpoint{2.868575in}{1.879631in}}%
\pgfpathcurveto{\pgfqpoint{2.868575in}{1.885156in}}{\pgfqpoint{2.866380in}{1.890455in}}{\pgfqpoint{2.862473in}{1.894362in}}%
\pgfpathcurveto{\pgfqpoint{2.858566in}{1.898269in}}{\pgfqpoint{2.853266in}{1.900464in}}{\pgfqpoint{2.847741in}{1.900464in}}%
\pgfpathcurveto{\pgfqpoint{2.842216in}{1.900464in}}{\pgfqpoint{2.836917in}{1.898269in}}{\pgfqpoint{2.833010in}{1.894362in}}%
\pgfpathcurveto{\pgfqpoint{2.829103in}{1.890455in}}{\pgfqpoint{2.826908in}{1.885156in}}{\pgfqpoint{2.826908in}{1.879631in}}%
\pgfpathcurveto{\pgfqpoint{2.826908in}{1.874106in}}{\pgfqpoint{2.829103in}{1.868806in}}{\pgfqpoint{2.833010in}{1.864899in}}%
\pgfpathcurveto{\pgfqpoint{2.836917in}{1.860993in}}{\pgfqpoint{2.842216in}{1.858798in}}{\pgfqpoint{2.847741in}{1.858798in}}%
\pgfpathclose%
\pgfusepath{stroke}%
\end{pgfscope}%
\begin{pgfscope}%
\pgfpathrectangle{\pgfqpoint{0.438556in}{0.383578in}}{\pgfqpoint{4.650000in}{2.310000in}}%
\pgfusepath{clip}%
\pgfsetbuttcap%
\pgfsetroundjoin%
\pgfsetlinewidth{0.803000pt}%
\definecolor{currentstroke}{rgb}{0.686275,0.352941,0.313725}%
\pgfsetstrokecolor{currentstroke}%
\pgfsetdash{}{0pt}%
\pgfpathmoveto{\pgfqpoint{2.838405in}{1.924324in}}%
\pgfpathcurveto{\pgfqpoint{2.843930in}{1.924324in}}{\pgfqpoint{2.849229in}{1.926519in}}{\pgfqpoint{2.853136in}{1.930426in}}%
\pgfpathcurveto{\pgfqpoint{2.857043in}{1.934333in}}{\pgfqpoint{2.859238in}{1.939632in}}{\pgfqpoint{2.859238in}{1.945157in}}%
\pgfpathcurveto{\pgfqpoint{2.859238in}{1.950682in}}{\pgfqpoint{2.857043in}{1.955982in}}{\pgfqpoint{2.853136in}{1.959889in}}%
\pgfpathcurveto{\pgfqpoint{2.849229in}{1.963795in}}{\pgfqpoint{2.843930in}{1.965991in}}{\pgfqpoint{2.838405in}{1.965991in}}%
\pgfpathcurveto{\pgfqpoint{2.832880in}{1.965991in}}{\pgfqpoint{2.827580in}{1.963795in}}{\pgfqpoint{2.823673in}{1.959889in}}%
\pgfpathcurveto{\pgfqpoint{2.819766in}{1.955982in}}{\pgfqpoint{2.817571in}{1.950682in}}{\pgfqpoint{2.817571in}{1.945157in}}%
\pgfpathcurveto{\pgfqpoint{2.817571in}{1.939632in}}{\pgfqpoint{2.819766in}{1.934333in}}{\pgfqpoint{2.823673in}{1.930426in}}%
\pgfpathcurveto{\pgfqpoint{2.827580in}{1.926519in}}{\pgfqpoint{2.832880in}{1.924324in}}{\pgfqpoint{2.838405in}{1.924324in}}%
\pgfpathclose%
\pgfusepath{stroke}%
\end{pgfscope}%
\begin{pgfscope}%
\pgfpathrectangle{\pgfqpoint{0.438556in}{0.383578in}}{\pgfqpoint{4.650000in}{2.310000in}}%
\pgfusepath{clip}%
\pgfsetbuttcap%
\pgfsetroundjoin%
\pgfsetlinewidth{0.803000pt}%
\definecolor{currentstroke}{rgb}{0.686275,0.352941,0.313725}%
\pgfsetstrokecolor{currentstroke}%
\pgfsetdash{}{0pt}%
\pgfpathmoveto{\pgfqpoint{2.815581in}{1.989850in}}%
\pgfpathcurveto{\pgfqpoint{2.821106in}{1.989850in}}{\pgfqpoint{2.826405in}{1.992045in}}{\pgfqpoint{2.830312in}{1.995952in}}%
\pgfpathcurveto{\pgfqpoint{2.834219in}{1.999859in}}{\pgfqpoint{2.836414in}{2.005159in}}{\pgfqpoint{2.836414in}{2.010684in}}%
\pgfpathcurveto{\pgfqpoint{2.836414in}{2.016209in}}{\pgfqpoint{2.834219in}{2.021508in}}{\pgfqpoint{2.830312in}{2.025415in}}%
\pgfpathcurveto{\pgfqpoint{2.826405in}{2.029322in}}{\pgfqpoint{2.821106in}{2.031517in}}{\pgfqpoint{2.815581in}{2.031517in}}%
\pgfpathcurveto{\pgfqpoint{2.810056in}{2.031517in}}{\pgfqpoint{2.804756in}{2.029322in}}{\pgfqpoint{2.800849in}{2.025415in}}%
\pgfpathcurveto{\pgfqpoint{2.796943in}{2.021508in}}{\pgfqpoint{2.794747in}{2.016209in}}{\pgfqpoint{2.794747in}{2.010684in}}%
\pgfpathcurveto{\pgfqpoint{2.794747in}{2.005159in}}{\pgfqpoint{2.796943in}{1.999859in}}{\pgfqpoint{2.800849in}{1.995952in}}%
\pgfpathcurveto{\pgfqpoint{2.804756in}{1.992045in}}{\pgfqpoint{2.810056in}{1.989850in}}{\pgfqpoint{2.815581in}{1.989850in}}%
\pgfpathclose%
\pgfusepath{stroke}%
\end{pgfscope}%
\begin{pgfscope}%
\pgfpathrectangle{\pgfqpoint{0.438556in}{0.383578in}}{\pgfqpoint{4.650000in}{2.310000in}}%
\pgfusepath{clip}%
\pgfsetbuttcap%
\pgfsetroundjoin%
\pgfsetlinewidth{0.803000pt}%
\definecolor{currentstroke}{rgb}{0.686275,0.352941,0.313725}%
\pgfsetstrokecolor{currentstroke}%
\pgfsetdash{}{0pt}%
\pgfpathmoveto{\pgfqpoint{2.839807in}{2.011692in}}%
\pgfpathcurveto{\pgfqpoint{2.845332in}{2.011692in}}{\pgfqpoint{2.850631in}{2.013888in}}{\pgfqpoint{2.854538in}{2.017794in}}%
\pgfpathcurveto{\pgfqpoint{2.858445in}{2.021701in}}{\pgfqpoint{2.860640in}{2.027001in}}{\pgfqpoint{2.860640in}{2.032526in}}%
\pgfpathcurveto{\pgfqpoint{2.860640in}{2.038051in}}{\pgfqpoint{2.858445in}{2.043350in}}{\pgfqpoint{2.854538in}{2.047257in}}%
\pgfpathcurveto{\pgfqpoint{2.850631in}{2.051164in}}{\pgfqpoint{2.845332in}{2.053359in}}{\pgfqpoint{2.839807in}{2.053359in}}%
\pgfpathcurveto{\pgfqpoint{2.834281in}{2.053359in}}{\pgfqpoint{2.828982in}{2.051164in}}{\pgfqpoint{2.825075in}{2.047257in}}%
\pgfpathcurveto{\pgfqpoint{2.821168in}{2.043350in}}{\pgfqpoint{2.818973in}{2.038051in}}{\pgfqpoint{2.818973in}{2.032526in}}%
\pgfpathcurveto{\pgfqpoint{2.818973in}{2.027001in}}{\pgfqpoint{2.821168in}{2.021701in}}{\pgfqpoint{2.825075in}{2.017794in}}%
\pgfpathcurveto{\pgfqpoint{2.828982in}{2.013888in}}{\pgfqpoint{2.834281in}{2.011692in}}{\pgfqpoint{2.839807in}{2.011692in}}%
\pgfpathclose%
\pgfusepath{stroke}%
\end{pgfscope}%
\begin{pgfscope}%
\pgfpathrectangle{\pgfqpoint{0.438556in}{0.383578in}}{\pgfqpoint{4.650000in}{2.310000in}}%
\pgfusepath{clip}%
\pgfsetbuttcap%
\pgfsetroundjoin%
\pgfsetlinewidth{0.803000pt}%
\definecolor{currentstroke}{rgb}{0.686275,0.352941,0.313725}%
\pgfsetstrokecolor{currentstroke}%
\pgfsetdash{}{0pt}%
\pgfpathmoveto{\pgfqpoint{2.819603in}{2.077219in}}%
\pgfpathcurveto{\pgfqpoint{2.825128in}{2.077219in}}{\pgfqpoint{2.830428in}{2.079414in}}{\pgfqpoint{2.834334in}{2.083321in}}%
\pgfpathcurveto{\pgfqpoint{2.838241in}{2.087228in}}{\pgfqpoint{2.840436in}{2.092527in}}{\pgfqpoint{2.840436in}{2.098052in}}%
\pgfpathcurveto{\pgfqpoint{2.840436in}{2.103577in}}{\pgfqpoint{2.838241in}{2.108877in}}{\pgfqpoint{2.834334in}{2.112784in}}%
\pgfpathcurveto{\pgfqpoint{2.830428in}{2.116690in}}{\pgfqpoint{2.825128in}{2.118885in}}{\pgfqpoint{2.819603in}{2.118885in}}%
\pgfpathcurveto{\pgfqpoint{2.814078in}{2.118885in}}{\pgfqpoint{2.808779in}{2.116690in}}{\pgfqpoint{2.804872in}{2.112784in}}%
\pgfpathcurveto{\pgfqpoint{2.800965in}{2.108877in}}{\pgfqpoint{2.798770in}{2.103577in}}{\pgfqpoint{2.798770in}{2.098052in}}%
\pgfpathcurveto{\pgfqpoint{2.798770in}{2.092527in}}{\pgfqpoint{2.800965in}{2.087228in}}{\pgfqpoint{2.804872in}{2.083321in}}%
\pgfpathcurveto{\pgfqpoint{2.808779in}{2.079414in}}{\pgfqpoint{2.814078in}{2.077219in}}{\pgfqpoint{2.819603in}{2.077219in}}%
\pgfpathclose%
\pgfusepath{stroke}%
\end{pgfscope}%
\begin{pgfscope}%
\pgfpathrectangle{\pgfqpoint{0.438556in}{0.383578in}}{\pgfqpoint{4.650000in}{2.310000in}}%
\pgfusepath{clip}%
\pgfsetbuttcap%
\pgfsetroundjoin%
\pgfsetlinewidth{0.803000pt}%
\definecolor{currentstroke}{rgb}{0.686275,0.352941,0.313725}%
\pgfsetstrokecolor{currentstroke}%
\pgfsetdash{}{0pt}%
\pgfpathmoveto{\pgfqpoint{2.857380in}{2.099061in}}%
\pgfpathcurveto{\pgfqpoint{2.862905in}{2.099061in}}{\pgfqpoint{2.868205in}{2.101256in}}{\pgfqpoint{2.872112in}{2.105163in}}%
\pgfpathcurveto{\pgfqpoint{2.876019in}{2.109070in}}{\pgfqpoint{2.878214in}{2.114369in}}{\pgfqpoint{2.878214in}{2.119894in}}%
\pgfpathcurveto{\pgfqpoint{2.878214in}{2.125419in}}{\pgfqpoint{2.876019in}{2.130719in}}{\pgfqpoint{2.872112in}{2.134626in}}%
\pgfpathcurveto{\pgfqpoint{2.868205in}{2.138532in}}{\pgfqpoint{2.862905in}{2.140728in}}{\pgfqpoint{2.857380in}{2.140728in}}%
\pgfpathcurveto{\pgfqpoint{2.851855in}{2.140728in}}{\pgfqpoint{2.846556in}{2.138532in}}{\pgfqpoint{2.842649in}{2.134626in}}%
\pgfpathcurveto{\pgfqpoint{2.838742in}{2.130719in}}{\pgfqpoint{2.836547in}{2.125419in}}{\pgfqpoint{2.836547in}{2.119894in}}%
\pgfpathcurveto{\pgfqpoint{2.836547in}{2.114369in}}{\pgfqpoint{2.838742in}{2.109070in}}{\pgfqpoint{2.842649in}{2.105163in}}%
\pgfpathcurveto{\pgfqpoint{2.846556in}{2.101256in}}{\pgfqpoint{2.851855in}{2.099061in}}{\pgfqpoint{2.857380in}{2.099061in}}%
\pgfpathclose%
\pgfusepath{stroke}%
\end{pgfscope}%
\begin{pgfscope}%
\pgfpathrectangle{\pgfqpoint{0.438556in}{0.383578in}}{\pgfqpoint{4.650000in}{2.310000in}}%
\pgfusepath{clip}%
\pgfsetbuttcap%
\pgfsetroundjoin%
\pgfsetlinewidth{0.803000pt}%
\definecolor{currentstroke}{rgb}{0.686275,0.352941,0.313725}%
\pgfsetstrokecolor{currentstroke}%
\pgfsetdash{}{0pt}%
\pgfpathmoveto{\pgfqpoint{2.879279in}{2.164587in}}%
\pgfpathcurveto{\pgfqpoint{2.884804in}{2.164587in}}{\pgfqpoint{2.890103in}{2.166782in}}{\pgfqpoint{2.894010in}{2.170689in}}%
\pgfpathcurveto{\pgfqpoint{2.897917in}{2.174596in}}{\pgfqpoint{2.900112in}{2.179896in}}{\pgfqpoint{2.900112in}{2.185421in}}%
\pgfpathcurveto{\pgfqpoint{2.900112in}{2.190946in}}{\pgfqpoint{2.897917in}{2.196245in}}{\pgfqpoint{2.894010in}{2.200152in}}%
\pgfpathcurveto{\pgfqpoint{2.890103in}{2.204059in}}{\pgfqpoint{2.884804in}{2.206254in}}{\pgfqpoint{2.879279in}{2.206254in}}%
\pgfpathcurveto{\pgfqpoint{2.873754in}{2.206254in}}{\pgfqpoint{2.868454in}{2.204059in}}{\pgfqpoint{2.864547in}{2.200152in}}%
\pgfpathcurveto{\pgfqpoint{2.860641in}{2.196245in}}{\pgfqpoint{2.858446in}{2.190946in}}{\pgfqpoint{2.858446in}{2.185421in}}%
\pgfpathcurveto{\pgfqpoint{2.858446in}{2.179896in}}{\pgfqpoint{2.860641in}{2.174596in}}{\pgfqpoint{2.864547in}{2.170689in}}%
\pgfpathcurveto{\pgfqpoint{2.868454in}{2.166782in}}{\pgfqpoint{2.873754in}{2.164587in}}{\pgfqpoint{2.879279in}{2.164587in}}%
\pgfpathclose%
\pgfusepath{stroke}%
\end{pgfscope}%
\begin{pgfscope}%
\pgfpathrectangle{\pgfqpoint{0.438556in}{0.383578in}}{\pgfqpoint{4.650000in}{2.310000in}}%
\pgfusepath{clip}%
\pgfsetbuttcap%
\pgfsetroundjoin%
\pgfsetlinewidth{0.803000pt}%
\definecolor{currentstroke}{rgb}{0.686275,0.352941,0.313725}%
\pgfsetstrokecolor{currentstroke}%
\pgfsetdash{}{0pt}%
\pgfpathmoveto{\pgfqpoint{2.847851in}{2.230114in}}%
\pgfpathcurveto{\pgfqpoint{2.853376in}{2.230114in}}{\pgfqpoint{2.858676in}{2.232309in}}{\pgfqpoint{2.862583in}{2.236216in}}%
\pgfpathcurveto{\pgfqpoint{2.866489in}{2.240123in}}{\pgfqpoint{2.868685in}{2.245422in}}{\pgfqpoint{2.868685in}{2.250947in}}%
\pgfpathcurveto{\pgfqpoint{2.868685in}{2.256472in}}{\pgfqpoint{2.866489in}{2.261772in}}{\pgfqpoint{2.862583in}{2.265678in}}%
\pgfpathcurveto{\pgfqpoint{2.858676in}{2.269585in}}{\pgfqpoint{2.853376in}{2.271780in}}{\pgfqpoint{2.847851in}{2.271780in}}%
\pgfpathcurveto{\pgfqpoint{2.842326in}{2.271780in}}{\pgfqpoint{2.837027in}{2.269585in}}{\pgfqpoint{2.833120in}{2.265678in}}%
\pgfpathcurveto{\pgfqpoint{2.829213in}{2.261772in}}{\pgfqpoint{2.827018in}{2.256472in}}{\pgfqpoint{2.827018in}{2.250947in}}%
\pgfpathcurveto{\pgfqpoint{2.827018in}{2.245422in}}{\pgfqpoint{2.829213in}{2.240123in}}{\pgfqpoint{2.833120in}{2.236216in}}%
\pgfpathcurveto{\pgfqpoint{2.837027in}{2.232309in}}{\pgfqpoint{2.842326in}{2.230114in}}{\pgfqpoint{2.847851in}{2.230114in}}%
\pgfpathclose%
\pgfusepath{stroke}%
\end{pgfscope}%
\begin{pgfscope}%
\pgfpathrectangle{\pgfqpoint{0.438556in}{0.383578in}}{\pgfqpoint{4.650000in}{2.310000in}}%
\pgfusepath{clip}%
\pgfsetbuttcap%
\pgfsetroundjoin%
\pgfsetlinewidth{0.803000pt}%
\definecolor{currentstroke}{rgb}{0.686275,0.352941,0.313725}%
\pgfsetstrokecolor{currentstroke}%
\pgfsetdash{}{0pt}%
\pgfpathmoveto{\pgfqpoint{2.837983in}{2.251956in}}%
\pgfpathcurveto{\pgfqpoint{2.843508in}{2.251956in}}{\pgfqpoint{2.848808in}{2.254151in}}{\pgfqpoint{2.852715in}{2.258058in}}%
\pgfpathcurveto{\pgfqpoint{2.856621in}{2.261965in}}{\pgfqpoint{2.858817in}{2.267264in}}{\pgfqpoint{2.858817in}{2.272789in}}%
\pgfpathcurveto{\pgfqpoint{2.858817in}{2.278314in}}{\pgfqpoint{2.856621in}{2.283614in}}{\pgfqpoint{2.852715in}{2.287521in}}%
\pgfpathcurveto{\pgfqpoint{2.848808in}{2.291427in}}{\pgfqpoint{2.843508in}{2.293623in}}{\pgfqpoint{2.837983in}{2.293623in}}%
\pgfpathcurveto{\pgfqpoint{2.832458in}{2.293623in}}{\pgfqpoint{2.827159in}{2.291427in}}{\pgfqpoint{2.823252in}{2.287521in}}%
\pgfpathcurveto{\pgfqpoint{2.819345in}{2.283614in}}{\pgfqpoint{2.817150in}{2.278314in}}{\pgfqpoint{2.817150in}{2.272789in}}%
\pgfpathcurveto{\pgfqpoint{2.817150in}{2.267264in}}{\pgfqpoint{2.819345in}{2.261965in}}{\pgfqpoint{2.823252in}{2.258058in}}%
\pgfpathcurveto{\pgfqpoint{2.827159in}{2.254151in}}{\pgfqpoint{2.832458in}{2.251956in}}{\pgfqpoint{2.837983in}{2.251956in}}%
\pgfpathclose%
\pgfusepath{stroke}%
\end{pgfscope}%
\begin{pgfscope}%
\pgfpathrectangle{\pgfqpoint{0.438556in}{0.383578in}}{\pgfqpoint{4.650000in}{2.310000in}}%
\pgfusepath{clip}%
\pgfsetbuttcap%
\pgfsetroundjoin%
\pgfsetlinewidth{0.803000pt}%
\definecolor{currentstroke}{rgb}{0.686275,0.352941,0.313725}%
\pgfsetstrokecolor{currentstroke}%
\pgfsetdash{}{0pt}%
\pgfpathmoveto{\pgfqpoint{2.846697in}{2.295640in}}%
\pgfpathcurveto{\pgfqpoint{2.852222in}{2.295640in}}{\pgfqpoint{2.857521in}{2.297835in}}{\pgfqpoint{2.861428in}{2.301742in}}%
\pgfpathcurveto{\pgfqpoint{2.865335in}{2.305649in}}{\pgfqpoint{2.867530in}{2.310948in}}{\pgfqpoint{2.867530in}{2.316473in}}%
\pgfpathcurveto{\pgfqpoint{2.867530in}{2.321999in}}{\pgfqpoint{2.865335in}{2.327298in}}{\pgfqpoint{2.861428in}{2.331205in}}%
\pgfpathcurveto{\pgfqpoint{2.857521in}{2.335112in}}{\pgfqpoint{2.852222in}{2.337307in}}{\pgfqpoint{2.846697in}{2.337307in}}%
\pgfpathcurveto{\pgfqpoint{2.841172in}{2.337307in}}{\pgfqpoint{2.835872in}{2.335112in}}{\pgfqpoint{2.831965in}{2.331205in}}%
\pgfpathcurveto{\pgfqpoint{2.828059in}{2.327298in}}{\pgfqpoint{2.825863in}{2.321999in}}{\pgfqpoint{2.825863in}{2.316473in}}%
\pgfpathcurveto{\pgfqpoint{2.825863in}{2.310948in}}{\pgfqpoint{2.828059in}{2.305649in}}{\pgfqpoint{2.831965in}{2.301742in}}%
\pgfpathcurveto{\pgfqpoint{2.835872in}{2.297835in}}{\pgfqpoint{2.841172in}{2.295640in}}{\pgfqpoint{2.846697in}{2.295640in}}%
\pgfpathclose%
\pgfusepath{stroke}%
\end{pgfscope}%
\begin{pgfscope}%
\pgfpathrectangle{\pgfqpoint{0.438556in}{0.383578in}}{\pgfqpoint{4.650000in}{2.310000in}}%
\pgfusepath{clip}%
\pgfsetbuttcap%
\pgfsetroundjoin%
\pgfsetlinewidth{0.803000pt}%
\definecolor{currentstroke}{rgb}{0.686275,0.352941,0.313725}%
\pgfsetstrokecolor{currentstroke}%
\pgfsetdash{}{0pt}%
\pgfpathmoveto{\pgfqpoint{2.846495in}{2.317482in}}%
\pgfpathcurveto{\pgfqpoint{2.852020in}{2.317482in}}{\pgfqpoint{2.857320in}{2.319677in}}{\pgfqpoint{2.861227in}{2.323584in}}%
\pgfpathcurveto{\pgfqpoint{2.865133in}{2.327491in}}{\pgfqpoint{2.867329in}{2.332791in}}{\pgfqpoint{2.867329in}{2.338316in}}%
\pgfpathcurveto{\pgfqpoint{2.867329in}{2.343841in}}{\pgfqpoint{2.865133in}{2.349140in}}{\pgfqpoint{2.861227in}{2.353047in}}%
\pgfpathcurveto{\pgfqpoint{2.857320in}{2.356954in}}{\pgfqpoint{2.852020in}{2.359149in}}{\pgfqpoint{2.846495in}{2.359149in}}%
\pgfpathcurveto{\pgfqpoint{2.840970in}{2.359149in}}{\pgfqpoint{2.835671in}{2.356954in}}{\pgfqpoint{2.831764in}{2.353047in}}%
\pgfpathcurveto{\pgfqpoint{2.827857in}{2.349140in}}{\pgfqpoint{2.825662in}{2.343841in}}{\pgfqpoint{2.825662in}{2.338316in}}%
\pgfpathcurveto{\pgfqpoint{2.825662in}{2.332791in}}{\pgfqpoint{2.827857in}{2.327491in}}{\pgfqpoint{2.831764in}{2.323584in}}%
\pgfpathcurveto{\pgfqpoint{2.835671in}{2.319677in}}{\pgfqpoint{2.840970in}{2.317482in}}{\pgfqpoint{2.846495in}{2.317482in}}%
\pgfpathclose%
\pgfusepath{stroke}%
\end{pgfscope}%
\begin{pgfscope}%
\pgfpathrectangle{\pgfqpoint{0.438556in}{0.383578in}}{\pgfqpoint{4.650000in}{2.310000in}}%
\pgfusepath{clip}%
\pgfsetbuttcap%
\pgfsetroundjoin%
\pgfsetlinewidth{0.803000pt}%
\definecolor{currentstroke}{rgb}{0.686275,0.352941,0.313725}%
\pgfsetstrokecolor{currentstroke}%
\pgfsetdash{}{0pt}%
\pgfpathmoveto{\pgfqpoint{2.864600in}{2.339324in}}%
\pgfpathcurveto{\pgfqpoint{2.870125in}{2.339324in}}{\pgfqpoint{2.875425in}{2.341520in}}{\pgfqpoint{2.879332in}{2.345426in}}%
\pgfpathcurveto{\pgfqpoint{2.883239in}{2.349333in}}{\pgfqpoint{2.885434in}{2.354633in}}{\pgfqpoint{2.885434in}{2.360158in}}%
\pgfpathcurveto{\pgfqpoint{2.885434in}{2.365683in}}{\pgfqpoint{2.883239in}{2.370982in}}{\pgfqpoint{2.879332in}{2.374889in}}%
\pgfpathcurveto{\pgfqpoint{2.875425in}{2.378796in}}{\pgfqpoint{2.870125in}{2.380991in}}{\pgfqpoint{2.864600in}{2.380991in}}%
\pgfpathcurveto{\pgfqpoint{2.859075in}{2.380991in}}{\pgfqpoint{2.853776in}{2.378796in}}{\pgfqpoint{2.849869in}{2.374889in}}%
\pgfpathcurveto{\pgfqpoint{2.845962in}{2.370982in}}{\pgfqpoint{2.843767in}{2.365683in}}{\pgfqpoint{2.843767in}{2.360158in}}%
\pgfpathcurveto{\pgfqpoint{2.843767in}{2.354633in}}{\pgfqpoint{2.845962in}{2.349333in}}{\pgfqpoint{2.849869in}{2.345426in}}%
\pgfpathcurveto{\pgfqpoint{2.853776in}{2.341520in}}{\pgfqpoint{2.859075in}{2.339324in}}{\pgfqpoint{2.864600in}{2.339324in}}%
\pgfpathclose%
\pgfusepath{stroke}%
\end{pgfscope}%
\begin{pgfscope}%
\pgfpathrectangle{\pgfqpoint{0.438556in}{0.383578in}}{\pgfqpoint{4.650000in}{2.310000in}}%
\pgfusepath{clip}%
\pgfsetbuttcap%
\pgfsetroundjoin%
\pgfsetlinewidth{0.803000pt}%
\definecolor{currentstroke}{rgb}{0.686275,0.352941,0.313725}%
\pgfsetstrokecolor{currentstroke}%
\pgfsetdash{}{0pt}%
\pgfpathmoveto{\pgfqpoint{2.884666in}{2.383009in}}%
\pgfpathcurveto{\pgfqpoint{2.890191in}{2.383009in}}{\pgfqpoint{2.895491in}{2.385204in}}{\pgfqpoint{2.899398in}{2.389111in}}%
\pgfpathcurveto{\pgfqpoint{2.903305in}{2.393017in}}{\pgfqpoint{2.905500in}{2.398317in}}{\pgfqpoint{2.905500in}{2.403842in}}%
\pgfpathcurveto{\pgfqpoint{2.905500in}{2.409367in}}{\pgfqpoint{2.903305in}{2.414667in}}{\pgfqpoint{2.899398in}{2.418573in}}%
\pgfpathcurveto{\pgfqpoint{2.895491in}{2.422480in}}{\pgfqpoint{2.890191in}{2.424675in}}{\pgfqpoint{2.884666in}{2.424675in}}%
\pgfpathcurveto{\pgfqpoint{2.879141in}{2.424675in}}{\pgfqpoint{2.873842in}{2.422480in}}{\pgfqpoint{2.869935in}{2.418573in}}%
\pgfpathcurveto{\pgfqpoint{2.866028in}{2.414667in}}{\pgfqpoint{2.863833in}{2.409367in}}{\pgfqpoint{2.863833in}{2.403842in}}%
\pgfpathcurveto{\pgfqpoint{2.863833in}{2.398317in}}{\pgfqpoint{2.866028in}{2.393017in}}{\pgfqpoint{2.869935in}{2.389111in}}%
\pgfpathcurveto{\pgfqpoint{2.873842in}{2.385204in}}{\pgfqpoint{2.879141in}{2.383009in}}{\pgfqpoint{2.884666in}{2.383009in}}%
\pgfpathclose%
\pgfusepath{stroke}%
\end{pgfscope}%
\begin{pgfscope}%
\pgfpathrectangle{\pgfqpoint{0.438556in}{0.383578in}}{\pgfqpoint{4.650000in}{2.310000in}}%
\pgfusepath{clip}%
\pgfsetbuttcap%
\pgfsetroundjoin%
\pgfsetlinewidth{0.803000pt}%
\definecolor{currentstroke}{rgb}{0.686275,0.352941,0.313725}%
\pgfsetstrokecolor{currentstroke}%
\pgfsetdash{}{0pt}%
\pgfpathmoveto{\pgfqpoint{2.813556in}{2.404851in}}%
\pgfpathcurveto{\pgfqpoint{2.819081in}{2.404851in}}{\pgfqpoint{2.824380in}{2.407046in}}{\pgfqpoint{2.828287in}{2.410953in}}%
\pgfpathcurveto{\pgfqpoint{2.832194in}{2.414860in}}{\pgfqpoint{2.834389in}{2.420159in}}{\pgfqpoint{2.834389in}{2.425684in}}%
\pgfpathcurveto{\pgfqpoint{2.834389in}{2.431209in}}{\pgfqpoint{2.832194in}{2.436509in}}{\pgfqpoint{2.828287in}{2.440416in}}%
\pgfpathcurveto{\pgfqpoint{2.824380in}{2.444322in}}{\pgfqpoint{2.819081in}{2.446517in}}{\pgfqpoint{2.813556in}{2.446517in}}%
\pgfpathcurveto{\pgfqpoint{2.808031in}{2.446517in}}{\pgfqpoint{2.802731in}{2.444322in}}{\pgfqpoint{2.798824in}{2.440416in}}%
\pgfpathcurveto{\pgfqpoint{2.794918in}{2.436509in}}{\pgfqpoint{2.792722in}{2.431209in}}{\pgfqpoint{2.792722in}{2.425684in}}%
\pgfpathcurveto{\pgfqpoint{2.792722in}{2.420159in}}{\pgfqpoint{2.794918in}{2.414860in}}{\pgfqpoint{2.798824in}{2.410953in}}%
\pgfpathcurveto{\pgfqpoint{2.802731in}{2.407046in}}{\pgfqpoint{2.808031in}{2.404851in}}{\pgfqpoint{2.813556in}{2.404851in}}%
\pgfpathclose%
\pgfusepath{stroke}%
\end{pgfscope}%
\begin{pgfscope}%
\pgfpathrectangle{\pgfqpoint{0.438556in}{0.383578in}}{\pgfqpoint{4.650000in}{2.310000in}}%
\pgfusepath{clip}%
\pgfsetbuttcap%
\pgfsetroundjoin%
\pgfsetlinewidth{0.803000pt}%
\definecolor{currentstroke}{rgb}{0.686275,0.352941,0.313725}%
\pgfsetstrokecolor{currentstroke}%
\pgfsetdash{}{0pt}%
\pgfpathmoveto{\pgfqpoint{2.831349in}{2.492219in}}%
\pgfpathcurveto{\pgfqpoint{2.836875in}{2.492219in}}{\pgfqpoint{2.842174in}{2.494414in}}{\pgfqpoint{2.846081in}{2.498321in}}%
\pgfpathcurveto{\pgfqpoint{2.849988in}{2.502228in}}{\pgfqpoint{2.852183in}{2.507528in}}{\pgfqpoint{2.852183in}{2.513053in}}%
\pgfpathcurveto{\pgfqpoint{2.852183in}{2.518578in}}{\pgfqpoint{2.849988in}{2.523877in}}{\pgfqpoint{2.846081in}{2.527784in}}%
\pgfpathcurveto{\pgfqpoint{2.842174in}{2.531691in}}{\pgfqpoint{2.836875in}{2.533886in}}{\pgfqpoint{2.831349in}{2.533886in}}%
\pgfpathcurveto{\pgfqpoint{2.825824in}{2.533886in}}{\pgfqpoint{2.820525in}{2.531691in}}{\pgfqpoint{2.816618in}{2.527784in}}%
\pgfpathcurveto{\pgfqpoint{2.812711in}{2.523877in}}{\pgfqpoint{2.810516in}{2.518578in}}{\pgfqpoint{2.810516in}{2.513053in}}%
\pgfpathcurveto{\pgfqpoint{2.810516in}{2.507528in}}{\pgfqpoint{2.812711in}{2.502228in}}{\pgfqpoint{2.816618in}{2.498321in}}%
\pgfpathcurveto{\pgfqpoint{2.820525in}{2.494414in}}{\pgfqpoint{2.825824in}{2.492219in}}{\pgfqpoint{2.831349in}{2.492219in}}%
\pgfpathclose%
\pgfusepath{stroke}%
\end{pgfscope}%
\begin{pgfscope}%
\pgfpathrectangle{\pgfqpoint{0.438556in}{0.383578in}}{\pgfqpoint{4.650000in}{2.310000in}}%
\pgfusepath{clip}%
\pgfsetbuttcap%
\pgfsetroundjoin%
\pgfsetlinewidth{0.803000pt}%
\definecolor{currentstroke}{rgb}{0.686275,0.352941,0.313725}%
\pgfsetstrokecolor{currentstroke}%
\pgfsetdash{}{0pt}%
\pgfpathmoveto{\pgfqpoint{2.805190in}{2.514061in}}%
\pgfpathcurveto{\pgfqpoint{2.810715in}{2.514061in}}{\pgfqpoint{2.816015in}{2.516257in}}{\pgfqpoint{2.819922in}{2.520163in}}%
\pgfpathcurveto{\pgfqpoint{2.823829in}{2.524070in}}{\pgfqpoint{2.826024in}{2.529370in}}{\pgfqpoint{2.826024in}{2.534895in}}%
\pgfpathcurveto{\pgfqpoint{2.826024in}{2.540420in}}{\pgfqpoint{2.823829in}{2.545719in}}{\pgfqpoint{2.819922in}{2.549626in}}%
\pgfpathcurveto{\pgfqpoint{2.816015in}{2.553533in}}{\pgfqpoint{2.810715in}{2.555728in}}{\pgfqpoint{2.805190in}{2.555728in}}%
\pgfpathcurveto{\pgfqpoint{2.799665in}{2.555728in}}{\pgfqpoint{2.794366in}{2.553533in}}{\pgfqpoint{2.790459in}{2.549626in}}%
\pgfpathcurveto{\pgfqpoint{2.786552in}{2.545719in}}{\pgfqpoint{2.784357in}{2.540420in}}{\pgfqpoint{2.784357in}{2.534895in}}%
\pgfpathcurveto{\pgfqpoint{2.784357in}{2.529370in}}{\pgfqpoint{2.786552in}{2.524070in}}{\pgfqpoint{2.790459in}{2.520163in}}%
\pgfpathcurveto{\pgfqpoint{2.794366in}{2.516257in}}{\pgfqpoint{2.799665in}{2.514061in}}{\pgfqpoint{2.805190in}{2.514061in}}%
\pgfpathclose%
\pgfusepath{stroke}%
\end{pgfscope}%
\begin{pgfscope}%
\pgfpathrectangle{\pgfqpoint{0.438556in}{0.383578in}}{\pgfqpoint{4.650000in}{2.310000in}}%
\pgfusepath{clip}%
\pgfsetbuttcap%
\pgfsetroundjoin%
\pgfsetlinewidth{0.803000pt}%
\definecolor{currentstroke}{rgb}{0.686275,0.352941,0.313725}%
\pgfsetstrokecolor{currentstroke}%
\pgfsetdash{}{0pt}%
\pgfpathmoveto{\pgfqpoint{3.381873in}{0.482743in}}%
\pgfpathcurveto{\pgfqpoint{3.387398in}{0.482743in}}{\pgfqpoint{3.392697in}{0.484938in}}{\pgfqpoint{3.396604in}{0.488845in}}%
\pgfpathcurveto{\pgfqpoint{3.400511in}{0.492752in}}{\pgfqpoint{3.402706in}{0.498051in}}{\pgfqpoint{3.402706in}{0.503577in}}%
\pgfpathcurveto{\pgfqpoint{3.402706in}{0.509102in}}{\pgfqpoint{3.400511in}{0.514401in}}{\pgfqpoint{3.396604in}{0.518308in}}%
\pgfpathcurveto{\pgfqpoint{3.392697in}{0.522215in}}{\pgfqpoint{3.387398in}{0.524410in}}{\pgfqpoint{3.381873in}{0.524410in}}%
\pgfpathcurveto{\pgfqpoint{3.376348in}{0.524410in}}{\pgfqpoint{3.371048in}{0.522215in}}{\pgfqpoint{3.367141in}{0.518308in}}%
\pgfpathcurveto{\pgfqpoint{3.363235in}{0.514401in}}{\pgfqpoint{3.361040in}{0.509102in}}{\pgfqpoint{3.361040in}{0.503577in}}%
\pgfpathcurveto{\pgfqpoint{3.361040in}{0.498051in}}{\pgfqpoint{3.363235in}{0.492752in}}{\pgfqpoint{3.367141in}{0.488845in}}%
\pgfpathcurveto{\pgfqpoint{3.371048in}{0.484938in}}{\pgfqpoint{3.376348in}{0.482743in}}{\pgfqpoint{3.381873in}{0.482743in}}%
\pgfpathclose%
\pgfusepath{stroke}%
\end{pgfscope}%
\begin{pgfscope}%
\pgfpathrectangle{\pgfqpoint{0.438556in}{0.383578in}}{\pgfqpoint{4.650000in}{2.310000in}}%
\pgfusepath{clip}%
\pgfsetbuttcap%
\pgfsetroundjoin%
\pgfsetlinewidth{0.803000pt}%
\definecolor{currentstroke}{rgb}{0.686275,0.352941,0.313725}%
\pgfsetstrokecolor{currentstroke}%
\pgfsetdash{}{0pt}%
\pgfpathmoveto{\pgfqpoint{3.380929in}{0.635638in}}%
\pgfpathcurveto{\pgfqpoint{3.386454in}{0.635638in}}{\pgfqpoint{3.391754in}{0.637833in}}{\pgfqpoint{3.395660in}{0.641740in}}%
\pgfpathcurveto{\pgfqpoint{3.399567in}{0.645647in}}{\pgfqpoint{3.401762in}{0.650946in}}{\pgfqpoint{3.401762in}{0.656471in}}%
\pgfpathcurveto{\pgfqpoint{3.401762in}{0.661997in}}{\pgfqpoint{3.399567in}{0.667296in}}{\pgfqpoint{3.395660in}{0.671203in}}%
\pgfpathcurveto{\pgfqpoint{3.391754in}{0.675110in}}{\pgfqpoint{3.386454in}{0.677305in}}{\pgfqpoint{3.380929in}{0.677305in}}%
\pgfpathcurveto{\pgfqpoint{3.375404in}{0.677305in}}{\pgfqpoint{3.370105in}{0.675110in}}{\pgfqpoint{3.366198in}{0.671203in}}%
\pgfpathcurveto{\pgfqpoint{3.362291in}{0.667296in}}{\pgfqpoint{3.360096in}{0.661997in}}{\pgfqpoint{3.360096in}{0.656471in}}%
\pgfpathcurveto{\pgfqpoint{3.360096in}{0.650946in}}{\pgfqpoint{3.362291in}{0.645647in}}{\pgfqpoint{3.366198in}{0.641740in}}%
\pgfpathcurveto{\pgfqpoint{3.370105in}{0.637833in}}{\pgfqpoint{3.375404in}{0.635638in}}{\pgfqpoint{3.380929in}{0.635638in}}%
\pgfpathclose%
\pgfusepath{stroke}%
\end{pgfscope}%
\begin{pgfscope}%
\pgfpathrectangle{\pgfqpoint{0.438556in}{0.383578in}}{\pgfqpoint{4.650000in}{2.310000in}}%
\pgfusepath{clip}%
\pgfsetbuttcap%
\pgfsetroundjoin%
\pgfsetlinewidth{0.803000pt}%
\definecolor{currentstroke}{rgb}{0.686275,0.352941,0.313725}%
\pgfsetstrokecolor{currentstroke}%
\pgfsetdash{}{0pt}%
\pgfpathmoveto{\pgfqpoint{3.451169in}{0.679322in}}%
\pgfpathcurveto{\pgfqpoint{3.456694in}{0.679322in}}{\pgfqpoint{3.461994in}{0.681518in}}{\pgfqpoint{3.465901in}{0.685424in}}%
\pgfpathcurveto{\pgfqpoint{3.469807in}{0.689331in}}{\pgfqpoint{3.472003in}{0.694631in}}{\pgfqpoint{3.472003in}{0.700156in}}%
\pgfpathcurveto{\pgfqpoint{3.472003in}{0.705681in}}{\pgfqpoint{3.469807in}{0.710980in}}{\pgfqpoint{3.465901in}{0.714887in}}%
\pgfpathcurveto{\pgfqpoint{3.461994in}{0.718794in}}{\pgfqpoint{3.456694in}{0.720989in}}{\pgfqpoint{3.451169in}{0.720989in}}%
\pgfpathcurveto{\pgfqpoint{3.445644in}{0.720989in}}{\pgfqpoint{3.440345in}{0.718794in}}{\pgfqpoint{3.436438in}{0.714887in}}%
\pgfpathcurveto{\pgfqpoint{3.432531in}{0.710980in}}{\pgfqpoint{3.430336in}{0.705681in}}{\pgfqpoint{3.430336in}{0.700156in}}%
\pgfpathcurveto{\pgfqpoint{3.430336in}{0.694631in}}{\pgfqpoint{3.432531in}{0.689331in}}{\pgfqpoint{3.436438in}{0.685424in}}%
\pgfpathcurveto{\pgfqpoint{3.440345in}{0.681518in}}{\pgfqpoint{3.445644in}{0.679322in}}{\pgfqpoint{3.451169in}{0.679322in}}%
\pgfpathclose%
\pgfusepath{stroke}%
\end{pgfscope}%
\begin{pgfscope}%
\pgfpathrectangle{\pgfqpoint{0.438556in}{0.383578in}}{\pgfqpoint{4.650000in}{2.310000in}}%
\pgfusepath{clip}%
\pgfsetbuttcap%
\pgfsetroundjoin%
\pgfsetlinewidth{0.803000pt}%
\definecolor{currentstroke}{rgb}{0.686275,0.352941,0.313725}%
\pgfsetstrokecolor{currentstroke}%
\pgfsetdash{}{0pt}%
\pgfpathmoveto{\pgfqpoint{3.410936in}{0.701165in}}%
\pgfpathcurveto{\pgfqpoint{3.416462in}{0.701165in}}{\pgfqpoint{3.421761in}{0.703360in}}{\pgfqpoint{3.425668in}{0.707266in}}%
\pgfpathcurveto{\pgfqpoint{3.429575in}{0.711173in}}{\pgfqpoint{3.431770in}{0.716473in}}{\pgfqpoint{3.431770in}{0.721998in}}%
\pgfpathcurveto{\pgfqpoint{3.431770in}{0.727523in}}{\pgfqpoint{3.429575in}{0.732822in}}{\pgfqpoint{3.425668in}{0.736729in}}%
\pgfpathcurveto{\pgfqpoint{3.421761in}{0.740636in}}{\pgfqpoint{3.416462in}{0.742831in}}{\pgfqpoint{3.410936in}{0.742831in}}%
\pgfpathcurveto{\pgfqpoint{3.405411in}{0.742831in}}{\pgfqpoint{3.400112in}{0.740636in}}{\pgfqpoint{3.396205in}{0.736729in}}%
\pgfpathcurveto{\pgfqpoint{3.392298in}{0.732822in}}{\pgfqpoint{3.390103in}{0.727523in}}{\pgfqpoint{3.390103in}{0.721998in}}%
\pgfpathcurveto{\pgfqpoint{3.390103in}{0.716473in}}{\pgfqpoint{3.392298in}{0.711173in}}{\pgfqpoint{3.396205in}{0.707266in}}%
\pgfpathcurveto{\pgfqpoint{3.400112in}{0.703360in}}{\pgfqpoint{3.405411in}{0.701165in}}{\pgfqpoint{3.410936in}{0.701165in}}%
\pgfpathclose%
\pgfusepath{stroke}%
\end{pgfscope}%
\begin{pgfscope}%
\pgfpathrectangle{\pgfqpoint{0.438556in}{0.383578in}}{\pgfqpoint{4.650000in}{2.310000in}}%
\pgfusepath{clip}%
\pgfsetbuttcap%
\pgfsetroundjoin%
\pgfsetlinewidth{0.803000pt}%
\definecolor{currentstroke}{rgb}{0.686275,0.352941,0.313725}%
\pgfsetstrokecolor{currentstroke}%
\pgfsetdash{}{0pt}%
\pgfpathmoveto{\pgfqpoint{3.457922in}{0.810375in}}%
\pgfpathcurveto{\pgfqpoint{3.463447in}{0.810375in}}{\pgfqpoint{3.468747in}{0.812570in}}{\pgfqpoint{3.472653in}{0.816477in}}%
\pgfpathcurveto{\pgfqpoint{3.476560in}{0.820384in}}{\pgfqpoint{3.478755in}{0.825683in}}{\pgfqpoint{3.478755in}{0.831209in}}%
\pgfpathcurveto{\pgfqpoint{3.478755in}{0.836734in}}{\pgfqpoint{3.476560in}{0.842033in}}{\pgfqpoint{3.472653in}{0.845940in}}%
\pgfpathcurveto{\pgfqpoint{3.468747in}{0.849847in}}{\pgfqpoint{3.463447in}{0.852042in}}{\pgfqpoint{3.457922in}{0.852042in}}%
\pgfpathcurveto{\pgfqpoint{3.452397in}{0.852042in}}{\pgfqpoint{3.447098in}{0.849847in}}{\pgfqpoint{3.443191in}{0.845940in}}%
\pgfpathcurveto{\pgfqpoint{3.439284in}{0.842033in}}{\pgfqpoint{3.437089in}{0.836734in}}{\pgfqpoint{3.437089in}{0.831209in}}%
\pgfpathcurveto{\pgfqpoint{3.437089in}{0.825683in}}{\pgfqpoint{3.439284in}{0.820384in}}{\pgfqpoint{3.443191in}{0.816477in}}%
\pgfpathcurveto{\pgfqpoint{3.447098in}{0.812570in}}{\pgfqpoint{3.452397in}{0.810375in}}{\pgfqpoint{3.457922in}{0.810375in}}%
\pgfpathclose%
\pgfusepath{stroke}%
\end{pgfscope}%
\begin{pgfscope}%
\pgfpathrectangle{\pgfqpoint{0.438556in}{0.383578in}}{\pgfqpoint{4.650000in}{2.310000in}}%
\pgfusepath{clip}%
\pgfsetbuttcap%
\pgfsetroundjoin%
\pgfsetlinewidth{0.803000pt}%
\definecolor{currentstroke}{rgb}{0.686275,0.352941,0.313725}%
\pgfsetstrokecolor{currentstroke}%
\pgfsetdash{}{0pt}%
\pgfpathmoveto{\pgfqpoint{3.407940in}{0.854059in}}%
\pgfpathcurveto{\pgfqpoint{3.413465in}{0.854059in}}{\pgfqpoint{3.418765in}{0.856255in}}{\pgfqpoint{3.422672in}{0.860161in}}%
\pgfpathcurveto{\pgfqpoint{3.426579in}{0.864068in}}{\pgfqpoint{3.428774in}{0.869368in}}{\pgfqpoint{3.428774in}{0.874893in}}%
\pgfpathcurveto{\pgfqpoint{3.428774in}{0.880418in}}{\pgfqpoint{3.426579in}{0.885717in}}{\pgfqpoint{3.422672in}{0.889624in}}%
\pgfpathcurveto{\pgfqpoint{3.418765in}{0.893531in}}{\pgfqpoint{3.413465in}{0.895726in}}{\pgfqpoint{3.407940in}{0.895726in}}%
\pgfpathcurveto{\pgfqpoint{3.402415in}{0.895726in}}{\pgfqpoint{3.397116in}{0.893531in}}{\pgfqpoint{3.393209in}{0.889624in}}%
\pgfpathcurveto{\pgfqpoint{3.389302in}{0.885717in}}{\pgfqpoint{3.387107in}{0.880418in}}{\pgfqpoint{3.387107in}{0.874893in}}%
\pgfpathcurveto{\pgfqpoint{3.387107in}{0.869368in}}{\pgfqpoint{3.389302in}{0.864068in}}{\pgfqpoint{3.393209in}{0.860161in}}%
\pgfpathcurveto{\pgfqpoint{3.397116in}{0.856255in}}{\pgfqpoint{3.402415in}{0.854059in}}{\pgfqpoint{3.407940in}{0.854059in}}%
\pgfpathclose%
\pgfusepath{stroke}%
\end{pgfscope}%
\begin{pgfscope}%
\pgfpathrectangle{\pgfqpoint{0.438556in}{0.383578in}}{\pgfqpoint{4.650000in}{2.310000in}}%
\pgfusepath{clip}%
\pgfsetbuttcap%
\pgfsetroundjoin%
\pgfsetlinewidth{0.803000pt}%
\definecolor{currentstroke}{rgb}{0.686275,0.352941,0.313725}%
\pgfsetstrokecolor{currentstroke}%
\pgfsetdash{}{0pt}%
\pgfpathmoveto{\pgfqpoint{3.458417in}{0.875902in}}%
\pgfpathcurveto{\pgfqpoint{3.463942in}{0.875902in}}{\pgfqpoint{3.469241in}{0.878097in}}{\pgfqpoint{3.473148in}{0.882004in}}%
\pgfpathcurveto{\pgfqpoint{3.477055in}{0.885910in}}{\pgfqpoint{3.479250in}{0.891210in}}{\pgfqpoint{3.479250in}{0.896735in}}%
\pgfpathcurveto{\pgfqpoint{3.479250in}{0.902260in}}{\pgfqpoint{3.477055in}{0.907560in}}{\pgfqpoint{3.473148in}{0.911466in}}%
\pgfpathcurveto{\pgfqpoint{3.469241in}{0.915373in}}{\pgfqpoint{3.463942in}{0.917568in}}{\pgfqpoint{3.458417in}{0.917568in}}%
\pgfpathcurveto{\pgfqpoint{3.452892in}{0.917568in}}{\pgfqpoint{3.447592in}{0.915373in}}{\pgfqpoint{3.443685in}{0.911466in}}%
\pgfpathcurveto{\pgfqpoint{3.439779in}{0.907560in}}{\pgfqpoint{3.437584in}{0.902260in}}{\pgfqpoint{3.437584in}{0.896735in}}%
\pgfpathcurveto{\pgfqpoint{3.437584in}{0.891210in}}{\pgfqpoint{3.439779in}{0.885910in}}{\pgfqpoint{3.443685in}{0.882004in}}%
\pgfpathcurveto{\pgfqpoint{3.447592in}{0.878097in}}{\pgfqpoint{3.452892in}{0.875902in}}{\pgfqpoint{3.458417in}{0.875902in}}%
\pgfpathclose%
\pgfusepath{stroke}%
\end{pgfscope}%
\begin{pgfscope}%
\pgfpathrectangle{\pgfqpoint{0.438556in}{0.383578in}}{\pgfqpoint{4.650000in}{2.310000in}}%
\pgfusepath{clip}%
\pgfsetbuttcap%
\pgfsetroundjoin%
\pgfsetlinewidth{0.803000pt}%
\definecolor{currentstroke}{rgb}{0.686275,0.352941,0.313725}%
\pgfsetstrokecolor{currentstroke}%
\pgfsetdash{}{0pt}%
\pgfpathmoveto{\pgfqpoint{3.380957in}{0.919586in}}%
\pgfpathcurveto{\pgfqpoint{3.386482in}{0.919586in}}{\pgfqpoint{3.391781in}{0.921781in}}{\pgfqpoint{3.395688in}{0.925688in}}%
\pgfpathcurveto{\pgfqpoint{3.399595in}{0.929595in}}{\pgfqpoint{3.401790in}{0.934894in}}{\pgfqpoint{3.401790in}{0.940419in}}%
\pgfpathcurveto{\pgfqpoint{3.401790in}{0.945944in}}{\pgfqpoint{3.399595in}{0.951244in}}{\pgfqpoint{3.395688in}{0.955151in}}%
\pgfpathcurveto{\pgfqpoint{3.391781in}{0.959057in}}{\pgfqpoint{3.386482in}{0.961253in}}{\pgfqpoint{3.380957in}{0.961253in}}%
\pgfpathcurveto{\pgfqpoint{3.375432in}{0.961253in}}{\pgfqpoint{3.370132in}{0.959057in}}{\pgfqpoint{3.366225in}{0.955151in}}%
\pgfpathcurveto{\pgfqpoint{3.362318in}{0.951244in}}{\pgfqpoint{3.360123in}{0.945944in}}{\pgfqpoint{3.360123in}{0.940419in}}%
\pgfpathcurveto{\pgfqpoint{3.360123in}{0.934894in}}{\pgfqpoint{3.362318in}{0.929595in}}{\pgfqpoint{3.366225in}{0.925688in}}%
\pgfpathcurveto{\pgfqpoint{3.370132in}{0.921781in}}{\pgfqpoint{3.375432in}{0.919586in}}{\pgfqpoint{3.380957in}{0.919586in}}%
\pgfpathclose%
\pgfusepath{stroke}%
\end{pgfscope}%
\begin{pgfscope}%
\pgfpathrectangle{\pgfqpoint{0.438556in}{0.383578in}}{\pgfqpoint{4.650000in}{2.310000in}}%
\pgfusepath{clip}%
\pgfsetbuttcap%
\pgfsetroundjoin%
\pgfsetlinewidth{0.803000pt}%
\definecolor{currentstroke}{rgb}{0.686275,0.352941,0.313725}%
\pgfsetstrokecolor{currentstroke}%
\pgfsetdash{}{0pt}%
\pgfpathmoveto{\pgfqpoint{3.386408in}{0.941428in}}%
\pgfpathcurveto{\pgfqpoint{3.391933in}{0.941428in}}{\pgfqpoint{3.397233in}{0.943623in}}{\pgfqpoint{3.401140in}{0.947530in}}%
\pgfpathcurveto{\pgfqpoint{3.405047in}{0.951437in}}{\pgfqpoint{3.407242in}{0.956736in}}{\pgfqpoint{3.407242in}{0.962261in}}%
\pgfpathcurveto{\pgfqpoint{3.407242in}{0.967786in}}{\pgfqpoint{3.405047in}{0.973086in}}{\pgfqpoint{3.401140in}{0.976993in}}%
\pgfpathcurveto{\pgfqpoint{3.397233in}{0.980900in}}{\pgfqpoint{3.391933in}{0.983095in}}{\pgfqpoint{3.386408in}{0.983095in}}%
\pgfpathcurveto{\pgfqpoint{3.380883in}{0.983095in}}{\pgfqpoint{3.375584in}{0.980900in}}{\pgfqpoint{3.371677in}{0.976993in}}%
\pgfpathcurveto{\pgfqpoint{3.367770in}{0.973086in}}{\pgfqpoint{3.365575in}{0.967786in}}{\pgfqpoint{3.365575in}{0.962261in}}%
\pgfpathcurveto{\pgfqpoint{3.365575in}{0.956736in}}{\pgfqpoint{3.367770in}{0.951437in}}{\pgfqpoint{3.371677in}{0.947530in}}%
\pgfpathcurveto{\pgfqpoint{3.375584in}{0.943623in}}{\pgfqpoint{3.380883in}{0.941428in}}{\pgfqpoint{3.386408in}{0.941428in}}%
\pgfpathclose%
\pgfusepath{stroke}%
\end{pgfscope}%
\begin{pgfscope}%
\pgfpathrectangle{\pgfqpoint{0.438556in}{0.383578in}}{\pgfqpoint{4.650000in}{2.310000in}}%
\pgfusepath{clip}%
\pgfsetbuttcap%
\pgfsetroundjoin%
\pgfsetlinewidth{0.803000pt}%
\definecolor{currentstroke}{rgb}{0.686275,0.352941,0.313725}%
\pgfsetstrokecolor{currentstroke}%
\pgfsetdash{}{0pt}%
\pgfpathmoveto{\pgfqpoint{3.437462in}{1.006954in}}%
\pgfpathcurveto{\pgfqpoint{3.442987in}{1.006954in}}{\pgfqpoint{3.448287in}{1.009150in}}{\pgfqpoint{3.452193in}{1.013056in}}%
\pgfpathcurveto{\pgfqpoint{3.456100in}{1.016963in}}{\pgfqpoint{3.458295in}{1.022263in}}{\pgfqpoint{3.458295in}{1.027788in}}%
\pgfpathcurveto{\pgfqpoint{3.458295in}{1.033313in}}{\pgfqpoint{3.456100in}{1.038612in}}{\pgfqpoint{3.452193in}{1.042519in}}%
\pgfpathcurveto{\pgfqpoint{3.448287in}{1.046426in}}{\pgfqpoint{3.442987in}{1.048621in}}{\pgfqpoint{3.437462in}{1.048621in}}%
\pgfpathcurveto{\pgfqpoint{3.431937in}{1.048621in}}{\pgfqpoint{3.426638in}{1.046426in}}{\pgfqpoint{3.422731in}{1.042519in}}%
\pgfpathcurveto{\pgfqpoint{3.418824in}{1.038612in}}{\pgfqpoint{3.416629in}{1.033313in}}{\pgfqpoint{3.416629in}{1.027788in}}%
\pgfpathcurveto{\pgfqpoint{3.416629in}{1.022263in}}{\pgfqpoint{3.418824in}{1.016963in}}{\pgfqpoint{3.422731in}{1.013056in}}%
\pgfpathcurveto{\pgfqpoint{3.426638in}{1.009150in}}{\pgfqpoint{3.431937in}{1.006954in}}{\pgfqpoint{3.437462in}{1.006954in}}%
\pgfpathclose%
\pgfusepath{stroke}%
\end{pgfscope}%
\begin{pgfscope}%
\pgfpathrectangle{\pgfqpoint{0.438556in}{0.383578in}}{\pgfqpoint{4.650000in}{2.310000in}}%
\pgfusepath{clip}%
\pgfsetbuttcap%
\pgfsetroundjoin%
\pgfsetlinewidth{0.803000pt}%
\definecolor{currentstroke}{rgb}{0.686275,0.352941,0.313725}%
\pgfsetstrokecolor{currentstroke}%
\pgfsetdash{}{0pt}%
\pgfpathmoveto{\pgfqpoint{3.402534in}{1.028797in}}%
\pgfpathcurveto{\pgfqpoint{3.408059in}{1.028797in}}{\pgfqpoint{3.413359in}{1.030992in}}{\pgfqpoint{3.417266in}{1.034898in}}%
\pgfpathcurveto{\pgfqpoint{3.421173in}{1.038805in}}{\pgfqpoint{3.423368in}{1.044105in}}{\pgfqpoint{3.423368in}{1.049630in}}%
\pgfpathcurveto{\pgfqpoint{3.423368in}{1.055155in}}{\pgfqpoint{3.421173in}{1.060454in}}{\pgfqpoint{3.417266in}{1.064361in}}%
\pgfpathcurveto{\pgfqpoint{3.413359in}{1.068268in}}{\pgfqpoint{3.408059in}{1.070463in}}{\pgfqpoint{3.402534in}{1.070463in}}%
\pgfpathcurveto{\pgfqpoint{3.397009in}{1.070463in}}{\pgfqpoint{3.391710in}{1.068268in}}{\pgfqpoint{3.387803in}{1.064361in}}%
\pgfpathcurveto{\pgfqpoint{3.383896in}{1.060454in}}{\pgfqpoint{3.381701in}{1.055155in}}{\pgfqpoint{3.381701in}{1.049630in}}%
\pgfpathcurveto{\pgfqpoint{3.381701in}{1.044105in}}{\pgfqpoint{3.383896in}{1.038805in}}{\pgfqpoint{3.387803in}{1.034898in}}%
\pgfpathcurveto{\pgfqpoint{3.391710in}{1.030992in}}{\pgfqpoint{3.397009in}{1.028797in}}{\pgfqpoint{3.402534in}{1.028797in}}%
\pgfpathclose%
\pgfusepath{stroke}%
\end{pgfscope}%
\begin{pgfscope}%
\pgfpathrectangle{\pgfqpoint{0.438556in}{0.383578in}}{\pgfqpoint{4.650000in}{2.310000in}}%
\pgfusepath{clip}%
\pgfsetbuttcap%
\pgfsetroundjoin%
\pgfsetlinewidth{0.803000pt}%
\definecolor{currentstroke}{rgb}{0.686275,0.352941,0.313725}%
\pgfsetstrokecolor{currentstroke}%
\pgfsetdash{}{0pt}%
\pgfpathmoveto{\pgfqpoint{3.404129in}{1.072481in}}%
\pgfpathcurveto{\pgfqpoint{3.409654in}{1.072481in}}{\pgfqpoint{3.414953in}{1.074676in}}{\pgfqpoint{3.418860in}{1.078583in}}%
\pgfpathcurveto{\pgfqpoint{3.422767in}{1.082490in}}{\pgfqpoint{3.424962in}{1.087789in}}{\pgfqpoint{3.424962in}{1.093314in}}%
\pgfpathcurveto{\pgfqpoint{3.424962in}{1.098839in}}{\pgfqpoint{3.422767in}{1.104139in}}{\pgfqpoint{3.418860in}{1.108045in}}%
\pgfpathcurveto{\pgfqpoint{3.414953in}{1.111952in}}{\pgfqpoint{3.409654in}{1.114147in}}{\pgfqpoint{3.404129in}{1.114147in}}%
\pgfpathcurveto{\pgfqpoint{3.398604in}{1.114147in}}{\pgfqpoint{3.393304in}{1.111952in}}{\pgfqpoint{3.389397in}{1.108045in}}%
\pgfpathcurveto{\pgfqpoint{3.385490in}{1.104139in}}{\pgfqpoint{3.383295in}{1.098839in}}{\pgfqpoint{3.383295in}{1.093314in}}%
\pgfpathcurveto{\pgfqpoint{3.383295in}{1.087789in}}{\pgfqpoint{3.385490in}{1.082490in}}{\pgfqpoint{3.389397in}{1.078583in}}%
\pgfpathcurveto{\pgfqpoint{3.393304in}{1.074676in}}{\pgfqpoint{3.398604in}{1.072481in}}{\pgfqpoint{3.404129in}{1.072481in}}%
\pgfpathclose%
\pgfusepath{stroke}%
\end{pgfscope}%
\begin{pgfscope}%
\pgfpathrectangle{\pgfqpoint{0.438556in}{0.383578in}}{\pgfqpoint{4.650000in}{2.310000in}}%
\pgfusepath{clip}%
\pgfsetbuttcap%
\pgfsetroundjoin%
\pgfsetlinewidth{0.803000pt}%
\definecolor{currentstroke}{rgb}{0.686275,0.352941,0.313725}%
\pgfsetstrokecolor{currentstroke}%
\pgfsetdash{}{0pt}%
\pgfpathmoveto{\pgfqpoint{3.453222in}{1.159849in}}%
\pgfpathcurveto{\pgfqpoint{3.458747in}{1.159849in}}{\pgfqpoint{3.464046in}{1.162044in}}{\pgfqpoint{3.467953in}{1.165951in}}%
\pgfpathcurveto{\pgfqpoint{3.471860in}{1.169858in}}{\pgfqpoint{3.474055in}{1.175158in}}{\pgfqpoint{3.474055in}{1.180683in}}%
\pgfpathcurveto{\pgfqpoint{3.474055in}{1.186208in}}{\pgfqpoint{3.471860in}{1.191507in}}{\pgfqpoint{3.467953in}{1.195414in}}%
\pgfpathcurveto{\pgfqpoint{3.464046in}{1.199321in}}{\pgfqpoint{3.458747in}{1.201516in}}{\pgfqpoint{3.453222in}{1.201516in}}%
\pgfpathcurveto{\pgfqpoint{3.447697in}{1.201516in}}{\pgfqpoint{3.442397in}{1.199321in}}{\pgfqpoint{3.438490in}{1.195414in}}%
\pgfpathcurveto{\pgfqpoint{3.434584in}{1.191507in}}{\pgfqpoint{3.432388in}{1.186208in}}{\pgfqpoint{3.432388in}{1.180683in}}%
\pgfpathcurveto{\pgfqpoint{3.432388in}{1.175158in}}{\pgfqpoint{3.434584in}{1.169858in}}{\pgfqpoint{3.438490in}{1.165951in}}%
\pgfpathcurveto{\pgfqpoint{3.442397in}{1.162044in}}{\pgfqpoint{3.447697in}{1.159849in}}{\pgfqpoint{3.453222in}{1.159849in}}%
\pgfpathclose%
\pgfusepath{stroke}%
\end{pgfscope}%
\begin{pgfscope}%
\pgfpathrectangle{\pgfqpoint{0.438556in}{0.383578in}}{\pgfqpoint{4.650000in}{2.310000in}}%
\pgfusepath{clip}%
\pgfsetbuttcap%
\pgfsetroundjoin%
\pgfsetlinewidth{0.803000pt}%
\definecolor{currentstroke}{rgb}{0.686275,0.352941,0.313725}%
\pgfsetstrokecolor{currentstroke}%
\pgfsetdash{}{0pt}%
\pgfpathmoveto{\pgfqpoint{3.399639in}{1.203534in}}%
\pgfpathcurveto{\pgfqpoint{3.405164in}{1.203534in}}{\pgfqpoint{3.410464in}{1.205729in}}{\pgfqpoint{3.414370in}{1.209636in}}%
\pgfpathcurveto{\pgfqpoint{3.418277in}{1.213542in}}{\pgfqpoint{3.420472in}{1.218842in}}{\pgfqpoint{3.420472in}{1.224367in}}%
\pgfpathcurveto{\pgfqpoint{3.420472in}{1.229892in}}{\pgfqpoint{3.418277in}{1.235191in}}{\pgfqpoint{3.414370in}{1.239098in}}%
\pgfpathcurveto{\pgfqpoint{3.410464in}{1.243005in}}{\pgfqpoint{3.405164in}{1.245200in}}{\pgfqpoint{3.399639in}{1.245200in}}%
\pgfpathcurveto{\pgfqpoint{3.394114in}{1.245200in}}{\pgfqpoint{3.388814in}{1.243005in}}{\pgfqpoint{3.384908in}{1.239098in}}%
\pgfpathcurveto{\pgfqpoint{3.381001in}{1.235191in}}{\pgfqpoint{3.378806in}{1.229892in}}{\pgfqpoint{3.378806in}{1.224367in}}%
\pgfpathcurveto{\pgfqpoint{3.378806in}{1.218842in}}{\pgfqpoint{3.381001in}{1.213542in}}{\pgfqpoint{3.384908in}{1.209636in}}%
\pgfpathcurveto{\pgfqpoint{3.388814in}{1.205729in}}{\pgfqpoint{3.394114in}{1.203534in}}{\pgfqpoint{3.399639in}{1.203534in}}%
\pgfpathclose%
\pgfusepath{stroke}%
\end{pgfscope}%
\begin{pgfscope}%
\pgfpathrectangle{\pgfqpoint{0.438556in}{0.383578in}}{\pgfqpoint{4.650000in}{2.310000in}}%
\pgfusepath{clip}%
\pgfsetbuttcap%
\pgfsetroundjoin%
\pgfsetlinewidth{0.803000pt}%
\definecolor{currentstroke}{rgb}{0.686275,0.352941,0.313725}%
\pgfsetstrokecolor{currentstroke}%
\pgfsetdash{}{0pt}%
\pgfpathmoveto{\pgfqpoint{3.393482in}{1.290902in}}%
\pgfpathcurveto{\pgfqpoint{3.399007in}{1.290902in}}{\pgfqpoint{3.404306in}{1.293097in}}{\pgfqpoint{3.408213in}{1.297004in}}%
\pgfpathcurveto{\pgfqpoint{3.412120in}{1.300911in}}{\pgfqpoint{3.414315in}{1.306210in}}{\pgfqpoint{3.414315in}{1.311735in}}%
\pgfpathcurveto{\pgfqpoint{3.414315in}{1.317260in}}{\pgfqpoint{3.412120in}{1.322560in}}{\pgfqpoint{3.408213in}{1.326467in}}%
\pgfpathcurveto{\pgfqpoint{3.404306in}{1.330374in}}{\pgfqpoint{3.399007in}{1.332569in}}{\pgfqpoint{3.393482in}{1.332569in}}%
\pgfpathcurveto{\pgfqpoint{3.387957in}{1.332569in}}{\pgfqpoint{3.382657in}{1.330374in}}{\pgfqpoint{3.378750in}{1.326467in}}%
\pgfpathcurveto{\pgfqpoint{3.374844in}{1.322560in}}{\pgfqpoint{3.372648in}{1.317260in}}{\pgfqpoint{3.372648in}{1.311735in}}%
\pgfpathcurveto{\pgfqpoint{3.372648in}{1.306210in}}{\pgfqpoint{3.374844in}{1.300911in}}{\pgfqpoint{3.378750in}{1.297004in}}%
\pgfpathcurveto{\pgfqpoint{3.382657in}{1.293097in}}{\pgfqpoint{3.387957in}{1.290902in}}{\pgfqpoint{3.393482in}{1.290902in}}%
\pgfpathclose%
\pgfusepath{stroke}%
\end{pgfscope}%
\begin{pgfscope}%
\pgfpathrectangle{\pgfqpoint{0.438556in}{0.383578in}}{\pgfqpoint{4.650000in}{2.310000in}}%
\pgfusepath{clip}%
\pgfsetbuttcap%
\pgfsetroundjoin%
\pgfsetlinewidth{0.803000pt}%
\definecolor{currentstroke}{rgb}{0.686275,0.352941,0.313725}%
\pgfsetstrokecolor{currentstroke}%
\pgfsetdash{}{0pt}%
\pgfpathmoveto{\pgfqpoint{3.374717in}{1.312744in}}%
\pgfpathcurveto{\pgfqpoint{3.380242in}{1.312744in}}{\pgfqpoint{3.385541in}{1.314939in}}{\pgfqpoint{3.389448in}{1.318846in}}%
\pgfpathcurveto{\pgfqpoint{3.393355in}{1.322753in}}{\pgfqpoint{3.395550in}{1.328052in}}{\pgfqpoint{3.395550in}{1.333578in}}%
\pgfpathcurveto{\pgfqpoint{3.395550in}{1.339103in}}{\pgfqpoint{3.393355in}{1.344402in}}{\pgfqpoint{3.389448in}{1.348309in}}%
\pgfpathcurveto{\pgfqpoint{3.385541in}{1.352216in}}{\pgfqpoint{3.380242in}{1.354411in}}{\pgfqpoint{3.374717in}{1.354411in}}%
\pgfpathcurveto{\pgfqpoint{3.369192in}{1.354411in}}{\pgfqpoint{3.363892in}{1.352216in}}{\pgfqpoint{3.359985in}{1.348309in}}%
\pgfpathcurveto{\pgfqpoint{3.356079in}{1.344402in}}{\pgfqpoint{3.353884in}{1.339103in}}{\pgfqpoint{3.353884in}{1.333578in}}%
\pgfpathcurveto{\pgfqpoint{3.353884in}{1.328052in}}{\pgfqpoint{3.356079in}{1.322753in}}{\pgfqpoint{3.359985in}{1.318846in}}%
\pgfpathcurveto{\pgfqpoint{3.363892in}{1.314939in}}{\pgfqpoint{3.369192in}{1.312744in}}{\pgfqpoint{3.374717in}{1.312744in}}%
\pgfpathclose%
\pgfusepath{stroke}%
\end{pgfscope}%
\begin{pgfscope}%
\pgfpathrectangle{\pgfqpoint{0.438556in}{0.383578in}}{\pgfqpoint{4.650000in}{2.310000in}}%
\pgfusepath{clip}%
\pgfsetbuttcap%
\pgfsetroundjoin%
\pgfsetlinewidth{0.803000pt}%
\definecolor{currentstroke}{rgb}{0.686275,0.352941,0.313725}%
\pgfsetstrokecolor{currentstroke}%
\pgfsetdash{}{0pt}%
\pgfpathmoveto{\pgfqpoint{3.447596in}{1.334586in}}%
\pgfpathcurveto{\pgfqpoint{3.453121in}{1.334586in}}{\pgfqpoint{3.458420in}{1.336781in}}{\pgfqpoint{3.462327in}{1.340688in}}%
\pgfpathcurveto{\pgfqpoint{3.466234in}{1.344595in}}{\pgfqpoint{3.468429in}{1.349895in}}{\pgfqpoint{3.468429in}{1.355420in}}%
\pgfpathcurveto{\pgfqpoint{3.468429in}{1.360945in}}{\pgfqpoint{3.466234in}{1.366244in}}{\pgfqpoint{3.462327in}{1.370151in}}%
\pgfpathcurveto{\pgfqpoint{3.458420in}{1.374058in}}{\pgfqpoint{3.453121in}{1.376253in}}{\pgfqpoint{3.447596in}{1.376253in}}%
\pgfpathcurveto{\pgfqpoint{3.442071in}{1.376253in}}{\pgfqpoint{3.436771in}{1.374058in}}{\pgfqpoint{3.432865in}{1.370151in}}%
\pgfpathcurveto{\pgfqpoint{3.428958in}{1.366244in}}{\pgfqpoint{3.426763in}{1.360945in}}{\pgfqpoint{3.426763in}{1.355420in}}%
\pgfpathcurveto{\pgfqpoint{3.426763in}{1.349895in}}{\pgfqpoint{3.428958in}{1.344595in}}{\pgfqpoint{3.432865in}{1.340688in}}%
\pgfpathcurveto{\pgfqpoint{3.436771in}{1.336781in}}{\pgfqpoint{3.442071in}{1.334586in}}{\pgfqpoint{3.447596in}{1.334586in}}%
\pgfpathclose%
\pgfusepath{stroke}%
\end{pgfscope}%
\begin{pgfscope}%
\pgfpathrectangle{\pgfqpoint{0.438556in}{0.383578in}}{\pgfqpoint{4.650000in}{2.310000in}}%
\pgfusepath{clip}%
\pgfsetbuttcap%
\pgfsetroundjoin%
\pgfsetlinewidth{0.803000pt}%
\definecolor{currentstroke}{rgb}{0.686275,0.352941,0.313725}%
\pgfsetstrokecolor{currentstroke}%
\pgfsetdash{}{0pt}%
\pgfpathmoveto{\pgfqpoint{3.395892in}{1.443797in}}%
\pgfpathcurveto{\pgfqpoint{3.401417in}{1.443797in}}{\pgfqpoint{3.406716in}{1.445992in}}{\pgfqpoint{3.410623in}{1.449899in}}%
\pgfpathcurveto{\pgfqpoint{3.414530in}{1.453806in}}{\pgfqpoint{3.416725in}{1.459105in}}{\pgfqpoint{3.416725in}{1.464630in}}%
\pgfpathcurveto{\pgfqpoint{3.416725in}{1.470155in}}{\pgfqpoint{3.414530in}{1.475455in}}{\pgfqpoint{3.410623in}{1.479362in}}%
\pgfpathcurveto{\pgfqpoint{3.406716in}{1.483269in}}{\pgfqpoint{3.401417in}{1.485464in}}{\pgfqpoint{3.395892in}{1.485464in}}%
\pgfpathcurveto{\pgfqpoint{3.390366in}{1.485464in}}{\pgfqpoint{3.385067in}{1.483269in}}{\pgfqpoint{3.381160in}{1.479362in}}%
\pgfpathcurveto{\pgfqpoint{3.377253in}{1.475455in}}{\pgfqpoint{3.375058in}{1.470155in}}{\pgfqpoint{3.375058in}{1.464630in}}%
\pgfpathcurveto{\pgfqpoint{3.375058in}{1.459105in}}{\pgfqpoint{3.377253in}{1.453806in}}{\pgfqpoint{3.381160in}{1.449899in}}%
\pgfpathcurveto{\pgfqpoint{3.385067in}{1.445992in}}{\pgfqpoint{3.390366in}{1.443797in}}{\pgfqpoint{3.395892in}{1.443797in}}%
\pgfpathclose%
\pgfusepath{stroke}%
\end{pgfscope}%
\begin{pgfscope}%
\pgfpathrectangle{\pgfqpoint{0.438556in}{0.383578in}}{\pgfqpoint{4.650000in}{2.310000in}}%
\pgfusepath{clip}%
\pgfsetbuttcap%
\pgfsetroundjoin%
\pgfsetlinewidth{0.803000pt}%
\definecolor{currentstroke}{rgb}{0.686275,0.352941,0.313725}%
\pgfsetstrokecolor{currentstroke}%
\pgfsetdash{}{0pt}%
\pgfpathmoveto{\pgfqpoint{3.374992in}{1.487481in}}%
\pgfpathcurveto{\pgfqpoint{3.380517in}{1.487481in}}{\pgfqpoint{3.385816in}{1.489676in}}{\pgfqpoint{3.389723in}{1.493583in}}%
\pgfpathcurveto{\pgfqpoint{3.393630in}{1.497490in}}{\pgfqpoint{3.395825in}{1.502790in}}{\pgfqpoint{3.395825in}{1.508315in}}%
\pgfpathcurveto{\pgfqpoint{3.395825in}{1.513840in}}{\pgfqpoint{3.393630in}{1.519139in}}{\pgfqpoint{3.389723in}{1.523046in}}%
\pgfpathcurveto{\pgfqpoint{3.385816in}{1.526953in}}{\pgfqpoint{3.380517in}{1.529148in}}{\pgfqpoint{3.374992in}{1.529148in}}%
\pgfpathcurveto{\pgfqpoint{3.369467in}{1.529148in}}{\pgfqpoint{3.364167in}{1.526953in}}{\pgfqpoint{3.360260in}{1.523046in}}%
\pgfpathcurveto{\pgfqpoint{3.356354in}{1.519139in}}{\pgfqpoint{3.354158in}{1.513840in}}{\pgfqpoint{3.354158in}{1.508315in}}%
\pgfpathcurveto{\pgfqpoint{3.354158in}{1.502790in}}{\pgfqpoint{3.356354in}{1.497490in}}{\pgfqpoint{3.360260in}{1.493583in}}%
\pgfpathcurveto{\pgfqpoint{3.364167in}{1.489676in}}{\pgfqpoint{3.369467in}{1.487481in}}{\pgfqpoint{3.374992in}{1.487481in}}%
\pgfpathclose%
\pgfusepath{stroke}%
\end{pgfscope}%
\begin{pgfscope}%
\pgfpathrectangle{\pgfqpoint{0.438556in}{0.383578in}}{\pgfqpoint{4.650000in}{2.310000in}}%
\pgfusepath{clip}%
\pgfsetbuttcap%
\pgfsetroundjoin%
\pgfsetlinewidth{0.803000pt}%
\definecolor{currentstroke}{rgb}{0.686275,0.352941,0.313725}%
\pgfsetstrokecolor{currentstroke}%
\pgfsetdash{}{0pt}%
\pgfpathmoveto{\pgfqpoint{3.388562in}{1.531166in}}%
\pgfpathcurveto{\pgfqpoint{3.394087in}{1.531166in}}{\pgfqpoint{3.399386in}{1.533361in}}{\pgfqpoint{3.403293in}{1.537267in}}%
\pgfpathcurveto{\pgfqpoint{3.407200in}{1.541174in}}{\pgfqpoint{3.409395in}{1.546474in}}{\pgfqpoint{3.409395in}{1.551999in}}%
\pgfpathcurveto{\pgfqpoint{3.409395in}{1.557524in}}{\pgfqpoint{3.407200in}{1.562823in}}{\pgfqpoint{3.403293in}{1.566730in}}%
\pgfpathcurveto{\pgfqpoint{3.399386in}{1.570637in}}{\pgfqpoint{3.394087in}{1.572832in}}{\pgfqpoint{3.388562in}{1.572832in}}%
\pgfpathcurveto{\pgfqpoint{3.383036in}{1.572832in}}{\pgfqpoint{3.377737in}{1.570637in}}{\pgfqpoint{3.373830in}{1.566730in}}%
\pgfpathcurveto{\pgfqpoint{3.369923in}{1.562823in}}{\pgfqpoint{3.367728in}{1.557524in}}{\pgfqpoint{3.367728in}{1.551999in}}%
\pgfpathcurveto{\pgfqpoint{3.367728in}{1.546474in}}{\pgfqpoint{3.369923in}{1.541174in}}{\pgfqpoint{3.373830in}{1.537267in}}%
\pgfpathcurveto{\pgfqpoint{3.377737in}{1.533361in}}{\pgfqpoint{3.383036in}{1.531166in}}{\pgfqpoint{3.388562in}{1.531166in}}%
\pgfpathclose%
\pgfusepath{stroke}%
\end{pgfscope}%
\begin{pgfscope}%
\pgfpathrectangle{\pgfqpoint{0.438556in}{0.383578in}}{\pgfqpoint{4.650000in}{2.310000in}}%
\pgfusepath{clip}%
\pgfsetbuttcap%
\pgfsetroundjoin%
\pgfsetlinewidth{0.803000pt}%
\definecolor{currentstroke}{rgb}{0.686275,0.352941,0.313725}%
\pgfsetstrokecolor{currentstroke}%
\pgfsetdash{}{0pt}%
\pgfpathmoveto{\pgfqpoint{3.440944in}{1.684060in}}%
\pgfpathcurveto{\pgfqpoint{3.446469in}{1.684060in}}{\pgfqpoint{3.451768in}{1.686256in}}{\pgfqpoint{3.455675in}{1.690162in}}%
\pgfpathcurveto{\pgfqpoint{3.459582in}{1.694069in}}{\pgfqpoint{3.461777in}{1.699369in}}{\pgfqpoint{3.461777in}{1.704894in}}%
\pgfpathcurveto{\pgfqpoint{3.461777in}{1.710419in}}{\pgfqpoint{3.459582in}{1.715718in}}{\pgfqpoint{3.455675in}{1.719625in}}%
\pgfpathcurveto{\pgfqpoint{3.451768in}{1.723532in}}{\pgfqpoint{3.446469in}{1.725727in}}{\pgfqpoint{3.440944in}{1.725727in}}%
\pgfpathcurveto{\pgfqpoint{3.435419in}{1.725727in}}{\pgfqpoint{3.430119in}{1.723532in}}{\pgfqpoint{3.426212in}{1.719625in}}%
\pgfpathcurveto{\pgfqpoint{3.422306in}{1.715718in}}{\pgfqpoint{3.420111in}{1.710419in}}{\pgfqpoint{3.420111in}{1.704894in}}%
\pgfpathcurveto{\pgfqpoint{3.420111in}{1.699369in}}{\pgfqpoint{3.422306in}{1.694069in}}{\pgfqpoint{3.426212in}{1.690162in}}%
\pgfpathcurveto{\pgfqpoint{3.430119in}{1.686256in}}{\pgfqpoint{3.435419in}{1.684060in}}{\pgfqpoint{3.440944in}{1.684060in}}%
\pgfpathclose%
\pgfusepath{stroke}%
\end{pgfscope}%
\begin{pgfscope}%
\pgfpathrectangle{\pgfqpoint{0.438556in}{0.383578in}}{\pgfqpoint{4.650000in}{2.310000in}}%
\pgfusepath{clip}%
\pgfsetbuttcap%
\pgfsetroundjoin%
\pgfsetlinewidth{0.803000pt}%
\definecolor{currentstroke}{rgb}{0.686275,0.352941,0.313725}%
\pgfsetstrokecolor{currentstroke}%
\pgfsetdash{}{0pt}%
\pgfpathmoveto{\pgfqpoint{3.395525in}{1.771429in}}%
\pgfpathcurveto{\pgfqpoint{3.401050in}{1.771429in}}{\pgfqpoint{3.406350in}{1.773624in}}{\pgfqpoint{3.410256in}{1.777531in}}%
\pgfpathcurveto{\pgfqpoint{3.414163in}{1.781438in}}{\pgfqpoint{3.416358in}{1.786737in}}{\pgfqpoint{3.416358in}{1.792262in}}%
\pgfpathcurveto{\pgfqpoint{3.416358in}{1.797787in}}{\pgfqpoint{3.414163in}{1.803087in}}{\pgfqpoint{3.410256in}{1.806994in}}%
\pgfpathcurveto{\pgfqpoint{3.406350in}{1.810901in}}{\pgfqpoint{3.401050in}{1.813096in}}{\pgfqpoint{3.395525in}{1.813096in}}%
\pgfpathcurveto{\pgfqpoint{3.390000in}{1.813096in}}{\pgfqpoint{3.384700in}{1.810901in}}{\pgfqpoint{3.380794in}{1.806994in}}%
\pgfpathcurveto{\pgfqpoint{3.376887in}{1.803087in}}{\pgfqpoint{3.374692in}{1.797787in}}{\pgfqpoint{3.374692in}{1.792262in}}%
\pgfpathcurveto{\pgfqpoint{3.374692in}{1.786737in}}{\pgfqpoint{3.376887in}{1.781438in}}{\pgfqpoint{3.380794in}{1.777531in}}%
\pgfpathcurveto{\pgfqpoint{3.384700in}{1.773624in}}{\pgfqpoint{3.390000in}{1.771429in}}{\pgfqpoint{3.395525in}{1.771429in}}%
\pgfpathclose%
\pgfusepath{stroke}%
\end{pgfscope}%
\begin{pgfscope}%
\pgfpathrectangle{\pgfqpoint{0.438556in}{0.383578in}}{\pgfqpoint{4.650000in}{2.310000in}}%
\pgfusepath{clip}%
\pgfsetbuttcap%
\pgfsetroundjoin%
\pgfsetlinewidth{0.803000pt}%
\definecolor{currentstroke}{rgb}{0.686275,0.352941,0.313725}%
\pgfsetstrokecolor{currentstroke}%
\pgfsetdash{}{0pt}%
\pgfpathmoveto{\pgfqpoint{3.380874in}{1.793271in}}%
\pgfpathcurveto{\pgfqpoint{3.386399in}{1.793271in}}{\pgfqpoint{3.391699in}{1.795466in}}{\pgfqpoint{3.395606in}{1.799373in}}%
\pgfpathcurveto{\pgfqpoint{3.399512in}{1.803280in}}{\pgfqpoint{3.401707in}{1.808579in}}{\pgfqpoint{3.401707in}{1.814104in}}%
\pgfpathcurveto{\pgfqpoint{3.401707in}{1.819630in}}{\pgfqpoint{3.399512in}{1.824929in}}{\pgfqpoint{3.395606in}{1.828836in}}%
\pgfpathcurveto{\pgfqpoint{3.391699in}{1.832743in}}{\pgfqpoint{3.386399in}{1.834938in}}{\pgfqpoint{3.380874in}{1.834938in}}%
\pgfpathcurveto{\pgfqpoint{3.375349in}{1.834938in}}{\pgfqpoint{3.370050in}{1.832743in}}{\pgfqpoint{3.366143in}{1.828836in}}%
\pgfpathcurveto{\pgfqpoint{3.362236in}{1.824929in}}{\pgfqpoint{3.360041in}{1.819630in}}{\pgfqpoint{3.360041in}{1.814104in}}%
\pgfpathcurveto{\pgfqpoint{3.360041in}{1.808579in}}{\pgfqpoint{3.362236in}{1.803280in}}{\pgfqpoint{3.366143in}{1.799373in}}%
\pgfpathcurveto{\pgfqpoint{3.370050in}{1.795466in}}{\pgfqpoint{3.375349in}{1.793271in}}{\pgfqpoint{3.380874in}{1.793271in}}%
\pgfpathclose%
\pgfusepath{stroke}%
\end{pgfscope}%
\begin{pgfscope}%
\pgfpathrectangle{\pgfqpoint{0.438556in}{0.383578in}}{\pgfqpoint{4.650000in}{2.310000in}}%
\pgfusepath{clip}%
\pgfsetbuttcap%
\pgfsetroundjoin%
\pgfsetlinewidth{0.803000pt}%
\definecolor{currentstroke}{rgb}{0.686275,0.352941,0.313725}%
\pgfsetstrokecolor{currentstroke}%
\pgfsetdash{}{0pt}%
\pgfpathmoveto{\pgfqpoint{3.416095in}{1.836955in}}%
\pgfpathcurveto{\pgfqpoint{3.421620in}{1.836955in}}{\pgfqpoint{3.426920in}{1.839151in}}{\pgfqpoint{3.430826in}{1.843057in}}%
\pgfpathcurveto{\pgfqpoint{3.434733in}{1.846964in}}{\pgfqpoint{3.436928in}{1.852264in}}{\pgfqpoint{3.436928in}{1.857789in}}%
\pgfpathcurveto{\pgfqpoint{3.436928in}{1.863314in}}{\pgfqpoint{3.434733in}{1.868613in}}{\pgfqpoint{3.430826in}{1.872520in}}%
\pgfpathcurveto{\pgfqpoint{3.426920in}{1.876427in}}{\pgfqpoint{3.421620in}{1.878622in}}{\pgfqpoint{3.416095in}{1.878622in}}%
\pgfpathcurveto{\pgfqpoint{3.410570in}{1.878622in}}{\pgfqpoint{3.405270in}{1.876427in}}{\pgfqpoint{3.401364in}{1.872520in}}%
\pgfpathcurveto{\pgfqpoint{3.397457in}{1.868613in}}{\pgfqpoint{3.395262in}{1.863314in}}{\pgfqpoint{3.395262in}{1.857789in}}%
\pgfpathcurveto{\pgfqpoint{3.395262in}{1.852264in}}{\pgfqpoint{3.397457in}{1.846964in}}{\pgfqpoint{3.401364in}{1.843057in}}%
\pgfpathcurveto{\pgfqpoint{3.405270in}{1.839151in}}{\pgfqpoint{3.410570in}{1.836955in}}{\pgfqpoint{3.416095in}{1.836955in}}%
\pgfpathclose%
\pgfusepath{stroke}%
\end{pgfscope}%
\begin{pgfscope}%
\pgfpathrectangle{\pgfqpoint{0.438556in}{0.383578in}}{\pgfqpoint{4.650000in}{2.310000in}}%
\pgfusepath{clip}%
\pgfsetbuttcap%
\pgfsetroundjoin%
\pgfsetlinewidth{0.803000pt}%
\definecolor{currentstroke}{rgb}{0.686275,0.352941,0.313725}%
\pgfsetstrokecolor{currentstroke}%
\pgfsetdash{}{0pt}%
\pgfpathmoveto{\pgfqpoint{3.420401in}{1.858798in}}%
\pgfpathcurveto{\pgfqpoint{3.425926in}{1.858798in}}{\pgfqpoint{3.431226in}{1.860993in}}{\pgfqpoint{3.435133in}{1.864899in}}%
\pgfpathcurveto{\pgfqpoint{3.439040in}{1.868806in}}{\pgfqpoint{3.441235in}{1.874106in}}{\pgfqpoint{3.441235in}{1.879631in}}%
\pgfpathcurveto{\pgfqpoint{3.441235in}{1.885156in}}{\pgfqpoint{3.439040in}{1.890455in}}{\pgfqpoint{3.435133in}{1.894362in}}%
\pgfpathcurveto{\pgfqpoint{3.431226in}{1.898269in}}{\pgfqpoint{3.425926in}{1.900464in}}{\pgfqpoint{3.420401in}{1.900464in}}%
\pgfpathcurveto{\pgfqpoint{3.414876in}{1.900464in}}{\pgfqpoint{3.409577in}{1.898269in}}{\pgfqpoint{3.405670in}{1.894362in}}%
\pgfpathcurveto{\pgfqpoint{3.401763in}{1.890455in}}{\pgfqpoint{3.399568in}{1.885156in}}{\pgfqpoint{3.399568in}{1.879631in}}%
\pgfpathcurveto{\pgfqpoint{3.399568in}{1.874106in}}{\pgfqpoint{3.401763in}{1.868806in}}{\pgfqpoint{3.405670in}{1.864899in}}%
\pgfpathcurveto{\pgfqpoint{3.409577in}{1.860993in}}{\pgfqpoint{3.414876in}{1.858798in}}{\pgfqpoint{3.420401in}{1.858798in}}%
\pgfpathclose%
\pgfusepath{stroke}%
\end{pgfscope}%
\begin{pgfscope}%
\pgfpathrectangle{\pgfqpoint{0.438556in}{0.383578in}}{\pgfqpoint{4.650000in}{2.310000in}}%
\pgfusepath{clip}%
\pgfsetbuttcap%
\pgfsetroundjoin%
\pgfsetlinewidth{0.803000pt}%
\definecolor{currentstroke}{rgb}{0.686275,0.352941,0.313725}%
\pgfsetstrokecolor{currentstroke}%
\pgfsetdash{}{0pt}%
\pgfpathmoveto{\pgfqpoint{3.411065in}{1.924324in}}%
\pgfpathcurveto{\pgfqpoint{3.416590in}{1.924324in}}{\pgfqpoint{3.421889in}{1.926519in}}{\pgfqpoint{3.425796in}{1.930426in}}%
\pgfpathcurveto{\pgfqpoint{3.429703in}{1.934333in}}{\pgfqpoint{3.431898in}{1.939632in}}{\pgfqpoint{3.431898in}{1.945157in}}%
\pgfpathcurveto{\pgfqpoint{3.431898in}{1.950682in}}{\pgfqpoint{3.429703in}{1.955982in}}{\pgfqpoint{3.425796in}{1.959889in}}%
\pgfpathcurveto{\pgfqpoint{3.421889in}{1.963795in}}{\pgfqpoint{3.416590in}{1.965991in}}{\pgfqpoint{3.411065in}{1.965991in}}%
\pgfpathcurveto{\pgfqpoint{3.405540in}{1.965991in}}{\pgfqpoint{3.400240in}{1.963795in}}{\pgfqpoint{3.396333in}{1.959889in}}%
\pgfpathcurveto{\pgfqpoint{3.392427in}{1.955982in}}{\pgfqpoint{3.390231in}{1.950682in}}{\pgfqpoint{3.390231in}{1.945157in}}%
\pgfpathcurveto{\pgfqpoint{3.390231in}{1.939632in}}{\pgfqpoint{3.392427in}{1.934333in}}{\pgfqpoint{3.396333in}{1.930426in}}%
\pgfpathcurveto{\pgfqpoint{3.400240in}{1.926519in}}{\pgfqpoint{3.405540in}{1.924324in}}{\pgfqpoint{3.411065in}{1.924324in}}%
\pgfpathclose%
\pgfusepath{stroke}%
\end{pgfscope}%
\begin{pgfscope}%
\pgfpathrectangle{\pgfqpoint{0.438556in}{0.383578in}}{\pgfqpoint{4.650000in}{2.310000in}}%
\pgfusepath{clip}%
\pgfsetbuttcap%
\pgfsetroundjoin%
\pgfsetlinewidth{0.803000pt}%
\definecolor{currentstroke}{rgb}{0.686275,0.352941,0.313725}%
\pgfsetstrokecolor{currentstroke}%
\pgfsetdash{}{0pt}%
\pgfpathmoveto{\pgfqpoint{3.388241in}{1.989850in}}%
\pgfpathcurveto{\pgfqpoint{3.393766in}{1.989850in}}{\pgfqpoint{3.399065in}{1.992045in}}{\pgfqpoint{3.402972in}{1.995952in}}%
\pgfpathcurveto{\pgfqpoint{3.406879in}{1.999859in}}{\pgfqpoint{3.409074in}{2.005159in}}{\pgfqpoint{3.409074in}{2.010684in}}%
\pgfpathcurveto{\pgfqpoint{3.409074in}{2.016209in}}{\pgfqpoint{3.406879in}{2.021508in}}{\pgfqpoint{3.402972in}{2.025415in}}%
\pgfpathcurveto{\pgfqpoint{3.399065in}{2.029322in}}{\pgfqpoint{3.393766in}{2.031517in}}{\pgfqpoint{3.388241in}{2.031517in}}%
\pgfpathcurveto{\pgfqpoint{3.382716in}{2.031517in}}{\pgfqpoint{3.377416in}{2.029322in}}{\pgfqpoint{3.373509in}{2.025415in}}%
\pgfpathcurveto{\pgfqpoint{3.369603in}{2.021508in}}{\pgfqpoint{3.367407in}{2.016209in}}{\pgfqpoint{3.367407in}{2.010684in}}%
\pgfpathcurveto{\pgfqpoint{3.367407in}{2.005159in}}{\pgfqpoint{3.369603in}{1.999859in}}{\pgfqpoint{3.373509in}{1.995952in}}%
\pgfpathcurveto{\pgfqpoint{3.377416in}{1.992045in}}{\pgfqpoint{3.382716in}{1.989850in}}{\pgfqpoint{3.388241in}{1.989850in}}%
\pgfpathclose%
\pgfusepath{stroke}%
\end{pgfscope}%
\begin{pgfscope}%
\pgfpathrectangle{\pgfqpoint{0.438556in}{0.383578in}}{\pgfqpoint{4.650000in}{2.310000in}}%
\pgfusepath{clip}%
\pgfsetbuttcap%
\pgfsetroundjoin%
\pgfsetlinewidth{0.803000pt}%
\definecolor{currentstroke}{rgb}{0.686275,0.352941,0.313725}%
\pgfsetstrokecolor{currentstroke}%
\pgfsetdash{}{0pt}%
\pgfpathmoveto{\pgfqpoint{3.412467in}{2.011692in}}%
\pgfpathcurveto{\pgfqpoint{3.417992in}{2.011692in}}{\pgfqpoint{3.423291in}{2.013888in}}{\pgfqpoint{3.427198in}{2.017794in}}%
\pgfpathcurveto{\pgfqpoint{3.431105in}{2.021701in}}{\pgfqpoint{3.433300in}{2.027001in}}{\pgfqpoint{3.433300in}{2.032526in}}%
\pgfpathcurveto{\pgfqpoint{3.433300in}{2.038051in}}{\pgfqpoint{3.431105in}{2.043350in}}{\pgfqpoint{3.427198in}{2.047257in}}%
\pgfpathcurveto{\pgfqpoint{3.423291in}{2.051164in}}{\pgfqpoint{3.417992in}{2.053359in}}{\pgfqpoint{3.412467in}{2.053359in}}%
\pgfpathcurveto{\pgfqpoint{3.406942in}{2.053359in}}{\pgfqpoint{3.401642in}{2.051164in}}{\pgfqpoint{3.397735in}{2.047257in}}%
\pgfpathcurveto{\pgfqpoint{3.393828in}{2.043350in}}{\pgfqpoint{3.391633in}{2.038051in}}{\pgfqpoint{3.391633in}{2.032526in}}%
\pgfpathcurveto{\pgfqpoint{3.391633in}{2.027001in}}{\pgfqpoint{3.393828in}{2.021701in}}{\pgfqpoint{3.397735in}{2.017794in}}%
\pgfpathcurveto{\pgfqpoint{3.401642in}{2.013888in}}{\pgfqpoint{3.406942in}{2.011692in}}{\pgfqpoint{3.412467in}{2.011692in}}%
\pgfpathclose%
\pgfusepath{stroke}%
\end{pgfscope}%
\begin{pgfscope}%
\pgfpathrectangle{\pgfqpoint{0.438556in}{0.383578in}}{\pgfqpoint{4.650000in}{2.310000in}}%
\pgfusepath{clip}%
\pgfsetbuttcap%
\pgfsetroundjoin%
\pgfsetlinewidth{0.803000pt}%
\definecolor{currentstroke}{rgb}{0.686275,0.352941,0.313725}%
\pgfsetstrokecolor{currentstroke}%
\pgfsetdash{}{0pt}%
\pgfpathmoveto{\pgfqpoint{3.392263in}{2.077219in}}%
\pgfpathcurveto{\pgfqpoint{3.397788in}{2.077219in}}{\pgfqpoint{3.403088in}{2.079414in}}{\pgfqpoint{3.406995in}{2.083321in}}%
\pgfpathcurveto{\pgfqpoint{3.410901in}{2.087228in}}{\pgfqpoint{3.413097in}{2.092527in}}{\pgfqpoint{3.413097in}{2.098052in}}%
\pgfpathcurveto{\pgfqpoint{3.413097in}{2.103577in}}{\pgfqpoint{3.410901in}{2.108877in}}{\pgfqpoint{3.406995in}{2.112784in}}%
\pgfpathcurveto{\pgfqpoint{3.403088in}{2.116690in}}{\pgfqpoint{3.397788in}{2.118885in}}{\pgfqpoint{3.392263in}{2.118885in}}%
\pgfpathcurveto{\pgfqpoint{3.386738in}{2.118885in}}{\pgfqpoint{3.381439in}{2.116690in}}{\pgfqpoint{3.377532in}{2.112784in}}%
\pgfpathcurveto{\pgfqpoint{3.373625in}{2.108877in}}{\pgfqpoint{3.371430in}{2.103577in}}{\pgfqpoint{3.371430in}{2.098052in}}%
\pgfpathcurveto{\pgfqpoint{3.371430in}{2.092527in}}{\pgfqpoint{3.373625in}{2.087228in}}{\pgfqpoint{3.377532in}{2.083321in}}%
\pgfpathcurveto{\pgfqpoint{3.381439in}{2.079414in}}{\pgfqpoint{3.386738in}{2.077219in}}{\pgfqpoint{3.392263in}{2.077219in}}%
\pgfpathclose%
\pgfusepath{stroke}%
\end{pgfscope}%
\begin{pgfscope}%
\pgfpathrectangle{\pgfqpoint{0.438556in}{0.383578in}}{\pgfqpoint{4.650000in}{2.310000in}}%
\pgfusepath{clip}%
\pgfsetbuttcap%
\pgfsetroundjoin%
\pgfsetlinewidth{0.803000pt}%
\definecolor{currentstroke}{rgb}{0.686275,0.352941,0.313725}%
\pgfsetstrokecolor{currentstroke}%
\pgfsetdash{}{0pt}%
\pgfpathmoveto{\pgfqpoint{3.430040in}{2.099061in}}%
\pgfpathcurveto{\pgfqpoint{3.435565in}{2.099061in}}{\pgfqpoint{3.440865in}{2.101256in}}{\pgfqpoint{3.444772in}{2.105163in}}%
\pgfpathcurveto{\pgfqpoint{3.448679in}{2.109070in}}{\pgfqpoint{3.450874in}{2.114369in}}{\pgfqpoint{3.450874in}{2.119894in}}%
\pgfpathcurveto{\pgfqpoint{3.450874in}{2.125419in}}{\pgfqpoint{3.448679in}{2.130719in}}{\pgfqpoint{3.444772in}{2.134626in}}%
\pgfpathcurveto{\pgfqpoint{3.440865in}{2.138532in}}{\pgfqpoint{3.435565in}{2.140728in}}{\pgfqpoint{3.430040in}{2.140728in}}%
\pgfpathcurveto{\pgfqpoint{3.424515in}{2.140728in}}{\pgfqpoint{3.419216in}{2.138532in}}{\pgfqpoint{3.415309in}{2.134626in}}%
\pgfpathcurveto{\pgfqpoint{3.411402in}{2.130719in}}{\pgfqpoint{3.409207in}{2.125419in}}{\pgfqpoint{3.409207in}{2.119894in}}%
\pgfpathcurveto{\pgfqpoint{3.409207in}{2.114369in}}{\pgfqpoint{3.411402in}{2.109070in}}{\pgfqpoint{3.415309in}{2.105163in}}%
\pgfpathcurveto{\pgfqpoint{3.419216in}{2.101256in}}{\pgfqpoint{3.424515in}{2.099061in}}{\pgfqpoint{3.430040in}{2.099061in}}%
\pgfpathclose%
\pgfusepath{stroke}%
\end{pgfscope}%
\begin{pgfscope}%
\pgfpathrectangle{\pgfqpoint{0.438556in}{0.383578in}}{\pgfqpoint{4.650000in}{2.310000in}}%
\pgfusepath{clip}%
\pgfsetbuttcap%
\pgfsetroundjoin%
\pgfsetlinewidth{0.803000pt}%
\definecolor{currentstroke}{rgb}{0.686275,0.352941,0.313725}%
\pgfsetstrokecolor{currentstroke}%
\pgfsetdash{}{0pt}%
\pgfpathmoveto{\pgfqpoint{3.451939in}{2.164587in}}%
\pgfpathcurveto{\pgfqpoint{3.457464in}{2.164587in}}{\pgfqpoint{3.462764in}{2.166782in}}{\pgfqpoint{3.466670in}{2.170689in}}%
\pgfpathcurveto{\pgfqpoint{3.470577in}{2.174596in}}{\pgfqpoint{3.472772in}{2.179896in}}{\pgfqpoint{3.472772in}{2.185421in}}%
\pgfpathcurveto{\pgfqpoint{3.472772in}{2.190946in}}{\pgfqpoint{3.470577in}{2.196245in}}{\pgfqpoint{3.466670in}{2.200152in}}%
\pgfpathcurveto{\pgfqpoint{3.462764in}{2.204059in}}{\pgfqpoint{3.457464in}{2.206254in}}{\pgfqpoint{3.451939in}{2.206254in}}%
\pgfpathcurveto{\pgfqpoint{3.446414in}{2.206254in}}{\pgfqpoint{3.441114in}{2.204059in}}{\pgfqpoint{3.437208in}{2.200152in}}%
\pgfpathcurveto{\pgfqpoint{3.433301in}{2.196245in}}{\pgfqpoint{3.431106in}{2.190946in}}{\pgfqpoint{3.431106in}{2.185421in}}%
\pgfpathcurveto{\pgfqpoint{3.431106in}{2.179896in}}{\pgfqpoint{3.433301in}{2.174596in}}{\pgfqpoint{3.437208in}{2.170689in}}%
\pgfpathcurveto{\pgfqpoint{3.441114in}{2.166782in}}{\pgfqpoint{3.446414in}{2.164587in}}{\pgfqpoint{3.451939in}{2.164587in}}%
\pgfpathclose%
\pgfusepath{stroke}%
\end{pgfscope}%
\begin{pgfscope}%
\pgfpathrectangle{\pgfqpoint{0.438556in}{0.383578in}}{\pgfqpoint{4.650000in}{2.310000in}}%
\pgfusepath{clip}%
\pgfsetbuttcap%
\pgfsetroundjoin%
\pgfsetlinewidth{0.803000pt}%
\definecolor{currentstroke}{rgb}{0.686275,0.352941,0.313725}%
\pgfsetstrokecolor{currentstroke}%
\pgfsetdash{}{0pt}%
\pgfpathmoveto{\pgfqpoint{3.420511in}{2.230114in}}%
\pgfpathcurveto{\pgfqpoint{3.426036in}{2.230114in}}{\pgfqpoint{3.431336in}{2.232309in}}{\pgfqpoint{3.435243in}{2.236216in}}%
\pgfpathcurveto{\pgfqpoint{3.439150in}{2.240123in}}{\pgfqpoint{3.441345in}{2.245422in}}{\pgfqpoint{3.441345in}{2.250947in}}%
\pgfpathcurveto{\pgfqpoint{3.441345in}{2.256472in}}{\pgfqpoint{3.439150in}{2.261772in}}{\pgfqpoint{3.435243in}{2.265678in}}%
\pgfpathcurveto{\pgfqpoint{3.431336in}{2.269585in}}{\pgfqpoint{3.426036in}{2.271780in}}{\pgfqpoint{3.420511in}{2.271780in}}%
\pgfpathcurveto{\pgfqpoint{3.414986in}{2.271780in}}{\pgfqpoint{3.409687in}{2.269585in}}{\pgfqpoint{3.405780in}{2.265678in}}%
\pgfpathcurveto{\pgfqpoint{3.401873in}{2.261772in}}{\pgfqpoint{3.399678in}{2.256472in}}{\pgfqpoint{3.399678in}{2.250947in}}%
\pgfpathcurveto{\pgfqpoint{3.399678in}{2.245422in}}{\pgfqpoint{3.401873in}{2.240123in}}{\pgfqpoint{3.405780in}{2.236216in}}%
\pgfpathcurveto{\pgfqpoint{3.409687in}{2.232309in}}{\pgfqpoint{3.414986in}{2.230114in}}{\pgfqpoint{3.420511in}{2.230114in}}%
\pgfpathclose%
\pgfusepath{stroke}%
\end{pgfscope}%
\begin{pgfscope}%
\pgfpathrectangle{\pgfqpoint{0.438556in}{0.383578in}}{\pgfqpoint{4.650000in}{2.310000in}}%
\pgfusepath{clip}%
\pgfsetbuttcap%
\pgfsetroundjoin%
\pgfsetlinewidth{0.803000pt}%
\definecolor{currentstroke}{rgb}{0.686275,0.352941,0.313725}%
\pgfsetstrokecolor{currentstroke}%
\pgfsetdash{}{0pt}%
\pgfpathmoveto{\pgfqpoint{3.410643in}{2.251956in}}%
\pgfpathcurveto{\pgfqpoint{3.416168in}{2.251956in}}{\pgfqpoint{3.421468in}{2.254151in}}{\pgfqpoint{3.425375in}{2.258058in}}%
\pgfpathcurveto{\pgfqpoint{3.429281in}{2.261965in}}{\pgfqpoint{3.431477in}{2.267264in}}{\pgfqpoint{3.431477in}{2.272789in}}%
\pgfpathcurveto{\pgfqpoint{3.431477in}{2.278314in}}{\pgfqpoint{3.429281in}{2.283614in}}{\pgfqpoint{3.425375in}{2.287521in}}%
\pgfpathcurveto{\pgfqpoint{3.421468in}{2.291427in}}{\pgfqpoint{3.416168in}{2.293623in}}{\pgfqpoint{3.410643in}{2.293623in}}%
\pgfpathcurveto{\pgfqpoint{3.405118in}{2.293623in}}{\pgfqpoint{3.399819in}{2.291427in}}{\pgfqpoint{3.395912in}{2.287521in}}%
\pgfpathcurveto{\pgfqpoint{3.392005in}{2.283614in}}{\pgfqpoint{3.389810in}{2.278314in}}{\pgfqpoint{3.389810in}{2.272789in}}%
\pgfpathcurveto{\pgfqpoint{3.389810in}{2.267264in}}{\pgfqpoint{3.392005in}{2.261965in}}{\pgfqpoint{3.395912in}{2.258058in}}%
\pgfpathcurveto{\pgfqpoint{3.399819in}{2.254151in}}{\pgfqpoint{3.405118in}{2.251956in}}{\pgfqpoint{3.410643in}{2.251956in}}%
\pgfpathclose%
\pgfusepath{stroke}%
\end{pgfscope}%
\begin{pgfscope}%
\pgfpathrectangle{\pgfqpoint{0.438556in}{0.383578in}}{\pgfqpoint{4.650000in}{2.310000in}}%
\pgfusepath{clip}%
\pgfsetbuttcap%
\pgfsetroundjoin%
\pgfsetlinewidth{0.803000pt}%
\definecolor{currentstroke}{rgb}{0.686275,0.352941,0.313725}%
\pgfsetstrokecolor{currentstroke}%
\pgfsetdash{}{0pt}%
\pgfpathmoveto{\pgfqpoint{3.419357in}{2.295640in}}%
\pgfpathcurveto{\pgfqpoint{3.424882in}{2.295640in}}{\pgfqpoint{3.430181in}{2.297835in}}{\pgfqpoint{3.434088in}{2.301742in}}%
\pgfpathcurveto{\pgfqpoint{3.437995in}{2.305649in}}{\pgfqpoint{3.440190in}{2.310948in}}{\pgfqpoint{3.440190in}{2.316473in}}%
\pgfpathcurveto{\pgfqpoint{3.440190in}{2.321999in}}{\pgfqpoint{3.437995in}{2.327298in}}{\pgfqpoint{3.434088in}{2.331205in}}%
\pgfpathcurveto{\pgfqpoint{3.430181in}{2.335112in}}{\pgfqpoint{3.424882in}{2.337307in}}{\pgfqpoint{3.419357in}{2.337307in}}%
\pgfpathcurveto{\pgfqpoint{3.413832in}{2.337307in}}{\pgfqpoint{3.408532in}{2.335112in}}{\pgfqpoint{3.404625in}{2.331205in}}%
\pgfpathcurveto{\pgfqpoint{3.400719in}{2.327298in}}{\pgfqpoint{3.398524in}{2.321999in}}{\pgfqpoint{3.398524in}{2.316473in}}%
\pgfpathcurveto{\pgfqpoint{3.398524in}{2.310948in}}{\pgfqpoint{3.400719in}{2.305649in}}{\pgfqpoint{3.404625in}{2.301742in}}%
\pgfpathcurveto{\pgfqpoint{3.408532in}{2.297835in}}{\pgfqpoint{3.413832in}{2.295640in}}{\pgfqpoint{3.419357in}{2.295640in}}%
\pgfpathclose%
\pgfusepath{stroke}%
\end{pgfscope}%
\begin{pgfscope}%
\pgfpathrectangle{\pgfqpoint{0.438556in}{0.383578in}}{\pgfqpoint{4.650000in}{2.310000in}}%
\pgfusepath{clip}%
\pgfsetbuttcap%
\pgfsetroundjoin%
\pgfsetlinewidth{0.803000pt}%
\definecolor{currentstroke}{rgb}{0.686275,0.352941,0.313725}%
\pgfsetstrokecolor{currentstroke}%
\pgfsetdash{}{0pt}%
\pgfpathmoveto{\pgfqpoint{3.419155in}{2.317482in}}%
\pgfpathcurveto{\pgfqpoint{3.424680in}{2.317482in}}{\pgfqpoint{3.429980in}{2.319677in}}{\pgfqpoint{3.433887in}{2.323584in}}%
\pgfpathcurveto{\pgfqpoint{3.437794in}{2.327491in}}{\pgfqpoint{3.439989in}{2.332791in}}{\pgfqpoint{3.439989in}{2.338316in}}%
\pgfpathcurveto{\pgfqpoint{3.439989in}{2.343841in}}{\pgfqpoint{3.437794in}{2.349140in}}{\pgfqpoint{3.433887in}{2.353047in}}%
\pgfpathcurveto{\pgfqpoint{3.429980in}{2.356954in}}{\pgfqpoint{3.424680in}{2.359149in}}{\pgfqpoint{3.419155in}{2.359149in}}%
\pgfpathcurveto{\pgfqpoint{3.413630in}{2.359149in}}{\pgfqpoint{3.408331in}{2.356954in}}{\pgfqpoint{3.404424in}{2.353047in}}%
\pgfpathcurveto{\pgfqpoint{3.400517in}{2.349140in}}{\pgfqpoint{3.398322in}{2.343841in}}{\pgfqpoint{3.398322in}{2.338316in}}%
\pgfpathcurveto{\pgfqpoint{3.398322in}{2.332791in}}{\pgfqpoint{3.400517in}{2.327491in}}{\pgfqpoint{3.404424in}{2.323584in}}%
\pgfpathcurveto{\pgfqpoint{3.408331in}{2.319677in}}{\pgfqpoint{3.413630in}{2.317482in}}{\pgfqpoint{3.419155in}{2.317482in}}%
\pgfpathclose%
\pgfusepath{stroke}%
\end{pgfscope}%
\begin{pgfscope}%
\pgfpathrectangle{\pgfqpoint{0.438556in}{0.383578in}}{\pgfqpoint{4.650000in}{2.310000in}}%
\pgfusepath{clip}%
\pgfsetbuttcap%
\pgfsetroundjoin%
\pgfsetlinewidth{0.803000pt}%
\definecolor{currentstroke}{rgb}{0.686275,0.352941,0.313725}%
\pgfsetstrokecolor{currentstroke}%
\pgfsetdash{}{0pt}%
\pgfpathmoveto{\pgfqpoint{3.437261in}{2.339324in}}%
\pgfpathcurveto{\pgfqpoint{3.442786in}{2.339324in}}{\pgfqpoint{3.448085in}{2.341520in}}{\pgfqpoint{3.451992in}{2.345426in}}%
\pgfpathcurveto{\pgfqpoint{3.455899in}{2.349333in}}{\pgfqpoint{3.458094in}{2.354633in}}{\pgfqpoint{3.458094in}{2.360158in}}%
\pgfpathcurveto{\pgfqpoint{3.458094in}{2.365683in}}{\pgfqpoint{3.455899in}{2.370982in}}{\pgfqpoint{3.451992in}{2.374889in}}%
\pgfpathcurveto{\pgfqpoint{3.448085in}{2.378796in}}{\pgfqpoint{3.442786in}{2.380991in}}{\pgfqpoint{3.437261in}{2.380991in}}%
\pgfpathcurveto{\pgfqpoint{3.431735in}{2.380991in}}{\pgfqpoint{3.426436in}{2.378796in}}{\pgfqpoint{3.422529in}{2.374889in}}%
\pgfpathcurveto{\pgfqpoint{3.418622in}{2.370982in}}{\pgfqpoint{3.416427in}{2.365683in}}{\pgfqpoint{3.416427in}{2.360158in}}%
\pgfpathcurveto{\pgfqpoint{3.416427in}{2.354633in}}{\pgfqpoint{3.418622in}{2.349333in}}{\pgfqpoint{3.422529in}{2.345426in}}%
\pgfpathcurveto{\pgfqpoint{3.426436in}{2.341520in}}{\pgfqpoint{3.431735in}{2.339324in}}{\pgfqpoint{3.437261in}{2.339324in}}%
\pgfpathclose%
\pgfusepath{stroke}%
\end{pgfscope}%
\begin{pgfscope}%
\pgfpathrectangle{\pgfqpoint{0.438556in}{0.383578in}}{\pgfqpoint{4.650000in}{2.310000in}}%
\pgfusepath{clip}%
\pgfsetbuttcap%
\pgfsetroundjoin%
\pgfsetlinewidth{0.803000pt}%
\definecolor{currentstroke}{rgb}{0.686275,0.352941,0.313725}%
\pgfsetstrokecolor{currentstroke}%
\pgfsetdash{}{0pt}%
\pgfpathmoveto{\pgfqpoint{3.457327in}{2.383009in}}%
\pgfpathcurveto{\pgfqpoint{3.462852in}{2.383009in}}{\pgfqpoint{3.468151in}{2.385204in}}{\pgfqpoint{3.472058in}{2.389111in}}%
\pgfpathcurveto{\pgfqpoint{3.475965in}{2.393017in}}{\pgfqpoint{3.478160in}{2.398317in}}{\pgfqpoint{3.478160in}{2.403842in}}%
\pgfpathcurveto{\pgfqpoint{3.478160in}{2.409367in}}{\pgfqpoint{3.475965in}{2.414667in}}{\pgfqpoint{3.472058in}{2.418573in}}%
\pgfpathcurveto{\pgfqpoint{3.468151in}{2.422480in}}{\pgfqpoint{3.462852in}{2.424675in}}{\pgfqpoint{3.457327in}{2.424675in}}%
\pgfpathcurveto{\pgfqpoint{3.451801in}{2.424675in}}{\pgfqpoint{3.446502in}{2.422480in}}{\pgfqpoint{3.442595in}{2.418573in}}%
\pgfpathcurveto{\pgfqpoint{3.438688in}{2.414667in}}{\pgfqpoint{3.436493in}{2.409367in}}{\pgfqpoint{3.436493in}{2.403842in}}%
\pgfpathcurveto{\pgfqpoint{3.436493in}{2.398317in}}{\pgfqpoint{3.438688in}{2.393017in}}{\pgfqpoint{3.442595in}{2.389111in}}%
\pgfpathcurveto{\pgfqpoint{3.446502in}{2.385204in}}{\pgfqpoint{3.451801in}{2.383009in}}{\pgfqpoint{3.457327in}{2.383009in}}%
\pgfpathclose%
\pgfusepath{stroke}%
\end{pgfscope}%
\begin{pgfscope}%
\pgfpathrectangle{\pgfqpoint{0.438556in}{0.383578in}}{\pgfqpoint{4.650000in}{2.310000in}}%
\pgfusepath{clip}%
\pgfsetbuttcap%
\pgfsetroundjoin%
\pgfsetlinewidth{0.803000pt}%
\definecolor{currentstroke}{rgb}{0.686275,0.352941,0.313725}%
\pgfsetstrokecolor{currentstroke}%
\pgfsetdash{}{0pt}%
\pgfpathmoveto{\pgfqpoint{3.386216in}{2.404851in}}%
\pgfpathcurveto{\pgfqpoint{3.391741in}{2.404851in}}{\pgfqpoint{3.397040in}{2.407046in}}{\pgfqpoint{3.400947in}{2.410953in}}%
\pgfpathcurveto{\pgfqpoint{3.404854in}{2.414860in}}{\pgfqpoint{3.407049in}{2.420159in}}{\pgfqpoint{3.407049in}{2.425684in}}%
\pgfpathcurveto{\pgfqpoint{3.407049in}{2.431209in}}{\pgfqpoint{3.404854in}{2.436509in}}{\pgfqpoint{3.400947in}{2.440416in}}%
\pgfpathcurveto{\pgfqpoint{3.397040in}{2.444322in}}{\pgfqpoint{3.391741in}{2.446517in}}{\pgfqpoint{3.386216in}{2.446517in}}%
\pgfpathcurveto{\pgfqpoint{3.380691in}{2.446517in}}{\pgfqpoint{3.375391in}{2.444322in}}{\pgfqpoint{3.371485in}{2.440416in}}%
\pgfpathcurveto{\pgfqpoint{3.367578in}{2.436509in}}{\pgfqpoint{3.365383in}{2.431209in}}{\pgfqpoint{3.365383in}{2.425684in}}%
\pgfpathcurveto{\pgfqpoint{3.365383in}{2.420159in}}{\pgfqpoint{3.367578in}{2.414860in}}{\pgfqpoint{3.371485in}{2.410953in}}%
\pgfpathcurveto{\pgfqpoint{3.375391in}{2.407046in}}{\pgfqpoint{3.380691in}{2.404851in}}{\pgfqpoint{3.386216in}{2.404851in}}%
\pgfpathclose%
\pgfusepath{stroke}%
\end{pgfscope}%
\begin{pgfscope}%
\pgfpathrectangle{\pgfqpoint{0.438556in}{0.383578in}}{\pgfqpoint{4.650000in}{2.310000in}}%
\pgfusepath{clip}%
\pgfsetbuttcap%
\pgfsetroundjoin%
\pgfsetlinewidth{0.803000pt}%
\definecolor{currentstroke}{rgb}{0.686275,0.352941,0.313725}%
\pgfsetstrokecolor{currentstroke}%
\pgfsetdash{}{0pt}%
\pgfpathmoveto{\pgfqpoint{3.404010in}{2.492219in}}%
\pgfpathcurveto{\pgfqpoint{3.409535in}{2.492219in}}{\pgfqpoint{3.414834in}{2.494414in}}{\pgfqpoint{3.418741in}{2.498321in}}%
\pgfpathcurveto{\pgfqpoint{3.422648in}{2.502228in}}{\pgfqpoint{3.424843in}{2.507528in}}{\pgfqpoint{3.424843in}{2.513053in}}%
\pgfpathcurveto{\pgfqpoint{3.424843in}{2.518578in}}{\pgfqpoint{3.422648in}{2.523877in}}{\pgfqpoint{3.418741in}{2.527784in}}%
\pgfpathcurveto{\pgfqpoint{3.414834in}{2.531691in}}{\pgfqpoint{3.409535in}{2.533886in}}{\pgfqpoint{3.404010in}{2.533886in}}%
\pgfpathcurveto{\pgfqpoint{3.398485in}{2.533886in}}{\pgfqpoint{3.393185in}{2.531691in}}{\pgfqpoint{3.389278in}{2.527784in}}%
\pgfpathcurveto{\pgfqpoint{3.385371in}{2.523877in}}{\pgfqpoint{3.383176in}{2.518578in}}{\pgfqpoint{3.383176in}{2.513053in}}%
\pgfpathcurveto{\pgfqpoint{3.383176in}{2.507528in}}{\pgfqpoint{3.385371in}{2.502228in}}{\pgfqpoint{3.389278in}{2.498321in}}%
\pgfpathcurveto{\pgfqpoint{3.393185in}{2.494414in}}{\pgfqpoint{3.398485in}{2.492219in}}{\pgfqpoint{3.404010in}{2.492219in}}%
\pgfpathclose%
\pgfusepath{stroke}%
\end{pgfscope}%
\begin{pgfscope}%
\pgfpathrectangle{\pgfqpoint{0.438556in}{0.383578in}}{\pgfqpoint{4.650000in}{2.310000in}}%
\pgfusepath{clip}%
\pgfsetbuttcap%
\pgfsetroundjoin%
\pgfsetlinewidth{0.803000pt}%
\definecolor{currentstroke}{rgb}{0.686275,0.352941,0.313725}%
\pgfsetstrokecolor{currentstroke}%
\pgfsetdash{}{0pt}%
\pgfpathmoveto{\pgfqpoint{3.377850in}{2.514061in}}%
\pgfpathcurveto{\pgfqpoint{3.383376in}{2.514061in}}{\pgfqpoint{3.388675in}{2.516257in}}{\pgfqpoint{3.392582in}{2.520163in}}%
\pgfpathcurveto{\pgfqpoint{3.396489in}{2.524070in}}{\pgfqpoint{3.398684in}{2.529370in}}{\pgfqpoint{3.398684in}{2.534895in}}%
\pgfpathcurveto{\pgfqpoint{3.398684in}{2.540420in}}{\pgfqpoint{3.396489in}{2.545719in}}{\pgfqpoint{3.392582in}{2.549626in}}%
\pgfpathcurveto{\pgfqpoint{3.388675in}{2.553533in}}{\pgfqpoint{3.383376in}{2.555728in}}{\pgfqpoint{3.377850in}{2.555728in}}%
\pgfpathcurveto{\pgfqpoint{3.372325in}{2.555728in}}{\pgfqpoint{3.367026in}{2.553533in}}{\pgfqpoint{3.363119in}{2.549626in}}%
\pgfpathcurveto{\pgfqpoint{3.359212in}{2.545719in}}{\pgfqpoint{3.357017in}{2.540420in}}{\pgfqpoint{3.357017in}{2.534895in}}%
\pgfpathcurveto{\pgfqpoint{3.357017in}{2.529370in}}{\pgfqpoint{3.359212in}{2.524070in}}{\pgfqpoint{3.363119in}{2.520163in}}%
\pgfpathcurveto{\pgfqpoint{3.367026in}{2.516257in}}{\pgfqpoint{3.372325in}{2.514061in}}{\pgfqpoint{3.377850in}{2.514061in}}%
\pgfpathclose%
\pgfusepath{stroke}%
\end{pgfscope}%
\begin{pgfscope}%
\pgfpathrectangle{\pgfqpoint{0.438556in}{0.383578in}}{\pgfqpoint{4.650000in}{2.310000in}}%
\pgfusepath{clip}%
\pgfsetbuttcap%
\pgfsetroundjoin%
\pgfsetlinewidth{0.803000pt}%
\definecolor{currentstroke}{rgb}{0.686275,0.352941,0.313725}%
\pgfsetstrokecolor{currentstroke}%
\pgfsetdash{}{0pt}%
\pgfpathmoveto{\pgfqpoint{3.954533in}{0.482743in}}%
\pgfpathcurveto{\pgfqpoint{3.960058in}{0.482743in}}{\pgfqpoint{3.965358in}{0.484938in}}{\pgfqpoint{3.969264in}{0.488845in}}%
\pgfpathcurveto{\pgfqpoint{3.973171in}{0.492752in}}{\pgfqpoint{3.975366in}{0.498051in}}{\pgfqpoint{3.975366in}{0.503577in}}%
\pgfpathcurveto{\pgfqpoint{3.975366in}{0.509102in}}{\pgfqpoint{3.973171in}{0.514401in}}{\pgfqpoint{3.969264in}{0.518308in}}%
\pgfpathcurveto{\pgfqpoint{3.965358in}{0.522215in}}{\pgfqpoint{3.960058in}{0.524410in}}{\pgfqpoint{3.954533in}{0.524410in}}%
\pgfpathcurveto{\pgfqpoint{3.949008in}{0.524410in}}{\pgfqpoint{3.943708in}{0.522215in}}{\pgfqpoint{3.939802in}{0.518308in}}%
\pgfpathcurveto{\pgfqpoint{3.935895in}{0.514401in}}{\pgfqpoint{3.933700in}{0.509102in}}{\pgfqpoint{3.933700in}{0.503577in}}%
\pgfpathcurveto{\pgfqpoint{3.933700in}{0.498051in}}{\pgfqpoint{3.935895in}{0.492752in}}{\pgfqpoint{3.939802in}{0.488845in}}%
\pgfpathcurveto{\pgfqpoint{3.943708in}{0.484938in}}{\pgfqpoint{3.949008in}{0.482743in}}{\pgfqpoint{3.954533in}{0.482743in}}%
\pgfpathclose%
\pgfusepath{stroke}%
\end{pgfscope}%
\begin{pgfscope}%
\pgfpathrectangle{\pgfqpoint{0.438556in}{0.383578in}}{\pgfqpoint{4.650000in}{2.310000in}}%
\pgfusepath{clip}%
\pgfsetbuttcap%
\pgfsetroundjoin%
\pgfsetlinewidth{0.803000pt}%
\definecolor{currentstroke}{rgb}{0.686275,0.352941,0.313725}%
\pgfsetstrokecolor{currentstroke}%
\pgfsetdash{}{0pt}%
\pgfpathmoveto{\pgfqpoint{3.953589in}{0.635638in}}%
\pgfpathcurveto{\pgfqpoint{3.959114in}{0.635638in}}{\pgfqpoint{3.964414in}{0.637833in}}{\pgfqpoint{3.968321in}{0.641740in}}%
\pgfpathcurveto{\pgfqpoint{3.972227in}{0.645647in}}{\pgfqpoint{3.974423in}{0.650946in}}{\pgfqpoint{3.974423in}{0.656471in}}%
\pgfpathcurveto{\pgfqpoint{3.974423in}{0.661997in}}{\pgfqpoint{3.972227in}{0.667296in}}{\pgfqpoint{3.968321in}{0.671203in}}%
\pgfpathcurveto{\pgfqpoint{3.964414in}{0.675110in}}{\pgfqpoint{3.959114in}{0.677305in}}{\pgfqpoint{3.953589in}{0.677305in}}%
\pgfpathcurveto{\pgfqpoint{3.948064in}{0.677305in}}{\pgfqpoint{3.942765in}{0.675110in}}{\pgfqpoint{3.938858in}{0.671203in}}%
\pgfpathcurveto{\pgfqpoint{3.934951in}{0.667296in}}{\pgfqpoint{3.932756in}{0.661997in}}{\pgfqpoint{3.932756in}{0.656471in}}%
\pgfpathcurveto{\pgfqpoint{3.932756in}{0.650946in}}{\pgfqpoint{3.934951in}{0.645647in}}{\pgfqpoint{3.938858in}{0.641740in}}%
\pgfpathcurveto{\pgfqpoint{3.942765in}{0.637833in}}{\pgfqpoint{3.948064in}{0.635638in}}{\pgfqpoint{3.953589in}{0.635638in}}%
\pgfpathclose%
\pgfusepath{stroke}%
\end{pgfscope}%
\begin{pgfscope}%
\pgfpathrectangle{\pgfqpoint{0.438556in}{0.383578in}}{\pgfqpoint{4.650000in}{2.310000in}}%
\pgfusepath{clip}%
\pgfsetbuttcap%
\pgfsetroundjoin%
\pgfsetlinewidth{0.803000pt}%
\definecolor{currentstroke}{rgb}{0.686275,0.352941,0.313725}%
\pgfsetstrokecolor{currentstroke}%
\pgfsetdash{}{0pt}%
\pgfpathmoveto{\pgfqpoint{4.023829in}{0.679322in}}%
\pgfpathcurveto{\pgfqpoint{4.029354in}{0.679322in}}{\pgfqpoint{4.034654in}{0.681518in}}{\pgfqpoint{4.038561in}{0.685424in}}%
\pgfpathcurveto{\pgfqpoint{4.042468in}{0.689331in}}{\pgfqpoint{4.044663in}{0.694631in}}{\pgfqpoint{4.044663in}{0.700156in}}%
\pgfpathcurveto{\pgfqpoint{4.044663in}{0.705681in}}{\pgfqpoint{4.042468in}{0.710980in}}{\pgfqpoint{4.038561in}{0.714887in}}%
\pgfpathcurveto{\pgfqpoint{4.034654in}{0.718794in}}{\pgfqpoint{4.029354in}{0.720989in}}{\pgfqpoint{4.023829in}{0.720989in}}%
\pgfpathcurveto{\pgfqpoint{4.018304in}{0.720989in}}{\pgfqpoint{4.013005in}{0.718794in}}{\pgfqpoint{4.009098in}{0.714887in}}%
\pgfpathcurveto{\pgfqpoint{4.005191in}{0.710980in}}{\pgfqpoint{4.002996in}{0.705681in}}{\pgfqpoint{4.002996in}{0.700156in}}%
\pgfpathcurveto{\pgfqpoint{4.002996in}{0.694631in}}{\pgfqpoint{4.005191in}{0.689331in}}{\pgfqpoint{4.009098in}{0.685424in}}%
\pgfpathcurveto{\pgfqpoint{4.013005in}{0.681518in}}{\pgfqpoint{4.018304in}{0.679322in}}{\pgfqpoint{4.023829in}{0.679322in}}%
\pgfpathclose%
\pgfusepath{stroke}%
\end{pgfscope}%
\begin{pgfscope}%
\pgfpathrectangle{\pgfqpoint{0.438556in}{0.383578in}}{\pgfqpoint{4.650000in}{2.310000in}}%
\pgfusepath{clip}%
\pgfsetbuttcap%
\pgfsetroundjoin%
\pgfsetlinewidth{0.803000pt}%
\definecolor{currentstroke}{rgb}{0.686275,0.352941,0.313725}%
\pgfsetstrokecolor{currentstroke}%
\pgfsetdash{}{0pt}%
\pgfpathmoveto{\pgfqpoint{3.983597in}{0.701165in}}%
\pgfpathcurveto{\pgfqpoint{3.989122in}{0.701165in}}{\pgfqpoint{3.994421in}{0.703360in}}{\pgfqpoint{3.998328in}{0.707266in}}%
\pgfpathcurveto{\pgfqpoint{4.002235in}{0.711173in}}{\pgfqpoint{4.004430in}{0.716473in}}{\pgfqpoint{4.004430in}{0.721998in}}%
\pgfpathcurveto{\pgfqpoint{4.004430in}{0.727523in}}{\pgfqpoint{4.002235in}{0.732822in}}{\pgfqpoint{3.998328in}{0.736729in}}%
\pgfpathcurveto{\pgfqpoint{3.994421in}{0.740636in}}{\pgfqpoint{3.989122in}{0.742831in}}{\pgfqpoint{3.983597in}{0.742831in}}%
\pgfpathcurveto{\pgfqpoint{3.978072in}{0.742831in}}{\pgfqpoint{3.972772in}{0.740636in}}{\pgfqpoint{3.968865in}{0.736729in}}%
\pgfpathcurveto{\pgfqpoint{3.964958in}{0.732822in}}{\pgfqpoint{3.962763in}{0.727523in}}{\pgfqpoint{3.962763in}{0.721998in}}%
\pgfpathcurveto{\pgfqpoint{3.962763in}{0.716473in}}{\pgfqpoint{3.964958in}{0.711173in}}{\pgfqpoint{3.968865in}{0.707266in}}%
\pgfpathcurveto{\pgfqpoint{3.972772in}{0.703360in}}{\pgfqpoint{3.978072in}{0.701165in}}{\pgfqpoint{3.983597in}{0.701165in}}%
\pgfpathclose%
\pgfusepath{stroke}%
\end{pgfscope}%
\begin{pgfscope}%
\pgfpathrectangle{\pgfqpoint{0.438556in}{0.383578in}}{\pgfqpoint{4.650000in}{2.310000in}}%
\pgfusepath{clip}%
\pgfsetbuttcap%
\pgfsetroundjoin%
\pgfsetlinewidth{0.803000pt}%
\definecolor{currentstroke}{rgb}{0.686275,0.352941,0.313725}%
\pgfsetstrokecolor{currentstroke}%
\pgfsetdash{}{0pt}%
\pgfpathmoveto{\pgfqpoint{4.030582in}{0.810375in}}%
\pgfpathcurveto{\pgfqpoint{4.036107in}{0.810375in}}{\pgfqpoint{4.041407in}{0.812570in}}{\pgfqpoint{4.045314in}{0.816477in}}%
\pgfpathcurveto{\pgfqpoint{4.049220in}{0.820384in}}{\pgfqpoint{4.051416in}{0.825683in}}{\pgfqpoint{4.051416in}{0.831209in}}%
\pgfpathcurveto{\pgfqpoint{4.051416in}{0.836734in}}{\pgfqpoint{4.049220in}{0.842033in}}{\pgfqpoint{4.045314in}{0.845940in}}%
\pgfpathcurveto{\pgfqpoint{4.041407in}{0.849847in}}{\pgfqpoint{4.036107in}{0.852042in}}{\pgfqpoint{4.030582in}{0.852042in}}%
\pgfpathcurveto{\pgfqpoint{4.025057in}{0.852042in}}{\pgfqpoint{4.019758in}{0.849847in}}{\pgfqpoint{4.015851in}{0.845940in}}%
\pgfpathcurveto{\pgfqpoint{4.011944in}{0.842033in}}{\pgfqpoint{4.009749in}{0.836734in}}{\pgfqpoint{4.009749in}{0.831209in}}%
\pgfpathcurveto{\pgfqpoint{4.009749in}{0.825683in}}{\pgfqpoint{4.011944in}{0.820384in}}{\pgfqpoint{4.015851in}{0.816477in}}%
\pgfpathcurveto{\pgfqpoint{4.019758in}{0.812570in}}{\pgfqpoint{4.025057in}{0.810375in}}{\pgfqpoint{4.030582in}{0.810375in}}%
\pgfpathclose%
\pgfusepath{stroke}%
\end{pgfscope}%
\begin{pgfscope}%
\pgfpathrectangle{\pgfqpoint{0.438556in}{0.383578in}}{\pgfqpoint{4.650000in}{2.310000in}}%
\pgfusepath{clip}%
\pgfsetbuttcap%
\pgfsetroundjoin%
\pgfsetlinewidth{0.803000pt}%
\definecolor{currentstroke}{rgb}{0.686275,0.352941,0.313725}%
\pgfsetstrokecolor{currentstroke}%
\pgfsetdash{}{0pt}%
\pgfpathmoveto{\pgfqpoint{3.980600in}{0.854059in}}%
\pgfpathcurveto{\pgfqpoint{3.986125in}{0.854059in}}{\pgfqpoint{3.991425in}{0.856255in}}{\pgfqpoint{3.995332in}{0.860161in}}%
\pgfpathcurveto{\pgfqpoint{3.999239in}{0.864068in}}{\pgfqpoint{4.001434in}{0.869368in}}{\pgfqpoint{4.001434in}{0.874893in}}%
\pgfpathcurveto{\pgfqpoint{4.001434in}{0.880418in}}{\pgfqpoint{3.999239in}{0.885717in}}{\pgfqpoint{3.995332in}{0.889624in}}%
\pgfpathcurveto{\pgfqpoint{3.991425in}{0.893531in}}{\pgfqpoint{3.986125in}{0.895726in}}{\pgfqpoint{3.980600in}{0.895726in}}%
\pgfpathcurveto{\pgfqpoint{3.975075in}{0.895726in}}{\pgfqpoint{3.969776in}{0.893531in}}{\pgfqpoint{3.965869in}{0.889624in}}%
\pgfpathcurveto{\pgfqpoint{3.961962in}{0.885717in}}{\pgfqpoint{3.959767in}{0.880418in}}{\pgfqpoint{3.959767in}{0.874893in}}%
\pgfpathcurveto{\pgfqpoint{3.959767in}{0.869368in}}{\pgfqpoint{3.961962in}{0.864068in}}{\pgfqpoint{3.965869in}{0.860161in}}%
\pgfpathcurveto{\pgfqpoint{3.969776in}{0.856255in}}{\pgfqpoint{3.975075in}{0.854059in}}{\pgfqpoint{3.980600in}{0.854059in}}%
\pgfpathclose%
\pgfusepath{stroke}%
\end{pgfscope}%
\begin{pgfscope}%
\pgfpathrectangle{\pgfqpoint{0.438556in}{0.383578in}}{\pgfqpoint{4.650000in}{2.310000in}}%
\pgfusepath{clip}%
\pgfsetbuttcap%
\pgfsetroundjoin%
\pgfsetlinewidth{0.803000pt}%
\definecolor{currentstroke}{rgb}{0.686275,0.352941,0.313725}%
\pgfsetstrokecolor{currentstroke}%
\pgfsetdash{}{0pt}%
\pgfpathmoveto{\pgfqpoint{4.031077in}{0.875902in}}%
\pgfpathcurveto{\pgfqpoint{4.036602in}{0.875902in}}{\pgfqpoint{4.041902in}{0.878097in}}{\pgfqpoint{4.045808in}{0.882004in}}%
\pgfpathcurveto{\pgfqpoint{4.049715in}{0.885910in}}{\pgfqpoint{4.051910in}{0.891210in}}{\pgfqpoint{4.051910in}{0.896735in}}%
\pgfpathcurveto{\pgfqpoint{4.051910in}{0.902260in}}{\pgfqpoint{4.049715in}{0.907560in}}{\pgfqpoint{4.045808in}{0.911466in}}%
\pgfpathcurveto{\pgfqpoint{4.041902in}{0.915373in}}{\pgfqpoint{4.036602in}{0.917568in}}{\pgfqpoint{4.031077in}{0.917568in}}%
\pgfpathcurveto{\pgfqpoint{4.025552in}{0.917568in}}{\pgfqpoint{4.020252in}{0.915373in}}{\pgfqpoint{4.016346in}{0.911466in}}%
\pgfpathcurveto{\pgfqpoint{4.012439in}{0.907560in}}{\pgfqpoint{4.010244in}{0.902260in}}{\pgfqpoint{4.010244in}{0.896735in}}%
\pgfpathcurveto{\pgfqpoint{4.010244in}{0.891210in}}{\pgfqpoint{4.012439in}{0.885910in}}{\pgfqpoint{4.016346in}{0.882004in}}%
\pgfpathcurveto{\pgfqpoint{4.020252in}{0.878097in}}{\pgfqpoint{4.025552in}{0.875902in}}{\pgfqpoint{4.031077in}{0.875902in}}%
\pgfpathclose%
\pgfusepath{stroke}%
\end{pgfscope}%
\begin{pgfscope}%
\pgfpathrectangle{\pgfqpoint{0.438556in}{0.383578in}}{\pgfqpoint{4.650000in}{2.310000in}}%
\pgfusepath{clip}%
\pgfsetbuttcap%
\pgfsetroundjoin%
\pgfsetlinewidth{0.803000pt}%
\definecolor{currentstroke}{rgb}{0.686275,0.352941,0.313725}%
\pgfsetstrokecolor{currentstroke}%
\pgfsetdash{}{0pt}%
\pgfpathmoveto{\pgfqpoint{3.953617in}{0.919586in}}%
\pgfpathcurveto{\pgfqpoint{3.959142in}{0.919586in}}{\pgfqpoint{3.964441in}{0.921781in}}{\pgfqpoint{3.968348in}{0.925688in}}%
\pgfpathcurveto{\pgfqpoint{3.972255in}{0.929595in}}{\pgfqpoint{3.974450in}{0.934894in}}{\pgfqpoint{3.974450in}{0.940419in}}%
\pgfpathcurveto{\pgfqpoint{3.974450in}{0.945944in}}{\pgfqpoint{3.972255in}{0.951244in}}{\pgfqpoint{3.968348in}{0.955151in}}%
\pgfpathcurveto{\pgfqpoint{3.964441in}{0.959057in}}{\pgfqpoint{3.959142in}{0.961253in}}{\pgfqpoint{3.953617in}{0.961253in}}%
\pgfpathcurveto{\pgfqpoint{3.948092in}{0.961253in}}{\pgfqpoint{3.942792in}{0.959057in}}{\pgfqpoint{3.938885in}{0.955151in}}%
\pgfpathcurveto{\pgfqpoint{3.934978in}{0.951244in}}{\pgfqpoint{3.932783in}{0.945944in}}{\pgfqpoint{3.932783in}{0.940419in}}%
\pgfpathcurveto{\pgfqpoint{3.932783in}{0.934894in}}{\pgfqpoint{3.934978in}{0.929595in}}{\pgfqpoint{3.938885in}{0.925688in}}%
\pgfpathcurveto{\pgfqpoint{3.942792in}{0.921781in}}{\pgfqpoint{3.948092in}{0.919586in}}{\pgfqpoint{3.953617in}{0.919586in}}%
\pgfpathclose%
\pgfusepath{stroke}%
\end{pgfscope}%
\begin{pgfscope}%
\pgfpathrectangle{\pgfqpoint{0.438556in}{0.383578in}}{\pgfqpoint{4.650000in}{2.310000in}}%
\pgfusepath{clip}%
\pgfsetbuttcap%
\pgfsetroundjoin%
\pgfsetlinewidth{0.803000pt}%
\definecolor{currentstroke}{rgb}{0.686275,0.352941,0.313725}%
\pgfsetstrokecolor{currentstroke}%
\pgfsetdash{}{0pt}%
\pgfpathmoveto{\pgfqpoint{3.959068in}{0.941428in}}%
\pgfpathcurveto{\pgfqpoint{3.964593in}{0.941428in}}{\pgfqpoint{3.969893in}{0.943623in}}{\pgfqpoint{3.973800in}{0.947530in}}%
\pgfpathcurveto{\pgfqpoint{3.977707in}{0.951437in}}{\pgfqpoint{3.979902in}{0.956736in}}{\pgfqpoint{3.979902in}{0.962261in}}%
\pgfpathcurveto{\pgfqpoint{3.979902in}{0.967786in}}{\pgfqpoint{3.977707in}{0.973086in}}{\pgfqpoint{3.973800in}{0.976993in}}%
\pgfpathcurveto{\pgfqpoint{3.969893in}{0.980900in}}{\pgfqpoint{3.964593in}{0.983095in}}{\pgfqpoint{3.959068in}{0.983095in}}%
\pgfpathcurveto{\pgfqpoint{3.953543in}{0.983095in}}{\pgfqpoint{3.948244in}{0.980900in}}{\pgfqpoint{3.944337in}{0.976993in}}%
\pgfpathcurveto{\pgfqpoint{3.940430in}{0.973086in}}{\pgfqpoint{3.938235in}{0.967786in}}{\pgfqpoint{3.938235in}{0.962261in}}%
\pgfpathcurveto{\pgfqpoint{3.938235in}{0.956736in}}{\pgfqpoint{3.940430in}{0.951437in}}{\pgfqpoint{3.944337in}{0.947530in}}%
\pgfpathcurveto{\pgfqpoint{3.948244in}{0.943623in}}{\pgfqpoint{3.953543in}{0.941428in}}{\pgfqpoint{3.959068in}{0.941428in}}%
\pgfpathclose%
\pgfusepath{stroke}%
\end{pgfscope}%
\begin{pgfscope}%
\pgfpathrectangle{\pgfqpoint{0.438556in}{0.383578in}}{\pgfqpoint{4.650000in}{2.310000in}}%
\pgfusepath{clip}%
\pgfsetbuttcap%
\pgfsetroundjoin%
\pgfsetlinewidth{0.803000pt}%
\definecolor{currentstroke}{rgb}{0.686275,0.352941,0.313725}%
\pgfsetstrokecolor{currentstroke}%
\pgfsetdash{}{0pt}%
\pgfpathmoveto{\pgfqpoint{4.010122in}{1.006954in}}%
\pgfpathcurveto{\pgfqpoint{4.015647in}{1.006954in}}{\pgfqpoint{4.020947in}{1.009150in}}{\pgfqpoint{4.024854in}{1.013056in}}%
\pgfpathcurveto{\pgfqpoint{4.028760in}{1.016963in}}{\pgfqpoint{4.030956in}{1.022263in}}{\pgfqpoint{4.030956in}{1.027788in}}%
\pgfpathcurveto{\pgfqpoint{4.030956in}{1.033313in}}{\pgfqpoint{4.028760in}{1.038612in}}{\pgfqpoint{4.024854in}{1.042519in}}%
\pgfpathcurveto{\pgfqpoint{4.020947in}{1.046426in}}{\pgfqpoint{4.015647in}{1.048621in}}{\pgfqpoint{4.010122in}{1.048621in}}%
\pgfpathcurveto{\pgfqpoint{4.004597in}{1.048621in}}{\pgfqpoint{3.999298in}{1.046426in}}{\pgfqpoint{3.995391in}{1.042519in}}%
\pgfpathcurveto{\pgfqpoint{3.991484in}{1.038612in}}{\pgfqpoint{3.989289in}{1.033313in}}{\pgfqpoint{3.989289in}{1.027788in}}%
\pgfpathcurveto{\pgfqpoint{3.989289in}{1.022263in}}{\pgfqpoint{3.991484in}{1.016963in}}{\pgfqpoint{3.995391in}{1.013056in}}%
\pgfpathcurveto{\pgfqpoint{3.999298in}{1.009150in}}{\pgfqpoint{4.004597in}{1.006954in}}{\pgfqpoint{4.010122in}{1.006954in}}%
\pgfpathclose%
\pgfusepath{stroke}%
\end{pgfscope}%
\begin{pgfscope}%
\pgfpathrectangle{\pgfqpoint{0.438556in}{0.383578in}}{\pgfqpoint{4.650000in}{2.310000in}}%
\pgfusepath{clip}%
\pgfsetbuttcap%
\pgfsetroundjoin%
\pgfsetlinewidth{0.803000pt}%
\definecolor{currentstroke}{rgb}{0.686275,0.352941,0.313725}%
\pgfsetstrokecolor{currentstroke}%
\pgfsetdash{}{0pt}%
\pgfpathmoveto{\pgfqpoint{3.975195in}{1.028797in}}%
\pgfpathcurveto{\pgfqpoint{3.980720in}{1.028797in}}{\pgfqpoint{3.986019in}{1.030992in}}{\pgfqpoint{3.989926in}{1.034898in}}%
\pgfpathcurveto{\pgfqpoint{3.993833in}{1.038805in}}{\pgfqpoint{3.996028in}{1.044105in}}{\pgfqpoint{3.996028in}{1.049630in}}%
\pgfpathcurveto{\pgfqpoint{3.996028in}{1.055155in}}{\pgfqpoint{3.993833in}{1.060454in}}{\pgfqpoint{3.989926in}{1.064361in}}%
\pgfpathcurveto{\pgfqpoint{3.986019in}{1.068268in}}{\pgfqpoint{3.980720in}{1.070463in}}{\pgfqpoint{3.975195in}{1.070463in}}%
\pgfpathcurveto{\pgfqpoint{3.969669in}{1.070463in}}{\pgfqpoint{3.964370in}{1.068268in}}{\pgfqpoint{3.960463in}{1.064361in}}%
\pgfpathcurveto{\pgfqpoint{3.956556in}{1.060454in}}{\pgfqpoint{3.954361in}{1.055155in}}{\pgfqpoint{3.954361in}{1.049630in}}%
\pgfpathcurveto{\pgfqpoint{3.954361in}{1.044105in}}{\pgfqpoint{3.956556in}{1.038805in}}{\pgfqpoint{3.960463in}{1.034898in}}%
\pgfpathcurveto{\pgfqpoint{3.964370in}{1.030992in}}{\pgfqpoint{3.969669in}{1.028797in}}{\pgfqpoint{3.975195in}{1.028797in}}%
\pgfpathclose%
\pgfusepath{stroke}%
\end{pgfscope}%
\begin{pgfscope}%
\pgfpathrectangle{\pgfqpoint{0.438556in}{0.383578in}}{\pgfqpoint{4.650000in}{2.310000in}}%
\pgfusepath{clip}%
\pgfsetbuttcap%
\pgfsetroundjoin%
\pgfsetlinewidth{0.803000pt}%
\definecolor{currentstroke}{rgb}{0.686275,0.352941,0.313725}%
\pgfsetstrokecolor{currentstroke}%
\pgfsetdash{}{0pt}%
\pgfpathmoveto{\pgfqpoint{3.976789in}{1.072481in}}%
\pgfpathcurveto{\pgfqpoint{3.982314in}{1.072481in}}{\pgfqpoint{3.987613in}{1.074676in}}{\pgfqpoint{3.991520in}{1.078583in}}%
\pgfpathcurveto{\pgfqpoint{3.995427in}{1.082490in}}{\pgfqpoint{3.997622in}{1.087789in}}{\pgfqpoint{3.997622in}{1.093314in}}%
\pgfpathcurveto{\pgfqpoint{3.997622in}{1.098839in}}{\pgfqpoint{3.995427in}{1.104139in}}{\pgfqpoint{3.991520in}{1.108045in}}%
\pgfpathcurveto{\pgfqpoint{3.987613in}{1.111952in}}{\pgfqpoint{3.982314in}{1.114147in}}{\pgfqpoint{3.976789in}{1.114147in}}%
\pgfpathcurveto{\pgfqpoint{3.971264in}{1.114147in}}{\pgfqpoint{3.965964in}{1.111952in}}{\pgfqpoint{3.962057in}{1.108045in}}%
\pgfpathcurveto{\pgfqpoint{3.958151in}{1.104139in}}{\pgfqpoint{3.955955in}{1.098839in}}{\pgfqpoint{3.955955in}{1.093314in}}%
\pgfpathcurveto{\pgfqpoint{3.955955in}{1.087789in}}{\pgfqpoint{3.958151in}{1.082490in}}{\pgfqpoint{3.962057in}{1.078583in}}%
\pgfpathcurveto{\pgfqpoint{3.965964in}{1.074676in}}{\pgfqpoint{3.971264in}{1.072481in}}{\pgfqpoint{3.976789in}{1.072481in}}%
\pgfpathclose%
\pgfusepath{stroke}%
\end{pgfscope}%
\begin{pgfscope}%
\pgfpathrectangle{\pgfqpoint{0.438556in}{0.383578in}}{\pgfqpoint{4.650000in}{2.310000in}}%
\pgfusepath{clip}%
\pgfsetbuttcap%
\pgfsetroundjoin%
\pgfsetlinewidth{0.803000pt}%
\definecolor{currentstroke}{rgb}{0.686275,0.352941,0.313725}%
\pgfsetstrokecolor{currentstroke}%
\pgfsetdash{}{0pt}%
\pgfpathmoveto{\pgfqpoint{4.025882in}{1.159849in}}%
\pgfpathcurveto{\pgfqpoint{4.031407in}{1.159849in}}{\pgfqpoint{4.036706in}{1.162044in}}{\pgfqpoint{4.040613in}{1.165951in}}%
\pgfpathcurveto{\pgfqpoint{4.044520in}{1.169858in}}{\pgfqpoint{4.046715in}{1.175158in}}{\pgfqpoint{4.046715in}{1.180683in}}%
\pgfpathcurveto{\pgfqpoint{4.046715in}{1.186208in}}{\pgfqpoint{4.044520in}{1.191507in}}{\pgfqpoint{4.040613in}{1.195414in}}%
\pgfpathcurveto{\pgfqpoint{4.036706in}{1.199321in}}{\pgfqpoint{4.031407in}{1.201516in}}{\pgfqpoint{4.025882in}{1.201516in}}%
\pgfpathcurveto{\pgfqpoint{4.020357in}{1.201516in}}{\pgfqpoint{4.015057in}{1.199321in}}{\pgfqpoint{4.011150in}{1.195414in}}%
\pgfpathcurveto{\pgfqpoint{4.007244in}{1.191507in}}{\pgfqpoint{4.005048in}{1.186208in}}{\pgfqpoint{4.005048in}{1.180683in}}%
\pgfpathcurveto{\pgfqpoint{4.005048in}{1.175158in}}{\pgfqpoint{4.007244in}{1.169858in}}{\pgfqpoint{4.011150in}{1.165951in}}%
\pgfpathcurveto{\pgfqpoint{4.015057in}{1.162044in}}{\pgfqpoint{4.020357in}{1.159849in}}{\pgfqpoint{4.025882in}{1.159849in}}%
\pgfpathclose%
\pgfusepath{stroke}%
\end{pgfscope}%
\begin{pgfscope}%
\pgfpathrectangle{\pgfqpoint{0.438556in}{0.383578in}}{\pgfqpoint{4.650000in}{2.310000in}}%
\pgfusepath{clip}%
\pgfsetbuttcap%
\pgfsetroundjoin%
\pgfsetlinewidth{0.803000pt}%
\definecolor{currentstroke}{rgb}{0.686275,0.352941,0.313725}%
\pgfsetstrokecolor{currentstroke}%
\pgfsetdash{}{0pt}%
\pgfpathmoveto{\pgfqpoint{3.972299in}{1.203534in}}%
\pgfpathcurveto{\pgfqpoint{3.977824in}{1.203534in}}{\pgfqpoint{3.983124in}{1.205729in}}{\pgfqpoint{3.987031in}{1.209636in}}%
\pgfpathcurveto{\pgfqpoint{3.990937in}{1.213542in}}{\pgfqpoint{3.993132in}{1.218842in}}{\pgfqpoint{3.993132in}{1.224367in}}%
\pgfpathcurveto{\pgfqpoint{3.993132in}{1.229892in}}{\pgfqpoint{3.990937in}{1.235191in}}{\pgfqpoint{3.987031in}{1.239098in}}%
\pgfpathcurveto{\pgfqpoint{3.983124in}{1.243005in}}{\pgfqpoint{3.977824in}{1.245200in}}{\pgfqpoint{3.972299in}{1.245200in}}%
\pgfpathcurveto{\pgfqpoint{3.966774in}{1.245200in}}{\pgfqpoint{3.961475in}{1.243005in}}{\pgfqpoint{3.957568in}{1.239098in}}%
\pgfpathcurveto{\pgfqpoint{3.953661in}{1.235191in}}{\pgfqpoint{3.951466in}{1.229892in}}{\pgfqpoint{3.951466in}{1.224367in}}%
\pgfpathcurveto{\pgfqpoint{3.951466in}{1.218842in}}{\pgfqpoint{3.953661in}{1.213542in}}{\pgfqpoint{3.957568in}{1.209636in}}%
\pgfpathcurveto{\pgfqpoint{3.961475in}{1.205729in}}{\pgfqpoint{3.966774in}{1.203534in}}{\pgfqpoint{3.972299in}{1.203534in}}%
\pgfpathclose%
\pgfusepath{stroke}%
\end{pgfscope}%
\begin{pgfscope}%
\pgfpathrectangle{\pgfqpoint{0.438556in}{0.383578in}}{\pgfqpoint{4.650000in}{2.310000in}}%
\pgfusepath{clip}%
\pgfsetbuttcap%
\pgfsetroundjoin%
\pgfsetlinewidth{0.803000pt}%
\definecolor{currentstroke}{rgb}{0.686275,0.352941,0.313725}%
\pgfsetstrokecolor{currentstroke}%
\pgfsetdash{}{0pt}%
\pgfpathmoveto{\pgfqpoint{3.966142in}{1.290902in}}%
\pgfpathcurveto{\pgfqpoint{3.971667in}{1.290902in}}{\pgfqpoint{3.976966in}{1.293097in}}{\pgfqpoint{3.980873in}{1.297004in}}%
\pgfpathcurveto{\pgfqpoint{3.984780in}{1.300911in}}{\pgfqpoint{3.986975in}{1.306210in}}{\pgfqpoint{3.986975in}{1.311735in}}%
\pgfpathcurveto{\pgfqpoint{3.986975in}{1.317260in}}{\pgfqpoint{3.984780in}{1.322560in}}{\pgfqpoint{3.980873in}{1.326467in}}%
\pgfpathcurveto{\pgfqpoint{3.976966in}{1.330374in}}{\pgfqpoint{3.971667in}{1.332569in}}{\pgfqpoint{3.966142in}{1.332569in}}%
\pgfpathcurveto{\pgfqpoint{3.960617in}{1.332569in}}{\pgfqpoint{3.955317in}{1.330374in}}{\pgfqpoint{3.951411in}{1.326467in}}%
\pgfpathcurveto{\pgfqpoint{3.947504in}{1.322560in}}{\pgfqpoint{3.945309in}{1.317260in}}{\pgfqpoint{3.945309in}{1.311735in}}%
\pgfpathcurveto{\pgfqpoint{3.945309in}{1.306210in}}{\pgfqpoint{3.947504in}{1.300911in}}{\pgfqpoint{3.951411in}{1.297004in}}%
\pgfpathcurveto{\pgfqpoint{3.955317in}{1.293097in}}{\pgfqpoint{3.960617in}{1.290902in}}{\pgfqpoint{3.966142in}{1.290902in}}%
\pgfpathclose%
\pgfusepath{stroke}%
\end{pgfscope}%
\begin{pgfscope}%
\pgfpathrectangle{\pgfqpoint{0.438556in}{0.383578in}}{\pgfqpoint{4.650000in}{2.310000in}}%
\pgfusepath{clip}%
\pgfsetbuttcap%
\pgfsetroundjoin%
\pgfsetlinewidth{0.803000pt}%
\definecolor{currentstroke}{rgb}{0.686275,0.352941,0.313725}%
\pgfsetstrokecolor{currentstroke}%
\pgfsetdash{}{0pt}%
\pgfpathmoveto{\pgfqpoint{3.947377in}{1.312744in}}%
\pgfpathcurveto{\pgfqpoint{3.952902in}{1.312744in}}{\pgfqpoint{3.958202in}{1.314939in}}{\pgfqpoint{3.962108in}{1.318846in}}%
\pgfpathcurveto{\pgfqpoint{3.966015in}{1.322753in}}{\pgfqpoint{3.968210in}{1.328052in}}{\pgfqpoint{3.968210in}{1.333578in}}%
\pgfpathcurveto{\pgfqpoint{3.968210in}{1.339103in}}{\pgfqpoint{3.966015in}{1.344402in}}{\pgfqpoint{3.962108in}{1.348309in}}%
\pgfpathcurveto{\pgfqpoint{3.958202in}{1.352216in}}{\pgfqpoint{3.952902in}{1.354411in}}{\pgfqpoint{3.947377in}{1.354411in}}%
\pgfpathcurveto{\pgfqpoint{3.941852in}{1.354411in}}{\pgfqpoint{3.936552in}{1.352216in}}{\pgfqpoint{3.932646in}{1.348309in}}%
\pgfpathcurveto{\pgfqpoint{3.928739in}{1.344402in}}{\pgfqpoint{3.926544in}{1.339103in}}{\pgfqpoint{3.926544in}{1.333578in}}%
\pgfpathcurveto{\pgfqpoint{3.926544in}{1.328052in}}{\pgfqpoint{3.928739in}{1.322753in}}{\pgfqpoint{3.932646in}{1.318846in}}%
\pgfpathcurveto{\pgfqpoint{3.936552in}{1.314939in}}{\pgfqpoint{3.941852in}{1.312744in}}{\pgfqpoint{3.947377in}{1.312744in}}%
\pgfpathclose%
\pgfusepath{stroke}%
\end{pgfscope}%
\begin{pgfscope}%
\pgfpathrectangle{\pgfqpoint{0.438556in}{0.383578in}}{\pgfqpoint{4.650000in}{2.310000in}}%
\pgfusepath{clip}%
\pgfsetbuttcap%
\pgfsetroundjoin%
\pgfsetlinewidth{0.803000pt}%
\definecolor{currentstroke}{rgb}{0.686275,0.352941,0.313725}%
\pgfsetstrokecolor{currentstroke}%
\pgfsetdash{}{0pt}%
\pgfpathmoveto{\pgfqpoint{4.020256in}{1.334586in}}%
\pgfpathcurveto{\pgfqpoint{4.025781in}{1.334586in}}{\pgfqpoint{4.031081in}{1.336781in}}{\pgfqpoint{4.034987in}{1.340688in}}%
\pgfpathcurveto{\pgfqpoint{4.038894in}{1.344595in}}{\pgfqpoint{4.041089in}{1.349895in}}{\pgfqpoint{4.041089in}{1.355420in}}%
\pgfpathcurveto{\pgfqpoint{4.041089in}{1.360945in}}{\pgfqpoint{4.038894in}{1.366244in}}{\pgfqpoint{4.034987in}{1.370151in}}%
\pgfpathcurveto{\pgfqpoint{4.031081in}{1.374058in}}{\pgfqpoint{4.025781in}{1.376253in}}{\pgfqpoint{4.020256in}{1.376253in}}%
\pgfpathcurveto{\pgfqpoint{4.014731in}{1.376253in}}{\pgfqpoint{4.009431in}{1.374058in}}{\pgfqpoint{4.005525in}{1.370151in}}%
\pgfpathcurveto{\pgfqpoint{4.001618in}{1.366244in}}{\pgfqpoint{3.999423in}{1.360945in}}{\pgfqpoint{3.999423in}{1.355420in}}%
\pgfpathcurveto{\pgfqpoint{3.999423in}{1.349895in}}{\pgfqpoint{4.001618in}{1.344595in}}{\pgfqpoint{4.005525in}{1.340688in}}%
\pgfpathcurveto{\pgfqpoint{4.009431in}{1.336781in}}{\pgfqpoint{4.014731in}{1.334586in}}{\pgfqpoint{4.020256in}{1.334586in}}%
\pgfpathclose%
\pgfusepath{stroke}%
\end{pgfscope}%
\begin{pgfscope}%
\pgfpathrectangle{\pgfqpoint{0.438556in}{0.383578in}}{\pgfqpoint{4.650000in}{2.310000in}}%
\pgfusepath{clip}%
\pgfsetbuttcap%
\pgfsetroundjoin%
\pgfsetlinewidth{0.803000pt}%
\definecolor{currentstroke}{rgb}{0.686275,0.352941,0.313725}%
\pgfsetstrokecolor{currentstroke}%
\pgfsetdash{}{0pt}%
\pgfpathmoveto{\pgfqpoint{3.968552in}{1.443797in}}%
\pgfpathcurveto{\pgfqpoint{3.974077in}{1.443797in}}{\pgfqpoint{3.979376in}{1.445992in}}{\pgfqpoint{3.983283in}{1.449899in}}%
\pgfpathcurveto{\pgfqpoint{3.987190in}{1.453806in}}{\pgfqpoint{3.989385in}{1.459105in}}{\pgfqpoint{3.989385in}{1.464630in}}%
\pgfpathcurveto{\pgfqpoint{3.989385in}{1.470155in}}{\pgfqpoint{3.987190in}{1.475455in}}{\pgfqpoint{3.983283in}{1.479362in}}%
\pgfpathcurveto{\pgfqpoint{3.979376in}{1.483269in}}{\pgfqpoint{3.974077in}{1.485464in}}{\pgfqpoint{3.968552in}{1.485464in}}%
\pgfpathcurveto{\pgfqpoint{3.963027in}{1.485464in}}{\pgfqpoint{3.957727in}{1.483269in}}{\pgfqpoint{3.953820in}{1.479362in}}%
\pgfpathcurveto{\pgfqpoint{3.949913in}{1.475455in}}{\pgfqpoint{3.947718in}{1.470155in}}{\pgfqpoint{3.947718in}{1.464630in}}%
\pgfpathcurveto{\pgfqpoint{3.947718in}{1.459105in}}{\pgfqpoint{3.949913in}{1.453806in}}{\pgfqpoint{3.953820in}{1.449899in}}%
\pgfpathcurveto{\pgfqpoint{3.957727in}{1.445992in}}{\pgfqpoint{3.963027in}{1.443797in}}{\pgfqpoint{3.968552in}{1.443797in}}%
\pgfpathclose%
\pgfusepath{stroke}%
\end{pgfscope}%
\begin{pgfscope}%
\pgfpathrectangle{\pgfqpoint{0.438556in}{0.383578in}}{\pgfqpoint{4.650000in}{2.310000in}}%
\pgfusepath{clip}%
\pgfsetbuttcap%
\pgfsetroundjoin%
\pgfsetlinewidth{0.803000pt}%
\definecolor{currentstroke}{rgb}{0.686275,0.352941,0.313725}%
\pgfsetstrokecolor{currentstroke}%
\pgfsetdash{}{0pt}%
\pgfpathmoveto{\pgfqpoint{3.947652in}{1.487481in}}%
\pgfpathcurveto{\pgfqpoint{3.953177in}{1.487481in}}{\pgfqpoint{3.958476in}{1.489676in}}{\pgfqpoint{3.962383in}{1.493583in}}%
\pgfpathcurveto{\pgfqpoint{3.966290in}{1.497490in}}{\pgfqpoint{3.968485in}{1.502790in}}{\pgfqpoint{3.968485in}{1.508315in}}%
\pgfpathcurveto{\pgfqpoint{3.968485in}{1.513840in}}{\pgfqpoint{3.966290in}{1.519139in}}{\pgfqpoint{3.962383in}{1.523046in}}%
\pgfpathcurveto{\pgfqpoint{3.958476in}{1.526953in}}{\pgfqpoint{3.953177in}{1.529148in}}{\pgfqpoint{3.947652in}{1.529148in}}%
\pgfpathcurveto{\pgfqpoint{3.942127in}{1.529148in}}{\pgfqpoint{3.936827in}{1.526953in}}{\pgfqpoint{3.932920in}{1.523046in}}%
\pgfpathcurveto{\pgfqpoint{3.929014in}{1.519139in}}{\pgfqpoint{3.926819in}{1.513840in}}{\pgfqpoint{3.926819in}{1.508315in}}%
\pgfpathcurveto{\pgfqpoint{3.926819in}{1.502790in}}{\pgfqpoint{3.929014in}{1.497490in}}{\pgfqpoint{3.932920in}{1.493583in}}%
\pgfpathcurveto{\pgfqpoint{3.936827in}{1.489676in}}{\pgfqpoint{3.942127in}{1.487481in}}{\pgfqpoint{3.947652in}{1.487481in}}%
\pgfpathclose%
\pgfusepath{stroke}%
\end{pgfscope}%
\begin{pgfscope}%
\pgfpathrectangle{\pgfqpoint{0.438556in}{0.383578in}}{\pgfqpoint{4.650000in}{2.310000in}}%
\pgfusepath{clip}%
\pgfsetbuttcap%
\pgfsetroundjoin%
\pgfsetlinewidth{0.803000pt}%
\definecolor{currentstroke}{rgb}{0.686275,0.352941,0.313725}%
\pgfsetstrokecolor{currentstroke}%
\pgfsetdash{}{0pt}%
\pgfpathmoveto{\pgfqpoint{3.961222in}{1.531166in}}%
\pgfpathcurveto{\pgfqpoint{3.966747in}{1.531166in}}{\pgfqpoint{3.972046in}{1.533361in}}{\pgfqpoint{3.975953in}{1.537267in}}%
\pgfpathcurveto{\pgfqpoint{3.979860in}{1.541174in}}{\pgfqpoint{3.982055in}{1.546474in}}{\pgfqpoint{3.982055in}{1.551999in}}%
\pgfpathcurveto{\pgfqpoint{3.982055in}{1.557524in}}{\pgfqpoint{3.979860in}{1.562823in}}{\pgfqpoint{3.975953in}{1.566730in}}%
\pgfpathcurveto{\pgfqpoint{3.972046in}{1.570637in}}{\pgfqpoint{3.966747in}{1.572832in}}{\pgfqpoint{3.961222in}{1.572832in}}%
\pgfpathcurveto{\pgfqpoint{3.955697in}{1.572832in}}{\pgfqpoint{3.950397in}{1.570637in}}{\pgfqpoint{3.946490in}{1.566730in}}%
\pgfpathcurveto{\pgfqpoint{3.942583in}{1.562823in}}{\pgfqpoint{3.940388in}{1.557524in}}{\pgfqpoint{3.940388in}{1.551999in}}%
\pgfpathcurveto{\pgfqpoint{3.940388in}{1.546474in}}{\pgfqpoint{3.942583in}{1.541174in}}{\pgfqpoint{3.946490in}{1.537267in}}%
\pgfpathcurveto{\pgfqpoint{3.950397in}{1.533361in}}{\pgfqpoint{3.955697in}{1.531166in}}{\pgfqpoint{3.961222in}{1.531166in}}%
\pgfpathclose%
\pgfusepath{stroke}%
\end{pgfscope}%
\begin{pgfscope}%
\pgfpathrectangle{\pgfqpoint{0.438556in}{0.383578in}}{\pgfqpoint{4.650000in}{2.310000in}}%
\pgfusepath{clip}%
\pgfsetbuttcap%
\pgfsetroundjoin%
\pgfsetlinewidth{0.803000pt}%
\definecolor{currentstroke}{rgb}{0.686275,0.352941,0.313725}%
\pgfsetstrokecolor{currentstroke}%
\pgfsetdash{}{0pt}%
\pgfpathmoveto{\pgfqpoint{4.013604in}{1.684060in}}%
\pgfpathcurveto{\pgfqpoint{4.019129in}{1.684060in}}{\pgfqpoint{4.024429in}{1.686256in}}{\pgfqpoint{4.028335in}{1.690162in}}%
\pgfpathcurveto{\pgfqpoint{4.032242in}{1.694069in}}{\pgfqpoint{4.034437in}{1.699369in}}{\pgfqpoint{4.034437in}{1.704894in}}%
\pgfpathcurveto{\pgfqpoint{4.034437in}{1.710419in}}{\pgfqpoint{4.032242in}{1.715718in}}{\pgfqpoint{4.028335in}{1.719625in}}%
\pgfpathcurveto{\pgfqpoint{4.024429in}{1.723532in}}{\pgfqpoint{4.019129in}{1.725727in}}{\pgfqpoint{4.013604in}{1.725727in}}%
\pgfpathcurveto{\pgfqpoint{4.008079in}{1.725727in}}{\pgfqpoint{4.002779in}{1.723532in}}{\pgfqpoint{3.998873in}{1.719625in}}%
\pgfpathcurveto{\pgfqpoint{3.994966in}{1.715718in}}{\pgfqpoint{3.992771in}{1.710419in}}{\pgfqpoint{3.992771in}{1.704894in}}%
\pgfpathcurveto{\pgfqpoint{3.992771in}{1.699369in}}{\pgfqpoint{3.994966in}{1.694069in}}{\pgfqpoint{3.998873in}{1.690162in}}%
\pgfpathcurveto{\pgfqpoint{4.002779in}{1.686256in}}{\pgfqpoint{4.008079in}{1.684060in}}{\pgfqpoint{4.013604in}{1.684060in}}%
\pgfpathclose%
\pgfusepath{stroke}%
\end{pgfscope}%
\begin{pgfscope}%
\pgfpathrectangle{\pgfqpoint{0.438556in}{0.383578in}}{\pgfqpoint{4.650000in}{2.310000in}}%
\pgfusepath{clip}%
\pgfsetbuttcap%
\pgfsetroundjoin%
\pgfsetlinewidth{0.803000pt}%
\definecolor{currentstroke}{rgb}{0.686275,0.352941,0.313725}%
\pgfsetstrokecolor{currentstroke}%
\pgfsetdash{}{0pt}%
\pgfpathmoveto{\pgfqpoint{3.968185in}{1.771429in}}%
\pgfpathcurveto{\pgfqpoint{3.973710in}{1.771429in}}{\pgfqpoint{3.979010in}{1.773624in}}{\pgfqpoint{3.982917in}{1.777531in}}%
\pgfpathcurveto{\pgfqpoint{3.986823in}{1.781438in}}{\pgfqpoint{3.989018in}{1.786737in}}{\pgfqpoint{3.989018in}{1.792262in}}%
\pgfpathcurveto{\pgfqpoint{3.989018in}{1.797787in}}{\pgfqpoint{3.986823in}{1.803087in}}{\pgfqpoint{3.982917in}{1.806994in}}%
\pgfpathcurveto{\pgfqpoint{3.979010in}{1.810901in}}{\pgfqpoint{3.973710in}{1.813096in}}{\pgfqpoint{3.968185in}{1.813096in}}%
\pgfpathcurveto{\pgfqpoint{3.962660in}{1.813096in}}{\pgfqpoint{3.957361in}{1.810901in}}{\pgfqpoint{3.953454in}{1.806994in}}%
\pgfpathcurveto{\pgfqpoint{3.949547in}{1.803087in}}{\pgfqpoint{3.947352in}{1.797787in}}{\pgfqpoint{3.947352in}{1.792262in}}%
\pgfpathcurveto{\pgfqpoint{3.947352in}{1.786737in}}{\pgfqpoint{3.949547in}{1.781438in}}{\pgfqpoint{3.953454in}{1.777531in}}%
\pgfpathcurveto{\pgfqpoint{3.957361in}{1.773624in}}{\pgfqpoint{3.962660in}{1.771429in}}{\pgfqpoint{3.968185in}{1.771429in}}%
\pgfpathclose%
\pgfusepath{stroke}%
\end{pgfscope}%
\begin{pgfscope}%
\pgfpathrectangle{\pgfqpoint{0.438556in}{0.383578in}}{\pgfqpoint{4.650000in}{2.310000in}}%
\pgfusepath{clip}%
\pgfsetbuttcap%
\pgfsetroundjoin%
\pgfsetlinewidth{0.803000pt}%
\definecolor{currentstroke}{rgb}{0.686275,0.352941,0.313725}%
\pgfsetstrokecolor{currentstroke}%
\pgfsetdash{}{0pt}%
\pgfpathmoveto{\pgfqpoint{3.953534in}{1.793271in}}%
\pgfpathcurveto{\pgfqpoint{3.959059in}{1.793271in}}{\pgfqpoint{3.964359in}{1.795466in}}{\pgfqpoint{3.968266in}{1.799373in}}%
\pgfpathcurveto{\pgfqpoint{3.972172in}{1.803280in}}{\pgfqpoint{3.974368in}{1.808579in}}{\pgfqpoint{3.974368in}{1.814104in}}%
\pgfpathcurveto{\pgfqpoint{3.974368in}{1.819630in}}{\pgfqpoint{3.972172in}{1.824929in}}{\pgfqpoint{3.968266in}{1.828836in}}%
\pgfpathcurveto{\pgfqpoint{3.964359in}{1.832743in}}{\pgfqpoint{3.959059in}{1.834938in}}{\pgfqpoint{3.953534in}{1.834938in}}%
\pgfpathcurveto{\pgfqpoint{3.948009in}{1.834938in}}{\pgfqpoint{3.942710in}{1.832743in}}{\pgfqpoint{3.938803in}{1.828836in}}%
\pgfpathcurveto{\pgfqpoint{3.934896in}{1.824929in}}{\pgfqpoint{3.932701in}{1.819630in}}{\pgfqpoint{3.932701in}{1.814104in}}%
\pgfpathcurveto{\pgfqpoint{3.932701in}{1.808579in}}{\pgfqpoint{3.934896in}{1.803280in}}{\pgfqpoint{3.938803in}{1.799373in}}%
\pgfpathcurveto{\pgfqpoint{3.942710in}{1.795466in}}{\pgfqpoint{3.948009in}{1.793271in}}{\pgfqpoint{3.953534in}{1.793271in}}%
\pgfpathclose%
\pgfusepath{stroke}%
\end{pgfscope}%
\begin{pgfscope}%
\pgfpathrectangle{\pgfqpoint{0.438556in}{0.383578in}}{\pgfqpoint{4.650000in}{2.310000in}}%
\pgfusepath{clip}%
\pgfsetbuttcap%
\pgfsetroundjoin%
\pgfsetlinewidth{0.803000pt}%
\definecolor{currentstroke}{rgb}{0.686275,0.352941,0.313725}%
\pgfsetstrokecolor{currentstroke}%
\pgfsetdash{}{0pt}%
\pgfpathmoveto{\pgfqpoint{3.988755in}{1.836955in}}%
\pgfpathcurveto{\pgfqpoint{3.994280in}{1.836955in}}{\pgfqpoint{3.999580in}{1.839151in}}{\pgfqpoint{4.003486in}{1.843057in}}%
\pgfpathcurveto{\pgfqpoint{4.007393in}{1.846964in}}{\pgfqpoint{4.009588in}{1.852264in}}{\pgfqpoint{4.009588in}{1.857789in}}%
\pgfpathcurveto{\pgfqpoint{4.009588in}{1.863314in}}{\pgfqpoint{4.007393in}{1.868613in}}{\pgfqpoint{4.003486in}{1.872520in}}%
\pgfpathcurveto{\pgfqpoint{3.999580in}{1.876427in}}{\pgfqpoint{3.994280in}{1.878622in}}{\pgfqpoint{3.988755in}{1.878622in}}%
\pgfpathcurveto{\pgfqpoint{3.983230in}{1.878622in}}{\pgfqpoint{3.977931in}{1.876427in}}{\pgfqpoint{3.974024in}{1.872520in}}%
\pgfpathcurveto{\pgfqpoint{3.970117in}{1.868613in}}{\pgfqpoint{3.967922in}{1.863314in}}{\pgfqpoint{3.967922in}{1.857789in}}%
\pgfpathcurveto{\pgfqpoint{3.967922in}{1.852264in}}{\pgfqpoint{3.970117in}{1.846964in}}{\pgfqpoint{3.974024in}{1.843057in}}%
\pgfpathcurveto{\pgfqpoint{3.977931in}{1.839151in}}{\pgfqpoint{3.983230in}{1.836955in}}{\pgfqpoint{3.988755in}{1.836955in}}%
\pgfpathclose%
\pgfusepath{stroke}%
\end{pgfscope}%
\begin{pgfscope}%
\pgfpathrectangle{\pgfqpoint{0.438556in}{0.383578in}}{\pgfqpoint{4.650000in}{2.310000in}}%
\pgfusepath{clip}%
\pgfsetbuttcap%
\pgfsetroundjoin%
\pgfsetlinewidth{0.803000pt}%
\definecolor{currentstroke}{rgb}{0.686275,0.352941,0.313725}%
\pgfsetstrokecolor{currentstroke}%
\pgfsetdash{}{0pt}%
\pgfpathmoveto{\pgfqpoint{3.993062in}{1.858798in}}%
\pgfpathcurveto{\pgfqpoint{3.998587in}{1.858798in}}{\pgfqpoint{4.003886in}{1.860993in}}{\pgfqpoint{4.007793in}{1.864899in}}%
\pgfpathcurveto{\pgfqpoint{4.011700in}{1.868806in}}{\pgfqpoint{4.013895in}{1.874106in}}{\pgfqpoint{4.013895in}{1.879631in}}%
\pgfpathcurveto{\pgfqpoint{4.013895in}{1.885156in}}{\pgfqpoint{4.011700in}{1.890455in}}{\pgfqpoint{4.007793in}{1.894362in}}%
\pgfpathcurveto{\pgfqpoint{4.003886in}{1.898269in}}{\pgfqpoint{3.998587in}{1.900464in}}{\pgfqpoint{3.993062in}{1.900464in}}%
\pgfpathcurveto{\pgfqpoint{3.987536in}{1.900464in}}{\pgfqpoint{3.982237in}{1.898269in}}{\pgfqpoint{3.978330in}{1.894362in}}%
\pgfpathcurveto{\pgfqpoint{3.974423in}{1.890455in}}{\pgfqpoint{3.972228in}{1.885156in}}{\pgfqpoint{3.972228in}{1.879631in}}%
\pgfpathcurveto{\pgfqpoint{3.972228in}{1.874106in}}{\pgfqpoint{3.974423in}{1.868806in}}{\pgfqpoint{3.978330in}{1.864899in}}%
\pgfpathcurveto{\pgfqpoint{3.982237in}{1.860993in}}{\pgfqpoint{3.987536in}{1.858798in}}{\pgfqpoint{3.993062in}{1.858798in}}%
\pgfpathclose%
\pgfusepath{stroke}%
\end{pgfscope}%
\begin{pgfscope}%
\pgfpathrectangle{\pgfqpoint{0.438556in}{0.383578in}}{\pgfqpoint{4.650000in}{2.310000in}}%
\pgfusepath{clip}%
\pgfsetbuttcap%
\pgfsetroundjoin%
\pgfsetlinewidth{0.803000pt}%
\definecolor{currentstroke}{rgb}{0.686275,0.352941,0.313725}%
\pgfsetstrokecolor{currentstroke}%
\pgfsetdash{}{0pt}%
\pgfpathmoveto{\pgfqpoint{3.983725in}{1.924324in}}%
\pgfpathcurveto{\pgfqpoint{3.989250in}{1.924324in}}{\pgfqpoint{3.994549in}{1.926519in}}{\pgfqpoint{3.998456in}{1.930426in}}%
\pgfpathcurveto{\pgfqpoint{4.002363in}{1.934333in}}{\pgfqpoint{4.004558in}{1.939632in}}{\pgfqpoint{4.004558in}{1.945157in}}%
\pgfpathcurveto{\pgfqpoint{4.004558in}{1.950682in}}{\pgfqpoint{4.002363in}{1.955982in}}{\pgfqpoint{3.998456in}{1.959889in}}%
\pgfpathcurveto{\pgfqpoint{3.994549in}{1.963795in}}{\pgfqpoint{3.989250in}{1.965991in}}{\pgfqpoint{3.983725in}{1.965991in}}%
\pgfpathcurveto{\pgfqpoint{3.978200in}{1.965991in}}{\pgfqpoint{3.972900in}{1.963795in}}{\pgfqpoint{3.968993in}{1.959889in}}%
\pgfpathcurveto{\pgfqpoint{3.965087in}{1.955982in}}{\pgfqpoint{3.962892in}{1.950682in}}{\pgfqpoint{3.962892in}{1.945157in}}%
\pgfpathcurveto{\pgfqpoint{3.962892in}{1.939632in}}{\pgfqpoint{3.965087in}{1.934333in}}{\pgfqpoint{3.968993in}{1.930426in}}%
\pgfpathcurveto{\pgfqpoint{3.972900in}{1.926519in}}{\pgfqpoint{3.978200in}{1.924324in}}{\pgfqpoint{3.983725in}{1.924324in}}%
\pgfpathclose%
\pgfusepath{stroke}%
\end{pgfscope}%
\begin{pgfscope}%
\pgfpathrectangle{\pgfqpoint{0.438556in}{0.383578in}}{\pgfqpoint{4.650000in}{2.310000in}}%
\pgfusepath{clip}%
\pgfsetbuttcap%
\pgfsetroundjoin%
\pgfsetlinewidth{0.803000pt}%
\definecolor{currentstroke}{rgb}{0.686275,0.352941,0.313725}%
\pgfsetstrokecolor{currentstroke}%
\pgfsetdash{}{0pt}%
\pgfpathmoveto{\pgfqpoint{3.960901in}{1.989850in}}%
\pgfpathcurveto{\pgfqpoint{3.966426in}{1.989850in}}{\pgfqpoint{3.971725in}{1.992045in}}{\pgfqpoint{3.975632in}{1.995952in}}%
\pgfpathcurveto{\pgfqpoint{3.979539in}{1.999859in}}{\pgfqpoint{3.981734in}{2.005159in}}{\pgfqpoint{3.981734in}{2.010684in}}%
\pgfpathcurveto{\pgfqpoint{3.981734in}{2.016209in}}{\pgfqpoint{3.979539in}{2.021508in}}{\pgfqpoint{3.975632in}{2.025415in}}%
\pgfpathcurveto{\pgfqpoint{3.971725in}{2.029322in}}{\pgfqpoint{3.966426in}{2.031517in}}{\pgfqpoint{3.960901in}{2.031517in}}%
\pgfpathcurveto{\pgfqpoint{3.955376in}{2.031517in}}{\pgfqpoint{3.950076in}{2.029322in}}{\pgfqpoint{3.946170in}{2.025415in}}%
\pgfpathcurveto{\pgfqpoint{3.942263in}{2.021508in}}{\pgfqpoint{3.940068in}{2.016209in}}{\pgfqpoint{3.940068in}{2.010684in}}%
\pgfpathcurveto{\pgfqpoint{3.940068in}{2.005159in}}{\pgfqpoint{3.942263in}{1.999859in}}{\pgfqpoint{3.946170in}{1.995952in}}%
\pgfpathcurveto{\pgfqpoint{3.950076in}{1.992045in}}{\pgfqpoint{3.955376in}{1.989850in}}{\pgfqpoint{3.960901in}{1.989850in}}%
\pgfpathclose%
\pgfusepath{stroke}%
\end{pgfscope}%
\begin{pgfscope}%
\pgfpathrectangle{\pgfqpoint{0.438556in}{0.383578in}}{\pgfqpoint{4.650000in}{2.310000in}}%
\pgfusepath{clip}%
\pgfsetbuttcap%
\pgfsetroundjoin%
\pgfsetlinewidth{0.803000pt}%
\definecolor{currentstroke}{rgb}{0.686275,0.352941,0.313725}%
\pgfsetstrokecolor{currentstroke}%
\pgfsetdash{}{0pt}%
\pgfpathmoveto{\pgfqpoint{3.985127in}{2.011692in}}%
\pgfpathcurveto{\pgfqpoint{3.990652in}{2.011692in}}{\pgfqpoint{3.995951in}{2.013888in}}{\pgfqpoint{3.999858in}{2.017794in}}%
\pgfpathcurveto{\pgfqpoint{4.003765in}{2.021701in}}{\pgfqpoint{4.005960in}{2.027001in}}{\pgfqpoint{4.005960in}{2.032526in}}%
\pgfpathcurveto{\pgfqpoint{4.005960in}{2.038051in}}{\pgfqpoint{4.003765in}{2.043350in}}{\pgfqpoint{3.999858in}{2.047257in}}%
\pgfpathcurveto{\pgfqpoint{3.995951in}{2.051164in}}{\pgfqpoint{3.990652in}{2.053359in}}{\pgfqpoint{3.985127in}{2.053359in}}%
\pgfpathcurveto{\pgfqpoint{3.979602in}{2.053359in}}{\pgfqpoint{3.974302in}{2.051164in}}{\pgfqpoint{3.970395in}{2.047257in}}%
\pgfpathcurveto{\pgfqpoint{3.966489in}{2.043350in}}{\pgfqpoint{3.964293in}{2.038051in}}{\pgfqpoint{3.964293in}{2.032526in}}%
\pgfpathcurveto{\pgfqpoint{3.964293in}{2.027001in}}{\pgfqpoint{3.966489in}{2.021701in}}{\pgfqpoint{3.970395in}{2.017794in}}%
\pgfpathcurveto{\pgfqpoint{3.974302in}{2.013888in}}{\pgfqpoint{3.979602in}{2.011692in}}{\pgfqpoint{3.985127in}{2.011692in}}%
\pgfpathclose%
\pgfusepath{stroke}%
\end{pgfscope}%
\begin{pgfscope}%
\pgfpathrectangle{\pgfqpoint{0.438556in}{0.383578in}}{\pgfqpoint{4.650000in}{2.310000in}}%
\pgfusepath{clip}%
\pgfsetbuttcap%
\pgfsetroundjoin%
\pgfsetlinewidth{0.803000pt}%
\definecolor{currentstroke}{rgb}{0.686275,0.352941,0.313725}%
\pgfsetstrokecolor{currentstroke}%
\pgfsetdash{}{0pt}%
\pgfpathmoveto{\pgfqpoint{3.964923in}{2.077219in}}%
\pgfpathcurveto{\pgfqpoint{3.970448in}{2.077219in}}{\pgfqpoint{3.975748in}{2.079414in}}{\pgfqpoint{3.979655in}{2.083321in}}%
\pgfpathcurveto{\pgfqpoint{3.983561in}{2.087228in}}{\pgfqpoint{3.985757in}{2.092527in}}{\pgfqpoint{3.985757in}{2.098052in}}%
\pgfpathcurveto{\pgfqpoint{3.985757in}{2.103577in}}{\pgfqpoint{3.983561in}{2.108877in}}{\pgfqpoint{3.979655in}{2.112784in}}%
\pgfpathcurveto{\pgfqpoint{3.975748in}{2.116690in}}{\pgfqpoint{3.970448in}{2.118885in}}{\pgfqpoint{3.964923in}{2.118885in}}%
\pgfpathcurveto{\pgfqpoint{3.959398in}{2.118885in}}{\pgfqpoint{3.954099in}{2.116690in}}{\pgfqpoint{3.950192in}{2.112784in}}%
\pgfpathcurveto{\pgfqpoint{3.946285in}{2.108877in}}{\pgfqpoint{3.944090in}{2.103577in}}{\pgfqpoint{3.944090in}{2.098052in}}%
\pgfpathcurveto{\pgfqpoint{3.944090in}{2.092527in}}{\pgfqpoint{3.946285in}{2.087228in}}{\pgfqpoint{3.950192in}{2.083321in}}%
\pgfpathcurveto{\pgfqpoint{3.954099in}{2.079414in}}{\pgfqpoint{3.959398in}{2.077219in}}{\pgfqpoint{3.964923in}{2.077219in}}%
\pgfpathclose%
\pgfusepath{stroke}%
\end{pgfscope}%
\begin{pgfscope}%
\pgfpathrectangle{\pgfqpoint{0.438556in}{0.383578in}}{\pgfqpoint{4.650000in}{2.310000in}}%
\pgfusepath{clip}%
\pgfsetbuttcap%
\pgfsetroundjoin%
\pgfsetlinewidth{0.803000pt}%
\definecolor{currentstroke}{rgb}{0.686275,0.352941,0.313725}%
\pgfsetstrokecolor{currentstroke}%
\pgfsetdash{}{0pt}%
\pgfpathmoveto{\pgfqpoint{4.002701in}{2.099061in}}%
\pgfpathcurveto{\pgfqpoint{4.008226in}{2.099061in}}{\pgfqpoint{4.013525in}{2.101256in}}{\pgfqpoint{4.017432in}{2.105163in}}%
\pgfpathcurveto{\pgfqpoint{4.021339in}{2.109070in}}{\pgfqpoint{4.023534in}{2.114369in}}{\pgfqpoint{4.023534in}{2.119894in}}%
\pgfpathcurveto{\pgfqpoint{4.023534in}{2.125419in}}{\pgfqpoint{4.021339in}{2.130719in}}{\pgfqpoint{4.017432in}{2.134626in}}%
\pgfpathcurveto{\pgfqpoint{4.013525in}{2.138532in}}{\pgfqpoint{4.008226in}{2.140728in}}{\pgfqpoint{4.002701in}{2.140728in}}%
\pgfpathcurveto{\pgfqpoint{3.997175in}{2.140728in}}{\pgfqpoint{3.991876in}{2.138532in}}{\pgfqpoint{3.987969in}{2.134626in}}%
\pgfpathcurveto{\pgfqpoint{3.984062in}{2.130719in}}{\pgfqpoint{3.981867in}{2.125419in}}{\pgfqpoint{3.981867in}{2.119894in}}%
\pgfpathcurveto{\pgfqpoint{3.981867in}{2.114369in}}{\pgfqpoint{3.984062in}{2.109070in}}{\pgfqpoint{3.987969in}{2.105163in}}%
\pgfpathcurveto{\pgfqpoint{3.991876in}{2.101256in}}{\pgfqpoint{3.997175in}{2.099061in}}{\pgfqpoint{4.002701in}{2.099061in}}%
\pgfpathclose%
\pgfusepath{stroke}%
\end{pgfscope}%
\begin{pgfscope}%
\pgfpathrectangle{\pgfqpoint{0.438556in}{0.383578in}}{\pgfqpoint{4.650000in}{2.310000in}}%
\pgfusepath{clip}%
\pgfsetbuttcap%
\pgfsetroundjoin%
\pgfsetlinewidth{0.803000pt}%
\definecolor{currentstroke}{rgb}{0.686275,0.352941,0.313725}%
\pgfsetstrokecolor{currentstroke}%
\pgfsetdash{}{0pt}%
\pgfpathmoveto{\pgfqpoint{4.024599in}{2.164587in}}%
\pgfpathcurveto{\pgfqpoint{4.030124in}{2.164587in}}{\pgfqpoint{4.035424in}{2.166782in}}{\pgfqpoint{4.039330in}{2.170689in}}%
\pgfpathcurveto{\pgfqpoint{4.043237in}{2.174596in}}{\pgfqpoint{4.045432in}{2.179896in}}{\pgfqpoint{4.045432in}{2.185421in}}%
\pgfpathcurveto{\pgfqpoint{4.045432in}{2.190946in}}{\pgfqpoint{4.043237in}{2.196245in}}{\pgfqpoint{4.039330in}{2.200152in}}%
\pgfpathcurveto{\pgfqpoint{4.035424in}{2.204059in}}{\pgfqpoint{4.030124in}{2.206254in}}{\pgfqpoint{4.024599in}{2.206254in}}%
\pgfpathcurveto{\pgfqpoint{4.019074in}{2.206254in}}{\pgfqpoint{4.013774in}{2.204059in}}{\pgfqpoint{4.009868in}{2.200152in}}%
\pgfpathcurveto{\pgfqpoint{4.005961in}{2.196245in}}{\pgfqpoint{4.003766in}{2.190946in}}{\pgfqpoint{4.003766in}{2.185421in}}%
\pgfpathcurveto{\pgfqpoint{4.003766in}{2.179896in}}{\pgfqpoint{4.005961in}{2.174596in}}{\pgfqpoint{4.009868in}{2.170689in}}%
\pgfpathcurveto{\pgfqpoint{4.013774in}{2.166782in}}{\pgfqpoint{4.019074in}{2.164587in}}{\pgfqpoint{4.024599in}{2.164587in}}%
\pgfpathclose%
\pgfusepath{stroke}%
\end{pgfscope}%
\begin{pgfscope}%
\pgfpathrectangle{\pgfqpoint{0.438556in}{0.383578in}}{\pgfqpoint{4.650000in}{2.310000in}}%
\pgfusepath{clip}%
\pgfsetbuttcap%
\pgfsetroundjoin%
\pgfsetlinewidth{0.803000pt}%
\definecolor{currentstroke}{rgb}{0.686275,0.352941,0.313725}%
\pgfsetstrokecolor{currentstroke}%
\pgfsetdash{}{0pt}%
\pgfpathmoveto{\pgfqpoint{3.993171in}{2.230114in}}%
\pgfpathcurveto{\pgfqpoint{3.998697in}{2.230114in}}{\pgfqpoint{4.003996in}{2.232309in}}{\pgfqpoint{4.007903in}{2.236216in}}%
\pgfpathcurveto{\pgfqpoint{4.011810in}{2.240123in}}{\pgfqpoint{4.014005in}{2.245422in}}{\pgfqpoint{4.014005in}{2.250947in}}%
\pgfpathcurveto{\pgfqpoint{4.014005in}{2.256472in}}{\pgfqpoint{4.011810in}{2.261772in}}{\pgfqpoint{4.007903in}{2.265678in}}%
\pgfpathcurveto{\pgfqpoint{4.003996in}{2.269585in}}{\pgfqpoint{3.998697in}{2.271780in}}{\pgfqpoint{3.993171in}{2.271780in}}%
\pgfpathcurveto{\pgfqpoint{3.987646in}{2.271780in}}{\pgfqpoint{3.982347in}{2.269585in}}{\pgfqpoint{3.978440in}{2.265678in}}%
\pgfpathcurveto{\pgfqpoint{3.974533in}{2.261772in}}{\pgfqpoint{3.972338in}{2.256472in}}{\pgfqpoint{3.972338in}{2.250947in}}%
\pgfpathcurveto{\pgfqpoint{3.972338in}{2.245422in}}{\pgfqpoint{3.974533in}{2.240123in}}{\pgfqpoint{3.978440in}{2.236216in}}%
\pgfpathcurveto{\pgfqpoint{3.982347in}{2.232309in}}{\pgfqpoint{3.987646in}{2.230114in}}{\pgfqpoint{3.993171in}{2.230114in}}%
\pgfpathclose%
\pgfusepath{stroke}%
\end{pgfscope}%
\begin{pgfscope}%
\pgfpathrectangle{\pgfqpoint{0.438556in}{0.383578in}}{\pgfqpoint{4.650000in}{2.310000in}}%
\pgfusepath{clip}%
\pgfsetbuttcap%
\pgfsetroundjoin%
\pgfsetlinewidth{0.803000pt}%
\definecolor{currentstroke}{rgb}{0.686275,0.352941,0.313725}%
\pgfsetstrokecolor{currentstroke}%
\pgfsetdash{}{0pt}%
\pgfpathmoveto{\pgfqpoint{3.983303in}{2.251956in}}%
\pgfpathcurveto{\pgfqpoint{3.988828in}{2.251956in}}{\pgfqpoint{3.994128in}{2.254151in}}{\pgfqpoint{3.998035in}{2.258058in}}%
\pgfpathcurveto{\pgfqpoint{4.001942in}{2.261965in}}{\pgfqpoint{4.004137in}{2.267264in}}{\pgfqpoint{4.004137in}{2.272789in}}%
\pgfpathcurveto{\pgfqpoint{4.004137in}{2.278314in}}{\pgfqpoint{4.001942in}{2.283614in}}{\pgfqpoint{3.998035in}{2.287521in}}%
\pgfpathcurveto{\pgfqpoint{3.994128in}{2.291427in}}{\pgfqpoint{3.988828in}{2.293623in}}{\pgfqpoint{3.983303in}{2.293623in}}%
\pgfpathcurveto{\pgfqpoint{3.977778in}{2.293623in}}{\pgfqpoint{3.972479in}{2.291427in}}{\pgfqpoint{3.968572in}{2.287521in}}%
\pgfpathcurveto{\pgfqpoint{3.964665in}{2.283614in}}{\pgfqpoint{3.962470in}{2.278314in}}{\pgfqpoint{3.962470in}{2.272789in}}%
\pgfpathcurveto{\pgfqpoint{3.962470in}{2.267264in}}{\pgfqpoint{3.964665in}{2.261965in}}{\pgfqpoint{3.968572in}{2.258058in}}%
\pgfpathcurveto{\pgfqpoint{3.972479in}{2.254151in}}{\pgfqpoint{3.977778in}{2.251956in}}{\pgfqpoint{3.983303in}{2.251956in}}%
\pgfpathclose%
\pgfusepath{stroke}%
\end{pgfscope}%
\begin{pgfscope}%
\pgfpathrectangle{\pgfqpoint{0.438556in}{0.383578in}}{\pgfqpoint{4.650000in}{2.310000in}}%
\pgfusepath{clip}%
\pgfsetbuttcap%
\pgfsetroundjoin%
\pgfsetlinewidth{0.803000pt}%
\definecolor{currentstroke}{rgb}{0.686275,0.352941,0.313725}%
\pgfsetstrokecolor{currentstroke}%
\pgfsetdash{}{0pt}%
\pgfpathmoveto{\pgfqpoint{3.992017in}{2.295640in}}%
\pgfpathcurveto{\pgfqpoint{3.997542in}{2.295640in}}{\pgfqpoint{4.002842in}{2.297835in}}{\pgfqpoint{4.006748in}{2.301742in}}%
\pgfpathcurveto{\pgfqpoint{4.010655in}{2.305649in}}{\pgfqpoint{4.012850in}{2.310948in}}{\pgfqpoint{4.012850in}{2.316473in}}%
\pgfpathcurveto{\pgfqpoint{4.012850in}{2.321999in}}{\pgfqpoint{4.010655in}{2.327298in}}{\pgfqpoint{4.006748in}{2.331205in}}%
\pgfpathcurveto{\pgfqpoint{4.002842in}{2.335112in}}{\pgfqpoint{3.997542in}{2.337307in}}{\pgfqpoint{3.992017in}{2.337307in}}%
\pgfpathcurveto{\pgfqpoint{3.986492in}{2.337307in}}{\pgfqpoint{3.981192in}{2.335112in}}{\pgfqpoint{3.977286in}{2.331205in}}%
\pgfpathcurveto{\pgfqpoint{3.973379in}{2.327298in}}{\pgfqpoint{3.971184in}{2.321999in}}{\pgfqpoint{3.971184in}{2.316473in}}%
\pgfpathcurveto{\pgfqpoint{3.971184in}{2.310948in}}{\pgfqpoint{3.973379in}{2.305649in}}{\pgfqpoint{3.977286in}{2.301742in}}%
\pgfpathcurveto{\pgfqpoint{3.981192in}{2.297835in}}{\pgfqpoint{3.986492in}{2.295640in}}{\pgfqpoint{3.992017in}{2.295640in}}%
\pgfpathclose%
\pgfusepath{stroke}%
\end{pgfscope}%
\begin{pgfscope}%
\pgfpathrectangle{\pgfqpoint{0.438556in}{0.383578in}}{\pgfqpoint{4.650000in}{2.310000in}}%
\pgfusepath{clip}%
\pgfsetbuttcap%
\pgfsetroundjoin%
\pgfsetlinewidth{0.803000pt}%
\definecolor{currentstroke}{rgb}{0.686275,0.352941,0.313725}%
\pgfsetstrokecolor{currentstroke}%
\pgfsetdash{}{0pt}%
\pgfpathmoveto{\pgfqpoint{3.991815in}{2.317482in}}%
\pgfpathcurveto{\pgfqpoint{3.997340in}{2.317482in}}{\pgfqpoint{4.002640in}{2.319677in}}{\pgfqpoint{4.006547in}{2.323584in}}%
\pgfpathcurveto{\pgfqpoint{4.010454in}{2.327491in}}{\pgfqpoint{4.012649in}{2.332791in}}{\pgfqpoint{4.012649in}{2.338316in}}%
\pgfpathcurveto{\pgfqpoint{4.012649in}{2.343841in}}{\pgfqpoint{4.010454in}{2.349140in}}{\pgfqpoint{4.006547in}{2.353047in}}%
\pgfpathcurveto{\pgfqpoint{4.002640in}{2.356954in}}{\pgfqpoint{3.997340in}{2.359149in}}{\pgfqpoint{3.991815in}{2.359149in}}%
\pgfpathcurveto{\pgfqpoint{3.986290in}{2.359149in}}{\pgfqpoint{3.980991in}{2.356954in}}{\pgfqpoint{3.977084in}{2.353047in}}%
\pgfpathcurveto{\pgfqpoint{3.973177in}{2.349140in}}{\pgfqpoint{3.970982in}{2.343841in}}{\pgfqpoint{3.970982in}{2.338316in}}%
\pgfpathcurveto{\pgfqpoint{3.970982in}{2.332791in}}{\pgfqpoint{3.973177in}{2.327491in}}{\pgfqpoint{3.977084in}{2.323584in}}%
\pgfpathcurveto{\pgfqpoint{3.980991in}{2.319677in}}{\pgfqpoint{3.986290in}{2.317482in}}{\pgfqpoint{3.991815in}{2.317482in}}%
\pgfpathclose%
\pgfusepath{stroke}%
\end{pgfscope}%
\begin{pgfscope}%
\pgfpathrectangle{\pgfqpoint{0.438556in}{0.383578in}}{\pgfqpoint{4.650000in}{2.310000in}}%
\pgfusepath{clip}%
\pgfsetbuttcap%
\pgfsetroundjoin%
\pgfsetlinewidth{0.803000pt}%
\definecolor{currentstroke}{rgb}{0.686275,0.352941,0.313725}%
\pgfsetstrokecolor{currentstroke}%
\pgfsetdash{}{0pt}%
\pgfpathmoveto{\pgfqpoint{4.009921in}{2.339324in}}%
\pgfpathcurveto{\pgfqpoint{4.015446in}{2.339324in}}{\pgfqpoint{4.020745in}{2.341520in}}{\pgfqpoint{4.024652in}{2.345426in}}%
\pgfpathcurveto{\pgfqpoint{4.028559in}{2.349333in}}{\pgfqpoint{4.030754in}{2.354633in}}{\pgfqpoint{4.030754in}{2.360158in}}%
\pgfpathcurveto{\pgfqpoint{4.030754in}{2.365683in}}{\pgfqpoint{4.028559in}{2.370982in}}{\pgfqpoint{4.024652in}{2.374889in}}%
\pgfpathcurveto{\pgfqpoint{4.020745in}{2.378796in}}{\pgfqpoint{4.015446in}{2.380991in}}{\pgfqpoint{4.009921in}{2.380991in}}%
\pgfpathcurveto{\pgfqpoint{4.004396in}{2.380991in}}{\pgfqpoint{3.999096in}{2.378796in}}{\pgfqpoint{3.995189in}{2.374889in}}%
\pgfpathcurveto{\pgfqpoint{3.991282in}{2.370982in}}{\pgfqpoint{3.989087in}{2.365683in}}{\pgfqpoint{3.989087in}{2.360158in}}%
\pgfpathcurveto{\pgfqpoint{3.989087in}{2.354633in}}{\pgfqpoint{3.991282in}{2.349333in}}{\pgfqpoint{3.995189in}{2.345426in}}%
\pgfpathcurveto{\pgfqpoint{3.999096in}{2.341520in}}{\pgfqpoint{4.004396in}{2.339324in}}{\pgfqpoint{4.009921in}{2.339324in}}%
\pgfpathclose%
\pgfusepath{stroke}%
\end{pgfscope}%
\begin{pgfscope}%
\pgfpathrectangle{\pgfqpoint{0.438556in}{0.383578in}}{\pgfqpoint{4.650000in}{2.310000in}}%
\pgfusepath{clip}%
\pgfsetbuttcap%
\pgfsetroundjoin%
\pgfsetlinewidth{0.803000pt}%
\definecolor{currentstroke}{rgb}{0.686275,0.352941,0.313725}%
\pgfsetstrokecolor{currentstroke}%
\pgfsetdash{}{0pt}%
\pgfpathmoveto{\pgfqpoint{4.029987in}{2.383009in}}%
\pgfpathcurveto{\pgfqpoint{4.035512in}{2.383009in}}{\pgfqpoint{4.040811in}{2.385204in}}{\pgfqpoint{4.044718in}{2.389111in}}%
\pgfpathcurveto{\pgfqpoint{4.048625in}{2.393017in}}{\pgfqpoint{4.050820in}{2.398317in}}{\pgfqpoint{4.050820in}{2.403842in}}%
\pgfpathcurveto{\pgfqpoint{4.050820in}{2.409367in}}{\pgfqpoint{4.048625in}{2.414667in}}{\pgfqpoint{4.044718in}{2.418573in}}%
\pgfpathcurveto{\pgfqpoint{4.040811in}{2.422480in}}{\pgfqpoint{4.035512in}{2.424675in}}{\pgfqpoint{4.029987in}{2.424675in}}%
\pgfpathcurveto{\pgfqpoint{4.024462in}{2.424675in}}{\pgfqpoint{4.019162in}{2.422480in}}{\pgfqpoint{4.015255in}{2.418573in}}%
\pgfpathcurveto{\pgfqpoint{4.011348in}{2.414667in}}{\pgfqpoint{4.009153in}{2.409367in}}{\pgfqpoint{4.009153in}{2.403842in}}%
\pgfpathcurveto{\pgfqpoint{4.009153in}{2.398317in}}{\pgfqpoint{4.011348in}{2.393017in}}{\pgfqpoint{4.015255in}{2.389111in}}%
\pgfpathcurveto{\pgfqpoint{4.019162in}{2.385204in}}{\pgfqpoint{4.024462in}{2.383009in}}{\pgfqpoint{4.029987in}{2.383009in}}%
\pgfpathclose%
\pgfusepath{stroke}%
\end{pgfscope}%
\begin{pgfscope}%
\pgfpathrectangle{\pgfqpoint{0.438556in}{0.383578in}}{\pgfqpoint{4.650000in}{2.310000in}}%
\pgfusepath{clip}%
\pgfsetbuttcap%
\pgfsetroundjoin%
\pgfsetlinewidth{0.803000pt}%
\definecolor{currentstroke}{rgb}{0.686275,0.352941,0.313725}%
\pgfsetstrokecolor{currentstroke}%
\pgfsetdash{}{0pt}%
\pgfpathmoveto{\pgfqpoint{3.958876in}{2.404851in}}%
\pgfpathcurveto{\pgfqpoint{3.964401in}{2.404851in}}{\pgfqpoint{3.969701in}{2.407046in}}{\pgfqpoint{3.973607in}{2.410953in}}%
\pgfpathcurveto{\pgfqpoint{3.977514in}{2.414860in}}{\pgfqpoint{3.979709in}{2.420159in}}{\pgfqpoint{3.979709in}{2.425684in}}%
\pgfpathcurveto{\pgfqpoint{3.979709in}{2.431209in}}{\pgfqpoint{3.977514in}{2.436509in}}{\pgfqpoint{3.973607in}{2.440416in}}%
\pgfpathcurveto{\pgfqpoint{3.969701in}{2.444322in}}{\pgfqpoint{3.964401in}{2.446517in}}{\pgfqpoint{3.958876in}{2.446517in}}%
\pgfpathcurveto{\pgfqpoint{3.953351in}{2.446517in}}{\pgfqpoint{3.948051in}{2.444322in}}{\pgfqpoint{3.944145in}{2.440416in}}%
\pgfpathcurveto{\pgfqpoint{3.940238in}{2.436509in}}{\pgfqpoint{3.938043in}{2.431209in}}{\pgfqpoint{3.938043in}{2.425684in}}%
\pgfpathcurveto{\pgfqpoint{3.938043in}{2.420159in}}{\pgfqpoint{3.940238in}{2.414860in}}{\pgfqpoint{3.944145in}{2.410953in}}%
\pgfpathcurveto{\pgfqpoint{3.948051in}{2.407046in}}{\pgfqpoint{3.953351in}{2.404851in}}{\pgfqpoint{3.958876in}{2.404851in}}%
\pgfpathclose%
\pgfusepath{stroke}%
\end{pgfscope}%
\begin{pgfscope}%
\pgfpathrectangle{\pgfqpoint{0.438556in}{0.383578in}}{\pgfqpoint{4.650000in}{2.310000in}}%
\pgfusepath{clip}%
\pgfsetbuttcap%
\pgfsetroundjoin%
\pgfsetlinewidth{0.803000pt}%
\definecolor{currentstroke}{rgb}{0.686275,0.352941,0.313725}%
\pgfsetstrokecolor{currentstroke}%
\pgfsetdash{}{0pt}%
\pgfpathmoveto{\pgfqpoint{3.976670in}{2.492219in}}%
\pgfpathcurveto{\pgfqpoint{3.982195in}{2.492219in}}{\pgfqpoint{3.987494in}{2.494414in}}{\pgfqpoint{3.991401in}{2.498321in}}%
\pgfpathcurveto{\pgfqpoint{3.995308in}{2.502228in}}{\pgfqpoint{3.997503in}{2.507528in}}{\pgfqpoint{3.997503in}{2.513053in}}%
\pgfpathcurveto{\pgfqpoint{3.997503in}{2.518578in}}{\pgfqpoint{3.995308in}{2.523877in}}{\pgfqpoint{3.991401in}{2.527784in}}%
\pgfpathcurveto{\pgfqpoint{3.987494in}{2.531691in}}{\pgfqpoint{3.982195in}{2.533886in}}{\pgfqpoint{3.976670in}{2.533886in}}%
\pgfpathcurveto{\pgfqpoint{3.971145in}{2.533886in}}{\pgfqpoint{3.965845in}{2.531691in}}{\pgfqpoint{3.961938in}{2.527784in}}%
\pgfpathcurveto{\pgfqpoint{3.958031in}{2.523877in}}{\pgfqpoint{3.955836in}{2.518578in}}{\pgfqpoint{3.955836in}{2.513053in}}%
\pgfpathcurveto{\pgfqpoint{3.955836in}{2.507528in}}{\pgfqpoint{3.958031in}{2.502228in}}{\pgfqpoint{3.961938in}{2.498321in}}%
\pgfpathcurveto{\pgfqpoint{3.965845in}{2.494414in}}{\pgfqpoint{3.971145in}{2.492219in}}{\pgfqpoint{3.976670in}{2.492219in}}%
\pgfpathclose%
\pgfusepath{stroke}%
\end{pgfscope}%
\begin{pgfscope}%
\pgfpathrectangle{\pgfqpoint{0.438556in}{0.383578in}}{\pgfqpoint{4.650000in}{2.310000in}}%
\pgfusepath{clip}%
\pgfsetbuttcap%
\pgfsetroundjoin%
\pgfsetlinewidth{0.803000pt}%
\definecolor{currentstroke}{rgb}{0.686275,0.352941,0.313725}%
\pgfsetstrokecolor{currentstroke}%
\pgfsetdash{}{0pt}%
\pgfpathmoveto{\pgfqpoint{3.950511in}{2.514061in}}%
\pgfpathcurveto{\pgfqpoint{3.956036in}{2.514061in}}{\pgfqpoint{3.961335in}{2.516257in}}{\pgfqpoint{3.965242in}{2.520163in}}%
\pgfpathcurveto{\pgfqpoint{3.969149in}{2.524070in}}{\pgfqpoint{3.971344in}{2.529370in}}{\pgfqpoint{3.971344in}{2.534895in}}%
\pgfpathcurveto{\pgfqpoint{3.971344in}{2.540420in}}{\pgfqpoint{3.969149in}{2.545719in}}{\pgfqpoint{3.965242in}{2.549626in}}%
\pgfpathcurveto{\pgfqpoint{3.961335in}{2.553533in}}{\pgfqpoint{3.956036in}{2.555728in}}{\pgfqpoint{3.950511in}{2.555728in}}%
\pgfpathcurveto{\pgfqpoint{3.944986in}{2.555728in}}{\pgfqpoint{3.939686in}{2.553533in}}{\pgfqpoint{3.935779in}{2.549626in}}%
\pgfpathcurveto{\pgfqpoint{3.931872in}{2.545719in}}{\pgfqpoint{3.929677in}{2.540420in}}{\pgfqpoint{3.929677in}{2.534895in}}%
\pgfpathcurveto{\pgfqpoint{3.929677in}{2.529370in}}{\pgfqpoint{3.931872in}{2.524070in}}{\pgfqpoint{3.935779in}{2.520163in}}%
\pgfpathcurveto{\pgfqpoint{3.939686in}{2.516257in}}{\pgfqpoint{3.944986in}{2.514061in}}{\pgfqpoint{3.950511in}{2.514061in}}%
\pgfpathclose%
\pgfusepath{stroke}%
\end{pgfscope}%
\begin{pgfscope}%
\pgfpathrectangle{\pgfqpoint{0.438556in}{0.383578in}}{\pgfqpoint{4.650000in}{2.310000in}}%
\pgfusepath{clip}%
\pgfsetbuttcap%
\pgfsetroundjoin%
\pgfsetlinewidth{0.803000pt}%
\definecolor{currentstroke}{rgb}{0.000000,0.356863,0.509804}%
\pgfsetstrokecolor{currentstroke}%
\pgfsetdash{}{0pt}%
\pgfpathmoveto{\pgfqpoint{4.552234in}{0.620740in}}%
\pgfpathcurveto{\pgfqpoint{4.555918in}{0.620740in}}{\pgfqpoint{4.559451in}{0.622204in}}{\pgfqpoint{4.562055in}{0.624808in}}%
\pgfpathcurveto{\pgfqpoint{4.564660in}{0.627413in}}{\pgfqpoint{4.566123in}{0.630946in}}{\pgfqpoint{4.566123in}{0.634629in}}%
\pgfpathcurveto{\pgfqpoint{4.566123in}{0.638313in}}{\pgfqpoint{4.564660in}{0.641846in}}{\pgfqpoint{4.562055in}{0.644450in}}%
\pgfpathcurveto{\pgfqpoint{4.559451in}{0.647055in}}{\pgfqpoint{4.555918in}{0.648518in}}{\pgfqpoint{4.552234in}{0.648518in}}%
\pgfpathcurveto{\pgfqpoint{4.548551in}{0.648518in}}{\pgfqpoint{4.545018in}{0.647055in}}{\pgfqpoint{4.542413in}{0.644450in}}%
\pgfpathcurveto{\pgfqpoint{4.539809in}{0.641846in}}{\pgfqpoint{4.538345in}{0.638313in}}{\pgfqpoint{4.538345in}{0.634629in}}%
\pgfpathcurveto{\pgfqpoint{4.538345in}{0.630946in}}{\pgfqpoint{4.539809in}{0.627413in}}{\pgfqpoint{4.542413in}{0.624808in}}%
\pgfpathcurveto{\pgfqpoint{4.545018in}{0.622204in}}{\pgfqpoint{4.548551in}{0.620740in}}{\pgfqpoint{4.552234in}{0.620740in}}%
\pgfpathclose%
\pgfusepath{stroke}%
\end{pgfscope}%
\begin{pgfscope}%
\pgfpathrectangle{\pgfqpoint{0.438556in}{0.383578in}}{\pgfqpoint{4.650000in}{2.310000in}}%
\pgfusepath{clip}%
\pgfsetbuttcap%
\pgfsetroundjoin%
\pgfsetlinewidth{0.803000pt}%
\definecolor{currentstroke}{rgb}{0.000000,0.356863,0.509804}%
\pgfsetstrokecolor{currentstroke}%
\pgfsetdash{}{0pt}%
\pgfpathmoveto{\pgfqpoint{4.526249in}{0.642583in}}%
\pgfpathcurveto{\pgfqpoint{4.529933in}{0.642583in}}{\pgfqpoint{4.533466in}{0.644046in}}{\pgfqpoint{4.536070in}{0.646651in}}%
\pgfpathcurveto{\pgfqpoint{4.538675in}{0.649255in}}{\pgfqpoint{4.540138in}{0.652788in}}{\pgfqpoint{4.540138in}{0.656471in}}%
\pgfpathcurveto{\pgfqpoint{4.540138in}{0.660155in}}{\pgfqpoint{4.538675in}{0.663688in}}{\pgfqpoint{4.536070in}{0.666292in}}%
\pgfpathcurveto{\pgfqpoint{4.533466in}{0.668897in}}{\pgfqpoint{4.529933in}{0.670360in}}{\pgfqpoint{4.526249in}{0.670360in}}%
\pgfpathcurveto{\pgfqpoint{4.522566in}{0.670360in}}{\pgfqpoint{4.519033in}{0.668897in}}{\pgfqpoint{4.516428in}{0.666292in}}%
\pgfpathcurveto{\pgfqpoint{4.513824in}{0.663688in}}{\pgfqpoint{4.512360in}{0.660155in}}{\pgfqpoint{4.512360in}{0.656471in}}%
\pgfpathcurveto{\pgfqpoint{4.512360in}{0.652788in}}{\pgfqpoint{4.513824in}{0.649255in}}{\pgfqpoint{4.516428in}{0.646651in}}%
\pgfpathcurveto{\pgfqpoint{4.519033in}{0.644046in}}{\pgfqpoint{4.522566in}{0.642583in}}{\pgfqpoint{4.526249in}{0.642583in}}%
\pgfpathclose%
\pgfusepath{stroke}%
\end{pgfscope}%
\begin{pgfscope}%
\pgfpathrectangle{\pgfqpoint{0.438556in}{0.383578in}}{\pgfqpoint{4.650000in}{2.310000in}}%
\pgfusepath{clip}%
\pgfsetbuttcap%
\pgfsetroundjoin%
\pgfsetlinewidth{0.803000pt}%
\definecolor{currentstroke}{rgb}{0.000000,0.356863,0.509804}%
\pgfsetstrokecolor{currentstroke}%
\pgfsetdash{}{0pt}%
\pgfpathmoveto{\pgfqpoint{4.532581in}{0.795478in}}%
\pgfpathcurveto{\pgfqpoint{4.536264in}{0.795478in}}{\pgfqpoint{4.539797in}{0.796941in}}{\pgfqpoint{4.542402in}{0.799545in}}%
\pgfpathcurveto{\pgfqpoint{4.545006in}{0.802150in}}{\pgfqpoint{4.546470in}{0.805683in}}{\pgfqpoint{4.546470in}{0.809366in}}%
\pgfpathcurveto{\pgfqpoint{4.546470in}{0.813050in}}{\pgfqpoint{4.545006in}{0.816583in}}{\pgfqpoint{4.542402in}{0.819187in}}%
\pgfpathcurveto{\pgfqpoint{4.539797in}{0.821792in}}{\pgfqpoint{4.536264in}{0.823255in}}{\pgfqpoint{4.532581in}{0.823255in}}%
\pgfpathcurveto{\pgfqpoint{4.528897in}{0.823255in}}{\pgfqpoint{4.525364in}{0.821792in}}{\pgfqpoint{4.522760in}{0.819187in}}%
\pgfpathcurveto{\pgfqpoint{4.520155in}{0.816583in}}{\pgfqpoint{4.518692in}{0.813050in}}{\pgfqpoint{4.518692in}{0.809366in}}%
\pgfpathcurveto{\pgfqpoint{4.518692in}{0.805683in}}{\pgfqpoint{4.520155in}{0.802150in}}{\pgfqpoint{4.522760in}{0.799545in}}%
\pgfpathcurveto{\pgfqpoint{4.525364in}{0.796941in}}{\pgfqpoint{4.528897in}{0.795478in}}{\pgfqpoint{4.532581in}{0.795478in}}%
\pgfpathclose%
\pgfusepath{stroke}%
\end{pgfscope}%
\begin{pgfscope}%
\pgfpathrectangle{\pgfqpoint{0.438556in}{0.383578in}}{\pgfqpoint{4.650000in}{2.310000in}}%
\pgfusepath{clip}%
\pgfsetbuttcap%
\pgfsetroundjoin%
\pgfsetlinewidth{0.803000pt}%
\definecolor{currentstroke}{rgb}{0.000000,0.356863,0.509804}%
\pgfsetstrokecolor{currentstroke}%
\pgfsetdash{}{0pt}%
\pgfpathmoveto{\pgfqpoint{4.553261in}{0.861004in}}%
\pgfpathcurveto{\pgfqpoint{4.556944in}{0.861004in}}{\pgfqpoint{4.560477in}{0.862467in}}{\pgfqpoint{4.563081in}{0.865072in}}%
\pgfpathcurveto{\pgfqpoint{4.565686in}{0.867676in}}{\pgfqpoint{4.567149in}{0.871209in}}{\pgfqpoint{4.567149in}{0.874893in}}%
\pgfpathcurveto{\pgfqpoint{4.567149in}{0.878576in}}{\pgfqpoint{4.565686in}{0.882109in}}{\pgfqpoint{4.563081in}{0.884714in}}%
\pgfpathcurveto{\pgfqpoint{4.560477in}{0.887318in}}{\pgfqpoint{4.556944in}{0.888782in}}{\pgfqpoint{4.553261in}{0.888782in}}%
\pgfpathcurveto{\pgfqpoint{4.549577in}{0.888782in}}{\pgfqpoint{4.546044in}{0.887318in}}{\pgfqpoint{4.543440in}{0.884714in}}%
\pgfpathcurveto{\pgfqpoint{4.540835in}{0.882109in}}{\pgfqpoint{4.539372in}{0.878576in}}{\pgfqpoint{4.539372in}{0.874893in}}%
\pgfpathcurveto{\pgfqpoint{4.539372in}{0.871209in}}{\pgfqpoint{4.540835in}{0.867676in}}{\pgfqpoint{4.543440in}{0.865072in}}%
\pgfpathcurveto{\pgfqpoint{4.546044in}{0.862467in}}{\pgfqpoint{4.549577in}{0.861004in}}{\pgfqpoint{4.553261in}{0.861004in}}%
\pgfpathclose%
\pgfusepath{stroke}%
\end{pgfscope}%
\begin{pgfscope}%
\pgfpathrectangle{\pgfqpoint{0.438556in}{0.383578in}}{\pgfqpoint{4.650000in}{2.310000in}}%
\pgfusepath{clip}%
\pgfsetbuttcap%
\pgfsetroundjoin%
\pgfsetlinewidth{0.803000pt}%
\definecolor{currentstroke}{rgb}{0.000000,0.356863,0.509804}%
\pgfsetstrokecolor{currentstroke}%
\pgfsetdash{}{0pt}%
\pgfpathmoveto{\pgfqpoint{4.603737in}{0.882846in}}%
\pgfpathcurveto{\pgfqpoint{4.607420in}{0.882846in}}{\pgfqpoint{4.610953in}{0.884309in}}{\pgfqpoint{4.613558in}{0.886914in}}%
\pgfpathcurveto{\pgfqpoint{4.616163in}{0.889519in}}{\pgfqpoint{4.617626in}{0.893052in}}{\pgfqpoint{4.617626in}{0.896735in}}%
\pgfpathcurveto{\pgfqpoint{4.617626in}{0.900418in}}{\pgfqpoint{4.616163in}{0.903951in}}{\pgfqpoint{4.613558in}{0.906556in}}%
\pgfpathcurveto{\pgfqpoint{4.610953in}{0.909160in}}{\pgfqpoint{4.607420in}{0.910624in}}{\pgfqpoint{4.603737in}{0.910624in}}%
\pgfpathcurveto{\pgfqpoint{4.600054in}{0.910624in}}{\pgfqpoint{4.596521in}{0.909160in}}{\pgfqpoint{4.593916in}{0.906556in}}%
\pgfpathcurveto{\pgfqpoint{4.591312in}{0.903951in}}{\pgfqpoint{4.589848in}{0.900418in}}{\pgfqpoint{4.589848in}{0.896735in}}%
\pgfpathcurveto{\pgfqpoint{4.589848in}{0.893052in}}{\pgfqpoint{4.591312in}{0.889519in}}{\pgfqpoint{4.593916in}{0.886914in}}%
\pgfpathcurveto{\pgfqpoint{4.596521in}{0.884309in}}{\pgfqpoint{4.600054in}{0.882846in}}{\pgfqpoint{4.603737in}{0.882846in}}%
\pgfpathclose%
\pgfusepath{stroke}%
\end{pgfscope}%
\begin{pgfscope}%
\pgfpathrectangle{\pgfqpoint{0.438556in}{0.383578in}}{\pgfqpoint{4.650000in}{2.310000in}}%
\pgfusepath{clip}%
\pgfsetbuttcap%
\pgfsetroundjoin%
\pgfsetlinewidth{0.803000pt}%
\definecolor{currentstroke}{rgb}{0.000000,0.356863,0.509804}%
\pgfsetstrokecolor{currentstroke}%
\pgfsetdash{}{0pt}%
\pgfpathmoveto{\pgfqpoint{4.582782in}{1.013899in}}%
\pgfpathcurveto{\pgfqpoint{4.586466in}{1.013899in}}{\pgfqpoint{4.589999in}{1.015362in}}{\pgfqpoint{4.592603in}{1.017967in}}%
\pgfpathcurveto{\pgfqpoint{4.595208in}{1.020571in}}{\pgfqpoint{4.596671in}{1.024104in}}{\pgfqpoint{4.596671in}{1.027788in}}%
\pgfpathcurveto{\pgfqpoint{4.596671in}{1.031471in}}{\pgfqpoint{4.595208in}{1.035004in}}{\pgfqpoint{4.592603in}{1.037609in}}%
\pgfpathcurveto{\pgfqpoint{4.589999in}{1.040213in}}{\pgfqpoint{4.586466in}{1.041677in}}{\pgfqpoint{4.582782in}{1.041677in}}%
\pgfpathcurveto{\pgfqpoint{4.579099in}{1.041677in}}{\pgfqpoint{4.575566in}{1.040213in}}{\pgfqpoint{4.572961in}{1.037609in}}%
\pgfpathcurveto{\pgfqpoint{4.570357in}{1.035004in}}{\pgfqpoint{4.568893in}{1.031471in}}{\pgfqpoint{4.568893in}{1.027788in}}%
\pgfpathcurveto{\pgfqpoint{4.568893in}{1.024104in}}{\pgfqpoint{4.570357in}{1.020571in}}{\pgfqpoint{4.572961in}{1.017967in}}%
\pgfpathcurveto{\pgfqpoint{4.575566in}{1.015362in}}{\pgfqpoint{4.579099in}{1.013899in}}{\pgfqpoint{4.582782in}{1.013899in}}%
\pgfpathclose%
\pgfusepath{stroke}%
\end{pgfscope}%
\begin{pgfscope}%
\pgfpathrectangle{\pgfqpoint{0.438556in}{0.383578in}}{\pgfqpoint{4.650000in}{2.310000in}}%
\pgfusepath{clip}%
\pgfsetbuttcap%
\pgfsetroundjoin%
\pgfsetlinewidth{0.803000pt}%
\definecolor{currentstroke}{rgb}{0.000000,0.356863,0.509804}%
\pgfsetstrokecolor{currentstroke}%
\pgfsetdash{}{0pt}%
\pgfpathmoveto{\pgfqpoint{4.547855in}{1.035741in}}%
\pgfpathcurveto{\pgfqpoint{4.551538in}{1.035741in}}{\pgfqpoint{4.555071in}{1.037204in}}{\pgfqpoint{4.557676in}{1.039809in}}%
\pgfpathcurveto{\pgfqpoint{4.560280in}{1.042413in}}{\pgfqpoint{4.561743in}{1.045946in}}{\pgfqpoint{4.561743in}{1.049630in}}%
\pgfpathcurveto{\pgfqpoint{4.561743in}{1.053313in}}{\pgfqpoint{4.560280in}{1.056846in}}{\pgfqpoint{4.557676in}{1.059451in}}%
\pgfpathcurveto{\pgfqpoint{4.555071in}{1.062055in}}{\pgfqpoint{4.551538in}{1.063519in}}{\pgfqpoint{4.547855in}{1.063519in}}%
\pgfpathcurveto{\pgfqpoint{4.544171in}{1.063519in}}{\pgfqpoint{4.540638in}{1.062055in}}{\pgfqpoint{4.538034in}{1.059451in}}%
\pgfpathcurveto{\pgfqpoint{4.535429in}{1.056846in}}{\pgfqpoint{4.533966in}{1.053313in}}{\pgfqpoint{4.533966in}{1.049630in}}%
\pgfpathcurveto{\pgfqpoint{4.533966in}{1.045946in}}{\pgfqpoint{4.535429in}{1.042413in}}{\pgfqpoint{4.538034in}{1.039809in}}%
\pgfpathcurveto{\pgfqpoint{4.540638in}{1.037204in}}{\pgfqpoint{4.544171in}{1.035741in}}{\pgfqpoint{4.547855in}{1.035741in}}%
\pgfpathclose%
\pgfusepath{stroke}%
\end{pgfscope}%
\begin{pgfscope}%
\pgfpathrectangle{\pgfqpoint{0.438556in}{0.383578in}}{\pgfqpoint{4.650000in}{2.310000in}}%
\pgfusepath{clip}%
\pgfsetbuttcap%
\pgfsetroundjoin%
\pgfsetlinewidth{0.803000pt}%
\definecolor{currentstroke}{rgb}{0.000000,0.356863,0.509804}%
\pgfsetstrokecolor{currentstroke}%
\pgfsetdash{}{0pt}%
\pgfpathmoveto{\pgfqpoint{4.558694in}{1.101267in}}%
\pgfpathcurveto{\pgfqpoint{4.562377in}{1.101267in}}{\pgfqpoint{4.565910in}{1.102731in}}{\pgfqpoint{4.568515in}{1.105335in}}%
\pgfpathcurveto{\pgfqpoint{4.571119in}{1.107940in}}{\pgfqpoint{4.572583in}{1.111473in}}{\pgfqpoint{4.572583in}{1.115156in}}%
\pgfpathcurveto{\pgfqpoint{4.572583in}{1.118840in}}{\pgfqpoint{4.571119in}{1.122373in}}{\pgfqpoint{4.568515in}{1.124977in}}%
\pgfpathcurveto{\pgfqpoint{4.565910in}{1.127582in}}{\pgfqpoint{4.562377in}{1.129045in}}{\pgfqpoint{4.558694in}{1.129045in}}%
\pgfpathcurveto{\pgfqpoint{4.555011in}{1.129045in}}{\pgfqpoint{4.551478in}{1.127582in}}{\pgfqpoint{4.548873in}{1.124977in}}%
\pgfpathcurveto{\pgfqpoint{4.546268in}{1.122373in}}{\pgfqpoint{4.544805in}{1.118840in}}{\pgfqpoint{4.544805in}{1.115156in}}%
\pgfpathcurveto{\pgfqpoint{4.544805in}{1.111473in}}{\pgfqpoint{4.546268in}{1.107940in}}{\pgfqpoint{4.548873in}{1.105335in}}%
\pgfpathcurveto{\pgfqpoint{4.551478in}{1.102731in}}{\pgfqpoint{4.555011in}{1.101267in}}{\pgfqpoint{4.558694in}{1.101267in}}%
\pgfpathclose%
\pgfusepath{stroke}%
\end{pgfscope}%
\begin{pgfscope}%
\pgfpathrectangle{\pgfqpoint{0.438556in}{0.383578in}}{\pgfqpoint{4.650000in}{2.310000in}}%
\pgfusepath{clip}%
\pgfsetbuttcap%
\pgfsetroundjoin%
\pgfsetlinewidth{0.803000pt}%
\definecolor{currentstroke}{rgb}{0.000000,0.356863,0.509804}%
\pgfsetstrokecolor{currentstroke}%
\pgfsetdash{}{0pt}%
\pgfpathmoveto{\pgfqpoint{4.544959in}{1.210478in}}%
\pgfpathcurveto{\pgfqpoint{4.548643in}{1.210478in}}{\pgfqpoint{4.552176in}{1.211941in}}{\pgfqpoint{4.554780in}{1.214546in}}%
\pgfpathcurveto{\pgfqpoint{4.557385in}{1.217151in}}{\pgfqpoint{4.558848in}{1.220684in}}{\pgfqpoint{4.558848in}{1.224367in}}%
\pgfpathcurveto{\pgfqpoint{4.558848in}{1.228050in}}{\pgfqpoint{4.557385in}{1.231583in}}{\pgfqpoint{4.554780in}{1.234188in}}%
\pgfpathcurveto{\pgfqpoint{4.552176in}{1.236792in}}{\pgfqpoint{4.548643in}{1.238256in}}{\pgfqpoint{4.544959in}{1.238256in}}%
\pgfpathcurveto{\pgfqpoint{4.541276in}{1.238256in}}{\pgfqpoint{4.537743in}{1.236792in}}{\pgfqpoint{4.535138in}{1.234188in}}%
\pgfpathcurveto{\pgfqpoint{4.532534in}{1.231583in}}{\pgfqpoint{4.531070in}{1.228050in}}{\pgfqpoint{4.531070in}{1.224367in}}%
\pgfpathcurveto{\pgfqpoint{4.531070in}{1.220684in}}{\pgfqpoint{4.532534in}{1.217151in}}{\pgfqpoint{4.535138in}{1.214546in}}%
\pgfpathcurveto{\pgfqpoint{4.537743in}{1.211941in}}{\pgfqpoint{4.541276in}{1.210478in}}{\pgfqpoint{4.544959in}{1.210478in}}%
\pgfpathclose%
\pgfusepath{stroke}%
\end{pgfscope}%
\begin{pgfscope}%
\pgfpathrectangle{\pgfqpoint{0.438556in}{0.383578in}}{\pgfqpoint{4.650000in}{2.310000in}}%
\pgfusepath{clip}%
\pgfsetbuttcap%
\pgfsetroundjoin%
\pgfsetlinewidth{0.803000pt}%
\definecolor{currentstroke}{rgb}{0.000000,0.356863,0.509804}%
\pgfsetstrokecolor{currentstroke}%
\pgfsetdash{}{0pt}%
\pgfpathmoveto{\pgfqpoint{4.605817in}{1.232320in}}%
\pgfpathcurveto{\pgfqpoint{4.609500in}{1.232320in}}{\pgfqpoint{4.613033in}{1.233784in}}{\pgfqpoint{4.615638in}{1.236388in}}%
\pgfpathcurveto{\pgfqpoint{4.618242in}{1.238993in}}{\pgfqpoint{4.619706in}{1.242526in}}{\pgfqpoint{4.619706in}{1.246209in}}%
\pgfpathcurveto{\pgfqpoint{4.619706in}{1.249892in}}{\pgfqpoint{4.618242in}{1.253425in}}{\pgfqpoint{4.615638in}{1.256030in}}%
\pgfpathcurveto{\pgfqpoint{4.613033in}{1.258634in}}{\pgfqpoint{4.609500in}{1.260098in}}{\pgfqpoint{4.605817in}{1.260098in}}%
\pgfpathcurveto{\pgfqpoint{4.602134in}{1.260098in}}{\pgfqpoint{4.598601in}{1.258634in}}{\pgfqpoint{4.595996in}{1.256030in}}%
\pgfpathcurveto{\pgfqpoint{4.593392in}{1.253425in}}{\pgfqpoint{4.591928in}{1.249892in}}{\pgfqpoint{4.591928in}{1.246209in}}%
\pgfpathcurveto{\pgfqpoint{4.591928in}{1.242526in}}{\pgfqpoint{4.593392in}{1.238993in}}{\pgfqpoint{4.595996in}{1.236388in}}%
\pgfpathcurveto{\pgfqpoint{4.598601in}{1.233784in}}{\pgfqpoint{4.602134in}{1.232320in}}{\pgfqpoint{4.605817in}{1.232320in}}%
\pgfpathclose%
\pgfusepath{stroke}%
\end{pgfscope}%
\begin{pgfscope}%
\pgfpathrectangle{\pgfqpoint{0.438556in}{0.383578in}}{\pgfqpoint{4.650000in}{2.310000in}}%
\pgfusepath{clip}%
\pgfsetbuttcap%
\pgfsetroundjoin%
\pgfsetlinewidth{0.803000pt}%
\definecolor{currentstroke}{rgb}{0.000000,0.356863,0.509804}%
\pgfsetstrokecolor{currentstroke}%
\pgfsetdash{}{0pt}%
\pgfpathmoveto{\pgfqpoint{4.541991in}{1.254162in}}%
\pgfpathcurveto{\pgfqpoint{4.545674in}{1.254162in}}{\pgfqpoint{4.549207in}{1.255626in}}{\pgfqpoint{4.551811in}{1.258230in}}%
\pgfpathcurveto{\pgfqpoint{4.554416in}{1.260835in}}{\pgfqpoint{4.555879in}{1.264368in}}{\pgfqpoint{4.555879in}{1.268051in}}%
\pgfpathcurveto{\pgfqpoint{4.555879in}{1.271735in}}{\pgfqpoint{4.554416in}{1.275268in}}{\pgfqpoint{4.551811in}{1.277872in}}%
\pgfpathcurveto{\pgfqpoint{4.549207in}{1.280477in}}{\pgfqpoint{4.545674in}{1.281940in}}{\pgfqpoint{4.541991in}{1.281940in}}%
\pgfpathcurveto{\pgfqpoint{4.538307in}{1.281940in}}{\pgfqpoint{4.534774in}{1.280477in}}{\pgfqpoint{4.532170in}{1.277872in}}%
\pgfpathcurveto{\pgfqpoint{4.529565in}{1.275268in}}{\pgfqpoint{4.528102in}{1.271735in}}{\pgfqpoint{4.528102in}{1.268051in}}%
\pgfpathcurveto{\pgfqpoint{4.528102in}{1.264368in}}{\pgfqpoint{4.529565in}{1.260835in}}{\pgfqpoint{4.532170in}{1.258230in}}%
\pgfpathcurveto{\pgfqpoint{4.534774in}{1.255626in}}{\pgfqpoint{4.538307in}{1.254162in}}{\pgfqpoint{4.541991in}{1.254162in}}%
\pgfpathclose%
\pgfusepath{stroke}%
\end{pgfscope}%
\begin{pgfscope}%
\pgfpathrectangle{\pgfqpoint{0.438556in}{0.383578in}}{\pgfqpoint{4.650000in}{2.310000in}}%
\pgfusepath{clip}%
\pgfsetbuttcap%
\pgfsetroundjoin%
\pgfsetlinewidth{0.803000pt}%
\definecolor{currentstroke}{rgb}{0.000000,0.356863,0.509804}%
\pgfsetstrokecolor{currentstroke}%
\pgfsetdash{}{0pt}%
\pgfpathmoveto{\pgfqpoint{4.590103in}{1.276004in}}%
\pgfpathcurveto{\pgfqpoint{4.593787in}{1.276004in}}{\pgfqpoint{4.597320in}{1.277468in}}{\pgfqpoint{4.599924in}{1.280072in}}%
\pgfpathcurveto{\pgfqpoint{4.602529in}{1.282677in}}{\pgfqpoint{4.603992in}{1.286210in}}{\pgfqpoint{4.603992in}{1.289893in}}%
\pgfpathcurveto{\pgfqpoint{4.603992in}{1.293577in}}{\pgfqpoint{4.602529in}{1.297110in}}{\pgfqpoint{4.599924in}{1.299714in}}%
\pgfpathcurveto{\pgfqpoint{4.597320in}{1.302319in}}{\pgfqpoint{4.593787in}{1.303782in}}{\pgfqpoint{4.590103in}{1.303782in}}%
\pgfpathcurveto{\pgfqpoint{4.586420in}{1.303782in}}{\pgfqpoint{4.582887in}{1.302319in}}{\pgfqpoint{4.580282in}{1.299714in}}%
\pgfpathcurveto{\pgfqpoint{4.577678in}{1.297110in}}{\pgfqpoint{4.576214in}{1.293577in}}{\pgfqpoint{4.576214in}{1.289893in}}%
\pgfpathcurveto{\pgfqpoint{4.576214in}{1.286210in}}{\pgfqpoint{4.577678in}{1.282677in}}{\pgfqpoint{4.580282in}{1.280072in}}%
\pgfpathcurveto{\pgfqpoint{4.582887in}{1.277468in}}{\pgfqpoint{4.586420in}{1.276004in}}{\pgfqpoint{4.590103in}{1.276004in}}%
\pgfpathclose%
\pgfusepath{stroke}%
\end{pgfscope}%
\begin{pgfscope}%
\pgfpathrectangle{\pgfqpoint{0.438556in}{0.383578in}}{\pgfqpoint{4.650000in}{2.310000in}}%
\pgfusepath{clip}%
\pgfsetbuttcap%
\pgfsetroundjoin%
\pgfsetlinewidth{0.803000pt}%
\definecolor{currentstroke}{rgb}{0.000000,0.356863,0.509804}%
\pgfsetstrokecolor{currentstroke}%
\pgfsetdash{}{0pt}%
\pgfpathmoveto{\pgfqpoint{4.538802in}{1.297847in}}%
\pgfpathcurveto{\pgfqpoint{4.542485in}{1.297847in}}{\pgfqpoint{4.546018in}{1.299310in}}{\pgfqpoint{4.548623in}{1.301914in}}%
\pgfpathcurveto{\pgfqpoint{4.551227in}{1.304519in}}{\pgfqpoint{4.552691in}{1.308052in}}{\pgfqpoint{4.552691in}{1.311735in}}%
\pgfpathcurveto{\pgfqpoint{4.552691in}{1.315419in}}{\pgfqpoint{4.551227in}{1.318952in}}{\pgfqpoint{4.548623in}{1.321556in}}%
\pgfpathcurveto{\pgfqpoint{4.546018in}{1.324161in}}{\pgfqpoint{4.542485in}{1.325624in}}{\pgfqpoint{4.538802in}{1.325624in}}%
\pgfpathcurveto{\pgfqpoint{4.535119in}{1.325624in}}{\pgfqpoint{4.531586in}{1.324161in}}{\pgfqpoint{4.528981in}{1.321556in}}%
\pgfpathcurveto{\pgfqpoint{4.526377in}{1.318952in}}{\pgfqpoint{4.524913in}{1.315419in}}{\pgfqpoint{4.524913in}{1.311735in}}%
\pgfpathcurveto{\pgfqpoint{4.524913in}{1.308052in}}{\pgfqpoint{4.526377in}{1.304519in}}{\pgfqpoint{4.528981in}{1.301914in}}%
\pgfpathcurveto{\pgfqpoint{4.531586in}{1.299310in}}{\pgfqpoint{4.535119in}{1.297847in}}{\pgfqpoint{4.538802in}{1.297847in}}%
\pgfpathclose%
\pgfusepath{stroke}%
\end{pgfscope}%
\begin{pgfscope}%
\pgfpathrectangle{\pgfqpoint{0.438556in}{0.383578in}}{\pgfqpoint{4.650000in}{2.310000in}}%
\pgfusepath{clip}%
\pgfsetbuttcap%
\pgfsetroundjoin%
\pgfsetlinewidth{0.803000pt}%
\definecolor{currentstroke}{rgb}{0.000000,0.356863,0.509804}%
\pgfsetstrokecolor{currentstroke}%
\pgfsetdash{}{0pt}%
\pgfpathmoveto{\pgfqpoint{4.592916in}{1.341531in}}%
\pgfpathcurveto{\pgfqpoint{4.596599in}{1.341531in}}{\pgfqpoint{4.600132in}{1.342994in}}{\pgfqpoint{4.602737in}{1.345599in}}%
\pgfpathcurveto{\pgfqpoint{4.605342in}{1.348203in}}{\pgfqpoint{4.606805in}{1.351736in}}{\pgfqpoint{4.606805in}{1.355420in}}%
\pgfpathcurveto{\pgfqpoint{4.606805in}{1.359103in}}{\pgfqpoint{4.605342in}{1.362636in}}{\pgfqpoint{4.602737in}{1.365241in}}%
\pgfpathcurveto{\pgfqpoint{4.600132in}{1.367845in}}{\pgfqpoint{4.596599in}{1.369309in}}{\pgfqpoint{4.592916in}{1.369309in}}%
\pgfpathcurveto{\pgfqpoint{4.589233in}{1.369309in}}{\pgfqpoint{4.585700in}{1.367845in}}{\pgfqpoint{4.583095in}{1.365241in}}%
\pgfpathcurveto{\pgfqpoint{4.580491in}{1.362636in}}{\pgfqpoint{4.579027in}{1.359103in}}{\pgfqpoint{4.579027in}{1.355420in}}%
\pgfpathcurveto{\pgfqpoint{4.579027in}{1.351736in}}{\pgfqpoint{4.580491in}{1.348203in}}{\pgfqpoint{4.583095in}{1.345599in}}%
\pgfpathcurveto{\pgfqpoint{4.585700in}{1.342994in}}{\pgfqpoint{4.589233in}{1.341531in}}{\pgfqpoint{4.592916in}{1.341531in}}%
\pgfpathclose%
\pgfusepath{stroke}%
\end{pgfscope}%
\begin{pgfscope}%
\pgfpathrectangle{\pgfqpoint{0.438556in}{0.383578in}}{\pgfqpoint{4.650000in}{2.310000in}}%
\pgfusepath{clip}%
\pgfsetbuttcap%
\pgfsetroundjoin%
\pgfsetlinewidth{0.803000pt}%
\definecolor{currentstroke}{rgb}{0.000000,0.356863,0.509804}%
\pgfsetstrokecolor{currentstroke}%
\pgfsetdash{}{0pt}%
\pgfpathmoveto{\pgfqpoint{4.541212in}{1.450741in}}%
\pgfpathcurveto{\pgfqpoint{4.544895in}{1.450741in}}{\pgfqpoint{4.548428in}{1.452205in}}{\pgfqpoint{4.551033in}{1.454809in}}%
\pgfpathcurveto{\pgfqpoint{4.553637in}{1.457414in}}{\pgfqpoint{4.555101in}{1.460947in}}{\pgfqpoint{4.555101in}{1.464630in}}%
\pgfpathcurveto{\pgfqpoint{4.555101in}{1.468314in}}{\pgfqpoint{4.553637in}{1.471847in}}{\pgfqpoint{4.551033in}{1.474451in}}%
\pgfpathcurveto{\pgfqpoint{4.548428in}{1.477056in}}{\pgfqpoint{4.544895in}{1.478519in}}{\pgfqpoint{4.541212in}{1.478519in}}%
\pgfpathcurveto{\pgfqpoint{4.537528in}{1.478519in}}{\pgfqpoint{4.533995in}{1.477056in}}{\pgfqpoint{4.531391in}{1.474451in}}%
\pgfpathcurveto{\pgfqpoint{4.528786in}{1.471847in}}{\pgfqpoint{4.527323in}{1.468314in}}{\pgfqpoint{4.527323in}{1.464630in}}%
\pgfpathcurveto{\pgfqpoint{4.527323in}{1.460947in}}{\pgfqpoint{4.528786in}{1.457414in}}{\pgfqpoint{4.531391in}{1.454809in}}%
\pgfpathcurveto{\pgfqpoint{4.533995in}{1.452205in}}{\pgfqpoint{4.537528in}{1.450741in}}{\pgfqpoint{4.541212in}{1.450741in}}%
\pgfpathclose%
\pgfusepath{stroke}%
\end{pgfscope}%
\begin{pgfscope}%
\pgfpathrectangle{\pgfqpoint{0.438556in}{0.383578in}}{\pgfqpoint{4.650000in}{2.310000in}}%
\pgfusepath{clip}%
\pgfsetbuttcap%
\pgfsetroundjoin%
\pgfsetlinewidth{0.803000pt}%
\definecolor{currentstroke}{rgb}{0.000000,0.356863,0.509804}%
\pgfsetstrokecolor{currentstroke}%
\pgfsetdash{}{0pt}%
\pgfpathmoveto{\pgfqpoint{4.551025in}{1.472584in}}%
\pgfpathcurveto{\pgfqpoint{4.554708in}{1.472584in}}{\pgfqpoint{4.558241in}{1.474047in}}{\pgfqpoint{4.560846in}{1.476652in}}%
\pgfpathcurveto{\pgfqpoint{4.563450in}{1.479256in}}{\pgfqpoint{4.564914in}{1.482789in}}{\pgfqpoint{4.564914in}{1.486472in}}%
\pgfpathcurveto{\pgfqpoint{4.564914in}{1.490156in}}{\pgfqpoint{4.563450in}{1.493689in}}{\pgfqpoint{4.560846in}{1.496293in}}%
\pgfpathcurveto{\pgfqpoint{4.558241in}{1.498898in}}{\pgfqpoint{4.554708in}{1.500361in}}{\pgfqpoint{4.551025in}{1.500361in}}%
\pgfpathcurveto{\pgfqpoint{4.547341in}{1.500361in}}{\pgfqpoint{4.543808in}{1.498898in}}{\pgfqpoint{4.541204in}{1.496293in}}%
\pgfpathcurveto{\pgfqpoint{4.538599in}{1.493689in}}{\pgfqpoint{4.537136in}{1.490156in}}{\pgfqpoint{4.537136in}{1.486472in}}%
\pgfpathcurveto{\pgfqpoint{4.537136in}{1.482789in}}{\pgfqpoint{4.538599in}{1.479256in}}{\pgfqpoint{4.541204in}{1.476652in}}%
\pgfpathcurveto{\pgfqpoint{4.543808in}{1.474047in}}{\pgfqpoint{4.547341in}{1.472584in}}{\pgfqpoint{4.551025in}{1.472584in}}%
\pgfpathclose%
\pgfusepath{stroke}%
\end{pgfscope}%
\begin{pgfscope}%
\pgfpathrectangle{\pgfqpoint{0.438556in}{0.383578in}}{\pgfqpoint{4.650000in}{2.310000in}}%
\pgfusepath{clip}%
\pgfsetbuttcap%
\pgfsetroundjoin%
\pgfsetlinewidth{0.803000pt}%
\definecolor{currentstroke}{rgb}{0.000000,0.356863,0.509804}%
\pgfsetstrokecolor{currentstroke}%
\pgfsetdash{}{0pt}%
\pgfpathmoveto{\pgfqpoint{4.553132in}{1.647321in}}%
\pgfpathcurveto{\pgfqpoint{4.556816in}{1.647321in}}{\pgfqpoint{4.560349in}{1.648784in}}{\pgfqpoint{4.562953in}{1.651389in}}%
\pgfpathcurveto{\pgfqpoint{4.565558in}{1.653993in}}{\pgfqpoint{4.567021in}{1.657526in}}{\pgfqpoint{4.567021in}{1.661210in}}%
\pgfpathcurveto{\pgfqpoint{4.567021in}{1.664893in}}{\pgfqpoint{4.565558in}{1.668426in}}{\pgfqpoint{4.562953in}{1.671030in}}%
\pgfpathcurveto{\pgfqpoint{4.560349in}{1.673635in}}{\pgfqpoint{4.556816in}{1.675098in}}{\pgfqpoint{4.553132in}{1.675098in}}%
\pgfpathcurveto{\pgfqpoint{4.549449in}{1.675098in}}{\pgfqpoint{4.545916in}{1.673635in}}{\pgfqpoint{4.543311in}{1.671030in}}%
\pgfpathcurveto{\pgfqpoint{4.540707in}{1.668426in}}{\pgfqpoint{4.539243in}{1.664893in}}{\pgfqpoint{4.539243in}{1.661210in}}%
\pgfpathcurveto{\pgfqpoint{4.539243in}{1.657526in}}{\pgfqpoint{4.540707in}{1.653993in}}{\pgfqpoint{4.543311in}{1.651389in}}%
\pgfpathcurveto{\pgfqpoint{4.545916in}{1.648784in}}{\pgfqpoint{4.549449in}{1.647321in}}{\pgfqpoint{4.553132in}{1.647321in}}%
\pgfpathclose%
\pgfusepath{stroke}%
\end{pgfscope}%
\begin{pgfscope}%
\pgfpathrectangle{\pgfqpoint{0.438556in}{0.383578in}}{\pgfqpoint{4.650000in}{2.310000in}}%
\pgfusepath{clip}%
\pgfsetbuttcap%
\pgfsetroundjoin%
\pgfsetlinewidth{0.803000pt}%
\definecolor{currentstroke}{rgb}{0.000000,0.356863,0.509804}%
\pgfsetstrokecolor{currentstroke}%
\pgfsetdash{}{0pt}%
\pgfpathmoveto{\pgfqpoint{4.593961in}{1.669163in}}%
\pgfpathcurveto{\pgfqpoint{4.597644in}{1.669163in}}{\pgfqpoint{4.601177in}{1.670626in}}{\pgfqpoint{4.603782in}{1.673231in}}%
\pgfpathcurveto{\pgfqpoint{4.606386in}{1.675835in}}{\pgfqpoint{4.607850in}{1.679368in}}{\pgfqpoint{4.607850in}{1.683052in}}%
\pgfpathcurveto{\pgfqpoint{4.607850in}{1.686735in}}{\pgfqpoint{4.606386in}{1.690268in}}{\pgfqpoint{4.603782in}{1.692873in}}%
\pgfpathcurveto{\pgfqpoint{4.601177in}{1.695477in}}{\pgfqpoint{4.597644in}{1.696941in}}{\pgfqpoint{4.593961in}{1.696941in}}%
\pgfpathcurveto{\pgfqpoint{4.590277in}{1.696941in}}{\pgfqpoint{4.586744in}{1.695477in}}{\pgfqpoint{4.584140in}{1.692873in}}%
\pgfpathcurveto{\pgfqpoint{4.581535in}{1.690268in}}{\pgfqpoint{4.580072in}{1.686735in}}{\pgfqpoint{4.580072in}{1.683052in}}%
\pgfpathcurveto{\pgfqpoint{4.580072in}{1.679368in}}{\pgfqpoint{4.581535in}{1.675835in}}{\pgfqpoint{4.584140in}{1.673231in}}%
\pgfpathcurveto{\pgfqpoint{4.586744in}{1.670626in}}{\pgfqpoint{4.590277in}{1.669163in}}{\pgfqpoint{4.593961in}{1.669163in}}%
\pgfpathclose%
\pgfusepath{stroke}%
\end{pgfscope}%
\begin{pgfscope}%
\pgfpathrectangle{\pgfqpoint{0.438556in}{0.383578in}}{\pgfqpoint{4.650000in}{2.310000in}}%
\pgfusepath{clip}%
\pgfsetbuttcap%
\pgfsetroundjoin%
\pgfsetlinewidth{0.803000pt}%
\definecolor{currentstroke}{rgb}{0.000000,0.356863,0.509804}%
\pgfsetstrokecolor{currentstroke}%
\pgfsetdash{}{0pt}%
\pgfpathmoveto{\pgfqpoint{4.586264in}{1.691005in}}%
\pgfpathcurveto{\pgfqpoint{4.589947in}{1.691005in}}{\pgfqpoint{4.593480in}{1.692468in}}{\pgfqpoint{4.596085in}{1.695073in}}%
\pgfpathcurveto{\pgfqpoint{4.598690in}{1.697677in}}{\pgfqpoint{4.600153in}{1.701210in}}{\pgfqpoint{4.600153in}{1.704894in}}%
\pgfpathcurveto{\pgfqpoint{4.600153in}{1.708577in}}{\pgfqpoint{4.598690in}{1.712110in}}{\pgfqpoint{4.596085in}{1.714715in}}%
\pgfpathcurveto{\pgfqpoint{4.593480in}{1.717319in}}{\pgfqpoint{4.589947in}{1.718783in}}{\pgfqpoint{4.586264in}{1.718783in}}%
\pgfpathcurveto{\pgfqpoint{4.582581in}{1.718783in}}{\pgfqpoint{4.579048in}{1.717319in}}{\pgfqpoint{4.576443in}{1.714715in}}%
\pgfpathcurveto{\pgfqpoint{4.573839in}{1.712110in}}{\pgfqpoint{4.572375in}{1.708577in}}{\pgfqpoint{4.572375in}{1.704894in}}%
\pgfpathcurveto{\pgfqpoint{4.572375in}{1.701210in}}{\pgfqpoint{4.573839in}{1.697677in}}{\pgfqpoint{4.576443in}{1.695073in}}%
\pgfpathcurveto{\pgfqpoint{4.579048in}{1.692468in}}{\pgfqpoint{4.582581in}{1.691005in}}{\pgfqpoint{4.586264in}{1.691005in}}%
\pgfpathclose%
\pgfusepath{stroke}%
\end{pgfscope}%
\begin{pgfscope}%
\pgfpathrectangle{\pgfqpoint{0.438556in}{0.383578in}}{\pgfqpoint{4.650000in}{2.310000in}}%
\pgfusepath{clip}%
\pgfsetbuttcap%
\pgfsetroundjoin%
\pgfsetlinewidth{0.803000pt}%
\definecolor{currentstroke}{rgb}{0.000000,0.356863,0.509804}%
\pgfsetstrokecolor{currentstroke}%
\pgfsetdash{}{0pt}%
\pgfpathmoveto{\pgfqpoint{4.533598in}{1.712847in}}%
\pgfpathcurveto{\pgfqpoint{4.537281in}{1.712847in}}{\pgfqpoint{4.540814in}{1.714310in}}{\pgfqpoint{4.543419in}{1.716915in}}%
\pgfpathcurveto{\pgfqpoint{4.546023in}{1.719520in}}{\pgfqpoint{4.547487in}{1.723053in}}{\pgfqpoint{4.547487in}{1.726736in}}%
\pgfpathcurveto{\pgfqpoint{4.547487in}{1.730419in}}{\pgfqpoint{4.546023in}{1.733952in}}{\pgfqpoint{4.543419in}{1.736557in}}%
\pgfpathcurveto{\pgfqpoint{4.540814in}{1.739161in}}{\pgfqpoint{4.537281in}{1.740625in}}{\pgfqpoint{4.533598in}{1.740625in}}%
\pgfpathcurveto{\pgfqpoint{4.529914in}{1.740625in}}{\pgfqpoint{4.526381in}{1.739161in}}{\pgfqpoint{4.523777in}{1.736557in}}%
\pgfpathcurveto{\pgfqpoint{4.521172in}{1.733952in}}{\pgfqpoint{4.519709in}{1.730419in}}{\pgfqpoint{4.519709in}{1.726736in}}%
\pgfpathcurveto{\pgfqpoint{4.519709in}{1.723053in}}{\pgfqpoint{4.521172in}{1.719520in}}{\pgfqpoint{4.523777in}{1.716915in}}%
\pgfpathcurveto{\pgfqpoint{4.526381in}{1.714310in}}{\pgfqpoint{4.529914in}{1.712847in}}{\pgfqpoint{4.533598in}{1.712847in}}%
\pgfpathclose%
\pgfusepath{stroke}%
\end{pgfscope}%
\begin{pgfscope}%
\pgfpathrectangle{\pgfqpoint{0.438556in}{0.383578in}}{\pgfqpoint{4.650000in}{2.310000in}}%
\pgfusepath{clip}%
\pgfsetbuttcap%
\pgfsetroundjoin%
\pgfsetlinewidth{0.803000pt}%
\definecolor{currentstroke}{rgb}{0.000000,0.356863,0.509804}%
\pgfsetstrokecolor{currentstroke}%
\pgfsetdash{}{0pt}%
\pgfpathmoveto{\pgfqpoint{4.540845in}{1.778373in}}%
\pgfpathcurveto{\pgfqpoint{4.544529in}{1.778373in}}{\pgfqpoint{4.548062in}{1.779837in}}{\pgfqpoint{4.550666in}{1.782441in}}%
\pgfpathcurveto{\pgfqpoint{4.553271in}{1.785046in}}{\pgfqpoint{4.554734in}{1.788579in}}{\pgfqpoint{4.554734in}{1.792262in}}%
\pgfpathcurveto{\pgfqpoint{4.554734in}{1.795946in}}{\pgfqpoint{4.553271in}{1.799479in}}{\pgfqpoint{4.550666in}{1.802083in}}%
\pgfpathcurveto{\pgfqpoint{4.548062in}{1.804688in}}{\pgfqpoint{4.544529in}{1.806151in}}{\pgfqpoint{4.540845in}{1.806151in}}%
\pgfpathcurveto{\pgfqpoint{4.537162in}{1.806151in}}{\pgfqpoint{4.533629in}{1.804688in}}{\pgfqpoint{4.531024in}{1.802083in}}%
\pgfpathcurveto{\pgfqpoint{4.528420in}{1.799479in}}{\pgfqpoint{4.526956in}{1.795946in}}{\pgfqpoint{4.526956in}{1.792262in}}%
\pgfpathcurveto{\pgfqpoint{4.526956in}{1.788579in}}{\pgfqpoint{4.528420in}{1.785046in}}{\pgfqpoint{4.531024in}{1.782441in}}%
\pgfpathcurveto{\pgfqpoint{4.533629in}{1.779837in}}{\pgfqpoint{4.537162in}{1.778373in}}{\pgfqpoint{4.540845in}{1.778373in}}%
\pgfpathclose%
\pgfusepath{stroke}%
\end{pgfscope}%
\begin{pgfscope}%
\pgfpathrectangle{\pgfqpoint{0.438556in}{0.383578in}}{\pgfqpoint{4.650000in}{2.310000in}}%
\pgfusepath{clip}%
\pgfsetbuttcap%
\pgfsetroundjoin%
\pgfsetlinewidth{0.803000pt}%
\definecolor{currentstroke}{rgb}{0.000000,0.356863,0.509804}%
\pgfsetstrokecolor{currentstroke}%
\pgfsetdash{}{0pt}%
\pgfpathmoveto{\pgfqpoint{4.526194in}{1.800216in}}%
\pgfpathcurveto{\pgfqpoint{4.529878in}{1.800216in}}{\pgfqpoint{4.533411in}{1.801679in}}{\pgfqpoint{4.536015in}{1.804284in}}%
\pgfpathcurveto{\pgfqpoint{4.538620in}{1.806888in}}{\pgfqpoint{4.540083in}{1.810421in}}{\pgfqpoint{4.540083in}{1.814104in}}%
\pgfpathcurveto{\pgfqpoint{4.540083in}{1.817788in}}{\pgfqpoint{4.538620in}{1.821321in}}{\pgfqpoint{4.536015in}{1.823925in}}%
\pgfpathcurveto{\pgfqpoint{4.533411in}{1.826530in}}{\pgfqpoint{4.529878in}{1.827993in}}{\pgfqpoint{4.526194in}{1.827993in}}%
\pgfpathcurveto{\pgfqpoint{4.522511in}{1.827993in}}{\pgfqpoint{4.518978in}{1.826530in}}{\pgfqpoint{4.516373in}{1.823925in}}%
\pgfpathcurveto{\pgfqpoint{4.513769in}{1.821321in}}{\pgfqpoint{4.512305in}{1.817788in}}{\pgfqpoint{4.512305in}{1.814104in}}%
\pgfpathcurveto{\pgfqpoint{4.512305in}{1.810421in}}{\pgfqpoint{4.513769in}{1.806888in}}{\pgfqpoint{4.516373in}{1.804284in}}%
\pgfpathcurveto{\pgfqpoint{4.518978in}{1.801679in}}{\pgfqpoint{4.522511in}{1.800216in}}{\pgfqpoint{4.526194in}{1.800216in}}%
\pgfpathclose%
\pgfusepath{stroke}%
\end{pgfscope}%
\begin{pgfscope}%
\pgfpathrectangle{\pgfqpoint{0.438556in}{0.383578in}}{\pgfqpoint{4.650000in}{2.310000in}}%
\pgfusepath{clip}%
\pgfsetbuttcap%
\pgfsetroundjoin%
\pgfsetlinewidth{0.803000pt}%
\definecolor{currentstroke}{rgb}{0.000000,0.356863,0.509804}%
\pgfsetstrokecolor{currentstroke}%
\pgfsetdash{}{0pt}%
\pgfpathmoveto{\pgfqpoint{4.551419in}{1.822058in}}%
\pgfpathcurveto{\pgfqpoint{4.555102in}{1.822058in}}{\pgfqpoint{4.558635in}{1.823521in}}{\pgfqpoint{4.561240in}{1.826126in}}%
\pgfpathcurveto{\pgfqpoint{4.563844in}{1.828730in}}{\pgfqpoint{4.565308in}{1.832263in}}{\pgfqpoint{4.565308in}{1.835947in}}%
\pgfpathcurveto{\pgfqpoint{4.565308in}{1.839630in}}{\pgfqpoint{4.563844in}{1.843163in}}{\pgfqpoint{4.561240in}{1.845768in}}%
\pgfpathcurveto{\pgfqpoint{4.558635in}{1.848372in}}{\pgfqpoint{4.555102in}{1.849835in}}{\pgfqpoint{4.551419in}{1.849835in}}%
\pgfpathcurveto{\pgfqpoint{4.547735in}{1.849835in}}{\pgfqpoint{4.544202in}{1.848372in}}{\pgfqpoint{4.541598in}{1.845768in}}%
\pgfpathcurveto{\pgfqpoint{4.538993in}{1.843163in}}{\pgfqpoint{4.537530in}{1.839630in}}{\pgfqpoint{4.537530in}{1.835947in}}%
\pgfpathcurveto{\pgfqpoint{4.537530in}{1.832263in}}{\pgfqpoint{4.538993in}{1.828730in}}{\pgfqpoint{4.541598in}{1.826126in}}%
\pgfpathcurveto{\pgfqpoint{4.544202in}{1.823521in}}{\pgfqpoint{4.547735in}{1.822058in}}{\pgfqpoint{4.551419in}{1.822058in}}%
\pgfpathclose%
\pgfusepath{stroke}%
\end{pgfscope}%
\begin{pgfscope}%
\pgfpathrectangle{\pgfqpoint{0.438556in}{0.383578in}}{\pgfqpoint{4.650000in}{2.310000in}}%
\pgfusepath{clip}%
\pgfsetbuttcap%
\pgfsetroundjoin%
\pgfsetlinewidth{0.803000pt}%
\definecolor{currentstroke}{rgb}{0.000000,0.356863,0.509804}%
\pgfsetstrokecolor{currentstroke}%
\pgfsetdash{}{0pt}%
\pgfpathmoveto{\pgfqpoint{4.561415in}{1.843900in}}%
\pgfpathcurveto{\pgfqpoint{4.565099in}{1.843900in}}{\pgfqpoint{4.568632in}{1.845363in}}{\pgfqpoint{4.571236in}{1.847968in}}%
\pgfpathcurveto{\pgfqpoint{4.573841in}{1.850572in}}{\pgfqpoint{4.575304in}{1.854105in}}{\pgfqpoint{4.575304in}{1.857789in}}%
\pgfpathcurveto{\pgfqpoint{4.575304in}{1.861472in}}{\pgfqpoint{4.573841in}{1.865005in}}{\pgfqpoint{4.571236in}{1.867610in}}%
\pgfpathcurveto{\pgfqpoint{4.568632in}{1.870214in}}{\pgfqpoint{4.565099in}{1.871678in}}{\pgfqpoint{4.561415in}{1.871678in}}%
\pgfpathcurveto{\pgfqpoint{4.557732in}{1.871678in}}{\pgfqpoint{4.554199in}{1.870214in}}{\pgfqpoint{4.551594in}{1.867610in}}%
\pgfpathcurveto{\pgfqpoint{4.548990in}{1.865005in}}{\pgfqpoint{4.547526in}{1.861472in}}{\pgfqpoint{4.547526in}{1.857789in}}%
\pgfpathcurveto{\pgfqpoint{4.547526in}{1.854105in}}{\pgfqpoint{4.548990in}{1.850572in}}{\pgfqpoint{4.551594in}{1.847968in}}%
\pgfpathcurveto{\pgfqpoint{4.554199in}{1.845363in}}{\pgfqpoint{4.557732in}{1.843900in}}{\pgfqpoint{4.561415in}{1.843900in}}%
\pgfpathclose%
\pgfusepath{stroke}%
\end{pgfscope}%
\begin{pgfscope}%
\pgfpathrectangle{\pgfqpoint{0.438556in}{0.383578in}}{\pgfqpoint{4.650000in}{2.310000in}}%
\pgfusepath{clip}%
\pgfsetbuttcap%
\pgfsetroundjoin%
\pgfsetlinewidth{0.803000pt}%
\definecolor{currentstroke}{rgb}{0.000000,0.356863,0.509804}%
\pgfsetstrokecolor{currentstroke}%
\pgfsetdash{}{0pt}%
\pgfpathmoveto{\pgfqpoint{4.565722in}{1.865742in}}%
\pgfpathcurveto{\pgfqpoint{4.569405in}{1.865742in}}{\pgfqpoint{4.572938in}{1.867205in}}{\pgfqpoint{4.575543in}{1.869810in}}%
\pgfpathcurveto{\pgfqpoint{4.578147in}{1.872414in}}{\pgfqpoint{4.579610in}{1.875947in}}{\pgfqpoint{4.579610in}{1.879631in}}%
\pgfpathcurveto{\pgfqpoint{4.579610in}{1.883314in}}{\pgfqpoint{4.578147in}{1.886847in}}{\pgfqpoint{4.575543in}{1.889452in}}%
\pgfpathcurveto{\pgfqpoint{4.572938in}{1.892056in}}{\pgfqpoint{4.569405in}{1.893520in}}{\pgfqpoint{4.565722in}{1.893520in}}%
\pgfpathcurveto{\pgfqpoint{4.562038in}{1.893520in}}{\pgfqpoint{4.558505in}{1.892056in}}{\pgfqpoint{4.555901in}{1.889452in}}%
\pgfpathcurveto{\pgfqpoint{4.553296in}{1.886847in}}{\pgfqpoint{4.551833in}{1.883314in}}{\pgfqpoint{4.551833in}{1.879631in}}%
\pgfpathcurveto{\pgfqpoint{4.551833in}{1.875947in}}{\pgfqpoint{4.553296in}{1.872414in}}{\pgfqpoint{4.555901in}{1.869810in}}%
\pgfpathcurveto{\pgfqpoint{4.558505in}{1.867205in}}{\pgfqpoint{4.562038in}{1.865742in}}{\pgfqpoint{4.565722in}{1.865742in}}%
\pgfpathclose%
\pgfusepath{stroke}%
\end{pgfscope}%
\begin{pgfscope}%
\pgfpathrectangle{\pgfqpoint{0.438556in}{0.383578in}}{\pgfqpoint{4.650000in}{2.310000in}}%
\pgfusepath{clip}%
\pgfsetbuttcap%
\pgfsetroundjoin%
\pgfsetlinewidth{0.803000pt}%
\definecolor{currentstroke}{rgb}{0.000000,0.356863,0.509804}%
\pgfsetstrokecolor{currentstroke}%
\pgfsetdash{}{0pt}%
\pgfpathmoveto{\pgfqpoint{4.600512in}{1.887584in}}%
\pgfpathcurveto{\pgfqpoint{4.604195in}{1.887584in}}{\pgfqpoint{4.607728in}{1.889048in}}{\pgfqpoint{4.610333in}{1.891652in}}%
\pgfpathcurveto{\pgfqpoint{4.612937in}{1.894257in}}{\pgfqpoint{4.614401in}{1.897790in}}{\pgfqpoint{4.614401in}{1.901473in}}%
\pgfpathcurveto{\pgfqpoint{4.614401in}{1.905156in}}{\pgfqpoint{4.612937in}{1.908689in}}{\pgfqpoint{4.610333in}{1.911294in}}%
\pgfpathcurveto{\pgfqpoint{4.607728in}{1.913898in}}{\pgfqpoint{4.604195in}{1.915362in}}{\pgfqpoint{4.600512in}{1.915362in}}%
\pgfpathcurveto{\pgfqpoint{4.596828in}{1.915362in}}{\pgfqpoint{4.593295in}{1.913898in}}{\pgfqpoint{4.590691in}{1.911294in}}%
\pgfpathcurveto{\pgfqpoint{4.588086in}{1.908689in}}{\pgfqpoint{4.586623in}{1.905156in}}{\pgfqpoint{4.586623in}{1.901473in}}%
\pgfpathcurveto{\pgfqpoint{4.586623in}{1.897790in}}{\pgfqpoint{4.588086in}{1.894257in}}{\pgfqpoint{4.590691in}{1.891652in}}%
\pgfpathcurveto{\pgfqpoint{4.593295in}{1.889048in}}{\pgfqpoint{4.596828in}{1.887584in}}{\pgfqpoint{4.600512in}{1.887584in}}%
\pgfpathclose%
\pgfusepath{stroke}%
\end{pgfscope}%
\begin{pgfscope}%
\pgfpathrectangle{\pgfqpoint{0.438556in}{0.383578in}}{\pgfqpoint{4.650000in}{2.310000in}}%
\pgfusepath{clip}%
\pgfsetbuttcap%
\pgfsetroundjoin%
\pgfsetlinewidth{0.803000pt}%
\definecolor{currentstroke}{rgb}{0.000000,0.356863,0.509804}%
\pgfsetstrokecolor{currentstroke}%
\pgfsetdash{}{0pt}%
\pgfpathmoveto{\pgfqpoint{4.583992in}{1.953110in}}%
\pgfpathcurveto{\pgfqpoint{4.587675in}{1.953110in}}{\pgfqpoint{4.591208in}{1.954574in}}{\pgfqpoint{4.593813in}{1.957178in}}%
\pgfpathcurveto{\pgfqpoint{4.596417in}{1.959783in}}{\pgfqpoint{4.597881in}{1.963316in}}{\pgfqpoint{4.597881in}{1.966999in}}%
\pgfpathcurveto{\pgfqpoint{4.597881in}{1.970683in}}{\pgfqpoint{4.596417in}{1.974216in}}{\pgfqpoint{4.593813in}{1.976820in}}%
\pgfpathcurveto{\pgfqpoint{4.591208in}{1.979425in}}{\pgfqpoint{4.587675in}{1.980888in}}{\pgfqpoint{4.583992in}{1.980888in}}%
\pgfpathcurveto{\pgfqpoint{4.580308in}{1.980888in}}{\pgfqpoint{4.576775in}{1.979425in}}{\pgfqpoint{4.574171in}{1.976820in}}%
\pgfpathcurveto{\pgfqpoint{4.571566in}{1.974216in}}{\pgfqpoint{4.570103in}{1.970683in}}{\pgfqpoint{4.570103in}{1.966999in}}%
\pgfpathcurveto{\pgfqpoint{4.570103in}{1.963316in}}{\pgfqpoint{4.571566in}{1.959783in}}{\pgfqpoint{4.574171in}{1.957178in}}%
\pgfpathcurveto{\pgfqpoint{4.576775in}{1.954574in}}{\pgfqpoint{4.580308in}{1.953110in}}{\pgfqpoint{4.583992in}{1.953110in}}%
\pgfpathclose%
\pgfusepath{stroke}%
\end{pgfscope}%
\begin{pgfscope}%
\pgfpathrectangle{\pgfqpoint{0.438556in}{0.383578in}}{\pgfqpoint{4.650000in}{2.310000in}}%
\pgfusepath{clip}%
\pgfsetbuttcap%
\pgfsetroundjoin%
\pgfsetlinewidth{0.803000pt}%
\definecolor{currentstroke}{rgb}{0.000000,0.356863,0.509804}%
\pgfsetstrokecolor{currentstroke}%
\pgfsetdash{}{0pt}%
\pgfpathmoveto{\pgfqpoint{4.533561in}{1.996795in}}%
\pgfpathcurveto{\pgfqpoint{4.537244in}{1.996795in}}{\pgfqpoint{4.540777in}{1.998258in}}{\pgfqpoint{4.543382in}{2.000863in}}%
\pgfpathcurveto{\pgfqpoint{4.545986in}{2.003467in}}{\pgfqpoint{4.547450in}{2.007000in}}{\pgfqpoint{4.547450in}{2.010684in}}%
\pgfpathcurveto{\pgfqpoint{4.547450in}{2.014367in}}{\pgfqpoint{4.545986in}{2.017900in}}{\pgfqpoint{4.543382in}{2.020505in}}%
\pgfpathcurveto{\pgfqpoint{4.540777in}{2.023109in}}{\pgfqpoint{4.537244in}{2.024573in}}{\pgfqpoint{4.533561in}{2.024573in}}%
\pgfpathcurveto{\pgfqpoint{4.529878in}{2.024573in}}{\pgfqpoint{4.526345in}{2.023109in}}{\pgfqpoint{4.523740in}{2.020505in}}%
\pgfpathcurveto{\pgfqpoint{4.521136in}{2.017900in}}{\pgfqpoint{4.519672in}{2.014367in}}{\pgfqpoint{4.519672in}{2.010684in}}%
\pgfpathcurveto{\pgfqpoint{4.519672in}{2.007000in}}{\pgfqpoint{4.521136in}{2.003467in}}{\pgfqpoint{4.523740in}{2.000863in}}%
\pgfpathcurveto{\pgfqpoint{4.526345in}{1.998258in}}{\pgfqpoint{4.529878in}{1.996795in}}{\pgfqpoint{4.533561in}{1.996795in}}%
\pgfpathclose%
\pgfusepath{stroke}%
\end{pgfscope}%
\begin{pgfscope}%
\pgfpathrectangle{\pgfqpoint{0.438556in}{0.383578in}}{\pgfqpoint{4.650000in}{2.310000in}}%
\pgfusepath{clip}%
\pgfsetbuttcap%
\pgfsetroundjoin%
\pgfsetlinewidth{0.803000pt}%
\definecolor{currentstroke}{rgb}{0.000000,0.356863,0.509804}%
\pgfsetstrokecolor{currentstroke}%
\pgfsetdash{}{0pt}%
\pgfpathmoveto{\pgfqpoint{4.565337in}{2.040479in}}%
\pgfpathcurveto{\pgfqpoint{4.569020in}{2.040479in}}{\pgfqpoint{4.572553in}{2.041942in}}{\pgfqpoint{4.575158in}{2.044547in}}%
\pgfpathcurveto{\pgfqpoint{4.577762in}{2.047152in}}{\pgfqpoint{4.579226in}{2.050685in}}{\pgfqpoint{4.579226in}{2.054368in}}%
\pgfpathcurveto{\pgfqpoint{4.579226in}{2.058051in}}{\pgfqpoint{4.577762in}{2.061584in}}{\pgfqpoint{4.575158in}{2.064189in}}%
\pgfpathcurveto{\pgfqpoint{4.572553in}{2.066793in}}{\pgfqpoint{4.569020in}{2.068257in}}{\pgfqpoint{4.565337in}{2.068257in}}%
\pgfpathcurveto{\pgfqpoint{4.561653in}{2.068257in}}{\pgfqpoint{4.558120in}{2.066793in}}{\pgfqpoint{4.555516in}{2.064189in}}%
\pgfpathcurveto{\pgfqpoint{4.552911in}{2.061584in}}{\pgfqpoint{4.551448in}{2.058051in}}{\pgfqpoint{4.551448in}{2.054368in}}%
\pgfpathcurveto{\pgfqpoint{4.551448in}{2.050685in}}{\pgfqpoint{4.552911in}{2.047152in}}{\pgfqpoint{4.555516in}{2.044547in}}%
\pgfpathcurveto{\pgfqpoint{4.558120in}{2.041942in}}{\pgfqpoint{4.561653in}{2.040479in}}{\pgfqpoint{4.565337in}{2.040479in}}%
\pgfpathclose%
\pgfusepath{stroke}%
\end{pgfscope}%
\begin{pgfscope}%
\pgfpathrectangle{\pgfqpoint{0.438556in}{0.383578in}}{\pgfqpoint{4.650000in}{2.310000in}}%
\pgfusepath{clip}%
\pgfsetbuttcap%
\pgfsetroundjoin%
\pgfsetlinewidth{0.803000pt}%
\definecolor{currentstroke}{rgb}{0.000000,0.356863,0.509804}%
\pgfsetstrokecolor{currentstroke}%
\pgfsetdash{}{0pt}%
\pgfpathmoveto{\pgfqpoint{4.537583in}{2.084163in}}%
\pgfpathcurveto{\pgfqpoint{4.541267in}{2.084163in}}{\pgfqpoint{4.544800in}{2.085627in}}{\pgfqpoint{4.547404in}{2.088231in}}%
\pgfpathcurveto{\pgfqpoint{4.550009in}{2.090836in}}{\pgfqpoint{4.551472in}{2.094369in}}{\pgfqpoint{4.551472in}{2.098052in}}%
\pgfpathcurveto{\pgfqpoint{4.551472in}{2.101736in}}{\pgfqpoint{4.550009in}{2.105269in}}{\pgfqpoint{4.547404in}{2.107873in}}%
\pgfpathcurveto{\pgfqpoint{4.544800in}{2.110478in}}{\pgfqpoint{4.541267in}{2.111941in}}{\pgfqpoint{4.537583in}{2.111941in}}%
\pgfpathcurveto{\pgfqpoint{4.533900in}{2.111941in}}{\pgfqpoint{4.530367in}{2.110478in}}{\pgfqpoint{4.527762in}{2.107873in}}%
\pgfpathcurveto{\pgfqpoint{4.525158in}{2.105269in}}{\pgfqpoint{4.523694in}{2.101736in}}{\pgfqpoint{4.523694in}{2.098052in}}%
\pgfpathcurveto{\pgfqpoint{4.523694in}{2.094369in}}{\pgfqpoint{4.525158in}{2.090836in}}{\pgfqpoint{4.527762in}{2.088231in}}%
\pgfpathcurveto{\pgfqpoint{4.530367in}{2.085627in}}{\pgfqpoint{4.533900in}{2.084163in}}{\pgfqpoint{4.537583in}{2.084163in}}%
\pgfpathclose%
\pgfusepath{stroke}%
\end{pgfscope}%
\begin{pgfscope}%
\pgfpathrectangle{\pgfqpoint{0.438556in}{0.383578in}}{\pgfqpoint{4.650000in}{2.310000in}}%
\pgfusepath{clip}%
\pgfsetbuttcap%
\pgfsetroundjoin%
\pgfsetlinewidth{0.803000pt}%
\definecolor{currentstroke}{rgb}{0.000000,0.356863,0.509804}%
\pgfsetstrokecolor{currentstroke}%
\pgfsetdash{}{0pt}%
\pgfpathmoveto{\pgfqpoint{4.534679in}{2.127848in}}%
\pgfpathcurveto{\pgfqpoint{4.538362in}{2.127848in}}{\pgfqpoint{4.541895in}{2.129311in}}{\pgfqpoint{4.544500in}{2.131915in}}%
\pgfpathcurveto{\pgfqpoint{4.547104in}{2.134520in}}{\pgfqpoint{4.548568in}{2.138053in}}{\pgfqpoint{4.548568in}{2.141736in}}%
\pgfpathcurveto{\pgfqpoint{4.548568in}{2.145420in}}{\pgfqpoint{4.547104in}{2.148953in}}{\pgfqpoint{4.544500in}{2.151557in}}%
\pgfpathcurveto{\pgfqpoint{4.541895in}{2.154162in}}{\pgfqpoint{4.538362in}{2.155625in}}{\pgfqpoint{4.534679in}{2.155625in}}%
\pgfpathcurveto{\pgfqpoint{4.530995in}{2.155625in}}{\pgfqpoint{4.527462in}{2.154162in}}{\pgfqpoint{4.524858in}{2.151557in}}%
\pgfpathcurveto{\pgfqpoint{4.522253in}{2.148953in}}{\pgfqpoint{4.520790in}{2.145420in}}{\pgfqpoint{4.520790in}{2.141736in}}%
\pgfpathcurveto{\pgfqpoint{4.520790in}{2.138053in}}{\pgfqpoint{4.522253in}{2.134520in}}{\pgfqpoint{4.524858in}{2.131915in}}%
\pgfpathcurveto{\pgfqpoint{4.527462in}{2.129311in}}{\pgfqpoint{4.530995in}{2.127848in}}{\pgfqpoint{4.534679in}{2.127848in}}%
\pgfpathclose%
\pgfusepath{stroke}%
\end{pgfscope}%
\begin{pgfscope}%
\pgfpathrectangle{\pgfqpoint{0.438556in}{0.383578in}}{\pgfqpoint{4.650000in}{2.310000in}}%
\pgfusepath{clip}%
\pgfsetbuttcap%
\pgfsetroundjoin%
\pgfsetlinewidth{0.803000pt}%
\definecolor{currentstroke}{rgb}{0.000000,0.356863,0.509804}%
\pgfsetstrokecolor{currentstroke}%
\pgfsetdash{}{0pt}%
\pgfpathmoveto{\pgfqpoint{4.586090in}{2.149690in}}%
\pgfpathcurveto{\pgfqpoint{4.589773in}{2.149690in}}{\pgfqpoint{4.593306in}{2.151153in}}{\pgfqpoint{4.595911in}{2.153758in}}%
\pgfpathcurveto{\pgfqpoint{4.598515in}{2.156362in}}{\pgfqpoint{4.599979in}{2.159895in}}{\pgfqpoint{4.599979in}{2.163579in}}%
\pgfpathcurveto{\pgfqpoint{4.599979in}{2.167262in}}{\pgfqpoint{4.598515in}{2.170795in}}{\pgfqpoint{4.595911in}{2.173399in}}%
\pgfpathcurveto{\pgfqpoint{4.593306in}{2.176004in}}{\pgfqpoint{4.589773in}{2.177467in}}{\pgfqpoint{4.586090in}{2.177467in}}%
\pgfpathcurveto{\pgfqpoint{4.582407in}{2.177467in}}{\pgfqpoint{4.578874in}{2.176004in}}{\pgfqpoint{4.576269in}{2.173399in}}%
\pgfpathcurveto{\pgfqpoint{4.573665in}{2.170795in}}{\pgfqpoint{4.572201in}{2.167262in}}{\pgfqpoint{4.572201in}{2.163579in}}%
\pgfpathcurveto{\pgfqpoint{4.572201in}{2.159895in}}{\pgfqpoint{4.573665in}{2.156362in}}{\pgfqpoint{4.576269in}{2.153758in}}%
\pgfpathcurveto{\pgfqpoint{4.578874in}{2.151153in}}{\pgfqpoint{4.582407in}{2.149690in}}{\pgfqpoint{4.586090in}{2.149690in}}%
\pgfpathclose%
\pgfusepath{stroke}%
\end{pgfscope}%
\begin{pgfscope}%
\pgfpathrectangle{\pgfqpoint{0.438556in}{0.383578in}}{\pgfqpoint{4.650000in}{2.310000in}}%
\pgfusepath{clip}%
\pgfsetbuttcap%
\pgfsetroundjoin%
\pgfsetlinewidth{0.803000pt}%
\definecolor{currentstroke}{rgb}{0.000000,0.356863,0.509804}%
\pgfsetstrokecolor{currentstroke}%
\pgfsetdash{}{0pt}%
\pgfpathmoveto{\pgfqpoint{4.597259in}{2.171532in}}%
\pgfpathcurveto{\pgfqpoint{4.600943in}{2.171532in}}{\pgfqpoint{4.604476in}{2.172995in}}{\pgfqpoint{4.607080in}{2.175600in}}%
\pgfpathcurveto{\pgfqpoint{4.609685in}{2.178204in}}{\pgfqpoint{4.611148in}{2.181737in}}{\pgfqpoint{4.611148in}{2.185421in}}%
\pgfpathcurveto{\pgfqpoint{4.611148in}{2.189104in}}{\pgfqpoint{4.609685in}{2.192637in}}{\pgfqpoint{4.607080in}{2.195242in}}%
\pgfpathcurveto{\pgfqpoint{4.604476in}{2.197846in}}{\pgfqpoint{4.600943in}{2.199310in}}{\pgfqpoint{4.597259in}{2.199310in}}%
\pgfpathcurveto{\pgfqpoint{4.593576in}{2.199310in}}{\pgfqpoint{4.590043in}{2.197846in}}{\pgfqpoint{4.587438in}{2.195242in}}%
\pgfpathcurveto{\pgfqpoint{4.584834in}{2.192637in}}{\pgfqpoint{4.583370in}{2.189104in}}{\pgfqpoint{4.583370in}{2.185421in}}%
\pgfpathcurveto{\pgfqpoint{4.583370in}{2.181737in}}{\pgfqpoint{4.584834in}{2.178204in}}{\pgfqpoint{4.587438in}{2.175600in}}%
\pgfpathcurveto{\pgfqpoint{4.590043in}{2.172995in}}{\pgfqpoint{4.593576in}{2.171532in}}{\pgfqpoint{4.597259in}{2.171532in}}%
\pgfpathclose%
\pgfusepath{stroke}%
\end{pgfscope}%
\begin{pgfscope}%
\pgfpathrectangle{\pgfqpoint{0.438556in}{0.383578in}}{\pgfqpoint{4.650000in}{2.310000in}}%
\pgfusepath{clip}%
\pgfsetbuttcap%
\pgfsetroundjoin%
\pgfsetlinewidth{0.803000pt}%
\definecolor{currentstroke}{rgb}{0.000000,0.356863,0.509804}%
\pgfsetstrokecolor{currentstroke}%
\pgfsetdash{}{0pt}%
\pgfpathmoveto{\pgfqpoint{4.555963in}{2.258900in}}%
\pgfpathcurveto{\pgfqpoint{4.559647in}{2.258900in}}{\pgfqpoint{4.563180in}{2.260364in}}{\pgfqpoint{4.565784in}{2.262968in}}%
\pgfpathcurveto{\pgfqpoint{4.568389in}{2.265573in}}{\pgfqpoint{4.569852in}{2.269106in}}{\pgfqpoint{4.569852in}{2.272789in}}%
\pgfpathcurveto{\pgfqpoint{4.569852in}{2.276473in}}{\pgfqpoint{4.568389in}{2.280006in}}{\pgfqpoint{4.565784in}{2.282610in}}%
\pgfpathcurveto{\pgfqpoint{4.563180in}{2.285215in}}{\pgfqpoint{4.559647in}{2.286678in}}{\pgfqpoint{4.555963in}{2.286678in}}%
\pgfpathcurveto{\pgfqpoint{4.552280in}{2.286678in}}{\pgfqpoint{4.548747in}{2.285215in}}{\pgfqpoint{4.546143in}{2.282610in}}%
\pgfpathcurveto{\pgfqpoint{4.543538in}{2.280006in}}{\pgfqpoint{4.542075in}{2.276473in}}{\pgfqpoint{4.542075in}{2.272789in}}%
\pgfpathcurveto{\pgfqpoint{4.542075in}{2.269106in}}{\pgfqpoint{4.543538in}{2.265573in}}{\pgfqpoint{4.546143in}{2.262968in}}%
\pgfpathcurveto{\pgfqpoint{4.548747in}{2.260364in}}{\pgfqpoint{4.552280in}{2.258900in}}{\pgfqpoint{4.555963in}{2.258900in}}%
\pgfpathclose%
\pgfusepath{stroke}%
\end{pgfscope}%
\begin{pgfscope}%
\pgfpathrectangle{\pgfqpoint{0.438556in}{0.383578in}}{\pgfqpoint{4.650000in}{2.310000in}}%
\pgfusepath{clip}%
\pgfsetbuttcap%
\pgfsetroundjoin%
\pgfsetlinewidth{0.803000pt}%
\definecolor{currentstroke}{rgb}{0.000000,0.356863,0.509804}%
\pgfsetstrokecolor{currentstroke}%
\pgfsetdash{}{0pt}%
\pgfpathmoveto{\pgfqpoint{4.561892in}{2.280742in}}%
\pgfpathcurveto{\pgfqpoint{4.565575in}{2.280742in}}{\pgfqpoint{4.569108in}{2.282206in}}{\pgfqpoint{4.571713in}{2.284810in}}%
\pgfpathcurveto{\pgfqpoint{4.574317in}{2.287415in}}{\pgfqpoint{4.575781in}{2.290948in}}{\pgfqpoint{4.575781in}{2.294631in}}%
\pgfpathcurveto{\pgfqpoint{4.575781in}{2.298315in}}{\pgfqpoint{4.574317in}{2.301848in}}{\pgfqpoint{4.571713in}{2.304452in}}%
\pgfpathcurveto{\pgfqpoint{4.569108in}{2.307057in}}{\pgfqpoint{4.565575in}{2.308520in}}{\pgfqpoint{4.561892in}{2.308520in}}%
\pgfpathcurveto{\pgfqpoint{4.558208in}{2.308520in}}{\pgfqpoint{4.554675in}{2.307057in}}{\pgfqpoint{4.552071in}{2.304452in}}%
\pgfpathcurveto{\pgfqpoint{4.549466in}{2.301848in}}{\pgfqpoint{4.548003in}{2.298315in}}{\pgfqpoint{4.548003in}{2.294631in}}%
\pgfpathcurveto{\pgfqpoint{4.548003in}{2.290948in}}{\pgfqpoint{4.549466in}{2.287415in}}{\pgfqpoint{4.552071in}{2.284810in}}%
\pgfpathcurveto{\pgfqpoint{4.554675in}{2.282206in}}{\pgfqpoint{4.558208in}{2.280742in}}{\pgfqpoint{4.561892in}{2.280742in}}%
\pgfpathclose%
\pgfusepath{stroke}%
\end{pgfscope}%
\begin{pgfscope}%
\pgfpathrectangle{\pgfqpoint{0.438556in}{0.383578in}}{\pgfqpoint{4.650000in}{2.310000in}}%
\pgfusepath{clip}%
\pgfsetbuttcap%
\pgfsetroundjoin%
\pgfsetlinewidth{0.803000pt}%
\definecolor{currentstroke}{rgb}{0.000000,0.356863,0.509804}%
\pgfsetstrokecolor{currentstroke}%
\pgfsetdash{}{0pt}%
\pgfpathmoveto{\pgfqpoint{4.564475in}{2.324427in}}%
\pgfpathcurveto{\pgfqpoint{4.568159in}{2.324427in}}{\pgfqpoint{4.571692in}{2.325890in}}{\pgfqpoint{4.574296in}{2.328495in}}%
\pgfpathcurveto{\pgfqpoint{4.576901in}{2.331099in}}{\pgfqpoint{4.578364in}{2.334632in}}{\pgfqpoint{4.578364in}{2.338316in}}%
\pgfpathcurveto{\pgfqpoint{4.578364in}{2.341999in}}{\pgfqpoint{4.576901in}{2.345532in}}{\pgfqpoint{4.574296in}{2.348137in}}%
\pgfpathcurveto{\pgfqpoint{4.571692in}{2.350741in}}{\pgfqpoint{4.568159in}{2.352204in}}{\pgfqpoint{4.564475in}{2.352204in}}%
\pgfpathcurveto{\pgfqpoint{4.560792in}{2.352204in}}{\pgfqpoint{4.557259in}{2.350741in}}{\pgfqpoint{4.554655in}{2.348137in}}%
\pgfpathcurveto{\pgfqpoint{4.552050in}{2.345532in}}{\pgfqpoint{4.550587in}{2.341999in}}{\pgfqpoint{4.550587in}{2.338316in}}%
\pgfpathcurveto{\pgfqpoint{4.550587in}{2.334632in}}{\pgfqpoint{4.552050in}{2.331099in}}{\pgfqpoint{4.554655in}{2.328495in}}%
\pgfpathcurveto{\pgfqpoint{4.557259in}{2.325890in}}{\pgfqpoint{4.560792in}{2.324427in}}{\pgfqpoint{4.564475in}{2.324427in}}%
\pgfpathclose%
\pgfusepath{stroke}%
\end{pgfscope}%
\begin{pgfscope}%
\pgfpathrectangle{\pgfqpoint{0.438556in}{0.383578in}}{\pgfqpoint{4.650000in}{2.310000in}}%
\pgfusepath{clip}%
\pgfsetbuttcap%
\pgfsetroundjoin%
\pgfsetlinewidth{0.803000pt}%
\definecolor{currentstroke}{rgb}{0.000000,0.356863,0.509804}%
\pgfsetstrokecolor{currentstroke}%
\pgfsetdash{}{0pt}%
\pgfpathmoveto{\pgfqpoint{4.582581in}{2.346269in}}%
\pgfpathcurveto{\pgfqpoint{4.586264in}{2.346269in}}{\pgfqpoint{4.589797in}{2.347732in}}{\pgfqpoint{4.592402in}{2.350337in}}%
\pgfpathcurveto{\pgfqpoint{4.595006in}{2.352941in}}{\pgfqpoint{4.596470in}{2.356474in}}{\pgfqpoint{4.596470in}{2.360158in}}%
\pgfpathcurveto{\pgfqpoint{4.596470in}{2.363841in}}{\pgfqpoint{4.595006in}{2.367374in}}{\pgfqpoint{4.592402in}{2.369979in}}%
\pgfpathcurveto{\pgfqpoint{4.589797in}{2.372583in}}{\pgfqpoint{4.586264in}{2.374047in}}{\pgfqpoint{4.582581in}{2.374047in}}%
\pgfpathcurveto{\pgfqpoint{4.578897in}{2.374047in}}{\pgfqpoint{4.575364in}{2.372583in}}{\pgfqpoint{4.572760in}{2.369979in}}%
\pgfpathcurveto{\pgfqpoint{4.570155in}{2.367374in}}{\pgfqpoint{4.568692in}{2.363841in}}{\pgfqpoint{4.568692in}{2.360158in}}%
\pgfpathcurveto{\pgfqpoint{4.568692in}{2.356474in}}{\pgfqpoint{4.570155in}{2.352941in}}{\pgfqpoint{4.572760in}{2.350337in}}%
\pgfpathcurveto{\pgfqpoint{4.575364in}{2.347732in}}{\pgfqpoint{4.578897in}{2.346269in}}{\pgfqpoint{4.582581in}{2.346269in}}%
\pgfpathclose%
\pgfusepath{stroke}%
\end{pgfscope}%
\begin{pgfscope}%
\pgfpathrectangle{\pgfqpoint{0.438556in}{0.383578in}}{\pgfqpoint{4.650000in}{2.310000in}}%
\pgfusepath{clip}%
\pgfsetbuttcap%
\pgfsetroundjoin%
\pgfsetlinewidth{0.803000pt}%
\definecolor{currentstroke}{rgb}{0.000000,0.356863,0.509804}%
\pgfsetstrokecolor{currentstroke}%
\pgfsetdash{}{0pt}%
\pgfpathmoveto{\pgfqpoint{4.536392in}{2.433637in}}%
\pgfpathcurveto{\pgfqpoint{4.540076in}{2.433637in}}{\pgfqpoint{4.543609in}{2.435101in}}{\pgfqpoint{4.546213in}{2.437705in}}%
\pgfpathcurveto{\pgfqpoint{4.548818in}{2.440310in}}{\pgfqpoint{4.550281in}{2.443843in}}{\pgfqpoint{4.550281in}{2.447526in}}%
\pgfpathcurveto{\pgfqpoint{4.550281in}{2.451210in}}{\pgfqpoint{4.548818in}{2.454743in}}{\pgfqpoint{4.546213in}{2.457347in}}%
\pgfpathcurveto{\pgfqpoint{4.543609in}{2.459952in}}{\pgfqpoint{4.540076in}{2.461415in}}{\pgfqpoint{4.536392in}{2.461415in}}%
\pgfpathcurveto{\pgfqpoint{4.532709in}{2.461415in}}{\pgfqpoint{4.529176in}{2.459952in}}{\pgfqpoint{4.526571in}{2.457347in}}%
\pgfpathcurveto{\pgfqpoint{4.523967in}{2.454743in}}{\pgfqpoint{4.522503in}{2.451210in}}{\pgfqpoint{4.522503in}{2.447526in}}%
\pgfpathcurveto{\pgfqpoint{4.522503in}{2.443843in}}{\pgfqpoint{4.523967in}{2.440310in}}{\pgfqpoint{4.526571in}{2.437705in}}%
\pgfpathcurveto{\pgfqpoint{4.529176in}{2.435101in}}{\pgfqpoint{4.532709in}{2.433637in}}{\pgfqpoint{4.536392in}{2.433637in}}%
\pgfpathclose%
\pgfusepath{stroke}%
\end{pgfscope}%
\begin{pgfscope}%
\pgfpathrectangle{\pgfqpoint{0.438556in}{0.383578in}}{\pgfqpoint{4.650000in}{2.310000in}}%
\pgfusepath{clip}%
\pgfsetbuttcap%
\pgfsetroundjoin%
\pgfsetlinewidth{0.803000pt}%
\definecolor{currentstroke}{rgb}{0.000000,0.356863,0.509804}%
\pgfsetstrokecolor{currentstroke}%
\pgfsetdash{}{0pt}%
\pgfpathmoveto{\pgfqpoint{4.597177in}{2.455480in}}%
\pgfpathcurveto{\pgfqpoint{4.600860in}{2.455480in}}{\pgfqpoint{4.604393in}{2.456943in}}{\pgfqpoint{4.606998in}{2.459547in}}%
\pgfpathcurveto{\pgfqpoint{4.609602in}{2.462152in}}{\pgfqpoint{4.611066in}{2.465685in}}{\pgfqpoint{4.611066in}{2.469368in}}%
\pgfpathcurveto{\pgfqpoint{4.611066in}{2.473052in}}{\pgfqpoint{4.609602in}{2.476585in}}{\pgfqpoint{4.606998in}{2.479189in}}%
\pgfpathcurveto{\pgfqpoint{4.604393in}{2.481794in}}{\pgfqpoint{4.600860in}{2.483257in}}{\pgfqpoint{4.597177in}{2.483257in}}%
\pgfpathcurveto{\pgfqpoint{4.593493in}{2.483257in}}{\pgfqpoint{4.589960in}{2.481794in}}{\pgfqpoint{4.587356in}{2.479189in}}%
\pgfpathcurveto{\pgfqpoint{4.584751in}{2.476585in}}{\pgfqpoint{4.583288in}{2.473052in}}{\pgfqpoint{4.583288in}{2.469368in}}%
\pgfpathcurveto{\pgfqpoint{4.583288in}{2.465685in}}{\pgfqpoint{4.584751in}{2.462152in}}{\pgfqpoint{4.587356in}{2.459547in}}%
\pgfpathcurveto{\pgfqpoint{4.589960in}{2.456943in}}{\pgfqpoint{4.593493in}{2.455480in}}{\pgfqpoint{4.597177in}{2.455480in}}%
\pgfpathclose%
\pgfusepath{stroke}%
\end{pgfscope}%
\begin{pgfscope}%
\pgfpathrectangle{\pgfqpoint{0.438556in}{0.383578in}}{\pgfqpoint{4.650000in}{2.310000in}}%
\pgfusepath{clip}%
\pgfsetbuttcap%
\pgfsetroundjoin%
\pgfsetlinewidth{0.803000pt}%
\definecolor{currentstroke}{rgb}{0.000000,0.356863,0.509804}%
\pgfsetstrokecolor{currentstroke}%
\pgfsetdash{}{0pt}%
\pgfpathmoveto{\pgfqpoint{4.573345in}{2.564690in}}%
\pgfpathcurveto{\pgfqpoint{4.577028in}{2.564690in}}{\pgfqpoint{4.580561in}{2.566154in}}{\pgfqpoint{4.583166in}{2.568758in}}%
\pgfpathcurveto{\pgfqpoint{4.585770in}{2.571363in}}{\pgfqpoint{4.587234in}{2.574896in}}{\pgfqpoint{4.587234in}{2.578579in}}%
\pgfpathcurveto{\pgfqpoint{4.587234in}{2.582262in}}{\pgfqpoint{4.585770in}{2.585795in}}{\pgfqpoint{4.583166in}{2.588400in}}%
\pgfpathcurveto{\pgfqpoint{4.580561in}{2.591005in}}{\pgfqpoint{4.577028in}{2.592468in}}{\pgfqpoint{4.573345in}{2.592468in}}%
\pgfpathcurveto{\pgfqpoint{4.569661in}{2.592468in}}{\pgfqpoint{4.566128in}{2.591005in}}{\pgfqpoint{4.563524in}{2.588400in}}%
\pgfpathcurveto{\pgfqpoint{4.560919in}{2.585795in}}{\pgfqpoint{4.559456in}{2.582262in}}{\pgfqpoint{4.559456in}{2.578579in}}%
\pgfpathcurveto{\pgfqpoint{4.559456in}{2.574896in}}{\pgfqpoint{4.560919in}{2.571363in}}{\pgfqpoint{4.563524in}{2.568758in}}%
\pgfpathcurveto{\pgfqpoint{4.566128in}{2.566154in}}{\pgfqpoint{4.569661in}{2.564690in}}{\pgfqpoint{4.573345in}{2.564690in}}%
\pgfpathclose%
\pgfusepath{stroke}%
\end{pgfscope}%
\begin{pgfscope}%
\pgfsetrectcap%
\pgfsetmiterjoin%
\pgfsetlinewidth{0.501875pt}%
\definecolor{currentstroke}{rgb}{0.317647,0.317647,0.317647}%
\pgfsetstrokecolor{currentstroke}%
\pgfsetdash{}{0pt}%
\pgfpathmoveto{\pgfqpoint{0.438556in}{0.383578in}}%
\pgfpathlineto{\pgfqpoint{0.438556in}{2.693578in}}%
\pgfusepath{stroke}%
\end{pgfscope}%
\begin{pgfscope}%
\pgfsetrectcap%
\pgfsetmiterjoin%
\pgfsetlinewidth{0.501875pt}%
\definecolor{currentstroke}{rgb}{0.317647,0.317647,0.317647}%
\pgfsetstrokecolor{currentstroke}%
\pgfsetdash{}{0pt}%
\pgfpathmoveto{\pgfqpoint{0.438556in}{0.383578in}}%
\pgfpathlineto{\pgfqpoint{5.088556in}{0.383578in}}%
\pgfusepath{stroke}%
\end{pgfscope}%
\begin{pgfscope}%
\pgfsetbuttcap%
\pgfsetroundjoin%
\pgfsetlinewidth{0.803000pt}%
\definecolor{currentstroke}{rgb}{0.333333,0.333333,0.333333}%
\pgfsetstrokecolor{currentstroke}%
\pgfsetdash{}{0pt}%
\pgfpathmoveto{\pgfqpoint{4.855082in}{2.597541in}}%
\pgfpathcurveto{\pgfqpoint{4.862449in}{2.597541in}}{\pgfqpoint{4.869515in}{2.600467in}}{\pgfqpoint{4.874724in}{2.605676in}}%
\pgfpathcurveto{\pgfqpoint{4.879933in}{2.610886in}}{\pgfqpoint{4.882860in}{2.617952in}}{\pgfqpoint{4.882860in}{2.625318in}}%
\pgfpathcurveto{\pgfqpoint{4.882860in}{2.632685in}}{\pgfqpoint{4.879933in}{2.639751in}}{\pgfqpoint{4.874724in}{2.644960in}}%
\pgfpathcurveto{\pgfqpoint{4.869515in}{2.650169in}}{\pgfqpoint{4.862449in}{2.653096in}}{\pgfqpoint{4.855082in}{2.653096in}}%
\pgfpathcurveto{\pgfqpoint{4.847716in}{2.653096in}}{\pgfqpoint{4.840650in}{2.650169in}}{\pgfqpoint{4.835441in}{2.644960in}}%
\pgfpathcurveto{\pgfqpoint{4.830231in}{2.639751in}}{\pgfqpoint{4.827305in}{2.632685in}}{\pgfqpoint{4.827305in}{2.625318in}}%
\pgfpathcurveto{\pgfqpoint{4.827305in}{2.617952in}}{\pgfqpoint{4.830231in}{2.610886in}}{\pgfqpoint{4.835441in}{2.605676in}}%
\pgfpathcurveto{\pgfqpoint{4.840650in}{2.600467in}}{\pgfqpoint{4.847716in}{2.597541in}}{\pgfqpoint{4.855082in}{2.597541in}}%
\pgfpathclose%
\pgfusepath{stroke}%
\end{pgfscope}%
\begin{pgfscope}%
\definecolor{textcolor}{rgb}{0.000000,0.000000,0.000000}%
\pgfsetstrokecolor{textcolor}%
\pgfsetfillcolor{textcolor}%
\pgftext[x=4.938382in,y=2.601023in,left,base]{\color{textcolor}\rmfamily\fontsize{6.664000}{7.996800}\selectfont \(\displaystyle S_{1}\)}%
\end{pgfscope}%
\begin{pgfscope}%
\pgfsetbuttcap%
\pgfsetroundjoin%
\pgfsetlinewidth{0.803000pt}%
\definecolor{currentstroke}{rgb}{0.686275,0.352941,0.313725}%
\pgfsetstrokecolor{currentstroke}%
\pgfsetdash{}{0pt}%
\pgfpathmoveto{\pgfqpoint{4.855082in}{2.484718in}}%
\pgfpathcurveto{\pgfqpoint{4.860607in}{2.484718in}}{\pgfqpoint{4.865907in}{2.486913in}}{\pgfqpoint{4.869814in}{2.490820in}}%
\pgfpathcurveto{\pgfqpoint{4.873721in}{2.494727in}}{\pgfqpoint{4.875916in}{2.500027in}}{\pgfqpoint{4.875916in}{2.505552in}}%
\pgfpathcurveto{\pgfqpoint{4.875916in}{2.511077in}}{\pgfqpoint{4.873721in}{2.516376in}}{\pgfqpoint{4.869814in}{2.520283in}}%
\pgfpathcurveto{\pgfqpoint{4.865907in}{2.524190in}}{\pgfqpoint{4.860607in}{2.526385in}}{\pgfqpoint{4.855082in}{2.526385in}}%
\pgfpathcurveto{\pgfqpoint{4.849557in}{2.526385in}}{\pgfqpoint{4.844258in}{2.524190in}}{\pgfqpoint{4.840351in}{2.520283in}}%
\pgfpathcurveto{\pgfqpoint{4.836444in}{2.516376in}}{\pgfqpoint{4.834249in}{2.511077in}}{\pgfqpoint{4.834249in}{2.505552in}}%
\pgfpathcurveto{\pgfqpoint{4.834249in}{2.500027in}}{\pgfqpoint{4.836444in}{2.494727in}}{\pgfqpoint{4.840351in}{2.490820in}}%
\pgfpathcurveto{\pgfqpoint{4.844258in}{2.486913in}}{\pgfqpoint{4.849557in}{2.484718in}}{\pgfqpoint{4.855082in}{2.484718in}}%
\pgfpathclose%
\pgfusepath{stroke}%
\end{pgfscope}%
\begin{pgfscope}%
\definecolor{textcolor}{rgb}{0.000000,0.000000,0.000000}%
\pgfsetstrokecolor{textcolor}%
\pgfsetfillcolor{textcolor}%
\pgftext[x=4.938382in,y=2.481256in,left,base]{\color{textcolor}\rmfamily\fontsize{6.664000}{7.996800}\selectfont \(\displaystyle S_{2}\)}%
\end{pgfscope}%
\begin{pgfscope}%
\pgfsetbuttcap%
\pgfsetroundjoin%
\pgfsetlinewidth{0.803000pt}%
\definecolor{currentstroke}{rgb}{0.000000,0.356863,0.509804}%
\pgfsetstrokecolor{currentstroke}%
\pgfsetdash{}{0pt}%
\pgfpathmoveto{\pgfqpoint{4.855082in}{2.371896in}}%
\pgfpathcurveto{\pgfqpoint{4.858766in}{2.371896in}}{\pgfqpoint{4.862299in}{2.373360in}}{\pgfqpoint{4.864903in}{2.375964in}}%
\pgfpathcurveto{\pgfqpoint{4.867508in}{2.378569in}}{\pgfqpoint{4.868971in}{2.382102in}}{\pgfqpoint{4.868971in}{2.385785in}}%
\pgfpathcurveto{\pgfqpoint{4.868971in}{2.389468in}}{\pgfqpoint{4.867508in}{2.393001in}}{\pgfqpoint{4.864903in}{2.395606in}}%
\pgfpathcurveto{\pgfqpoint{4.862299in}{2.398210in}}{\pgfqpoint{4.858766in}{2.399674in}}{\pgfqpoint{4.855082in}{2.399674in}}%
\pgfpathcurveto{\pgfqpoint{4.851399in}{2.399674in}}{\pgfqpoint{4.847866in}{2.398210in}}{\pgfqpoint{4.845261in}{2.395606in}}%
\pgfpathcurveto{\pgfqpoint{4.842657in}{2.393001in}}{\pgfqpoint{4.841193in}{2.389468in}}{\pgfqpoint{4.841193in}{2.385785in}}%
\pgfpathcurveto{\pgfqpoint{4.841193in}{2.382102in}}{\pgfqpoint{4.842657in}{2.378569in}}{\pgfqpoint{4.845261in}{2.375964in}}%
\pgfpathcurveto{\pgfqpoint{4.847866in}{2.373360in}}{\pgfqpoint{4.851399in}{2.371896in}}{\pgfqpoint{4.855082in}{2.371896in}}%
\pgfpathclose%
\pgfusepath{stroke}%
\end{pgfscope}%
\begin{pgfscope}%
\definecolor{textcolor}{rgb}{0.000000,0.000000,0.000000}%
\pgfsetstrokecolor{textcolor}%
\pgfsetfillcolor{textcolor}%
\pgftext[x=4.938382in,y=2.361489in,left,base]{\color{textcolor}\rmfamily\fontsize{6.664000}{7.996800}\selectfont \(\displaystyle S_{3}\)}%
\end{pgfscope}%
\begin{pgfscope}%
\pgfsetbuttcap%
\pgfsetroundjoin%
\definecolor{currentfill}{rgb}{0.490196,0.588235,0.431373}%
\pgfsetfillcolor{currentfill}%
\pgfsetlinewidth{0.803000pt}%
\definecolor{currentstroke}{rgb}{0.490196,0.588235,0.431373}%
\pgfsetstrokecolor{currentstroke}%
\pgfsetdash{}{0pt}%
\pgfsys@defobject{currentmarker}{\pgfqpoint{-0.006944in}{-0.006944in}}{\pgfqpoint{0.006944in}{0.006944in}}{%
\pgfpathmoveto{\pgfqpoint{0.000000in}{-0.006944in}}%
\pgfpathcurveto{\pgfqpoint{0.001842in}{-0.006944in}}{\pgfqpoint{0.003608in}{-0.006213in}}{\pgfqpoint{0.004910in}{-0.004910in}}%
\pgfpathcurveto{\pgfqpoint{0.006213in}{-0.003608in}}{\pgfqpoint{0.006944in}{-0.001842in}}{\pgfqpoint{0.006944in}{0.000000in}}%
\pgfpathcurveto{\pgfqpoint{0.006944in}{0.001842in}}{\pgfqpoint{0.006213in}{0.003608in}}{\pgfqpoint{0.004910in}{0.004910in}}%
\pgfpathcurveto{\pgfqpoint{0.003608in}{0.006213in}}{\pgfqpoint{0.001842in}{0.006944in}}{\pgfqpoint{0.000000in}{0.006944in}}%
\pgfpathcurveto{\pgfqpoint{-0.001842in}{0.006944in}}{\pgfqpoint{-0.003608in}{0.006213in}}{\pgfqpoint{-0.004910in}{0.004910in}}%
\pgfpathcurveto{\pgfqpoint{-0.006213in}{0.003608in}}{\pgfqpoint{-0.006944in}{0.001842in}}{\pgfqpoint{-0.006944in}{0.000000in}}%
\pgfpathcurveto{\pgfqpoint{-0.006944in}{-0.001842in}}{\pgfqpoint{-0.006213in}{-0.003608in}}{\pgfqpoint{-0.004910in}{-0.004910in}}%
\pgfpathcurveto{\pgfqpoint{-0.003608in}{-0.006213in}}{\pgfqpoint{-0.001842in}{-0.006944in}}{\pgfqpoint{0.000000in}{-0.006944in}}%
\pgfpathclose%
\pgfusepath{stroke,fill}%
}%
\begin{pgfscope}%
\pgfsys@transformshift{4.855082in}{2.266018in}%
\pgfsys@useobject{currentmarker}{}%
\end{pgfscope}%
\end{pgfscope}%
\begin{pgfscope}%
\definecolor{textcolor}{rgb}{0.000000,0.000000,0.000000}%
\pgfsetstrokecolor{textcolor}%
\pgfsetfillcolor{textcolor}%
\pgftext[x=4.938382in,y=2.241723in,left,base]{\color{textcolor}\rmfamily\fontsize{6.664000}{7.996800}\selectfont \(\displaystyle S_{4}\)}%
\end{pgfscope}%
\end{pgfpicture}%
\makeatother%
\endgroup%

	\caption[Example of a training batch.]{Example of a training batch. After The chosen batch-size for the XOR-related task is eight. The network evaluates}
	\label{inputofabatch}
\end{figure}


\begin{figure}
	\centering
	%% Creator: Matplotlib, PGF backend
%%
%% To include the figure in your LaTeX document, write
%%   \input{<filename>.pgf}
%%
%% Make sure the required packages are loaded in your preamble
%%   \usepackage{pgf}
%%
%% Figures using additional raster images can only be included by \input if
%% they are in the same directory as the main LaTeX file. For loading figures
%% from other directories you can use the `import` package
%%   \usepackage{import}
%% and then include the figures with
%%   \import{<path to file>}{<filename>.pgf}
%%
%% Matplotlib used the following preamble
%%   \usepackage{amsmath} \usepackage{pifont} \usepackage{xcolor} \definecolor{green}{HTML}{467821} \definecolor{red}{HTML}{CF4457} \usepackage[detect-all]{siunitx}
%%   \usepackage{fontspec}
%%
\begingroup%
\makeatletter%
\begin{pgfpicture}%
\pgfpathrectangle{\pgfpointorigin}{\pgfqpoint{5.198215in}{2.941978in}}%
\pgfusepath{use as bounding box, clip}%
\begin{pgfscope}%
\pgfsetbuttcap%
\pgfsetmiterjoin%
\pgfsetlinewidth{0.000000pt}%
\definecolor{currentstroke}{rgb}{0.000000,0.000000,0.000000}%
\pgfsetstrokecolor{currentstroke}%
\pgfsetstrokeopacity{0.000000}%
\pgfsetdash{}{0pt}%
\pgfpathmoveto{\pgfqpoint{0.000000in}{0.000000in}}%
\pgfpathlineto{\pgfqpoint{5.198215in}{0.000000in}}%
\pgfpathlineto{\pgfqpoint{5.198215in}{2.941978in}}%
\pgfpathlineto{\pgfqpoint{0.000000in}{2.941978in}}%
\pgfpathclose%
\pgfusepath{}%
\end{pgfscope}%
\begin{pgfscope}%
\pgfsetbuttcap%
\pgfsetmiterjoin%
\pgfsetlinewidth{0.000000pt}%
\definecolor{currentstroke}{rgb}{0.000000,0.000000,0.000000}%
\pgfsetstrokecolor{currentstroke}%
\pgfsetstrokeopacity{0.000000}%
\pgfsetdash{}{0pt}%
\pgfpathmoveto{\pgfqpoint{0.448215in}{1.689230in}}%
\pgfpathlineto{\pgfqpoint{1.671899in}{1.689230in}}%
\pgfpathlineto{\pgfqpoint{1.671899in}{2.693578in}}%
\pgfpathlineto{\pgfqpoint{0.448215in}{2.693578in}}%
\pgfpathclose%
\pgfusepath{}%
\end{pgfscope}%
\begin{pgfscope}%
\pgfsetbuttcap%
\pgfsetroundjoin%
\definecolor{currentfill}{rgb}{0.317647,0.317647,0.317647}%
\pgfsetfillcolor{currentfill}%
\pgfsetlinewidth{0.501875pt}%
\definecolor{currentstroke}{rgb}{0.317647,0.317647,0.317647}%
\pgfsetstrokecolor{currentstroke}%
\pgfsetdash{}{0pt}%
\pgfsys@defobject{currentmarker}{\pgfqpoint{0.000000in}{-0.020833in}}{\pgfqpoint{0.000000in}{0.000000in}}{%
\pgfpathmoveto{\pgfqpoint{0.000000in}{0.000000in}}%
\pgfpathlineto{\pgfqpoint{0.000000in}{-0.020833in}}%
\pgfusepath{stroke,fill}%
}%
\begin{pgfscope}%
\pgfsys@transformshift{0.497163in}{1.689230in}%
\pgfsys@useobject{currentmarker}{}%
\end{pgfscope}%
\end{pgfscope}%
\begin{pgfscope}%
\definecolor{textcolor}{rgb}{0.317647,0.317647,0.317647}%
\pgfsetstrokecolor{textcolor}%
\pgfsetfillcolor{textcolor}%
\pgftext[x=0.497163in,y=1.640619in,,top]{\color{textcolor}\rmfamily\fontsize{6.664000}{7.996800}\selectfont \(\displaystyle 0\)}%
\end{pgfscope}%
\begin{pgfscope}%
\pgfsetbuttcap%
\pgfsetroundjoin%
\definecolor{currentfill}{rgb}{0.317647,0.317647,0.317647}%
\pgfsetfillcolor{currentfill}%
\pgfsetlinewidth{0.501875pt}%
\definecolor{currentstroke}{rgb}{0.317647,0.317647,0.317647}%
\pgfsetstrokecolor{currentstroke}%
\pgfsetdash{}{0pt}%
\pgfsys@defobject{currentmarker}{\pgfqpoint{0.000000in}{-0.020833in}}{\pgfqpoint{0.000000in}{0.000000in}}{%
\pgfpathmoveto{\pgfqpoint{0.000000in}{0.000000in}}%
\pgfpathlineto{\pgfqpoint{0.000000in}{-0.020833in}}%
\pgfusepath{stroke,fill}%
}%
\begin{pgfscope}%
\pgfsys@transformshift{0.986636in}{1.689230in}%
\pgfsys@useobject{currentmarker}{}%
\end{pgfscope}%
\end{pgfscope}%
\begin{pgfscope}%
\definecolor{textcolor}{rgb}{0.317647,0.317647,0.317647}%
\pgfsetstrokecolor{textcolor}%
\pgfsetfillcolor{textcolor}%
\pgftext[x=0.986636in,y=1.640619in,,top]{\color{textcolor}\rmfamily\fontsize{6.664000}{7.996800}\selectfont \(\displaystyle 50\)}%
\end{pgfscope}%
\begin{pgfscope}%
\pgfsetbuttcap%
\pgfsetroundjoin%
\definecolor{currentfill}{rgb}{0.317647,0.317647,0.317647}%
\pgfsetfillcolor{currentfill}%
\pgfsetlinewidth{0.501875pt}%
\definecolor{currentstroke}{rgb}{0.317647,0.317647,0.317647}%
\pgfsetstrokecolor{currentstroke}%
\pgfsetdash{}{0pt}%
\pgfsys@defobject{currentmarker}{\pgfqpoint{0.000000in}{-0.020833in}}{\pgfqpoint{0.000000in}{0.000000in}}{%
\pgfpathmoveto{\pgfqpoint{0.000000in}{0.000000in}}%
\pgfpathlineto{\pgfqpoint{0.000000in}{-0.020833in}}%
\pgfusepath{stroke,fill}%
}%
\begin{pgfscope}%
\pgfsys@transformshift{1.476110in}{1.689230in}%
\pgfsys@useobject{currentmarker}{}%
\end{pgfscope}%
\end{pgfscope}%
\begin{pgfscope}%
\definecolor{textcolor}{rgb}{0.317647,0.317647,0.317647}%
\pgfsetstrokecolor{textcolor}%
\pgfsetfillcolor{textcolor}%
\pgftext[x=1.476110in,y=1.640619in,,top]{\color{textcolor}\rmfamily\fontsize{6.664000}{7.996800}\selectfont \(\displaystyle 100\)}%
\end{pgfscope}%
\begin{pgfscope}%
\pgfsetbuttcap%
\pgfsetroundjoin%
\definecolor{currentfill}{rgb}{0.317647,0.317647,0.317647}%
\pgfsetfillcolor{currentfill}%
\pgfsetlinewidth{0.501875pt}%
\definecolor{currentstroke}{rgb}{0.317647,0.317647,0.317647}%
\pgfsetstrokecolor{currentstroke}%
\pgfsetdash{}{0pt}%
\pgfsys@defobject{currentmarker}{\pgfqpoint{-0.020833in}{0.000000in}}{\pgfqpoint{0.000000in}{0.000000in}}{%
\pgfpathmoveto{\pgfqpoint{0.000000in}{0.000000in}}%
\pgfpathlineto{\pgfqpoint{-0.020833in}{0.000000in}}%
\pgfusepath{stroke,fill}%
}%
\begin{pgfscope}%
\pgfsys@transformshift{0.448215in}{1.709727in}%
\pgfsys@useobject{currentmarker}{}%
\end{pgfscope}%
\end{pgfscope}%
\begin{pgfscope}%
\definecolor{textcolor}{rgb}{0.317647,0.317647,0.317647}%
\pgfsetstrokecolor{textcolor}%
\pgfsetfillcolor{textcolor}%
\pgftext[x=0.358130in,y=1.677610in,left,base]{\color{textcolor}\rmfamily\fontsize{6.664000}{7.996800}\selectfont \(\displaystyle 0\)}%
\end{pgfscope}%
\begin{pgfscope}%
\pgfsetbuttcap%
\pgfsetroundjoin%
\definecolor{currentfill}{rgb}{0.317647,0.317647,0.317647}%
\pgfsetfillcolor{currentfill}%
\pgfsetlinewidth{0.501875pt}%
\definecolor{currentstroke}{rgb}{0.317647,0.317647,0.317647}%
\pgfsetstrokecolor{currentstroke}%
\pgfsetdash{}{0pt}%
\pgfsys@defobject{currentmarker}{\pgfqpoint{-0.020833in}{0.000000in}}{\pgfqpoint{0.000000in}{0.000000in}}{%
\pgfpathmoveto{\pgfqpoint{0.000000in}{0.000000in}}%
\pgfpathlineto{\pgfqpoint{-0.020833in}{0.000000in}}%
\pgfusepath{stroke,fill}%
}%
\begin{pgfscope}%
\pgfsys@transformshift{0.448215in}{1.914696in}%
\pgfsys@useobject{currentmarker}{}%
\end{pgfscope}%
\end{pgfscope}%
\begin{pgfscope}%
\definecolor{textcolor}{rgb}{0.317647,0.317647,0.317647}%
\pgfsetstrokecolor{textcolor}%
\pgfsetfillcolor{textcolor}%
\pgftext[x=0.302767in,y=1.882579in,left,base]{\color{textcolor}\rmfamily\fontsize{6.664000}{7.996800}\selectfont \(\displaystyle 20\)}%
\end{pgfscope}%
\begin{pgfscope}%
\pgfsetbuttcap%
\pgfsetroundjoin%
\definecolor{currentfill}{rgb}{0.317647,0.317647,0.317647}%
\pgfsetfillcolor{currentfill}%
\pgfsetlinewidth{0.501875pt}%
\definecolor{currentstroke}{rgb}{0.317647,0.317647,0.317647}%
\pgfsetstrokecolor{currentstroke}%
\pgfsetdash{}{0pt}%
\pgfsys@defobject{currentmarker}{\pgfqpoint{-0.020833in}{0.000000in}}{\pgfqpoint{0.000000in}{0.000000in}}{%
\pgfpathmoveto{\pgfqpoint{0.000000in}{0.000000in}}%
\pgfpathlineto{\pgfqpoint{-0.020833in}{0.000000in}}%
\pgfusepath{stroke,fill}%
}%
\begin{pgfscope}%
\pgfsys@transformshift{0.448215in}{2.119665in}%
\pgfsys@useobject{currentmarker}{}%
\end{pgfscope}%
\end{pgfscope}%
\begin{pgfscope}%
\definecolor{textcolor}{rgb}{0.317647,0.317647,0.317647}%
\pgfsetstrokecolor{textcolor}%
\pgfsetfillcolor{textcolor}%
\pgftext[x=0.302767in,y=2.087548in,left,base]{\color{textcolor}\rmfamily\fontsize{6.664000}{7.996800}\selectfont \(\displaystyle 40\)}%
\end{pgfscope}%
\begin{pgfscope}%
\pgfsetbuttcap%
\pgfsetroundjoin%
\definecolor{currentfill}{rgb}{0.317647,0.317647,0.317647}%
\pgfsetfillcolor{currentfill}%
\pgfsetlinewidth{0.501875pt}%
\definecolor{currentstroke}{rgb}{0.317647,0.317647,0.317647}%
\pgfsetstrokecolor{currentstroke}%
\pgfsetdash{}{0pt}%
\pgfsys@defobject{currentmarker}{\pgfqpoint{-0.020833in}{0.000000in}}{\pgfqpoint{0.000000in}{0.000000in}}{%
\pgfpathmoveto{\pgfqpoint{0.000000in}{0.000000in}}%
\pgfpathlineto{\pgfqpoint{-0.020833in}{0.000000in}}%
\pgfusepath{stroke,fill}%
}%
\begin{pgfscope}%
\pgfsys@transformshift{0.448215in}{2.324634in}%
\pgfsys@useobject{currentmarker}{}%
\end{pgfscope}%
\end{pgfscope}%
\begin{pgfscope}%
\definecolor{textcolor}{rgb}{0.317647,0.317647,0.317647}%
\pgfsetstrokecolor{textcolor}%
\pgfsetfillcolor{textcolor}%
\pgftext[x=0.302767in,y=2.292517in,left,base]{\color{textcolor}\rmfamily\fontsize{6.664000}{7.996800}\selectfont \(\displaystyle 60\)}%
\end{pgfscope}%
\begin{pgfscope}%
\pgfsetbuttcap%
\pgfsetroundjoin%
\definecolor{currentfill}{rgb}{0.317647,0.317647,0.317647}%
\pgfsetfillcolor{currentfill}%
\pgfsetlinewidth{0.501875pt}%
\definecolor{currentstroke}{rgb}{0.317647,0.317647,0.317647}%
\pgfsetstrokecolor{currentstroke}%
\pgfsetdash{}{0pt}%
\pgfsys@defobject{currentmarker}{\pgfqpoint{-0.020833in}{0.000000in}}{\pgfqpoint{0.000000in}{0.000000in}}{%
\pgfpathmoveto{\pgfqpoint{0.000000in}{0.000000in}}%
\pgfpathlineto{\pgfqpoint{-0.020833in}{0.000000in}}%
\pgfusepath{stroke,fill}%
}%
\begin{pgfscope}%
\pgfsys@transformshift{0.448215in}{2.529603in}%
\pgfsys@useobject{currentmarker}{}%
\end{pgfscope}%
\end{pgfscope}%
\begin{pgfscope}%
\definecolor{textcolor}{rgb}{0.317647,0.317647,0.317647}%
\pgfsetstrokecolor{textcolor}%
\pgfsetfillcolor{textcolor}%
\pgftext[x=0.302767in,y=2.497486in,left,base]{\color{textcolor}\rmfamily\fontsize{6.664000}{7.996800}\selectfont \(\displaystyle 80\)}%
\end{pgfscope}%
\begin{pgfscope}%
\definecolor{textcolor}{rgb}{0.317647,0.317647,0.317647}%
\pgfsetstrokecolor{textcolor}%
\pgfsetfillcolor{textcolor}%
\pgftext[x=0.247212in,y=2.191404in,,bottom,rotate=90.000000]{\color{textcolor}\rmfamily\fontsize{6.664000}{7.996800}\selectfont \(\displaystyle S_j^{(h)}\)}%
\end{pgfscope}%
\begin{pgfscope}%
\pgfpathrectangle{\pgfqpoint{0.448215in}{1.689230in}}{\pgfqpoint{1.223684in}{1.004348in}}%
\pgfusepath{clip}%
\pgfsetbuttcap%
\pgfsetroundjoin%
\pgfsetlinewidth{1.003750pt}%
\definecolor{currentstroke}{rgb}{0.333333,0.333333,0.333333}%
\pgfsetstrokecolor{currentstroke}%
\pgfsetdash{}{0pt}%
\pgfpathmoveto{\pgfqpoint{0.634450in}{1.719976in}}%
\pgfpathlineto{\pgfqpoint{0.634450in}{1.699479in}}%
\pgfusepath{stroke}%
\end{pgfscope}%
\begin{pgfscope}%
\pgfpathrectangle{\pgfqpoint{0.448215in}{1.689230in}}{\pgfqpoint{1.223684in}{1.004348in}}%
\pgfusepath{clip}%
\pgfsetbuttcap%
\pgfsetroundjoin%
\pgfsetlinewidth{1.003750pt}%
\definecolor{currentstroke}{rgb}{0.333333,0.333333,0.333333}%
\pgfsetstrokecolor{currentstroke}%
\pgfsetdash{}{0pt}%
\pgfpathmoveto{\pgfqpoint{0.534206in}{1.730224in}}%
\pgfpathlineto{\pgfqpoint{0.534206in}{1.709727in}}%
\pgfusepath{stroke}%
\end{pgfscope}%
\begin{pgfscope}%
\pgfpathrectangle{\pgfqpoint{0.448215in}{1.689230in}}{\pgfqpoint{1.223684in}{1.004348in}}%
\pgfusepath{clip}%
\pgfsetbuttcap%
\pgfsetroundjoin%
\pgfsetlinewidth{1.003750pt}%
\definecolor{currentstroke}{rgb}{0.333333,0.333333,0.333333}%
\pgfsetstrokecolor{currentstroke}%
\pgfsetdash{}{0pt}%
\pgfpathmoveto{\pgfqpoint{0.559424in}{1.801963in}}%
\pgfpathlineto{\pgfqpoint{0.559424in}{1.781466in}}%
\pgfusepath{stroke}%
\end{pgfscope}%
\begin{pgfscope}%
\pgfpathrectangle{\pgfqpoint{0.448215in}{1.689230in}}{\pgfqpoint{1.223684in}{1.004348in}}%
\pgfusepath{clip}%
\pgfsetbuttcap%
\pgfsetroundjoin%
\pgfsetlinewidth{1.003750pt}%
\definecolor{currentstroke}{rgb}{0.333333,0.333333,0.333333}%
\pgfsetstrokecolor{currentstroke}%
\pgfsetdash{}{0pt}%
\pgfpathmoveto{\pgfqpoint{0.634763in}{1.832708in}}%
\pgfpathlineto{\pgfqpoint{0.634763in}{1.812212in}}%
\pgfusepath{stroke}%
\end{pgfscope}%
\begin{pgfscope}%
\pgfpathrectangle{\pgfqpoint{0.448215in}{1.689230in}}{\pgfqpoint{1.223684in}{1.004348in}}%
\pgfusepath{clip}%
\pgfsetbuttcap%
\pgfsetroundjoin%
\pgfsetlinewidth{1.003750pt}%
\definecolor{currentstroke}{rgb}{0.333333,0.333333,0.333333}%
\pgfsetstrokecolor{currentstroke}%
\pgfsetdash{}{0pt}%
\pgfpathmoveto{\pgfqpoint{0.859295in}{1.842957in}}%
\pgfpathlineto{\pgfqpoint{0.859295in}{1.822460in}}%
\pgfusepath{stroke}%
\end{pgfscope}%
\begin{pgfscope}%
\pgfpathrectangle{\pgfqpoint{0.448215in}{1.689230in}}{\pgfqpoint{1.223684in}{1.004348in}}%
\pgfusepath{clip}%
\pgfsetbuttcap%
\pgfsetroundjoin%
\pgfsetlinewidth{1.003750pt}%
\definecolor{currentstroke}{rgb}{0.333333,0.333333,0.333333}%
\pgfsetstrokecolor{currentstroke}%
\pgfsetdash{}{0pt}%
\pgfpathmoveto{\pgfqpoint{0.759129in}{1.904448in}}%
\pgfpathlineto{\pgfqpoint{0.759129in}{1.883951in}}%
\pgfusepath{stroke}%
\end{pgfscope}%
\begin{pgfscope}%
\pgfpathrectangle{\pgfqpoint{0.448215in}{1.689230in}}{\pgfqpoint{1.223684in}{1.004348in}}%
\pgfusepath{clip}%
\pgfsetbuttcap%
\pgfsetroundjoin%
\pgfsetlinewidth{1.003750pt}%
\definecolor{currentstroke}{rgb}{0.333333,0.333333,0.333333}%
\pgfsetstrokecolor{currentstroke}%
\pgfsetdash{}{0pt}%
\pgfpathmoveto{\pgfqpoint{0.609154in}{1.914696in}}%
\pgfpathlineto{\pgfqpoint{0.609154in}{1.894199in}}%
\pgfusepath{stroke}%
\end{pgfscope}%
\begin{pgfscope}%
\pgfpathrectangle{\pgfqpoint{0.448215in}{1.689230in}}{\pgfqpoint{1.223684in}{1.004348in}}%
\pgfusepath{clip}%
\pgfsetbuttcap%
\pgfsetroundjoin%
\pgfsetlinewidth{1.003750pt}%
\definecolor{currentstroke}{rgb}{0.333333,0.333333,0.333333}%
\pgfsetstrokecolor{currentstroke}%
\pgfsetdash{}{0pt}%
\pgfpathmoveto{\pgfqpoint{0.659355in}{1.945441in}}%
\pgfpathlineto{\pgfqpoint{0.659355in}{1.924945in}}%
\pgfusepath{stroke}%
\end{pgfscope}%
\begin{pgfscope}%
\pgfpathrectangle{\pgfqpoint{0.448215in}{1.689230in}}{\pgfqpoint{1.223684in}{1.004348in}}%
\pgfusepath{clip}%
\pgfsetbuttcap%
\pgfsetroundjoin%
\pgfsetlinewidth{1.003750pt}%
\definecolor{currentstroke}{rgb}{0.333333,0.333333,0.333333}%
\pgfsetstrokecolor{currentstroke}%
\pgfsetdash{}{0pt}%
\pgfpathmoveto{\pgfqpoint{0.609389in}{1.996684in}}%
\pgfpathlineto{\pgfqpoint{0.609389in}{1.976187in}}%
\pgfusepath{stroke}%
\end{pgfscope}%
\begin{pgfscope}%
\pgfpathrectangle{\pgfqpoint{0.448215in}{1.689230in}}{\pgfqpoint{1.223684in}{1.004348in}}%
\pgfusepath{clip}%
\pgfsetbuttcap%
\pgfsetroundjoin%
\pgfsetlinewidth{1.003750pt}%
\definecolor{currentstroke}{rgb}{0.333333,0.333333,0.333333}%
\pgfsetstrokecolor{currentstroke}%
\pgfsetdash{}{0pt}%
\pgfpathmoveto{\pgfqpoint{0.859060in}{2.006932in}}%
\pgfpathlineto{\pgfqpoint{0.859060in}{1.986435in}}%
\pgfusepath{stroke}%
\end{pgfscope}%
\begin{pgfscope}%
\pgfpathrectangle{\pgfqpoint{0.448215in}{1.689230in}}{\pgfqpoint{1.223684in}{1.004348in}}%
\pgfusepath{clip}%
\pgfsetbuttcap%
\pgfsetroundjoin%
\pgfsetlinewidth{1.003750pt}%
\definecolor{currentstroke}{rgb}{0.333333,0.333333,0.333333}%
\pgfsetstrokecolor{currentstroke}%
\pgfsetdash{}{0pt}%
\pgfpathmoveto{\pgfqpoint{0.584485in}{2.017181in}}%
\pgfpathlineto{\pgfqpoint{0.584485in}{1.996684in}}%
\pgfusepath{stroke}%
\end{pgfscope}%
\begin{pgfscope}%
\pgfpathrectangle{\pgfqpoint{0.448215in}{1.689230in}}{\pgfqpoint{1.223684in}{1.004348in}}%
\pgfusepath{clip}%
\pgfsetbuttcap%
\pgfsetroundjoin%
\pgfsetlinewidth{1.003750pt}%
\definecolor{currentstroke}{rgb}{0.333333,0.333333,0.333333}%
\pgfsetstrokecolor{currentstroke}%
\pgfsetdash{}{0pt}%
\pgfpathmoveto{\pgfqpoint{0.809094in}{2.027429in}}%
\pgfpathlineto{\pgfqpoint{0.809094in}{2.006932in}}%
\pgfusepath{stroke}%
\end{pgfscope}%
\begin{pgfscope}%
\pgfpathrectangle{\pgfqpoint{0.448215in}{1.689230in}}{\pgfqpoint{1.223684in}{1.004348in}}%
\pgfusepath{clip}%
\pgfsetbuttcap%
\pgfsetroundjoin%
\pgfsetlinewidth{1.003750pt}%
\definecolor{currentstroke}{rgb}{0.333333,0.333333,0.333333}%
\pgfsetstrokecolor{currentstroke}%
\pgfsetdash{}{0pt}%
\pgfpathmoveto{\pgfqpoint{0.584798in}{2.037677in}}%
\pgfpathlineto{\pgfqpoint{0.584798in}{2.017181in}}%
\pgfusepath{stroke}%
\end{pgfscope}%
\begin{pgfscope}%
\pgfpathrectangle{\pgfqpoint{0.448215in}{1.689230in}}{\pgfqpoint{1.223684in}{1.004348in}}%
\pgfusepath{clip}%
\pgfsetbuttcap%
\pgfsetroundjoin%
\pgfsetlinewidth{1.003750pt}%
\definecolor{currentstroke}{rgb}{0.333333,0.333333,0.333333}%
\pgfsetstrokecolor{currentstroke}%
\pgfsetdash{}{0pt}%
\pgfpathmoveto{\pgfqpoint{0.809329in}{2.058174in}}%
\pgfpathlineto{\pgfqpoint{0.809329in}{2.037677in}}%
\pgfusepath{stroke}%
\end{pgfscope}%
\begin{pgfscope}%
\pgfpathrectangle{\pgfqpoint{0.448215in}{1.689230in}}{\pgfqpoint{1.223684in}{1.004348in}}%
\pgfusepath{clip}%
\pgfsetbuttcap%
\pgfsetroundjoin%
\pgfsetlinewidth{1.003750pt}%
\definecolor{currentstroke}{rgb}{0.333333,0.333333,0.333333}%
\pgfsetstrokecolor{currentstroke}%
\pgfsetdash{}{0pt}%
\pgfpathmoveto{\pgfqpoint{0.584955in}{2.109417in}}%
\pgfpathlineto{\pgfqpoint{0.584955in}{2.088920in}}%
\pgfusepath{stroke}%
\end{pgfscope}%
\begin{pgfscope}%
\pgfpathrectangle{\pgfqpoint{0.448215in}{1.689230in}}{\pgfqpoint{1.223684in}{1.004348in}}%
\pgfusepath{clip}%
\pgfsetbuttcap%
\pgfsetroundjoin%
\pgfsetlinewidth{1.003750pt}%
\definecolor{currentstroke}{rgb}{0.333333,0.333333,0.333333}%
\pgfsetstrokecolor{currentstroke}%
\pgfsetdash{}{0pt}%
\pgfpathmoveto{\pgfqpoint{0.634842in}{2.119665in}}%
\pgfpathlineto{\pgfqpoint{0.634842in}{2.099168in}}%
\pgfusepath{stroke}%
\end{pgfscope}%
\begin{pgfscope}%
\pgfpathrectangle{\pgfqpoint{0.448215in}{1.689230in}}{\pgfqpoint{1.223684in}{1.004348in}}%
\pgfusepath{clip}%
\pgfsetbuttcap%
\pgfsetroundjoin%
\pgfsetlinewidth{1.003750pt}%
\definecolor{currentstroke}{rgb}{0.333333,0.333333,0.333333}%
\pgfsetstrokecolor{currentstroke}%
\pgfsetdash{}{0pt}%
\pgfpathmoveto{\pgfqpoint{0.634372in}{2.201653in}}%
\pgfpathlineto{\pgfqpoint{0.634372in}{2.181156in}}%
\pgfusepath{stroke}%
\end{pgfscope}%
\begin{pgfscope}%
\pgfpathrectangle{\pgfqpoint{0.448215in}{1.689230in}}{\pgfqpoint{1.223684in}{1.004348in}}%
\pgfusepath{clip}%
\pgfsetbuttcap%
\pgfsetroundjoin%
\pgfsetlinewidth{1.003750pt}%
\definecolor{currentstroke}{rgb}{0.333333,0.333333,0.333333}%
\pgfsetstrokecolor{currentstroke}%
\pgfsetdash{}{0pt}%
\pgfpathmoveto{\pgfqpoint{0.809408in}{2.211901in}}%
\pgfpathlineto{\pgfqpoint{0.809408in}{2.191404in}}%
\pgfusepath{stroke}%
\end{pgfscope}%
\begin{pgfscope}%
\pgfpathrectangle{\pgfqpoint{0.448215in}{1.689230in}}{\pgfqpoint{1.223684in}{1.004348in}}%
\pgfusepath{clip}%
\pgfsetbuttcap%
\pgfsetroundjoin%
\pgfsetlinewidth{1.003750pt}%
\definecolor{currentstroke}{rgb}{0.333333,0.333333,0.333333}%
\pgfsetstrokecolor{currentstroke}%
\pgfsetdash{}{0pt}%
\pgfpathmoveto{\pgfqpoint{0.784347in}{2.222149in}}%
\pgfpathlineto{\pgfqpoint{0.784347in}{2.201653in}}%
\pgfusepath{stroke}%
\end{pgfscope}%
\begin{pgfscope}%
\pgfpathrectangle{\pgfqpoint{0.448215in}{1.689230in}}{\pgfqpoint{1.223684in}{1.004348in}}%
\pgfusepath{clip}%
\pgfsetbuttcap%
\pgfsetroundjoin%
\pgfsetlinewidth{1.003750pt}%
\definecolor{currentstroke}{rgb}{0.333333,0.333333,0.333333}%
\pgfsetstrokecolor{currentstroke}%
\pgfsetdash{}{0pt}%
\pgfpathmoveto{\pgfqpoint{0.559189in}{2.232398in}}%
\pgfpathlineto{\pgfqpoint{0.559189in}{2.211901in}}%
\pgfusepath{stroke}%
\end{pgfscope}%
\begin{pgfscope}%
\pgfpathrectangle{\pgfqpoint{0.448215in}{1.689230in}}{\pgfqpoint{1.223684in}{1.004348in}}%
\pgfusepath{clip}%
\pgfsetbuttcap%
\pgfsetroundjoin%
\pgfsetlinewidth{1.003750pt}%
\definecolor{currentstroke}{rgb}{0.333333,0.333333,0.333333}%
\pgfsetstrokecolor{currentstroke}%
\pgfsetdash{}{0pt}%
\pgfpathmoveto{\pgfqpoint{0.584876in}{2.263143in}}%
\pgfpathlineto{\pgfqpoint{0.584876in}{2.242646in}}%
\pgfusepath{stroke}%
\end{pgfscope}%
\begin{pgfscope}%
\pgfpathrectangle{\pgfqpoint{0.448215in}{1.689230in}}{\pgfqpoint{1.223684in}{1.004348in}}%
\pgfusepath{clip}%
\pgfsetbuttcap%
\pgfsetroundjoin%
\pgfsetlinewidth{1.003750pt}%
\definecolor{currentstroke}{rgb}{0.333333,0.333333,0.333333}%
\pgfsetstrokecolor{currentstroke}%
\pgfsetdash{}{0pt}%
\pgfpathmoveto{\pgfqpoint{0.534441in}{2.273392in}}%
\pgfpathlineto{\pgfqpoint{0.534441in}{2.252895in}}%
\pgfusepath{stroke}%
\end{pgfscope}%
\begin{pgfscope}%
\pgfpathrectangle{\pgfqpoint{0.448215in}{1.689230in}}{\pgfqpoint{1.223684in}{1.004348in}}%
\pgfusepath{clip}%
\pgfsetbuttcap%
\pgfsetroundjoin%
\pgfsetlinewidth{1.003750pt}%
\definecolor{currentstroke}{rgb}{0.333333,0.333333,0.333333}%
\pgfsetstrokecolor{currentstroke}%
\pgfsetdash{}{0pt}%
\pgfpathmoveto{\pgfqpoint{0.634137in}{2.283640in}}%
\pgfpathlineto{\pgfqpoint{0.634137in}{2.263143in}}%
\pgfusepath{stroke}%
\end{pgfscope}%
\begin{pgfscope}%
\pgfpathrectangle{\pgfqpoint{0.448215in}{1.689230in}}{\pgfqpoint{1.223684in}{1.004348in}}%
\pgfusepath{clip}%
\pgfsetbuttcap%
\pgfsetroundjoin%
\pgfsetlinewidth{1.003750pt}%
\definecolor{currentstroke}{rgb}{0.333333,0.333333,0.333333}%
\pgfsetstrokecolor{currentstroke}%
\pgfsetdash{}{0pt}%
\pgfpathmoveto{\pgfqpoint{0.684729in}{2.293889in}}%
\pgfpathlineto{\pgfqpoint{0.684729in}{2.273392in}}%
\pgfusepath{stroke}%
\end{pgfscope}%
\begin{pgfscope}%
\pgfpathrectangle{\pgfqpoint{0.448215in}{1.689230in}}{\pgfqpoint{1.223684in}{1.004348in}}%
\pgfusepath{clip}%
\pgfsetbuttcap%
\pgfsetroundjoin%
\pgfsetlinewidth{1.003750pt}%
\definecolor{currentstroke}{rgb}{0.333333,0.333333,0.333333}%
\pgfsetstrokecolor{currentstroke}%
\pgfsetdash{}{0pt}%
\pgfpathmoveto{\pgfqpoint{0.709163in}{2.304137in}}%
\pgfpathlineto{\pgfqpoint{0.709163in}{2.283640in}}%
\pgfusepath{stroke}%
\end{pgfscope}%
\begin{pgfscope}%
\pgfpathrectangle{\pgfqpoint{0.448215in}{1.689230in}}{\pgfqpoint{1.223684in}{1.004348in}}%
\pgfusepath{clip}%
\pgfsetbuttcap%
\pgfsetroundjoin%
\pgfsetlinewidth{1.003750pt}%
\definecolor{currentstroke}{rgb}{0.333333,0.333333,0.333333}%
\pgfsetstrokecolor{currentstroke}%
\pgfsetdash{}{0pt}%
\pgfpathmoveto{\pgfqpoint{0.834077in}{2.314386in}}%
\pgfpathlineto{\pgfqpoint{0.834077in}{2.293889in}}%
\pgfusepath{stroke}%
\end{pgfscope}%
\begin{pgfscope}%
\pgfpathrectangle{\pgfqpoint{0.448215in}{1.689230in}}{\pgfqpoint{1.223684in}{1.004348in}}%
\pgfusepath{clip}%
\pgfsetbuttcap%
\pgfsetroundjoin%
\pgfsetlinewidth{1.003750pt}%
\definecolor{currentstroke}{rgb}{0.333333,0.333333,0.333333}%
\pgfsetstrokecolor{currentstroke}%
\pgfsetdash{}{0pt}%
\pgfpathmoveto{\pgfqpoint{0.784425in}{2.345131in}}%
\pgfpathlineto{\pgfqpoint{0.784425in}{2.324634in}}%
\pgfusepath{stroke}%
\end{pgfscope}%
\begin{pgfscope}%
\pgfpathrectangle{\pgfqpoint{0.448215in}{1.689230in}}{\pgfqpoint{1.223684in}{1.004348in}}%
\pgfusepath{clip}%
\pgfsetbuttcap%
\pgfsetroundjoin%
\pgfsetlinewidth{1.003750pt}%
\definecolor{currentstroke}{rgb}{0.333333,0.333333,0.333333}%
\pgfsetstrokecolor{currentstroke}%
\pgfsetdash{}{0pt}%
\pgfpathmoveto{\pgfqpoint{0.559815in}{2.365628in}}%
\pgfpathlineto{\pgfqpoint{0.559815in}{2.345131in}}%
\pgfusepath{stroke}%
\end{pgfscope}%
\begin{pgfscope}%
\pgfpathrectangle{\pgfqpoint{0.448215in}{1.689230in}}{\pgfqpoint{1.223684in}{1.004348in}}%
\pgfusepath{clip}%
\pgfsetbuttcap%
\pgfsetroundjoin%
\pgfsetlinewidth{1.003750pt}%
\definecolor{currentstroke}{rgb}{0.333333,0.333333,0.333333}%
\pgfsetstrokecolor{currentstroke}%
\pgfsetdash{}{0pt}%
\pgfpathmoveto{\pgfqpoint{0.684102in}{2.386125in}}%
\pgfpathlineto{\pgfqpoint{0.684102in}{2.365628in}}%
\pgfusepath{stroke}%
\end{pgfscope}%
\begin{pgfscope}%
\pgfpathrectangle{\pgfqpoint{0.448215in}{1.689230in}}{\pgfqpoint{1.223684in}{1.004348in}}%
\pgfusepath{clip}%
\pgfsetbuttcap%
\pgfsetroundjoin%
\pgfsetlinewidth{1.003750pt}%
\definecolor{currentstroke}{rgb}{0.333333,0.333333,0.333333}%
\pgfsetstrokecolor{currentstroke}%
\pgfsetdash{}{0pt}%
\pgfpathmoveto{\pgfqpoint{0.584406in}{2.406622in}}%
\pgfpathlineto{\pgfqpoint{0.584406in}{2.386125in}}%
\pgfusepath{stroke}%
\end{pgfscope}%
\begin{pgfscope}%
\pgfpathrectangle{\pgfqpoint{0.448215in}{1.689230in}}{\pgfqpoint{1.223684in}{1.004348in}}%
\pgfusepath{clip}%
\pgfsetbuttcap%
\pgfsetroundjoin%
\pgfsetlinewidth{1.003750pt}%
\definecolor{currentstroke}{rgb}{0.333333,0.333333,0.333333}%
\pgfsetstrokecolor{currentstroke}%
\pgfsetdash{}{0pt}%
\pgfpathmoveto{\pgfqpoint{0.559502in}{2.427118in}}%
\pgfpathlineto{\pgfqpoint{0.559502in}{2.406622in}}%
\pgfusepath{stroke}%
\end{pgfscope}%
\begin{pgfscope}%
\pgfpathrectangle{\pgfqpoint{0.448215in}{1.689230in}}{\pgfqpoint{1.223684in}{1.004348in}}%
\pgfusepath{clip}%
\pgfsetbuttcap%
\pgfsetroundjoin%
\pgfsetlinewidth{1.003750pt}%
\definecolor{currentstroke}{rgb}{0.333333,0.333333,0.333333}%
\pgfsetstrokecolor{currentstroke}%
\pgfsetdash{}{0pt}%
\pgfpathmoveto{\pgfqpoint{0.784112in}{2.437367in}}%
\pgfpathlineto{\pgfqpoint{0.784112in}{2.416870in}}%
\pgfusepath{stroke}%
\end{pgfscope}%
\begin{pgfscope}%
\pgfpathrectangle{\pgfqpoint{0.448215in}{1.689230in}}{\pgfqpoint{1.223684in}{1.004348in}}%
\pgfusepath{clip}%
\pgfsetbuttcap%
\pgfsetroundjoin%
\pgfsetlinewidth{1.003750pt}%
\definecolor{currentstroke}{rgb}{0.333333,0.333333,0.333333}%
\pgfsetstrokecolor{currentstroke}%
\pgfsetdash{}{0pt}%
\pgfpathmoveto{\pgfqpoint{0.834312in}{2.447615in}}%
\pgfpathlineto{\pgfqpoint{0.834312in}{2.427118in}}%
\pgfusepath{stroke}%
\end{pgfscope}%
\begin{pgfscope}%
\pgfpathrectangle{\pgfqpoint{0.448215in}{1.689230in}}{\pgfqpoint{1.223684in}{1.004348in}}%
\pgfusepath{clip}%
\pgfsetbuttcap%
\pgfsetroundjoin%
\pgfsetlinewidth{1.003750pt}%
\definecolor{currentstroke}{rgb}{0.333333,0.333333,0.333333}%
\pgfsetstrokecolor{currentstroke}%
\pgfsetdash{}{0pt}%
\pgfpathmoveto{\pgfqpoint{0.659120in}{2.488609in}}%
\pgfpathlineto{\pgfqpoint{0.659120in}{2.468112in}}%
\pgfusepath{stroke}%
\end{pgfscope}%
\begin{pgfscope}%
\pgfpathrectangle{\pgfqpoint{0.448215in}{1.689230in}}{\pgfqpoint{1.223684in}{1.004348in}}%
\pgfusepath{clip}%
\pgfsetbuttcap%
\pgfsetroundjoin%
\pgfsetlinewidth{1.003750pt}%
\definecolor{currentstroke}{rgb}{0.333333,0.333333,0.333333}%
\pgfsetstrokecolor{currentstroke}%
\pgfsetdash{}{0pt}%
\pgfpathmoveto{\pgfqpoint{0.684416in}{2.498858in}}%
\pgfpathlineto{\pgfqpoint{0.684416in}{2.478361in}}%
\pgfusepath{stroke}%
\end{pgfscope}%
\begin{pgfscope}%
\pgfpathrectangle{\pgfqpoint{0.448215in}{1.689230in}}{\pgfqpoint{1.223684in}{1.004348in}}%
\pgfusepath{clip}%
\pgfsetbuttcap%
\pgfsetroundjoin%
\pgfsetlinewidth{1.003750pt}%
\definecolor{currentstroke}{rgb}{0.333333,0.333333,0.333333}%
\pgfsetstrokecolor{currentstroke}%
\pgfsetdash{}{0pt}%
\pgfpathmoveto{\pgfqpoint{0.684337in}{2.519354in}}%
\pgfpathlineto{\pgfqpoint{0.684337in}{2.498858in}}%
\pgfusepath{stroke}%
\end{pgfscope}%
\begin{pgfscope}%
\pgfpathrectangle{\pgfqpoint{0.448215in}{1.689230in}}{\pgfqpoint{1.223684in}{1.004348in}}%
\pgfusepath{clip}%
\pgfsetbuttcap%
\pgfsetroundjoin%
\pgfsetlinewidth{1.003750pt}%
\definecolor{currentstroke}{rgb}{0.333333,0.333333,0.333333}%
\pgfsetstrokecolor{currentstroke}%
\pgfsetdash{}{0pt}%
\pgfpathmoveto{\pgfqpoint{0.759364in}{2.529603in}}%
\pgfpathlineto{\pgfqpoint{0.759364in}{2.509106in}}%
\pgfusepath{stroke}%
\end{pgfscope}%
\begin{pgfscope}%
\pgfpathrectangle{\pgfqpoint{0.448215in}{1.689230in}}{\pgfqpoint{1.223684in}{1.004348in}}%
\pgfusepath{clip}%
\pgfsetbuttcap%
\pgfsetroundjoin%
\pgfsetlinewidth{1.003750pt}%
\definecolor{currentstroke}{rgb}{0.333333,0.333333,0.333333}%
\pgfsetstrokecolor{currentstroke}%
\pgfsetdash{}{0pt}%
\pgfpathmoveto{\pgfqpoint{0.584171in}{2.570597in}}%
\pgfpathlineto{\pgfqpoint{0.584171in}{2.550100in}}%
\pgfusepath{stroke}%
\end{pgfscope}%
\begin{pgfscope}%
\pgfpathrectangle{\pgfqpoint{0.448215in}{1.689230in}}{\pgfqpoint{1.223684in}{1.004348in}}%
\pgfusepath{clip}%
\pgfsetbuttcap%
\pgfsetroundjoin%
\pgfsetlinewidth{1.003750pt}%
\definecolor{currentstroke}{rgb}{0.333333,0.333333,0.333333}%
\pgfsetstrokecolor{currentstroke}%
\pgfsetdash{}{0pt}%
\pgfpathmoveto{\pgfqpoint{0.834390in}{2.580845in}}%
\pgfpathlineto{\pgfqpoint{0.834390in}{2.560348in}}%
\pgfusepath{stroke}%
\end{pgfscope}%
\begin{pgfscope}%
\pgfpathrectangle{\pgfqpoint{0.448215in}{1.689230in}}{\pgfqpoint{1.223684in}{1.004348in}}%
\pgfusepath{clip}%
\pgfsetbuttcap%
\pgfsetroundjoin%
\pgfsetlinewidth{1.003750pt}%
\definecolor{currentstroke}{rgb}{0.333333,0.333333,0.333333}%
\pgfsetstrokecolor{currentstroke}%
\pgfsetdash{}{0pt}%
\pgfpathmoveto{\pgfqpoint{0.734146in}{2.632087in}}%
\pgfpathlineto{\pgfqpoint{0.734146in}{2.611590in}}%
\pgfusepath{stroke}%
\end{pgfscope}%
\begin{pgfscope}%
\pgfsetrectcap%
\pgfsetmiterjoin%
\pgfsetlinewidth{0.501875pt}%
\definecolor{currentstroke}{rgb}{0.317647,0.317647,0.317647}%
\pgfsetstrokecolor{currentstroke}%
\pgfsetdash{}{0pt}%
\pgfpathmoveto{\pgfqpoint{0.448215in}{1.689230in}}%
\pgfpathlineto{\pgfqpoint{0.448215in}{2.693578in}}%
\pgfusepath{stroke}%
\end{pgfscope}%
\begin{pgfscope}%
\pgfsetrectcap%
\pgfsetmiterjoin%
\pgfsetlinewidth{0.501875pt}%
\definecolor{currentstroke}{rgb}{0.317647,0.317647,0.317647}%
\pgfsetstrokecolor{currentstroke}%
\pgfsetdash{}{0pt}%
\pgfpathmoveto{\pgfqpoint{0.448215in}{1.689230in}}%
\pgfpathlineto{\pgfqpoint{1.671899in}{1.689230in}}%
\pgfusepath{stroke}%
\end{pgfscope}%
\begin{pgfscope}%
\definecolor{textcolor}{rgb}{0.000000,0.000000,0.000000}%
\pgfsetstrokecolor{textcolor}%
\pgfsetfillcolor{textcolor}%
\pgftext[x=1.060057in,y=2.776911in,,base]{\color{textcolor}\rmfamily\fontsize{6.664000}{7.996800}\selectfont Presynaptic Spikes}%
\end{pgfscope}%
\begin{pgfscope}%
\pgfsetbuttcap%
\pgfsetmiterjoin%
\pgfsetlinewidth{0.000000pt}%
\definecolor{currentstroke}{rgb}{0.000000,0.000000,0.000000}%
\pgfsetstrokecolor{currentstroke}%
\pgfsetstrokeopacity{0.000000}%
\pgfsetdash{}{0pt}%
\pgfpathmoveto{\pgfqpoint{2.161373in}{1.689230in}}%
\pgfpathlineto{\pgfqpoint{3.385057in}{1.689230in}}%
\pgfpathlineto{\pgfqpoint{3.385057in}{2.693578in}}%
\pgfpathlineto{\pgfqpoint{2.161373in}{2.693578in}}%
\pgfpathclose%
\pgfusepath{}%
\end{pgfscope}%
\begin{pgfscope}%
\pgfsetbuttcap%
\pgfsetroundjoin%
\definecolor{currentfill}{rgb}{0.317647,0.317647,0.317647}%
\pgfsetfillcolor{currentfill}%
\pgfsetlinewidth{0.501875pt}%
\definecolor{currentstroke}{rgb}{0.317647,0.317647,0.317647}%
\pgfsetstrokecolor{currentstroke}%
\pgfsetdash{}{0pt}%
\pgfsys@defobject{currentmarker}{\pgfqpoint{0.000000in}{-0.020833in}}{\pgfqpoint{0.000000in}{0.000000in}}{%
\pgfpathmoveto{\pgfqpoint{0.000000in}{0.000000in}}%
\pgfpathlineto{\pgfqpoint{0.000000in}{-0.020833in}}%
\pgfusepath{stroke,fill}%
}%
\begin{pgfscope}%
\pgfsys@transformshift{2.210320in}{1.689230in}%
\pgfsys@useobject{currentmarker}{}%
\end{pgfscope}%
\end{pgfscope}%
\begin{pgfscope}%
\definecolor{textcolor}{rgb}{0.317647,0.317647,0.317647}%
\pgfsetstrokecolor{textcolor}%
\pgfsetfillcolor{textcolor}%
\pgftext[x=2.210320in,y=1.640619in,,top]{\color{textcolor}\rmfamily\fontsize{6.664000}{7.996800}\selectfont \(\displaystyle 0\)}%
\end{pgfscope}%
\begin{pgfscope}%
\pgfsetbuttcap%
\pgfsetroundjoin%
\definecolor{currentfill}{rgb}{0.317647,0.317647,0.317647}%
\pgfsetfillcolor{currentfill}%
\pgfsetlinewidth{0.501875pt}%
\definecolor{currentstroke}{rgb}{0.317647,0.317647,0.317647}%
\pgfsetstrokecolor{currentstroke}%
\pgfsetdash{}{0pt}%
\pgfsys@defobject{currentmarker}{\pgfqpoint{0.000000in}{-0.020833in}}{\pgfqpoint{0.000000in}{0.000000in}}{%
\pgfpathmoveto{\pgfqpoint{0.000000in}{0.000000in}}%
\pgfpathlineto{\pgfqpoint{0.000000in}{-0.020833in}}%
\pgfusepath{stroke,fill}%
}%
\begin{pgfscope}%
\pgfsys@transformshift{2.699794in}{1.689230in}%
\pgfsys@useobject{currentmarker}{}%
\end{pgfscope}%
\end{pgfscope}%
\begin{pgfscope}%
\definecolor{textcolor}{rgb}{0.317647,0.317647,0.317647}%
\pgfsetstrokecolor{textcolor}%
\pgfsetfillcolor{textcolor}%
\pgftext[x=2.699794in,y=1.640619in,,top]{\color{textcolor}\rmfamily\fontsize{6.664000}{7.996800}\selectfont \(\displaystyle 50\)}%
\end{pgfscope}%
\begin{pgfscope}%
\pgfsetbuttcap%
\pgfsetroundjoin%
\definecolor{currentfill}{rgb}{0.317647,0.317647,0.317647}%
\pgfsetfillcolor{currentfill}%
\pgfsetlinewidth{0.501875pt}%
\definecolor{currentstroke}{rgb}{0.317647,0.317647,0.317647}%
\pgfsetstrokecolor{currentstroke}%
\pgfsetdash{}{0pt}%
\pgfsys@defobject{currentmarker}{\pgfqpoint{0.000000in}{-0.020833in}}{\pgfqpoint{0.000000in}{0.000000in}}{%
\pgfpathmoveto{\pgfqpoint{0.000000in}{0.000000in}}%
\pgfpathlineto{\pgfqpoint{0.000000in}{-0.020833in}}%
\pgfusepath{stroke,fill}%
}%
\begin{pgfscope}%
\pgfsys@transformshift{3.189268in}{1.689230in}%
\pgfsys@useobject{currentmarker}{}%
\end{pgfscope}%
\end{pgfscope}%
\begin{pgfscope}%
\definecolor{textcolor}{rgb}{0.317647,0.317647,0.317647}%
\pgfsetstrokecolor{textcolor}%
\pgfsetfillcolor{textcolor}%
\pgftext[x=3.189268in,y=1.640619in,,top]{\color{textcolor}\rmfamily\fontsize{6.664000}{7.996800}\selectfont \(\displaystyle 100\)}%
\end{pgfscope}%
\begin{pgfscope}%
\pgfsetbuttcap%
\pgfsetroundjoin%
\definecolor{currentfill}{rgb}{0.317647,0.317647,0.317647}%
\pgfsetfillcolor{currentfill}%
\pgfsetlinewidth{0.501875pt}%
\definecolor{currentstroke}{rgb}{0.317647,0.317647,0.317647}%
\pgfsetstrokecolor{currentstroke}%
\pgfsetdash{}{0pt}%
\pgfsys@defobject{currentmarker}{\pgfqpoint{-0.020833in}{0.000000in}}{\pgfqpoint{0.000000in}{0.000000in}}{%
\pgfpathmoveto{\pgfqpoint{0.000000in}{0.000000in}}%
\pgfpathlineto{\pgfqpoint{-0.020833in}{0.000000in}}%
\pgfusepath{stroke,fill}%
}%
\begin{pgfscope}%
\pgfsys@transformshift{2.161373in}{1.785891in}%
\pgfsys@useobject{currentmarker}{}%
\end{pgfscope}%
\end{pgfscope}%
\begin{pgfscope}%
\definecolor{textcolor}{rgb}{0.317647,0.317647,0.317647}%
\pgfsetstrokecolor{textcolor}%
\pgfsetfillcolor{textcolor}%
\pgftext[x=1.982939in,y=1.753774in,left,base]{\color{textcolor}\rmfamily\fontsize{6.664000}{7.996800}\selectfont \(\displaystyle 0.2\)}%
\end{pgfscope}%
\begin{pgfscope}%
\pgfsetbuttcap%
\pgfsetroundjoin%
\definecolor{currentfill}{rgb}{0.317647,0.317647,0.317647}%
\pgfsetfillcolor{currentfill}%
\pgfsetlinewidth{0.501875pt}%
\definecolor{currentstroke}{rgb}{0.317647,0.317647,0.317647}%
\pgfsetstrokecolor{currentstroke}%
\pgfsetdash{}{0pt}%
\pgfsys@defobject{currentmarker}{\pgfqpoint{-0.020833in}{0.000000in}}{\pgfqpoint{0.000000in}{0.000000in}}{%
\pgfpathmoveto{\pgfqpoint{0.000000in}{0.000000in}}%
\pgfpathlineto{\pgfqpoint{-0.020833in}{0.000000in}}%
\pgfusepath{stroke,fill}%
}%
\begin{pgfscope}%
\pgfsys@transformshift{2.161373in}{2.082597in}%
\pgfsys@useobject{currentmarker}{}%
\end{pgfscope}%
\end{pgfscope}%
\begin{pgfscope}%
\definecolor{textcolor}{rgb}{0.317647,0.317647,0.317647}%
\pgfsetstrokecolor{textcolor}%
\pgfsetfillcolor{textcolor}%
\pgftext[x=1.982939in,y=2.050480in,left,base]{\color{textcolor}\rmfamily\fontsize{6.664000}{7.996800}\selectfont \(\displaystyle 0.4\)}%
\end{pgfscope}%
\begin{pgfscope}%
\pgfsetbuttcap%
\pgfsetroundjoin%
\definecolor{currentfill}{rgb}{0.317647,0.317647,0.317647}%
\pgfsetfillcolor{currentfill}%
\pgfsetlinewidth{0.501875pt}%
\definecolor{currentstroke}{rgb}{0.317647,0.317647,0.317647}%
\pgfsetstrokecolor{currentstroke}%
\pgfsetdash{}{0pt}%
\pgfsys@defobject{currentmarker}{\pgfqpoint{-0.020833in}{0.000000in}}{\pgfqpoint{0.000000in}{0.000000in}}{%
\pgfpathmoveto{\pgfqpoint{0.000000in}{0.000000in}}%
\pgfpathlineto{\pgfqpoint{-0.020833in}{0.000000in}}%
\pgfusepath{stroke,fill}%
}%
\begin{pgfscope}%
\pgfsys@transformshift{2.161373in}{2.379303in}%
\pgfsys@useobject{currentmarker}{}%
\end{pgfscope}%
\end{pgfscope}%
\begin{pgfscope}%
\definecolor{textcolor}{rgb}{0.317647,0.317647,0.317647}%
\pgfsetstrokecolor{textcolor}%
\pgfsetfillcolor{textcolor}%
\pgftext[x=1.982939in,y=2.347186in,left,base]{\color{textcolor}\rmfamily\fontsize{6.664000}{7.996800}\selectfont \(\displaystyle 0.6\)}%
\end{pgfscope}%
\begin{pgfscope}%
\pgfsetbuttcap%
\pgfsetroundjoin%
\definecolor{currentfill}{rgb}{0.317647,0.317647,0.317647}%
\pgfsetfillcolor{currentfill}%
\pgfsetlinewidth{0.501875pt}%
\definecolor{currentstroke}{rgb}{0.317647,0.317647,0.317647}%
\pgfsetstrokecolor{currentstroke}%
\pgfsetdash{}{0pt}%
\pgfsys@defobject{currentmarker}{\pgfqpoint{-0.020833in}{0.000000in}}{\pgfqpoint{0.000000in}{0.000000in}}{%
\pgfpathmoveto{\pgfqpoint{0.000000in}{0.000000in}}%
\pgfpathlineto{\pgfqpoint{-0.020833in}{0.000000in}}%
\pgfusepath{stroke,fill}%
}%
\begin{pgfscope}%
\pgfsys@transformshift{2.161373in}{2.676009in}%
\pgfsys@useobject{currentmarker}{}%
\end{pgfscope}%
\end{pgfscope}%
\begin{pgfscope}%
\definecolor{textcolor}{rgb}{0.317647,0.317647,0.317647}%
\pgfsetstrokecolor{textcolor}%
\pgfsetfillcolor{textcolor}%
\pgftext[x=1.982939in,y=2.643892in,left,base]{\color{textcolor}\rmfamily\fontsize{6.664000}{7.996800}\selectfont \(\displaystyle 0.8\)}%
\end{pgfscope}%
\begin{pgfscope}%
\definecolor{textcolor}{rgb}{0.317647,0.317647,0.317647}%
\pgfsetstrokecolor{textcolor}%
\pgfsetfillcolor{textcolor}%
\pgftext[x=1.927383in,y=2.191404in,,bottom,rotate=90.000000]{\color{textcolor}\rmfamily\fontsize{6.664000}{7.996800}\selectfont \(\displaystyle V_{\mathrm{m}}^{(h)} \; (\si{\V})\)}%
\end{pgfscope}%
\begin{pgfscope}%
\pgfpathrectangle{\pgfqpoint{2.161373in}{1.689230in}}{\pgfqpoint{1.223684in}{1.004348in}}%
\pgfusepath{clip}%
\pgfsetrectcap%
\pgfsetroundjoin%
\pgfsetlinewidth{0.803000pt}%
\definecolor{currentstroke}{rgb}{0.333333,0.333333,0.333333}%
\pgfsetstrokecolor{currentstroke}%
\pgfsetdash{}{0pt}%
\pgfpathmoveto{\pgfqpoint{2.210320in}{2.465317in}}%
\pgfpathlineto{\pgfqpoint{2.235312in}{2.465317in}}%
\pgfpathlineto{\pgfqpoint{2.260304in}{2.497926in}}%
\pgfpathlineto{\pgfqpoint{2.285296in}{2.471839in}}%
\pgfpathlineto{\pgfqpoint{2.310288in}{2.432708in}}%
\pgfpathlineto{\pgfqpoint{2.335280in}{2.471839in}}%
\pgfpathlineto{\pgfqpoint{2.360271in}{2.426187in}}%
\pgfpathlineto{\pgfqpoint{2.385263in}{2.387056in}}%
\pgfpathlineto{\pgfqpoint{2.410255in}{2.360969in}}%
\pgfpathlineto{\pgfqpoint{2.435247in}{2.360969in}}%
\pgfpathlineto{\pgfqpoint{2.460239in}{2.367491in}}%
\pgfpathlineto{\pgfqpoint{2.485231in}{2.380535in}}%
\pgfpathlineto{\pgfqpoint{2.510222in}{2.347926in}}%
\pgfpathlineto{\pgfqpoint{2.535214in}{2.315317in}}%
\pgfpathlineto{\pgfqpoint{2.560206in}{2.289230in}}%
\pgfpathlineto{\pgfqpoint{2.585198in}{2.250100in}}%
\pgfpathlineto{\pgfqpoint{2.610190in}{2.250100in}}%
\pgfpathlineto{\pgfqpoint{2.635182in}{2.263143in}}%
\pgfpathlineto{\pgfqpoint{2.660173in}{2.289230in}}%
\pgfpathlineto{\pgfqpoint{2.685165in}{2.328361in}}%
\pgfpathlineto{\pgfqpoint{2.710157in}{2.354448in}}%
\pgfpathlineto{\pgfqpoint{2.735149in}{2.380535in}}%
\pgfpathlineto{\pgfqpoint{2.760141in}{2.400100in}}%
\pgfpathlineto{\pgfqpoint{2.785133in}{2.406622in}}%
\pgfpathlineto{\pgfqpoint{2.810124in}{2.419665in}}%
\pgfpathlineto{\pgfqpoint{2.835116in}{2.426187in}}%
\pgfpathlineto{\pgfqpoint{2.860108in}{2.432708in}}%
\pgfpathlineto{\pgfqpoint{2.885100in}{2.439230in}}%
\pgfpathlineto{\pgfqpoint{2.910092in}{2.439230in}}%
\pgfpathlineto{\pgfqpoint{2.935084in}{2.439230in}}%
\pgfpathlineto{\pgfqpoint{2.960075in}{2.452274in}}%
\pgfpathlineto{\pgfqpoint{2.985067in}{2.458795in}}%
\pgfpathlineto{\pgfqpoint{3.010059in}{2.458795in}}%
\pgfpathlineto{\pgfqpoint{3.035051in}{2.465317in}}%
\pgfpathlineto{\pgfqpoint{3.060043in}{2.465317in}}%
\pgfpathlineto{\pgfqpoint{3.085035in}{2.452274in}}%
\pgfpathlineto{\pgfqpoint{3.110026in}{2.458795in}}%
\pgfpathlineto{\pgfqpoint{3.135018in}{2.458795in}}%
\pgfpathlineto{\pgfqpoint{3.160010in}{2.458795in}}%
\pgfpathlineto{\pgfqpoint{3.185002in}{2.452274in}}%
\pgfpathlineto{\pgfqpoint{3.209994in}{2.452274in}}%
\pgfpathlineto{\pgfqpoint{3.234986in}{2.458795in}}%
\pgfpathlineto{\pgfqpoint{3.259977in}{2.465317in}}%
\pgfpathlineto{\pgfqpoint{3.284969in}{2.458795in}}%
\pgfpathlineto{\pgfqpoint{3.309961in}{2.458795in}}%
\pgfpathlineto{\pgfqpoint{3.334953in}{2.458795in}}%
\pgfpathlineto{\pgfqpoint{3.359945in}{2.458795in}}%
\pgfpathlineto{\pgfqpoint{3.384937in}{2.458795in}}%
\pgfpathlineto{\pgfqpoint{3.395057in}{2.461436in}}%
\pgfusepath{stroke}%
\end{pgfscope}%
\begin{pgfscope}%
\pgfpathrectangle{\pgfqpoint{2.161373in}{1.689230in}}{\pgfqpoint{1.223684in}{1.004348in}}%
\pgfusepath{clip}%
\pgfsetrectcap%
\pgfsetroundjoin%
\pgfsetlinewidth{0.803000pt}%
\definecolor{currentstroke}{rgb}{0.686275,0.352941,0.313725}%
\pgfsetstrokecolor{currentstroke}%
\pgfsetdash{}{0pt}%
\pgfpathmoveto{\pgfqpoint{2.210320in}{2.132708in}}%
\pgfpathlineto{\pgfqpoint{2.235312in}{2.139230in}}%
\pgfpathlineto{\pgfqpoint{2.260304in}{2.197926in}}%
\pgfpathlineto{\pgfqpoint{2.285296in}{2.197926in}}%
\pgfpathlineto{\pgfqpoint{2.310288in}{2.263143in}}%
\pgfpathlineto{\pgfqpoint{2.335280in}{2.367491in}}%
\pgfpathlineto{\pgfqpoint{2.360271in}{2.497926in}}%
\pgfpathlineto{\pgfqpoint{2.385263in}{2.641404in}}%
\pgfpathlineto{\pgfqpoint{2.410255in}{2.008795in}}%
\pgfpathlineto{\pgfqpoint{2.435247in}{2.008795in}}%
\pgfpathlineto{\pgfqpoint{2.460239in}{2.002274in}}%
\pgfpathlineto{\pgfqpoint{2.485231in}{2.002274in}}%
\pgfpathlineto{\pgfqpoint{2.510222in}{1.995752in}}%
\pgfpathlineto{\pgfqpoint{2.535214in}{2.002274in}}%
\pgfpathlineto{\pgfqpoint{2.560206in}{1.995752in}}%
\pgfpathlineto{\pgfqpoint{2.585198in}{1.995752in}}%
\pgfpathlineto{\pgfqpoint{2.610190in}{1.995752in}}%
\pgfpathlineto{\pgfqpoint{2.635182in}{1.989230in}}%
\pgfpathlineto{\pgfqpoint{2.660173in}{1.989230in}}%
\pgfpathlineto{\pgfqpoint{2.685165in}{1.989230in}}%
\pgfpathlineto{\pgfqpoint{2.710157in}{1.982708in}}%
\pgfpathlineto{\pgfqpoint{2.735149in}{2.041404in}}%
\pgfpathlineto{\pgfqpoint{2.760141in}{2.080535in}}%
\pgfpathlineto{\pgfqpoint{2.785133in}{2.100100in}}%
\pgfpathlineto{\pgfqpoint{2.810124in}{2.119665in}}%
\pgfpathlineto{\pgfqpoint{2.835116in}{2.132708in}}%
\pgfpathlineto{\pgfqpoint{2.860108in}{2.132708in}}%
\pgfpathlineto{\pgfqpoint{2.885100in}{2.139230in}}%
\pgfpathlineto{\pgfqpoint{2.910092in}{2.139230in}}%
\pgfpathlineto{\pgfqpoint{2.935084in}{2.139230in}}%
\pgfpathlineto{\pgfqpoint{2.960075in}{2.139230in}}%
\pgfpathlineto{\pgfqpoint{2.985067in}{2.139230in}}%
\pgfpathlineto{\pgfqpoint{3.010059in}{2.139230in}}%
\pgfpathlineto{\pgfqpoint{3.035051in}{2.139230in}}%
\pgfpathlineto{\pgfqpoint{3.060043in}{2.139230in}}%
\pgfpathlineto{\pgfqpoint{3.085035in}{2.145752in}}%
\pgfpathlineto{\pgfqpoint{3.110026in}{2.145752in}}%
\pgfpathlineto{\pgfqpoint{3.135018in}{2.145752in}}%
\pgfpathlineto{\pgfqpoint{3.160010in}{2.145752in}}%
\pgfpathlineto{\pgfqpoint{3.185002in}{2.145752in}}%
\pgfpathlineto{\pgfqpoint{3.209994in}{2.152274in}}%
\pgfpathlineto{\pgfqpoint{3.234986in}{2.152274in}}%
\pgfpathlineto{\pgfqpoint{3.259977in}{2.152274in}}%
\pgfpathlineto{\pgfqpoint{3.284969in}{2.152274in}}%
\pgfpathlineto{\pgfqpoint{3.309961in}{2.145752in}}%
\pgfpathlineto{\pgfqpoint{3.334953in}{2.152274in}}%
\pgfpathlineto{\pgfqpoint{3.359945in}{2.152274in}}%
\pgfpathlineto{\pgfqpoint{3.384937in}{2.139230in}}%
\pgfpathlineto{\pgfqpoint{3.395057in}{2.141871in}}%
\pgfusepath{stroke}%
\end{pgfscope}%
\begin{pgfscope}%
\pgfpathrectangle{\pgfqpoint{2.161373in}{1.689230in}}{\pgfqpoint{1.223684in}{1.004348in}}%
\pgfusepath{clip}%
\pgfsetrectcap%
\pgfsetroundjoin%
\pgfsetlinewidth{0.803000pt}%
\definecolor{currentstroke}{rgb}{0.000000,0.356863,0.509804}%
\pgfsetstrokecolor{currentstroke}%
\pgfsetdash{}{0pt}%
\pgfpathmoveto{\pgfqpoint{2.210320in}{2.321839in}}%
\pgfpathlineto{\pgfqpoint{2.235312in}{2.321839in}}%
\pgfpathlineto{\pgfqpoint{2.260304in}{2.347926in}}%
\pgfpathlineto{\pgfqpoint{2.285296in}{2.341404in}}%
\pgfpathlineto{\pgfqpoint{2.310288in}{2.380535in}}%
\pgfpathlineto{\pgfqpoint{2.335280in}{2.393578in}}%
\pgfpathlineto{\pgfqpoint{2.360271in}{2.360969in}}%
\pgfpathlineto{\pgfqpoint{2.385263in}{2.367491in}}%
\pgfpathlineto{\pgfqpoint{2.410255in}{2.354448in}}%
\pgfpathlineto{\pgfqpoint{2.435247in}{2.341404in}}%
\pgfpathlineto{\pgfqpoint{2.460239in}{2.360969in}}%
\pgfpathlineto{\pgfqpoint{2.485231in}{2.354448in}}%
\pgfpathlineto{\pgfqpoint{2.510222in}{2.354448in}}%
\pgfpathlineto{\pgfqpoint{2.535214in}{2.354448in}}%
\pgfpathlineto{\pgfqpoint{2.560206in}{2.367491in}}%
\pgfpathlineto{\pgfqpoint{2.585198in}{2.374013in}}%
\pgfpathlineto{\pgfqpoint{2.610190in}{2.374013in}}%
\pgfpathlineto{\pgfqpoint{2.635182in}{2.367491in}}%
\pgfpathlineto{\pgfqpoint{2.660173in}{2.354448in}}%
\pgfpathlineto{\pgfqpoint{2.685165in}{2.347926in}}%
\pgfpathlineto{\pgfqpoint{2.710157in}{2.341404in}}%
\pgfpathlineto{\pgfqpoint{2.735149in}{2.341404in}}%
\pgfpathlineto{\pgfqpoint{2.760141in}{2.334882in}}%
\pgfpathlineto{\pgfqpoint{2.785133in}{2.334882in}}%
\pgfpathlineto{\pgfqpoint{2.810124in}{2.334882in}}%
\pgfpathlineto{\pgfqpoint{2.835116in}{2.334882in}}%
\pgfpathlineto{\pgfqpoint{2.860108in}{2.328361in}}%
\pgfpathlineto{\pgfqpoint{2.885100in}{2.328361in}}%
\pgfpathlineto{\pgfqpoint{2.910092in}{2.328361in}}%
\pgfpathlineto{\pgfqpoint{2.935084in}{2.321839in}}%
\pgfpathlineto{\pgfqpoint{2.960075in}{2.321839in}}%
\pgfpathlineto{\pgfqpoint{2.985067in}{2.328361in}}%
\pgfpathlineto{\pgfqpoint{3.010059in}{2.321839in}}%
\pgfpathlineto{\pgfqpoint{3.035051in}{2.328361in}}%
\pgfpathlineto{\pgfqpoint{3.060043in}{2.328361in}}%
\pgfpathlineto{\pgfqpoint{3.085035in}{2.321839in}}%
\pgfpathlineto{\pgfqpoint{3.110026in}{2.321839in}}%
\pgfpathlineto{\pgfqpoint{3.135018in}{2.315317in}}%
\pgfpathlineto{\pgfqpoint{3.160010in}{2.315317in}}%
\pgfpathlineto{\pgfqpoint{3.185002in}{2.315317in}}%
\pgfpathlineto{\pgfqpoint{3.209994in}{2.315317in}}%
\pgfpathlineto{\pgfqpoint{3.234986in}{2.315317in}}%
\pgfpathlineto{\pgfqpoint{3.259977in}{2.315317in}}%
\pgfpathlineto{\pgfqpoint{3.284969in}{2.321839in}}%
\pgfpathlineto{\pgfqpoint{3.309961in}{2.315317in}}%
\pgfpathlineto{\pgfqpoint{3.334953in}{2.321839in}}%
\pgfpathlineto{\pgfqpoint{3.359945in}{2.321839in}}%
\pgfpathlineto{\pgfqpoint{3.384937in}{2.315317in}}%
\pgfpathlineto{\pgfqpoint{3.395057in}{2.317958in}}%
\pgfusepath{stroke}%
\end{pgfscope}%
\begin{pgfscope}%
\pgfpathrectangle{\pgfqpoint{2.161373in}{1.689230in}}{\pgfqpoint{1.223684in}{1.004348in}}%
\pgfusepath{clip}%
\pgfsetrectcap%
\pgfsetroundjoin%
\pgfsetlinewidth{0.803000pt}%
\definecolor{currentstroke}{rgb}{0.490196,0.588235,0.431373}%
\pgfsetstrokecolor{currentstroke}%
\pgfsetdash{}{0pt}%
\pgfpathmoveto{\pgfqpoint{2.210320in}{2.178361in}}%
\pgfpathlineto{\pgfqpoint{2.235312in}{2.178361in}}%
\pgfpathlineto{\pgfqpoint{2.260304in}{2.145752in}}%
\pgfpathlineto{\pgfqpoint{2.285296in}{2.132708in}}%
\pgfpathlineto{\pgfqpoint{2.310288in}{2.165317in}}%
\pgfpathlineto{\pgfqpoint{2.335280in}{2.171839in}}%
\pgfpathlineto{\pgfqpoint{2.360271in}{2.145752in}}%
\pgfpathlineto{\pgfqpoint{2.385263in}{2.093578in}}%
\pgfpathlineto{\pgfqpoint{2.410255in}{2.041404in}}%
\pgfpathlineto{\pgfqpoint{2.435247in}{2.002274in}}%
\pgfpathlineto{\pgfqpoint{2.460239in}{2.015317in}}%
\pgfpathlineto{\pgfqpoint{2.485231in}{2.034882in}}%
\pgfpathlineto{\pgfqpoint{2.510222in}{2.054448in}}%
\pgfpathlineto{\pgfqpoint{2.535214in}{2.041404in}}%
\pgfpathlineto{\pgfqpoint{2.560206in}{1.995752in}}%
\pgfpathlineto{\pgfqpoint{2.585198in}{1.976187in}}%
\pgfpathlineto{\pgfqpoint{2.610190in}{1.982708in}}%
\pgfpathlineto{\pgfqpoint{2.635182in}{2.008795in}}%
\pgfpathlineto{\pgfqpoint{2.660173in}{2.034882in}}%
\pgfpathlineto{\pgfqpoint{2.685165in}{2.074013in}}%
\pgfpathlineto{\pgfqpoint{2.710157in}{2.080535in}}%
\pgfpathlineto{\pgfqpoint{2.735149in}{2.093578in}}%
\pgfpathlineto{\pgfqpoint{2.760141in}{2.126187in}}%
\pgfpathlineto{\pgfqpoint{2.785133in}{2.126187in}}%
\pgfpathlineto{\pgfqpoint{2.810124in}{2.132708in}}%
\pgfpathlineto{\pgfqpoint{2.835116in}{2.132708in}}%
\pgfpathlineto{\pgfqpoint{2.860108in}{2.139230in}}%
\pgfpathlineto{\pgfqpoint{2.885100in}{2.165317in}}%
\pgfpathlineto{\pgfqpoint{2.910092in}{2.165317in}}%
\pgfpathlineto{\pgfqpoint{2.935084in}{2.158795in}}%
\pgfpathlineto{\pgfqpoint{2.960075in}{2.171839in}}%
\pgfpathlineto{\pgfqpoint{2.985067in}{2.178361in}}%
\pgfpathlineto{\pgfqpoint{3.010059in}{2.171839in}}%
\pgfpathlineto{\pgfqpoint{3.035051in}{2.178361in}}%
\pgfpathlineto{\pgfqpoint{3.060043in}{2.184882in}}%
\pgfpathlineto{\pgfqpoint{3.085035in}{2.191404in}}%
\pgfpathlineto{\pgfqpoint{3.110026in}{2.184882in}}%
\pgfpathlineto{\pgfqpoint{3.135018in}{2.178361in}}%
\pgfpathlineto{\pgfqpoint{3.160010in}{2.165317in}}%
\pgfpathlineto{\pgfqpoint{3.185002in}{2.165317in}}%
\pgfpathlineto{\pgfqpoint{3.209994in}{2.158795in}}%
\pgfpathlineto{\pgfqpoint{3.234986in}{2.145752in}}%
\pgfpathlineto{\pgfqpoint{3.259977in}{2.152274in}}%
\pgfpathlineto{\pgfqpoint{3.284969in}{2.165317in}}%
\pgfpathlineto{\pgfqpoint{3.309961in}{2.178361in}}%
\pgfpathlineto{\pgfqpoint{3.334953in}{2.178361in}}%
\pgfpathlineto{\pgfqpoint{3.359945in}{2.184882in}}%
\pgfpathlineto{\pgfqpoint{3.384937in}{2.178361in}}%
\pgfpathlineto{\pgfqpoint{3.395057in}{2.173079in}}%
\pgfusepath{stroke}%
\end{pgfscope}%
\begin{pgfscope}%
\pgfpathrectangle{\pgfqpoint{2.161373in}{1.689230in}}{\pgfqpoint{1.223684in}{1.004348in}}%
\pgfusepath{clip}%
\pgfsetrectcap%
\pgfsetroundjoin%
\pgfsetlinewidth{0.803000pt}%
\definecolor{currentstroke}{rgb}{0.843137,0.666667,0.313725}%
\pgfsetstrokecolor{currentstroke}%
\pgfsetdash{}{0pt}%
\pgfpathmoveto{\pgfqpoint{2.210320in}{2.178361in}}%
\pgfpathlineto{\pgfqpoint{2.235312in}{2.178361in}}%
\pgfpathlineto{\pgfqpoint{2.260304in}{2.139230in}}%
\pgfpathlineto{\pgfqpoint{2.285296in}{2.145752in}}%
\pgfpathlineto{\pgfqpoint{2.310288in}{2.145752in}}%
\pgfpathlineto{\pgfqpoint{2.335280in}{2.119665in}}%
\pgfpathlineto{\pgfqpoint{2.360271in}{2.087056in}}%
\pgfpathlineto{\pgfqpoint{2.385263in}{2.106622in}}%
\pgfpathlineto{\pgfqpoint{2.410255in}{2.113143in}}%
\pgfpathlineto{\pgfqpoint{2.435247in}{2.145752in}}%
\pgfpathlineto{\pgfqpoint{2.460239in}{2.184882in}}%
\pgfpathlineto{\pgfqpoint{2.485231in}{2.178361in}}%
\pgfpathlineto{\pgfqpoint{2.510222in}{2.191404in}}%
\pgfpathlineto{\pgfqpoint{2.535214in}{2.237056in}}%
\pgfpathlineto{\pgfqpoint{2.560206in}{2.263143in}}%
\pgfpathlineto{\pgfqpoint{2.585198in}{2.243578in}}%
\pgfpathlineto{\pgfqpoint{2.610190in}{2.217491in}}%
\pgfpathlineto{\pgfqpoint{2.635182in}{2.191404in}}%
\pgfpathlineto{\pgfqpoint{2.660173in}{2.184882in}}%
\pgfpathlineto{\pgfqpoint{2.685165in}{2.171839in}}%
\pgfpathlineto{\pgfqpoint{2.710157in}{2.171839in}}%
\pgfpathlineto{\pgfqpoint{2.735149in}{2.171839in}}%
\pgfpathlineto{\pgfqpoint{2.760141in}{2.171839in}}%
\pgfpathlineto{\pgfqpoint{2.785133in}{2.171839in}}%
\pgfpathlineto{\pgfqpoint{2.810124in}{2.178361in}}%
\pgfpathlineto{\pgfqpoint{2.835116in}{2.171839in}}%
\pgfpathlineto{\pgfqpoint{2.860108in}{2.171839in}}%
\pgfpathlineto{\pgfqpoint{2.885100in}{2.178361in}}%
\pgfpathlineto{\pgfqpoint{2.910092in}{2.171839in}}%
\pgfpathlineto{\pgfqpoint{2.935084in}{2.171839in}}%
\pgfpathlineto{\pgfqpoint{2.960075in}{2.171839in}}%
\pgfpathlineto{\pgfqpoint{2.985067in}{2.171839in}}%
\pgfpathlineto{\pgfqpoint{3.010059in}{2.171839in}}%
\pgfpathlineto{\pgfqpoint{3.035051in}{2.178361in}}%
\pgfpathlineto{\pgfqpoint{3.060043in}{2.178361in}}%
\pgfpathlineto{\pgfqpoint{3.085035in}{2.178361in}}%
\pgfpathlineto{\pgfqpoint{3.110026in}{2.178361in}}%
\pgfpathlineto{\pgfqpoint{3.135018in}{2.178361in}}%
\pgfpathlineto{\pgfqpoint{3.160010in}{2.178361in}}%
\pgfpathlineto{\pgfqpoint{3.185002in}{2.178361in}}%
\pgfpathlineto{\pgfqpoint{3.209994in}{2.178361in}}%
\pgfpathlineto{\pgfqpoint{3.234986in}{2.171839in}}%
\pgfpathlineto{\pgfqpoint{3.259977in}{2.171839in}}%
\pgfpathlineto{\pgfqpoint{3.284969in}{2.178361in}}%
\pgfpathlineto{\pgfqpoint{3.309961in}{2.178361in}}%
\pgfpathlineto{\pgfqpoint{3.334953in}{2.178361in}}%
\pgfpathlineto{\pgfqpoint{3.359945in}{2.178361in}}%
\pgfpathlineto{\pgfqpoint{3.384937in}{2.178361in}}%
\pgfpathlineto{\pgfqpoint{3.395057in}{2.175720in}}%
\pgfusepath{stroke}%
\end{pgfscope}%
\begin{pgfscope}%
\pgfpathrectangle{\pgfqpoint{2.161373in}{1.689230in}}{\pgfqpoint{1.223684in}{1.004348in}}%
\pgfusepath{clip}%
\pgfsetrectcap%
\pgfsetroundjoin%
\pgfsetlinewidth{0.803000pt}%
\definecolor{currentstroke}{rgb}{0.333333,0.333333,0.333333}%
\pgfsetstrokecolor{currentstroke}%
\pgfsetdash{}{0pt}%
\pgfpathmoveto{\pgfqpoint{2.210320in}{2.321839in}}%
\pgfpathlineto{\pgfqpoint{2.235312in}{2.321839in}}%
\pgfpathlineto{\pgfqpoint{2.260304in}{2.367491in}}%
\pgfpathlineto{\pgfqpoint{2.285296in}{2.439230in}}%
\pgfpathlineto{\pgfqpoint{2.310288in}{2.608795in}}%
\pgfpathlineto{\pgfqpoint{2.335280in}{2.028361in}}%
\pgfpathlineto{\pgfqpoint{2.360271in}{2.021839in}}%
\pgfpathlineto{\pgfqpoint{2.385263in}{2.021839in}}%
\pgfpathlineto{\pgfqpoint{2.410255in}{2.008795in}}%
\pgfpathlineto{\pgfqpoint{2.435247in}{2.015317in}}%
\pgfpathlineto{\pgfqpoint{2.460239in}{2.015317in}}%
\pgfpathlineto{\pgfqpoint{2.485231in}{2.015317in}}%
\pgfpathlineto{\pgfqpoint{2.510222in}{2.008795in}}%
\pgfpathlineto{\pgfqpoint{2.535214in}{2.002274in}}%
\pgfpathlineto{\pgfqpoint{2.560206in}{2.008795in}}%
\pgfpathlineto{\pgfqpoint{2.585198in}{2.008795in}}%
\pgfpathlineto{\pgfqpoint{2.610190in}{2.008795in}}%
\pgfpathlineto{\pgfqpoint{2.635182in}{2.008795in}}%
\pgfpathlineto{\pgfqpoint{2.660173in}{2.060969in}}%
\pgfpathlineto{\pgfqpoint{2.685165in}{2.126187in}}%
\pgfpathlineto{\pgfqpoint{2.710157in}{2.171839in}}%
\pgfpathlineto{\pgfqpoint{2.735149in}{2.217491in}}%
\pgfpathlineto{\pgfqpoint{2.760141in}{2.250100in}}%
\pgfpathlineto{\pgfqpoint{2.785133in}{2.269665in}}%
\pgfpathlineto{\pgfqpoint{2.810124in}{2.295752in}}%
\pgfpathlineto{\pgfqpoint{2.835116in}{2.302274in}}%
\pgfpathlineto{\pgfqpoint{2.860108in}{2.302274in}}%
\pgfpathlineto{\pgfqpoint{2.885100in}{2.308795in}}%
\pgfpathlineto{\pgfqpoint{2.910092in}{2.315317in}}%
\pgfpathlineto{\pgfqpoint{2.935084in}{2.315317in}}%
\pgfpathlineto{\pgfqpoint{2.960075in}{2.315317in}}%
\pgfpathlineto{\pgfqpoint{2.985067in}{2.321839in}}%
\pgfpathlineto{\pgfqpoint{3.010059in}{2.321839in}}%
\pgfpathlineto{\pgfqpoint{3.035051in}{2.328361in}}%
\pgfpathlineto{\pgfqpoint{3.060043in}{2.328361in}}%
\pgfpathlineto{\pgfqpoint{3.085035in}{2.321839in}}%
\pgfpathlineto{\pgfqpoint{3.110026in}{2.328361in}}%
\pgfpathlineto{\pgfqpoint{3.135018in}{2.321839in}}%
\pgfpathlineto{\pgfqpoint{3.160010in}{2.321839in}}%
\pgfpathlineto{\pgfqpoint{3.185002in}{2.328361in}}%
\pgfpathlineto{\pgfqpoint{3.209994in}{2.328361in}}%
\pgfpathlineto{\pgfqpoint{3.234986in}{2.328361in}}%
\pgfpathlineto{\pgfqpoint{3.259977in}{2.334882in}}%
\pgfpathlineto{\pgfqpoint{3.284969in}{2.334882in}}%
\pgfpathlineto{\pgfqpoint{3.309961in}{2.334882in}}%
\pgfpathlineto{\pgfqpoint{3.334953in}{2.334882in}}%
\pgfpathlineto{\pgfqpoint{3.359945in}{2.341404in}}%
\pgfpathlineto{\pgfqpoint{3.384937in}{2.334882in}}%
\pgfpathlineto{\pgfqpoint{3.395057in}{2.334882in}}%
\pgfusepath{stroke}%
\end{pgfscope}%
\begin{pgfscope}%
\pgfpathrectangle{\pgfqpoint{2.161373in}{1.689230in}}{\pgfqpoint{1.223684in}{1.004348in}}%
\pgfusepath{clip}%
\pgfsetrectcap%
\pgfsetroundjoin%
\pgfsetlinewidth{0.803000pt}%
\definecolor{currentstroke}{rgb}{0.686275,0.352941,0.313725}%
\pgfsetstrokecolor{currentstroke}%
\pgfsetdash{}{0pt}%
\pgfpathmoveto{\pgfqpoint{2.210320in}{2.387056in}}%
\pgfpathlineto{\pgfqpoint{2.235312in}{2.387056in}}%
\pgfpathlineto{\pgfqpoint{2.260304in}{2.367491in}}%
\pgfpathlineto{\pgfqpoint{2.285296in}{2.334882in}}%
\pgfpathlineto{\pgfqpoint{2.310288in}{2.276187in}}%
\pgfpathlineto{\pgfqpoint{2.335280in}{2.243578in}}%
\pgfpathlineto{\pgfqpoint{2.360271in}{2.191404in}}%
\pgfpathlineto{\pgfqpoint{2.385263in}{2.132708in}}%
\pgfpathlineto{\pgfqpoint{2.410255in}{2.093578in}}%
\pgfpathlineto{\pgfqpoint{2.435247in}{2.060969in}}%
\pgfpathlineto{\pgfqpoint{2.460239in}{2.080535in}}%
\pgfpathlineto{\pgfqpoint{2.485231in}{2.113143in}}%
\pgfpathlineto{\pgfqpoint{2.510222in}{2.158795in}}%
\pgfpathlineto{\pgfqpoint{2.535214in}{2.237056in}}%
\pgfpathlineto{\pgfqpoint{2.560206in}{2.250100in}}%
\pgfpathlineto{\pgfqpoint{2.585198in}{2.224013in}}%
\pgfpathlineto{\pgfqpoint{2.610190in}{2.210969in}}%
\pgfpathlineto{\pgfqpoint{2.635182in}{2.224013in}}%
\pgfpathlineto{\pgfqpoint{2.660173in}{2.243578in}}%
\pgfpathlineto{\pgfqpoint{2.685165in}{2.263143in}}%
\pgfpathlineto{\pgfqpoint{2.710157in}{2.282708in}}%
\pgfpathlineto{\pgfqpoint{2.735149in}{2.308795in}}%
\pgfpathlineto{\pgfqpoint{2.760141in}{2.328361in}}%
\pgfpathlineto{\pgfqpoint{2.785133in}{2.334882in}}%
\pgfpathlineto{\pgfqpoint{2.810124in}{2.347926in}}%
\pgfpathlineto{\pgfqpoint{2.835116in}{2.347926in}}%
\pgfpathlineto{\pgfqpoint{2.860108in}{2.354448in}}%
\pgfpathlineto{\pgfqpoint{2.885100in}{2.367491in}}%
\pgfpathlineto{\pgfqpoint{2.910092in}{2.374013in}}%
\pgfpathlineto{\pgfqpoint{2.935084in}{2.380535in}}%
\pgfpathlineto{\pgfqpoint{2.960075in}{2.387056in}}%
\pgfpathlineto{\pgfqpoint{2.985067in}{2.387056in}}%
\pgfpathlineto{\pgfqpoint{3.010059in}{2.387056in}}%
\pgfpathlineto{\pgfqpoint{3.035051in}{2.393578in}}%
\pgfpathlineto{\pgfqpoint{3.060043in}{2.393578in}}%
\pgfpathlineto{\pgfqpoint{3.085035in}{2.387056in}}%
\pgfpathlineto{\pgfqpoint{3.110026in}{2.393578in}}%
\pgfpathlineto{\pgfqpoint{3.135018in}{2.393578in}}%
\pgfpathlineto{\pgfqpoint{3.160010in}{2.393578in}}%
\pgfpathlineto{\pgfqpoint{3.185002in}{2.393578in}}%
\pgfpathlineto{\pgfqpoint{3.209994in}{2.393578in}}%
\pgfpathlineto{\pgfqpoint{3.234986in}{2.393578in}}%
\pgfpathlineto{\pgfqpoint{3.259977in}{2.387056in}}%
\pgfpathlineto{\pgfqpoint{3.284969in}{2.387056in}}%
\pgfpathlineto{\pgfqpoint{3.309961in}{2.387056in}}%
\pgfpathlineto{\pgfqpoint{3.334953in}{2.387056in}}%
\pgfpathlineto{\pgfqpoint{3.359945in}{2.387056in}}%
\pgfpathlineto{\pgfqpoint{3.384937in}{2.387056in}}%
\pgfpathlineto{\pgfqpoint{3.395057in}{2.387056in}}%
\pgfusepath{stroke}%
\end{pgfscope}%
\begin{pgfscope}%
\pgfpathrectangle{\pgfqpoint{2.161373in}{1.689230in}}{\pgfqpoint{1.223684in}{1.004348in}}%
\pgfusepath{clip}%
\pgfsetrectcap%
\pgfsetroundjoin%
\pgfsetlinewidth{0.803000pt}%
\definecolor{currentstroke}{rgb}{0.000000,0.356863,0.509804}%
\pgfsetstrokecolor{currentstroke}%
\pgfsetdash{}{0pt}%
\pgfpathmoveto{\pgfqpoint{2.210320in}{2.067491in}}%
\pgfpathlineto{\pgfqpoint{2.235312in}{2.060969in}}%
\pgfpathlineto{\pgfqpoint{2.260304in}{2.074013in}}%
\pgfpathlineto{\pgfqpoint{2.285296in}{2.100100in}}%
\pgfpathlineto{\pgfqpoint{2.310288in}{2.047926in}}%
\pgfpathlineto{\pgfqpoint{2.335280in}{2.008795in}}%
\pgfpathlineto{\pgfqpoint{2.360271in}{2.021839in}}%
\pgfpathlineto{\pgfqpoint{2.385263in}{2.054448in}}%
\pgfpathlineto{\pgfqpoint{2.410255in}{2.093578in}}%
\pgfpathlineto{\pgfqpoint{2.435247in}{2.126187in}}%
\pgfpathlineto{\pgfqpoint{2.460239in}{2.126187in}}%
\pgfpathlineto{\pgfqpoint{2.485231in}{2.074013in}}%
\pgfpathlineto{\pgfqpoint{2.510222in}{2.021839in}}%
\pgfpathlineto{\pgfqpoint{2.535214in}{1.976187in}}%
\pgfpathlineto{\pgfqpoint{2.560206in}{1.930535in}}%
\pgfpathlineto{\pgfqpoint{2.585198in}{1.904448in}}%
\pgfpathlineto{\pgfqpoint{2.610190in}{1.897926in}}%
\pgfpathlineto{\pgfqpoint{2.635182in}{1.910969in}}%
\pgfpathlineto{\pgfqpoint{2.660173in}{1.937056in}}%
\pgfpathlineto{\pgfqpoint{2.685165in}{1.963143in}}%
\pgfpathlineto{\pgfqpoint{2.710157in}{1.982708in}}%
\pgfpathlineto{\pgfqpoint{2.735149in}{2.002274in}}%
\pgfpathlineto{\pgfqpoint{2.760141in}{2.015317in}}%
\pgfpathlineto{\pgfqpoint{2.785133in}{2.028361in}}%
\pgfpathlineto{\pgfqpoint{2.810124in}{2.041404in}}%
\pgfpathlineto{\pgfqpoint{2.835116in}{2.047926in}}%
\pgfpathlineto{\pgfqpoint{2.860108in}{2.047926in}}%
\pgfpathlineto{\pgfqpoint{2.885100in}{2.054448in}}%
\pgfpathlineto{\pgfqpoint{2.910092in}{2.054448in}}%
\pgfpathlineto{\pgfqpoint{2.935084in}{2.054448in}}%
\pgfpathlineto{\pgfqpoint{2.960075in}{2.060969in}}%
\pgfpathlineto{\pgfqpoint{2.985067in}{2.060969in}}%
\pgfpathlineto{\pgfqpoint{3.010059in}{2.060969in}}%
\pgfpathlineto{\pgfqpoint{3.035051in}{2.060969in}}%
\pgfpathlineto{\pgfqpoint{3.060043in}{2.060969in}}%
\pgfpathlineto{\pgfqpoint{3.085035in}{2.060969in}}%
\pgfpathlineto{\pgfqpoint{3.110026in}{2.067491in}}%
\pgfpathlineto{\pgfqpoint{3.135018in}{2.060969in}}%
\pgfpathlineto{\pgfqpoint{3.160010in}{2.060969in}}%
\pgfpathlineto{\pgfqpoint{3.185002in}{2.060969in}}%
\pgfpathlineto{\pgfqpoint{3.209994in}{2.060969in}}%
\pgfpathlineto{\pgfqpoint{3.234986in}{2.060969in}}%
\pgfpathlineto{\pgfqpoint{3.259977in}{2.060969in}}%
\pgfpathlineto{\pgfqpoint{3.284969in}{2.060969in}}%
\pgfpathlineto{\pgfqpoint{3.309961in}{2.060969in}}%
\pgfpathlineto{\pgfqpoint{3.334953in}{2.060969in}}%
\pgfpathlineto{\pgfqpoint{3.359945in}{2.067491in}}%
\pgfpathlineto{\pgfqpoint{3.384937in}{2.060969in}}%
\pgfpathlineto{\pgfqpoint{3.395057in}{2.060969in}}%
\pgfusepath{stroke}%
\end{pgfscope}%
\begin{pgfscope}%
\pgfpathrectangle{\pgfqpoint{2.161373in}{1.689230in}}{\pgfqpoint{1.223684in}{1.004348in}}%
\pgfusepath{clip}%
\pgfsetrectcap%
\pgfsetroundjoin%
\pgfsetlinewidth{0.803000pt}%
\definecolor{currentstroke}{rgb}{0.490196,0.588235,0.431373}%
\pgfsetstrokecolor{currentstroke}%
\pgfsetdash{}{0pt}%
\pgfpathmoveto{\pgfqpoint{2.210320in}{2.243578in}}%
\pgfpathlineto{\pgfqpoint{2.235312in}{2.243578in}}%
\pgfpathlineto{\pgfqpoint{2.260304in}{2.224013in}}%
\pgfpathlineto{\pgfqpoint{2.285296in}{2.263143in}}%
\pgfpathlineto{\pgfqpoint{2.310288in}{2.263143in}}%
\pgfpathlineto{\pgfqpoint{2.335280in}{2.243578in}}%
\pgfpathlineto{\pgfqpoint{2.360271in}{2.184882in}}%
\pgfpathlineto{\pgfqpoint{2.385263in}{2.132708in}}%
\pgfpathlineto{\pgfqpoint{2.410255in}{2.100100in}}%
\pgfpathlineto{\pgfqpoint{2.435247in}{2.126187in}}%
\pgfpathlineto{\pgfqpoint{2.460239in}{2.184882in}}%
\pgfpathlineto{\pgfqpoint{2.485231in}{2.217491in}}%
\pgfpathlineto{\pgfqpoint{2.510222in}{2.210969in}}%
\pgfpathlineto{\pgfqpoint{2.535214in}{2.243578in}}%
\pgfpathlineto{\pgfqpoint{2.560206in}{2.243578in}}%
\pgfpathlineto{\pgfqpoint{2.585198in}{2.237056in}}%
\pgfpathlineto{\pgfqpoint{2.610190in}{2.237056in}}%
\pgfpathlineto{\pgfqpoint{2.635182in}{2.237056in}}%
\pgfpathlineto{\pgfqpoint{2.660173in}{2.237056in}}%
\pgfpathlineto{\pgfqpoint{2.685165in}{2.237056in}}%
\pgfpathlineto{\pgfqpoint{2.710157in}{2.237056in}}%
\pgfpathlineto{\pgfqpoint{2.735149in}{2.243578in}}%
\pgfpathlineto{\pgfqpoint{2.760141in}{2.243578in}}%
\pgfpathlineto{\pgfqpoint{2.785133in}{2.243578in}}%
\pgfpathlineto{\pgfqpoint{2.810124in}{2.243578in}}%
\pgfpathlineto{\pgfqpoint{2.835116in}{2.250100in}}%
\pgfpathlineto{\pgfqpoint{2.860108in}{2.243578in}}%
\pgfpathlineto{\pgfqpoint{2.885100in}{2.243578in}}%
\pgfpathlineto{\pgfqpoint{2.910092in}{2.243578in}}%
\pgfpathlineto{\pgfqpoint{2.935084in}{2.243578in}}%
\pgfpathlineto{\pgfqpoint{2.960075in}{2.237056in}}%
\pgfpathlineto{\pgfqpoint{2.985067in}{2.237056in}}%
\pgfpathlineto{\pgfqpoint{3.010059in}{2.237056in}}%
\pgfpathlineto{\pgfqpoint{3.035051in}{2.243578in}}%
\pgfpathlineto{\pgfqpoint{3.060043in}{2.243578in}}%
\pgfpathlineto{\pgfqpoint{3.085035in}{2.243578in}}%
\pgfpathlineto{\pgfqpoint{3.110026in}{2.243578in}}%
\pgfpathlineto{\pgfqpoint{3.135018in}{2.243578in}}%
\pgfpathlineto{\pgfqpoint{3.160010in}{2.243578in}}%
\pgfpathlineto{\pgfqpoint{3.185002in}{2.243578in}}%
\pgfpathlineto{\pgfqpoint{3.209994in}{2.243578in}}%
\pgfpathlineto{\pgfqpoint{3.234986in}{2.243578in}}%
\pgfpathlineto{\pgfqpoint{3.259977in}{2.243578in}}%
\pgfpathlineto{\pgfqpoint{3.284969in}{2.243578in}}%
\pgfpathlineto{\pgfqpoint{3.309961in}{2.230535in}}%
\pgfpathlineto{\pgfqpoint{3.334953in}{2.237056in}}%
\pgfpathlineto{\pgfqpoint{3.359945in}{2.237056in}}%
\pgfpathlineto{\pgfqpoint{3.384937in}{2.237056in}}%
\pgfpathlineto{\pgfqpoint{3.395057in}{2.237056in}}%
\pgfusepath{stroke}%
\end{pgfscope}%
\begin{pgfscope}%
\pgfpathrectangle{\pgfqpoint{2.161373in}{1.689230in}}{\pgfqpoint{1.223684in}{1.004348in}}%
\pgfusepath{clip}%
\pgfsetrectcap%
\pgfsetroundjoin%
\pgfsetlinewidth{0.803000pt}%
\definecolor{currentstroke}{rgb}{0.843137,0.666667,0.313725}%
\pgfsetstrokecolor{currentstroke}%
\pgfsetdash{}{0pt}%
\pgfpathmoveto{\pgfqpoint{2.210320in}{2.047926in}}%
\pgfpathlineto{\pgfqpoint{2.235312in}{2.047926in}}%
\pgfpathlineto{\pgfqpoint{2.260304in}{2.028361in}}%
\pgfpathlineto{\pgfqpoint{2.285296in}{2.021839in}}%
\pgfpathlineto{\pgfqpoint{2.310288in}{1.963143in}}%
\pgfpathlineto{\pgfqpoint{2.335280in}{1.858795in}}%
\pgfpathlineto{\pgfqpoint{2.360271in}{1.754448in}}%
\pgfpathlineto{\pgfqpoint{2.385263in}{1.734882in}}%
\pgfpathlineto{\pgfqpoint{2.410255in}{1.780535in}}%
\pgfpathlineto{\pgfqpoint{2.435247in}{1.839230in}}%
\pgfpathlineto{\pgfqpoint{2.460239in}{1.878361in}}%
\pgfpathlineto{\pgfqpoint{2.485231in}{1.904448in}}%
\pgfpathlineto{\pgfqpoint{2.510222in}{1.904448in}}%
\pgfpathlineto{\pgfqpoint{2.535214in}{1.904448in}}%
\pgfpathlineto{\pgfqpoint{2.560206in}{1.924013in}}%
\pgfpathlineto{\pgfqpoint{2.585198in}{1.969665in}}%
\pgfpathlineto{\pgfqpoint{2.610190in}{2.008795in}}%
\pgfpathlineto{\pgfqpoint{2.635182in}{2.028361in}}%
\pgfpathlineto{\pgfqpoint{2.660173in}{2.034882in}}%
\pgfpathlineto{\pgfqpoint{2.685165in}{2.041404in}}%
\pgfpathlineto{\pgfqpoint{2.710157in}{2.041404in}}%
\pgfpathlineto{\pgfqpoint{2.735149in}{2.047926in}}%
\pgfpathlineto{\pgfqpoint{2.760141in}{2.047926in}}%
\pgfpathlineto{\pgfqpoint{2.785133in}{2.047926in}}%
\pgfpathlineto{\pgfqpoint{2.810124in}{2.054448in}}%
\pgfpathlineto{\pgfqpoint{2.835116in}{2.054448in}}%
\pgfpathlineto{\pgfqpoint{2.860108in}{2.047926in}}%
\pgfpathlineto{\pgfqpoint{2.885100in}{2.054448in}}%
\pgfpathlineto{\pgfqpoint{2.910092in}{2.054448in}}%
\pgfpathlineto{\pgfqpoint{2.935084in}{2.047926in}}%
\pgfpathlineto{\pgfqpoint{2.960075in}{2.054448in}}%
\pgfpathlineto{\pgfqpoint{2.985067in}{2.054448in}}%
\pgfpathlineto{\pgfqpoint{3.010059in}{2.054448in}}%
\pgfpathlineto{\pgfqpoint{3.035051in}{2.054448in}}%
\pgfpathlineto{\pgfqpoint{3.060043in}{2.054448in}}%
\pgfpathlineto{\pgfqpoint{3.085035in}{2.054448in}}%
\pgfpathlineto{\pgfqpoint{3.110026in}{2.054448in}}%
\pgfpathlineto{\pgfqpoint{3.135018in}{2.054448in}}%
\pgfpathlineto{\pgfqpoint{3.160010in}{2.054448in}}%
\pgfpathlineto{\pgfqpoint{3.185002in}{2.054448in}}%
\pgfpathlineto{\pgfqpoint{3.209994in}{2.054448in}}%
\pgfpathlineto{\pgfqpoint{3.234986in}{2.047926in}}%
\pgfpathlineto{\pgfqpoint{3.259977in}{2.047926in}}%
\pgfpathlineto{\pgfqpoint{3.284969in}{2.047926in}}%
\pgfpathlineto{\pgfqpoint{3.309961in}{2.041404in}}%
\pgfpathlineto{\pgfqpoint{3.334953in}{2.047926in}}%
\pgfpathlineto{\pgfqpoint{3.359945in}{2.047926in}}%
\pgfpathlineto{\pgfqpoint{3.384937in}{2.047926in}}%
\pgfpathlineto{\pgfqpoint{3.395057in}{2.047926in}}%
\pgfusepath{stroke}%
\end{pgfscope}%
\begin{pgfscope}%
\pgfpathrectangle{\pgfqpoint{2.161373in}{1.689230in}}{\pgfqpoint{1.223684in}{1.004348in}}%
\pgfusepath{clip}%
\pgfsetrectcap%
\pgfsetroundjoin%
\pgfsetlinewidth{0.803000pt}%
\definecolor{currentstroke}{rgb}{0.333333,0.333333,0.333333}%
\pgfsetstrokecolor{currentstroke}%
\pgfsetdash{}{0pt}%
\pgfpathmoveto{\pgfqpoint{2.210320in}{2.113143in}}%
\pgfpathlineto{\pgfqpoint{2.235312in}{2.119665in}}%
\pgfpathlineto{\pgfqpoint{2.260304in}{2.074013in}}%
\pgfpathlineto{\pgfqpoint{2.285296in}{2.060969in}}%
\pgfpathlineto{\pgfqpoint{2.310288in}{2.080535in}}%
\pgfpathlineto{\pgfqpoint{2.335280in}{2.074013in}}%
\pgfpathlineto{\pgfqpoint{2.360271in}{2.074013in}}%
\pgfpathlineto{\pgfqpoint{2.385263in}{2.119665in}}%
\pgfpathlineto{\pgfqpoint{2.410255in}{2.093578in}}%
\pgfpathlineto{\pgfqpoint{2.435247in}{2.067491in}}%
\pgfpathlineto{\pgfqpoint{2.460239in}{2.015317in}}%
\pgfpathlineto{\pgfqpoint{2.485231in}{1.989230in}}%
\pgfpathlineto{\pgfqpoint{2.510222in}{2.034882in}}%
\pgfpathlineto{\pgfqpoint{2.535214in}{2.054448in}}%
\pgfpathlineto{\pgfqpoint{2.560206in}{2.015317in}}%
\pgfpathlineto{\pgfqpoint{2.585198in}{1.976187in}}%
\pgfpathlineto{\pgfqpoint{2.610190in}{1.976187in}}%
\pgfpathlineto{\pgfqpoint{2.635182in}{1.995752in}}%
\pgfpathlineto{\pgfqpoint{2.660173in}{2.015317in}}%
\pgfpathlineto{\pgfqpoint{2.685165in}{2.041404in}}%
\pgfpathlineto{\pgfqpoint{2.710157in}{2.054448in}}%
\pgfpathlineto{\pgfqpoint{2.735149in}{2.074013in}}%
\pgfpathlineto{\pgfqpoint{2.760141in}{2.080535in}}%
\pgfpathlineto{\pgfqpoint{2.785133in}{2.087056in}}%
\pgfpathlineto{\pgfqpoint{2.810124in}{2.093578in}}%
\pgfpathlineto{\pgfqpoint{2.835116in}{2.100100in}}%
\pgfpathlineto{\pgfqpoint{2.860108in}{2.100100in}}%
\pgfpathlineto{\pgfqpoint{2.885100in}{2.106622in}}%
\pgfpathlineto{\pgfqpoint{2.910092in}{2.106622in}}%
\pgfpathlineto{\pgfqpoint{2.935084in}{2.100100in}}%
\pgfpathlineto{\pgfqpoint{2.960075in}{2.106622in}}%
\pgfpathlineto{\pgfqpoint{2.985067in}{2.106622in}}%
\pgfpathlineto{\pgfqpoint{3.010059in}{2.106622in}}%
\pgfpathlineto{\pgfqpoint{3.035051in}{2.106622in}}%
\pgfpathlineto{\pgfqpoint{3.060043in}{2.100100in}}%
\pgfpathlineto{\pgfqpoint{3.085035in}{2.106622in}}%
\pgfpathlineto{\pgfqpoint{3.110026in}{2.106622in}}%
\pgfpathlineto{\pgfqpoint{3.135018in}{2.106622in}}%
\pgfpathlineto{\pgfqpoint{3.160010in}{2.113143in}}%
\pgfpathlineto{\pgfqpoint{3.185002in}{2.106622in}}%
\pgfpathlineto{\pgfqpoint{3.209994in}{2.106622in}}%
\pgfpathlineto{\pgfqpoint{3.234986in}{2.106622in}}%
\pgfpathlineto{\pgfqpoint{3.259977in}{2.106622in}}%
\pgfpathlineto{\pgfqpoint{3.284969in}{2.106622in}}%
\pgfpathlineto{\pgfqpoint{3.309961in}{2.106622in}}%
\pgfpathlineto{\pgfqpoint{3.334953in}{2.106622in}}%
\pgfpathlineto{\pgfqpoint{3.359945in}{2.106622in}}%
\pgfpathlineto{\pgfqpoint{3.384937in}{2.106622in}}%
\pgfpathlineto{\pgfqpoint{3.395057in}{2.106622in}}%
\pgfusepath{stroke}%
\end{pgfscope}%
\begin{pgfscope}%
\pgfpathrectangle{\pgfqpoint{2.161373in}{1.689230in}}{\pgfqpoint{1.223684in}{1.004348in}}%
\pgfusepath{clip}%
\pgfsetrectcap%
\pgfsetroundjoin%
\pgfsetlinewidth{0.803000pt}%
\definecolor{currentstroke}{rgb}{0.686275,0.352941,0.313725}%
\pgfsetstrokecolor{currentstroke}%
\pgfsetdash{}{0pt}%
\pgfpathmoveto{\pgfqpoint{2.210320in}{2.158795in}}%
\pgfpathlineto{\pgfqpoint{2.235312in}{2.158795in}}%
\pgfpathlineto{\pgfqpoint{2.260304in}{2.139230in}}%
\pgfpathlineto{\pgfqpoint{2.285296in}{2.126187in}}%
\pgfpathlineto{\pgfqpoint{2.310288in}{1.937056in}}%
\pgfpathlineto{\pgfqpoint{2.335280in}{1.806622in}}%
\pgfpathlineto{\pgfqpoint{2.360271in}{1.760969in}}%
\pgfpathlineto{\pgfqpoint{2.385263in}{1.774013in}}%
\pgfpathlineto{\pgfqpoint{2.410255in}{1.852274in}}%
\pgfpathlineto{\pgfqpoint{2.435247in}{1.904448in}}%
\pgfpathlineto{\pgfqpoint{2.460239in}{1.937056in}}%
\pgfpathlineto{\pgfqpoint{2.485231in}{1.995752in}}%
\pgfpathlineto{\pgfqpoint{2.510222in}{2.002274in}}%
\pgfpathlineto{\pgfqpoint{2.535214in}{1.982708in}}%
\pgfpathlineto{\pgfqpoint{2.560206in}{1.969665in}}%
\pgfpathlineto{\pgfqpoint{2.585198in}{1.956622in}}%
\pgfpathlineto{\pgfqpoint{2.610190in}{1.963143in}}%
\pgfpathlineto{\pgfqpoint{2.635182in}{1.989230in}}%
\pgfpathlineto{\pgfqpoint{2.660173in}{2.021839in}}%
\pgfpathlineto{\pgfqpoint{2.685165in}{2.047926in}}%
\pgfpathlineto{\pgfqpoint{2.710157in}{2.074013in}}%
\pgfpathlineto{\pgfqpoint{2.735149in}{2.093578in}}%
\pgfpathlineto{\pgfqpoint{2.760141in}{2.106622in}}%
\pgfpathlineto{\pgfqpoint{2.785133in}{2.119665in}}%
\pgfpathlineto{\pgfqpoint{2.810124in}{2.132708in}}%
\pgfpathlineto{\pgfqpoint{2.835116in}{2.139230in}}%
\pgfpathlineto{\pgfqpoint{2.860108in}{2.145752in}}%
\pgfpathlineto{\pgfqpoint{2.885100in}{2.145752in}}%
\pgfpathlineto{\pgfqpoint{2.910092in}{2.152274in}}%
\pgfpathlineto{\pgfqpoint{2.935084in}{2.145752in}}%
\pgfpathlineto{\pgfqpoint{2.960075in}{2.152274in}}%
\pgfpathlineto{\pgfqpoint{2.985067in}{2.152274in}}%
\pgfpathlineto{\pgfqpoint{3.010059in}{2.145752in}}%
\pgfpathlineto{\pgfqpoint{3.035051in}{2.152274in}}%
\pgfpathlineto{\pgfqpoint{3.060043in}{2.152274in}}%
\pgfpathlineto{\pgfqpoint{3.085035in}{2.152274in}}%
\pgfpathlineto{\pgfqpoint{3.110026in}{2.158795in}}%
\pgfpathlineto{\pgfqpoint{3.135018in}{2.158795in}}%
\pgfpathlineto{\pgfqpoint{3.160010in}{2.152274in}}%
\pgfpathlineto{\pgfqpoint{3.185002in}{2.158795in}}%
\pgfpathlineto{\pgfqpoint{3.209994in}{2.158795in}}%
\pgfpathlineto{\pgfqpoint{3.234986in}{2.152274in}}%
\pgfpathlineto{\pgfqpoint{3.259977in}{2.158795in}}%
\pgfpathlineto{\pgfqpoint{3.284969in}{2.158795in}}%
\pgfpathlineto{\pgfqpoint{3.309961in}{2.152274in}}%
\pgfpathlineto{\pgfqpoint{3.334953in}{2.152274in}}%
\pgfpathlineto{\pgfqpoint{3.359945in}{2.158795in}}%
\pgfpathlineto{\pgfqpoint{3.384937in}{2.158795in}}%
\pgfpathlineto{\pgfqpoint{3.395057in}{2.156154in}}%
\pgfusepath{stroke}%
\end{pgfscope}%
\begin{pgfscope}%
\pgfpathrectangle{\pgfqpoint{2.161373in}{1.689230in}}{\pgfqpoint{1.223684in}{1.004348in}}%
\pgfusepath{clip}%
\pgfsetrectcap%
\pgfsetroundjoin%
\pgfsetlinewidth{0.803000pt}%
\definecolor{currentstroke}{rgb}{0.000000,0.356863,0.509804}%
\pgfsetstrokecolor{currentstroke}%
\pgfsetdash{}{0pt}%
\pgfpathmoveto{\pgfqpoint{2.210320in}{2.347926in}}%
\pgfpathlineto{\pgfqpoint{2.235312in}{2.347926in}}%
\pgfpathlineto{\pgfqpoint{2.260304in}{2.367491in}}%
\pgfpathlineto{\pgfqpoint{2.285296in}{2.458795in}}%
\pgfpathlineto{\pgfqpoint{2.310288in}{2.582708in}}%
\pgfpathlineto{\pgfqpoint{2.335280in}{2.308795in}}%
\pgfpathlineto{\pgfqpoint{2.360271in}{2.015317in}}%
\pgfpathlineto{\pgfqpoint{2.385263in}{2.008795in}}%
\pgfpathlineto{\pgfqpoint{2.410255in}{2.008795in}}%
\pgfpathlineto{\pgfqpoint{2.435247in}{2.008795in}}%
\pgfpathlineto{\pgfqpoint{2.460239in}{2.002274in}}%
\pgfpathlineto{\pgfqpoint{2.485231in}{2.002274in}}%
\pgfpathlineto{\pgfqpoint{2.510222in}{2.015317in}}%
\pgfpathlineto{\pgfqpoint{2.535214in}{2.015317in}}%
\pgfpathlineto{\pgfqpoint{2.560206in}{2.002274in}}%
\pgfpathlineto{\pgfqpoint{2.585198in}{2.008795in}}%
\pgfpathlineto{\pgfqpoint{2.610190in}{2.008795in}}%
\pgfpathlineto{\pgfqpoint{2.635182in}{2.002274in}}%
\pgfpathlineto{\pgfqpoint{2.660173in}{2.041404in}}%
\pgfpathlineto{\pgfqpoint{2.685165in}{2.132708in}}%
\pgfpathlineto{\pgfqpoint{2.710157in}{2.191404in}}%
\pgfpathlineto{\pgfqpoint{2.735149in}{2.237056in}}%
\pgfpathlineto{\pgfqpoint{2.760141in}{2.263143in}}%
\pgfpathlineto{\pgfqpoint{2.785133in}{2.282708in}}%
\pgfpathlineto{\pgfqpoint{2.810124in}{2.302274in}}%
\pgfpathlineto{\pgfqpoint{2.835116in}{2.315317in}}%
\pgfpathlineto{\pgfqpoint{2.860108in}{2.321839in}}%
\pgfpathlineto{\pgfqpoint{2.885100in}{2.334882in}}%
\pgfpathlineto{\pgfqpoint{2.910092in}{2.334882in}}%
\pgfpathlineto{\pgfqpoint{2.935084in}{2.334882in}}%
\pgfpathlineto{\pgfqpoint{2.960075in}{2.341404in}}%
\pgfpathlineto{\pgfqpoint{2.985067in}{2.341404in}}%
\pgfpathlineto{\pgfqpoint{3.010059in}{2.341404in}}%
\pgfpathlineto{\pgfqpoint{3.035051in}{2.347926in}}%
\pgfpathlineto{\pgfqpoint{3.060043in}{2.347926in}}%
\pgfpathlineto{\pgfqpoint{3.085035in}{2.341404in}}%
\pgfpathlineto{\pgfqpoint{3.110026in}{2.347926in}}%
\pgfpathlineto{\pgfqpoint{3.135018in}{2.347926in}}%
\pgfpathlineto{\pgfqpoint{3.160010in}{2.347926in}}%
\pgfpathlineto{\pgfqpoint{3.185002in}{2.347926in}}%
\pgfpathlineto{\pgfqpoint{3.209994in}{2.347926in}}%
\pgfpathlineto{\pgfqpoint{3.234986in}{2.347926in}}%
\pgfpathlineto{\pgfqpoint{3.259977in}{2.347926in}}%
\pgfpathlineto{\pgfqpoint{3.284969in}{2.354448in}}%
\pgfpathlineto{\pgfqpoint{3.309961in}{2.347926in}}%
\pgfpathlineto{\pgfqpoint{3.334953in}{2.354448in}}%
\pgfpathlineto{\pgfqpoint{3.359945in}{2.354448in}}%
\pgfpathlineto{\pgfqpoint{3.384937in}{2.354448in}}%
\pgfpathlineto{\pgfqpoint{3.395057in}{2.354448in}}%
\pgfusepath{stroke}%
\end{pgfscope}%
\begin{pgfscope}%
\pgfpathrectangle{\pgfqpoint{2.161373in}{1.689230in}}{\pgfqpoint{1.223684in}{1.004348in}}%
\pgfusepath{clip}%
\pgfsetrectcap%
\pgfsetroundjoin%
\pgfsetlinewidth{0.803000pt}%
\definecolor{currentstroke}{rgb}{0.490196,0.588235,0.431373}%
\pgfsetstrokecolor{currentstroke}%
\pgfsetdash{}{0pt}%
\pgfpathmoveto{\pgfqpoint{2.210320in}{2.119665in}}%
\pgfpathlineto{\pgfqpoint{2.235312in}{2.119665in}}%
\pgfpathlineto{\pgfqpoint{2.260304in}{2.158795in}}%
\pgfpathlineto{\pgfqpoint{2.285296in}{2.171839in}}%
\pgfpathlineto{\pgfqpoint{2.310288in}{2.158795in}}%
\pgfpathlineto{\pgfqpoint{2.335280in}{2.184882in}}%
\pgfpathlineto{\pgfqpoint{2.360271in}{2.113143in}}%
\pgfpathlineto{\pgfqpoint{2.385263in}{2.074013in}}%
\pgfpathlineto{\pgfqpoint{2.410255in}{2.034882in}}%
\pgfpathlineto{\pgfqpoint{2.435247in}{2.021839in}}%
\pgfpathlineto{\pgfqpoint{2.460239in}{2.034882in}}%
\pgfpathlineto{\pgfqpoint{2.485231in}{2.041404in}}%
\pgfpathlineto{\pgfqpoint{2.510222in}{2.060969in}}%
\pgfpathlineto{\pgfqpoint{2.535214in}{1.989230in}}%
\pgfpathlineto{\pgfqpoint{2.560206in}{1.956622in}}%
\pgfpathlineto{\pgfqpoint{2.585198in}{1.989230in}}%
\pgfpathlineto{\pgfqpoint{2.610190in}{2.021839in}}%
\pgfpathlineto{\pgfqpoint{2.635182in}{2.041404in}}%
\pgfpathlineto{\pgfqpoint{2.660173in}{2.054448in}}%
\pgfpathlineto{\pgfqpoint{2.685165in}{2.067491in}}%
\pgfpathlineto{\pgfqpoint{2.710157in}{2.080535in}}%
\pgfpathlineto{\pgfqpoint{2.735149in}{2.093578in}}%
\pgfpathlineto{\pgfqpoint{2.760141in}{2.093578in}}%
\pgfpathlineto{\pgfqpoint{2.785133in}{2.100100in}}%
\pgfpathlineto{\pgfqpoint{2.810124in}{2.106622in}}%
\pgfpathlineto{\pgfqpoint{2.835116in}{2.106622in}}%
\pgfpathlineto{\pgfqpoint{2.860108in}{2.106622in}}%
\pgfpathlineto{\pgfqpoint{2.885100in}{2.113143in}}%
\pgfpathlineto{\pgfqpoint{2.910092in}{2.113143in}}%
\pgfpathlineto{\pgfqpoint{2.935084in}{2.113143in}}%
\pgfpathlineto{\pgfqpoint{2.960075in}{2.113143in}}%
\pgfpathlineto{\pgfqpoint{2.985067in}{2.113143in}}%
\pgfpathlineto{\pgfqpoint{3.010059in}{2.106622in}}%
\pgfpathlineto{\pgfqpoint{3.035051in}{2.106622in}}%
\pgfpathlineto{\pgfqpoint{3.060043in}{2.106622in}}%
\pgfpathlineto{\pgfqpoint{3.085035in}{2.106622in}}%
\pgfpathlineto{\pgfqpoint{3.110026in}{2.113143in}}%
\pgfpathlineto{\pgfqpoint{3.135018in}{2.113143in}}%
\pgfpathlineto{\pgfqpoint{3.160010in}{2.113143in}}%
\pgfpathlineto{\pgfqpoint{3.185002in}{2.113143in}}%
\pgfpathlineto{\pgfqpoint{3.209994in}{2.113143in}}%
\pgfpathlineto{\pgfqpoint{3.234986in}{2.113143in}}%
\pgfpathlineto{\pgfqpoint{3.259977in}{2.113143in}}%
\pgfpathlineto{\pgfqpoint{3.284969in}{2.113143in}}%
\pgfpathlineto{\pgfqpoint{3.309961in}{2.106622in}}%
\pgfpathlineto{\pgfqpoint{3.334953in}{2.113143in}}%
\pgfpathlineto{\pgfqpoint{3.359945in}{2.113143in}}%
\pgfpathlineto{\pgfqpoint{3.384937in}{2.113143in}}%
\pgfpathlineto{\pgfqpoint{3.395057in}{2.115784in}}%
\pgfusepath{stroke}%
\end{pgfscope}%
\begin{pgfscope}%
\pgfpathrectangle{\pgfqpoint{2.161373in}{1.689230in}}{\pgfqpoint{1.223684in}{1.004348in}}%
\pgfusepath{clip}%
\pgfsetrectcap%
\pgfsetroundjoin%
\pgfsetlinewidth{0.803000pt}%
\definecolor{currentstroke}{rgb}{0.843137,0.666667,0.313725}%
\pgfsetstrokecolor{currentstroke}%
\pgfsetdash{}{0pt}%
\pgfpathmoveto{\pgfqpoint{2.210320in}{2.282708in}}%
\pgfpathlineto{\pgfqpoint{2.235312in}{2.276187in}}%
\pgfpathlineto{\pgfqpoint{2.260304in}{2.243578in}}%
\pgfpathlineto{\pgfqpoint{2.285296in}{2.158795in}}%
\pgfpathlineto{\pgfqpoint{2.310288in}{2.178361in}}%
\pgfpathlineto{\pgfqpoint{2.335280in}{2.243578in}}%
\pgfpathlineto{\pgfqpoint{2.360271in}{2.289230in}}%
\pgfpathlineto{\pgfqpoint{2.385263in}{2.387056in}}%
\pgfpathlineto{\pgfqpoint{2.410255in}{2.491404in}}%
\pgfpathlineto{\pgfqpoint{2.435247in}{2.569665in}}%
\pgfpathlineto{\pgfqpoint{2.460239in}{2.576187in}}%
\pgfpathlineto{\pgfqpoint{2.485231in}{2.589230in}}%
\pgfpathlineto{\pgfqpoint{2.510222in}{2.550100in}}%
\pgfpathlineto{\pgfqpoint{2.535214in}{2.484882in}}%
\pgfpathlineto{\pgfqpoint{2.560206in}{2.478361in}}%
\pgfpathlineto{\pgfqpoint{2.585198in}{2.484882in}}%
\pgfpathlineto{\pgfqpoint{2.610190in}{2.491404in}}%
\pgfpathlineto{\pgfqpoint{2.635182in}{2.471839in}}%
\pgfpathlineto{\pgfqpoint{2.660173in}{2.445752in}}%
\pgfpathlineto{\pgfqpoint{2.685165in}{2.419665in}}%
\pgfpathlineto{\pgfqpoint{2.710157in}{2.393578in}}%
\pgfpathlineto{\pgfqpoint{2.735149in}{2.380535in}}%
\pgfpathlineto{\pgfqpoint{2.760141in}{2.360969in}}%
\pgfpathlineto{\pgfqpoint{2.785133in}{2.341404in}}%
\pgfpathlineto{\pgfqpoint{2.810124in}{2.334882in}}%
\pgfpathlineto{\pgfqpoint{2.835116in}{2.321839in}}%
\pgfpathlineto{\pgfqpoint{2.860108in}{2.308795in}}%
\pgfpathlineto{\pgfqpoint{2.885100in}{2.302274in}}%
\pgfpathlineto{\pgfqpoint{2.910092in}{2.302274in}}%
\pgfpathlineto{\pgfqpoint{2.935084in}{2.289230in}}%
\pgfpathlineto{\pgfqpoint{2.960075in}{2.295752in}}%
\pgfpathlineto{\pgfqpoint{2.985067in}{2.289230in}}%
\pgfpathlineto{\pgfqpoint{3.010059in}{2.282708in}}%
\pgfpathlineto{\pgfqpoint{3.035051in}{2.289230in}}%
\pgfpathlineto{\pgfqpoint{3.060043in}{2.282708in}}%
\pgfpathlineto{\pgfqpoint{3.085035in}{2.282708in}}%
\pgfpathlineto{\pgfqpoint{3.110026in}{2.289230in}}%
\pgfpathlineto{\pgfqpoint{3.135018in}{2.289230in}}%
\pgfpathlineto{\pgfqpoint{3.160010in}{2.289230in}}%
\pgfpathlineto{\pgfqpoint{3.185002in}{2.289230in}}%
\pgfpathlineto{\pgfqpoint{3.209994in}{2.289230in}}%
\pgfpathlineto{\pgfqpoint{3.234986in}{2.289230in}}%
\pgfpathlineto{\pgfqpoint{3.259977in}{2.289230in}}%
\pgfpathlineto{\pgfqpoint{3.284969in}{2.289230in}}%
\pgfpathlineto{\pgfqpoint{3.309961in}{2.289230in}}%
\pgfpathlineto{\pgfqpoint{3.334953in}{2.289230in}}%
\pgfpathlineto{\pgfqpoint{3.359945in}{2.289230in}}%
\pgfpathlineto{\pgfqpoint{3.384937in}{2.282708in}}%
\pgfpathlineto{\pgfqpoint{3.395057in}{2.282708in}}%
\pgfusepath{stroke}%
\end{pgfscope}%
\begin{pgfscope}%
\pgfpathrectangle{\pgfqpoint{2.161373in}{1.689230in}}{\pgfqpoint{1.223684in}{1.004348in}}%
\pgfusepath{clip}%
\pgfsetrectcap%
\pgfsetroundjoin%
\pgfsetlinewidth{0.803000pt}%
\definecolor{currentstroke}{rgb}{0.333333,0.333333,0.333333}%
\pgfsetstrokecolor{currentstroke}%
\pgfsetdash{}{0pt}%
\pgfpathmoveto{\pgfqpoint{2.210320in}{2.191404in}}%
\pgfpathlineto{\pgfqpoint{2.235312in}{2.184882in}}%
\pgfpathlineto{\pgfqpoint{2.260304in}{2.197926in}}%
\pgfpathlineto{\pgfqpoint{2.285296in}{2.152274in}}%
\pgfpathlineto{\pgfqpoint{2.310288in}{2.028361in}}%
\pgfpathlineto{\pgfqpoint{2.335280in}{1.937056in}}%
\pgfpathlineto{\pgfqpoint{2.360271in}{1.930535in}}%
\pgfpathlineto{\pgfqpoint{2.385263in}{1.976187in}}%
\pgfpathlineto{\pgfqpoint{2.410255in}{1.969665in}}%
\pgfpathlineto{\pgfqpoint{2.435247in}{1.976187in}}%
\pgfpathlineto{\pgfqpoint{2.460239in}{2.021839in}}%
\pgfpathlineto{\pgfqpoint{2.485231in}{2.002274in}}%
\pgfpathlineto{\pgfqpoint{2.510222in}{1.995752in}}%
\pgfpathlineto{\pgfqpoint{2.535214in}{2.054448in}}%
\pgfpathlineto{\pgfqpoint{2.560206in}{2.080535in}}%
\pgfpathlineto{\pgfqpoint{2.585198in}{2.106622in}}%
\pgfpathlineto{\pgfqpoint{2.610190in}{2.132708in}}%
\pgfpathlineto{\pgfqpoint{2.635182in}{2.152274in}}%
\pgfpathlineto{\pgfqpoint{2.660173in}{2.165317in}}%
\pgfpathlineto{\pgfqpoint{2.685165in}{2.178361in}}%
\pgfpathlineto{\pgfqpoint{2.710157in}{2.178361in}}%
\pgfpathlineto{\pgfqpoint{2.735149in}{2.184882in}}%
\pgfpathlineto{\pgfqpoint{2.760141in}{2.191404in}}%
\pgfpathlineto{\pgfqpoint{2.785133in}{2.184882in}}%
\pgfpathlineto{\pgfqpoint{2.810124in}{2.191404in}}%
\pgfpathlineto{\pgfqpoint{2.835116in}{2.191404in}}%
\pgfpathlineto{\pgfqpoint{2.860108in}{2.191404in}}%
\pgfpathlineto{\pgfqpoint{2.885100in}{2.191404in}}%
\pgfpathlineto{\pgfqpoint{2.910092in}{2.184882in}}%
\pgfpathlineto{\pgfqpoint{2.935084in}{2.184882in}}%
\pgfpathlineto{\pgfqpoint{2.960075in}{2.184882in}}%
\pgfpathlineto{\pgfqpoint{2.985067in}{2.184882in}}%
\pgfpathlineto{\pgfqpoint{3.010059in}{2.178361in}}%
\pgfpathlineto{\pgfqpoint{3.035051in}{2.184882in}}%
\pgfpathlineto{\pgfqpoint{3.060043in}{2.184882in}}%
\pgfpathlineto{\pgfqpoint{3.085035in}{2.184882in}}%
\pgfpathlineto{\pgfqpoint{3.110026in}{2.184882in}}%
\pgfpathlineto{\pgfqpoint{3.135018in}{2.184882in}}%
\pgfpathlineto{\pgfqpoint{3.160010in}{2.178361in}}%
\pgfpathlineto{\pgfqpoint{3.185002in}{2.178361in}}%
\pgfpathlineto{\pgfqpoint{3.209994in}{2.178361in}}%
\pgfpathlineto{\pgfqpoint{3.234986in}{2.178361in}}%
\pgfpathlineto{\pgfqpoint{3.259977in}{2.184882in}}%
\pgfpathlineto{\pgfqpoint{3.284969in}{2.191404in}}%
\pgfpathlineto{\pgfqpoint{3.309961in}{2.184882in}}%
\pgfpathlineto{\pgfqpoint{3.334953in}{2.184882in}}%
\pgfpathlineto{\pgfqpoint{3.359945in}{2.191404in}}%
\pgfpathlineto{\pgfqpoint{3.384937in}{2.184882in}}%
\pgfpathlineto{\pgfqpoint{3.395057in}{2.184882in}}%
\pgfusepath{stroke}%
\end{pgfscope}%
\begin{pgfscope}%
\pgfpathrectangle{\pgfqpoint{2.161373in}{1.689230in}}{\pgfqpoint{1.223684in}{1.004348in}}%
\pgfusepath{clip}%
\pgfsetrectcap%
\pgfsetroundjoin%
\pgfsetlinewidth{0.803000pt}%
\definecolor{currentstroke}{rgb}{0.686275,0.352941,0.313725}%
\pgfsetstrokecolor{currentstroke}%
\pgfsetdash{}{0pt}%
\pgfpathmoveto{\pgfqpoint{2.210320in}{2.165317in}}%
\pgfpathlineto{\pgfqpoint{2.235312in}{2.171839in}}%
\pgfpathlineto{\pgfqpoint{2.260304in}{2.145752in}}%
\pgfpathlineto{\pgfqpoint{2.285296in}{2.100100in}}%
\pgfpathlineto{\pgfqpoint{2.310288in}{2.054448in}}%
\pgfpathlineto{\pgfqpoint{2.335280in}{2.021839in}}%
\pgfpathlineto{\pgfqpoint{2.360271in}{2.080535in}}%
\pgfpathlineto{\pgfqpoint{2.385263in}{2.113143in}}%
\pgfpathlineto{\pgfqpoint{2.410255in}{2.106622in}}%
\pgfpathlineto{\pgfqpoint{2.435247in}{2.067491in}}%
\pgfpathlineto{\pgfqpoint{2.460239in}{2.067491in}}%
\pgfpathlineto{\pgfqpoint{2.485231in}{2.034882in}}%
\pgfpathlineto{\pgfqpoint{2.510222in}{1.937056in}}%
\pgfpathlineto{\pgfqpoint{2.535214in}{1.897926in}}%
\pgfpathlineto{\pgfqpoint{2.560206in}{1.910969in}}%
\pgfpathlineto{\pgfqpoint{2.585198in}{1.963143in}}%
\pgfpathlineto{\pgfqpoint{2.610190in}{2.021839in}}%
\pgfpathlineto{\pgfqpoint{2.635182in}{2.060969in}}%
\pgfpathlineto{\pgfqpoint{2.660173in}{2.080535in}}%
\pgfpathlineto{\pgfqpoint{2.685165in}{2.106622in}}%
\pgfpathlineto{\pgfqpoint{2.710157in}{2.119665in}}%
\pgfpathlineto{\pgfqpoint{2.735149in}{2.132708in}}%
\pgfpathlineto{\pgfqpoint{2.760141in}{2.145752in}}%
\pgfpathlineto{\pgfqpoint{2.785133in}{2.152274in}}%
\pgfpathlineto{\pgfqpoint{2.810124in}{2.158795in}}%
\pgfpathlineto{\pgfqpoint{2.835116in}{2.158795in}}%
\pgfpathlineto{\pgfqpoint{2.860108in}{2.158795in}}%
\pgfpathlineto{\pgfqpoint{2.885100in}{2.165317in}}%
\pgfpathlineto{\pgfqpoint{2.910092in}{2.165317in}}%
\pgfpathlineto{\pgfqpoint{2.935084in}{2.171839in}}%
\pgfpathlineto{\pgfqpoint{2.960075in}{2.178361in}}%
\pgfpathlineto{\pgfqpoint{2.985067in}{2.171839in}}%
\pgfpathlineto{\pgfqpoint{3.010059in}{2.178361in}}%
\pgfpathlineto{\pgfqpoint{3.035051in}{2.178361in}}%
\pgfpathlineto{\pgfqpoint{3.060043in}{2.165317in}}%
\pgfpathlineto{\pgfqpoint{3.085035in}{2.165317in}}%
\pgfpathlineto{\pgfqpoint{3.110026in}{2.165317in}}%
\pgfpathlineto{\pgfqpoint{3.135018in}{2.165317in}}%
\pgfpathlineto{\pgfqpoint{3.160010in}{2.165317in}}%
\pgfpathlineto{\pgfqpoint{3.185002in}{2.158795in}}%
\pgfpathlineto{\pgfqpoint{3.209994in}{2.152274in}}%
\pgfpathlineto{\pgfqpoint{3.234986in}{2.152274in}}%
\pgfpathlineto{\pgfqpoint{3.259977in}{2.152274in}}%
\pgfpathlineto{\pgfqpoint{3.284969in}{2.152274in}}%
\pgfpathlineto{\pgfqpoint{3.309961in}{2.158795in}}%
\pgfpathlineto{\pgfqpoint{3.334953in}{2.158795in}}%
\pgfpathlineto{\pgfqpoint{3.359945in}{2.165317in}}%
\pgfpathlineto{\pgfqpoint{3.384937in}{2.165317in}}%
\pgfpathlineto{\pgfqpoint{3.395057in}{2.162676in}}%
\pgfusepath{stroke}%
\end{pgfscope}%
\begin{pgfscope}%
\pgfpathrectangle{\pgfqpoint{2.161373in}{1.689230in}}{\pgfqpoint{1.223684in}{1.004348in}}%
\pgfusepath{clip}%
\pgfsetrectcap%
\pgfsetroundjoin%
\pgfsetlinewidth{0.803000pt}%
\definecolor{currentstroke}{rgb}{0.000000,0.356863,0.509804}%
\pgfsetstrokecolor{currentstroke}%
\pgfsetdash{}{0pt}%
\pgfpathmoveto{\pgfqpoint{2.210320in}{2.184882in}}%
\pgfpathlineto{\pgfqpoint{2.235312in}{2.184882in}}%
\pgfpathlineto{\pgfqpoint{2.260304in}{2.178361in}}%
\pgfpathlineto{\pgfqpoint{2.285296in}{2.132708in}}%
\pgfpathlineto{\pgfqpoint{2.310288in}{2.093578in}}%
\pgfpathlineto{\pgfqpoint{2.335280in}{2.100100in}}%
\pgfpathlineto{\pgfqpoint{2.360271in}{2.113143in}}%
\pgfpathlineto{\pgfqpoint{2.385263in}{2.087056in}}%
\pgfpathlineto{\pgfqpoint{2.410255in}{2.060969in}}%
\pgfpathlineto{\pgfqpoint{2.435247in}{2.034882in}}%
\pgfpathlineto{\pgfqpoint{2.460239in}{2.015317in}}%
\pgfpathlineto{\pgfqpoint{2.485231in}{1.989230in}}%
\pgfpathlineto{\pgfqpoint{2.510222in}{1.963143in}}%
\pgfpathlineto{\pgfqpoint{2.535214in}{1.917491in}}%
\pgfpathlineto{\pgfqpoint{2.560206in}{1.904448in}}%
\pgfpathlineto{\pgfqpoint{2.585198in}{1.956622in}}%
\pgfpathlineto{\pgfqpoint{2.610190in}{2.008795in}}%
\pgfpathlineto{\pgfqpoint{2.635182in}{2.034882in}}%
\pgfpathlineto{\pgfqpoint{2.660173in}{2.047926in}}%
\pgfpathlineto{\pgfqpoint{2.685165in}{2.054448in}}%
\pgfpathlineto{\pgfqpoint{2.710157in}{2.080535in}}%
\pgfpathlineto{\pgfqpoint{2.735149in}{2.100100in}}%
\pgfpathlineto{\pgfqpoint{2.760141in}{2.126187in}}%
\pgfpathlineto{\pgfqpoint{2.785133in}{2.139230in}}%
\pgfpathlineto{\pgfqpoint{2.810124in}{2.145752in}}%
\pgfpathlineto{\pgfqpoint{2.835116in}{2.119665in}}%
\pgfpathlineto{\pgfqpoint{2.860108in}{2.139230in}}%
\pgfpathlineto{\pgfqpoint{2.885100in}{2.145752in}}%
\pgfpathlineto{\pgfqpoint{2.910092in}{2.152274in}}%
\pgfpathlineto{\pgfqpoint{2.935084in}{2.145752in}}%
\pgfpathlineto{\pgfqpoint{2.960075in}{2.119665in}}%
\pgfpathlineto{\pgfqpoint{2.985067in}{2.093578in}}%
\pgfpathlineto{\pgfqpoint{3.010059in}{2.074013in}}%
\pgfpathlineto{\pgfqpoint{3.035051in}{2.067491in}}%
\pgfpathlineto{\pgfqpoint{3.060043in}{2.067491in}}%
\pgfpathlineto{\pgfqpoint{3.085035in}{2.067491in}}%
\pgfpathlineto{\pgfqpoint{3.110026in}{2.067491in}}%
\pgfpathlineto{\pgfqpoint{3.135018in}{2.060969in}}%
\pgfpathlineto{\pgfqpoint{3.160010in}{2.060969in}}%
\pgfpathlineto{\pgfqpoint{3.185002in}{2.067491in}}%
\pgfpathlineto{\pgfqpoint{3.209994in}{2.060969in}}%
\pgfpathlineto{\pgfqpoint{3.234986in}{2.067491in}}%
\pgfpathlineto{\pgfqpoint{3.259977in}{2.060969in}}%
\pgfpathlineto{\pgfqpoint{3.284969in}{2.060969in}}%
\pgfpathlineto{\pgfqpoint{3.309961in}{2.067491in}}%
\pgfpathlineto{\pgfqpoint{3.334953in}{2.100100in}}%
\pgfpathlineto{\pgfqpoint{3.359945in}{2.119665in}}%
\pgfpathlineto{\pgfqpoint{3.384937in}{2.126187in}}%
\pgfpathlineto{\pgfqpoint{3.395057in}{2.118264in}}%
\pgfusepath{stroke}%
\end{pgfscope}%
\begin{pgfscope}%
\pgfpathrectangle{\pgfqpoint{2.161373in}{1.689230in}}{\pgfqpoint{1.223684in}{1.004348in}}%
\pgfusepath{clip}%
\pgfsetrectcap%
\pgfsetroundjoin%
\pgfsetlinewidth{0.803000pt}%
\definecolor{currentstroke}{rgb}{0.490196,0.588235,0.431373}%
\pgfsetstrokecolor{currentstroke}%
\pgfsetdash{}{0pt}%
\pgfpathmoveto{\pgfqpoint{2.210320in}{2.230535in}}%
\pgfpathlineto{\pgfqpoint{2.235312in}{2.230535in}}%
\pgfpathlineto{\pgfqpoint{2.260304in}{2.197926in}}%
\pgfpathlineto{\pgfqpoint{2.285296in}{2.184882in}}%
\pgfpathlineto{\pgfqpoint{2.310288in}{2.106622in}}%
\pgfpathlineto{\pgfqpoint{2.335280in}{2.021839in}}%
\pgfpathlineto{\pgfqpoint{2.360271in}{2.002274in}}%
\pgfpathlineto{\pgfqpoint{2.385263in}{2.067491in}}%
\pgfpathlineto{\pgfqpoint{2.410255in}{2.152274in}}%
\pgfpathlineto{\pgfqpoint{2.435247in}{2.230535in}}%
\pgfpathlineto{\pgfqpoint{2.460239in}{2.315317in}}%
\pgfpathlineto{\pgfqpoint{2.485231in}{2.347926in}}%
\pgfpathlineto{\pgfqpoint{2.510222in}{2.295752in}}%
\pgfpathlineto{\pgfqpoint{2.535214in}{2.256622in}}%
\pgfpathlineto{\pgfqpoint{2.560206in}{2.269665in}}%
\pgfpathlineto{\pgfqpoint{2.585198in}{2.276187in}}%
\pgfpathlineto{\pgfqpoint{2.610190in}{2.276187in}}%
\pgfpathlineto{\pgfqpoint{2.635182in}{2.276187in}}%
\pgfpathlineto{\pgfqpoint{2.660173in}{2.256622in}}%
\pgfpathlineto{\pgfqpoint{2.685165in}{2.256622in}}%
\pgfpathlineto{\pgfqpoint{2.710157in}{2.250100in}}%
\pgfpathlineto{\pgfqpoint{2.735149in}{2.243578in}}%
\pgfpathlineto{\pgfqpoint{2.760141in}{2.243578in}}%
\pgfpathlineto{\pgfqpoint{2.785133in}{2.237056in}}%
\pgfpathlineto{\pgfqpoint{2.810124in}{2.237056in}}%
\pgfpathlineto{\pgfqpoint{2.835116in}{2.237056in}}%
\pgfpathlineto{\pgfqpoint{2.860108in}{2.230535in}}%
\pgfpathlineto{\pgfqpoint{2.885100in}{2.230535in}}%
\pgfpathlineto{\pgfqpoint{2.910092in}{2.230535in}}%
\pgfpathlineto{\pgfqpoint{2.935084in}{2.230535in}}%
\pgfpathlineto{\pgfqpoint{2.960075in}{2.230535in}}%
\pgfpathlineto{\pgfqpoint{2.985067in}{2.224013in}}%
\pgfpathlineto{\pgfqpoint{3.010059in}{2.224013in}}%
\pgfpathlineto{\pgfqpoint{3.035051in}{2.224013in}}%
\pgfpathlineto{\pgfqpoint{3.060043in}{2.224013in}}%
\pgfpathlineto{\pgfqpoint{3.085035in}{2.224013in}}%
\pgfpathlineto{\pgfqpoint{3.110026in}{2.224013in}}%
\pgfpathlineto{\pgfqpoint{3.135018in}{2.224013in}}%
\pgfpathlineto{\pgfqpoint{3.160010in}{2.224013in}}%
\pgfpathlineto{\pgfqpoint{3.185002in}{2.224013in}}%
\pgfpathlineto{\pgfqpoint{3.209994in}{2.217491in}}%
\pgfpathlineto{\pgfqpoint{3.234986in}{2.217491in}}%
\pgfpathlineto{\pgfqpoint{3.259977in}{2.217491in}}%
\pgfpathlineto{\pgfqpoint{3.284969in}{2.210969in}}%
\pgfpathlineto{\pgfqpoint{3.309961in}{2.217491in}}%
\pgfpathlineto{\pgfqpoint{3.334953in}{2.217491in}}%
\pgfpathlineto{\pgfqpoint{3.359945in}{2.217491in}}%
\pgfpathlineto{\pgfqpoint{3.384937in}{2.217491in}}%
\pgfpathlineto{\pgfqpoint{3.395057in}{2.217491in}}%
\pgfusepath{stroke}%
\end{pgfscope}%
\begin{pgfscope}%
\pgfpathrectangle{\pgfqpoint{2.161373in}{1.689230in}}{\pgfqpoint{1.223684in}{1.004348in}}%
\pgfusepath{clip}%
\pgfsetrectcap%
\pgfsetroundjoin%
\pgfsetlinewidth{0.803000pt}%
\definecolor{currentstroke}{rgb}{0.843137,0.666667,0.313725}%
\pgfsetstrokecolor{currentstroke}%
\pgfsetdash{}{0pt}%
\pgfpathmoveto{\pgfqpoint{2.210320in}{2.276187in}}%
\pgfpathlineto{\pgfqpoint{2.235312in}{2.269665in}}%
\pgfpathlineto{\pgfqpoint{2.260304in}{2.263143in}}%
\pgfpathlineto{\pgfqpoint{2.285296in}{2.204448in}}%
\pgfpathlineto{\pgfqpoint{2.310288in}{2.171839in}}%
\pgfpathlineto{\pgfqpoint{2.335280in}{2.152274in}}%
\pgfpathlineto{\pgfqpoint{2.360271in}{2.165317in}}%
\pgfpathlineto{\pgfqpoint{2.385263in}{2.158795in}}%
\pgfpathlineto{\pgfqpoint{2.410255in}{2.158795in}}%
\pgfpathlineto{\pgfqpoint{2.435247in}{2.204448in}}%
\pgfpathlineto{\pgfqpoint{2.460239in}{2.237056in}}%
\pgfpathlineto{\pgfqpoint{2.485231in}{2.217491in}}%
\pgfpathlineto{\pgfqpoint{2.510222in}{2.191404in}}%
\pgfpathlineto{\pgfqpoint{2.535214in}{2.165317in}}%
\pgfpathlineto{\pgfqpoint{2.560206in}{2.217491in}}%
\pgfpathlineto{\pgfqpoint{2.585198in}{2.295752in}}%
\pgfpathlineto{\pgfqpoint{2.610190in}{2.321839in}}%
\pgfpathlineto{\pgfqpoint{2.635182in}{2.328361in}}%
\pgfpathlineto{\pgfqpoint{2.660173in}{2.321839in}}%
\pgfpathlineto{\pgfqpoint{2.685165in}{2.315317in}}%
\pgfpathlineto{\pgfqpoint{2.710157in}{2.302274in}}%
\pgfpathlineto{\pgfqpoint{2.735149in}{2.302274in}}%
\pgfpathlineto{\pgfqpoint{2.760141in}{2.295752in}}%
\pgfpathlineto{\pgfqpoint{2.785133in}{2.295752in}}%
\pgfpathlineto{\pgfqpoint{2.810124in}{2.289230in}}%
\pgfpathlineto{\pgfqpoint{2.835116in}{2.295752in}}%
\pgfpathlineto{\pgfqpoint{2.860108in}{2.289230in}}%
\pgfpathlineto{\pgfqpoint{2.885100in}{2.282708in}}%
\pgfpathlineto{\pgfqpoint{2.910092in}{2.282708in}}%
\pgfpathlineto{\pgfqpoint{2.935084in}{2.289230in}}%
\pgfpathlineto{\pgfqpoint{2.960075in}{2.289230in}}%
\pgfpathlineto{\pgfqpoint{2.985067in}{2.295752in}}%
\pgfpathlineto{\pgfqpoint{3.010059in}{2.289230in}}%
\pgfpathlineto{\pgfqpoint{3.035051in}{2.289230in}}%
\pgfpathlineto{\pgfqpoint{3.060043in}{2.282708in}}%
\pgfpathlineto{\pgfqpoint{3.085035in}{2.289230in}}%
\pgfpathlineto{\pgfqpoint{3.110026in}{2.282708in}}%
\pgfpathlineto{\pgfqpoint{3.135018in}{2.269665in}}%
\pgfpathlineto{\pgfqpoint{3.160010in}{2.276187in}}%
\pgfpathlineto{\pgfqpoint{3.185002in}{2.276187in}}%
\pgfpathlineto{\pgfqpoint{3.209994in}{2.269665in}}%
\pgfpathlineto{\pgfqpoint{3.234986in}{2.282708in}}%
\pgfpathlineto{\pgfqpoint{3.259977in}{2.289230in}}%
\pgfpathlineto{\pgfqpoint{3.284969in}{2.282708in}}%
\pgfpathlineto{\pgfqpoint{3.309961in}{2.282708in}}%
\pgfpathlineto{\pgfqpoint{3.334953in}{2.276187in}}%
\pgfpathlineto{\pgfqpoint{3.359945in}{2.276187in}}%
\pgfpathlineto{\pgfqpoint{3.384937in}{2.282708in}}%
\pgfpathlineto{\pgfqpoint{3.395057in}{2.282708in}}%
\pgfusepath{stroke}%
\end{pgfscope}%
\begin{pgfscope}%
\pgfpathrectangle{\pgfqpoint{2.161373in}{1.689230in}}{\pgfqpoint{1.223684in}{1.004348in}}%
\pgfusepath{clip}%
\pgfsetrectcap%
\pgfsetroundjoin%
\pgfsetlinewidth{0.803000pt}%
\definecolor{currentstroke}{rgb}{0.333333,0.333333,0.333333}%
\pgfsetstrokecolor{currentstroke}%
\pgfsetdash{}{0pt}%
\pgfpathmoveto{\pgfqpoint{2.210320in}{2.256622in}}%
\pgfpathlineto{\pgfqpoint{2.235312in}{2.250100in}}%
\pgfpathlineto{\pgfqpoint{2.260304in}{2.276187in}}%
\pgfpathlineto{\pgfqpoint{2.285296in}{2.393578in}}%
\pgfpathlineto{\pgfqpoint{2.310288in}{2.491404in}}%
\pgfpathlineto{\pgfqpoint{2.335280in}{2.569665in}}%
\pgfpathlineto{\pgfqpoint{2.360271in}{2.100100in}}%
\pgfpathlineto{\pgfqpoint{2.385263in}{2.008795in}}%
\pgfpathlineto{\pgfqpoint{2.410255in}{2.002274in}}%
\pgfpathlineto{\pgfqpoint{2.435247in}{1.995752in}}%
\pgfpathlineto{\pgfqpoint{2.460239in}{2.002274in}}%
\pgfpathlineto{\pgfqpoint{2.485231in}{2.002274in}}%
\pgfpathlineto{\pgfqpoint{2.510222in}{2.008795in}}%
\pgfpathlineto{\pgfqpoint{2.535214in}{2.002274in}}%
\pgfpathlineto{\pgfqpoint{2.560206in}{1.995752in}}%
\pgfpathlineto{\pgfqpoint{2.585198in}{1.995752in}}%
\pgfpathlineto{\pgfqpoint{2.610190in}{2.002274in}}%
\pgfpathlineto{\pgfqpoint{2.635182in}{2.002274in}}%
\pgfpathlineto{\pgfqpoint{2.660173in}{2.002274in}}%
\pgfpathlineto{\pgfqpoint{2.685165in}{2.015317in}}%
\pgfpathlineto{\pgfqpoint{2.710157in}{2.060969in}}%
\pgfpathlineto{\pgfqpoint{2.735149in}{2.100100in}}%
\pgfpathlineto{\pgfqpoint{2.760141in}{2.132708in}}%
\pgfpathlineto{\pgfqpoint{2.785133in}{2.158795in}}%
\pgfpathlineto{\pgfqpoint{2.810124in}{2.178361in}}%
\pgfpathlineto{\pgfqpoint{2.835116in}{2.197926in}}%
\pgfpathlineto{\pgfqpoint{2.860108in}{2.210969in}}%
\pgfpathlineto{\pgfqpoint{2.885100in}{2.217491in}}%
\pgfpathlineto{\pgfqpoint{2.910092in}{2.224013in}}%
\pgfpathlineto{\pgfqpoint{2.935084in}{2.237056in}}%
\pgfpathlineto{\pgfqpoint{2.960075in}{2.237056in}}%
\pgfpathlineto{\pgfqpoint{2.985067in}{2.243578in}}%
\pgfpathlineto{\pgfqpoint{3.010059in}{2.243578in}}%
\pgfpathlineto{\pgfqpoint{3.035051in}{2.243578in}}%
\pgfpathlineto{\pgfqpoint{3.060043in}{2.243578in}}%
\pgfpathlineto{\pgfqpoint{3.085035in}{2.243578in}}%
\pgfpathlineto{\pgfqpoint{3.110026in}{2.243578in}}%
\pgfpathlineto{\pgfqpoint{3.135018in}{2.243578in}}%
\pgfpathlineto{\pgfqpoint{3.160010in}{2.243578in}}%
\pgfpathlineto{\pgfqpoint{3.185002in}{2.243578in}}%
\pgfpathlineto{\pgfqpoint{3.209994in}{2.243578in}}%
\pgfpathlineto{\pgfqpoint{3.234986in}{2.243578in}}%
\pgfpathlineto{\pgfqpoint{3.259977in}{2.243578in}}%
\pgfpathlineto{\pgfqpoint{3.284969in}{2.243578in}}%
\pgfpathlineto{\pgfqpoint{3.309961in}{2.250100in}}%
\pgfpathlineto{\pgfqpoint{3.334953in}{2.243578in}}%
\pgfpathlineto{\pgfqpoint{3.359945in}{2.250100in}}%
\pgfpathlineto{\pgfqpoint{3.384937in}{2.250100in}}%
\pgfpathlineto{\pgfqpoint{3.395057in}{2.250100in}}%
\pgfusepath{stroke}%
\end{pgfscope}%
\begin{pgfscope}%
\pgfpathrectangle{\pgfqpoint{2.161373in}{1.689230in}}{\pgfqpoint{1.223684in}{1.004348in}}%
\pgfusepath{clip}%
\pgfsetrectcap%
\pgfsetroundjoin%
\pgfsetlinewidth{0.803000pt}%
\definecolor{currentstroke}{rgb}{0.686275,0.352941,0.313725}%
\pgfsetstrokecolor{currentstroke}%
\pgfsetdash{}{0pt}%
\pgfpathmoveto{\pgfqpoint{2.210320in}{2.315317in}}%
\pgfpathlineto{\pgfqpoint{2.235312in}{2.308795in}}%
\pgfpathlineto{\pgfqpoint{2.260304in}{2.276187in}}%
\pgfpathlineto{\pgfqpoint{2.285296in}{2.230535in}}%
\pgfpathlineto{\pgfqpoint{2.310288in}{2.321839in}}%
\pgfpathlineto{\pgfqpoint{2.335280in}{2.478361in}}%
\pgfpathlineto{\pgfqpoint{2.360271in}{2.615317in}}%
\pgfpathlineto{\pgfqpoint{2.385263in}{1.976187in}}%
\pgfpathlineto{\pgfqpoint{2.410255in}{1.976187in}}%
\pgfpathlineto{\pgfqpoint{2.435247in}{1.976187in}}%
\pgfpathlineto{\pgfqpoint{2.460239in}{1.969665in}}%
\pgfpathlineto{\pgfqpoint{2.485231in}{1.963143in}}%
\pgfpathlineto{\pgfqpoint{2.510222in}{1.950100in}}%
\pgfpathlineto{\pgfqpoint{2.535214in}{1.963143in}}%
\pgfpathlineto{\pgfqpoint{2.560206in}{1.956622in}}%
\pgfpathlineto{\pgfqpoint{2.585198in}{1.956622in}}%
\pgfpathlineto{\pgfqpoint{2.610190in}{1.956622in}}%
\pgfpathlineto{\pgfqpoint{2.635182in}{1.956622in}}%
\pgfpathlineto{\pgfqpoint{2.660173in}{1.956622in}}%
\pgfpathlineto{\pgfqpoint{2.685165in}{1.956622in}}%
\pgfpathlineto{\pgfqpoint{2.710157in}{2.041404in}}%
\pgfpathlineto{\pgfqpoint{2.735149in}{2.119665in}}%
\pgfpathlineto{\pgfqpoint{2.760141in}{2.171839in}}%
\pgfpathlineto{\pgfqpoint{2.785133in}{2.210969in}}%
\pgfpathlineto{\pgfqpoint{2.810124in}{2.230535in}}%
\pgfpathlineto{\pgfqpoint{2.835116in}{2.250100in}}%
\pgfpathlineto{\pgfqpoint{2.860108in}{2.269665in}}%
\pgfpathlineto{\pgfqpoint{2.885100in}{2.276187in}}%
\pgfpathlineto{\pgfqpoint{2.910092in}{2.289230in}}%
\pgfpathlineto{\pgfqpoint{2.935084in}{2.295752in}}%
\pgfpathlineto{\pgfqpoint{2.960075in}{2.289230in}}%
\pgfpathlineto{\pgfqpoint{2.985067in}{2.289230in}}%
\pgfpathlineto{\pgfqpoint{3.010059in}{2.295752in}}%
\pgfpathlineto{\pgfqpoint{3.035051in}{2.302274in}}%
\pgfpathlineto{\pgfqpoint{3.060043in}{2.302274in}}%
\pgfpathlineto{\pgfqpoint{3.085035in}{2.302274in}}%
\pgfpathlineto{\pgfqpoint{3.110026in}{2.302274in}}%
\pgfpathlineto{\pgfqpoint{3.135018in}{2.295752in}}%
\pgfpathlineto{\pgfqpoint{3.160010in}{2.302274in}}%
\pgfpathlineto{\pgfqpoint{3.185002in}{2.302274in}}%
\pgfpathlineto{\pgfqpoint{3.209994in}{2.302274in}}%
\pgfpathlineto{\pgfqpoint{3.234986in}{2.302274in}}%
\pgfpathlineto{\pgfqpoint{3.259977in}{2.295752in}}%
\pgfpathlineto{\pgfqpoint{3.284969in}{2.302274in}}%
\pgfpathlineto{\pgfqpoint{3.309961in}{2.302274in}}%
\pgfpathlineto{\pgfqpoint{3.334953in}{2.302274in}}%
\pgfpathlineto{\pgfqpoint{3.359945in}{2.302274in}}%
\pgfpathlineto{\pgfqpoint{3.384937in}{2.302274in}}%
\pgfpathlineto{\pgfqpoint{3.395057in}{2.302274in}}%
\pgfusepath{stroke}%
\end{pgfscope}%
\begin{pgfscope}%
\pgfpathrectangle{\pgfqpoint{2.161373in}{1.689230in}}{\pgfqpoint{1.223684in}{1.004348in}}%
\pgfusepath{clip}%
\pgfsetrectcap%
\pgfsetroundjoin%
\pgfsetlinewidth{0.803000pt}%
\definecolor{currentstroke}{rgb}{0.000000,0.356863,0.509804}%
\pgfsetstrokecolor{currentstroke}%
\pgfsetdash{}{0pt}%
\pgfpathmoveto{\pgfqpoint{2.210320in}{2.256622in}}%
\pgfpathlineto{\pgfqpoint{2.235312in}{2.256622in}}%
\pgfpathlineto{\pgfqpoint{2.260304in}{2.367491in}}%
\pgfpathlineto{\pgfqpoint{2.285296in}{2.576187in}}%
\pgfpathlineto{\pgfqpoint{2.310288in}{1.995752in}}%
\pgfpathlineto{\pgfqpoint{2.335280in}{1.989230in}}%
\pgfpathlineto{\pgfqpoint{2.360271in}{1.989230in}}%
\pgfpathlineto{\pgfqpoint{2.385263in}{1.995752in}}%
\pgfpathlineto{\pgfqpoint{2.410255in}{1.989230in}}%
\pgfpathlineto{\pgfqpoint{2.435247in}{1.976187in}}%
\pgfpathlineto{\pgfqpoint{2.460239in}{1.989230in}}%
\pgfpathlineto{\pgfqpoint{2.485231in}{1.982708in}}%
\pgfpathlineto{\pgfqpoint{2.510222in}{1.989230in}}%
\pgfpathlineto{\pgfqpoint{2.535214in}{1.982708in}}%
\pgfpathlineto{\pgfqpoint{2.560206in}{1.989230in}}%
\pgfpathlineto{\pgfqpoint{2.585198in}{1.976187in}}%
\pgfpathlineto{\pgfqpoint{2.610190in}{1.976187in}}%
\pgfpathlineto{\pgfqpoint{2.635182in}{2.054448in}}%
\pgfpathlineto{\pgfqpoint{2.660173in}{2.119665in}}%
\pgfpathlineto{\pgfqpoint{2.685165in}{2.165317in}}%
\pgfpathlineto{\pgfqpoint{2.710157in}{2.197926in}}%
\pgfpathlineto{\pgfqpoint{2.735149in}{2.210969in}}%
\pgfpathlineto{\pgfqpoint{2.760141in}{2.230535in}}%
\pgfpathlineto{\pgfqpoint{2.785133in}{2.243578in}}%
\pgfpathlineto{\pgfqpoint{2.810124in}{2.250100in}}%
\pgfpathlineto{\pgfqpoint{2.835116in}{2.256622in}}%
\pgfpathlineto{\pgfqpoint{2.860108in}{2.263143in}}%
\pgfpathlineto{\pgfqpoint{2.885100in}{2.269665in}}%
\pgfpathlineto{\pgfqpoint{2.910092in}{2.276187in}}%
\pgfpathlineto{\pgfqpoint{2.935084in}{2.276187in}}%
\pgfpathlineto{\pgfqpoint{2.960075in}{2.282708in}}%
\pgfpathlineto{\pgfqpoint{2.985067in}{2.282708in}}%
\pgfpathlineto{\pgfqpoint{3.010059in}{2.282708in}}%
\pgfpathlineto{\pgfqpoint{3.035051in}{2.282708in}}%
\pgfpathlineto{\pgfqpoint{3.060043in}{2.282708in}}%
\pgfpathlineto{\pgfqpoint{3.085035in}{2.282708in}}%
\pgfpathlineto{\pgfqpoint{3.110026in}{2.282708in}}%
\pgfpathlineto{\pgfqpoint{3.135018in}{2.276187in}}%
\pgfpathlineto{\pgfqpoint{3.160010in}{2.282708in}}%
\pgfpathlineto{\pgfqpoint{3.185002in}{2.282708in}}%
\pgfpathlineto{\pgfqpoint{3.209994in}{2.282708in}}%
\pgfpathlineto{\pgfqpoint{3.234986in}{2.282708in}}%
\pgfpathlineto{\pgfqpoint{3.259977in}{2.276187in}}%
\pgfpathlineto{\pgfqpoint{3.284969in}{2.276187in}}%
\pgfpathlineto{\pgfqpoint{3.309961in}{2.282708in}}%
\pgfpathlineto{\pgfqpoint{3.334953in}{2.282708in}}%
\pgfpathlineto{\pgfqpoint{3.359945in}{2.289230in}}%
\pgfpathlineto{\pgfqpoint{3.384937in}{2.282708in}}%
\pgfpathlineto{\pgfqpoint{3.395057in}{2.282708in}}%
\pgfusepath{stroke}%
\end{pgfscope}%
\begin{pgfscope}%
\pgfpathrectangle{\pgfqpoint{2.161373in}{1.689230in}}{\pgfqpoint{1.223684in}{1.004348in}}%
\pgfusepath{clip}%
\pgfsetrectcap%
\pgfsetroundjoin%
\pgfsetlinewidth{0.803000pt}%
\definecolor{currentstroke}{rgb}{0.490196,0.588235,0.431373}%
\pgfsetstrokecolor{currentstroke}%
\pgfsetdash{}{0pt}%
\pgfpathmoveto{\pgfqpoint{2.210320in}{2.145752in}}%
\pgfpathlineto{\pgfqpoint{2.235312in}{2.139230in}}%
\pgfpathlineto{\pgfqpoint{2.260304in}{2.145752in}}%
\pgfpathlineto{\pgfqpoint{2.285296in}{2.145752in}}%
\pgfpathlineto{\pgfqpoint{2.310288in}{2.126187in}}%
\pgfpathlineto{\pgfqpoint{2.335280in}{2.008795in}}%
\pgfpathlineto{\pgfqpoint{2.360271in}{1.871839in}}%
\pgfpathlineto{\pgfqpoint{2.385263in}{1.839230in}}%
\pgfpathlineto{\pgfqpoint{2.410255in}{1.904448in}}%
\pgfpathlineto{\pgfqpoint{2.435247in}{1.982708in}}%
\pgfpathlineto{\pgfqpoint{2.460239in}{2.008795in}}%
\pgfpathlineto{\pgfqpoint{2.485231in}{1.943578in}}%
\pgfpathlineto{\pgfqpoint{2.510222in}{1.858795in}}%
\pgfpathlineto{\pgfqpoint{2.535214in}{1.845752in}}%
\pgfpathlineto{\pgfqpoint{2.560206in}{1.897926in}}%
\pgfpathlineto{\pgfqpoint{2.585198in}{1.904448in}}%
\pgfpathlineto{\pgfqpoint{2.610190in}{1.904448in}}%
\pgfpathlineto{\pgfqpoint{2.635182in}{1.930535in}}%
\pgfpathlineto{\pgfqpoint{2.660173in}{1.956622in}}%
\pgfpathlineto{\pgfqpoint{2.685165in}{1.989230in}}%
\pgfpathlineto{\pgfqpoint{2.710157in}{2.021839in}}%
\pgfpathlineto{\pgfqpoint{2.735149in}{2.041404in}}%
\pgfpathlineto{\pgfqpoint{2.760141in}{2.067491in}}%
\pgfpathlineto{\pgfqpoint{2.785133in}{2.080535in}}%
\pgfpathlineto{\pgfqpoint{2.810124in}{2.093578in}}%
\pgfpathlineto{\pgfqpoint{2.835116in}{2.106622in}}%
\pgfpathlineto{\pgfqpoint{2.860108in}{2.119665in}}%
\pgfpathlineto{\pgfqpoint{2.885100in}{2.126187in}}%
\pgfpathlineto{\pgfqpoint{2.910092in}{2.126187in}}%
\pgfpathlineto{\pgfqpoint{2.935084in}{2.139230in}}%
\pgfpathlineto{\pgfqpoint{2.960075in}{2.132708in}}%
\pgfpathlineto{\pgfqpoint{2.985067in}{2.139230in}}%
\pgfpathlineto{\pgfqpoint{3.010059in}{2.139230in}}%
\pgfpathlineto{\pgfqpoint{3.035051in}{2.145752in}}%
\pgfpathlineto{\pgfqpoint{3.060043in}{2.145752in}}%
\pgfpathlineto{\pgfqpoint{3.085035in}{2.145752in}}%
\pgfpathlineto{\pgfqpoint{3.110026in}{2.145752in}}%
\pgfpathlineto{\pgfqpoint{3.135018in}{2.145752in}}%
\pgfpathlineto{\pgfqpoint{3.160010in}{2.152274in}}%
\pgfpathlineto{\pgfqpoint{3.185002in}{2.145752in}}%
\pgfpathlineto{\pgfqpoint{3.209994in}{2.152274in}}%
\pgfpathlineto{\pgfqpoint{3.234986in}{2.152274in}}%
\pgfpathlineto{\pgfqpoint{3.259977in}{2.152274in}}%
\pgfpathlineto{\pgfqpoint{3.284969in}{2.145752in}}%
\pgfpathlineto{\pgfqpoint{3.309961in}{2.152274in}}%
\pgfpathlineto{\pgfqpoint{3.334953in}{2.145752in}}%
\pgfpathlineto{\pgfqpoint{3.359945in}{2.152274in}}%
\pgfpathlineto{\pgfqpoint{3.384937in}{2.152274in}}%
\pgfpathlineto{\pgfqpoint{3.395057in}{2.152274in}}%
\pgfusepath{stroke}%
\end{pgfscope}%
\begin{pgfscope}%
\pgfpathrectangle{\pgfqpoint{2.161373in}{1.689230in}}{\pgfqpoint{1.223684in}{1.004348in}}%
\pgfusepath{clip}%
\pgfsetrectcap%
\pgfsetroundjoin%
\pgfsetlinewidth{0.803000pt}%
\definecolor{currentstroke}{rgb}{0.843137,0.666667,0.313725}%
\pgfsetstrokecolor{currentstroke}%
\pgfsetdash{}{0pt}%
\pgfpathmoveto{\pgfqpoint{2.210320in}{2.210969in}}%
\pgfpathlineto{\pgfqpoint{2.235312in}{2.210969in}}%
\pgfpathlineto{\pgfqpoint{2.260304in}{2.250100in}}%
\pgfpathlineto{\pgfqpoint{2.285296in}{2.308795in}}%
\pgfpathlineto{\pgfqpoint{2.310288in}{2.530535in}}%
\pgfpathlineto{\pgfqpoint{2.335280in}{2.647926in}}%
\pgfpathlineto{\pgfqpoint{2.360271in}{2.008795in}}%
\pgfpathlineto{\pgfqpoint{2.385263in}{2.008795in}}%
\pgfpathlineto{\pgfqpoint{2.410255in}{2.008795in}}%
\pgfpathlineto{\pgfqpoint{2.435247in}{2.002274in}}%
\pgfpathlineto{\pgfqpoint{2.460239in}{2.002274in}}%
\pgfpathlineto{\pgfqpoint{2.485231in}{1.995752in}}%
\pgfpathlineto{\pgfqpoint{2.510222in}{2.002274in}}%
\pgfpathlineto{\pgfqpoint{2.535214in}{1.995752in}}%
\pgfpathlineto{\pgfqpoint{2.560206in}{2.008795in}}%
\pgfpathlineto{\pgfqpoint{2.585198in}{2.002274in}}%
\pgfpathlineto{\pgfqpoint{2.610190in}{1.995752in}}%
\pgfpathlineto{\pgfqpoint{2.635182in}{1.995752in}}%
\pgfpathlineto{\pgfqpoint{2.660173in}{1.989230in}}%
\pgfpathlineto{\pgfqpoint{2.685165in}{2.074013in}}%
\pgfpathlineto{\pgfqpoint{2.710157in}{2.126187in}}%
\pgfpathlineto{\pgfqpoint{2.735149in}{2.152274in}}%
\pgfpathlineto{\pgfqpoint{2.760141in}{2.171839in}}%
\pgfpathlineto{\pgfqpoint{2.785133in}{2.191404in}}%
\pgfpathlineto{\pgfqpoint{2.810124in}{2.197926in}}%
\pgfpathlineto{\pgfqpoint{2.835116in}{2.197926in}}%
\pgfpathlineto{\pgfqpoint{2.860108in}{2.204448in}}%
\pgfpathlineto{\pgfqpoint{2.885100in}{2.210969in}}%
\pgfpathlineto{\pgfqpoint{2.910092in}{2.210969in}}%
\pgfpathlineto{\pgfqpoint{2.935084in}{2.217491in}}%
\pgfpathlineto{\pgfqpoint{2.960075in}{2.217491in}}%
\pgfpathlineto{\pgfqpoint{2.985067in}{2.217491in}}%
\pgfpathlineto{\pgfqpoint{3.010059in}{2.217491in}}%
\pgfpathlineto{\pgfqpoint{3.035051in}{2.217491in}}%
\pgfpathlineto{\pgfqpoint{3.060043in}{2.217491in}}%
\pgfpathlineto{\pgfqpoint{3.085035in}{2.217491in}}%
\pgfpathlineto{\pgfqpoint{3.110026in}{2.217491in}}%
\pgfpathlineto{\pgfqpoint{3.135018in}{2.210969in}}%
\pgfpathlineto{\pgfqpoint{3.160010in}{2.210969in}}%
\pgfpathlineto{\pgfqpoint{3.185002in}{2.210969in}}%
\pgfpathlineto{\pgfqpoint{3.209994in}{2.210969in}}%
\pgfpathlineto{\pgfqpoint{3.234986in}{2.210969in}}%
\pgfpathlineto{\pgfqpoint{3.259977in}{2.210969in}}%
\pgfpathlineto{\pgfqpoint{3.284969in}{2.204448in}}%
\pgfpathlineto{\pgfqpoint{3.309961in}{2.204448in}}%
\pgfpathlineto{\pgfqpoint{3.334953in}{2.204448in}}%
\pgfpathlineto{\pgfqpoint{3.359945in}{2.204448in}}%
\pgfpathlineto{\pgfqpoint{3.384937in}{2.210969in}}%
\pgfpathlineto{\pgfqpoint{3.395057in}{2.210969in}}%
\pgfusepath{stroke}%
\end{pgfscope}%
\begin{pgfscope}%
\pgfpathrectangle{\pgfqpoint{2.161373in}{1.689230in}}{\pgfqpoint{1.223684in}{1.004348in}}%
\pgfusepath{clip}%
\pgfsetrectcap%
\pgfsetroundjoin%
\pgfsetlinewidth{0.803000pt}%
\definecolor{currentstroke}{rgb}{0.333333,0.333333,0.333333}%
\pgfsetstrokecolor{currentstroke}%
\pgfsetdash{}{0pt}%
\pgfpathmoveto{\pgfqpoint{2.210320in}{2.224013in}}%
\pgfpathlineto{\pgfqpoint{2.235312in}{2.230535in}}%
\pgfpathlineto{\pgfqpoint{2.260304in}{2.145752in}}%
\pgfpathlineto{\pgfqpoint{2.285296in}{2.087056in}}%
\pgfpathlineto{\pgfqpoint{2.310288in}{2.028361in}}%
\pgfpathlineto{\pgfqpoint{2.335280in}{1.995752in}}%
\pgfpathlineto{\pgfqpoint{2.360271in}{2.021839in}}%
\pgfpathlineto{\pgfqpoint{2.385263in}{2.047926in}}%
\pgfpathlineto{\pgfqpoint{2.410255in}{2.113143in}}%
\pgfpathlineto{\pgfqpoint{2.435247in}{2.113143in}}%
\pgfpathlineto{\pgfqpoint{2.460239in}{2.132708in}}%
\pgfpathlineto{\pgfqpoint{2.485231in}{2.158795in}}%
\pgfpathlineto{\pgfqpoint{2.510222in}{2.119665in}}%
\pgfpathlineto{\pgfqpoint{2.535214in}{2.067491in}}%
\pgfpathlineto{\pgfqpoint{2.560206in}{2.041404in}}%
\pgfpathlineto{\pgfqpoint{2.585198in}{2.021839in}}%
\pgfpathlineto{\pgfqpoint{2.610190in}{2.021839in}}%
\pgfpathlineto{\pgfqpoint{2.635182in}{2.047926in}}%
\pgfpathlineto{\pgfqpoint{2.660173in}{2.074013in}}%
\pgfpathlineto{\pgfqpoint{2.685165in}{2.100100in}}%
\pgfpathlineto{\pgfqpoint{2.710157in}{2.126187in}}%
\pgfpathlineto{\pgfqpoint{2.735149in}{2.145752in}}%
\pgfpathlineto{\pgfqpoint{2.760141in}{2.165317in}}%
\pgfpathlineto{\pgfqpoint{2.785133in}{2.184882in}}%
\pgfpathlineto{\pgfqpoint{2.810124in}{2.191404in}}%
\pgfpathlineto{\pgfqpoint{2.835116in}{2.204448in}}%
\pgfpathlineto{\pgfqpoint{2.860108in}{2.210969in}}%
\pgfpathlineto{\pgfqpoint{2.885100in}{2.217491in}}%
\pgfpathlineto{\pgfqpoint{2.910092in}{2.217491in}}%
\pgfpathlineto{\pgfqpoint{2.935084in}{2.224013in}}%
\pgfpathlineto{\pgfqpoint{2.960075in}{2.224013in}}%
\pgfpathlineto{\pgfqpoint{2.985067in}{2.230535in}}%
\pgfpathlineto{\pgfqpoint{3.010059in}{2.230535in}}%
\pgfpathlineto{\pgfqpoint{3.035051in}{2.230535in}}%
\pgfpathlineto{\pgfqpoint{3.060043in}{2.217491in}}%
\pgfpathlineto{\pgfqpoint{3.085035in}{2.210969in}}%
\pgfpathlineto{\pgfqpoint{3.110026in}{2.204448in}}%
\pgfpathlineto{\pgfqpoint{3.135018in}{2.197926in}}%
\pgfpathlineto{\pgfqpoint{3.160010in}{2.197926in}}%
\pgfpathlineto{\pgfqpoint{3.185002in}{2.191404in}}%
\pgfpathlineto{\pgfqpoint{3.209994in}{2.191404in}}%
\pgfpathlineto{\pgfqpoint{3.234986in}{2.191404in}}%
\pgfpathlineto{\pgfqpoint{3.259977in}{2.191404in}}%
\pgfpathlineto{\pgfqpoint{3.284969in}{2.184882in}}%
\pgfpathlineto{\pgfqpoint{3.309961in}{2.184882in}}%
\pgfpathlineto{\pgfqpoint{3.334953in}{2.178361in}}%
\pgfpathlineto{\pgfqpoint{3.359945in}{2.178361in}}%
\pgfpathlineto{\pgfqpoint{3.384937in}{2.178361in}}%
\pgfpathlineto{\pgfqpoint{3.395057in}{2.178361in}}%
\pgfusepath{stroke}%
\end{pgfscope}%
\begin{pgfscope}%
\pgfpathrectangle{\pgfqpoint{2.161373in}{1.689230in}}{\pgfqpoint{1.223684in}{1.004348in}}%
\pgfusepath{clip}%
\pgfsetrectcap%
\pgfsetroundjoin%
\pgfsetlinewidth{0.803000pt}%
\definecolor{currentstroke}{rgb}{0.686275,0.352941,0.313725}%
\pgfsetstrokecolor{currentstroke}%
\pgfsetdash{}{0pt}%
\pgfpathmoveto{\pgfqpoint{2.210320in}{2.139230in}}%
\pgfpathlineto{\pgfqpoint{2.235312in}{2.139230in}}%
\pgfpathlineto{\pgfqpoint{2.260304in}{2.152274in}}%
\pgfpathlineto{\pgfqpoint{2.285296in}{2.178361in}}%
\pgfpathlineto{\pgfqpoint{2.310288in}{2.250100in}}%
\pgfpathlineto{\pgfqpoint{2.335280in}{2.308795in}}%
\pgfpathlineto{\pgfqpoint{2.360271in}{2.230535in}}%
\pgfpathlineto{\pgfqpoint{2.385263in}{2.165317in}}%
\pgfpathlineto{\pgfqpoint{2.410255in}{2.060969in}}%
\pgfpathlineto{\pgfqpoint{2.435247in}{1.950100in}}%
\pgfpathlineto{\pgfqpoint{2.460239in}{1.930535in}}%
\pgfpathlineto{\pgfqpoint{2.485231in}{1.950100in}}%
\pgfpathlineto{\pgfqpoint{2.510222in}{2.041404in}}%
\pgfpathlineto{\pgfqpoint{2.535214in}{2.145752in}}%
\pgfpathlineto{\pgfqpoint{2.560206in}{2.210969in}}%
\pgfpathlineto{\pgfqpoint{2.585198in}{2.230535in}}%
\pgfpathlineto{\pgfqpoint{2.610190in}{2.224013in}}%
\pgfpathlineto{\pgfqpoint{2.635182in}{2.217491in}}%
\pgfpathlineto{\pgfqpoint{2.660173in}{2.197926in}}%
\pgfpathlineto{\pgfqpoint{2.685165in}{2.197926in}}%
\pgfpathlineto{\pgfqpoint{2.710157in}{2.184882in}}%
\pgfpathlineto{\pgfqpoint{2.735149in}{2.171839in}}%
\pgfpathlineto{\pgfqpoint{2.760141in}{2.165317in}}%
\pgfpathlineto{\pgfqpoint{2.785133in}{2.165317in}}%
\pgfpathlineto{\pgfqpoint{2.810124in}{2.158795in}}%
\pgfpathlineto{\pgfqpoint{2.835116in}{2.152274in}}%
\pgfpathlineto{\pgfqpoint{2.860108in}{2.152274in}}%
\pgfpathlineto{\pgfqpoint{2.885100in}{2.152274in}}%
\pgfpathlineto{\pgfqpoint{2.910092in}{2.152274in}}%
\pgfpathlineto{\pgfqpoint{2.935084in}{2.152274in}}%
\pgfpathlineto{\pgfqpoint{2.960075in}{2.145752in}}%
\pgfpathlineto{\pgfqpoint{2.985067in}{2.145752in}}%
\pgfpathlineto{\pgfqpoint{3.010059in}{2.145752in}}%
\pgfpathlineto{\pgfqpoint{3.035051in}{2.145752in}}%
\pgfpathlineto{\pgfqpoint{3.060043in}{2.145752in}}%
\pgfpathlineto{\pgfqpoint{3.085035in}{2.145752in}}%
\pgfpathlineto{\pgfqpoint{3.110026in}{2.145752in}}%
\pgfpathlineto{\pgfqpoint{3.135018in}{2.145752in}}%
\pgfpathlineto{\pgfqpoint{3.160010in}{2.145752in}}%
\pgfpathlineto{\pgfqpoint{3.185002in}{2.145752in}}%
\pgfpathlineto{\pgfqpoint{3.209994in}{2.145752in}}%
\pgfpathlineto{\pgfqpoint{3.234986in}{2.145752in}}%
\pgfpathlineto{\pgfqpoint{3.259977in}{2.145752in}}%
\pgfpathlineto{\pgfqpoint{3.284969in}{2.145752in}}%
\pgfpathlineto{\pgfqpoint{3.309961in}{2.145752in}}%
\pgfpathlineto{\pgfqpoint{3.334953in}{2.145752in}}%
\pgfpathlineto{\pgfqpoint{3.359945in}{2.145752in}}%
\pgfpathlineto{\pgfqpoint{3.384937in}{2.139230in}}%
\pgfpathlineto{\pgfqpoint{3.395057in}{2.139230in}}%
\pgfusepath{stroke}%
\end{pgfscope}%
\begin{pgfscope}%
\pgfpathrectangle{\pgfqpoint{2.161373in}{1.689230in}}{\pgfqpoint{1.223684in}{1.004348in}}%
\pgfusepath{clip}%
\pgfsetrectcap%
\pgfsetroundjoin%
\pgfsetlinewidth{0.803000pt}%
\definecolor{currentstroke}{rgb}{0.000000,0.356863,0.509804}%
\pgfsetstrokecolor{currentstroke}%
\pgfsetdash{}{0pt}%
\pgfpathmoveto{\pgfqpoint{2.210320in}{2.217491in}}%
\pgfpathlineto{\pgfqpoint{2.235312in}{2.210969in}}%
\pgfpathlineto{\pgfqpoint{2.260304in}{2.224013in}}%
\pgfpathlineto{\pgfqpoint{2.285296in}{2.250100in}}%
\pgfpathlineto{\pgfqpoint{2.310288in}{2.289230in}}%
\pgfpathlineto{\pgfqpoint{2.335280in}{2.302274in}}%
\pgfpathlineto{\pgfqpoint{2.360271in}{2.328361in}}%
\pgfpathlineto{\pgfqpoint{2.385263in}{2.289230in}}%
\pgfpathlineto{\pgfqpoint{2.410255in}{2.237056in}}%
\pgfpathlineto{\pgfqpoint{2.435247in}{2.217491in}}%
\pgfpathlineto{\pgfqpoint{2.460239in}{2.224013in}}%
\pgfpathlineto{\pgfqpoint{2.485231in}{2.243578in}}%
\pgfpathlineto{\pgfqpoint{2.510222in}{2.230535in}}%
\pgfpathlineto{\pgfqpoint{2.535214in}{2.197926in}}%
\pgfpathlineto{\pgfqpoint{2.560206in}{2.204448in}}%
\pgfpathlineto{\pgfqpoint{2.585198in}{2.165317in}}%
\pgfpathlineto{\pgfqpoint{2.610190in}{2.139230in}}%
\pgfpathlineto{\pgfqpoint{2.635182in}{2.145752in}}%
\pgfpathlineto{\pgfqpoint{2.660173in}{2.145752in}}%
\pgfpathlineto{\pgfqpoint{2.685165in}{2.158795in}}%
\pgfpathlineto{\pgfqpoint{2.710157in}{2.171839in}}%
\pgfpathlineto{\pgfqpoint{2.735149in}{2.178361in}}%
\pgfpathlineto{\pgfqpoint{2.760141in}{2.184882in}}%
\pgfpathlineto{\pgfqpoint{2.785133in}{2.197926in}}%
\pgfpathlineto{\pgfqpoint{2.810124in}{2.204448in}}%
\pgfpathlineto{\pgfqpoint{2.835116in}{2.204448in}}%
\pgfpathlineto{\pgfqpoint{2.860108in}{2.204448in}}%
\pgfpathlineto{\pgfqpoint{2.885100in}{2.204448in}}%
\pgfpathlineto{\pgfqpoint{2.910092in}{2.204448in}}%
\pgfpathlineto{\pgfqpoint{2.935084in}{2.204448in}}%
\pgfpathlineto{\pgfqpoint{2.960075in}{2.204448in}}%
\pgfpathlineto{\pgfqpoint{2.985067in}{2.204448in}}%
\pgfpathlineto{\pgfqpoint{3.010059in}{2.204448in}}%
\pgfpathlineto{\pgfqpoint{3.035051in}{2.210969in}}%
\pgfpathlineto{\pgfqpoint{3.060043in}{2.210969in}}%
\pgfpathlineto{\pgfqpoint{3.085035in}{2.217491in}}%
\pgfpathlineto{\pgfqpoint{3.110026in}{2.217491in}}%
\pgfpathlineto{\pgfqpoint{3.135018in}{2.210969in}}%
\pgfpathlineto{\pgfqpoint{3.160010in}{2.217491in}}%
\pgfpathlineto{\pgfqpoint{3.185002in}{2.210969in}}%
\pgfpathlineto{\pgfqpoint{3.209994in}{2.204448in}}%
\pgfpathlineto{\pgfqpoint{3.234986in}{2.204448in}}%
\pgfpathlineto{\pgfqpoint{3.259977in}{2.204448in}}%
\pgfpathlineto{\pgfqpoint{3.284969in}{2.204448in}}%
\pgfpathlineto{\pgfqpoint{3.309961in}{2.204448in}}%
\pgfpathlineto{\pgfqpoint{3.334953in}{2.197926in}}%
\pgfpathlineto{\pgfqpoint{3.359945in}{2.204448in}}%
\pgfpathlineto{\pgfqpoint{3.384937in}{2.204448in}}%
\pgfpathlineto{\pgfqpoint{3.395057in}{2.204448in}}%
\pgfusepath{stroke}%
\end{pgfscope}%
\begin{pgfscope}%
\pgfpathrectangle{\pgfqpoint{2.161373in}{1.689230in}}{\pgfqpoint{1.223684in}{1.004348in}}%
\pgfusepath{clip}%
\pgfsetrectcap%
\pgfsetroundjoin%
\pgfsetlinewidth{0.803000pt}%
\definecolor{currentstroke}{rgb}{0.490196,0.588235,0.431373}%
\pgfsetstrokecolor{currentstroke}%
\pgfsetdash{}{0pt}%
\pgfpathmoveto{\pgfqpoint{2.210320in}{2.315317in}}%
\pgfpathlineto{\pgfqpoint{2.235312in}{2.315317in}}%
\pgfpathlineto{\pgfqpoint{2.260304in}{2.308795in}}%
\pgfpathlineto{\pgfqpoint{2.285296in}{2.374013in}}%
\pgfpathlineto{\pgfqpoint{2.310288in}{2.484882in}}%
\pgfpathlineto{\pgfqpoint{2.335280in}{2.543578in}}%
\pgfpathlineto{\pgfqpoint{2.360271in}{2.641404in}}%
\pgfpathlineto{\pgfqpoint{2.385263in}{2.002274in}}%
\pgfpathlineto{\pgfqpoint{2.410255in}{2.002274in}}%
\pgfpathlineto{\pgfqpoint{2.435247in}{1.989230in}}%
\pgfpathlineto{\pgfqpoint{2.460239in}{1.995752in}}%
\pgfpathlineto{\pgfqpoint{2.485231in}{1.995752in}}%
\pgfpathlineto{\pgfqpoint{2.510222in}{1.976187in}}%
\pgfpathlineto{\pgfqpoint{2.535214in}{1.995752in}}%
\pgfpathlineto{\pgfqpoint{2.560206in}{1.995752in}}%
\pgfpathlineto{\pgfqpoint{2.585198in}{1.989230in}}%
\pgfpathlineto{\pgfqpoint{2.610190in}{1.989230in}}%
\pgfpathlineto{\pgfqpoint{2.635182in}{1.989230in}}%
\pgfpathlineto{\pgfqpoint{2.660173in}{1.989230in}}%
\pgfpathlineto{\pgfqpoint{2.685165in}{1.995752in}}%
\pgfpathlineto{\pgfqpoint{2.710157in}{2.067491in}}%
\pgfpathlineto{\pgfqpoint{2.735149in}{2.126187in}}%
\pgfpathlineto{\pgfqpoint{2.760141in}{2.171839in}}%
\pgfpathlineto{\pgfqpoint{2.785133in}{2.210969in}}%
\pgfpathlineto{\pgfqpoint{2.810124in}{2.230535in}}%
\pgfpathlineto{\pgfqpoint{2.835116in}{2.256622in}}%
\pgfpathlineto{\pgfqpoint{2.860108in}{2.269665in}}%
\pgfpathlineto{\pgfqpoint{2.885100in}{2.282708in}}%
\pgfpathlineto{\pgfqpoint{2.910092in}{2.289230in}}%
\pgfpathlineto{\pgfqpoint{2.935084in}{2.295752in}}%
\pgfpathlineto{\pgfqpoint{2.960075in}{2.302274in}}%
\pgfpathlineto{\pgfqpoint{2.985067in}{2.315317in}}%
\pgfpathlineto{\pgfqpoint{3.010059in}{2.315317in}}%
\pgfpathlineto{\pgfqpoint{3.035051in}{2.315317in}}%
\pgfpathlineto{\pgfqpoint{3.060043in}{2.315317in}}%
\pgfpathlineto{\pgfqpoint{3.085035in}{2.315317in}}%
\pgfpathlineto{\pgfqpoint{3.110026in}{2.315317in}}%
\pgfpathlineto{\pgfqpoint{3.135018in}{2.315317in}}%
\pgfpathlineto{\pgfqpoint{3.160010in}{2.315317in}}%
\pgfpathlineto{\pgfqpoint{3.185002in}{2.308795in}}%
\pgfpathlineto{\pgfqpoint{3.209994in}{2.308795in}}%
\pgfpathlineto{\pgfqpoint{3.234986in}{2.315317in}}%
\pgfpathlineto{\pgfqpoint{3.259977in}{2.308795in}}%
\pgfpathlineto{\pgfqpoint{3.284969in}{2.308795in}}%
\pgfpathlineto{\pgfqpoint{3.309961in}{2.308795in}}%
\pgfpathlineto{\pgfqpoint{3.334953in}{2.308795in}}%
\pgfpathlineto{\pgfqpoint{3.359945in}{2.315317in}}%
\pgfpathlineto{\pgfqpoint{3.384937in}{2.315317in}}%
\pgfpathlineto{\pgfqpoint{3.395057in}{2.315317in}}%
\pgfusepath{stroke}%
\end{pgfscope}%
\begin{pgfscope}%
\pgfpathrectangle{\pgfqpoint{2.161373in}{1.689230in}}{\pgfqpoint{1.223684in}{1.004348in}}%
\pgfusepath{clip}%
\pgfsetrectcap%
\pgfsetroundjoin%
\pgfsetlinewidth{0.803000pt}%
\definecolor{currentstroke}{rgb}{0.843137,0.666667,0.313725}%
\pgfsetstrokecolor{currentstroke}%
\pgfsetdash{}{0pt}%
\pgfpathmoveto{\pgfqpoint{2.210320in}{2.204448in}}%
\pgfpathlineto{\pgfqpoint{2.235312in}{2.197926in}}%
\pgfpathlineto{\pgfqpoint{2.260304in}{2.171839in}}%
\pgfpathlineto{\pgfqpoint{2.285296in}{2.139230in}}%
\pgfpathlineto{\pgfqpoint{2.310288in}{2.165317in}}%
\pgfpathlineto{\pgfqpoint{2.335280in}{2.224013in}}%
\pgfpathlineto{\pgfqpoint{2.360271in}{2.256622in}}%
\pgfpathlineto{\pgfqpoint{2.385263in}{2.250100in}}%
\pgfpathlineto{\pgfqpoint{2.410255in}{2.230535in}}%
\pgfpathlineto{\pgfqpoint{2.435247in}{2.191404in}}%
\pgfpathlineto{\pgfqpoint{2.460239in}{2.184882in}}%
\pgfpathlineto{\pgfqpoint{2.485231in}{2.184882in}}%
\pgfpathlineto{\pgfqpoint{2.510222in}{2.165317in}}%
\pgfpathlineto{\pgfqpoint{2.535214in}{2.152274in}}%
\pgfpathlineto{\pgfqpoint{2.560206in}{2.158795in}}%
\pgfpathlineto{\pgfqpoint{2.585198in}{2.145752in}}%
\pgfpathlineto{\pgfqpoint{2.610190in}{2.139230in}}%
\pgfpathlineto{\pgfqpoint{2.635182in}{2.145752in}}%
\pgfpathlineto{\pgfqpoint{2.660173in}{2.152274in}}%
\pgfpathlineto{\pgfqpoint{2.685165in}{2.165317in}}%
\pgfpathlineto{\pgfqpoint{2.710157in}{2.171839in}}%
\pgfpathlineto{\pgfqpoint{2.735149in}{2.178361in}}%
\pgfpathlineto{\pgfqpoint{2.760141in}{2.184882in}}%
\pgfpathlineto{\pgfqpoint{2.785133in}{2.191404in}}%
\pgfpathlineto{\pgfqpoint{2.810124in}{2.191404in}}%
\pgfpathlineto{\pgfqpoint{2.835116in}{2.197926in}}%
\pgfpathlineto{\pgfqpoint{2.860108in}{2.197926in}}%
\pgfpathlineto{\pgfqpoint{2.885100in}{2.204448in}}%
\pgfpathlineto{\pgfqpoint{2.910092in}{2.197926in}}%
\pgfpathlineto{\pgfqpoint{2.935084in}{2.204448in}}%
\pgfpathlineto{\pgfqpoint{2.960075in}{2.197926in}}%
\pgfpathlineto{\pgfqpoint{2.985067in}{2.197926in}}%
\pgfpathlineto{\pgfqpoint{3.010059in}{2.197926in}}%
\pgfpathlineto{\pgfqpoint{3.035051in}{2.197926in}}%
\pgfpathlineto{\pgfqpoint{3.060043in}{2.197926in}}%
\pgfpathlineto{\pgfqpoint{3.085035in}{2.197926in}}%
\pgfpathlineto{\pgfqpoint{3.110026in}{2.197926in}}%
\pgfpathlineto{\pgfqpoint{3.135018in}{2.197926in}}%
\pgfpathlineto{\pgfqpoint{3.160010in}{2.204448in}}%
\pgfpathlineto{\pgfqpoint{3.185002in}{2.197926in}}%
\pgfpathlineto{\pgfqpoint{3.209994in}{2.197926in}}%
\pgfpathlineto{\pgfqpoint{3.234986in}{2.197926in}}%
\pgfpathlineto{\pgfqpoint{3.259977in}{2.197926in}}%
\pgfpathlineto{\pgfqpoint{3.284969in}{2.197926in}}%
\pgfpathlineto{\pgfqpoint{3.309961in}{2.197926in}}%
\pgfpathlineto{\pgfqpoint{3.334953in}{2.197926in}}%
\pgfpathlineto{\pgfqpoint{3.359945in}{2.204448in}}%
\pgfpathlineto{\pgfqpoint{3.384937in}{2.197926in}}%
\pgfpathlineto{\pgfqpoint{3.395057in}{2.200567in}}%
\pgfusepath{stroke}%
\end{pgfscope}%
\begin{pgfscope}%
\pgfsetrectcap%
\pgfsetmiterjoin%
\pgfsetlinewidth{0.501875pt}%
\definecolor{currentstroke}{rgb}{0.317647,0.317647,0.317647}%
\pgfsetstrokecolor{currentstroke}%
\pgfsetdash{}{0pt}%
\pgfpathmoveto{\pgfqpoint{2.161373in}{1.689230in}}%
\pgfpathlineto{\pgfqpoint{2.161373in}{2.693578in}}%
\pgfusepath{stroke}%
\end{pgfscope}%
\begin{pgfscope}%
\pgfsetrectcap%
\pgfsetmiterjoin%
\pgfsetlinewidth{0.501875pt}%
\definecolor{currentstroke}{rgb}{0.317647,0.317647,0.317647}%
\pgfsetstrokecolor{currentstroke}%
\pgfsetdash{}{0pt}%
\pgfpathmoveto{\pgfqpoint{2.161373in}{1.689230in}}%
\pgfpathlineto{\pgfqpoint{3.385057in}{1.689230in}}%
\pgfusepath{stroke}%
\end{pgfscope}%
\begin{pgfscope}%
\definecolor{textcolor}{rgb}{0.000000,0.000000,0.000000}%
\pgfsetstrokecolor{textcolor}%
\pgfsetfillcolor{textcolor}%
\pgftext[x=2.773215in,y=2.776911in,,base]{\color{textcolor}\rmfamily\fontsize{6.664000}{7.996800}\selectfont Membrane Potential}%
\end{pgfscope}%
\begin{pgfscope}%
\pgfsetbuttcap%
\pgfsetmiterjoin%
\pgfsetlinewidth{0.000000pt}%
\definecolor{currentstroke}{rgb}{0.000000,0.000000,0.000000}%
\pgfsetstrokecolor{currentstroke}%
\pgfsetstrokeopacity{0.000000}%
\pgfsetdash{}{0pt}%
\pgfpathmoveto{\pgfqpoint{3.874531in}{1.689230in}}%
\pgfpathlineto{\pgfqpoint{5.098215in}{1.689230in}}%
\pgfpathlineto{\pgfqpoint{5.098215in}{2.693578in}}%
\pgfpathlineto{\pgfqpoint{3.874531in}{2.693578in}}%
\pgfpathclose%
\pgfusepath{}%
\end{pgfscope}%
\begin{pgfscope}%
\pgfsetbuttcap%
\pgfsetroundjoin%
\definecolor{currentfill}{rgb}{0.317647,0.317647,0.317647}%
\pgfsetfillcolor{currentfill}%
\pgfsetlinewidth{0.501875pt}%
\definecolor{currentstroke}{rgb}{0.317647,0.317647,0.317647}%
\pgfsetstrokecolor{currentstroke}%
\pgfsetdash{}{0pt}%
\pgfsys@defobject{currentmarker}{\pgfqpoint{0.000000in}{-0.020833in}}{\pgfqpoint{0.000000in}{0.000000in}}{%
\pgfpathmoveto{\pgfqpoint{0.000000in}{0.000000in}}%
\pgfpathlineto{\pgfqpoint{0.000000in}{-0.020833in}}%
\pgfusepath{stroke,fill}%
}%
\begin{pgfscope}%
\pgfsys@transformshift{3.923478in}{1.689230in}%
\pgfsys@useobject{currentmarker}{}%
\end{pgfscope}%
\end{pgfscope}%
\begin{pgfscope}%
\definecolor{textcolor}{rgb}{0.317647,0.317647,0.317647}%
\pgfsetstrokecolor{textcolor}%
\pgfsetfillcolor{textcolor}%
\pgftext[x=3.923478in,y=1.640619in,,top]{\color{textcolor}\rmfamily\fontsize{6.664000}{7.996800}\selectfont \(\displaystyle 0\)}%
\end{pgfscope}%
\begin{pgfscope}%
\pgfsetbuttcap%
\pgfsetroundjoin%
\definecolor{currentfill}{rgb}{0.317647,0.317647,0.317647}%
\pgfsetfillcolor{currentfill}%
\pgfsetlinewidth{0.501875pt}%
\definecolor{currentstroke}{rgb}{0.317647,0.317647,0.317647}%
\pgfsetstrokecolor{currentstroke}%
\pgfsetdash{}{0pt}%
\pgfsys@defobject{currentmarker}{\pgfqpoint{0.000000in}{-0.020833in}}{\pgfqpoint{0.000000in}{0.000000in}}{%
\pgfpathmoveto{\pgfqpoint{0.000000in}{0.000000in}}%
\pgfpathlineto{\pgfqpoint{0.000000in}{-0.020833in}}%
\pgfusepath{stroke,fill}%
}%
\begin{pgfscope}%
\pgfsys@transformshift{4.412952in}{1.689230in}%
\pgfsys@useobject{currentmarker}{}%
\end{pgfscope}%
\end{pgfscope}%
\begin{pgfscope}%
\definecolor{textcolor}{rgb}{0.317647,0.317647,0.317647}%
\pgfsetstrokecolor{textcolor}%
\pgfsetfillcolor{textcolor}%
\pgftext[x=4.412952in,y=1.640619in,,top]{\color{textcolor}\rmfamily\fontsize{6.664000}{7.996800}\selectfont \(\displaystyle 50\)}%
\end{pgfscope}%
\begin{pgfscope}%
\pgfsetbuttcap%
\pgfsetroundjoin%
\definecolor{currentfill}{rgb}{0.317647,0.317647,0.317647}%
\pgfsetfillcolor{currentfill}%
\pgfsetlinewidth{0.501875pt}%
\definecolor{currentstroke}{rgb}{0.317647,0.317647,0.317647}%
\pgfsetstrokecolor{currentstroke}%
\pgfsetdash{}{0pt}%
\pgfsys@defobject{currentmarker}{\pgfqpoint{0.000000in}{-0.020833in}}{\pgfqpoint{0.000000in}{0.000000in}}{%
\pgfpathmoveto{\pgfqpoint{0.000000in}{0.000000in}}%
\pgfpathlineto{\pgfqpoint{0.000000in}{-0.020833in}}%
\pgfusepath{stroke,fill}%
}%
\begin{pgfscope}%
\pgfsys@transformshift{4.902426in}{1.689230in}%
\pgfsys@useobject{currentmarker}{}%
\end{pgfscope}%
\end{pgfscope}%
\begin{pgfscope}%
\definecolor{textcolor}{rgb}{0.317647,0.317647,0.317647}%
\pgfsetstrokecolor{textcolor}%
\pgfsetfillcolor{textcolor}%
\pgftext[x=4.902426in,y=1.640619in,,top]{\color{textcolor}\rmfamily\fontsize{6.664000}{7.996800}\selectfont \(\displaystyle 100\)}%
\end{pgfscope}%
\begin{pgfscope}%
\pgfsetbuttcap%
\pgfsetroundjoin%
\definecolor{currentfill}{rgb}{0.317647,0.317647,0.317647}%
\pgfsetfillcolor{currentfill}%
\pgfsetlinewidth{0.501875pt}%
\definecolor{currentstroke}{rgb}{0.317647,0.317647,0.317647}%
\pgfsetstrokecolor{currentstroke}%
\pgfsetdash{}{0pt}%
\pgfsys@defobject{currentmarker}{\pgfqpoint{-0.020833in}{0.000000in}}{\pgfqpoint{0.000000in}{0.000000in}}{%
\pgfpathmoveto{\pgfqpoint{0.000000in}{0.000000in}}%
\pgfpathlineto{\pgfqpoint{-0.020833in}{0.000000in}}%
\pgfusepath{stroke,fill}%
}%
\begin{pgfscope}%
\pgfsys@transformshift{3.874531in}{1.691862in}%
\pgfsys@useobject{currentmarker}{}%
\end{pgfscope}%
\end{pgfscope}%
\begin{pgfscope}%
\definecolor{textcolor}{rgb}{0.317647,0.317647,0.317647}%
\pgfsetstrokecolor{textcolor}%
\pgfsetfillcolor{textcolor}%
\pgftext[x=3.609291in,y=1.659745in,left,base]{\color{textcolor}\rmfamily\fontsize{6.664000}{7.996800}\selectfont \(\displaystyle -5.0\)}%
\end{pgfscope}%
\begin{pgfscope}%
\pgfsetbuttcap%
\pgfsetroundjoin%
\definecolor{currentfill}{rgb}{0.317647,0.317647,0.317647}%
\pgfsetfillcolor{currentfill}%
\pgfsetlinewidth{0.501875pt}%
\definecolor{currentstroke}{rgb}{0.317647,0.317647,0.317647}%
\pgfsetstrokecolor{currentstroke}%
\pgfsetdash{}{0pt}%
\pgfsys@defobject{currentmarker}{\pgfqpoint{-0.020833in}{0.000000in}}{\pgfqpoint{0.000000in}{0.000000in}}{%
\pgfpathmoveto{\pgfqpoint{0.000000in}{0.000000in}}%
\pgfpathlineto{\pgfqpoint{-0.020833in}{0.000000in}}%
\pgfusepath{stroke,fill}%
}%
\begin{pgfscope}%
\pgfsys@transformshift{3.874531in}{1.914089in}%
\pgfsys@useobject{currentmarker}{}%
\end{pgfscope}%
\end{pgfscope}%
\begin{pgfscope}%
\definecolor{textcolor}{rgb}{0.317647,0.317647,0.317647}%
\pgfsetstrokecolor{textcolor}%
\pgfsetfillcolor{textcolor}%
\pgftext[x=3.609291in,y=1.881972in,left,base]{\color{textcolor}\rmfamily\fontsize{6.664000}{7.996800}\selectfont \(\displaystyle -2.5\)}%
\end{pgfscope}%
\begin{pgfscope}%
\pgfsetbuttcap%
\pgfsetroundjoin%
\definecolor{currentfill}{rgb}{0.317647,0.317647,0.317647}%
\pgfsetfillcolor{currentfill}%
\pgfsetlinewidth{0.501875pt}%
\definecolor{currentstroke}{rgb}{0.317647,0.317647,0.317647}%
\pgfsetstrokecolor{currentstroke}%
\pgfsetdash{}{0pt}%
\pgfsys@defobject{currentmarker}{\pgfqpoint{-0.020833in}{0.000000in}}{\pgfqpoint{0.000000in}{0.000000in}}{%
\pgfpathmoveto{\pgfqpoint{0.000000in}{0.000000in}}%
\pgfpathlineto{\pgfqpoint{-0.020833in}{0.000000in}}%
\pgfusepath{stroke,fill}%
}%
\begin{pgfscope}%
\pgfsys@transformshift{3.874531in}{2.136316in}%
\pgfsys@useobject{currentmarker}{}%
\end{pgfscope}%
\end{pgfscope}%
\begin{pgfscope}%
\definecolor{textcolor}{rgb}{0.317647,0.317647,0.317647}%
\pgfsetstrokecolor{textcolor}%
\pgfsetfillcolor{textcolor}%
\pgftext[x=3.696097in,y=2.104199in,left,base]{\color{textcolor}\rmfamily\fontsize{6.664000}{7.996800}\selectfont \(\displaystyle 0.0\)}%
\end{pgfscope}%
\begin{pgfscope}%
\pgfsetbuttcap%
\pgfsetroundjoin%
\definecolor{currentfill}{rgb}{0.317647,0.317647,0.317647}%
\pgfsetfillcolor{currentfill}%
\pgfsetlinewidth{0.501875pt}%
\definecolor{currentstroke}{rgb}{0.317647,0.317647,0.317647}%
\pgfsetstrokecolor{currentstroke}%
\pgfsetdash{}{0pt}%
\pgfsys@defobject{currentmarker}{\pgfqpoint{-0.020833in}{0.000000in}}{\pgfqpoint{0.000000in}{0.000000in}}{%
\pgfpathmoveto{\pgfqpoint{0.000000in}{0.000000in}}%
\pgfpathlineto{\pgfqpoint{-0.020833in}{0.000000in}}%
\pgfusepath{stroke,fill}%
}%
\begin{pgfscope}%
\pgfsys@transformshift{3.874531in}{2.358543in}%
\pgfsys@useobject{currentmarker}{}%
\end{pgfscope}%
\end{pgfscope}%
\begin{pgfscope}%
\definecolor{textcolor}{rgb}{0.317647,0.317647,0.317647}%
\pgfsetstrokecolor{textcolor}%
\pgfsetfillcolor{textcolor}%
\pgftext[x=3.696097in,y=2.326426in,left,base]{\color{textcolor}\rmfamily\fontsize{6.664000}{7.996800}\selectfont \(\displaystyle 2.5\)}%
\end{pgfscope}%
\begin{pgfscope}%
\pgfsetbuttcap%
\pgfsetroundjoin%
\definecolor{currentfill}{rgb}{0.317647,0.317647,0.317647}%
\pgfsetfillcolor{currentfill}%
\pgfsetlinewidth{0.501875pt}%
\definecolor{currentstroke}{rgb}{0.317647,0.317647,0.317647}%
\pgfsetstrokecolor{currentstroke}%
\pgfsetdash{}{0pt}%
\pgfsys@defobject{currentmarker}{\pgfqpoint{-0.020833in}{0.000000in}}{\pgfqpoint{0.000000in}{0.000000in}}{%
\pgfpathmoveto{\pgfqpoint{0.000000in}{0.000000in}}%
\pgfpathlineto{\pgfqpoint{-0.020833in}{0.000000in}}%
\pgfusepath{stroke,fill}%
}%
\begin{pgfscope}%
\pgfsys@transformshift{3.874531in}{2.580770in}%
\pgfsys@useobject{currentmarker}{}%
\end{pgfscope}%
\end{pgfscope}%
\begin{pgfscope}%
\definecolor{textcolor}{rgb}{0.317647,0.317647,0.317647}%
\pgfsetstrokecolor{textcolor}%
\pgfsetfillcolor{textcolor}%
\pgftext[x=3.696097in,y=2.548653in,left,base]{\color{textcolor}\rmfamily\fontsize{6.664000}{7.996800}\selectfont \(\displaystyle 5.0\)}%
\end{pgfscope}%
\begin{pgfscope}%
\definecolor{textcolor}{rgb}{0.317647,0.317647,0.317647}%
\pgfsetstrokecolor{textcolor}%
\pgfsetfillcolor{textcolor}%
\pgftext[x=3.553736in,y=2.191404in,,bottom,rotate=90.000000]{\color{textcolor}\rmfamily\fontsize{6.664000}{7.996800}\selectfont \(\displaystyle e^{(h)}\)}%
\end{pgfscope}%
\begin{pgfscope}%
\pgfpathrectangle{\pgfqpoint{3.874531in}{1.689230in}}{\pgfqpoint{1.223684in}{1.004348in}}%
\pgfusepath{clip}%
\pgfsetrectcap%
\pgfsetroundjoin%
\pgfsetlinewidth{0.803000pt}%
\definecolor{currentstroke}{rgb}{0.333333,0.333333,0.333333}%
\pgfsetstrokecolor{currentstroke}%
\pgfsetdash{}{0pt}%
\pgfpathmoveto{\pgfqpoint{3.923478in}{2.136316in}}%
\pgfpathlineto{\pgfqpoint{3.948470in}{2.136316in}}%
\pgfpathlineto{\pgfqpoint{3.973462in}{2.136316in}}%
\pgfpathlineto{\pgfqpoint{3.998454in}{2.136316in}}%
\pgfpathlineto{\pgfqpoint{4.023446in}{2.136316in}}%
\pgfpathlineto{\pgfqpoint{4.048437in}{2.136316in}}%
\pgfpathlineto{\pgfqpoint{4.073429in}{2.136316in}}%
\pgfpathlineto{\pgfqpoint{4.098421in}{2.136316in}}%
\pgfpathlineto{\pgfqpoint{4.123413in}{2.136316in}}%
\pgfpathlineto{\pgfqpoint{4.148405in}{2.103928in}}%
\pgfpathlineto{\pgfqpoint{4.173397in}{2.083282in}}%
\pgfpathlineto{\pgfqpoint{4.198388in}{2.071185in}}%
\pgfpathlineto{\pgfqpoint{4.223380in}{2.065217in}}%
\pgfpathlineto{\pgfqpoint{4.248372in}{2.063552in}}%
\pgfpathlineto{\pgfqpoint{4.273364in}{2.004257in}}%
\pgfpathlineto{\pgfqpoint{4.298356in}{1.968850in}}%
\pgfpathlineto{\pgfqpoint{4.323348in}{1.950618in}}%
\pgfpathlineto{\pgfqpoint{4.348339in}{1.944499in}}%
\pgfpathlineto{\pgfqpoint{4.373331in}{1.946700in}}%
\pgfpathlineto{\pgfqpoint{4.398323in}{1.954406in}}%
\pgfpathlineto{\pgfqpoint{4.423315in}{1.965548in}}%
\pgfpathlineto{\pgfqpoint{4.448307in}{1.978629in}}%
\pgfpathlineto{\pgfqpoint{4.473299in}{1.992578in}}%
\pgfpathlineto{\pgfqpoint{4.498290in}{2.006652in}}%
\pgfpathlineto{\pgfqpoint{4.523282in}{2.020346in}}%
\pgfpathlineto{\pgfqpoint{4.548274in}{2.033336in}}%
\pgfpathlineto{\pgfqpoint{4.573266in}{2.045428in}}%
\pgfpathlineto{\pgfqpoint{4.598258in}{2.056519in}}%
\pgfpathlineto{\pgfqpoint{4.623250in}{2.066576in}}%
\pgfpathlineto{\pgfqpoint{4.648241in}{2.075609in}}%
\pgfpathlineto{\pgfqpoint{4.673233in}{2.083659in}}%
\pgfpathlineto{\pgfqpoint{4.698225in}{2.090785in}}%
\pgfpathlineto{\pgfqpoint{4.723217in}{2.097057in}}%
\pgfpathlineto{\pgfqpoint{4.748209in}{2.102552in}}%
\pgfpathlineto{\pgfqpoint{4.773201in}{2.107345in}}%
\pgfpathlineto{\pgfqpoint{4.798192in}{2.111510in}}%
\pgfpathlineto{\pgfqpoint{4.823184in}{2.115116in}}%
\pgfpathlineto{\pgfqpoint{4.848176in}{2.118230in}}%
\pgfpathlineto{\pgfqpoint{4.873168in}{2.120912in}}%
\pgfpathlineto{\pgfqpoint{4.898160in}{2.123216in}}%
\pgfpathlineto{\pgfqpoint{4.923152in}{2.125191in}}%
\pgfpathlineto{\pgfqpoint{4.948143in}{2.126880in}}%
\pgfpathlineto{\pgfqpoint{4.973135in}{2.128322in}}%
\pgfpathlineto{\pgfqpoint{4.998127in}{2.129552in}}%
\pgfpathlineto{\pgfqpoint{5.023119in}{2.130598in}}%
\pgfpathlineto{\pgfqpoint{5.048111in}{2.131488in}}%
\pgfpathlineto{\pgfqpoint{5.073103in}{2.132242in}}%
\pgfpathlineto{\pgfqpoint{5.098095in}{2.132882in}}%
\pgfpathlineto{\pgfqpoint{5.108215in}{2.133102in}}%
\pgfusepath{stroke}%
\end{pgfscope}%
\begin{pgfscope}%
\pgfpathrectangle{\pgfqpoint{3.874531in}{1.689230in}}{\pgfqpoint{1.223684in}{1.004348in}}%
\pgfusepath{clip}%
\pgfsetrectcap%
\pgfsetroundjoin%
\pgfsetlinewidth{0.803000pt}%
\definecolor{currentstroke}{rgb}{0.686275,0.352941,0.313725}%
\pgfsetstrokecolor{currentstroke}%
\pgfsetdash{}{0pt}%
\pgfpathmoveto{\pgfqpoint{3.923478in}{2.136316in}}%
\pgfpathlineto{\pgfqpoint{3.948470in}{2.136316in}}%
\pgfpathlineto{\pgfqpoint{3.973462in}{2.136316in}}%
\pgfpathlineto{\pgfqpoint{3.998454in}{2.136316in}}%
\pgfpathlineto{\pgfqpoint{4.023446in}{2.136316in}}%
\pgfpathlineto{\pgfqpoint{4.048437in}{2.136316in}}%
\pgfpathlineto{\pgfqpoint{4.073429in}{2.136316in}}%
\pgfpathlineto{\pgfqpoint{4.098421in}{2.136316in}}%
\pgfpathlineto{\pgfqpoint{4.123413in}{2.136316in}}%
\pgfpathlineto{\pgfqpoint{4.148405in}{2.155453in}}%
\pgfpathlineto{\pgfqpoint{4.173397in}{2.167653in}}%
\pgfpathlineto{\pgfqpoint{4.198388in}{2.174801in}}%
\pgfpathlineto{\pgfqpoint{4.223380in}{2.178327in}}%
\pgfpathlineto{\pgfqpoint{4.248372in}{2.179311in}}%
\pgfpathlineto{\pgfqpoint{4.273364in}{2.212640in}}%
\pgfpathlineto{\pgfqpoint{4.298356in}{2.232473in}}%
\pgfpathlineto{\pgfqpoint{4.323348in}{2.242608in}}%
\pgfpathlineto{\pgfqpoint{4.348339in}{2.245909in}}%
\pgfpathlineto{\pgfqpoint{4.373331in}{2.244521in}}%
\pgfpathlineto{\pgfqpoint{4.398323in}{2.240034in}}%
\pgfpathlineto{\pgfqpoint{4.423315in}{2.233619in}}%
\pgfpathlineto{\pgfqpoint{4.448307in}{2.226122in}}%
\pgfpathlineto{\pgfqpoint{4.473299in}{2.218145in}}%
\pgfpathlineto{\pgfqpoint{4.498290in}{2.210110in}}%
\pgfpathlineto{\pgfqpoint{4.523282in}{2.202298in}}%
\pgfpathlineto{\pgfqpoint{4.548274in}{2.194894in}}%
\pgfpathlineto{\pgfqpoint{4.573266in}{2.188006in}}%
\pgfpathlineto{\pgfqpoint{4.598258in}{2.181691in}}%
\pgfpathlineto{\pgfqpoint{4.623250in}{2.175966in}}%
\pgfpathlineto{\pgfqpoint{4.648241in}{2.170826in}}%
\pgfpathlineto{\pgfqpoint{4.673233in}{2.166247in}}%
\pgfpathlineto{\pgfqpoint{4.698225in}{2.162194in}}%
\pgfpathlineto{\pgfqpoint{4.723217in}{2.158626in}}%
\pgfpathlineto{\pgfqpoint{4.748209in}{2.155502in}}%
\pgfpathlineto{\pgfqpoint{4.773201in}{2.152777in}}%
\pgfpathlineto{\pgfqpoint{4.798192in}{2.150410in}}%
\pgfpathlineto{\pgfqpoint{4.823184in}{2.148360in}}%
\pgfpathlineto{\pgfqpoint{4.848176in}{2.146590in}}%
\pgfpathlineto{\pgfqpoint{4.873168in}{2.145066in}}%
\pgfpathlineto{\pgfqpoint{4.898160in}{2.143757in}}%
\pgfpathlineto{\pgfqpoint{4.923152in}{2.142635in}}%
\pgfpathlineto{\pgfqpoint{4.948143in}{2.141675in}}%
\pgfpathlineto{\pgfqpoint{4.973135in}{2.140856in}}%
\pgfpathlineto{\pgfqpoint{4.998127in}{2.140157in}}%
\pgfpathlineto{\pgfqpoint{5.023119in}{2.139563in}}%
\pgfpathlineto{\pgfqpoint{5.048111in}{2.139058in}}%
\pgfpathlineto{\pgfqpoint{5.073103in}{2.138629in}}%
\pgfpathlineto{\pgfqpoint{5.098095in}{2.138266in}}%
\pgfpathlineto{\pgfqpoint{5.108215in}{2.138141in}}%
\pgfusepath{stroke}%
\end{pgfscope}%
\begin{pgfscope}%
\pgfpathrectangle{\pgfqpoint{3.874531in}{1.689230in}}{\pgfqpoint{1.223684in}{1.004348in}}%
\pgfusepath{clip}%
\pgfsetrectcap%
\pgfsetroundjoin%
\pgfsetlinewidth{0.803000pt}%
\definecolor{currentstroke}{rgb}{0.000000,0.356863,0.509804}%
\pgfsetstrokecolor{currentstroke}%
\pgfsetdash{}{0pt}%
\pgfpathmoveto{\pgfqpoint{3.923478in}{2.136316in}}%
\pgfpathlineto{\pgfqpoint{3.948470in}{2.136316in}}%
\pgfpathlineto{\pgfqpoint{3.973462in}{2.136316in}}%
\pgfpathlineto{\pgfqpoint{3.998454in}{2.136316in}}%
\pgfpathlineto{\pgfqpoint{4.023446in}{2.136316in}}%
\pgfpathlineto{\pgfqpoint{4.048437in}{2.136316in}}%
\pgfpathlineto{\pgfqpoint{4.073429in}{2.136316in}}%
\pgfpathlineto{\pgfqpoint{4.098421in}{2.136316in}}%
\pgfpathlineto{\pgfqpoint{4.123413in}{2.136316in}}%
\pgfpathlineto{\pgfqpoint{4.148405in}{2.098343in}}%
\pgfpathlineto{\pgfqpoint{4.173397in}{2.074137in}}%
\pgfpathlineto{\pgfqpoint{4.198388in}{2.059954in}}%
\pgfpathlineto{\pgfqpoint{4.223380in}{2.052956in}}%
\pgfpathlineto{\pgfqpoint{4.248372in}{2.051005in}}%
\pgfpathlineto{\pgfqpoint{4.273364in}{2.143256in}}%
\pgfpathlineto{\pgfqpoint{4.298356in}{2.204866in}}%
\pgfpathlineto{\pgfqpoint{4.323348in}{2.243912in}}%
\pgfpathlineto{\pgfqpoint{4.348339in}{2.266550in}}%
\pgfpathlineto{\pgfqpoint{4.373331in}{2.277445in}}%
\pgfpathlineto{\pgfqpoint{4.398323in}{2.280111in}}%
\pgfpathlineto{\pgfqpoint{4.423315in}{2.277173in}}%
\pgfpathlineto{\pgfqpoint{4.448307in}{2.270575in}}%
\pgfpathlineto{\pgfqpoint{4.473299in}{2.261741in}}%
\pgfpathlineto{\pgfqpoint{4.498290in}{2.251698in}}%
\pgfpathlineto{\pgfqpoint{4.523282in}{2.241175in}}%
\pgfpathlineto{\pgfqpoint{4.548274in}{2.230676in}}%
\pgfpathlineto{\pgfqpoint{4.573266in}{2.220537in}}%
\pgfpathlineto{\pgfqpoint{4.598258in}{2.210974in}}%
\pgfpathlineto{\pgfqpoint{4.623250in}{2.202110in}}%
\pgfpathlineto{\pgfqpoint{4.648241in}{2.194006in}}%
\pgfpathlineto{\pgfqpoint{4.673233in}{2.186679in}}%
\pgfpathlineto{\pgfqpoint{4.698225in}{2.180112in}}%
\pgfpathlineto{\pgfqpoint{4.723217in}{2.174271in}}%
\pgfpathlineto{\pgfqpoint{4.748209in}{2.169109in}}%
\pgfpathlineto{\pgfqpoint{4.773201in}{2.164571in}}%
\pgfpathlineto{\pgfqpoint{4.798192in}{2.160601in}}%
\pgfpathlineto{\pgfqpoint{4.823184in}{2.157141in}}%
\pgfpathlineto{\pgfqpoint{4.848176in}{2.154138in}}%
\pgfpathlineto{\pgfqpoint{4.873168in}{2.151539in}}%
\pgfpathlineto{\pgfqpoint{4.898160in}{2.149297in}}%
\pgfpathlineto{\pgfqpoint{4.923152in}{2.147368in}}%
\pgfpathlineto{\pgfqpoint{4.948143in}{2.145711in}}%
\pgfpathlineto{\pgfqpoint{4.973135in}{2.144292in}}%
\pgfpathlineto{\pgfqpoint{4.998127in}{2.143079in}}%
\pgfpathlineto{\pgfqpoint{5.023119in}{2.142043in}}%
\pgfpathlineto{\pgfqpoint{5.048111in}{2.141161in}}%
\pgfpathlineto{\pgfqpoint{5.073103in}{2.140410in}}%
\pgfpathlineto{\pgfqpoint{5.098095in}{2.139772in}}%
\pgfpathlineto{\pgfqpoint{5.108215in}{2.139553in}}%
\pgfusepath{stroke}%
\end{pgfscope}%
\begin{pgfscope}%
\pgfpathrectangle{\pgfqpoint{3.874531in}{1.689230in}}{\pgfqpoint{1.223684in}{1.004348in}}%
\pgfusepath{clip}%
\pgfsetrectcap%
\pgfsetroundjoin%
\pgfsetlinewidth{0.803000pt}%
\definecolor{currentstroke}{rgb}{0.490196,0.588235,0.431373}%
\pgfsetstrokecolor{currentstroke}%
\pgfsetdash{}{0pt}%
\pgfpathmoveto{\pgfqpoint{3.923478in}{2.136316in}}%
\pgfpathlineto{\pgfqpoint{3.948470in}{2.136316in}}%
\pgfpathlineto{\pgfqpoint{3.973462in}{2.136316in}}%
\pgfpathlineto{\pgfqpoint{3.998454in}{2.136316in}}%
\pgfpathlineto{\pgfqpoint{4.023446in}{2.136316in}}%
\pgfpathlineto{\pgfqpoint{4.048437in}{2.136316in}}%
\pgfpathlineto{\pgfqpoint{4.073429in}{2.136316in}}%
\pgfpathlineto{\pgfqpoint{4.098421in}{2.136316in}}%
\pgfpathlineto{\pgfqpoint{4.123413in}{2.136316in}}%
\pgfpathlineto{\pgfqpoint{4.148405in}{2.115708in}}%
\pgfpathlineto{\pgfqpoint{4.173397in}{2.102571in}}%
\pgfpathlineto{\pgfqpoint{4.198388in}{2.094874in}}%
\pgfpathlineto{\pgfqpoint{4.223380in}{2.091076in}}%
\pgfpathlineto{\pgfqpoint{4.248372in}{2.090017in}}%
\pgfpathlineto{\pgfqpoint{4.273364in}{2.057782in}}%
\pgfpathlineto{\pgfqpoint{4.298356in}{2.038755in}}%
\pgfpathlineto{\pgfqpoint{4.323348in}{2.029207in}}%
\pgfpathlineto{\pgfqpoint{4.348339in}{2.026325in}}%
\pgfpathlineto{\pgfqpoint{4.373331in}{2.028008in}}%
\pgfpathlineto{\pgfqpoint{4.398323in}{2.032695in}}%
\pgfpathlineto{\pgfqpoint{4.423315in}{2.039242in}}%
\pgfpathlineto{\pgfqpoint{4.448307in}{2.046820in}}%
\pgfpathlineto{\pgfqpoint{4.473299in}{2.054840in}}%
\pgfpathlineto{\pgfqpoint{4.498290in}{2.062894in}}%
\pgfpathlineto{\pgfqpoint{4.523282in}{2.070704in}}%
\pgfpathlineto{\pgfqpoint{4.548274in}{2.078096in}}%
\pgfpathlineto{\pgfqpoint{4.573266in}{2.084964in}}%
\pgfpathlineto{\pgfqpoint{4.598258in}{2.091255in}}%
\pgfpathlineto{\pgfqpoint{4.623250in}{2.096953in}}%
\pgfpathlineto{\pgfqpoint{4.648241in}{2.102065in}}%
\pgfpathlineto{\pgfqpoint{4.673233in}{2.106618in}}%
\pgfpathlineto{\pgfqpoint{4.698225in}{2.110645in}}%
\pgfpathlineto{\pgfqpoint{4.723217in}{2.114188in}}%
\pgfpathlineto{\pgfqpoint{4.748209in}{2.117291in}}%
\pgfpathlineto{\pgfqpoint{4.773201in}{2.119995in}}%
\pgfpathlineto{\pgfqpoint{4.798192in}{2.122344in}}%
\pgfpathlineto{\pgfqpoint{4.823184in}{2.124378in}}%
\pgfpathlineto{\pgfqpoint{4.848176in}{2.126134in}}%
\pgfpathlineto{\pgfqpoint{4.873168in}{2.127645in}}%
\pgfpathlineto{\pgfqpoint{4.898160in}{2.128943in}}%
\pgfpathlineto{\pgfqpoint{4.923152in}{2.130055in}}%
\pgfpathlineto{\pgfqpoint{4.948143in}{2.131007in}}%
\pgfpathlineto{\pgfqpoint{4.973135in}{2.131819in}}%
\pgfpathlineto{\pgfqpoint{4.998127in}{2.132511in}}%
\pgfpathlineto{\pgfqpoint{5.023119in}{2.133100in}}%
\pgfpathlineto{\pgfqpoint{5.048111in}{2.133600in}}%
\pgfpathlineto{\pgfqpoint{5.073103in}{2.134025in}}%
\pgfpathlineto{\pgfqpoint{5.098095in}{2.134385in}}%
\pgfpathlineto{\pgfqpoint{5.108215in}{2.134509in}}%
\pgfusepath{stroke}%
\end{pgfscope}%
\begin{pgfscope}%
\pgfpathrectangle{\pgfqpoint{3.874531in}{1.689230in}}{\pgfqpoint{1.223684in}{1.004348in}}%
\pgfusepath{clip}%
\pgfsetrectcap%
\pgfsetroundjoin%
\pgfsetlinewidth{0.803000pt}%
\definecolor{currentstroke}{rgb}{0.843137,0.666667,0.313725}%
\pgfsetstrokecolor{currentstroke}%
\pgfsetdash{}{0pt}%
\pgfpathmoveto{\pgfqpoint{3.923478in}{2.136316in}}%
\pgfpathlineto{\pgfqpoint{3.948470in}{2.136316in}}%
\pgfpathlineto{\pgfqpoint{3.973462in}{2.136316in}}%
\pgfpathlineto{\pgfqpoint{3.998454in}{2.136316in}}%
\pgfpathlineto{\pgfqpoint{4.023446in}{2.136316in}}%
\pgfpathlineto{\pgfqpoint{4.048437in}{2.136316in}}%
\pgfpathlineto{\pgfqpoint{4.073429in}{2.136316in}}%
\pgfpathlineto{\pgfqpoint{4.098421in}{2.136316in}}%
\pgfpathlineto{\pgfqpoint{4.123413in}{2.136316in}}%
\pgfpathlineto{\pgfqpoint{4.148405in}{2.210203in}}%
\pgfpathlineto{\pgfqpoint{4.173397in}{2.257303in}}%
\pgfpathlineto{\pgfqpoint{4.198388in}{2.284900in}}%
\pgfpathlineto{\pgfqpoint{4.223380in}{2.298516in}}%
\pgfpathlineto{\pgfqpoint{4.248372in}{2.302314in}}%
\pgfpathlineto{\pgfqpoint{4.273364in}{2.339110in}}%
\pgfpathlineto{\pgfqpoint{4.298356in}{2.357112in}}%
\pgfpathlineto{\pgfqpoint{4.323348in}{2.361924in}}%
\pgfpathlineto{\pgfqpoint{4.348339in}{2.357736in}}%
\pgfpathlineto{\pgfqpoint{4.373331in}{2.347654in}}%
\pgfpathlineto{\pgfqpoint{4.398323in}{2.333951in}}%
\pgfpathlineto{\pgfqpoint{4.423315in}{2.318271in}}%
\pgfpathlineto{\pgfqpoint{4.448307in}{2.301783in}}%
\pgfpathlineto{\pgfqpoint{4.473299in}{2.285293in}}%
\pgfpathlineto{\pgfqpoint{4.498290in}{2.269344in}}%
\pgfpathlineto{\pgfqpoint{4.523282in}{2.254282in}}%
\pgfpathlineto{\pgfqpoint{4.548274in}{2.240310in}}%
\pgfpathlineto{\pgfqpoint{4.573266in}{2.227527in}}%
\pgfpathlineto{\pgfqpoint{4.598258in}{2.215961in}}%
\pgfpathlineto{\pgfqpoint{4.623250in}{2.205591in}}%
\pgfpathlineto{\pgfqpoint{4.648241in}{2.196364in}}%
\pgfpathlineto{\pgfqpoint{4.673233in}{2.188205in}}%
\pgfpathlineto{\pgfqpoint{4.698225in}{2.181032in}}%
\pgfpathlineto{\pgfqpoint{4.723217in}{2.174754in}}%
\pgfpathlineto{\pgfqpoint{4.748209in}{2.169282in}}%
\pgfpathlineto{\pgfqpoint{4.773201in}{2.164532in}}%
\pgfpathlineto{\pgfqpoint{4.798192in}{2.160420in}}%
\pgfpathlineto{\pgfqpoint{4.823184in}{2.156872in}}%
\pgfpathlineto{\pgfqpoint{4.848176in}{2.153818in}}%
\pgfpathlineto{\pgfqpoint{4.873168in}{2.151196in}}%
\pgfpathlineto{\pgfqpoint{4.898160in}{2.148949in}}%
\pgfpathlineto{\pgfqpoint{4.923152in}{2.147028in}}%
\pgfpathlineto{\pgfqpoint{4.948143in}{2.145388in}}%
\pgfpathlineto{\pgfqpoint{4.973135in}{2.143991in}}%
\pgfpathlineto{\pgfqpoint{4.998127in}{2.142802in}}%
\pgfpathlineto{\pgfqpoint{5.023119in}{2.141792in}}%
\pgfpathlineto{\pgfqpoint{5.048111in}{2.140935in}}%
\pgfpathlineto{\pgfqpoint{5.073103in}{2.140209in}}%
\pgfpathlineto{\pgfqpoint{5.098095in}{2.139594in}}%
\pgfpathlineto{\pgfqpoint{5.108215in}{2.139384in}}%
\pgfusepath{stroke}%
\end{pgfscope}%
\begin{pgfscope}%
\pgfpathrectangle{\pgfqpoint{3.874531in}{1.689230in}}{\pgfqpoint{1.223684in}{1.004348in}}%
\pgfusepath{clip}%
\pgfsetrectcap%
\pgfsetroundjoin%
\pgfsetlinewidth{0.803000pt}%
\definecolor{currentstroke}{rgb}{0.333333,0.333333,0.333333}%
\pgfsetstrokecolor{currentstroke}%
\pgfsetdash{}{0pt}%
\pgfpathmoveto{\pgfqpoint{3.923478in}{2.136316in}}%
\pgfpathlineto{\pgfqpoint{3.948470in}{2.136316in}}%
\pgfpathlineto{\pgfqpoint{3.973462in}{2.136316in}}%
\pgfpathlineto{\pgfqpoint{3.998454in}{2.136316in}}%
\pgfpathlineto{\pgfqpoint{4.023446in}{2.136316in}}%
\pgfpathlineto{\pgfqpoint{4.048437in}{2.136316in}}%
\pgfpathlineto{\pgfqpoint{4.073429in}{2.136316in}}%
\pgfpathlineto{\pgfqpoint{4.098421in}{2.136316in}}%
\pgfpathlineto{\pgfqpoint{4.123413in}{2.136316in}}%
\pgfpathlineto{\pgfqpoint{4.148405in}{2.081639in}}%
\pgfpathlineto{\pgfqpoint{4.173397in}{2.046785in}}%
\pgfpathlineto{\pgfqpoint{4.198388in}{2.026363in}}%
\pgfpathlineto{\pgfqpoint{4.223380in}{2.016287in}}%
\pgfpathlineto{\pgfqpoint{4.248372in}{2.013476in}}%
\pgfpathlineto{\pgfqpoint{4.273364in}{2.096903in}}%
\pgfpathlineto{\pgfqpoint{4.298356in}{2.154120in}}%
\pgfpathlineto{\pgfqpoint{4.323348in}{2.191889in}}%
\pgfpathlineto{\pgfqpoint{4.348339in}{2.215380in}}%
\pgfpathlineto{\pgfqpoint{4.373331in}{2.228529in}}%
\pgfpathlineto{\pgfqpoint{4.398323in}{2.234313in}}%
\pgfpathlineto{\pgfqpoint{4.423315in}{2.234970in}}%
\pgfpathlineto{\pgfqpoint{4.448307in}{2.232168in}}%
\pgfpathlineto{\pgfqpoint{4.473299in}{2.227141in}}%
\pgfpathlineto{\pgfqpoint{4.498290in}{2.220787in}}%
\pgfpathlineto{\pgfqpoint{4.523282in}{2.213752in}}%
\pgfpathlineto{\pgfqpoint{4.548274in}{2.206492in}}%
\pgfpathlineto{\pgfqpoint{4.573266in}{2.199319in}}%
\pgfpathlineto{\pgfqpoint{4.598258in}{2.192441in}}%
\pgfpathlineto{\pgfqpoint{4.623250in}{2.185986in}}%
\pgfpathlineto{\pgfqpoint{4.648241in}{2.180028in}}%
\pgfpathlineto{\pgfqpoint{4.673233in}{2.174597in}}%
\pgfpathlineto{\pgfqpoint{4.698225in}{2.169699in}}%
\pgfpathlineto{\pgfqpoint{4.723217in}{2.165319in}}%
\pgfpathlineto{\pgfqpoint{4.748209in}{2.161430in}}%
\pgfpathlineto{\pgfqpoint{4.773201in}{2.157997in}}%
\pgfpathlineto{\pgfqpoint{4.798192in}{2.154984in}}%
\pgfpathlineto{\pgfqpoint{4.823184in}{2.152351in}}%
\pgfpathlineto{\pgfqpoint{4.848176in}{2.150059in}}%
\pgfpathlineto{\pgfqpoint{4.873168in}{2.148071in}}%
\pgfpathlineto{\pgfqpoint{4.898160in}{2.146352in}}%
\pgfpathlineto{\pgfqpoint{4.923152in}{2.144871in}}%
\pgfpathlineto{\pgfqpoint{4.948143in}{2.143596in}}%
\pgfpathlineto{\pgfqpoint{4.973135in}{2.142503in}}%
\pgfpathlineto{\pgfqpoint{4.998127in}{2.141566in}}%
\pgfpathlineto{\pgfqpoint{5.023119in}{2.140766in}}%
\pgfpathlineto{\pgfqpoint{5.048111in}{2.140084in}}%
\pgfpathlineto{\pgfqpoint{5.073103in}{2.139502in}}%
\pgfpathlineto{\pgfqpoint{5.098095in}{2.139008in}}%
\pgfpathlineto{\pgfqpoint{5.108215in}{2.138838in}}%
\pgfusepath{stroke}%
\end{pgfscope}%
\begin{pgfscope}%
\pgfpathrectangle{\pgfqpoint{3.874531in}{1.689230in}}{\pgfqpoint{1.223684in}{1.004348in}}%
\pgfusepath{clip}%
\pgfsetrectcap%
\pgfsetroundjoin%
\pgfsetlinewidth{0.803000pt}%
\definecolor{currentstroke}{rgb}{0.686275,0.352941,0.313725}%
\pgfsetstrokecolor{currentstroke}%
\pgfsetdash{}{0pt}%
\pgfpathmoveto{\pgfqpoint{3.923478in}{2.136316in}}%
\pgfpathlineto{\pgfqpoint{3.948470in}{2.136316in}}%
\pgfpathlineto{\pgfqpoint{3.973462in}{2.136316in}}%
\pgfpathlineto{\pgfqpoint{3.998454in}{2.136316in}}%
\pgfpathlineto{\pgfqpoint{4.023446in}{2.136316in}}%
\pgfpathlineto{\pgfqpoint{4.048437in}{2.136316in}}%
\pgfpathlineto{\pgfqpoint{4.073429in}{2.136316in}}%
\pgfpathlineto{\pgfqpoint{4.098421in}{2.136316in}}%
\pgfpathlineto{\pgfqpoint{4.123413in}{2.136316in}}%
\pgfpathlineto{\pgfqpoint{4.148405in}{2.112996in}}%
\pgfpathlineto{\pgfqpoint{4.173397in}{2.098130in}}%
\pgfpathlineto{\pgfqpoint{4.198388in}{2.089420in}}%
\pgfpathlineto{\pgfqpoint{4.223380in}{2.085122in}}%
\pgfpathlineto{\pgfqpoint{4.248372in}{2.083923in}}%
\pgfpathlineto{\pgfqpoint{4.273364in}{1.973521in}}%
\pgfpathlineto{\pgfqpoint{4.298356in}{1.904865in}}%
\pgfpathlineto{\pgfqpoint{4.323348in}{1.866448in}}%
\pgfpathlineto{\pgfqpoint{4.348339in}{1.849564in}}%
\pgfpathlineto{\pgfqpoint{4.373331in}{1.847669in}}%
\pgfpathlineto{\pgfqpoint{4.398323in}{1.855883in}}%
\pgfpathlineto{\pgfqpoint{4.423315in}{1.870604in}}%
\pgfpathlineto{\pgfqpoint{4.448307in}{1.889202in}}%
\pgfpathlineto{\pgfqpoint{4.473299in}{1.909789in}}%
\pgfpathlineto{\pgfqpoint{4.498290in}{1.931032in}}%
\pgfpathlineto{\pgfqpoint{4.523282in}{1.952017in}}%
\pgfpathlineto{\pgfqpoint{4.548274in}{1.972139in}}%
\pgfpathlineto{\pgfqpoint{4.573266in}{1.991022in}}%
\pgfpathlineto{\pgfqpoint{4.598258in}{2.008454in}}%
\pgfpathlineto{\pgfqpoint{4.623250in}{2.024341in}}%
\pgfpathlineto{\pgfqpoint{4.648241in}{2.038669in}}%
\pgfpathlineto{\pgfqpoint{4.673233in}{2.051482in}}%
\pgfpathlineto{\pgfqpoint{4.698225in}{2.062858in}}%
\pgfpathlineto{\pgfqpoint{4.723217in}{2.072898in}}%
\pgfpathlineto{\pgfqpoint{4.748209in}{2.081711in}}%
\pgfpathlineto{\pgfqpoint{4.773201in}{2.089413in}}%
\pgfpathlineto{\pgfqpoint{4.798192in}{2.096117in}}%
\pgfpathlineto{\pgfqpoint{4.823184in}{2.101932in}}%
\pgfpathlineto{\pgfqpoint{4.848176in}{2.106960in}}%
\pgfpathlineto{\pgfqpoint{4.873168in}{2.111294in}}%
\pgfpathlineto{\pgfqpoint{4.898160in}{2.115022in}}%
\pgfpathlineto{\pgfqpoint{4.923152in}{2.118220in}}%
\pgfpathlineto{\pgfqpoint{4.948143in}{2.120959in}}%
\pgfpathlineto{\pgfqpoint{4.973135in}{2.123299in}}%
\pgfpathlineto{\pgfqpoint{4.998127in}{2.125296in}}%
\pgfpathlineto{\pgfqpoint{5.023119in}{2.126996in}}%
\pgfpathlineto{\pgfqpoint{5.048111in}{2.128442in}}%
\pgfpathlineto{\pgfqpoint{5.073103in}{2.129670in}}%
\pgfpathlineto{\pgfqpoint{5.098095in}{2.130712in}}%
\pgfpathlineto{\pgfqpoint{5.108215in}{2.131069in}}%
\pgfusepath{stroke}%
\end{pgfscope}%
\begin{pgfscope}%
\pgfpathrectangle{\pgfqpoint{3.874531in}{1.689230in}}{\pgfqpoint{1.223684in}{1.004348in}}%
\pgfusepath{clip}%
\pgfsetrectcap%
\pgfsetroundjoin%
\pgfsetlinewidth{0.803000pt}%
\definecolor{currentstroke}{rgb}{0.000000,0.356863,0.509804}%
\pgfsetstrokecolor{currentstroke}%
\pgfsetdash{}{0pt}%
\pgfpathmoveto{\pgfqpoint{3.923478in}{2.136316in}}%
\pgfpathlineto{\pgfqpoint{3.948470in}{2.136316in}}%
\pgfpathlineto{\pgfqpoint{3.973462in}{2.136316in}}%
\pgfpathlineto{\pgfqpoint{3.998454in}{2.136316in}}%
\pgfpathlineto{\pgfqpoint{4.023446in}{2.136316in}}%
\pgfpathlineto{\pgfqpoint{4.048437in}{2.136316in}}%
\pgfpathlineto{\pgfqpoint{4.073429in}{2.136316in}}%
\pgfpathlineto{\pgfqpoint{4.098421in}{2.136316in}}%
\pgfpathlineto{\pgfqpoint{4.123413in}{2.136316in}}%
\pgfpathlineto{\pgfqpoint{4.148405in}{1.962078in}}%
\pgfpathlineto{\pgfqpoint{4.173397in}{1.851008in}}%
\pgfpathlineto{\pgfqpoint{4.198388in}{1.785930in}}%
\pgfpathlineto{\pgfqpoint{4.223380in}{1.753821in}}%
\pgfpathlineto{\pgfqpoint{4.248372in}{1.744865in}}%
\pgfpathlineto{\pgfqpoint{4.273364in}{1.814011in}}%
\pgfpathlineto{\pgfqpoint{4.298356in}{1.870951in}}%
\pgfpathlineto{\pgfqpoint{4.323348in}{1.917838in}}%
\pgfpathlineto{\pgfqpoint{4.348339in}{1.956447in}}%
\pgfpathlineto{\pgfqpoint{4.373331in}{1.988237in}}%
\pgfpathlineto{\pgfqpoint{4.398323in}{2.014412in}}%
\pgfpathlineto{\pgfqpoint{4.423315in}{2.035964in}}%
\pgfpathlineto{\pgfqpoint{4.448307in}{2.053708in}}%
\pgfpathlineto{\pgfqpoint{4.473299in}{2.068316in}}%
\pgfpathlineto{\pgfqpoint{4.498290in}{2.080343in}}%
\pgfpathlineto{\pgfqpoint{4.523282in}{2.090244in}}%
\pgfpathlineto{\pgfqpoint{4.548274in}{2.098394in}}%
\pgfpathlineto{\pgfqpoint{4.573266in}{2.105104in}}%
\pgfpathlineto{\pgfqpoint{4.598258in}{2.110627in}}%
\pgfpathlineto{\pgfqpoint{4.623250in}{2.115173in}}%
\pgfpathlineto{\pgfqpoint{4.648241in}{2.118916in}}%
\pgfpathlineto{\pgfqpoint{4.673233in}{2.121996in}}%
\pgfpathlineto{\pgfqpoint{4.698225in}{2.124531in}}%
\pgfpathlineto{\pgfqpoint{4.723217in}{2.126618in}}%
\pgfpathlineto{\pgfqpoint{4.748209in}{2.128335in}}%
\pgfpathlineto{\pgfqpoint{4.773201in}{2.129749in}}%
\pgfpathlineto{\pgfqpoint{4.798192in}{2.130912in}}%
\pgfpathlineto{\pgfqpoint{4.823184in}{2.131869in}}%
\pgfpathlineto{\pgfqpoint{4.848176in}{2.132657in}}%
\pgfpathlineto{\pgfqpoint{4.873168in}{2.133305in}}%
\pgfpathlineto{\pgfqpoint{4.898160in}{2.133839in}}%
\pgfpathlineto{\pgfqpoint{4.923152in}{2.134278in}}%
\pgfpathlineto{\pgfqpoint{4.948143in}{2.134639in}}%
\pgfpathlineto{\pgfqpoint{4.973135in}{2.134936in}}%
\pgfpathlineto{\pgfqpoint{4.998127in}{2.135181in}}%
\pgfpathlineto{\pgfqpoint{5.023119in}{2.135382in}}%
\pgfpathlineto{\pgfqpoint{5.048111in}{2.135548in}}%
\pgfpathlineto{\pgfqpoint{5.073103in}{2.135684in}}%
\pgfpathlineto{\pgfqpoint{5.098095in}{2.135796in}}%
\pgfpathlineto{\pgfqpoint{5.108215in}{2.135833in}}%
\pgfusepath{stroke}%
\end{pgfscope}%
\begin{pgfscope}%
\pgfpathrectangle{\pgfqpoint{3.874531in}{1.689230in}}{\pgfqpoint{1.223684in}{1.004348in}}%
\pgfusepath{clip}%
\pgfsetrectcap%
\pgfsetroundjoin%
\pgfsetlinewidth{0.803000pt}%
\definecolor{currentstroke}{rgb}{0.490196,0.588235,0.431373}%
\pgfsetstrokecolor{currentstroke}%
\pgfsetdash{}{0pt}%
\pgfpathmoveto{\pgfqpoint{3.923478in}{2.136316in}}%
\pgfpathlineto{\pgfqpoint{3.948470in}{2.136316in}}%
\pgfpathlineto{\pgfqpoint{3.973462in}{2.136316in}}%
\pgfpathlineto{\pgfqpoint{3.998454in}{2.136316in}}%
\pgfpathlineto{\pgfqpoint{4.023446in}{2.136316in}}%
\pgfpathlineto{\pgfqpoint{4.048437in}{2.136316in}}%
\pgfpathlineto{\pgfqpoint{4.073429in}{2.136316in}}%
\pgfpathlineto{\pgfqpoint{4.098421in}{2.136316in}}%
\pgfpathlineto{\pgfqpoint{4.123413in}{2.136316in}}%
\pgfpathlineto{\pgfqpoint{4.148405in}{2.113804in}}%
\pgfpathlineto{\pgfqpoint{4.173397in}{2.099453in}}%
\pgfpathlineto{\pgfqpoint{4.198388in}{2.091045in}}%
\pgfpathlineto{\pgfqpoint{4.223380in}{2.086896in}}%
\pgfpathlineto{\pgfqpoint{4.248372in}{2.085739in}}%
\pgfpathlineto{\pgfqpoint{4.273364in}{2.003875in}}%
\pgfpathlineto{\pgfqpoint{4.298356in}{1.953351in}}%
\pgfpathlineto{\pgfqpoint{4.323348in}{1.925496in}}%
\pgfpathlineto{\pgfqpoint{4.348339in}{1.913752in}}%
\pgfpathlineto{\pgfqpoint{4.373331in}{1.913192in}}%
\pgfpathlineto{\pgfqpoint{4.398323in}{1.920149in}}%
\pgfpathlineto{\pgfqpoint{4.423315in}{1.931915in}}%
\pgfpathlineto{\pgfqpoint{4.448307in}{1.946518in}}%
\pgfpathlineto{\pgfqpoint{4.473299in}{1.962543in}}%
\pgfpathlineto{\pgfqpoint{4.498290in}{1.978995in}}%
\pgfpathlineto{\pgfqpoint{4.523282in}{1.995193in}}%
\pgfpathlineto{\pgfqpoint{4.548274in}{2.010687in}}%
\pgfpathlineto{\pgfqpoint{4.573266in}{2.025202in}}%
\pgfpathlineto{\pgfqpoint{4.598258in}{2.038582in}}%
\pgfpathlineto{\pgfqpoint{4.623250in}{2.050762in}}%
\pgfpathlineto{\pgfqpoint{4.648241in}{2.061738in}}%
\pgfpathlineto{\pgfqpoint{4.673233in}{2.071546in}}%
\pgfpathlineto{\pgfqpoint{4.698225in}{2.080249in}}%
\pgfpathlineto{\pgfqpoint{4.723217in}{2.087925in}}%
\pgfpathlineto{\pgfqpoint{4.748209in}{2.094660in}}%
\pgfpathlineto{\pgfqpoint{4.773201in}{2.100544in}}%
\pgfpathlineto{\pgfqpoint{4.798192in}{2.105663in}}%
\pgfpathlineto{\pgfqpoint{4.823184in}{2.110102in}}%
\pgfpathlineto{\pgfqpoint{4.848176in}{2.113939in}}%
\pgfpathlineto{\pgfqpoint{4.873168in}{2.117246in}}%
\pgfpathlineto{\pgfqpoint{4.898160in}{2.120089in}}%
\pgfpathlineto{\pgfqpoint{4.923152in}{2.122528in}}%
\pgfpathlineto{\pgfqpoint{4.948143in}{2.124616in}}%
\pgfpathlineto{\pgfqpoint{4.973135in}{2.126401in}}%
\pgfpathlineto{\pgfqpoint{4.998127in}{2.127922in}}%
\pgfpathlineto{\pgfqpoint{5.023119in}{2.129218in}}%
\pgfpathlineto{\pgfqpoint{5.048111in}{2.130320in}}%
\pgfpathlineto{\pgfqpoint{5.073103in}{2.131256in}}%
\pgfpathlineto{\pgfqpoint{5.098095in}{2.132049in}}%
\pgfpathlineto{\pgfqpoint{5.108215in}{2.132321in}}%
\pgfusepath{stroke}%
\end{pgfscope}%
\begin{pgfscope}%
\pgfpathrectangle{\pgfqpoint{3.874531in}{1.689230in}}{\pgfqpoint{1.223684in}{1.004348in}}%
\pgfusepath{clip}%
\pgfsetrectcap%
\pgfsetroundjoin%
\pgfsetlinewidth{0.803000pt}%
\definecolor{currentstroke}{rgb}{0.843137,0.666667,0.313725}%
\pgfsetstrokecolor{currentstroke}%
\pgfsetdash{}{0pt}%
\pgfpathmoveto{\pgfqpoint{3.923478in}{2.136316in}}%
\pgfpathlineto{\pgfqpoint{3.948470in}{2.136316in}}%
\pgfpathlineto{\pgfqpoint{3.973462in}{2.136316in}}%
\pgfpathlineto{\pgfqpoint{3.998454in}{2.136316in}}%
\pgfpathlineto{\pgfqpoint{4.023446in}{2.136316in}}%
\pgfpathlineto{\pgfqpoint{4.048437in}{2.136316in}}%
\pgfpathlineto{\pgfqpoint{4.073429in}{2.136316in}}%
\pgfpathlineto{\pgfqpoint{4.098421in}{2.136316in}}%
\pgfpathlineto{\pgfqpoint{4.123413in}{2.136316in}}%
\pgfpathlineto{\pgfqpoint{4.148405in}{2.159218in}}%
\pgfpathlineto{\pgfqpoint{4.173397in}{2.173818in}}%
\pgfpathlineto{\pgfqpoint{4.198388in}{2.182372in}}%
\pgfpathlineto{\pgfqpoint{4.223380in}{2.186593in}}%
\pgfpathlineto{\pgfqpoint{4.248372in}{2.187770in}}%
\pgfpathlineto{\pgfqpoint{4.273364in}{1.991336in}}%
\pgfpathlineto{\pgfqpoint{4.298356in}{1.864426in}}%
\pgfpathlineto{\pgfqpoint{4.323348in}{1.788290in}}%
\pgfpathlineto{\pgfqpoint{4.348339in}{1.748690in}}%
\pgfpathlineto{\pgfqpoint{4.373331in}{1.734882in}}%
\pgfpathlineto{\pgfqpoint{4.398323in}{1.738817in}}%
\pgfpathlineto{\pgfqpoint{4.423315in}{1.754516in}}%
\pgfpathlineto{\pgfqpoint{4.448307in}{1.777585in}}%
\pgfpathlineto{\pgfqpoint{4.473299in}{1.804835in}}%
\pgfpathlineto{\pgfqpoint{4.498290in}{1.833994in}}%
\pgfpathlineto{\pgfqpoint{4.523282in}{1.863474in}}%
\pgfpathlineto{\pgfqpoint{4.548274in}{1.892199in}}%
\pgfpathlineto{\pgfqpoint{4.573266in}{1.919476in}}%
\pgfpathlineto{\pgfqpoint{4.598258in}{1.944886in}}%
\pgfpathlineto{\pgfqpoint{4.623250in}{1.968208in}}%
\pgfpathlineto{\pgfqpoint{4.648241in}{1.989365in}}%
\pgfpathlineto{\pgfqpoint{4.673233in}{2.008376in}}%
\pgfpathlineto{\pgfqpoint{4.698225in}{2.025323in}}%
\pgfpathlineto{\pgfqpoint{4.723217in}{2.040330in}}%
\pgfpathlineto{\pgfqpoint{4.748209in}{2.053544in}}%
\pgfpathlineto{\pgfqpoint{4.773201in}{2.065121in}}%
\pgfpathlineto{\pgfqpoint{4.798192in}{2.075221in}}%
\pgfpathlineto{\pgfqpoint{4.823184in}{2.083998in}}%
\pgfpathlineto{\pgfqpoint{4.848176in}{2.091601in}}%
\pgfpathlineto{\pgfqpoint{4.873168in}{2.098167in}}%
\pgfpathlineto{\pgfqpoint{4.898160in}{2.103821in}}%
\pgfpathlineto{\pgfqpoint{4.923152in}{2.108679in}}%
\pgfpathlineto{\pgfqpoint{4.948143in}{2.112844in}}%
\pgfpathlineto{\pgfqpoint{4.973135in}{2.116407in}}%
\pgfpathlineto{\pgfqpoint{4.998127in}{2.119449in}}%
\pgfpathlineto{\pgfqpoint{5.023119in}{2.122043in}}%
\pgfpathlineto{\pgfqpoint{5.048111in}{2.124250in}}%
\pgfpathlineto{\pgfqpoint{5.073103in}{2.126127in}}%
\pgfpathlineto{\pgfqpoint{5.098095in}{2.127719in}}%
\pgfpathlineto{\pgfqpoint{5.108215in}{2.128266in}}%
\pgfusepath{stroke}%
\end{pgfscope}%
\begin{pgfscope}%
\pgfpathrectangle{\pgfqpoint{3.874531in}{1.689230in}}{\pgfqpoint{1.223684in}{1.004348in}}%
\pgfusepath{clip}%
\pgfsetrectcap%
\pgfsetroundjoin%
\pgfsetlinewidth{0.803000pt}%
\definecolor{currentstroke}{rgb}{0.333333,0.333333,0.333333}%
\pgfsetstrokecolor{currentstroke}%
\pgfsetdash{}{0pt}%
\pgfpathmoveto{\pgfqpoint{3.923478in}{2.136316in}}%
\pgfpathlineto{\pgfqpoint{3.948470in}{2.136316in}}%
\pgfpathlineto{\pgfqpoint{3.973462in}{2.136316in}}%
\pgfpathlineto{\pgfqpoint{3.998454in}{2.136316in}}%
\pgfpathlineto{\pgfqpoint{4.023446in}{2.136316in}}%
\pgfpathlineto{\pgfqpoint{4.048437in}{2.136316in}}%
\pgfpathlineto{\pgfqpoint{4.073429in}{2.136316in}}%
\pgfpathlineto{\pgfqpoint{4.098421in}{2.136316in}}%
\pgfpathlineto{\pgfqpoint{4.123413in}{2.136316in}}%
\pgfpathlineto{\pgfqpoint{4.148405in}{2.062825in}}%
\pgfpathlineto{\pgfqpoint{4.173397in}{2.015978in}}%
\pgfpathlineto{\pgfqpoint{4.198388in}{1.988529in}}%
\pgfpathlineto{\pgfqpoint{4.223380in}{1.974985in}}%
\pgfpathlineto{\pgfqpoint{4.248372in}{1.971208in}}%
\pgfpathlineto{\pgfqpoint{4.273364in}{2.010027in}}%
\pgfpathlineto{\pgfqpoint{4.298356in}{2.040198in}}%
\pgfpathlineto{\pgfqpoint{4.323348in}{2.063581in}}%
\pgfpathlineto{\pgfqpoint{4.348339in}{2.081645in}}%
\pgfpathlineto{\pgfqpoint{4.373331in}{2.095550in}}%
\pgfpathlineto{\pgfqpoint{4.398323in}{2.106210in}}%
\pgfpathlineto{\pgfqpoint{4.423315in}{2.114345in}}%
\pgfpathlineto{\pgfqpoint{4.448307in}{2.120520in}}%
\pgfpathlineto{\pgfqpoint{4.473299in}{2.125178in}}%
\pgfpathlineto{\pgfqpoint{4.498290in}{2.128667in}}%
\pgfpathlineto{\pgfqpoint{4.523282in}{2.131256in}}%
\pgfpathlineto{\pgfqpoint{4.548274in}{2.133159in}}%
\pgfpathlineto{\pgfqpoint{4.573266in}{2.134537in}}%
\pgfpathlineto{\pgfqpoint{4.598258in}{2.135520in}}%
\pgfpathlineto{\pgfqpoint{4.623250in}{2.136205in}}%
\pgfpathlineto{\pgfqpoint{4.648241in}{2.136668in}}%
\pgfpathlineto{\pgfqpoint{4.673233in}{2.136966in}}%
\pgfpathlineto{\pgfqpoint{4.698225in}{2.137145in}}%
\pgfpathlineto{\pgfqpoint{4.723217in}{2.137238in}}%
\pgfpathlineto{\pgfqpoint{4.748209in}{2.137269in}}%
\pgfpathlineto{\pgfqpoint{4.773201in}{2.137259in}}%
\pgfpathlineto{\pgfqpoint{4.798192in}{2.137222in}}%
\pgfpathlineto{\pgfqpoint{4.823184in}{2.137167in}}%
\pgfpathlineto{\pgfqpoint{4.848176in}{2.137102in}}%
\pgfpathlineto{\pgfqpoint{4.873168in}{2.137032in}}%
\pgfpathlineto{\pgfqpoint{4.898160in}{2.136962in}}%
\pgfpathlineto{\pgfqpoint{4.923152in}{2.136894in}}%
\pgfpathlineto{\pgfqpoint{4.948143in}{2.136830in}}%
\pgfpathlineto{\pgfqpoint{4.973135in}{2.136769in}}%
\pgfpathlineto{\pgfqpoint{4.998127in}{2.136714in}}%
\pgfpathlineto{\pgfqpoint{5.023119in}{2.136664in}}%
\pgfpathlineto{\pgfqpoint{5.048111in}{2.136619in}}%
\pgfpathlineto{\pgfqpoint{5.073103in}{2.136579in}}%
\pgfpathlineto{\pgfqpoint{5.098095in}{2.136543in}}%
\pgfpathlineto{\pgfqpoint{5.108215in}{2.136530in}}%
\pgfusepath{stroke}%
\end{pgfscope}%
\begin{pgfscope}%
\pgfpathrectangle{\pgfqpoint{3.874531in}{1.689230in}}{\pgfqpoint{1.223684in}{1.004348in}}%
\pgfusepath{clip}%
\pgfsetrectcap%
\pgfsetroundjoin%
\pgfsetlinewidth{0.803000pt}%
\definecolor{currentstroke}{rgb}{0.686275,0.352941,0.313725}%
\pgfsetstrokecolor{currentstroke}%
\pgfsetdash{}{0pt}%
\pgfpathmoveto{\pgfqpoint{3.923478in}{2.136316in}}%
\pgfpathlineto{\pgfqpoint{3.948470in}{2.136316in}}%
\pgfpathlineto{\pgfqpoint{3.973462in}{2.136316in}}%
\pgfpathlineto{\pgfqpoint{3.998454in}{2.136316in}}%
\pgfpathlineto{\pgfqpoint{4.023446in}{2.136316in}}%
\pgfpathlineto{\pgfqpoint{4.048437in}{2.136316in}}%
\pgfpathlineto{\pgfqpoint{4.073429in}{2.136316in}}%
\pgfpathlineto{\pgfqpoint{4.098421in}{2.136316in}}%
\pgfpathlineto{\pgfqpoint{4.123413in}{2.136316in}}%
\pgfpathlineto{\pgfqpoint{4.148405in}{2.183510in}}%
\pgfpathlineto{\pgfqpoint{4.173397in}{2.213594in}}%
\pgfpathlineto{\pgfqpoint{4.198388in}{2.231221in}}%
\pgfpathlineto{\pgfqpoint{4.223380in}{2.239918in}}%
\pgfpathlineto{\pgfqpoint{4.248372in}{2.242344in}}%
\pgfpathlineto{\pgfqpoint{4.273364in}{2.335865in}}%
\pgfpathlineto{\pgfqpoint{4.298356in}{2.391997in}}%
\pgfpathlineto{\pgfqpoint{4.323348in}{2.421223in}}%
\pgfpathlineto{\pgfqpoint{4.348339in}{2.431452in}}%
\pgfpathlineto{\pgfqpoint{4.373331in}{2.428610in}}%
\pgfpathlineto{\pgfqpoint{4.398323in}{2.417101in}}%
\pgfpathlineto{\pgfqpoint{4.423315in}{2.400161in}}%
\pgfpathlineto{\pgfqpoint{4.448307in}{2.380135in}}%
\pgfpathlineto{\pgfqpoint{4.473299in}{2.358700in}}%
\pgfpathlineto{\pgfqpoint{4.498290in}{2.337025in}}%
\pgfpathlineto{\pgfqpoint{4.523282in}{2.315900in}}%
\pgfpathlineto{\pgfqpoint{4.548274in}{2.295839in}}%
\pgfpathlineto{\pgfqpoint{4.573266in}{2.277150in}}%
\pgfpathlineto{\pgfqpoint{4.598258in}{2.259995in}}%
\pgfpathlineto{\pgfqpoint{4.623250in}{2.244431in}}%
\pgfpathlineto{\pgfqpoint{4.648241in}{2.230446in}}%
\pgfpathlineto{\pgfqpoint{4.673233in}{2.217979in}}%
\pgfpathlineto{\pgfqpoint{4.698225in}{2.206939in}}%
\pgfpathlineto{\pgfqpoint{4.723217in}{2.197217in}}%
\pgfpathlineto{\pgfqpoint{4.748209in}{2.188700in}}%
\pgfpathlineto{\pgfqpoint{4.773201in}{2.181269in}}%
\pgfpathlineto{\pgfqpoint{4.798192in}{2.174811in}}%
\pgfpathlineto{\pgfqpoint{4.823184in}{2.169217in}}%
\pgfpathlineto{\pgfqpoint{4.848176in}{2.164387in}}%
\pgfpathlineto{\pgfqpoint{4.873168in}{2.160226in}}%
\pgfpathlineto{\pgfqpoint{4.898160in}{2.156651in}}%
\pgfpathlineto{\pgfqpoint{4.923152in}{2.153587in}}%
\pgfpathlineto{\pgfqpoint{4.948143in}{2.150966in}}%
\pgfpathlineto{\pgfqpoint{4.973135in}{2.148727in}}%
\pgfpathlineto{\pgfqpoint{4.998127in}{2.146819in}}%
\pgfpathlineto{\pgfqpoint{5.023119in}{2.145194in}}%
\pgfpathlineto{\pgfqpoint{5.048111in}{2.143814in}}%
\pgfpathlineto{\pgfqpoint{5.073103in}{2.142642in}}%
\pgfpathlineto{\pgfqpoint{5.098095in}{2.141648in}}%
\pgfpathlineto{\pgfqpoint{5.108215in}{2.141308in}}%
\pgfusepath{stroke}%
\end{pgfscope}%
\begin{pgfscope}%
\pgfpathrectangle{\pgfqpoint{3.874531in}{1.689230in}}{\pgfqpoint{1.223684in}{1.004348in}}%
\pgfusepath{clip}%
\pgfsetrectcap%
\pgfsetroundjoin%
\pgfsetlinewidth{0.803000pt}%
\definecolor{currentstroke}{rgb}{0.000000,0.356863,0.509804}%
\pgfsetstrokecolor{currentstroke}%
\pgfsetdash{}{0pt}%
\pgfpathmoveto{\pgfqpoint{3.923478in}{2.136316in}}%
\pgfpathlineto{\pgfqpoint{3.948470in}{2.136316in}}%
\pgfpathlineto{\pgfqpoint{3.973462in}{2.136316in}}%
\pgfpathlineto{\pgfqpoint{3.998454in}{2.136316in}}%
\pgfpathlineto{\pgfqpoint{4.023446in}{2.136316in}}%
\pgfpathlineto{\pgfqpoint{4.048437in}{2.136316in}}%
\pgfpathlineto{\pgfqpoint{4.073429in}{2.136316in}}%
\pgfpathlineto{\pgfqpoint{4.098421in}{2.136316in}}%
\pgfpathlineto{\pgfqpoint{4.123413in}{2.136316in}}%
\pgfpathlineto{\pgfqpoint{4.148405in}{2.155707in}}%
\pgfpathlineto{\pgfqpoint{4.173397in}{2.168069in}}%
\pgfpathlineto{\pgfqpoint{4.198388in}{2.175311in}}%
\pgfpathlineto{\pgfqpoint{4.223380in}{2.178885in}}%
\pgfpathlineto{\pgfqpoint{4.248372in}{2.179881in}}%
\pgfpathlineto{\pgfqpoint{4.273364in}{2.250279in}}%
\pgfpathlineto{\pgfqpoint{4.298356in}{2.293723in}}%
\pgfpathlineto{\pgfqpoint{4.323348in}{2.317672in}}%
\pgfpathlineto{\pgfqpoint{4.348339in}{2.327767in}}%
\pgfpathlineto{\pgfqpoint{4.373331in}{2.328243in}}%
\pgfpathlineto{\pgfqpoint{4.398323in}{2.322255in}}%
\pgfpathlineto{\pgfqpoint{4.423315in}{2.312132in}}%
\pgfpathlineto{\pgfqpoint{4.448307in}{2.299569in}}%
\pgfpathlineto{\pgfqpoint{4.473299in}{2.285784in}}%
\pgfpathlineto{\pgfqpoint{4.498290in}{2.271632in}}%
\pgfpathlineto{\pgfqpoint{4.523282in}{2.257699in}}%
\pgfpathlineto{\pgfqpoint{4.548274in}{2.244371in}}%
\pgfpathlineto{\pgfqpoint{4.573266in}{2.231887in}}%
\pgfpathlineto{\pgfqpoint{4.598258in}{2.220378in}}%
\pgfpathlineto{\pgfqpoint{4.623250in}{2.209901in}}%
\pgfpathlineto{\pgfqpoint{4.648241in}{2.200460in}}%
\pgfpathlineto{\pgfqpoint{4.673233in}{2.192024in}}%
\pgfpathlineto{\pgfqpoint{4.698225in}{2.184539in}}%
\pgfpathlineto{\pgfqpoint{4.723217in}{2.177937in}}%
\pgfpathlineto{\pgfqpoint{4.748209in}{2.172144in}}%
\pgfpathlineto{\pgfqpoint{4.773201in}{2.167083in}}%
\pgfpathlineto{\pgfqpoint{4.798192in}{2.162680in}}%
\pgfpathlineto{\pgfqpoint{4.823184in}{2.158862in}}%
\pgfpathlineto{\pgfqpoint{4.848176in}{2.155562in}}%
\pgfpathlineto{\pgfqpoint{4.873168in}{2.152718in}}%
\pgfpathlineto{\pgfqpoint{4.898160in}{2.150272in}}%
\pgfpathlineto{\pgfqpoint{4.923152in}{2.148174in}}%
\pgfpathlineto{\pgfqpoint{4.948143in}{2.146378in}}%
\pgfpathlineto{\pgfqpoint{4.973135in}{2.144844in}}%
\pgfpathlineto{\pgfqpoint{4.998127in}{2.143535in}}%
\pgfpathlineto{\pgfqpoint{5.023119in}{2.142420in}}%
\pgfpathlineto{\pgfqpoint{5.048111in}{2.141473in}}%
\pgfpathlineto{\pgfqpoint{5.073103in}{2.140668in}}%
\pgfpathlineto{\pgfqpoint{5.098095in}{2.139986in}}%
\pgfpathlineto{\pgfqpoint{5.108215in}{2.139752in}}%
\pgfusepath{stroke}%
\end{pgfscope}%
\begin{pgfscope}%
\pgfpathrectangle{\pgfqpoint{3.874531in}{1.689230in}}{\pgfqpoint{1.223684in}{1.004348in}}%
\pgfusepath{clip}%
\pgfsetrectcap%
\pgfsetroundjoin%
\pgfsetlinewidth{0.803000pt}%
\definecolor{currentstroke}{rgb}{0.490196,0.588235,0.431373}%
\pgfsetstrokecolor{currentstroke}%
\pgfsetdash{}{0pt}%
\pgfpathmoveto{\pgfqpoint{3.923478in}{2.136316in}}%
\pgfpathlineto{\pgfqpoint{3.948470in}{2.136316in}}%
\pgfpathlineto{\pgfqpoint{3.973462in}{2.136316in}}%
\pgfpathlineto{\pgfqpoint{3.998454in}{2.136316in}}%
\pgfpathlineto{\pgfqpoint{4.023446in}{2.136316in}}%
\pgfpathlineto{\pgfqpoint{4.048437in}{2.136316in}}%
\pgfpathlineto{\pgfqpoint{4.073429in}{2.136316in}}%
\pgfpathlineto{\pgfqpoint{4.098421in}{2.136316in}}%
\pgfpathlineto{\pgfqpoint{4.123413in}{2.136316in}}%
\pgfpathlineto{\pgfqpoint{4.148405in}{2.258359in}}%
\pgfpathlineto{\pgfqpoint{4.173397in}{2.336157in}}%
\pgfpathlineto{\pgfqpoint{4.198388in}{2.381739in}}%
\pgfpathlineto{\pgfqpoint{4.223380in}{2.404230in}}%
\pgfpathlineto{\pgfqpoint{4.248372in}{2.410503in}}%
\pgfpathlineto{\pgfqpoint{4.273364in}{2.408981in}}%
\pgfpathlineto{\pgfqpoint{4.298356in}{2.399001in}}%
\pgfpathlineto{\pgfqpoint{4.323348in}{2.383680in}}%
\pgfpathlineto{\pgfqpoint{4.348339in}{2.365282in}}%
\pgfpathlineto{\pgfqpoint{4.373331in}{2.345426in}}%
\pgfpathlineto{\pgfqpoint{4.398323in}{2.325245in}}%
\pgfpathlineto{\pgfqpoint{4.423315in}{2.305509in}}%
\pgfpathlineto{\pgfqpoint{4.448307in}{2.286720in}}%
\pgfpathlineto{\pgfqpoint{4.473299in}{2.269184in}}%
\pgfpathlineto{\pgfqpoint{4.498290in}{2.253063in}}%
\pgfpathlineto{\pgfqpoint{4.523282in}{2.238421in}}%
\pgfpathlineto{\pgfqpoint{4.548274in}{2.225251in}}%
\pgfpathlineto{\pgfqpoint{4.573266in}{2.213500in}}%
\pgfpathlineto{\pgfqpoint{4.598258in}{2.203088in}}%
\pgfpathlineto{\pgfqpoint{4.623250in}{2.193914in}}%
\pgfpathlineto{\pgfqpoint{4.648241in}{2.185872in}}%
\pgfpathlineto{\pgfqpoint{4.673233in}{2.178853in}}%
\pgfpathlineto{\pgfqpoint{4.698225in}{2.172750in}}%
\pgfpathlineto{\pgfqpoint{4.723217in}{2.167462in}}%
\pgfpathlineto{\pgfqpoint{4.748209in}{2.162894in}}%
\pgfpathlineto{\pgfqpoint{4.773201in}{2.158959in}}%
\pgfpathlineto{\pgfqpoint{4.798192in}{2.155577in}}%
\pgfpathlineto{\pgfqpoint{4.823184in}{2.152677in}}%
\pgfpathlineto{\pgfqpoint{4.848176in}{2.150195in}}%
\pgfpathlineto{\pgfqpoint{4.873168in}{2.148076in}}%
\pgfpathlineto{\pgfqpoint{4.898160in}{2.146269in}}%
\pgfpathlineto{\pgfqpoint{4.923152in}{2.144730in}}%
\pgfpathlineto{\pgfqpoint{4.948143in}{2.143423in}}%
\pgfpathlineto{\pgfqpoint{4.973135in}{2.142313in}}%
\pgfpathlineto{\pgfqpoint{4.998127in}{2.141371in}}%
\pgfpathlineto{\pgfqpoint{5.023119in}{2.140574in}}%
\pgfpathlineto{\pgfqpoint{5.048111in}{2.139900in}}%
\pgfpathlineto{\pgfqpoint{5.073103in}{2.139331in}}%
\pgfpathlineto{\pgfqpoint{5.098095in}{2.138850in}}%
\pgfpathlineto{\pgfqpoint{5.108215in}{2.138685in}}%
\pgfusepath{stroke}%
\end{pgfscope}%
\begin{pgfscope}%
\pgfpathrectangle{\pgfqpoint{3.874531in}{1.689230in}}{\pgfqpoint{1.223684in}{1.004348in}}%
\pgfusepath{clip}%
\pgfsetrectcap%
\pgfsetroundjoin%
\pgfsetlinewidth{0.803000pt}%
\definecolor{currentstroke}{rgb}{0.843137,0.666667,0.313725}%
\pgfsetstrokecolor{currentstroke}%
\pgfsetdash{}{0pt}%
\pgfpathmoveto{\pgfqpoint{3.923478in}{2.136316in}}%
\pgfpathlineto{\pgfqpoint{3.948470in}{2.136316in}}%
\pgfpathlineto{\pgfqpoint{3.973462in}{2.136316in}}%
\pgfpathlineto{\pgfqpoint{3.998454in}{2.136316in}}%
\pgfpathlineto{\pgfqpoint{4.023446in}{2.136316in}}%
\pgfpathlineto{\pgfqpoint{4.048437in}{2.136316in}}%
\pgfpathlineto{\pgfqpoint{4.073429in}{2.136316in}}%
\pgfpathlineto{\pgfqpoint{4.098421in}{2.136316in}}%
\pgfpathlineto{\pgfqpoint{4.123413in}{2.136316in}}%
\pgfpathlineto{\pgfqpoint{4.148405in}{2.072595in}}%
\pgfpathlineto{\pgfqpoint{4.173397in}{2.031975in}}%
\pgfpathlineto{\pgfqpoint{4.198388in}{2.008175in}}%
\pgfpathlineto{\pgfqpoint{4.223380in}{1.996432in}}%
\pgfpathlineto{\pgfqpoint{4.248372in}{1.993157in}}%
\pgfpathlineto{\pgfqpoint{4.273364in}{1.988764in}}%
\pgfpathlineto{\pgfqpoint{4.298356in}{1.990667in}}%
\pgfpathlineto{\pgfqpoint{4.323348in}{1.996729in}}%
\pgfpathlineto{\pgfqpoint{4.348339in}{2.005379in}}%
\pgfpathlineto{\pgfqpoint{4.373331in}{2.015480in}}%
\pgfpathlineto{\pgfqpoint{4.398323in}{2.026221in}}%
\pgfpathlineto{\pgfqpoint{4.423315in}{2.037038in}}%
\pgfpathlineto{\pgfqpoint{4.448307in}{2.047552in}}%
\pgfpathlineto{\pgfqpoint{4.473299in}{2.057516in}}%
\pgfpathlineto{\pgfqpoint{4.498290in}{2.066784in}}%
\pgfpathlineto{\pgfqpoint{4.523282in}{2.075281in}}%
\pgfpathlineto{\pgfqpoint{4.548274in}{2.082983in}}%
\pgfpathlineto{\pgfqpoint{4.573266in}{2.089898in}}%
\pgfpathlineto{\pgfqpoint{4.598258in}{2.096058in}}%
\pgfpathlineto{\pgfqpoint{4.623250in}{2.101511in}}%
\pgfpathlineto{\pgfqpoint{4.648241in}{2.106309in}}%
\pgfpathlineto{\pgfqpoint{4.673233in}{2.110511in}}%
\pgfpathlineto{\pgfqpoint{4.698225in}{2.114176in}}%
\pgfpathlineto{\pgfqpoint{4.723217in}{2.117360in}}%
\pgfpathlineto{\pgfqpoint{4.748209in}{2.120118in}}%
\pgfpathlineto{\pgfqpoint{4.773201in}{2.122498in}}%
\pgfpathlineto{\pgfqpoint{4.798192in}{2.124548in}}%
\pgfpathlineto{\pgfqpoint{4.823184in}{2.126308in}}%
\pgfpathlineto{\pgfqpoint{4.848176in}{2.127818in}}%
\pgfpathlineto{\pgfqpoint{4.873168in}{2.129108in}}%
\pgfpathlineto{\pgfqpoint{4.898160in}{2.130210in}}%
\pgfpathlineto{\pgfqpoint{4.923152in}{2.131150in}}%
\pgfpathlineto{\pgfqpoint{4.948143in}{2.131949in}}%
\pgfpathlineto{\pgfqpoint{4.973135in}{2.132628in}}%
\pgfpathlineto{\pgfqpoint{4.998127in}{2.133205in}}%
\pgfpathlineto{\pgfqpoint{5.023119in}{2.133694in}}%
\pgfpathlineto{\pgfqpoint{5.048111in}{2.134107in}}%
\pgfpathlineto{\pgfqpoint{5.073103in}{2.134457in}}%
\pgfpathlineto{\pgfqpoint{5.098095in}{2.134753in}}%
\pgfpathlineto{\pgfqpoint{5.108215in}{2.134854in}}%
\pgfusepath{stroke}%
\end{pgfscope}%
\begin{pgfscope}%
\pgfpathrectangle{\pgfqpoint{3.874531in}{1.689230in}}{\pgfqpoint{1.223684in}{1.004348in}}%
\pgfusepath{clip}%
\pgfsetrectcap%
\pgfsetroundjoin%
\pgfsetlinewidth{0.803000pt}%
\definecolor{currentstroke}{rgb}{0.333333,0.333333,0.333333}%
\pgfsetstrokecolor{currentstroke}%
\pgfsetdash{}{0pt}%
\pgfpathmoveto{\pgfqpoint{3.923478in}{2.136316in}}%
\pgfpathlineto{\pgfqpoint{3.948470in}{2.136316in}}%
\pgfpathlineto{\pgfqpoint{3.973462in}{2.136316in}}%
\pgfpathlineto{\pgfqpoint{3.998454in}{2.136316in}}%
\pgfpathlineto{\pgfqpoint{4.023446in}{2.136316in}}%
\pgfpathlineto{\pgfqpoint{4.048437in}{2.136316in}}%
\pgfpathlineto{\pgfqpoint{4.073429in}{2.136316in}}%
\pgfpathlineto{\pgfqpoint{4.098421in}{2.136316in}}%
\pgfpathlineto{\pgfqpoint{4.123413in}{2.136316in}}%
\pgfpathlineto{\pgfqpoint{4.148405in}{2.020326in}}%
\pgfpathlineto{\pgfqpoint{4.173397in}{1.946388in}}%
\pgfpathlineto{\pgfqpoint{4.198388in}{1.903066in}}%
\pgfpathlineto{\pgfqpoint{4.223380in}{1.881690in}}%
\pgfpathlineto{\pgfqpoint{4.248372in}{1.875729in}}%
\pgfpathlineto{\pgfqpoint{4.273364in}{1.868512in}}%
\pgfpathlineto{\pgfqpoint{4.298356in}{1.872474in}}%
\pgfpathlineto{\pgfqpoint{4.323348in}{1.883799in}}%
\pgfpathlineto{\pgfqpoint{4.348339in}{1.899688in}}%
\pgfpathlineto{\pgfqpoint{4.373331in}{1.918114in}}%
\pgfpathlineto{\pgfqpoint{4.398323in}{1.937635in}}%
\pgfpathlineto{\pgfqpoint{4.423315in}{1.957248in}}%
\pgfpathlineto{\pgfqpoint{4.448307in}{1.976280in}}%
\pgfpathlineto{\pgfqpoint{4.473299in}{1.994295in}}%
\pgfpathlineto{\pgfqpoint{4.498290in}{2.011038in}}%
\pgfpathlineto{\pgfqpoint{4.523282in}{2.026378in}}%
\pgfpathlineto{\pgfqpoint{4.548274in}{2.040273in}}%
\pgfpathlineto{\pgfqpoint{4.573266in}{2.052742in}}%
\pgfpathlineto{\pgfqpoint{4.598258in}{2.063847in}}%
\pgfpathlineto{\pgfqpoint{4.623250in}{2.073672in}}%
\pgfpathlineto{\pgfqpoint{4.648241in}{2.082316in}}%
\pgfpathlineto{\pgfqpoint{4.673233in}{2.089885in}}%
\pgfpathlineto{\pgfqpoint{4.698225in}{2.096484in}}%
\pgfpathlineto{\pgfqpoint{4.723217in}{2.102217in}}%
\pgfpathlineto{\pgfqpoint{4.748209in}{2.107180in}}%
\pgfpathlineto{\pgfqpoint{4.773201in}{2.111464in}}%
\pgfpathlineto{\pgfqpoint{4.798192in}{2.115152in}}%
\pgfpathlineto{\pgfqpoint{4.823184in}{2.118320in}}%
\pgfpathlineto{\pgfqpoint{4.848176in}{2.121035in}}%
\pgfpathlineto{\pgfqpoint{4.873168in}{2.123357in}}%
\pgfpathlineto{\pgfqpoint{4.898160in}{2.125339in}}%
\pgfpathlineto{\pgfqpoint{4.923152in}{2.127028in}}%
\pgfpathlineto{\pgfqpoint{4.948143in}{2.128466in}}%
\pgfpathlineto{\pgfqpoint{4.973135in}{2.129687in}}%
\pgfpathlineto{\pgfqpoint{4.998127in}{2.130724in}}%
\pgfpathlineto{\pgfqpoint{5.023119in}{2.131603in}}%
\pgfpathlineto{\pgfqpoint{5.048111in}{2.132347in}}%
\pgfpathlineto{\pgfqpoint{5.073103in}{2.132976in}}%
\pgfpathlineto{\pgfqpoint{5.098095in}{2.133507in}}%
\pgfpathlineto{\pgfqpoint{5.108215in}{2.133689in}}%
\pgfusepath{stroke}%
\end{pgfscope}%
\begin{pgfscope}%
\pgfpathrectangle{\pgfqpoint{3.874531in}{1.689230in}}{\pgfqpoint{1.223684in}{1.004348in}}%
\pgfusepath{clip}%
\pgfsetrectcap%
\pgfsetroundjoin%
\pgfsetlinewidth{0.803000pt}%
\definecolor{currentstroke}{rgb}{0.686275,0.352941,0.313725}%
\pgfsetstrokecolor{currentstroke}%
\pgfsetdash{}{0pt}%
\pgfpathmoveto{\pgfqpoint{3.923478in}{2.136316in}}%
\pgfpathlineto{\pgfqpoint{3.948470in}{2.136316in}}%
\pgfpathlineto{\pgfqpoint{3.973462in}{2.136316in}}%
\pgfpathlineto{\pgfqpoint{3.998454in}{2.136316in}}%
\pgfpathlineto{\pgfqpoint{4.023446in}{2.136316in}}%
\pgfpathlineto{\pgfqpoint{4.048437in}{2.136316in}}%
\pgfpathlineto{\pgfqpoint{4.073429in}{2.136316in}}%
\pgfpathlineto{\pgfqpoint{4.098421in}{2.136316in}}%
\pgfpathlineto{\pgfqpoint{4.123413in}{2.136316in}}%
\pgfpathlineto{\pgfqpoint{4.148405in}{2.242700in}}%
\pgfpathlineto{\pgfqpoint{4.173397in}{2.310515in}}%
\pgfpathlineto{\pgfqpoint{4.198388in}{2.350249in}}%
\pgfpathlineto{\pgfqpoint{4.223380in}{2.369854in}}%
\pgfpathlineto{\pgfqpoint{4.248372in}{2.375322in}}%
\pgfpathlineto{\pgfqpoint{4.273364in}{2.336464in}}%
\pgfpathlineto{\pgfqpoint{4.298356in}{2.303841in}}%
\pgfpathlineto{\pgfqpoint{4.323348in}{2.276468in}}%
\pgfpathlineto{\pgfqpoint{4.348339in}{2.253514in}}%
\pgfpathlineto{\pgfqpoint{4.373331in}{2.234276in}}%
\pgfpathlineto{\pgfqpoint{4.398323in}{2.218162in}}%
\pgfpathlineto{\pgfqpoint{4.423315in}{2.204671in}}%
\pgfpathlineto{\pgfqpoint{4.448307in}{2.193382in}}%
\pgfpathlineto{\pgfqpoint{4.473299in}{2.183940in}}%
\pgfpathlineto{\pgfqpoint{4.498290in}{2.176045in}}%
\pgfpathlineto{\pgfqpoint{4.523282in}{2.169448in}}%
\pgfpathlineto{\pgfqpoint{4.548274in}{2.163937in}}%
\pgfpathlineto{\pgfqpoint{4.573266in}{2.159335in}}%
\pgfpathlineto{\pgfqpoint{4.598258in}{2.155494in}}%
\pgfpathlineto{\pgfqpoint{4.623250in}{2.152290in}}%
\pgfpathlineto{\pgfqpoint{4.648241in}{2.149616in}}%
\pgfpathlineto{\pgfqpoint{4.673233in}{2.147387in}}%
\pgfpathlineto{\pgfqpoint{4.698225in}{2.145530in}}%
\pgfpathlineto{\pgfqpoint{4.723217in}{2.143981in}}%
\pgfpathlineto{\pgfqpoint{4.748209in}{2.142692in}}%
\pgfpathlineto{\pgfqpoint{4.773201in}{2.141618in}}%
\pgfpathlineto{\pgfqpoint{4.798192in}{2.140724in}}%
\pgfpathlineto{\pgfqpoint{4.823184in}{2.139980in}}%
\pgfpathlineto{\pgfqpoint{4.848176in}{2.139360in}}%
\pgfpathlineto{\pgfqpoint{4.873168in}{2.138845in}}%
\pgfpathlineto{\pgfqpoint{4.898160in}{2.138417in}}%
\pgfpathlineto{\pgfqpoint{4.923152in}{2.138061in}}%
\pgfpathlineto{\pgfqpoint{4.948143in}{2.137765in}}%
\pgfpathlineto{\pgfqpoint{4.973135in}{2.137518in}}%
\pgfpathlineto{\pgfqpoint{4.998127in}{2.137314in}}%
\pgfpathlineto{\pgfqpoint{5.023119in}{2.137144in}}%
\pgfpathlineto{\pgfqpoint{5.048111in}{2.137003in}}%
\pgfpathlineto{\pgfqpoint{5.073103in}{2.136886in}}%
\pgfpathlineto{\pgfqpoint{5.098095in}{2.136789in}}%
\pgfpathlineto{\pgfqpoint{5.108215in}{2.136756in}}%
\pgfusepath{stroke}%
\end{pgfscope}%
\begin{pgfscope}%
\pgfpathrectangle{\pgfqpoint{3.874531in}{1.689230in}}{\pgfqpoint{1.223684in}{1.004348in}}%
\pgfusepath{clip}%
\pgfsetrectcap%
\pgfsetroundjoin%
\pgfsetlinewidth{0.803000pt}%
\definecolor{currentstroke}{rgb}{0.000000,0.356863,0.509804}%
\pgfsetstrokecolor{currentstroke}%
\pgfsetdash{}{0pt}%
\pgfpathmoveto{\pgfqpoint{3.923478in}{2.136316in}}%
\pgfpathlineto{\pgfqpoint{3.948470in}{2.136316in}}%
\pgfpathlineto{\pgfqpoint{3.973462in}{2.136316in}}%
\pgfpathlineto{\pgfqpoint{3.998454in}{2.136316in}}%
\pgfpathlineto{\pgfqpoint{4.023446in}{2.136316in}}%
\pgfpathlineto{\pgfqpoint{4.048437in}{2.136316in}}%
\pgfpathlineto{\pgfqpoint{4.073429in}{2.136316in}}%
\pgfpathlineto{\pgfqpoint{4.098421in}{2.136316in}}%
\pgfpathlineto{\pgfqpoint{4.123413in}{2.136316in}}%
\pgfpathlineto{\pgfqpoint{4.148405in}{2.092184in}}%
\pgfpathlineto{\pgfqpoint{4.173397in}{2.064052in}}%
\pgfpathlineto{\pgfqpoint{4.198388in}{2.047569in}}%
\pgfpathlineto{\pgfqpoint{4.223380in}{2.039436in}}%
\pgfpathlineto{\pgfqpoint{4.248372in}{2.037168in}}%
\pgfpathlineto{\pgfqpoint{4.273364in}{1.946080in}}%
\pgfpathlineto{\pgfqpoint{4.298356in}{1.891273in}}%
\pgfpathlineto{\pgfqpoint{4.323348in}{1.862586in}}%
\pgfpathlineto{\pgfqpoint{4.348339in}{1.852351in}}%
\pgfpathlineto{\pgfqpoint{4.373331in}{1.854821in}}%
\pgfpathlineto{\pgfqpoint{4.398323in}{1.865726in}}%
\pgfpathlineto{\pgfqpoint{4.423315in}{1.881927in}}%
\pgfpathlineto{\pgfqpoint{4.448307in}{1.901146in}}%
\pgfpathlineto{\pgfqpoint{4.473299in}{1.921756in}}%
\pgfpathlineto{\pgfqpoint{4.498290in}{1.942622in}}%
\pgfpathlineto{\pgfqpoint{4.523282in}{1.962973in}}%
\pgfpathlineto{\pgfqpoint{4.548274in}{1.982310in}}%
\pgfpathlineto{\pgfqpoint{4.573266in}{2.000333in}}%
\pgfpathlineto{\pgfqpoint{4.598258in}{2.016882in}}%
\pgfpathlineto{\pgfqpoint{4.623250in}{2.031900in}}%
\pgfpathlineto{\pgfqpoint{4.648241in}{2.045397in}}%
\pgfpathlineto{\pgfqpoint{4.673233in}{2.057433in}}%
\pgfpathlineto{\pgfqpoint{4.698225in}{2.068092in}}%
\pgfpathlineto{\pgfqpoint{4.723217in}{2.077479in}}%
\pgfpathlineto{\pgfqpoint{4.748209in}{2.085705in}}%
\pgfpathlineto{\pgfqpoint{4.773201in}{2.092881in}}%
\pgfpathlineto{\pgfqpoint{4.798192in}{2.099119in}}%
\pgfpathlineto{\pgfqpoint{4.823184in}{2.104523in}}%
\pgfpathlineto{\pgfqpoint{4.848176in}{2.109190in}}%
\pgfpathlineto{\pgfqpoint{4.873168in}{2.113209in}}%
\pgfpathlineto{\pgfqpoint{4.898160in}{2.116663in}}%
\pgfpathlineto{\pgfqpoint{4.923152in}{2.119624in}}%
\pgfpathlineto{\pgfqpoint{4.948143in}{2.122157in}}%
\pgfpathlineto{\pgfqpoint{4.973135in}{2.124320in}}%
\pgfpathlineto{\pgfqpoint{4.998127in}{2.126164in}}%
\pgfpathlineto{\pgfqpoint{5.023119in}{2.127734in}}%
\pgfpathlineto{\pgfqpoint{5.048111in}{2.129069in}}%
\pgfpathlineto{\pgfqpoint{5.073103in}{2.130201in}}%
\pgfpathlineto{\pgfqpoint{5.098095in}{2.131161in}}%
\pgfpathlineto{\pgfqpoint{5.108215in}{2.131490in}}%
\pgfusepath{stroke}%
\end{pgfscope}%
\begin{pgfscope}%
\pgfpathrectangle{\pgfqpoint{3.874531in}{1.689230in}}{\pgfqpoint{1.223684in}{1.004348in}}%
\pgfusepath{clip}%
\pgfsetrectcap%
\pgfsetroundjoin%
\pgfsetlinewidth{0.803000pt}%
\definecolor{currentstroke}{rgb}{0.490196,0.588235,0.431373}%
\pgfsetstrokecolor{currentstroke}%
\pgfsetdash{}{0pt}%
\pgfpathmoveto{\pgfqpoint{3.923478in}{2.136316in}}%
\pgfpathlineto{\pgfqpoint{3.948470in}{2.136316in}}%
\pgfpathlineto{\pgfqpoint{3.973462in}{2.136316in}}%
\pgfpathlineto{\pgfqpoint{3.998454in}{2.136316in}}%
\pgfpathlineto{\pgfqpoint{4.023446in}{2.136316in}}%
\pgfpathlineto{\pgfqpoint{4.048437in}{2.136316in}}%
\pgfpathlineto{\pgfqpoint{4.073429in}{2.136316in}}%
\pgfpathlineto{\pgfqpoint{4.098421in}{2.136316in}}%
\pgfpathlineto{\pgfqpoint{4.123413in}{2.136316in}}%
\pgfpathlineto{\pgfqpoint{4.148405in}{2.154947in}}%
\pgfpathlineto{\pgfqpoint{4.173397in}{2.166824in}}%
\pgfpathlineto{\pgfqpoint{4.198388in}{2.173783in}}%
\pgfpathlineto{\pgfqpoint{4.223380in}{2.177217in}}%
\pgfpathlineto{\pgfqpoint{4.248372in}{2.178174in}}%
\pgfpathlineto{\pgfqpoint{4.273364in}{2.138512in}}%
\pgfpathlineto{\pgfqpoint{4.298356in}{2.111853in}}%
\pgfpathlineto{\pgfqpoint{4.323348in}{2.094786in}}%
\pgfpathlineto{\pgfqpoint{4.348339in}{2.084711in}}%
\pgfpathlineto{\pgfqpoint{4.373331in}{2.079653in}}%
\pgfpathlineto{\pgfqpoint{4.398323in}{2.078124in}}%
\pgfpathlineto{\pgfqpoint{4.423315in}{2.079012in}}%
\pgfpathlineto{\pgfqpoint{4.448307in}{2.081490in}}%
\pgfpathlineto{\pgfqpoint{4.473299in}{2.084952in}}%
\pgfpathlineto{\pgfqpoint{4.498290in}{2.088961in}}%
\pgfpathlineto{\pgfqpoint{4.523282in}{2.093204in}}%
\pgfpathlineto{\pgfqpoint{4.548274in}{2.097465in}}%
\pgfpathlineto{\pgfqpoint{4.573266in}{2.101597in}}%
\pgfpathlineto{\pgfqpoint{4.598258in}{2.105508in}}%
\pgfpathlineto{\pgfqpoint{4.623250in}{2.109142in}}%
\pgfpathlineto{\pgfqpoint{4.648241in}{2.112471in}}%
\pgfpathlineto{\pgfqpoint{4.673233in}{2.115486in}}%
\pgfpathlineto{\pgfqpoint{4.698225in}{2.118191in}}%
\pgfpathlineto{\pgfqpoint{4.723217in}{2.120600in}}%
\pgfpathlineto{\pgfqpoint{4.748209in}{2.122732in}}%
\pgfpathlineto{\pgfqpoint{4.773201in}{2.124606in}}%
\pgfpathlineto{\pgfqpoint{4.798192in}{2.126248in}}%
\pgfpathlineto{\pgfqpoint{4.823184in}{2.127679in}}%
\pgfpathlineto{\pgfqpoint{4.848176in}{2.128922in}}%
\pgfpathlineto{\pgfqpoint{4.873168in}{2.129999in}}%
\pgfpathlineto{\pgfqpoint{4.898160in}{2.130928in}}%
\pgfpathlineto{\pgfqpoint{4.923152in}{2.131727in}}%
\pgfpathlineto{\pgfqpoint{4.948143in}{2.132414in}}%
\pgfpathlineto{\pgfqpoint{4.973135in}{2.133003in}}%
\pgfpathlineto{\pgfqpoint{4.998127in}{2.133506in}}%
\pgfpathlineto{\pgfqpoint{5.023119in}{2.133936in}}%
\pgfpathlineto{\pgfqpoint{5.048111in}{2.134302in}}%
\pgfpathlineto{\pgfqpoint{5.073103in}{2.134614in}}%
\pgfpathlineto{\pgfqpoint{5.098095in}{2.134879in}}%
\pgfpathlineto{\pgfqpoint{5.108215in}{2.134970in}}%
\pgfusepath{stroke}%
\end{pgfscope}%
\begin{pgfscope}%
\pgfpathrectangle{\pgfqpoint{3.874531in}{1.689230in}}{\pgfqpoint{1.223684in}{1.004348in}}%
\pgfusepath{clip}%
\pgfsetrectcap%
\pgfsetroundjoin%
\pgfsetlinewidth{0.803000pt}%
\definecolor{currentstroke}{rgb}{0.843137,0.666667,0.313725}%
\pgfsetstrokecolor{currentstroke}%
\pgfsetdash{}{0pt}%
\pgfpathmoveto{\pgfqpoint{3.923478in}{2.136316in}}%
\pgfpathlineto{\pgfqpoint{3.948470in}{2.136316in}}%
\pgfpathlineto{\pgfqpoint{3.973462in}{2.136316in}}%
\pgfpathlineto{\pgfqpoint{3.998454in}{2.136316in}}%
\pgfpathlineto{\pgfqpoint{4.023446in}{2.136316in}}%
\pgfpathlineto{\pgfqpoint{4.048437in}{2.136316in}}%
\pgfpathlineto{\pgfqpoint{4.073429in}{2.136316in}}%
\pgfpathlineto{\pgfqpoint{4.098421in}{2.136316in}}%
\pgfpathlineto{\pgfqpoint{4.123413in}{2.136316in}}%
\pgfpathlineto{\pgfqpoint{4.148405in}{2.169641in}}%
\pgfpathlineto{\pgfqpoint{4.173397in}{2.190884in}}%
\pgfpathlineto{\pgfqpoint{4.198388in}{2.203331in}}%
\pgfpathlineto{\pgfqpoint{4.223380in}{2.209472in}}%
\pgfpathlineto{\pgfqpoint{4.248372in}{2.211185in}}%
\pgfpathlineto{\pgfqpoint{4.273364in}{2.206092in}}%
\pgfpathlineto{\pgfqpoint{4.298356in}{2.200386in}}%
\pgfpathlineto{\pgfqpoint{4.323348in}{2.194456in}}%
\pgfpathlineto{\pgfqpoint{4.348339in}{2.188570in}}%
\pgfpathlineto{\pgfqpoint{4.373331in}{2.182908in}}%
\pgfpathlineto{\pgfqpoint{4.398323in}{2.177581in}}%
\pgfpathlineto{\pgfqpoint{4.423315in}{2.172655in}}%
\pgfpathlineto{\pgfqpoint{4.448307in}{2.168159in}}%
\pgfpathlineto{\pgfqpoint{4.473299in}{2.164099in}}%
\pgfpathlineto{\pgfqpoint{4.498290in}{2.160464in}}%
\pgfpathlineto{\pgfqpoint{4.523282in}{2.157234in}}%
\pgfpathlineto{\pgfqpoint{4.548274in}{2.154382in}}%
\pgfpathlineto{\pgfqpoint{4.573266in}{2.151876in}}%
\pgfpathlineto{\pgfqpoint{4.598258in}{2.149685in}}%
\pgfpathlineto{\pgfqpoint{4.623250in}{2.147777in}}%
\pgfpathlineto{\pgfqpoint{4.648241in}{2.146122in}}%
\pgfpathlineto{\pgfqpoint{4.673233in}{2.144690in}}%
\pgfpathlineto{\pgfqpoint{4.698225in}{2.143455in}}%
\pgfpathlineto{\pgfqpoint{4.723217in}{2.142393in}}%
\pgfpathlineto{\pgfqpoint{4.748209in}{2.141481in}}%
\pgfpathlineto{\pgfqpoint{4.773201in}{2.140700in}}%
\pgfpathlineto{\pgfqpoint{4.798192in}{2.140032in}}%
\pgfpathlineto{\pgfqpoint{4.823184in}{2.139463in}}%
\pgfpathlineto{\pgfqpoint{4.848176in}{2.138977in}}%
\pgfpathlineto{\pgfqpoint{4.873168in}{2.138565in}}%
\pgfpathlineto{\pgfqpoint{4.898160in}{2.138214in}}%
\pgfpathlineto{\pgfqpoint{4.923152in}{2.137917in}}%
\pgfpathlineto{\pgfqpoint{4.948143in}{2.137665in}}%
\pgfpathlineto{\pgfqpoint{4.973135in}{2.137452in}}%
\pgfpathlineto{\pgfqpoint{4.998127in}{2.137272in}}%
\pgfpathlineto{\pgfqpoint{5.023119in}{2.137119in}}%
\pgfpathlineto{\pgfqpoint{5.048111in}{2.136991in}}%
\pgfpathlineto{\pgfqpoint{5.073103in}{2.136883in}}%
\pgfpathlineto{\pgfqpoint{5.098095in}{2.136791in}}%
\pgfpathlineto{\pgfqpoint{5.108215in}{2.136760in}}%
\pgfusepath{stroke}%
\end{pgfscope}%
\begin{pgfscope}%
\pgfpathrectangle{\pgfqpoint{3.874531in}{1.689230in}}{\pgfqpoint{1.223684in}{1.004348in}}%
\pgfusepath{clip}%
\pgfsetrectcap%
\pgfsetroundjoin%
\pgfsetlinewidth{0.803000pt}%
\definecolor{currentstroke}{rgb}{0.333333,0.333333,0.333333}%
\pgfsetstrokecolor{currentstroke}%
\pgfsetdash{}{0pt}%
\pgfpathmoveto{\pgfqpoint{3.923478in}{2.136316in}}%
\pgfpathlineto{\pgfqpoint{3.948470in}{2.136316in}}%
\pgfpathlineto{\pgfqpoint{3.973462in}{2.136316in}}%
\pgfpathlineto{\pgfqpoint{3.998454in}{2.136316in}}%
\pgfpathlineto{\pgfqpoint{4.023446in}{2.136316in}}%
\pgfpathlineto{\pgfqpoint{4.048437in}{2.136316in}}%
\pgfpathlineto{\pgfqpoint{4.073429in}{2.136316in}}%
\pgfpathlineto{\pgfqpoint{4.098421in}{2.136316in}}%
\pgfpathlineto{\pgfqpoint{4.123413in}{2.136316in}}%
\pgfpathlineto{\pgfqpoint{4.148405in}{2.161183in}}%
\pgfpathlineto{\pgfqpoint{4.173397in}{2.177035in}}%
\pgfpathlineto{\pgfqpoint{4.198388in}{2.186322in}}%
\pgfpathlineto{\pgfqpoint{4.223380in}{2.190905in}}%
\pgfpathlineto{\pgfqpoint{4.248372in}{2.192183in}}%
\pgfpathlineto{\pgfqpoint{4.273364in}{2.257716in}}%
\pgfpathlineto{\pgfqpoint{4.298356in}{2.297655in}}%
\pgfpathlineto{\pgfqpoint{4.323348in}{2.319125in}}%
\pgfpathlineto{\pgfqpoint{4.348339in}{2.327510in}}%
\pgfpathlineto{\pgfqpoint{4.373331in}{2.326849in}}%
\pgfpathlineto{\pgfqpoint{4.398323in}{2.320144in}}%
\pgfpathlineto{\pgfqpoint{4.423315in}{2.309610in}}%
\pgfpathlineto{\pgfqpoint{4.448307in}{2.296855in}}%
\pgfpathlineto{\pgfqpoint{4.473299in}{2.283029in}}%
\pgfpathlineto{\pgfqpoint{4.498290in}{2.268941in}}%
\pgfpathlineto{\pgfqpoint{4.523282in}{2.255140in}}%
\pgfpathlineto{\pgfqpoint{4.548274in}{2.241984in}}%
\pgfpathlineto{\pgfqpoint{4.573266in}{2.229693in}}%
\pgfpathlineto{\pgfqpoint{4.598258in}{2.218386in}}%
\pgfpathlineto{\pgfqpoint{4.623250in}{2.208110in}}%
\pgfpathlineto{\pgfqpoint{4.648241in}{2.198863in}}%
\pgfpathlineto{\pgfqpoint{4.673233in}{2.190609in}}%
\pgfpathlineto{\pgfqpoint{4.698225in}{2.183292in}}%
\pgfpathlineto{\pgfqpoint{4.723217in}{2.176844in}}%
\pgfpathlineto{\pgfqpoint{4.748209in}{2.171190in}}%
\pgfpathlineto{\pgfqpoint{4.773201in}{2.166254in}}%
\pgfpathlineto{\pgfqpoint{4.798192in}{2.161962in}}%
\pgfpathlineto{\pgfqpoint{4.823184in}{2.158242in}}%
\pgfpathlineto{\pgfqpoint{4.848176in}{2.155028in}}%
\pgfpathlineto{\pgfqpoint{4.873168in}{2.152259in}}%
\pgfpathlineto{\pgfqpoint{4.898160in}{2.149879in}}%
\pgfpathlineto{\pgfqpoint{4.923152in}{2.147837in}}%
\pgfpathlineto{\pgfqpoint{4.948143in}{2.146091in}}%
\pgfpathlineto{\pgfqpoint{4.973135in}{2.144599in}}%
\pgfpathlineto{\pgfqpoint{4.998127in}{2.143326in}}%
\pgfpathlineto{\pgfqpoint{5.023119in}{2.142243in}}%
\pgfpathlineto{\pgfqpoint{5.048111in}{2.141322in}}%
\pgfpathlineto{\pgfqpoint{5.073103in}{2.140540in}}%
\pgfpathlineto{\pgfqpoint{5.098095in}{2.139878in}}%
\pgfpathlineto{\pgfqpoint{5.108215in}{2.139650in}}%
\pgfusepath{stroke}%
\end{pgfscope}%
\begin{pgfscope}%
\pgfpathrectangle{\pgfqpoint{3.874531in}{1.689230in}}{\pgfqpoint{1.223684in}{1.004348in}}%
\pgfusepath{clip}%
\pgfsetrectcap%
\pgfsetroundjoin%
\pgfsetlinewidth{0.803000pt}%
\definecolor{currentstroke}{rgb}{0.686275,0.352941,0.313725}%
\pgfsetstrokecolor{currentstroke}%
\pgfsetdash{}{0pt}%
\pgfpathmoveto{\pgfqpoint{3.923478in}{2.136316in}}%
\pgfpathlineto{\pgfqpoint{3.948470in}{2.136316in}}%
\pgfpathlineto{\pgfqpoint{3.973462in}{2.136316in}}%
\pgfpathlineto{\pgfqpoint{3.998454in}{2.136316in}}%
\pgfpathlineto{\pgfqpoint{4.023446in}{2.136316in}}%
\pgfpathlineto{\pgfqpoint{4.048437in}{2.136316in}}%
\pgfpathlineto{\pgfqpoint{4.073429in}{2.136316in}}%
\pgfpathlineto{\pgfqpoint{4.098421in}{2.136316in}}%
\pgfpathlineto{\pgfqpoint{4.123413in}{2.136316in}}%
\pgfpathlineto{\pgfqpoint{4.148405in}{2.082315in}}%
\pgfpathlineto{\pgfqpoint{4.173397in}{2.047891in}}%
\pgfpathlineto{\pgfqpoint{4.198388in}{2.027722in}}%
\pgfpathlineto{\pgfqpoint{4.223380in}{2.017770in}}%
\pgfpathlineto{\pgfqpoint{4.248372in}{2.014995in}}%
\pgfpathlineto{\pgfqpoint{4.273364in}{2.042814in}}%
\pgfpathlineto{\pgfqpoint{4.298356in}{2.064534in}}%
\pgfpathlineto{\pgfqpoint{4.323348in}{2.081452in}}%
\pgfpathlineto{\pgfqpoint{4.348339in}{2.094595in}}%
\pgfpathlineto{\pgfqpoint{4.373331in}{2.104776in}}%
\pgfpathlineto{\pgfqpoint{4.398323in}{2.112637in}}%
\pgfpathlineto{\pgfqpoint{4.423315in}{2.118685in}}%
\pgfpathlineto{\pgfqpoint{4.448307in}{2.123318in}}%
\pgfpathlineto{\pgfqpoint{4.473299in}{2.126850in}}%
\pgfpathlineto{\pgfqpoint{4.498290in}{2.129529in}}%
\pgfpathlineto{\pgfqpoint{4.523282in}{2.131548in}}%
\pgfpathlineto{\pgfqpoint{4.548274in}{2.133058in}}%
\pgfpathlineto{\pgfqpoint{4.573266in}{2.134177in}}%
\pgfpathlineto{\pgfqpoint{4.598258in}{2.134998in}}%
\pgfpathlineto{\pgfqpoint{4.623250in}{2.135591in}}%
\pgfpathlineto{\pgfqpoint{4.648241in}{2.136013in}}%
\pgfpathlineto{\pgfqpoint{4.673233in}{2.136305in}}%
\pgfpathlineto{\pgfqpoint{4.698225in}{2.136501in}}%
\pgfpathlineto{\pgfqpoint{4.723217in}{2.136627in}}%
\pgfpathlineto{\pgfqpoint{4.748209in}{2.136701in}}%
\pgfpathlineto{\pgfqpoint{4.773201in}{2.136738in}}%
\pgfpathlineto{\pgfqpoint{4.798192in}{2.136749in}}%
\pgfpathlineto{\pgfqpoint{4.823184in}{2.136742in}}%
\pgfpathlineto{\pgfqpoint{4.848176in}{2.136723in}}%
\pgfpathlineto{\pgfqpoint{4.873168in}{2.136697in}}%
\pgfpathlineto{\pgfqpoint{4.898160in}{2.136667in}}%
\pgfpathlineto{\pgfqpoint{4.923152in}{2.136636in}}%
\pgfpathlineto{\pgfqpoint{4.948143in}{2.136604in}}%
\pgfpathlineto{\pgfqpoint{4.973135in}{2.136573in}}%
\pgfpathlineto{\pgfqpoint{4.998127in}{2.136544in}}%
\pgfpathlineto{\pgfqpoint{5.023119in}{2.136517in}}%
\pgfpathlineto{\pgfqpoint{5.048111in}{2.136493in}}%
\pgfpathlineto{\pgfqpoint{5.073103in}{2.136470in}}%
\pgfpathlineto{\pgfqpoint{5.098095in}{2.136450in}}%
\pgfpathlineto{\pgfqpoint{5.108215in}{2.136443in}}%
\pgfusepath{stroke}%
\end{pgfscope}%
\begin{pgfscope}%
\pgfpathrectangle{\pgfqpoint{3.874531in}{1.689230in}}{\pgfqpoint{1.223684in}{1.004348in}}%
\pgfusepath{clip}%
\pgfsetrectcap%
\pgfsetroundjoin%
\pgfsetlinewidth{0.803000pt}%
\definecolor{currentstroke}{rgb}{0.000000,0.356863,0.509804}%
\pgfsetstrokecolor{currentstroke}%
\pgfsetdash{}{0pt}%
\pgfpathmoveto{\pgfqpoint{3.923478in}{2.136316in}}%
\pgfpathlineto{\pgfqpoint{3.948470in}{2.136316in}}%
\pgfpathlineto{\pgfqpoint{3.973462in}{2.136316in}}%
\pgfpathlineto{\pgfqpoint{3.998454in}{2.136316in}}%
\pgfpathlineto{\pgfqpoint{4.023446in}{2.136316in}}%
\pgfpathlineto{\pgfqpoint{4.048437in}{2.136316in}}%
\pgfpathlineto{\pgfqpoint{4.073429in}{2.136316in}}%
\pgfpathlineto{\pgfqpoint{4.098421in}{2.136316in}}%
\pgfpathlineto{\pgfqpoint{4.123413in}{2.136316in}}%
\pgfpathlineto{\pgfqpoint{4.148405in}{2.263440in}}%
\pgfpathlineto{\pgfqpoint{4.173397in}{2.344476in}}%
\pgfpathlineto{\pgfqpoint{4.198388in}{2.391957in}}%
\pgfpathlineto{\pgfqpoint{4.223380in}{2.415384in}}%
\pgfpathlineto{\pgfqpoint{4.248372in}{2.421918in}}%
\pgfpathlineto{\pgfqpoint{4.273364in}{2.544743in}}%
\pgfpathlineto{\pgfqpoint{4.298356in}{2.613655in}}%
\pgfpathlineto{\pgfqpoint{4.323348in}{2.644163in}}%
\pgfpathlineto{\pgfqpoint{4.348339in}{2.647926in}}%
\pgfpathlineto{\pgfqpoint{4.373331in}{2.633638in}}%
\pgfpathlineto{\pgfqpoint{4.398323in}{2.607719in}}%
\pgfpathlineto{\pgfqpoint{4.423315in}{2.574855in}}%
\pgfpathlineto{\pgfqpoint{4.448307in}{2.538416in}}%
\pgfpathlineto{\pgfqpoint{4.473299in}{2.500778in}}%
\pgfpathlineto{\pgfqpoint{4.498290in}{2.463573in}}%
\pgfpathlineto{\pgfqpoint{4.523282in}{2.427880in}}%
\pgfpathlineto{\pgfqpoint{4.548274in}{2.394374in}}%
\pgfpathlineto{\pgfqpoint{4.573266in}{2.363435in}}%
\pgfpathlineto{\pgfqpoint{4.598258in}{2.335233in}}%
\pgfpathlineto{\pgfqpoint{4.623250in}{2.309793in}}%
\pgfpathlineto{\pgfqpoint{4.648241in}{2.287039in}}%
\pgfpathlineto{\pgfqpoint{4.673233in}{2.266835in}}%
\pgfpathlineto{\pgfqpoint{4.698225in}{2.249002in}}%
\pgfpathlineto{\pgfqpoint{4.723217in}{2.233346in}}%
\pgfpathlineto{\pgfqpoint{4.748209in}{2.219664in}}%
\pgfpathlineto{\pgfqpoint{4.773201in}{2.207753in}}%
\pgfpathlineto{\pgfqpoint{4.798192in}{2.197422in}}%
\pgfpathlineto{\pgfqpoint{4.823184in}{2.188489in}}%
\pgfpathlineto{\pgfqpoint{4.848176in}{2.180786in}}%
\pgfpathlineto{\pgfqpoint{4.873168in}{2.174162in}}%
\pgfpathlineto{\pgfqpoint{4.898160in}{2.168478in}}%
\pgfpathlineto{\pgfqpoint{4.923152in}{2.163611in}}%
\pgfpathlineto{\pgfqpoint{4.948143in}{2.159452in}}%
\pgfpathlineto{\pgfqpoint{4.973135in}{2.155904in}}%
\pgfpathlineto{\pgfqpoint{4.998127in}{2.152882in}}%
\pgfpathlineto{\pgfqpoint{5.023119in}{2.150312in}}%
\pgfpathlineto{\pgfqpoint{5.048111in}{2.148129in}}%
\pgfpathlineto{\pgfqpoint{5.073103in}{2.146278in}}%
\pgfpathlineto{\pgfqpoint{5.098095in}{2.144709in}}%
\pgfpathlineto{\pgfqpoint{5.108215in}{2.144172in}}%
\pgfusepath{stroke}%
\end{pgfscope}%
\begin{pgfscope}%
\pgfpathrectangle{\pgfqpoint{3.874531in}{1.689230in}}{\pgfqpoint{1.223684in}{1.004348in}}%
\pgfusepath{clip}%
\pgfsetrectcap%
\pgfsetroundjoin%
\pgfsetlinewidth{0.803000pt}%
\definecolor{currentstroke}{rgb}{0.490196,0.588235,0.431373}%
\pgfsetstrokecolor{currentstroke}%
\pgfsetdash{}{0pt}%
\pgfpathmoveto{\pgfqpoint{3.923478in}{2.136316in}}%
\pgfpathlineto{\pgfqpoint{3.948470in}{2.136316in}}%
\pgfpathlineto{\pgfqpoint{3.973462in}{2.136316in}}%
\pgfpathlineto{\pgfqpoint{3.998454in}{2.136316in}}%
\pgfpathlineto{\pgfqpoint{4.023446in}{2.136316in}}%
\pgfpathlineto{\pgfqpoint{4.048437in}{2.136316in}}%
\pgfpathlineto{\pgfqpoint{4.073429in}{2.136316in}}%
\pgfpathlineto{\pgfqpoint{4.098421in}{2.136316in}}%
\pgfpathlineto{\pgfqpoint{4.123413in}{2.136316in}}%
\pgfpathlineto{\pgfqpoint{4.148405in}{2.157566in}}%
\pgfpathlineto{\pgfqpoint{4.173397in}{2.171112in}}%
\pgfpathlineto{\pgfqpoint{4.198388in}{2.179049in}}%
\pgfpathlineto{\pgfqpoint{4.223380in}{2.182965in}}%
\pgfpathlineto{\pgfqpoint{4.248372in}{2.184057in}}%
\pgfpathlineto{\pgfqpoint{4.273364in}{2.023466in}}%
\pgfpathlineto{\pgfqpoint{4.298356in}{1.919527in}}%
\pgfpathlineto{\pgfqpoint{4.323348in}{1.856978in}}%
\pgfpathlineto{\pgfqpoint{4.348339in}{1.824228in}}%
\pgfpathlineto{\pgfqpoint{4.373331in}{1.812530in}}%
\pgfpathlineto{\pgfqpoint{4.398323in}{1.815328in}}%
\pgfpathlineto{\pgfqpoint{4.423315in}{1.827751in}}%
\pgfpathlineto{\pgfqpoint{4.448307in}{1.846217in}}%
\pgfpathlineto{\pgfqpoint{4.473299in}{1.868127in}}%
\pgfpathlineto{\pgfqpoint{4.498290in}{1.891627in}}%
\pgfpathlineto{\pgfqpoint{4.523282in}{1.915419in}}%
\pgfpathlineto{\pgfqpoint{4.548274in}{1.938626in}}%
\pgfpathlineto{\pgfqpoint{4.573266in}{1.960677in}}%
\pgfpathlineto{\pgfqpoint{4.598258in}{1.981230in}}%
\pgfpathlineto{\pgfqpoint{4.623250in}{2.000103in}}%
\pgfpathlineto{\pgfqpoint{4.648241in}{2.017230in}}%
\pgfpathlineto{\pgfqpoint{4.673233in}{2.032623in}}%
\pgfpathlineto{\pgfqpoint{4.698225in}{2.046349in}}%
\pgfpathlineto{\pgfqpoint{4.723217in}{2.058506in}}%
\pgfpathlineto{\pgfqpoint{4.748209in}{2.069211in}}%
\pgfpathlineto{\pgfqpoint{4.773201in}{2.078592in}}%
\pgfpathlineto{\pgfqpoint{4.798192in}{2.086778in}}%
\pgfpathlineto{\pgfqpoint{4.823184in}{2.093892in}}%
\pgfpathlineto{\pgfqpoint{4.848176in}{2.100055in}}%
\pgfpathlineto{\pgfqpoint{4.873168in}{2.105377in}}%
\pgfpathlineto{\pgfqpoint{4.898160in}{2.109962in}}%
\pgfpathlineto{\pgfqpoint{4.923152in}{2.113901in}}%
\pgfpathlineto{\pgfqpoint{4.948143in}{2.117278in}}%
\pgfpathlineto{\pgfqpoint{4.973135in}{2.120167in}}%
\pgfpathlineto{\pgfqpoint{4.998127in}{2.122634in}}%
\pgfpathlineto{\pgfqpoint{5.023119in}{2.124738in}}%
\pgfpathlineto{\pgfqpoint{5.048111in}{2.126528in}}%
\pgfpathlineto{\pgfqpoint{5.073103in}{2.128050in}}%
\pgfpathlineto{\pgfqpoint{5.098095in}{2.129342in}}%
\pgfpathlineto{\pgfqpoint{5.108215in}{2.129785in}}%
\pgfusepath{stroke}%
\end{pgfscope}%
\begin{pgfscope}%
\pgfpathrectangle{\pgfqpoint{3.874531in}{1.689230in}}{\pgfqpoint{1.223684in}{1.004348in}}%
\pgfusepath{clip}%
\pgfsetrectcap%
\pgfsetroundjoin%
\pgfsetlinewidth{0.803000pt}%
\definecolor{currentstroke}{rgb}{0.843137,0.666667,0.313725}%
\pgfsetstrokecolor{currentstroke}%
\pgfsetdash{}{0pt}%
\pgfpathmoveto{\pgfqpoint{3.923478in}{2.136316in}}%
\pgfpathlineto{\pgfqpoint{3.948470in}{2.136316in}}%
\pgfpathlineto{\pgfqpoint{3.973462in}{2.136316in}}%
\pgfpathlineto{\pgfqpoint{3.998454in}{2.136316in}}%
\pgfpathlineto{\pgfqpoint{4.023446in}{2.136316in}}%
\pgfpathlineto{\pgfqpoint{4.048437in}{2.136316in}}%
\pgfpathlineto{\pgfqpoint{4.073429in}{2.136316in}}%
\pgfpathlineto{\pgfqpoint{4.098421in}{2.136316in}}%
\pgfpathlineto{\pgfqpoint{4.123413in}{2.136316in}}%
\pgfpathlineto{\pgfqpoint{4.148405in}{2.067581in}}%
\pgfpathlineto{\pgfqpoint{4.173397in}{2.023765in}}%
\pgfpathlineto{\pgfqpoint{4.198388in}{1.998092in}}%
\pgfpathlineto{\pgfqpoint{4.223380in}{1.985425in}}%
\pgfpathlineto{\pgfqpoint{4.248372in}{1.981892in}}%
\pgfpathlineto{\pgfqpoint{4.273364in}{2.058596in}}%
\pgfpathlineto{\pgfqpoint{4.298356in}{2.112566in}}%
\pgfpathlineto{\pgfqpoint{4.323348in}{2.149524in}}%
\pgfpathlineto{\pgfqpoint{4.348339in}{2.173863in}}%
\pgfpathlineto{\pgfqpoint{4.373331in}{2.188944in}}%
\pgfpathlineto{\pgfqpoint{4.398323in}{2.197324in}}%
\pgfpathlineto{\pgfqpoint{4.423315in}{2.200937in}}%
\pgfpathlineto{\pgfqpoint{4.448307in}{2.201235in}}%
\pgfpathlineto{\pgfqpoint{4.473299in}{2.199302in}}%
\pgfpathlineto{\pgfqpoint{4.498290in}{2.195937in}}%
\pgfpathlineto{\pgfqpoint{4.523282in}{2.191722in}}%
\pgfpathlineto{\pgfqpoint{4.548274in}{2.187076in}}%
\pgfpathlineto{\pgfqpoint{4.573266in}{2.182293in}}%
\pgfpathlineto{\pgfqpoint{4.598258in}{2.177577in}}%
\pgfpathlineto{\pgfqpoint{4.623250in}{2.173060in}}%
\pgfpathlineto{\pgfqpoint{4.648241in}{2.168824in}}%
\pgfpathlineto{\pgfqpoint{4.673233in}{2.164917in}}%
\pgfpathlineto{\pgfqpoint{4.698225in}{2.161358in}}%
\pgfpathlineto{\pgfqpoint{4.723217in}{2.158150in}}%
\pgfpathlineto{\pgfqpoint{4.748209in}{2.155282in}}%
\pgfpathlineto{\pgfqpoint{4.773201in}{2.152736in}}%
\pgfpathlineto{\pgfqpoint{4.798192in}{2.150490in}}%
\pgfpathlineto{\pgfqpoint{4.823184in}{2.148519in}}%
\pgfpathlineto{\pgfqpoint{4.848176in}{2.146796in}}%
\pgfpathlineto{\pgfqpoint{4.873168in}{2.145297in}}%
\pgfpathlineto{\pgfqpoint{4.898160in}{2.143998in}}%
\pgfpathlineto{\pgfqpoint{4.923152in}{2.142874in}}%
\pgfpathlineto{\pgfqpoint{4.948143in}{2.141905in}}%
\pgfpathlineto{\pgfqpoint{4.973135in}{2.141072in}}%
\pgfpathlineto{\pgfqpoint{4.998127in}{2.140357in}}%
\pgfpathlineto{\pgfqpoint{5.023119in}{2.139746in}}%
\pgfpathlineto{\pgfqpoint{5.048111in}{2.139223in}}%
\pgfpathlineto{\pgfqpoint{5.073103in}{2.138777in}}%
\pgfpathlineto{\pgfqpoint{5.098095in}{2.138397in}}%
\pgfpathlineto{\pgfqpoint{5.108215in}{2.138266in}}%
\pgfusepath{stroke}%
\end{pgfscope}%
\begin{pgfscope}%
\pgfpathrectangle{\pgfqpoint{3.874531in}{1.689230in}}{\pgfqpoint{1.223684in}{1.004348in}}%
\pgfusepath{clip}%
\pgfsetrectcap%
\pgfsetroundjoin%
\pgfsetlinewidth{0.803000pt}%
\definecolor{currentstroke}{rgb}{0.333333,0.333333,0.333333}%
\pgfsetstrokecolor{currentstroke}%
\pgfsetdash{}{0pt}%
\pgfpathmoveto{\pgfqpoint{3.923478in}{2.136316in}}%
\pgfpathlineto{\pgfqpoint{3.948470in}{2.136316in}}%
\pgfpathlineto{\pgfqpoint{3.973462in}{2.136316in}}%
\pgfpathlineto{\pgfqpoint{3.998454in}{2.136316in}}%
\pgfpathlineto{\pgfqpoint{4.023446in}{2.136316in}}%
\pgfpathlineto{\pgfqpoint{4.048437in}{2.136316in}}%
\pgfpathlineto{\pgfqpoint{4.073429in}{2.136316in}}%
\pgfpathlineto{\pgfqpoint{4.098421in}{2.136316in}}%
\pgfpathlineto{\pgfqpoint{4.123413in}{2.136316in}}%
\pgfpathlineto{\pgfqpoint{4.148405in}{2.192621in}}%
\pgfpathlineto{\pgfqpoint{4.173397in}{2.228513in}}%
\pgfpathlineto{\pgfqpoint{4.198388in}{2.249543in}}%
\pgfpathlineto{\pgfqpoint{4.223380in}{2.259919in}}%
\pgfpathlineto{\pgfqpoint{4.248372in}{2.262813in}}%
\pgfpathlineto{\pgfqpoint{4.273364in}{2.138270in}}%
\pgfpathlineto{\pgfqpoint{4.298356in}{2.054723in}}%
\pgfpathlineto{\pgfqpoint{4.323348in}{2.001400in}}%
\pgfpathlineto{\pgfqpoint{4.348339in}{1.970090in}}%
\pgfpathlineto{\pgfqpoint{4.373331in}{1.954564in}}%
\pgfpathlineto{\pgfqpoint{4.398323in}{1.950129in}}%
\pgfpathlineto{\pgfqpoint{4.423315in}{1.953274in}}%
\pgfpathlineto{\pgfqpoint{4.448307in}{1.961396in}}%
\pgfpathlineto{\pgfqpoint{4.473299in}{1.972589in}}%
\pgfpathlineto{\pgfqpoint{4.498290in}{1.985471in}}%
\pgfpathlineto{\pgfqpoint{4.523282in}{1.999063in}}%
\pgfpathlineto{\pgfqpoint{4.548274in}{2.012684in}}%
\pgfpathlineto{\pgfqpoint{4.573266in}{2.025876in}}%
\pgfpathlineto{\pgfqpoint{4.598258in}{2.038348in}}%
\pgfpathlineto{\pgfqpoint{4.623250in}{2.049928in}}%
\pgfpathlineto{\pgfqpoint{4.648241in}{2.060528in}}%
\pgfpathlineto{\pgfqpoint{4.673233in}{2.070124in}}%
\pgfpathlineto{\pgfqpoint{4.698225in}{2.078732in}}%
\pgfpathlineto{\pgfqpoint{4.723217in}{2.086393in}}%
\pgfpathlineto{\pgfqpoint{4.748209in}{2.093170in}}%
\pgfpathlineto{\pgfqpoint{4.773201in}{2.099130in}}%
\pgfpathlineto{\pgfqpoint{4.798192in}{2.104347in}}%
\pgfpathlineto{\pgfqpoint{4.823184in}{2.108894in}}%
\pgfpathlineto{\pgfqpoint{4.848176in}{2.112843in}}%
\pgfpathlineto{\pgfqpoint{4.873168in}{2.116262in}}%
\pgfpathlineto{\pgfqpoint{4.898160in}{2.119213in}}%
\pgfpathlineto{\pgfqpoint{4.923152in}{2.121752in}}%
\pgfpathlineto{\pgfqpoint{4.948143in}{2.123933in}}%
\pgfpathlineto{\pgfqpoint{4.973135in}{2.125802in}}%
\pgfpathlineto{\pgfqpoint{4.998127in}{2.127400in}}%
\pgfpathlineto{\pgfqpoint{5.023119in}{2.128765in}}%
\pgfpathlineto{\pgfqpoint{5.048111in}{2.129927in}}%
\pgfpathlineto{\pgfqpoint{5.073103in}{2.130916in}}%
\pgfpathlineto{\pgfqpoint{5.098095in}{2.131757in}}%
\pgfpathlineto{\pgfqpoint{5.108215in}{2.132046in}}%
\pgfusepath{stroke}%
\end{pgfscope}%
\begin{pgfscope}%
\pgfpathrectangle{\pgfqpoint{3.874531in}{1.689230in}}{\pgfqpoint{1.223684in}{1.004348in}}%
\pgfusepath{clip}%
\pgfsetrectcap%
\pgfsetroundjoin%
\pgfsetlinewidth{0.803000pt}%
\definecolor{currentstroke}{rgb}{0.686275,0.352941,0.313725}%
\pgfsetstrokecolor{currentstroke}%
\pgfsetdash{}{0pt}%
\pgfpathmoveto{\pgfqpoint{3.923478in}{2.136316in}}%
\pgfpathlineto{\pgfqpoint{3.948470in}{2.136316in}}%
\pgfpathlineto{\pgfqpoint{3.973462in}{2.136316in}}%
\pgfpathlineto{\pgfqpoint{3.998454in}{2.136316in}}%
\pgfpathlineto{\pgfqpoint{4.023446in}{2.136316in}}%
\pgfpathlineto{\pgfqpoint{4.048437in}{2.136316in}}%
\pgfpathlineto{\pgfqpoint{4.073429in}{2.136316in}}%
\pgfpathlineto{\pgfqpoint{4.098421in}{2.136316in}}%
\pgfpathlineto{\pgfqpoint{4.123413in}{2.136316in}}%
\pgfpathlineto{\pgfqpoint{4.148405in}{2.120627in}}%
\pgfpathlineto{\pgfqpoint{4.173397in}{2.110626in}}%
\pgfpathlineto{\pgfqpoint{4.198388in}{2.104766in}}%
\pgfpathlineto{\pgfqpoint{4.223380in}{2.101875in}}%
\pgfpathlineto{\pgfqpoint{4.248372in}{2.101069in}}%
\pgfpathlineto{\pgfqpoint{4.273364in}{2.061405in}}%
\pgfpathlineto{\pgfqpoint{4.298356in}{2.037280in}}%
\pgfpathlineto{\pgfqpoint{4.323348in}{2.024362in}}%
\pgfpathlineto{\pgfqpoint{4.348339in}{2.019381in}}%
\pgfpathlineto{\pgfqpoint{4.373331in}{2.019885in}}%
\pgfpathlineto{\pgfqpoint{4.398323in}{2.024049in}}%
\pgfpathlineto{\pgfqpoint{4.423315in}{2.030529in}}%
\pgfpathlineto{\pgfqpoint{4.448307in}{2.038348in}}%
\pgfpathlineto{\pgfqpoint{4.473299in}{2.046809in}}%
\pgfpathlineto{\pgfqpoint{4.498290in}{2.055421in}}%
\pgfpathlineto{\pgfqpoint{4.523282in}{2.063852in}}%
\pgfpathlineto{\pgfqpoint{4.548274in}{2.071885in}}%
\pgfpathlineto{\pgfqpoint{4.573266in}{2.079387in}}%
\pgfpathlineto{\pgfqpoint{4.598258in}{2.086286in}}%
\pgfpathlineto{\pgfqpoint{4.623250in}{2.092554in}}%
\pgfpathlineto{\pgfqpoint{4.648241in}{2.098194in}}%
\pgfpathlineto{\pgfqpoint{4.673233in}{2.103227in}}%
\pgfpathlineto{\pgfqpoint{4.698225in}{2.107688in}}%
\pgfpathlineto{\pgfqpoint{4.723217in}{2.111619in}}%
\pgfpathlineto{\pgfqpoint{4.748209in}{2.115066in}}%
\pgfpathlineto{\pgfqpoint{4.773201in}{2.118074in}}%
\pgfpathlineto{\pgfqpoint{4.798192in}{2.120690in}}%
\pgfpathlineto{\pgfqpoint{4.823184in}{2.122957in}}%
\pgfpathlineto{\pgfqpoint{4.848176in}{2.124916in}}%
\pgfpathlineto{\pgfqpoint{4.873168in}{2.126603in}}%
\pgfpathlineto{\pgfqpoint{4.898160in}{2.128054in}}%
\pgfpathlineto{\pgfqpoint{4.923152in}{2.129297in}}%
\pgfpathlineto{\pgfqpoint{4.948143in}{2.130361in}}%
\pgfpathlineto{\pgfqpoint{4.973135in}{2.131271in}}%
\pgfpathlineto{\pgfqpoint{4.998127in}{2.132046in}}%
\pgfpathlineto{\pgfqpoint{5.023119in}{2.132706in}}%
\pgfpathlineto{\pgfqpoint{5.048111in}{2.133267in}}%
\pgfpathlineto{\pgfqpoint{5.073103in}{2.133743in}}%
\pgfpathlineto{\pgfqpoint{5.098095in}{2.134147in}}%
\pgfpathlineto{\pgfqpoint{5.108215in}{2.134285in}}%
\pgfusepath{stroke}%
\end{pgfscope}%
\begin{pgfscope}%
\pgfpathrectangle{\pgfqpoint{3.874531in}{1.689230in}}{\pgfqpoint{1.223684in}{1.004348in}}%
\pgfusepath{clip}%
\pgfsetrectcap%
\pgfsetroundjoin%
\pgfsetlinewidth{0.803000pt}%
\definecolor{currentstroke}{rgb}{0.000000,0.356863,0.509804}%
\pgfsetstrokecolor{currentstroke}%
\pgfsetdash{}{0pt}%
\pgfpathmoveto{\pgfqpoint{3.923478in}{2.136316in}}%
\pgfpathlineto{\pgfqpoint{3.948470in}{2.136316in}}%
\pgfpathlineto{\pgfqpoint{3.973462in}{2.136316in}}%
\pgfpathlineto{\pgfqpoint{3.998454in}{2.136316in}}%
\pgfpathlineto{\pgfqpoint{4.023446in}{2.136316in}}%
\pgfpathlineto{\pgfqpoint{4.048437in}{2.136316in}}%
\pgfpathlineto{\pgfqpoint{4.073429in}{2.136316in}}%
\pgfpathlineto{\pgfqpoint{4.098421in}{2.136316in}}%
\pgfpathlineto{\pgfqpoint{4.123413in}{2.136316in}}%
\pgfpathlineto{\pgfqpoint{4.148405in}{2.097529in}}%
\pgfpathlineto{\pgfqpoint{4.173397in}{2.072804in}}%
\pgfpathlineto{\pgfqpoint{4.198388in}{2.058317in}}%
\pgfpathlineto{\pgfqpoint{4.223380in}{2.051169in}}%
\pgfpathlineto{\pgfqpoint{4.248372in}{2.049176in}}%
\pgfpathlineto{\pgfqpoint{4.273364in}{2.040625in}}%
\pgfpathlineto{\pgfqpoint{4.298356in}{2.038038in}}%
\pgfpathlineto{\pgfqpoint{4.323348in}{2.039532in}}%
\pgfpathlineto{\pgfqpoint{4.348339in}{2.043715in}}%
\pgfpathlineto{\pgfqpoint{4.373331in}{2.049561in}}%
\pgfpathlineto{\pgfqpoint{4.398323in}{2.056330in}}%
\pgfpathlineto{\pgfqpoint{4.423315in}{2.063496in}}%
\pgfpathlineto{\pgfqpoint{4.448307in}{2.070692in}}%
\pgfpathlineto{\pgfqpoint{4.473299in}{2.077672in}}%
\pgfpathlineto{\pgfqpoint{4.498290in}{2.084278in}}%
\pgfpathlineto{\pgfqpoint{4.523282in}{2.090416in}}%
\pgfpathlineto{\pgfqpoint{4.548274in}{2.096039in}}%
\pgfpathlineto{\pgfqpoint{4.573266in}{2.101131in}}%
\pgfpathlineto{\pgfqpoint{4.598258in}{2.105701in}}%
\pgfpathlineto{\pgfqpoint{4.623250in}{2.109770in}}%
\pgfpathlineto{\pgfqpoint{4.648241in}{2.113370in}}%
\pgfpathlineto{\pgfqpoint{4.673233in}{2.116537in}}%
\pgfpathlineto{\pgfqpoint{4.698225in}{2.119309in}}%
\pgfpathlineto{\pgfqpoint{4.723217in}{2.121727in}}%
\pgfpathlineto{\pgfqpoint{4.748209in}{2.123827in}}%
\pgfpathlineto{\pgfqpoint{4.773201in}{2.125645in}}%
\pgfpathlineto{\pgfqpoint{4.798192in}{2.127214in}}%
\pgfpathlineto{\pgfqpoint{4.823184in}{2.128565in}}%
\pgfpathlineto{\pgfqpoint{4.848176in}{2.129725in}}%
\pgfpathlineto{\pgfqpoint{4.873168in}{2.130720in}}%
\pgfpathlineto{\pgfqpoint{4.898160in}{2.131570in}}%
\pgfpathlineto{\pgfqpoint{4.923152in}{2.132296in}}%
\pgfpathlineto{\pgfqpoint{4.948143in}{2.132915in}}%
\pgfpathlineto{\pgfqpoint{4.973135in}{2.133441in}}%
\pgfpathlineto{\pgfqpoint{4.998127in}{2.133889in}}%
\pgfpathlineto{\pgfqpoint{5.023119in}{2.134268in}}%
\pgfpathlineto{\pgfqpoint{5.048111in}{2.134590in}}%
\pgfpathlineto{\pgfqpoint{5.073103in}{2.134862in}}%
\pgfpathlineto{\pgfqpoint{5.098095in}{2.135093in}}%
\pgfpathlineto{\pgfqpoint{5.108215in}{2.135171in}}%
\pgfusepath{stroke}%
\end{pgfscope}%
\begin{pgfscope}%
\pgfpathrectangle{\pgfqpoint{3.874531in}{1.689230in}}{\pgfqpoint{1.223684in}{1.004348in}}%
\pgfusepath{clip}%
\pgfsetrectcap%
\pgfsetroundjoin%
\pgfsetlinewidth{0.803000pt}%
\definecolor{currentstroke}{rgb}{0.490196,0.588235,0.431373}%
\pgfsetstrokecolor{currentstroke}%
\pgfsetdash{}{0pt}%
\pgfpathmoveto{\pgfqpoint{3.923478in}{2.136316in}}%
\pgfpathlineto{\pgfqpoint{3.948470in}{2.136316in}}%
\pgfpathlineto{\pgfqpoint{3.973462in}{2.136316in}}%
\pgfpathlineto{\pgfqpoint{3.998454in}{2.136316in}}%
\pgfpathlineto{\pgfqpoint{4.023446in}{2.136316in}}%
\pgfpathlineto{\pgfqpoint{4.048437in}{2.136316in}}%
\pgfpathlineto{\pgfqpoint{4.073429in}{2.136316in}}%
\pgfpathlineto{\pgfqpoint{4.098421in}{2.136316in}}%
\pgfpathlineto{\pgfqpoint{4.123413in}{2.136316in}}%
\pgfpathlineto{\pgfqpoint{4.148405in}{2.157062in}}%
\pgfpathlineto{\pgfqpoint{4.173397in}{2.170287in}}%
\pgfpathlineto{\pgfqpoint{4.198388in}{2.178036in}}%
\pgfpathlineto{\pgfqpoint{4.223380in}{2.181859in}}%
\pgfpathlineto{\pgfqpoint{4.248372in}{2.182926in}}%
\pgfpathlineto{\pgfqpoint{4.273364in}{2.185755in}}%
\pgfpathlineto{\pgfqpoint{4.298356in}{2.186026in}}%
\pgfpathlineto{\pgfqpoint{4.323348in}{2.184575in}}%
\pgfpathlineto{\pgfqpoint{4.348339in}{2.182017in}}%
\pgfpathlineto{\pgfqpoint{4.373331in}{2.178800in}}%
\pgfpathlineto{\pgfqpoint{4.398323in}{2.175248in}}%
\pgfpathlineto{\pgfqpoint{4.423315in}{2.171587in}}%
\pgfpathlineto{\pgfqpoint{4.448307in}{2.167975in}}%
\pgfpathlineto{\pgfqpoint{4.473299in}{2.164513in}}%
\pgfpathlineto{\pgfqpoint{4.498290in}{2.161266in}}%
\pgfpathlineto{\pgfqpoint{4.523282in}{2.158269in}}%
\pgfpathlineto{\pgfqpoint{4.548274in}{2.155540in}}%
\pgfpathlineto{\pgfqpoint{4.573266in}{2.153078in}}%
\pgfpathlineto{\pgfqpoint{4.598258in}{2.150877in}}%
\pgfpathlineto{\pgfqpoint{4.623250in}{2.148923in}}%
\pgfpathlineto{\pgfqpoint{4.648241in}{2.147200in}}%
\pgfpathlineto{\pgfqpoint{4.673233in}{2.145686in}}%
\pgfpathlineto{\pgfqpoint{4.698225in}{2.144364in}}%
\pgfpathlineto{\pgfqpoint{4.723217in}{2.143213in}}%
\pgfpathlineto{\pgfqpoint{4.748209in}{2.142215in}}%
\pgfpathlineto{\pgfqpoint{4.773201in}{2.141353in}}%
\pgfpathlineto{\pgfqpoint{4.798192in}{2.140609in}}%
\pgfpathlineto{\pgfqpoint{4.823184in}{2.139969in}}%
\pgfpathlineto{\pgfqpoint{4.848176in}{2.139420in}}%
\pgfpathlineto{\pgfqpoint{4.873168in}{2.138950in}}%
\pgfpathlineto{\pgfqpoint{4.898160in}{2.138549in}}%
\pgfpathlineto{\pgfqpoint{4.923152in}{2.138206in}}%
\pgfpathlineto{\pgfqpoint{4.948143in}{2.137914in}}%
\pgfpathlineto{\pgfqpoint{4.973135in}{2.137666in}}%
\pgfpathlineto{\pgfqpoint{4.998127in}{2.137456in}}%
\pgfpathlineto{\pgfqpoint{5.023119in}{2.137277in}}%
\pgfpathlineto{\pgfqpoint{5.048111in}{2.137126in}}%
\pgfpathlineto{\pgfqpoint{5.073103in}{2.136998in}}%
\pgfpathlineto{\pgfqpoint{5.098095in}{2.136889in}}%
\pgfpathlineto{\pgfqpoint{5.108215in}{2.136852in}}%
\pgfusepath{stroke}%
\end{pgfscope}%
\begin{pgfscope}%
\pgfpathrectangle{\pgfqpoint{3.874531in}{1.689230in}}{\pgfqpoint{1.223684in}{1.004348in}}%
\pgfusepath{clip}%
\pgfsetrectcap%
\pgfsetroundjoin%
\pgfsetlinewidth{0.803000pt}%
\definecolor{currentstroke}{rgb}{0.843137,0.666667,0.313725}%
\pgfsetstrokecolor{currentstroke}%
\pgfsetdash{}{0pt}%
\pgfpathmoveto{\pgfqpoint{3.923478in}{2.136316in}}%
\pgfpathlineto{\pgfqpoint{3.948470in}{2.136316in}}%
\pgfpathlineto{\pgfqpoint{3.973462in}{2.136316in}}%
\pgfpathlineto{\pgfqpoint{3.998454in}{2.136316in}}%
\pgfpathlineto{\pgfqpoint{4.023446in}{2.136316in}}%
\pgfpathlineto{\pgfqpoint{4.048437in}{2.136316in}}%
\pgfpathlineto{\pgfqpoint{4.073429in}{2.136316in}}%
\pgfpathlineto{\pgfqpoint{4.098421in}{2.136316in}}%
\pgfpathlineto{\pgfqpoint{4.123413in}{2.136316in}}%
\pgfpathlineto{\pgfqpoint{4.148405in}{2.141054in}}%
\pgfpathlineto{\pgfqpoint{4.173397in}{2.144075in}}%
\pgfpathlineto{\pgfqpoint{4.198388in}{2.145845in}}%
\pgfpathlineto{\pgfqpoint{4.223380in}{2.146718in}}%
\pgfpathlineto{\pgfqpoint{4.248372in}{2.146962in}}%
\pgfpathlineto{\pgfqpoint{4.273364in}{2.148741in}}%
\pgfpathlineto{\pgfqpoint{4.298356in}{2.149526in}}%
\pgfpathlineto{\pgfqpoint{4.323348in}{2.149618in}}%
\pgfpathlineto{\pgfqpoint{4.348339in}{2.149242in}}%
\pgfpathlineto{\pgfqpoint{4.373331in}{2.148566in}}%
\pgfpathlineto{\pgfqpoint{4.398323in}{2.147710in}}%
\pgfpathlineto{\pgfqpoint{4.423315in}{2.146762in}}%
\pgfpathlineto{\pgfqpoint{4.448307in}{2.145783in}}%
\pgfpathlineto{\pgfqpoint{4.473299in}{2.144816in}}%
\pgfpathlineto{\pgfqpoint{4.498290in}{2.143888in}}%
\pgfpathlineto{\pgfqpoint{4.523282in}{2.143017in}}%
\pgfpathlineto{\pgfqpoint{4.548274in}{2.142214in}}%
\pgfpathlineto{\pgfqpoint{4.573266in}{2.141481in}}%
\pgfpathlineto{\pgfqpoint{4.598258in}{2.140820in}}%
\pgfpathlineto{\pgfqpoint{4.623250in}{2.140229in}}%
\pgfpathlineto{\pgfqpoint{4.648241in}{2.139705in}}%
\pgfpathlineto{\pgfqpoint{4.673233in}{2.139241in}}%
\pgfpathlineto{\pgfqpoint{4.698225in}{2.138835in}}%
\pgfpathlineto{\pgfqpoint{4.723217in}{2.138480in}}%
\pgfpathlineto{\pgfqpoint{4.748209in}{2.138170in}}%
\pgfpathlineto{\pgfqpoint{4.773201in}{2.137902in}}%
\pgfpathlineto{\pgfqpoint{4.798192in}{2.137670in}}%
\pgfpathlineto{\pgfqpoint{4.823184in}{2.137470in}}%
\pgfpathlineto{\pgfqpoint{4.848176in}{2.137298in}}%
\pgfpathlineto{\pgfqpoint{4.873168in}{2.137151in}}%
\pgfpathlineto{\pgfqpoint{4.898160in}{2.137024in}}%
\pgfpathlineto{\pgfqpoint{4.923152in}{2.136916in}}%
\pgfpathlineto{\pgfqpoint{4.948143in}{2.136824in}}%
\pgfpathlineto{\pgfqpoint{4.973135in}{2.136746in}}%
\pgfpathlineto{\pgfqpoint{4.998127in}{2.136679in}}%
\pgfpathlineto{\pgfqpoint{5.023119in}{2.136622in}}%
\pgfpathlineto{\pgfqpoint{5.048111in}{2.136574in}}%
\pgfpathlineto{\pgfqpoint{5.073103in}{2.136534in}}%
\pgfpathlineto{\pgfqpoint{5.098095in}{2.136499in}}%
\pgfpathlineto{\pgfqpoint{5.108215in}{2.136487in}}%
\pgfusepath{stroke}%
\end{pgfscope}%
\begin{pgfscope}%
\pgfsetrectcap%
\pgfsetmiterjoin%
\pgfsetlinewidth{0.501875pt}%
\definecolor{currentstroke}{rgb}{0.317647,0.317647,0.317647}%
\pgfsetstrokecolor{currentstroke}%
\pgfsetdash{}{0pt}%
\pgfpathmoveto{\pgfqpoint{3.874531in}{1.689230in}}%
\pgfpathlineto{\pgfqpoint{3.874531in}{2.693578in}}%
\pgfusepath{stroke}%
\end{pgfscope}%
\begin{pgfscope}%
\pgfsetrectcap%
\pgfsetmiterjoin%
\pgfsetlinewidth{0.501875pt}%
\definecolor{currentstroke}{rgb}{0.317647,0.317647,0.317647}%
\pgfsetstrokecolor{currentstroke}%
\pgfsetdash{}{0pt}%
\pgfpathmoveto{\pgfqpoint{3.874531in}{1.689230in}}%
\pgfpathlineto{\pgfqpoint{5.098215in}{1.689230in}}%
\pgfusepath{stroke}%
\end{pgfscope}%
\begin{pgfscope}%
\definecolor{textcolor}{rgb}{0.000000,0.000000,0.000000}%
\pgfsetstrokecolor{textcolor}%
\pgfsetfillcolor{textcolor}%
\pgftext[x=4.486373in,y=2.776911in,,base]{\color{textcolor}\rmfamily\fontsize{6.664000}{7.996800}\selectfont Error}%
\end{pgfscope}%
\begin{pgfscope}%
\pgfsetbuttcap%
\pgfsetmiterjoin%
\pgfsetlinewidth{0.000000pt}%
\definecolor{currentstroke}{rgb}{0.000000,0.000000,0.000000}%
\pgfsetstrokecolor{currentstroke}%
\pgfsetstrokeopacity{0.000000}%
\pgfsetdash{}{0pt}%
\pgfpathmoveto{\pgfqpoint{0.448215in}{0.383578in}}%
\pgfpathlineto{\pgfqpoint{1.671899in}{0.383578in}}%
\pgfpathlineto{\pgfqpoint{1.671899in}{1.387926in}}%
\pgfpathlineto{\pgfqpoint{0.448215in}{1.387926in}}%
\pgfpathclose%
\pgfusepath{}%
\end{pgfscope}%
\begin{pgfscope}%
\pgfsetbuttcap%
\pgfsetroundjoin%
\definecolor{currentfill}{rgb}{0.317647,0.317647,0.317647}%
\pgfsetfillcolor{currentfill}%
\pgfsetlinewidth{0.501875pt}%
\definecolor{currentstroke}{rgb}{0.317647,0.317647,0.317647}%
\pgfsetstrokecolor{currentstroke}%
\pgfsetdash{}{0pt}%
\pgfsys@defobject{currentmarker}{\pgfqpoint{0.000000in}{-0.020833in}}{\pgfqpoint{0.000000in}{0.000000in}}{%
\pgfpathmoveto{\pgfqpoint{0.000000in}{0.000000in}}%
\pgfpathlineto{\pgfqpoint{0.000000in}{-0.020833in}}%
\pgfusepath{stroke,fill}%
}%
\begin{pgfscope}%
\pgfsys@transformshift{0.497163in}{0.383578in}%
\pgfsys@useobject{currentmarker}{}%
\end{pgfscope}%
\end{pgfscope}%
\begin{pgfscope}%
\definecolor{textcolor}{rgb}{0.317647,0.317647,0.317647}%
\pgfsetstrokecolor{textcolor}%
\pgfsetfillcolor{textcolor}%
\pgftext[x=0.497163in,y=0.334967in,,top]{\color{textcolor}\rmfamily\fontsize{6.664000}{7.996800}\selectfont \(\displaystyle 0\)}%
\end{pgfscope}%
\begin{pgfscope}%
\pgfsetbuttcap%
\pgfsetroundjoin%
\definecolor{currentfill}{rgb}{0.317647,0.317647,0.317647}%
\pgfsetfillcolor{currentfill}%
\pgfsetlinewidth{0.501875pt}%
\definecolor{currentstroke}{rgb}{0.317647,0.317647,0.317647}%
\pgfsetstrokecolor{currentstroke}%
\pgfsetdash{}{0pt}%
\pgfsys@defobject{currentmarker}{\pgfqpoint{0.000000in}{-0.020833in}}{\pgfqpoint{0.000000in}{0.000000in}}{%
\pgfpathmoveto{\pgfqpoint{0.000000in}{0.000000in}}%
\pgfpathlineto{\pgfqpoint{0.000000in}{-0.020833in}}%
\pgfusepath{stroke,fill}%
}%
\begin{pgfscope}%
\pgfsys@transformshift{0.986636in}{0.383578in}%
\pgfsys@useobject{currentmarker}{}%
\end{pgfscope}%
\end{pgfscope}%
\begin{pgfscope}%
\definecolor{textcolor}{rgb}{0.317647,0.317647,0.317647}%
\pgfsetstrokecolor{textcolor}%
\pgfsetfillcolor{textcolor}%
\pgftext[x=0.986636in,y=0.334967in,,top]{\color{textcolor}\rmfamily\fontsize{6.664000}{7.996800}\selectfont \(\displaystyle 50\)}%
\end{pgfscope}%
\begin{pgfscope}%
\pgfsetbuttcap%
\pgfsetroundjoin%
\definecolor{currentfill}{rgb}{0.317647,0.317647,0.317647}%
\pgfsetfillcolor{currentfill}%
\pgfsetlinewidth{0.501875pt}%
\definecolor{currentstroke}{rgb}{0.317647,0.317647,0.317647}%
\pgfsetstrokecolor{currentstroke}%
\pgfsetdash{}{0pt}%
\pgfsys@defobject{currentmarker}{\pgfqpoint{0.000000in}{-0.020833in}}{\pgfqpoint{0.000000in}{0.000000in}}{%
\pgfpathmoveto{\pgfqpoint{0.000000in}{0.000000in}}%
\pgfpathlineto{\pgfqpoint{0.000000in}{-0.020833in}}%
\pgfusepath{stroke,fill}%
}%
\begin{pgfscope}%
\pgfsys@transformshift{1.476110in}{0.383578in}%
\pgfsys@useobject{currentmarker}{}%
\end{pgfscope}%
\end{pgfscope}%
\begin{pgfscope}%
\definecolor{textcolor}{rgb}{0.317647,0.317647,0.317647}%
\pgfsetstrokecolor{textcolor}%
\pgfsetfillcolor{textcolor}%
\pgftext[x=1.476110in,y=0.334967in,,top]{\color{textcolor}\rmfamily\fontsize{6.664000}{7.996800}\selectfont \(\displaystyle 100\)}%
\end{pgfscope}%
\begin{pgfscope}%
\definecolor{textcolor}{rgb}{0.317647,0.317647,0.317647}%
\pgfsetstrokecolor{textcolor}%
\pgfsetfillcolor{textcolor}%
\pgftext[x=1.060057in,y=0.197222in,,top]{\color{textcolor}\rmfamily\fontsize{6.664000}{7.996800}\selectfont time\(\displaystyle \;(\si{\s})\)}%
\end{pgfscope}%
\begin{pgfscope}%
\pgfsetbuttcap%
\pgfsetroundjoin%
\definecolor{currentfill}{rgb}{0.317647,0.317647,0.317647}%
\pgfsetfillcolor{currentfill}%
\pgfsetlinewidth{0.501875pt}%
\definecolor{currentstroke}{rgb}{0.317647,0.317647,0.317647}%
\pgfsetstrokecolor{currentstroke}%
\pgfsetdash{}{0pt}%
\pgfsys@defobject{currentmarker}{\pgfqpoint{-0.020833in}{0.000000in}}{\pgfqpoint{0.000000in}{0.000000in}}{%
\pgfpathmoveto{\pgfqpoint{0.000000in}{0.000000in}}%
\pgfpathlineto{\pgfqpoint{-0.020833in}{0.000000in}}%
\pgfusepath{stroke,fill}%
}%
\begin{pgfscope}%
\pgfsys@transformshift{0.448215in}{0.446350in}%
\pgfsys@useobject{currentmarker}{}%
\end{pgfscope}%
\end{pgfscope}%
\begin{pgfscope}%
\definecolor{textcolor}{rgb}{0.317647,0.317647,0.317647}%
\pgfsetstrokecolor{textcolor}%
\pgfsetfillcolor{textcolor}%
\pgftext[x=0.358130in,y=0.414233in,left,base]{\color{textcolor}\rmfamily\fontsize{6.664000}{7.996800}\selectfont \(\displaystyle 0\)}%
\end{pgfscope}%
\begin{pgfscope}%
\pgfsetbuttcap%
\pgfsetroundjoin%
\definecolor{currentfill}{rgb}{0.317647,0.317647,0.317647}%
\pgfsetfillcolor{currentfill}%
\pgfsetlinewidth{0.501875pt}%
\definecolor{currentstroke}{rgb}{0.317647,0.317647,0.317647}%
\pgfsetstrokecolor{currentstroke}%
\pgfsetdash{}{0pt}%
\pgfsys@defobject{currentmarker}{\pgfqpoint{-0.020833in}{0.000000in}}{\pgfqpoint{0.000000in}{0.000000in}}{%
\pgfpathmoveto{\pgfqpoint{0.000000in}{0.000000in}}%
\pgfpathlineto{\pgfqpoint{-0.020833in}{0.000000in}}%
\pgfusepath{stroke,fill}%
}%
\begin{pgfscope}%
\pgfsys@transformshift{0.448215in}{0.760208in}%
\pgfsys@useobject{currentmarker}{}%
\end{pgfscope}%
\end{pgfscope}%
\begin{pgfscope}%
\definecolor{textcolor}{rgb}{0.317647,0.317647,0.317647}%
\pgfsetstrokecolor{textcolor}%
\pgfsetfillcolor{textcolor}%
\pgftext[x=0.302767in,y=0.728092in,left,base]{\color{textcolor}\rmfamily\fontsize{6.664000}{7.996800}\selectfont \(\displaystyle 10\)}%
\end{pgfscope}%
\begin{pgfscope}%
\pgfsetbuttcap%
\pgfsetroundjoin%
\definecolor{currentfill}{rgb}{0.317647,0.317647,0.317647}%
\pgfsetfillcolor{currentfill}%
\pgfsetlinewidth{0.501875pt}%
\definecolor{currentstroke}{rgb}{0.317647,0.317647,0.317647}%
\pgfsetstrokecolor{currentstroke}%
\pgfsetdash{}{0pt}%
\pgfsys@defobject{currentmarker}{\pgfqpoint{-0.020833in}{0.000000in}}{\pgfqpoint{0.000000in}{0.000000in}}{%
\pgfpathmoveto{\pgfqpoint{0.000000in}{0.000000in}}%
\pgfpathlineto{\pgfqpoint{-0.020833in}{0.000000in}}%
\pgfusepath{stroke,fill}%
}%
\begin{pgfscope}%
\pgfsys@transformshift{0.448215in}{1.074067in}%
\pgfsys@useobject{currentmarker}{}%
\end{pgfscope}%
\end{pgfscope}%
\begin{pgfscope}%
\definecolor{textcolor}{rgb}{0.317647,0.317647,0.317647}%
\pgfsetstrokecolor{textcolor}%
\pgfsetfillcolor{textcolor}%
\pgftext[x=0.302767in,y=1.041950in,left,base]{\color{textcolor}\rmfamily\fontsize{6.664000}{7.996800}\selectfont \(\displaystyle 20\)}%
\end{pgfscope}%
\begin{pgfscope}%
\pgfsetbuttcap%
\pgfsetroundjoin%
\definecolor{currentfill}{rgb}{0.317647,0.317647,0.317647}%
\pgfsetfillcolor{currentfill}%
\pgfsetlinewidth{0.501875pt}%
\definecolor{currentstroke}{rgb}{0.317647,0.317647,0.317647}%
\pgfsetstrokecolor{currentstroke}%
\pgfsetdash{}{0pt}%
\pgfsys@defobject{currentmarker}{\pgfqpoint{-0.020833in}{0.000000in}}{\pgfqpoint{0.000000in}{0.000000in}}{%
\pgfpathmoveto{\pgfqpoint{0.000000in}{0.000000in}}%
\pgfpathlineto{\pgfqpoint{-0.020833in}{0.000000in}}%
\pgfusepath{stroke,fill}%
}%
\begin{pgfscope}%
\pgfsys@transformshift{0.448215in}{1.387926in}%
\pgfsys@useobject{currentmarker}{}%
\end{pgfscope}%
\end{pgfscope}%
\begin{pgfscope}%
\definecolor{textcolor}{rgb}{0.317647,0.317647,0.317647}%
\pgfsetstrokecolor{textcolor}%
\pgfsetfillcolor{textcolor}%
\pgftext[x=0.302767in,y=1.355809in,left,base]{\color{textcolor}\rmfamily\fontsize{6.664000}{7.996800}\selectfont \(\displaystyle 30\)}%
\end{pgfscope}%
\begin{pgfscope}%
\definecolor{textcolor}{rgb}{0.317647,0.317647,0.317647}%
\pgfsetstrokecolor{textcolor}%
\pgfsetfillcolor{textcolor}%
\pgftext[x=0.247212in,y=0.885752in,,bottom,rotate=90.000000]{\color{textcolor}\rmfamily\fontsize{6.664000}{7.996800}\selectfont \(\displaystyle S_j^{(o)}\)}%
\end{pgfscope}%
\begin{pgfscope}%
\pgfpathrectangle{\pgfqpoint{0.448215in}{0.383578in}}{\pgfqpoint{1.223684in}{1.004348in}}%
\pgfusepath{clip}%
\pgfsetbuttcap%
\pgfsetroundjoin%
\pgfsetlinewidth{1.003750pt}%
\definecolor{currentstroke}{rgb}{0.333333,0.333333,0.333333}%
\pgfsetstrokecolor{currentstroke}%
\pgfsetdash{}{0pt}%
\pgfpathmoveto{\pgfqpoint{0.682144in}{0.493429in}}%
\pgfpathlineto{\pgfqpoint{0.682144in}{0.462043in}}%
\pgfusepath{stroke}%
\end{pgfscope}%
\begin{pgfscope}%
\pgfpathrectangle{\pgfqpoint{0.448215in}{0.383578in}}{\pgfqpoint{1.223684in}{1.004348in}}%
\pgfusepath{clip}%
\pgfsetbuttcap%
\pgfsetroundjoin%
\pgfsetlinewidth{1.003750pt}%
\definecolor{currentstroke}{rgb}{0.333333,0.333333,0.333333}%
\pgfsetstrokecolor{currentstroke}%
\pgfsetdash{}{0pt}%
\pgfpathmoveto{\pgfqpoint{0.613461in}{0.618972in}}%
\pgfpathlineto{\pgfqpoint{0.613461in}{0.587586in}}%
\pgfusepath{stroke}%
\end{pgfscope}%
\begin{pgfscope}%
\pgfpathrectangle{\pgfqpoint{0.448215in}{0.383578in}}{\pgfqpoint{1.223684in}{1.004348in}}%
\pgfusepath{clip}%
\pgfsetbuttcap%
\pgfsetroundjoin%
\pgfsetlinewidth{1.003750pt}%
\definecolor{currentstroke}{rgb}{0.333333,0.333333,0.333333}%
\pgfsetstrokecolor{currentstroke}%
\pgfsetdash{}{0pt}%
\pgfpathmoveto{\pgfqpoint{0.624347in}{0.838673in}}%
\pgfpathlineto{\pgfqpoint{0.624347in}{0.807287in}}%
\pgfusepath{stroke}%
\end{pgfscope}%
\begin{pgfscope}%
\pgfpathrectangle{\pgfqpoint{0.448215in}{0.383578in}}{\pgfqpoint{1.223684in}{1.004348in}}%
\pgfusepath{clip}%
\pgfsetbuttcap%
\pgfsetroundjoin%
\pgfsetlinewidth{1.003750pt}%
\definecolor{currentstroke}{rgb}{0.333333,0.333333,0.333333}%
\pgfsetstrokecolor{currentstroke}%
\pgfsetdash{}{0pt}%
\pgfpathmoveto{\pgfqpoint{0.645728in}{1.089760in}}%
\pgfpathlineto{\pgfqpoint{0.645728in}{1.058374in}}%
\pgfusepath{stroke}%
\end{pgfscope}%
\begin{pgfscope}%
\pgfpathrectangle{\pgfqpoint{0.448215in}{0.383578in}}{\pgfqpoint{1.223684in}{1.004348in}}%
\pgfusepath{clip}%
\pgfsetbuttcap%
\pgfsetroundjoin%
\pgfsetlinewidth{1.003750pt}%
\definecolor{currentstroke}{rgb}{0.333333,0.333333,0.333333}%
\pgfsetstrokecolor{currentstroke}%
\pgfsetdash{}{0pt}%
\pgfpathmoveto{\pgfqpoint{0.663349in}{1.121146in}}%
\pgfpathlineto{\pgfqpoint{0.663349in}{1.089760in}}%
\pgfusepath{stroke}%
\end{pgfscope}%
\begin{pgfscope}%
\pgfpathrectangle{\pgfqpoint{0.448215in}{0.383578in}}{\pgfqpoint{1.223684in}{1.004348in}}%
\pgfusepath{clip}%
\pgfsetbuttcap%
\pgfsetroundjoin%
\pgfsetlinewidth{1.003750pt}%
\definecolor{currentstroke}{rgb}{0.333333,0.333333,0.333333}%
\pgfsetstrokecolor{currentstroke}%
\pgfsetdash{}{0pt}%
\pgfpathmoveto{\pgfqpoint{0.589732in}{1.152532in}}%
\pgfpathlineto{\pgfqpoint{0.589732in}{1.121146in}}%
\pgfusepath{stroke}%
\end{pgfscope}%
\begin{pgfscope}%
\pgfpathrectangle{\pgfqpoint{0.448215in}{0.383578in}}{\pgfqpoint{1.223684in}{1.004348in}}%
\pgfusepath{clip}%
\pgfsetbuttcap%
\pgfsetroundjoin%
\pgfsetlinewidth{1.003750pt}%
\definecolor{currentstroke}{rgb}{0.333333,0.333333,0.333333}%
\pgfsetstrokecolor{currentstroke}%
\pgfsetdash{}{0pt}%
\pgfpathmoveto{\pgfqpoint{0.633745in}{1.215304in}}%
\pgfpathlineto{\pgfqpoint{0.633745in}{1.183918in}}%
\pgfusepath{stroke}%
\end{pgfscope}%
\begin{pgfscope}%
\pgfpathrectangle{\pgfqpoint{0.448215in}{0.383578in}}{\pgfqpoint{1.223684in}{1.004348in}}%
\pgfusepath{clip}%
\pgfsetbuttcap%
\pgfsetroundjoin%
\pgfsetlinewidth{1.003750pt}%
\definecolor{currentstroke}{rgb}{0.333333,0.333333,0.333333}%
\pgfsetstrokecolor{currentstroke}%
\pgfsetdash{}{0pt}%
\pgfpathmoveto{\pgfqpoint{0.659511in}{1.340847in}}%
\pgfpathlineto{\pgfqpoint{0.659511in}{1.309461in}}%
\pgfusepath{stroke}%
\end{pgfscope}%
\begin{pgfscope}%
\pgfsetrectcap%
\pgfsetmiterjoin%
\pgfsetlinewidth{0.501875pt}%
\definecolor{currentstroke}{rgb}{0.317647,0.317647,0.317647}%
\pgfsetstrokecolor{currentstroke}%
\pgfsetdash{}{0pt}%
\pgfpathmoveto{\pgfqpoint{0.448215in}{0.383578in}}%
\pgfpathlineto{\pgfqpoint{0.448215in}{1.387926in}}%
\pgfusepath{stroke}%
\end{pgfscope}%
\begin{pgfscope}%
\pgfsetrectcap%
\pgfsetmiterjoin%
\pgfsetlinewidth{0.501875pt}%
\definecolor{currentstroke}{rgb}{0.317647,0.317647,0.317647}%
\pgfsetstrokecolor{currentstroke}%
\pgfsetdash{}{0pt}%
\pgfpathmoveto{\pgfqpoint{0.448215in}{0.383578in}}%
\pgfpathlineto{\pgfqpoint{1.671899in}{0.383578in}}%
\pgfusepath{stroke}%
\end{pgfscope}%
\begin{pgfscope}%
\pgfsetbuttcap%
\pgfsetmiterjoin%
\pgfsetlinewidth{0.000000pt}%
\definecolor{currentstroke}{rgb}{0.000000,0.000000,0.000000}%
\pgfsetstrokecolor{currentstroke}%
\pgfsetstrokeopacity{0.000000}%
\pgfsetdash{}{0pt}%
\pgfpathmoveto{\pgfqpoint{2.161373in}{0.383578in}}%
\pgfpathlineto{\pgfqpoint{3.385057in}{0.383578in}}%
\pgfpathlineto{\pgfqpoint{3.385057in}{1.387926in}}%
\pgfpathlineto{\pgfqpoint{2.161373in}{1.387926in}}%
\pgfpathclose%
\pgfusepath{}%
\end{pgfscope}%
\begin{pgfscope}%
\pgfsetbuttcap%
\pgfsetroundjoin%
\definecolor{currentfill}{rgb}{0.317647,0.317647,0.317647}%
\pgfsetfillcolor{currentfill}%
\pgfsetlinewidth{0.501875pt}%
\definecolor{currentstroke}{rgb}{0.317647,0.317647,0.317647}%
\pgfsetstrokecolor{currentstroke}%
\pgfsetdash{}{0pt}%
\pgfsys@defobject{currentmarker}{\pgfqpoint{0.000000in}{-0.020833in}}{\pgfqpoint{0.000000in}{0.000000in}}{%
\pgfpathmoveto{\pgfqpoint{0.000000in}{0.000000in}}%
\pgfpathlineto{\pgfqpoint{0.000000in}{-0.020833in}}%
\pgfusepath{stroke,fill}%
}%
\begin{pgfscope}%
\pgfsys@transformshift{2.210320in}{0.383578in}%
\pgfsys@useobject{currentmarker}{}%
\end{pgfscope}%
\end{pgfscope}%
\begin{pgfscope}%
\definecolor{textcolor}{rgb}{0.317647,0.317647,0.317647}%
\pgfsetstrokecolor{textcolor}%
\pgfsetfillcolor{textcolor}%
\pgftext[x=2.210320in,y=0.334967in,,top]{\color{textcolor}\rmfamily\fontsize{6.664000}{7.996800}\selectfont \(\displaystyle 0\)}%
\end{pgfscope}%
\begin{pgfscope}%
\pgfsetbuttcap%
\pgfsetroundjoin%
\definecolor{currentfill}{rgb}{0.317647,0.317647,0.317647}%
\pgfsetfillcolor{currentfill}%
\pgfsetlinewidth{0.501875pt}%
\definecolor{currentstroke}{rgb}{0.317647,0.317647,0.317647}%
\pgfsetstrokecolor{currentstroke}%
\pgfsetdash{}{0pt}%
\pgfsys@defobject{currentmarker}{\pgfqpoint{0.000000in}{-0.020833in}}{\pgfqpoint{0.000000in}{0.000000in}}{%
\pgfpathmoveto{\pgfqpoint{0.000000in}{0.000000in}}%
\pgfpathlineto{\pgfqpoint{0.000000in}{-0.020833in}}%
\pgfusepath{stroke,fill}%
}%
\begin{pgfscope}%
\pgfsys@transformshift{2.699794in}{0.383578in}%
\pgfsys@useobject{currentmarker}{}%
\end{pgfscope}%
\end{pgfscope}%
\begin{pgfscope}%
\definecolor{textcolor}{rgb}{0.317647,0.317647,0.317647}%
\pgfsetstrokecolor{textcolor}%
\pgfsetfillcolor{textcolor}%
\pgftext[x=2.699794in,y=0.334967in,,top]{\color{textcolor}\rmfamily\fontsize{6.664000}{7.996800}\selectfont \(\displaystyle 50\)}%
\end{pgfscope}%
\begin{pgfscope}%
\pgfsetbuttcap%
\pgfsetroundjoin%
\definecolor{currentfill}{rgb}{0.317647,0.317647,0.317647}%
\pgfsetfillcolor{currentfill}%
\pgfsetlinewidth{0.501875pt}%
\definecolor{currentstroke}{rgb}{0.317647,0.317647,0.317647}%
\pgfsetstrokecolor{currentstroke}%
\pgfsetdash{}{0pt}%
\pgfsys@defobject{currentmarker}{\pgfqpoint{0.000000in}{-0.020833in}}{\pgfqpoint{0.000000in}{0.000000in}}{%
\pgfpathmoveto{\pgfqpoint{0.000000in}{0.000000in}}%
\pgfpathlineto{\pgfqpoint{0.000000in}{-0.020833in}}%
\pgfusepath{stroke,fill}%
}%
\begin{pgfscope}%
\pgfsys@transformshift{3.189268in}{0.383578in}%
\pgfsys@useobject{currentmarker}{}%
\end{pgfscope}%
\end{pgfscope}%
\begin{pgfscope}%
\definecolor{textcolor}{rgb}{0.317647,0.317647,0.317647}%
\pgfsetstrokecolor{textcolor}%
\pgfsetfillcolor{textcolor}%
\pgftext[x=3.189268in,y=0.334967in,,top]{\color{textcolor}\rmfamily\fontsize{6.664000}{7.996800}\selectfont \(\displaystyle 100\)}%
\end{pgfscope}%
\begin{pgfscope}%
\definecolor{textcolor}{rgb}{0.317647,0.317647,0.317647}%
\pgfsetstrokecolor{textcolor}%
\pgfsetfillcolor{textcolor}%
\pgftext[x=2.773215in,y=0.197222in,,top]{\color{textcolor}\rmfamily\fontsize{6.664000}{7.996800}\selectfont time\(\displaystyle \;(\si{\s})\)}%
\end{pgfscope}%
\begin{pgfscope}%
\pgfsetbuttcap%
\pgfsetroundjoin%
\definecolor{currentfill}{rgb}{0.317647,0.317647,0.317647}%
\pgfsetfillcolor{currentfill}%
\pgfsetlinewidth{0.501875pt}%
\definecolor{currentstroke}{rgb}{0.317647,0.317647,0.317647}%
\pgfsetstrokecolor{currentstroke}%
\pgfsetdash{}{0pt}%
\pgfsys@defobject{currentmarker}{\pgfqpoint{-0.020833in}{0.000000in}}{\pgfqpoint{0.000000in}{0.000000in}}{%
\pgfpathmoveto{\pgfqpoint{0.000000in}{0.000000in}}%
\pgfpathlineto{\pgfqpoint{-0.020833in}{0.000000in}}%
\pgfusepath{stroke,fill}%
}%
\begin{pgfscope}%
\pgfsys@transformshift{2.161373in}{0.540044in}%
\pgfsys@useobject{currentmarker}{}%
\end{pgfscope}%
\end{pgfscope}%
\begin{pgfscope}%
\definecolor{textcolor}{rgb}{0.317647,0.317647,0.317647}%
\pgfsetstrokecolor{textcolor}%
\pgfsetfillcolor{textcolor}%
\pgftext[x=1.982939in,y=0.507927in,left,base]{\color{textcolor}\rmfamily\fontsize{6.664000}{7.996800}\selectfont \(\displaystyle 0.4\)}%
\end{pgfscope}%
\begin{pgfscope}%
\pgfsetbuttcap%
\pgfsetroundjoin%
\definecolor{currentfill}{rgb}{0.317647,0.317647,0.317647}%
\pgfsetfillcolor{currentfill}%
\pgfsetlinewidth{0.501875pt}%
\definecolor{currentstroke}{rgb}{0.317647,0.317647,0.317647}%
\pgfsetstrokecolor{currentstroke}%
\pgfsetdash{}{0pt}%
\pgfsys@defobject{currentmarker}{\pgfqpoint{-0.020833in}{0.000000in}}{\pgfqpoint{0.000000in}{0.000000in}}{%
\pgfpathmoveto{\pgfqpoint{0.000000in}{0.000000in}}%
\pgfpathlineto{\pgfqpoint{-0.020833in}{0.000000in}}%
\pgfusepath{stroke,fill}%
}%
\begin{pgfscope}%
\pgfsys@transformshift{2.161373in}{0.784390in}%
\pgfsys@useobject{currentmarker}{}%
\end{pgfscope}%
\end{pgfscope}%
\begin{pgfscope}%
\definecolor{textcolor}{rgb}{0.317647,0.317647,0.317647}%
\pgfsetstrokecolor{textcolor}%
\pgfsetfillcolor{textcolor}%
\pgftext[x=1.982939in,y=0.752273in,left,base]{\color{textcolor}\rmfamily\fontsize{6.664000}{7.996800}\selectfont \(\displaystyle 0.5\)}%
\end{pgfscope}%
\begin{pgfscope}%
\pgfsetbuttcap%
\pgfsetroundjoin%
\definecolor{currentfill}{rgb}{0.317647,0.317647,0.317647}%
\pgfsetfillcolor{currentfill}%
\pgfsetlinewidth{0.501875pt}%
\definecolor{currentstroke}{rgb}{0.317647,0.317647,0.317647}%
\pgfsetstrokecolor{currentstroke}%
\pgfsetdash{}{0pt}%
\pgfsys@defobject{currentmarker}{\pgfqpoint{-0.020833in}{0.000000in}}{\pgfqpoint{0.000000in}{0.000000in}}{%
\pgfpathmoveto{\pgfqpoint{0.000000in}{0.000000in}}%
\pgfpathlineto{\pgfqpoint{-0.020833in}{0.000000in}}%
\pgfusepath{stroke,fill}%
}%
\begin{pgfscope}%
\pgfsys@transformshift{2.161373in}{1.028736in}%
\pgfsys@useobject{currentmarker}{}%
\end{pgfscope}%
\end{pgfscope}%
\begin{pgfscope}%
\definecolor{textcolor}{rgb}{0.317647,0.317647,0.317647}%
\pgfsetstrokecolor{textcolor}%
\pgfsetfillcolor{textcolor}%
\pgftext[x=1.982939in,y=0.996619in,left,base]{\color{textcolor}\rmfamily\fontsize{6.664000}{7.996800}\selectfont \(\displaystyle 0.6\)}%
\end{pgfscope}%
\begin{pgfscope}%
\pgfsetbuttcap%
\pgfsetroundjoin%
\definecolor{currentfill}{rgb}{0.317647,0.317647,0.317647}%
\pgfsetfillcolor{currentfill}%
\pgfsetlinewidth{0.501875pt}%
\definecolor{currentstroke}{rgb}{0.317647,0.317647,0.317647}%
\pgfsetstrokecolor{currentstroke}%
\pgfsetdash{}{0pt}%
\pgfsys@defobject{currentmarker}{\pgfqpoint{-0.020833in}{0.000000in}}{\pgfqpoint{0.000000in}{0.000000in}}{%
\pgfpathmoveto{\pgfqpoint{0.000000in}{0.000000in}}%
\pgfpathlineto{\pgfqpoint{-0.020833in}{0.000000in}}%
\pgfusepath{stroke,fill}%
}%
\begin{pgfscope}%
\pgfsys@transformshift{2.161373in}{1.273082in}%
\pgfsys@useobject{currentmarker}{}%
\end{pgfscope}%
\end{pgfscope}%
\begin{pgfscope}%
\definecolor{textcolor}{rgb}{0.317647,0.317647,0.317647}%
\pgfsetstrokecolor{textcolor}%
\pgfsetfillcolor{textcolor}%
\pgftext[x=1.982939in,y=1.240965in,left,base]{\color{textcolor}\rmfamily\fontsize{6.664000}{7.996800}\selectfont \(\displaystyle 0.7\)}%
\end{pgfscope}%
\begin{pgfscope}%
\definecolor{textcolor}{rgb}{0.317647,0.317647,0.317647}%
\pgfsetstrokecolor{textcolor}%
\pgfsetfillcolor{textcolor}%
\pgftext[x=1.927383in,y=0.885752in,,bottom,rotate=90.000000]{\color{textcolor}\rmfamily\fontsize{6.664000}{7.996800}\selectfont \(\displaystyle V_\mathrm{m}^{(o)} \; (\si{\V})\)}%
\end{pgfscope}%
\begin{pgfscope}%
\pgfpathrectangle{\pgfqpoint{2.161373in}{0.383578in}}{\pgfqpoint{1.223684in}{1.004348in}}%
\pgfusepath{clip}%
\pgfsetrectcap%
\pgfsetroundjoin%
\pgfsetlinewidth{0.803000pt}%
\definecolor{currentstroke}{rgb}{0.333333,0.333333,0.333333}%
\pgfsetstrokecolor{currentstroke}%
\pgfsetdash{}{0pt}%
\pgfpathmoveto{\pgfqpoint{2.210320in}{0.676289in}}%
\pgfpathlineto{\pgfqpoint{2.235312in}{0.676289in}}%
\pgfpathlineto{\pgfqpoint{2.260304in}{0.676289in}}%
\pgfpathlineto{\pgfqpoint{2.285296in}{0.676289in}}%
\pgfpathlineto{\pgfqpoint{2.310288in}{0.687031in}}%
\pgfpathlineto{\pgfqpoint{2.335280in}{0.794448in}}%
\pgfpathlineto{\pgfqpoint{2.360271in}{1.073732in}}%
\pgfpathlineto{\pgfqpoint{2.385263in}{1.213373in}}%
\pgfpathlineto{\pgfqpoint{2.410255in}{1.342274in}}%
\pgfpathlineto{\pgfqpoint{2.435247in}{0.439972in}}%
\pgfpathlineto{\pgfqpoint{2.460239in}{0.439972in}}%
\pgfpathlineto{\pgfqpoint{2.485231in}{0.429230in}}%
\pgfpathlineto{\pgfqpoint{2.510222in}{0.429230in}}%
\pgfpathlineto{\pgfqpoint{2.535214in}{0.429230in}}%
\pgfpathlineto{\pgfqpoint{2.560206in}{0.429230in}}%
\pgfpathlineto{\pgfqpoint{2.585198in}{0.429230in}}%
\pgfpathlineto{\pgfqpoint{2.610190in}{0.429230in}}%
\pgfpathlineto{\pgfqpoint{2.635182in}{0.429230in}}%
\pgfpathlineto{\pgfqpoint{2.660173in}{0.429230in}}%
\pgfpathlineto{\pgfqpoint{2.685165in}{0.429230in}}%
\pgfpathlineto{\pgfqpoint{2.710157in}{0.429230in}}%
\pgfpathlineto{\pgfqpoint{2.735149in}{0.429230in}}%
\pgfpathlineto{\pgfqpoint{2.760141in}{0.515164in}}%
\pgfpathlineto{\pgfqpoint{2.785133in}{0.568872in}}%
\pgfpathlineto{\pgfqpoint{2.810124in}{0.601097in}}%
\pgfpathlineto{\pgfqpoint{2.835116in}{0.633322in}}%
\pgfpathlineto{\pgfqpoint{2.860108in}{0.644064in}}%
\pgfpathlineto{\pgfqpoint{2.885100in}{0.665547in}}%
\pgfpathlineto{\pgfqpoint{2.910092in}{0.676289in}}%
\pgfpathlineto{\pgfqpoint{2.935084in}{0.687031in}}%
\pgfpathlineto{\pgfqpoint{2.960075in}{0.697772in}}%
\pgfpathlineto{\pgfqpoint{2.985067in}{0.697772in}}%
\pgfpathlineto{\pgfqpoint{3.010059in}{0.708514in}}%
\pgfpathlineto{\pgfqpoint{3.035051in}{0.708514in}}%
\pgfpathlineto{\pgfqpoint{3.060043in}{0.708514in}}%
\pgfpathlineto{\pgfqpoint{3.085035in}{0.697772in}}%
\pgfpathlineto{\pgfqpoint{3.110026in}{0.697772in}}%
\pgfpathlineto{\pgfqpoint{3.135018in}{0.697772in}}%
\pgfpathlineto{\pgfqpoint{3.160010in}{0.697772in}}%
\pgfpathlineto{\pgfqpoint{3.185002in}{0.697772in}}%
\pgfpathlineto{\pgfqpoint{3.209994in}{0.697772in}}%
\pgfpathlineto{\pgfqpoint{3.234986in}{0.697772in}}%
\pgfpathlineto{\pgfqpoint{3.259977in}{0.708514in}}%
\pgfpathlineto{\pgfqpoint{3.284969in}{0.708514in}}%
\pgfpathlineto{\pgfqpoint{3.309961in}{0.708514in}}%
\pgfpathlineto{\pgfqpoint{3.334953in}{0.708514in}}%
\pgfpathlineto{\pgfqpoint{3.359945in}{0.719256in}}%
\pgfpathlineto{\pgfqpoint{3.384937in}{0.729997in}}%
\pgfpathlineto{\pgfqpoint{3.395057in}{0.729997in}}%
\pgfusepath{stroke}%
\end{pgfscope}%
\begin{pgfscope}%
\pgfpathrectangle{\pgfqpoint{2.161373in}{0.383578in}}{\pgfqpoint{1.223684in}{1.004348in}}%
\pgfusepath{clip}%
\pgfsetrectcap%
\pgfsetroundjoin%
\pgfsetlinewidth{0.803000pt}%
\definecolor{currentstroke}{rgb}{0.686275,0.352941,0.313725}%
\pgfsetstrokecolor{currentstroke}%
\pgfsetdash{}{0pt}%
\pgfpathmoveto{\pgfqpoint{2.210320in}{0.826673in}}%
\pgfpathlineto{\pgfqpoint{2.235312in}{0.826673in}}%
\pgfpathlineto{\pgfqpoint{2.260304in}{0.815931in}}%
\pgfpathlineto{\pgfqpoint{2.285296in}{0.826673in}}%
\pgfpathlineto{\pgfqpoint{2.310288in}{0.912606in}}%
\pgfpathlineto{\pgfqpoint{2.335280in}{1.052248in}}%
\pgfpathlineto{\pgfqpoint{2.360271in}{1.127440in}}%
\pgfpathlineto{\pgfqpoint{2.385263in}{1.224115in}}%
\pgfpathlineto{\pgfqpoint{2.410255in}{1.277824in}}%
\pgfpathlineto{\pgfqpoint{2.435247in}{1.234857in}}%
\pgfpathlineto{\pgfqpoint{2.460239in}{1.181148in}}%
\pgfpathlineto{\pgfqpoint{2.485231in}{1.105957in}}%
\pgfpathlineto{\pgfqpoint{2.510222in}{1.041506in}}%
\pgfpathlineto{\pgfqpoint{2.535214in}{0.987798in}}%
\pgfpathlineto{\pgfqpoint{2.560206in}{0.955573in}}%
\pgfpathlineto{\pgfqpoint{2.585198in}{0.923348in}}%
\pgfpathlineto{\pgfqpoint{2.610190in}{0.901865in}}%
\pgfpathlineto{\pgfqpoint{2.635182in}{0.880381in}}%
\pgfpathlineto{\pgfqpoint{2.660173in}{0.858898in}}%
\pgfpathlineto{\pgfqpoint{2.685165in}{0.858898in}}%
\pgfpathlineto{\pgfqpoint{2.710157in}{0.848156in}}%
\pgfpathlineto{\pgfqpoint{2.735149in}{0.837414in}}%
\pgfpathlineto{\pgfqpoint{2.760141in}{0.837414in}}%
\pgfpathlineto{\pgfqpoint{2.785133in}{0.837414in}}%
\pgfpathlineto{\pgfqpoint{2.810124in}{0.826673in}}%
\pgfpathlineto{\pgfqpoint{2.835116in}{0.826673in}}%
\pgfpathlineto{\pgfqpoint{2.860108in}{0.826673in}}%
\pgfpathlineto{\pgfqpoint{2.885100in}{0.815931in}}%
\pgfpathlineto{\pgfqpoint{2.910092in}{0.815931in}}%
\pgfpathlineto{\pgfqpoint{2.935084in}{0.826673in}}%
\pgfpathlineto{\pgfqpoint{2.960075in}{0.826673in}}%
\pgfpathlineto{\pgfqpoint{2.985067in}{0.826673in}}%
\pgfpathlineto{\pgfqpoint{3.010059in}{0.826673in}}%
\pgfpathlineto{\pgfqpoint{3.035051in}{0.826673in}}%
\pgfpathlineto{\pgfqpoint{3.060043in}{0.826673in}}%
\pgfpathlineto{\pgfqpoint{3.085035in}{0.826673in}}%
\pgfpathlineto{\pgfqpoint{3.110026in}{0.826673in}}%
\pgfpathlineto{\pgfqpoint{3.135018in}{0.826673in}}%
\pgfpathlineto{\pgfqpoint{3.160010in}{0.826673in}}%
\pgfpathlineto{\pgfqpoint{3.185002in}{0.826673in}}%
\pgfpathlineto{\pgfqpoint{3.209994in}{0.815931in}}%
\pgfpathlineto{\pgfqpoint{3.234986in}{0.826673in}}%
\pgfpathlineto{\pgfqpoint{3.259977in}{0.826673in}}%
\pgfpathlineto{\pgfqpoint{3.284969in}{0.826673in}}%
\pgfpathlineto{\pgfqpoint{3.309961in}{0.826673in}}%
\pgfpathlineto{\pgfqpoint{3.334953in}{0.826673in}}%
\pgfpathlineto{\pgfqpoint{3.359945in}{0.826673in}}%
\pgfpathlineto{\pgfqpoint{3.384937in}{0.826673in}}%
\pgfpathlineto{\pgfqpoint{3.395057in}{0.826673in}}%
\pgfusepath{stroke}%
\end{pgfscope}%
\begin{pgfscope}%
\pgfpathrectangle{\pgfqpoint{2.161373in}{0.383578in}}{\pgfqpoint{1.223684in}{1.004348in}}%
\pgfusepath{clip}%
\pgfsetbuttcap%
\pgfsetroundjoin%
\pgfsetlinewidth{0.803000pt}%
\definecolor{currentstroke}{rgb}{0.686275,0.352941,0.313725}%
\pgfsetstrokecolor{currentstroke}%
\pgfsetdash{{2.960000pt}{1.280000pt}}{0.000000pt}%
\pgfpathmoveto{\pgfqpoint{2.210320in}{0.383578in}}%
\pgfpathlineto{\pgfqpoint{2.210320in}{1.387926in}}%
\pgfusepath{stroke}%
\end{pgfscope}%
\begin{pgfscope}%
\pgfpathrectangle{\pgfqpoint{2.161373in}{0.383578in}}{\pgfqpoint{1.223684in}{1.004348in}}%
\pgfusepath{clip}%
\pgfsetbuttcap%
\pgfsetroundjoin%
\pgfsetlinewidth{0.803000pt}%
\definecolor{currentstroke}{rgb}{0.686275,0.352941,0.313725}%
\pgfsetstrokecolor{currentstroke}%
\pgfsetdash{{2.960000pt}{1.280000pt}}{0.000000pt}%
\pgfpathmoveto{\pgfqpoint{2.993478in}{0.383578in}}%
\pgfpathlineto{\pgfqpoint{2.993478in}{1.387926in}}%
\pgfusepath{stroke}%
\end{pgfscope}%
\begin{pgfscope}%
\pgfsetrectcap%
\pgfsetmiterjoin%
\pgfsetlinewidth{0.501875pt}%
\definecolor{currentstroke}{rgb}{0.317647,0.317647,0.317647}%
\pgfsetstrokecolor{currentstroke}%
\pgfsetdash{}{0pt}%
\pgfpathmoveto{\pgfqpoint{2.161373in}{0.383578in}}%
\pgfpathlineto{\pgfqpoint{2.161373in}{1.387926in}}%
\pgfusepath{stroke}%
\end{pgfscope}%
\begin{pgfscope}%
\pgfsetrectcap%
\pgfsetmiterjoin%
\pgfsetlinewidth{0.501875pt}%
\definecolor{currentstroke}{rgb}{0.317647,0.317647,0.317647}%
\pgfsetstrokecolor{currentstroke}%
\pgfsetdash{}{0pt}%
\pgfpathmoveto{\pgfqpoint{2.161373in}{0.383578in}}%
\pgfpathlineto{\pgfqpoint{3.385057in}{0.383578in}}%
\pgfusepath{stroke}%
\end{pgfscope}%
\begin{pgfscope}%
\pgfsetrectcap%
\pgfsetroundjoin%
\pgfsetlinewidth{0.803000pt}%
\definecolor{currentstroke}{rgb}{0.333333,0.333333,0.333333}%
\pgfsetstrokecolor{currentstroke}%
\pgfsetdash{}{0pt}%
\pgfpathmoveto{\pgfqpoint{3.184304in}{1.327765in}}%
\pgfpathlineto{\pgfqpoint{3.258349in}{1.327765in}}%
\pgfusepath{stroke}%
\end{pgfscope}%
\begin{pgfscope}%
\definecolor{textcolor}{rgb}{0.000000,0.000000,0.000000}%
\pgfsetstrokecolor{textcolor}%
\pgfsetfillcolor{textcolor}%
\pgftext[x=3.304626in,y=1.295370in,left,base]{\color{textcolor}\rmfamily\fontsize{6.664000}{7.996800}\selectfont 0}%
\end{pgfscope}%
\begin{pgfscope}%
\pgfsetrectcap%
\pgfsetroundjoin%
\pgfsetlinewidth{0.803000pt}%
\definecolor{currentstroke}{rgb}{0.686275,0.352941,0.313725}%
\pgfsetstrokecolor{currentstroke}%
\pgfsetdash{}{0pt}%
\pgfpathmoveto{\pgfqpoint{3.184304in}{1.207998in}}%
\pgfpathlineto{\pgfqpoint{3.258349in}{1.207998in}}%
\pgfusepath{stroke}%
\end{pgfscope}%
\begin{pgfscope}%
\definecolor{textcolor}{rgb}{0.000000,0.000000,0.000000}%
\pgfsetstrokecolor{textcolor}%
\pgfsetfillcolor{textcolor}%
\pgftext[x=3.304626in,y=1.175604in,left,base]{\color{textcolor}\rmfamily\fontsize{6.664000}{7.996800}\selectfont 1}%
\end{pgfscope}%
\begin{pgfscope}%
\pgfsetbuttcap%
\pgfsetmiterjoin%
\pgfsetlinewidth{0.000000pt}%
\definecolor{currentstroke}{rgb}{0.000000,0.000000,0.000000}%
\pgfsetstrokecolor{currentstroke}%
\pgfsetstrokeopacity{0.000000}%
\pgfsetdash{}{0pt}%
\pgfpathmoveto{\pgfqpoint{3.874531in}{0.383578in}}%
\pgfpathlineto{\pgfqpoint{5.098215in}{0.383578in}}%
\pgfpathlineto{\pgfqpoint{5.098215in}{1.387926in}}%
\pgfpathlineto{\pgfqpoint{3.874531in}{1.387926in}}%
\pgfpathclose%
\pgfusepath{}%
\end{pgfscope}%
\begin{pgfscope}%
\pgfsetbuttcap%
\pgfsetroundjoin%
\definecolor{currentfill}{rgb}{0.317647,0.317647,0.317647}%
\pgfsetfillcolor{currentfill}%
\pgfsetlinewidth{0.501875pt}%
\definecolor{currentstroke}{rgb}{0.317647,0.317647,0.317647}%
\pgfsetstrokecolor{currentstroke}%
\pgfsetdash{}{0pt}%
\pgfsys@defobject{currentmarker}{\pgfqpoint{0.000000in}{-0.020833in}}{\pgfqpoint{0.000000in}{0.000000in}}{%
\pgfpathmoveto{\pgfqpoint{0.000000in}{0.000000in}}%
\pgfpathlineto{\pgfqpoint{0.000000in}{-0.020833in}}%
\pgfusepath{stroke,fill}%
}%
\begin{pgfscope}%
\pgfsys@transformshift{3.923478in}{0.383578in}%
\pgfsys@useobject{currentmarker}{}%
\end{pgfscope}%
\end{pgfscope}%
\begin{pgfscope}%
\definecolor{textcolor}{rgb}{0.317647,0.317647,0.317647}%
\pgfsetstrokecolor{textcolor}%
\pgfsetfillcolor{textcolor}%
\pgftext[x=3.923478in,y=0.334967in,,top]{\color{textcolor}\rmfamily\fontsize{6.664000}{7.996800}\selectfont \(\displaystyle 0\)}%
\end{pgfscope}%
\begin{pgfscope}%
\pgfsetbuttcap%
\pgfsetroundjoin%
\definecolor{currentfill}{rgb}{0.317647,0.317647,0.317647}%
\pgfsetfillcolor{currentfill}%
\pgfsetlinewidth{0.501875pt}%
\definecolor{currentstroke}{rgb}{0.317647,0.317647,0.317647}%
\pgfsetstrokecolor{currentstroke}%
\pgfsetdash{}{0pt}%
\pgfsys@defobject{currentmarker}{\pgfqpoint{0.000000in}{-0.020833in}}{\pgfqpoint{0.000000in}{0.000000in}}{%
\pgfpathmoveto{\pgfqpoint{0.000000in}{0.000000in}}%
\pgfpathlineto{\pgfqpoint{0.000000in}{-0.020833in}}%
\pgfusepath{stroke,fill}%
}%
\begin{pgfscope}%
\pgfsys@transformshift{4.412952in}{0.383578in}%
\pgfsys@useobject{currentmarker}{}%
\end{pgfscope}%
\end{pgfscope}%
\begin{pgfscope}%
\definecolor{textcolor}{rgb}{0.317647,0.317647,0.317647}%
\pgfsetstrokecolor{textcolor}%
\pgfsetfillcolor{textcolor}%
\pgftext[x=4.412952in,y=0.334967in,,top]{\color{textcolor}\rmfamily\fontsize{6.664000}{7.996800}\selectfont \(\displaystyle 50\)}%
\end{pgfscope}%
\begin{pgfscope}%
\pgfsetbuttcap%
\pgfsetroundjoin%
\definecolor{currentfill}{rgb}{0.317647,0.317647,0.317647}%
\pgfsetfillcolor{currentfill}%
\pgfsetlinewidth{0.501875pt}%
\definecolor{currentstroke}{rgb}{0.317647,0.317647,0.317647}%
\pgfsetstrokecolor{currentstroke}%
\pgfsetdash{}{0pt}%
\pgfsys@defobject{currentmarker}{\pgfqpoint{0.000000in}{-0.020833in}}{\pgfqpoint{0.000000in}{0.000000in}}{%
\pgfpathmoveto{\pgfqpoint{0.000000in}{0.000000in}}%
\pgfpathlineto{\pgfqpoint{0.000000in}{-0.020833in}}%
\pgfusepath{stroke,fill}%
}%
\begin{pgfscope}%
\pgfsys@transformshift{4.902426in}{0.383578in}%
\pgfsys@useobject{currentmarker}{}%
\end{pgfscope}%
\end{pgfscope}%
\begin{pgfscope}%
\definecolor{textcolor}{rgb}{0.317647,0.317647,0.317647}%
\pgfsetstrokecolor{textcolor}%
\pgfsetfillcolor{textcolor}%
\pgftext[x=4.902426in,y=0.334967in,,top]{\color{textcolor}\rmfamily\fontsize{6.664000}{7.996800}\selectfont \(\displaystyle 100\)}%
\end{pgfscope}%
\begin{pgfscope}%
\definecolor{textcolor}{rgb}{0.317647,0.317647,0.317647}%
\pgfsetstrokecolor{textcolor}%
\pgfsetfillcolor{textcolor}%
\pgftext[x=4.486373in,y=0.197222in,,top]{\color{textcolor}\rmfamily\fontsize{6.664000}{7.996800}\selectfont time\(\displaystyle \;(\si{\s})\)}%
\end{pgfscope}%
\begin{pgfscope}%
\pgfsetbuttcap%
\pgfsetroundjoin%
\definecolor{currentfill}{rgb}{0.317647,0.317647,0.317647}%
\pgfsetfillcolor{currentfill}%
\pgfsetlinewidth{0.501875pt}%
\definecolor{currentstroke}{rgb}{0.317647,0.317647,0.317647}%
\pgfsetstrokecolor{currentstroke}%
\pgfsetdash{}{0pt}%
\pgfsys@defobject{currentmarker}{\pgfqpoint{-0.020833in}{0.000000in}}{\pgfqpoint{0.000000in}{0.000000in}}{%
\pgfpathmoveto{\pgfqpoint{0.000000in}{0.000000in}}%
\pgfpathlineto{\pgfqpoint{-0.020833in}{0.000000in}}%
\pgfusepath{stroke,fill}%
}%
\begin{pgfscope}%
\pgfsys@transformshift{3.874531in}{0.576545in}%
\pgfsys@useobject{currentmarker}{}%
\end{pgfscope}%
\end{pgfscope}%
\begin{pgfscope}%
\definecolor{textcolor}{rgb}{0.317647,0.317647,0.317647}%
\pgfsetstrokecolor{textcolor}%
\pgfsetfillcolor{textcolor}%
\pgftext[x=3.553928in,y=0.544429in,left,base]{\color{textcolor}\rmfamily\fontsize{6.664000}{7.996800}\selectfont \(\displaystyle -0.05\)}%
\end{pgfscope}%
\begin{pgfscope}%
\pgfsetbuttcap%
\pgfsetroundjoin%
\definecolor{currentfill}{rgb}{0.317647,0.317647,0.317647}%
\pgfsetfillcolor{currentfill}%
\pgfsetlinewidth{0.501875pt}%
\definecolor{currentstroke}{rgb}{0.317647,0.317647,0.317647}%
\pgfsetstrokecolor{currentstroke}%
\pgfsetdash{}{0pt}%
\pgfsys@defobject{currentmarker}{\pgfqpoint{-0.020833in}{0.000000in}}{\pgfqpoint{0.000000in}{0.000000in}}{%
\pgfpathmoveto{\pgfqpoint{0.000000in}{0.000000in}}%
\pgfpathlineto{\pgfqpoint{-0.020833in}{0.000000in}}%
\pgfusepath{stroke,fill}%
}%
\begin{pgfscope}%
\pgfsys@transformshift{3.874531in}{0.885752in}%
\pgfsys@useobject{currentmarker}{}%
\end{pgfscope}%
\end{pgfscope}%
\begin{pgfscope}%
\definecolor{textcolor}{rgb}{0.317647,0.317647,0.317647}%
\pgfsetstrokecolor{textcolor}%
\pgfsetfillcolor{textcolor}%
\pgftext[x=3.640734in,y=0.853635in,left,base]{\color{textcolor}\rmfamily\fontsize{6.664000}{7.996800}\selectfont \(\displaystyle 0.00\)}%
\end{pgfscope}%
\begin{pgfscope}%
\pgfsetbuttcap%
\pgfsetroundjoin%
\definecolor{currentfill}{rgb}{0.317647,0.317647,0.317647}%
\pgfsetfillcolor{currentfill}%
\pgfsetlinewidth{0.501875pt}%
\definecolor{currentstroke}{rgb}{0.317647,0.317647,0.317647}%
\pgfsetstrokecolor{currentstroke}%
\pgfsetdash{}{0pt}%
\pgfsys@defobject{currentmarker}{\pgfqpoint{-0.020833in}{0.000000in}}{\pgfqpoint{0.000000in}{0.000000in}}{%
\pgfpathmoveto{\pgfqpoint{0.000000in}{0.000000in}}%
\pgfpathlineto{\pgfqpoint{-0.020833in}{0.000000in}}%
\pgfusepath{stroke,fill}%
}%
\begin{pgfscope}%
\pgfsys@transformshift{3.874531in}{1.194959in}%
\pgfsys@useobject{currentmarker}{}%
\end{pgfscope}%
\end{pgfscope}%
\begin{pgfscope}%
\definecolor{textcolor}{rgb}{0.317647,0.317647,0.317647}%
\pgfsetstrokecolor{textcolor}%
\pgfsetfillcolor{textcolor}%
\pgftext[x=3.640734in,y=1.162842in,left,base]{\color{textcolor}\rmfamily\fontsize{6.664000}{7.996800}\selectfont \(\displaystyle 0.05\)}%
\end{pgfscope}%
\begin{pgfscope}%
\definecolor{textcolor}{rgb}{0.317647,0.317647,0.317647}%
\pgfsetstrokecolor{textcolor}%
\pgfsetfillcolor{textcolor}%
\pgftext[x=3.498373in,y=0.885752in,,bottom,rotate=90.000000]{\color{textcolor}\rmfamily\fontsize{6.664000}{7.996800}\selectfont \(\displaystyle e^{(o)}\)}%
\end{pgfscope}%
\begin{pgfscope}%
\pgfpathrectangle{\pgfqpoint{3.874531in}{0.383578in}}{\pgfqpoint{1.223684in}{1.004348in}}%
\pgfusepath{clip}%
\pgfsetrectcap%
\pgfsetroundjoin%
\pgfsetlinewidth{0.803000pt}%
\definecolor{currentstroke}{rgb}{0.333333,0.333333,0.333333}%
\pgfsetstrokecolor{currentstroke}%
\pgfsetdash{}{0pt}%
\pgfpathmoveto{\pgfqpoint{3.923478in}{0.885752in}}%
\pgfpathlineto{\pgfqpoint{3.948470in}{0.885752in}}%
\pgfpathlineto{\pgfqpoint{3.973462in}{0.885752in}}%
\pgfpathlineto{\pgfqpoint{3.998454in}{0.885752in}}%
\pgfpathlineto{\pgfqpoint{4.023446in}{0.885752in}}%
\pgfpathlineto{\pgfqpoint{4.048437in}{0.885752in}}%
\pgfpathlineto{\pgfqpoint{4.073429in}{0.885752in}}%
\pgfpathlineto{\pgfqpoint{4.098421in}{0.885752in}}%
\pgfpathlineto{\pgfqpoint{4.123413in}{0.885752in}}%
\pgfpathlineto{\pgfqpoint{4.148405in}{0.682550in}}%
\pgfpathlineto{\pgfqpoint{4.173397in}{0.553017in}}%
\pgfpathlineto{\pgfqpoint{4.198388in}{0.477122in}}%
\pgfpathlineto{\pgfqpoint{4.223380in}{0.439674in}}%
\pgfpathlineto{\pgfqpoint{4.248372in}{0.429230in}}%
\pgfpathlineto{\pgfqpoint{4.273364in}{0.437230in}}%
\pgfpathlineto{\pgfqpoint{4.298356in}{0.457330in}}%
\pgfpathlineto{\pgfqpoint{4.323348in}{0.484881in}}%
\pgfpathlineto{\pgfqpoint{4.348339in}{0.516521in}}%
\pgfpathlineto{\pgfqpoint{4.373331in}{0.549862in}}%
\pgfpathlineto{\pgfqpoint{4.398323in}{0.583248in}}%
\pgfpathlineto{\pgfqpoint{4.423315in}{0.615567in}}%
\pgfpathlineto{\pgfqpoint{4.448307in}{0.646110in}}%
\pgfpathlineto{\pgfqpoint{4.473299in}{0.674457in}}%
\pgfpathlineto{\pgfqpoint{4.498290in}{0.700401in}}%
\pgfpathlineto{\pgfqpoint{4.523282in}{0.723883in}}%
\pgfpathlineto{\pgfqpoint{4.548274in}{0.744942in}}%
\pgfpathlineto{\pgfqpoint{4.573266in}{0.763685in}}%
\pgfpathlineto{\pgfqpoint{4.598258in}{0.780260in}}%
\pgfpathlineto{\pgfqpoint{4.623250in}{0.794836in}}%
\pgfpathlineto{\pgfqpoint{4.648241in}{0.807595in}}%
\pgfpathlineto{\pgfqpoint{4.673233in}{0.818715in}}%
\pgfpathlineto{\pgfqpoint{4.698225in}{0.828372in}}%
\pgfpathlineto{\pgfqpoint{4.723217in}{0.836731in}}%
\pgfpathlineto{\pgfqpoint{4.748209in}{0.843945in}}%
\pgfpathlineto{\pgfqpoint{4.773201in}{0.850154in}}%
\pgfpathlineto{\pgfqpoint{4.798192in}{0.855486in}}%
\pgfpathlineto{\pgfqpoint{4.823184in}{0.860054in}}%
\pgfpathlineto{\pgfqpoint{4.848176in}{0.863961in}}%
\pgfpathlineto{\pgfqpoint{4.873168in}{0.867296in}}%
\pgfpathlineto{\pgfqpoint{4.898160in}{0.870138in}}%
\pgfpathlineto{\pgfqpoint{4.923152in}{0.872556in}}%
\pgfpathlineto{\pgfqpoint{4.948143in}{0.874610in}}%
\pgfpathlineto{\pgfqpoint{4.973135in}{0.876353in}}%
\pgfpathlineto{\pgfqpoint{4.998127in}{0.877831in}}%
\pgfpathlineto{\pgfqpoint{5.023119in}{0.879081in}}%
\pgfpathlineto{\pgfqpoint{5.048111in}{0.880139in}}%
\pgfpathlineto{\pgfqpoint{5.073103in}{0.881032in}}%
\pgfpathlineto{\pgfqpoint{5.098095in}{0.881786in}}%
\pgfpathlineto{\pgfqpoint{5.108215in}{0.882043in}}%
\pgfusepath{stroke}%
\end{pgfscope}%
\begin{pgfscope}%
\pgfpathrectangle{\pgfqpoint{3.874531in}{0.383578in}}{\pgfqpoint{1.223684in}{1.004348in}}%
\pgfusepath{clip}%
\pgfsetrectcap%
\pgfsetroundjoin%
\pgfsetlinewidth{0.803000pt}%
\definecolor{currentstroke}{rgb}{0.686275,0.352941,0.313725}%
\pgfsetstrokecolor{currentstroke}%
\pgfsetdash{}{0pt}%
\pgfpathmoveto{\pgfqpoint{3.923478in}{0.885752in}}%
\pgfpathlineto{\pgfqpoint{3.948470in}{0.885752in}}%
\pgfpathlineto{\pgfqpoint{3.973462in}{0.885752in}}%
\pgfpathlineto{\pgfqpoint{3.998454in}{0.885752in}}%
\pgfpathlineto{\pgfqpoint{4.023446in}{0.885752in}}%
\pgfpathlineto{\pgfqpoint{4.048437in}{0.885752in}}%
\pgfpathlineto{\pgfqpoint{4.073429in}{0.885752in}}%
\pgfpathlineto{\pgfqpoint{4.098421in}{0.885752in}}%
\pgfpathlineto{\pgfqpoint{4.123413in}{0.885752in}}%
\pgfpathlineto{\pgfqpoint{4.148405in}{0.885752in}}%
\pgfpathlineto{\pgfqpoint{4.173397in}{0.885752in}}%
\pgfpathlineto{\pgfqpoint{4.198388in}{0.885752in}}%
\pgfpathlineto{\pgfqpoint{4.223380in}{0.885752in}}%
\pgfpathlineto{\pgfqpoint{4.248372in}{0.885752in}}%
\pgfpathlineto{\pgfqpoint{4.273364in}{1.088954in}}%
\pgfpathlineto{\pgfqpoint{4.298356in}{1.218487in}}%
\pgfpathlineto{\pgfqpoint{4.323348in}{1.294382in}}%
\pgfpathlineto{\pgfqpoint{4.348339in}{1.331829in}}%
\pgfpathlineto{\pgfqpoint{4.373331in}{1.342274in}}%
\pgfpathlineto{\pgfqpoint{4.398323in}{1.334274in}}%
\pgfpathlineto{\pgfqpoint{4.423315in}{1.314174in}}%
\pgfpathlineto{\pgfqpoint{4.448307in}{1.286623in}}%
\pgfpathlineto{\pgfqpoint{4.473299in}{1.254983in}}%
\pgfpathlineto{\pgfqpoint{4.498290in}{1.221642in}}%
\pgfpathlineto{\pgfqpoint{4.523282in}{1.188256in}}%
\pgfpathlineto{\pgfqpoint{4.548274in}{1.155936in}}%
\pgfpathlineto{\pgfqpoint{4.573266in}{1.125394in}}%
\pgfpathlineto{\pgfqpoint{4.598258in}{1.097047in}}%
\pgfpathlineto{\pgfqpoint{4.623250in}{1.071102in}}%
\pgfpathlineto{\pgfqpoint{4.648241in}{1.047621in}}%
\pgfpathlineto{\pgfqpoint{4.673233in}{1.026562in}}%
\pgfpathlineto{\pgfqpoint{4.698225in}{1.007819in}}%
\pgfpathlineto{\pgfqpoint{4.723217in}{0.991244in}}%
\pgfpathlineto{\pgfqpoint{4.748209in}{0.976667in}}%
\pgfpathlineto{\pgfqpoint{4.773201in}{0.963909in}}%
\pgfpathlineto{\pgfqpoint{4.798192in}{0.952789in}}%
\pgfpathlineto{\pgfqpoint{4.823184in}{0.943132in}}%
\pgfpathlineto{\pgfqpoint{4.848176in}{0.934773in}}%
\pgfpathlineto{\pgfqpoint{4.873168in}{0.927559in}}%
\pgfpathlineto{\pgfqpoint{4.898160in}{0.921350in}}%
\pgfpathlineto{\pgfqpoint{4.923152in}{0.916018in}}%
\pgfpathlineto{\pgfqpoint{4.948143in}{0.911450in}}%
\pgfpathlineto{\pgfqpoint{4.973135in}{0.907543in}}%
\pgfpathlineto{\pgfqpoint{4.998127in}{0.904208in}}%
\pgfpathlineto{\pgfqpoint{5.023119in}{0.901366in}}%
\pgfpathlineto{\pgfqpoint{5.048111in}{0.898948in}}%
\pgfpathlineto{\pgfqpoint{5.073103in}{0.896894in}}%
\pgfpathlineto{\pgfqpoint{5.098095in}{0.895151in}}%
\pgfpathlineto{\pgfqpoint{5.108215in}{0.894552in}}%
\pgfusepath{stroke}%
\end{pgfscope}%
\begin{pgfscope}%
\pgfpathrectangle{\pgfqpoint{3.874531in}{0.383578in}}{\pgfqpoint{1.223684in}{1.004348in}}%
\pgfusepath{clip}%
\pgfsetbuttcap%
\pgfsetroundjoin%
\pgfsetlinewidth{0.803000pt}%
\definecolor{currentstroke}{rgb}{0.686275,0.352941,0.313725}%
\pgfsetstrokecolor{currentstroke}%
\pgfsetdash{{2.960000pt}{1.280000pt}}{0.000000pt}%
\pgfpathmoveto{\pgfqpoint{3.923478in}{0.383578in}}%
\pgfpathlineto{\pgfqpoint{3.923478in}{1.387926in}}%
\pgfusepath{stroke}%
\end{pgfscope}%
\begin{pgfscope}%
\pgfpathrectangle{\pgfqpoint{3.874531in}{0.383578in}}{\pgfqpoint{1.223684in}{1.004348in}}%
\pgfusepath{clip}%
\pgfsetbuttcap%
\pgfsetroundjoin%
\pgfsetlinewidth{0.803000pt}%
\definecolor{currentstroke}{rgb}{0.686275,0.352941,0.313725}%
\pgfsetstrokecolor{currentstroke}%
\pgfsetdash{{2.960000pt}{1.280000pt}}{0.000000pt}%
\pgfpathmoveto{\pgfqpoint{4.706636in}{0.383578in}}%
\pgfpathlineto{\pgfqpoint{4.706636in}{1.387926in}}%
\pgfusepath{stroke}%
\end{pgfscope}%
\begin{pgfscope}%
\pgfsetrectcap%
\pgfsetmiterjoin%
\pgfsetlinewidth{0.501875pt}%
\definecolor{currentstroke}{rgb}{0.317647,0.317647,0.317647}%
\pgfsetstrokecolor{currentstroke}%
\pgfsetdash{}{0pt}%
\pgfpathmoveto{\pgfqpoint{3.874531in}{0.383578in}}%
\pgfpathlineto{\pgfqpoint{3.874531in}{1.387926in}}%
\pgfusepath{stroke}%
\end{pgfscope}%
\begin{pgfscope}%
\pgfsetrectcap%
\pgfsetmiterjoin%
\pgfsetlinewidth{0.501875pt}%
\definecolor{currentstroke}{rgb}{0.317647,0.317647,0.317647}%
\pgfsetstrokecolor{currentstroke}%
\pgfsetdash{}{0pt}%
\pgfpathmoveto{\pgfqpoint{3.874531in}{0.383578in}}%
\pgfpathlineto{\pgfqpoint{5.098215in}{0.383578in}}%
\pgfusepath{stroke}%
\end{pgfscope}%
\begin{pgfscope}%
\pgfsetrectcap%
\pgfsetroundjoin%
\pgfsetlinewidth{0.803000pt}%
\definecolor{currentstroke}{rgb}{0.333333,0.333333,0.333333}%
\pgfsetstrokecolor{currentstroke}%
\pgfsetdash{}{0pt}%
\pgfpathmoveto{\pgfqpoint{4.897462in}{1.327765in}}%
\pgfpathlineto{\pgfqpoint{4.971507in}{1.327765in}}%
\pgfusepath{stroke}%
\end{pgfscope}%
\begin{pgfscope}%
\definecolor{textcolor}{rgb}{0.000000,0.000000,0.000000}%
\pgfsetstrokecolor{textcolor}%
\pgfsetfillcolor{textcolor}%
\pgftext[x=5.017784in,y=1.295370in,left,base]{\color{textcolor}\rmfamily\fontsize{6.664000}{7.996800}\selectfont 0}%
\end{pgfscope}%
\begin{pgfscope}%
\pgfsetrectcap%
\pgfsetroundjoin%
\pgfsetlinewidth{0.803000pt}%
\definecolor{currentstroke}{rgb}{0.686275,0.352941,0.313725}%
\pgfsetstrokecolor{currentstroke}%
\pgfsetdash{}{0pt}%
\pgfpathmoveto{\pgfqpoint{4.897462in}{1.207998in}}%
\pgfpathlineto{\pgfqpoint{4.971507in}{1.207998in}}%
\pgfusepath{stroke}%
\end{pgfscope}%
\begin{pgfscope}%
\definecolor{textcolor}{rgb}{0.000000,0.000000,0.000000}%
\pgfsetstrokecolor{textcolor}%
\pgfsetfillcolor{textcolor}%
\pgftext[x=5.017784in,y=1.175604in,left,base]{\color{textcolor}\rmfamily\fontsize{6.664000}{7.996800}\selectfont 1}%
\end{pgfscope}%
\end{pgfpicture}%
\makeatother%
\endgroup%

	\caption[Monitoring of the analog traces measured on the \gls{hx}.]{Monitoring of the analog traces measured on the \gls{hx}. Each column represents an important observable for the backward pass. The hidden and output layer are discriminated row-wise. The first column, contains the information of the presynaptic spike train spike train $S_j^{(l)}$, the second column shows the evolution of the membrane potential \gls{v_mem} with respect to the total synaptic input generated by the incoming spikes and in the third column the adapted von Rossum error $e^{(l)}$ is displayed.}
	\label{debugplot}
\end{figure}

...
\begin{figure}
	\centering
	%% Creator: Matplotlib, PGF backend
%%
%% To include the figure in your LaTeX document, write
%%   \input{<filename>.pgf}
%%
%% Make sure the required packages are loaded in your preamble
%%   \usepackage{pgf}
%%
%% Figures using additional raster images can only be included by \input if
%% they are in the same directory as the main LaTeX file. For loading figures
%% from other directories you can use the `import` package
%%   \usepackage{import}
%% and then include the figures with
%%   \import{<path to file>}{<filename>.pgf}
%%
%% Matplotlib used the following preamble
%%   \usepackage{amsmath} \usepackage{pifont} \usepackage{xcolor} \definecolor{green}{HTML}{467821} \definecolor{red}{HTML}{CF4457} \usepackage[detect-all]{siunitx}
%%   \usepackage{fontspec}
%%
\begingroup%
\makeatletter%
\begin{pgfpicture}%
\pgfpathrectangle{\pgfpointorigin}{\pgfqpoint{6.157463in}{2.023578in}}%
\pgfusepath{use as bounding box, clip}%
\begin{pgfscope}%
\pgfsetbuttcap%
\pgfsetmiterjoin%
\pgfsetlinewidth{0.000000pt}%
\definecolor{currentstroke}{rgb}{0.000000,0.000000,0.000000}%
\pgfsetstrokecolor{currentstroke}%
\pgfsetstrokeopacity{0.000000}%
\pgfsetdash{}{0pt}%
\pgfpathmoveto{\pgfqpoint{-0.000000in}{0.000000in}}%
\pgfpathlineto{\pgfqpoint{6.157463in}{0.000000in}}%
\pgfpathlineto{\pgfqpoint{6.157463in}{2.023578in}}%
\pgfpathlineto{\pgfqpoint{-0.000000in}{2.023578in}}%
\pgfpathclose%
\pgfusepath{}%
\end{pgfscope}%
\begin{pgfscope}%
\pgfsetbuttcap%
\pgfsetmiterjoin%
\pgfsetlinewidth{0.000000pt}%
\definecolor{currentstroke}{rgb}{0.000000,0.000000,0.000000}%
\pgfsetstrokecolor{currentstroke}%
\pgfsetstrokeopacity{0.000000}%
\pgfsetdash{}{0pt}%
\pgfpathmoveto{\pgfqpoint{0.632463in}{0.383578in}}%
\pgfpathlineto{\pgfqpoint{1.988713in}{0.383578in}}%
\pgfpathlineto{\pgfqpoint{1.988713in}{1.923578in}}%
\pgfpathlineto{\pgfqpoint{0.632463in}{1.923578in}}%
\pgfpathclose%
\pgfusepath{}%
\end{pgfscope}%
\begin{pgfscope}%
\pgfsetbuttcap%
\pgfsetroundjoin%
\definecolor{currentfill}{rgb}{0.317647,0.317647,0.317647}%
\pgfsetfillcolor{currentfill}%
\pgfsetlinewidth{0.501875pt}%
\definecolor{currentstroke}{rgb}{0.317647,0.317647,0.317647}%
\pgfsetstrokecolor{currentstroke}%
\pgfsetdash{}{0pt}%
\pgfsys@defobject{currentmarker}{\pgfqpoint{0.000000in}{-0.020833in}}{\pgfqpoint{0.000000in}{0.000000in}}{%
\pgfpathmoveto{\pgfqpoint{0.000000in}{0.000000in}}%
\pgfpathlineto{\pgfqpoint{0.000000in}{-0.020833in}}%
\pgfusepath{stroke,fill}%
}%
\begin{pgfscope}%
\pgfsys@transformshift{0.686713in}{0.383578in}%
\pgfsys@useobject{currentmarker}{}%
\end{pgfscope}%
\end{pgfscope}%
\begin{pgfscope}%
\definecolor{textcolor}{rgb}{0.317647,0.317647,0.317647}%
\pgfsetstrokecolor{textcolor}%
\pgfsetfillcolor{textcolor}%
\pgftext[x=0.686713in,y=0.334967in,,top]{\color{textcolor}\rmfamily\fontsize{6.664000}{7.996800}\selectfont \(\displaystyle 0\)}%
\end{pgfscope}%
\begin{pgfscope}%
\pgfsetbuttcap%
\pgfsetroundjoin%
\definecolor{currentfill}{rgb}{0.317647,0.317647,0.317647}%
\pgfsetfillcolor{currentfill}%
\pgfsetlinewidth{0.501875pt}%
\definecolor{currentstroke}{rgb}{0.317647,0.317647,0.317647}%
\pgfsetstrokecolor{currentstroke}%
\pgfsetdash{}{0pt}%
\pgfsys@defobject{currentmarker}{\pgfqpoint{0.000000in}{-0.020833in}}{\pgfqpoint{0.000000in}{0.000000in}}{%
\pgfpathmoveto{\pgfqpoint{0.000000in}{0.000000in}}%
\pgfpathlineto{\pgfqpoint{0.000000in}{-0.020833in}}%
\pgfusepath{stroke,fill}%
}%
\begin{pgfscope}%
\pgfsys@transformshift{1.229213in}{0.383578in}%
\pgfsys@useobject{currentmarker}{}%
\end{pgfscope}%
\end{pgfscope}%
\begin{pgfscope}%
\definecolor{textcolor}{rgb}{0.317647,0.317647,0.317647}%
\pgfsetstrokecolor{textcolor}%
\pgfsetfillcolor{textcolor}%
\pgftext[x=1.229213in,y=0.334967in,,top]{\color{textcolor}\rmfamily\fontsize{6.664000}{7.996800}\selectfont \(\displaystyle 50\)}%
\end{pgfscope}%
\begin{pgfscope}%
\pgfsetbuttcap%
\pgfsetroundjoin%
\definecolor{currentfill}{rgb}{0.317647,0.317647,0.317647}%
\pgfsetfillcolor{currentfill}%
\pgfsetlinewidth{0.501875pt}%
\definecolor{currentstroke}{rgb}{0.317647,0.317647,0.317647}%
\pgfsetstrokecolor{currentstroke}%
\pgfsetdash{}{0pt}%
\pgfsys@defobject{currentmarker}{\pgfqpoint{0.000000in}{-0.020833in}}{\pgfqpoint{0.000000in}{0.000000in}}{%
\pgfpathmoveto{\pgfqpoint{0.000000in}{0.000000in}}%
\pgfpathlineto{\pgfqpoint{0.000000in}{-0.020833in}}%
\pgfusepath{stroke,fill}%
}%
\begin{pgfscope}%
\pgfsys@transformshift{1.771713in}{0.383578in}%
\pgfsys@useobject{currentmarker}{}%
\end{pgfscope}%
\end{pgfscope}%
\begin{pgfscope}%
\definecolor{textcolor}{rgb}{0.317647,0.317647,0.317647}%
\pgfsetstrokecolor{textcolor}%
\pgfsetfillcolor{textcolor}%
\pgftext[x=1.771713in,y=0.334967in,,top]{\color{textcolor}\rmfamily\fontsize{6.664000}{7.996800}\selectfont \(\displaystyle 100\)}%
\end{pgfscope}%
\begin{pgfscope}%
\definecolor{textcolor}{rgb}{0.317647,0.317647,0.317647}%
\pgfsetstrokecolor{textcolor}%
\pgfsetfillcolor{textcolor}%
\pgftext[x=1.310588in,y=0.197222in,,top]{\color{textcolor}\rmfamily\fontsize{6.664000}{7.996800}\selectfont time\(\displaystyle \;(\si{\micro \s})\)}%
\end{pgfscope}%
\begin{pgfscope}%
\pgfsetbuttcap%
\pgfsetroundjoin%
\definecolor{currentfill}{rgb}{0.317647,0.317647,0.317647}%
\pgfsetfillcolor{currentfill}%
\pgfsetlinewidth{0.501875pt}%
\definecolor{currentstroke}{rgb}{0.317647,0.317647,0.317647}%
\pgfsetstrokecolor{currentstroke}%
\pgfsetdash{}{0pt}%
\pgfsys@defobject{currentmarker}{\pgfqpoint{-0.020833in}{0.000000in}}{\pgfqpoint{0.000000in}{0.000000in}}{%
\pgfpathmoveto{\pgfqpoint{0.000000in}{0.000000in}}%
\pgfpathlineto{\pgfqpoint{-0.020833in}{0.000000in}}%
\pgfusepath{stroke,fill}%
}%
\begin{pgfscope}%
\pgfsys@transformshift{0.632463in}{0.442403in}%
\pgfsys@useobject{currentmarker}{}%
\end{pgfscope}%
\end{pgfscope}%
\begin{pgfscope}%
\definecolor{textcolor}{rgb}{0.317647,0.317647,0.317647}%
\pgfsetstrokecolor{textcolor}%
\pgfsetfillcolor{textcolor}%
\pgftext[x=0.256497in,y=0.410286in,left,base]{\color{textcolor}\rmfamily\fontsize{6.664000}{7.996800}\selectfont \(\displaystyle -0.075\)}%
\end{pgfscope}%
\begin{pgfscope}%
\pgfsetbuttcap%
\pgfsetroundjoin%
\definecolor{currentfill}{rgb}{0.317647,0.317647,0.317647}%
\pgfsetfillcolor{currentfill}%
\pgfsetlinewidth{0.501875pt}%
\definecolor{currentstroke}{rgb}{0.317647,0.317647,0.317647}%
\pgfsetstrokecolor{currentstroke}%
\pgfsetdash{}{0pt}%
\pgfsys@defobject{currentmarker}{\pgfqpoint{-0.020833in}{0.000000in}}{\pgfqpoint{0.000000in}{0.000000in}}{%
\pgfpathmoveto{\pgfqpoint{0.000000in}{0.000000in}}%
\pgfpathlineto{\pgfqpoint{-0.020833in}{0.000000in}}%
\pgfusepath{stroke,fill}%
}%
\begin{pgfscope}%
\pgfsys@transformshift{0.632463in}{0.679461in}%
\pgfsys@useobject{currentmarker}{}%
\end{pgfscope}%
\end{pgfscope}%
\begin{pgfscope}%
\definecolor{textcolor}{rgb}{0.317647,0.317647,0.317647}%
\pgfsetstrokecolor{textcolor}%
\pgfsetfillcolor{textcolor}%
\pgftext[x=0.256497in,y=0.647344in,left,base]{\color{textcolor}\rmfamily\fontsize{6.664000}{7.996800}\selectfont \(\displaystyle -0.050\)}%
\end{pgfscope}%
\begin{pgfscope}%
\pgfsetbuttcap%
\pgfsetroundjoin%
\definecolor{currentfill}{rgb}{0.317647,0.317647,0.317647}%
\pgfsetfillcolor{currentfill}%
\pgfsetlinewidth{0.501875pt}%
\definecolor{currentstroke}{rgb}{0.317647,0.317647,0.317647}%
\pgfsetstrokecolor{currentstroke}%
\pgfsetdash{}{0pt}%
\pgfsys@defobject{currentmarker}{\pgfqpoint{-0.020833in}{0.000000in}}{\pgfqpoint{0.000000in}{0.000000in}}{%
\pgfpathmoveto{\pgfqpoint{0.000000in}{0.000000in}}%
\pgfpathlineto{\pgfqpoint{-0.020833in}{0.000000in}}%
\pgfusepath{stroke,fill}%
}%
\begin{pgfscope}%
\pgfsys@transformshift{0.632463in}{0.916520in}%
\pgfsys@useobject{currentmarker}{}%
\end{pgfscope}%
\end{pgfscope}%
\begin{pgfscope}%
\definecolor{textcolor}{rgb}{0.317647,0.317647,0.317647}%
\pgfsetstrokecolor{textcolor}%
\pgfsetfillcolor{textcolor}%
\pgftext[x=0.256497in,y=0.884403in,left,base]{\color{textcolor}\rmfamily\fontsize{6.664000}{7.996800}\selectfont \(\displaystyle -0.025\)}%
\end{pgfscope}%
\begin{pgfscope}%
\pgfsetbuttcap%
\pgfsetroundjoin%
\definecolor{currentfill}{rgb}{0.317647,0.317647,0.317647}%
\pgfsetfillcolor{currentfill}%
\pgfsetlinewidth{0.501875pt}%
\definecolor{currentstroke}{rgb}{0.317647,0.317647,0.317647}%
\pgfsetstrokecolor{currentstroke}%
\pgfsetdash{}{0pt}%
\pgfsys@defobject{currentmarker}{\pgfqpoint{-0.020833in}{0.000000in}}{\pgfqpoint{0.000000in}{0.000000in}}{%
\pgfpathmoveto{\pgfqpoint{0.000000in}{0.000000in}}%
\pgfpathlineto{\pgfqpoint{-0.020833in}{0.000000in}}%
\pgfusepath{stroke,fill}%
}%
\begin{pgfscope}%
\pgfsys@transformshift{0.632463in}{1.153578in}%
\pgfsys@useobject{currentmarker}{}%
\end{pgfscope}%
\end{pgfscope}%
\begin{pgfscope}%
\definecolor{textcolor}{rgb}{0.317647,0.317647,0.317647}%
\pgfsetstrokecolor{textcolor}%
\pgfsetfillcolor{textcolor}%
\pgftext[x=0.343303in,y=1.121461in,left,base]{\color{textcolor}\rmfamily\fontsize{6.664000}{7.996800}\selectfont \(\displaystyle 0.000\)}%
\end{pgfscope}%
\begin{pgfscope}%
\pgfsetbuttcap%
\pgfsetroundjoin%
\definecolor{currentfill}{rgb}{0.317647,0.317647,0.317647}%
\pgfsetfillcolor{currentfill}%
\pgfsetlinewidth{0.501875pt}%
\definecolor{currentstroke}{rgb}{0.317647,0.317647,0.317647}%
\pgfsetstrokecolor{currentstroke}%
\pgfsetdash{}{0pt}%
\pgfsys@defobject{currentmarker}{\pgfqpoint{-0.020833in}{0.000000in}}{\pgfqpoint{0.000000in}{0.000000in}}{%
\pgfpathmoveto{\pgfqpoint{0.000000in}{0.000000in}}%
\pgfpathlineto{\pgfqpoint{-0.020833in}{0.000000in}}%
\pgfusepath{stroke,fill}%
}%
\begin{pgfscope}%
\pgfsys@transformshift{0.632463in}{1.390636in}%
\pgfsys@useobject{currentmarker}{}%
\end{pgfscope}%
\end{pgfscope}%
\begin{pgfscope}%
\definecolor{textcolor}{rgb}{0.317647,0.317647,0.317647}%
\pgfsetstrokecolor{textcolor}%
\pgfsetfillcolor{textcolor}%
\pgftext[x=0.343303in,y=1.358520in,left,base]{\color{textcolor}\rmfamily\fontsize{6.664000}{7.996800}\selectfont \(\displaystyle 0.025\)}%
\end{pgfscope}%
\begin{pgfscope}%
\pgfsetbuttcap%
\pgfsetroundjoin%
\definecolor{currentfill}{rgb}{0.317647,0.317647,0.317647}%
\pgfsetfillcolor{currentfill}%
\pgfsetlinewidth{0.501875pt}%
\definecolor{currentstroke}{rgb}{0.317647,0.317647,0.317647}%
\pgfsetstrokecolor{currentstroke}%
\pgfsetdash{}{0pt}%
\pgfsys@defobject{currentmarker}{\pgfqpoint{-0.020833in}{0.000000in}}{\pgfqpoint{0.000000in}{0.000000in}}{%
\pgfpathmoveto{\pgfqpoint{0.000000in}{0.000000in}}%
\pgfpathlineto{\pgfqpoint{-0.020833in}{0.000000in}}%
\pgfusepath{stroke,fill}%
}%
\begin{pgfscope}%
\pgfsys@transformshift{0.632463in}{1.627695in}%
\pgfsys@useobject{currentmarker}{}%
\end{pgfscope}%
\end{pgfscope}%
\begin{pgfscope}%
\definecolor{textcolor}{rgb}{0.317647,0.317647,0.317647}%
\pgfsetstrokecolor{textcolor}%
\pgfsetfillcolor{textcolor}%
\pgftext[x=0.343303in,y=1.595578in,left,base]{\color{textcolor}\rmfamily\fontsize{6.664000}{7.996800}\selectfont \(\displaystyle 0.050\)}%
\end{pgfscope}%
\begin{pgfscope}%
\pgfsetbuttcap%
\pgfsetroundjoin%
\definecolor{currentfill}{rgb}{0.317647,0.317647,0.317647}%
\pgfsetfillcolor{currentfill}%
\pgfsetlinewidth{0.501875pt}%
\definecolor{currentstroke}{rgb}{0.317647,0.317647,0.317647}%
\pgfsetstrokecolor{currentstroke}%
\pgfsetdash{}{0pt}%
\pgfsys@defobject{currentmarker}{\pgfqpoint{-0.020833in}{0.000000in}}{\pgfqpoint{0.000000in}{0.000000in}}{%
\pgfpathmoveto{\pgfqpoint{0.000000in}{0.000000in}}%
\pgfpathlineto{\pgfqpoint{-0.020833in}{0.000000in}}%
\pgfusepath{stroke,fill}%
}%
\begin{pgfscope}%
\pgfsys@transformshift{0.632463in}{1.864753in}%
\pgfsys@useobject{currentmarker}{}%
\end{pgfscope}%
\end{pgfscope}%
\begin{pgfscope}%
\definecolor{textcolor}{rgb}{0.317647,0.317647,0.317647}%
\pgfsetstrokecolor{textcolor}%
\pgfsetfillcolor{textcolor}%
\pgftext[x=0.343303in,y=1.832636in,left,base]{\color{textcolor}\rmfamily\fontsize{6.664000}{7.996800}\selectfont \(\displaystyle 0.075\)}%
\end{pgfscope}%
\begin{pgfscope}%
\definecolor{textcolor}{rgb}{0.317647,0.317647,0.317647}%
\pgfsetstrokecolor{textcolor}%
\pgfsetfillcolor{textcolor}%
\pgftext[x=0.200942in,y=1.153578in,,bottom,rotate=90.000000]{\color{textcolor}\rmfamily\fontsize{6.664000}{7.996800}\selectfont \(\displaystyle e^{(o)}\)}%
\end{pgfscope}%
\begin{pgfscope}%
\pgfpathrectangle{\pgfqpoint{0.632463in}{0.383578in}}{\pgfqpoint{1.356250in}{1.540000in}}%
\pgfusepath{clip}%
\pgfsetrectcap%
\pgfsetroundjoin%
\pgfsetlinewidth{0.803000pt}%
\definecolor{currentstroke}{rgb}{0.333333,0.333333,0.333333}%
\pgfsetstrokecolor{currentstroke}%
\pgfsetdash{}{0pt}%
\pgfpathmoveto{\pgfqpoint{0.686713in}{1.153578in}}%
\pgfpathlineto{\pgfqpoint{0.714412in}{1.153578in}}%
\pgfpathlineto{\pgfqpoint{0.742111in}{1.153578in}}%
\pgfpathlineto{\pgfqpoint{0.769811in}{1.153578in}}%
\pgfpathlineto{\pgfqpoint{0.797510in}{1.153578in}}%
\pgfpathlineto{\pgfqpoint{0.825209in}{1.153578in}}%
\pgfpathlineto{\pgfqpoint{0.852908in}{1.153578in}}%
\pgfpathlineto{\pgfqpoint{0.880608in}{1.153578in}}%
\pgfpathlineto{\pgfqpoint{0.908307in}{1.153578in}}%
\pgfpathlineto{\pgfqpoint{0.936006in}{0.842002in}}%
\pgfpathlineto{\pgfqpoint{0.963706in}{0.643385in}}%
\pgfpathlineto{\pgfqpoint{0.991405in}{0.527012in}}%
\pgfpathlineto{\pgfqpoint{1.019104in}{0.469593in}}%
\pgfpathlineto{\pgfqpoint{1.046803in}{0.453578in}}%
\pgfpathlineto{\pgfqpoint{1.074503in}{0.465844in}}%
\pgfpathlineto{\pgfqpoint{1.102202in}{0.496664in}}%
\pgfpathlineto{\pgfqpoint{1.129901in}{0.538909in}}%
\pgfpathlineto{\pgfqpoint{1.157601in}{0.587424in}}%
\pgfpathlineto{\pgfqpoint{1.185300in}{0.638547in}}%
\pgfpathlineto{\pgfqpoint{1.212999in}{0.689739in}}%
\pgfpathlineto{\pgfqpoint{1.240698in}{0.739295in}}%
\pgfpathlineto{\pgfqpoint{1.268398in}{0.786126in}}%
\pgfpathlineto{\pgfqpoint{1.296097in}{0.829592in}}%
\pgfpathlineto{\pgfqpoint{1.323796in}{0.869374in}}%
\pgfpathlineto{\pgfqpoint{1.351496in}{0.905379in}}%
\pgfpathlineto{\pgfqpoint{1.379195in}{0.937669in}}%
\pgfpathlineto{\pgfqpoint{1.406894in}{0.966409in}}%
\pgfpathlineto{\pgfqpoint{1.434593in}{0.991823in}}%
\pgfpathlineto{\pgfqpoint{1.462293in}{1.014174in}}%
\pgfpathlineto{\pgfqpoint{1.489992in}{1.033737in}}%
\pgfpathlineto{\pgfqpoint{1.517691in}{1.050788in}}%
\pgfpathlineto{\pgfqpoint{1.545391in}{1.065596in}}%
\pgfpathlineto{\pgfqpoint{1.573090in}{1.078412in}}%
\pgfpathlineto{\pgfqpoint{1.600789in}{1.089473in}}%
\pgfpathlineto{\pgfqpoint{1.628488in}{1.098994in}}%
\pgfpathlineto{\pgfqpoint{1.656188in}{1.107170in}}%
\pgfpathlineto{\pgfqpoint{1.683887in}{1.114175in}}%
\pgfpathlineto{\pgfqpoint{1.711586in}{1.120165in}}%
\pgfpathlineto{\pgfqpoint{1.739286in}{1.125279in}}%
\pgfpathlineto{\pgfqpoint{1.766985in}{1.129636in}}%
\pgfpathlineto{\pgfqpoint{1.794684in}{1.133344in}}%
\pgfpathlineto{\pgfqpoint{1.822383in}{1.136494in}}%
\pgfpathlineto{\pgfqpoint{1.850083in}{1.139167in}}%
\pgfpathlineto{\pgfqpoint{1.877782in}{1.141432in}}%
\pgfpathlineto{\pgfqpoint{1.905481in}{1.143350in}}%
\pgfpathlineto{\pgfqpoint{1.933181in}{1.144971in}}%
\pgfpathlineto{\pgfqpoint{1.960880in}{1.146341in}}%
\pgfpathlineto{\pgfqpoint{1.988579in}{1.147497in}}%
\pgfpathlineto{\pgfqpoint{1.998713in}{1.147853in}}%
\pgfusepath{stroke}%
\end{pgfscope}%
\begin{pgfscope}%
\pgfpathrectangle{\pgfqpoint{0.632463in}{0.383578in}}{\pgfqpoint{1.356250in}{1.540000in}}%
\pgfusepath{clip}%
\pgfsetrectcap%
\pgfsetroundjoin%
\pgfsetlinewidth{0.803000pt}%
\definecolor{currentstroke}{rgb}{0.686275,0.352941,0.313725}%
\pgfsetstrokecolor{currentstroke}%
\pgfsetdash{}{0pt}%
\pgfpathmoveto{\pgfqpoint{0.686713in}{1.153578in}}%
\pgfpathlineto{\pgfqpoint{0.714412in}{1.153578in}}%
\pgfpathlineto{\pgfqpoint{0.742111in}{1.153578in}}%
\pgfpathlineto{\pgfqpoint{0.769811in}{1.153578in}}%
\pgfpathlineto{\pgfqpoint{0.797510in}{1.153578in}}%
\pgfpathlineto{\pgfqpoint{0.825209in}{1.153578in}}%
\pgfpathlineto{\pgfqpoint{0.852908in}{1.153578in}}%
\pgfpathlineto{\pgfqpoint{0.880608in}{1.153578in}}%
\pgfpathlineto{\pgfqpoint{0.908307in}{1.153578in}}%
\pgfpathlineto{\pgfqpoint{0.936006in}{1.153578in}}%
\pgfpathlineto{\pgfqpoint{0.963706in}{1.153578in}}%
\pgfpathlineto{\pgfqpoint{0.991405in}{1.153578in}}%
\pgfpathlineto{\pgfqpoint{1.019104in}{1.153578in}}%
\pgfpathlineto{\pgfqpoint{1.046803in}{1.153578in}}%
\pgfpathlineto{\pgfqpoint{1.074503in}{1.465154in}}%
\pgfpathlineto{\pgfqpoint{1.102202in}{1.663771in}}%
\pgfpathlineto{\pgfqpoint{1.129901in}{1.780144in}}%
\pgfpathlineto{\pgfqpoint{1.157601in}{1.837564in}}%
\pgfpathlineto{\pgfqpoint{1.185300in}{1.853578in}}%
\pgfpathlineto{\pgfqpoint{1.212999in}{1.841312in}}%
\pgfpathlineto{\pgfqpoint{1.240698in}{1.810492in}}%
\pgfpathlineto{\pgfqpoint{1.268398in}{1.768247in}}%
\pgfpathlineto{\pgfqpoint{1.296097in}{1.719733in}}%
\pgfpathlineto{\pgfqpoint{1.323796in}{1.668609in}}%
\pgfpathlineto{\pgfqpoint{1.351496in}{1.617417in}}%
\pgfpathlineto{\pgfqpoint{1.379195in}{1.567861in}}%
\pgfpathlineto{\pgfqpoint{1.406894in}{1.521030in}}%
\pgfpathlineto{\pgfqpoint{1.434593in}{1.477564in}}%
\pgfpathlineto{\pgfqpoint{1.462293in}{1.437782in}}%
\pgfpathlineto{\pgfqpoint{1.489992in}{1.401777in}}%
\pgfpathlineto{\pgfqpoint{1.517691in}{1.369487in}}%
\pgfpathlineto{\pgfqpoint{1.545391in}{1.340747in}}%
\pgfpathlineto{\pgfqpoint{1.573090in}{1.315333in}}%
\pgfpathlineto{\pgfqpoint{1.600789in}{1.292982in}}%
\pgfpathlineto{\pgfqpoint{1.628488in}{1.273419in}}%
\pgfpathlineto{\pgfqpoint{1.656188in}{1.256368in}}%
\pgfpathlineto{\pgfqpoint{1.683887in}{1.241560in}}%
\pgfpathlineto{\pgfqpoint{1.711586in}{1.228744in}}%
\pgfpathlineto{\pgfqpoint{1.739286in}{1.217683in}}%
\pgfpathlineto{\pgfqpoint{1.766985in}{1.208162in}}%
\pgfpathlineto{\pgfqpoint{1.794684in}{1.199986in}}%
\pgfpathlineto{\pgfqpoint{1.822383in}{1.192981in}}%
\pgfpathlineto{\pgfqpoint{1.850083in}{1.186991in}}%
\pgfpathlineto{\pgfqpoint{1.877782in}{1.181877in}}%
\pgfpathlineto{\pgfqpoint{1.905481in}{1.177520in}}%
\pgfpathlineto{\pgfqpoint{1.933181in}{1.173812in}}%
\pgfpathlineto{\pgfqpoint{1.960880in}{1.170662in}}%
\pgfpathlineto{\pgfqpoint{1.988579in}{1.167989in}}%
\pgfpathlineto{\pgfqpoint{1.998713in}{1.167160in}}%
\pgfusepath{stroke}%
\end{pgfscope}%
\begin{pgfscope}%
\pgfsetrectcap%
\pgfsetmiterjoin%
\pgfsetlinewidth{0.501875pt}%
\definecolor{currentstroke}{rgb}{0.317647,0.317647,0.317647}%
\pgfsetstrokecolor{currentstroke}%
\pgfsetdash{}{0pt}%
\pgfpathmoveto{\pgfqpoint{0.632463in}{0.383578in}}%
\pgfpathlineto{\pgfqpoint{0.632463in}{1.923578in}}%
\pgfusepath{stroke}%
\end{pgfscope}%
\begin{pgfscope}%
\pgfsetrectcap%
\pgfsetmiterjoin%
\pgfsetlinewidth{0.501875pt}%
\definecolor{currentstroke}{rgb}{0.317647,0.317647,0.317647}%
\pgfsetstrokecolor{currentstroke}%
\pgfsetdash{}{0pt}%
\pgfpathmoveto{\pgfqpoint{0.632463in}{0.383578in}}%
\pgfpathlineto{\pgfqpoint{1.988713in}{0.383578in}}%
\pgfusepath{stroke}%
\end{pgfscope}%
\begin{pgfscope}%
\pgfsetrectcap%
\pgfsetroundjoin%
\pgfsetlinewidth{0.803000pt}%
\definecolor{currentstroke}{rgb}{0.333333,0.333333,0.333333}%
\pgfsetstrokecolor{currentstroke}%
\pgfsetdash{}{0pt}%
\pgfpathmoveto{\pgfqpoint{1.787960in}{1.863417in}}%
\pgfpathlineto{\pgfqpoint{1.862004in}{1.863417in}}%
\pgfusepath{stroke}%
\end{pgfscope}%
\begin{pgfscope}%
\definecolor{textcolor}{rgb}{0.000000,0.000000,0.000000}%
\pgfsetstrokecolor{textcolor}%
\pgfsetfillcolor{textcolor}%
\pgftext[x=1.908282in,y=1.831022in,left,base]{\color{textcolor}\rmfamily\fontsize{6.664000}{7.996800}\selectfont 0}%
\end{pgfscope}%
\begin{pgfscope}%
\pgfsetrectcap%
\pgfsetroundjoin%
\pgfsetlinewidth{0.803000pt}%
\definecolor{currentstroke}{rgb}{0.686275,0.352941,0.313725}%
\pgfsetstrokecolor{currentstroke}%
\pgfsetdash{}{0pt}%
\pgfpathmoveto{\pgfqpoint{1.787960in}{1.743650in}}%
\pgfpathlineto{\pgfqpoint{1.862004in}{1.743650in}}%
\pgfusepath{stroke}%
\end{pgfscope}%
\begin{pgfscope}%
\definecolor{textcolor}{rgb}{0.000000,0.000000,0.000000}%
\pgfsetstrokecolor{textcolor}%
\pgfsetfillcolor{textcolor}%
\pgftext[x=1.908282in,y=1.711256in,left,base]{\color{textcolor}\rmfamily\fontsize{6.664000}{7.996800}\selectfont 1}%
\end{pgfscope}%
\begin{pgfscope}%
\pgfsetbuttcap%
\pgfsetmiterjoin%
\pgfsetlinewidth{0.000000pt}%
\definecolor{currentstroke}{rgb}{0.000000,0.000000,0.000000}%
\pgfsetstrokecolor{currentstroke}%
\pgfsetstrokeopacity{0.000000}%
\pgfsetdash{}{0pt}%
\pgfpathmoveto{\pgfqpoint{2.666838in}{0.383578in}}%
\pgfpathlineto{\pgfqpoint{4.023088in}{0.383578in}}%
\pgfpathlineto{\pgfqpoint{4.023088in}{1.923578in}}%
\pgfpathlineto{\pgfqpoint{2.666838in}{1.923578in}}%
\pgfpathclose%
\pgfusepath{}%
\end{pgfscope}%
\begin{pgfscope}%
\pgfsetbuttcap%
\pgfsetroundjoin%
\definecolor{currentfill}{rgb}{0.317647,0.317647,0.317647}%
\pgfsetfillcolor{currentfill}%
\pgfsetlinewidth{0.501875pt}%
\definecolor{currentstroke}{rgb}{0.317647,0.317647,0.317647}%
\pgfsetstrokecolor{currentstroke}%
\pgfsetdash{}{0pt}%
\pgfsys@defobject{currentmarker}{\pgfqpoint{0.000000in}{-0.020833in}}{\pgfqpoint{0.000000in}{0.000000in}}{%
\pgfpathmoveto{\pgfqpoint{0.000000in}{0.000000in}}%
\pgfpathlineto{\pgfqpoint{0.000000in}{-0.020833in}}%
\pgfusepath{stroke,fill}%
}%
\begin{pgfscope}%
\pgfsys@transformshift{2.721088in}{0.383578in}%
\pgfsys@useobject{currentmarker}{}%
\end{pgfscope}%
\end{pgfscope}%
\begin{pgfscope}%
\definecolor{textcolor}{rgb}{0.317647,0.317647,0.317647}%
\pgfsetstrokecolor{textcolor}%
\pgfsetfillcolor{textcolor}%
\pgftext[x=2.721088in,y=0.334967in,,top]{\color{textcolor}\rmfamily\fontsize{6.664000}{7.996800}\selectfont \(\displaystyle 0\)}%
\end{pgfscope}%
\begin{pgfscope}%
\pgfsetbuttcap%
\pgfsetroundjoin%
\definecolor{currentfill}{rgb}{0.317647,0.317647,0.317647}%
\pgfsetfillcolor{currentfill}%
\pgfsetlinewidth{0.501875pt}%
\definecolor{currentstroke}{rgb}{0.317647,0.317647,0.317647}%
\pgfsetstrokecolor{currentstroke}%
\pgfsetdash{}{0pt}%
\pgfsys@defobject{currentmarker}{\pgfqpoint{0.000000in}{-0.020833in}}{\pgfqpoint{0.000000in}{0.000000in}}{%
\pgfpathmoveto{\pgfqpoint{0.000000in}{0.000000in}}%
\pgfpathlineto{\pgfqpoint{0.000000in}{-0.020833in}}%
\pgfusepath{stroke,fill}%
}%
\begin{pgfscope}%
\pgfsys@transformshift{3.263588in}{0.383578in}%
\pgfsys@useobject{currentmarker}{}%
\end{pgfscope}%
\end{pgfscope}%
\begin{pgfscope}%
\definecolor{textcolor}{rgb}{0.317647,0.317647,0.317647}%
\pgfsetstrokecolor{textcolor}%
\pgfsetfillcolor{textcolor}%
\pgftext[x=3.263588in,y=0.334967in,,top]{\color{textcolor}\rmfamily\fontsize{6.664000}{7.996800}\selectfont \(\displaystyle 50\)}%
\end{pgfscope}%
\begin{pgfscope}%
\pgfsetbuttcap%
\pgfsetroundjoin%
\definecolor{currentfill}{rgb}{0.317647,0.317647,0.317647}%
\pgfsetfillcolor{currentfill}%
\pgfsetlinewidth{0.501875pt}%
\definecolor{currentstroke}{rgb}{0.317647,0.317647,0.317647}%
\pgfsetstrokecolor{currentstroke}%
\pgfsetdash{}{0pt}%
\pgfsys@defobject{currentmarker}{\pgfqpoint{0.000000in}{-0.020833in}}{\pgfqpoint{0.000000in}{0.000000in}}{%
\pgfpathmoveto{\pgfqpoint{0.000000in}{0.000000in}}%
\pgfpathlineto{\pgfqpoint{0.000000in}{-0.020833in}}%
\pgfusepath{stroke,fill}%
}%
\begin{pgfscope}%
\pgfsys@transformshift{3.806088in}{0.383578in}%
\pgfsys@useobject{currentmarker}{}%
\end{pgfscope}%
\end{pgfscope}%
\begin{pgfscope}%
\definecolor{textcolor}{rgb}{0.317647,0.317647,0.317647}%
\pgfsetstrokecolor{textcolor}%
\pgfsetfillcolor{textcolor}%
\pgftext[x=3.806088in,y=0.334967in,,top]{\color{textcolor}\rmfamily\fontsize{6.664000}{7.996800}\selectfont \(\displaystyle 100\)}%
\end{pgfscope}%
\begin{pgfscope}%
\definecolor{textcolor}{rgb}{0.317647,0.317647,0.317647}%
\pgfsetstrokecolor{textcolor}%
\pgfsetfillcolor{textcolor}%
\pgftext[x=3.344963in,y=0.197222in,,top]{\color{textcolor}\rmfamily\fontsize{6.664000}{7.996800}\selectfont time\(\displaystyle \;(\si{\micro \s})\)}%
\end{pgfscope}%
\begin{pgfscope}%
\pgfsetbuttcap%
\pgfsetroundjoin%
\definecolor{currentfill}{rgb}{0.317647,0.317647,0.317647}%
\pgfsetfillcolor{currentfill}%
\pgfsetlinewidth{0.501875pt}%
\definecolor{currentstroke}{rgb}{0.317647,0.317647,0.317647}%
\pgfsetstrokecolor{currentstroke}%
\pgfsetdash{}{0pt}%
\pgfsys@defobject{currentmarker}{\pgfqpoint{-0.020833in}{0.000000in}}{\pgfqpoint{0.000000in}{0.000000in}}{%
\pgfpathmoveto{\pgfqpoint{0.000000in}{0.000000in}}%
\pgfpathlineto{\pgfqpoint{-0.020833in}{0.000000in}}%
\pgfusepath{stroke,fill}%
}%
\begin{pgfscope}%
\pgfsys@transformshift{2.666838in}{0.453578in}%
\pgfsys@useobject{currentmarker}{}%
\end{pgfscope}%
\end{pgfscope}%
\begin{pgfscope}%
\definecolor{textcolor}{rgb}{0.317647,0.317647,0.317647}%
\pgfsetstrokecolor{textcolor}%
\pgfsetfillcolor{textcolor}%
\pgftext[x=2.377678in,y=0.421461in,left,base]{\color{textcolor}\rmfamily\fontsize{6.664000}{7.996800}\selectfont \(\displaystyle 0.000\)}%
\end{pgfscope}%
\begin{pgfscope}%
\pgfsetbuttcap%
\pgfsetroundjoin%
\definecolor{currentfill}{rgb}{0.317647,0.317647,0.317647}%
\pgfsetfillcolor{currentfill}%
\pgfsetlinewidth{0.501875pt}%
\definecolor{currentstroke}{rgb}{0.317647,0.317647,0.317647}%
\pgfsetstrokecolor{currentstroke}%
\pgfsetdash{}{0pt}%
\pgfsys@defobject{currentmarker}{\pgfqpoint{-0.020833in}{0.000000in}}{\pgfqpoint{0.000000in}{0.000000in}}{%
\pgfpathmoveto{\pgfqpoint{0.000000in}{0.000000in}}%
\pgfpathlineto{\pgfqpoint{-0.020833in}{0.000000in}}%
\pgfusepath{stroke,fill}%
}%
\begin{pgfscope}%
\pgfsys@transformshift{2.666838in}{0.831987in}%
\pgfsys@useobject{currentmarker}{}%
\end{pgfscope}%
\end{pgfscope}%
\begin{pgfscope}%
\definecolor{textcolor}{rgb}{0.317647,0.317647,0.317647}%
\pgfsetstrokecolor{textcolor}%
\pgfsetfillcolor{textcolor}%
\pgftext[x=2.377678in,y=0.799870in,left,base]{\color{textcolor}\rmfamily\fontsize{6.664000}{7.996800}\selectfont \(\displaystyle 0.002\)}%
\end{pgfscope}%
\begin{pgfscope}%
\pgfsetbuttcap%
\pgfsetroundjoin%
\definecolor{currentfill}{rgb}{0.317647,0.317647,0.317647}%
\pgfsetfillcolor{currentfill}%
\pgfsetlinewidth{0.501875pt}%
\definecolor{currentstroke}{rgb}{0.317647,0.317647,0.317647}%
\pgfsetstrokecolor{currentstroke}%
\pgfsetdash{}{0pt}%
\pgfsys@defobject{currentmarker}{\pgfqpoint{-0.020833in}{0.000000in}}{\pgfqpoint{0.000000in}{0.000000in}}{%
\pgfpathmoveto{\pgfqpoint{0.000000in}{0.000000in}}%
\pgfpathlineto{\pgfqpoint{-0.020833in}{0.000000in}}%
\pgfusepath{stroke,fill}%
}%
\begin{pgfscope}%
\pgfsys@transformshift{2.666838in}{1.210396in}%
\pgfsys@useobject{currentmarker}{}%
\end{pgfscope}%
\end{pgfscope}%
\begin{pgfscope}%
\definecolor{textcolor}{rgb}{0.317647,0.317647,0.317647}%
\pgfsetstrokecolor{textcolor}%
\pgfsetfillcolor{textcolor}%
\pgftext[x=2.377678in,y=1.178279in,left,base]{\color{textcolor}\rmfamily\fontsize{6.664000}{7.996800}\selectfont \(\displaystyle 0.004\)}%
\end{pgfscope}%
\begin{pgfscope}%
\pgfsetbuttcap%
\pgfsetroundjoin%
\definecolor{currentfill}{rgb}{0.317647,0.317647,0.317647}%
\pgfsetfillcolor{currentfill}%
\pgfsetlinewidth{0.501875pt}%
\definecolor{currentstroke}{rgb}{0.317647,0.317647,0.317647}%
\pgfsetstrokecolor{currentstroke}%
\pgfsetdash{}{0pt}%
\pgfsys@defobject{currentmarker}{\pgfqpoint{-0.020833in}{0.000000in}}{\pgfqpoint{0.000000in}{0.000000in}}{%
\pgfpathmoveto{\pgfqpoint{0.000000in}{0.000000in}}%
\pgfpathlineto{\pgfqpoint{-0.020833in}{0.000000in}}%
\pgfusepath{stroke,fill}%
}%
\begin{pgfscope}%
\pgfsys@transformshift{2.666838in}{1.588805in}%
\pgfsys@useobject{currentmarker}{}%
\end{pgfscope}%
\end{pgfscope}%
\begin{pgfscope}%
\definecolor{textcolor}{rgb}{0.317647,0.317647,0.317647}%
\pgfsetstrokecolor{textcolor}%
\pgfsetfillcolor{textcolor}%
\pgftext[x=2.377678in,y=1.556688in,left,base]{\color{textcolor}\rmfamily\fontsize{6.664000}{7.996800}\selectfont \(\displaystyle 0.006\)}%
\end{pgfscope}%
\begin{pgfscope}%
\definecolor{textcolor}{rgb}{0.317647,0.317647,0.317647}%
\pgfsetstrokecolor{textcolor}%
\pgfsetfillcolor{textcolor}%
\pgftext[x=2.322122in,y=1.153578in,,bottom,rotate=90.000000]{\color{textcolor}\rmfamily\fontsize{6.664000}{7.996800}\selectfont \(\displaystyle \lambda^{(o)}_{ij}s\)}%
\end{pgfscope}%
\begin{pgfscope}%
\pgfpathrectangle{\pgfqpoint{2.666838in}{0.383578in}}{\pgfqpoint{1.356250in}{1.540000in}}%
\pgfusepath{clip}%
\pgfsetrectcap%
\pgfsetroundjoin%
\pgfsetlinewidth{0.803000pt}%
\definecolor{currentstroke}{rgb}{0.333333,0.333333,0.333333}%
\pgfsetstrokecolor{currentstroke}%
\pgfsetstrokeopacity{0.300000}%
\pgfsetdash{}{0pt}%
\pgfpathmoveto{\pgfqpoint{2.721088in}{0.453578in}}%
\pgfpathlineto{\pgfqpoint{2.748787in}{0.453578in}}%
\pgfpathlineto{\pgfqpoint{2.776486in}{0.453578in}}%
\pgfpathlineto{\pgfqpoint{2.804186in}{0.453578in}}%
\pgfpathlineto{\pgfqpoint{2.831885in}{0.453578in}}%
\pgfpathlineto{\pgfqpoint{2.859584in}{0.453578in}}%
\pgfpathlineto{\pgfqpoint{2.887283in}{0.453578in}}%
\pgfpathlineto{\pgfqpoint{2.914983in}{0.453578in}}%
\pgfpathlineto{\pgfqpoint{2.942682in}{0.453578in}}%
\pgfpathlineto{\pgfqpoint{2.970381in}{0.453578in}}%
\pgfpathlineto{\pgfqpoint{2.998081in}{0.453578in}}%
\pgfpathlineto{\pgfqpoint{3.025780in}{0.453578in}}%
\pgfpathlineto{\pgfqpoint{3.053479in}{0.453578in}}%
\pgfpathlineto{\pgfqpoint{3.081178in}{0.453578in}}%
\pgfpathlineto{\pgfqpoint{3.108878in}{0.453578in}}%
\pgfpathlineto{\pgfqpoint{3.136577in}{0.453578in}}%
\pgfpathlineto{\pgfqpoint{3.164276in}{0.453578in}}%
\pgfpathlineto{\pgfqpoint{3.191976in}{0.453578in}}%
\pgfpathlineto{\pgfqpoint{3.219675in}{0.453578in}}%
\pgfpathlineto{\pgfqpoint{3.247374in}{0.453578in}}%
\pgfpathlineto{\pgfqpoint{3.275073in}{0.453578in}}%
\pgfpathlineto{\pgfqpoint{3.302773in}{0.453578in}}%
\pgfpathlineto{\pgfqpoint{3.330472in}{0.453578in}}%
\pgfpathlineto{\pgfqpoint{3.358171in}{0.453578in}}%
\pgfpathlineto{\pgfqpoint{3.385871in}{0.453578in}}%
\pgfpathlineto{\pgfqpoint{3.413570in}{0.453578in}}%
\pgfpathlineto{\pgfqpoint{3.441269in}{0.453578in}}%
\pgfpathlineto{\pgfqpoint{3.468968in}{0.453578in}}%
\pgfpathlineto{\pgfqpoint{3.496668in}{0.453578in}}%
\pgfpathlineto{\pgfqpoint{3.524367in}{0.453578in}}%
\pgfpathlineto{\pgfqpoint{3.552066in}{0.453578in}}%
\pgfpathlineto{\pgfqpoint{3.579766in}{0.453578in}}%
\pgfpathlineto{\pgfqpoint{3.607465in}{0.453578in}}%
\pgfpathlineto{\pgfqpoint{3.635164in}{0.453578in}}%
\pgfpathlineto{\pgfqpoint{3.662863in}{0.453578in}}%
\pgfpathlineto{\pgfqpoint{3.690563in}{0.453578in}}%
\pgfpathlineto{\pgfqpoint{3.718262in}{0.453578in}}%
\pgfpathlineto{\pgfqpoint{3.745961in}{0.453578in}}%
\pgfpathlineto{\pgfqpoint{3.773661in}{0.453578in}}%
\pgfpathlineto{\pgfqpoint{3.801360in}{0.453578in}}%
\pgfpathlineto{\pgfqpoint{3.829059in}{0.453578in}}%
\pgfpathlineto{\pgfqpoint{3.856758in}{0.453578in}}%
\pgfpathlineto{\pgfqpoint{3.884458in}{0.453578in}}%
\pgfpathlineto{\pgfqpoint{3.912157in}{0.453578in}}%
\pgfpathlineto{\pgfqpoint{3.939856in}{0.453578in}}%
\pgfpathlineto{\pgfqpoint{3.967556in}{0.453578in}}%
\pgfpathlineto{\pgfqpoint{3.995255in}{0.453578in}}%
\pgfpathlineto{\pgfqpoint{4.022954in}{0.453578in}}%
\pgfpathlineto{\pgfqpoint{4.033088in}{0.453578in}}%
\pgfusepath{stroke}%
\end{pgfscope}%
\begin{pgfscope}%
\pgfpathrectangle{\pgfqpoint{2.666838in}{0.383578in}}{\pgfqpoint{1.356250in}{1.540000in}}%
\pgfusepath{clip}%
\pgfsetrectcap%
\pgfsetroundjoin%
\pgfsetlinewidth{0.803000pt}%
\definecolor{currentstroke}{rgb}{0.333333,0.333333,0.333333}%
\pgfsetstrokecolor{currentstroke}%
\pgfsetstrokeopacity{0.300000}%
\pgfsetdash{}{0pt}%
\pgfpathmoveto{\pgfqpoint{2.721088in}{0.453578in}}%
\pgfpathlineto{\pgfqpoint{2.748787in}{0.453578in}}%
\pgfpathlineto{\pgfqpoint{2.776486in}{0.453578in}}%
\pgfpathlineto{\pgfqpoint{2.804186in}{0.453578in}}%
\pgfpathlineto{\pgfqpoint{2.831885in}{0.453578in}}%
\pgfpathlineto{\pgfqpoint{2.859584in}{0.453578in}}%
\pgfpathlineto{\pgfqpoint{2.887283in}{0.453578in}}%
\pgfpathlineto{\pgfqpoint{2.914983in}{0.453578in}}%
\pgfpathlineto{\pgfqpoint{2.942682in}{0.453578in}}%
\pgfpathlineto{\pgfqpoint{2.970381in}{0.483017in}}%
\pgfpathlineto{\pgfqpoint{2.998081in}{0.549990in}}%
\pgfpathlineto{\pgfqpoint{3.025780in}{0.655915in}}%
\pgfpathlineto{\pgfqpoint{3.053479in}{0.802534in}}%
\pgfpathlineto{\pgfqpoint{3.081178in}{0.945499in}}%
\pgfpathlineto{\pgfqpoint{3.108878in}{1.061524in}}%
\pgfpathlineto{\pgfqpoint{3.136577in}{1.144052in}}%
\pgfpathlineto{\pgfqpoint{3.164276in}{1.190503in}}%
\pgfpathlineto{\pgfqpoint{3.191976in}{1.210154in}}%
\pgfpathlineto{\pgfqpoint{3.219675in}{1.212226in}}%
\pgfpathlineto{\pgfqpoint{3.247374in}{1.206148in}}%
\pgfpathlineto{\pgfqpoint{3.275073in}{1.203251in}}%
\pgfpathlineto{\pgfqpoint{3.302773in}{1.208674in}}%
\pgfpathlineto{\pgfqpoint{3.330472in}{1.226464in}}%
\pgfpathlineto{\pgfqpoint{3.358171in}{1.256225in}}%
\pgfpathlineto{\pgfqpoint{3.385871in}{1.290560in}}%
\pgfpathlineto{\pgfqpoint{3.413570in}{1.327277in}}%
\pgfpathlineto{\pgfqpoint{3.441269in}{1.361623in}}%
\pgfpathlineto{\pgfqpoint{3.468968in}{1.390519in}}%
\pgfpathlineto{\pgfqpoint{3.496668in}{1.412140in}}%
\pgfpathlineto{\pgfqpoint{3.524367in}{1.423539in}}%
\pgfpathlineto{\pgfqpoint{3.552066in}{1.423669in}}%
\pgfpathlineto{\pgfqpoint{3.579766in}{1.415808in}}%
\pgfpathlineto{\pgfqpoint{3.607465in}{1.400767in}}%
\pgfpathlineto{\pgfqpoint{3.635164in}{1.378193in}}%
\pgfpathlineto{\pgfqpoint{3.662863in}{1.349605in}}%
\pgfpathlineto{\pgfqpoint{3.690563in}{1.315389in}}%
\pgfpathlineto{\pgfqpoint{3.718262in}{1.274852in}}%
\pgfpathlineto{\pgfqpoint{3.745961in}{1.230404in}}%
\pgfpathlineto{\pgfqpoint{3.773661in}{1.183309in}}%
\pgfpathlineto{\pgfqpoint{3.801360in}{1.134721in}}%
\pgfpathlineto{\pgfqpoint{3.829059in}{1.085273in}}%
\pgfpathlineto{\pgfqpoint{3.856758in}{1.036017in}}%
\pgfpathlineto{\pgfqpoint{3.884458in}{0.988050in}}%
\pgfpathlineto{\pgfqpoint{3.912157in}{0.942100in}}%
\pgfpathlineto{\pgfqpoint{3.939856in}{0.898220in}}%
\pgfpathlineto{\pgfqpoint{3.967556in}{0.856698in}}%
\pgfpathlineto{\pgfqpoint{3.995255in}{0.817722in}}%
\pgfpathlineto{\pgfqpoint{4.022954in}{0.781398in}}%
\pgfpathlineto{\pgfqpoint{4.033088in}{0.769092in}}%
\pgfusepath{stroke}%
\end{pgfscope}%
\begin{pgfscope}%
\pgfpathrectangle{\pgfqpoint{2.666838in}{0.383578in}}{\pgfqpoint{1.356250in}{1.540000in}}%
\pgfusepath{clip}%
\pgfsetrectcap%
\pgfsetroundjoin%
\pgfsetlinewidth{0.803000pt}%
\definecolor{currentstroke}{rgb}{0.333333,0.333333,0.333333}%
\pgfsetstrokecolor{currentstroke}%
\pgfsetstrokeopacity{0.300000}%
\pgfsetdash{}{0pt}%
\pgfpathmoveto{\pgfqpoint{2.721088in}{0.453578in}}%
\pgfpathlineto{\pgfqpoint{2.748787in}{0.453578in}}%
\pgfpathlineto{\pgfqpoint{2.776486in}{0.453578in}}%
\pgfpathlineto{\pgfqpoint{2.804186in}{0.453578in}}%
\pgfpathlineto{\pgfqpoint{2.831885in}{0.453578in}}%
\pgfpathlineto{\pgfqpoint{2.859584in}{0.453578in}}%
\pgfpathlineto{\pgfqpoint{2.887283in}{0.453578in}}%
\pgfpathlineto{\pgfqpoint{2.914983in}{0.453578in}}%
\pgfpathlineto{\pgfqpoint{2.942682in}{0.453578in}}%
\pgfpathlineto{\pgfqpoint{2.970381in}{0.453578in}}%
\pgfpathlineto{\pgfqpoint{2.998081in}{0.453578in}}%
\pgfpathlineto{\pgfqpoint{3.025780in}{0.453578in}}%
\pgfpathlineto{\pgfqpoint{3.053479in}{0.453578in}}%
\pgfpathlineto{\pgfqpoint{3.081178in}{0.453578in}}%
\pgfpathlineto{\pgfqpoint{3.108878in}{0.453578in}}%
\pgfpathlineto{\pgfqpoint{3.136577in}{0.453578in}}%
\pgfpathlineto{\pgfqpoint{3.164276in}{0.453578in}}%
\pgfpathlineto{\pgfqpoint{3.191976in}{0.453578in}}%
\pgfpathlineto{\pgfqpoint{3.219675in}{0.453578in}}%
\pgfpathlineto{\pgfqpoint{3.247374in}{0.453578in}}%
\pgfpathlineto{\pgfqpoint{3.275073in}{0.453578in}}%
\pgfpathlineto{\pgfqpoint{3.302773in}{0.453578in}}%
\pgfpathlineto{\pgfqpoint{3.330472in}{0.453578in}}%
\pgfpathlineto{\pgfqpoint{3.358171in}{0.453578in}}%
\pgfpathlineto{\pgfqpoint{3.385871in}{0.453578in}}%
\pgfpathlineto{\pgfqpoint{3.413570in}{0.453578in}}%
\pgfpathlineto{\pgfqpoint{3.441269in}{0.453578in}}%
\pgfpathlineto{\pgfqpoint{3.468968in}{0.453578in}}%
\pgfpathlineto{\pgfqpoint{3.496668in}{0.453578in}}%
\pgfpathlineto{\pgfqpoint{3.524367in}{0.453578in}}%
\pgfpathlineto{\pgfqpoint{3.552066in}{0.453578in}}%
\pgfpathlineto{\pgfqpoint{3.579766in}{0.453578in}}%
\pgfpathlineto{\pgfqpoint{3.607465in}{0.453578in}}%
\pgfpathlineto{\pgfqpoint{3.635164in}{0.453578in}}%
\pgfpathlineto{\pgfqpoint{3.662863in}{0.453578in}}%
\pgfpathlineto{\pgfqpoint{3.690563in}{0.453578in}}%
\pgfpathlineto{\pgfqpoint{3.718262in}{0.453578in}}%
\pgfpathlineto{\pgfqpoint{3.745961in}{0.453578in}}%
\pgfpathlineto{\pgfqpoint{3.773661in}{0.453578in}}%
\pgfpathlineto{\pgfqpoint{3.801360in}{0.453578in}}%
\pgfpathlineto{\pgfqpoint{3.829059in}{0.453578in}}%
\pgfpathlineto{\pgfqpoint{3.856758in}{0.453578in}}%
\pgfpathlineto{\pgfqpoint{3.884458in}{0.453578in}}%
\pgfpathlineto{\pgfqpoint{3.912157in}{0.453578in}}%
\pgfpathlineto{\pgfqpoint{3.939856in}{0.453578in}}%
\pgfpathlineto{\pgfqpoint{3.967556in}{0.453578in}}%
\pgfpathlineto{\pgfqpoint{3.995255in}{0.453578in}}%
\pgfpathlineto{\pgfqpoint{4.022954in}{0.453578in}}%
\pgfpathlineto{\pgfqpoint{4.033088in}{0.453578in}}%
\pgfusepath{stroke}%
\end{pgfscope}%
\begin{pgfscope}%
\pgfpathrectangle{\pgfqpoint{2.666838in}{0.383578in}}{\pgfqpoint{1.356250in}{1.540000in}}%
\pgfusepath{clip}%
\pgfsetrectcap%
\pgfsetroundjoin%
\pgfsetlinewidth{0.803000pt}%
\definecolor{currentstroke}{rgb}{0.333333,0.333333,0.333333}%
\pgfsetstrokecolor{currentstroke}%
\pgfsetstrokeopacity{0.300000}%
\pgfsetdash{}{0pt}%
\pgfpathmoveto{\pgfqpoint{2.721088in}{0.453578in}}%
\pgfpathlineto{\pgfqpoint{2.748787in}{0.453578in}}%
\pgfpathlineto{\pgfqpoint{2.776486in}{0.453578in}}%
\pgfpathlineto{\pgfqpoint{2.804186in}{0.453578in}}%
\pgfpathlineto{\pgfqpoint{2.831885in}{0.453578in}}%
\pgfpathlineto{\pgfqpoint{2.859584in}{0.453578in}}%
\pgfpathlineto{\pgfqpoint{2.887283in}{0.453578in}}%
\pgfpathlineto{\pgfqpoint{2.914983in}{0.453578in}}%
\pgfpathlineto{\pgfqpoint{2.942682in}{0.453578in}}%
\pgfpathlineto{\pgfqpoint{2.970381in}{0.453578in}}%
\pgfpathlineto{\pgfqpoint{2.998081in}{0.453578in}}%
\pgfpathlineto{\pgfqpoint{3.025780in}{0.453578in}}%
\pgfpathlineto{\pgfqpoint{3.053479in}{0.453578in}}%
\pgfpathlineto{\pgfqpoint{3.081178in}{0.453578in}}%
\pgfpathlineto{\pgfqpoint{3.108878in}{0.453578in}}%
\pgfpathlineto{\pgfqpoint{3.136577in}{0.453578in}}%
\pgfpathlineto{\pgfqpoint{3.164276in}{0.453578in}}%
\pgfpathlineto{\pgfqpoint{3.191976in}{0.453578in}}%
\pgfpathlineto{\pgfqpoint{3.219675in}{0.453578in}}%
\pgfpathlineto{\pgfqpoint{3.247374in}{0.453578in}}%
\pgfpathlineto{\pgfqpoint{3.275073in}{0.453578in}}%
\pgfpathlineto{\pgfqpoint{3.302773in}{0.453578in}}%
\pgfpathlineto{\pgfqpoint{3.330472in}{0.453578in}}%
\pgfpathlineto{\pgfqpoint{3.358171in}{0.453578in}}%
\pgfpathlineto{\pgfqpoint{3.385871in}{0.453578in}}%
\pgfpathlineto{\pgfqpoint{3.413570in}{0.453578in}}%
\pgfpathlineto{\pgfqpoint{3.441269in}{0.453578in}}%
\pgfpathlineto{\pgfqpoint{3.468968in}{0.453578in}}%
\pgfpathlineto{\pgfqpoint{3.496668in}{0.453578in}}%
\pgfpathlineto{\pgfqpoint{3.524367in}{0.453578in}}%
\pgfpathlineto{\pgfqpoint{3.552066in}{0.453578in}}%
\pgfpathlineto{\pgfqpoint{3.579766in}{0.453578in}}%
\pgfpathlineto{\pgfqpoint{3.607465in}{0.453578in}}%
\pgfpathlineto{\pgfqpoint{3.635164in}{0.453578in}}%
\pgfpathlineto{\pgfqpoint{3.662863in}{0.453578in}}%
\pgfpathlineto{\pgfqpoint{3.690563in}{0.453578in}}%
\pgfpathlineto{\pgfqpoint{3.718262in}{0.453578in}}%
\pgfpathlineto{\pgfqpoint{3.745961in}{0.453578in}}%
\pgfpathlineto{\pgfqpoint{3.773661in}{0.453578in}}%
\pgfpathlineto{\pgfqpoint{3.801360in}{0.453578in}}%
\pgfpathlineto{\pgfqpoint{3.829059in}{0.453578in}}%
\pgfpathlineto{\pgfqpoint{3.856758in}{0.453578in}}%
\pgfpathlineto{\pgfqpoint{3.884458in}{0.453578in}}%
\pgfpathlineto{\pgfqpoint{3.912157in}{0.453578in}}%
\pgfpathlineto{\pgfqpoint{3.939856in}{0.453578in}}%
\pgfpathlineto{\pgfqpoint{3.967556in}{0.453578in}}%
\pgfpathlineto{\pgfqpoint{3.995255in}{0.453578in}}%
\pgfpathlineto{\pgfqpoint{4.022954in}{0.453578in}}%
\pgfpathlineto{\pgfqpoint{4.033088in}{0.453578in}}%
\pgfusepath{stroke}%
\end{pgfscope}%
\begin{pgfscope}%
\pgfpathrectangle{\pgfqpoint{2.666838in}{0.383578in}}{\pgfqpoint{1.356250in}{1.540000in}}%
\pgfusepath{clip}%
\pgfsetrectcap%
\pgfsetroundjoin%
\pgfsetlinewidth{0.803000pt}%
\definecolor{currentstroke}{rgb}{0.333333,0.333333,0.333333}%
\pgfsetstrokecolor{currentstroke}%
\pgfsetstrokeopacity{0.300000}%
\pgfsetdash{}{0pt}%
\pgfpathmoveto{\pgfqpoint{2.721088in}{0.453578in}}%
\pgfpathlineto{\pgfqpoint{2.748787in}{0.453578in}}%
\pgfpathlineto{\pgfqpoint{2.776486in}{0.453578in}}%
\pgfpathlineto{\pgfqpoint{2.804186in}{0.453578in}}%
\pgfpathlineto{\pgfqpoint{2.831885in}{0.453578in}}%
\pgfpathlineto{\pgfqpoint{2.859584in}{0.453578in}}%
\pgfpathlineto{\pgfqpoint{2.887283in}{0.453578in}}%
\pgfpathlineto{\pgfqpoint{2.914983in}{0.453578in}}%
\pgfpathlineto{\pgfqpoint{2.942682in}{0.453578in}}%
\pgfpathlineto{\pgfqpoint{2.970381in}{0.453578in}}%
\pgfpathlineto{\pgfqpoint{2.998081in}{0.453578in}}%
\pgfpathlineto{\pgfqpoint{3.025780in}{0.453578in}}%
\pgfpathlineto{\pgfqpoint{3.053479in}{0.453578in}}%
\pgfpathlineto{\pgfqpoint{3.081178in}{0.453578in}}%
\pgfpathlineto{\pgfqpoint{3.108878in}{0.453578in}}%
\pgfpathlineto{\pgfqpoint{3.136577in}{0.453578in}}%
\pgfpathlineto{\pgfqpoint{3.164276in}{0.453578in}}%
\pgfpathlineto{\pgfqpoint{3.191976in}{0.453578in}}%
\pgfpathlineto{\pgfqpoint{3.219675in}{0.453578in}}%
\pgfpathlineto{\pgfqpoint{3.247374in}{0.453578in}}%
\pgfpathlineto{\pgfqpoint{3.275073in}{0.453578in}}%
\pgfpathlineto{\pgfqpoint{3.302773in}{0.453578in}}%
\pgfpathlineto{\pgfqpoint{3.330472in}{0.453578in}}%
\pgfpathlineto{\pgfqpoint{3.358171in}{0.453578in}}%
\pgfpathlineto{\pgfqpoint{3.385871in}{0.453578in}}%
\pgfpathlineto{\pgfqpoint{3.413570in}{0.453578in}}%
\pgfpathlineto{\pgfqpoint{3.441269in}{0.453578in}}%
\pgfpathlineto{\pgfqpoint{3.468968in}{0.453578in}}%
\pgfpathlineto{\pgfqpoint{3.496668in}{0.453578in}}%
\pgfpathlineto{\pgfqpoint{3.524367in}{0.453578in}}%
\pgfpathlineto{\pgfqpoint{3.552066in}{0.453578in}}%
\pgfpathlineto{\pgfqpoint{3.579766in}{0.453578in}}%
\pgfpathlineto{\pgfqpoint{3.607465in}{0.453578in}}%
\pgfpathlineto{\pgfqpoint{3.635164in}{0.453578in}}%
\pgfpathlineto{\pgfqpoint{3.662863in}{0.453578in}}%
\pgfpathlineto{\pgfqpoint{3.690563in}{0.453578in}}%
\pgfpathlineto{\pgfqpoint{3.718262in}{0.453578in}}%
\pgfpathlineto{\pgfqpoint{3.745961in}{0.453578in}}%
\pgfpathlineto{\pgfqpoint{3.773661in}{0.453578in}}%
\pgfpathlineto{\pgfqpoint{3.801360in}{0.453578in}}%
\pgfpathlineto{\pgfqpoint{3.829059in}{0.453578in}}%
\pgfpathlineto{\pgfqpoint{3.856758in}{0.453578in}}%
\pgfpathlineto{\pgfqpoint{3.884458in}{0.453578in}}%
\pgfpathlineto{\pgfqpoint{3.912157in}{0.453578in}}%
\pgfpathlineto{\pgfqpoint{3.939856in}{0.453578in}}%
\pgfpathlineto{\pgfqpoint{3.967556in}{0.453578in}}%
\pgfpathlineto{\pgfqpoint{3.995255in}{0.453578in}}%
\pgfpathlineto{\pgfqpoint{4.022954in}{0.453578in}}%
\pgfpathlineto{\pgfqpoint{4.033088in}{0.453578in}}%
\pgfusepath{stroke}%
\end{pgfscope}%
\begin{pgfscope}%
\pgfpathrectangle{\pgfqpoint{2.666838in}{0.383578in}}{\pgfqpoint{1.356250in}{1.540000in}}%
\pgfusepath{clip}%
\pgfsetrectcap%
\pgfsetroundjoin%
\pgfsetlinewidth{0.803000pt}%
\definecolor{currentstroke}{rgb}{0.333333,0.333333,0.333333}%
\pgfsetstrokecolor{currentstroke}%
\pgfsetstrokeopacity{0.300000}%
\pgfsetdash{}{0pt}%
\pgfpathmoveto{\pgfqpoint{2.721088in}{0.453578in}}%
\pgfpathlineto{\pgfqpoint{2.748787in}{0.453578in}}%
\pgfpathlineto{\pgfqpoint{2.776486in}{0.453578in}}%
\pgfpathlineto{\pgfqpoint{2.804186in}{0.453578in}}%
\pgfpathlineto{\pgfqpoint{2.831885in}{0.453578in}}%
\pgfpathlineto{\pgfqpoint{2.859584in}{0.453578in}}%
\pgfpathlineto{\pgfqpoint{2.887283in}{0.453578in}}%
\pgfpathlineto{\pgfqpoint{2.914983in}{0.517571in}}%
\pgfpathlineto{\pgfqpoint{2.942682in}{0.624774in}}%
\pgfpathlineto{\pgfqpoint{2.970381in}{0.750212in}}%
\pgfpathlineto{\pgfqpoint{2.998081in}{0.889175in}}%
\pgfpathlineto{\pgfqpoint{3.025780in}{1.039735in}}%
\pgfpathlineto{\pgfqpoint{3.053479in}{1.204113in}}%
\pgfpathlineto{\pgfqpoint{3.081178in}{1.342332in}}%
\pgfpathlineto{\pgfqpoint{3.108878in}{1.438273in}}%
\pgfpathlineto{\pgfqpoint{3.136577in}{1.491542in}}%
\pgfpathlineto{\pgfqpoint{3.164276in}{1.504723in}}%
\pgfpathlineto{\pgfqpoint{3.191976in}{1.489446in}}%
\pgfpathlineto{\pgfqpoint{3.219675in}{1.456237in}}%
\pgfpathlineto{\pgfqpoint{3.247374in}{1.414864in}}%
\pgfpathlineto{\pgfqpoint{3.275073in}{1.375802in}}%
\pgfpathlineto{\pgfqpoint{3.302773in}{1.344065in}}%
\pgfpathlineto{\pgfqpoint{3.330472in}{1.323381in}}%
\pgfpathlineto{\pgfqpoint{3.358171in}{1.313780in}}%
\pgfpathlineto{\pgfqpoint{3.385871in}{1.309816in}}%
\pgfpathlineto{\pgfqpoint{3.413570in}{1.309893in}}%
\pgfpathlineto{\pgfqpoint{3.441269in}{1.310439in}}%
\pgfpathlineto{\pgfqpoint{3.468968in}{1.309107in}}%
\pgfpathlineto{\pgfqpoint{3.496668in}{1.304463in}}%
\pgfpathlineto{\pgfqpoint{3.524367in}{1.294244in}}%
\pgfpathlineto{\pgfqpoint{3.552066in}{1.277585in}}%
\pgfpathlineto{\pgfqpoint{3.579766in}{1.256776in}}%
\pgfpathlineto{\pgfqpoint{3.607465in}{1.232299in}}%
\pgfpathlineto{\pgfqpoint{3.635164in}{1.203793in}}%
\pgfpathlineto{\pgfqpoint{3.662863in}{1.172251in}}%
\pgfpathlineto{\pgfqpoint{3.690563in}{1.137854in}}%
\pgfpathlineto{\pgfqpoint{3.718262in}{1.100009in}}%
\pgfpathlineto{\pgfqpoint{3.745961in}{1.060358in}}%
\pgfpathlineto{\pgfqpoint{3.773661in}{1.019722in}}%
\pgfpathlineto{\pgfqpoint{3.801360in}{0.978846in}}%
\pgfpathlineto{\pgfqpoint{3.829059in}{0.938105in}}%
\pgfpathlineto{\pgfqpoint{3.856758in}{0.898178in}}%
\pgfpathlineto{\pgfqpoint{3.884458in}{0.859778in}}%
\pgfpathlineto{\pgfqpoint{3.912157in}{0.823360in}}%
\pgfpathlineto{\pgfqpoint{3.939856in}{0.788901in}}%
\pgfpathlineto{\pgfqpoint{3.967556in}{0.756551in}}%
\pgfpathlineto{\pgfqpoint{3.995255in}{0.726395in}}%
\pgfpathlineto{\pgfqpoint{4.022954in}{0.698463in}}%
\pgfpathlineto{\pgfqpoint{4.033088in}{0.689051in}}%
\pgfusepath{stroke}%
\end{pgfscope}%
\begin{pgfscope}%
\pgfpathrectangle{\pgfqpoint{2.666838in}{0.383578in}}{\pgfqpoint{1.356250in}{1.540000in}}%
\pgfusepath{clip}%
\pgfsetrectcap%
\pgfsetroundjoin%
\pgfsetlinewidth{0.803000pt}%
\definecolor{currentstroke}{rgb}{0.333333,0.333333,0.333333}%
\pgfsetstrokecolor{currentstroke}%
\pgfsetstrokeopacity{0.300000}%
\pgfsetdash{}{0pt}%
\pgfpathmoveto{\pgfqpoint{2.721088in}{0.453578in}}%
\pgfpathlineto{\pgfqpoint{2.748787in}{0.453578in}}%
\pgfpathlineto{\pgfqpoint{2.776486in}{0.453578in}}%
\pgfpathlineto{\pgfqpoint{2.804186in}{0.453578in}}%
\pgfpathlineto{\pgfqpoint{2.831885in}{0.453578in}}%
\pgfpathlineto{\pgfqpoint{2.859584in}{0.453578in}}%
\pgfpathlineto{\pgfqpoint{2.887283in}{0.453578in}}%
\pgfpathlineto{\pgfqpoint{2.914983in}{0.453578in}}%
\pgfpathlineto{\pgfqpoint{2.942682in}{0.453578in}}%
\pgfpathlineto{\pgfqpoint{2.970381in}{0.453578in}}%
\pgfpathlineto{\pgfqpoint{2.998081in}{0.453578in}}%
\pgfpathlineto{\pgfqpoint{3.025780in}{0.453578in}}%
\pgfpathlineto{\pgfqpoint{3.053479in}{0.453578in}}%
\pgfpathlineto{\pgfqpoint{3.081178in}{0.453578in}}%
\pgfpathlineto{\pgfqpoint{3.108878in}{0.453578in}}%
\pgfpathlineto{\pgfqpoint{3.136577in}{0.453578in}}%
\pgfpathlineto{\pgfqpoint{3.164276in}{0.453578in}}%
\pgfpathlineto{\pgfqpoint{3.191976in}{0.453578in}}%
\pgfpathlineto{\pgfqpoint{3.219675in}{0.453578in}}%
\pgfpathlineto{\pgfqpoint{3.247374in}{0.453578in}}%
\pgfpathlineto{\pgfqpoint{3.275073in}{0.453578in}}%
\pgfpathlineto{\pgfqpoint{3.302773in}{0.453578in}}%
\pgfpathlineto{\pgfqpoint{3.330472in}{0.453578in}}%
\pgfpathlineto{\pgfqpoint{3.358171in}{0.453578in}}%
\pgfpathlineto{\pgfqpoint{3.385871in}{0.453578in}}%
\pgfpathlineto{\pgfqpoint{3.413570in}{0.453578in}}%
\pgfpathlineto{\pgfqpoint{3.441269in}{0.453578in}}%
\pgfpathlineto{\pgfqpoint{3.468968in}{0.453578in}}%
\pgfpathlineto{\pgfqpoint{3.496668in}{0.453578in}}%
\pgfpathlineto{\pgfqpoint{3.524367in}{0.453578in}}%
\pgfpathlineto{\pgfqpoint{3.552066in}{0.453578in}}%
\pgfpathlineto{\pgfqpoint{3.579766in}{0.453578in}}%
\pgfpathlineto{\pgfqpoint{3.607465in}{0.453578in}}%
\pgfpathlineto{\pgfqpoint{3.635164in}{0.453578in}}%
\pgfpathlineto{\pgfqpoint{3.662863in}{0.453578in}}%
\pgfpathlineto{\pgfqpoint{3.690563in}{0.453578in}}%
\pgfpathlineto{\pgfqpoint{3.718262in}{0.453578in}}%
\pgfpathlineto{\pgfqpoint{3.745961in}{0.453578in}}%
\pgfpathlineto{\pgfqpoint{3.773661in}{0.453578in}}%
\pgfpathlineto{\pgfqpoint{3.801360in}{0.453578in}}%
\pgfpathlineto{\pgfqpoint{3.829059in}{0.453578in}}%
\pgfpathlineto{\pgfqpoint{3.856758in}{0.453578in}}%
\pgfpathlineto{\pgfqpoint{3.884458in}{0.453578in}}%
\pgfpathlineto{\pgfqpoint{3.912157in}{0.453578in}}%
\pgfpathlineto{\pgfqpoint{3.939856in}{0.453578in}}%
\pgfpathlineto{\pgfqpoint{3.967556in}{0.453578in}}%
\pgfpathlineto{\pgfqpoint{3.995255in}{0.453578in}}%
\pgfpathlineto{\pgfqpoint{4.022954in}{0.453578in}}%
\pgfpathlineto{\pgfqpoint{4.033088in}{0.453578in}}%
\pgfusepath{stroke}%
\end{pgfscope}%
\begin{pgfscope}%
\pgfpathrectangle{\pgfqpoint{2.666838in}{0.383578in}}{\pgfqpoint{1.356250in}{1.540000in}}%
\pgfusepath{clip}%
\pgfsetrectcap%
\pgfsetroundjoin%
\pgfsetlinewidth{0.803000pt}%
\definecolor{currentstroke}{rgb}{0.333333,0.333333,0.333333}%
\pgfsetstrokecolor{currentstroke}%
\pgfsetstrokeopacity{0.300000}%
\pgfsetdash{}{0pt}%
\pgfpathmoveto{\pgfqpoint{2.721088in}{0.453578in}}%
\pgfpathlineto{\pgfqpoint{2.748787in}{0.453578in}}%
\pgfpathlineto{\pgfqpoint{2.776486in}{0.453578in}}%
\pgfpathlineto{\pgfqpoint{2.804186in}{0.453578in}}%
\pgfpathlineto{\pgfqpoint{2.831885in}{0.453578in}}%
\pgfpathlineto{\pgfqpoint{2.859584in}{0.453578in}}%
\pgfpathlineto{\pgfqpoint{2.887283in}{0.453578in}}%
\pgfpathlineto{\pgfqpoint{2.914983in}{0.453578in}}%
\pgfpathlineto{\pgfqpoint{2.942682in}{0.453578in}}%
\pgfpathlineto{\pgfqpoint{2.970381in}{0.453578in}}%
\pgfpathlineto{\pgfqpoint{2.998081in}{0.453578in}}%
\pgfpathlineto{\pgfqpoint{3.025780in}{0.453578in}}%
\pgfpathlineto{\pgfqpoint{3.053479in}{0.453578in}}%
\pgfpathlineto{\pgfqpoint{3.081178in}{0.453578in}}%
\pgfpathlineto{\pgfqpoint{3.108878in}{0.453578in}}%
\pgfpathlineto{\pgfqpoint{3.136577in}{0.453578in}}%
\pgfpathlineto{\pgfqpoint{3.164276in}{0.453578in}}%
\pgfpathlineto{\pgfqpoint{3.191976in}{0.453578in}}%
\pgfpathlineto{\pgfqpoint{3.219675in}{0.453578in}}%
\pgfpathlineto{\pgfqpoint{3.247374in}{0.453578in}}%
\pgfpathlineto{\pgfqpoint{3.275073in}{0.453578in}}%
\pgfpathlineto{\pgfqpoint{3.302773in}{0.453578in}}%
\pgfpathlineto{\pgfqpoint{3.330472in}{0.453578in}}%
\pgfpathlineto{\pgfqpoint{3.358171in}{0.453578in}}%
\pgfpathlineto{\pgfqpoint{3.385871in}{0.453578in}}%
\pgfpathlineto{\pgfqpoint{3.413570in}{0.453578in}}%
\pgfpathlineto{\pgfqpoint{3.441269in}{0.453578in}}%
\pgfpathlineto{\pgfqpoint{3.468968in}{0.453578in}}%
\pgfpathlineto{\pgfqpoint{3.496668in}{0.453578in}}%
\pgfpathlineto{\pgfqpoint{3.524367in}{0.453578in}}%
\pgfpathlineto{\pgfqpoint{3.552066in}{0.453578in}}%
\pgfpathlineto{\pgfqpoint{3.579766in}{0.453578in}}%
\pgfpathlineto{\pgfqpoint{3.607465in}{0.453578in}}%
\pgfpathlineto{\pgfqpoint{3.635164in}{0.453578in}}%
\pgfpathlineto{\pgfqpoint{3.662863in}{0.453578in}}%
\pgfpathlineto{\pgfqpoint{3.690563in}{0.453578in}}%
\pgfpathlineto{\pgfqpoint{3.718262in}{0.453578in}}%
\pgfpathlineto{\pgfqpoint{3.745961in}{0.453578in}}%
\pgfpathlineto{\pgfqpoint{3.773661in}{0.453578in}}%
\pgfpathlineto{\pgfqpoint{3.801360in}{0.453578in}}%
\pgfpathlineto{\pgfqpoint{3.829059in}{0.453578in}}%
\pgfpathlineto{\pgfqpoint{3.856758in}{0.453578in}}%
\pgfpathlineto{\pgfqpoint{3.884458in}{0.453578in}}%
\pgfpathlineto{\pgfqpoint{3.912157in}{0.453578in}}%
\pgfpathlineto{\pgfqpoint{3.939856in}{0.453578in}}%
\pgfpathlineto{\pgfqpoint{3.967556in}{0.453578in}}%
\pgfpathlineto{\pgfqpoint{3.995255in}{0.453578in}}%
\pgfpathlineto{\pgfqpoint{4.022954in}{0.453578in}}%
\pgfpathlineto{\pgfqpoint{4.033088in}{0.453578in}}%
\pgfusepath{stroke}%
\end{pgfscope}%
\begin{pgfscope}%
\pgfpathrectangle{\pgfqpoint{2.666838in}{0.383578in}}{\pgfqpoint{1.356250in}{1.540000in}}%
\pgfusepath{clip}%
\pgfsetrectcap%
\pgfsetroundjoin%
\pgfsetlinewidth{0.803000pt}%
\definecolor{currentstroke}{rgb}{0.333333,0.333333,0.333333}%
\pgfsetstrokecolor{currentstroke}%
\pgfsetstrokeopacity{0.300000}%
\pgfsetdash{}{0pt}%
\pgfpathmoveto{\pgfqpoint{2.721088in}{0.453578in}}%
\pgfpathlineto{\pgfqpoint{2.748787in}{0.453578in}}%
\pgfpathlineto{\pgfqpoint{2.776486in}{0.453578in}}%
\pgfpathlineto{\pgfqpoint{2.804186in}{0.453578in}}%
\pgfpathlineto{\pgfqpoint{2.831885in}{0.453578in}}%
\pgfpathlineto{\pgfqpoint{2.859584in}{0.453578in}}%
\pgfpathlineto{\pgfqpoint{2.887283in}{0.453578in}}%
\pgfpathlineto{\pgfqpoint{2.914983in}{0.453578in}}%
\pgfpathlineto{\pgfqpoint{2.942682in}{0.453578in}}%
\pgfpathlineto{\pgfqpoint{2.970381in}{0.453578in}}%
\pgfpathlineto{\pgfqpoint{2.998081in}{0.453578in}}%
\pgfpathlineto{\pgfqpoint{3.025780in}{0.453578in}}%
\pgfpathlineto{\pgfqpoint{3.053479in}{0.453578in}}%
\pgfpathlineto{\pgfqpoint{3.081178in}{0.453578in}}%
\pgfpathlineto{\pgfqpoint{3.108878in}{0.453578in}}%
\pgfpathlineto{\pgfqpoint{3.136577in}{0.453578in}}%
\pgfpathlineto{\pgfqpoint{3.164276in}{0.453578in}}%
\pgfpathlineto{\pgfqpoint{3.191976in}{0.453578in}}%
\pgfpathlineto{\pgfqpoint{3.219675in}{0.453578in}}%
\pgfpathlineto{\pgfqpoint{3.247374in}{0.453578in}}%
\pgfpathlineto{\pgfqpoint{3.275073in}{0.453578in}}%
\pgfpathlineto{\pgfqpoint{3.302773in}{0.453578in}}%
\pgfpathlineto{\pgfqpoint{3.330472in}{0.453578in}}%
\pgfpathlineto{\pgfqpoint{3.358171in}{0.453578in}}%
\pgfpathlineto{\pgfqpoint{3.385871in}{0.453578in}}%
\pgfpathlineto{\pgfqpoint{3.413570in}{0.453578in}}%
\pgfpathlineto{\pgfqpoint{3.441269in}{0.453578in}}%
\pgfpathlineto{\pgfqpoint{3.468968in}{0.453578in}}%
\pgfpathlineto{\pgfqpoint{3.496668in}{0.453578in}}%
\pgfpathlineto{\pgfqpoint{3.524367in}{0.453578in}}%
\pgfpathlineto{\pgfqpoint{3.552066in}{0.453578in}}%
\pgfpathlineto{\pgfqpoint{3.579766in}{0.453578in}}%
\pgfpathlineto{\pgfqpoint{3.607465in}{0.453578in}}%
\pgfpathlineto{\pgfqpoint{3.635164in}{0.453578in}}%
\pgfpathlineto{\pgfqpoint{3.662863in}{0.453578in}}%
\pgfpathlineto{\pgfqpoint{3.690563in}{0.453578in}}%
\pgfpathlineto{\pgfqpoint{3.718262in}{0.453578in}}%
\pgfpathlineto{\pgfqpoint{3.745961in}{0.453578in}}%
\pgfpathlineto{\pgfqpoint{3.773661in}{0.453578in}}%
\pgfpathlineto{\pgfqpoint{3.801360in}{0.453578in}}%
\pgfpathlineto{\pgfqpoint{3.829059in}{0.453578in}}%
\pgfpathlineto{\pgfqpoint{3.856758in}{0.453578in}}%
\pgfpathlineto{\pgfqpoint{3.884458in}{0.453578in}}%
\pgfpathlineto{\pgfqpoint{3.912157in}{0.453578in}}%
\pgfpathlineto{\pgfqpoint{3.939856in}{0.453578in}}%
\pgfpathlineto{\pgfqpoint{3.967556in}{0.453578in}}%
\pgfpathlineto{\pgfqpoint{3.995255in}{0.453578in}}%
\pgfpathlineto{\pgfqpoint{4.022954in}{0.453578in}}%
\pgfpathlineto{\pgfqpoint{4.033088in}{0.453578in}}%
\pgfusepath{stroke}%
\end{pgfscope}%
\begin{pgfscope}%
\pgfpathrectangle{\pgfqpoint{2.666838in}{0.383578in}}{\pgfqpoint{1.356250in}{1.540000in}}%
\pgfusepath{clip}%
\pgfsetrectcap%
\pgfsetroundjoin%
\pgfsetlinewidth{0.803000pt}%
\definecolor{currentstroke}{rgb}{0.333333,0.333333,0.333333}%
\pgfsetstrokecolor{currentstroke}%
\pgfsetstrokeopacity{0.300000}%
\pgfsetdash{}{0pt}%
\pgfpathmoveto{\pgfqpoint{2.721088in}{0.453578in}}%
\pgfpathlineto{\pgfqpoint{2.748787in}{0.453578in}}%
\pgfpathlineto{\pgfqpoint{2.776486in}{0.453578in}}%
\pgfpathlineto{\pgfqpoint{2.804186in}{0.453578in}}%
\pgfpathlineto{\pgfqpoint{2.831885in}{0.453578in}}%
\pgfpathlineto{\pgfqpoint{2.859584in}{0.453578in}}%
\pgfpathlineto{\pgfqpoint{2.887283in}{0.453578in}}%
\pgfpathlineto{\pgfqpoint{2.914983in}{0.453578in}}%
\pgfpathlineto{\pgfqpoint{2.942682in}{0.453578in}}%
\pgfpathlineto{\pgfqpoint{2.970381in}{0.453578in}}%
\pgfpathlineto{\pgfqpoint{2.998081in}{0.453578in}}%
\pgfpathlineto{\pgfqpoint{3.025780in}{0.453578in}}%
\pgfpathlineto{\pgfqpoint{3.053479in}{0.453578in}}%
\pgfpathlineto{\pgfqpoint{3.081178in}{0.453578in}}%
\pgfpathlineto{\pgfqpoint{3.108878in}{0.453578in}}%
\pgfpathlineto{\pgfqpoint{3.136577in}{0.453578in}}%
\pgfpathlineto{\pgfqpoint{3.164276in}{0.453578in}}%
\pgfpathlineto{\pgfqpoint{3.191976in}{0.453578in}}%
\pgfpathlineto{\pgfqpoint{3.219675in}{0.453578in}}%
\pgfpathlineto{\pgfqpoint{3.247374in}{0.453578in}}%
\pgfpathlineto{\pgfqpoint{3.275073in}{0.453578in}}%
\pgfpathlineto{\pgfqpoint{3.302773in}{0.453578in}}%
\pgfpathlineto{\pgfqpoint{3.330472in}{0.453578in}}%
\pgfpathlineto{\pgfqpoint{3.358171in}{0.453578in}}%
\pgfpathlineto{\pgfqpoint{3.385871in}{0.453578in}}%
\pgfpathlineto{\pgfqpoint{3.413570in}{0.453578in}}%
\pgfpathlineto{\pgfqpoint{3.441269in}{0.453578in}}%
\pgfpathlineto{\pgfqpoint{3.468968in}{0.453578in}}%
\pgfpathlineto{\pgfqpoint{3.496668in}{0.453578in}}%
\pgfpathlineto{\pgfqpoint{3.524367in}{0.453578in}}%
\pgfpathlineto{\pgfqpoint{3.552066in}{0.453578in}}%
\pgfpathlineto{\pgfqpoint{3.579766in}{0.453578in}}%
\pgfpathlineto{\pgfqpoint{3.607465in}{0.453578in}}%
\pgfpathlineto{\pgfqpoint{3.635164in}{0.453578in}}%
\pgfpathlineto{\pgfqpoint{3.662863in}{0.453578in}}%
\pgfpathlineto{\pgfqpoint{3.690563in}{0.453578in}}%
\pgfpathlineto{\pgfqpoint{3.718262in}{0.453578in}}%
\pgfpathlineto{\pgfqpoint{3.745961in}{0.453578in}}%
\pgfpathlineto{\pgfqpoint{3.773661in}{0.453578in}}%
\pgfpathlineto{\pgfqpoint{3.801360in}{0.453578in}}%
\pgfpathlineto{\pgfqpoint{3.829059in}{0.453578in}}%
\pgfpathlineto{\pgfqpoint{3.856758in}{0.453578in}}%
\pgfpathlineto{\pgfqpoint{3.884458in}{0.453578in}}%
\pgfpathlineto{\pgfqpoint{3.912157in}{0.453578in}}%
\pgfpathlineto{\pgfqpoint{3.939856in}{0.453578in}}%
\pgfpathlineto{\pgfqpoint{3.967556in}{0.453578in}}%
\pgfpathlineto{\pgfqpoint{3.995255in}{0.453578in}}%
\pgfpathlineto{\pgfqpoint{4.022954in}{0.453578in}}%
\pgfpathlineto{\pgfqpoint{4.033088in}{0.453578in}}%
\pgfusepath{stroke}%
\end{pgfscope}%
\begin{pgfscope}%
\pgfpathrectangle{\pgfqpoint{2.666838in}{0.383578in}}{\pgfqpoint{1.356250in}{1.540000in}}%
\pgfusepath{clip}%
\pgfsetrectcap%
\pgfsetroundjoin%
\pgfsetlinewidth{0.803000pt}%
\definecolor{currentstroke}{rgb}{0.333333,0.333333,0.333333}%
\pgfsetstrokecolor{currentstroke}%
\pgfsetstrokeopacity{0.300000}%
\pgfsetdash{}{0pt}%
\pgfpathmoveto{\pgfqpoint{2.721088in}{0.453578in}}%
\pgfpathlineto{\pgfqpoint{2.748787in}{0.453578in}}%
\pgfpathlineto{\pgfqpoint{2.776486in}{0.453578in}}%
\pgfpathlineto{\pgfqpoint{2.804186in}{0.453578in}}%
\pgfpathlineto{\pgfqpoint{2.831885in}{0.453578in}}%
\pgfpathlineto{\pgfqpoint{2.859584in}{0.453578in}}%
\pgfpathlineto{\pgfqpoint{2.887283in}{0.453578in}}%
\pgfpathlineto{\pgfqpoint{2.914983in}{0.453578in}}%
\pgfpathlineto{\pgfqpoint{2.942682in}{0.453578in}}%
\pgfpathlineto{\pgfqpoint{2.970381in}{0.453578in}}%
\pgfpathlineto{\pgfqpoint{2.998081in}{0.453578in}}%
\pgfpathlineto{\pgfqpoint{3.025780in}{0.453578in}}%
\pgfpathlineto{\pgfqpoint{3.053479in}{0.453578in}}%
\pgfpathlineto{\pgfqpoint{3.081178in}{0.453578in}}%
\pgfpathlineto{\pgfqpoint{3.108878in}{0.453578in}}%
\pgfpathlineto{\pgfqpoint{3.136577in}{0.453578in}}%
\pgfpathlineto{\pgfqpoint{3.164276in}{0.453578in}}%
\pgfpathlineto{\pgfqpoint{3.191976in}{0.453578in}}%
\pgfpathlineto{\pgfqpoint{3.219675in}{0.453578in}}%
\pgfpathlineto{\pgfqpoint{3.247374in}{0.453578in}}%
\pgfpathlineto{\pgfqpoint{3.275073in}{0.453578in}}%
\pgfpathlineto{\pgfqpoint{3.302773in}{0.453578in}}%
\pgfpathlineto{\pgfqpoint{3.330472in}{0.453578in}}%
\pgfpathlineto{\pgfqpoint{3.358171in}{0.453578in}}%
\pgfpathlineto{\pgfqpoint{3.385871in}{0.453578in}}%
\pgfpathlineto{\pgfqpoint{3.413570in}{0.453578in}}%
\pgfpathlineto{\pgfqpoint{3.441269in}{0.453578in}}%
\pgfpathlineto{\pgfqpoint{3.468968in}{0.453578in}}%
\pgfpathlineto{\pgfqpoint{3.496668in}{0.453578in}}%
\pgfpathlineto{\pgfqpoint{3.524367in}{0.453578in}}%
\pgfpathlineto{\pgfqpoint{3.552066in}{0.453578in}}%
\pgfpathlineto{\pgfqpoint{3.579766in}{0.453578in}}%
\pgfpathlineto{\pgfqpoint{3.607465in}{0.453578in}}%
\pgfpathlineto{\pgfqpoint{3.635164in}{0.453578in}}%
\pgfpathlineto{\pgfqpoint{3.662863in}{0.453578in}}%
\pgfpathlineto{\pgfqpoint{3.690563in}{0.453578in}}%
\pgfpathlineto{\pgfqpoint{3.718262in}{0.453578in}}%
\pgfpathlineto{\pgfqpoint{3.745961in}{0.453578in}}%
\pgfpathlineto{\pgfqpoint{3.773661in}{0.453578in}}%
\pgfpathlineto{\pgfqpoint{3.801360in}{0.453578in}}%
\pgfpathlineto{\pgfqpoint{3.829059in}{0.453578in}}%
\pgfpathlineto{\pgfqpoint{3.856758in}{0.453578in}}%
\pgfpathlineto{\pgfqpoint{3.884458in}{0.453578in}}%
\pgfpathlineto{\pgfqpoint{3.912157in}{0.453578in}}%
\pgfpathlineto{\pgfqpoint{3.939856in}{0.453578in}}%
\pgfpathlineto{\pgfqpoint{3.967556in}{0.453578in}}%
\pgfpathlineto{\pgfqpoint{3.995255in}{0.453578in}}%
\pgfpathlineto{\pgfqpoint{4.022954in}{0.453578in}}%
\pgfpathlineto{\pgfqpoint{4.033088in}{0.453578in}}%
\pgfusepath{stroke}%
\end{pgfscope}%
\begin{pgfscope}%
\pgfpathrectangle{\pgfqpoint{2.666838in}{0.383578in}}{\pgfqpoint{1.356250in}{1.540000in}}%
\pgfusepath{clip}%
\pgfsetrectcap%
\pgfsetroundjoin%
\pgfsetlinewidth{0.803000pt}%
\definecolor{currentstroke}{rgb}{0.333333,0.333333,0.333333}%
\pgfsetstrokecolor{currentstroke}%
\pgfsetstrokeopacity{0.300000}%
\pgfsetdash{}{0pt}%
\pgfpathmoveto{\pgfqpoint{2.721088in}{0.453578in}}%
\pgfpathlineto{\pgfqpoint{2.748787in}{0.453578in}}%
\pgfpathlineto{\pgfqpoint{2.776486in}{0.453578in}}%
\pgfpathlineto{\pgfqpoint{2.804186in}{0.453578in}}%
\pgfpathlineto{\pgfqpoint{2.831885in}{0.453578in}}%
\pgfpathlineto{\pgfqpoint{2.859584in}{0.453578in}}%
\pgfpathlineto{\pgfqpoint{2.887283in}{0.453578in}}%
\pgfpathlineto{\pgfqpoint{2.914983in}{0.453578in}}%
\pgfpathlineto{\pgfqpoint{2.942682in}{0.453578in}}%
\pgfpathlineto{\pgfqpoint{2.970381in}{0.453578in}}%
\pgfpathlineto{\pgfqpoint{2.998081in}{0.453578in}}%
\pgfpathlineto{\pgfqpoint{3.025780in}{0.453578in}}%
\pgfpathlineto{\pgfqpoint{3.053479in}{0.453578in}}%
\pgfpathlineto{\pgfqpoint{3.081178in}{0.453578in}}%
\pgfpathlineto{\pgfqpoint{3.108878in}{0.453578in}}%
\pgfpathlineto{\pgfqpoint{3.136577in}{0.453578in}}%
\pgfpathlineto{\pgfqpoint{3.164276in}{0.453578in}}%
\pgfpathlineto{\pgfqpoint{3.191976in}{0.453578in}}%
\pgfpathlineto{\pgfqpoint{3.219675in}{0.453578in}}%
\pgfpathlineto{\pgfqpoint{3.247374in}{0.453578in}}%
\pgfpathlineto{\pgfqpoint{3.275073in}{0.453578in}}%
\pgfpathlineto{\pgfqpoint{3.302773in}{0.453578in}}%
\pgfpathlineto{\pgfqpoint{3.330472in}{0.453578in}}%
\pgfpathlineto{\pgfqpoint{3.358171in}{0.453578in}}%
\pgfpathlineto{\pgfqpoint{3.385871in}{0.453578in}}%
\pgfpathlineto{\pgfqpoint{3.413570in}{0.453578in}}%
\pgfpathlineto{\pgfqpoint{3.441269in}{0.453578in}}%
\pgfpathlineto{\pgfqpoint{3.468968in}{0.453578in}}%
\pgfpathlineto{\pgfqpoint{3.496668in}{0.453578in}}%
\pgfpathlineto{\pgfqpoint{3.524367in}{0.453578in}}%
\pgfpathlineto{\pgfqpoint{3.552066in}{0.453578in}}%
\pgfpathlineto{\pgfqpoint{3.579766in}{0.453578in}}%
\pgfpathlineto{\pgfqpoint{3.607465in}{0.453578in}}%
\pgfpathlineto{\pgfqpoint{3.635164in}{0.453578in}}%
\pgfpathlineto{\pgfqpoint{3.662863in}{0.453578in}}%
\pgfpathlineto{\pgfqpoint{3.690563in}{0.453578in}}%
\pgfpathlineto{\pgfqpoint{3.718262in}{0.453578in}}%
\pgfpathlineto{\pgfqpoint{3.745961in}{0.453578in}}%
\pgfpathlineto{\pgfqpoint{3.773661in}{0.453578in}}%
\pgfpathlineto{\pgfqpoint{3.801360in}{0.453578in}}%
\pgfpathlineto{\pgfqpoint{3.829059in}{0.453578in}}%
\pgfpathlineto{\pgfqpoint{3.856758in}{0.453578in}}%
\pgfpathlineto{\pgfqpoint{3.884458in}{0.453578in}}%
\pgfpathlineto{\pgfqpoint{3.912157in}{0.453578in}}%
\pgfpathlineto{\pgfqpoint{3.939856in}{0.453578in}}%
\pgfpathlineto{\pgfqpoint{3.967556in}{0.453578in}}%
\pgfpathlineto{\pgfqpoint{3.995255in}{0.453578in}}%
\pgfpathlineto{\pgfqpoint{4.022954in}{0.453578in}}%
\pgfpathlineto{\pgfqpoint{4.033088in}{0.453578in}}%
\pgfusepath{stroke}%
\end{pgfscope}%
\begin{pgfscope}%
\pgfpathrectangle{\pgfqpoint{2.666838in}{0.383578in}}{\pgfqpoint{1.356250in}{1.540000in}}%
\pgfusepath{clip}%
\pgfsetrectcap%
\pgfsetroundjoin%
\pgfsetlinewidth{0.803000pt}%
\definecolor{currentstroke}{rgb}{0.333333,0.333333,0.333333}%
\pgfsetstrokecolor{currentstroke}%
\pgfsetstrokeopacity{0.300000}%
\pgfsetdash{}{0pt}%
\pgfpathmoveto{\pgfqpoint{2.721088in}{0.453578in}}%
\pgfpathlineto{\pgfqpoint{2.748787in}{0.453578in}}%
\pgfpathlineto{\pgfqpoint{2.776486in}{0.453578in}}%
\pgfpathlineto{\pgfqpoint{2.804186in}{0.453578in}}%
\pgfpathlineto{\pgfqpoint{2.831885in}{0.453578in}}%
\pgfpathlineto{\pgfqpoint{2.859584in}{0.453578in}}%
\pgfpathlineto{\pgfqpoint{2.887283in}{0.453578in}}%
\pgfpathlineto{\pgfqpoint{2.914983in}{0.517571in}}%
\pgfpathlineto{\pgfqpoint{2.942682in}{0.624774in}}%
\pgfpathlineto{\pgfqpoint{2.970381in}{0.750212in}}%
\pgfpathlineto{\pgfqpoint{2.998081in}{0.889175in}}%
\pgfpathlineto{\pgfqpoint{3.025780in}{1.039735in}}%
\pgfpathlineto{\pgfqpoint{3.053479in}{1.204113in}}%
\pgfpathlineto{\pgfqpoint{3.081178in}{1.342332in}}%
\pgfpathlineto{\pgfqpoint{3.108878in}{1.438273in}}%
\pgfpathlineto{\pgfqpoint{3.136577in}{1.491542in}}%
\pgfpathlineto{\pgfqpoint{3.164276in}{1.504723in}}%
\pgfpathlineto{\pgfqpoint{3.191976in}{1.489446in}}%
\pgfpathlineto{\pgfqpoint{3.219675in}{1.456237in}}%
\pgfpathlineto{\pgfqpoint{3.247374in}{1.414864in}}%
\pgfpathlineto{\pgfqpoint{3.275073in}{1.375802in}}%
\pgfpathlineto{\pgfqpoint{3.302773in}{1.344065in}}%
\pgfpathlineto{\pgfqpoint{3.330472in}{1.323381in}}%
\pgfpathlineto{\pgfqpoint{3.358171in}{1.313780in}}%
\pgfpathlineto{\pgfqpoint{3.385871in}{1.309816in}}%
\pgfpathlineto{\pgfqpoint{3.413570in}{1.309893in}}%
\pgfpathlineto{\pgfqpoint{3.441269in}{1.310439in}}%
\pgfpathlineto{\pgfqpoint{3.468968in}{1.309107in}}%
\pgfpathlineto{\pgfqpoint{3.496668in}{1.304463in}}%
\pgfpathlineto{\pgfqpoint{3.524367in}{1.294244in}}%
\pgfpathlineto{\pgfqpoint{3.552066in}{1.277585in}}%
\pgfpathlineto{\pgfqpoint{3.579766in}{1.256776in}}%
\pgfpathlineto{\pgfqpoint{3.607465in}{1.232299in}}%
\pgfpathlineto{\pgfqpoint{3.635164in}{1.203793in}}%
\pgfpathlineto{\pgfqpoint{3.662863in}{1.172251in}}%
\pgfpathlineto{\pgfqpoint{3.690563in}{1.137854in}}%
\pgfpathlineto{\pgfqpoint{3.718262in}{1.100009in}}%
\pgfpathlineto{\pgfqpoint{3.745961in}{1.060358in}}%
\pgfpathlineto{\pgfqpoint{3.773661in}{1.019722in}}%
\pgfpathlineto{\pgfqpoint{3.801360in}{0.978846in}}%
\pgfpathlineto{\pgfqpoint{3.829059in}{0.938105in}}%
\pgfpathlineto{\pgfqpoint{3.856758in}{0.898178in}}%
\pgfpathlineto{\pgfqpoint{3.884458in}{0.859778in}}%
\pgfpathlineto{\pgfqpoint{3.912157in}{0.823360in}}%
\pgfpathlineto{\pgfqpoint{3.939856in}{0.788901in}}%
\pgfpathlineto{\pgfqpoint{3.967556in}{0.756551in}}%
\pgfpathlineto{\pgfqpoint{3.995255in}{0.726395in}}%
\pgfpathlineto{\pgfqpoint{4.022954in}{0.698463in}}%
\pgfpathlineto{\pgfqpoint{4.033088in}{0.689051in}}%
\pgfusepath{stroke}%
\end{pgfscope}%
\begin{pgfscope}%
\pgfpathrectangle{\pgfqpoint{2.666838in}{0.383578in}}{\pgfqpoint{1.356250in}{1.540000in}}%
\pgfusepath{clip}%
\pgfsetrectcap%
\pgfsetroundjoin%
\pgfsetlinewidth{0.803000pt}%
\definecolor{currentstroke}{rgb}{0.333333,0.333333,0.333333}%
\pgfsetstrokecolor{currentstroke}%
\pgfsetstrokeopacity{0.300000}%
\pgfsetdash{}{0pt}%
\pgfpathmoveto{\pgfqpoint{2.721088in}{0.453578in}}%
\pgfpathlineto{\pgfqpoint{2.748787in}{0.453578in}}%
\pgfpathlineto{\pgfqpoint{2.776486in}{0.453578in}}%
\pgfpathlineto{\pgfqpoint{2.804186in}{0.453578in}}%
\pgfpathlineto{\pgfqpoint{2.831885in}{0.453578in}}%
\pgfpathlineto{\pgfqpoint{2.859584in}{0.453578in}}%
\pgfpathlineto{\pgfqpoint{2.887283in}{0.453578in}}%
\pgfpathlineto{\pgfqpoint{2.914983in}{0.453578in}}%
\pgfpathlineto{\pgfqpoint{2.942682in}{0.453578in}}%
\pgfpathlineto{\pgfqpoint{2.970381in}{0.453578in}}%
\pgfpathlineto{\pgfqpoint{2.998081in}{0.453578in}}%
\pgfpathlineto{\pgfqpoint{3.025780in}{0.453578in}}%
\pgfpathlineto{\pgfqpoint{3.053479in}{0.453578in}}%
\pgfpathlineto{\pgfqpoint{3.081178in}{0.453578in}}%
\pgfpathlineto{\pgfqpoint{3.108878in}{0.453578in}}%
\pgfpathlineto{\pgfqpoint{3.136577in}{0.453578in}}%
\pgfpathlineto{\pgfqpoint{3.164276in}{0.453578in}}%
\pgfpathlineto{\pgfqpoint{3.191976in}{0.453578in}}%
\pgfpathlineto{\pgfqpoint{3.219675in}{0.453578in}}%
\pgfpathlineto{\pgfqpoint{3.247374in}{0.453578in}}%
\pgfpathlineto{\pgfqpoint{3.275073in}{0.453578in}}%
\pgfpathlineto{\pgfqpoint{3.302773in}{0.453578in}}%
\pgfpathlineto{\pgfqpoint{3.330472in}{0.453578in}}%
\pgfpathlineto{\pgfqpoint{3.358171in}{0.453578in}}%
\pgfpathlineto{\pgfqpoint{3.385871in}{0.453578in}}%
\pgfpathlineto{\pgfqpoint{3.413570in}{0.453578in}}%
\pgfpathlineto{\pgfqpoint{3.441269in}{0.453578in}}%
\pgfpathlineto{\pgfqpoint{3.468968in}{0.453578in}}%
\pgfpathlineto{\pgfqpoint{3.496668in}{0.453578in}}%
\pgfpathlineto{\pgfqpoint{3.524367in}{0.453578in}}%
\pgfpathlineto{\pgfqpoint{3.552066in}{0.453578in}}%
\pgfpathlineto{\pgfqpoint{3.579766in}{0.453578in}}%
\pgfpathlineto{\pgfqpoint{3.607465in}{0.453578in}}%
\pgfpathlineto{\pgfqpoint{3.635164in}{0.453578in}}%
\pgfpathlineto{\pgfqpoint{3.662863in}{0.453578in}}%
\pgfpathlineto{\pgfqpoint{3.690563in}{0.453578in}}%
\pgfpathlineto{\pgfqpoint{3.718262in}{0.453578in}}%
\pgfpathlineto{\pgfqpoint{3.745961in}{0.453578in}}%
\pgfpathlineto{\pgfqpoint{3.773661in}{0.453578in}}%
\pgfpathlineto{\pgfqpoint{3.801360in}{0.453578in}}%
\pgfpathlineto{\pgfqpoint{3.829059in}{0.453578in}}%
\pgfpathlineto{\pgfqpoint{3.856758in}{0.453578in}}%
\pgfpathlineto{\pgfqpoint{3.884458in}{0.453578in}}%
\pgfpathlineto{\pgfqpoint{3.912157in}{0.453578in}}%
\pgfpathlineto{\pgfqpoint{3.939856in}{0.453578in}}%
\pgfpathlineto{\pgfqpoint{3.967556in}{0.453578in}}%
\pgfpathlineto{\pgfqpoint{3.995255in}{0.453578in}}%
\pgfpathlineto{\pgfqpoint{4.022954in}{0.453578in}}%
\pgfpathlineto{\pgfqpoint{4.033088in}{0.453578in}}%
\pgfusepath{stroke}%
\end{pgfscope}%
\begin{pgfscope}%
\pgfpathrectangle{\pgfqpoint{2.666838in}{0.383578in}}{\pgfqpoint{1.356250in}{1.540000in}}%
\pgfusepath{clip}%
\pgfsetrectcap%
\pgfsetroundjoin%
\pgfsetlinewidth{0.803000pt}%
\definecolor{currentstroke}{rgb}{0.333333,0.333333,0.333333}%
\pgfsetstrokecolor{currentstroke}%
\pgfsetstrokeopacity{0.300000}%
\pgfsetdash{}{0pt}%
\pgfpathmoveto{\pgfqpoint{2.721088in}{0.453578in}}%
\pgfpathlineto{\pgfqpoint{2.748787in}{0.453578in}}%
\pgfpathlineto{\pgfqpoint{2.776486in}{0.453578in}}%
\pgfpathlineto{\pgfqpoint{2.804186in}{0.453578in}}%
\pgfpathlineto{\pgfqpoint{2.831885in}{0.453578in}}%
\pgfpathlineto{\pgfqpoint{2.859584in}{0.453578in}}%
\pgfpathlineto{\pgfqpoint{2.887283in}{0.453578in}}%
\pgfpathlineto{\pgfqpoint{2.914983in}{0.453578in}}%
\pgfpathlineto{\pgfqpoint{2.942682in}{0.453578in}}%
\pgfpathlineto{\pgfqpoint{2.970381in}{0.453578in}}%
\pgfpathlineto{\pgfqpoint{2.998081in}{0.453578in}}%
\pgfpathlineto{\pgfqpoint{3.025780in}{0.453578in}}%
\pgfpathlineto{\pgfqpoint{3.053479in}{0.453578in}}%
\pgfpathlineto{\pgfqpoint{3.081178in}{0.453578in}}%
\pgfpathlineto{\pgfqpoint{3.108878in}{0.453578in}}%
\pgfpathlineto{\pgfqpoint{3.136577in}{0.453578in}}%
\pgfpathlineto{\pgfqpoint{3.164276in}{0.453578in}}%
\pgfpathlineto{\pgfqpoint{3.191976in}{0.453578in}}%
\pgfpathlineto{\pgfqpoint{3.219675in}{0.453578in}}%
\pgfpathlineto{\pgfqpoint{3.247374in}{0.453578in}}%
\pgfpathlineto{\pgfqpoint{3.275073in}{0.453578in}}%
\pgfpathlineto{\pgfqpoint{3.302773in}{0.453578in}}%
\pgfpathlineto{\pgfqpoint{3.330472in}{0.453578in}}%
\pgfpathlineto{\pgfqpoint{3.358171in}{0.453578in}}%
\pgfpathlineto{\pgfqpoint{3.385871in}{0.453578in}}%
\pgfpathlineto{\pgfqpoint{3.413570in}{0.453578in}}%
\pgfpathlineto{\pgfqpoint{3.441269in}{0.453578in}}%
\pgfpathlineto{\pgfqpoint{3.468968in}{0.453578in}}%
\pgfpathlineto{\pgfqpoint{3.496668in}{0.453578in}}%
\pgfpathlineto{\pgfqpoint{3.524367in}{0.453578in}}%
\pgfpathlineto{\pgfqpoint{3.552066in}{0.453578in}}%
\pgfpathlineto{\pgfqpoint{3.579766in}{0.453578in}}%
\pgfpathlineto{\pgfqpoint{3.607465in}{0.453578in}}%
\pgfpathlineto{\pgfqpoint{3.635164in}{0.453578in}}%
\pgfpathlineto{\pgfqpoint{3.662863in}{0.453578in}}%
\pgfpathlineto{\pgfqpoint{3.690563in}{0.453578in}}%
\pgfpathlineto{\pgfqpoint{3.718262in}{0.453578in}}%
\pgfpathlineto{\pgfqpoint{3.745961in}{0.453578in}}%
\pgfpathlineto{\pgfqpoint{3.773661in}{0.453578in}}%
\pgfpathlineto{\pgfqpoint{3.801360in}{0.453578in}}%
\pgfpathlineto{\pgfqpoint{3.829059in}{0.453578in}}%
\pgfpathlineto{\pgfqpoint{3.856758in}{0.453578in}}%
\pgfpathlineto{\pgfqpoint{3.884458in}{0.453578in}}%
\pgfpathlineto{\pgfqpoint{3.912157in}{0.453578in}}%
\pgfpathlineto{\pgfqpoint{3.939856in}{0.453578in}}%
\pgfpathlineto{\pgfqpoint{3.967556in}{0.453578in}}%
\pgfpathlineto{\pgfqpoint{3.995255in}{0.453578in}}%
\pgfpathlineto{\pgfqpoint{4.022954in}{0.453578in}}%
\pgfpathlineto{\pgfqpoint{4.033088in}{0.453578in}}%
\pgfusepath{stroke}%
\end{pgfscope}%
\begin{pgfscope}%
\pgfpathrectangle{\pgfqpoint{2.666838in}{0.383578in}}{\pgfqpoint{1.356250in}{1.540000in}}%
\pgfusepath{clip}%
\pgfsetrectcap%
\pgfsetroundjoin%
\pgfsetlinewidth{0.803000pt}%
\definecolor{currentstroke}{rgb}{0.333333,0.333333,0.333333}%
\pgfsetstrokecolor{currentstroke}%
\pgfsetstrokeopacity{0.300000}%
\pgfsetdash{}{0pt}%
\pgfpathmoveto{\pgfqpoint{2.721088in}{0.453578in}}%
\pgfpathlineto{\pgfqpoint{2.748787in}{0.453578in}}%
\pgfpathlineto{\pgfqpoint{2.776486in}{0.453578in}}%
\pgfpathlineto{\pgfqpoint{2.804186in}{0.453578in}}%
\pgfpathlineto{\pgfqpoint{2.831885in}{0.453578in}}%
\pgfpathlineto{\pgfqpoint{2.859584in}{0.453578in}}%
\pgfpathlineto{\pgfqpoint{2.887283in}{0.453578in}}%
\pgfpathlineto{\pgfqpoint{2.914983in}{0.453578in}}%
\pgfpathlineto{\pgfqpoint{2.942682in}{0.453578in}}%
\pgfpathlineto{\pgfqpoint{2.970381in}{0.453578in}}%
\pgfpathlineto{\pgfqpoint{2.998081in}{0.453578in}}%
\pgfpathlineto{\pgfqpoint{3.025780in}{0.453578in}}%
\pgfpathlineto{\pgfqpoint{3.053479in}{0.453578in}}%
\pgfpathlineto{\pgfqpoint{3.081178in}{0.453578in}}%
\pgfpathlineto{\pgfqpoint{3.108878in}{0.453578in}}%
\pgfpathlineto{\pgfqpoint{3.136577in}{0.453578in}}%
\pgfpathlineto{\pgfqpoint{3.164276in}{0.453578in}}%
\pgfpathlineto{\pgfqpoint{3.191976in}{0.453578in}}%
\pgfpathlineto{\pgfqpoint{3.219675in}{0.453578in}}%
\pgfpathlineto{\pgfqpoint{3.247374in}{0.453578in}}%
\pgfpathlineto{\pgfqpoint{3.275073in}{0.453578in}}%
\pgfpathlineto{\pgfqpoint{3.302773in}{0.453578in}}%
\pgfpathlineto{\pgfqpoint{3.330472in}{0.453578in}}%
\pgfpathlineto{\pgfqpoint{3.358171in}{0.453578in}}%
\pgfpathlineto{\pgfqpoint{3.385871in}{0.453578in}}%
\pgfpathlineto{\pgfqpoint{3.413570in}{0.453578in}}%
\pgfpathlineto{\pgfqpoint{3.441269in}{0.453578in}}%
\pgfpathlineto{\pgfqpoint{3.468968in}{0.453578in}}%
\pgfpathlineto{\pgfqpoint{3.496668in}{0.453578in}}%
\pgfpathlineto{\pgfqpoint{3.524367in}{0.453578in}}%
\pgfpathlineto{\pgfqpoint{3.552066in}{0.453578in}}%
\pgfpathlineto{\pgfqpoint{3.579766in}{0.453578in}}%
\pgfpathlineto{\pgfqpoint{3.607465in}{0.453578in}}%
\pgfpathlineto{\pgfqpoint{3.635164in}{0.453578in}}%
\pgfpathlineto{\pgfqpoint{3.662863in}{0.453578in}}%
\pgfpathlineto{\pgfqpoint{3.690563in}{0.453578in}}%
\pgfpathlineto{\pgfqpoint{3.718262in}{0.453578in}}%
\pgfpathlineto{\pgfqpoint{3.745961in}{0.453578in}}%
\pgfpathlineto{\pgfqpoint{3.773661in}{0.453578in}}%
\pgfpathlineto{\pgfqpoint{3.801360in}{0.453578in}}%
\pgfpathlineto{\pgfqpoint{3.829059in}{0.453578in}}%
\pgfpathlineto{\pgfqpoint{3.856758in}{0.453578in}}%
\pgfpathlineto{\pgfqpoint{3.884458in}{0.453578in}}%
\pgfpathlineto{\pgfqpoint{3.912157in}{0.453578in}}%
\pgfpathlineto{\pgfqpoint{3.939856in}{0.453578in}}%
\pgfpathlineto{\pgfqpoint{3.967556in}{0.453578in}}%
\pgfpathlineto{\pgfqpoint{3.995255in}{0.453578in}}%
\pgfpathlineto{\pgfqpoint{4.022954in}{0.453578in}}%
\pgfpathlineto{\pgfqpoint{4.033088in}{0.453578in}}%
\pgfusepath{stroke}%
\end{pgfscope}%
\begin{pgfscope}%
\pgfpathrectangle{\pgfqpoint{2.666838in}{0.383578in}}{\pgfqpoint{1.356250in}{1.540000in}}%
\pgfusepath{clip}%
\pgfsetrectcap%
\pgfsetroundjoin%
\pgfsetlinewidth{0.803000pt}%
\definecolor{currentstroke}{rgb}{0.333333,0.333333,0.333333}%
\pgfsetstrokecolor{currentstroke}%
\pgfsetstrokeopacity{0.300000}%
\pgfsetdash{}{0pt}%
\pgfpathmoveto{\pgfqpoint{2.721088in}{0.453578in}}%
\pgfpathlineto{\pgfqpoint{2.748787in}{0.453578in}}%
\pgfpathlineto{\pgfqpoint{2.776486in}{0.453578in}}%
\pgfpathlineto{\pgfqpoint{2.804186in}{0.453578in}}%
\pgfpathlineto{\pgfqpoint{2.831885in}{0.453578in}}%
\pgfpathlineto{\pgfqpoint{2.859584in}{0.453578in}}%
\pgfpathlineto{\pgfqpoint{2.887283in}{0.453578in}}%
\pgfpathlineto{\pgfqpoint{2.914983in}{0.453578in}}%
\pgfpathlineto{\pgfqpoint{2.942682in}{0.453578in}}%
\pgfpathlineto{\pgfqpoint{2.970381in}{0.453578in}}%
\pgfpathlineto{\pgfqpoint{2.998081in}{0.453578in}}%
\pgfpathlineto{\pgfqpoint{3.025780in}{0.453578in}}%
\pgfpathlineto{\pgfqpoint{3.053479in}{0.453578in}}%
\pgfpathlineto{\pgfqpoint{3.081178in}{0.453578in}}%
\pgfpathlineto{\pgfqpoint{3.108878in}{0.453578in}}%
\pgfpathlineto{\pgfqpoint{3.136577in}{0.453578in}}%
\pgfpathlineto{\pgfqpoint{3.164276in}{0.453578in}}%
\pgfpathlineto{\pgfqpoint{3.191976in}{0.453578in}}%
\pgfpathlineto{\pgfqpoint{3.219675in}{0.453578in}}%
\pgfpathlineto{\pgfqpoint{3.247374in}{0.453578in}}%
\pgfpathlineto{\pgfqpoint{3.275073in}{0.453578in}}%
\pgfpathlineto{\pgfqpoint{3.302773in}{0.453578in}}%
\pgfpathlineto{\pgfqpoint{3.330472in}{0.453578in}}%
\pgfpathlineto{\pgfqpoint{3.358171in}{0.453578in}}%
\pgfpathlineto{\pgfqpoint{3.385871in}{0.453578in}}%
\pgfpathlineto{\pgfqpoint{3.413570in}{0.453578in}}%
\pgfpathlineto{\pgfqpoint{3.441269in}{0.453578in}}%
\pgfpathlineto{\pgfqpoint{3.468968in}{0.453578in}}%
\pgfpathlineto{\pgfqpoint{3.496668in}{0.453578in}}%
\pgfpathlineto{\pgfqpoint{3.524367in}{0.453578in}}%
\pgfpathlineto{\pgfqpoint{3.552066in}{0.453578in}}%
\pgfpathlineto{\pgfqpoint{3.579766in}{0.453578in}}%
\pgfpathlineto{\pgfqpoint{3.607465in}{0.453578in}}%
\pgfpathlineto{\pgfqpoint{3.635164in}{0.453578in}}%
\pgfpathlineto{\pgfqpoint{3.662863in}{0.453578in}}%
\pgfpathlineto{\pgfqpoint{3.690563in}{0.453578in}}%
\pgfpathlineto{\pgfqpoint{3.718262in}{0.453578in}}%
\pgfpathlineto{\pgfqpoint{3.745961in}{0.453578in}}%
\pgfpathlineto{\pgfqpoint{3.773661in}{0.453578in}}%
\pgfpathlineto{\pgfqpoint{3.801360in}{0.453578in}}%
\pgfpathlineto{\pgfqpoint{3.829059in}{0.453578in}}%
\pgfpathlineto{\pgfqpoint{3.856758in}{0.453578in}}%
\pgfpathlineto{\pgfqpoint{3.884458in}{0.453578in}}%
\pgfpathlineto{\pgfqpoint{3.912157in}{0.453578in}}%
\pgfpathlineto{\pgfqpoint{3.939856in}{0.453578in}}%
\pgfpathlineto{\pgfqpoint{3.967556in}{0.453578in}}%
\pgfpathlineto{\pgfqpoint{3.995255in}{0.453578in}}%
\pgfpathlineto{\pgfqpoint{4.022954in}{0.453578in}}%
\pgfpathlineto{\pgfqpoint{4.033088in}{0.453578in}}%
\pgfusepath{stroke}%
\end{pgfscope}%
\begin{pgfscope}%
\pgfpathrectangle{\pgfqpoint{2.666838in}{0.383578in}}{\pgfqpoint{1.356250in}{1.540000in}}%
\pgfusepath{clip}%
\pgfsetrectcap%
\pgfsetroundjoin%
\pgfsetlinewidth{0.803000pt}%
\definecolor{currentstroke}{rgb}{0.333333,0.333333,0.333333}%
\pgfsetstrokecolor{currentstroke}%
\pgfsetstrokeopacity{0.300000}%
\pgfsetdash{}{0pt}%
\pgfpathmoveto{\pgfqpoint{2.721088in}{0.453578in}}%
\pgfpathlineto{\pgfqpoint{2.748787in}{0.453578in}}%
\pgfpathlineto{\pgfqpoint{2.776486in}{0.453578in}}%
\pgfpathlineto{\pgfqpoint{2.804186in}{0.453578in}}%
\pgfpathlineto{\pgfqpoint{2.831885in}{0.453578in}}%
\pgfpathlineto{\pgfqpoint{2.859584in}{0.453578in}}%
\pgfpathlineto{\pgfqpoint{2.887283in}{0.453578in}}%
\pgfpathlineto{\pgfqpoint{2.914983in}{0.453578in}}%
\pgfpathlineto{\pgfqpoint{2.942682in}{0.453578in}}%
\pgfpathlineto{\pgfqpoint{2.970381in}{0.453578in}}%
\pgfpathlineto{\pgfqpoint{2.998081in}{0.453578in}}%
\pgfpathlineto{\pgfqpoint{3.025780in}{0.453578in}}%
\pgfpathlineto{\pgfqpoint{3.053479in}{0.453578in}}%
\pgfpathlineto{\pgfqpoint{3.081178in}{0.453578in}}%
\pgfpathlineto{\pgfqpoint{3.108878in}{0.453578in}}%
\pgfpathlineto{\pgfqpoint{3.136577in}{0.453578in}}%
\pgfpathlineto{\pgfqpoint{3.164276in}{0.453578in}}%
\pgfpathlineto{\pgfqpoint{3.191976in}{0.453578in}}%
\pgfpathlineto{\pgfqpoint{3.219675in}{0.453578in}}%
\pgfpathlineto{\pgfqpoint{3.247374in}{0.453578in}}%
\pgfpathlineto{\pgfqpoint{3.275073in}{0.453578in}}%
\pgfpathlineto{\pgfqpoint{3.302773in}{0.453578in}}%
\pgfpathlineto{\pgfqpoint{3.330472in}{0.453578in}}%
\pgfpathlineto{\pgfqpoint{3.358171in}{0.453578in}}%
\pgfpathlineto{\pgfqpoint{3.385871in}{0.453578in}}%
\pgfpathlineto{\pgfqpoint{3.413570in}{0.453578in}}%
\pgfpathlineto{\pgfqpoint{3.441269in}{0.453578in}}%
\pgfpathlineto{\pgfqpoint{3.468968in}{0.453578in}}%
\pgfpathlineto{\pgfqpoint{3.496668in}{0.453578in}}%
\pgfpathlineto{\pgfqpoint{3.524367in}{0.453578in}}%
\pgfpathlineto{\pgfqpoint{3.552066in}{0.453578in}}%
\pgfpathlineto{\pgfqpoint{3.579766in}{0.453578in}}%
\pgfpathlineto{\pgfqpoint{3.607465in}{0.453578in}}%
\pgfpathlineto{\pgfqpoint{3.635164in}{0.453578in}}%
\pgfpathlineto{\pgfqpoint{3.662863in}{0.453578in}}%
\pgfpathlineto{\pgfqpoint{3.690563in}{0.453578in}}%
\pgfpathlineto{\pgfqpoint{3.718262in}{0.453578in}}%
\pgfpathlineto{\pgfqpoint{3.745961in}{0.453578in}}%
\pgfpathlineto{\pgfqpoint{3.773661in}{0.453578in}}%
\pgfpathlineto{\pgfqpoint{3.801360in}{0.453578in}}%
\pgfpathlineto{\pgfqpoint{3.829059in}{0.453578in}}%
\pgfpathlineto{\pgfqpoint{3.856758in}{0.453578in}}%
\pgfpathlineto{\pgfqpoint{3.884458in}{0.453578in}}%
\pgfpathlineto{\pgfqpoint{3.912157in}{0.453578in}}%
\pgfpathlineto{\pgfqpoint{3.939856in}{0.453578in}}%
\pgfpathlineto{\pgfqpoint{3.967556in}{0.453578in}}%
\pgfpathlineto{\pgfqpoint{3.995255in}{0.453578in}}%
\pgfpathlineto{\pgfqpoint{4.022954in}{0.453578in}}%
\pgfpathlineto{\pgfqpoint{4.033088in}{0.453578in}}%
\pgfusepath{stroke}%
\end{pgfscope}%
\begin{pgfscope}%
\pgfpathrectangle{\pgfqpoint{2.666838in}{0.383578in}}{\pgfqpoint{1.356250in}{1.540000in}}%
\pgfusepath{clip}%
\pgfsetrectcap%
\pgfsetroundjoin%
\pgfsetlinewidth{0.803000pt}%
\definecolor{currentstroke}{rgb}{0.333333,0.333333,0.333333}%
\pgfsetstrokecolor{currentstroke}%
\pgfsetstrokeopacity{0.300000}%
\pgfsetdash{}{0pt}%
\pgfpathmoveto{\pgfqpoint{2.721088in}{0.453578in}}%
\pgfpathlineto{\pgfqpoint{2.748787in}{0.453578in}}%
\pgfpathlineto{\pgfqpoint{2.776486in}{0.453578in}}%
\pgfpathlineto{\pgfqpoint{2.804186in}{0.453578in}}%
\pgfpathlineto{\pgfqpoint{2.831885in}{0.453578in}}%
\pgfpathlineto{\pgfqpoint{2.859584in}{0.453578in}}%
\pgfpathlineto{\pgfqpoint{2.887283in}{0.453578in}}%
\pgfpathlineto{\pgfqpoint{2.914983in}{0.453578in}}%
\pgfpathlineto{\pgfqpoint{2.942682in}{0.453578in}}%
\pgfpathlineto{\pgfqpoint{2.970381in}{0.453578in}}%
\pgfpathlineto{\pgfqpoint{2.998081in}{0.453578in}}%
\pgfpathlineto{\pgfqpoint{3.025780in}{0.453578in}}%
\pgfpathlineto{\pgfqpoint{3.053479in}{0.453578in}}%
\pgfpathlineto{\pgfqpoint{3.081178in}{0.453578in}}%
\pgfpathlineto{\pgfqpoint{3.108878in}{0.453578in}}%
\pgfpathlineto{\pgfqpoint{3.136577in}{0.453578in}}%
\pgfpathlineto{\pgfqpoint{3.164276in}{0.453578in}}%
\pgfpathlineto{\pgfqpoint{3.191976in}{0.453578in}}%
\pgfpathlineto{\pgfqpoint{3.219675in}{0.453578in}}%
\pgfpathlineto{\pgfqpoint{3.247374in}{0.453578in}}%
\pgfpathlineto{\pgfqpoint{3.275073in}{0.453578in}}%
\pgfpathlineto{\pgfqpoint{3.302773in}{0.453578in}}%
\pgfpathlineto{\pgfqpoint{3.330472in}{0.453578in}}%
\pgfpathlineto{\pgfqpoint{3.358171in}{0.453578in}}%
\pgfpathlineto{\pgfqpoint{3.385871in}{0.453578in}}%
\pgfpathlineto{\pgfqpoint{3.413570in}{0.453578in}}%
\pgfpathlineto{\pgfqpoint{3.441269in}{0.453578in}}%
\pgfpathlineto{\pgfqpoint{3.468968in}{0.453578in}}%
\pgfpathlineto{\pgfqpoint{3.496668in}{0.453578in}}%
\pgfpathlineto{\pgfqpoint{3.524367in}{0.453578in}}%
\pgfpathlineto{\pgfqpoint{3.552066in}{0.453578in}}%
\pgfpathlineto{\pgfqpoint{3.579766in}{0.453578in}}%
\pgfpathlineto{\pgfqpoint{3.607465in}{0.453578in}}%
\pgfpathlineto{\pgfqpoint{3.635164in}{0.453578in}}%
\pgfpathlineto{\pgfqpoint{3.662863in}{0.453578in}}%
\pgfpathlineto{\pgfqpoint{3.690563in}{0.453578in}}%
\pgfpathlineto{\pgfqpoint{3.718262in}{0.453578in}}%
\pgfpathlineto{\pgfqpoint{3.745961in}{0.453578in}}%
\pgfpathlineto{\pgfqpoint{3.773661in}{0.453578in}}%
\pgfpathlineto{\pgfqpoint{3.801360in}{0.453578in}}%
\pgfpathlineto{\pgfqpoint{3.829059in}{0.453578in}}%
\pgfpathlineto{\pgfqpoint{3.856758in}{0.453578in}}%
\pgfpathlineto{\pgfqpoint{3.884458in}{0.453578in}}%
\pgfpathlineto{\pgfqpoint{3.912157in}{0.453578in}}%
\pgfpathlineto{\pgfqpoint{3.939856in}{0.453578in}}%
\pgfpathlineto{\pgfqpoint{3.967556in}{0.453578in}}%
\pgfpathlineto{\pgfqpoint{3.995255in}{0.453578in}}%
\pgfpathlineto{\pgfqpoint{4.022954in}{0.453578in}}%
\pgfpathlineto{\pgfqpoint{4.033088in}{0.453578in}}%
\pgfusepath{stroke}%
\end{pgfscope}%
\begin{pgfscope}%
\pgfpathrectangle{\pgfqpoint{2.666838in}{0.383578in}}{\pgfqpoint{1.356250in}{1.540000in}}%
\pgfusepath{clip}%
\pgfsetrectcap%
\pgfsetroundjoin%
\pgfsetlinewidth{0.803000pt}%
\definecolor{currentstroke}{rgb}{0.333333,0.333333,0.333333}%
\pgfsetstrokecolor{currentstroke}%
\pgfsetstrokeopacity{0.300000}%
\pgfsetdash{}{0pt}%
\pgfpathmoveto{\pgfqpoint{2.721088in}{0.453578in}}%
\pgfpathlineto{\pgfqpoint{2.748787in}{0.453578in}}%
\pgfpathlineto{\pgfqpoint{2.776486in}{0.453578in}}%
\pgfpathlineto{\pgfqpoint{2.804186in}{0.453578in}}%
\pgfpathlineto{\pgfqpoint{2.831885in}{0.453578in}}%
\pgfpathlineto{\pgfqpoint{2.859584in}{0.453578in}}%
\pgfpathlineto{\pgfqpoint{2.887283in}{0.453578in}}%
\pgfpathlineto{\pgfqpoint{2.914983in}{0.453578in}}%
\pgfpathlineto{\pgfqpoint{2.942682in}{0.453578in}}%
\pgfpathlineto{\pgfqpoint{2.970381in}{0.453578in}}%
\pgfpathlineto{\pgfqpoint{2.998081in}{0.453578in}}%
\pgfpathlineto{\pgfqpoint{3.025780in}{0.453578in}}%
\pgfpathlineto{\pgfqpoint{3.053479in}{0.453578in}}%
\pgfpathlineto{\pgfqpoint{3.081178in}{0.453578in}}%
\pgfpathlineto{\pgfqpoint{3.108878in}{0.453578in}}%
\pgfpathlineto{\pgfqpoint{3.136577in}{0.453578in}}%
\pgfpathlineto{\pgfqpoint{3.164276in}{0.453578in}}%
\pgfpathlineto{\pgfqpoint{3.191976in}{0.453578in}}%
\pgfpathlineto{\pgfqpoint{3.219675in}{0.453578in}}%
\pgfpathlineto{\pgfqpoint{3.247374in}{0.453578in}}%
\pgfpathlineto{\pgfqpoint{3.275073in}{0.453578in}}%
\pgfpathlineto{\pgfqpoint{3.302773in}{0.453578in}}%
\pgfpathlineto{\pgfqpoint{3.330472in}{0.453578in}}%
\pgfpathlineto{\pgfqpoint{3.358171in}{0.453578in}}%
\pgfpathlineto{\pgfqpoint{3.385871in}{0.453578in}}%
\pgfpathlineto{\pgfqpoint{3.413570in}{0.453578in}}%
\pgfpathlineto{\pgfqpoint{3.441269in}{0.453578in}}%
\pgfpathlineto{\pgfqpoint{3.468968in}{0.453578in}}%
\pgfpathlineto{\pgfqpoint{3.496668in}{0.453578in}}%
\pgfpathlineto{\pgfqpoint{3.524367in}{0.453578in}}%
\pgfpathlineto{\pgfqpoint{3.552066in}{0.453578in}}%
\pgfpathlineto{\pgfqpoint{3.579766in}{0.453578in}}%
\pgfpathlineto{\pgfqpoint{3.607465in}{0.453578in}}%
\pgfpathlineto{\pgfqpoint{3.635164in}{0.453578in}}%
\pgfpathlineto{\pgfqpoint{3.662863in}{0.453578in}}%
\pgfpathlineto{\pgfqpoint{3.690563in}{0.453578in}}%
\pgfpathlineto{\pgfqpoint{3.718262in}{0.453578in}}%
\pgfpathlineto{\pgfqpoint{3.745961in}{0.453578in}}%
\pgfpathlineto{\pgfqpoint{3.773661in}{0.453578in}}%
\pgfpathlineto{\pgfqpoint{3.801360in}{0.453578in}}%
\pgfpathlineto{\pgfqpoint{3.829059in}{0.453578in}}%
\pgfpathlineto{\pgfqpoint{3.856758in}{0.453578in}}%
\pgfpathlineto{\pgfqpoint{3.884458in}{0.453578in}}%
\pgfpathlineto{\pgfqpoint{3.912157in}{0.453578in}}%
\pgfpathlineto{\pgfqpoint{3.939856in}{0.453578in}}%
\pgfpathlineto{\pgfqpoint{3.967556in}{0.453578in}}%
\pgfpathlineto{\pgfqpoint{3.995255in}{0.453578in}}%
\pgfpathlineto{\pgfqpoint{4.022954in}{0.453578in}}%
\pgfpathlineto{\pgfqpoint{4.033088in}{0.453578in}}%
\pgfusepath{stroke}%
\end{pgfscope}%
\begin{pgfscope}%
\pgfpathrectangle{\pgfqpoint{2.666838in}{0.383578in}}{\pgfqpoint{1.356250in}{1.540000in}}%
\pgfusepath{clip}%
\pgfsetrectcap%
\pgfsetroundjoin%
\pgfsetlinewidth{0.803000pt}%
\definecolor{currentstroke}{rgb}{0.333333,0.333333,0.333333}%
\pgfsetstrokecolor{currentstroke}%
\pgfsetstrokeopacity{0.300000}%
\pgfsetdash{}{0pt}%
\pgfpathmoveto{\pgfqpoint{2.721088in}{0.453578in}}%
\pgfpathlineto{\pgfqpoint{2.748787in}{0.453578in}}%
\pgfpathlineto{\pgfqpoint{2.776486in}{0.453578in}}%
\pgfpathlineto{\pgfqpoint{2.804186in}{0.453578in}}%
\pgfpathlineto{\pgfqpoint{2.831885in}{0.453578in}}%
\pgfpathlineto{\pgfqpoint{2.859584in}{0.453578in}}%
\pgfpathlineto{\pgfqpoint{2.887283in}{0.453578in}}%
\pgfpathlineto{\pgfqpoint{2.914983in}{0.453578in}}%
\pgfpathlineto{\pgfqpoint{2.942682in}{0.494135in}}%
\pgfpathlineto{\pgfqpoint{2.970381in}{0.568195in}}%
\pgfpathlineto{\pgfqpoint{2.998081in}{0.673274in}}%
\pgfpathlineto{\pgfqpoint{3.025780in}{0.806575in}}%
\pgfpathlineto{\pgfqpoint{3.053479in}{0.968520in}}%
\pgfpathlineto{\pgfqpoint{3.081178in}{1.115322in}}%
\pgfpathlineto{\pgfqpoint{3.108878in}{1.226708in}}%
\pgfpathlineto{\pgfqpoint{3.136577in}{1.299109in}}%
\pgfpathlineto{\pgfqpoint{3.164276in}{1.332586in}}%
\pgfpathlineto{\pgfqpoint{3.191976in}{1.337722in}}%
\pgfpathlineto{\pgfqpoint{3.219675in}{1.324505in}}%
\pgfpathlineto{\pgfqpoint{3.247374in}{1.302624in}}%
\pgfpathlineto{\pgfqpoint{3.275073in}{1.283035in}}%
\pgfpathlineto{\pgfqpoint{3.302773in}{1.270863in}}%
\pgfpathlineto{\pgfqpoint{3.330472in}{1.270030in}}%
\pgfpathlineto{\pgfqpoint{3.358171in}{1.280399in}}%
\pgfpathlineto{\pgfqpoint{3.385871in}{1.295639in}}%
\pgfpathlineto{\pgfqpoint{3.413570in}{1.313905in}}%
\pgfpathlineto{\pgfqpoint{3.441269in}{1.331101in}}%
\pgfpathlineto{\pgfqpoint{3.468968in}{1.344570in}}%
\pgfpathlineto{\pgfqpoint{3.496668in}{1.352724in}}%
\pgfpathlineto{\pgfqpoint{3.524367in}{1.353007in}}%
\pgfpathlineto{\pgfqpoint{3.552066in}{1.344494in}}%
\pgfpathlineto{\pgfqpoint{3.579766in}{1.329955in}}%
\pgfpathlineto{\pgfqpoint{3.607465in}{1.310044in}}%
\pgfpathlineto{\pgfqpoint{3.635164in}{1.284420in}}%
\pgfpathlineto{\pgfqpoint{3.662863in}{1.254333in}}%
\pgfpathlineto{\pgfqpoint{3.690563in}{1.220071in}}%
\pgfpathlineto{\pgfqpoint{3.718262in}{1.181006in}}%
\pgfpathlineto{\pgfqpoint{3.745961in}{1.139142in}}%
\pgfpathlineto{\pgfqpoint{3.773661in}{1.095512in}}%
\pgfpathlineto{\pgfqpoint{3.801360in}{1.051054in}}%
\pgfpathlineto{\pgfqpoint{3.829059in}{1.006266in}}%
\pgfpathlineto{\pgfqpoint{3.856758in}{0.962003in}}%
\pgfpathlineto{\pgfqpoint{3.884458in}{0.919156in}}%
\pgfpathlineto{\pgfqpoint{3.912157in}{0.878309in}}%
\pgfpathlineto{\pgfqpoint{3.939856in}{0.839473in}}%
\pgfpathlineto{\pgfqpoint{3.967556in}{0.802865in}}%
\pgfpathlineto{\pgfqpoint{3.995255in}{0.768615in}}%
\pgfpathlineto{\pgfqpoint{4.022954in}{0.736790in}}%
\pgfpathlineto{\pgfqpoint{4.033088in}{0.726036in}}%
\pgfusepath{stroke}%
\end{pgfscope}%
\begin{pgfscope}%
\pgfpathrectangle{\pgfqpoint{2.666838in}{0.383578in}}{\pgfqpoint{1.356250in}{1.540000in}}%
\pgfusepath{clip}%
\pgfsetrectcap%
\pgfsetroundjoin%
\pgfsetlinewidth{0.803000pt}%
\definecolor{currentstroke}{rgb}{0.333333,0.333333,0.333333}%
\pgfsetstrokecolor{currentstroke}%
\pgfsetstrokeopacity{0.300000}%
\pgfsetdash{}{0pt}%
\pgfpathmoveto{\pgfqpoint{2.721088in}{0.453578in}}%
\pgfpathlineto{\pgfqpoint{2.748787in}{0.453578in}}%
\pgfpathlineto{\pgfqpoint{2.776486in}{0.453578in}}%
\pgfpathlineto{\pgfqpoint{2.804186in}{0.453578in}}%
\pgfpathlineto{\pgfqpoint{2.831885in}{0.453578in}}%
\pgfpathlineto{\pgfqpoint{2.859584in}{0.453578in}}%
\pgfpathlineto{\pgfqpoint{2.887283in}{0.453578in}}%
\pgfpathlineto{\pgfqpoint{2.914983in}{0.453578in}}%
\pgfpathlineto{\pgfqpoint{2.942682in}{0.453578in}}%
\pgfpathlineto{\pgfqpoint{2.970381in}{0.483017in}}%
\pgfpathlineto{\pgfqpoint{2.998081in}{0.549990in}}%
\pgfpathlineto{\pgfqpoint{3.025780in}{0.655915in}}%
\pgfpathlineto{\pgfqpoint{3.053479in}{0.802534in}}%
\pgfpathlineto{\pgfqpoint{3.081178in}{0.945499in}}%
\pgfpathlineto{\pgfqpoint{3.108878in}{1.061524in}}%
\pgfpathlineto{\pgfqpoint{3.136577in}{1.144052in}}%
\pgfpathlineto{\pgfqpoint{3.164276in}{1.190503in}}%
\pgfpathlineto{\pgfqpoint{3.191976in}{1.210154in}}%
\pgfpathlineto{\pgfqpoint{3.219675in}{1.212226in}}%
\pgfpathlineto{\pgfqpoint{3.247374in}{1.206148in}}%
\pgfpathlineto{\pgfqpoint{3.275073in}{1.203251in}}%
\pgfpathlineto{\pgfqpoint{3.302773in}{1.208674in}}%
\pgfpathlineto{\pgfqpoint{3.330472in}{1.226464in}}%
\pgfpathlineto{\pgfqpoint{3.358171in}{1.256225in}}%
\pgfpathlineto{\pgfqpoint{3.385871in}{1.290560in}}%
\pgfpathlineto{\pgfqpoint{3.413570in}{1.327277in}}%
\pgfpathlineto{\pgfqpoint{3.441269in}{1.361623in}}%
\pgfpathlineto{\pgfqpoint{3.468968in}{1.390519in}}%
\pgfpathlineto{\pgfqpoint{3.496668in}{1.412140in}}%
\pgfpathlineto{\pgfqpoint{3.524367in}{1.423539in}}%
\pgfpathlineto{\pgfqpoint{3.552066in}{1.423669in}}%
\pgfpathlineto{\pgfqpoint{3.579766in}{1.415808in}}%
\pgfpathlineto{\pgfqpoint{3.607465in}{1.400767in}}%
\pgfpathlineto{\pgfqpoint{3.635164in}{1.378193in}}%
\pgfpathlineto{\pgfqpoint{3.662863in}{1.349605in}}%
\pgfpathlineto{\pgfqpoint{3.690563in}{1.315389in}}%
\pgfpathlineto{\pgfqpoint{3.718262in}{1.274852in}}%
\pgfpathlineto{\pgfqpoint{3.745961in}{1.230404in}}%
\pgfpathlineto{\pgfqpoint{3.773661in}{1.183309in}}%
\pgfpathlineto{\pgfqpoint{3.801360in}{1.134721in}}%
\pgfpathlineto{\pgfqpoint{3.829059in}{1.085273in}}%
\pgfpathlineto{\pgfqpoint{3.856758in}{1.036017in}}%
\pgfpathlineto{\pgfqpoint{3.884458in}{0.988050in}}%
\pgfpathlineto{\pgfqpoint{3.912157in}{0.942100in}}%
\pgfpathlineto{\pgfqpoint{3.939856in}{0.898220in}}%
\pgfpathlineto{\pgfqpoint{3.967556in}{0.856698in}}%
\pgfpathlineto{\pgfqpoint{3.995255in}{0.817722in}}%
\pgfpathlineto{\pgfqpoint{4.022954in}{0.781398in}}%
\pgfpathlineto{\pgfqpoint{4.033088in}{0.769092in}}%
\pgfusepath{stroke}%
\end{pgfscope}%
\begin{pgfscope}%
\pgfpathrectangle{\pgfqpoint{2.666838in}{0.383578in}}{\pgfqpoint{1.356250in}{1.540000in}}%
\pgfusepath{clip}%
\pgfsetrectcap%
\pgfsetroundjoin%
\pgfsetlinewidth{0.803000pt}%
\definecolor{currentstroke}{rgb}{0.333333,0.333333,0.333333}%
\pgfsetstrokecolor{currentstroke}%
\pgfsetstrokeopacity{0.300000}%
\pgfsetdash{}{0pt}%
\pgfpathmoveto{\pgfqpoint{2.721088in}{0.453578in}}%
\pgfpathlineto{\pgfqpoint{2.748787in}{0.453578in}}%
\pgfpathlineto{\pgfqpoint{2.776486in}{0.453578in}}%
\pgfpathlineto{\pgfqpoint{2.804186in}{0.453578in}}%
\pgfpathlineto{\pgfqpoint{2.831885in}{0.453578in}}%
\pgfpathlineto{\pgfqpoint{2.859584in}{0.453578in}}%
\pgfpathlineto{\pgfqpoint{2.887283in}{0.556461in}}%
\pgfpathlineto{\pgfqpoint{2.914983in}{0.726830in}}%
\pgfpathlineto{\pgfqpoint{2.942682in}{0.913612in}}%
\pgfpathlineto{\pgfqpoint{2.970381in}{1.088327in}}%
\pgfpathlineto{\pgfqpoint{2.998081in}{1.250725in}}%
\pgfpathlineto{\pgfqpoint{3.025780in}{1.403858in}}%
\pgfpathlineto{\pgfqpoint{3.053479in}{1.554289in}}%
\pgfpathlineto{\pgfqpoint{3.081178in}{1.668263in}}%
\pgfpathlineto{\pgfqpoint{3.108878in}{1.734679in}}%
\pgfpathlineto{\pgfqpoint{3.136577in}{1.756492in}}%
\pgfpathlineto{\pgfqpoint{3.164276in}{1.738731in}}%
\pgfpathlineto{\pgfqpoint{3.191976in}{1.693910in}}%
\pgfpathlineto{\pgfqpoint{3.219675in}{1.632927in}}%
\pgfpathlineto{\pgfqpoint{3.247374in}{1.565494in}}%
\pgfpathlineto{\pgfqpoint{3.275073in}{1.501528in}}%
\pgfpathlineto{\pgfqpoint{3.302773in}{1.445829in}}%
\pgfpathlineto{\pgfqpoint{3.330472in}{1.401857in}}%
\pgfpathlineto{\pgfqpoint{3.358171in}{1.369694in}}%
\pgfpathlineto{\pgfqpoint{3.385871in}{1.344596in}}%
\pgfpathlineto{\pgfqpoint{3.413570in}{1.325098in}}%
\pgfpathlineto{\pgfqpoint{3.441269in}{1.308005in}}%
\pgfpathlineto{\pgfqpoint{3.468968in}{1.291165in}}%
\pgfpathlineto{\pgfqpoint{3.496668in}{1.273206in}}%
\pgfpathlineto{\pgfqpoint{3.524367in}{1.252051in}}%
\pgfpathlineto{\pgfqpoint{3.552066in}{1.226826in}}%
\pgfpathlineto{\pgfqpoint{3.579766in}{1.199354in}}%
\pgfpathlineto{\pgfqpoint{3.607465in}{1.169922in}}%
\pgfpathlineto{\pgfqpoint{3.635164in}{1.138107in}}%
\pgfpathlineto{\pgfqpoint{3.662863in}{1.104648in}}%
\pgfpathlineto{\pgfqpoint{3.690563in}{1.069602in}}%
\pgfpathlineto{\pgfqpoint{3.718262in}{1.032380in}}%
\pgfpathlineto{\pgfqpoint{3.745961in}{0.994291in}}%
\pgfpathlineto{\pgfqpoint{3.773661in}{0.955957in}}%
\pgfpathlineto{\pgfqpoint{3.801360in}{0.917943in}}%
\pgfpathlineto{\pgfqpoint{3.829059in}{0.880505in}}%
\pgfpathlineto{\pgfqpoint{3.856758in}{0.844164in}}%
\pgfpathlineto{\pgfqpoint{3.884458in}{0.809471in}}%
\pgfpathlineto{\pgfqpoint{3.912157in}{0.776766in}}%
\pgfpathlineto{\pgfqpoint{3.939856in}{0.745990in}}%
\pgfpathlineto{\pgfqpoint{3.967556in}{0.717237in}}%
\pgfpathlineto{\pgfqpoint{3.995255in}{0.690545in}}%
\pgfpathlineto{\pgfqpoint{4.022954in}{0.665911in}}%
\pgfpathlineto{\pgfqpoint{4.033088in}{0.657638in}}%
\pgfusepath{stroke}%
\end{pgfscope}%
\begin{pgfscope}%
\pgfpathrectangle{\pgfqpoint{2.666838in}{0.383578in}}{\pgfqpoint{1.356250in}{1.540000in}}%
\pgfusepath{clip}%
\pgfsetrectcap%
\pgfsetroundjoin%
\pgfsetlinewidth{0.803000pt}%
\definecolor{currentstroke}{rgb}{0.333333,0.333333,0.333333}%
\pgfsetstrokecolor{currentstroke}%
\pgfsetstrokeopacity{0.300000}%
\pgfsetdash{}{0pt}%
\pgfpathmoveto{\pgfqpoint{2.721088in}{0.453578in}}%
\pgfpathlineto{\pgfqpoint{2.748787in}{0.453578in}}%
\pgfpathlineto{\pgfqpoint{2.776486in}{0.453578in}}%
\pgfpathlineto{\pgfqpoint{2.804186in}{0.453578in}}%
\pgfpathlineto{\pgfqpoint{2.831885in}{0.453578in}}%
\pgfpathlineto{\pgfqpoint{2.859584in}{0.453578in}}%
\pgfpathlineto{\pgfqpoint{2.887283in}{0.453578in}}%
\pgfpathlineto{\pgfqpoint{2.914983in}{0.453578in}}%
\pgfpathlineto{\pgfqpoint{2.942682in}{0.453578in}}%
\pgfpathlineto{\pgfqpoint{2.970381in}{0.453578in}}%
\pgfpathlineto{\pgfqpoint{2.998081in}{0.453578in}}%
\pgfpathlineto{\pgfqpoint{3.025780in}{0.453578in}}%
\pgfpathlineto{\pgfqpoint{3.053479in}{0.453578in}}%
\pgfpathlineto{\pgfqpoint{3.081178in}{0.453578in}}%
\pgfpathlineto{\pgfqpoint{3.108878in}{0.453578in}}%
\pgfpathlineto{\pgfqpoint{3.136577in}{0.453578in}}%
\pgfpathlineto{\pgfqpoint{3.164276in}{0.453578in}}%
\pgfpathlineto{\pgfqpoint{3.191976in}{0.453578in}}%
\pgfpathlineto{\pgfqpoint{3.219675in}{0.453578in}}%
\pgfpathlineto{\pgfqpoint{3.247374in}{0.453578in}}%
\pgfpathlineto{\pgfqpoint{3.275073in}{0.453578in}}%
\pgfpathlineto{\pgfqpoint{3.302773in}{0.453578in}}%
\pgfpathlineto{\pgfqpoint{3.330472in}{0.453578in}}%
\pgfpathlineto{\pgfqpoint{3.358171in}{0.453578in}}%
\pgfpathlineto{\pgfqpoint{3.385871in}{0.453578in}}%
\pgfpathlineto{\pgfqpoint{3.413570in}{0.453578in}}%
\pgfpathlineto{\pgfqpoint{3.441269in}{0.453578in}}%
\pgfpathlineto{\pgfqpoint{3.468968in}{0.453578in}}%
\pgfpathlineto{\pgfqpoint{3.496668in}{0.453578in}}%
\pgfpathlineto{\pgfqpoint{3.524367in}{0.453578in}}%
\pgfpathlineto{\pgfqpoint{3.552066in}{0.453578in}}%
\pgfpathlineto{\pgfqpoint{3.579766in}{0.453578in}}%
\pgfpathlineto{\pgfqpoint{3.607465in}{0.453578in}}%
\pgfpathlineto{\pgfqpoint{3.635164in}{0.453578in}}%
\pgfpathlineto{\pgfqpoint{3.662863in}{0.453578in}}%
\pgfpathlineto{\pgfqpoint{3.690563in}{0.453578in}}%
\pgfpathlineto{\pgfqpoint{3.718262in}{0.453578in}}%
\pgfpathlineto{\pgfqpoint{3.745961in}{0.453578in}}%
\pgfpathlineto{\pgfqpoint{3.773661in}{0.453578in}}%
\pgfpathlineto{\pgfqpoint{3.801360in}{0.453578in}}%
\pgfpathlineto{\pgfqpoint{3.829059in}{0.453578in}}%
\pgfpathlineto{\pgfqpoint{3.856758in}{0.453578in}}%
\pgfpathlineto{\pgfqpoint{3.884458in}{0.453578in}}%
\pgfpathlineto{\pgfqpoint{3.912157in}{0.453578in}}%
\pgfpathlineto{\pgfqpoint{3.939856in}{0.453578in}}%
\pgfpathlineto{\pgfqpoint{3.967556in}{0.453578in}}%
\pgfpathlineto{\pgfqpoint{3.995255in}{0.453578in}}%
\pgfpathlineto{\pgfqpoint{4.022954in}{0.453578in}}%
\pgfpathlineto{\pgfqpoint{4.033088in}{0.453578in}}%
\pgfusepath{stroke}%
\end{pgfscope}%
\begin{pgfscope}%
\pgfpathrectangle{\pgfqpoint{2.666838in}{0.383578in}}{\pgfqpoint{1.356250in}{1.540000in}}%
\pgfusepath{clip}%
\pgfsetrectcap%
\pgfsetroundjoin%
\pgfsetlinewidth{0.803000pt}%
\definecolor{currentstroke}{rgb}{0.333333,0.333333,0.333333}%
\pgfsetstrokecolor{currentstroke}%
\pgfsetstrokeopacity{0.300000}%
\pgfsetdash{}{0pt}%
\pgfpathmoveto{\pgfqpoint{2.721088in}{0.453578in}}%
\pgfpathlineto{\pgfqpoint{2.748787in}{0.453578in}}%
\pgfpathlineto{\pgfqpoint{2.776486in}{0.453578in}}%
\pgfpathlineto{\pgfqpoint{2.804186in}{0.453578in}}%
\pgfpathlineto{\pgfqpoint{2.831885in}{0.453578in}}%
\pgfpathlineto{\pgfqpoint{2.859584in}{0.453578in}}%
\pgfpathlineto{\pgfqpoint{2.887283in}{0.453578in}}%
\pgfpathlineto{\pgfqpoint{2.914983in}{0.517571in}}%
\pgfpathlineto{\pgfqpoint{2.942682in}{0.624774in}}%
\pgfpathlineto{\pgfqpoint{2.970381in}{0.750212in}}%
\pgfpathlineto{\pgfqpoint{2.998081in}{0.889175in}}%
\pgfpathlineto{\pgfqpoint{3.025780in}{1.039735in}}%
\pgfpathlineto{\pgfqpoint{3.053479in}{1.204113in}}%
\pgfpathlineto{\pgfqpoint{3.081178in}{1.342332in}}%
\pgfpathlineto{\pgfqpoint{3.108878in}{1.438273in}}%
\pgfpathlineto{\pgfqpoint{3.136577in}{1.491542in}}%
\pgfpathlineto{\pgfqpoint{3.164276in}{1.504723in}}%
\pgfpathlineto{\pgfqpoint{3.191976in}{1.489446in}}%
\pgfpathlineto{\pgfqpoint{3.219675in}{1.456237in}}%
\pgfpathlineto{\pgfqpoint{3.247374in}{1.414864in}}%
\pgfpathlineto{\pgfqpoint{3.275073in}{1.375802in}}%
\pgfpathlineto{\pgfqpoint{3.302773in}{1.344065in}}%
\pgfpathlineto{\pgfqpoint{3.330472in}{1.323381in}}%
\pgfpathlineto{\pgfqpoint{3.358171in}{1.313780in}}%
\pgfpathlineto{\pgfqpoint{3.385871in}{1.309816in}}%
\pgfpathlineto{\pgfqpoint{3.413570in}{1.309893in}}%
\pgfpathlineto{\pgfqpoint{3.441269in}{1.310439in}}%
\pgfpathlineto{\pgfqpoint{3.468968in}{1.309107in}}%
\pgfpathlineto{\pgfqpoint{3.496668in}{1.304463in}}%
\pgfpathlineto{\pgfqpoint{3.524367in}{1.294244in}}%
\pgfpathlineto{\pgfqpoint{3.552066in}{1.277585in}}%
\pgfpathlineto{\pgfqpoint{3.579766in}{1.256776in}}%
\pgfpathlineto{\pgfqpoint{3.607465in}{1.232299in}}%
\pgfpathlineto{\pgfqpoint{3.635164in}{1.203793in}}%
\pgfpathlineto{\pgfqpoint{3.662863in}{1.172251in}}%
\pgfpathlineto{\pgfqpoint{3.690563in}{1.137854in}}%
\pgfpathlineto{\pgfqpoint{3.718262in}{1.100009in}}%
\pgfpathlineto{\pgfqpoint{3.745961in}{1.060358in}}%
\pgfpathlineto{\pgfqpoint{3.773661in}{1.019722in}}%
\pgfpathlineto{\pgfqpoint{3.801360in}{0.978846in}}%
\pgfpathlineto{\pgfqpoint{3.829059in}{0.938105in}}%
\pgfpathlineto{\pgfqpoint{3.856758in}{0.898178in}}%
\pgfpathlineto{\pgfqpoint{3.884458in}{0.859778in}}%
\pgfpathlineto{\pgfqpoint{3.912157in}{0.823360in}}%
\pgfpathlineto{\pgfqpoint{3.939856in}{0.788901in}}%
\pgfpathlineto{\pgfqpoint{3.967556in}{0.756551in}}%
\pgfpathlineto{\pgfqpoint{3.995255in}{0.726395in}}%
\pgfpathlineto{\pgfqpoint{4.022954in}{0.698463in}}%
\pgfpathlineto{\pgfqpoint{4.033088in}{0.689051in}}%
\pgfusepath{stroke}%
\end{pgfscope}%
\begin{pgfscope}%
\pgfpathrectangle{\pgfqpoint{2.666838in}{0.383578in}}{\pgfqpoint{1.356250in}{1.540000in}}%
\pgfusepath{clip}%
\pgfsetrectcap%
\pgfsetroundjoin%
\pgfsetlinewidth{0.803000pt}%
\definecolor{currentstroke}{rgb}{0.333333,0.333333,0.333333}%
\pgfsetstrokecolor{currentstroke}%
\pgfsetstrokeopacity{0.300000}%
\pgfsetdash{}{0pt}%
\pgfpathmoveto{\pgfqpoint{2.721088in}{0.453578in}}%
\pgfpathlineto{\pgfqpoint{2.748787in}{0.453578in}}%
\pgfpathlineto{\pgfqpoint{2.776486in}{0.453578in}}%
\pgfpathlineto{\pgfqpoint{2.804186in}{0.453578in}}%
\pgfpathlineto{\pgfqpoint{2.831885in}{0.453578in}}%
\pgfpathlineto{\pgfqpoint{2.859584in}{0.453578in}}%
\pgfpathlineto{\pgfqpoint{2.887283in}{0.453578in}}%
\pgfpathlineto{\pgfqpoint{2.914983in}{0.453578in}}%
\pgfpathlineto{\pgfqpoint{2.942682in}{0.453578in}}%
\pgfpathlineto{\pgfqpoint{2.970381in}{0.453578in}}%
\pgfpathlineto{\pgfqpoint{2.998081in}{0.453578in}}%
\pgfpathlineto{\pgfqpoint{3.025780in}{0.453578in}}%
\pgfpathlineto{\pgfqpoint{3.053479in}{0.453578in}}%
\pgfpathlineto{\pgfqpoint{3.081178in}{0.453578in}}%
\pgfpathlineto{\pgfqpoint{3.108878in}{0.453578in}}%
\pgfpathlineto{\pgfqpoint{3.136577in}{0.453578in}}%
\pgfpathlineto{\pgfqpoint{3.164276in}{0.453578in}}%
\pgfpathlineto{\pgfqpoint{3.191976in}{0.453578in}}%
\pgfpathlineto{\pgfqpoint{3.219675in}{0.453578in}}%
\pgfpathlineto{\pgfqpoint{3.247374in}{0.453578in}}%
\pgfpathlineto{\pgfqpoint{3.275073in}{0.453578in}}%
\pgfpathlineto{\pgfqpoint{3.302773in}{0.453578in}}%
\pgfpathlineto{\pgfqpoint{3.330472in}{0.453578in}}%
\pgfpathlineto{\pgfqpoint{3.358171in}{0.453578in}}%
\pgfpathlineto{\pgfqpoint{3.385871in}{0.453578in}}%
\pgfpathlineto{\pgfqpoint{3.413570in}{0.453578in}}%
\pgfpathlineto{\pgfqpoint{3.441269in}{0.453578in}}%
\pgfpathlineto{\pgfqpoint{3.468968in}{0.453578in}}%
\pgfpathlineto{\pgfqpoint{3.496668in}{0.453578in}}%
\pgfpathlineto{\pgfqpoint{3.524367in}{0.453578in}}%
\pgfpathlineto{\pgfqpoint{3.552066in}{0.453578in}}%
\pgfpathlineto{\pgfqpoint{3.579766in}{0.453578in}}%
\pgfpathlineto{\pgfqpoint{3.607465in}{0.453578in}}%
\pgfpathlineto{\pgfqpoint{3.635164in}{0.453578in}}%
\pgfpathlineto{\pgfqpoint{3.662863in}{0.453578in}}%
\pgfpathlineto{\pgfqpoint{3.690563in}{0.453578in}}%
\pgfpathlineto{\pgfqpoint{3.718262in}{0.453578in}}%
\pgfpathlineto{\pgfqpoint{3.745961in}{0.453578in}}%
\pgfpathlineto{\pgfqpoint{3.773661in}{0.453578in}}%
\pgfpathlineto{\pgfqpoint{3.801360in}{0.453578in}}%
\pgfpathlineto{\pgfqpoint{3.829059in}{0.453578in}}%
\pgfpathlineto{\pgfqpoint{3.856758in}{0.453578in}}%
\pgfpathlineto{\pgfqpoint{3.884458in}{0.453578in}}%
\pgfpathlineto{\pgfqpoint{3.912157in}{0.453578in}}%
\pgfpathlineto{\pgfqpoint{3.939856in}{0.453578in}}%
\pgfpathlineto{\pgfqpoint{3.967556in}{0.453578in}}%
\pgfpathlineto{\pgfqpoint{3.995255in}{0.453578in}}%
\pgfpathlineto{\pgfqpoint{4.022954in}{0.453578in}}%
\pgfpathlineto{\pgfqpoint{4.033088in}{0.453578in}}%
\pgfusepath{stroke}%
\end{pgfscope}%
\begin{pgfscope}%
\pgfpathrectangle{\pgfqpoint{2.666838in}{0.383578in}}{\pgfqpoint{1.356250in}{1.540000in}}%
\pgfusepath{clip}%
\pgfsetrectcap%
\pgfsetroundjoin%
\pgfsetlinewidth{0.803000pt}%
\definecolor{currentstroke}{rgb}{0.333333,0.333333,0.333333}%
\pgfsetstrokecolor{currentstroke}%
\pgfsetstrokeopacity{0.300000}%
\pgfsetdash{}{0pt}%
\pgfpathmoveto{\pgfqpoint{2.721088in}{0.453578in}}%
\pgfpathlineto{\pgfqpoint{2.748787in}{0.453578in}}%
\pgfpathlineto{\pgfqpoint{2.776486in}{0.453578in}}%
\pgfpathlineto{\pgfqpoint{2.804186in}{0.453578in}}%
\pgfpathlineto{\pgfqpoint{2.831885in}{0.453578in}}%
\pgfpathlineto{\pgfqpoint{2.859584in}{0.453578in}}%
\pgfpathlineto{\pgfqpoint{2.887283in}{0.453578in}}%
\pgfpathlineto{\pgfqpoint{2.914983in}{0.453578in}}%
\pgfpathlineto{\pgfqpoint{2.942682in}{0.453578in}}%
\pgfpathlineto{\pgfqpoint{2.970381in}{0.453578in}}%
\pgfpathlineto{\pgfqpoint{2.998081in}{0.453578in}}%
\pgfpathlineto{\pgfqpoint{3.025780in}{0.453578in}}%
\pgfpathlineto{\pgfqpoint{3.053479in}{0.453578in}}%
\pgfpathlineto{\pgfqpoint{3.081178in}{0.453578in}}%
\pgfpathlineto{\pgfqpoint{3.108878in}{0.453578in}}%
\pgfpathlineto{\pgfqpoint{3.136577in}{0.453578in}}%
\pgfpathlineto{\pgfqpoint{3.164276in}{0.453578in}}%
\pgfpathlineto{\pgfqpoint{3.191976in}{0.453578in}}%
\pgfpathlineto{\pgfqpoint{3.219675in}{0.453578in}}%
\pgfpathlineto{\pgfqpoint{3.247374in}{0.453578in}}%
\pgfpathlineto{\pgfqpoint{3.275073in}{0.453578in}}%
\pgfpathlineto{\pgfqpoint{3.302773in}{0.453578in}}%
\pgfpathlineto{\pgfqpoint{3.330472in}{0.453578in}}%
\pgfpathlineto{\pgfqpoint{3.358171in}{0.453578in}}%
\pgfpathlineto{\pgfqpoint{3.385871in}{0.453578in}}%
\pgfpathlineto{\pgfqpoint{3.413570in}{0.453578in}}%
\pgfpathlineto{\pgfqpoint{3.441269in}{0.453578in}}%
\pgfpathlineto{\pgfqpoint{3.468968in}{0.453578in}}%
\pgfpathlineto{\pgfqpoint{3.496668in}{0.453578in}}%
\pgfpathlineto{\pgfqpoint{3.524367in}{0.453578in}}%
\pgfpathlineto{\pgfqpoint{3.552066in}{0.453578in}}%
\pgfpathlineto{\pgfqpoint{3.579766in}{0.453578in}}%
\pgfpathlineto{\pgfqpoint{3.607465in}{0.453578in}}%
\pgfpathlineto{\pgfqpoint{3.635164in}{0.453578in}}%
\pgfpathlineto{\pgfqpoint{3.662863in}{0.453578in}}%
\pgfpathlineto{\pgfqpoint{3.690563in}{0.453578in}}%
\pgfpathlineto{\pgfqpoint{3.718262in}{0.453578in}}%
\pgfpathlineto{\pgfqpoint{3.745961in}{0.453578in}}%
\pgfpathlineto{\pgfqpoint{3.773661in}{0.453578in}}%
\pgfpathlineto{\pgfqpoint{3.801360in}{0.453578in}}%
\pgfpathlineto{\pgfqpoint{3.829059in}{0.453578in}}%
\pgfpathlineto{\pgfqpoint{3.856758in}{0.453578in}}%
\pgfpathlineto{\pgfqpoint{3.884458in}{0.453578in}}%
\pgfpathlineto{\pgfqpoint{3.912157in}{0.453578in}}%
\pgfpathlineto{\pgfqpoint{3.939856in}{0.453578in}}%
\pgfpathlineto{\pgfqpoint{3.967556in}{0.453578in}}%
\pgfpathlineto{\pgfqpoint{3.995255in}{0.453578in}}%
\pgfpathlineto{\pgfqpoint{4.022954in}{0.453578in}}%
\pgfpathlineto{\pgfqpoint{4.033088in}{0.453578in}}%
\pgfusepath{stroke}%
\end{pgfscope}%
\begin{pgfscope}%
\pgfpathrectangle{\pgfqpoint{2.666838in}{0.383578in}}{\pgfqpoint{1.356250in}{1.540000in}}%
\pgfusepath{clip}%
\pgfsetrectcap%
\pgfsetroundjoin%
\pgfsetlinewidth{0.803000pt}%
\definecolor{currentstroke}{rgb}{0.333333,0.333333,0.333333}%
\pgfsetstrokecolor{currentstroke}%
\pgfsetstrokeopacity{0.300000}%
\pgfsetdash{}{0pt}%
\pgfpathmoveto{\pgfqpoint{2.721088in}{0.453578in}}%
\pgfpathlineto{\pgfqpoint{2.748787in}{0.453578in}}%
\pgfpathlineto{\pgfqpoint{2.776486in}{0.453578in}}%
\pgfpathlineto{\pgfqpoint{2.804186in}{0.453578in}}%
\pgfpathlineto{\pgfqpoint{2.831885in}{0.453578in}}%
\pgfpathlineto{\pgfqpoint{2.859584in}{0.453578in}}%
\pgfpathlineto{\pgfqpoint{2.887283in}{0.453578in}}%
\pgfpathlineto{\pgfqpoint{2.914983in}{0.453578in}}%
\pgfpathlineto{\pgfqpoint{2.942682in}{0.453578in}}%
\pgfpathlineto{\pgfqpoint{2.970381in}{0.453578in}}%
\pgfpathlineto{\pgfqpoint{2.998081in}{0.453578in}}%
\pgfpathlineto{\pgfqpoint{3.025780in}{0.453578in}}%
\pgfpathlineto{\pgfqpoint{3.053479in}{0.453578in}}%
\pgfpathlineto{\pgfqpoint{3.081178in}{0.453578in}}%
\pgfpathlineto{\pgfqpoint{3.108878in}{0.453578in}}%
\pgfpathlineto{\pgfqpoint{3.136577in}{0.453578in}}%
\pgfpathlineto{\pgfqpoint{3.164276in}{0.453578in}}%
\pgfpathlineto{\pgfqpoint{3.191976in}{0.453578in}}%
\pgfpathlineto{\pgfqpoint{3.219675in}{0.453578in}}%
\pgfpathlineto{\pgfqpoint{3.247374in}{0.453578in}}%
\pgfpathlineto{\pgfqpoint{3.275073in}{0.453578in}}%
\pgfpathlineto{\pgfqpoint{3.302773in}{0.453578in}}%
\pgfpathlineto{\pgfqpoint{3.330472in}{0.453578in}}%
\pgfpathlineto{\pgfqpoint{3.358171in}{0.453578in}}%
\pgfpathlineto{\pgfqpoint{3.385871in}{0.453578in}}%
\pgfpathlineto{\pgfqpoint{3.413570in}{0.453578in}}%
\pgfpathlineto{\pgfqpoint{3.441269in}{0.453578in}}%
\pgfpathlineto{\pgfqpoint{3.468968in}{0.453578in}}%
\pgfpathlineto{\pgfqpoint{3.496668in}{0.453578in}}%
\pgfpathlineto{\pgfqpoint{3.524367in}{0.453578in}}%
\pgfpathlineto{\pgfqpoint{3.552066in}{0.453578in}}%
\pgfpathlineto{\pgfqpoint{3.579766in}{0.453578in}}%
\pgfpathlineto{\pgfqpoint{3.607465in}{0.453578in}}%
\pgfpathlineto{\pgfqpoint{3.635164in}{0.453578in}}%
\pgfpathlineto{\pgfqpoint{3.662863in}{0.453578in}}%
\pgfpathlineto{\pgfqpoint{3.690563in}{0.453578in}}%
\pgfpathlineto{\pgfqpoint{3.718262in}{0.453578in}}%
\pgfpathlineto{\pgfqpoint{3.745961in}{0.453578in}}%
\pgfpathlineto{\pgfqpoint{3.773661in}{0.453578in}}%
\pgfpathlineto{\pgfqpoint{3.801360in}{0.453578in}}%
\pgfpathlineto{\pgfqpoint{3.829059in}{0.453578in}}%
\pgfpathlineto{\pgfqpoint{3.856758in}{0.453578in}}%
\pgfpathlineto{\pgfqpoint{3.884458in}{0.453578in}}%
\pgfpathlineto{\pgfqpoint{3.912157in}{0.453578in}}%
\pgfpathlineto{\pgfqpoint{3.939856in}{0.453578in}}%
\pgfpathlineto{\pgfqpoint{3.967556in}{0.453578in}}%
\pgfpathlineto{\pgfqpoint{3.995255in}{0.453578in}}%
\pgfpathlineto{\pgfqpoint{4.022954in}{0.453578in}}%
\pgfpathlineto{\pgfqpoint{4.033088in}{0.453578in}}%
\pgfusepath{stroke}%
\end{pgfscope}%
\begin{pgfscope}%
\pgfpathrectangle{\pgfqpoint{2.666838in}{0.383578in}}{\pgfqpoint{1.356250in}{1.540000in}}%
\pgfusepath{clip}%
\pgfsetrectcap%
\pgfsetroundjoin%
\pgfsetlinewidth{0.803000pt}%
\definecolor{currentstroke}{rgb}{0.333333,0.333333,0.333333}%
\pgfsetstrokecolor{currentstroke}%
\pgfsetstrokeopacity{0.300000}%
\pgfsetdash{}{0pt}%
\pgfpathmoveto{\pgfqpoint{2.721088in}{0.453578in}}%
\pgfpathlineto{\pgfqpoint{2.748787in}{0.453578in}}%
\pgfpathlineto{\pgfqpoint{2.776486in}{0.453578in}}%
\pgfpathlineto{\pgfqpoint{2.804186in}{0.453578in}}%
\pgfpathlineto{\pgfqpoint{2.831885in}{0.453578in}}%
\pgfpathlineto{\pgfqpoint{2.859584in}{0.453578in}}%
\pgfpathlineto{\pgfqpoint{2.887283in}{0.453578in}}%
\pgfpathlineto{\pgfqpoint{2.914983in}{0.453578in}}%
\pgfpathlineto{\pgfqpoint{2.942682in}{0.494135in}}%
\pgfpathlineto{\pgfqpoint{2.970381in}{0.568195in}}%
\pgfpathlineto{\pgfqpoint{2.998081in}{0.673274in}}%
\pgfpathlineto{\pgfqpoint{3.025780in}{0.806575in}}%
\pgfpathlineto{\pgfqpoint{3.053479in}{0.968520in}}%
\pgfpathlineto{\pgfqpoint{3.081178in}{1.115322in}}%
\pgfpathlineto{\pgfqpoint{3.108878in}{1.226708in}}%
\pgfpathlineto{\pgfqpoint{3.136577in}{1.299109in}}%
\pgfpathlineto{\pgfqpoint{3.164276in}{1.332586in}}%
\pgfpathlineto{\pgfqpoint{3.191976in}{1.337722in}}%
\pgfpathlineto{\pgfqpoint{3.219675in}{1.324505in}}%
\pgfpathlineto{\pgfqpoint{3.247374in}{1.302624in}}%
\pgfpathlineto{\pgfqpoint{3.275073in}{1.283035in}}%
\pgfpathlineto{\pgfqpoint{3.302773in}{1.270863in}}%
\pgfpathlineto{\pgfqpoint{3.330472in}{1.270030in}}%
\pgfpathlineto{\pgfqpoint{3.358171in}{1.280399in}}%
\pgfpathlineto{\pgfqpoint{3.385871in}{1.295639in}}%
\pgfpathlineto{\pgfqpoint{3.413570in}{1.313905in}}%
\pgfpathlineto{\pgfqpoint{3.441269in}{1.331101in}}%
\pgfpathlineto{\pgfqpoint{3.468968in}{1.344570in}}%
\pgfpathlineto{\pgfqpoint{3.496668in}{1.352724in}}%
\pgfpathlineto{\pgfqpoint{3.524367in}{1.353007in}}%
\pgfpathlineto{\pgfqpoint{3.552066in}{1.344494in}}%
\pgfpathlineto{\pgfqpoint{3.579766in}{1.329955in}}%
\pgfpathlineto{\pgfqpoint{3.607465in}{1.310044in}}%
\pgfpathlineto{\pgfqpoint{3.635164in}{1.284420in}}%
\pgfpathlineto{\pgfqpoint{3.662863in}{1.254333in}}%
\pgfpathlineto{\pgfqpoint{3.690563in}{1.220071in}}%
\pgfpathlineto{\pgfqpoint{3.718262in}{1.181006in}}%
\pgfpathlineto{\pgfqpoint{3.745961in}{1.139142in}}%
\pgfpathlineto{\pgfqpoint{3.773661in}{1.095512in}}%
\pgfpathlineto{\pgfqpoint{3.801360in}{1.051054in}}%
\pgfpathlineto{\pgfqpoint{3.829059in}{1.006266in}}%
\pgfpathlineto{\pgfqpoint{3.856758in}{0.962003in}}%
\pgfpathlineto{\pgfqpoint{3.884458in}{0.919156in}}%
\pgfpathlineto{\pgfqpoint{3.912157in}{0.878309in}}%
\pgfpathlineto{\pgfqpoint{3.939856in}{0.839473in}}%
\pgfpathlineto{\pgfqpoint{3.967556in}{0.802865in}}%
\pgfpathlineto{\pgfqpoint{3.995255in}{0.768615in}}%
\pgfpathlineto{\pgfqpoint{4.022954in}{0.736790in}}%
\pgfpathlineto{\pgfqpoint{4.033088in}{0.726036in}}%
\pgfusepath{stroke}%
\end{pgfscope}%
\begin{pgfscope}%
\pgfpathrectangle{\pgfqpoint{2.666838in}{0.383578in}}{\pgfqpoint{1.356250in}{1.540000in}}%
\pgfusepath{clip}%
\pgfsetrectcap%
\pgfsetroundjoin%
\pgfsetlinewidth{0.803000pt}%
\definecolor{currentstroke}{rgb}{0.333333,0.333333,0.333333}%
\pgfsetstrokecolor{currentstroke}%
\pgfsetstrokeopacity{0.300000}%
\pgfsetdash{}{0pt}%
\pgfpathmoveto{\pgfqpoint{2.721088in}{0.453578in}}%
\pgfpathlineto{\pgfqpoint{2.748787in}{0.453578in}}%
\pgfpathlineto{\pgfqpoint{2.776486in}{0.453578in}}%
\pgfpathlineto{\pgfqpoint{2.804186in}{0.453578in}}%
\pgfpathlineto{\pgfqpoint{2.831885in}{0.453578in}}%
\pgfpathlineto{\pgfqpoint{2.859584in}{0.453578in}}%
\pgfpathlineto{\pgfqpoint{2.887283in}{0.453578in}}%
\pgfpathlineto{\pgfqpoint{2.914983in}{0.453578in}}%
\pgfpathlineto{\pgfqpoint{2.942682in}{0.453578in}}%
\pgfpathlineto{\pgfqpoint{2.970381in}{0.453578in}}%
\pgfpathlineto{\pgfqpoint{2.998081in}{0.453578in}}%
\pgfpathlineto{\pgfqpoint{3.025780in}{0.453578in}}%
\pgfpathlineto{\pgfqpoint{3.053479in}{0.453578in}}%
\pgfpathlineto{\pgfqpoint{3.081178in}{0.453578in}}%
\pgfpathlineto{\pgfqpoint{3.108878in}{0.453578in}}%
\pgfpathlineto{\pgfqpoint{3.136577in}{0.453578in}}%
\pgfpathlineto{\pgfqpoint{3.164276in}{0.453578in}}%
\pgfpathlineto{\pgfqpoint{3.191976in}{0.453578in}}%
\pgfpathlineto{\pgfqpoint{3.219675in}{0.453578in}}%
\pgfpathlineto{\pgfqpoint{3.247374in}{0.453578in}}%
\pgfpathlineto{\pgfqpoint{3.275073in}{0.453578in}}%
\pgfpathlineto{\pgfqpoint{3.302773in}{0.453578in}}%
\pgfpathlineto{\pgfqpoint{3.330472in}{0.453578in}}%
\pgfpathlineto{\pgfqpoint{3.358171in}{0.453578in}}%
\pgfpathlineto{\pgfqpoint{3.385871in}{0.453578in}}%
\pgfpathlineto{\pgfqpoint{3.413570in}{0.453578in}}%
\pgfpathlineto{\pgfqpoint{3.441269in}{0.453578in}}%
\pgfpathlineto{\pgfqpoint{3.468968in}{0.453578in}}%
\pgfpathlineto{\pgfqpoint{3.496668in}{0.453578in}}%
\pgfpathlineto{\pgfqpoint{3.524367in}{0.453578in}}%
\pgfpathlineto{\pgfqpoint{3.552066in}{0.453578in}}%
\pgfpathlineto{\pgfqpoint{3.579766in}{0.453578in}}%
\pgfpathlineto{\pgfqpoint{3.607465in}{0.453578in}}%
\pgfpathlineto{\pgfqpoint{3.635164in}{0.453578in}}%
\pgfpathlineto{\pgfqpoint{3.662863in}{0.453578in}}%
\pgfpathlineto{\pgfqpoint{3.690563in}{0.453578in}}%
\pgfpathlineto{\pgfqpoint{3.718262in}{0.453578in}}%
\pgfpathlineto{\pgfqpoint{3.745961in}{0.453578in}}%
\pgfpathlineto{\pgfqpoint{3.773661in}{0.453578in}}%
\pgfpathlineto{\pgfqpoint{3.801360in}{0.453578in}}%
\pgfpathlineto{\pgfqpoint{3.829059in}{0.453578in}}%
\pgfpathlineto{\pgfqpoint{3.856758in}{0.453578in}}%
\pgfpathlineto{\pgfqpoint{3.884458in}{0.453578in}}%
\pgfpathlineto{\pgfqpoint{3.912157in}{0.453578in}}%
\pgfpathlineto{\pgfqpoint{3.939856in}{0.453578in}}%
\pgfpathlineto{\pgfqpoint{3.967556in}{0.453578in}}%
\pgfpathlineto{\pgfqpoint{3.995255in}{0.453578in}}%
\pgfpathlineto{\pgfqpoint{4.022954in}{0.453578in}}%
\pgfpathlineto{\pgfqpoint{4.033088in}{0.453578in}}%
\pgfusepath{stroke}%
\end{pgfscope}%
\begin{pgfscope}%
\pgfpathrectangle{\pgfqpoint{2.666838in}{0.383578in}}{\pgfqpoint{1.356250in}{1.540000in}}%
\pgfusepath{clip}%
\pgfsetrectcap%
\pgfsetroundjoin%
\pgfsetlinewidth{0.803000pt}%
\definecolor{currentstroke}{rgb}{0.686275,0.352941,0.313725}%
\pgfsetstrokecolor{currentstroke}%
\pgfsetstrokeopacity{0.300000}%
\pgfsetdash{}{0pt}%
\pgfpathmoveto{\pgfqpoint{2.721088in}{0.453578in}}%
\pgfpathlineto{\pgfqpoint{2.748787in}{0.453578in}}%
\pgfpathlineto{\pgfqpoint{2.776486in}{0.453578in}}%
\pgfpathlineto{\pgfqpoint{2.804186in}{0.453578in}}%
\pgfpathlineto{\pgfqpoint{2.831885in}{0.453578in}}%
\pgfpathlineto{\pgfqpoint{2.859584in}{0.453578in}}%
\pgfpathlineto{\pgfqpoint{2.887283in}{0.453578in}}%
\pgfpathlineto{\pgfqpoint{2.914983in}{0.453578in}}%
\pgfpathlineto{\pgfqpoint{2.942682in}{0.453578in}}%
\pgfpathlineto{\pgfqpoint{2.970381in}{0.453578in}}%
\pgfpathlineto{\pgfqpoint{2.998081in}{0.453578in}}%
\pgfpathlineto{\pgfqpoint{3.025780in}{0.453578in}}%
\pgfpathlineto{\pgfqpoint{3.053479in}{0.453578in}}%
\pgfpathlineto{\pgfqpoint{3.081178in}{0.453578in}}%
\pgfpathlineto{\pgfqpoint{3.108878in}{0.453578in}}%
\pgfpathlineto{\pgfqpoint{3.136577in}{0.453578in}}%
\pgfpathlineto{\pgfqpoint{3.164276in}{0.453578in}}%
\pgfpathlineto{\pgfqpoint{3.191976in}{0.453578in}}%
\pgfpathlineto{\pgfqpoint{3.219675in}{0.453578in}}%
\pgfpathlineto{\pgfqpoint{3.247374in}{0.453578in}}%
\pgfpathlineto{\pgfqpoint{3.275073in}{0.453578in}}%
\pgfpathlineto{\pgfqpoint{3.302773in}{0.453578in}}%
\pgfpathlineto{\pgfqpoint{3.330472in}{0.453578in}}%
\pgfpathlineto{\pgfqpoint{3.358171in}{0.453578in}}%
\pgfpathlineto{\pgfqpoint{3.385871in}{0.453578in}}%
\pgfpathlineto{\pgfqpoint{3.413570in}{0.453578in}}%
\pgfpathlineto{\pgfqpoint{3.441269in}{0.453578in}}%
\pgfpathlineto{\pgfqpoint{3.468968in}{0.453578in}}%
\pgfpathlineto{\pgfqpoint{3.496668in}{0.453578in}}%
\pgfpathlineto{\pgfqpoint{3.524367in}{0.453578in}}%
\pgfpathlineto{\pgfqpoint{3.552066in}{0.453578in}}%
\pgfpathlineto{\pgfqpoint{3.579766in}{0.453578in}}%
\pgfpathlineto{\pgfqpoint{3.607465in}{0.453578in}}%
\pgfpathlineto{\pgfqpoint{3.635164in}{0.453578in}}%
\pgfpathlineto{\pgfqpoint{3.662863in}{0.453578in}}%
\pgfpathlineto{\pgfqpoint{3.690563in}{0.453578in}}%
\pgfpathlineto{\pgfqpoint{3.718262in}{0.453578in}}%
\pgfpathlineto{\pgfqpoint{3.745961in}{0.453578in}}%
\pgfpathlineto{\pgfqpoint{3.773661in}{0.453578in}}%
\pgfpathlineto{\pgfqpoint{3.801360in}{0.453578in}}%
\pgfpathlineto{\pgfqpoint{3.829059in}{0.453578in}}%
\pgfpathlineto{\pgfqpoint{3.856758in}{0.453578in}}%
\pgfpathlineto{\pgfqpoint{3.884458in}{0.453578in}}%
\pgfpathlineto{\pgfqpoint{3.912157in}{0.453578in}}%
\pgfpathlineto{\pgfqpoint{3.939856in}{0.453578in}}%
\pgfpathlineto{\pgfqpoint{3.967556in}{0.453578in}}%
\pgfpathlineto{\pgfqpoint{3.995255in}{0.453578in}}%
\pgfpathlineto{\pgfqpoint{4.022954in}{0.453578in}}%
\pgfpathlineto{\pgfqpoint{4.033088in}{0.453578in}}%
\pgfusepath{stroke}%
\end{pgfscope}%
\begin{pgfscope}%
\pgfpathrectangle{\pgfqpoint{2.666838in}{0.383578in}}{\pgfqpoint{1.356250in}{1.540000in}}%
\pgfusepath{clip}%
\pgfsetrectcap%
\pgfsetroundjoin%
\pgfsetlinewidth{0.803000pt}%
\definecolor{currentstroke}{rgb}{0.686275,0.352941,0.313725}%
\pgfsetstrokecolor{currentstroke}%
\pgfsetstrokeopacity{0.300000}%
\pgfsetdash{}{0pt}%
\pgfpathmoveto{\pgfqpoint{2.721088in}{0.453578in}}%
\pgfpathlineto{\pgfqpoint{2.748787in}{0.453578in}}%
\pgfpathlineto{\pgfqpoint{2.776486in}{0.453578in}}%
\pgfpathlineto{\pgfqpoint{2.804186in}{0.453578in}}%
\pgfpathlineto{\pgfqpoint{2.831885in}{0.453578in}}%
\pgfpathlineto{\pgfqpoint{2.859584in}{0.453578in}}%
\pgfpathlineto{\pgfqpoint{2.887283in}{0.453578in}}%
\pgfpathlineto{\pgfqpoint{2.914983in}{0.453578in}}%
\pgfpathlineto{\pgfqpoint{2.942682in}{0.453578in}}%
\pgfpathlineto{\pgfqpoint{2.970381in}{0.453853in}}%
\pgfpathlineto{\pgfqpoint{2.998081in}{0.454480in}}%
\pgfpathlineto{\pgfqpoint{3.025780in}{0.455377in}}%
\pgfpathlineto{\pgfqpoint{3.053479in}{0.456473in}}%
\pgfpathlineto{\pgfqpoint{3.081178in}{0.457631in}}%
\pgfpathlineto{\pgfqpoint{3.108878in}{0.458831in}}%
\pgfpathlineto{\pgfqpoint{3.136577in}{0.459950in}}%
\pgfpathlineto{\pgfqpoint{3.164276in}{0.460946in}}%
\pgfpathlineto{\pgfqpoint{3.191976in}{0.461791in}}%
\pgfpathlineto{\pgfqpoint{3.219675in}{0.462437in}}%
\pgfpathlineto{\pgfqpoint{3.247374in}{0.462894in}}%
\pgfpathlineto{\pgfqpoint{3.275073in}{0.463176in}}%
\pgfpathlineto{\pgfqpoint{3.302773in}{0.463278in}}%
\pgfpathlineto{\pgfqpoint{3.330472in}{0.463471in}}%
\pgfpathlineto{\pgfqpoint{3.358171in}{0.463936in}}%
\pgfpathlineto{\pgfqpoint{3.385871in}{0.464657in}}%
\pgfpathlineto{\pgfqpoint{3.413570in}{0.465625in}}%
\pgfpathlineto{\pgfqpoint{3.441269in}{0.466754in}}%
\pgfpathlineto{\pgfqpoint{3.468968in}{0.467835in}}%
\pgfpathlineto{\pgfqpoint{3.496668in}{0.468804in}}%
\pgfpathlineto{\pgfqpoint{3.524367in}{0.469566in}}%
\pgfpathlineto{\pgfqpoint{3.552066in}{0.470077in}}%
\pgfpathlineto{\pgfqpoint{3.579766in}{0.470327in}}%
\pgfpathlineto{\pgfqpoint{3.607465in}{0.470328in}}%
\pgfpathlineto{\pgfqpoint{3.635164in}{0.470107in}}%
\pgfpathlineto{\pgfqpoint{3.662863in}{0.469694in}}%
\pgfpathlineto{\pgfqpoint{3.690563in}{0.469125in}}%
\pgfpathlineto{\pgfqpoint{3.718262in}{0.468452in}}%
\pgfpathlineto{\pgfqpoint{3.745961in}{0.467699in}}%
\pgfpathlineto{\pgfqpoint{3.773661in}{0.466888in}}%
\pgfpathlineto{\pgfqpoint{3.801360in}{0.466041in}}%
\pgfpathlineto{\pgfqpoint{3.829059in}{0.465176in}}%
\pgfpathlineto{\pgfqpoint{3.856758in}{0.464320in}}%
\pgfpathlineto{\pgfqpoint{3.884458in}{0.463480in}}%
\pgfpathlineto{\pgfqpoint{3.912157in}{0.462666in}}%
\pgfpathlineto{\pgfqpoint{3.939856in}{0.461884in}}%
\pgfpathlineto{\pgfqpoint{3.967556in}{0.461135in}}%
\pgfpathlineto{\pgfqpoint{3.995255in}{0.460429in}}%
\pgfpathlineto{\pgfqpoint{4.022954in}{0.459767in}}%
\pgfpathlineto{\pgfqpoint{4.033088in}{0.459540in}}%
\pgfusepath{stroke}%
\end{pgfscope}%
\begin{pgfscope}%
\pgfpathrectangle{\pgfqpoint{2.666838in}{0.383578in}}{\pgfqpoint{1.356250in}{1.540000in}}%
\pgfusepath{clip}%
\pgfsetrectcap%
\pgfsetroundjoin%
\pgfsetlinewidth{0.803000pt}%
\definecolor{currentstroke}{rgb}{0.686275,0.352941,0.313725}%
\pgfsetstrokecolor{currentstroke}%
\pgfsetstrokeopacity{0.300000}%
\pgfsetdash{}{0pt}%
\pgfpathmoveto{\pgfqpoint{2.721088in}{0.453578in}}%
\pgfpathlineto{\pgfqpoint{2.748787in}{0.453578in}}%
\pgfpathlineto{\pgfqpoint{2.776486in}{0.453578in}}%
\pgfpathlineto{\pgfqpoint{2.804186in}{0.453578in}}%
\pgfpathlineto{\pgfqpoint{2.831885in}{0.453578in}}%
\pgfpathlineto{\pgfqpoint{2.859584in}{0.453578in}}%
\pgfpathlineto{\pgfqpoint{2.887283in}{0.453578in}}%
\pgfpathlineto{\pgfqpoint{2.914983in}{0.453578in}}%
\pgfpathlineto{\pgfqpoint{2.942682in}{0.453578in}}%
\pgfpathlineto{\pgfqpoint{2.970381in}{0.453578in}}%
\pgfpathlineto{\pgfqpoint{2.998081in}{0.453578in}}%
\pgfpathlineto{\pgfqpoint{3.025780in}{0.453578in}}%
\pgfpathlineto{\pgfqpoint{3.053479in}{0.453578in}}%
\pgfpathlineto{\pgfqpoint{3.081178in}{0.453578in}}%
\pgfpathlineto{\pgfqpoint{3.108878in}{0.453578in}}%
\pgfpathlineto{\pgfqpoint{3.136577in}{0.453578in}}%
\pgfpathlineto{\pgfqpoint{3.164276in}{0.453578in}}%
\pgfpathlineto{\pgfqpoint{3.191976in}{0.453578in}}%
\pgfpathlineto{\pgfqpoint{3.219675in}{0.453578in}}%
\pgfpathlineto{\pgfqpoint{3.247374in}{0.453578in}}%
\pgfpathlineto{\pgfqpoint{3.275073in}{0.453578in}}%
\pgfpathlineto{\pgfqpoint{3.302773in}{0.453578in}}%
\pgfpathlineto{\pgfqpoint{3.330472in}{0.453578in}}%
\pgfpathlineto{\pgfqpoint{3.358171in}{0.453578in}}%
\pgfpathlineto{\pgfqpoint{3.385871in}{0.453578in}}%
\pgfpathlineto{\pgfqpoint{3.413570in}{0.453578in}}%
\pgfpathlineto{\pgfqpoint{3.441269in}{0.453578in}}%
\pgfpathlineto{\pgfqpoint{3.468968in}{0.453578in}}%
\pgfpathlineto{\pgfqpoint{3.496668in}{0.453578in}}%
\pgfpathlineto{\pgfqpoint{3.524367in}{0.453578in}}%
\pgfpathlineto{\pgfqpoint{3.552066in}{0.453578in}}%
\pgfpathlineto{\pgfqpoint{3.579766in}{0.453578in}}%
\pgfpathlineto{\pgfqpoint{3.607465in}{0.453578in}}%
\pgfpathlineto{\pgfqpoint{3.635164in}{0.453578in}}%
\pgfpathlineto{\pgfqpoint{3.662863in}{0.453578in}}%
\pgfpathlineto{\pgfqpoint{3.690563in}{0.453578in}}%
\pgfpathlineto{\pgfqpoint{3.718262in}{0.453578in}}%
\pgfpathlineto{\pgfqpoint{3.745961in}{0.453578in}}%
\pgfpathlineto{\pgfqpoint{3.773661in}{0.453578in}}%
\pgfpathlineto{\pgfqpoint{3.801360in}{0.453578in}}%
\pgfpathlineto{\pgfqpoint{3.829059in}{0.453578in}}%
\pgfpathlineto{\pgfqpoint{3.856758in}{0.453578in}}%
\pgfpathlineto{\pgfqpoint{3.884458in}{0.453578in}}%
\pgfpathlineto{\pgfqpoint{3.912157in}{0.453578in}}%
\pgfpathlineto{\pgfqpoint{3.939856in}{0.453578in}}%
\pgfpathlineto{\pgfqpoint{3.967556in}{0.453578in}}%
\pgfpathlineto{\pgfqpoint{3.995255in}{0.453578in}}%
\pgfpathlineto{\pgfqpoint{4.022954in}{0.453578in}}%
\pgfpathlineto{\pgfqpoint{4.033088in}{0.453578in}}%
\pgfusepath{stroke}%
\end{pgfscope}%
\begin{pgfscope}%
\pgfpathrectangle{\pgfqpoint{2.666838in}{0.383578in}}{\pgfqpoint{1.356250in}{1.540000in}}%
\pgfusepath{clip}%
\pgfsetrectcap%
\pgfsetroundjoin%
\pgfsetlinewidth{0.803000pt}%
\definecolor{currentstroke}{rgb}{0.686275,0.352941,0.313725}%
\pgfsetstrokecolor{currentstroke}%
\pgfsetstrokeopacity{0.300000}%
\pgfsetdash{}{0pt}%
\pgfpathmoveto{\pgfqpoint{2.721088in}{0.453578in}}%
\pgfpathlineto{\pgfqpoint{2.748787in}{0.453578in}}%
\pgfpathlineto{\pgfqpoint{2.776486in}{0.453578in}}%
\pgfpathlineto{\pgfqpoint{2.804186in}{0.453578in}}%
\pgfpathlineto{\pgfqpoint{2.831885in}{0.453578in}}%
\pgfpathlineto{\pgfqpoint{2.859584in}{0.453578in}}%
\pgfpathlineto{\pgfqpoint{2.887283in}{0.453578in}}%
\pgfpathlineto{\pgfqpoint{2.914983in}{0.453578in}}%
\pgfpathlineto{\pgfqpoint{2.942682in}{0.453578in}}%
\pgfpathlineto{\pgfqpoint{2.970381in}{0.453578in}}%
\pgfpathlineto{\pgfqpoint{2.998081in}{0.453578in}}%
\pgfpathlineto{\pgfqpoint{3.025780in}{0.453578in}}%
\pgfpathlineto{\pgfqpoint{3.053479in}{0.453578in}}%
\pgfpathlineto{\pgfqpoint{3.081178in}{0.453578in}}%
\pgfpathlineto{\pgfqpoint{3.108878in}{0.453578in}}%
\pgfpathlineto{\pgfqpoint{3.136577in}{0.453578in}}%
\pgfpathlineto{\pgfqpoint{3.164276in}{0.453578in}}%
\pgfpathlineto{\pgfqpoint{3.191976in}{0.453578in}}%
\pgfpathlineto{\pgfqpoint{3.219675in}{0.453578in}}%
\pgfpathlineto{\pgfqpoint{3.247374in}{0.453578in}}%
\pgfpathlineto{\pgfqpoint{3.275073in}{0.453578in}}%
\pgfpathlineto{\pgfqpoint{3.302773in}{0.453578in}}%
\pgfpathlineto{\pgfqpoint{3.330472in}{0.453578in}}%
\pgfpathlineto{\pgfqpoint{3.358171in}{0.453578in}}%
\pgfpathlineto{\pgfqpoint{3.385871in}{0.453578in}}%
\pgfpathlineto{\pgfqpoint{3.413570in}{0.453578in}}%
\pgfpathlineto{\pgfqpoint{3.441269in}{0.453578in}}%
\pgfpathlineto{\pgfqpoint{3.468968in}{0.453578in}}%
\pgfpathlineto{\pgfqpoint{3.496668in}{0.453578in}}%
\pgfpathlineto{\pgfqpoint{3.524367in}{0.453578in}}%
\pgfpathlineto{\pgfqpoint{3.552066in}{0.453578in}}%
\pgfpathlineto{\pgfqpoint{3.579766in}{0.453578in}}%
\pgfpathlineto{\pgfqpoint{3.607465in}{0.453578in}}%
\pgfpathlineto{\pgfqpoint{3.635164in}{0.453578in}}%
\pgfpathlineto{\pgfqpoint{3.662863in}{0.453578in}}%
\pgfpathlineto{\pgfqpoint{3.690563in}{0.453578in}}%
\pgfpathlineto{\pgfqpoint{3.718262in}{0.453578in}}%
\pgfpathlineto{\pgfqpoint{3.745961in}{0.453578in}}%
\pgfpathlineto{\pgfqpoint{3.773661in}{0.453578in}}%
\pgfpathlineto{\pgfqpoint{3.801360in}{0.453578in}}%
\pgfpathlineto{\pgfqpoint{3.829059in}{0.453578in}}%
\pgfpathlineto{\pgfqpoint{3.856758in}{0.453578in}}%
\pgfpathlineto{\pgfqpoint{3.884458in}{0.453578in}}%
\pgfpathlineto{\pgfqpoint{3.912157in}{0.453578in}}%
\pgfpathlineto{\pgfqpoint{3.939856in}{0.453578in}}%
\pgfpathlineto{\pgfqpoint{3.967556in}{0.453578in}}%
\pgfpathlineto{\pgfqpoint{3.995255in}{0.453578in}}%
\pgfpathlineto{\pgfqpoint{4.022954in}{0.453578in}}%
\pgfpathlineto{\pgfqpoint{4.033088in}{0.453578in}}%
\pgfusepath{stroke}%
\end{pgfscope}%
\begin{pgfscope}%
\pgfpathrectangle{\pgfqpoint{2.666838in}{0.383578in}}{\pgfqpoint{1.356250in}{1.540000in}}%
\pgfusepath{clip}%
\pgfsetrectcap%
\pgfsetroundjoin%
\pgfsetlinewidth{0.803000pt}%
\definecolor{currentstroke}{rgb}{0.686275,0.352941,0.313725}%
\pgfsetstrokecolor{currentstroke}%
\pgfsetstrokeopacity{0.300000}%
\pgfsetdash{}{0pt}%
\pgfpathmoveto{\pgfqpoint{2.721088in}{0.453578in}}%
\pgfpathlineto{\pgfqpoint{2.748787in}{0.453578in}}%
\pgfpathlineto{\pgfqpoint{2.776486in}{0.453578in}}%
\pgfpathlineto{\pgfqpoint{2.804186in}{0.453578in}}%
\pgfpathlineto{\pgfqpoint{2.831885in}{0.453578in}}%
\pgfpathlineto{\pgfqpoint{2.859584in}{0.453578in}}%
\pgfpathlineto{\pgfqpoint{2.887283in}{0.453578in}}%
\pgfpathlineto{\pgfqpoint{2.914983in}{0.453578in}}%
\pgfpathlineto{\pgfqpoint{2.942682in}{0.453578in}}%
\pgfpathlineto{\pgfqpoint{2.970381in}{0.453578in}}%
\pgfpathlineto{\pgfqpoint{2.998081in}{0.453578in}}%
\pgfpathlineto{\pgfqpoint{3.025780in}{0.453578in}}%
\pgfpathlineto{\pgfqpoint{3.053479in}{0.453578in}}%
\pgfpathlineto{\pgfqpoint{3.081178in}{0.453578in}}%
\pgfpathlineto{\pgfqpoint{3.108878in}{0.453578in}}%
\pgfpathlineto{\pgfqpoint{3.136577in}{0.453578in}}%
\pgfpathlineto{\pgfqpoint{3.164276in}{0.453578in}}%
\pgfpathlineto{\pgfqpoint{3.191976in}{0.453578in}}%
\pgfpathlineto{\pgfqpoint{3.219675in}{0.453578in}}%
\pgfpathlineto{\pgfqpoint{3.247374in}{0.453578in}}%
\pgfpathlineto{\pgfqpoint{3.275073in}{0.453578in}}%
\pgfpathlineto{\pgfqpoint{3.302773in}{0.453578in}}%
\pgfpathlineto{\pgfqpoint{3.330472in}{0.453578in}}%
\pgfpathlineto{\pgfqpoint{3.358171in}{0.453578in}}%
\pgfpathlineto{\pgfqpoint{3.385871in}{0.453578in}}%
\pgfpathlineto{\pgfqpoint{3.413570in}{0.453578in}}%
\pgfpathlineto{\pgfqpoint{3.441269in}{0.453578in}}%
\pgfpathlineto{\pgfqpoint{3.468968in}{0.453578in}}%
\pgfpathlineto{\pgfqpoint{3.496668in}{0.453578in}}%
\pgfpathlineto{\pgfqpoint{3.524367in}{0.453578in}}%
\pgfpathlineto{\pgfqpoint{3.552066in}{0.453578in}}%
\pgfpathlineto{\pgfqpoint{3.579766in}{0.453578in}}%
\pgfpathlineto{\pgfqpoint{3.607465in}{0.453578in}}%
\pgfpathlineto{\pgfqpoint{3.635164in}{0.453578in}}%
\pgfpathlineto{\pgfqpoint{3.662863in}{0.453578in}}%
\pgfpathlineto{\pgfqpoint{3.690563in}{0.453578in}}%
\pgfpathlineto{\pgfqpoint{3.718262in}{0.453578in}}%
\pgfpathlineto{\pgfqpoint{3.745961in}{0.453578in}}%
\pgfpathlineto{\pgfqpoint{3.773661in}{0.453578in}}%
\pgfpathlineto{\pgfqpoint{3.801360in}{0.453578in}}%
\pgfpathlineto{\pgfqpoint{3.829059in}{0.453578in}}%
\pgfpathlineto{\pgfqpoint{3.856758in}{0.453578in}}%
\pgfpathlineto{\pgfqpoint{3.884458in}{0.453578in}}%
\pgfpathlineto{\pgfqpoint{3.912157in}{0.453578in}}%
\pgfpathlineto{\pgfqpoint{3.939856in}{0.453578in}}%
\pgfpathlineto{\pgfqpoint{3.967556in}{0.453578in}}%
\pgfpathlineto{\pgfqpoint{3.995255in}{0.453578in}}%
\pgfpathlineto{\pgfqpoint{4.022954in}{0.453578in}}%
\pgfpathlineto{\pgfqpoint{4.033088in}{0.453578in}}%
\pgfusepath{stroke}%
\end{pgfscope}%
\begin{pgfscope}%
\pgfpathrectangle{\pgfqpoint{2.666838in}{0.383578in}}{\pgfqpoint{1.356250in}{1.540000in}}%
\pgfusepath{clip}%
\pgfsetrectcap%
\pgfsetroundjoin%
\pgfsetlinewidth{0.803000pt}%
\definecolor{currentstroke}{rgb}{0.686275,0.352941,0.313725}%
\pgfsetstrokecolor{currentstroke}%
\pgfsetstrokeopacity{0.300000}%
\pgfsetdash{}{0pt}%
\pgfpathmoveto{\pgfqpoint{2.721088in}{0.453578in}}%
\pgfpathlineto{\pgfqpoint{2.748787in}{0.453578in}}%
\pgfpathlineto{\pgfqpoint{2.776486in}{0.453578in}}%
\pgfpathlineto{\pgfqpoint{2.804186in}{0.453578in}}%
\pgfpathlineto{\pgfqpoint{2.831885in}{0.453578in}}%
\pgfpathlineto{\pgfqpoint{2.859584in}{0.453578in}}%
\pgfpathlineto{\pgfqpoint{2.887283in}{0.453578in}}%
\pgfpathlineto{\pgfqpoint{2.914983in}{0.583181in}}%
\pgfpathlineto{\pgfqpoint{2.942682in}{0.977584in}}%
\pgfpathlineto{\pgfqpoint{2.970381in}{1.225297in}}%
\pgfpathlineto{\pgfqpoint{2.998081in}{1.366590in}}%
\pgfpathlineto{\pgfqpoint{3.025780in}{1.431868in}}%
\pgfpathlineto{\pgfqpoint{3.053479in}{1.444037in}}%
\pgfpathlineto{\pgfqpoint{3.081178in}{1.420135in}}%
\pgfpathlineto{\pgfqpoint{3.108878in}{1.372852in}}%
\pgfpathlineto{\pgfqpoint{3.136577in}{1.311370in}}%
\pgfpathlineto{\pgfqpoint{3.164276in}{1.242354in}}%
\pgfpathlineto{\pgfqpoint{3.191976in}{1.170517in}}%
\pgfpathlineto{\pgfqpoint{3.219675in}{1.099086in}}%
\pgfpathlineto{\pgfqpoint{3.247374in}{1.030244in}}%
\pgfpathlineto{\pgfqpoint{3.275073in}{0.965369in}}%
\pgfpathlineto{\pgfqpoint{3.302773in}{0.905243in}}%
\pgfpathlineto{\pgfqpoint{3.330472in}{0.850438in}}%
\pgfpathlineto{\pgfqpoint{3.358171in}{0.801162in}}%
\pgfpathlineto{\pgfqpoint{3.385871in}{0.757276in}}%
\pgfpathlineto{\pgfqpoint{3.413570in}{0.718514in}}%
\pgfpathlineto{\pgfqpoint{3.441269in}{0.684479in}}%
\pgfpathlineto{\pgfqpoint{3.468968in}{0.654639in}}%
\pgfpathlineto{\pgfqpoint{3.496668in}{0.628555in}}%
\pgfpathlineto{\pgfqpoint{3.524367in}{0.605771in}}%
\pgfpathlineto{\pgfqpoint{3.552066in}{0.585882in}}%
\pgfpathlineto{\pgfqpoint{3.579766in}{0.568530in}}%
\pgfpathlineto{\pgfqpoint{3.607465in}{0.553401in}}%
\pgfpathlineto{\pgfqpoint{3.635164in}{0.540216in}}%
\pgfpathlineto{\pgfqpoint{3.662863in}{0.528733in}}%
\pgfpathlineto{\pgfqpoint{3.690563in}{0.518737in}}%
\pgfpathlineto{\pgfqpoint{3.718262in}{0.510053in}}%
\pgfpathlineto{\pgfqpoint{3.745961in}{0.502510in}}%
\pgfpathlineto{\pgfqpoint{3.773661in}{0.495959in}}%
\pgfpathlineto{\pgfqpoint{3.801360in}{0.490270in}}%
\pgfpathlineto{\pgfqpoint{3.829059in}{0.485332in}}%
\pgfpathlineto{\pgfqpoint{3.856758in}{0.481052in}}%
\pgfpathlineto{\pgfqpoint{3.884458in}{0.477343in}}%
\pgfpathlineto{\pgfqpoint{3.912157in}{0.474128in}}%
\pgfpathlineto{\pgfqpoint{3.939856in}{0.471342in}}%
\pgfpathlineto{\pgfqpoint{3.967556in}{0.468924in}}%
\pgfpathlineto{\pgfqpoint{3.995255in}{0.466831in}}%
\pgfpathlineto{\pgfqpoint{4.022954in}{0.465018in}}%
\pgfpathlineto{\pgfqpoint{4.033088in}{0.464443in}}%
\pgfusepath{stroke}%
\end{pgfscope}%
\begin{pgfscope}%
\pgfpathrectangle{\pgfqpoint{2.666838in}{0.383578in}}{\pgfqpoint{1.356250in}{1.540000in}}%
\pgfusepath{clip}%
\pgfsetrectcap%
\pgfsetroundjoin%
\pgfsetlinewidth{0.803000pt}%
\definecolor{currentstroke}{rgb}{0.686275,0.352941,0.313725}%
\pgfsetstrokecolor{currentstroke}%
\pgfsetstrokeopacity{0.300000}%
\pgfsetdash{}{0pt}%
\pgfpathmoveto{\pgfqpoint{2.721088in}{0.453578in}}%
\pgfpathlineto{\pgfqpoint{2.748787in}{0.453578in}}%
\pgfpathlineto{\pgfqpoint{2.776486in}{0.453578in}}%
\pgfpathlineto{\pgfqpoint{2.804186in}{0.453578in}}%
\pgfpathlineto{\pgfqpoint{2.831885in}{0.453578in}}%
\pgfpathlineto{\pgfqpoint{2.859584in}{0.453578in}}%
\pgfpathlineto{\pgfqpoint{2.887283in}{0.453578in}}%
\pgfpathlineto{\pgfqpoint{2.914983in}{0.453578in}}%
\pgfpathlineto{\pgfqpoint{2.942682in}{0.453578in}}%
\pgfpathlineto{\pgfqpoint{2.970381in}{0.453578in}}%
\pgfpathlineto{\pgfqpoint{2.998081in}{0.453578in}}%
\pgfpathlineto{\pgfqpoint{3.025780in}{0.453578in}}%
\pgfpathlineto{\pgfqpoint{3.053479in}{0.453578in}}%
\pgfpathlineto{\pgfqpoint{3.081178in}{0.453578in}}%
\pgfpathlineto{\pgfqpoint{3.108878in}{0.453578in}}%
\pgfpathlineto{\pgfqpoint{3.136577in}{0.453578in}}%
\pgfpathlineto{\pgfqpoint{3.164276in}{0.453578in}}%
\pgfpathlineto{\pgfqpoint{3.191976in}{0.453578in}}%
\pgfpathlineto{\pgfqpoint{3.219675in}{0.453578in}}%
\pgfpathlineto{\pgfqpoint{3.247374in}{0.453578in}}%
\pgfpathlineto{\pgfqpoint{3.275073in}{0.453578in}}%
\pgfpathlineto{\pgfqpoint{3.302773in}{0.453578in}}%
\pgfpathlineto{\pgfqpoint{3.330472in}{0.453578in}}%
\pgfpathlineto{\pgfqpoint{3.358171in}{0.453578in}}%
\pgfpathlineto{\pgfqpoint{3.385871in}{0.453578in}}%
\pgfpathlineto{\pgfqpoint{3.413570in}{0.453578in}}%
\pgfpathlineto{\pgfqpoint{3.441269in}{0.453578in}}%
\pgfpathlineto{\pgfqpoint{3.468968in}{0.453578in}}%
\pgfpathlineto{\pgfqpoint{3.496668in}{0.453578in}}%
\pgfpathlineto{\pgfqpoint{3.524367in}{0.453578in}}%
\pgfpathlineto{\pgfqpoint{3.552066in}{0.453578in}}%
\pgfpathlineto{\pgfqpoint{3.579766in}{0.453578in}}%
\pgfpathlineto{\pgfqpoint{3.607465in}{0.453578in}}%
\pgfpathlineto{\pgfqpoint{3.635164in}{0.453578in}}%
\pgfpathlineto{\pgfqpoint{3.662863in}{0.453578in}}%
\pgfpathlineto{\pgfqpoint{3.690563in}{0.453578in}}%
\pgfpathlineto{\pgfqpoint{3.718262in}{0.453578in}}%
\pgfpathlineto{\pgfqpoint{3.745961in}{0.453578in}}%
\pgfpathlineto{\pgfqpoint{3.773661in}{0.453578in}}%
\pgfpathlineto{\pgfqpoint{3.801360in}{0.453578in}}%
\pgfpathlineto{\pgfqpoint{3.829059in}{0.453578in}}%
\pgfpathlineto{\pgfqpoint{3.856758in}{0.453578in}}%
\pgfpathlineto{\pgfqpoint{3.884458in}{0.453578in}}%
\pgfpathlineto{\pgfqpoint{3.912157in}{0.453578in}}%
\pgfpathlineto{\pgfqpoint{3.939856in}{0.453578in}}%
\pgfpathlineto{\pgfqpoint{3.967556in}{0.453578in}}%
\pgfpathlineto{\pgfqpoint{3.995255in}{0.453578in}}%
\pgfpathlineto{\pgfqpoint{4.022954in}{0.453578in}}%
\pgfpathlineto{\pgfqpoint{4.033088in}{0.453578in}}%
\pgfusepath{stroke}%
\end{pgfscope}%
\begin{pgfscope}%
\pgfpathrectangle{\pgfqpoint{2.666838in}{0.383578in}}{\pgfqpoint{1.356250in}{1.540000in}}%
\pgfusepath{clip}%
\pgfsetrectcap%
\pgfsetroundjoin%
\pgfsetlinewidth{0.803000pt}%
\definecolor{currentstroke}{rgb}{0.686275,0.352941,0.313725}%
\pgfsetstrokecolor{currentstroke}%
\pgfsetstrokeopacity{0.300000}%
\pgfsetdash{}{0pt}%
\pgfpathmoveto{\pgfqpoint{2.721088in}{0.453578in}}%
\pgfpathlineto{\pgfqpoint{2.748787in}{0.453578in}}%
\pgfpathlineto{\pgfqpoint{2.776486in}{0.453578in}}%
\pgfpathlineto{\pgfqpoint{2.804186in}{0.453578in}}%
\pgfpathlineto{\pgfqpoint{2.831885in}{0.453578in}}%
\pgfpathlineto{\pgfqpoint{2.859584in}{0.453578in}}%
\pgfpathlineto{\pgfqpoint{2.887283in}{0.453578in}}%
\pgfpathlineto{\pgfqpoint{2.914983in}{0.453578in}}%
\pgfpathlineto{\pgfqpoint{2.942682in}{0.453578in}}%
\pgfpathlineto{\pgfqpoint{2.970381in}{0.453578in}}%
\pgfpathlineto{\pgfqpoint{2.998081in}{0.453578in}}%
\pgfpathlineto{\pgfqpoint{3.025780in}{0.453578in}}%
\pgfpathlineto{\pgfqpoint{3.053479in}{0.453578in}}%
\pgfpathlineto{\pgfqpoint{3.081178in}{0.453578in}}%
\pgfpathlineto{\pgfqpoint{3.108878in}{0.453578in}}%
\pgfpathlineto{\pgfqpoint{3.136577in}{0.453578in}}%
\pgfpathlineto{\pgfqpoint{3.164276in}{0.453578in}}%
\pgfpathlineto{\pgfqpoint{3.191976in}{0.453578in}}%
\pgfpathlineto{\pgfqpoint{3.219675in}{0.453578in}}%
\pgfpathlineto{\pgfqpoint{3.247374in}{0.453578in}}%
\pgfpathlineto{\pgfqpoint{3.275073in}{0.453578in}}%
\pgfpathlineto{\pgfqpoint{3.302773in}{0.453578in}}%
\pgfpathlineto{\pgfqpoint{3.330472in}{0.453578in}}%
\pgfpathlineto{\pgfqpoint{3.358171in}{0.453578in}}%
\pgfpathlineto{\pgfqpoint{3.385871in}{0.453578in}}%
\pgfpathlineto{\pgfqpoint{3.413570in}{0.453578in}}%
\pgfpathlineto{\pgfqpoint{3.441269in}{0.453578in}}%
\pgfpathlineto{\pgfqpoint{3.468968in}{0.453578in}}%
\pgfpathlineto{\pgfqpoint{3.496668in}{0.453578in}}%
\pgfpathlineto{\pgfqpoint{3.524367in}{0.453578in}}%
\pgfpathlineto{\pgfqpoint{3.552066in}{0.453578in}}%
\pgfpathlineto{\pgfqpoint{3.579766in}{0.453578in}}%
\pgfpathlineto{\pgfqpoint{3.607465in}{0.453578in}}%
\pgfpathlineto{\pgfqpoint{3.635164in}{0.453578in}}%
\pgfpathlineto{\pgfqpoint{3.662863in}{0.453578in}}%
\pgfpathlineto{\pgfqpoint{3.690563in}{0.453578in}}%
\pgfpathlineto{\pgfqpoint{3.718262in}{0.453578in}}%
\pgfpathlineto{\pgfqpoint{3.745961in}{0.453578in}}%
\pgfpathlineto{\pgfqpoint{3.773661in}{0.453578in}}%
\pgfpathlineto{\pgfqpoint{3.801360in}{0.453578in}}%
\pgfpathlineto{\pgfqpoint{3.829059in}{0.453578in}}%
\pgfpathlineto{\pgfqpoint{3.856758in}{0.453578in}}%
\pgfpathlineto{\pgfqpoint{3.884458in}{0.453578in}}%
\pgfpathlineto{\pgfqpoint{3.912157in}{0.453578in}}%
\pgfpathlineto{\pgfqpoint{3.939856in}{0.453578in}}%
\pgfpathlineto{\pgfqpoint{3.967556in}{0.453578in}}%
\pgfpathlineto{\pgfqpoint{3.995255in}{0.453578in}}%
\pgfpathlineto{\pgfqpoint{4.022954in}{0.453578in}}%
\pgfpathlineto{\pgfqpoint{4.033088in}{0.453578in}}%
\pgfusepath{stroke}%
\end{pgfscope}%
\begin{pgfscope}%
\pgfpathrectangle{\pgfqpoint{2.666838in}{0.383578in}}{\pgfqpoint{1.356250in}{1.540000in}}%
\pgfusepath{clip}%
\pgfsetrectcap%
\pgfsetroundjoin%
\pgfsetlinewidth{0.803000pt}%
\definecolor{currentstroke}{rgb}{0.686275,0.352941,0.313725}%
\pgfsetstrokecolor{currentstroke}%
\pgfsetstrokeopacity{0.300000}%
\pgfsetdash{}{0pt}%
\pgfpathmoveto{\pgfqpoint{2.721088in}{0.453578in}}%
\pgfpathlineto{\pgfqpoint{2.748787in}{0.453578in}}%
\pgfpathlineto{\pgfqpoint{2.776486in}{0.453578in}}%
\pgfpathlineto{\pgfqpoint{2.804186in}{0.453578in}}%
\pgfpathlineto{\pgfqpoint{2.831885in}{0.453578in}}%
\pgfpathlineto{\pgfqpoint{2.859584in}{0.453578in}}%
\pgfpathlineto{\pgfqpoint{2.887283in}{0.453578in}}%
\pgfpathlineto{\pgfqpoint{2.914983in}{0.453578in}}%
\pgfpathlineto{\pgfqpoint{2.942682in}{0.453578in}}%
\pgfpathlineto{\pgfqpoint{2.970381in}{0.453578in}}%
\pgfpathlineto{\pgfqpoint{2.998081in}{0.453578in}}%
\pgfpathlineto{\pgfqpoint{3.025780in}{0.453578in}}%
\pgfpathlineto{\pgfqpoint{3.053479in}{0.453578in}}%
\pgfpathlineto{\pgfqpoint{3.081178in}{0.453578in}}%
\pgfpathlineto{\pgfqpoint{3.108878in}{0.453578in}}%
\pgfpathlineto{\pgfqpoint{3.136577in}{0.453578in}}%
\pgfpathlineto{\pgfqpoint{3.164276in}{0.453578in}}%
\pgfpathlineto{\pgfqpoint{3.191976in}{0.453578in}}%
\pgfpathlineto{\pgfqpoint{3.219675in}{0.453578in}}%
\pgfpathlineto{\pgfqpoint{3.247374in}{0.453578in}}%
\pgfpathlineto{\pgfqpoint{3.275073in}{0.453578in}}%
\pgfpathlineto{\pgfqpoint{3.302773in}{0.453578in}}%
\pgfpathlineto{\pgfqpoint{3.330472in}{0.453578in}}%
\pgfpathlineto{\pgfqpoint{3.358171in}{0.453578in}}%
\pgfpathlineto{\pgfqpoint{3.385871in}{0.453578in}}%
\pgfpathlineto{\pgfqpoint{3.413570in}{0.453578in}}%
\pgfpathlineto{\pgfqpoint{3.441269in}{0.453578in}}%
\pgfpathlineto{\pgfqpoint{3.468968in}{0.453578in}}%
\pgfpathlineto{\pgfqpoint{3.496668in}{0.453578in}}%
\pgfpathlineto{\pgfqpoint{3.524367in}{0.453578in}}%
\pgfpathlineto{\pgfqpoint{3.552066in}{0.453578in}}%
\pgfpathlineto{\pgfqpoint{3.579766in}{0.453578in}}%
\pgfpathlineto{\pgfqpoint{3.607465in}{0.453578in}}%
\pgfpathlineto{\pgfqpoint{3.635164in}{0.453578in}}%
\pgfpathlineto{\pgfqpoint{3.662863in}{0.453578in}}%
\pgfpathlineto{\pgfqpoint{3.690563in}{0.453578in}}%
\pgfpathlineto{\pgfqpoint{3.718262in}{0.453578in}}%
\pgfpathlineto{\pgfqpoint{3.745961in}{0.453578in}}%
\pgfpathlineto{\pgfqpoint{3.773661in}{0.453578in}}%
\pgfpathlineto{\pgfqpoint{3.801360in}{0.453578in}}%
\pgfpathlineto{\pgfqpoint{3.829059in}{0.453578in}}%
\pgfpathlineto{\pgfqpoint{3.856758in}{0.453578in}}%
\pgfpathlineto{\pgfqpoint{3.884458in}{0.453578in}}%
\pgfpathlineto{\pgfqpoint{3.912157in}{0.453578in}}%
\pgfpathlineto{\pgfqpoint{3.939856in}{0.453578in}}%
\pgfpathlineto{\pgfqpoint{3.967556in}{0.453578in}}%
\pgfpathlineto{\pgfqpoint{3.995255in}{0.453578in}}%
\pgfpathlineto{\pgfqpoint{4.022954in}{0.453578in}}%
\pgfpathlineto{\pgfqpoint{4.033088in}{0.453578in}}%
\pgfusepath{stroke}%
\end{pgfscope}%
\begin{pgfscope}%
\pgfpathrectangle{\pgfqpoint{2.666838in}{0.383578in}}{\pgfqpoint{1.356250in}{1.540000in}}%
\pgfusepath{clip}%
\pgfsetrectcap%
\pgfsetroundjoin%
\pgfsetlinewidth{0.803000pt}%
\definecolor{currentstroke}{rgb}{0.686275,0.352941,0.313725}%
\pgfsetstrokecolor{currentstroke}%
\pgfsetstrokeopacity{0.300000}%
\pgfsetdash{}{0pt}%
\pgfpathmoveto{\pgfqpoint{2.721088in}{0.453578in}}%
\pgfpathlineto{\pgfqpoint{2.748787in}{0.453578in}}%
\pgfpathlineto{\pgfqpoint{2.776486in}{0.453578in}}%
\pgfpathlineto{\pgfqpoint{2.804186in}{0.453578in}}%
\pgfpathlineto{\pgfqpoint{2.831885in}{0.453578in}}%
\pgfpathlineto{\pgfqpoint{2.859584in}{0.453578in}}%
\pgfpathlineto{\pgfqpoint{2.887283in}{0.453578in}}%
\pgfpathlineto{\pgfqpoint{2.914983in}{0.453578in}}%
\pgfpathlineto{\pgfqpoint{2.942682in}{0.453578in}}%
\pgfpathlineto{\pgfqpoint{2.970381in}{0.453578in}}%
\pgfpathlineto{\pgfqpoint{2.998081in}{0.453578in}}%
\pgfpathlineto{\pgfqpoint{3.025780in}{0.453578in}}%
\pgfpathlineto{\pgfqpoint{3.053479in}{0.453578in}}%
\pgfpathlineto{\pgfqpoint{3.081178in}{0.453578in}}%
\pgfpathlineto{\pgfqpoint{3.108878in}{0.453578in}}%
\pgfpathlineto{\pgfqpoint{3.136577in}{0.453578in}}%
\pgfpathlineto{\pgfqpoint{3.164276in}{0.453578in}}%
\pgfpathlineto{\pgfqpoint{3.191976in}{0.453578in}}%
\pgfpathlineto{\pgfqpoint{3.219675in}{0.453578in}}%
\pgfpathlineto{\pgfqpoint{3.247374in}{0.453578in}}%
\pgfpathlineto{\pgfqpoint{3.275073in}{0.453578in}}%
\pgfpathlineto{\pgfqpoint{3.302773in}{0.453578in}}%
\pgfpathlineto{\pgfqpoint{3.330472in}{0.453578in}}%
\pgfpathlineto{\pgfqpoint{3.358171in}{0.453578in}}%
\pgfpathlineto{\pgfqpoint{3.385871in}{0.453578in}}%
\pgfpathlineto{\pgfqpoint{3.413570in}{0.453578in}}%
\pgfpathlineto{\pgfqpoint{3.441269in}{0.453578in}}%
\pgfpathlineto{\pgfqpoint{3.468968in}{0.453578in}}%
\pgfpathlineto{\pgfqpoint{3.496668in}{0.453578in}}%
\pgfpathlineto{\pgfqpoint{3.524367in}{0.453578in}}%
\pgfpathlineto{\pgfqpoint{3.552066in}{0.453578in}}%
\pgfpathlineto{\pgfqpoint{3.579766in}{0.453578in}}%
\pgfpathlineto{\pgfqpoint{3.607465in}{0.453578in}}%
\pgfpathlineto{\pgfqpoint{3.635164in}{0.453578in}}%
\pgfpathlineto{\pgfqpoint{3.662863in}{0.453578in}}%
\pgfpathlineto{\pgfqpoint{3.690563in}{0.453578in}}%
\pgfpathlineto{\pgfqpoint{3.718262in}{0.453578in}}%
\pgfpathlineto{\pgfqpoint{3.745961in}{0.453578in}}%
\pgfpathlineto{\pgfqpoint{3.773661in}{0.453578in}}%
\pgfpathlineto{\pgfqpoint{3.801360in}{0.453578in}}%
\pgfpathlineto{\pgfqpoint{3.829059in}{0.453578in}}%
\pgfpathlineto{\pgfqpoint{3.856758in}{0.453578in}}%
\pgfpathlineto{\pgfqpoint{3.884458in}{0.453578in}}%
\pgfpathlineto{\pgfqpoint{3.912157in}{0.453578in}}%
\pgfpathlineto{\pgfqpoint{3.939856in}{0.453578in}}%
\pgfpathlineto{\pgfqpoint{3.967556in}{0.453578in}}%
\pgfpathlineto{\pgfqpoint{3.995255in}{0.453578in}}%
\pgfpathlineto{\pgfqpoint{4.022954in}{0.453578in}}%
\pgfpathlineto{\pgfqpoint{4.033088in}{0.453578in}}%
\pgfusepath{stroke}%
\end{pgfscope}%
\begin{pgfscope}%
\pgfpathrectangle{\pgfqpoint{2.666838in}{0.383578in}}{\pgfqpoint{1.356250in}{1.540000in}}%
\pgfusepath{clip}%
\pgfsetrectcap%
\pgfsetroundjoin%
\pgfsetlinewidth{0.803000pt}%
\definecolor{currentstroke}{rgb}{0.686275,0.352941,0.313725}%
\pgfsetstrokecolor{currentstroke}%
\pgfsetstrokeopacity{0.300000}%
\pgfsetdash{}{0pt}%
\pgfpathmoveto{\pgfqpoint{2.721088in}{0.453578in}}%
\pgfpathlineto{\pgfqpoint{2.748787in}{0.453578in}}%
\pgfpathlineto{\pgfqpoint{2.776486in}{0.453578in}}%
\pgfpathlineto{\pgfqpoint{2.804186in}{0.453578in}}%
\pgfpathlineto{\pgfqpoint{2.831885in}{0.453578in}}%
\pgfpathlineto{\pgfqpoint{2.859584in}{0.453578in}}%
\pgfpathlineto{\pgfqpoint{2.887283in}{0.453578in}}%
\pgfpathlineto{\pgfqpoint{2.914983in}{0.453578in}}%
\pgfpathlineto{\pgfqpoint{2.942682in}{0.453578in}}%
\pgfpathlineto{\pgfqpoint{2.970381in}{0.453578in}}%
\pgfpathlineto{\pgfqpoint{2.998081in}{0.453578in}}%
\pgfpathlineto{\pgfqpoint{3.025780in}{0.453578in}}%
\pgfpathlineto{\pgfqpoint{3.053479in}{0.453578in}}%
\pgfpathlineto{\pgfqpoint{3.081178in}{0.453578in}}%
\pgfpathlineto{\pgfqpoint{3.108878in}{0.453578in}}%
\pgfpathlineto{\pgfqpoint{3.136577in}{0.453578in}}%
\pgfpathlineto{\pgfqpoint{3.164276in}{0.453578in}}%
\pgfpathlineto{\pgfqpoint{3.191976in}{0.453578in}}%
\pgfpathlineto{\pgfqpoint{3.219675in}{0.453578in}}%
\pgfpathlineto{\pgfqpoint{3.247374in}{0.453578in}}%
\pgfpathlineto{\pgfqpoint{3.275073in}{0.453578in}}%
\pgfpathlineto{\pgfqpoint{3.302773in}{0.453578in}}%
\pgfpathlineto{\pgfqpoint{3.330472in}{0.453578in}}%
\pgfpathlineto{\pgfqpoint{3.358171in}{0.453578in}}%
\pgfpathlineto{\pgfqpoint{3.385871in}{0.453578in}}%
\pgfpathlineto{\pgfqpoint{3.413570in}{0.453578in}}%
\pgfpathlineto{\pgfqpoint{3.441269in}{0.453578in}}%
\pgfpathlineto{\pgfqpoint{3.468968in}{0.453578in}}%
\pgfpathlineto{\pgfqpoint{3.496668in}{0.453578in}}%
\pgfpathlineto{\pgfqpoint{3.524367in}{0.453578in}}%
\pgfpathlineto{\pgfqpoint{3.552066in}{0.453578in}}%
\pgfpathlineto{\pgfqpoint{3.579766in}{0.453578in}}%
\pgfpathlineto{\pgfqpoint{3.607465in}{0.453578in}}%
\pgfpathlineto{\pgfqpoint{3.635164in}{0.453578in}}%
\pgfpathlineto{\pgfqpoint{3.662863in}{0.453578in}}%
\pgfpathlineto{\pgfqpoint{3.690563in}{0.453578in}}%
\pgfpathlineto{\pgfqpoint{3.718262in}{0.453578in}}%
\pgfpathlineto{\pgfqpoint{3.745961in}{0.453578in}}%
\pgfpathlineto{\pgfqpoint{3.773661in}{0.453578in}}%
\pgfpathlineto{\pgfqpoint{3.801360in}{0.453578in}}%
\pgfpathlineto{\pgfqpoint{3.829059in}{0.453578in}}%
\pgfpathlineto{\pgfqpoint{3.856758in}{0.453578in}}%
\pgfpathlineto{\pgfqpoint{3.884458in}{0.453578in}}%
\pgfpathlineto{\pgfqpoint{3.912157in}{0.453578in}}%
\pgfpathlineto{\pgfqpoint{3.939856in}{0.453578in}}%
\pgfpathlineto{\pgfqpoint{3.967556in}{0.453578in}}%
\pgfpathlineto{\pgfqpoint{3.995255in}{0.453578in}}%
\pgfpathlineto{\pgfqpoint{4.022954in}{0.453578in}}%
\pgfpathlineto{\pgfqpoint{4.033088in}{0.453578in}}%
\pgfusepath{stroke}%
\end{pgfscope}%
\begin{pgfscope}%
\pgfpathrectangle{\pgfqpoint{2.666838in}{0.383578in}}{\pgfqpoint{1.356250in}{1.540000in}}%
\pgfusepath{clip}%
\pgfsetrectcap%
\pgfsetroundjoin%
\pgfsetlinewidth{0.803000pt}%
\definecolor{currentstroke}{rgb}{0.686275,0.352941,0.313725}%
\pgfsetstrokecolor{currentstroke}%
\pgfsetstrokeopacity{0.300000}%
\pgfsetdash{}{0pt}%
\pgfpathmoveto{\pgfqpoint{2.721088in}{0.453578in}}%
\pgfpathlineto{\pgfqpoint{2.748787in}{0.453578in}}%
\pgfpathlineto{\pgfqpoint{2.776486in}{0.453578in}}%
\pgfpathlineto{\pgfqpoint{2.804186in}{0.453578in}}%
\pgfpathlineto{\pgfqpoint{2.831885in}{0.453578in}}%
\pgfpathlineto{\pgfqpoint{2.859584in}{0.453578in}}%
\pgfpathlineto{\pgfqpoint{2.887283in}{0.453578in}}%
\pgfpathlineto{\pgfqpoint{2.914983in}{0.453578in}}%
\pgfpathlineto{\pgfqpoint{2.942682in}{0.453578in}}%
\pgfpathlineto{\pgfqpoint{2.970381in}{0.453578in}}%
\pgfpathlineto{\pgfqpoint{2.998081in}{0.453578in}}%
\pgfpathlineto{\pgfqpoint{3.025780in}{0.453578in}}%
\pgfpathlineto{\pgfqpoint{3.053479in}{0.453578in}}%
\pgfpathlineto{\pgfqpoint{3.081178in}{0.453578in}}%
\pgfpathlineto{\pgfqpoint{3.108878in}{0.453578in}}%
\pgfpathlineto{\pgfqpoint{3.136577in}{0.453578in}}%
\pgfpathlineto{\pgfqpoint{3.164276in}{0.453578in}}%
\pgfpathlineto{\pgfqpoint{3.191976in}{0.453578in}}%
\pgfpathlineto{\pgfqpoint{3.219675in}{0.453578in}}%
\pgfpathlineto{\pgfqpoint{3.247374in}{0.453578in}}%
\pgfpathlineto{\pgfqpoint{3.275073in}{0.453578in}}%
\pgfpathlineto{\pgfqpoint{3.302773in}{0.453578in}}%
\pgfpathlineto{\pgfqpoint{3.330472in}{0.453578in}}%
\pgfpathlineto{\pgfqpoint{3.358171in}{0.453578in}}%
\pgfpathlineto{\pgfqpoint{3.385871in}{0.453578in}}%
\pgfpathlineto{\pgfqpoint{3.413570in}{0.453578in}}%
\pgfpathlineto{\pgfqpoint{3.441269in}{0.453578in}}%
\pgfpathlineto{\pgfqpoint{3.468968in}{0.453578in}}%
\pgfpathlineto{\pgfqpoint{3.496668in}{0.453578in}}%
\pgfpathlineto{\pgfqpoint{3.524367in}{0.453578in}}%
\pgfpathlineto{\pgfqpoint{3.552066in}{0.453578in}}%
\pgfpathlineto{\pgfqpoint{3.579766in}{0.453578in}}%
\pgfpathlineto{\pgfqpoint{3.607465in}{0.453578in}}%
\pgfpathlineto{\pgfqpoint{3.635164in}{0.453578in}}%
\pgfpathlineto{\pgfqpoint{3.662863in}{0.453578in}}%
\pgfpathlineto{\pgfqpoint{3.690563in}{0.453578in}}%
\pgfpathlineto{\pgfqpoint{3.718262in}{0.453578in}}%
\pgfpathlineto{\pgfqpoint{3.745961in}{0.453578in}}%
\pgfpathlineto{\pgfqpoint{3.773661in}{0.453578in}}%
\pgfpathlineto{\pgfqpoint{3.801360in}{0.453578in}}%
\pgfpathlineto{\pgfqpoint{3.829059in}{0.453578in}}%
\pgfpathlineto{\pgfqpoint{3.856758in}{0.453578in}}%
\pgfpathlineto{\pgfqpoint{3.884458in}{0.453578in}}%
\pgfpathlineto{\pgfqpoint{3.912157in}{0.453578in}}%
\pgfpathlineto{\pgfqpoint{3.939856in}{0.453578in}}%
\pgfpathlineto{\pgfqpoint{3.967556in}{0.453578in}}%
\pgfpathlineto{\pgfqpoint{3.995255in}{0.453578in}}%
\pgfpathlineto{\pgfqpoint{4.022954in}{0.453578in}}%
\pgfpathlineto{\pgfqpoint{4.033088in}{0.453578in}}%
\pgfusepath{stroke}%
\end{pgfscope}%
\begin{pgfscope}%
\pgfpathrectangle{\pgfqpoint{2.666838in}{0.383578in}}{\pgfqpoint{1.356250in}{1.540000in}}%
\pgfusepath{clip}%
\pgfsetrectcap%
\pgfsetroundjoin%
\pgfsetlinewidth{0.803000pt}%
\definecolor{currentstroke}{rgb}{0.686275,0.352941,0.313725}%
\pgfsetstrokecolor{currentstroke}%
\pgfsetstrokeopacity{0.300000}%
\pgfsetdash{}{0pt}%
\pgfpathmoveto{\pgfqpoint{2.721088in}{0.453578in}}%
\pgfpathlineto{\pgfqpoint{2.748787in}{0.453578in}}%
\pgfpathlineto{\pgfqpoint{2.776486in}{0.453578in}}%
\pgfpathlineto{\pgfqpoint{2.804186in}{0.453578in}}%
\pgfpathlineto{\pgfqpoint{2.831885in}{0.453578in}}%
\pgfpathlineto{\pgfqpoint{2.859584in}{0.453578in}}%
\pgfpathlineto{\pgfqpoint{2.887283in}{0.453578in}}%
\pgfpathlineto{\pgfqpoint{2.914983in}{0.583181in}}%
\pgfpathlineto{\pgfqpoint{2.942682in}{0.977584in}}%
\pgfpathlineto{\pgfqpoint{2.970381in}{1.225297in}}%
\pgfpathlineto{\pgfqpoint{2.998081in}{1.366590in}}%
\pgfpathlineto{\pgfqpoint{3.025780in}{1.431868in}}%
\pgfpathlineto{\pgfqpoint{3.053479in}{1.444037in}}%
\pgfpathlineto{\pgfqpoint{3.081178in}{1.420135in}}%
\pgfpathlineto{\pgfqpoint{3.108878in}{1.372852in}}%
\pgfpathlineto{\pgfqpoint{3.136577in}{1.311370in}}%
\pgfpathlineto{\pgfqpoint{3.164276in}{1.242354in}}%
\pgfpathlineto{\pgfqpoint{3.191976in}{1.170517in}}%
\pgfpathlineto{\pgfqpoint{3.219675in}{1.099086in}}%
\pgfpathlineto{\pgfqpoint{3.247374in}{1.030244in}}%
\pgfpathlineto{\pgfqpoint{3.275073in}{0.965369in}}%
\pgfpathlineto{\pgfqpoint{3.302773in}{0.905243in}}%
\pgfpathlineto{\pgfqpoint{3.330472in}{0.850438in}}%
\pgfpathlineto{\pgfqpoint{3.358171in}{0.801162in}}%
\pgfpathlineto{\pgfqpoint{3.385871in}{0.757276in}}%
\pgfpathlineto{\pgfqpoint{3.413570in}{0.718514in}}%
\pgfpathlineto{\pgfqpoint{3.441269in}{0.684479in}}%
\pgfpathlineto{\pgfqpoint{3.468968in}{0.654639in}}%
\pgfpathlineto{\pgfqpoint{3.496668in}{0.628555in}}%
\pgfpathlineto{\pgfqpoint{3.524367in}{0.605771in}}%
\pgfpathlineto{\pgfqpoint{3.552066in}{0.585882in}}%
\pgfpathlineto{\pgfqpoint{3.579766in}{0.568530in}}%
\pgfpathlineto{\pgfqpoint{3.607465in}{0.553401in}}%
\pgfpathlineto{\pgfqpoint{3.635164in}{0.540216in}}%
\pgfpathlineto{\pgfqpoint{3.662863in}{0.528733in}}%
\pgfpathlineto{\pgfqpoint{3.690563in}{0.518737in}}%
\pgfpathlineto{\pgfqpoint{3.718262in}{0.510053in}}%
\pgfpathlineto{\pgfqpoint{3.745961in}{0.502510in}}%
\pgfpathlineto{\pgfqpoint{3.773661in}{0.495959in}}%
\pgfpathlineto{\pgfqpoint{3.801360in}{0.490270in}}%
\pgfpathlineto{\pgfqpoint{3.829059in}{0.485332in}}%
\pgfpathlineto{\pgfqpoint{3.856758in}{0.481052in}}%
\pgfpathlineto{\pgfqpoint{3.884458in}{0.477343in}}%
\pgfpathlineto{\pgfqpoint{3.912157in}{0.474128in}}%
\pgfpathlineto{\pgfqpoint{3.939856in}{0.471342in}}%
\pgfpathlineto{\pgfqpoint{3.967556in}{0.468924in}}%
\pgfpathlineto{\pgfqpoint{3.995255in}{0.466831in}}%
\pgfpathlineto{\pgfqpoint{4.022954in}{0.465018in}}%
\pgfpathlineto{\pgfqpoint{4.033088in}{0.464443in}}%
\pgfusepath{stroke}%
\end{pgfscope}%
\begin{pgfscope}%
\pgfpathrectangle{\pgfqpoint{2.666838in}{0.383578in}}{\pgfqpoint{1.356250in}{1.540000in}}%
\pgfusepath{clip}%
\pgfsetrectcap%
\pgfsetroundjoin%
\pgfsetlinewidth{0.803000pt}%
\definecolor{currentstroke}{rgb}{0.686275,0.352941,0.313725}%
\pgfsetstrokecolor{currentstroke}%
\pgfsetstrokeopacity{0.300000}%
\pgfsetdash{}{0pt}%
\pgfpathmoveto{\pgfqpoint{2.721088in}{0.453578in}}%
\pgfpathlineto{\pgfqpoint{2.748787in}{0.453578in}}%
\pgfpathlineto{\pgfqpoint{2.776486in}{0.453578in}}%
\pgfpathlineto{\pgfqpoint{2.804186in}{0.453578in}}%
\pgfpathlineto{\pgfqpoint{2.831885in}{0.453578in}}%
\pgfpathlineto{\pgfqpoint{2.859584in}{0.453578in}}%
\pgfpathlineto{\pgfqpoint{2.887283in}{0.453578in}}%
\pgfpathlineto{\pgfqpoint{2.914983in}{0.453578in}}%
\pgfpathlineto{\pgfqpoint{2.942682in}{0.453578in}}%
\pgfpathlineto{\pgfqpoint{2.970381in}{0.453578in}}%
\pgfpathlineto{\pgfqpoint{2.998081in}{0.453578in}}%
\pgfpathlineto{\pgfqpoint{3.025780in}{0.453578in}}%
\pgfpathlineto{\pgfqpoint{3.053479in}{0.453578in}}%
\pgfpathlineto{\pgfqpoint{3.081178in}{0.453578in}}%
\pgfpathlineto{\pgfqpoint{3.108878in}{0.453578in}}%
\pgfpathlineto{\pgfqpoint{3.136577in}{0.453578in}}%
\pgfpathlineto{\pgfqpoint{3.164276in}{0.453578in}}%
\pgfpathlineto{\pgfqpoint{3.191976in}{0.453578in}}%
\pgfpathlineto{\pgfqpoint{3.219675in}{0.453578in}}%
\pgfpathlineto{\pgfqpoint{3.247374in}{0.453578in}}%
\pgfpathlineto{\pgfqpoint{3.275073in}{0.453578in}}%
\pgfpathlineto{\pgfqpoint{3.302773in}{0.453578in}}%
\pgfpathlineto{\pgfqpoint{3.330472in}{0.453578in}}%
\pgfpathlineto{\pgfqpoint{3.358171in}{0.453578in}}%
\pgfpathlineto{\pgfqpoint{3.385871in}{0.453578in}}%
\pgfpathlineto{\pgfqpoint{3.413570in}{0.453578in}}%
\pgfpathlineto{\pgfqpoint{3.441269in}{0.453578in}}%
\pgfpathlineto{\pgfqpoint{3.468968in}{0.453578in}}%
\pgfpathlineto{\pgfqpoint{3.496668in}{0.453578in}}%
\pgfpathlineto{\pgfqpoint{3.524367in}{0.453578in}}%
\pgfpathlineto{\pgfqpoint{3.552066in}{0.453578in}}%
\pgfpathlineto{\pgfqpoint{3.579766in}{0.453578in}}%
\pgfpathlineto{\pgfqpoint{3.607465in}{0.453578in}}%
\pgfpathlineto{\pgfqpoint{3.635164in}{0.453578in}}%
\pgfpathlineto{\pgfqpoint{3.662863in}{0.453578in}}%
\pgfpathlineto{\pgfqpoint{3.690563in}{0.453578in}}%
\pgfpathlineto{\pgfqpoint{3.718262in}{0.453578in}}%
\pgfpathlineto{\pgfqpoint{3.745961in}{0.453578in}}%
\pgfpathlineto{\pgfqpoint{3.773661in}{0.453578in}}%
\pgfpathlineto{\pgfqpoint{3.801360in}{0.453578in}}%
\pgfpathlineto{\pgfqpoint{3.829059in}{0.453578in}}%
\pgfpathlineto{\pgfqpoint{3.856758in}{0.453578in}}%
\pgfpathlineto{\pgfqpoint{3.884458in}{0.453578in}}%
\pgfpathlineto{\pgfqpoint{3.912157in}{0.453578in}}%
\pgfpathlineto{\pgfqpoint{3.939856in}{0.453578in}}%
\pgfpathlineto{\pgfqpoint{3.967556in}{0.453578in}}%
\pgfpathlineto{\pgfqpoint{3.995255in}{0.453578in}}%
\pgfpathlineto{\pgfqpoint{4.022954in}{0.453578in}}%
\pgfpathlineto{\pgfqpoint{4.033088in}{0.453578in}}%
\pgfusepath{stroke}%
\end{pgfscope}%
\begin{pgfscope}%
\pgfpathrectangle{\pgfqpoint{2.666838in}{0.383578in}}{\pgfqpoint{1.356250in}{1.540000in}}%
\pgfusepath{clip}%
\pgfsetrectcap%
\pgfsetroundjoin%
\pgfsetlinewidth{0.803000pt}%
\definecolor{currentstroke}{rgb}{0.686275,0.352941,0.313725}%
\pgfsetstrokecolor{currentstroke}%
\pgfsetstrokeopacity{0.300000}%
\pgfsetdash{}{0pt}%
\pgfpathmoveto{\pgfqpoint{2.721088in}{0.453578in}}%
\pgfpathlineto{\pgfqpoint{2.748787in}{0.453578in}}%
\pgfpathlineto{\pgfqpoint{2.776486in}{0.453578in}}%
\pgfpathlineto{\pgfqpoint{2.804186in}{0.453578in}}%
\pgfpathlineto{\pgfqpoint{2.831885in}{0.453578in}}%
\pgfpathlineto{\pgfqpoint{2.859584in}{0.453578in}}%
\pgfpathlineto{\pgfqpoint{2.887283in}{0.453578in}}%
\pgfpathlineto{\pgfqpoint{2.914983in}{0.453578in}}%
\pgfpathlineto{\pgfqpoint{2.942682in}{0.453578in}}%
\pgfpathlineto{\pgfqpoint{2.970381in}{0.453578in}}%
\pgfpathlineto{\pgfqpoint{2.998081in}{0.453578in}}%
\pgfpathlineto{\pgfqpoint{3.025780in}{0.453578in}}%
\pgfpathlineto{\pgfqpoint{3.053479in}{0.453578in}}%
\pgfpathlineto{\pgfqpoint{3.081178in}{0.453578in}}%
\pgfpathlineto{\pgfqpoint{3.108878in}{0.453578in}}%
\pgfpathlineto{\pgfqpoint{3.136577in}{0.453578in}}%
\pgfpathlineto{\pgfqpoint{3.164276in}{0.453578in}}%
\pgfpathlineto{\pgfqpoint{3.191976in}{0.453578in}}%
\pgfpathlineto{\pgfqpoint{3.219675in}{0.453578in}}%
\pgfpathlineto{\pgfqpoint{3.247374in}{0.453578in}}%
\pgfpathlineto{\pgfqpoint{3.275073in}{0.453578in}}%
\pgfpathlineto{\pgfqpoint{3.302773in}{0.453578in}}%
\pgfpathlineto{\pgfqpoint{3.330472in}{0.453578in}}%
\pgfpathlineto{\pgfqpoint{3.358171in}{0.453578in}}%
\pgfpathlineto{\pgfqpoint{3.385871in}{0.453578in}}%
\pgfpathlineto{\pgfqpoint{3.413570in}{0.453578in}}%
\pgfpathlineto{\pgfqpoint{3.441269in}{0.453578in}}%
\pgfpathlineto{\pgfqpoint{3.468968in}{0.453578in}}%
\pgfpathlineto{\pgfqpoint{3.496668in}{0.453578in}}%
\pgfpathlineto{\pgfqpoint{3.524367in}{0.453578in}}%
\pgfpathlineto{\pgfqpoint{3.552066in}{0.453578in}}%
\pgfpathlineto{\pgfqpoint{3.579766in}{0.453578in}}%
\pgfpathlineto{\pgfqpoint{3.607465in}{0.453578in}}%
\pgfpathlineto{\pgfqpoint{3.635164in}{0.453578in}}%
\pgfpathlineto{\pgfqpoint{3.662863in}{0.453578in}}%
\pgfpathlineto{\pgfqpoint{3.690563in}{0.453578in}}%
\pgfpathlineto{\pgfqpoint{3.718262in}{0.453578in}}%
\pgfpathlineto{\pgfqpoint{3.745961in}{0.453578in}}%
\pgfpathlineto{\pgfqpoint{3.773661in}{0.453578in}}%
\pgfpathlineto{\pgfqpoint{3.801360in}{0.453578in}}%
\pgfpathlineto{\pgfqpoint{3.829059in}{0.453578in}}%
\pgfpathlineto{\pgfqpoint{3.856758in}{0.453578in}}%
\pgfpathlineto{\pgfqpoint{3.884458in}{0.453578in}}%
\pgfpathlineto{\pgfqpoint{3.912157in}{0.453578in}}%
\pgfpathlineto{\pgfqpoint{3.939856in}{0.453578in}}%
\pgfpathlineto{\pgfqpoint{3.967556in}{0.453578in}}%
\pgfpathlineto{\pgfqpoint{3.995255in}{0.453578in}}%
\pgfpathlineto{\pgfqpoint{4.022954in}{0.453578in}}%
\pgfpathlineto{\pgfqpoint{4.033088in}{0.453578in}}%
\pgfusepath{stroke}%
\end{pgfscope}%
\begin{pgfscope}%
\pgfpathrectangle{\pgfqpoint{2.666838in}{0.383578in}}{\pgfqpoint{1.356250in}{1.540000in}}%
\pgfusepath{clip}%
\pgfsetrectcap%
\pgfsetroundjoin%
\pgfsetlinewidth{0.803000pt}%
\definecolor{currentstroke}{rgb}{0.686275,0.352941,0.313725}%
\pgfsetstrokecolor{currentstroke}%
\pgfsetstrokeopacity{0.300000}%
\pgfsetdash{}{0pt}%
\pgfpathmoveto{\pgfqpoint{2.721088in}{0.453578in}}%
\pgfpathlineto{\pgfqpoint{2.748787in}{0.453578in}}%
\pgfpathlineto{\pgfqpoint{2.776486in}{0.453578in}}%
\pgfpathlineto{\pgfqpoint{2.804186in}{0.453578in}}%
\pgfpathlineto{\pgfqpoint{2.831885in}{0.453578in}}%
\pgfpathlineto{\pgfqpoint{2.859584in}{0.453578in}}%
\pgfpathlineto{\pgfqpoint{2.887283in}{0.453578in}}%
\pgfpathlineto{\pgfqpoint{2.914983in}{0.453578in}}%
\pgfpathlineto{\pgfqpoint{2.942682in}{0.453578in}}%
\pgfpathlineto{\pgfqpoint{2.970381in}{0.453578in}}%
\pgfpathlineto{\pgfqpoint{2.998081in}{0.453578in}}%
\pgfpathlineto{\pgfqpoint{3.025780in}{0.453578in}}%
\pgfpathlineto{\pgfqpoint{3.053479in}{0.453578in}}%
\pgfpathlineto{\pgfqpoint{3.081178in}{0.453578in}}%
\pgfpathlineto{\pgfqpoint{3.108878in}{0.453578in}}%
\pgfpathlineto{\pgfqpoint{3.136577in}{0.453578in}}%
\pgfpathlineto{\pgfqpoint{3.164276in}{0.453578in}}%
\pgfpathlineto{\pgfqpoint{3.191976in}{0.453578in}}%
\pgfpathlineto{\pgfqpoint{3.219675in}{0.453578in}}%
\pgfpathlineto{\pgfqpoint{3.247374in}{0.453578in}}%
\pgfpathlineto{\pgfqpoint{3.275073in}{0.453578in}}%
\pgfpathlineto{\pgfqpoint{3.302773in}{0.453578in}}%
\pgfpathlineto{\pgfqpoint{3.330472in}{0.453578in}}%
\pgfpathlineto{\pgfqpoint{3.358171in}{0.453578in}}%
\pgfpathlineto{\pgfqpoint{3.385871in}{0.453578in}}%
\pgfpathlineto{\pgfqpoint{3.413570in}{0.453578in}}%
\pgfpathlineto{\pgfqpoint{3.441269in}{0.453578in}}%
\pgfpathlineto{\pgfqpoint{3.468968in}{0.453578in}}%
\pgfpathlineto{\pgfqpoint{3.496668in}{0.453578in}}%
\pgfpathlineto{\pgfqpoint{3.524367in}{0.453578in}}%
\pgfpathlineto{\pgfqpoint{3.552066in}{0.453578in}}%
\pgfpathlineto{\pgfqpoint{3.579766in}{0.453578in}}%
\pgfpathlineto{\pgfqpoint{3.607465in}{0.453578in}}%
\pgfpathlineto{\pgfqpoint{3.635164in}{0.453578in}}%
\pgfpathlineto{\pgfqpoint{3.662863in}{0.453578in}}%
\pgfpathlineto{\pgfqpoint{3.690563in}{0.453578in}}%
\pgfpathlineto{\pgfqpoint{3.718262in}{0.453578in}}%
\pgfpathlineto{\pgfqpoint{3.745961in}{0.453578in}}%
\pgfpathlineto{\pgfqpoint{3.773661in}{0.453578in}}%
\pgfpathlineto{\pgfqpoint{3.801360in}{0.453578in}}%
\pgfpathlineto{\pgfqpoint{3.829059in}{0.453578in}}%
\pgfpathlineto{\pgfqpoint{3.856758in}{0.453578in}}%
\pgfpathlineto{\pgfqpoint{3.884458in}{0.453578in}}%
\pgfpathlineto{\pgfqpoint{3.912157in}{0.453578in}}%
\pgfpathlineto{\pgfqpoint{3.939856in}{0.453578in}}%
\pgfpathlineto{\pgfqpoint{3.967556in}{0.453578in}}%
\pgfpathlineto{\pgfqpoint{3.995255in}{0.453578in}}%
\pgfpathlineto{\pgfqpoint{4.022954in}{0.453578in}}%
\pgfpathlineto{\pgfqpoint{4.033088in}{0.453578in}}%
\pgfusepath{stroke}%
\end{pgfscope}%
\begin{pgfscope}%
\pgfpathrectangle{\pgfqpoint{2.666838in}{0.383578in}}{\pgfqpoint{1.356250in}{1.540000in}}%
\pgfusepath{clip}%
\pgfsetrectcap%
\pgfsetroundjoin%
\pgfsetlinewidth{0.803000pt}%
\definecolor{currentstroke}{rgb}{0.686275,0.352941,0.313725}%
\pgfsetstrokecolor{currentstroke}%
\pgfsetstrokeopacity{0.300000}%
\pgfsetdash{}{0pt}%
\pgfpathmoveto{\pgfqpoint{2.721088in}{0.453578in}}%
\pgfpathlineto{\pgfqpoint{2.748787in}{0.453578in}}%
\pgfpathlineto{\pgfqpoint{2.776486in}{0.453578in}}%
\pgfpathlineto{\pgfqpoint{2.804186in}{0.453578in}}%
\pgfpathlineto{\pgfqpoint{2.831885in}{0.453578in}}%
\pgfpathlineto{\pgfqpoint{2.859584in}{0.453578in}}%
\pgfpathlineto{\pgfqpoint{2.887283in}{0.453578in}}%
\pgfpathlineto{\pgfqpoint{2.914983in}{0.453578in}}%
\pgfpathlineto{\pgfqpoint{2.942682in}{0.453578in}}%
\pgfpathlineto{\pgfqpoint{2.970381in}{0.453578in}}%
\pgfpathlineto{\pgfqpoint{2.998081in}{0.453578in}}%
\pgfpathlineto{\pgfqpoint{3.025780in}{0.453578in}}%
\pgfpathlineto{\pgfqpoint{3.053479in}{0.453578in}}%
\pgfpathlineto{\pgfqpoint{3.081178in}{0.453578in}}%
\pgfpathlineto{\pgfqpoint{3.108878in}{0.453578in}}%
\pgfpathlineto{\pgfqpoint{3.136577in}{0.453578in}}%
\pgfpathlineto{\pgfqpoint{3.164276in}{0.453578in}}%
\pgfpathlineto{\pgfqpoint{3.191976in}{0.453578in}}%
\pgfpathlineto{\pgfqpoint{3.219675in}{0.453578in}}%
\pgfpathlineto{\pgfqpoint{3.247374in}{0.453578in}}%
\pgfpathlineto{\pgfqpoint{3.275073in}{0.453578in}}%
\pgfpathlineto{\pgfqpoint{3.302773in}{0.453578in}}%
\pgfpathlineto{\pgfqpoint{3.330472in}{0.453578in}}%
\pgfpathlineto{\pgfqpoint{3.358171in}{0.453578in}}%
\pgfpathlineto{\pgfqpoint{3.385871in}{0.453578in}}%
\pgfpathlineto{\pgfqpoint{3.413570in}{0.453578in}}%
\pgfpathlineto{\pgfqpoint{3.441269in}{0.453578in}}%
\pgfpathlineto{\pgfqpoint{3.468968in}{0.453578in}}%
\pgfpathlineto{\pgfqpoint{3.496668in}{0.453578in}}%
\pgfpathlineto{\pgfqpoint{3.524367in}{0.453578in}}%
\pgfpathlineto{\pgfqpoint{3.552066in}{0.453578in}}%
\pgfpathlineto{\pgfqpoint{3.579766in}{0.453578in}}%
\pgfpathlineto{\pgfqpoint{3.607465in}{0.453578in}}%
\pgfpathlineto{\pgfqpoint{3.635164in}{0.453578in}}%
\pgfpathlineto{\pgfqpoint{3.662863in}{0.453578in}}%
\pgfpathlineto{\pgfqpoint{3.690563in}{0.453578in}}%
\pgfpathlineto{\pgfqpoint{3.718262in}{0.453578in}}%
\pgfpathlineto{\pgfqpoint{3.745961in}{0.453578in}}%
\pgfpathlineto{\pgfqpoint{3.773661in}{0.453578in}}%
\pgfpathlineto{\pgfqpoint{3.801360in}{0.453578in}}%
\pgfpathlineto{\pgfqpoint{3.829059in}{0.453578in}}%
\pgfpathlineto{\pgfqpoint{3.856758in}{0.453578in}}%
\pgfpathlineto{\pgfqpoint{3.884458in}{0.453578in}}%
\pgfpathlineto{\pgfqpoint{3.912157in}{0.453578in}}%
\pgfpathlineto{\pgfqpoint{3.939856in}{0.453578in}}%
\pgfpathlineto{\pgfqpoint{3.967556in}{0.453578in}}%
\pgfpathlineto{\pgfqpoint{3.995255in}{0.453578in}}%
\pgfpathlineto{\pgfqpoint{4.022954in}{0.453578in}}%
\pgfpathlineto{\pgfqpoint{4.033088in}{0.453578in}}%
\pgfusepath{stroke}%
\end{pgfscope}%
\begin{pgfscope}%
\pgfpathrectangle{\pgfqpoint{2.666838in}{0.383578in}}{\pgfqpoint{1.356250in}{1.540000in}}%
\pgfusepath{clip}%
\pgfsetrectcap%
\pgfsetroundjoin%
\pgfsetlinewidth{0.803000pt}%
\definecolor{currentstroke}{rgb}{0.686275,0.352941,0.313725}%
\pgfsetstrokecolor{currentstroke}%
\pgfsetstrokeopacity{0.300000}%
\pgfsetdash{}{0pt}%
\pgfpathmoveto{\pgfqpoint{2.721088in}{0.453578in}}%
\pgfpathlineto{\pgfqpoint{2.748787in}{0.453578in}}%
\pgfpathlineto{\pgfqpoint{2.776486in}{0.453578in}}%
\pgfpathlineto{\pgfqpoint{2.804186in}{0.453578in}}%
\pgfpathlineto{\pgfqpoint{2.831885in}{0.453578in}}%
\pgfpathlineto{\pgfqpoint{2.859584in}{0.453578in}}%
\pgfpathlineto{\pgfqpoint{2.887283in}{0.453578in}}%
\pgfpathlineto{\pgfqpoint{2.914983in}{0.453578in}}%
\pgfpathlineto{\pgfqpoint{2.942682in}{0.453578in}}%
\pgfpathlineto{\pgfqpoint{2.970381in}{0.453578in}}%
\pgfpathlineto{\pgfqpoint{2.998081in}{0.453578in}}%
\pgfpathlineto{\pgfqpoint{3.025780in}{0.453578in}}%
\pgfpathlineto{\pgfqpoint{3.053479in}{0.453578in}}%
\pgfpathlineto{\pgfqpoint{3.081178in}{0.453578in}}%
\pgfpathlineto{\pgfqpoint{3.108878in}{0.453578in}}%
\pgfpathlineto{\pgfqpoint{3.136577in}{0.453578in}}%
\pgfpathlineto{\pgfqpoint{3.164276in}{0.453578in}}%
\pgfpathlineto{\pgfqpoint{3.191976in}{0.453578in}}%
\pgfpathlineto{\pgfqpoint{3.219675in}{0.453578in}}%
\pgfpathlineto{\pgfqpoint{3.247374in}{0.453578in}}%
\pgfpathlineto{\pgfqpoint{3.275073in}{0.453578in}}%
\pgfpathlineto{\pgfqpoint{3.302773in}{0.453578in}}%
\pgfpathlineto{\pgfqpoint{3.330472in}{0.453578in}}%
\pgfpathlineto{\pgfqpoint{3.358171in}{0.453578in}}%
\pgfpathlineto{\pgfqpoint{3.385871in}{0.453578in}}%
\pgfpathlineto{\pgfqpoint{3.413570in}{0.453578in}}%
\pgfpathlineto{\pgfqpoint{3.441269in}{0.453578in}}%
\pgfpathlineto{\pgfqpoint{3.468968in}{0.453578in}}%
\pgfpathlineto{\pgfqpoint{3.496668in}{0.453578in}}%
\pgfpathlineto{\pgfqpoint{3.524367in}{0.453578in}}%
\pgfpathlineto{\pgfqpoint{3.552066in}{0.453578in}}%
\pgfpathlineto{\pgfqpoint{3.579766in}{0.453578in}}%
\pgfpathlineto{\pgfqpoint{3.607465in}{0.453578in}}%
\pgfpathlineto{\pgfqpoint{3.635164in}{0.453578in}}%
\pgfpathlineto{\pgfqpoint{3.662863in}{0.453578in}}%
\pgfpathlineto{\pgfqpoint{3.690563in}{0.453578in}}%
\pgfpathlineto{\pgfqpoint{3.718262in}{0.453578in}}%
\pgfpathlineto{\pgfqpoint{3.745961in}{0.453578in}}%
\pgfpathlineto{\pgfqpoint{3.773661in}{0.453578in}}%
\pgfpathlineto{\pgfqpoint{3.801360in}{0.453578in}}%
\pgfpathlineto{\pgfqpoint{3.829059in}{0.453578in}}%
\pgfpathlineto{\pgfqpoint{3.856758in}{0.453578in}}%
\pgfpathlineto{\pgfqpoint{3.884458in}{0.453578in}}%
\pgfpathlineto{\pgfqpoint{3.912157in}{0.453578in}}%
\pgfpathlineto{\pgfqpoint{3.939856in}{0.453578in}}%
\pgfpathlineto{\pgfqpoint{3.967556in}{0.453578in}}%
\pgfpathlineto{\pgfqpoint{3.995255in}{0.453578in}}%
\pgfpathlineto{\pgfqpoint{4.022954in}{0.453578in}}%
\pgfpathlineto{\pgfqpoint{4.033088in}{0.453578in}}%
\pgfusepath{stroke}%
\end{pgfscope}%
\begin{pgfscope}%
\pgfpathrectangle{\pgfqpoint{2.666838in}{0.383578in}}{\pgfqpoint{1.356250in}{1.540000in}}%
\pgfusepath{clip}%
\pgfsetrectcap%
\pgfsetroundjoin%
\pgfsetlinewidth{0.803000pt}%
\definecolor{currentstroke}{rgb}{0.686275,0.352941,0.313725}%
\pgfsetstrokecolor{currentstroke}%
\pgfsetstrokeopacity{0.300000}%
\pgfsetdash{}{0pt}%
\pgfpathmoveto{\pgfqpoint{2.721088in}{0.453578in}}%
\pgfpathlineto{\pgfqpoint{2.748787in}{0.453578in}}%
\pgfpathlineto{\pgfqpoint{2.776486in}{0.453578in}}%
\pgfpathlineto{\pgfqpoint{2.804186in}{0.453578in}}%
\pgfpathlineto{\pgfqpoint{2.831885in}{0.453578in}}%
\pgfpathlineto{\pgfqpoint{2.859584in}{0.453578in}}%
\pgfpathlineto{\pgfqpoint{2.887283in}{0.453578in}}%
\pgfpathlineto{\pgfqpoint{2.914983in}{0.453578in}}%
\pgfpathlineto{\pgfqpoint{2.942682in}{0.453578in}}%
\pgfpathlineto{\pgfqpoint{2.970381in}{0.453578in}}%
\pgfpathlineto{\pgfqpoint{2.998081in}{0.453578in}}%
\pgfpathlineto{\pgfqpoint{3.025780in}{0.453578in}}%
\pgfpathlineto{\pgfqpoint{3.053479in}{0.453578in}}%
\pgfpathlineto{\pgfqpoint{3.081178in}{0.453578in}}%
\pgfpathlineto{\pgfqpoint{3.108878in}{0.453578in}}%
\pgfpathlineto{\pgfqpoint{3.136577in}{0.453578in}}%
\pgfpathlineto{\pgfqpoint{3.164276in}{0.453578in}}%
\pgfpathlineto{\pgfqpoint{3.191976in}{0.453578in}}%
\pgfpathlineto{\pgfqpoint{3.219675in}{0.453578in}}%
\pgfpathlineto{\pgfqpoint{3.247374in}{0.453578in}}%
\pgfpathlineto{\pgfqpoint{3.275073in}{0.453578in}}%
\pgfpathlineto{\pgfqpoint{3.302773in}{0.453578in}}%
\pgfpathlineto{\pgfqpoint{3.330472in}{0.453578in}}%
\pgfpathlineto{\pgfqpoint{3.358171in}{0.453578in}}%
\pgfpathlineto{\pgfqpoint{3.385871in}{0.453578in}}%
\pgfpathlineto{\pgfqpoint{3.413570in}{0.453578in}}%
\pgfpathlineto{\pgfqpoint{3.441269in}{0.453578in}}%
\pgfpathlineto{\pgfqpoint{3.468968in}{0.453578in}}%
\pgfpathlineto{\pgfqpoint{3.496668in}{0.453578in}}%
\pgfpathlineto{\pgfqpoint{3.524367in}{0.453578in}}%
\pgfpathlineto{\pgfqpoint{3.552066in}{0.453578in}}%
\pgfpathlineto{\pgfqpoint{3.579766in}{0.453578in}}%
\pgfpathlineto{\pgfqpoint{3.607465in}{0.453578in}}%
\pgfpathlineto{\pgfqpoint{3.635164in}{0.453578in}}%
\pgfpathlineto{\pgfqpoint{3.662863in}{0.453578in}}%
\pgfpathlineto{\pgfqpoint{3.690563in}{0.453578in}}%
\pgfpathlineto{\pgfqpoint{3.718262in}{0.453578in}}%
\pgfpathlineto{\pgfqpoint{3.745961in}{0.453578in}}%
\pgfpathlineto{\pgfqpoint{3.773661in}{0.453578in}}%
\pgfpathlineto{\pgfqpoint{3.801360in}{0.453578in}}%
\pgfpathlineto{\pgfqpoint{3.829059in}{0.453578in}}%
\pgfpathlineto{\pgfqpoint{3.856758in}{0.453578in}}%
\pgfpathlineto{\pgfqpoint{3.884458in}{0.453578in}}%
\pgfpathlineto{\pgfqpoint{3.912157in}{0.453578in}}%
\pgfpathlineto{\pgfqpoint{3.939856in}{0.453578in}}%
\pgfpathlineto{\pgfqpoint{3.967556in}{0.453578in}}%
\pgfpathlineto{\pgfqpoint{3.995255in}{0.453578in}}%
\pgfpathlineto{\pgfqpoint{4.022954in}{0.453578in}}%
\pgfpathlineto{\pgfqpoint{4.033088in}{0.453578in}}%
\pgfusepath{stroke}%
\end{pgfscope}%
\begin{pgfscope}%
\pgfpathrectangle{\pgfqpoint{2.666838in}{0.383578in}}{\pgfqpoint{1.356250in}{1.540000in}}%
\pgfusepath{clip}%
\pgfsetrectcap%
\pgfsetroundjoin%
\pgfsetlinewidth{0.803000pt}%
\definecolor{currentstroke}{rgb}{0.686275,0.352941,0.313725}%
\pgfsetstrokecolor{currentstroke}%
\pgfsetstrokeopacity{0.300000}%
\pgfsetdash{}{0pt}%
\pgfpathmoveto{\pgfqpoint{2.721088in}{0.453578in}}%
\pgfpathlineto{\pgfqpoint{2.748787in}{0.453578in}}%
\pgfpathlineto{\pgfqpoint{2.776486in}{0.453578in}}%
\pgfpathlineto{\pgfqpoint{2.804186in}{0.453578in}}%
\pgfpathlineto{\pgfqpoint{2.831885in}{0.453578in}}%
\pgfpathlineto{\pgfqpoint{2.859584in}{0.453578in}}%
\pgfpathlineto{\pgfqpoint{2.887283in}{0.453578in}}%
\pgfpathlineto{\pgfqpoint{2.914983in}{0.453578in}}%
\pgfpathlineto{\pgfqpoint{2.942682in}{0.453578in}}%
\pgfpathlineto{\pgfqpoint{2.970381in}{0.453578in}}%
\pgfpathlineto{\pgfqpoint{2.998081in}{0.453578in}}%
\pgfpathlineto{\pgfqpoint{3.025780in}{0.453578in}}%
\pgfpathlineto{\pgfqpoint{3.053479in}{0.453578in}}%
\pgfpathlineto{\pgfqpoint{3.081178in}{0.453578in}}%
\pgfpathlineto{\pgfqpoint{3.108878in}{0.453578in}}%
\pgfpathlineto{\pgfqpoint{3.136577in}{0.453578in}}%
\pgfpathlineto{\pgfqpoint{3.164276in}{0.453578in}}%
\pgfpathlineto{\pgfqpoint{3.191976in}{0.453578in}}%
\pgfpathlineto{\pgfqpoint{3.219675in}{0.453578in}}%
\pgfpathlineto{\pgfqpoint{3.247374in}{0.453578in}}%
\pgfpathlineto{\pgfqpoint{3.275073in}{0.453578in}}%
\pgfpathlineto{\pgfqpoint{3.302773in}{0.453578in}}%
\pgfpathlineto{\pgfqpoint{3.330472in}{0.453578in}}%
\pgfpathlineto{\pgfqpoint{3.358171in}{0.453578in}}%
\pgfpathlineto{\pgfqpoint{3.385871in}{0.453578in}}%
\pgfpathlineto{\pgfqpoint{3.413570in}{0.453578in}}%
\pgfpathlineto{\pgfqpoint{3.441269in}{0.453578in}}%
\pgfpathlineto{\pgfqpoint{3.468968in}{0.453578in}}%
\pgfpathlineto{\pgfqpoint{3.496668in}{0.453578in}}%
\pgfpathlineto{\pgfqpoint{3.524367in}{0.453578in}}%
\pgfpathlineto{\pgfqpoint{3.552066in}{0.453578in}}%
\pgfpathlineto{\pgfqpoint{3.579766in}{0.453578in}}%
\pgfpathlineto{\pgfqpoint{3.607465in}{0.453578in}}%
\pgfpathlineto{\pgfqpoint{3.635164in}{0.453578in}}%
\pgfpathlineto{\pgfqpoint{3.662863in}{0.453578in}}%
\pgfpathlineto{\pgfqpoint{3.690563in}{0.453578in}}%
\pgfpathlineto{\pgfqpoint{3.718262in}{0.453578in}}%
\pgfpathlineto{\pgfqpoint{3.745961in}{0.453578in}}%
\pgfpathlineto{\pgfqpoint{3.773661in}{0.453578in}}%
\pgfpathlineto{\pgfqpoint{3.801360in}{0.453578in}}%
\pgfpathlineto{\pgfqpoint{3.829059in}{0.453578in}}%
\pgfpathlineto{\pgfqpoint{3.856758in}{0.453578in}}%
\pgfpathlineto{\pgfqpoint{3.884458in}{0.453578in}}%
\pgfpathlineto{\pgfqpoint{3.912157in}{0.453578in}}%
\pgfpathlineto{\pgfqpoint{3.939856in}{0.453578in}}%
\pgfpathlineto{\pgfqpoint{3.967556in}{0.453578in}}%
\pgfpathlineto{\pgfqpoint{3.995255in}{0.453578in}}%
\pgfpathlineto{\pgfqpoint{4.022954in}{0.453578in}}%
\pgfpathlineto{\pgfqpoint{4.033088in}{0.453578in}}%
\pgfusepath{stroke}%
\end{pgfscope}%
\begin{pgfscope}%
\pgfpathrectangle{\pgfqpoint{2.666838in}{0.383578in}}{\pgfqpoint{1.356250in}{1.540000in}}%
\pgfusepath{clip}%
\pgfsetrectcap%
\pgfsetroundjoin%
\pgfsetlinewidth{0.803000pt}%
\definecolor{currentstroke}{rgb}{0.686275,0.352941,0.313725}%
\pgfsetstrokecolor{currentstroke}%
\pgfsetstrokeopacity{0.300000}%
\pgfsetdash{}{0pt}%
\pgfpathmoveto{\pgfqpoint{2.721088in}{0.453578in}}%
\pgfpathlineto{\pgfqpoint{2.748787in}{0.453578in}}%
\pgfpathlineto{\pgfqpoint{2.776486in}{0.453578in}}%
\pgfpathlineto{\pgfqpoint{2.804186in}{0.453578in}}%
\pgfpathlineto{\pgfqpoint{2.831885in}{0.453578in}}%
\pgfpathlineto{\pgfqpoint{2.859584in}{0.453578in}}%
\pgfpathlineto{\pgfqpoint{2.887283in}{0.453578in}}%
\pgfpathlineto{\pgfqpoint{2.914983in}{0.453578in}}%
\pgfpathlineto{\pgfqpoint{2.942682in}{0.643987in}}%
\pgfpathlineto{\pgfqpoint{2.970381in}{0.765816in}}%
\pgfpathlineto{\pgfqpoint{2.998081in}{0.837775in}}%
\pgfpathlineto{\pgfqpoint{3.025780in}{0.873939in}}%
\pgfpathlineto{\pgfqpoint{3.053479in}{0.884935in}}%
\pgfpathlineto{\pgfqpoint{3.081178in}{0.878642in}}%
\pgfpathlineto{\pgfqpoint{3.108878in}{0.860988in}}%
\pgfpathlineto{\pgfqpoint{3.136577in}{0.836223in}}%
\pgfpathlineto{\pgfqpoint{3.164276in}{0.807469in}}%
\pgfpathlineto{\pgfqpoint{3.191976in}{0.776948in}}%
\pgfpathlineto{\pgfqpoint{3.219675in}{0.746179in}}%
\pgfpathlineto{\pgfqpoint{3.247374in}{0.716220in}}%
\pgfpathlineto{\pgfqpoint{3.275073in}{0.687758in}}%
\pgfpathlineto{\pgfqpoint{3.302773in}{0.661186in}}%
\pgfpathlineto{\pgfqpoint{3.330472in}{0.636942in}}%
\pgfpathlineto{\pgfqpoint{3.358171in}{0.615242in}}%
\pgfpathlineto{\pgfqpoint{3.385871in}{0.596036in}}%
\pgfpathlineto{\pgfqpoint{3.413570in}{0.579216in}}%
\pgfpathlineto{\pgfqpoint{3.441269in}{0.564571in}}%
\pgfpathlineto{\pgfqpoint{3.468968in}{0.551760in}}%
\pgfpathlineto{\pgfqpoint{3.496668in}{0.540559in}}%
\pgfpathlineto{\pgfqpoint{3.524367in}{0.530718in}}%
\pgfpathlineto{\pgfqpoint{3.552066in}{0.522036in}}%
\pgfpathlineto{\pgfqpoint{3.579766in}{0.514351in}}%
\pgfpathlineto{\pgfqpoint{3.607465in}{0.507531in}}%
\pgfpathlineto{\pgfqpoint{3.635164in}{0.501465in}}%
\pgfpathlineto{\pgfqpoint{3.662863in}{0.496063in}}%
\pgfpathlineto{\pgfqpoint{3.690563in}{0.491247in}}%
\pgfpathlineto{\pgfqpoint{3.718262in}{0.486968in}}%
\pgfpathlineto{\pgfqpoint{3.745961in}{0.483160in}}%
\pgfpathlineto{\pgfqpoint{3.773661in}{0.479769in}}%
\pgfpathlineto{\pgfqpoint{3.801360in}{0.476748in}}%
\pgfpathlineto{\pgfqpoint{3.829059in}{0.474057in}}%
\pgfpathlineto{\pgfqpoint{3.856758in}{0.471668in}}%
\pgfpathlineto{\pgfqpoint{3.884458in}{0.469547in}}%
\pgfpathlineto{\pgfqpoint{3.912157in}{0.467662in}}%
\pgfpathlineto{\pgfqpoint{3.939856in}{0.465988in}}%
\pgfpathlineto{\pgfqpoint{3.967556in}{0.464499in}}%
\pgfpathlineto{\pgfqpoint{3.995255in}{0.463179in}}%
\pgfpathlineto{\pgfqpoint{4.022954in}{0.462010in}}%
\pgfpathlineto{\pgfqpoint{4.033088in}{0.461629in}}%
\pgfusepath{stroke}%
\end{pgfscope}%
\begin{pgfscope}%
\pgfpathrectangle{\pgfqpoint{2.666838in}{0.383578in}}{\pgfqpoint{1.356250in}{1.540000in}}%
\pgfusepath{clip}%
\pgfsetrectcap%
\pgfsetroundjoin%
\pgfsetlinewidth{0.803000pt}%
\definecolor{currentstroke}{rgb}{0.686275,0.352941,0.313725}%
\pgfsetstrokecolor{currentstroke}%
\pgfsetstrokeopacity{0.300000}%
\pgfsetdash{}{0pt}%
\pgfpathmoveto{\pgfqpoint{2.721088in}{0.453578in}}%
\pgfpathlineto{\pgfqpoint{2.748787in}{0.453578in}}%
\pgfpathlineto{\pgfqpoint{2.776486in}{0.453578in}}%
\pgfpathlineto{\pgfqpoint{2.804186in}{0.453578in}}%
\pgfpathlineto{\pgfqpoint{2.831885in}{0.453578in}}%
\pgfpathlineto{\pgfqpoint{2.859584in}{0.453578in}}%
\pgfpathlineto{\pgfqpoint{2.887283in}{0.453578in}}%
\pgfpathlineto{\pgfqpoint{2.914983in}{0.453578in}}%
\pgfpathlineto{\pgfqpoint{2.942682in}{0.453578in}}%
\pgfpathlineto{\pgfqpoint{2.970381in}{0.453853in}}%
\pgfpathlineto{\pgfqpoint{2.998081in}{0.454480in}}%
\pgfpathlineto{\pgfqpoint{3.025780in}{0.455377in}}%
\pgfpathlineto{\pgfqpoint{3.053479in}{0.456473in}}%
\pgfpathlineto{\pgfqpoint{3.081178in}{0.457631in}}%
\pgfpathlineto{\pgfqpoint{3.108878in}{0.458831in}}%
\pgfpathlineto{\pgfqpoint{3.136577in}{0.459950in}}%
\pgfpathlineto{\pgfqpoint{3.164276in}{0.460946in}}%
\pgfpathlineto{\pgfqpoint{3.191976in}{0.461791in}}%
\pgfpathlineto{\pgfqpoint{3.219675in}{0.462437in}}%
\pgfpathlineto{\pgfqpoint{3.247374in}{0.462894in}}%
\pgfpathlineto{\pgfqpoint{3.275073in}{0.463176in}}%
\pgfpathlineto{\pgfqpoint{3.302773in}{0.463278in}}%
\pgfpathlineto{\pgfqpoint{3.330472in}{0.463471in}}%
\pgfpathlineto{\pgfqpoint{3.358171in}{0.463936in}}%
\pgfpathlineto{\pgfqpoint{3.385871in}{0.464657in}}%
\pgfpathlineto{\pgfqpoint{3.413570in}{0.465625in}}%
\pgfpathlineto{\pgfqpoint{3.441269in}{0.466754in}}%
\pgfpathlineto{\pgfqpoint{3.468968in}{0.467835in}}%
\pgfpathlineto{\pgfqpoint{3.496668in}{0.468804in}}%
\pgfpathlineto{\pgfqpoint{3.524367in}{0.469566in}}%
\pgfpathlineto{\pgfqpoint{3.552066in}{0.470077in}}%
\pgfpathlineto{\pgfqpoint{3.579766in}{0.470327in}}%
\pgfpathlineto{\pgfqpoint{3.607465in}{0.470328in}}%
\pgfpathlineto{\pgfqpoint{3.635164in}{0.470107in}}%
\pgfpathlineto{\pgfqpoint{3.662863in}{0.469694in}}%
\pgfpathlineto{\pgfqpoint{3.690563in}{0.469125in}}%
\pgfpathlineto{\pgfqpoint{3.718262in}{0.468452in}}%
\pgfpathlineto{\pgfqpoint{3.745961in}{0.467699in}}%
\pgfpathlineto{\pgfqpoint{3.773661in}{0.466888in}}%
\pgfpathlineto{\pgfqpoint{3.801360in}{0.466041in}}%
\pgfpathlineto{\pgfqpoint{3.829059in}{0.465176in}}%
\pgfpathlineto{\pgfqpoint{3.856758in}{0.464320in}}%
\pgfpathlineto{\pgfqpoint{3.884458in}{0.463480in}}%
\pgfpathlineto{\pgfqpoint{3.912157in}{0.462666in}}%
\pgfpathlineto{\pgfqpoint{3.939856in}{0.461884in}}%
\pgfpathlineto{\pgfqpoint{3.967556in}{0.461135in}}%
\pgfpathlineto{\pgfqpoint{3.995255in}{0.460429in}}%
\pgfpathlineto{\pgfqpoint{4.022954in}{0.459767in}}%
\pgfpathlineto{\pgfqpoint{4.033088in}{0.459540in}}%
\pgfusepath{stroke}%
\end{pgfscope}%
\begin{pgfscope}%
\pgfpathrectangle{\pgfqpoint{2.666838in}{0.383578in}}{\pgfqpoint{1.356250in}{1.540000in}}%
\pgfusepath{clip}%
\pgfsetrectcap%
\pgfsetroundjoin%
\pgfsetlinewidth{0.803000pt}%
\definecolor{currentstroke}{rgb}{0.686275,0.352941,0.313725}%
\pgfsetstrokecolor{currentstroke}%
\pgfsetstrokeopacity{0.300000}%
\pgfsetdash{}{0pt}%
\pgfpathmoveto{\pgfqpoint{2.721088in}{0.453578in}}%
\pgfpathlineto{\pgfqpoint{2.748787in}{0.453578in}}%
\pgfpathlineto{\pgfqpoint{2.776486in}{0.453578in}}%
\pgfpathlineto{\pgfqpoint{2.804186in}{0.453578in}}%
\pgfpathlineto{\pgfqpoint{2.831885in}{0.453578in}}%
\pgfpathlineto{\pgfqpoint{2.859584in}{0.453578in}}%
\pgfpathlineto{\pgfqpoint{2.887283in}{0.485503in}}%
\pgfpathlineto{\pgfqpoint{2.914983in}{0.718073in}}%
\pgfpathlineto{\pgfqpoint{2.942682in}{1.248183in}}%
\pgfpathlineto{\pgfqpoint{2.970381in}{1.578021in}}%
\pgfpathlineto{\pgfqpoint{2.998081in}{1.762790in}}%
\pgfpathlineto{\pgfqpoint{3.025780in}{1.844181in}}%
\pgfpathlineto{\pgfqpoint{3.053479in}{1.853578in}}%
\pgfpathlineto{\pgfqpoint{3.081178in}{1.814331in}}%
\pgfpathlineto{\pgfqpoint{3.108878in}{1.743768in}}%
\pgfpathlineto{\pgfqpoint{3.136577in}{1.654457in}}%
\pgfpathlineto{\pgfqpoint{3.164276in}{1.555474in}}%
\pgfpathlineto{\pgfqpoint{3.191976in}{1.453224in}}%
\pgfpathlineto{\pgfqpoint{3.219675in}{1.352087in}}%
\pgfpathlineto{\pgfqpoint{3.247374in}{1.254989in}}%
\pgfpathlineto{\pgfqpoint{3.275073in}{1.163762in}}%
\pgfpathlineto{\pgfqpoint{3.302773in}{1.079429in}}%
\pgfpathlineto{\pgfqpoint{3.330472in}{1.002631in}}%
\pgfpathlineto{\pgfqpoint{3.358171in}{0.933545in}}%
\pgfpathlineto{\pgfqpoint{3.385871in}{0.871953in}}%
\pgfpathlineto{\pgfqpoint{3.413570in}{0.817467in}}%
\pgfpathlineto{\pgfqpoint{3.441269in}{0.769548in}}%
\pgfpathlineto{\pgfqpoint{3.468968in}{0.727527in}}%
\pgfpathlineto{\pgfqpoint{3.496668in}{0.690807in}}%
\pgfpathlineto{\pgfqpoint{3.524367in}{0.658782in}}%
\pgfpathlineto{\pgfqpoint{3.552066in}{0.630900in}}%
\pgfpathlineto{\pgfqpoint{3.579766in}{0.606663in}}%
\pgfpathlineto{\pgfqpoint{3.607465in}{0.585623in}}%
\pgfpathlineto{\pgfqpoint{3.635164in}{0.567382in}}%
\pgfpathlineto{\pgfqpoint{3.662863in}{0.551586in}}%
\pgfpathlineto{\pgfqpoint{3.690563in}{0.537922in}}%
\pgfpathlineto{\pgfqpoint{3.718262in}{0.526125in}}%
\pgfpathlineto{\pgfqpoint{3.745961in}{0.515947in}}%
\pgfpathlineto{\pgfqpoint{3.773661in}{0.507170in}}%
\pgfpathlineto{\pgfqpoint{3.801360in}{0.499606in}}%
\pgfpathlineto{\pgfqpoint{3.829059in}{0.493091in}}%
\pgfpathlineto{\pgfqpoint{3.856758in}{0.487488in}}%
\pgfpathlineto{\pgfqpoint{3.884458in}{0.482670in}}%
\pgfpathlineto{\pgfqpoint{3.912157in}{0.478529in}}%
\pgfpathlineto{\pgfqpoint{3.939856in}{0.474970in}}%
\pgfpathlineto{\pgfqpoint{3.967556in}{0.471910in}}%
\pgfpathlineto{\pgfqpoint{3.995255in}{0.469283in}}%
\pgfpathlineto{\pgfqpoint{4.022954in}{0.467027in}}%
\pgfpathlineto{\pgfqpoint{4.033088in}{0.466318in}}%
\pgfusepath{stroke}%
\end{pgfscope}%
\begin{pgfscope}%
\pgfpathrectangle{\pgfqpoint{2.666838in}{0.383578in}}{\pgfqpoint{1.356250in}{1.540000in}}%
\pgfusepath{clip}%
\pgfsetrectcap%
\pgfsetroundjoin%
\pgfsetlinewidth{0.803000pt}%
\definecolor{currentstroke}{rgb}{0.686275,0.352941,0.313725}%
\pgfsetstrokecolor{currentstroke}%
\pgfsetstrokeopacity{0.300000}%
\pgfsetdash{}{0pt}%
\pgfpathmoveto{\pgfqpoint{2.721088in}{0.453578in}}%
\pgfpathlineto{\pgfqpoint{2.748787in}{0.453578in}}%
\pgfpathlineto{\pgfqpoint{2.776486in}{0.453578in}}%
\pgfpathlineto{\pgfqpoint{2.804186in}{0.453578in}}%
\pgfpathlineto{\pgfqpoint{2.831885in}{0.453578in}}%
\pgfpathlineto{\pgfqpoint{2.859584in}{0.453578in}}%
\pgfpathlineto{\pgfqpoint{2.887283in}{0.453578in}}%
\pgfpathlineto{\pgfqpoint{2.914983in}{0.453578in}}%
\pgfpathlineto{\pgfqpoint{2.942682in}{0.453578in}}%
\pgfpathlineto{\pgfqpoint{2.970381in}{0.453578in}}%
\pgfpathlineto{\pgfqpoint{2.998081in}{0.453578in}}%
\pgfpathlineto{\pgfqpoint{3.025780in}{0.453578in}}%
\pgfpathlineto{\pgfqpoint{3.053479in}{0.453578in}}%
\pgfpathlineto{\pgfqpoint{3.081178in}{0.453578in}}%
\pgfpathlineto{\pgfqpoint{3.108878in}{0.453578in}}%
\pgfpathlineto{\pgfqpoint{3.136577in}{0.453578in}}%
\pgfpathlineto{\pgfqpoint{3.164276in}{0.453578in}}%
\pgfpathlineto{\pgfqpoint{3.191976in}{0.453578in}}%
\pgfpathlineto{\pgfqpoint{3.219675in}{0.453578in}}%
\pgfpathlineto{\pgfqpoint{3.247374in}{0.453578in}}%
\pgfpathlineto{\pgfqpoint{3.275073in}{0.453578in}}%
\pgfpathlineto{\pgfqpoint{3.302773in}{0.453578in}}%
\pgfpathlineto{\pgfqpoint{3.330472in}{0.453578in}}%
\pgfpathlineto{\pgfqpoint{3.358171in}{0.453578in}}%
\pgfpathlineto{\pgfqpoint{3.385871in}{0.453578in}}%
\pgfpathlineto{\pgfqpoint{3.413570in}{0.453578in}}%
\pgfpathlineto{\pgfqpoint{3.441269in}{0.453578in}}%
\pgfpathlineto{\pgfqpoint{3.468968in}{0.453578in}}%
\pgfpathlineto{\pgfqpoint{3.496668in}{0.453578in}}%
\pgfpathlineto{\pgfqpoint{3.524367in}{0.453578in}}%
\pgfpathlineto{\pgfqpoint{3.552066in}{0.453578in}}%
\pgfpathlineto{\pgfqpoint{3.579766in}{0.453578in}}%
\pgfpathlineto{\pgfqpoint{3.607465in}{0.453578in}}%
\pgfpathlineto{\pgfqpoint{3.635164in}{0.453578in}}%
\pgfpathlineto{\pgfqpoint{3.662863in}{0.453578in}}%
\pgfpathlineto{\pgfqpoint{3.690563in}{0.453578in}}%
\pgfpathlineto{\pgfqpoint{3.718262in}{0.453578in}}%
\pgfpathlineto{\pgfqpoint{3.745961in}{0.453578in}}%
\pgfpathlineto{\pgfqpoint{3.773661in}{0.453578in}}%
\pgfpathlineto{\pgfqpoint{3.801360in}{0.453578in}}%
\pgfpathlineto{\pgfqpoint{3.829059in}{0.453578in}}%
\pgfpathlineto{\pgfqpoint{3.856758in}{0.453578in}}%
\pgfpathlineto{\pgfqpoint{3.884458in}{0.453578in}}%
\pgfpathlineto{\pgfqpoint{3.912157in}{0.453578in}}%
\pgfpathlineto{\pgfqpoint{3.939856in}{0.453578in}}%
\pgfpathlineto{\pgfqpoint{3.967556in}{0.453578in}}%
\pgfpathlineto{\pgfqpoint{3.995255in}{0.453578in}}%
\pgfpathlineto{\pgfqpoint{4.022954in}{0.453578in}}%
\pgfpathlineto{\pgfqpoint{4.033088in}{0.453578in}}%
\pgfusepath{stroke}%
\end{pgfscope}%
\begin{pgfscope}%
\pgfpathrectangle{\pgfqpoint{2.666838in}{0.383578in}}{\pgfqpoint{1.356250in}{1.540000in}}%
\pgfusepath{clip}%
\pgfsetrectcap%
\pgfsetroundjoin%
\pgfsetlinewidth{0.803000pt}%
\definecolor{currentstroke}{rgb}{0.686275,0.352941,0.313725}%
\pgfsetstrokecolor{currentstroke}%
\pgfsetstrokeopacity{0.300000}%
\pgfsetdash{}{0pt}%
\pgfpathmoveto{\pgfqpoint{2.721088in}{0.453578in}}%
\pgfpathlineto{\pgfqpoint{2.748787in}{0.453578in}}%
\pgfpathlineto{\pgfqpoint{2.776486in}{0.453578in}}%
\pgfpathlineto{\pgfqpoint{2.804186in}{0.453578in}}%
\pgfpathlineto{\pgfqpoint{2.831885in}{0.453578in}}%
\pgfpathlineto{\pgfqpoint{2.859584in}{0.453578in}}%
\pgfpathlineto{\pgfqpoint{2.887283in}{0.453578in}}%
\pgfpathlineto{\pgfqpoint{2.914983in}{0.583181in}}%
\pgfpathlineto{\pgfqpoint{2.942682in}{0.977584in}}%
\pgfpathlineto{\pgfqpoint{2.970381in}{1.225297in}}%
\pgfpathlineto{\pgfqpoint{2.998081in}{1.366590in}}%
\pgfpathlineto{\pgfqpoint{3.025780in}{1.431868in}}%
\pgfpathlineto{\pgfqpoint{3.053479in}{1.444037in}}%
\pgfpathlineto{\pgfqpoint{3.081178in}{1.420135in}}%
\pgfpathlineto{\pgfqpoint{3.108878in}{1.372852in}}%
\pgfpathlineto{\pgfqpoint{3.136577in}{1.311370in}}%
\pgfpathlineto{\pgfqpoint{3.164276in}{1.242354in}}%
\pgfpathlineto{\pgfqpoint{3.191976in}{1.170517in}}%
\pgfpathlineto{\pgfqpoint{3.219675in}{1.099086in}}%
\pgfpathlineto{\pgfqpoint{3.247374in}{1.030244in}}%
\pgfpathlineto{\pgfqpoint{3.275073in}{0.965369in}}%
\pgfpathlineto{\pgfqpoint{3.302773in}{0.905243in}}%
\pgfpathlineto{\pgfqpoint{3.330472in}{0.850438in}}%
\pgfpathlineto{\pgfqpoint{3.358171in}{0.801162in}}%
\pgfpathlineto{\pgfqpoint{3.385871in}{0.757276in}}%
\pgfpathlineto{\pgfqpoint{3.413570in}{0.718514in}}%
\pgfpathlineto{\pgfqpoint{3.441269in}{0.684479in}}%
\pgfpathlineto{\pgfqpoint{3.468968in}{0.654639in}}%
\pgfpathlineto{\pgfqpoint{3.496668in}{0.628555in}}%
\pgfpathlineto{\pgfqpoint{3.524367in}{0.605771in}}%
\pgfpathlineto{\pgfqpoint{3.552066in}{0.585882in}}%
\pgfpathlineto{\pgfqpoint{3.579766in}{0.568530in}}%
\pgfpathlineto{\pgfqpoint{3.607465in}{0.553401in}}%
\pgfpathlineto{\pgfqpoint{3.635164in}{0.540216in}}%
\pgfpathlineto{\pgfqpoint{3.662863in}{0.528733in}}%
\pgfpathlineto{\pgfqpoint{3.690563in}{0.518737in}}%
\pgfpathlineto{\pgfqpoint{3.718262in}{0.510053in}}%
\pgfpathlineto{\pgfqpoint{3.745961in}{0.502510in}}%
\pgfpathlineto{\pgfqpoint{3.773661in}{0.495959in}}%
\pgfpathlineto{\pgfqpoint{3.801360in}{0.490270in}}%
\pgfpathlineto{\pgfqpoint{3.829059in}{0.485332in}}%
\pgfpathlineto{\pgfqpoint{3.856758in}{0.481052in}}%
\pgfpathlineto{\pgfqpoint{3.884458in}{0.477343in}}%
\pgfpathlineto{\pgfqpoint{3.912157in}{0.474128in}}%
\pgfpathlineto{\pgfqpoint{3.939856in}{0.471342in}}%
\pgfpathlineto{\pgfqpoint{3.967556in}{0.468924in}}%
\pgfpathlineto{\pgfqpoint{3.995255in}{0.466831in}}%
\pgfpathlineto{\pgfqpoint{4.022954in}{0.465018in}}%
\pgfpathlineto{\pgfqpoint{4.033088in}{0.464443in}}%
\pgfusepath{stroke}%
\end{pgfscope}%
\begin{pgfscope}%
\pgfpathrectangle{\pgfqpoint{2.666838in}{0.383578in}}{\pgfqpoint{1.356250in}{1.540000in}}%
\pgfusepath{clip}%
\pgfsetrectcap%
\pgfsetroundjoin%
\pgfsetlinewidth{0.803000pt}%
\definecolor{currentstroke}{rgb}{0.686275,0.352941,0.313725}%
\pgfsetstrokecolor{currentstroke}%
\pgfsetstrokeopacity{0.300000}%
\pgfsetdash{}{0pt}%
\pgfpathmoveto{\pgfqpoint{2.721088in}{0.453578in}}%
\pgfpathlineto{\pgfqpoint{2.748787in}{0.453578in}}%
\pgfpathlineto{\pgfqpoint{2.776486in}{0.453578in}}%
\pgfpathlineto{\pgfqpoint{2.804186in}{0.453578in}}%
\pgfpathlineto{\pgfqpoint{2.831885in}{0.453578in}}%
\pgfpathlineto{\pgfqpoint{2.859584in}{0.453578in}}%
\pgfpathlineto{\pgfqpoint{2.887283in}{0.453578in}}%
\pgfpathlineto{\pgfqpoint{2.914983in}{0.453578in}}%
\pgfpathlineto{\pgfqpoint{2.942682in}{0.453578in}}%
\pgfpathlineto{\pgfqpoint{2.970381in}{0.453578in}}%
\pgfpathlineto{\pgfqpoint{2.998081in}{0.453578in}}%
\pgfpathlineto{\pgfqpoint{3.025780in}{0.453578in}}%
\pgfpathlineto{\pgfqpoint{3.053479in}{0.453578in}}%
\pgfpathlineto{\pgfqpoint{3.081178in}{0.453578in}}%
\pgfpathlineto{\pgfqpoint{3.108878in}{0.453578in}}%
\pgfpathlineto{\pgfqpoint{3.136577in}{0.453578in}}%
\pgfpathlineto{\pgfqpoint{3.164276in}{0.453578in}}%
\pgfpathlineto{\pgfqpoint{3.191976in}{0.453578in}}%
\pgfpathlineto{\pgfqpoint{3.219675in}{0.453578in}}%
\pgfpathlineto{\pgfqpoint{3.247374in}{0.453578in}}%
\pgfpathlineto{\pgfqpoint{3.275073in}{0.453578in}}%
\pgfpathlineto{\pgfqpoint{3.302773in}{0.453578in}}%
\pgfpathlineto{\pgfqpoint{3.330472in}{0.453578in}}%
\pgfpathlineto{\pgfqpoint{3.358171in}{0.453578in}}%
\pgfpathlineto{\pgfqpoint{3.385871in}{0.453578in}}%
\pgfpathlineto{\pgfqpoint{3.413570in}{0.453578in}}%
\pgfpathlineto{\pgfqpoint{3.441269in}{0.453578in}}%
\pgfpathlineto{\pgfqpoint{3.468968in}{0.453578in}}%
\pgfpathlineto{\pgfqpoint{3.496668in}{0.453578in}}%
\pgfpathlineto{\pgfqpoint{3.524367in}{0.453578in}}%
\pgfpathlineto{\pgfqpoint{3.552066in}{0.453578in}}%
\pgfpathlineto{\pgfqpoint{3.579766in}{0.453578in}}%
\pgfpathlineto{\pgfqpoint{3.607465in}{0.453578in}}%
\pgfpathlineto{\pgfqpoint{3.635164in}{0.453578in}}%
\pgfpathlineto{\pgfqpoint{3.662863in}{0.453578in}}%
\pgfpathlineto{\pgfqpoint{3.690563in}{0.453578in}}%
\pgfpathlineto{\pgfqpoint{3.718262in}{0.453578in}}%
\pgfpathlineto{\pgfqpoint{3.745961in}{0.453578in}}%
\pgfpathlineto{\pgfqpoint{3.773661in}{0.453578in}}%
\pgfpathlineto{\pgfqpoint{3.801360in}{0.453578in}}%
\pgfpathlineto{\pgfqpoint{3.829059in}{0.453578in}}%
\pgfpathlineto{\pgfqpoint{3.856758in}{0.453578in}}%
\pgfpathlineto{\pgfqpoint{3.884458in}{0.453578in}}%
\pgfpathlineto{\pgfqpoint{3.912157in}{0.453578in}}%
\pgfpathlineto{\pgfqpoint{3.939856in}{0.453578in}}%
\pgfpathlineto{\pgfqpoint{3.967556in}{0.453578in}}%
\pgfpathlineto{\pgfqpoint{3.995255in}{0.453578in}}%
\pgfpathlineto{\pgfqpoint{4.022954in}{0.453578in}}%
\pgfpathlineto{\pgfqpoint{4.033088in}{0.453578in}}%
\pgfusepath{stroke}%
\end{pgfscope}%
\begin{pgfscope}%
\pgfpathrectangle{\pgfqpoint{2.666838in}{0.383578in}}{\pgfqpoint{1.356250in}{1.540000in}}%
\pgfusepath{clip}%
\pgfsetrectcap%
\pgfsetroundjoin%
\pgfsetlinewidth{0.803000pt}%
\definecolor{currentstroke}{rgb}{0.686275,0.352941,0.313725}%
\pgfsetstrokecolor{currentstroke}%
\pgfsetstrokeopacity{0.300000}%
\pgfsetdash{}{0pt}%
\pgfpathmoveto{\pgfqpoint{2.721088in}{0.453578in}}%
\pgfpathlineto{\pgfqpoint{2.748787in}{0.453578in}}%
\pgfpathlineto{\pgfqpoint{2.776486in}{0.453578in}}%
\pgfpathlineto{\pgfqpoint{2.804186in}{0.453578in}}%
\pgfpathlineto{\pgfqpoint{2.831885in}{0.453578in}}%
\pgfpathlineto{\pgfqpoint{2.859584in}{0.453578in}}%
\pgfpathlineto{\pgfqpoint{2.887283in}{0.453578in}}%
\pgfpathlineto{\pgfqpoint{2.914983in}{0.453578in}}%
\pgfpathlineto{\pgfqpoint{2.942682in}{0.453578in}}%
\pgfpathlineto{\pgfqpoint{2.970381in}{0.453578in}}%
\pgfpathlineto{\pgfqpoint{2.998081in}{0.453578in}}%
\pgfpathlineto{\pgfqpoint{3.025780in}{0.453578in}}%
\pgfpathlineto{\pgfqpoint{3.053479in}{0.453578in}}%
\pgfpathlineto{\pgfqpoint{3.081178in}{0.453578in}}%
\pgfpathlineto{\pgfqpoint{3.108878in}{0.453578in}}%
\pgfpathlineto{\pgfqpoint{3.136577in}{0.453578in}}%
\pgfpathlineto{\pgfqpoint{3.164276in}{0.453578in}}%
\pgfpathlineto{\pgfqpoint{3.191976in}{0.453578in}}%
\pgfpathlineto{\pgfqpoint{3.219675in}{0.453578in}}%
\pgfpathlineto{\pgfqpoint{3.247374in}{0.453578in}}%
\pgfpathlineto{\pgfqpoint{3.275073in}{0.453578in}}%
\pgfpathlineto{\pgfqpoint{3.302773in}{0.453578in}}%
\pgfpathlineto{\pgfqpoint{3.330472in}{0.453578in}}%
\pgfpathlineto{\pgfqpoint{3.358171in}{0.453578in}}%
\pgfpathlineto{\pgfqpoint{3.385871in}{0.453578in}}%
\pgfpathlineto{\pgfqpoint{3.413570in}{0.453578in}}%
\pgfpathlineto{\pgfqpoint{3.441269in}{0.453578in}}%
\pgfpathlineto{\pgfqpoint{3.468968in}{0.453578in}}%
\pgfpathlineto{\pgfqpoint{3.496668in}{0.453578in}}%
\pgfpathlineto{\pgfqpoint{3.524367in}{0.453578in}}%
\pgfpathlineto{\pgfqpoint{3.552066in}{0.453578in}}%
\pgfpathlineto{\pgfqpoint{3.579766in}{0.453578in}}%
\pgfpathlineto{\pgfqpoint{3.607465in}{0.453578in}}%
\pgfpathlineto{\pgfqpoint{3.635164in}{0.453578in}}%
\pgfpathlineto{\pgfqpoint{3.662863in}{0.453578in}}%
\pgfpathlineto{\pgfqpoint{3.690563in}{0.453578in}}%
\pgfpathlineto{\pgfqpoint{3.718262in}{0.453578in}}%
\pgfpathlineto{\pgfqpoint{3.745961in}{0.453578in}}%
\pgfpathlineto{\pgfqpoint{3.773661in}{0.453578in}}%
\pgfpathlineto{\pgfqpoint{3.801360in}{0.453578in}}%
\pgfpathlineto{\pgfqpoint{3.829059in}{0.453578in}}%
\pgfpathlineto{\pgfqpoint{3.856758in}{0.453578in}}%
\pgfpathlineto{\pgfqpoint{3.884458in}{0.453578in}}%
\pgfpathlineto{\pgfqpoint{3.912157in}{0.453578in}}%
\pgfpathlineto{\pgfqpoint{3.939856in}{0.453578in}}%
\pgfpathlineto{\pgfqpoint{3.967556in}{0.453578in}}%
\pgfpathlineto{\pgfqpoint{3.995255in}{0.453578in}}%
\pgfpathlineto{\pgfqpoint{4.022954in}{0.453578in}}%
\pgfpathlineto{\pgfqpoint{4.033088in}{0.453578in}}%
\pgfusepath{stroke}%
\end{pgfscope}%
\begin{pgfscope}%
\pgfpathrectangle{\pgfqpoint{2.666838in}{0.383578in}}{\pgfqpoint{1.356250in}{1.540000in}}%
\pgfusepath{clip}%
\pgfsetrectcap%
\pgfsetroundjoin%
\pgfsetlinewidth{0.803000pt}%
\definecolor{currentstroke}{rgb}{0.686275,0.352941,0.313725}%
\pgfsetstrokecolor{currentstroke}%
\pgfsetstrokeopacity{0.300000}%
\pgfsetdash{}{0pt}%
\pgfpathmoveto{\pgfqpoint{2.721088in}{0.453578in}}%
\pgfpathlineto{\pgfqpoint{2.748787in}{0.453578in}}%
\pgfpathlineto{\pgfqpoint{2.776486in}{0.453578in}}%
\pgfpathlineto{\pgfqpoint{2.804186in}{0.453578in}}%
\pgfpathlineto{\pgfqpoint{2.831885in}{0.453578in}}%
\pgfpathlineto{\pgfqpoint{2.859584in}{0.453578in}}%
\pgfpathlineto{\pgfqpoint{2.887283in}{0.453578in}}%
\pgfpathlineto{\pgfqpoint{2.914983in}{0.453578in}}%
\pgfpathlineto{\pgfqpoint{2.942682in}{0.453578in}}%
\pgfpathlineto{\pgfqpoint{2.970381in}{0.453578in}}%
\pgfpathlineto{\pgfqpoint{2.998081in}{0.453578in}}%
\pgfpathlineto{\pgfqpoint{3.025780in}{0.453578in}}%
\pgfpathlineto{\pgfqpoint{3.053479in}{0.453578in}}%
\pgfpathlineto{\pgfqpoint{3.081178in}{0.453578in}}%
\pgfpathlineto{\pgfqpoint{3.108878in}{0.453578in}}%
\pgfpathlineto{\pgfqpoint{3.136577in}{0.453578in}}%
\pgfpathlineto{\pgfqpoint{3.164276in}{0.453578in}}%
\pgfpathlineto{\pgfqpoint{3.191976in}{0.453578in}}%
\pgfpathlineto{\pgfqpoint{3.219675in}{0.453578in}}%
\pgfpathlineto{\pgfqpoint{3.247374in}{0.453578in}}%
\pgfpathlineto{\pgfqpoint{3.275073in}{0.453578in}}%
\pgfpathlineto{\pgfqpoint{3.302773in}{0.453578in}}%
\pgfpathlineto{\pgfqpoint{3.330472in}{0.453578in}}%
\pgfpathlineto{\pgfqpoint{3.358171in}{0.453578in}}%
\pgfpathlineto{\pgfqpoint{3.385871in}{0.453578in}}%
\pgfpathlineto{\pgfqpoint{3.413570in}{0.453578in}}%
\pgfpathlineto{\pgfqpoint{3.441269in}{0.453578in}}%
\pgfpathlineto{\pgfqpoint{3.468968in}{0.453578in}}%
\pgfpathlineto{\pgfqpoint{3.496668in}{0.453578in}}%
\pgfpathlineto{\pgfqpoint{3.524367in}{0.453578in}}%
\pgfpathlineto{\pgfqpoint{3.552066in}{0.453578in}}%
\pgfpathlineto{\pgfqpoint{3.579766in}{0.453578in}}%
\pgfpathlineto{\pgfqpoint{3.607465in}{0.453578in}}%
\pgfpathlineto{\pgfqpoint{3.635164in}{0.453578in}}%
\pgfpathlineto{\pgfqpoint{3.662863in}{0.453578in}}%
\pgfpathlineto{\pgfqpoint{3.690563in}{0.453578in}}%
\pgfpathlineto{\pgfqpoint{3.718262in}{0.453578in}}%
\pgfpathlineto{\pgfqpoint{3.745961in}{0.453578in}}%
\pgfpathlineto{\pgfqpoint{3.773661in}{0.453578in}}%
\pgfpathlineto{\pgfqpoint{3.801360in}{0.453578in}}%
\pgfpathlineto{\pgfqpoint{3.829059in}{0.453578in}}%
\pgfpathlineto{\pgfqpoint{3.856758in}{0.453578in}}%
\pgfpathlineto{\pgfqpoint{3.884458in}{0.453578in}}%
\pgfpathlineto{\pgfqpoint{3.912157in}{0.453578in}}%
\pgfpathlineto{\pgfqpoint{3.939856in}{0.453578in}}%
\pgfpathlineto{\pgfqpoint{3.967556in}{0.453578in}}%
\pgfpathlineto{\pgfqpoint{3.995255in}{0.453578in}}%
\pgfpathlineto{\pgfqpoint{4.022954in}{0.453578in}}%
\pgfpathlineto{\pgfqpoint{4.033088in}{0.453578in}}%
\pgfusepath{stroke}%
\end{pgfscope}%
\begin{pgfscope}%
\pgfpathrectangle{\pgfqpoint{2.666838in}{0.383578in}}{\pgfqpoint{1.356250in}{1.540000in}}%
\pgfusepath{clip}%
\pgfsetrectcap%
\pgfsetroundjoin%
\pgfsetlinewidth{0.803000pt}%
\definecolor{currentstroke}{rgb}{0.686275,0.352941,0.313725}%
\pgfsetstrokecolor{currentstroke}%
\pgfsetstrokeopacity{0.300000}%
\pgfsetdash{}{0pt}%
\pgfpathmoveto{\pgfqpoint{2.721088in}{0.453578in}}%
\pgfpathlineto{\pgfqpoint{2.748787in}{0.453578in}}%
\pgfpathlineto{\pgfqpoint{2.776486in}{0.453578in}}%
\pgfpathlineto{\pgfqpoint{2.804186in}{0.453578in}}%
\pgfpathlineto{\pgfqpoint{2.831885in}{0.453578in}}%
\pgfpathlineto{\pgfqpoint{2.859584in}{0.453578in}}%
\pgfpathlineto{\pgfqpoint{2.887283in}{0.453578in}}%
\pgfpathlineto{\pgfqpoint{2.914983in}{0.453578in}}%
\pgfpathlineto{\pgfqpoint{2.942682in}{0.643987in}}%
\pgfpathlineto{\pgfqpoint{2.970381in}{0.765816in}}%
\pgfpathlineto{\pgfqpoint{2.998081in}{0.837775in}}%
\pgfpathlineto{\pgfqpoint{3.025780in}{0.873939in}}%
\pgfpathlineto{\pgfqpoint{3.053479in}{0.884935in}}%
\pgfpathlineto{\pgfqpoint{3.081178in}{0.878642in}}%
\pgfpathlineto{\pgfqpoint{3.108878in}{0.860988in}}%
\pgfpathlineto{\pgfqpoint{3.136577in}{0.836223in}}%
\pgfpathlineto{\pgfqpoint{3.164276in}{0.807469in}}%
\pgfpathlineto{\pgfqpoint{3.191976in}{0.776948in}}%
\pgfpathlineto{\pgfqpoint{3.219675in}{0.746179in}}%
\pgfpathlineto{\pgfqpoint{3.247374in}{0.716220in}}%
\pgfpathlineto{\pgfqpoint{3.275073in}{0.687758in}}%
\pgfpathlineto{\pgfqpoint{3.302773in}{0.661186in}}%
\pgfpathlineto{\pgfqpoint{3.330472in}{0.636942in}}%
\pgfpathlineto{\pgfqpoint{3.358171in}{0.615242in}}%
\pgfpathlineto{\pgfqpoint{3.385871in}{0.596036in}}%
\pgfpathlineto{\pgfqpoint{3.413570in}{0.579216in}}%
\pgfpathlineto{\pgfqpoint{3.441269in}{0.564571in}}%
\pgfpathlineto{\pgfqpoint{3.468968in}{0.551760in}}%
\pgfpathlineto{\pgfqpoint{3.496668in}{0.540559in}}%
\pgfpathlineto{\pgfqpoint{3.524367in}{0.530718in}}%
\pgfpathlineto{\pgfqpoint{3.552066in}{0.522036in}}%
\pgfpathlineto{\pgfqpoint{3.579766in}{0.514351in}}%
\pgfpathlineto{\pgfqpoint{3.607465in}{0.507531in}}%
\pgfpathlineto{\pgfqpoint{3.635164in}{0.501465in}}%
\pgfpathlineto{\pgfqpoint{3.662863in}{0.496063in}}%
\pgfpathlineto{\pgfqpoint{3.690563in}{0.491247in}}%
\pgfpathlineto{\pgfqpoint{3.718262in}{0.486968in}}%
\pgfpathlineto{\pgfqpoint{3.745961in}{0.483160in}}%
\pgfpathlineto{\pgfqpoint{3.773661in}{0.479769in}}%
\pgfpathlineto{\pgfqpoint{3.801360in}{0.476748in}}%
\pgfpathlineto{\pgfqpoint{3.829059in}{0.474057in}}%
\pgfpathlineto{\pgfqpoint{3.856758in}{0.471668in}}%
\pgfpathlineto{\pgfqpoint{3.884458in}{0.469547in}}%
\pgfpathlineto{\pgfqpoint{3.912157in}{0.467662in}}%
\pgfpathlineto{\pgfqpoint{3.939856in}{0.465988in}}%
\pgfpathlineto{\pgfqpoint{3.967556in}{0.464499in}}%
\pgfpathlineto{\pgfqpoint{3.995255in}{0.463179in}}%
\pgfpathlineto{\pgfqpoint{4.022954in}{0.462010in}}%
\pgfpathlineto{\pgfqpoint{4.033088in}{0.461629in}}%
\pgfusepath{stroke}%
\end{pgfscope}%
\begin{pgfscope}%
\pgfpathrectangle{\pgfqpoint{2.666838in}{0.383578in}}{\pgfqpoint{1.356250in}{1.540000in}}%
\pgfusepath{clip}%
\pgfsetrectcap%
\pgfsetroundjoin%
\pgfsetlinewidth{0.803000pt}%
\definecolor{currentstroke}{rgb}{0.686275,0.352941,0.313725}%
\pgfsetstrokecolor{currentstroke}%
\pgfsetstrokeopacity{0.300000}%
\pgfsetdash{}{0pt}%
\pgfpathmoveto{\pgfqpoint{2.721088in}{0.453578in}}%
\pgfpathlineto{\pgfqpoint{2.748787in}{0.453578in}}%
\pgfpathlineto{\pgfqpoint{2.776486in}{0.453578in}}%
\pgfpathlineto{\pgfqpoint{2.804186in}{0.453578in}}%
\pgfpathlineto{\pgfqpoint{2.831885in}{0.453578in}}%
\pgfpathlineto{\pgfqpoint{2.859584in}{0.453578in}}%
\pgfpathlineto{\pgfqpoint{2.887283in}{0.453578in}}%
\pgfpathlineto{\pgfqpoint{2.914983in}{0.453578in}}%
\pgfpathlineto{\pgfqpoint{2.942682in}{0.453578in}}%
\pgfpathlineto{\pgfqpoint{2.970381in}{0.453578in}}%
\pgfpathlineto{\pgfqpoint{2.998081in}{0.453578in}}%
\pgfpathlineto{\pgfqpoint{3.025780in}{0.453578in}}%
\pgfpathlineto{\pgfqpoint{3.053479in}{0.453578in}}%
\pgfpathlineto{\pgfqpoint{3.081178in}{0.453578in}}%
\pgfpathlineto{\pgfqpoint{3.108878in}{0.453578in}}%
\pgfpathlineto{\pgfqpoint{3.136577in}{0.453578in}}%
\pgfpathlineto{\pgfqpoint{3.164276in}{0.453578in}}%
\pgfpathlineto{\pgfqpoint{3.191976in}{0.453578in}}%
\pgfpathlineto{\pgfqpoint{3.219675in}{0.453578in}}%
\pgfpathlineto{\pgfqpoint{3.247374in}{0.453578in}}%
\pgfpathlineto{\pgfqpoint{3.275073in}{0.453578in}}%
\pgfpathlineto{\pgfqpoint{3.302773in}{0.453578in}}%
\pgfpathlineto{\pgfqpoint{3.330472in}{0.453578in}}%
\pgfpathlineto{\pgfqpoint{3.358171in}{0.453578in}}%
\pgfpathlineto{\pgfqpoint{3.385871in}{0.453578in}}%
\pgfpathlineto{\pgfqpoint{3.413570in}{0.453578in}}%
\pgfpathlineto{\pgfqpoint{3.441269in}{0.453578in}}%
\pgfpathlineto{\pgfqpoint{3.468968in}{0.453578in}}%
\pgfpathlineto{\pgfqpoint{3.496668in}{0.453578in}}%
\pgfpathlineto{\pgfqpoint{3.524367in}{0.453578in}}%
\pgfpathlineto{\pgfqpoint{3.552066in}{0.453578in}}%
\pgfpathlineto{\pgfqpoint{3.579766in}{0.453578in}}%
\pgfpathlineto{\pgfqpoint{3.607465in}{0.453578in}}%
\pgfpathlineto{\pgfqpoint{3.635164in}{0.453578in}}%
\pgfpathlineto{\pgfqpoint{3.662863in}{0.453578in}}%
\pgfpathlineto{\pgfqpoint{3.690563in}{0.453578in}}%
\pgfpathlineto{\pgfqpoint{3.718262in}{0.453578in}}%
\pgfpathlineto{\pgfqpoint{3.745961in}{0.453578in}}%
\pgfpathlineto{\pgfqpoint{3.773661in}{0.453578in}}%
\pgfpathlineto{\pgfqpoint{3.801360in}{0.453578in}}%
\pgfpathlineto{\pgfqpoint{3.829059in}{0.453578in}}%
\pgfpathlineto{\pgfqpoint{3.856758in}{0.453578in}}%
\pgfpathlineto{\pgfqpoint{3.884458in}{0.453578in}}%
\pgfpathlineto{\pgfqpoint{3.912157in}{0.453578in}}%
\pgfpathlineto{\pgfqpoint{3.939856in}{0.453578in}}%
\pgfpathlineto{\pgfqpoint{3.967556in}{0.453578in}}%
\pgfpathlineto{\pgfqpoint{3.995255in}{0.453578in}}%
\pgfpathlineto{\pgfqpoint{4.022954in}{0.453578in}}%
\pgfpathlineto{\pgfqpoint{4.033088in}{0.453578in}}%
\pgfusepath{stroke}%
\end{pgfscope}%
\begin{pgfscope}%
\pgfsetrectcap%
\pgfsetmiterjoin%
\pgfsetlinewidth{0.501875pt}%
\definecolor{currentstroke}{rgb}{0.317647,0.317647,0.317647}%
\pgfsetstrokecolor{currentstroke}%
\pgfsetdash{}{0pt}%
\pgfpathmoveto{\pgfqpoint{2.666838in}{0.383578in}}%
\pgfpathlineto{\pgfqpoint{2.666838in}{1.923578in}}%
\pgfusepath{stroke}%
\end{pgfscope}%
\begin{pgfscope}%
\pgfsetrectcap%
\pgfsetmiterjoin%
\pgfsetlinewidth{0.501875pt}%
\definecolor{currentstroke}{rgb}{0.317647,0.317647,0.317647}%
\pgfsetstrokecolor{currentstroke}%
\pgfsetdash{}{0pt}%
\pgfpathmoveto{\pgfqpoint{2.666838in}{0.383578in}}%
\pgfpathlineto{\pgfqpoint{4.023088in}{0.383578in}}%
\pgfusepath{stroke}%
\end{pgfscope}%
\begin{pgfscope}%
\pgfsetrectcap%
\pgfsetroundjoin%
\pgfsetlinewidth{0.803000pt}%
\definecolor{currentstroke}{rgb}{0.333333,0.333333,0.333333}%
\pgfsetstrokecolor{currentstroke}%
\pgfsetdash{}{0pt}%
\pgfpathmoveto{\pgfqpoint{3.822335in}{1.863417in}}%
\pgfpathlineto{\pgfqpoint{3.896379in}{1.863417in}}%
\pgfusepath{stroke}%
\end{pgfscope}%
\begin{pgfscope}%
\definecolor{textcolor}{rgb}{0.000000,0.000000,0.000000}%
\pgfsetstrokecolor{textcolor}%
\pgfsetfillcolor{textcolor}%
\pgftext[x=3.942657in,y=1.831022in,left,base]{\color{textcolor}\rmfamily\fontsize{6.664000}{7.996800}\selectfont 0}%
\end{pgfscope}%
\begin{pgfscope}%
\pgfsetrectcap%
\pgfsetroundjoin%
\pgfsetlinewidth{0.803000pt}%
\definecolor{currentstroke}{rgb}{0.686275,0.352941,0.313725}%
\pgfsetstrokecolor{currentstroke}%
\pgfsetdash{}{0pt}%
\pgfpathmoveto{\pgfqpoint{3.822335in}{1.743650in}}%
\pgfpathlineto{\pgfqpoint{3.896379in}{1.743650in}}%
\pgfusepath{stroke}%
\end{pgfscope}%
\begin{pgfscope}%
\definecolor{textcolor}{rgb}{0.000000,0.000000,0.000000}%
\pgfsetstrokecolor{textcolor}%
\pgfsetfillcolor{textcolor}%
\pgftext[x=3.942657in,y=1.711256in,left,base]{\color{textcolor}\rmfamily\fontsize{6.664000}{7.996800}\selectfont 1}%
\end{pgfscope}%
\begin{pgfscope}%
\pgfsetbuttcap%
\pgfsetmiterjoin%
\pgfsetlinewidth{0.000000pt}%
\definecolor{currentstroke}{rgb}{0.000000,0.000000,0.000000}%
\pgfsetstrokecolor{currentstroke}%
\pgfsetstrokeopacity{0.000000}%
\pgfsetdash{}{0pt}%
\pgfpathmoveto{\pgfqpoint{4.701213in}{0.383578in}}%
\pgfpathlineto{\pgfqpoint{6.057463in}{0.383578in}}%
\pgfpathlineto{\pgfqpoint{6.057463in}{1.923578in}}%
\pgfpathlineto{\pgfqpoint{4.701213in}{1.923578in}}%
\pgfpathclose%
\pgfusepath{}%
\end{pgfscope}%
\begin{pgfscope}%
\pgfsetbuttcap%
\pgfsetroundjoin%
\definecolor{currentfill}{rgb}{0.317647,0.317647,0.317647}%
\pgfsetfillcolor{currentfill}%
\pgfsetlinewidth{0.501875pt}%
\definecolor{currentstroke}{rgb}{0.317647,0.317647,0.317647}%
\pgfsetstrokecolor{currentstroke}%
\pgfsetdash{}{0pt}%
\pgfsys@defobject{currentmarker}{\pgfqpoint{0.000000in}{-0.020833in}}{\pgfqpoint{0.000000in}{0.000000in}}{%
\pgfpathmoveto{\pgfqpoint{0.000000in}{0.000000in}}%
\pgfpathlineto{\pgfqpoint{0.000000in}{-0.020833in}}%
\pgfusepath{stroke,fill}%
}%
\begin{pgfscope}%
\pgfsys@transformshift{4.755463in}{0.383578in}%
\pgfsys@useobject{currentmarker}{}%
\end{pgfscope}%
\end{pgfscope}%
\begin{pgfscope}%
\definecolor{textcolor}{rgb}{0.317647,0.317647,0.317647}%
\pgfsetstrokecolor{textcolor}%
\pgfsetfillcolor{textcolor}%
\pgftext[x=4.755463in,y=0.334967in,,top]{\color{textcolor}\rmfamily\fontsize{6.664000}{7.996800}\selectfont \(\displaystyle 0\)}%
\end{pgfscope}%
\begin{pgfscope}%
\pgfsetbuttcap%
\pgfsetroundjoin%
\definecolor{currentfill}{rgb}{0.317647,0.317647,0.317647}%
\pgfsetfillcolor{currentfill}%
\pgfsetlinewidth{0.501875pt}%
\definecolor{currentstroke}{rgb}{0.317647,0.317647,0.317647}%
\pgfsetstrokecolor{currentstroke}%
\pgfsetdash{}{0pt}%
\pgfsys@defobject{currentmarker}{\pgfqpoint{0.000000in}{-0.020833in}}{\pgfqpoint{0.000000in}{0.000000in}}{%
\pgfpathmoveto{\pgfqpoint{0.000000in}{0.000000in}}%
\pgfpathlineto{\pgfqpoint{0.000000in}{-0.020833in}}%
\pgfusepath{stroke,fill}%
}%
\begin{pgfscope}%
\pgfsys@transformshift{5.297963in}{0.383578in}%
\pgfsys@useobject{currentmarker}{}%
\end{pgfscope}%
\end{pgfscope}%
\begin{pgfscope}%
\definecolor{textcolor}{rgb}{0.317647,0.317647,0.317647}%
\pgfsetstrokecolor{textcolor}%
\pgfsetfillcolor{textcolor}%
\pgftext[x=5.297963in,y=0.334967in,,top]{\color{textcolor}\rmfamily\fontsize{6.664000}{7.996800}\selectfont \(\displaystyle 50\)}%
\end{pgfscope}%
\begin{pgfscope}%
\pgfsetbuttcap%
\pgfsetroundjoin%
\definecolor{currentfill}{rgb}{0.317647,0.317647,0.317647}%
\pgfsetfillcolor{currentfill}%
\pgfsetlinewidth{0.501875pt}%
\definecolor{currentstroke}{rgb}{0.317647,0.317647,0.317647}%
\pgfsetstrokecolor{currentstroke}%
\pgfsetdash{}{0pt}%
\pgfsys@defobject{currentmarker}{\pgfqpoint{0.000000in}{-0.020833in}}{\pgfqpoint{0.000000in}{0.000000in}}{%
\pgfpathmoveto{\pgfqpoint{0.000000in}{0.000000in}}%
\pgfpathlineto{\pgfqpoint{0.000000in}{-0.020833in}}%
\pgfusepath{stroke,fill}%
}%
\begin{pgfscope}%
\pgfsys@transformshift{5.840463in}{0.383578in}%
\pgfsys@useobject{currentmarker}{}%
\end{pgfscope}%
\end{pgfscope}%
\begin{pgfscope}%
\definecolor{textcolor}{rgb}{0.317647,0.317647,0.317647}%
\pgfsetstrokecolor{textcolor}%
\pgfsetfillcolor{textcolor}%
\pgftext[x=5.840463in,y=0.334967in,,top]{\color{textcolor}\rmfamily\fontsize{6.664000}{7.996800}\selectfont \(\displaystyle 100\)}%
\end{pgfscope}%
\begin{pgfscope}%
\definecolor{textcolor}{rgb}{0.317647,0.317647,0.317647}%
\pgfsetstrokecolor{textcolor}%
\pgfsetfillcolor{textcolor}%
\pgftext[x=5.379338in,y=0.197222in,,top]{\color{textcolor}\rmfamily\fontsize{6.664000}{7.996800}\selectfont time\(\displaystyle \;(\si{\micro \s})\)}%
\end{pgfscope}%
\begin{pgfscope}%
\pgfsetbuttcap%
\pgfsetroundjoin%
\definecolor{currentfill}{rgb}{0.317647,0.317647,0.317647}%
\pgfsetfillcolor{currentfill}%
\pgfsetlinewidth{0.501875pt}%
\definecolor{currentstroke}{rgb}{0.317647,0.317647,0.317647}%
\pgfsetstrokecolor{currentstroke}%
\pgfsetdash{}{0pt}%
\pgfsys@defobject{currentmarker}{\pgfqpoint{-0.020833in}{0.000000in}}{\pgfqpoint{0.000000in}{0.000000in}}{%
\pgfpathmoveto{\pgfqpoint{0.000000in}{0.000000in}}%
\pgfpathlineto{\pgfqpoint{-0.020833in}{0.000000in}}%
\pgfusepath{stroke,fill}%
}%
\begin{pgfscope}%
\pgfsys@transformshift{4.701213in}{0.599163in}%
\pgfsys@useobject{currentmarker}{}%
\end{pgfscope}%
\end{pgfscope}%
\begin{pgfscope}%
\definecolor{textcolor}{rgb}{0.317647,0.317647,0.317647}%
\pgfsetstrokecolor{textcolor}%
\pgfsetfillcolor{textcolor}%
\pgftext[x=4.269884in,y=0.567047in,left,base]{\color{textcolor}\rmfamily\fontsize{6.664000}{7.996800}\selectfont \(\displaystyle -0.0004\)}%
\end{pgfscope}%
\begin{pgfscope}%
\pgfsetbuttcap%
\pgfsetroundjoin%
\definecolor{currentfill}{rgb}{0.317647,0.317647,0.317647}%
\pgfsetfillcolor{currentfill}%
\pgfsetlinewidth{0.501875pt}%
\definecolor{currentstroke}{rgb}{0.317647,0.317647,0.317647}%
\pgfsetstrokecolor{currentstroke}%
\pgfsetdash{}{0pt}%
\pgfsys@defobject{currentmarker}{\pgfqpoint{-0.020833in}{0.000000in}}{\pgfqpoint{0.000000in}{0.000000in}}{%
\pgfpathmoveto{\pgfqpoint{0.000000in}{0.000000in}}%
\pgfpathlineto{\pgfqpoint{-0.020833in}{0.000000in}}%
\pgfusepath{stroke,fill}%
}%
\begin{pgfscope}%
\pgfsys@transformshift{4.701213in}{0.918832in}%
\pgfsys@useobject{currentmarker}{}%
\end{pgfscope}%
\end{pgfscope}%
\begin{pgfscope}%
\definecolor{textcolor}{rgb}{0.317647,0.317647,0.317647}%
\pgfsetstrokecolor{textcolor}%
\pgfsetfillcolor{textcolor}%
\pgftext[x=4.269884in,y=0.886715in,left,base]{\color{textcolor}\rmfamily\fontsize{6.664000}{7.996800}\selectfont \(\displaystyle -0.0002\)}%
\end{pgfscope}%
\begin{pgfscope}%
\pgfsetbuttcap%
\pgfsetroundjoin%
\definecolor{currentfill}{rgb}{0.317647,0.317647,0.317647}%
\pgfsetfillcolor{currentfill}%
\pgfsetlinewidth{0.501875pt}%
\definecolor{currentstroke}{rgb}{0.317647,0.317647,0.317647}%
\pgfsetstrokecolor{currentstroke}%
\pgfsetdash{}{0pt}%
\pgfsys@defobject{currentmarker}{\pgfqpoint{-0.020833in}{0.000000in}}{\pgfqpoint{0.000000in}{0.000000in}}{%
\pgfpathmoveto{\pgfqpoint{0.000000in}{0.000000in}}%
\pgfpathlineto{\pgfqpoint{-0.020833in}{0.000000in}}%
\pgfusepath{stroke,fill}%
}%
\begin{pgfscope}%
\pgfsys@transformshift{4.701213in}{1.238500in}%
\pgfsys@useobject{currentmarker}{}%
\end{pgfscope}%
\end{pgfscope}%
\begin{pgfscope}%
\definecolor{textcolor}{rgb}{0.317647,0.317647,0.317647}%
\pgfsetstrokecolor{textcolor}%
\pgfsetfillcolor{textcolor}%
\pgftext[x=4.356690in,y=1.206383in,left,base]{\color{textcolor}\rmfamily\fontsize{6.664000}{7.996800}\selectfont \(\displaystyle 0.0000\)}%
\end{pgfscope}%
\begin{pgfscope}%
\pgfsetbuttcap%
\pgfsetroundjoin%
\definecolor{currentfill}{rgb}{0.317647,0.317647,0.317647}%
\pgfsetfillcolor{currentfill}%
\pgfsetlinewidth{0.501875pt}%
\definecolor{currentstroke}{rgb}{0.317647,0.317647,0.317647}%
\pgfsetstrokecolor{currentstroke}%
\pgfsetdash{}{0pt}%
\pgfsys@defobject{currentmarker}{\pgfqpoint{-0.020833in}{0.000000in}}{\pgfqpoint{0.000000in}{0.000000in}}{%
\pgfpathmoveto{\pgfqpoint{0.000000in}{0.000000in}}%
\pgfpathlineto{\pgfqpoint{-0.020833in}{0.000000in}}%
\pgfusepath{stroke,fill}%
}%
\begin{pgfscope}%
\pgfsys@transformshift{4.701213in}{1.558168in}%
\pgfsys@useobject{currentmarker}{}%
\end{pgfscope}%
\end{pgfscope}%
\begin{pgfscope}%
\definecolor{textcolor}{rgb}{0.317647,0.317647,0.317647}%
\pgfsetstrokecolor{textcolor}%
\pgfsetfillcolor{textcolor}%
\pgftext[x=4.356690in,y=1.526051in,left,base]{\color{textcolor}\rmfamily\fontsize{6.664000}{7.996800}\selectfont \(\displaystyle 0.0002\)}%
\end{pgfscope}%
\begin{pgfscope}%
\pgfsetbuttcap%
\pgfsetroundjoin%
\definecolor{currentfill}{rgb}{0.317647,0.317647,0.317647}%
\pgfsetfillcolor{currentfill}%
\pgfsetlinewidth{0.501875pt}%
\definecolor{currentstroke}{rgb}{0.317647,0.317647,0.317647}%
\pgfsetstrokecolor{currentstroke}%
\pgfsetdash{}{0pt}%
\pgfsys@defobject{currentmarker}{\pgfqpoint{-0.020833in}{0.000000in}}{\pgfqpoint{0.000000in}{0.000000in}}{%
\pgfpathmoveto{\pgfqpoint{0.000000in}{0.000000in}}%
\pgfpathlineto{\pgfqpoint{-0.020833in}{0.000000in}}%
\pgfusepath{stroke,fill}%
}%
\begin{pgfscope}%
\pgfsys@transformshift{4.701213in}{1.877836in}%
\pgfsys@useobject{currentmarker}{}%
\end{pgfscope}%
\end{pgfscope}%
\begin{pgfscope}%
\definecolor{textcolor}{rgb}{0.317647,0.317647,0.317647}%
\pgfsetstrokecolor{textcolor}%
\pgfsetfillcolor{textcolor}%
\pgftext[x=4.356690in,y=1.845719in,left,base]{\color{textcolor}\rmfamily\fontsize{6.664000}{7.996800}\selectfont \(\displaystyle 0.0004\)}%
\end{pgfscope}%
\begin{pgfscope}%
\definecolor{textcolor}{rgb}{0.317647,0.317647,0.317647}%
\pgfsetstrokecolor{textcolor}%
\pgfsetfillcolor{textcolor}%
\pgftext[x=4.214329in,y=1.153578in,,bottom,rotate=90.000000]{\color{textcolor}\rmfamily\fontsize{6.664000}{7.996800}\selectfont \(\displaystyle \Delta w^{(o)}_{ij}\)}%
\end{pgfscope}%
\begin{pgfscope}%
\pgfpathrectangle{\pgfqpoint{4.701213in}{0.383578in}}{\pgfqpoint{1.356250in}{1.540000in}}%
\pgfusepath{clip}%
\pgfsetrectcap%
\pgfsetroundjoin%
\pgfsetlinewidth{0.803000pt}%
\definecolor{currentstroke}{rgb}{0.333333,0.333333,0.333333}%
\pgfsetstrokecolor{currentstroke}%
\pgfsetstrokeopacity{0.300000}%
\pgfsetdash{}{0pt}%
\pgfpathmoveto{\pgfqpoint{4.755463in}{1.238500in}}%
\pgfpathlineto{\pgfqpoint{4.783162in}{1.238500in}}%
\pgfpathlineto{\pgfqpoint{4.810861in}{1.238500in}}%
\pgfpathlineto{\pgfqpoint{4.838561in}{1.238500in}}%
\pgfpathlineto{\pgfqpoint{4.866260in}{1.238500in}}%
\pgfpathlineto{\pgfqpoint{4.893959in}{1.238500in}}%
\pgfpathlineto{\pgfqpoint{4.921658in}{1.238500in}}%
\pgfpathlineto{\pgfqpoint{4.949358in}{1.238500in}}%
\pgfpathlineto{\pgfqpoint{4.977057in}{1.238500in}}%
\pgfpathlineto{\pgfqpoint{5.004756in}{1.238500in}}%
\pgfpathlineto{\pgfqpoint{5.032456in}{1.238500in}}%
\pgfpathlineto{\pgfqpoint{5.060155in}{1.238500in}}%
\pgfpathlineto{\pgfqpoint{5.087854in}{1.238500in}}%
\pgfpathlineto{\pgfqpoint{5.115553in}{1.238500in}}%
\pgfpathlineto{\pgfqpoint{5.143253in}{1.238500in}}%
\pgfpathlineto{\pgfqpoint{5.170952in}{1.238500in}}%
\pgfpathlineto{\pgfqpoint{5.198651in}{1.238500in}}%
\pgfpathlineto{\pgfqpoint{5.226351in}{1.238500in}}%
\pgfpathlineto{\pgfqpoint{5.254050in}{1.238500in}}%
\pgfpathlineto{\pgfqpoint{5.281749in}{1.238500in}}%
\pgfpathlineto{\pgfqpoint{5.309448in}{1.238500in}}%
\pgfpathlineto{\pgfqpoint{5.337148in}{1.238500in}}%
\pgfpathlineto{\pgfqpoint{5.364847in}{1.238500in}}%
\pgfpathlineto{\pgfqpoint{5.392546in}{1.238500in}}%
\pgfpathlineto{\pgfqpoint{5.420246in}{1.238500in}}%
\pgfpathlineto{\pgfqpoint{5.447945in}{1.238500in}}%
\pgfpathlineto{\pgfqpoint{5.475644in}{1.238500in}}%
\pgfpathlineto{\pgfqpoint{5.503343in}{1.238500in}}%
\pgfpathlineto{\pgfqpoint{5.531043in}{1.238500in}}%
\pgfpathlineto{\pgfqpoint{5.558742in}{1.238500in}}%
\pgfpathlineto{\pgfqpoint{5.586441in}{1.238500in}}%
\pgfpathlineto{\pgfqpoint{5.614141in}{1.238500in}}%
\pgfpathlineto{\pgfqpoint{5.641840in}{1.238500in}}%
\pgfpathlineto{\pgfqpoint{5.669539in}{1.238500in}}%
\pgfpathlineto{\pgfqpoint{5.697238in}{1.238500in}}%
\pgfpathlineto{\pgfqpoint{5.724938in}{1.238500in}}%
\pgfpathlineto{\pgfqpoint{5.752637in}{1.238500in}}%
\pgfpathlineto{\pgfqpoint{5.780336in}{1.238500in}}%
\pgfpathlineto{\pgfqpoint{5.808036in}{1.238500in}}%
\pgfpathlineto{\pgfqpoint{5.835735in}{1.238500in}}%
\pgfpathlineto{\pgfqpoint{5.863434in}{1.238500in}}%
\pgfpathlineto{\pgfqpoint{5.891133in}{1.238500in}}%
\pgfpathlineto{\pgfqpoint{5.918833in}{1.238500in}}%
\pgfpathlineto{\pgfqpoint{5.946532in}{1.238500in}}%
\pgfpathlineto{\pgfqpoint{5.974231in}{1.238500in}}%
\pgfpathlineto{\pgfqpoint{6.001931in}{1.238500in}}%
\pgfpathlineto{\pgfqpoint{6.029630in}{1.238500in}}%
\pgfpathlineto{\pgfqpoint{6.057329in}{1.238500in}}%
\pgfpathlineto{\pgfqpoint{6.067463in}{1.238500in}}%
\pgfusepath{stroke}%
\end{pgfscope}%
\begin{pgfscope}%
\pgfpathrectangle{\pgfqpoint{4.701213in}{0.383578in}}{\pgfqpoint{1.356250in}{1.540000in}}%
\pgfusepath{clip}%
\pgfsetrectcap%
\pgfsetroundjoin%
\pgfsetlinewidth{0.803000pt}%
\definecolor{currentstroke}{rgb}{0.333333,0.333333,0.333333}%
\pgfsetstrokecolor{currentstroke}%
\pgfsetstrokeopacity{0.300000}%
\pgfsetdash{}{0pt}%
\pgfpathmoveto{\pgfqpoint{4.755463in}{1.238500in}}%
\pgfpathlineto{\pgfqpoint{4.783162in}{1.238500in}}%
\pgfpathlineto{\pgfqpoint{4.810861in}{1.238500in}}%
\pgfpathlineto{\pgfqpoint{4.838561in}{1.238500in}}%
\pgfpathlineto{\pgfqpoint{4.866260in}{1.238500in}}%
\pgfpathlineto{\pgfqpoint{4.893959in}{1.238500in}}%
\pgfpathlineto{\pgfqpoint{4.921658in}{1.238500in}}%
\pgfpathlineto{\pgfqpoint{4.949358in}{1.238500in}}%
\pgfpathlineto{\pgfqpoint{4.977057in}{1.238500in}}%
\pgfpathlineto{\pgfqpoint{5.004756in}{1.230328in}}%
\pgfpathlineto{\pgfqpoint{5.032456in}{1.194678in}}%
\pgfpathlineto{\pgfqpoint{5.060155in}{1.125555in}}%
\pgfpathlineto{\pgfqpoint{5.087854in}{1.025862in}}%
\pgfpathlineto{\pgfqpoint{5.115553in}{0.931727in}}%
\pgfpathlineto{\pgfqpoint{5.143253in}{0.866015in}}%
\pgfpathlineto{\pgfqpoint{5.170952in}{0.834409in}}%
\pgfpathlineto{\pgfqpoint{5.198651in}{0.834959in}}%
\pgfpathlineto{\pgfqpoint{5.226351in}{0.856898in}}%
\pgfpathlineto{\pgfqpoint{5.254050in}{0.890406in}}%
\pgfpathlineto{\pgfqpoint{5.281749in}{0.927516in}}%
\pgfpathlineto{\pgfqpoint{5.309448in}{0.961811in}}%
\pgfpathlineto{\pgfqpoint{5.337148in}{0.991313in}}%
\pgfpathlineto{\pgfqpoint{5.364847in}{1.015418in}}%
\pgfpathlineto{\pgfqpoint{5.392546in}{1.035275in}}%
\pgfpathlineto{\pgfqpoint{5.420246in}{1.053428in}}%
\pgfpathlineto{\pgfqpoint{5.447945in}{1.070443in}}%
\pgfpathlineto{\pgfqpoint{5.475644in}{1.087086in}}%
\pgfpathlineto{\pgfqpoint{5.503343in}{1.103482in}}%
\pgfpathlineto{\pgfqpoint{5.531043in}{1.119453in}}%
\pgfpathlineto{\pgfqpoint{5.558742in}{1.134942in}}%
\pgfpathlineto{\pgfqpoint{5.586441in}{1.149665in}}%
\pgfpathlineto{\pgfqpoint{5.614141in}{1.163078in}}%
\pgfpathlineto{\pgfqpoint{5.641840in}{1.175072in}}%
\pgfpathlineto{\pgfqpoint{5.669539in}{1.185695in}}%
\pgfpathlineto{\pgfqpoint{5.697238in}{1.194928in}}%
\pgfpathlineto{\pgfqpoint{5.724938in}{1.202869in}}%
\pgfpathlineto{\pgfqpoint{5.752637in}{1.209670in}}%
\pgfpathlineto{\pgfqpoint{5.780336in}{1.215376in}}%
\pgfpathlineto{\pgfqpoint{5.808036in}{1.220102in}}%
\pgfpathlineto{\pgfqpoint{5.835735in}{1.223971in}}%
\pgfpathlineto{\pgfqpoint{5.863434in}{1.227113in}}%
\pgfpathlineto{\pgfqpoint{5.891133in}{1.229635in}}%
\pgfpathlineto{\pgfqpoint{5.918833in}{1.231638in}}%
\pgfpathlineto{\pgfqpoint{5.946532in}{1.233214in}}%
\pgfpathlineto{\pgfqpoint{5.974231in}{1.234448in}}%
\pgfpathlineto{\pgfqpoint{6.001931in}{1.235409in}}%
\pgfpathlineto{\pgfqpoint{6.029630in}{1.236152in}}%
\pgfpathlineto{\pgfqpoint{6.057329in}{1.236724in}}%
\pgfpathlineto{\pgfqpoint{6.067463in}{1.236884in}}%
\pgfusepath{stroke}%
\end{pgfscope}%
\begin{pgfscope}%
\pgfpathrectangle{\pgfqpoint{4.701213in}{0.383578in}}{\pgfqpoint{1.356250in}{1.540000in}}%
\pgfusepath{clip}%
\pgfsetrectcap%
\pgfsetroundjoin%
\pgfsetlinewidth{0.803000pt}%
\definecolor{currentstroke}{rgb}{0.333333,0.333333,0.333333}%
\pgfsetstrokecolor{currentstroke}%
\pgfsetstrokeopacity{0.300000}%
\pgfsetdash{}{0pt}%
\pgfpathmoveto{\pgfqpoint{4.755463in}{1.238500in}}%
\pgfpathlineto{\pgfqpoint{4.783162in}{1.238500in}}%
\pgfpathlineto{\pgfqpoint{4.810861in}{1.238500in}}%
\pgfpathlineto{\pgfqpoint{4.838561in}{1.238500in}}%
\pgfpathlineto{\pgfqpoint{4.866260in}{1.238500in}}%
\pgfpathlineto{\pgfqpoint{4.893959in}{1.238500in}}%
\pgfpathlineto{\pgfqpoint{4.921658in}{1.238500in}}%
\pgfpathlineto{\pgfqpoint{4.949358in}{1.238500in}}%
\pgfpathlineto{\pgfqpoint{4.977057in}{1.238500in}}%
\pgfpathlineto{\pgfqpoint{5.004756in}{1.238500in}}%
\pgfpathlineto{\pgfqpoint{5.032456in}{1.238500in}}%
\pgfpathlineto{\pgfqpoint{5.060155in}{1.238500in}}%
\pgfpathlineto{\pgfqpoint{5.087854in}{1.238500in}}%
\pgfpathlineto{\pgfqpoint{5.115553in}{1.238500in}}%
\pgfpathlineto{\pgfqpoint{5.143253in}{1.238500in}}%
\pgfpathlineto{\pgfqpoint{5.170952in}{1.238500in}}%
\pgfpathlineto{\pgfqpoint{5.198651in}{1.238500in}}%
\pgfpathlineto{\pgfqpoint{5.226351in}{1.238500in}}%
\pgfpathlineto{\pgfqpoint{5.254050in}{1.238500in}}%
\pgfpathlineto{\pgfqpoint{5.281749in}{1.238500in}}%
\pgfpathlineto{\pgfqpoint{5.309448in}{1.238500in}}%
\pgfpathlineto{\pgfqpoint{5.337148in}{1.238500in}}%
\pgfpathlineto{\pgfqpoint{5.364847in}{1.238500in}}%
\pgfpathlineto{\pgfqpoint{5.392546in}{1.238500in}}%
\pgfpathlineto{\pgfqpoint{5.420246in}{1.238500in}}%
\pgfpathlineto{\pgfqpoint{5.447945in}{1.238500in}}%
\pgfpathlineto{\pgfqpoint{5.475644in}{1.238500in}}%
\pgfpathlineto{\pgfqpoint{5.503343in}{1.238500in}}%
\pgfpathlineto{\pgfqpoint{5.531043in}{1.238500in}}%
\pgfpathlineto{\pgfqpoint{5.558742in}{1.238500in}}%
\pgfpathlineto{\pgfqpoint{5.586441in}{1.238500in}}%
\pgfpathlineto{\pgfqpoint{5.614141in}{1.238500in}}%
\pgfpathlineto{\pgfqpoint{5.641840in}{1.238500in}}%
\pgfpathlineto{\pgfqpoint{5.669539in}{1.238500in}}%
\pgfpathlineto{\pgfqpoint{5.697238in}{1.238500in}}%
\pgfpathlineto{\pgfqpoint{5.724938in}{1.238500in}}%
\pgfpathlineto{\pgfqpoint{5.752637in}{1.238500in}}%
\pgfpathlineto{\pgfqpoint{5.780336in}{1.238500in}}%
\pgfpathlineto{\pgfqpoint{5.808036in}{1.238500in}}%
\pgfpathlineto{\pgfqpoint{5.835735in}{1.238500in}}%
\pgfpathlineto{\pgfqpoint{5.863434in}{1.238500in}}%
\pgfpathlineto{\pgfqpoint{5.891133in}{1.238500in}}%
\pgfpathlineto{\pgfqpoint{5.918833in}{1.238500in}}%
\pgfpathlineto{\pgfqpoint{5.946532in}{1.238500in}}%
\pgfpathlineto{\pgfqpoint{5.974231in}{1.238500in}}%
\pgfpathlineto{\pgfqpoint{6.001931in}{1.238500in}}%
\pgfpathlineto{\pgfqpoint{6.029630in}{1.238500in}}%
\pgfpathlineto{\pgfqpoint{6.057329in}{1.238500in}}%
\pgfpathlineto{\pgfqpoint{6.067463in}{1.238500in}}%
\pgfusepath{stroke}%
\end{pgfscope}%
\begin{pgfscope}%
\pgfpathrectangle{\pgfqpoint{4.701213in}{0.383578in}}{\pgfqpoint{1.356250in}{1.540000in}}%
\pgfusepath{clip}%
\pgfsetrectcap%
\pgfsetroundjoin%
\pgfsetlinewidth{0.803000pt}%
\definecolor{currentstroke}{rgb}{0.333333,0.333333,0.333333}%
\pgfsetstrokecolor{currentstroke}%
\pgfsetstrokeopacity{0.300000}%
\pgfsetdash{}{0pt}%
\pgfpathmoveto{\pgfqpoint{4.755463in}{1.238500in}}%
\pgfpathlineto{\pgfqpoint{4.783162in}{1.238500in}}%
\pgfpathlineto{\pgfqpoint{4.810861in}{1.238500in}}%
\pgfpathlineto{\pgfqpoint{4.838561in}{1.238500in}}%
\pgfpathlineto{\pgfqpoint{4.866260in}{1.238500in}}%
\pgfpathlineto{\pgfqpoint{4.893959in}{1.238500in}}%
\pgfpathlineto{\pgfqpoint{4.921658in}{1.238500in}}%
\pgfpathlineto{\pgfqpoint{4.949358in}{1.238500in}}%
\pgfpathlineto{\pgfqpoint{4.977057in}{1.238500in}}%
\pgfpathlineto{\pgfqpoint{5.004756in}{1.238500in}}%
\pgfpathlineto{\pgfqpoint{5.032456in}{1.238500in}}%
\pgfpathlineto{\pgfqpoint{5.060155in}{1.238500in}}%
\pgfpathlineto{\pgfqpoint{5.087854in}{1.238500in}}%
\pgfpathlineto{\pgfqpoint{5.115553in}{1.238500in}}%
\pgfpathlineto{\pgfqpoint{5.143253in}{1.238500in}}%
\pgfpathlineto{\pgfqpoint{5.170952in}{1.238500in}}%
\pgfpathlineto{\pgfqpoint{5.198651in}{1.238500in}}%
\pgfpathlineto{\pgfqpoint{5.226351in}{1.238500in}}%
\pgfpathlineto{\pgfqpoint{5.254050in}{1.238500in}}%
\pgfpathlineto{\pgfqpoint{5.281749in}{1.238500in}}%
\pgfpathlineto{\pgfqpoint{5.309448in}{1.238500in}}%
\pgfpathlineto{\pgfqpoint{5.337148in}{1.238500in}}%
\pgfpathlineto{\pgfqpoint{5.364847in}{1.238500in}}%
\pgfpathlineto{\pgfqpoint{5.392546in}{1.238500in}}%
\pgfpathlineto{\pgfqpoint{5.420246in}{1.238500in}}%
\pgfpathlineto{\pgfqpoint{5.447945in}{1.238500in}}%
\pgfpathlineto{\pgfqpoint{5.475644in}{1.238500in}}%
\pgfpathlineto{\pgfqpoint{5.503343in}{1.238500in}}%
\pgfpathlineto{\pgfqpoint{5.531043in}{1.238500in}}%
\pgfpathlineto{\pgfqpoint{5.558742in}{1.238500in}}%
\pgfpathlineto{\pgfqpoint{5.586441in}{1.238500in}}%
\pgfpathlineto{\pgfqpoint{5.614141in}{1.238500in}}%
\pgfpathlineto{\pgfqpoint{5.641840in}{1.238500in}}%
\pgfpathlineto{\pgfqpoint{5.669539in}{1.238500in}}%
\pgfpathlineto{\pgfqpoint{5.697238in}{1.238500in}}%
\pgfpathlineto{\pgfqpoint{5.724938in}{1.238500in}}%
\pgfpathlineto{\pgfqpoint{5.752637in}{1.238500in}}%
\pgfpathlineto{\pgfqpoint{5.780336in}{1.238500in}}%
\pgfpathlineto{\pgfqpoint{5.808036in}{1.238500in}}%
\pgfpathlineto{\pgfqpoint{5.835735in}{1.238500in}}%
\pgfpathlineto{\pgfqpoint{5.863434in}{1.238500in}}%
\pgfpathlineto{\pgfqpoint{5.891133in}{1.238500in}}%
\pgfpathlineto{\pgfqpoint{5.918833in}{1.238500in}}%
\pgfpathlineto{\pgfqpoint{5.946532in}{1.238500in}}%
\pgfpathlineto{\pgfqpoint{5.974231in}{1.238500in}}%
\pgfpathlineto{\pgfqpoint{6.001931in}{1.238500in}}%
\pgfpathlineto{\pgfqpoint{6.029630in}{1.238500in}}%
\pgfpathlineto{\pgfqpoint{6.057329in}{1.238500in}}%
\pgfpathlineto{\pgfqpoint{6.067463in}{1.238500in}}%
\pgfusepath{stroke}%
\end{pgfscope}%
\begin{pgfscope}%
\pgfpathrectangle{\pgfqpoint{4.701213in}{0.383578in}}{\pgfqpoint{1.356250in}{1.540000in}}%
\pgfusepath{clip}%
\pgfsetrectcap%
\pgfsetroundjoin%
\pgfsetlinewidth{0.803000pt}%
\definecolor{currentstroke}{rgb}{0.333333,0.333333,0.333333}%
\pgfsetstrokecolor{currentstroke}%
\pgfsetstrokeopacity{0.300000}%
\pgfsetdash{}{0pt}%
\pgfpathmoveto{\pgfqpoint{4.755463in}{1.238500in}}%
\pgfpathlineto{\pgfqpoint{4.783162in}{1.238500in}}%
\pgfpathlineto{\pgfqpoint{4.810861in}{1.238500in}}%
\pgfpathlineto{\pgfqpoint{4.838561in}{1.238500in}}%
\pgfpathlineto{\pgfqpoint{4.866260in}{1.238500in}}%
\pgfpathlineto{\pgfqpoint{4.893959in}{1.238500in}}%
\pgfpathlineto{\pgfqpoint{4.921658in}{1.238500in}}%
\pgfpathlineto{\pgfqpoint{4.949358in}{1.238500in}}%
\pgfpathlineto{\pgfqpoint{4.977057in}{1.238500in}}%
\pgfpathlineto{\pgfqpoint{5.004756in}{1.238500in}}%
\pgfpathlineto{\pgfqpoint{5.032456in}{1.238500in}}%
\pgfpathlineto{\pgfqpoint{5.060155in}{1.238500in}}%
\pgfpathlineto{\pgfqpoint{5.087854in}{1.238500in}}%
\pgfpathlineto{\pgfqpoint{5.115553in}{1.238500in}}%
\pgfpathlineto{\pgfqpoint{5.143253in}{1.238500in}}%
\pgfpathlineto{\pgfqpoint{5.170952in}{1.238500in}}%
\pgfpathlineto{\pgfqpoint{5.198651in}{1.238500in}}%
\pgfpathlineto{\pgfqpoint{5.226351in}{1.238500in}}%
\pgfpathlineto{\pgfqpoint{5.254050in}{1.238500in}}%
\pgfpathlineto{\pgfqpoint{5.281749in}{1.238500in}}%
\pgfpathlineto{\pgfqpoint{5.309448in}{1.238500in}}%
\pgfpathlineto{\pgfqpoint{5.337148in}{1.238500in}}%
\pgfpathlineto{\pgfqpoint{5.364847in}{1.238500in}}%
\pgfpathlineto{\pgfqpoint{5.392546in}{1.238500in}}%
\pgfpathlineto{\pgfqpoint{5.420246in}{1.238500in}}%
\pgfpathlineto{\pgfqpoint{5.447945in}{1.238500in}}%
\pgfpathlineto{\pgfqpoint{5.475644in}{1.238500in}}%
\pgfpathlineto{\pgfqpoint{5.503343in}{1.238500in}}%
\pgfpathlineto{\pgfqpoint{5.531043in}{1.238500in}}%
\pgfpathlineto{\pgfqpoint{5.558742in}{1.238500in}}%
\pgfpathlineto{\pgfqpoint{5.586441in}{1.238500in}}%
\pgfpathlineto{\pgfqpoint{5.614141in}{1.238500in}}%
\pgfpathlineto{\pgfqpoint{5.641840in}{1.238500in}}%
\pgfpathlineto{\pgfqpoint{5.669539in}{1.238500in}}%
\pgfpathlineto{\pgfqpoint{5.697238in}{1.238500in}}%
\pgfpathlineto{\pgfqpoint{5.724938in}{1.238500in}}%
\pgfpathlineto{\pgfqpoint{5.752637in}{1.238500in}}%
\pgfpathlineto{\pgfqpoint{5.780336in}{1.238500in}}%
\pgfpathlineto{\pgfqpoint{5.808036in}{1.238500in}}%
\pgfpathlineto{\pgfqpoint{5.835735in}{1.238500in}}%
\pgfpathlineto{\pgfqpoint{5.863434in}{1.238500in}}%
\pgfpathlineto{\pgfqpoint{5.891133in}{1.238500in}}%
\pgfpathlineto{\pgfqpoint{5.918833in}{1.238500in}}%
\pgfpathlineto{\pgfqpoint{5.946532in}{1.238500in}}%
\pgfpathlineto{\pgfqpoint{5.974231in}{1.238500in}}%
\pgfpathlineto{\pgfqpoint{6.001931in}{1.238500in}}%
\pgfpathlineto{\pgfqpoint{6.029630in}{1.238500in}}%
\pgfpathlineto{\pgfqpoint{6.057329in}{1.238500in}}%
\pgfpathlineto{\pgfqpoint{6.067463in}{1.238500in}}%
\pgfusepath{stroke}%
\end{pgfscope}%
\begin{pgfscope}%
\pgfpathrectangle{\pgfqpoint{4.701213in}{0.383578in}}{\pgfqpoint{1.356250in}{1.540000in}}%
\pgfusepath{clip}%
\pgfsetrectcap%
\pgfsetroundjoin%
\pgfsetlinewidth{0.803000pt}%
\definecolor{currentstroke}{rgb}{0.333333,0.333333,0.333333}%
\pgfsetstrokecolor{currentstroke}%
\pgfsetstrokeopacity{0.300000}%
\pgfsetdash{}{0pt}%
\pgfpathmoveto{\pgfqpoint{4.755463in}{1.238500in}}%
\pgfpathlineto{\pgfqpoint{4.783162in}{1.238500in}}%
\pgfpathlineto{\pgfqpoint{4.810861in}{1.238500in}}%
\pgfpathlineto{\pgfqpoint{4.838561in}{1.238500in}}%
\pgfpathlineto{\pgfqpoint{4.866260in}{1.238500in}}%
\pgfpathlineto{\pgfqpoint{4.893959in}{1.238500in}}%
\pgfpathlineto{\pgfqpoint{4.921658in}{1.238500in}}%
\pgfpathlineto{\pgfqpoint{4.949358in}{1.238500in}}%
\pgfpathlineto{\pgfqpoint{4.977057in}{1.238500in}}%
\pgfpathlineto{\pgfqpoint{5.004756in}{1.156161in}}%
\pgfpathlineto{\pgfqpoint{5.032456in}{1.040510in}}%
\pgfpathlineto{\pgfqpoint{5.060155in}{0.911307in}}%
\pgfpathlineto{\pgfqpoint{5.087854in}{0.781158in}}%
\pgfpathlineto{\pgfqpoint{5.115553in}{0.684254in}}%
\pgfpathlineto{\pgfqpoint{5.143253in}{0.635184in}}%
\pgfpathlineto{\pgfqpoint{5.170952in}{0.631046in}}%
\pgfpathlineto{\pgfqpoint{5.198651in}{0.662892in}}%
\pgfpathlineto{\pgfqpoint{5.226351in}{0.716029in}}%
\pgfpathlineto{\pgfqpoint{5.254050in}{0.778445in}}%
\pgfpathlineto{\pgfqpoint{5.281749in}{0.841269in}}%
\pgfpathlineto{\pgfqpoint{5.309448in}{0.898126in}}%
\pgfpathlineto{\pgfqpoint{5.337148in}{0.946992in}}%
\pgfpathlineto{\pgfqpoint{5.364847in}{0.987444in}}%
\pgfpathlineto{\pgfqpoint{5.392546in}{1.020702in}}%
\pgfpathlineto{\pgfqpoint{5.420246in}{1.049171in}}%
\pgfpathlineto{\pgfqpoint{5.447945in}{1.073787in}}%
\pgfpathlineto{\pgfqpoint{5.475644in}{1.095621in}}%
\pgfpathlineto{\pgfqpoint{5.503343in}{1.115214in}}%
\pgfpathlineto{\pgfqpoint{5.531043in}{1.132826in}}%
\pgfpathlineto{\pgfqpoint{5.558742in}{1.148746in}}%
\pgfpathlineto{\pgfqpoint{5.586441in}{1.163042in}}%
\pgfpathlineto{\pgfqpoint{5.614141in}{1.175543in}}%
\pgfpathlineto{\pgfqpoint{5.641840in}{1.186353in}}%
\pgfpathlineto{\pgfqpoint{5.669539in}{1.195655in}}%
\pgfpathlineto{\pgfqpoint{5.697238in}{1.203552in}}%
\pgfpathlineto{\pgfqpoint{5.724938in}{1.210209in}}%
\pgfpathlineto{\pgfqpoint{5.752637in}{1.215808in}}%
\pgfpathlineto{\pgfqpoint{5.780336in}{1.220438in}}%
\pgfpathlineto{\pgfqpoint{5.808036in}{1.224226in}}%
\pgfpathlineto{\pgfqpoint{5.835735in}{1.227296in}}%
\pgfpathlineto{\pgfqpoint{5.863434in}{1.229765in}}%
\pgfpathlineto{\pgfqpoint{5.891133in}{1.231733in}}%
\pgfpathlineto{\pgfqpoint{5.918833in}{1.233285in}}%
\pgfpathlineto{\pgfqpoint{5.946532in}{1.234498in}}%
\pgfpathlineto{\pgfqpoint{5.974231in}{1.235444in}}%
\pgfpathlineto{\pgfqpoint{6.001931in}{1.236177in}}%
\pgfpathlineto{\pgfqpoint{6.029630in}{1.236741in}}%
\pgfpathlineto{\pgfqpoint{6.057329in}{1.237173in}}%
\pgfpathlineto{\pgfqpoint{6.067463in}{1.237294in}}%
\pgfusepath{stroke}%
\end{pgfscope}%
\begin{pgfscope}%
\pgfpathrectangle{\pgfqpoint{4.701213in}{0.383578in}}{\pgfqpoint{1.356250in}{1.540000in}}%
\pgfusepath{clip}%
\pgfsetrectcap%
\pgfsetroundjoin%
\pgfsetlinewidth{0.803000pt}%
\definecolor{currentstroke}{rgb}{0.333333,0.333333,0.333333}%
\pgfsetstrokecolor{currentstroke}%
\pgfsetstrokeopacity{0.300000}%
\pgfsetdash{}{0pt}%
\pgfpathmoveto{\pgfqpoint{4.755463in}{1.238500in}}%
\pgfpathlineto{\pgfqpoint{4.783162in}{1.238500in}}%
\pgfpathlineto{\pgfqpoint{4.810861in}{1.238500in}}%
\pgfpathlineto{\pgfqpoint{4.838561in}{1.238500in}}%
\pgfpathlineto{\pgfqpoint{4.866260in}{1.238500in}}%
\pgfpathlineto{\pgfqpoint{4.893959in}{1.238500in}}%
\pgfpathlineto{\pgfqpoint{4.921658in}{1.238500in}}%
\pgfpathlineto{\pgfqpoint{4.949358in}{1.238500in}}%
\pgfpathlineto{\pgfqpoint{4.977057in}{1.238500in}}%
\pgfpathlineto{\pgfqpoint{5.004756in}{1.238500in}}%
\pgfpathlineto{\pgfqpoint{5.032456in}{1.238500in}}%
\pgfpathlineto{\pgfqpoint{5.060155in}{1.238500in}}%
\pgfpathlineto{\pgfqpoint{5.087854in}{1.238500in}}%
\pgfpathlineto{\pgfqpoint{5.115553in}{1.238500in}}%
\pgfpathlineto{\pgfqpoint{5.143253in}{1.238500in}}%
\pgfpathlineto{\pgfqpoint{5.170952in}{1.238500in}}%
\pgfpathlineto{\pgfqpoint{5.198651in}{1.238500in}}%
\pgfpathlineto{\pgfqpoint{5.226351in}{1.238500in}}%
\pgfpathlineto{\pgfqpoint{5.254050in}{1.238500in}}%
\pgfpathlineto{\pgfqpoint{5.281749in}{1.238500in}}%
\pgfpathlineto{\pgfqpoint{5.309448in}{1.238500in}}%
\pgfpathlineto{\pgfqpoint{5.337148in}{1.238500in}}%
\pgfpathlineto{\pgfqpoint{5.364847in}{1.238500in}}%
\pgfpathlineto{\pgfqpoint{5.392546in}{1.238500in}}%
\pgfpathlineto{\pgfqpoint{5.420246in}{1.238500in}}%
\pgfpathlineto{\pgfqpoint{5.447945in}{1.238500in}}%
\pgfpathlineto{\pgfqpoint{5.475644in}{1.238500in}}%
\pgfpathlineto{\pgfqpoint{5.503343in}{1.238500in}}%
\pgfpathlineto{\pgfqpoint{5.531043in}{1.238500in}}%
\pgfpathlineto{\pgfqpoint{5.558742in}{1.238500in}}%
\pgfpathlineto{\pgfqpoint{5.586441in}{1.238500in}}%
\pgfpathlineto{\pgfqpoint{5.614141in}{1.238500in}}%
\pgfpathlineto{\pgfqpoint{5.641840in}{1.238500in}}%
\pgfpathlineto{\pgfqpoint{5.669539in}{1.238500in}}%
\pgfpathlineto{\pgfqpoint{5.697238in}{1.238500in}}%
\pgfpathlineto{\pgfqpoint{5.724938in}{1.238500in}}%
\pgfpathlineto{\pgfqpoint{5.752637in}{1.238500in}}%
\pgfpathlineto{\pgfqpoint{5.780336in}{1.238500in}}%
\pgfpathlineto{\pgfqpoint{5.808036in}{1.238500in}}%
\pgfpathlineto{\pgfqpoint{5.835735in}{1.238500in}}%
\pgfpathlineto{\pgfqpoint{5.863434in}{1.238500in}}%
\pgfpathlineto{\pgfqpoint{5.891133in}{1.238500in}}%
\pgfpathlineto{\pgfqpoint{5.918833in}{1.238500in}}%
\pgfpathlineto{\pgfqpoint{5.946532in}{1.238500in}}%
\pgfpathlineto{\pgfqpoint{5.974231in}{1.238500in}}%
\pgfpathlineto{\pgfqpoint{6.001931in}{1.238500in}}%
\pgfpathlineto{\pgfqpoint{6.029630in}{1.238500in}}%
\pgfpathlineto{\pgfqpoint{6.057329in}{1.238500in}}%
\pgfpathlineto{\pgfqpoint{6.067463in}{1.238500in}}%
\pgfusepath{stroke}%
\end{pgfscope}%
\begin{pgfscope}%
\pgfpathrectangle{\pgfqpoint{4.701213in}{0.383578in}}{\pgfqpoint{1.356250in}{1.540000in}}%
\pgfusepath{clip}%
\pgfsetrectcap%
\pgfsetroundjoin%
\pgfsetlinewidth{0.803000pt}%
\definecolor{currentstroke}{rgb}{0.333333,0.333333,0.333333}%
\pgfsetstrokecolor{currentstroke}%
\pgfsetstrokeopacity{0.300000}%
\pgfsetdash{}{0pt}%
\pgfpathmoveto{\pgfqpoint{4.755463in}{1.238500in}}%
\pgfpathlineto{\pgfqpoint{4.783162in}{1.238500in}}%
\pgfpathlineto{\pgfqpoint{4.810861in}{1.238500in}}%
\pgfpathlineto{\pgfqpoint{4.838561in}{1.238500in}}%
\pgfpathlineto{\pgfqpoint{4.866260in}{1.238500in}}%
\pgfpathlineto{\pgfqpoint{4.893959in}{1.238500in}}%
\pgfpathlineto{\pgfqpoint{4.921658in}{1.238500in}}%
\pgfpathlineto{\pgfqpoint{4.949358in}{1.238500in}}%
\pgfpathlineto{\pgfqpoint{4.977057in}{1.238500in}}%
\pgfpathlineto{\pgfqpoint{5.004756in}{1.238500in}}%
\pgfpathlineto{\pgfqpoint{5.032456in}{1.238500in}}%
\pgfpathlineto{\pgfqpoint{5.060155in}{1.238500in}}%
\pgfpathlineto{\pgfqpoint{5.087854in}{1.238500in}}%
\pgfpathlineto{\pgfqpoint{5.115553in}{1.238500in}}%
\pgfpathlineto{\pgfqpoint{5.143253in}{1.238500in}}%
\pgfpathlineto{\pgfqpoint{5.170952in}{1.238500in}}%
\pgfpathlineto{\pgfqpoint{5.198651in}{1.238500in}}%
\pgfpathlineto{\pgfqpoint{5.226351in}{1.238500in}}%
\pgfpathlineto{\pgfqpoint{5.254050in}{1.238500in}}%
\pgfpathlineto{\pgfqpoint{5.281749in}{1.238500in}}%
\pgfpathlineto{\pgfqpoint{5.309448in}{1.238500in}}%
\pgfpathlineto{\pgfqpoint{5.337148in}{1.238500in}}%
\pgfpathlineto{\pgfqpoint{5.364847in}{1.238500in}}%
\pgfpathlineto{\pgfqpoint{5.392546in}{1.238500in}}%
\pgfpathlineto{\pgfqpoint{5.420246in}{1.238500in}}%
\pgfpathlineto{\pgfqpoint{5.447945in}{1.238500in}}%
\pgfpathlineto{\pgfqpoint{5.475644in}{1.238500in}}%
\pgfpathlineto{\pgfqpoint{5.503343in}{1.238500in}}%
\pgfpathlineto{\pgfqpoint{5.531043in}{1.238500in}}%
\pgfpathlineto{\pgfqpoint{5.558742in}{1.238500in}}%
\pgfpathlineto{\pgfqpoint{5.586441in}{1.238500in}}%
\pgfpathlineto{\pgfqpoint{5.614141in}{1.238500in}}%
\pgfpathlineto{\pgfqpoint{5.641840in}{1.238500in}}%
\pgfpathlineto{\pgfqpoint{5.669539in}{1.238500in}}%
\pgfpathlineto{\pgfqpoint{5.697238in}{1.238500in}}%
\pgfpathlineto{\pgfqpoint{5.724938in}{1.238500in}}%
\pgfpathlineto{\pgfqpoint{5.752637in}{1.238500in}}%
\pgfpathlineto{\pgfqpoint{5.780336in}{1.238500in}}%
\pgfpathlineto{\pgfqpoint{5.808036in}{1.238500in}}%
\pgfpathlineto{\pgfqpoint{5.835735in}{1.238500in}}%
\pgfpathlineto{\pgfqpoint{5.863434in}{1.238500in}}%
\pgfpathlineto{\pgfqpoint{5.891133in}{1.238500in}}%
\pgfpathlineto{\pgfqpoint{5.918833in}{1.238500in}}%
\pgfpathlineto{\pgfqpoint{5.946532in}{1.238500in}}%
\pgfpathlineto{\pgfqpoint{5.974231in}{1.238500in}}%
\pgfpathlineto{\pgfqpoint{6.001931in}{1.238500in}}%
\pgfpathlineto{\pgfqpoint{6.029630in}{1.238500in}}%
\pgfpathlineto{\pgfqpoint{6.057329in}{1.238500in}}%
\pgfpathlineto{\pgfqpoint{6.067463in}{1.238500in}}%
\pgfusepath{stroke}%
\end{pgfscope}%
\begin{pgfscope}%
\pgfpathrectangle{\pgfqpoint{4.701213in}{0.383578in}}{\pgfqpoint{1.356250in}{1.540000in}}%
\pgfusepath{clip}%
\pgfsetrectcap%
\pgfsetroundjoin%
\pgfsetlinewidth{0.803000pt}%
\definecolor{currentstroke}{rgb}{0.333333,0.333333,0.333333}%
\pgfsetstrokecolor{currentstroke}%
\pgfsetstrokeopacity{0.300000}%
\pgfsetdash{}{0pt}%
\pgfpathmoveto{\pgfqpoint{4.755463in}{1.238500in}}%
\pgfpathlineto{\pgfqpoint{4.783162in}{1.238500in}}%
\pgfpathlineto{\pgfqpoint{4.810861in}{1.238500in}}%
\pgfpathlineto{\pgfqpoint{4.838561in}{1.238500in}}%
\pgfpathlineto{\pgfqpoint{4.866260in}{1.238500in}}%
\pgfpathlineto{\pgfqpoint{4.893959in}{1.238500in}}%
\pgfpathlineto{\pgfqpoint{4.921658in}{1.238500in}}%
\pgfpathlineto{\pgfqpoint{4.949358in}{1.238500in}}%
\pgfpathlineto{\pgfqpoint{4.977057in}{1.238500in}}%
\pgfpathlineto{\pgfqpoint{5.004756in}{1.238500in}}%
\pgfpathlineto{\pgfqpoint{5.032456in}{1.238500in}}%
\pgfpathlineto{\pgfqpoint{5.060155in}{1.238500in}}%
\pgfpathlineto{\pgfqpoint{5.087854in}{1.238500in}}%
\pgfpathlineto{\pgfqpoint{5.115553in}{1.238500in}}%
\pgfpathlineto{\pgfqpoint{5.143253in}{1.238500in}}%
\pgfpathlineto{\pgfqpoint{5.170952in}{1.238500in}}%
\pgfpathlineto{\pgfqpoint{5.198651in}{1.238500in}}%
\pgfpathlineto{\pgfqpoint{5.226351in}{1.238500in}}%
\pgfpathlineto{\pgfqpoint{5.254050in}{1.238500in}}%
\pgfpathlineto{\pgfqpoint{5.281749in}{1.238500in}}%
\pgfpathlineto{\pgfqpoint{5.309448in}{1.238500in}}%
\pgfpathlineto{\pgfqpoint{5.337148in}{1.238500in}}%
\pgfpathlineto{\pgfqpoint{5.364847in}{1.238500in}}%
\pgfpathlineto{\pgfqpoint{5.392546in}{1.238500in}}%
\pgfpathlineto{\pgfqpoint{5.420246in}{1.238500in}}%
\pgfpathlineto{\pgfqpoint{5.447945in}{1.238500in}}%
\pgfpathlineto{\pgfqpoint{5.475644in}{1.238500in}}%
\pgfpathlineto{\pgfqpoint{5.503343in}{1.238500in}}%
\pgfpathlineto{\pgfqpoint{5.531043in}{1.238500in}}%
\pgfpathlineto{\pgfqpoint{5.558742in}{1.238500in}}%
\pgfpathlineto{\pgfqpoint{5.586441in}{1.238500in}}%
\pgfpathlineto{\pgfqpoint{5.614141in}{1.238500in}}%
\pgfpathlineto{\pgfqpoint{5.641840in}{1.238500in}}%
\pgfpathlineto{\pgfqpoint{5.669539in}{1.238500in}}%
\pgfpathlineto{\pgfqpoint{5.697238in}{1.238500in}}%
\pgfpathlineto{\pgfqpoint{5.724938in}{1.238500in}}%
\pgfpathlineto{\pgfqpoint{5.752637in}{1.238500in}}%
\pgfpathlineto{\pgfqpoint{5.780336in}{1.238500in}}%
\pgfpathlineto{\pgfqpoint{5.808036in}{1.238500in}}%
\pgfpathlineto{\pgfqpoint{5.835735in}{1.238500in}}%
\pgfpathlineto{\pgfqpoint{5.863434in}{1.238500in}}%
\pgfpathlineto{\pgfqpoint{5.891133in}{1.238500in}}%
\pgfpathlineto{\pgfqpoint{5.918833in}{1.238500in}}%
\pgfpathlineto{\pgfqpoint{5.946532in}{1.238500in}}%
\pgfpathlineto{\pgfqpoint{5.974231in}{1.238500in}}%
\pgfpathlineto{\pgfqpoint{6.001931in}{1.238500in}}%
\pgfpathlineto{\pgfqpoint{6.029630in}{1.238500in}}%
\pgfpathlineto{\pgfqpoint{6.057329in}{1.238500in}}%
\pgfpathlineto{\pgfqpoint{6.067463in}{1.238500in}}%
\pgfusepath{stroke}%
\end{pgfscope}%
\begin{pgfscope}%
\pgfpathrectangle{\pgfqpoint{4.701213in}{0.383578in}}{\pgfqpoint{1.356250in}{1.540000in}}%
\pgfusepath{clip}%
\pgfsetrectcap%
\pgfsetroundjoin%
\pgfsetlinewidth{0.803000pt}%
\definecolor{currentstroke}{rgb}{0.333333,0.333333,0.333333}%
\pgfsetstrokecolor{currentstroke}%
\pgfsetstrokeopacity{0.300000}%
\pgfsetdash{}{0pt}%
\pgfpathmoveto{\pgfqpoint{4.755463in}{1.238500in}}%
\pgfpathlineto{\pgfqpoint{4.783162in}{1.238500in}}%
\pgfpathlineto{\pgfqpoint{4.810861in}{1.238500in}}%
\pgfpathlineto{\pgfqpoint{4.838561in}{1.238500in}}%
\pgfpathlineto{\pgfqpoint{4.866260in}{1.238500in}}%
\pgfpathlineto{\pgfqpoint{4.893959in}{1.238500in}}%
\pgfpathlineto{\pgfqpoint{4.921658in}{1.238500in}}%
\pgfpathlineto{\pgfqpoint{4.949358in}{1.238500in}}%
\pgfpathlineto{\pgfqpoint{4.977057in}{1.238500in}}%
\pgfpathlineto{\pgfqpoint{5.004756in}{1.238500in}}%
\pgfpathlineto{\pgfqpoint{5.032456in}{1.238500in}}%
\pgfpathlineto{\pgfqpoint{5.060155in}{1.238500in}}%
\pgfpathlineto{\pgfqpoint{5.087854in}{1.238500in}}%
\pgfpathlineto{\pgfqpoint{5.115553in}{1.238500in}}%
\pgfpathlineto{\pgfqpoint{5.143253in}{1.238500in}}%
\pgfpathlineto{\pgfqpoint{5.170952in}{1.238500in}}%
\pgfpathlineto{\pgfqpoint{5.198651in}{1.238500in}}%
\pgfpathlineto{\pgfqpoint{5.226351in}{1.238500in}}%
\pgfpathlineto{\pgfqpoint{5.254050in}{1.238500in}}%
\pgfpathlineto{\pgfqpoint{5.281749in}{1.238500in}}%
\pgfpathlineto{\pgfqpoint{5.309448in}{1.238500in}}%
\pgfpathlineto{\pgfqpoint{5.337148in}{1.238500in}}%
\pgfpathlineto{\pgfqpoint{5.364847in}{1.238500in}}%
\pgfpathlineto{\pgfqpoint{5.392546in}{1.238500in}}%
\pgfpathlineto{\pgfqpoint{5.420246in}{1.238500in}}%
\pgfpathlineto{\pgfqpoint{5.447945in}{1.238500in}}%
\pgfpathlineto{\pgfqpoint{5.475644in}{1.238500in}}%
\pgfpathlineto{\pgfqpoint{5.503343in}{1.238500in}}%
\pgfpathlineto{\pgfqpoint{5.531043in}{1.238500in}}%
\pgfpathlineto{\pgfqpoint{5.558742in}{1.238500in}}%
\pgfpathlineto{\pgfqpoint{5.586441in}{1.238500in}}%
\pgfpathlineto{\pgfqpoint{5.614141in}{1.238500in}}%
\pgfpathlineto{\pgfqpoint{5.641840in}{1.238500in}}%
\pgfpathlineto{\pgfqpoint{5.669539in}{1.238500in}}%
\pgfpathlineto{\pgfqpoint{5.697238in}{1.238500in}}%
\pgfpathlineto{\pgfqpoint{5.724938in}{1.238500in}}%
\pgfpathlineto{\pgfqpoint{5.752637in}{1.238500in}}%
\pgfpathlineto{\pgfqpoint{5.780336in}{1.238500in}}%
\pgfpathlineto{\pgfqpoint{5.808036in}{1.238500in}}%
\pgfpathlineto{\pgfqpoint{5.835735in}{1.238500in}}%
\pgfpathlineto{\pgfqpoint{5.863434in}{1.238500in}}%
\pgfpathlineto{\pgfqpoint{5.891133in}{1.238500in}}%
\pgfpathlineto{\pgfqpoint{5.918833in}{1.238500in}}%
\pgfpathlineto{\pgfqpoint{5.946532in}{1.238500in}}%
\pgfpathlineto{\pgfqpoint{5.974231in}{1.238500in}}%
\pgfpathlineto{\pgfqpoint{6.001931in}{1.238500in}}%
\pgfpathlineto{\pgfqpoint{6.029630in}{1.238500in}}%
\pgfpathlineto{\pgfqpoint{6.057329in}{1.238500in}}%
\pgfpathlineto{\pgfqpoint{6.067463in}{1.238500in}}%
\pgfusepath{stroke}%
\end{pgfscope}%
\begin{pgfscope}%
\pgfpathrectangle{\pgfqpoint{4.701213in}{0.383578in}}{\pgfqpoint{1.356250in}{1.540000in}}%
\pgfusepath{clip}%
\pgfsetrectcap%
\pgfsetroundjoin%
\pgfsetlinewidth{0.803000pt}%
\definecolor{currentstroke}{rgb}{0.333333,0.333333,0.333333}%
\pgfsetstrokecolor{currentstroke}%
\pgfsetstrokeopacity{0.300000}%
\pgfsetdash{}{0pt}%
\pgfpathmoveto{\pgfqpoint{4.755463in}{1.238500in}}%
\pgfpathlineto{\pgfqpoint{4.783162in}{1.238500in}}%
\pgfpathlineto{\pgfqpoint{4.810861in}{1.238500in}}%
\pgfpathlineto{\pgfqpoint{4.838561in}{1.238500in}}%
\pgfpathlineto{\pgfqpoint{4.866260in}{1.238500in}}%
\pgfpathlineto{\pgfqpoint{4.893959in}{1.238500in}}%
\pgfpathlineto{\pgfqpoint{4.921658in}{1.238500in}}%
\pgfpathlineto{\pgfqpoint{4.949358in}{1.238500in}}%
\pgfpathlineto{\pgfqpoint{4.977057in}{1.238500in}}%
\pgfpathlineto{\pgfqpoint{5.004756in}{1.238500in}}%
\pgfpathlineto{\pgfqpoint{5.032456in}{1.238500in}}%
\pgfpathlineto{\pgfqpoint{5.060155in}{1.238500in}}%
\pgfpathlineto{\pgfqpoint{5.087854in}{1.238500in}}%
\pgfpathlineto{\pgfqpoint{5.115553in}{1.238500in}}%
\pgfpathlineto{\pgfqpoint{5.143253in}{1.238500in}}%
\pgfpathlineto{\pgfqpoint{5.170952in}{1.238500in}}%
\pgfpathlineto{\pgfqpoint{5.198651in}{1.238500in}}%
\pgfpathlineto{\pgfqpoint{5.226351in}{1.238500in}}%
\pgfpathlineto{\pgfqpoint{5.254050in}{1.238500in}}%
\pgfpathlineto{\pgfqpoint{5.281749in}{1.238500in}}%
\pgfpathlineto{\pgfqpoint{5.309448in}{1.238500in}}%
\pgfpathlineto{\pgfqpoint{5.337148in}{1.238500in}}%
\pgfpathlineto{\pgfqpoint{5.364847in}{1.238500in}}%
\pgfpathlineto{\pgfqpoint{5.392546in}{1.238500in}}%
\pgfpathlineto{\pgfqpoint{5.420246in}{1.238500in}}%
\pgfpathlineto{\pgfqpoint{5.447945in}{1.238500in}}%
\pgfpathlineto{\pgfqpoint{5.475644in}{1.238500in}}%
\pgfpathlineto{\pgfqpoint{5.503343in}{1.238500in}}%
\pgfpathlineto{\pgfqpoint{5.531043in}{1.238500in}}%
\pgfpathlineto{\pgfqpoint{5.558742in}{1.238500in}}%
\pgfpathlineto{\pgfqpoint{5.586441in}{1.238500in}}%
\pgfpathlineto{\pgfqpoint{5.614141in}{1.238500in}}%
\pgfpathlineto{\pgfqpoint{5.641840in}{1.238500in}}%
\pgfpathlineto{\pgfqpoint{5.669539in}{1.238500in}}%
\pgfpathlineto{\pgfqpoint{5.697238in}{1.238500in}}%
\pgfpathlineto{\pgfqpoint{5.724938in}{1.238500in}}%
\pgfpathlineto{\pgfqpoint{5.752637in}{1.238500in}}%
\pgfpathlineto{\pgfqpoint{5.780336in}{1.238500in}}%
\pgfpathlineto{\pgfqpoint{5.808036in}{1.238500in}}%
\pgfpathlineto{\pgfqpoint{5.835735in}{1.238500in}}%
\pgfpathlineto{\pgfqpoint{5.863434in}{1.238500in}}%
\pgfpathlineto{\pgfqpoint{5.891133in}{1.238500in}}%
\pgfpathlineto{\pgfqpoint{5.918833in}{1.238500in}}%
\pgfpathlineto{\pgfqpoint{5.946532in}{1.238500in}}%
\pgfpathlineto{\pgfqpoint{5.974231in}{1.238500in}}%
\pgfpathlineto{\pgfqpoint{6.001931in}{1.238500in}}%
\pgfpathlineto{\pgfqpoint{6.029630in}{1.238500in}}%
\pgfpathlineto{\pgfqpoint{6.057329in}{1.238500in}}%
\pgfpathlineto{\pgfqpoint{6.067463in}{1.238500in}}%
\pgfusepath{stroke}%
\end{pgfscope}%
\begin{pgfscope}%
\pgfpathrectangle{\pgfqpoint{4.701213in}{0.383578in}}{\pgfqpoint{1.356250in}{1.540000in}}%
\pgfusepath{clip}%
\pgfsetrectcap%
\pgfsetroundjoin%
\pgfsetlinewidth{0.803000pt}%
\definecolor{currentstroke}{rgb}{0.333333,0.333333,0.333333}%
\pgfsetstrokecolor{currentstroke}%
\pgfsetstrokeopacity{0.300000}%
\pgfsetdash{}{0pt}%
\pgfpathmoveto{\pgfqpoint{4.755463in}{1.238500in}}%
\pgfpathlineto{\pgfqpoint{4.783162in}{1.238500in}}%
\pgfpathlineto{\pgfqpoint{4.810861in}{1.238500in}}%
\pgfpathlineto{\pgfqpoint{4.838561in}{1.238500in}}%
\pgfpathlineto{\pgfqpoint{4.866260in}{1.238500in}}%
\pgfpathlineto{\pgfqpoint{4.893959in}{1.238500in}}%
\pgfpathlineto{\pgfqpoint{4.921658in}{1.238500in}}%
\pgfpathlineto{\pgfqpoint{4.949358in}{1.238500in}}%
\pgfpathlineto{\pgfqpoint{4.977057in}{1.238500in}}%
\pgfpathlineto{\pgfqpoint{5.004756in}{1.238500in}}%
\pgfpathlineto{\pgfqpoint{5.032456in}{1.238500in}}%
\pgfpathlineto{\pgfqpoint{5.060155in}{1.238500in}}%
\pgfpathlineto{\pgfqpoint{5.087854in}{1.238500in}}%
\pgfpathlineto{\pgfqpoint{5.115553in}{1.238500in}}%
\pgfpathlineto{\pgfqpoint{5.143253in}{1.238500in}}%
\pgfpathlineto{\pgfqpoint{5.170952in}{1.238500in}}%
\pgfpathlineto{\pgfqpoint{5.198651in}{1.238500in}}%
\pgfpathlineto{\pgfqpoint{5.226351in}{1.238500in}}%
\pgfpathlineto{\pgfqpoint{5.254050in}{1.238500in}}%
\pgfpathlineto{\pgfqpoint{5.281749in}{1.238500in}}%
\pgfpathlineto{\pgfqpoint{5.309448in}{1.238500in}}%
\pgfpathlineto{\pgfqpoint{5.337148in}{1.238500in}}%
\pgfpathlineto{\pgfqpoint{5.364847in}{1.238500in}}%
\pgfpathlineto{\pgfqpoint{5.392546in}{1.238500in}}%
\pgfpathlineto{\pgfqpoint{5.420246in}{1.238500in}}%
\pgfpathlineto{\pgfqpoint{5.447945in}{1.238500in}}%
\pgfpathlineto{\pgfqpoint{5.475644in}{1.238500in}}%
\pgfpathlineto{\pgfqpoint{5.503343in}{1.238500in}}%
\pgfpathlineto{\pgfqpoint{5.531043in}{1.238500in}}%
\pgfpathlineto{\pgfqpoint{5.558742in}{1.238500in}}%
\pgfpathlineto{\pgfqpoint{5.586441in}{1.238500in}}%
\pgfpathlineto{\pgfqpoint{5.614141in}{1.238500in}}%
\pgfpathlineto{\pgfqpoint{5.641840in}{1.238500in}}%
\pgfpathlineto{\pgfqpoint{5.669539in}{1.238500in}}%
\pgfpathlineto{\pgfqpoint{5.697238in}{1.238500in}}%
\pgfpathlineto{\pgfqpoint{5.724938in}{1.238500in}}%
\pgfpathlineto{\pgfqpoint{5.752637in}{1.238500in}}%
\pgfpathlineto{\pgfqpoint{5.780336in}{1.238500in}}%
\pgfpathlineto{\pgfqpoint{5.808036in}{1.238500in}}%
\pgfpathlineto{\pgfqpoint{5.835735in}{1.238500in}}%
\pgfpathlineto{\pgfqpoint{5.863434in}{1.238500in}}%
\pgfpathlineto{\pgfqpoint{5.891133in}{1.238500in}}%
\pgfpathlineto{\pgfqpoint{5.918833in}{1.238500in}}%
\pgfpathlineto{\pgfqpoint{5.946532in}{1.238500in}}%
\pgfpathlineto{\pgfqpoint{5.974231in}{1.238500in}}%
\pgfpathlineto{\pgfqpoint{6.001931in}{1.238500in}}%
\pgfpathlineto{\pgfqpoint{6.029630in}{1.238500in}}%
\pgfpathlineto{\pgfqpoint{6.057329in}{1.238500in}}%
\pgfpathlineto{\pgfqpoint{6.067463in}{1.238500in}}%
\pgfusepath{stroke}%
\end{pgfscope}%
\begin{pgfscope}%
\pgfpathrectangle{\pgfqpoint{4.701213in}{0.383578in}}{\pgfqpoint{1.356250in}{1.540000in}}%
\pgfusepath{clip}%
\pgfsetrectcap%
\pgfsetroundjoin%
\pgfsetlinewidth{0.803000pt}%
\definecolor{currentstroke}{rgb}{0.333333,0.333333,0.333333}%
\pgfsetstrokecolor{currentstroke}%
\pgfsetstrokeopacity{0.300000}%
\pgfsetdash{}{0pt}%
\pgfpathmoveto{\pgfqpoint{4.755463in}{1.238500in}}%
\pgfpathlineto{\pgfqpoint{4.783162in}{1.238500in}}%
\pgfpathlineto{\pgfqpoint{4.810861in}{1.238500in}}%
\pgfpathlineto{\pgfqpoint{4.838561in}{1.238500in}}%
\pgfpathlineto{\pgfqpoint{4.866260in}{1.238500in}}%
\pgfpathlineto{\pgfqpoint{4.893959in}{1.238500in}}%
\pgfpathlineto{\pgfqpoint{4.921658in}{1.238500in}}%
\pgfpathlineto{\pgfqpoint{4.949358in}{1.238500in}}%
\pgfpathlineto{\pgfqpoint{4.977057in}{1.238500in}}%
\pgfpathlineto{\pgfqpoint{5.004756in}{1.156161in}}%
\pgfpathlineto{\pgfqpoint{5.032456in}{1.040510in}}%
\pgfpathlineto{\pgfqpoint{5.060155in}{0.911307in}}%
\pgfpathlineto{\pgfqpoint{5.087854in}{0.781158in}}%
\pgfpathlineto{\pgfqpoint{5.115553in}{0.684254in}}%
\pgfpathlineto{\pgfqpoint{5.143253in}{0.635184in}}%
\pgfpathlineto{\pgfqpoint{5.170952in}{0.631046in}}%
\pgfpathlineto{\pgfqpoint{5.198651in}{0.662892in}}%
\pgfpathlineto{\pgfqpoint{5.226351in}{0.716029in}}%
\pgfpathlineto{\pgfqpoint{5.254050in}{0.778445in}}%
\pgfpathlineto{\pgfqpoint{5.281749in}{0.841269in}}%
\pgfpathlineto{\pgfqpoint{5.309448in}{0.898126in}}%
\pgfpathlineto{\pgfqpoint{5.337148in}{0.946992in}}%
\pgfpathlineto{\pgfqpoint{5.364847in}{0.987444in}}%
\pgfpathlineto{\pgfqpoint{5.392546in}{1.020702in}}%
\pgfpathlineto{\pgfqpoint{5.420246in}{1.049171in}}%
\pgfpathlineto{\pgfqpoint{5.447945in}{1.073787in}}%
\pgfpathlineto{\pgfqpoint{5.475644in}{1.095621in}}%
\pgfpathlineto{\pgfqpoint{5.503343in}{1.115214in}}%
\pgfpathlineto{\pgfqpoint{5.531043in}{1.132826in}}%
\pgfpathlineto{\pgfqpoint{5.558742in}{1.148746in}}%
\pgfpathlineto{\pgfqpoint{5.586441in}{1.163042in}}%
\pgfpathlineto{\pgfqpoint{5.614141in}{1.175543in}}%
\pgfpathlineto{\pgfqpoint{5.641840in}{1.186353in}}%
\pgfpathlineto{\pgfqpoint{5.669539in}{1.195655in}}%
\pgfpathlineto{\pgfqpoint{5.697238in}{1.203552in}}%
\pgfpathlineto{\pgfqpoint{5.724938in}{1.210209in}}%
\pgfpathlineto{\pgfqpoint{5.752637in}{1.215808in}}%
\pgfpathlineto{\pgfqpoint{5.780336in}{1.220438in}}%
\pgfpathlineto{\pgfqpoint{5.808036in}{1.224226in}}%
\pgfpathlineto{\pgfqpoint{5.835735in}{1.227296in}}%
\pgfpathlineto{\pgfqpoint{5.863434in}{1.229765in}}%
\pgfpathlineto{\pgfqpoint{5.891133in}{1.231733in}}%
\pgfpathlineto{\pgfqpoint{5.918833in}{1.233285in}}%
\pgfpathlineto{\pgfqpoint{5.946532in}{1.234498in}}%
\pgfpathlineto{\pgfqpoint{5.974231in}{1.235444in}}%
\pgfpathlineto{\pgfqpoint{6.001931in}{1.236177in}}%
\pgfpathlineto{\pgfqpoint{6.029630in}{1.236741in}}%
\pgfpathlineto{\pgfqpoint{6.057329in}{1.237173in}}%
\pgfpathlineto{\pgfqpoint{6.067463in}{1.237294in}}%
\pgfusepath{stroke}%
\end{pgfscope}%
\begin{pgfscope}%
\pgfpathrectangle{\pgfqpoint{4.701213in}{0.383578in}}{\pgfqpoint{1.356250in}{1.540000in}}%
\pgfusepath{clip}%
\pgfsetrectcap%
\pgfsetroundjoin%
\pgfsetlinewidth{0.803000pt}%
\definecolor{currentstroke}{rgb}{0.333333,0.333333,0.333333}%
\pgfsetstrokecolor{currentstroke}%
\pgfsetstrokeopacity{0.300000}%
\pgfsetdash{}{0pt}%
\pgfpathmoveto{\pgfqpoint{4.755463in}{1.238500in}}%
\pgfpathlineto{\pgfqpoint{4.783162in}{1.238500in}}%
\pgfpathlineto{\pgfqpoint{4.810861in}{1.238500in}}%
\pgfpathlineto{\pgfqpoint{4.838561in}{1.238500in}}%
\pgfpathlineto{\pgfqpoint{4.866260in}{1.238500in}}%
\pgfpathlineto{\pgfqpoint{4.893959in}{1.238500in}}%
\pgfpathlineto{\pgfqpoint{4.921658in}{1.238500in}}%
\pgfpathlineto{\pgfqpoint{4.949358in}{1.238500in}}%
\pgfpathlineto{\pgfqpoint{4.977057in}{1.238500in}}%
\pgfpathlineto{\pgfqpoint{5.004756in}{1.238500in}}%
\pgfpathlineto{\pgfqpoint{5.032456in}{1.238500in}}%
\pgfpathlineto{\pgfqpoint{5.060155in}{1.238500in}}%
\pgfpathlineto{\pgfqpoint{5.087854in}{1.238500in}}%
\pgfpathlineto{\pgfqpoint{5.115553in}{1.238500in}}%
\pgfpathlineto{\pgfqpoint{5.143253in}{1.238500in}}%
\pgfpathlineto{\pgfqpoint{5.170952in}{1.238500in}}%
\pgfpathlineto{\pgfqpoint{5.198651in}{1.238500in}}%
\pgfpathlineto{\pgfqpoint{5.226351in}{1.238500in}}%
\pgfpathlineto{\pgfqpoint{5.254050in}{1.238500in}}%
\pgfpathlineto{\pgfqpoint{5.281749in}{1.238500in}}%
\pgfpathlineto{\pgfqpoint{5.309448in}{1.238500in}}%
\pgfpathlineto{\pgfqpoint{5.337148in}{1.238500in}}%
\pgfpathlineto{\pgfqpoint{5.364847in}{1.238500in}}%
\pgfpathlineto{\pgfqpoint{5.392546in}{1.238500in}}%
\pgfpathlineto{\pgfqpoint{5.420246in}{1.238500in}}%
\pgfpathlineto{\pgfqpoint{5.447945in}{1.238500in}}%
\pgfpathlineto{\pgfqpoint{5.475644in}{1.238500in}}%
\pgfpathlineto{\pgfqpoint{5.503343in}{1.238500in}}%
\pgfpathlineto{\pgfqpoint{5.531043in}{1.238500in}}%
\pgfpathlineto{\pgfqpoint{5.558742in}{1.238500in}}%
\pgfpathlineto{\pgfqpoint{5.586441in}{1.238500in}}%
\pgfpathlineto{\pgfqpoint{5.614141in}{1.238500in}}%
\pgfpathlineto{\pgfqpoint{5.641840in}{1.238500in}}%
\pgfpathlineto{\pgfqpoint{5.669539in}{1.238500in}}%
\pgfpathlineto{\pgfqpoint{5.697238in}{1.238500in}}%
\pgfpathlineto{\pgfqpoint{5.724938in}{1.238500in}}%
\pgfpathlineto{\pgfqpoint{5.752637in}{1.238500in}}%
\pgfpathlineto{\pgfqpoint{5.780336in}{1.238500in}}%
\pgfpathlineto{\pgfqpoint{5.808036in}{1.238500in}}%
\pgfpathlineto{\pgfqpoint{5.835735in}{1.238500in}}%
\pgfpathlineto{\pgfqpoint{5.863434in}{1.238500in}}%
\pgfpathlineto{\pgfqpoint{5.891133in}{1.238500in}}%
\pgfpathlineto{\pgfqpoint{5.918833in}{1.238500in}}%
\pgfpathlineto{\pgfqpoint{5.946532in}{1.238500in}}%
\pgfpathlineto{\pgfqpoint{5.974231in}{1.238500in}}%
\pgfpathlineto{\pgfqpoint{6.001931in}{1.238500in}}%
\pgfpathlineto{\pgfqpoint{6.029630in}{1.238500in}}%
\pgfpathlineto{\pgfqpoint{6.057329in}{1.238500in}}%
\pgfpathlineto{\pgfqpoint{6.067463in}{1.238500in}}%
\pgfusepath{stroke}%
\end{pgfscope}%
\begin{pgfscope}%
\pgfpathrectangle{\pgfqpoint{4.701213in}{0.383578in}}{\pgfqpoint{1.356250in}{1.540000in}}%
\pgfusepath{clip}%
\pgfsetrectcap%
\pgfsetroundjoin%
\pgfsetlinewidth{0.803000pt}%
\definecolor{currentstroke}{rgb}{0.333333,0.333333,0.333333}%
\pgfsetstrokecolor{currentstroke}%
\pgfsetstrokeopacity{0.300000}%
\pgfsetdash{}{0pt}%
\pgfpathmoveto{\pgfqpoint{4.755463in}{1.238500in}}%
\pgfpathlineto{\pgfqpoint{4.783162in}{1.238500in}}%
\pgfpathlineto{\pgfqpoint{4.810861in}{1.238500in}}%
\pgfpathlineto{\pgfqpoint{4.838561in}{1.238500in}}%
\pgfpathlineto{\pgfqpoint{4.866260in}{1.238500in}}%
\pgfpathlineto{\pgfqpoint{4.893959in}{1.238500in}}%
\pgfpathlineto{\pgfqpoint{4.921658in}{1.238500in}}%
\pgfpathlineto{\pgfqpoint{4.949358in}{1.238500in}}%
\pgfpathlineto{\pgfqpoint{4.977057in}{1.238500in}}%
\pgfpathlineto{\pgfqpoint{5.004756in}{1.238500in}}%
\pgfpathlineto{\pgfqpoint{5.032456in}{1.238500in}}%
\pgfpathlineto{\pgfqpoint{5.060155in}{1.238500in}}%
\pgfpathlineto{\pgfqpoint{5.087854in}{1.238500in}}%
\pgfpathlineto{\pgfqpoint{5.115553in}{1.238500in}}%
\pgfpathlineto{\pgfqpoint{5.143253in}{1.238500in}}%
\pgfpathlineto{\pgfqpoint{5.170952in}{1.238500in}}%
\pgfpathlineto{\pgfqpoint{5.198651in}{1.238500in}}%
\pgfpathlineto{\pgfqpoint{5.226351in}{1.238500in}}%
\pgfpathlineto{\pgfqpoint{5.254050in}{1.238500in}}%
\pgfpathlineto{\pgfqpoint{5.281749in}{1.238500in}}%
\pgfpathlineto{\pgfqpoint{5.309448in}{1.238500in}}%
\pgfpathlineto{\pgfqpoint{5.337148in}{1.238500in}}%
\pgfpathlineto{\pgfqpoint{5.364847in}{1.238500in}}%
\pgfpathlineto{\pgfqpoint{5.392546in}{1.238500in}}%
\pgfpathlineto{\pgfqpoint{5.420246in}{1.238500in}}%
\pgfpathlineto{\pgfqpoint{5.447945in}{1.238500in}}%
\pgfpathlineto{\pgfqpoint{5.475644in}{1.238500in}}%
\pgfpathlineto{\pgfqpoint{5.503343in}{1.238500in}}%
\pgfpathlineto{\pgfqpoint{5.531043in}{1.238500in}}%
\pgfpathlineto{\pgfqpoint{5.558742in}{1.238500in}}%
\pgfpathlineto{\pgfqpoint{5.586441in}{1.238500in}}%
\pgfpathlineto{\pgfqpoint{5.614141in}{1.238500in}}%
\pgfpathlineto{\pgfqpoint{5.641840in}{1.238500in}}%
\pgfpathlineto{\pgfqpoint{5.669539in}{1.238500in}}%
\pgfpathlineto{\pgfqpoint{5.697238in}{1.238500in}}%
\pgfpathlineto{\pgfqpoint{5.724938in}{1.238500in}}%
\pgfpathlineto{\pgfqpoint{5.752637in}{1.238500in}}%
\pgfpathlineto{\pgfqpoint{5.780336in}{1.238500in}}%
\pgfpathlineto{\pgfqpoint{5.808036in}{1.238500in}}%
\pgfpathlineto{\pgfqpoint{5.835735in}{1.238500in}}%
\pgfpathlineto{\pgfqpoint{5.863434in}{1.238500in}}%
\pgfpathlineto{\pgfqpoint{5.891133in}{1.238500in}}%
\pgfpathlineto{\pgfqpoint{5.918833in}{1.238500in}}%
\pgfpathlineto{\pgfqpoint{5.946532in}{1.238500in}}%
\pgfpathlineto{\pgfqpoint{5.974231in}{1.238500in}}%
\pgfpathlineto{\pgfqpoint{6.001931in}{1.238500in}}%
\pgfpathlineto{\pgfqpoint{6.029630in}{1.238500in}}%
\pgfpathlineto{\pgfqpoint{6.057329in}{1.238500in}}%
\pgfpathlineto{\pgfqpoint{6.067463in}{1.238500in}}%
\pgfusepath{stroke}%
\end{pgfscope}%
\begin{pgfscope}%
\pgfpathrectangle{\pgfqpoint{4.701213in}{0.383578in}}{\pgfqpoint{1.356250in}{1.540000in}}%
\pgfusepath{clip}%
\pgfsetrectcap%
\pgfsetroundjoin%
\pgfsetlinewidth{0.803000pt}%
\definecolor{currentstroke}{rgb}{0.333333,0.333333,0.333333}%
\pgfsetstrokecolor{currentstroke}%
\pgfsetstrokeopacity{0.300000}%
\pgfsetdash{}{0pt}%
\pgfpathmoveto{\pgfqpoint{4.755463in}{1.238500in}}%
\pgfpathlineto{\pgfqpoint{4.783162in}{1.238500in}}%
\pgfpathlineto{\pgfqpoint{4.810861in}{1.238500in}}%
\pgfpathlineto{\pgfqpoint{4.838561in}{1.238500in}}%
\pgfpathlineto{\pgfqpoint{4.866260in}{1.238500in}}%
\pgfpathlineto{\pgfqpoint{4.893959in}{1.238500in}}%
\pgfpathlineto{\pgfqpoint{4.921658in}{1.238500in}}%
\pgfpathlineto{\pgfqpoint{4.949358in}{1.238500in}}%
\pgfpathlineto{\pgfqpoint{4.977057in}{1.238500in}}%
\pgfpathlineto{\pgfqpoint{5.004756in}{1.238500in}}%
\pgfpathlineto{\pgfqpoint{5.032456in}{1.238500in}}%
\pgfpathlineto{\pgfqpoint{5.060155in}{1.238500in}}%
\pgfpathlineto{\pgfqpoint{5.087854in}{1.238500in}}%
\pgfpathlineto{\pgfqpoint{5.115553in}{1.238500in}}%
\pgfpathlineto{\pgfqpoint{5.143253in}{1.238500in}}%
\pgfpathlineto{\pgfqpoint{5.170952in}{1.238500in}}%
\pgfpathlineto{\pgfqpoint{5.198651in}{1.238500in}}%
\pgfpathlineto{\pgfqpoint{5.226351in}{1.238500in}}%
\pgfpathlineto{\pgfqpoint{5.254050in}{1.238500in}}%
\pgfpathlineto{\pgfqpoint{5.281749in}{1.238500in}}%
\pgfpathlineto{\pgfqpoint{5.309448in}{1.238500in}}%
\pgfpathlineto{\pgfqpoint{5.337148in}{1.238500in}}%
\pgfpathlineto{\pgfqpoint{5.364847in}{1.238500in}}%
\pgfpathlineto{\pgfqpoint{5.392546in}{1.238500in}}%
\pgfpathlineto{\pgfqpoint{5.420246in}{1.238500in}}%
\pgfpathlineto{\pgfqpoint{5.447945in}{1.238500in}}%
\pgfpathlineto{\pgfqpoint{5.475644in}{1.238500in}}%
\pgfpathlineto{\pgfqpoint{5.503343in}{1.238500in}}%
\pgfpathlineto{\pgfqpoint{5.531043in}{1.238500in}}%
\pgfpathlineto{\pgfqpoint{5.558742in}{1.238500in}}%
\pgfpathlineto{\pgfqpoint{5.586441in}{1.238500in}}%
\pgfpathlineto{\pgfqpoint{5.614141in}{1.238500in}}%
\pgfpathlineto{\pgfqpoint{5.641840in}{1.238500in}}%
\pgfpathlineto{\pgfqpoint{5.669539in}{1.238500in}}%
\pgfpathlineto{\pgfqpoint{5.697238in}{1.238500in}}%
\pgfpathlineto{\pgfqpoint{5.724938in}{1.238500in}}%
\pgfpathlineto{\pgfqpoint{5.752637in}{1.238500in}}%
\pgfpathlineto{\pgfqpoint{5.780336in}{1.238500in}}%
\pgfpathlineto{\pgfqpoint{5.808036in}{1.238500in}}%
\pgfpathlineto{\pgfqpoint{5.835735in}{1.238500in}}%
\pgfpathlineto{\pgfqpoint{5.863434in}{1.238500in}}%
\pgfpathlineto{\pgfqpoint{5.891133in}{1.238500in}}%
\pgfpathlineto{\pgfqpoint{5.918833in}{1.238500in}}%
\pgfpathlineto{\pgfqpoint{5.946532in}{1.238500in}}%
\pgfpathlineto{\pgfqpoint{5.974231in}{1.238500in}}%
\pgfpathlineto{\pgfqpoint{6.001931in}{1.238500in}}%
\pgfpathlineto{\pgfqpoint{6.029630in}{1.238500in}}%
\pgfpathlineto{\pgfqpoint{6.057329in}{1.238500in}}%
\pgfpathlineto{\pgfqpoint{6.067463in}{1.238500in}}%
\pgfusepath{stroke}%
\end{pgfscope}%
\begin{pgfscope}%
\pgfpathrectangle{\pgfqpoint{4.701213in}{0.383578in}}{\pgfqpoint{1.356250in}{1.540000in}}%
\pgfusepath{clip}%
\pgfsetrectcap%
\pgfsetroundjoin%
\pgfsetlinewidth{0.803000pt}%
\definecolor{currentstroke}{rgb}{0.333333,0.333333,0.333333}%
\pgfsetstrokecolor{currentstroke}%
\pgfsetstrokeopacity{0.300000}%
\pgfsetdash{}{0pt}%
\pgfpathmoveto{\pgfqpoint{4.755463in}{1.238500in}}%
\pgfpathlineto{\pgfqpoint{4.783162in}{1.238500in}}%
\pgfpathlineto{\pgfqpoint{4.810861in}{1.238500in}}%
\pgfpathlineto{\pgfqpoint{4.838561in}{1.238500in}}%
\pgfpathlineto{\pgfqpoint{4.866260in}{1.238500in}}%
\pgfpathlineto{\pgfqpoint{4.893959in}{1.238500in}}%
\pgfpathlineto{\pgfqpoint{4.921658in}{1.238500in}}%
\pgfpathlineto{\pgfqpoint{4.949358in}{1.238500in}}%
\pgfpathlineto{\pgfqpoint{4.977057in}{1.238500in}}%
\pgfpathlineto{\pgfqpoint{5.004756in}{1.238500in}}%
\pgfpathlineto{\pgfqpoint{5.032456in}{1.238500in}}%
\pgfpathlineto{\pgfqpoint{5.060155in}{1.238500in}}%
\pgfpathlineto{\pgfqpoint{5.087854in}{1.238500in}}%
\pgfpathlineto{\pgfqpoint{5.115553in}{1.238500in}}%
\pgfpathlineto{\pgfqpoint{5.143253in}{1.238500in}}%
\pgfpathlineto{\pgfqpoint{5.170952in}{1.238500in}}%
\pgfpathlineto{\pgfqpoint{5.198651in}{1.238500in}}%
\pgfpathlineto{\pgfqpoint{5.226351in}{1.238500in}}%
\pgfpathlineto{\pgfqpoint{5.254050in}{1.238500in}}%
\pgfpathlineto{\pgfqpoint{5.281749in}{1.238500in}}%
\pgfpathlineto{\pgfqpoint{5.309448in}{1.238500in}}%
\pgfpathlineto{\pgfqpoint{5.337148in}{1.238500in}}%
\pgfpathlineto{\pgfqpoint{5.364847in}{1.238500in}}%
\pgfpathlineto{\pgfqpoint{5.392546in}{1.238500in}}%
\pgfpathlineto{\pgfqpoint{5.420246in}{1.238500in}}%
\pgfpathlineto{\pgfqpoint{5.447945in}{1.238500in}}%
\pgfpathlineto{\pgfqpoint{5.475644in}{1.238500in}}%
\pgfpathlineto{\pgfqpoint{5.503343in}{1.238500in}}%
\pgfpathlineto{\pgfqpoint{5.531043in}{1.238500in}}%
\pgfpathlineto{\pgfqpoint{5.558742in}{1.238500in}}%
\pgfpathlineto{\pgfqpoint{5.586441in}{1.238500in}}%
\pgfpathlineto{\pgfqpoint{5.614141in}{1.238500in}}%
\pgfpathlineto{\pgfqpoint{5.641840in}{1.238500in}}%
\pgfpathlineto{\pgfqpoint{5.669539in}{1.238500in}}%
\pgfpathlineto{\pgfqpoint{5.697238in}{1.238500in}}%
\pgfpathlineto{\pgfqpoint{5.724938in}{1.238500in}}%
\pgfpathlineto{\pgfqpoint{5.752637in}{1.238500in}}%
\pgfpathlineto{\pgfqpoint{5.780336in}{1.238500in}}%
\pgfpathlineto{\pgfqpoint{5.808036in}{1.238500in}}%
\pgfpathlineto{\pgfqpoint{5.835735in}{1.238500in}}%
\pgfpathlineto{\pgfqpoint{5.863434in}{1.238500in}}%
\pgfpathlineto{\pgfqpoint{5.891133in}{1.238500in}}%
\pgfpathlineto{\pgfqpoint{5.918833in}{1.238500in}}%
\pgfpathlineto{\pgfqpoint{5.946532in}{1.238500in}}%
\pgfpathlineto{\pgfqpoint{5.974231in}{1.238500in}}%
\pgfpathlineto{\pgfqpoint{6.001931in}{1.238500in}}%
\pgfpathlineto{\pgfqpoint{6.029630in}{1.238500in}}%
\pgfpathlineto{\pgfqpoint{6.057329in}{1.238500in}}%
\pgfpathlineto{\pgfqpoint{6.067463in}{1.238500in}}%
\pgfusepath{stroke}%
\end{pgfscope}%
\begin{pgfscope}%
\pgfpathrectangle{\pgfqpoint{4.701213in}{0.383578in}}{\pgfqpoint{1.356250in}{1.540000in}}%
\pgfusepath{clip}%
\pgfsetrectcap%
\pgfsetroundjoin%
\pgfsetlinewidth{0.803000pt}%
\definecolor{currentstroke}{rgb}{0.333333,0.333333,0.333333}%
\pgfsetstrokecolor{currentstroke}%
\pgfsetstrokeopacity{0.300000}%
\pgfsetdash{}{0pt}%
\pgfpathmoveto{\pgfqpoint{4.755463in}{1.238500in}}%
\pgfpathlineto{\pgfqpoint{4.783162in}{1.238500in}}%
\pgfpathlineto{\pgfqpoint{4.810861in}{1.238500in}}%
\pgfpathlineto{\pgfqpoint{4.838561in}{1.238500in}}%
\pgfpathlineto{\pgfqpoint{4.866260in}{1.238500in}}%
\pgfpathlineto{\pgfqpoint{4.893959in}{1.238500in}}%
\pgfpathlineto{\pgfqpoint{4.921658in}{1.238500in}}%
\pgfpathlineto{\pgfqpoint{4.949358in}{1.238500in}}%
\pgfpathlineto{\pgfqpoint{4.977057in}{1.238500in}}%
\pgfpathlineto{\pgfqpoint{5.004756in}{1.238500in}}%
\pgfpathlineto{\pgfqpoint{5.032456in}{1.238500in}}%
\pgfpathlineto{\pgfqpoint{5.060155in}{1.238500in}}%
\pgfpathlineto{\pgfqpoint{5.087854in}{1.238500in}}%
\pgfpathlineto{\pgfqpoint{5.115553in}{1.238500in}}%
\pgfpathlineto{\pgfqpoint{5.143253in}{1.238500in}}%
\pgfpathlineto{\pgfqpoint{5.170952in}{1.238500in}}%
\pgfpathlineto{\pgfqpoint{5.198651in}{1.238500in}}%
\pgfpathlineto{\pgfqpoint{5.226351in}{1.238500in}}%
\pgfpathlineto{\pgfqpoint{5.254050in}{1.238500in}}%
\pgfpathlineto{\pgfqpoint{5.281749in}{1.238500in}}%
\pgfpathlineto{\pgfqpoint{5.309448in}{1.238500in}}%
\pgfpathlineto{\pgfqpoint{5.337148in}{1.238500in}}%
\pgfpathlineto{\pgfqpoint{5.364847in}{1.238500in}}%
\pgfpathlineto{\pgfqpoint{5.392546in}{1.238500in}}%
\pgfpathlineto{\pgfqpoint{5.420246in}{1.238500in}}%
\pgfpathlineto{\pgfqpoint{5.447945in}{1.238500in}}%
\pgfpathlineto{\pgfqpoint{5.475644in}{1.238500in}}%
\pgfpathlineto{\pgfqpoint{5.503343in}{1.238500in}}%
\pgfpathlineto{\pgfqpoint{5.531043in}{1.238500in}}%
\pgfpathlineto{\pgfqpoint{5.558742in}{1.238500in}}%
\pgfpathlineto{\pgfqpoint{5.586441in}{1.238500in}}%
\pgfpathlineto{\pgfqpoint{5.614141in}{1.238500in}}%
\pgfpathlineto{\pgfqpoint{5.641840in}{1.238500in}}%
\pgfpathlineto{\pgfqpoint{5.669539in}{1.238500in}}%
\pgfpathlineto{\pgfqpoint{5.697238in}{1.238500in}}%
\pgfpathlineto{\pgfqpoint{5.724938in}{1.238500in}}%
\pgfpathlineto{\pgfqpoint{5.752637in}{1.238500in}}%
\pgfpathlineto{\pgfqpoint{5.780336in}{1.238500in}}%
\pgfpathlineto{\pgfqpoint{5.808036in}{1.238500in}}%
\pgfpathlineto{\pgfqpoint{5.835735in}{1.238500in}}%
\pgfpathlineto{\pgfqpoint{5.863434in}{1.238500in}}%
\pgfpathlineto{\pgfqpoint{5.891133in}{1.238500in}}%
\pgfpathlineto{\pgfqpoint{5.918833in}{1.238500in}}%
\pgfpathlineto{\pgfqpoint{5.946532in}{1.238500in}}%
\pgfpathlineto{\pgfqpoint{5.974231in}{1.238500in}}%
\pgfpathlineto{\pgfqpoint{6.001931in}{1.238500in}}%
\pgfpathlineto{\pgfqpoint{6.029630in}{1.238500in}}%
\pgfpathlineto{\pgfqpoint{6.057329in}{1.238500in}}%
\pgfpathlineto{\pgfqpoint{6.067463in}{1.238500in}}%
\pgfusepath{stroke}%
\end{pgfscope}%
\begin{pgfscope}%
\pgfpathrectangle{\pgfqpoint{4.701213in}{0.383578in}}{\pgfqpoint{1.356250in}{1.540000in}}%
\pgfusepath{clip}%
\pgfsetrectcap%
\pgfsetroundjoin%
\pgfsetlinewidth{0.803000pt}%
\definecolor{currentstroke}{rgb}{0.333333,0.333333,0.333333}%
\pgfsetstrokecolor{currentstroke}%
\pgfsetstrokeopacity{0.300000}%
\pgfsetdash{}{0pt}%
\pgfpathmoveto{\pgfqpoint{4.755463in}{1.238500in}}%
\pgfpathlineto{\pgfqpoint{4.783162in}{1.238500in}}%
\pgfpathlineto{\pgfqpoint{4.810861in}{1.238500in}}%
\pgfpathlineto{\pgfqpoint{4.838561in}{1.238500in}}%
\pgfpathlineto{\pgfqpoint{4.866260in}{1.238500in}}%
\pgfpathlineto{\pgfqpoint{4.893959in}{1.238500in}}%
\pgfpathlineto{\pgfqpoint{4.921658in}{1.238500in}}%
\pgfpathlineto{\pgfqpoint{4.949358in}{1.238500in}}%
\pgfpathlineto{\pgfqpoint{4.977057in}{1.238500in}}%
\pgfpathlineto{\pgfqpoint{5.004756in}{1.238500in}}%
\pgfpathlineto{\pgfqpoint{5.032456in}{1.238500in}}%
\pgfpathlineto{\pgfqpoint{5.060155in}{1.238500in}}%
\pgfpathlineto{\pgfqpoint{5.087854in}{1.238500in}}%
\pgfpathlineto{\pgfqpoint{5.115553in}{1.238500in}}%
\pgfpathlineto{\pgfqpoint{5.143253in}{1.238500in}}%
\pgfpathlineto{\pgfqpoint{5.170952in}{1.238500in}}%
\pgfpathlineto{\pgfqpoint{5.198651in}{1.238500in}}%
\pgfpathlineto{\pgfqpoint{5.226351in}{1.238500in}}%
\pgfpathlineto{\pgfqpoint{5.254050in}{1.238500in}}%
\pgfpathlineto{\pgfqpoint{5.281749in}{1.238500in}}%
\pgfpathlineto{\pgfqpoint{5.309448in}{1.238500in}}%
\pgfpathlineto{\pgfqpoint{5.337148in}{1.238500in}}%
\pgfpathlineto{\pgfqpoint{5.364847in}{1.238500in}}%
\pgfpathlineto{\pgfqpoint{5.392546in}{1.238500in}}%
\pgfpathlineto{\pgfqpoint{5.420246in}{1.238500in}}%
\pgfpathlineto{\pgfqpoint{5.447945in}{1.238500in}}%
\pgfpathlineto{\pgfqpoint{5.475644in}{1.238500in}}%
\pgfpathlineto{\pgfqpoint{5.503343in}{1.238500in}}%
\pgfpathlineto{\pgfqpoint{5.531043in}{1.238500in}}%
\pgfpathlineto{\pgfqpoint{5.558742in}{1.238500in}}%
\pgfpathlineto{\pgfqpoint{5.586441in}{1.238500in}}%
\pgfpathlineto{\pgfqpoint{5.614141in}{1.238500in}}%
\pgfpathlineto{\pgfqpoint{5.641840in}{1.238500in}}%
\pgfpathlineto{\pgfqpoint{5.669539in}{1.238500in}}%
\pgfpathlineto{\pgfqpoint{5.697238in}{1.238500in}}%
\pgfpathlineto{\pgfqpoint{5.724938in}{1.238500in}}%
\pgfpathlineto{\pgfqpoint{5.752637in}{1.238500in}}%
\pgfpathlineto{\pgfqpoint{5.780336in}{1.238500in}}%
\pgfpathlineto{\pgfqpoint{5.808036in}{1.238500in}}%
\pgfpathlineto{\pgfqpoint{5.835735in}{1.238500in}}%
\pgfpathlineto{\pgfqpoint{5.863434in}{1.238500in}}%
\pgfpathlineto{\pgfqpoint{5.891133in}{1.238500in}}%
\pgfpathlineto{\pgfqpoint{5.918833in}{1.238500in}}%
\pgfpathlineto{\pgfqpoint{5.946532in}{1.238500in}}%
\pgfpathlineto{\pgfqpoint{5.974231in}{1.238500in}}%
\pgfpathlineto{\pgfqpoint{6.001931in}{1.238500in}}%
\pgfpathlineto{\pgfqpoint{6.029630in}{1.238500in}}%
\pgfpathlineto{\pgfqpoint{6.057329in}{1.238500in}}%
\pgfpathlineto{\pgfqpoint{6.067463in}{1.238500in}}%
\pgfusepath{stroke}%
\end{pgfscope}%
\begin{pgfscope}%
\pgfpathrectangle{\pgfqpoint{4.701213in}{0.383578in}}{\pgfqpoint{1.356250in}{1.540000in}}%
\pgfusepath{clip}%
\pgfsetrectcap%
\pgfsetroundjoin%
\pgfsetlinewidth{0.803000pt}%
\definecolor{currentstroke}{rgb}{0.333333,0.333333,0.333333}%
\pgfsetstrokecolor{currentstroke}%
\pgfsetstrokeopacity{0.300000}%
\pgfsetdash{}{0pt}%
\pgfpathmoveto{\pgfqpoint{4.755463in}{1.238500in}}%
\pgfpathlineto{\pgfqpoint{4.783162in}{1.238500in}}%
\pgfpathlineto{\pgfqpoint{4.810861in}{1.238500in}}%
\pgfpathlineto{\pgfqpoint{4.838561in}{1.238500in}}%
\pgfpathlineto{\pgfqpoint{4.866260in}{1.238500in}}%
\pgfpathlineto{\pgfqpoint{4.893959in}{1.238500in}}%
\pgfpathlineto{\pgfqpoint{4.921658in}{1.238500in}}%
\pgfpathlineto{\pgfqpoint{4.949358in}{1.238500in}}%
\pgfpathlineto{\pgfqpoint{4.977057in}{1.238500in}}%
\pgfpathlineto{\pgfqpoint{5.004756in}{1.238500in}}%
\pgfpathlineto{\pgfqpoint{5.032456in}{1.238500in}}%
\pgfpathlineto{\pgfqpoint{5.060155in}{1.238500in}}%
\pgfpathlineto{\pgfqpoint{5.087854in}{1.238500in}}%
\pgfpathlineto{\pgfqpoint{5.115553in}{1.238500in}}%
\pgfpathlineto{\pgfqpoint{5.143253in}{1.238500in}}%
\pgfpathlineto{\pgfqpoint{5.170952in}{1.238500in}}%
\pgfpathlineto{\pgfqpoint{5.198651in}{1.238500in}}%
\pgfpathlineto{\pgfqpoint{5.226351in}{1.238500in}}%
\pgfpathlineto{\pgfqpoint{5.254050in}{1.238500in}}%
\pgfpathlineto{\pgfqpoint{5.281749in}{1.238500in}}%
\pgfpathlineto{\pgfqpoint{5.309448in}{1.238500in}}%
\pgfpathlineto{\pgfqpoint{5.337148in}{1.238500in}}%
\pgfpathlineto{\pgfqpoint{5.364847in}{1.238500in}}%
\pgfpathlineto{\pgfqpoint{5.392546in}{1.238500in}}%
\pgfpathlineto{\pgfqpoint{5.420246in}{1.238500in}}%
\pgfpathlineto{\pgfqpoint{5.447945in}{1.238500in}}%
\pgfpathlineto{\pgfqpoint{5.475644in}{1.238500in}}%
\pgfpathlineto{\pgfqpoint{5.503343in}{1.238500in}}%
\pgfpathlineto{\pgfqpoint{5.531043in}{1.238500in}}%
\pgfpathlineto{\pgfqpoint{5.558742in}{1.238500in}}%
\pgfpathlineto{\pgfqpoint{5.586441in}{1.238500in}}%
\pgfpathlineto{\pgfqpoint{5.614141in}{1.238500in}}%
\pgfpathlineto{\pgfqpoint{5.641840in}{1.238500in}}%
\pgfpathlineto{\pgfqpoint{5.669539in}{1.238500in}}%
\pgfpathlineto{\pgfqpoint{5.697238in}{1.238500in}}%
\pgfpathlineto{\pgfqpoint{5.724938in}{1.238500in}}%
\pgfpathlineto{\pgfqpoint{5.752637in}{1.238500in}}%
\pgfpathlineto{\pgfqpoint{5.780336in}{1.238500in}}%
\pgfpathlineto{\pgfqpoint{5.808036in}{1.238500in}}%
\pgfpathlineto{\pgfqpoint{5.835735in}{1.238500in}}%
\pgfpathlineto{\pgfqpoint{5.863434in}{1.238500in}}%
\pgfpathlineto{\pgfqpoint{5.891133in}{1.238500in}}%
\pgfpathlineto{\pgfqpoint{5.918833in}{1.238500in}}%
\pgfpathlineto{\pgfqpoint{5.946532in}{1.238500in}}%
\pgfpathlineto{\pgfqpoint{5.974231in}{1.238500in}}%
\pgfpathlineto{\pgfqpoint{6.001931in}{1.238500in}}%
\pgfpathlineto{\pgfqpoint{6.029630in}{1.238500in}}%
\pgfpathlineto{\pgfqpoint{6.057329in}{1.238500in}}%
\pgfpathlineto{\pgfqpoint{6.067463in}{1.238500in}}%
\pgfusepath{stroke}%
\end{pgfscope}%
\begin{pgfscope}%
\pgfpathrectangle{\pgfqpoint{4.701213in}{0.383578in}}{\pgfqpoint{1.356250in}{1.540000in}}%
\pgfusepath{clip}%
\pgfsetrectcap%
\pgfsetroundjoin%
\pgfsetlinewidth{0.803000pt}%
\definecolor{currentstroke}{rgb}{0.333333,0.333333,0.333333}%
\pgfsetstrokecolor{currentstroke}%
\pgfsetstrokeopacity{0.300000}%
\pgfsetdash{}{0pt}%
\pgfpathmoveto{\pgfqpoint{4.755463in}{1.238500in}}%
\pgfpathlineto{\pgfqpoint{4.783162in}{1.238500in}}%
\pgfpathlineto{\pgfqpoint{4.810861in}{1.238500in}}%
\pgfpathlineto{\pgfqpoint{4.838561in}{1.238500in}}%
\pgfpathlineto{\pgfqpoint{4.866260in}{1.238500in}}%
\pgfpathlineto{\pgfqpoint{4.893959in}{1.238500in}}%
\pgfpathlineto{\pgfqpoint{4.921658in}{1.238500in}}%
\pgfpathlineto{\pgfqpoint{4.949358in}{1.238500in}}%
\pgfpathlineto{\pgfqpoint{4.977057in}{1.238500in}}%
\pgfpathlineto{\pgfqpoint{5.004756in}{1.206684in}}%
\pgfpathlineto{\pgfqpoint{5.032456in}{1.138642in}}%
\pgfpathlineto{\pgfqpoint{5.060155in}{1.041457in}}%
\pgfpathlineto{\pgfqpoint{5.087854in}{0.924718in}}%
\pgfpathlineto{\pgfqpoint{5.115553in}{0.825822in}}%
\pgfpathlineto{\pgfqpoint{5.143253in}{0.764808in}}%
\pgfpathlineto{\pgfqpoint{5.170952in}{0.743665in}}%
\pgfpathlineto{\pgfqpoint{5.198651in}{0.757154in}}%
\pgfpathlineto{\pgfqpoint{5.226351in}{0.792555in}}%
\pgfpathlineto{\pgfqpoint{5.254050in}{0.838888in}}%
\pgfpathlineto{\pgfqpoint{5.281749in}{0.887650in}}%
\pgfpathlineto{\pgfqpoint{5.309448in}{0.932364in}}%
\pgfpathlineto{\pgfqpoint{5.337148in}{0.970955in}}%
\pgfpathlineto{\pgfqpoint{5.364847in}{1.002843in}}%
\pgfpathlineto{\pgfqpoint{5.392546in}{1.029154in}}%
\pgfpathlineto{\pgfqpoint{5.420246in}{1.052305in}}%
\pgfpathlineto{\pgfqpoint{5.447945in}{1.073016in}}%
\pgfpathlineto{\pgfqpoint{5.475644in}{1.092176in}}%
\pgfpathlineto{\pgfqpoint{5.503343in}{1.110103in}}%
\pgfpathlineto{\pgfqpoint{5.531043in}{1.126832in}}%
\pgfpathlineto{\pgfqpoint{5.558742in}{1.142473in}}%
\pgfpathlineto{\pgfqpoint{5.586441in}{1.156915in}}%
\pgfpathlineto{\pgfqpoint{5.614141in}{1.169807in}}%
\pgfpathlineto{\pgfqpoint{5.641840in}{1.181147in}}%
\pgfpathlineto{\pgfqpoint{5.669539in}{1.191050in}}%
\pgfpathlineto{\pgfqpoint{5.697238in}{1.199561in}}%
\pgfpathlineto{\pgfqpoint{5.724938in}{1.206809in}}%
\pgfpathlineto{\pgfqpoint{5.752637in}{1.212964in}}%
\pgfpathlineto{\pgfqpoint{5.780336in}{1.218093in}}%
\pgfpathlineto{\pgfqpoint{5.808036in}{1.222316in}}%
\pgfpathlineto{\pgfqpoint{5.835735in}{1.225756in}}%
\pgfpathlineto{\pgfqpoint{5.863434in}{1.228537in}}%
\pgfpathlineto{\pgfqpoint{5.891133in}{1.230762in}}%
\pgfpathlineto{\pgfqpoint{5.918833in}{1.232522in}}%
\pgfpathlineto{\pgfqpoint{5.946532in}{1.233904in}}%
\pgfpathlineto{\pgfqpoint{5.974231in}{1.234983in}}%
\pgfpathlineto{\pgfqpoint{6.001931in}{1.235822in}}%
\pgfpathlineto{\pgfqpoint{6.029630in}{1.236469in}}%
\pgfpathlineto{\pgfqpoint{6.057329in}{1.236965in}}%
\pgfpathlineto{\pgfqpoint{6.067463in}{1.237104in}}%
\pgfusepath{stroke}%
\end{pgfscope}%
\begin{pgfscope}%
\pgfpathrectangle{\pgfqpoint{4.701213in}{0.383578in}}{\pgfqpoint{1.356250in}{1.540000in}}%
\pgfusepath{clip}%
\pgfsetrectcap%
\pgfsetroundjoin%
\pgfsetlinewidth{0.803000pt}%
\definecolor{currentstroke}{rgb}{0.333333,0.333333,0.333333}%
\pgfsetstrokecolor{currentstroke}%
\pgfsetstrokeopacity{0.300000}%
\pgfsetdash{}{0pt}%
\pgfpathmoveto{\pgfqpoint{4.755463in}{1.238500in}}%
\pgfpathlineto{\pgfqpoint{4.783162in}{1.238500in}}%
\pgfpathlineto{\pgfqpoint{4.810861in}{1.238500in}}%
\pgfpathlineto{\pgfqpoint{4.838561in}{1.238500in}}%
\pgfpathlineto{\pgfqpoint{4.866260in}{1.238500in}}%
\pgfpathlineto{\pgfqpoint{4.893959in}{1.238500in}}%
\pgfpathlineto{\pgfqpoint{4.921658in}{1.238500in}}%
\pgfpathlineto{\pgfqpoint{4.949358in}{1.238500in}}%
\pgfpathlineto{\pgfqpoint{4.977057in}{1.238500in}}%
\pgfpathlineto{\pgfqpoint{5.004756in}{1.230328in}}%
\pgfpathlineto{\pgfqpoint{5.032456in}{1.194678in}}%
\pgfpathlineto{\pgfqpoint{5.060155in}{1.125555in}}%
\pgfpathlineto{\pgfqpoint{5.087854in}{1.025862in}}%
\pgfpathlineto{\pgfqpoint{5.115553in}{0.931727in}}%
\pgfpathlineto{\pgfqpoint{5.143253in}{0.866015in}}%
\pgfpathlineto{\pgfqpoint{5.170952in}{0.834409in}}%
\pgfpathlineto{\pgfqpoint{5.198651in}{0.834959in}}%
\pgfpathlineto{\pgfqpoint{5.226351in}{0.856898in}}%
\pgfpathlineto{\pgfqpoint{5.254050in}{0.890406in}}%
\pgfpathlineto{\pgfqpoint{5.281749in}{0.927516in}}%
\pgfpathlineto{\pgfqpoint{5.309448in}{0.961811in}}%
\pgfpathlineto{\pgfqpoint{5.337148in}{0.991313in}}%
\pgfpathlineto{\pgfqpoint{5.364847in}{1.015418in}}%
\pgfpathlineto{\pgfqpoint{5.392546in}{1.035275in}}%
\pgfpathlineto{\pgfqpoint{5.420246in}{1.053428in}}%
\pgfpathlineto{\pgfqpoint{5.447945in}{1.070443in}}%
\pgfpathlineto{\pgfqpoint{5.475644in}{1.087086in}}%
\pgfpathlineto{\pgfqpoint{5.503343in}{1.103482in}}%
\pgfpathlineto{\pgfqpoint{5.531043in}{1.119453in}}%
\pgfpathlineto{\pgfqpoint{5.558742in}{1.134942in}}%
\pgfpathlineto{\pgfqpoint{5.586441in}{1.149665in}}%
\pgfpathlineto{\pgfqpoint{5.614141in}{1.163078in}}%
\pgfpathlineto{\pgfqpoint{5.641840in}{1.175072in}}%
\pgfpathlineto{\pgfqpoint{5.669539in}{1.185695in}}%
\pgfpathlineto{\pgfqpoint{5.697238in}{1.194928in}}%
\pgfpathlineto{\pgfqpoint{5.724938in}{1.202869in}}%
\pgfpathlineto{\pgfqpoint{5.752637in}{1.209670in}}%
\pgfpathlineto{\pgfqpoint{5.780336in}{1.215376in}}%
\pgfpathlineto{\pgfqpoint{5.808036in}{1.220102in}}%
\pgfpathlineto{\pgfqpoint{5.835735in}{1.223971in}}%
\pgfpathlineto{\pgfqpoint{5.863434in}{1.227113in}}%
\pgfpathlineto{\pgfqpoint{5.891133in}{1.229635in}}%
\pgfpathlineto{\pgfqpoint{5.918833in}{1.231638in}}%
\pgfpathlineto{\pgfqpoint{5.946532in}{1.233214in}}%
\pgfpathlineto{\pgfqpoint{5.974231in}{1.234448in}}%
\pgfpathlineto{\pgfqpoint{6.001931in}{1.235409in}}%
\pgfpathlineto{\pgfqpoint{6.029630in}{1.236152in}}%
\pgfpathlineto{\pgfqpoint{6.057329in}{1.236724in}}%
\pgfpathlineto{\pgfqpoint{6.067463in}{1.236884in}}%
\pgfusepath{stroke}%
\end{pgfscope}%
\begin{pgfscope}%
\pgfpathrectangle{\pgfqpoint{4.701213in}{0.383578in}}{\pgfqpoint{1.356250in}{1.540000in}}%
\pgfusepath{clip}%
\pgfsetrectcap%
\pgfsetroundjoin%
\pgfsetlinewidth{0.803000pt}%
\definecolor{currentstroke}{rgb}{0.333333,0.333333,0.333333}%
\pgfsetstrokecolor{currentstroke}%
\pgfsetstrokeopacity{0.300000}%
\pgfsetdash{}{0pt}%
\pgfpathmoveto{\pgfqpoint{4.755463in}{1.238500in}}%
\pgfpathlineto{\pgfqpoint{4.783162in}{1.238500in}}%
\pgfpathlineto{\pgfqpoint{4.810861in}{1.238500in}}%
\pgfpathlineto{\pgfqpoint{4.838561in}{1.238500in}}%
\pgfpathlineto{\pgfqpoint{4.866260in}{1.238500in}}%
\pgfpathlineto{\pgfqpoint{4.893959in}{1.238500in}}%
\pgfpathlineto{\pgfqpoint{4.921658in}{1.238500in}}%
\pgfpathlineto{\pgfqpoint{4.949358in}{1.238500in}}%
\pgfpathlineto{\pgfqpoint{4.977057in}{1.238500in}}%
\pgfpathlineto{\pgfqpoint{5.004756in}{1.062307in}}%
\pgfpathlineto{\pgfqpoint{5.032456in}{0.876177in}}%
\pgfpathlineto{\pgfqpoint{5.060155in}{0.708054in}}%
\pgfpathlineto{\pgfqpoint{5.087854in}{0.567777in}}%
\pgfpathlineto{\pgfqpoint{5.115553in}{0.480996in}}%
\pgfpathlineto{\pgfqpoint{5.143253in}{0.453578in}}%
\pgfpathlineto{\pgfqpoint{5.170952in}{0.475987in}}%
\pgfpathlineto{\pgfqpoint{5.198651in}{0.534749in}}%
\pgfpathlineto{\pgfqpoint{5.226351in}{0.612901in}}%
\pgfpathlineto{\pgfqpoint{5.254050in}{0.697374in}}%
\pgfpathlineto{\pgfqpoint{5.281749in}{0.779025in}}%
\pgfpathlineto{\pgfqpoint{5.309448in}{0.851723in}}%
\pgfpathlineto{\pgfqpoint{5.337148in}{0.913678in}}%
\pgfpathlineto{\pgfqpoint{5.364847in}{0.964793in}}%
\pgfpathlineto{\pgfqpoint{5.392546in}{1.006545in}}%
\pgfpathlineto{\pgfqpoint{5.420246in}{1.041480in}}%
\pgfpathlineto{\pgfqpoint{5.447945in}{1.070863in}}%
\pgfpathlineto{\pgfqpoint{5.475644in}{1.096027in}}%
\pgfpathlineto{\pgfqpoint{5.503343in}{1.117799in}}%
\pgfpathlineto{\pgfqpoint{5.531043in}{1.136708in}}%
\pgfpathlineto{\pgfqpoint{5.558742in}{1.153251in}}%
\pgfpathlineto{\pgfqpoint{5.586441in}{1.167690in}}%
\pgfpathlineto{\pgfqpoint{5.614141in}{1.180044in}}%
\pgfpathlineto{\pgfqpoint{5.641840in}{1.190530in}}%
\pgfpathlineto{\pgfqpoint{5.669539in}{1.199406in}}%
\pgfpathlineto{\pgfqpoint{5.697238in}{1.206839in}}%
\pgfpathlineto{\pgfqpoint{5.724938in}{1.213030in}}%
\pgfpathlineto{\pgfqpoint{5.752637in}{1.218182in}}%
\pgfpathlineto{\pgfqpoint{5.780336in}{1.222404in}}%
\pgfpathlineto{\pgfqpoint{5.808036in}{1.225834in}}%
\pgfpathlineto{\pgfqpoint{5.835735in}{1.228595in}}%
\pgfpathlineto{\pgfqpoint{5.863434in}{1.230804in}}%
\pgfpathlineto{\pgfqpoint{5.891133in}{1.232555in}}%
\pgfpathlineto{\pgfqpoint{5.918833in}{1.233931in}}%
\pgfpathlineto{\pgfqpoint{5.946532in}{1.235003in}}%
\pgfpathlineto{\pgfqpoint{5.974231in}{1.235835in}}%
\pgfpathlineto{\pgfqpoint{6.001931in}{1.236478in}}%
\pgfpathlineto{\pgfqpoint{6.029630in}{1.236972in}}%
\pgfpathlineto{\pgfqpoint{6.057329in}{1.237349in}}%
\pgfpathlineto{\pgfqpoint{6.067463in}{1.237454in}}%
\pgfusepath{stroke}%
\end{pgfscope}%
\begin{pgfscope}%
\pgfpathrectangle{\pgfqpoint{4.701213in}{0.383578in}}{\pgfqpoint{1.356250in}{1.540000in}}%
\pgfusepath{clip}%
\pgfsetrectcap%
\pgfsetroundjoin%
\pgfsetlinewidth{0.803000pt}%
\definecolor{currentstroke}{rgb}{0.333333,0.333333,0.333333}%
\pgfsetstrokecolor{currentstroke}%
\pgfsetstrokeopacity{0.300000}%
\pgfsetdash{}{0pt}%
\pgfpathmoveto{\pgfqpoint{4.755463in}{1.238500in}}%
\pgfpathlineto{\pgfqpoint{4.783162in}{1.238500in}}%
\pgfpathlineto{\pgfqpoint{4.810861in}{1.238500in}}%
\pgfpathlineto{\pgfqpoint{4.838561in}{1.238500in}}%
\pgfpathlineto{\pgfqpoint{4.866260in}{1.238500in}}%
\pgfpathlineto{\pgfqpoint{4.893959in}{1.238500in}}%
\pgfpathlineto{\pgfqpoint{4.921658in}{1.238500in}}%
\pgfpathlineto{\pgfqpoint{4.949358in}{1.238500in}}%
\pgfpathlineto{\pgfqpoint{4.977057in}{1.238500in}}%
\pgfpathlineto{\pgfqpoint{5.004756in}{1.238500in}}%
\pgfpathlineto{\pgfqpoint{5.032456in}{1.238500in}}%
\pgfpathlineto{\pgfqpoint{5.060155in}{1.238500in}}%
\pgfpathlineto{\pgfqpoint{5.087854in}{1.238500in}}%
\pgfpathlineto{\pgfqpoint{5.115553in}{1.238500in}}%
\pgfpathlineto{\pgfqpoint{5.143253in}{1.238500in}}%
\pgfpathlineto{\pgfqpoint{5.170952in}{1.238500in}}%
\pgfpathlineto{\pgfqpoint{5.198651in}{1.238500in}}%
\pgfpathlineto{\pgfqpoint{5.226351in}{1.238500in}}%
\pgfpathlineto{\pgfqpoint{5.254050in}{1.238500in}}%
\pgfpathlineto{\pgfqpoint{5.281749in}{1.238500in}}%
\pgfpathlineto{\pgfqpoint{5.309448in}{1.238500in}}%
\pgfpathlineto{\pgfqpoint{5.337148in}{1.238500in}}%
\pgfpathlineto{\pgfqpoint{5.364847in}{1.238500in}}%
\pgfpathlineto{\pgfqpoint{5.392546in}{1.238500in}}%
\pgfpathlineto{\pgfqpoint{5.420246in}{1.238500in}}%
\pgfpathlineto{\pgfqpoint{5.447945in}{1.238500in}}%
\pgfpathlineto{\pgfqpoint{5.475644in}{1.238500in}}%
\pgfpathlineto{\pgfqpoint{5.503343in}{1.238500in}}%
\pgfpathlineto{\pgfqpoint{5.531043in}{1.238500in}}%
\pgfpathlineto{\pgfqpoint{5.558742in}{1.238500in}}%
\pgfpathlineto{\pgfqpoint{5.586441in}{1.238500in}}%
\pgfpathlineto{\pgfqpoint{5.614141in}{1.238500in}}%
\pgfpathlineto{\pgfqpoint{5.641840in}{1.238500in}}%
\pgfpathlineto{\pgfqpoint{5.669539in}{1.238500in}}%
\pgfpathlineto{\pgfqpoint{5.697238in}{1.238500in}}%
\pgfpathlineto{\pgfqpoint{5.724938in}{1.238500in}}%
\pgfpathlineto{\pgfqpoint{5.752637in}{1.238500in}}%
\pgfpathlineto{\pgfqpoint{5.780336in}{1.238500in}}%
\pgfpathlineto{\pgfqpoint{5.808036in}{1.238500in}}%
\pgfpathlineto{\pgfqpoint{5.835735in}{1.238500in}}%
\pgfpathlineto{\pgfqpoint{5.863434in}{1.238500in}}%
\pgfpathlineto{\pgfqpoint{5.891133in}{1.238500in}}%
\pgfpathlineto{\pgfqpoint{5.918833in}{1.238500in}}%
\pgfpathlineto{\pgfqpoint{5.946532in}{1.238500in}}%
\pgfpathlineto{\pgfqpoint{5.974231in}{1.238500in}}%
\pgfpathlineto{\pgfqpoint{6.001931in}{1.238500in}}%
\pgfpathlineto{\pgfqpoint{6.029630in}{1.238500in}}%
\pgfpathlineto{\pgfqpoint{6.057329in}{1.238500in}}%
\pgfpathlineto{\pgfqpoint{6.067463in}{1.238500in}}%
\pgfusepath{stroke}%
\end{pgfscope}%
\begin{pgfscope}%
\pgfpathrectangle{\pgfqpoint{4.701213in}{0.383578in}}{\pgfqpoint{1.356250in}{1.540000in}}%
\pgfusepath{clip}%
\pgfsetrectcap%
\pgfsetroundjoin%
\pgfsetlinewidth{0.803000pt}%
\definecolor{currentstroke}{rgb}{0.333333,0.333333,0.333333}%
\pgfsetstrokecolor{currentstroke}%
\pgfsetstrokeopacity{0.300000}%
\pgfsetdash{}{0pt}%
\pgfpathmoveto{\pgfqpoint{4.755463in}{1.238500in}}%
\pgfpathlineto{\pgfqpoint{4.783162in}{1.238500in}}%
\pgfpathlineto{\pgfqpoint{4.810861in}{1.238500in}}%
\pgfpathlineto{\pgfqpoint{4.838561in}{1.238500in}}%
\pgfpathlineto{\pgfqpoint{4.866260in}{1.238500in}}%
\pgfpathlineto{\pgfqpoint{4.893959in}{1.238500in}}%
\pgfpathlineto{\pgfqpoint{4.921658in}{1.238500in}}%
\pgfpathlineto{\pgfqpoint{4.949358in}{1.238500in}}%
\pgfpathlineto{\pgfqpoint{4.977057in}{1.238500in}}%
\pgfpathlineto{\pgfqpoint{5.004756in}{1.156161in}}%
\pgfpathlineto{\pgfqpoint{5.032456in}{1.040510in}}%
\pgfpathlineto{\pgfqpoint{5.060155in}{0.911307in}}%
\pgfpathlineto{\pgfqpoint{5.087854in}{0.781158in}}%
\pgfpathlineto{\pgfqpoint{5.115553in}{0.684254in}}%
\pgfpathlineto{\pgfqpoint{5.143253in}{0.635184in}}%
\pgfpathlineto{\pgfqpoint{5.170952in}{0.631046in}}%
\pgfpathlineto{\pgfqpoint{5.198651in}{0.662892in}}%
\pgfpathlineto{\pgfqpoint{5.226351in}{0.716029in}}%
\pgfpathlineto{\pgfqpoint{5.254050in}{0.778445in}}%
\pgfpathlineto{\pgfqpoint{5.281749in}{0.841269in}}%
\pgfpathlineto{\pgfqpoint{5.309448in}{0.898126in}}%
\pgfpathlineto{\pgfqpoint{5.337148in}{0.946992in}}%
\pgfpathlineto{\pgfqpoint{5.364847in}{0.987444in}}%
\pgfpathlineto{\pgfqpoint{5.392546in}{1.020702in}}%
\pgfpathlineto{\pgfqpoint{5.420246in}{1.049171in}}%
\pgfpathlineto{\pgfqpoint{5.447945in}{1.073787in}}%
\pgfpathlineto{\pgfqpoint{5.475644in}{1.095621in}}%
\pgfpathlineto{\pgfqpoint{5.503343in}{1.115214in}}%
\pgfpathlineto{\pgfqpoint{5.531043in}{1.132826in}}%
\pgfpathlineto{\pgfqpoint{5.558742in}{1.148746in}}%
\pgfpathlineto{\pgfqpoint{5.586441in}{1.163042in}}%
\pgfpathlineto{\pgfqpoint{5.614141in}{1.175543in}}%
\pgfpathlineto{\pgfqpoint{5.641840in}{1.186353in}}%
\pgfpathlineto{\pgfqpoint{5.669539in}{1.195655in}}%
\pgfpathlineto{\pgfqpoint{5.697238in}{1.203552in}}%
\pgfpathlineto{\pgfqpoint{5.724938in}{1.210209in}}%
\pgfpathlineto{\pgfqpoint{5.752637in}{1.215808in}}%
\pgfpathlineto{\pgfqpoint{5.780336in}{1.220438in}}%
\pgfpathlineto{\pgfqpoint{5.808036in}{1.224226in}}%
\pgfpathlineto{\pgfqpoint{5.835735in}{1.227296in}}%
\pgfpathlineto{\pgfqpoint{5.863434in}{1.229765in}}%
\pgfpathlineto{\pgfqpoint{5.891133in}{1.231733in}}%
\pgfpathlineto{\pgfqpoint{5.918833in}{1.233285in}}%
\pgfpathlineto{\pgfqpoint{5.946532in}{1.234498in}}%
\pgfpathlineto{\pgfqpoint{5.974231in}{1.235444in}}%
\pgfpathlineto{\pgfqpoint{6.001931in}{1.236177in}}%
\pgfpathlineto{\pgfqpoint{6.029630in}{1.236741in}}%
\pgfpathlineto{\pgfqpoint{6.057329in}{1.237173in}}%
\pgfpathlineto{\pgfqpoint{6.067463in}{1.237294in}}%
\pgfusepath{stroke}%
\end{pgfscope}%
\begin{pgfscope}%
\pgfpathrectangle{\pgfqpoint{4.701213in}{0.383578in}}{\pgfqpoint{1.356250in}{1.540000in}}%
\pgfusepath{clip}%
\pgfsetrectcap%
\pgfsetroundjoin%
\pgfsetlinewidth{0.803000pt}%
\definecolor{currentstroke}{rgb}{0.333333,0.333333,0.333333}%
\pgfsetstrokecolor{currentstroke}%
\pgfsetstrokeopacity{0.300000}%
\pgfsetdash{}{0pt}%
\pgfpathmoveto{\pgfqpoint{4.755463in}{1.238500in}}%
\pgfpathlineto{\pgfqpoint{4.783162in}{1.238500in}}%
\pgfpathlineto{\pgfqpoint{4.810861in}{1.238500in}}%
\pgfpathlineto{\pgfqpoint{4.838561in}{1.238500in}}%
\pgfpathlineto{\pgfqpoint{4.866260in}{1.238500in}}%
\pgfpathlineto{\pgfqpoint{4.893959in}{1.238500in}}%
\pgfpathlineto{\pgfqpoint{4.921658in}{1.238500in}}%
\pgfpathlineto{\pgfqpoint{4.949358in}{1.238500in}}%
\pgfpathlineto{\pgfqpoint{4.977057in}{1.238500in}}%
\pgfpathlineto{\pgfqpoint{5.004756in}{1.238500in}}%
\pgfpathlineto{\pgfqpoint{5.032456in}{1.238500in}}%
\pgfpathlineto{\pgfqpoint{5.060155in}{1.238500in}}%
\pgfpathlineto{\pgfqpoint{5.087854in}{1.238500in}}%
\pgfpathlineto{\pgfqpoint{5.115553in}{1.238500in}}%
\pgfpathlineto{\pgfqpoint{5.143253in}{1.238500in}}%
\pgfpathlineto{\pgfqpoint{5.170952in}{1.238500in}}%
\pgfpathlineto{\pgfqpoint{5.198651in}{1.238500in}}%
\pgfpathlineto{\pgfqpoint{5.226351in}{1.238500in}}%
\pgfpathlineto{\pgfqpoint{5.254050in}{1.238500in}}%
\pgfpathlineto{\pgfqpoint{5.281749in}{1.238500in}}%
\pgfpathlineto{\pgfqpoint{5.309448in}{1.238500in}}%
\pgfpathlineto{\pgfqpoint{5.337148in}{1.238500in}}%
\pgfpathlineto{\pgfqpoint{5.364847in}{1.238500in}}%
\pgfpathlineto{\pgfqpoint{5.392546in}{1.238500in}}%
\pgfpathlineto{\pgfqpoint{5.420246in}{1.238500in}}%
\pgfpathlineto{\pgfqpoint{5.447945in}{1.238500in}}%
\pgfpathlineto{\pgfqpoint{5.475644in}{1.238500in}}%
\pgfpathlineto{\pgfqpoint{5.503343in}{1.238500in}}%
\pgfpathlineto{\pgfqpoint{5.531043in}{1.238500in}}%
\pgfpathlineto{\pgfqpoint{5.558742in}{1.238500in}}%
\pgfpathlineto{\pgfqpoint{5.586441in}{1.238500in}}%
\pgfpathlineto{\pgfqpoint{5.614141in}{1.238500in}}%
\pgfpathlineto{\pgfqpoint{5.641840in}{1.238500in}}%
\pgfpathlineto{\pgfqpoint{5.669539in}{1.238500in}}%
\pgfpathlineto{\pgfqpoint{5.697238in}{1.238500in}}%
\pgfpathlineto{\pgfqpoint{5.724938in}{1.238500in}}%
\pgfpathlineto{\pgfqpoint{5.752637in}{1.238500in}}%
\pgfpathlineto{\pgfqpoint{5.780336in}{1.238500in}}%
\pgfpathlineto{\pgfqpoint{5.808036in}{1.238500in}}%
\pgfpathlineto{\pgfqpoint{5.835735in}{1.238500in}}%
\pgfpathlineto{\pgfqpoint{5.863434in}{1.238500in}}%
\pgfpathlineto{\pgfqpoint{5.891133in}{1.238500in}}%
\pgfpathlineto{\pgfqpoint{5.918833in}{1.238500in}}%
\pgfpathlineto{\pgfqpoint{5.946532in}{1.238500in}}%
\pgfpathlineto{\pgfqpoint{5.974231in}{1.238500in}}%
\pgfpathlineto{\pgfqpoint{6.001931in}{1.238500in}}%
\pgfpathlineto{\pgfqpoint{6.029630in}{1.238500in}}%
\pgfpathlineto{\pgfqpoint{6.057329in}{1.238500in}}%
\pgfpathlineto{\pgfqpoint{6.067463in}{1.238500in}}%
\pgfusepath{stroke}%
\end{pgfscope}%
\begin{pgfscope}%
\pgfpathrectangle{\pgfqpoint{4.701213in}{0.383578in}}{\pgfqpoint{1.356250in}{1.540000in}}%
\pgfusepath{clip}%
\pgfsetrectcap%
\pgfsetroundjoin%
\pgfsetlinewidth{0.803000pt}%
\definecolor{currentstroke}{rgb}{0.333333,0.333333,0.333333}%
\pgfsetstrokecolor{currentstroke}%
\pgfsetstrokeopacity{0.300000}%
\pgfsetdash{}{0pt}%
\pgfpathmoveto{\pgfqpoint{4.755463in}{1.238500in}}%
\pgfpathlineto{\pgfqpoint{4.783162in}{1.238500in}}%
\pgfpathlineto{\pgfqpoint{4.810861in}{1.238500in}}%
\pgfpathlineto{\pgfqpoint{4.838561in}{1.238500in}}%
\pgfpathlineto{\pgfqpoint{4.866260in}{1.238500in}}%
\pgfpathlineto{\pgfqpoint{4.893959in}{1.238500in}}%
\pgfpathlineto{\pgfqpoint{4.921658in}{1.238500in}}%
\pgfpathlineto{\pgfqpoint{4.949358in}{1.238500in}}%
\pgfpathlineto{\pgfqpoint{4.977057in}{1.238500in}}%
\pgfpathlineto{\pgfqpoint{5.004756in}{1.238500in}}%
\pgfpathlineto{\pgfqpoint{5.032456in}{1.238500in}}%
\pgfpathlineto{\pgfqpoint{5.060155in}{1.238500in}}%
\pgfpathlineto{\pgfqpoint{5.087854in}{1.238500in}}%
\pgfpathlineto{\pgfqpoint{5.115553in}{1.238500in}}%
\pgfpathlineto{\pgfqpoint{5.143253in}{1.238500in}}%
\pgfpathlineto{\pgfqpoint{5.170952in}{1.238500in}}%
\pgfpathlineto{\pgfqpoint{5.198651in}{1.238500in}}%
\pgfpathlineto{\pgfqpoint{5.226351in}{1.238500in}}%
\pgfpathlineto{\pgfqpoint{5.254050in}{1.238500in}}%
\pgfpathlineto{\pgfqpoint{5.281749in}{1.238500in}}%
\pgfpathlineto{\pgfqpoint{5.309448in}{1.238500in}}%
\pgfpathlineto{\pgfqpoint{5.337148in}{1.238500in}}%
\pgfpathlineto{\pgfqpoint{5.364847in}{1.238500in}}%
\pgfpathlineto{\pgfqpoint{5.392546in}{1.238500in}}%
\pgfpathlineto{\pgfqpoint{5.420246in}{1.238500in}}%
\pgfpathlineto{\pgfqpoint{5.447945in}{1.238500in}}%
\pgfpathlineto{\pgfqpoint{5.475644in}{1.238500in}}%
\pgfpathlineto{\pgfqpoint{5.503343in}{1.238500in}}%
\pgfpathlineto{\pgfqpoint{5.531043in}{1.238500in}}%
\pgfpathlineto{\pgfqpoint{5.558742in}{1.238500in}}%
\pgfpathlineto{\pgfqpoint{5.586441in}{1.238500in}}%
\pgfpathlineto{\pgfqpoint{5.614141in}{1.238500in}}%
\pgfpathlineto{\pgfqpoint{5.641840in}{1.238500in}}%
\pgfpathlineto{\pgfqpoint{5.669539in}{1.238500in}}%
\pgfpathlineto{\pgfqpoint{5.697238in}{1.238500in}}%
\pgfpathlineto{\pgfqpoint{5.724938in}{1.238500in}}%
\pgfpathlineto{\pgfqpoint{5.752637in}{1.238500in}}%
\pgfpathlineto{\pgfqpoint{5.780336in}{1.238500in}}%
\pgfpathlineto{\pgfqpoint{5.808036in}{1.238500in}}%
\pgfpathlineto{\pgfqpoint{5.835735in}{1.238500in}}%
\pgfpathlineto{\pgfqpoint{5.863434in}{1.238500in}}%
\pgfpathlineto{\pgfqpoint{5.891133in}{1.238500in}}%
\pgfpathlineto{\pgfqpoint{5.918833in}{1.238500in}}%
\pgfpathlineto{\pgfqpoint{5.946532in}{1.238500in}}%
\pgfpathlineto{\pgfqpoint{5.974231in}{1.238500in}}%
\pgfpathlineto{\pgfqpoint{6.001931in}{1.238500in}}%
\pgfpathlineto{\pgfqpoint{6.029630in}{1.238500in}}%
\pgfpathlineto{\pgfqpoint{6.057329in}{1.238500in}}%
\pgfpathlineto{\pgfqpoint{6.067463in}{1.238500in}}%
\pgfusepath{stroke}%
\end{pgfscope}%
\begin{pgfscope}%
\pgfpathrectangle{\pgfqpoint{4.701213in}{0.383578in}}{\pgfqpoint{1.356250in}{1.540000in}}%
\pgfusepath{clip}%
\pgfsetrectcap%
\pgfsetroundjoin%
\pgfsetlinewidth{0.803000pt}%
\definecolor{currentstroke}{rgb}{0.333333,0.333333,0.333333}%
\pgfsetstrokecolor{currentstroke}%
\pgfsetstrokeopacity{0.300000}%
\pgfsetdash{}{0pt}%
\pgfpathmoveto{\pgfqpoint{4.755463in}{1.238500in}}%
\pgfpathlineto{\pgfqpoint{4.783162in}{1.238500in}}%
\pgfpathlineto{\pgfqpoint{4.810861in}{1.238500in}}%
\pgfpathlineto{\pgfqpoint{4.838561in}{1.238500in}}%
\pgfpathlineto{\pgfqpoint{4.866260in}{1.238500in}}%
\pgfpathlineto{\pgfqpoint{4.893959in}{1.238500in}}%
\pgfpathlineto{\pgfqpoint{4.921658in}{1.238500in}}%
\pgfpathlineto{\pgfqpoint{4.949358in}{1.238500in}}%
\pgfpathlineto{\pgfqpoint{4.977057in}{1.238500in}}%
\pgfpathlineto{\pgfqpoint{5.004756in}{1.238500in}}%
\pgfpathlineto{\pgfqpoint{5.032456in}{1.238500in}}%
\pgfpathlineto{\pgfqpoint{5.060155in}{1.238500in}}%
\pgfpathlineto{\pgfqpoint{5.087854in}{1.238500in}}%
\pgfpathlineto{\pgfqpoint{5.115553in}{1.238500in}}%
\pgfpathlineto{\pgfqpoint{5.143253in}{1.238500in}}%
\pgfpathlineto{\pgfqpoint{5.170952in}{1.238500in}}%
\pgfpathlineto{\pgfqpoint{5.198651in}{1.238500in}}%
\pgfpathlineto{\pgfqpoint{5.226351in}{1.238500in}}%
\pgfpathlineto{\pgfqpoint{5.254050in}{1.238500in}}%
\pgfpathlineto{\pgfqpoint{5.281749in}{1.238500in}}%
\pgfpathlineto{\pgfqpoint{5.309448in}{1.238500in}}%
\pgfpathlineto{\pgfqpoint{5.337148in}{1.238500in}}%
\pgfpathlineto{\pgfqpoint{5.364847in}{1.238500in}}%
\pgfpathlineto{\pgfqpoint{5.392546in}{1.238500in}}%
\pgfpathlineto{\pgfqpoint{5.420246in}{1.238500in}}%
\pgfpathlineto{\pgfqpoint{5.447945in}{1.238500in}}%
\pgfpathlineto{\pgfqpoint{5.475644in}{1.238500in}}%
\pgfpathlineto{\pgfqpoint{5.503343in}{1.238500in}}%
\pgfpathlineto{\pgfqpoint{5.531043in}{1.238500in}}%
\pgfpathlineto{\pgfqpoint{5.558742in}{1.238500in}}%
\pgfpathlineto{\pgfqpoint{5.586441in}{1.238500in}}%
\pgfpathlineto{\pgfqpoint{5.614141in}{1.238500in}}%
\pgfpathlineto{\pgfqpoint{5.641840in}{1.238500in}}%
\pgfpathlineto{\pgfqpoint{5.669539in}{1.238500in}}%
\pgfpathlineto{\pgfqpoint{5.697238in}{1.238500in}}%
\pgfpathlineto{\pgfqpoint{5.724938in}{1.238500in}}%
\pgfpathlineto{\pgfqpoint{5.752637in}{1.238500in}}%
\pgfpathlineto{\pgfqpoint{5.780336in}{1.238500in}}%
\pgfpathlineto{\pgfqpoint{5.808036in}{1.238500in}}%
\pgfpathlineto{\pgfqpoint{5.835735in}{1.238500in}}%
\pgfpathlineto{\pgfqpoint{5.863434in}{1.238500in}}%
\pgfpathlineto{\pgfqpoint{5.891133in}{1.238500in}}%
\pgfpathlineto{\pgfqpoint{5.918833in}{1.238500in}}%
\pgfpathlineto{\pgfqpoint{5.946532in}{1.238500in}}%
\pgfpathlineto{\pgfqpoint{5.974231in}{1.238500in}}%
\pgfpathlineto{\pgfqpoint{6.001931in}{1.238500in}}%
\pgfpathlineto{\pgfqpoint{6.029630in}{1.238500in}}%
\pgfpathlineto{\pgfqpoint{6.057329in}{1.238500in}}%
\pgfpathlineto{\pgfqpoint{6.067463in}{1.238500in}}%
\pgfusepath{stroke}%
\end{pgfscope}%
\begin{pgfscope}%
\pgfpathrectangle{\pgfqpoint{4.701213in}{0.383578in}}{\pgfqpoint{1.356250in}{1.540000in}}%
\pgfusepath{clip}%
\pgfsetrectcap%
\pgfsetroundjoin%
\pgfsetlinewidth{0.803000pt}%
\definecolor{currentstroke}{rgb}{0.333333,0.333333,0.333333}%
\pgfsetstrokecolor{currentstroke}%
\pgfsetstrokeopacity{0.300000}%
\pgfsetdash{}{0pt}%
\pgfpathmoveto{\pgfqpoint{4.755463in}{1.238500in}}%
\pgfpathlineto{\pgfqpoint{4.783162in}{1.238500in}}%
\pgfpathlineto{\pgfqpoint{4.810861in}{1.238500in}}%
\pgfpathlineto{\pgfqpoint{4.838561in}{1.238500in}}%
\pgfpathlineto{\pgfqpoint{4.866260in}{1.238500in}}%
\pgfpathlineto{\pgfqpoint{4.893959in}{1.238500in}}%
\pgfpathlineto{\pgfqpoint{4.921658in}{1.238500in}}%
\pgfpathlineto{\pgfqpoint{4.949358in}{1.238500in}}%
\pgfpathlineto{\pgfqpoint{4.977057in}{1.238500in}}%
\pgfpathlineto{\pgfqpoint{5.004756in}{1.206684in}}%
\pgfpathlineto{\pgfqpoint{5.032456in}{1.138642in}}%
\pgfpathlineto{\pgfqpoint{5.060155in}{1.041457in}}%
\pgfpathlineto{\pgfqpoint{5.087854in}{0.924718in}}%
\pgfpathlineto{\pgfqpoint{5.115553in}{0.825822in}}%
\pgfpathlineto{\pgfqpoint{5.143253in}{0.764808in}}%
\pgfpathlineto{\pgfqpoint{5.170952in}{0.743665in}}%
\pgfpathlineto{\pgfqpoint{5.198651in}{0.757154in}}%
\pgfpathlineto{\pgfqpoint{5.226351in}{0.792555in}}%
\pgfpathlineto{\pgfqpoint{5.254050in}{0.838888in}}%
\pgfpathlineto{\pgfqpoint{5.281749in}{0.887650in}}%
\pgfpathlineto{\pgfqpoint{5.309448in}{0.932364in}}%
\pgfpathlineto{\pgfqpoint{5.337148in}{0.970955in}}%
\pgfpathlineto{\pgfqpoint{5.364847in}{1.002843in}}%
\pgfpathlineto{\pgfqpoint{5.392546in}{1.029154in}}%
\pgfpathlineto{\pgfqpoint{5.420246in}{1.052305in}}%
\pgfpathlineto{\pgfqpoint{5.447945in}{1.073016in}}%
\pgfpathlineto{\pgfqpoint{5.475644in}{1.092176in}}%
\pgfpathlineto{\pgfqpoint{5.503343in}{1.110103in}}%
\pgfpathlineto{\pgfqpoint{5.531043in}{1.126832in}}%
\pgfpathlineto{\pgfqpoint{5.558742in}{1.142473in}}%
\pgfpathlineto{\pgfqpoint{5.586441in}{1.156915in}}%
\pgfpathlineto{\pgfqpoint{5.614141in}{1.169807in}}%
\pgfpathlineto{\pgfqpoint{5.641840in}{1.181147in}}%
\pgfpathlineto{\pgfqpoint{5.669539in}{1.191050in}}%
\pgfpathlineto{\pgfqpoint{5.697238in}{1.199561in}}%
\pgfpathlineto{\pgfqpoint{5.724938in}{1.206809in}}%
\pgfpathlineto{\pgfqpoint{5.752637in}{1.212964in}}%
\pgfpathlineto{\pgfqpoint{5.780336in}{1.218093in}}%
\pgfpathlineto{\pgfqpoint{5.808036in}{1.222316in}}%
\pgfpathlineto{\pgfqpoint{5.835735in}{1.225756in}}%
\pgfpathlineto{\pgfqpoint{5.863434in}{1.228537in}}%
\pgfpathlineto{\pgfqpoint{5.891133in}{1.230762in}}%
\pgfpathlineto{\pgfqpoint{5.918833in}{1.232522in}}%
\pgfpathlineto{\pgfqpoint{5.946532in}{1.233904in}}%
\pgfpathlineto{\pgfqpoint{5.974231in}{1.234983in}}%
\pgfpathlineto{\pgfqpoint{6.001931in}{1.235822in}}%
\pgfpathlineto{\pgfqpoint{6.029630in}{1.236469in}}%
\pgfpathlineto{\pgfqpoint{6.057329in}{1.236965in}}%
\pgfpathlineto{\pgfqpoint{6.067463in}{1.237104in}}%
\pgfusepath{stroke}%
\end{pgfscope}%
\begin{pgfscope}%
\pgfpathrectangle{\pgfqpoint{4.701213in}{0.383578in}}{\pgfqpoint{1.356250in}{1.540000in}}%
\pgfusepath{clip}%
\pgfsetrectcap%
\pgfsetroundjoin%
\pgfsetlinewidth{0.803000pt}%
\definecolor{currentstroke}{rgb}{0.333333,0.333333,0.333333}%
\pgfsetstrokecolor{currentstroke}%
\pgfsetstrokeopacity{0.300000}%
\pgfsetdash{}{0pt}%
\pgfpathmoveto{\pgfqpoint{4.755463in}{1.238500in}}%
\pgfpathlineto{\pgfqpoint{4.783162in}{1.238500in}}%
\pgfpathlineto{\pgfqpoint{4.810861in}{1.238500in}}%
\pgfpathlineto{\pgfqpoint{4.838561in}{1.238500in}}%
\pgfpathlineto{\pgfqpoint{4.866260in}{1.238500in}}%
\pgfpathlineto{\pgfqpoint{4.893959in}{1.238500in}}%
\pgfpathlineto{\pgfqpoint{4.921658in}{1.238500in}}%
\pgfpathlineto{\pgfqpoint{4.949358in}{1.238500in}}%
\pgfpathlineto{\pgfqpoint{4.977057in}{1.238500in}}%
\pgfpathlineto{\pgfqpoint{5.004756in}{1.238500in}}%
\pgfpathlineto{\pgfqpoint{5.032456in}{1.238500in}}%
\pgfpathlineto{\pgfqpoint{5.060155in}{1.238500in}}%
\pgfpathlineto{\pgfqpoint{5.087854in}{1.238500in}}%
\pgfpathlineto{\pgfqpoint{5.115553in}{1.238500in}}%
\pgfpathlineto{\pgfqpoint{5.143253in}{1.238500in}}%
\pgfpathlineto{\pgfqpoint{5.170952in}{1.238500in}}%
\pgfpathlineto{\pgfqpoint{5.198651in}{1.238500in}}%
\pgfpathlineto{\pgfqpoint{5.226351in}{1.238500in}}%
\pgfpathlineto{\pgfqpoint{5.254050in}{1.238500in}}%
\pgfpathlineto{\pgfqpoint{5.281749in}{1.238500in}}%
\pgfpathlineto{\pgfqpoint{5.309448in}{1.238500in}}%
\pgfpathlineto{\pgfqpoint{5.337148in}{1.238500in}}%
\pgfpathlineto{\pgfqpoint{5.364847in}{1.238500in}}%
\pgfpathlineto{\pgfqpoint{5.392546in}{1.238500in}}%
\pgfpathlineto{\pgfqpoint{5.420246in}{1.238500in}}%
\pgfpathlineto{\pgfqpoint{5.447945in}{1.238500in}}%
\pgfpathlineto{\pgfqpoint{5.475644in}{1.238500in}}%
\pgfpathlineto{\pgfqpoint{5.503343in}{1.238500in}}%
\pgfpathlineto{\pgfqpoint{5.531043in}{1.238500in}}%
\pgfpathlineto{\pgfqpoint{5.558742in}{1.238500in}}%
\pgfpathlineto{\pgfqpoint{5.586441in}{1.238500in}}%
\pgfpathlineto{\pgfqpoint{5.614141in}{1.238500in}}%
\pgfpathlineto{\pgfqpoint{5.641840in}{1.238500in}}%
\pgfpathlineto{\pgfqpoint{5.669539in}{1.238500in}}%
\pgfpathlineto{\pgfqpoint{5.697238in}{1.238500in}}%
\pgfpathlineto{\pgfqpoint{5.724938in}{1.238500in}}%
\pgfpathlineto{\pgfqpoint{5.752637in}{1.238500in}}%
\pgfpathlineto{\pgfqpoint{5.780336in}{1.238500in}}%
\pgfpathlineto{\pgfqpoint{5.808036in}{1.238500in}}%
\pgfpathlineto{\pgfqpoint{5.835735in}{1.238500in}}%
\pgfpathlineto{\pgfqpoint{5.863434in}{1.238500in}}%
\pgfpathlineto{\pgfqpoint{5.891133in}{1.238500in}}%
\pgfpathlineto{\pgfqpoint{5.918833in}{1.238500in}}%
\pgfpathlineto{\pgfqpoint{5.946532in}{1.238500in}}%
\pgfpathlineto{\pgfqpoint{5.974231in}{1.238500in}}%
\pgfpathlineto{\pgfqpoint{6.001931in}{1.238500in}}%
\pgfpathlineto{\pgfqpoint{6.029630in}{1.238500in}}%
\pgfpathlineto{\pgfqpoint{6.057329in}{1.238500in}}%
\pgfpathlineto{\pgfqpoint{6.067463in}{1.238500in}}%
\pgfusepath{stroke}%
\end{pgfscope}%
\begin{pgfscope}%
\pgfpathrectangle{\pgfqpoint{4.701213in}{0.383578in}}{\pgfqpoint{1.356250in}{1.540000in}}%
\pgfusepath{clip}%
\pgfsetrectcap%
\pgfsetroundjoin%
\pgfsetlinewidth{0.803000pt}%
\definecolor{currentstroke}{rgb}{0.686275,0.352941,0.313725}%
\pgfsetstrokecolor{currentstroke}%
\pgfsetstrokeopacity{0.300000}%
\pgfsetdash{}{0pt}%
\pgfpathmoveto{\pgfqpoint{4.755463in}{1.238500in}}%
\pgfpathlineto{\pgfqpoint{4.783162in}{1.238500in}}%
\pgfpathlineto{\pgfqpoint{4.810861in}{1.238500in}}%
\pgfpathlineto{\pgfqpoint{4.838561in}{1.238500in}}%
\pgfpathlineto{\pgfqpoint{4.866260in}{1.238500in}}%
\pgfpathlineto{\pgfqpoint{4.893959in}{1.238500in}}%
\pgfpathlineto{\pgfqpoint{4.921658in}{1.238500in}}%
\pgfpathlineto{\pgfqpoint{4.949358in}{1.238500in}}%
\pgfpathlineto{\pgfqpoint{4.977057in}{1.238500in}}%
\pgfpathlineto{\pgfqpoint{5.004756in}{1.238500in}}%
\pgfpathlineto{\pgfqpoint{5.032456in}{1.238500in}}%
\pgfpathlineto{\pgfqpoint{5.060155in}{1.238500in}}%
\pgfpathlineto{\pgfqpoint{5.087854in}{1.238500in}}%
\pgfpathlineto{\pgfqpoint{5.115553in}{1.238500in}}%
\pgfpathlineto{\pgfqpoint{5.143253in}{1.238500in}}%
\pgfpathlineto{\pgfqpoint{5.170952in}{1.238500in}}%
\pgfpathlineto{\pgfqpoint{5.198651in}{1.238500in}}%
\pgfpathlineto{\pgfqpoint{5.226351in}{1.238500in}}%
\pgfpathlineto{\pgfqpoint{5.254050in}{1.238500in}}%
\pgfpathlineto{\pgfqpoint{5.281749in}{1.238500in}}%
\pgfpathlineto{\pgfqpoint{5.309448in}{1.238500in}}%
\pgfpathlineto{\pgfqpoint{5.337148in}{1.238500in}}%
\pgfpathlineto{\pgfqpoint{5.364847in}{1.238500in}}%
\pgfpathlineto{\pgfqpoint{5.392546in}{1.238500in}}%
\pgfpathlineto{\pgfqpoint{5.420246in}{1.238500in}}%
\pgfpathlineto{\pgfqpoint{5.447945in}{1.238500in}}%
\pgfpathlineto{\pgfqpoint{5.475644in}{1.238500in}}%
\pgfpathlineto{\pgfqpoint{5.503343in}{1.238500in}}%
\pgfpathlineto{\pgfqpoint{5.531043in}{1.238500in}}%
\pgfpathlineto{\pgfqpoint{5.558742in}{1.238500in}}%
\pgfpathlineto{\pgfqpoint{5.586441in}{1.238500in}}%
\pgfpathlineto{\pgfqpoint{5.614141in}{1.238500in}}%
\pgfpathlineto{\pgfqpoint{5.641840in}{1.238500in}}%
\pgfpathlineto{\pgfqpoint{5.669539in}{1.238500in}}%
\pgfpathlineto{\pgfqpoint{5.697238in}{1.238500in}}%
\pgfpathlineto{\pgfqpoint{5.724938in}{1.238500in}}%
\pgfpathlineto{\pgfqpoint{5.752637in}{1.238500in}}%
\pgfpathlineto{\pgfqpoint{5.780336in}{1.238500in}}%
\pgfpathlineto{\pgfqpoint{5.808036in}{1.238500in}}%
\pgfpathlineto{\pgfqpoint{5.835735in}{1.238500in}}%
\pgfpathlineto{\pgfqpoint{5.863434in}{1.238500in}}%
\pgfpathlineto{\pgfqpoint{5.891133in}{1.238500in}}%
\pgfpathlineto{\pgfqpoint{5.918833in}{1.238500in}}%
\pgfpathlineto{\pgfqpoint{5.946532in}{1.238500in}}%
\pgfpathlineto{\pgfqpoint{5.974231in}{1.238500in}}%
\pgfpathlineto{\pgfqpoint{6.001931in}{1.238500in}}%
\pgfpathlineto{\pgfqpoint{6.029630in}{1.238500in}}%
\pgfpathlineto{\pgfqpoint{6.057329in}{1.238500in}}%
\pgfpathlineto{\pgfqpoint{6.067463in}{1.238500in}}%
\pgfusepath{stroke}%
\end{pgfscope}%
\begin{pgfscope}%
\pgfpathrectangle{\pgfqpoint{4.701213in}{0.383578in}}{\pgfqpoint{1.356250in}{1.540000in}}%
\pgfusepath{clip}%
\pgfsetrectcap%
\pgfsetroundjoin%
\pgfsetlinewidth{0.803000pt}%
\definecolor{currentstroke}{rgb}{0.686275,0.352941,0.313725}%
\pgfsetstrokecolor{currentstroke}%
\pgfsetstrokeopacity{0.300000}%
\pgfsetdash{}{0pt}%
\pgfpathmoveto{\pgfqpoint{4.755463in}{1.238500in}}%
\pgfpathlineto{\pgfqpoint{4.783162in}{1.238500in}}%
\pgfpathlineto{\pgfqpoint{4.810861in}{1.238500in}}%
\pgfpathlineto{\pgfqpoint{4.838561in}{1.238500in}}%
\pgfpathlineto{\pgfqpoint{4.866260in}{1.238500in}}%
\pgfpathlineto{\pgfqpoint{4.893959in}{1.238500in}}%
\pgfpathlineto{\pgfqpoint{4.921658in}{1.238500in}}%
\pgfpathlineto{\pgfqpoint{4.949358in}{1.238500in}}%
\pgfpathlineto{\pgfqpoint{4.977057in}{1.238500in}}%
\pgfpathlineto{\pgfqpoint{5.004756in}{1.238500in}}%
\pgfpathlineto{\pgfqpoint{5.032456in}{1.238500in}}%
\pgfpathlineto{\pgfqpoint{5.060155in}{1.238500in}}%
\pgfpathlineto{\pgfqpoint{5.087854in}{1.238500in}}%
\pgfpathlineto{\pgfqpoint{5.115553in}{1.238500in}}%
\pgfpathlineto{\pgfqpoint{5.143253in}{1.239958in}}%
\pgfpathlineto{\pgfqpoint{5.170952in}{1.241396in}}%
\pgfpathlineto{\pgfqpoint{5.198651in}{1.242612in}}%
\pgfpathlineto{\pgfqpoint{5.226351in}{1.243504in}}%
\pgfpathlineto{\pgfqpoint{5.254050in}{1.244025in}}%
\pgfpathlineto{\pgfqpoint{5.281749in}{1.244208in}}%
\pgfpathlineto{\pgfqpoint{5.309448in}{1.244117in}}%
\pgfpathlineto{\pgfqpoint{5.337148in}{1.243811in}}%
\pgfpathlineto{\pgfqpoint{5.364847in}{1.243490in}}%
\pgfpathlineto{\pgfqpoint{5.392546in}{1.243252in}}%
\pgfpathlineto{\pgfqpoint{5.420246in}{1.243078in}}%
\pgfpathlineto{\pgfqpoint{5.447945in}{1.242946in}}%
\pgfpathlineto{\pgfqpoint{5.475644in}{1.242813in}}%
\pgfpathlineto{\pgfqpoint{5.503343in}{1.242615in}}%
\pgfpathlineto{\pgfqpoint{5.531043in}{1.242355in}}%
\pgfpathlineto{\pgfqpoint{5.558742in}{1.242035in}}%
\pgfpathlineto{\pgfqpoint{5.586441in}{1.241673in}}%
\pgfpathlineto{\pgfqpoint{5.614141in}{1.241293in}}%
\pgfpathlineto{\pgfqpoint{5.641840in}{1.240914in}}%
\pgfpathlineto{\pgfqpoint{5.669539in}{1.240552in}}%
\pgfpathlineto{\pgfqpoint{5.697238in}{1.240220in}}%
\pgfpathlineto{\pgfqpoint{5.724938in}{1.239923in}}%
\pgfpathlineto{\pgfqpoint{5.752637in}{1.239666in}}%
\pgfpathlineto{\pgfqpoint{5.780336in}{1.239445in}}%
\pgfpathlineto{\pgfqpoint{5.808036in}{1.239260in}}%
\pgfpathlineto{\pgfqpoint{5.835735in}{1.239106in}}%
\pgfpathlineto{\pgfqpoint{5.863434in}{1.238979in}}%
\pgfpathlineto{\pgfqpoint{5.891133in}{1.238877in}}%
\pgfpathlineto{\pgfqpoint{5.918833in}{1.238795in}}%
\pgfpathlineto{\pgfqpoint{5.946532in}{1.238729in}}%
\pgfpathlineto{\pgfqpoint{5.974231in}{1.238677in}}%
\pgfpathlineto{\pgfqpoint{6.001931in}{1.238636in}}%
\pgfpathlineto{\pgfqpoint{6.029630in}{1.238604in}}%
\pgfpathlineto{\pgfqpoint{6.057329in}{1.238579in}}%
\pgfpathlineto{\pgfqpoint{6.067463in}{1.238572in}}%
\pgfusepath{stroke}%
\end{pgfscope}%
\begin{pgfscope}%
\pgfpathrectangle{\pgfqpoint{4.701213in}{0.383578in}}{\pgfqpoint{1.356250in}{1.540000in}}%
\pgfusepath{clip}%
\pgfsetrectcap%
\pgfsetroundjoin%
\pgfsetlinewidth{0.803000pt}%
\definecolor{currentstroke}{rgb}{0.686275,0.352941,0.313725}%
\pgfsetstrokecolor{currentstroke}%
\pgfsetstrokeopacity{0.300000}%
\pgfsetdash{}{0pt}%
\pgfpathmoveto{\pgfqpoint{4.755463in}{1.238500in}}%
\pgfpathlineto{\pgfqpoint{4.783162in}{1.238500in}}%
\pgfpathlineto{\pgfqpoint{4.810861in}{1.238500in}}%
\pgfpathlineto{\pgfqpoint{4.838561in}{1.238500in}}%
\pgfpathlineto{\pgfqpoint{4.866260in}{1.238500in}}%
\pgfpathlineto{\pgfqpoint{4.893959in}{1.238500in}}%
\pgfpathlineto{\pgfqpoint{4.921658in}{1.238500in}}%
\pgfpathlineto{\pgfqpoint{4.949358in}{1.238500in}}%
\pgfpathlineto{\pgfqpoint{4.977057in}{1.238500in}}%
\pgfpathlineto{\pgfqpoint{5.004756in}{1.238500in}}%
\pgfpathlineto{\pgfqpoint{5.032456in}{1.238500in}}%
\pgfpathlineto{\pgfqpoint{5.060155in}{1.238500in}}%
\pgfpathlineto{\pgfqpoint{5.087854in}{1.238500in}}%
\pgfpathlineto{\pgfqpoint{5.115553in}{1.238500in}}%
\pgfpathlineto{\pgfqpoint{5.143253in}{1.238500in}}%
\pgfpathlineto{\pgfqpoint{5.170952in}{1.238500in}}%
\pgfpathlineto{\pgfqpoint{5.198651in}{1.238500in}}%
\pgfpathlineto{\pgfqpoint{5.226351in}{1.238500in}}%
\pgfpathlineto{\pgfqpoint{5.254050in}{1.238500in}}%
\pgfpathlineto{\pgfqpoint{5.281749in}{1.238500in}}%
\pgfpathlineto{\pgfqpoint{5.309448in}{1.238500in}}%
\pgfpathlineto{\pgfqpoint{5.337148in}{1.238500in}}%
\pgfpathlineto{\pgfqpoint{5.364847in}{1.238500in}}%
\pgfpathlineto{\pgfqpoint{5.392546in}{1.238500in}}%
\pgfpathlineto{\pgfqpoint{5.420246in}{1.238500in}}%
\pgfpathlineto{\pgfqpoint{5.447945in}{1.238500in}}%
\pgfpathlineto{\pgfqpoint{5.475644in}{1.238500in}}%
\pgfpathlineto{\pgfqpoint{5.503343in}{1.238500in}}%
\pgfpathlineto{\pgfqpoint{5.531043in}{1.238500in}}%
\pgfpathlineto{\pgfqpoint{5.558742in}{1.238500in}}%
\pgfpathlineto{\pgfqpoint{5.586441in}{1.238500in}}%
\pgfpathlineto{\pgfqpoint{5.614141in}{1.238500in}}%
\pgfpathlineto{\pgfqpoint{5.641840in}{1.238500in}}%
\pgfpathlineto{\pgfqpoint{5.669539in}{1.238500in}}%
\pgfpathlineto{\pgfqpoint{5.697238in}{1.238500in}}%
\pgfpathlineto{\pgfqpoint{5.724938in}{1.238500in}}%
\pgfpathlineto{\pgfqpoint{5.752637in}{1.238500in}}%
\pgfpathlineto{\pgfqpoint{5.780336in}{1.238500in}}%
\pgfpathlineto{\pgfqpoint{5.808036in}{1.238500in}}%
\pgfpathlineto{\pgfqpoint{5.835735in}{1.238500in}}%
\pgfpathlineto{\pgfqpoint{5.863434in}{1.238500in}}%
\pgfpathlineto{\pgfqpoint{5.891133in}{1.238500in}}%
\pgfpathlineto{\pgfqpoint{5.918833in}{1.238500in}}%
\pgfpathlineto{\pgfqpoint{5.946532in}{1.238500in}}%
\pgfpathlineto{\pgfqpoint{5.974231in}{1.238500in}}%
\pgfpathlineto{\pgfqpoint{6.001931in}{1.238500in}}%
\pgfpathlineto{\pgfqpoint{6.029630in}{1.238500in}}%
\pgfpathlineto{\pgfqpoint{6.057329in}{1.238500in}}%
\pgfpathlineto{\pgfqpoint{6.067463in}{1.238500in}}%
\pgfusepath{stroke}%
\end{pgfscope}%
\begin{pgfscope}%
\pgfpathrectangle{\pgfqpoint{4.701213in}{0.383578in}}{\pgfqpoint{1.356250in}{1.540000in}}%
\pgfusepath{clip}%
\pgfsetrectcap%
\pgfsetroundjoin%
\pgfsetlinewidth{0.803000pt}%
\definecolor{currentstroke}{rgb}{0.686275,0.352941,0.313725}%
\pgfsetstrokecolor{currentstroke}%
\pgfsetstrokeopacity{0.300000}%
\pgfsetdash{}{0pt}%
\pgfpathmoveto{\pgfqpoint{4.755463in}{1.238500in}}%
\pgfpathlineto{\pgfqpoint{4.783162in}{1.238500in}}%
\pgfpathlineto{\pgfqpoint{4.810861in}{1.238500in}}%
\pgfpathlineto{\pgfqpoint{4.838561in}{1.238500in}}%
\pgfpathlineto{\pgfqpoint{4.866260in}{1.238500in}}%
\pgfpathlineto{\pgfqpoint{4.893959in}{1.238500in}}%
\pgfpathlineto{\pgfqpoint{4.921658in}{1.238500in}}%
\pgfpathlineto{\pgfqpoint{4.949358in}{1.238500in}}%
\pgfpathlineto{\pgfqpoint{4.977057in}{1.238500in}}%
\pgfpathlineto{\pgfqpoint{5.004756in}{1.238500in}}%
\pgfpathlineto{\pgfqpoint{5.032456in}{1.238500in}}%
\pgfpathlineto{\pgfqpoint{5.060155in}{1.238500in}}%
\pgfpathlineto{\pgfqpoint{5.087854in}{1.238500in}}%
\pgfpathlineto{\pgfqpoint{5.115553in}{1.238500in}}%
\pgfpathlineto{\pgfqpoint{5.143253in}{1.238500in}}%
\pgfpathlineto{\pgfqpoint{5.170952in}{1.238500in}}%
\pgfpathlineto{\pgfqpoint{5.198651in}{1.238500in}}%
\pgfpathlineto{\pgfqpoint{5.226351in}{1.238500in}}%
\pgfpathlineto{\pgfqpoint{5.254050in}{1.238500in}}%
\pgfpathlineto{\pgfqpoint{5.281749in}{1.238500in}}%
\pgfpathlineto{\pgfqpoint{5.309448in}{1.238500in}}%
\pgfpathlineto{\pgfqpoint{5.337148in}{1.238500in}}%
\pgfpathlineto{\pgfqpoint{5.364847in}{1.238500in}}%
\pgfpathlineto{\pgfqpoint{5.392546in}{1.238500in}}%
\pgfpathlineto{\pgfqpoint{5.420246in}{1.238500in}}%
\pgfpathlineto{\pgfqpoint{5.447945in}{1.238500in}}%
\pgfpathlineto{\pgfqpoint{5.475644in}{1.238500in}}%
\pgfpathlineto{\pgfqpoint{5.503343in}{1.238500in}}%
\pgfpathlineto{\pgfqpoint{5.531043in}{1.238500in}}%
\pgfpathlineto{\pgfqpoint{5.558742in}{1.238500in}}%
\pgfpathlineto{\pgfqpoint{5.586441in}{1.238500in}}%
\pgfpathlineto{\pgfqpoint{5.614141in}{1.238500in}}%
\pgfpathlineto{\pgfqpoint{5.641840in}{1.238500in}}%
\pgfpathlineto{\pgfqpoint{5.669539in}{1.238500in}}%
\pgfpathlineto{\pgfqpoint{5.697238in}{1.238500in}}%
\pgfpathlineto{\pgfqpoint{5.724938in}{1.238500in}}%
\pgfpathlineto{\pgfqpoint{5.752637in}{1.238500in}}%
\pgfpathlineto{\pgfqpoint{5.780336in}{1.238500in}}%
\pgfpathlineto{\pgfqpoint{5.808036in}{1.238500in}}%
\pgfpathlineto{\pgfqpoint{5.835735in}{1.238500in}}%
\pgfpathlineto{\pgfqpoint{5.863434in}{1.238500in}}%
\pgfpathlineto{\pgfqpoint{5.891133in}{1.238500in}}%
\pgfpathlineto{\pgfqpoint{5.918833in}{1.238500in}}%
\pgfpathlineto{\pgfqpoint{5.946532in}{1.238500in}}%
\pgfpathlineto{\pgfqpoint{5.974231in}{1.238500in}}%
\pgfpathlineto{\pgfqpoint{6.001931in}{1.238500in}}%
\pgfpathlineto{\pgfqpoint{6.029630in}{1.238500in}}%
\pgfpathlineto{\pgfqpoint{6.057329in}{1.238500in}}%
\pgfpathlineto{\pgfqpoint{6.067463in}{1.238500in}}%
\pgfusepath{stroke}%
\end{pgfscope}%
\begin{pgfscope}%
\pgfpathrectangle{\pgfqpoint{4.701213in}{0.383578in}}{\pgfqpoint{1.356250in}{1.540000in}}%
\pgfusepath{clip}%
\pgfsetrectcap%
\pgfsetroundjoin%
\pgfsetlinewidth{0.803000pt}%
\definecolor{currentstroke}{rgb}{0.686275,0.352941,0.313725}%
\pgfsetstrokecolor{currentstroke}%
\pgfsetstrokeopacity{0.300000}%
\pgfsetdash{}{0pt}%
\pgfpathmoveto{\pgfqpoint{4.755463in}{1.238500in}}%
\pgfpathlineto{\pgfqpoint{4.783162in}{1.238500in}}%
\pgfpathlineto{\pgfqpoint{4.810861in}{1.238500in}}%
\pgfpathlineto{\pgfqpoint{4.838561in}{1.238500in}}%
\pgfpathlineto{\pgfqpoint{4.866260in}{1.238500in}}%
\pgfpathlineto{\pgfqpoint{4.893959in}{1.238500in}}%
\pgfpathlineto{\pgfqpoint{4.921658in}{1.238500in}}%
\pgfpathlineto{\pgfqpoint{4.949358in}{1.238500in}}%
\pgfpathlineto{\pgfqpoint{4.977057in}{1.238500in}}%
\pgfpathlineto{\pgfqpoint{5.004756in}{1.238500in}}%
\pgfpathlineto{\pgfqpoint{5.032456in}{1.238500in}}%
\pgfpathlineto{\pgfqpoint{5.060155in}{1.238500in}}%
\pgfpathlineto{\pgfqpoint{5.087854in}{1.238500in}}%
\pgfpathlineto{\pgfqpoint{5.115553in}{1.238500in}}%
\pgfpathlineto{\pgfqpoint{5.143253in}{1.238500in}}%
\pgfpathlineto{\pgfqpoint{5.170952in}{1.238500in}}%
\pgfpathlineto{\pgfqpoint{5.198651in}{1.238500in}}%
\pgfpathlineto{\pgfqpoint{5.226351in}{1.238500in}}%
\pgfpathlineto{\pgfqpoint{5.254050in}{1.238500in}}%
\pgfpathlineto{\pgfqpoint{5.281749in}{1.238500in}}%
\pgfpathlineto{\pgfqpoint{5.309448in}{1.238500in}}%
\pgfpathlineto{\pgfqpoint{5.337148in}{1.238500in}}%
\pgfpathlineto{\pgfqpoint{5.364847in}{1.238500in}}%
\pgfpathlineto{\pgfqpoint{5.392546in}{1.238500in}}%
\pgfpathlineto{\pgfqpoint{5.420246in}{1.238500in}}%
\pgfpathlineto{\pgfqpoint{5.447945in}{1.238500in}}%
\pgfpathlineto{\pgfqpoint{5.475644in}{1.238500in}}%
\pgfpathlineto{\pgfqpoint{5.503343in}{1.238500in}}%
\pgfpathlineto{\pgfqpoint{5.531043in}{1.238500in}}%
\pgfpathlineto{\pgfqpoint{5.558742in}{1.238500in}}%
\pgfpathlineto{\pgfqpoint{5.586441in}{1.238500in}}%
\pgfpathlineto{\pgfqpoint{5.614141in}{1.238500in}}%
\pgfpathlineto{\pgfqpoint{5.641840in}{1.238500in}}%
\pgfpathlineto{\pgfqpoint{5.669539in}{1.238500in}}%
\pgfpathlineto{\pgfqpoint{5.697238in}{1.238500in}}%
\pgfpathlineto{\pgfqpoint{5.724938in}{1.238500in}}%
\pgfpathlineto{\pgfqpoint{5.752637in}{1.238500in}}%
\pgfpathlineto{\pgfqpoint{5.780336in}{1.238500in}}%
\pgfpathlineto{\pgfqpoint{5.808036in}{1.238500in}}%
\pgfpathlineto{\pgfqpoint{5.835735in}{1.238500in}}%
\pgfpathlineto{\pgfqpoint{5.863434in}{1.238500in}}%
\pgfpathlineto{\pgfqpoint{5.891133in}{1.238500in}}%
\pgfpathlineto{\pgfqpoint{5.918833in}{1.238500in}}%
\pgfpathlineto{\pgfqpoint{5.946532in}{1.238500in}}%
\pgfpathlineto{\pgfqpoint{5.974231in}{1.238500in}}%
\pgfpathlineto{\pgfqpoint{6.001931in}{1.238500in}}%
\pgfpathlineto{\pgfqpoint{6.029630in}{1.238500in}}%
\pgfpathlineto{\pgfqpoint{6.057329in}{1.238500in}}%
\pgfpathlineto{\pgfqpoint{6.067463in}{1.238500in}}%
\pgfusepath{stroke}%
\end{pgfscope}%
\begin{pgfscope}%
\pgfpathrectangle{\pgfqpoint{4.701213in}{0.383578in}}{\pgfqpoint{1.356250in}{1.540000in}}%
\pgfusepath{clip}%
\pgfsetrectcap%
\pgfsetroundjoin%
\pgfsetlinewidth{0.803000pt}%
\definecolor{currentstroke}{rgb}{0.686275,0.352941,0.313725}%
\pgfsetstrokecolor{currentstroke}%
\pgfsetstrokeopacity{0.300000}%
\pgfsetdash{}{0pt}%
\pgfpathmoveto{\pgfqpoint{4.755463in}{1.238500in}}%
\pgfpathlineto{\pgfqpoint{4.783162in}{1.238500in}}%
\pgfpathlineto{\pgfqpoint{4.810861in}{1.238500in}}%
\pgfpathlineto{\pgfqpoint{4.838561in}{1.238500in}}%
\pgfpathlineto{\pgfqpoint{4.866260in}{1.238500in}}%
\pgfpathlineto{\pgfqpoint{4.893959in}{1.238500in}}%
\pgfpathlineto{\pgfqpoint{4.921658in}{1.238500in}}%
\pgfpathlineto{\pgfqpoint{4.949358in}{1.238500in}}%
\pgfpathlineto{\pgfqpoint{4.977057in}{1.238500in}}%
\pgfpathlineto{\pgfqpoint{5.004756in}{1.238500in}}%
\pgfpathlineto{\pgfqpoint{5.032456in}{1.238500in}}%
\pgfpathlineto{\pgfqpoint{5.060155in}{1.238500in}}%
\pgfpathlineto{\pgfqpoint{5.087854in}{1.238500in}}%
\pgfpathlineto{\pgfqpoint{5.115553in}{1.238500in}}%
\pgfpathlineto{\pgfqpoint{5.143253in}{1.493671in}}%
\pgfpathlineto{\pgfqpoint{5.170952in}{1.628387in}}%
\pgfpathlineto{\pgfqpoint{5.198651in}{1.678794in}}%
\pgfpathlineto{\pgfqpoint{5.226351in}{1.675369in}}%
\pgfpathlineto{\pgfqpoint{5.254050in}{1.641052in}}%
\pgfpathlineto{\pgfqpoint{5.281749in}{1.591819in}}%
\pgfpathlineto{\pgfqpoint{5.309448in}{1.538018in}}%
\pgfpathlineto{\pgfqpoint{5.337148in}{1.485832in}}%
\pgfpathlineto{\pgfqpoint{5.364847in}{1.438668in}}%
\pgfpathlineto{\pgfqpoint{5.392546in}{1.397983in}}%
\pgfpathlineto{\pgfqpoint{5.420246in}{1.363996in}}%
\pgfpathlineto{\pgfqpoint{5.447945in}{1.336282in}}%
\pgfpathlineto{\pgfqpoint{5.475644in}{1.314087in}}%
\pgfpathlineto{\pgfqpoint{5.503343in}{1.296533in}}%
\pgfpathlineto{\pgfqpoint{5.531043in}{1.282803in}}%
\pgfpathlineto{\pgfqpoint{5.558742in}{1.272152in}}%
\pgfpathlineto{\pgfqpoint{5.586441in}{1.263948in}}%
\pgfpathlineto{\pgfqpoint{5.614141in}{1.257668in}}%
\pgfpathlineto{\pgfqpoint{5.641840in}{1.252885in}}%
\pgfpathlineto{\pgfqpoint{5.669539in}{1.249260in}}%
\pgfpathlineto{\pgfqpoint{5.697238in}{1.246524in}}%
\pgfpathlineto{\pgfqpoint{5.724938in}{1.244467in}}%
\pgfpathlineto{\pgfqpoint{5.752637in}{1.242926in}}%
\pgfpathlineto{\pgfqpoint{5.780336in}{1.241776in}}%
\pgfpathlineto{\pgfqpoint{5.808036in}{1.240920in}}%
\pgfpathlineto{\pgfqpoint{5.835735in}{1.240284in}}%
\pgfpathlineto{\pgfqpoint{5.863434in}{1.239813in}}%
\pgfpathlineto{\pgfqpoint{5.891133in}{1.239464in}}%
\pgfpathlineto{\pgfqpoint{5.918833in}{1.239207in}}%
\pgfpathlineto{\pgfqpoint{5.946532in}{1.239018in}}%
\pgfpathlineto{\pgfqpoint{5.974231in}{1.238879in}}%
\pgfpathlineto{\pgfqpoint{6.001931in}{1.238776in}}%
\pgfpathlineto{\pgfqpoint{6.029630in}{1.238701in}}%
\pgfpathlineto{\pgfqpoint{6.057329in}{1.238647in}}%
\pgfpathlineto{\pgfqpoint{6.067463in}{1.238632in}}%
\pgfusepath{stroke}%
\end{pgfscope}%
\begin{pgfscope}%
\pgfpathrectangle{\pgfqpoint{4.701213in}{0.383578in}}{\pgfqpoint{1.356250in}{1.540000in}}%
\pgfusepath{clip}%
\pgfsetrectcap%
\pgfsetroundjoin%
\pgfsetlinewidth{0.803000pt}%
\definecolor{currentstroke}{rgb}{0.686275,0.352941,0.313725}%
\pgfsetstrokecolor{currentstroke}%
\pgfsetstrokeopacity{0.300000}%
\pgfsetdash{}{0pt}%
\pgfpathmoveto{\pgfqpoint{4.755463in}{1.238500in}}%
\pgfpathlineto{\pgfqpoint{4.783162in}{1.238500in}}%
\pgfpathlineto{\pgfqpoint{4.810861in}{1.238500in}}%
\pgfpathlineto{\pgfqpoint{4.838561in}{1.238500in}}%
\pgfpathlineto{\pgfqpoint{4.866260in}{1.238500in}}%
\pgfpathlineto{\pgfqpoint{4.893959in}{1.238500in}}%
\pgfpathlineto{\pgfqpoint{4.921658in}{1.238500in}}%
\pgfpathlineto{\pgfqpoint{4.949358in}{1.238500in}}%
\pgfpathlineto{\pgfqpoint{4.977057in}{1.238500in}}%
\pgfpathlineto{\pgfqpoint{5.004756in}{1.238500in}}%
\pgfpathlineto{\pgfqpoint{5.032456in}{1.238500in}}%
\pgfpathlineto{\pgfqpoint{5.060155in}{1.238500in}}%
\pgfpathlineto{\pgfqpoint{5.087854in}{1.238500in}}%
\pgfpathlineto{\pgfqpoint{5.115553in}{1.238500in}}%
\pgfpathlineto{\pgfqpoint{5.143253in}{1.238500in}}%
\pgfpathlineto{\pgfqpoint{5.170952in}{1.238500in}}%
\pgfpathlineto{\pgfqpoint{5.198651in}{1.238500in}}%
\pgfpathlineto{\pgfqpoint{5.226351in}{1.238500in}}%
\pgfpathlineto{\pgfqpoint{5.254050in}{1.238500in}}%
\pgfpathlineto{\pgfqpoint{5.281749in}{1.238500in}}%
\pgfpathlineto{\pgfqpoint{5.309448in}{1.238500in}}%
\pgfpathlineto{\pgfqpoint{5.337148in}{1.238500in}}%
\pgfpathlineto{\pgfqpoint{5.364847in}{1.238500in}}%
\pgfpathlineto{\pgfqpoint{5.392546in}{1.238500in}}%
\pgfpathlineto{\pgfqpoint{5.420246in}{1.238500in}}%
\pgfpathlineto{\pgfqpoint{5.447945in}{1.238500in}}%
\pgfpathlineto{\pgfqpoint{5.475644in}{1.238500in}}%
\pgfpathlineto{\pgfqpoint{5.503343in}{1.238500in}}%
\pgfpathlineto{\pgfqpoint{5.531043in}{1.238500in}}%
\pgfpathlineto{\pgfqpoint{5.558742in}{1.238500in}}%
\pgfpathlineto{\pgfqpoint{5.586441in}{1.238500in}}%
\pgfpathlineto{\pgfqpoint{5.614141in}{1.238500in}}%
\pgfpathlineto{\pgfqpoint{5.641840in}{1.238500in}}%
\pgfpathlineto{\pgfqpoint{5.669539in}{1.238500in}}%
\pgfpathlineto{\pgfqpoint{5.697238in}{1.238500in}}%
\pgfpathlineto{\pgfqpoint{5.724938in}{1.238500in}}%
\pgfpathlineto{\pgfqpoint{5.752637in}{1.238500in}}%
\pgfpathlineto{\pgfqpoint{5.780336in}{1.238500in}}%
\pgfpathlineto{\pgfqpoint{5.808036in}{1.238500in}}%
\pgfpathlineto{\pgfqpoint{5.835735in}{1.238500in}}%
\pgfpathlineto{\pgfqpoint{5.863434in}{1.238500in}}%
\pgfpathlineto{\pgfqpoint{5.891133in}{1.238500in}}%
\pgfpathlineto{\pgfqpoint{5.918833in}{1.238500in}}%
\pgfpathlineto{\pgfqpoint{5.946532in}{1.238500in}}%
\pgfpathlineto{\pgfqpoint{5.974231in}{1.238500in}}%
\pgfpathlineto{\pgfqpoint{6.001931in}{1.238500in}}%
\pgfpathlineto{\pgfqpoint{6.029630in}{1.238500in}}%
\pgfpathlineto{\pgfqpoint{6.057329in}{1.238500in}}%
\pgfpathlineto{\pgfqpoint{6.067463in}{1.238500in}}%
\pgfusepath{stroke}%
\end{pgfscope}%
\begin{pgfscope}%
\pgfpathrectangle{\pgfqpoint{4.701213in}{0.383578in}}{\pgfqpoint{1.356250in}{1.540000in}}%
\pgfusepath{clip}%
\pgfsetrectcap%
\pgfsetroundjoin%
\pgfsetlinewidth{0.803000pt}%
\definecolor{currentstroke}{rgb}{0.686275,0.352941,0.313725}%
\pgfsetstrokecolor{currentstroke}%
\pgfsetstrokeopacity{0.300000}%
\pgfsetdash{}{0pt}%
\pgfpathmoveto{\pgfqpoint{4.755463in}{1.238500in}}%
\pgfpathlineto{\pgfqpoint{4.783162in}{1.238500in}}%
\pgfpathlineto{\pgfqpoint{4.810861in}{1.238500in}}%
\pgfpathlineto{\pgfqpoint{4.838561in}{1.238500in}}%
\pgfpathlineto{\pgfqpoint{4.866260in}{1.238500in}}%
\pgfpathlineto{\pgfqpoint{4.893959in}{1.238500in}}%
\pgfpathlineto{\pgfqpoint{4.921658in}{1.238500in}}%
\pgfpathlineto{\pgfqpoint{4.949358in}{1.238500in}}%
\pgfpathlineto{\pgfqpoint{4.977057in}{1.238500in}}%
\pgfpathlineto{\pgfqpoint{5.004756in}{1.238500in}}%
\pgfpathlineto{\pgfqpoint{5.032456in}{1.238500in}}%
\pgfpathlineto{\pgfqpoint{5.060155in}{1.238500in}}%
\pgfpathlineto{\pgfqpoint{5.087854in}{1.238500in}}%
\pgfpathlineto{\pgfqpoint{5.115553in}{1.238500in}}%
\pgfpathlineto{\pgfqpoint{5.143253in}{1.238500in}}%
\pgfpathlineto{\pgfqpoint{5.170952in}{1.238500in}}%
\pgfpathlineto{\pgfqpoint{5.198651in}{1.238500in}}%
\pgfpathlineto{\pgfqpoint{5.226351in}{1.238500in}}%
\pgfpathlineto{\pgfqpoint{5.254050in}{1.238500in}}%
\pgfpathlineto{\pgfqpoint{5.281749in}{1.238500in}}%
\pgfpathlineto{\pgfqpoint{5.309448in}{1.238500in}}%
\pgfpathlineto{\pgfqpoint{5.337148in}{1.238500in}}%
\pgfpathlineto{\pgfqpoint{5.364847in}{1.238500in}}%
\pgfpathlineto{\pgfqpoint{5.392546in}{1.238500in}}%
\pgfpathlineto{\pgfqpoint{5.420246in}{1.238500in}}%
\pgfpathlineto{\pgfqpoint{5.447945in}{1.238500in}}%
\pgfpathlineto{\pgfqpoint{5.475644in}{1.238500in}}%
\pgfpathlineto{\pgfqpoint{5.503343in}{1.238500in}}%
\pgfpathlineto{\pgfqpoint{5.531043in}{1.238500in}}%
\pgfpathlineto{\pgfqpoint{5.558742in}{1.238500in}}%
\pgfpathlineto{\pgfqpoint{5.586441in}{1.238500in}}%
\pgfpathlineto{\pgfqpoint{5.614141in}{1.238500in}}%
\pgfpathlineto{\pgfqpoint{5.641840in}{1.238500in}}%
\pgfpathlineto{\pgfqpoint{5.669539in}{1.238500in}}%
\pgfpathlineto{\pgfqpoint{5.697238in}{1.238500in}}%
\pgfpathlineto{\pgfqpoint{5.724938in}{1.238500in}}%
\pgfpathlineto{\pgfqpoint{5.752637in}{1.238500in}}%
\pgfpathlineto{\pgfqpoint{5.780336in}{1.238500in}}%
\pgfpathlineto{\pgfqpoint{5.808036in}{1.238500in}}%
\pgfpathlineto{\pgfqpoint{5.835735in}{1.238500in}}%
\pgfpathlineto{\pgfqpoint{5.863434in}{1.238500in}}%
\pgfpathlineto{\pgfqpoint{5.891133in}{1.238500in}}%
\pgfpathlineto{\pgfqpoint{5.918833in}{1.238500in}}%
\pgfpathlineto{\pgfqpoint{5.946532in}{1.238500in}}%
\pgfpathlineto{\pgfqpoint{5.974231in}{1.238500in}}%
\pgfpathlineto{\pgfqpoint{6.001931in}{1.238500in}}%
\pgfpathlineto{\pgfqpoint{6.029630in}{1.238500in}}%
\pgfpathlineto{\pgfqpoint{6.057329in}{1.238500in}}%
\pgfpathlineto{\pgfqpoint{6.067463in}{1.238500in}}%
\pgfusepath{stroke}%
\end{pgfscope}%
\begin{pgfscope}%
\pgfpathrectangle{\pgfqpoint{4.701213in}{0.383578in}}{\pgfqpoint{1.356250in}{1.540000in}}%
\pgfusepath{clip}%
\pgfsetrectcap%
\pgfsetroundjoin%
\pgfsetlinewidth{0.803000pt}%
\definecolor{currentstroke}{rgb}{0.686275,0.352941,0.313725}%
\pgfsetstrokecolor{currentstroke}%
\pgfsetstrokeopacity{0.300000}%
\pgfsetdash{}{0pt}%
\pgfpathmoveto{\pgfqpoint{4.755463in}{1.238500in}}%
\pgfpathlineto{\pgfqpoint{4.783162in}{1.238500in}}%
\pgfpathlineto{\pgfqpoint{4.810861in}{1.238500in}}%
\pgfpathlineto{\pgfqpoint{4.838561in}{1.238500in}}%
\pgfpathlineto{\pgfqpoint{4.866260in}{1.238500in}}%
\pgfpathlineto{\pgfqpoint{4.893959in}{1.238500in}}%
\pgfpathlineto{\pgfqpoint{4.921658in}{1.238500in}}%
\pgfpathlineto{\pgfqpoint{4.949358in}{1.238500in}}%
\pgfpathlineto{\pgfqpoint{4.977057in}{1.238500in}}%
\pgfpathlineto{\pgfqpoint{5.004756in}{1.238500in}}%
\pgfpathlineto{\pgfqpoint{5.032456in}{1.238500in}}%
\pgfpathlineto{\pgfqpoint{5.060155in}{1.238500in}}%
\pgfpathlineto{\pgfqpoint{5.087854in}{1.238500in}}%
\pgfpathlineto{\pgfqpoint{5.115553in}{1.238500in}}%
\pgfpathlineto{\pgfqpoint{5.143253in}{1.238500in}}%
\pgfpathlineto{\pgfqpoint{5.170952in}{1.238500in}}%
\pgfpathlineto{\pgfqpoint{5.198651in}{1.238500in}}%
\pgfpathlineto{\pgfqpoint{5.226351in}{1.238500in}}%
\pgfpathlineto{\pgfqpoint{5.254050in}{1.238500in}}%
\pgfpathlineto{\pgfqpoint{5.281749in}{1.238500in}}%
\pgfpathlineto{\pgfqpoint{5.309448in}{1.238500in}}%
\pgfpathlineto{\pgfqpoint{5.337148in}{1.238500in}}%
\pgfpathlineto{\pgfqpoint{5.364847in}{1.238500in}}%
\pgfpathlineto{\pgfqpoint{5.392546in}{1.238500in}}%
\pgfpathlineto{\pgfqpoint{5.420246in}{1.238500in}}%
\pgfpathlineto{\pgfqpoint{5.447945in}{1.238500in}}%
\pgfpathlineto{\pgfqpoint{5.475644in}{1.238500in}}%
\pgfpathlineto{\pgfqpoint{5.503343in}{1.238500in}}%
\pgfpathlineto{\pgfqpoint{5.531043in}{1.238500in}}%
\pgfpathlineto{\pgfqpoint{5.558742in}{1.238500in}}%
\pgfpathlineto{\pgfqpoint{5.586441in}{1.238500in}}%
\pgfpathlineto{\pgfqpoint{5.614141in}{1.238500in}}%
\pgfpathlineto{\pgfqpoint{5.641840in}{1.238500in}}%
\pgfpathlineto{\pgfqpoint{5.669539in}{1.238500in}}%
\pgfpathlineto{\pgfqpoint{5.697238in}{1.238500in}}%
\pgfpathlineto{\pgfqpoint{5.724938in}{1.238500in}}%
\pgfpathlineto{\pgfqpoint{5.752637in}{1.238500in}}%
\pgfpathlineto{\pgfqpoint{5.780336in}{1.238500in}}%
\pgfpathlineto{\pgfqpoint{5.808036in}{1.238500in}}%
\pgfpathlineto{\pgfqpoint{5.835735in}{1.238500in}}%
\pgfpathlineto{\pgfqpoint{5.863434in}{1.238500in}}%
\pgfpathlineto{\pgfqpoint{5.891133in}{1.238500in}}%
\pgfpathlineto{\pgfqpoint{5.918833in}{1.238500in}}%
\pgfpathlineto{\pgfqpoint{5.946532in}{1.238500in}}%
\pgfpathlineto{\pgfqpoint{5.974231in}{1.238500in}}%
\pgfpathlineto{\pgfqpoint{6.001931in}{1.238500in}}%
\pgfpathlineto{\pgfqpoint{6.029630in}{1.238500in}}%
\pgfpathlineto{\pgfqpoint{6.057329in}{1.238500in}}%
\pgfpathlineto{\pgfqpoint{6.067463in}{1.238500in}}%
\pgfusepath{stroke}%
\end{pgfscope}%
\begin{pgfscope}%
\pgfpathrectangle{\pgfqpoint{4.701213in}{0.383578in}}{\pgfqpoint{1.356250in}{1.540000in}}%
\pgfusepath{clip}%
\pgfsetrectcap%
\pgfsetroundjoin%
\pgfsetlinewidth{0.803000pt}%
\definecolor{currentstroke}{rgb}{0.686275,0.352941,0.313725}%
\pgfsetstrokecolor{currentstroke}%
\pgfsetstrokeopacity{0.300000}%
\pgfsetdash{}{0pt}%
\pgfpathmoveto{\pgfqpoint{4.755463in}{1.238500in}}%
\pgfpathlineto{\pgfqpoint{4.783162in}{1.238500in}}%
\pgfpathlineto{\pgfqpoint{4.810861in}{1.238500in}}%
\pgfpathlineto{\pgfqpoint{4.838561in}{1.238500in}}%
\pgfpathlineto{\pgfqpoint{4.866260in}{1.238500in}}%
\pgfpathlineto{\pgfqpoint{4.893959in}{1.238500in}}%
\pgfpathlineto{\pgfqpoint{4.921658in}{1.238500in}}%
\pgfpathlineto{\pgfqpoint{4.949358in}{1.238500in}}%
\pgfpathlineto{\pgfqpoint{4.977057in}{1.238500in}}%
\pgfpathlineto{\pgfqpoint{5.004756in}{1.238500in}}%
\pgfpathlineto{\pgfqpoint{5.032456in}{1.238500in}}%
\pgfpathlineto{\pgfqpoint{5.060155in}{1.238500in}}%
\pgfpathlineto{\pgfqpoint{5.087854in}{1.238500in}}%
\pgfpathlineto{\pgfqpoint{5.115553in}{1.238500in}}%
\pgfpathlineto{\pgfqpoint{5.143253in}{1.238500in}}%
\pgfpathlineto{\pgfqpoint{5.170952in}{1.238500in}}%
\pgfpathlineto{\pgfqpoint{5.198651in}{1.238500in}}%
\pgfpathlineto{\pgfqpoint{5.226351in}{1.238500in}}%
\pgfpathlineto{\pgfqpoint{5.254050in}{1.238500in}}%
\pgfpathlineto{\pgfqpoint{5.281749in}{1.238500in}}%
\pgfpathlineto{\pgfqpoint{5.309448in}{1.238500in}}%
\pgfpathlineto{\pgfqpoint{5.337148in}{1.238500in}}%
\pgfpathlineto{\pgfqpoint{5.364847in}{1.238500in}}%
\pgfpathlineto{\pgfqpoint{5.392546in}{1.238500in}}%
\pgfpathlineto{\pgfqpoint{5.420246in}{1.238500in}}%
\pgfpathlineto{\pgfqpoint{5.447945in}{1.238500in}}%
\pgfpathlineto{\pgfqpoint{5.475644in}{1.238500in}}%
\pgfpathlineto{\pgfqpoint{5.503343in}{1.238500in}}%
\pgfpathlineto{\pgfqpoint{5.531043in}{1.238500in}}%
\pgfpathlineto{\pgfqpoint{5.558742in}{1.238500in}}%
\pgfpathlineto{\pgfqpoint{5.586441in}{1.238500in}}%
\pgfpathlineto{\pgfqpoint{5.614141in}{1.238500in}}%
\pgfpathlineto{\pgfqpoint{5.641840in}{1.238500in}}%
\pgfpathlineto{\pgfqpoint{5.669539in}{1.238500in}}%
\pgfpathlineto{\pgfqpoint{5.697238in}{1.238500in}}%
\pgfpathlineto{\pgfqpoint{5.724938in}{1.238500in}}%
\pgfpathlineto{\pgfqpoint{5.752637in}{1.238500in}}%
\pgfpathlineto{\pgfqpoint{5.780336in}{1.238500in}}%
\pgfpathlineto{\pgfqpoint{5.808036in}{1.238500in}}%
\pgfpathlineto{\pgfqpoint{5.835735in}{1.238500in}}%
\pgfpathlineto{\pgfqpoint{5.863434in}{1.238500in}}%
\pgfpathlineto{\pgfqpoint{5.891133in}{1.238500in}}%
\pgfpathlineto{\pgfqpoint{5.918833in}{1.238500in}}%
\pgfpathlineto{\pgfqpoint{5.946532in}{1.238500in}}%
\pgfpathlineto{\pgfqpoint{5.974231in}{1.238500in}}%
\pgfpathlineto{\pgfqpoint{6.001931in}{1.238500in}}%
\pgfpathlineto{\pgfqpoint{6.029630in}{1.238500in}}%
\pgfpathlineto{\pgfqpoint{6.057329in}{1.238500in}}%
\pgfpathlineto{\pgfqpoint{6.067463in}{1.238500in}}%
\pgfusepath{stroke}%
\end{pgfscope}%
\begin{pgfscope}%
\pgfpathrectangle{\pgfqpoint{4.701213in}{0.383578in}}{\pgfqpoint{1.356250in}{1.540000in}}%
\pgfusepath{clip}%
\pgfsetrectcap%
\pgfsetroundjoin%
\pgfsetlinewidth{0.803000pt}%
\definecolor{currentstroke}{rgb}{0.686275,0.352941,0.313725}%
\pgfsetstrokecolor{currentstroke}%
\pgfsetstrokeopacity{0.300000}%
\pgfsetdash{}{0pt}%
\pgfpathmoveto{\pgfqpoint{4.755463in}{1.238500in}}%
\pgfpathlineto{\pgfqpoint{4.783162in}{1.238500in}}%
\pgfpathlineto{\pgfqpoint{4.810861in}{1.238500in}}%
\pgfpathlineto{\pgfqpoint{4.838561in}{1.238500in}}%
\pgfpathlineto{\pgfqpoint{4.866260in}{1.238500in}}%
\pgfpathlineto{\pgfqpoint{4.893959in}{1.238500in}}%
\pgfpathlineto{\pgfqpoint{4.921658in}{1.238500in}}%
\pgfpathlineto{\pgfqpoint{4.949358in}{1.238500in}}%
\pgfpathlineto{\pgfqpoint{4.977057in}{1.238500in}}%
\pgfpathlineto{\pgfqpoint{5.004756in}{1.238500in}}%
\pgfpathlineto{\pgfqpoint{5.032456in}{1.238500in}}%
\pgfpathlineto{\pgfqpoint{5.060155in}{1.238500in}}%
\pgfpathlineto{\pgfqpoint{5.087854in}{1.238500in}}%
\pgfpathlineto{\pgfqpoint{5.115553in}{1.238500in}}%
\pgfpathlineto{\pgfqpoint{5.143253in}{1.238500in}}%
\pgfpathlineto{\pgfqpoint{5.170952in}{1.238500in}}%
\pgfpathlineto{\pgfqpoint{5.198651in}{1.238500in}}%
\pgfpathlineto{\pgfqpoint{5.226351in}{1.238500in}}%
\pgfpathlineto{\pgfqpoint{5.254050in}{1.238500in}}%
\pgfpathlineto{\pgfqpoint{5.281749in}{1.238500in}}%
\pgfpathlineto{\pgfqpoint{5.309448in}{1.238500in}}%
\pgfpathlineto{\pgfqpoint{5.337148in}{1.238500in}}%
\pgfpathlineto{\pgfqpoint{5.364847in}{1.238500in}}%
\pgfpathlineto{\pgfqpoint{5.392546in}{1.238500in}}%
\pgfpathlineto{\pgfqpoint{5.420246in}{1.238500in}}%
\pgfpathlineto{\pgfqpoint{5.447945in}{1.238500in}}%
\pgfpathlineto{\pgfqpoint{5.475644in}{1.238500in}}%
\pgfpathlineto{\pgfqpoint{5.503343in}{1.238500in}}%
\pgfpathlineto{\pgfqpoint{5.531043in}{1.238500in}}%
\pgfpathlineto{\pgfqpoint{5.558742in}{1.238500in}}%
\pgfpathlineto{\pgfqpoint{5.586441in}{1.238500in}}%
\pgfpathlineto{\pgfqpoint{5.614141in}{1.238500in}}%
\pgfpathlineto{\pgfqpoint{5.641840in}{1.238500in}}%
\pgfpathlineto{\pgfqpoint{5.669539in}{1.238500in}}%
\pgfpathlineto{\pgfqpoint{5.697238in}{1.238500in}}%
\pgfpathlineto{\pgfqpoint{5.724938in}{1.238500in}}%
\pgfpathlineto{\pgfqpoint{5.752637in}{1.238500in}}%
\pgfpathlineto{\pgfqpoint{5.780336in}{1.238500in}}%
\pgfpathlineto{\pgfqpoint{5.808036in}{1.238500in}}%
\pgfpathlineto{\pgfqpoint{5.835735in}{1.238500in}}%
\pgfpathlineto{\pgfqpoint{5.863434in}{1.238500in}}%
\pgfpathlineto{\pgfqpoint{5.891133in}{1.238500in}}%
\pgfpathlineto{\pgfqpoint{5.918833in}{1.238500in}}%
\pgfpathlineto{\pgfqpoint{5.946532in}{1.238500in}}%
\pgfpathlineto{\pgfqpoint{5.974231in}{1.238500in}}%
\pgfpathlineto{\pgfqpoint{6.001931in}{1.238500in}}%
\pgfpathlineto{\pgfqpoint{6.029630in}{1.238500in}}%
\pgfpathlineto{\pgfqpoint{6.057329in}{1.238500in}}%
\pgfpathlineto{\pgfqpoint{6.067463in}{1.238500in}}%
\pgfusepath{stroke}%
\end{pgfscope}%
\begin{pgfscope}%
\pgfpathrectangle{\pgfqpoint{4.701213in}{0.383578in}}{\pgfqpoint{1.356250in}{1.540000in}}%
\pgfusepath{clip}%
\pgfsetrectcap%
\pgfsetroundjoin%
\pgfsetlinewidth{0.803000pt}%
\definecolor{currentstroke}{rgb}{0.686275,0.352941,0.313725}%
\pgfsetstrokecolor{currentstroke}%
\pgfsetstrokeopacity{0.300000}%
\pgfsetdash{}{0pt}%
\pgfpathmoveto{\pgfqpoint{4.755463in}{1.238500in}}%
\pgfpathlineto{\pgfqpoint{4.783162in}{1.238500in}}%
\pgfpathlineto{\pgfqpoint{4.810861in}{1.238500in}}%
\pgfpathlineto{\pgfqpoint{4.838561in}{1.238500in}}%
\pgfpathlineto{\pgfqpoint{4.866260in}{1.238500in}}%
\pgfpathlineto{\pgfqpoint{4.893959in}{1.238500in}}%
\pgfpathlineto{\pgfqpoint{4.921658in}{1.238500in}}%
\pgfpathlineto{\pgfqpoint{4.949358in}{1.238500in}}%
\pgfpathlineto{\pgfqpoint{4.977057in}{1.238500in}}%
\pgfpathlineto{\pgfqpoint{5.004756in}{1.238500in}}%
\pgfpathlineto{\pgfqpoint{5.032456in}{1.238500in}}%
\pgfpathlineto{\pgfqpoint{5.060155in}{1.238500in}}%
\pgfpathlineto{\pgfqpoint{5.087854in}{1.238500in}}%
\pgfpathlineto{\pgfqpoint{5.115553in}{1.238500in}}%
\pgfpathlineto{\pgfqpoint{5.143253in}{1.238500in}}%
\pgfpathlineto{\pgfqpoint{5.170952in}{1.238500in}}%
\pgfpathlineto{\pgfqpoint{5.198651in}{1.238500in}}%
\pgfpathlineto{\pgfqpoint{5.226351in}{1.238500in}}%
\pgfpathlineto{\pgfqpoint{5.254050in}{1.238500in}}%
\pgfpathlineto{\pgfqpoint{5.281749in}{1.238500in}}%
\pgfpathlineto{\pgfqpoint{5.309448in}{1.238500in}}%
\pgfpathlineto{\pgfqpoint{5.337148in}{1.238500in}}%
\pgfpathlineto{\pgfqpoint{5.364847in}{1.238500in}}%
\pgfpathlineto{\pgfqpoint{5.392546in}{1.238500in}}%
\pgfpathlineto{\pgfqpoint{5.420246in}{1.238500in}}%
\pgfpathlineto{\pgfqpoint{5.447945in}{1.238500in}}%
\pgfpathlineto{\pgfqpoint{5.475644in}{1.238500in}}%
\pgfpathlineto{\pgfqpoint{5.503343in}{1.238500in}}%
\pgfpathlineto{\pgfqpoint{5.531043in}{1.238500in}}%
\pgfpathlineto{\pgfqpoint{5.558742in}{1.238500in}}%
\pgfpathlineto{\pgfqpoint{5.586441in}{1.238500in}}%
\pgfpathlineto{\pgfqpoint{5.614141in}{1.238500in}}%
\pgfpathlineto{\pgfqpoint{5.641840in}{1.238500in}}%
\pgfpathlineto{\pgfqpoint{5.669539in}{1.238500in}}%
\pgfpathlineto{\pgfqpoint{5.697238in}{1.238500in}}%
\pgfpathlineto{\pgfqpoint{5.724938in}{1.238500in}}%
\pgfpathlineto{\pgfqpoint{5.752637in}{1.238500in}}%
\pgfpathlineto{\pgfqpoint{5.780336in}{1.238500in}}%
\pgfpathlineto{\pgfqpoint{5.808036in}{1.238500in}}%
\pgfpathlineto{\pgfqpoint{5.835735in}{1.238500in}}%
\pgfpathlineto{\pgfqpoint{5.863434in}{1.238500in}}%
\pgfpathlineto{\pgfqpoint{5.891133in}{1.238500in}}%
\pgfpathlineto{\pgfqpoint{5.918833in}{1.238500in}}%
\pgfpathlineto{\pgfqpoint{5.946532in}{1.238500in}}%
\pgfpathlineto{\pgfqpoint{5.974231in}{1.238500in}}%
\pgfpathlineto{\pgfqpoint{6.001931in}{1.238500in}}%
\pgfpathlineto{\pgfqpoint{6.029630in}{1.238500in}}%
\pgfpathlineto{\pgfqpoint{6.057329in}{1.238500in}}%
\pgfpathlineto{\pgfqpoint{6.067463in}{1.238500in}}%
\pgfusepath{stroke}%
\end{pgfscope}%
\begin{pgfscope}%
\pgfpathrectangle{\pgfqpoint{4.701213in}{0.383578in}}{\pgfqpoint{1.356250in}{1.540000in}}%
\pgfusepath{clip}%
\pgfsetrectcap%
\pgfsetroundjoin%
\pgfsetlinewidth{0.803000pt}%
\definecolor{currentstroke}{rgb}{0.686275,0.352941,0.313725}%
\pgfsetstrokecolor{currentstroke}%
\pgfsetstrokeopacity{0.300000}%
\pgfsetdash{}{0pt}%
\pgfpathmoveto{\pgfqpoint{4.755463in}{1.238500in}}%
\pgfpathlineto{\pgfqpoint{4.783162in}{1.238500in}}%
\pgfpathlineto{\pgfqpoint{4.810861in}{1.238500in}}%
\pgfpathlineto{\pgfqpoint{4.838561in}{1.238500in}}%
\pgfpathlineto{\pgfqpoint{4.866260in}{1.238500in}}%
\pgfpathlineto{\pgfqpoint{4.893959in}{1.238500in}}%
\pgfpathlineto{\pgfqpoint{4.921658in}{1.238500in}}%
\pgfpathlineto{\pgfqpoint{4.949358in}{1.238500in}}%
\pgfpathlineto{\pgfqpoint{4.977057in}{1.238500in}}%
\pgfpathlineto{\pgfqpoint{5.004756in}{1.238500in}}%
\pgfpathlineto{\pgfqpoint{5.032456in}{1.238500in}}%
\pgfpathlineto{\pgfqpoint{5.060155in}{1.238500in}}%
\pgfpathlineto{\pgfqpoint{5.087854in}{1.238500in}}%
\pgfpathlineto{\pgfqpoint{5.115553in}{1.238500in}}%
\pgfpathlineto{\pgfqpoint{5.143253in}{1.493671in}}%
\pgfpathlineto{\pgfqpoint{5.170952in}{1.628387in}}%
\pgfpathlineto{\pgfqpoint{5.198651in}{1.678794in}}%
\pgfpathlineto{\pgfqpoint{5.226351in}{1.675369in}}%
\pgfpathlineto{\pgfqpoint{5.254050in}{1.641052in}}%
\pgfpathlineto{\pgfqpoint{5.281749in}{1.591819in}}%
\pgfpathlineto{\pgfqpoint{5.309448in}{1.538018in}}%
\pgfpathlineto{\pgfqpoint{5.337148in}{1.485832in}}%
\pgfpathlineto{\pgfqpoint{5.364847in}{1.438668in}}%
\pgfpathlineto{\pgfqpoint{5.392546in}{1.397983in}}%
\pgfpathlineto{\pgfqpoint{5.420246in}{1.363996in}}%
\pgfpathlineto{\pgfqpoint{5.447945in}{1.336282in}}%
\pgfpathlineto{\pgfqpoint{5.475644in}{1.314087in}}%
\pgfpathlineto{\pgfqpoint{5.503343in}{1.296533in}}%
\pgfpathlineto{\pgfqpoint{5.531043in}{1.282803in}}%
\pgfpathlineto{\pgfqpoint{5.558742in}{1.272152in}}%
\pgfpathlineto{\pgfqpoint{5.586441in}{1.263948in}}%
\pgfpathlineto{\pgfqpoint{5.614141in}{1.257668in}}%
\pgfpathlineto{\pgfqpoint{5.641840in}{1.252885in}}%
\pgfpathlineto{\pgfqpoint{5.669539in}{1.249260in}}%
\pgfpathlineto{\pgfqpoint{5.697238in}{1.246524in}}%
\pgfpathlineto{\pgfqpoint{5.724938in}{1.244467in}}%
\pgfpathlineto{\pgfqpoint{5.752637in}{1.242926in}}%
\pgfpathlineto{\pgfqpoint{5.780336in}{1.241776in}}%
\pgfpathlineto{\pgfqpoint{5.808036in}{1.240920in}}%
\pgfpathlineto{\pgfqpoint{5.835735in}{1.240284in}}%
\pgfpathlineto{\pgfqpoint{5.863434in}{1.239813in}}%
\pgfpathlineto{\pgfqpoint{5.891133in}{1.239464in}}%
\pgfpathlineto{\pgfqpoint{5.918833in}{1.239207in}}%
\pgfpathlineto{\pgfqpoint{5.946532in}{1.239018in}}%
\pgfpathlineto{\pgfqpoint{5.974231in}{1.238879in}}%
\pgfpathlineto{\pgfqpoint{6.001931in}{1.238776in}}%
\pgfpathlineto{\pgfqpoint{6.029630in}{1.238701in}}%
\pgfpathlineto{\pgfqpoint{6.057329in}{1.238647in}}%
\pgfpathlineto{\pgfqpoint{6.067463in}{1.238632in}}%
\pgfusepath{stroke}%
\end{pgfscope}%
\begin{pgfscope}%
\pgfpathrectangle{\pgfqpoint{4.701213in}{0.383578in}}{\pgfqpoint{1.356250in}{1.540000in}}%
\pgfusepath{clip}%
\pgfsetrectcap%
\pgfsetroundjoin%
\pgfsetlinewidth{0.803000pt}%
\definecolor{currentstroke}{rgb}{0.686275,0.352941,0.313725}%
\pgfsetstrokecolor{currentstroke}%
\pgfsetstrokeopacity{0.300000}%
\pgfsetdash{}{0pt}%
\pgfpathmoveto{\pgfqpoint{4.755463in}{1.238500in}}%
\pgfpathlineto{\pgfqpoint{4.783162in}{1.238500in}}%
\pgfpathlineto{\pgfqpoint{4.810861in}{1.238500in}}%
\pgfpathlineto{\pgfqpoint{4.838561in}{1.238500in}}%
\pgfpathlineto{\pgfqpoint{4.866260in}{1.238500in}}%
\pgfpathlineto{\pgfqpoint{4.893959in}{1.238500in}}%
\pgfpathlineto{\pgfqpoint{4.921658in}{1.238500in}}%
\pgfpathlineto{\pgfqpoint{4.949358in}{1.238500in}}%
\pgfpathlineto{\pgfqpoint{4.977057in}{1.238500in}}%
\pgfpathlineto{\pgfqpoint{5.004756in}{1.238500in}}%
\pgfpathlineto{\pgfqpoint{5.032456in}{1.238500in}}%
\pgfpathlineto{\pgfqpoint{5.060155in}{1.238500in}}%
\pgfpathlineto{\pgfqpoint{5.087854in}{1.238500in}}%
\pgfpathlineto{\pgfqpoint{5.115553in}{1.238500in}}%
\pgfpathlineto{\pgfqpoint{5.143253in}{1.238500in}}%
\pgfpathlineto{\pgfqpoint{5.170952in}{1.238500in}}%
\pgfpathlineto{\pgfqpoint{5.198651in}{1.238500in}}%
\pgfpathlineto{\pgfqpoint{5.226351in}{1.238500in}}%
\pgfpathlineto{\pgfqpoint{5.254050in}{1.238500in}}%
\pgfpathlineto{\pgfqpoint{5.281749in}{1.238500in}}%
\pgfpathlineto{\pgfqpoint{5.309448in}{1.238500in}}%
\pgfpathlineto{\pgfqpoint{5.337148in}{1.238500in}}%
\pgfpathlineto{\pgfqpoint{5.364847in}{1.238500in}}%
\pgfpathlineto{\pgfqpoint{5.392546in}{1.238500in}}%
\pgfpathlineto{\pgfqpoint{5.420246in}{1.238500in}}%
\pgfpathlineto{\pgfqpoint{5.447945in}{1.238500in}}%
\pgfpathlineto{\pgfqpoint{5.475644in}{1.238500in}}%
\pgfpathlineto{\pgfqpoint{5.503343in}{1.238500in}}%
\pgfpathlineto{\pgfqpoint{5.531043in}{1.238500in}}%
\pgfpathlineto{\pgfqpoint{5.558742in}{1.238500in}}%
\pgfpathlineto{\pgfqpoint{5.586441in}{1.238500in}}%
\pgfpathlineto{\pgfqpoint{5.614141in}{1.238500in}}%
\pgfpathlineto{\pgfqpoint{5.641840in}{1.238500in}}%
\pgfpathlineto{\pgfqpoint{5.669539in}{1.238500in}}%
\pgfpathlineto{\pgfqpoint{5.697238in}{1.238500in}}%
\pgfpathlineto{\pgfqpoint{5.724938in}{1.238500in}}%
\pgfpathlineto{\pgfqpoint{5.752637in}{1.238500in}}%
\pgfpathlineto{\pgfqpoint{5.780336in}{1.238500in}}%
\pgfpathlineto{\pgfqpoint{5.808036in}{1.238500in}}%
\pgfpathlineto{\pgfqpoint{5.835735in}{1.238500in}}%
\pgfpathlineto{\pgfqpoint{5.863434in}{1.238500in}}%
\pgfpathlineto{\pgfqpoint{5.891133in}{1.238500in}}%
\pgfpathlineto{\pgfqpoint{5.918833in}{1.238500in}}%
\pgfpathlineto{\pgfqpoint{5.946532in}{1.238500in}}%
\pgfpathlineto{\pgfqpoint{5.974231in}{1.238500in}}%
\pgfpathlineto{\pgfqpoint{6.001931in}{1.238500in}}%
\pgfpathlineto{\pgfqpoint{6.029630in}{1.238500in}}%
\pgfpathlineto{\pgfqpoint{6.057329in}{1.238500in}}%
\pgfpathlineto{\pgfqpoint{6.067463in}{1.238500in}}%
\pgfusepath{stroke}%
\end{pgfscope}%
\begin{pgfscope}%
\pgfpathrectangle{\pgfqpoint{4.701213in}{0.383578in}}{\pgfqpoint{1.356250in}{1.540000in}}%
\pgfusepath{clip}%
\pgfsetrectcap%
\pgfsetroundjoin%
\pgfsetlinewidth{0.803000pt}%
\definecolor{currentstroke}{rgb}{0.686275,0.352941,0.313725}%
\pgfsetstrokecolor{currentstroke}%
\pgfsetstrokeopacity{0.300000}%
\pgfsetdash{}{0pt}%
\pgfpathmoveto{\pgfqpoint{4.755463in}{1.238500in}}%
\pgfpathlineto{\pgfqpoint{4.783162in}{1.238500in}}%
\pgfpathlineto{\pgfqpoint{4.810861in}{1.238500in}}%
\pgfpathlineto{\pgfqpoint{4.838561in}{1.238500in}}%
\pgfpathlineto{\pgfqpoint{4.866260in}{1.238500in}}%
\pgfpathlineto{\pgfqpoint{4.893959in}{1.238500in}}%
\pgfpathlineto{\pgfqpoint{4.921658in}{1.238500in}}%
\pgfpathlineto{\pgfqpoint{4.949358in}{1.238500in}}%
\pgfpathlineto{\pgfqpoint{4.977057in}{1.238500in}}%
\pgfpathlineto{\pgfqpoint{5.004756in}{1.238500in}}%
\pgfpathlineto{\pgfqpoint{5.032456in}{1.238500in}}%
\pgfpathlineto{\pgfqpoint{5.060155in}{1.238500in}}%
\pgfpathlineto{\pgfqpoint{5.087854in}{1.238500in}}%
\pgfpathlineto{\pgfqpoint{5.115553in}{1.238500in}}%
\pgfpathlineto{\pgfqpoint{5.143253in}{1.238500in}}%
\pgfpathlineto{\pgfqpoint{5.170952in}{1.238500in}}%
\pgfpathlineto{\pgfqpoint{5.198651in}{1.238500in}}%
\pgfpathlineto{\pgfqpoint{5.226351in}{1.238500in}}%
\pgfpathlineto{\pgfqpoint{5.254050in}{1.238500in}}%
\pgfpathlineto{\pgfqpoint{5.281749in}{1.238500in}}%
\pgfpathlineto{\pgfqpoint{5.309448in}{1.238500in}}%
\pgfpathlineto{\pgfqpoint{5.337148in}{1.238500in}}%
\pgfpathlineto{\pgfqpoint{5.364847in}{1.238500in}}%
\pgfpathlineto{\pgfqpoint{5.392546in}{1.238500in}}%
\pgfpathlineto{\pgfqpoint{5.420246in}{1.238500in}}%
\pgfpathlineto{\pgfqpoint{5.447945in}{1.238500in}}%
\pgfpathlineto{\pgfqpoint{5.475644in}{1.238500in}}%
\pgfpathlineto{\pgfqpoint{5.503343in}{1.238500in}}%
\pgfpathlineto{\pgfqpoint{5.531043in}{1.238500in}}%
\pgfpathlineto{\pgfqpoint{5.558742in}{1.238500in}}%
\pgfpathlineto{\pgfqpoint{5.586441in}{1.238500in}}%
\pgfpathlineto{\pgfqpoint{5.614141in}{1.238500in}}%
\pgfpathlineto{\pgfqpoint{5.641840in}{1.238500in}}%
\pgfpathlineto{\pgfqpoint{5.669539in}{1.238500in}}%
\pgfpathlineto{\pgfqpoint{5.697238in}{1.238500in}}%
\pgfpathlineto{\pgfqpoint{5.724938in}{1.238500in}}%
\pgfpathlineto{\pgfqpoint{5.752637in}{1.238500in}}%
\pgfpathlineto{\pgfqpoint{5.780336in}{1.238500in}}%
\pgfpathlineto{\pgfqpoint{5.808036in}{1.238500in}}%
\pgfpathlineto{\pgfqpoint{5.835735in}{1.238500in}}%
\pgfpathlineto{\pgfqpoint{5.863434in}{1.238500in}}%
\pgfpathlineto{\pgfqpoint{5.891133in}{1.238500in}}%
\pgfpathlineto{\pgfqpoint{5.918833in}{1.238500in}}%
\pgfpathlineto{\pgfqpoint{5.946532in}{1.238500in}}%
\pgfpathlineto{\pgfqpoint{5.974231in}{1.238500in}}%
\pgfpathlineto{\pgfqpoint{6.001931in}{1.238500in}}%
\pgfpathlineto{\pgfqpoint{6.029630in}{1.238500in}}%
\pgfpathlineto{\pgfqpoint{6.057329in}{1.238500in}}%
\pgfpathlineto{\pgfqpoint{6.067463in}{1.238500in}}%
\pgfusepath{stroke}%
\end{pgfscope}%
\begin{pgfscope}%
\pgfpathrectangle{\pgfqpoint{4.701213in}{0.383578in}}{\pgfqpoint{1.356250in}{1.540000in}}%
\pgfusepath{clip}%
\pgfsetrectcap%
\pgfsetroundjoin%
\pgfsetlinewidth{0.803000pt}%
\definecolor{currentstroke}{rgb}{0.686275,0.352941,0.313725}%
\pgfsetstrokecolor{currentstroke}%
\pgfsetstrokeopacity{0.300000}%
\pgfsetdash{}{0pt}%
\pgfpathmoveto{\pgfqpoint{4.755463in}{1.238500in}}%
\pgfpathlineto{\pgfqpoint{4.783162in}{1.238500in}}%
\pgfpathlineto{\pgfqpoint{4.810861in}{1.238500in}}%
\pgfpathlineto{\pgfqpoint{4.838561in}{1.238500in}}%
\pgfpathlineto{\pgfqpoint{4.866260in}{1.238500in}}%
\pgfpathlineto{\pgfqpoint{4.893959in}{1.238500in}}%
\pgfpathlineto{\pgfqpoint{4.921658in}{1.238500in}}%
\pgfpathlineto{\pgfqpoint{4.949358in}{1.238500in}}%
\pgfpathlineto{\pgfqpoint{4.977057in}{1.238500in}}%
\pgfpathlineto{\pgfqpoint{5.004756in}{1.238500in}}%
\pgfpathlineto{\pgfqpoint{5.032456in}{1.238500in}}%
\pgfpathlineto{\pgfqpoint{5.060155in}{1.238500in}}%
\pgfpathlineto{\pgfqpoint{5.087854in}{1.238500in}}%
\pgfpathlineto{\pgfqpoint{5.115553in}{1.238500in}}%
\pgfpathlineto{\pgfqpoint{5.143253in}{1.238500in}}%
\pgfpathlineto{\pgfqpoint{5.170952in}{1.238500in}}%
\pgfpathlineto{\pgfqpoint{5.198651in}{1.238500in}}%
\pgfpathlineto{\pgfqpoint{5.226351in}{1.238500in}}%
\pgfpathlineto{\pgfqpoint{5.254050in}{1.238500in}}%
\pgfpathlineto{\pgfqpoint{5.281749in}{1.238500in}}%
\pgfpathlineto{\pgfqpoint{5.309448in}{1.238500in}}%
\pgfpathlineto{\pgfqpoint{5.337148in}{1.238500in}}%
\pgfpathlineto{\pgfqpoint{5.364847in}{1.238500in}}%
\pgfpathlineto{\pgfqpoint{5.392546in}{1.238500in}}%
\pgfpathlineto{\pgfqpoint{5.420246in}{1.238500in}}%
\pgfpathlineto{\pgfqpoint{5.447945in}{1.238500in}}%
\pgfpathlineto{\pgfqpoint{5.475644in}{1.238500in}}%
\pgfpathlineto{\pgfqpoint{5.503343in}{1.238500in}}%
\pgfpathlineto{\pgfqpoint{5.531043in}{1.238500in}}%
\pgfpathlineto{\pgfqpoint{5.558742in}{1.238500in}}%
\pgfpathlineto{\pgfqpoint{5.586441in}{1.238500in}}%
\pgfpathlineto{\pgfqpoint{5.614141in}{1.238500in}}%
\pgfpathlineto{\pgfqpoint{5.641840in}{1.238500in}}%
\pgfpathlineto{\pgfqpoint{5.669539in}{1.238500in}}%
\pgfpathlineto{\pgfqpoint{5.697238in}{1.238500in}}%
\pgfpathlineto{\pgfqpoint{5.724938in}{1.238500in}}%
\pgfpathlineto{\pgfqpoint{5.752637in}{1.238500in}}%
\pgfpathlineto{\pgfqpoint{5.780336in}{1.238500in}}%
\pgfpathlineto{\pgfqpoint{5.808036in}{1.238500in}}%
\pgfpathlineto{\pgfqpoint{5.835735in}{1.238500in}}%
\pgfpathlineto{\pgfqpoint{5.863434in}{1.238500in}}%
\pgfpathlineto{\pgfqpoint{5.891133in}{1.238500in}}%
\pgfpathlineto{\pgfqpoint{5.918833in}{1.238500in}}%
\pgfpathlineto{\pgfqpoint{5.946532in}{1.238500in}}%
\pgfpathlineto{\pgfqpoint{5.974231in}{1.238500in}}%
\pgfpathlineto{\pgfqpoint{6.001931in}{1.238500in}}%
\pgfpathlineto{\pgfqpoint{6.029630in}{1.238500in}}%
\pgfpathlineto{\pgfqpoint{6.057329in}{1.238500in}}%
\pgfpathlineto{\pgfqpoint{6.067463in}{1.238500in}}%
\pgfusepath{stroke}%
\end{pgfscope}%
\begin{pgfscope}%
\pgfpathrectangle{\pgfqpoint{4.701213in}{0.383578in}}{\pgfqpoint{1.356250in}{1.540000in}}%
\pgfusepath{clip}%
\pgfsetrectcap%
\pgfsetroundjoin%
\pgfsetlinewidth{0.803000pt}%
\definecolor{currentstroke}{rgb}{0.686275,0.352941,0.313725}%
\pgfsetstrokecolor{currentstroke}%
\pgfsetstrokeopacity{0.300000}%
\pgfsetdash{}{0pt}%
\pgfpathmoveto{\pgfqpoint{4.755463in}{1.238500in}}%
\pgfpathlineto{\pgfqpoint{4.783162in}{1.238500in}}%
\pgfpathlineto{\pgfqpoint{4.810861in}{1.238500in}}%
\pgfpathlineto{\pgfqpoint{4.838561in}{1.238500in}}%
\pgfpathlineto{\pgfqpoint{4.866260in}{1.238500in}}%
\pgfpathlineto{\pgfqpoint{4.893959in}{1.238500in}}%
\pgfpathlineto{\pgfqpoint{4.921658in}{1.238500in}}%
\pgfpathlineto{\pgfqpoint{4.949358in}{1.238500in}}%
\pgfpathlineto{\pgfqpoint{4.977057in}{1.238500in}}%
\pgfpathlineto{\pgfqpoint{5.004756in}{1.238500in}}%
\pgfpathlineto{\pgfqpoint{5.032456in}{1.238500in}}%
\pgfpathlineto{\pgfqpoint{5.060155in}{1.238500in}}%
\pgfpathlineto{\pgfqpoint{5.087854in}{1.238500in}}%
\pgfpathlineto{\pgfqpoint{5.115553in}{1.238500in}}%
\pgfpathlineto{\pgfqpoint{5.143253in}{1.238500in}}%
\pgfpathlineto{\pgfqpoint{5.170952in}{1.238500in}}%
\pgfpathlineto{\pgfqpoint{5.198651in}{1.238500in}}%
\pgfpathlineto{\pgfqpoint{5.226351in}{1.238500in}}%
\pgfpathlineto{\pgfqpoint{5.254050in}{1.238500in}}%
\pgfpathlineto{\pgfqpoint{5.281749in}{1.238500in}}%
\pgfpathlineto{\pgfqpoint{5.309448in}{1.238500in}}%
\pgfpathlineto{\pgfqpoint{5.337148in}{1.238500in}}%
\pgfpathlineto{\pgfqpoint{5.364847in}{1.238500in}}%
\pgfpathlineto{\pgfqpoint{5.392546in}{1.238500in}}%
\pgfpathlineto{\pgfqpoint{5.420246in}{1.238500in}}%
\pgfpathlineto{\pgfqpoint{5.447945in}{1.238500in}}%
\pgfpathlineto{\pgfqpoint{5.475644in}{1.238500in}}%
\pgfpathlineto{\pgfqpoint{5.503343in}{1.238500in}}%
\pgfpathlineto{\pgfqpoint{5.531043in}{1.238500in}}%
\pgfpathlineto{\pgfqpoint{5.558742in}{1.238500in}}%
\pgfpathlineto{\pgfqpoint{5.586441in}{1.238500in}}%
\pgfpathlineto{\pgfqpoint{5.614141in}{1.238500in}}%
\pgfpathlineto{\pgfqpoint{5.641840in}{1.238500in}}%
\pgfpathlineto{\pgfqpoint{5.669539in}{1.238500in}}%
\pgfpathlineto{\pgfqpoint{5.697238in}{1.238500in}}%
\pgfpathlineto{\pgfqpoint{5.724938in}{1.238500in}}%
\pgfpathlineto{\pgfqpoint{5.752637in}{1.238500in}}%
\pgfpathlineto{\pgfqpoint{5.780336in}{1.238500in}}%
\pgfpathlineto{\pgfqpoint{5.808036in}{1.238500in}}%
\pgfpathlineto{\pgfqpoint{5.835735in}{1.238500in}}%
\pgfpathlineto{\pgfqpoint{5.863434in}{1.238500in}}%
\pgfpathlineto{\pgfqpoint{5.891133in}{1.238500in}}%
\pgfpathlineto{\pgfqpoint{5.918833in}{1.238500in}}%
\pgfpathlineto{\pgfqpoint{5.946532in}{1.238500in}}%
\pgfpathlineto{\pgfqpoint{5.974231in}{1.238500in}}%
\pgfpathlineto{\pgfqpoint{6.001931in}{1.238500in}}%
\pgfpathlineto{\pgfqpoint{6.029630in}{1.238500in}}%
\pgfpathlineto{\pgfqpoint{6.057329in}{1.238500in}}%
\pgfpathlineto{\pgfqpoint{6.067463in}{1.238500in}}%
\pgfusepath{stroke}%
\end{pgfscope}%
\begin{pgfscope}%
\pgfpathrectangle{\pgfqpoint{4.701213in}{0.383578in}}{\pgfqpoint{1.356250in}{1.540000in}}%
\pgfusepath{clip}%
\pgfsetrectcap%
\pgfsetroundjoin%
\pgfsetlinewidth{0.803000pt}%
\definecolor{currentstroke}{rgb}{0.686275,0.352941,0.313725}%
\pgfsetstrokecolor{currentstroke}%
\pgfsetstrokeopacity{0.300000}%
\pgfsetdash{}{0pt}%
\pgfpathmoveto{\pgfqpoint{4.755463in}{1.238500in}}%
\pgfpathlineto{\pgfqpoint{4.783162in}{1.238500in}}%
\pgfpathlineto{\pgfqpoint{4.810861in}{1.238500in}}%
\pgfpathlineto{\pgfqpoint{4.838561in}{1.238500in}}%
\pgfpathlineto{\pgfqpoint{4.866260in}{1.238500in}}%
\pgfpathlineto{\pgfqpoint{4.893959in}{1.238500in}}%
\pgfpathlineto{\pgfqpoint{4.921658in}{1.238500in}}%
\pgfpathlineto{\pgfqpoint{4.949358in}{1.238500in}}%
\pgfpathlineto{\pgfqpoint{4.977057in}{1.238500in}}%
\pgfpathlineto{\pgfqpoint{5.004756in}{1.238500in}}%
\pgfpathlineto{\pgfqpoint{5.032456in}{1.238500in}}%
\pgfpathlineto{\pgfqpoint{5.060155in}{1.238500in}}%
\pgfpathlineto{\pgfqpoint{5.087854in}{1.238500in}}%
\pgfpathlineto{\pgfqpoint{5.115553in}{1.238500in}}%
\pgfpathlineto{\pgfqpoint{5.143253in}{1.238500in}}%
\pgfpathlineto{\pgfqpoint{5.170952in}{1.238500in}}%
\pgfpathlineto{\pgfqpoint{5.198651in}{1.238500in}}%
\pgfpathlineto{\pgfqpoint{5.226351in}{1.238500in}}%
\pgfpathlineto{\pgfqpoint{5.254050in}{1.238500in}}%
\pgfpathlineto{\pgfqpoint{5.281749in}{1.238500in}}%
\pgfpathlineto{\pgfqpoint{5.309448in}{1.238500in}}%
\pgfpathlineto{\pgfqpoint{5.337148in}{1.238500in}}%
\pgfpathlineto{\pgfqpoint{5.364847in}{1.238500in}}%
\pgfpathlineto{\pgfqpoint{5.392546in}{1.238500in}}%
\pgfpathlineto{\pgfqpoint{5.420246in}{1.238500in}}%
\pgfpathlineto{\pgfqpoint{5.447945in}{1.238500in}}%
\pgfpathlineto{\pgfqpoint{5.475644in}{1.238500in}}%
\pgfpathlineto{\pgfqpoint{5.503343in}{1.238500in}}%
\pgfpathlineto{\pgfqpoint{5.531043in}{1.238500in}}%
\pgfpathlineto{\pgfqpoint{5.558742in}{1.238500in}}%
\pgfpathlineto{\pgfqpoint{5.586441in}{1.238500in}}%
\pgfpathlineto{\pgfqpoint{5.614141in}{1.238500in}}%
\pgfpathlineto{\pgfqpoint{5.641840in}{1.238500in}}%
\pgfpathlineto{\pgfqpoint{5.669539in}{1.238500in}}%
\pgfpathlineto{\pgfqpoint{5.697238in}{1.238500in}}%
\pgfpathlineto{\pgfqpoint{5.724938in}{1.238500in}}%
\pgfpathlineto{\pgfqpoint{5.752637in}{1.238500in}}%
\pgfpathlineto{\pgfqpoint{5.780336in}{1.238500in}}%
\pgfpathlineto{\pgfqpoint{5.808036in}{1.238500in}}%
\pgfpathlineto{\pgfqpoint{5.835735in}{1.238500in}}%
\pgfpathlineto{\pgfqpoint{5.863434in}{1.238500in}}%
\pgfpathlineto{\pgfqpoint{5.891133in}{1.238500in}}%
\pgfpathlineto{\pgfqpoint{5.918833in}{1.238500in}}%
\pgfpathlineto{\pgfqpoint{5.946532in}{1.238500in}}%
\pgfpathlineto{\pgfqpoint{5.974231in}{1.238500in}}%
\pgfpathlineto{\pgfqpoint{6.001931in}{1.238500in}}%
\pgfpathlineto{\pgfqpoint{6.029630in}{1.238500in}}%
\pgfpathlineto{\pgfqpoint{6.057329in}{1.238500in}}%
\pgfpathlineto{\pgfqpoint{6.067463in}{1.238500in}}%
\pgfusepath{stroke}%
\end{pgfscope}%
\begin{pgfscope}%
\pgfpathrectangle{\pgfqpoint{4.701213in}{0.383578in}}{\pgfqpoint{1.356250in}{1.540000in}}%
\pgfusepath{clip}%
\pgfsetrectcap%
\pgfsetroundjoin%
\pgfsetlinewidth{0.803000pt}%
\definecolor{currentstroke}{rgb}{0.686275,0.352941,0.313725}%
\pgfsetstrokecolor{currentstroke}%
\pgfsetstrokeopacity{0.300000}%
\pgfsetdash{}{0pt}%
\pgfpathmoveto{\pgfqpoint{4.755463in}{1.238500in}}%
\pgfpathlineto{\pgfqpoint{4.783162in}{1.238500in}}%
\pgfpathlineto{\pgfqpoint{4.810861in}{1.238500in}}%
\pgfpathlineto{\pgfqpoint{4.838561in}{1.238500in}}%
\pgfpathlineto{\pgfqpoint{4.866260in}{1.238500in}}%
\pgfpathlineto{\pgfqpoint{4.893959in}{1.238500in}}%
\pgfpathlineto{\pgfqpoint{4.921658in}{1.238500in}}%
\pgfpathlineto{\pgfqpoint{4.949358in}{1.238500in}}%
\pgfpathlineto{\pgfqpoint{4.977057in}{1.238500in}}%
\pgfpathlineto{\pgfqpoint{5.004756in}{1.238500in}}%
\pgfpathlineto{\pgfqpoint{5.032456in}{1.238500in}}%
\pgfpathlineto{\pgfqpoint{5.060155in}{1.238500in}}%
\pgfpathlineto{\pgfqpoint{5.087854in}{1.238500in}}%
\pgfpathlineto{\pgfqpoint{5.115553in}{1.238500in}}%
\pgfpathlineto{\pgfqpoint{5.143253in}{1.238500in}}%
\pgfpathlineto{\pgfqpoint{5.170952in}{1.238500in}}%
\pgfpathlineto{\pgfqpoint{5.198651in}{1.238500in}}%
\pgfpathlineto{\pgfqpoint{5.226351in}{1.238500in}}%
\pgfpathlineto{\pgfqpoint{5.254050in}{1.238500in}}%
\pgfpathlineto{\pgfqpoint{5.281749in}{1.238500in}}%
\pgfpathlineto{\pgfqpoint{5.309448in}{1.238500in}}%
\pgfpathlineto{\pgfqpoint{5.337148in}{1.238500in}}%
\pgfpathlineto{\pgfqpoint{5.364847in}{1.238500in}}%
\pgfpathlineto{\pgfqpoint{5.392546in}{1.238500in}}%
\pgfpathlineto{\pgfqpoint{5.420246in}{1.238500in}}%
\pgfpathlineto{\pgfqpoint{5.447945in}{1.238500in}}%
\pgfpathlineto{\pgfqpoint{5.475644in}{1.238500in}}%
\pgfpathlineto{\pgfqpoint{5.503343in}{1.238500in}}%
\pgfpathlineto{\pgfqpoint{5.531043in}{1.238500in}}%
\pgfpathlineto{\pgfqpoint{5.558742in}{1.238500in}}%
\pgfpathlineto{\pgfqpoint{5.586441in}{1.238500in}}%
\pgfpathlineto{\pgfqpoint{5.614141in}{1.238500in}}%
\pgfpathlineto{\pgfqpoint{5.641840in}{1.238500in}}%
\pgfpathlineto{\pgfqpoint{5.669539in}{1.238500in}}%
\pgfpathlineto{\pgfqpoint{5.697238in}{1.238500in}}%
\pgfpathlineto{\pgfqpoint{5.724938in}{1.238500in}}%
\pgfpathlineto{\pgfqpoint{5.752637in}{1.238500in}}%
\pgfpathlineto{\pgfqpoint{5.780336in}{1.238500in}}%
\pgfpathlineto{\pgfqpoint{5.808036in}{1.238500in}}%
\pgfpathlineto{\pgfqpoint{5.835735in}{1.238500in}}%
\pgfpathlineto{\pgfqpoint{5.863434in}{1.238500in}}%
\pgfpathlineto{\pgfqpoint{5.891133in}{1.238500in}}%
\pgfpathlineto{\pgfqpoint{5.918833in}{1.238500in}}%
\pgfpathlineto{\pgfqpoint{5.946532in}{1.238500in}}%
\pgfpathlineto{\pgfqpoint{5.974231in}{1.238500in}}%
\pgfpathlineto{\pgfqpoint{6.001931in}{1.238500in}}%
\pgfpathlineto{\pgfqpoint{6.029630in}{1.238500in}}%
\pgfpathlineto{\pgfqpoint{6.057329in}{1.238500in}}%
\pgfpathlineto{\pgfqpoint{6.067463in}{1.238500in}}%
\pgfusepath{stroke}%
\end{pgfscope}%
\begin{pgfscope}%
\pgfpathrectangle{\pgfqpoint{4.701213in}{0.383578in}}{\pgfqpoint{1.356250in}{1.540000in}}%
\pgfusepath{clip}%
\pgfsetrectcap%
\pgfsetroundjoin%
\pgfsetlinewidth{0.803000pt}%
\definecolor{currentstroke}{rgb}{0.686275,0.352941,0.313725}%
\pgfsetstrokecolor{currentstroke}%
\pgfsetstrokeopacity{0.300000}%
\pgfsetdash{}{0pt}%
\pgfpathmoveto{\pgfqpoint{4.755463in}{1.238500in}}%
\pgfpathlineto{\pgfqpoint{4.783162in}{1.238500in}}%
\pgfpathlineto{\pgfqpoint{4.810861in}{1.238500in}}%
\pgfpathlineto{\pgfqpoint{4.838561in}{1.238500in}}%
\pgfpathlineto{\pgfqpoint{4.866260in}{1.238500in}}%
\pgfpathlineto{\pgfqpoint{4.893959in}{1.238500in}}%
\pgfpathlineto{\pgfqpoint{4.921658in}{1.238500in}}%
\pgfpathlineto{\pgfqpoint{4.949358in}{1.238500in}}%
\pgfpathlineto{\pgfqpoint{4.977057in}{1.238500in}}%
\pgfpathlineto{\pgfqpoint{5.004756in}{1.238500in}}%
\pgfpathlineto{\pgfqpoint{5.032456in}{1.238500in}}%
\pgfpathlineto{\pgfqpoint{5.060155in}{1.238500in}}%
\pgfpathlineto{\pgfqpoint{5.087854in}{1.238500in}}%
\pgfpathlineto{\pgfqpoint{5.115553in}{1.238500in}}%
\pgfpathlineto{\pgfqpoint{5.143253in}{1.238500in}}%
\pgfpathlineto{\pgfqpoint{5.170952in}{1.238500in}}%
\pgfpathlineto{\pgfqpoint{5.198651in}{1.238500in}}%
\pgfpathlineto{\pgfqpoint{5.226351in}{1.238500in}}%
\pgfpathlineto{\pgfqpoint{5.254050in}{1.238500in}}%
\pgfpathlineto{\pgfqpoint{5.281749in}{1.238500in}}%
\pgfpathlineto{\pgfqpoint{5.309448in}{1.238500in}}%
\pgfpathlineto{\pgfqpoint{5.337148in}{1.238500in}}%
\pgfpathlineto{\pgfqpoint{5.364847in}{1.238500in}}%
\pgfpathlineto{\pgfqpoint{5.392546in}{1.238500in}}%
\pgfpathlineto{\pgfqpoint{5.420246in}{1.238500in}}%
\pgfpathlineto{\pgfqpoint{5.447945in}{1.238500in}}%
\pgfpathlineto{\pgfqpoint{5.475644in}{1.238500in}}%
\pgfpathlineto{\pgfqpoint{5.503343in}{1.238500in}}%
\pgfpathlineto{\pgfqpoint{5.531043in}{1.238500in}}%
\pgfpathlineto{\pgfqpoint{5.558742in}{1.238500in}}%
\pgfpathlineto{\pgfqpoint{5.586441in}{1.238500in}}%
\pgfpathlineto{\pgfqpoint{5.614141in}{1.238500in}}%
\pgfpathlineto{\pgfqpoint{5.641840in}{1.238500in}}%
\pgfpathlineto{\pgfqpoint{5.669539in}{1.238500in}}%
\pgfpathlineto{\pgfqpoint{5.697238in}{1.238500in}}%
\pgfpathlineto{\pgfqpoint{5.724938in}{1.238500in}}%
\pgfpathlineto{\pgfqpoint{5.752637in}{1.238500in}}%
\pgfpathlineto{\pgfqpoint{5.780336in}{1.238500in}}%
\pgfpathlineto{\pgfqpoint{5.808036in}{1.238500in}}%
\pgfpathlineto{\pgfqpoint{5.835735in}{1.238500in}}%
\pgfpathlineto{\pgfqpoint{5.863434in}{1.238500in}}%
\pgfpathlineto{\pgfqpoint{5.891133in}{1.238500in}}%
\pgfpathlineto{\pgfqpoint{5.918833in}{1.238500in}}%
\pgfpathlineto{\pgfqpoint{5.946532in}{1.238500in}}%
\pgfpathlineto{\pgfqpoint{5.974231in}{1.238500in}}%
\pgfpathlineto{\pgfqpoint{6.001931in}{1.238500in}}%
\pgfpathlineto{\pgfqpoint{6.029630in}{1.238500in}}%
\pgfpathlineto{\pgfqpoint{6.057329in}{1.238500in}}%
\pgfpathlineto{\pgfqpoint{6.067463in}{1.238500in}}%
\pgfusepath{stroke}%
\end{pgfscope}%
\begin{pgfscope}%
\pgfpathrectangle{\pgfqpoint{4.701213in}{0.383578in}}{\pgfqpoint{1.356250in}{1.540000in}}%
\pgfusepath{clip}%
\pgfsetrectcap%
\pgfsetroundjoin%
\pgfsetlinewidth{0.803000pt}%
\definecolor{currentstroke}{rgb}{0.686275,0.352941,0.313725}%
\pgfsetstrokecolor{currentstroke}%
\pgfsetstrokeopacity{0.300000}%
\pgfsetdash{}{0pt}%
\pgfpathmoveto{\pgfqpoint{4.755463in}{1.238500in}}%
\pgfpathlineto{\pgfqpoint{4.783162in}{1.238500in}}%
\pgfpathlineto{\pgfqpoint{4.810861in}{1.238500in}}%
\pgfpathlineto{\pgfqpoint{4.838561in}{1.238500in}}%
\pgfpathlineto{\pgfqpoint{4.866260in}{1.238500in}}%
\pgfpathlineto{\pgfqpoint{4.893959in}{1.238500in}}%
\pgfpathlineto{\pgfqpoint{4.921658in}{1.238500in}}%
\pgfpathlineto{\pgfqpoint{4.949358in}{1.238500in}}%
\pgfpathlineto{\pgfqpoint{4.977057in}{1.238500in}}%
\pgfpathlineto{\pgfqpoint{5.004756in}{1.238500in}}%
\pgfpathlineto{\pgfqpoint{5.032456in}{1.238500in}}%
\pgfpathlineto{\pgfqpoint{5.060155in}{1.238500in}}%
\pgfpathlineto{\pgfqpoint{5.087854in}{1.238500in}}%
\pgfpathlineto{\pgfqpoint{5.115553in}{1.238500in}}%
\pgfpathlineto{\pgfqpoint{5.143253in}{1.351588in}}%
\pgfpathlineto{\pgfqpoint{5.170952in}{1.412421in}}%
\pgfpathlineto{\pgfqpoint{5.198651in}{1.436042in}}%
\pgfpathlineto{\pgfqpoint{5.226351in}{1.435547in}}%
\pgfpathlineto{\pgfqpoint{5.254050in}{1.420972in}}%
\pgfpathlineto{\pgfqpoint{5.281749in}{1.399419in}}%
\pgfpathlineto{\pgfqpoint{5.309448in}{1.375550in}}%
\pgfpathlineto{\pgfqpoint{5.337148in}{1.352186in}}%
\pgfpathlineto{\pgfqpoint{5.364847in}{1.330985in}}%
\pgfpathlineto{\pgfqpoint{5.392546in}{1.312677in}}%
\pgfpathlineto{\pgfqpoint{5.420246in}{1.297367in}}%
\pgfpathlineto{\pgfqpoint{5.447945in}{1.284870in}}%
\pgfpathlineto{\pgfqpoint{5.475644in}{1.274834in}}%
\pgfpathlineto{\pgfqpoint{5.503343in}{1.266838in}}%
\pgfpathlineto{\pgfqpoint{5.531043in}{1.260523in}}%
\pgfpathlineto{\pgfqpoint{5.558742in}{1.255557in}}%
\pgfpathlineto{\pgfqpoint{5.586441in}{1.251668in}}%
\pgfpathlineto{\pgfqpoint{5.614141in}{1.248634in}}%
\pgfpathlineto{\pgfqpoint{5.641840in}{1.246275in}}%
\pgfpathlineto{\pgfqpoint{5.669539in}{1.244447in}}%
\pgfpathlineto{\pgfqpoint{5.697238in}{1.243036in}}%
\pgfpathlineto{\pgfqpoint{5.724938in}{1.241949in}}%
\pgfpathlineto{\pgfqpoint{5.752637in}{1.241117in}}%
\pgfpathlineto{\pgfqpoint{5.780336in}{1.240481in}}%
\pgfpathlineto{\pgfqpoint{5.808036in}{1.239996in}}%
\pgfpathlineto{\pgfqpoint{5.835735in}{1.239626in}}%
\pgfpathlineto{\pgfqpoint{5.863434in}{1.239346in}}%
\pgfpathlineto{\pgfqpoint{5.891133in}{1.239135in}}%
\pgfpathlineto{\pgfqpoint{5.918833in}{1.238975in}}%
\pgfpathlineto{\pgfqpoint{5.946532in}{1.238855in}}%
\pgfpathlineto{\pgfqpoint{5.974231in}{1.238764in}}%
\pgfpathlineto{\pgfqpoint{6.001931in}{1.238697in}}%
\pgfpathlineto{\pgfqpoint{6.029630in}{1.238646in}}%
\pgfpathlineto{\pgfqpoint{6.057329in}{1.238608in}}%
\pgfpathlineto{\pgfqpoint{6.067463in}{1.238598in}}%
\pgfusepath{stroke}%
\end{pgfscope}%
\begin{pgfscope}%
\pgfpathrectangle{\pgfqpoint{4.701213in}{0.383578in}}{\pgfqpoint{1.356250in}{1.540000in}}%
\pgfusepath{clip}%
\pgfsetrectcap%
\pgfsetroundjoin%
\pgfsetlinewidth{0.803000pt}%
\definecolor{currentstroke}{rgb}{0.686275,0.352941,0.313725}%
\pgfsetstrokecolor{currentstroke}%
\pgfsetstrokeopacity{0.300000}%
\pgfsetdash{}{0pt}%
\pgfpathmoveto{\pgfqpoint{4.755463in}{1.238500in}}%
\pgfpathlineto{\pgfqpoint{4.783162in}{1.238500in}}%
\pgfpathlineto{\pgfqpoint{4.810861in}{1.238500in}}%
\pgfpathlineto{\pgfqpoint{4.838561in}{1.238500in}}%
\pgfpathlineto{\pgfqpoint{4.866260in}{1.238500in}}%
\pgfpathlineto{\pgfqpoint{4.893959in}{1.238500in}}%
\pgfpathlineto{\pgfqpoint{4.921658in}{1.238500in}}%
\pgfpathlineto{\pgfqpoint{4.949358in}{1.238500in}}%
\pgfpathlineto{\pgfqpoint{4.977057in}{1.238500in}}%
\pgfpathlineto{\pgfqpoint{5.004756in}{1.238500in}}%
\pgfpathlineto{\pgfqpoint{5.032456in}{1.238500in}}%
\pgfpathlineto{\pgfqpoint{5.060155in}{1.238500in}}%
\pgfpathlineto{\pgfqpoint{5.087854in}{1.238500in}}%
\pgfpathlineto{\pgfqpoint{5.115553in}{1.238500in}}%
\pgfpathlineto{\pgfqpoint{5.143253in}{1.239958in}}%
\pgfpathlineto{\pgfqpoint{5.170952in}{1.241396in}}%
\pgfpathlineto{\pgfqpoint{5.198651in}{1.242612in}}%
\pgfpathlineto{\pgfqpoint{5.226351in}{1.243504in}}%
\pgfpathlineto{\pgfqpoint{5.254050in}{1.244025in}}%
\pgfpathlineto{\pgfqpoint{5.281749in}{1.244208in}}%
\pgfpathlineto{\pgfqpoint{5.309448in}{1.244117in}}%
\pgfpathlineto{\pgfqpoint{5.337148in}{1.243811in}}%
\pgfpathlineto{\pgfqpoint{5.364847in}{1.243490in}}%
\pgfpathlineto{\pgfqpoint{5.392546in}{1.243252in}}%
\pgfpathlineto{\pgfqpoint{5.420246in}{1.243078in}}%
\pgfpathlineto{\pgfqpoint{5.447945in}{1.242946in}}%
\pgfpathlineto{\pgfqpoint{5.475644in}{1.242813in}}%
\pgfpathlineto{\pgfqpoint{5.503343in}{1.242615in}}%
\pgfpathlineto{\pgfqpoint{5.531043in}{1.242355in}}%
\pgfpathlineto{\pgfqpoint{5.558742in}{1.242035in}}%
\pgfpathlineto{\pgfqpoint{5.586441in}{1.241673in}}%
\pgfpathlineto{\pgfqpoint{5.614141in}{1.241293in}}%
\pgfpathlineto{\pgfqpoint{5.641840in}{1.240914in}}%
\pgfpathlineto{\pgfqpoint{5.669539in}{1.240552in}}%
\pgfpathlineto{\pgfqpoint{5.697238in}{1.240220in}}%
\pgfpathlineto{\pgfqpoint{5.724938in}{1.239923in}}%
\pgfpathlineto{\pgfqpoint{5.752637in}{1.239666in}}%
\pgfpathlineto{\pgfqpoint{5.780336in}{1.239445in}}%
\pgfpathlineto{\pgfqpoint{5.808036in}{1.239260in}}%
\pgfpathlineto{\pgfqpoint{5.835735in}{1.239106in}}%
\pgfpathlineto{\pgfqpoint{5.863434in}{1.238979in}}%
\pgfpathlineto{\pgfqpoint{5.891133in}{1.238877in}}%
\pgfpathlineto{\pgfqpoint{5.918833in}{1.238795in}}%
\pgfpathlineto{\pgfqpoint{5.946532in}{1.238729in}}%
\pgfpathlineto{\pgfqpoint{5.974231in}{1.238677in}}%
\pgfpathlineto{\pgfqpoint{6.001931in}{1.238636in}}%
\pgfpathlineto{\pgfqpoint{6.029630in}{1.238604in}}%
\pgfpathlineto{\pgfqpoint{6.057329in}{1.238579in}}%
\pgfpathlineto{\pgfqpoint{6.067463in}{1.238572in}}%
\pgfusepath{stroke}%
\end{pgfscope}%
\begin{pgfscope}%
\pgfpathrectangle{\pgfqpoint{4.701213in}{0.383578in}}{\pgfqpoint{1.356250in}{1.540000in}}%
\pgfusepath{clip}%
\pgfsetrectcap%
\pgfsetroundjoin%
\pgfsetlinewidth{0.803000pt}%
\definecolor{currentstroke}{rgb}{0.686275,0.352941,0.313725}%
\pgfsetstrokecolor{currentstroke}%
\pgfsetstrokeopacity{0.300000}%
\pgfsetdash{}{0pt}%
\pgfpathmoveto{\pgfqpoint{4.755463in}{1.238500in}}%
\pgfpathlineto{\pgfqpoint{4.783162in}{1.238500in}}%
\pgfpathlineto{\pgfqpoint{4.810861in}{1.238500in}}%
\pgfpathlineto{\pgfqpoint{4.838561in}{1.238500in}}%
\pgfpathlineto{\pgfqpoint{4.866260in}{1.238500in}}%
\pgfpathlineto{\pgfqpoint{4.893959in}{1.238500in}}%
\pgfpathlineto{\pgfqpoint{4.921658in}{1.238500in}}%
\pgfpathlineto{\pgfqpoint{4.949358in}{1.238500in}}%
\pgfpathlineto{\pgfqpoint{4.977057in}{1.238500in}}%
\pgfpathlineto{\pgfqpoint{5.004756in}{1.238500in}}%
\pgfpathlineto{\pgfqpoint{5.032456in}{1.238500in}}%
\pgfpathlineto{\pgfqpoint{5.060155in}{1.238500in}}%
\pgfpathlineto{\pgfqpoint{5.087854in}{1.238500in}}%
\pgfpathlineto{\pgfqpoint{5.115553in}{1.238500in}}%
\pgfpathlineto{\pgfqpoint{5.143253in}{1.596629in}}%
\pgfpathlineto{\pgfqpoint{5.170952in}{1.784329in}}%
\pgfpathlineto{\pgfqpoint{5.198651in}{1.853578in}}%
\pgfpathlineto{\pgfqpoint{5.226351in}{1.847638in}}%
\pgfpathlineto{\pgfqpoint{5.254050in}{1.798829in}}%
\pgfpathlineto{\pgfqpoint{5.281749in}{1.729519in}}%
\pgfpathlineto{\pgfqpoint{5.309448in}{1.654125in}}%
\pgfpathlineto{\pgfqpoint{5.337148in}{1.581216in}}%
\pgfpathlineto{\pgfqpoint{5.364847in}{1.515431in}}%
\pgfpathlineto{\pgfqpoint{5.392546in}{1.458725in}}%
\pgfpathlineto{\pgfqpoint{5.420246in}{1.411384in}}%
\pgfpathlineto{\pgfqpoint{5.447945in}{1.372804in}}%
\pgfpathlineto{\pgfqpoint{5.475644in}{1.341935in}}%
\pgfpathlineto{\pgfqpoint{5.503343in}{1.317571in}}%
\pgfpathlineto{\pgfqpoint{5.531043in}{1.298565in}}%
\pgfpathlineto{\pgfqpoint{5.558742in}{1.283874in}}%
\pgfpathlineto{\pgfqpoint{5.586441in}{1.272608in}}%
\pgfpathlineto{\pgfqpoint{5.614141in}{1.264026in}}%
\pgfpathlineto{\pgfqpoint{5.641840in}{1.257528in}}%
\pgfpathlineto{\pgfqpoint{5.669539in}{1.252633in}}%
\pgfpathlineto{\pgfqpoint{5.697238in}{1.248964in}}%
\pgfpathlineto{\pgfqpoint{5.724938in}{1.246223in}}%
\pgfpathlineto{\pgfqpoint{5.752637in}{1.244186in}}%
\pgfpathlineto{\pgfqpoint{5.780336in}{1.242676in}}%
\pgfpathlineto{\pgfqpoint{5.808036in}{1.241560in}}%
\pgfpathlineto{\pgfqpoint{5.835735in}{1.240738in}}%
\pgfpathlineto{\pgfqpoint{5.863434in}{1.240133in}}%
\pgfpathlineto{\pgfqpoint{5.891133in}{1.239690in}}%
\pgfpathlineto{\pgfqpoint{5.918833in}{1.239366in}}%
\pgfpathlineto{\pgfqpoint{5.946532in}{1.239129in}}%
\pgfpathlineto{\pgfqpoint{5.974231in}{1.238956in}}%
\pgfpathlineto{\pgfqpoint{6.001931in}{1.238830in}}%
\pgfpathlineto{\pgfqpoint{6.029630in}{1.238739in}}%
\pgfpathlineto{\pgfqpoint{6.057329in}{1.238672in}}%
\pgfpathlineto{\pgfqpoint{6.067463in}{1.238655in}}%
\pgfusepath{stroke}%
\end{pgfscope}%
\begin{pgfscope}%
\pgfpathrectangle{\pgfqpoint{4.701213in}{0.383578in}}{\pgfqpoint{1.356250in}{1.540000in}}%
\pgfusepath{clip}%
\pgfsetrectcap%
\pgfsetroundjoin%
\pgfsetlinewidth{0.803000pt}%
\definecolor{currentstroke}{rgb}{0.686275,0.352941,0.313725}%
\pgfsetstrokecolor{currentstroke}%
\pgfsetstrokeopacity{0.300000}%
\pgfsetdash{}{0pt}%
\pgfpathmoveto{\pgfqpoint{4.755463in}{1.238500in}}%
\pgfpathlineto{\pgfqpoint{4.783162in}{1.238500in}}%
\pgfpathlineto{\pgfqpoint{4.810861in}{1.238500in}}%
\pgfpathlineto{\pgfqpoint{4.838561in}{1.238500in}}%
\pgfpathlineto{\pgfqpoint{4.866260in}{1.238500in}}%
\pgfpathlineto{\pgfqpoint{4.893959in}{1.238500in}}%
\pgfpathlineto{\pgfqpoint{4.921658in}{1.238500in}}%
\pgfpathlineto{\pgfqpoint{4.949358in}{1.238500in}}%
\pgfpathlineto{\pgfqpoint{4.977057in}{1.238500in}}%
\pgfpathlineto{\pgfqpoint{5.004756in}{1.238500in}}%
\pgfpathlineto{\pgfqpoint{5.032456in}{1.238500in}}%
\pgfpathlineto{\pgfqpoint{5.060155in}{1.238500in}}%
\pgfpathlineto{\pgfqpoint{5.087854in}{1.238500in}}%
\pgfpathlineto{\pgfqpoint{5.115553in}{1.238500in}}%
\pgfpathlineto{\pgfqpoint{5.143253in}{1.238500in}}%
\pgfpathlineto{\pgfqpoint{5.170952in}{1.238500in}}%
\pgfpathlineto{\pgfqpoint{5.198651in}{1.238500in}}%
\pgfpathlineto{\pgfqpoint{5.226351in}{1.238500in}}%
\pgfpathlineto{\pgfqpoint{5.254050in}{1.238500in}}%
\pgfpathlineto{\pgfqpoint{5.281749in}{1.238500in}}%
\pgfpathlineto{\pgfqpoint{5.309448in}{1.238500in}}%
\pgfpathlineto{\pgfqpoint{5.337148in}{1.238500in}}%
\pgfpathlineto{\pgfqpoint{5.364847in}{1.238500in}}%
\pgfpathlineto{\pgfqpoint{5.392546in}{1.238500in}}%
\pgfpathlineto{\pgfqpoint{5.420246in}{1.238500in}}%
\pgfpathlineto{\pgfqpoint{5.447945in}{1.238500in}}%
\pgfpathlineto{\pgfqpoint{5.475644in}{1.238500in}}%
\pgfpathlineto{\pgfqpoint{5.503343in}{1.238500in}}%
\pgfpathlineto{\pgfqpoint{5.531043in}{1.238500in}}%
\pgfpathlineto{\pgfqpoint{5.558742in}{1.238500in}}%
\pgfpathlineto{\pgfqpoint{5.586441in}{1.238500in}}%
\pgfpathlineto{\pgfqpoint{5.614141in}{1.238500in}}%
\pgfpathlineto{\pgfqpoint{5.641840in}{1.238500in}}%
\pgfpathlineto{\pgfqpoint{5.669539in}{1.238500in}}%
\pgfpathlineto{\pgfqpoint{5.697238in}{1.238500in}}%
\pgfpathlineto{\pgfqpoint{5.724938in}{1.238500in}}%
\pgfpathlineto{\pgfqpoint{5.752637in}{1.238500in}}%
\pgfpathlineto{\pgfqpoint{5.780336in}{1.238500in}}%
\pgfpathlineto{\pgfqpoint{5.808036in}{1.238500in}}%
\pgfpathlineto{\pgfqpoint{5.835735in}{1.238500in}}%
\pgfpathlineto{\pgfqpoint{5.863434in}{1.238500in}}%
\pgfpathlineto{\pgfqpoint{5.891133in}{1.238500in}}%
\pgfpathlineto{\pgfqpoint{5.918833in}{1.238500in}}%
\pgfpathlineto{\pgfqpoint{5.946532in}{1.238500in}}%
\pgfpathlineto{\pgfqpoint{5.974231in}{1.238500in}}%
\pgfpathlineto{\pgfqpoint{6.001931in}{1.238500in}}%
\pgfpathlineto{\pgfqpoint{6.029630in}{1.238500in}}%
\pgfpathlineto{\pgfqpoint{6.057329in}{1.238500in}}%
\pgfpathlineto{\pgfqpoint{6.067463in}{1.238500in}}%
\pgfusepath{stroke}%
\end{pgfscope}%
\begin{pgfscope}%
\pgfpathrectangle{\pgfqpoint{4.701213in}{0.383578in}}{\pgfqpoint{1.356250in}{1.540000in}}%
\pgfusepath{clip}%
\pgfsetrectcap%
\pgfsetroundjoin%
\pgfsetlinewidth{0.803000pt}%
\definecolor{currentstroke}{rgb}{0.686275,0.352941,0.313725}%
\pgfsetstrokecolor{currentstroke}%
\pgfsetstrokeopacity{0.300000}%
\pgfsetdash{}{0pt}%
\pgfpathmoveto{\pgfqpoint{4.755463in}{1.238500in}}%
\pgfpathlineto{\pgfqpoint{4.783162in}{1.238500in}}%
\pgfpathlineto{\pgfqpoint{4.810861in}{1.238500in}}%
\pgfpathlineto{\pgfqpoint{4.838561in}{1.238500in}}%
\pgfpathlineto{\pgfqpoint{4.866260in}{1.238500in}}%
\pgfpathlineto{\pgfqpoint{4.893959in}{1.238500in}}%
\pgfpathlineto{\pgfqpoint{4.921658in}{1.238500in}}%
\pgfpathlineto{\pgfqpoint{4.949358in}{1.238500in}}%
\pgfpathlineto{\pgfqpoint{4.977057in}{1.238500in}}%
\pgfpathlineto{\pgfqpoint{5.004756in}{1.238500in}}%
\pgfpathlineto{\pgfqpoint{5.032456in}{1.238500in}}%
\pgfpathlineto{\pgfqpoint{5.060155in}{1.238500in}}%
\pgfpathlineto{\pgfqpoint{5.087854in}{1.238500in}}%
\pgfpathlineto{\pgfqpoint{5.115553in}{1.238500in}}%
\pgfpathlineto{\pgfqpoint{5.143253in}{1.493671in}}%
\pgfpathlineto{\pgfqpoint{5.170952in}{1.628387in}}%
\pgfpathlineto{\pgfqpoint{5.198651in}{1.678794in}}%
\pgfpathlineto{\pgfqpoint{5.226351in}{1.675369in}}%
\pgfpathlineto{\pgfqpoint{5.254050in}{1.641052in}}%
\pgfpathlineto{\pgfqpoint{5.281749in}{1.591819in}}%
\pgfpathlineto{\pgfqpoint{5.309448in}{1.538018in}}%
\pgfpathlineto{\pgfqpoint{5.337148in}{1.485832in}}%
\pgfpathlineto{\pgfqpoint{5.364847in}{1.438668in}}%
\pgfpathlineto{\pgfqpoint{5.392546in}{1.397983in}}%
\pgfpathlineto{\pgfqpoint{5.420246in}{1.363996in}}%
\pgfpathlineto{\pgfqpoint{5.447945in}{1.336282in}}%
\pgfpathlineto{\pgfqpoint{5.475644in}{1.314087in}}%
\pgfpathlineto{\pgfqpoint{5.503343in}{1.296533in}}%
\pgfpathlineto{\pgfqpoint{5.531043in}{1.282803in}}%
\pgfpathlineto{\pgfqpoint{5.558742in}{1.272152in}}%
\pgfpathlineto{\pgfqpoint{5.586441in}{1.263948in}}%
\pgfpathlineto{\pgfqpoint{5.614141in}{1.257668in}}%
\pgfpathlineto{\pgfqpoint{5.641840in}{1.252885in}}%
\pgfpathlineto{\pgfqpoint{5.669539in}{1.249260in}}%
\pgfpathlineto{\pgfqpoint{5.697238in}{1.246524in}}%
\pgfpathlineto{\pgfqpoint{5.724938in}{1.244467in}}%
\pgfpathlineto{\pgfqpoint{5.752637in}{1.242926in}}%
\pgfpathlineto{\pgfqpoint{5.780336in}{1.241776in}}%
\pgfpathlineto{\pgfqpoint{5.808036in}{1.240920in}}%
\pgfpathlineto{\pgfqpoint{5.835735in}{1.240284in}}%
\pgfpathlineto{\pgfqpoint{5.863434in}{1.239813in}}%
\pgfpathlineto{\pgfqpoint{5.891133in}{1.239464in}}%
\pgfpathlineto{\pgfqpoint{5.918833in}{1.239207in}}%
\pgfpathlineto{\pgfqpoint{5.946532in}{1.239018in}}%
\pgfpathlineto{\pgfqpoint{5.974231in}{1.238879in}}%
\pgfpathlineto{\pgfqpoint{6.001931in}{1.238776in}}%
\pgfpathlineto{\pgfqpoint{6.029630in}{1.238701in}}%
\pgfpathlineto{\pgfqpoint{6.057329in}{1.238647in}}%
\pgfpathlineto{\pgfqpoint{6.067463in}{1.238632in}}%
\pgfusepath{stroke}%
\end{pgfscope}%
\begin{pgfscope}%
\pgfpathrectangle{\pgfqpoint{4.701213in}{0.383578in}}{\pgfqpoint{1.356250in}{1.540000in}}%
\pgfusepath{clip}%
\pgfsetrectcap%
\pgfsetroundjoin%
\pgfsetlinewidth{0.803000pt}%
\definecolor{currentstroke}{rgb}{0.686275,0.352941,0.313725}%
\pgfsetstrokecolor{currentstroke}%
\pgfsetstrokeopacity{0.300000}%
\pgfsetdash{}{0pt}%
\pgfpathmoveto{\pgfqpoint{4.755463in}{1.238500in}}%
\pgfpathlineto{\pgfqpoint{4.783162in}{1.238500in}}%
\pgfpathlineto{\pgfqpoint{4.810861in}{1.238500in}}%
\pgfpathlineto{\pgfqpoint{4.838561in}{1.238500in}}%
\pgfpathlineto{\pgfqpoint{4.866260in}{1.238500in}}%
\pgfpathlineto{\pgfqpoint{4.893959in}{1.238500in}}%
\pgfpathlineto{\pgfqpoint{4.921658in}{1.238500in}}%
\pgfpathlineto{\pgfqpoint{4.949358in}{1.238500in}}%
\pgfpathlineto{\pgfqpoint{4.977057in}{1.238500in}}%
\pgfpathlineto{\pgfqpoint{5.004756in}{1.238500in}}%
\pgfpathlineto{\pgfqpoint{5.032456in}{1.238500in}}%
\pgfpathlineto{\pgfqpoint{5.060155in}{1.238500in}}%
\pgfpathlineto{\pgfqpoint{5.087854in}{1.238500in}}%
\pgfpathlineto{\pgfqpoint{5.115553in}{1.238500in}}%
\pgfpathlineto{\pgfqpoint{5.143253in}{1.238500in}}%
\pgfpathlineto{\pgfqpoint{5.170952in}{1.238500in}}%
\pgfpathlineto{\pgfqpoint{5.198651in}{1.238500in}}%
\pgfpathlineto{\pgfqpoint{5.226351in}{1.238500in}}%
\pgfpathlineto{\pgfqpoint{5.254050in}{1.238500in}}%
\pgfpathlineto{\pgfqpoint{5.281749in}{1.238500in}}%
\pgfpathlineto{\pgfqpoint{5.309448in}{1.238500in}}%
\pgfpathlineto{\pgfqpoint{5.337148in}{1.238500in}}%
\pgfpathlineto{\pgfqpoint{5.364847in}{1.238500in}}%
\pgfpathlineto{\pgfqpoint{5.392546in}{1.238500in}}%
\pgfpathlineto{\pgfqpoint{5.420246in}{1.238500in}}%
\pgfpathlineto{\pgfqpoint{5.447945in}{1.238500in}}%
\pgfpathlineto{\pgfqpoint{5.475644in}{1.238500in}}%
\pgfpathlineto{\pgfqpoint{5.503343in}{1.238500in}}%
\pgfpathlineto{\pgfqpoint{5.531043in}{1.238500in}}%
\pgfpathlineto{\pgfqpoint{5.558742in}{1.238500in}}%
\pgfpathlineto{\pgfqpoint{5.586441in}{1.238500in}}%
\pgfpathlineto{\pgfqpoint{5.614141in}{1.238500in}}%
\pgfpathlineto{\pgfqpoint{5.641840in}{1.238500in}}%
\pgfpathlineto{\pgfqpoint{5.669539in}{1.238500in}}%
\pgfpathlineto{\pgfqpoint{5.697238in}{1.238500in}}%
\pgfpathlineto{\pgfqpoint{5.724938in}{1.238500in}}%
\pgfpathlineto{\pgfqpoint{5.752637in}{1.238500in}}%
\pgfpathlineto{\pgfqpoint{5.780336in}{1.238500in}}%
\pgfpathlineto{\pgfqpoint{5.808036in}{1.238500in}}%
\pgfpathlineto{\pgfqpoint{5.835735in}{1.238500in}}%
\pgfpathlineto{\pgfqpoint{5.863434in}{1.238500in}}%
\pgfpathlineto{\pgfqpoint{5.891133in}{1.238500in}}%
\pgfpathlineto{\pgfqpoint{5.918833in}{1.238500in}}%
\pgfpathlineto{\pgfqpoint{5.946532in}{1.238500in}}%
\pgfpathlineto{\pgfqpoint{5.974231in}{1.238500in}}%
\pgfpathlineto{\pgfqpoint{6.001931in}{1.238500in}}%
\pgfpathlineto{\pgfqpoint{6.029630in}{1.238500in}}%
\pgfpathlineto{\pgfqpoint{6.057329in}{1.238500in}}%
\pgfpathlineto{\pgfqpoint{6.067463in}{1.238500in}}%
\pgfusepath{stroke}%
\end{pgfscope}%
\begin{pgfscope}%
\pgfpathrectangle{\pgfqpoint{4.701213in}{0.383578in}}{\pgfqpoint{1.356250in}{1.540000in}}%
\pgfusepath{clip}%
\pgfsetrectcap%
\pgfsetroundjoin%
\pgfsetlinewidth{0.803000pt}%
\definecolor{currentstroke}{rgb}{0.686275,0.352941,0.313725}%
\pgfsetstrokecolor{currentstroke}%
\pgfsetstrokeopacity{0.300000}%
\pgfsetdash{}{0pt}%
\pgfpathmoveto{\pgfqpoint{4.755463in}{1.238500in}}%
\pgfpathlineto{\pgfqpoint{4.783162in}{1.238500in}}%
\pgfpathlineto{\pgfqpoint{4.810861in}{1.238500in}}%
\pgfpathlineto{\pgfqpoint{4.838561in}{1.238500in}}%
\pgfpathlineto{\pgfqpoint{4.866260in}{1.238500in}}%
\pgfpathlineto{\pgfqpoint{4.893959in}{1.238500in}}%
\pgfpathlineto{\pgfqpoint{4.921658in}{1.238500in}}%
\pgfpathlineto{\pgfqpoint{4.949358in}{1.238500in}}%
\pgfpathlineto{\pgfqpoint{4.977057in}{1.238500in}}%
\pgfpathlineto{\pgfqpoint{5.004756in}{1.238500in}}%
\pgfpathlineto{\pgfqpoint{5.032456in}{1.238500in}}%
\pgfpathlineto{\pgfqpoint{5.060155in}{1.238500in}}%
\pgfpathlineto{\pgfqpoint{5.087854in}{1.238500in}}%
\pgfpathlineto{\pgfqpoint{5.115553in}{1.238500in}}%
\pgfpathlineto{\pgfqpoint{5.143253in}{1.238500in}}%
\pgfpathlineto{\pgfqpoint{5.170952in}{1.238500in}}%
\pgfpathlineto{\pgfqpoint{5.198651in}{1.238500in}}%
\pgfpathlineto{\pgfqpoint{5.226351in}{1.238500in}}%
\pgfpathlineto{\pgfqpoint{5.254050in}{1.238500in}}%
\pgfpathlineto{\pgfqpoint{5.281749in}{1.238500in}}%
\pgfpathlineto{\pgfqpoint{5.309448in}{1.238500in}}%
\pgfpathlineto{\pgfqpoint{5.337148in}{1.238500in}}%
\pgfpathlineto{\pgfqpoint{5.364847in}{1.238500in}}%
\pgfpathlineto{\pgfqpoint{5.392546in}{1.238500in}}%
\pgfpathlineto{\pgfqpoint{5.420246in}{1.238500in}}%
\pgfpathlineto{\pgfqpoint{5.447945in}{1.238500in}}%
\pgfpathlineto{\pgfqpoint{5.475644in}{1.238500in}}%
\pgfpathlineto{\pgfqpoint{5.503343in}{1.238500in}}%
\pgfpathlineto{\pgfqpoint{5.531043in}{1.238500in}}%
\pgfpathlineto{\pgfqpoint{5.558742in}{1.238500in}}%
\pgfpathlineto{\pgfqpoint{5.586441in}{1.238500in}}%
\pgfpathlineto{\pgfqpoint{5.614141in}{1.238500in}}%
\pgfpathlineto{\pgfqpoint{5.641840in}{1.238500in}}%
\pgfpathlineto{\pgfqpoint{5.669539in}{1.238500in}}%
\pgfpathlineto{\pgfqpoint{5.697238in}{1.238500in}}%
\pgfpathlineto{\pgfqpoint{5.724938in}{1.238500in}}%
\pgfpathlineto{\pgfqpoint{5.752637in}{1.238500in}}%
\pgfpathlineto{\pgfqpoint{5.780336in}{1.238500in}}%
\pgfpathlineto{\pgfqpoint{5.808036in}{1.238500in}}%
\pgfpathlineto{\pgfqpoint{5.835735in}{1.238500in}}%
\pgfpathlineto{\pgfqpoint{5.863434in}{1.238500in}}%
\pgfpathlineto{\pgfqpoint{5.891133in}{1.238500in}}%
\pgfpathlineto{\pgfqpoint{5.918833in}{1.238500in}}%
\pgfpathlineto{\pgfqpoint{5.946532in}{1.238500in}}%
\pgfpathlineto{\pgfqpoint{5.974231in}{1.238500in}}%
\pgfpathlineto{\pgfqpoint{6.001931in}{1.238500in}}%
\pgfpathlineto{\pgfqpoint{6.029630in}{1.238500in}}%
\pgfpathlineto{\pgfqpoint{6.057329in}{1.238500in}}%
\pgfpathlineto{\pgfqpoint{6.067463in}{1.238500in}}%
\pgfusepath{stroke}%
\end{pgfscope}%
\begin{pgfscope}%
\pgfpathrectangle{\pgfqpoint{4.701213in}{0.383578in}}{\pgfqpoint{1.356250in}{1.540000in}}%
\pgfusepath{clip}%
\pgfsetrectcap%
\pgfsetroundjoin%
\pgfsetlinewidth{0.803000pt}%
\definecolor{currentstroke}{rgb}{0.686275,0.352941,0.313725}%
\pgfsetstrokecolor{currentstroke}%
\pgfsetstrokeopacity{0.300000}%
\pgfsetdash{}{0pt}%
\pgfpathmoveto{\pgfqpoint{4.755463in}{1.238500in}}%
\pgfpathlineto{\pgfqpoint{4.783162in}{1.238500in}}%
\pgfpathlineto{\pgfqpoint{4.810861in}{1.238500in}}%
\pgfpathlineto{\pgfqpoint{4.838561in}{1.238500in}}%
\pgfpathlineto{\pgfqpoint{4.866260in}{1.238500in}}%
\pgfpathlineto{\pgfqpoint{4.893959in}{1.238500in}}%
\pgfpathlineto{\pgfqpoint{4.921658in}{1.238500in}}%
\pgfpathlineto{\pgfqpoint{4.949358in}{1.238500in}}%
\pgfpathlineto{\pgfqpoint{4.977057in}{1.238500in}}%
\pgfpathlineto{\pgfqpoint{5.004756in}{1.238500in}}%
\pgfpathlineto{\pgfqpoint{5.032456in}{1.238500in}}%
\pgfpathlineto{\pgfqpoint{5.060155in}{1.238500in}}%
\pgfpathlineto{\pgfqpoint{5.087854in}{1.238500in}}%
\pgfpathlineto{\pgfqpoint{5.115553in}{1.238500in}}%
\pgfpathlineto{\pgfqpoint{5.143253in}{1.238500in}}%
\pgfpathlineto{\pgfqpoint{5.170952in}{1.238500in}}%
\pgfpathlineto{\pgfqpoint{5.198651in}{1.238500in}}%
\pgfpathlineto{\pgfqpoint{5.226351in}{1.238500in}}%
\pgfpathlineto{\pgfqpoint{5.254050in}{1.238500in}}%
\pgfpathlineto{\pgfqpoint{5.281749in}{1.238500in}}%
\pgfpathlineto{\pgfqpoint{5.309448in}{1.238500in}}%
\pgfpathlineto{\pgfqpoint{5.337148in}{1.238500in}}%
\pgfpathlineto{\pgfqpoint{5.364847in}{1.238500in}}%
\pgfpathlineto{\pgfqpoint{5.392546in}{1.238500in}}%
\pgfpathlineto{\pgfqpoint{5.420246in}{1.238500in}}%
\pgfpathlineto{\pgfqpoint{5.447945in}{1.238500in}}%
\pgfpathlineto{\pgfqpoint{5.475644in}{1.238500in}}%
\pgfpathlineto{\pgfqpoint{5.503343in}{1.238500in}}%
\pgfpathlineto{\pgfqpoint{5.531043in}{1.238500in}}%
\pgfpathlineto{\pgfqpoint{5.558742in}{1.238500in}}%
\pgfpathlineto{\pgfqpoint{5.586441in}{1.238500in}}%
\pgfpathlineto{\pgfqpoint{5.614141in}{1.238500in}}%
\pgfpathlineto{\pgfqpoint{5.641840in}{1.238500in}}%
\pgfpathlineto{\pgfqpoint{5.669539in}{1.238500in}}%
\pgfpathlineto{\pgfqpoint{5.697238in}{1.238500in}}%
\pgfpathlineto{\pgfqpoint{5.724938in}{1.238500in}}%
\pgfpathlineto{\pgfqpoint{5.752637in}{1.238500in}}%
\pgfpathlineto{\pgfqpoint{5.780336in}{1.238500in}}%
\pgfpathlineto{\pgfqpoint{5.808036in}{1.238500in}}%
\pgfpathlineto{\pgfqpoint{5.835735in}{1.238500in}}%
\pgfpathlineto{\pgfqpoint{5.863434in}{1.238500in}}%
\pgfpathlineto{\pgfqpoint{5.891133in}{1.238500in}}%
\pgfpathlineto{\pgfqpoint{5.918833in}{1.238500in}}%
\pgfpathlineto{\pgfqpoint{5.946532in}{1.238500in}}%
\pgfpathlineto{\pgfqpoint{5.974231in}{1.238500in}}%
\pgfpathlineto{\pgfqpoint{6.001931in}{1.238500in}}%
\pgfpathlineto{\pgfqpoint{6.029630in}{1.238500in}}%
\pgfpathlineto{\pgfqpoint{6.057329in}{1.238500in}}%
\pgfpathlineto{\pgfqpoint{6.067463in}{1.238500in}}%
\pgfusepath{stroke}%
\end{pgfscope}%
\begin{pgfscope}%
\pgfpathrectangle{\pgfqpoint{4.701213in}{0.383578in}}{\pgfqpoint{1.356250in}{1.540000in}}%
\pgfusepath{clip}%
\pgfsetrectcap%
\pgfsetroundjoin%
\pgfsetlinewidth{0.803000pt}%
\definecolor{currentstroke}{rgb}{0.686275,0.352941,0.313725}%
\pgfsetstrokecolor{currentstroke}%
\pgfsetstrokeopacity{0.300000}%
\pgfsetdash{}{0pt}%
\pgfpathmoveto{\pgfqpoint{4.755463in}{1.238500in}}%
\pgfpathlineto{\pgfqpoint{4.783162in}{1.238500in}}%
\pgfpathlineto{\pgfqpoint{4.810861in}{1.238500in}}%
\pgfpathlineto{\pgfqpoint{4.838561in}{1.238500in}}%
\pgfpathlineto{\pgfqpoint{4.866260in}{1.238500in}}%
\pgfpathlineto{\pgfqpoint{4.893959in}{1.238500in}}%
\pgfpathlineto{\pgfqpoint{4.921658in}{1.238500in}}%
\pgfpathlineto{\pgfqpoint{4.949358in}{1.238500in}}%
\pgfpathlineto{\pgfqpoint{4.977057in}{1.238500in}}%
\pgfpathlineto{\pgfqpoint{5.004756in}{1.238500in}}%
\pgfpathlineto{\pgfqpoint{5.032456in}{1.238500in}}%
\pgfpathlineto{\pgfqpoint{5.060155in}{1.238500in}}%
\pgfpathlineto{\pgfqpoint{5.087854in}{1.238500in}}%
\pgfpathlineto{\pgfqpoint{5.115553in}{1.238500in}}%
\pgfpathlineto{\pgfqpoint{5.143253in}{1.351588in}}%
\pgfpathlineto{\pgfqpoint{5.170952in}{1.412421in}}%
\pgfpathlineto{\pgfqpoint{5.198651in}{1.436042in}}%
\pgfpathlineto{\pgfqpoint{5.226351in}{1.435547in}}%
\pgfpathlineto{\pgfqpoint{5.254050in}{1.420972in}}%
\pgfpathlineto{\pgfqpoint{5.281749in}{1.399419in}}%
\pgfpathlineto{\pgfqpoint{5.309448in}{1.375550in}}%
\pgfpathlineto{\pgfqpoint{5.337148in}{1.352186in}}%
\pgfpathlineto{\pgfqpoint{5.364847in}{1.330985in}}%
\pgfpathlineto{\pgfqpoint{5.392546in}{1.312677in}}%
\pgfpathlineto{\pgfqpoint{5.420246in}{1.297367in}}%
\pgfpathlineto{\pgfqpoint{5.447945in}{1.284870in}}%
\pgfpathlineto{\pgfqpoint{5.475644in}{1.274834in}}%
\pgfpathlineto{\pgfqpoint{5.503343in}{1.266838in}}%
\pgfpathlineto{\pgfqpoint{5.531043in}{1.260523in}}%
\pgfpathlineto{\pgfqpoint{5.558742in}{1.255557in}}%
\pgfpathlineto{\pgfqpoint{5.586441in}{1.251668in}}%
\pgfpathlineto{\pgfqpoint{5.614141in}{1.248634in}}%
\pgfpathlineto{\pgfqpoint{5.641840in}{1.246275in}}%
\pgfpathlineto{\pgfqpoint{5.669539in}{1.244447in}}%
\pgfpathlineto{\pgfqpoint{5.697238in}{1.243036in}}%
\pgfpathlineto{\pgfqpoint{5.724938in}{1.241949in}}%
\pgfpathlineto{\pgfqpoint{5.752637in}{1.241117in}}%
\pgfpathlineto{\pgfqpoint{5.780336in}{1.240481in}}%
\pgfpathlineto{\pgfqpoint{5.808036in}{1.239996in}}%
\pgfpathlineto{\pgfqpoint{5.835735in}{1.239626in}}%
\pgfpathlineto{\pgfqpoint{5.863434in}{1.239346in}}%
\pgfpathlineto{\pgfqpoint{5.891133in}{1.239135in}}%
\pgfpathlineto{\pgfqpoint{5.918833in}{1.238975in}}%
\pgfpathlineto{\pgfqpoint{5.946532in}{1.238855in}}%
\pgfpathlineto{\pgfqpoint{5.974231in}{1.238764in}}%
\pgfpathlineto{\pgfqpoint{6.001931in}{1.238697in}}%
\pgfpathlineto{\pgfqpoint{6.029630in}{1.238646in}}%
\pgfpathlineto{\pgfqpoint{6.057329in}{1.238608in}}%
\pgfpathlineto{\pgfqpoint{6.067463in}{1.238598in}}%
\pgfusepath{stroke}%
\end{pgfscope}%
\begin{pgfscope}%
\pgfpathrectangle{\pgfqpoint{4.701213in}{0.383578in}}{\pgfqpoint{1.356250in}{1.540000in}}%
\pgfusepath{clip}%
\pgfsetrectcap%
\pgfsetroundjoin%
\pgfsetlinewidth{0.803000pt}%
\definecolor{currentstroke}{rgb}{0.686275,0.352941,0.313725}%
\pgfsetstrokecolor{currentstroke}%
\pgfsetstrokeopacity{0.300000}%
\pgfsetdash{}{0pt}%
\pgfpathmoveto{\pgfqpoint{4.755463in}{1.238500in}}%
\pgfpathlineto{\pgfqpoint{4.783162in}{1.238500in}}%
\pgfpathlineto{\pgfqpoint{4.810861in}{1.238500in}}%
\pgfpathlineto{\pgfqpoint{4.838561in}{1.238500in}}%
\pgfpathlineto{\pgfqpoint{4.866260in}{1.238500in}}%
\pgfpathlineto{\pgfqpoint{4.893959in}{1.238500in}}%
\pgfpathlineto{\pgfqpoint{4.921658in}{1.238500in}}%
\pgfpathlineto{\pgfqpoint{4.949358in}{1.238500in}}%
\pgfpathlineto{\pgfqpoint{4.977057in}{1.238500in}}%
\pgfpathlineto{\pgfqpoint{5.004756in}{1.238500in}}%
\pgfpathlineto{\pgfqpoint{5.032456in}{1.238500in}}%
\pgfpathlineto{\pgfqpoint{5.060155in}{1.238500in}}%
\pgfpathlineto{\pgfqpoint{5.087854in}{1.238500in}}%
\pgfpathlineto{\pgfqpoint{5.115553in}{1.238500in}}%
\pgfpathlineto{\pgfqpoint{5.143253in}{1.238500in}}%
\pgfpathlineto{\pgfqpoint{5.170952in}{1.238500in}}%
\pgfpathlineto{\pgfqpoint{5.198651in}{1.238500in}}%
\pgfpathlineto{\pgfqpoint{5.226351in}{1.238500in}}%
\pgfpathlineto{\pgfqpoint{5.254050in}{1.238500in}}%
\pgfpathlineto{\pgfqpoint{5.281749in}{1.238500in}}%
\pgfpathlineto{\pgfqpoint{5.309448in}{1.238500in}}%
\pgfpathlineto{\pgfqpoint{5.337148in}{1.238500in}}%
\pgfpathlineto{\pgfqpoint{5.364847in}{1.238500in}}%
\pgfpathlineto{\pgfqpoint{5.392546in}{1.238500in}}%
\pgfpathlineto{\pgfqpoint{5.420246in}{1.238500in}}%
\pgfpathlineto{\pgfqpoint{5.447945in}{1.238500in}}%
\pgfpathlineto{\pgfqpoint{5.475644in}{1.238500in}}%
\pgfpathlineto{\pgfqpoint{5.503343in}{1.238500in}}%
\pgfpathlineto{\pgfqpoint{5.531043in}{1.238500in}}%
\pgfpathlineto{\pgfqpoint{5.558742in}{1.238500in}}%
\pgfpathlineto{\pgfqpoint{5.586441in}{1.238500in}}%
\pgfpathlineto{\pgfqpoint{5.614141in}{1.238500in}}%
\pgfpathlineto{\pgfqpoint{5.641840in}{1.238500in}}%
\pgfpathlineto{\pgfqpoint{5.669539in}{1.238500in}}%
\pgfpathlineto{\pgfqpoint{5.697238in}{1.238500in}}%
\pgfpathlineto{\pgfqpoint{5.724938in}{1.238500in}}%
\pgfpathlineto{\pgfqpoint{5.752637in}{1.238500in}}%
\pgfpathlineto{\pgfqpoint{5.780336in}{1.238500in}}%
\pgfpathlineto{\pgfqpoint{5.808036in}{1.238500in}}%
\pgfpathlineto{\pgfqpoint{5.835735in}{1.238500in}}%
\pgfpathlineto{\pgfqpoint{5.863434in}{1.238500in}}%
\pgfpathlineto{\pgfqpoint{5.891133in}{1.238500in}}%
\pgfpathlineto{\pgfqpoint{5.918833in}{1.238500in}}%
\pgfpathlineto{\pgfqpoint{5.946532in}{1.238500in}}%
\pgfpathlineto{\pgfqpoint{5.974231in}{1.238500in}}%
\pgfpathlineto{\pgfqpoint{6.001931in}{1.238500in}}%
\pgfpathlineto{\pgfqpoint{6.029630in}{1.238500in}}%
\pgfpathlineto{\pgfqpoint{6.057329in}{1.238500in}}%
\pgfpathlineto{\pgfqpoint{6.067463in}{1.238500in}}%
\pgfusepath{stroke}%
\end{pgfscope}%
\begin{pgfscope}%
\pgfsetrectcap%
\pgfsetmiterjoin%
\pgfsetlinewidth{0.501875pt}%
\definecolor{currentstroke}{rgb}{0.317647,0.317647,0.317647}%
\pgfsetstrokecolor{currentstroke}%
\pgfsetdash{}{0pt}%
\pgfpathmoveto{\pgfqpoint{4.701213in}{0.383578in}}%
\pgfpathlineto{\pgfqpoint{4.701213in}{1.923578in}}%
\pgfusepath{stroke}%
\end{pgfscope}%
\begin{pgfscope}%
\pgfsetrectcap%
\pgfsetmiterjoin%
\pgfsetlinewidth{0.501875pt}%
\definecolor{currentstroke}{rgb}{0.317647,0.317647,0.317647}%
\pgfsetstrokecolor{currentstroke}%
\pgfsetdash{}{0pt}%
\pgfpathmoveto{\pgfqpoint{4.701213in}{0.383578in}}%
\pgfpathlineto{\pgfqpoint{6.057463in}{0.383578in}}%
\pgfusepath{stroke}%
\end{pgfscope}%
\begin{pgfscope}%
\pgfsetrectcap%
\pgfsetroundjoin%
\pgfsetlinewidth{0.803000pt}%
\definecolor{currentstroke}{rgb}{0.333333,0.333333,0.333333}%
\pgfsetstrokecolor{currentstroke}%
\pgfsetdash{}{0pt}%
\pgfpathmoveto{\pgfqpoint{5.856710in}{1.863417in}}%
\pgfpathlineto{\pgfqpoint{5.930754in}{1.863417in}}%
\pgfusepath{stroke}%
\end{pgfscope}%
\begin{pgfscope}%
\definecolor{textcolor}{rgb}{0.000000,0.000000,0.000000}%
\pgfsetstrokecolor{textcolor}%
\pgfsetfillcolor{textcolor}%
\pgftext[x=5.977032in,y=1.831022in,left,base]{\color{textcolor}\rmfamily\fontsize{6.664000}{7.996800}\selectfont 0}%
\end{pgfscope}%
\begin{pgfscope}%
\pgfsetrectcap%
\pgfsetroundjoin%
\pgfsetlinewidth{0.803000pt}%
\definecolor{currentstroke}{rgb}{0.686275,0.352941,0.313725}%
\pgfsetstrokecolor{currentstroke}%
\pgfsetdash{}{0pt}%
\pgfpathmoveto{\pgfqpoint{5.856710in}{1.743650in}}%
\pgfpathlineto{\pgfqpoint{5.930754in}{1.743650in}}%
\pgfusepath{stroke}%
\end{pgfscope}%
\begin{pgfscope}%
\definecolor{textcolor}{rgb}{0.000000,0.000000,0.000000}%
\pgfsetstrokecolor{textcolor}%
\pgfsetfillcolor{textcolor}%
\pgftext[x=5.977032in,y=1.711256in,left,base]{\color{textcolor}\rmfamily\fontsize{6.664000}{7.996800}\selectfont 1}%
\end{pgfscope}%
\end{pgfpicture}%
\makeatother%
\endgroup%

	\caption[Traces in the backward pass using SuperSpike.]{Computation traces of the backward pass using SuperSpike. The integrand of the weight update $\Delta w_{ij}^{(o)}$ is given by the adapted von Rossum error $e^{(o)}$ (first column) and the convolution of the surrogate gradient with the presynaptic spike activity $\lambda_{ij}^{(o)} = \alpha \ast \left(\sigma'(V_{\text{m},i}^{(o)}) \left(\epsilon \ast S_j^{(o)}\right)\right)$ (second column). The final weight update is given by the sum over temporal traces of the last column.}
	\label{weightchangesplot}
\end{figure}

	
\subsection{Results}
\begin{itemize}
	\item Backprop weights, FA, no hidden layer learning
\end{itemize}