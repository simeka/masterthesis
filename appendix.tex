\appendix
\chapter{Appendix}
%\renewcommand{\thechapter}{A}
\section{Activation Function Parameters}
\begin{table}[h!]\centering\ra{1.3}
	\begin{tabular}{@{}rlll@{}}\toprule
		& Parameter							& 	Simulation 				& 	Hardware 				\\ \midrule
		& noise weight $w_\text{noise}$		& 	8						&	15						\\
		& input weight $w_\text{in}$		&  	4						& 	30						\\
		& input rate $\nu_\text{in}$		&	\SIrange{-1000}{1000}{\kilo \Hz}& \SIrange{-560}{560}{\kilo \Hz}		\\
		& noise rate $\nu_\text{noise}$		&	\SI{240}{\kilo \Hz}		& 	\SI{70}{\kilo \Hz}		\\
		& resting potential \gls{v_leak}	&	\SI{0}{\V}				& 	\SI{0.54}{\V}			\\
		& reset potential \gls{v_reset}		&	\SI{-200}{\V}			& 	\SI{0.01}{\V}			\\
		& threshold \gls{thres}				&\SIrange{-4.25}{5.75}{\V}	& 	\SIrange{0.45}{0.63}{\V}\\
		& refractory period \gls{refrac}	&	\SI{100}{\micro \s}		&	\SI{9}{\micro \s}		\\
		& synaptic time constant \gls{tau_syn} 	&\SI{30}{\micro \s}		&	\SI{5}{\micro \s}		\\
		& membrane time constant \gls{tau_m}	&\SI{3}{\micro \s}		&	\SI{5}{\micro \s}		\\
		\bottomrule
	\end{tabular}
	\caption[Parameter setting for a sigmoid activation function of the \gls{lif} neuron.]{Parameter setting for a sigmoid activation function of the \gls{lif} neuron. The different parameters for the simulated and on hardware recorded activation function are shown. The choice of the unit for the simulated potentials is arbitrary.}
	\label{hardwarevssimulationtable}
\end{table}
\newpage
\section{Monitoring Plots}
\label{monitoringplots}
\begin{figure}[!htb]
	\centering
    %% Creator: Matplotlib, PGF backend
%%
%% To include the figure in your LaTeX document, write
%%   \input{<filename>.pgf}
%%
%% Make sure the required packages are loaded in your preamble
%%   \usepackage{pgf}
%%
%% Figures using additional raster images can only be included by \input if
%% they are in the same directory as the main LaTeX file. For loading figures
%% from other directories you can use the `import` package
%%   \usepackage{import}
%% and then include the figures with
%%   \import{<path to file>}{<filename>.pgf}
%%
%% Matplotlib used the following preamble
%%   \usepackage{amsmath} \usepackage{pifont} \usepackage{xcolor} \definecolor{green}{HTML}{467821} \definecolor{red}{HTML}{CF4457} \usepackage[detect-all]{siunitx}
%%   \usepackage{fontspec}
%%
\begingroup%
\makeatletter%
\begin{pgfpicture}%
\pgfpathrectangle{\pgfpointorigin}{\pgfqpoint{6.138952in}{3.548545in}}%
\pgfusepath{use as bounding box, clip}%
\begin{pgfscope}%
\pgfsetbuttcap%
\pgfsetmiterjoin%
\pgfsetlinewidth{0.000000pt}%
\definecolor{currentstroke}{rgb}{0.000000,0.000000,0.000000}%
\pgfsetstrokecolor{currentstroke}%
\pgfsetstrokeopacity{0.000000}%
\pgfsetdash{}{0pt}%
\pgfpathmoveto{\pgfqpoint{0.000000in}{0.000000in}}%
\pgfpathlineto{\pgfqpoint{6.138952in}{0.000000in}}%
\pgfpathlineto{\pgfqpoint{6.138952in}{3.548545in}}%
\pgfpathlineto{\pgfqpoint{0.000000in}{3.548545in}}%
\pgfpathclose%
\pgfusepath{}%
\end{pgfscope}%
\begin{pgfscope}%
\pgfsetbuttcap%
\pgfsetmiterjoin%
\pgfsetlinewidth{0.000000pt}%
\definecolor{currentstroke}{rgb}{0.000000,0.000000,0.000000}%
\pgfsetstrokecolor{currentstroke}%
\pgfsetstrokeopacity{0.000000}%
\pgfsetdash{}{0pt}%
\pgfpathmoveto{\pgfqpoint{0.488751in}{2.165212in}}%
\pgfpathlineto{\pgfqpoint{2.749168in}{2.165212in}}%
\pgfpathlineto{\pgfqpoint{2.749168in}{3.448545in}}%
\pgfpathlineto{\pgfqpoint{0.488751in}{3.448545in}}%
\pgfpathclose%
\pgfusepath{}%
\end{pgfscope}%
\begin{pgfscope}%
\pgfsetbuttcap%
\pgfsetroundjoin%
\definecolor{currentfill}{rgb}{0.317647,0.317647,0.317647}%
\pgfsetfillcolor{currentfill}%
\pgfsetlinewidth{0.501875pt}%
\definecolor{currentstroke}{rgb}{0.317647,0.317647,0.317647}%
\pgfsetstrokecolor{currentstroke}%
\pgfsetdash{}{0pt}%
\pgfsys@defobject{currentmarker}{\pgfqpoint{0.000000in}{-0.020833in}}{\pgfqpoint{0.000000in}{0.000000in}}{%
\pgfpathmoveto{\pgfqpoint{0.000000in}{0.000000in}}%
\pgfpathlineto{\pgfqpoint{0.000000in}{-0.020833in}}%
\pgfusepath{stroke,fill}%
}%
\begin{pgfscope}%
\pgfsys@transformshift{0.591497in}{2.165212in}%
\pgfsys@useobject{currentmarker}{}%
\end{pgfscope}%
\end{pgfscope}%
\begin{pgfscope}%
\pgfsetbuttcap%
\pgfsetroundjoin%
\definecolor{currentfill}{rgb}{0.317647,0.317647,0.317647}%
\pgfsetfillcolor{currentfill}%
\pgfsetlinewidth{0.501875pt}%
\definecolor{currentstroke}{rgb}{0.317647,0.317647,0.317647}%
\pgfsetstrokecolor{currentstroke}%
\pgfsetdash{}{0pt}%
\pgfsys@defobject{currentmarker}{\pgfqpoint{0.000000in}{-0.020833in}}{\pgfqpoint{0.000000in}{0.000000in}}{%
\pgfpathmoveto{\pgfqpoint{0.000000in}{0.000000in}}%
\pgfpathlineto{\pgfqpoint{0.000000in}{-0.020833in}}%
\pgfusepath{stroke,fill}%
}%
\begin{pgfscope}%
\pgfsys@transformshift{1.003306in}{2.165212in}%
\pgfsys@useobject{currentmarker}{}%
\end{pgfscope}%
\end{pgfscope}%
\begin{pgfscope}%
\pgfsetbuttcap%
\pgfsetroundjoin%
\definecolor{currentfill}{rgb}{0.317647,0.317647,0.317647}%
\pgfsetfillcolor{currentfill}%
\pgfsetlinewidth{0.501875pt}%
\definecolor{currentstroke}{rgb}{0.317647,0.317647,0.317647}%
\pgfsetstrokecolor{currentstroke}%
\pgfsetdash{}{0pt}%
\pgfsys@defobject{currentmarker}{\pgfqpoint{0.000000in}{-0.020833in}}{\pgfqpoint{0.000000in}{0.000000in}}{%
\pgfpathmoveto{\pgfqpoint{0.000000in}{0.000000in}}%
\pgfpathlineto{\pgfqpoint{0.000000in}{-0.020833in}}%
\pgfusepath{stroke,fill}%
}%
\begin{pgfscope}%
\pgfsys@transformshift{1.415114in}{2.165212in}%
\pgfsys@useobject{currentmarker}{}%
\end{pgfscope}%
\end{pgfscope}%
\begin{pgfscope}%
\pgfsetbuttcap%
\pgfsetroundjoin%
\definecolor{currentfill}{rgb}{0.317647,0.317647,0.317647}%
\pgfsetfillcolor{currentfill}%
\pgfsetlinewidth{0.501875pt}%
\definecolor{currentstroke}{rgb}{0.317647,0.317647,0.317647}%
\pgfsetstrokecolor{currentstroke}%
\pgfsetdash{}{0pt}%
\pgfsys@defobject{currentmarker}{\pgfqpoint{0.000000in}{-0.020833in}}{\pgfqpoint{0.000000in}{0.000000in}}{%
\pgfpathmoveto{\pgfqpoint{0.000000in}{0.000000in}}%
\pgfpathlineto{\pgfqpoint{0.000000in}{-0.020833in}}%
\pgfusepath{stroke,fill}%
}%
\begin{pgfscope}%
\pgfsys@transformshift{1.826922in}{2.165212in}%
\pgfsys@useobject{currentmarker}{}%
\end{pgfscope}%
\end{pgfscope}%
\begin{pgfscope}%
\pgfsetbuttcap%
\pgfsetroundjoin%
\definecolor{currentfill}{rgb}{0.317647,0.317647,0.317647}%
\pgfsetfillcolor{currentfill}%
\pgfsetlinewidth{0.501875pt}%
\definecolor{currentstroke}{rgb}{0.317647,0.317647,0.317647}%
\pgfsetstrokecolor{currentstroke}%
\pgfsetdash{}{0pt}%
\pgfsys@defobject{currentmarker}{\pgfqpoint{0.000000in}{-0.020833in}}{\pgfqpoint{0.000000in}{0.000000in}}{%
\pgfpathmoveto{\pgfqpoint{0.000000in}{0.000000in}}%
\pgfpathlineto{\pgfqpoint{0.000000in}{-0.020833in}}%
\pgfusepath{stroke,fill}%
}%
\begin{pgfscope}%
\pgfsys@transformshift{2.238731in}{2.165212in}%
\pgfsys@useobject{currentmarker}{}%
\end{pgfscope}%
\end{pgfscope}%
\begin{pgfscope}%
\pgfsetbuttcap%
\pgfsetroundjoin%
\definecolor{currentfill}{rgb}{0.317647,0.317647,0.317647}%
\pgfsetfillcolor{currentfill}%
\pgfsetlinewidth{0.501875pt}%
\definecolor{currentstroke}{rgb}{0.317647,0.317647,0.317647}%
\pgfsetstrokecolor{currentstroke}%
\pgfsetdash{}{0pt}%
\pgfsys@defobject{currentmarker}{\pgfqpoint{0.000000in}{-0.020833in}}{\pgfqpoint{0.000000in}{0.000000in}}{%
\pgfpathmoveto{\pgfqpoint{0.000000in}{0.000000in}}%
\pgfpathlineto{\pgfqpoint{0.000000in}{-0.020833in}}%
\pgfusepath{stroke,fill}%
}%
\begin{pgfscope}%
\pgfsys@transformshift{2.650539in}{2.165212in}%
\pgfsys@useobject{currentmarker}{}%
\end{pgfscope}%
\end{pgfscope}%
\begin{pgfscope}%
\pgfsetbuttcap%
\pgfsetroundjoin%
\definecolor{currentfill}{rgb}{0.317647,0.317647,0.317647}%
\pgfsetfillcolor{currentfill}%
\pgfsetlinewidth{0.501875pt}%
\definecolor{currentstroke}{rgb}{0.317647,0.317647,0.317647}%
\pgfsetstrokecolor{currentstroke}%
\pgfsetdash{}{0pt}%
\pgfsys@defobject{currentmarker}{\pgfqpoint{-0.020833in}{0.000000in}}{\pgfqpoint{0.000000in}{0.000000in}}{%
\pgfpathmoveto{\pgfqpoint{0.000000in}{0.000000in}}%
\pgfpathlineto{\pgfqpoint{-0.020833in}{0.000000in}}%
\pgfusepath{stroke,fill}%
}%
\begin{pgfscope}%
\pgfsys@transformshift{0.488751in}{2.434388in}%
\pgfsys@useobject{currentmarker}{}%
\end{pgfscope}%
\end{pgfscope}%
\begin{pgfscope}%
\definecolor{textcolor}{rgb}{0.317647,0.317647,0.317647}%
\pgfsetstrokecolor{textcolor}%
\pgfsetfillcolor{textcolor}%
\pgftext[x=0.256497in,y=2.402272in,left,base]{\color{textcolor}\rmfamily\fontsize{6.664000}{7.996800}\selectfont \(\displaystyle -20\)}%
\end{pgfscope}%
\begin{pgfscope}%
\pgfsetbuttcap%
\pgfsetroundjoin%
\definecolor{currentfill}{rgb}{0.317647,0.317647,0.317647}%
\pgfsetfillcolor{currentfill}%
\pgfsetlinewidth{0.501875pt}%
\definecolor{currentstroke}{rgb}{0.317647,0.317647,0.317647}%
\pgfsetstrokecolor{currentstroke}%
\pgfsetdash{}{0pt}%
\pgfsys@defobject{currentmarker}{\pgfqpoint{-0.020833in}{0.000000in}}{\pgfqpoint{0.000000in}{0.000000in}}{%
\pgfpathmoveto{\pgfqpoint{0.000000in}{0.000000in}}%
\pgfpathlineto{\pgfqpoint{-0.020833in}{0.000000in}}%
\pgfusepath{stroke,fill}%
}%
\begin{pgfscope}%
\pgfsys@transformshift{0.488751in}{2.715513in}%
\pgfsys@useobject{currentmarker}{}%
\end{pgfscope}%
\end{pgfscope}%
\begin{pgfscope}%
\definecolor{textcolor}{rgb}{0.317647,0.317647,0.317647}%
\pgfsetstrokecolor{textcolor}%
\pgfsetfillcolor{textcolor}%
\pgftext[x=0.398666in,y=2.683396in,left,base]{\color{textcolor}\rmfamily\fontsize{6.664000}{7.996800}\selectfont \(\displaystyle 0\)}%
\end{pgfscope}%
\begin{pgfscope}%
\pgfsetbuttcap%
\pgfsetroundjoin%
\definecolor{currentfill}{rgb}{0.317647,0.317647,0.317647}%
\pgfsetfillcolor{currentfill}%
\pgfsetlinewidth{0.501875pt}%
\definecolor{currentstroke}{rgb}{0.317647,0.317647,0.317647}%
\pgfsetstrokecolor{currentstroke}%
\pgfsetdash{}{0pt}%
\pgfsys@defobject{currentmarker}{\pgfqpoint{-0.020833in}{0.000000in}}{\pgfqpoint{0.000000in}{0.000000in}}{%
\pgfpathmoveto{\pgfqpoint{0.000000in}{0.000000in}}%
\pgfpathlineto{\pgfqpoint{-0.020833in}{0.000000in}}%
\pgfusepath{stroke,fill}%
}%
\begin{pgfscope}%
\pgfsys@transformshift{0.488751in}{2.996637in}%
\pgfsys@useobject{currentmarker}{}%
\end{pgfscope}%
\end{pgfscope}%
\begin{pgfscope}%
\definecolor{textcolor}{rgb}{0.317647,0.317647,0.317647}%
\pgfsetstrokecolor{textcolor}%
\pgfsetfillcolor{textcolor}%
\pgftext[x=0.343303in,y=2.964521in,left,base]{\color{textcolor}\rmfamily\fontsize{6.664000}{7.996800}\selectfont \(\displaystyle 20\)}%
\end{pgfscope}%
\begin{pgfscope}%
\pgfsetbuttcap%
\pgfsetroundjoin%
\definecolor{currentfill}{rgb}{0.317647,0.317647,0.317647}%
\pgfsetfillcolor{currentfill}%
\pgfsetlinewidth{0.501875pt}%
\definecolor{currentstroke}{rgb}{0.317647,0.317647,0.317647}%
\pgfsetstrokecolor{currentstroke}%
\pgfsetdash{}{0pt}%
\pgfsys@defobject{currentmarker}{\pgfqpoint{-0.020833in}{0.000000in}}{\pgfqpoint{0.000000in}{0.000000in}}{%
\pgfpathmoveto{\pgfqpoint{0.000000in}{0.000000in}}%
\pgfpathlineto{\pgfqpoint{-0.020833in}{0.000000in}}%
\pgfusepath{stroke,fill}%
}%
\begin{pgfscope}%
\pgfsys@transformshift{0.488751in}{3.277762in}%
\pgfsys@useobject{currentmarker}{}%
\end{pgfscope}%
\end{pgfscope}%
\begin{pgfscope}%
\definecolor{textcolor}{rgb}{0.317647,0.317647,0.317647}%
\pgfsetstrokecolor{textcolor}%
\pgfsetfillcolor{textcolor}%
\pgftext[x=0.343303in,y=3.245645in,left,base]{\color{textcolor}\rmfamily\fontsize{6.664000}{7.996800}\selectfont \(\displaystyle 40\)}%
\end{pgfscope}%
\begin{pgfscope}%
\definecolor{textcolor}{rgb}{0.317647,0.317647,0.317647}%
\pgfsetstrokecolor{textcolor}%
\pgfsetfillcolor{textcolor}%
\pgftext[x=0.200942in,y=2.806878in,,bottom,rotate=90.000000]{\color{textcolor}\rmfamily\fontsize{6.664000}{7.996800}\selectfont \(\displaystyle W^{(\mathrm{h})}\)}%
\end{pgfscope}%
\begin{pgfscope}%
\pgfpathrectangle{\pgfqpoint{0.488751in}{2.165212in}}{\pgfqpoint{2.260417in}{1.283333in}}%
\pgfusepath{clip}%
\pgfsetrectcap%
\pgfsetroundjoin%
\pgfsetlinewidth{0.803000pt}%
\definecolor{currentstroke}{rgb}{0.333333,0.333333,0.333333}%
\pgfsetstrokecolor{currentstroke}%
\pgfsetdash{}{0pt}%
\pgfpathmoveto{\pgfqpoint{0.591497in}{3.010694in}}%
\pgfpathlineto{\pgfqpoint{0.595615in}{3.052862in}}%
\pgfpathlineto{\pgfqpoint{0.599733in}{3.052862in}}%
\pgfpathlineto{\pgfqpoint{0.603851in}{3.038806in}}%
\pgfpathlineto{\pgfqpoint{0.612087in}{3.137200in}}%
\pgfpathlineto{\pgfqpoint{0.616206in}{3.109087in}}%
\pgfpathlineto{\pgfqpoint{0.620324in}{3.095031in}}%
\pgfpathlineto{\pgfqpoint{0.624442in}{3.095031in}}%
\pgfpathlineto{\pgfqpoint{0.628560in}{3.151256in}}%
\pgfpathlineto{\pgfqpoint{0.636796in}{3.151256in}}%
\pgfpathlineto{\pgfqpoint{0.640914in}{3.137200in}}%
\pgfpathlineto{\pgfqpoint{0.645032in}{3.193425in}}%
\pgfpathlineto{\pgfqpoint{0.649150in}{3.179368in}}%
\pgfpathlineto{\pgfqpoint{0.653268in}{3.123143in}}%
\pgfpathlineto{\pgfqpoint{0.657386in}{3.151256in}}%
\pgfpathlineto{\pgfqpoint{0.661504in}{3.193425in}}%
\pgfpathlineto{\pgfqpoint{0.665623in}{3.221537in}}%
\pgfpathlineto{\pgfqpoint{0.669741in}{3.109087in}}%
\pgfpathlineto{\pgfqpoint{0.673859in}{3.123143in}}%
\pgfpathlineto{\pgfqpoint{0.682095in}{3.038806in}}%
\pgfpathlineto{\pgfqpoint{0.686213in}{3.052862in}}%
\pgfpathlineto{\pgfqpoint{0.690331in}{3.024750in}}%
\pgfpathlineto{\pgfqpoint{0.694449in}{3.024750in}}%
\pgfpathlineto{\pgfqpoint{0.706803in}{2.982581in}}%
\pgfpathlineto{\pgfqpoint{0.710922in}{2.996637in}}%
\pgfpathlineto{\pgfqpoint{0.715040in}{2.968525in}}%
\pgfpathlineto{\pgfqpoint{0.719158in}{2.982581in}}%
\pgfpathlineto{\pgfqpoint{0.723276in}{2.968525in}}%
\pgfpathlineto{\pgfqpoint{0.727394in}{2.968525in}}%
\pgfpathlineto{\pgfqpoint{0.731512in}{2.982581in}}%
\pgfpathlineto{\pgfqpoint{0.739748in}{2.982581in}}%
\pgfpathlineto{\pgfqpoint{0.743866in}{2.996637in}}%
\pgfpathlineto{\pgfqpoint{0.747984in}{2.968525in}}%
\pgfpathlineto{\pgfqpoint{0.768575in}{2.968525in}}%
\pgfpathlineto{\pgfqpoint{0.772693in}{2.982581in}}%
\pgfpathlineto{\pgfqpoint{0.776811in}{2.982581in}}%
\pgfpathlineto{\pgfqpoint{0.785047in}{2.954469in}}%
\pgfpathlineto{\pgfqpoint{0.789165in}{2.968525in}}%
\pgfpathlineto{\pgfqpoint{0.805637in}{2.968525in}}%
\pgfpathlineto{\pgfqpoint{0.809756in}{3.010694in}}%
\pgfpathlineto{\pgfqpoint{0.813874in}{3.024750in}}%
\pgfpathlineto{\pgfqpoint{0.822110in}{3.024750in}}%
\pgfpathlineto{\pgfqpoint{0.826228in}{3.010694in}}%
\pgfpathlineto{\pgfqpoint{0.830346in}{2.982581in}}%
\pgfpathlineto{\pgfqpoint{0.834464in}{2.982581in}}%
\pgfpathlineto{\pgfqpoint{0.838582in}{3.010694in}}%
\pgfpathlineto{\pgfqpoint{0.842700in}{3.010694in}}%
\pgfpathlineto{\pgfqpoint{0.846818in}{3.024750in}}%
\pgfpathlineto{\pgfqpoint{0.850936in}{3.052862in}}%
\pgfpathlineto{\pgfqpoint{0.855054in}{3.038806in}}%
\pgfpathlineto{\pgfqpoint{0.883881in}{3.038806in}}%
\pgfpathlineto{\pgfqpoint{0.892117in}{3.066918in}}%
\pgfpathlineto{\pgfqpoint{0.896235in}{3.038806in}}%
\pgfpathlineto{\pgfqpoint{0.900353in}{3.066918in}}%
\pgfpathlineto{\pgfqpoint{0.904471in}{3.080975in}}%
\pgfpathlineto{\pgfqpoint{0.908590in}{3.080975in}}%
\pgfpathlineto{\pgfqpoint{0.912708in}{3.066918in}}%
\pgfpathlineto{\pgfqpoint{0.916826in}{3.066918in}}%
\pgfpathlineto{\pgfqpoint{0.920944in}{3.080975in}}%
\pgfpathlineto{\pgfqpoint{0.925062in}{3.080975in}}%
\pgfpathlineto{\pgfqpoint{0.933298in}{3.052862in}}%
\pgfpathlineto{\pgfqpoint{0.937416in}{3.052862in}}%
\pgfpathlineto{\pgfqpoint{0.941534in}{3.038806in}}%
\pgfpathlineto{\pgfqpoint{0.978597in}{3.038806in}}%
\pgfpathlineto{\pgfqpoint{0.982715in}{3.024750in}}%
\pgfpathlineto{\pgfqpoint{0.990951in}{3.024750in}}%
\pgfpathlineto{\pgfqpoint{0.995069in}{3.010694in}}%
\pgfpathlineto{\pgfqpoint{0.999187in}{3.024750in}}%
\pgfpathlineto{\pgfqpoint{1.003306in}{3.010694in}}%
\pgfpathlineto{\pgfqpoint{1.069195in}{3.010694in}}%
\pgfpathlineto{\pgfqpoint{1.073313in}{3.024750in}}%
\pgfpathlineto{\pgfqpoint{1.081549in}{3.024750in}}%
\pgfpathlineto{\pgfqpoint{1.085667in}{3.052862in}}%
\pgfpathlineto{\pgfqpoint{1.089785in}{3.052862in}}%
\pgfpathlineto{\pgfqpoint{1.093903in}{3.038806in}}%
\pgfpathlineto{\pgfqpoint{1.110376in}{3.038806in}}%
\pgfpathlineto{\pgfqpoint{1.114494in}{3.052862in}}%
\pgfpathlineto{\pgfqpoint{1.118612in}{3.052862in}}%
\pgfpathlineto{\pgfqpoint{1.122730in}{3.038806in}}%
\pgfpathlineto{\pgfqpoint{1.126848in}{2.996637in}}%
\pgfpathlineto{\pgfqpoint{1.130966in}{2.982581in}}%
\pgfpathlineto{\pgfqpoint{1.168029in}{2.982581in}}%
\pgfpathlineto{\pgfqpoint{1.172147in}{2.996637in}}%
\pgfpathlineto{\pgfqpoint{1.180383in}{2.996637in}}%
\pgfpathlineto{\pgfqpoint{1.184501in}{3.024750in}}%
\pgfpathlineto{\pgfqpoint{1.188619in}{2.996637in}}%
\pgfpathlineto{\pgfqpoint{1.196855in}{2.996637in}}%
\pgfpathlineto{\pgfqpoint{1.200974in}{3.010694in}}%
\pgfpathlineto{\pgfqpoint{1.217446in}{3.010694in}}%
\pgfpathlineto{\pgfqpoint{1.221564in}{3.024750in}}%
\pgfpathlineto{\pgfqpoint{1.225682in}{3.024750in}}%
\pgfpathlineto{\pgfqpoint{1.229800in}{3.052862in}}%
\pgfpathlineto{\pgfqpoint{1.246273in}{3.052862in}}%
\pgfpathlineto{\pgfqpoint{1.250391in}{3.024750in}}%
\pgfpathlineto{\pgfqpoint{1.254509in}{3.024750in}}%
\pgfpathlineto{\pgfqpoint{1.258627in}{3.010694in}}%
\pgfpathlineto{\pgfqpoint{1.262745in}{3.024750in}}%
\pgfpathlineto{\pgfqpoint{1.266863in}{3.024750in}}%
\pgfpathlineto{\pgfqpoint{1.270981in}{3.038806in}}%
\pgfpathlineto{\pgfqpoint{1.303926in}{3.038806in}}%
\pgfpathlineto{\pgfqpoint{1.308044in}{3.010694in}}%
\pgfpathlineto{\pgfqpoint{1.312162in}{3.024750in}}%
\pgfpathlineto{\pgfqpoint{1.316280in}{3.024750in}}%
\pgfpathlineto{\pgfqpoint{1.320398in}{2.996637in}}%
\pgfpathlineto{\pgfqpoint{1.340988in}{2.996637in}}%
\pgfpathlineto{\pgfqpoint{1.345107in}{3.010694in}}%
\pgfpathlineto{\pgfqpoint{1.382169in}{3.010694in}}%
\pgfpathlineto{\pgfqpoint{1.394524in}{3.052862in}}%
\pgfpathlineto{\pgfqpoint{1.419232in}{3.052862in}}%
\pgfpathlineto{\pgfqpoint{1.423350in}{3.038806in}}%
\pgfpathlineto{\pgfqpoint{1.435704in}{3.038806in}}%
\pgfpathlineto{\pgfqpoint{1.439822in}{3.066918in}}%
\pgfpathlineto{\pgfqpoint{1.464531in}{3.066918in}}%
\pgfpathlineto{\pgfqpoint{1.468649in}{3.038806in}}%
\pgfpathlineto{\pgfqpoint{1.481003in}{3.038806in}}%
\pgfpathlineto{\pgfqpoint{1.489240in}{3.095031in}}%
\pgfpathlineto{\pgfqpoint{1.493358in}{3.080975in}}%
\pgfpathlineto{\pgfqpoint{1.497476in}{3.080975in}}%
\pgfpathlineto{\pgfqpoint{1.501594in}{3.066918in}}%
\pgfpathlineto{\pgfqpoint{1.509830in}{3.066918in}}%
\pgfpathlineto{\pgfqpoint{1.513948in}{3.052862in}}%
\pgfpathlineto{\pgfqpoint{1.518066in}{3.066918in}}%
\pgfpathlineto{\pgfqpoint{1.522184in}{3.066918in}}%
\pgfpathlineto{\pgfqpoint{1.530420in}{3.038806in}}%
\pgfpathlineto{\pgfqpoint{1.534538in}{3.010694in}}%
\pgfpathlineto{\pgfqpoint{1.563365in}{3.010694in}}%
\pgfpathlineto{\pgfqpoint{1.567483in}{3.024750in}}%
\pgfpathlineto{\pgfqpoint{1.579837in}{3.024750in}}%
\pgfpathlineto{\pgfqpoint{1.583955in}{3.052862in}}%
\pgfpathlineto{\pgfqpoint{1.596310in}{3.052862in}}%
\pgfpathlineto{\pgfqpoint{1.600428in}{3.024750in}}%
\pgfpathlineto{\pgfqpoint{1.604546in}{3.024750in}}%
\pgfpathlineto{\pgfqpoint{1.608664in}{3.010694in}}%
\pgfpathlineto{\pgfqpoint{1.612782in}{3.010694in}}%
\pgfpathlineto{\pgfqpoint{1.616900in}{2.996637in}}%
\pgfpathlineto{\pgfqpoint{1.621018in}{3.010694in}}%
\pgfpathlineto{\pgfqpoint{1.625136in}{3.038806in}}%
\pgfpathlineto{\pgfqpoint{1.645727in}{3.038806in}}%
\pgfpathlineto{\pgfqpoint{1.649845in}{3.052862in}}%
\pgfpathlineto{\pgfqpoint{1.658081in}{3.052862in}}%
\pgfpathlineto{\pgfqpoint{1.662199in}{3.024750in}}%
\pgfpathlineto{\pgfqpoint{1.674553in}{3.024750in}}%
\pgfpathlineto{\pgfqpoint{1.678671in}{3.052862in}}%
\pgfpathlineto{\pgfqpoint{1.682789in}{3.066918in}}%
\pgfpathlineto{\pgfqpoint{1.703380in}{3.066918in}}%
\pgfpathlineto{\pgfqpoint{1.707498in}{3.052862in}}%
\pgfpathlineto{\pgfqpoint{1.711616in}{3.024750in}}%
\pgfpathlineto{\pgfqpoint{1.715734in}{3.010694in}}%
\pgfpathlineto{\pgfqpoint{1.723970in}{3.010694in}}%
\pgfpathlineto{\pgfqpoint{1.732206in}{3.066918in}}%
\pgfpathlineto{\pgfqpoint{1.756915in}{3.066918in}}%
\pgfpathlineto{\pgfqpoint{1.761033in}{3.095031in}}%
\pgfpathlineto{\pgfqpoint{1.773387in}{3.095031in}}%
\pgfpathlineto{\pgfqpoint{1.777505in}{3.109087in}}%
\pgfpathlineto{\pgfqpoint{1.781624in}{3.109087in}}%
\pgfpathlineto{\pgfqpoint{1.785742in}{3.123143in}}%
\pgfpathlineto{\pgfqpoint{1.793978in}{3.123143in}}%
\pgfpathlineto{\pgfqpoint{1.798096in}{3.109087in}}%
\pgfpathlineto{\pgfqpoint{1.802214in}{3.123143in}}%
\pgfpathlineto{\pgfqpoint{1.814568in}{3.123143in}}%
\pgfpathlineto{\pgfqpoint{1.818686in}{3.095031in}}%
\pgfpathlineto{\pgfqpoint{1.822804in}{3.109087in}}%
\pgfpathlineto{\pgfqpoint{1.826922in}{3.095031in}}%
\pgfpathlineto{\pgfqpoint{1.847513in}{3.095031in}}%
\pgfpathlineto{\pgfqpoint{1.855749in}{3.066918in}}%
\pgfpathlineto{\pgfqpoint{1.896930in}{3.066918in}}%
\pgfpathlineto{\pgfqpoint{1.901048in}{3.038806in}}%
\pgfpathlineto{\pgfqpoint{1.917520in}{3.038806in}}%
\pgfpathlineto{\pgfqpoint{1.921638in}{3.024750in}}%
\pgfpathlineto{\pgfqpoint{1.938111in}{3.024750in}}%
\pgfpathlineto{\pgfqpoint{1.942229in}{3.010694in}}%
\pgfpathlineto{\pgfqpoint{1.946347in}{3.024750in}}%
\pgfpathlineto{\pgfqpoint{1.971055in}{3.024750in}}%
\pgfpathlineto{\pgfqpoint{1.975173in}{3.038806in}}%
\pgfpathlineto{\pgfqpoint{2.028709in}{3.038806in}}%
\pgfpathlineto{\pgfqpoint{2.032827in}{3.052862in}}%
\pgfpathlineto{\pgfqpoint{2.036945in}{3.038806in}}%
\pgfpathlineto{\pgfqpoint{2.041063in}{3.038806in}}%
\pgfpathlineto{\pgfqpoint{2.045181in}{3.010694in}}%
\pgfpathlineto{\pgfqpoint{2.061653in}{3.010694in}}%
\pgfpathlineto{\pgfqpoint{2.065771in}{2.982581in}}%
\pgfpathlineto{\pgfqpoint{2.069889in}{2.982581in}}%
\pgfpathlineto{\pgfqpoint{2.074008in}{2.968525in}}%
\pgfpathlineto{\pgfqpoint{2.168723in}{2.968525in}}%
\pgfpathlineto{\pgfqpoint{2.172842in}{2.982581in}}%
\pgfpathlineto{\pgfqpoint{2.181078in}{2.982581in}}%
\pgfpathlineto{\pgfqpoint{2.185196in}{3.024750in}}%
\pgfpathlineto{\pgfqpoint{2.189314in}{3.024750in}}%
\pgfpathlineto{\pgfqpoint{2.193432in}{3.052862in}}%
\pgfpathlineto{\pgfqpoint{2.197550in}{3.038806in}}%
\pgfpathlineto{\pgfqpoint{2.201668in}{3.080975in}}%
\pgfpathlineto{\pgfqpoint{2.205786in}{3.080975in}}%
\pgfpathlineto{\pgfqpoint{2.209904in}{3.109087in}}%
\pgfpathlineto{\pgfqpoint{2.218140in}{3.137200in}}%
\pgfpathlineto{\pgfqpoint{2.222259in}{3.123143in}}%
\pgfpathlineto{\pgfqpoint{2.226377in}{3.151256in}}%
\pgfpathlineto{\pgfqpoint{2.230495in}{3.137200in}}%
\pgfpathlineto{\pgfqpoint{2.234613in}{3.151256in}}%
\pgfpathlineto{\pgfqpoint{2.242849in}{3.123143in}}%
\pgfpathlineto{\pgfqpoint{2.259321in}{3.123143in}}%
\pgfpathlineto{\pgfqpoint{2.267557in}{3.151256in}}%
\pgfpathlineto{\pgfqpoint{2.271676in}{3.179368in}}%
\pgfpathlineto{\pgfqpoint{2.275794in}{3.193425in}}%
\pgfpathlineto{\pgfqpoint{2.279912in}{3.165312in}}%
\pgfpathlineto{\pgfqpoint{2.284030in}{3.123143in}}%
\pgfpathlineto{\pgfqpoint{2.288148in}{3.123143in}}%
\pgfpathlineto{\pgfqpoint{2.292266in}{3.151256in}}%
\pgfpathlineto{\pgfqpoint{2.296384in}{3.151256in}}%
\pgfpathlineto{\pgfqpoint{2.300502in}{3.179368in}}%
\pgfpathlineto{\pgfqpoint{2.304620in}{3.137200in}}%
\pgfpathlineto{\pgfqpoint{2.308738in}{3.151256in}}%
\pgfpathlineto{\pgfqpoint{2.325211in}{3.151256in}}%
\pgfpathlineto{\pgfqpoint{2.329329in}{3.179368in}}%
\pgfpathlineto{\pgfqpoint{2.333447in}{3.151256in}}%
\pgfpathlineto{\pgfqpoint{2.337565in}{3.151256in}}%
\pgfpathlineto{\pgfqpoint{2.345801in}{3.179368in}}%
\pgfpathlineto{\pgfqpoint{2.349919in}{3.151256in}}%
\pgfpathlineto{\pgfqpoint{2.362273in}{3.151256in}}%
\pgfpathlineto{\pgfqpoint{2.366392in}{3.165312in}}%
\pgfpathlineto{\pgfqpoint{2.370510in}{3.151256in}}%
\pgfpathlineto{\pgfqpoint{2.374628in}{3.193425in}}%
\pgfpathlineto{\pgfqpoint{2.378746in}{3.179368in}}%
\pgfpathlineto{\pgfqpoint{2.382864in}{3.179368in}}%
\pgfpathlineto{\pgfqpoint{2.386982in}{3.151256in}}%
\pgfpathlineto{\pgfqpoint{2.391100in}{3.151256in}}%
\pgfpathlineto{\pgfqpoint{2.395218in}{3.165312in}}%
\pgfpathlineto{\pgfqpoint{2.399336in}{3.193425in}}%
\pgfpathlineto{\pgfqpoint{2.403454in}{3.193425in}}%
\pgfpathlineto{\pgfqpoint{2.407572in}{3.221537in}}%
\pgfpathlineto{\pgfqpoint{2.411690in}{3.193425in}}%
\pgfpathlineto{\pgfqpoint{2.415809in}{3.151256in}}%
\pgfpathlineto{\pgfqpoint{2.419927in}{3.151256in}}%
\pgfpathlineto{\pgfqpoint{2.424045in}{3.137200in}}%
\pgfpathlineto{\pgfqpoint{2.428163in}{3.151256in}}%
\pgfpathlineto{\pgfqpoint{2.436399in}{3.151256in}}%
\pgfpathlineto{\pgfqpoint{2.440517in}{3.123143in}}%
\pgfpathlineto{\pgfqpoint{2.448753in}{3.123143in}}%
\pgfpathlineto{\pgfqpoint{2.452871in}{3.137200in}}%
\pgfpathlineto{\pgfqpoint{2.456989in}{3.123143in}}%
\pgfpathlineto{\pgfqpoint{2.461107in}{3.137200in}}%
\pgfpathlineto{\pgfqpoint{2.469344in}{3.109087in}}%
\pgfpathlineto{\pgfqpoint{2.473462in}{3.137200in}}%
\pgfpathlineto{\pgfqpoint{2.477580in}{3.109087in}}%
\pgfpathlineto{\pgfqpoint{2.481698in}{3.095031in}}%
\pgfpathlineto{\pgfqpoint{2.485816in}{3.095031in}}%
\pgfpathlineto{\pgfqpoint{2.489934in}{3.066918in}}%
\pgfpathlineto{\pgfqpoint{2.494052in}{3.095031in}}%
\pgfpathlineto{\pgfqpoint{2.498170in}{3.080975in}}%
\pgfpathlineto{\pgfqpoint{2.502288in}{3.095031in}}%
\pgfpathlineto{\pgfqpoint{2.506406in}{3.095031in}}%
\pgfpathlineto{\pgfqpoint{2.510524in}{3.080975in}}%
\pgfpathlineto{\pgfqpoint{2.514643in}{3.080975in}}%
\pgfpathlineto{\pgfqpoint{2.522879in}{3.109087in}}%
\pgfpathlineto{\pgfqpoint{2.526997in}{3.080975in}}%
\pgfpathlineto{\pgfqpoint{2.531115in}{3.080975in}}%
\pgfpathlineto{\pgfqpoint{2.535233in}{3.095031in}}%
\pgfpathlineto{\pgfqpoint{2.539351in}{3.080975in}}%
\pgfpathlineto{\pgfqpoint{2.543469in}{3.080975in}}%
\pgfpathlineto{\pgfqpoint{2.547587in}{3.095031in}}%
\pgfpathlineto{\pgfqpoint{2.551705in}{3.080975in}}%
\pgfpathlineto{\pgfqpoint{2.564060in}{3.080975in}}%
\pgfpathlineto{\pgfqpoint{2.568178in}{3.066918in}}%
\pgfpathlineto{\pgfqpoint{2.588768in}{3.066918in}}%
\pgfpathlineto{\pgfqpoint{2.592886in}{3.052862in}}%
\pgfpathlineto{\pgfqpoint{2.605240in}{3.052862in}}%
\pgfpathlineto{\pgfqpoint{2.613477in}{3.080975in}}%
\pgfpathlineto{\pgfqpoint{2.617595in}{3.080975in}}%
\pgfpathlineto{\pgfqpoint{2.621713in}{3.066918in}}%
\pgfpathlineto{\pgfqpoint{2.634067in}{3.066918in}}%
\pgfpathlineto{\pgfqpoint{2.638185in}{3.052862in}}%
\pgfpathlineto{\pgfqpoint{2.642303in}{3.066918in}}%
\pgfpathlineto{\pgfqpoint{2.646421in}{3.066918in}}%
\pgfpathlineto{\pgfqpoint{2.646421in}{3.066918in}}%
\pgfusepath{stroke}%
\end{pgfscope}%
\begin{pgfscope}%
\pgfpathrectangle{\pgfqpoint{0.488751in}{2.165212in}}{\pgfqpoint{2.260417in}{1.283333in}}%
\pgfusepath{clip}%
\pgfsetrectcap%
\pgfsetroundjoin%
\pgfsetlinewidth{0.803000pt}%
\definecolor{currentstroke}{rgb}{0.686275,0.352941,0.313725}%
\pgfsetstrokecolor{currentstroke}%
\pgfsetdash{}{0pt}%
\pgfpathmoveto{\pgfqpoint{0.591497in}{2.532782in}}%
\pgfpathlineto{\pgfqpoint{0.603851in}{2.532782in}}%
\pgfpathlineto{\pgfqpoint{0.607969in}{2.518726in}}%
\pgfpathlineto{\pgfqpoint{0.612087in}{2.518726in}}%
\pgfpathlineto{\pgfqpoint{0.616206in}{2.490613in}}%
\pgfpathlineto{\pgfqpoint{0.620324in}{2.504669in}}%
\pgfpathlineto{\pgfqpoint{0.628560in}{2.504669in}}%
\pgfpathlineto{\pgfqpoint{0.640914in}{2.462501in}}%
\pgfpathlineto{\pgfqpoint{0.653268in}{2.462501in}}%
\pgfpathlineto{\pgfqpoint{0.657386in}{2.448445in}}%
\pgfpathlineto{\pgfqpoint{0.669741in}{2.448445in}}%
\pgfpathlineto{\pgfqpoint{0.673859in}{2.434388in}}%
\pgfpathlineto{\pgfqpoint{0.690331in}{2.434388in}}%
\pgfpathlineto{\pgfqpoint{0.694449in}{2.420332in}}%
\pgfpathlineto{\pgfqpoint{0.702685in}{2.420332in}}%
\pgfpathlineto{\pgfqpoint{0.706803in}{2.406276in}}%
\pgfpathlineto{\pgfqpoint{0.756220in}{2.406276in}}%
\pgfpathlineto{\pgfqpoint{0.760339in}{2.420332in}}%
\pgfpathlineto{\pgfqpoint{0.780929in}{2.420332in}}%
\pgfpathlineto{\pgfqpoint{0.785047in}{2.406276in}}%
\pgfpathlineto{\pgfqpoint{0.789165in}{2.406276in}}%
\pgfpathlineto{\pgfqpoint{0.793283in}{2.420332in}}%
\pgfpathlineto{\pgfqpoint{0.801519in}{2.420332in}}%
\pgfpathlineto{\pgfqpoint{0.805637in}{2.434388in}}%
\pgfpathlineto{\pgfqpoint{0.846818in}{2.434388in}}%
\pgfpathlineto{\pgfqpoint{0.855054in}{2.406276in}}%
\pgfpathlineto{\pgfqpoint{0.883881in}{2.406276in}}%
\pgfpathlineto{\pgfqpoint{0.887999in}{2.420332in}}%
\pgfpathlineto{\pgfqpoint{0.982715in}{2.420332in}}%
\pgfpathlineto{\pgfqpoint{0.986833in}{2.434388in}}%
\pgfpathlineto{\pgfqpoint{1.028014in}{2.434388in}}%
\pgfpathlineto{\pgfqpoint{1.032132in}{2.420332in}}%
\pgfpathlineto{\pgfqpoint{1.106258in}{2.420332in}}%
\pgfpathlineto{\pgfqpoint{1.110376in}{2.434388in}}%
\pgfpathlineto{\pgfqpoint{1.114494in}{2.434388in}}%
\pgfpathlineto{\pgfqpoint{1.118612in}{2.420332in}}%
\pgfpathlineto{\pgfqpoint{1.122730in}{2.420332in}}%
\pgfpathlineto{\pgfqpoint{1.126848in}{2.434388in}}%
\pgfpathlineto{\pgfqpoint{1.155675in}{2.434388in}}%
\pgfpathlineto{\pgfqpoint{1.159793in}{2.448445in}}%
\pgfpathlineto{\pgfqpoint{1.184501in}{2.448445in}}%
\pgfpathlineto{\pgfqpoint{1.192737in}{2.476557in}}%
\pgfpathlineto{\pgfqpoint{1.213328in}{2.476557in}}%
\pgfpathlineto{\pgfqpoint{1.217446in}{2.504669in}}%
\pgfpathlineto{\pgfqpoint{1.221564in}{2.504669in}}%
\pgfpathlineto{\pgfqpoint{1.225682in}{2.518726in}}%
\pgfpathlineto{\pgfqpoint{1.258627in}{2.518726in}}%
\pgfpathlineto{\pgfqpoint{1.262745in}{2.532782in}}%
\pgfpathlineto{\pgfqpoint{1.266863in}{2.532782in}}%
\pgfpathlineto{\pgfqpoint{1.270981in}{2.546838in}}%
\pgfpathlineto{\pgfqpoint{1.291571in}{2.546838in}}%
\pgfpathlineto{\pgfqpoint{1.295690in}{2.589007in}}%
\pgfpathlineto{\pgfqpoint{1.299808in}{2.589007in}}%
\pgfpathlineto{\pgfqpoint{1.303926in}{2.603063in}}%
\pgfpathlineto{\pgfqpoint{1.308044in}{2.631176in}}%
\pgfpathlineto{\pgfqpoint{1.312162in}{2.603063in}}%
\pgfpathlineto{\pgfqpoint{1.320398in}{2.603063in}}%
\pgfpathlineto{\pgfqpoint{1.324516in}{2.617119in}}%
\pgfpathlineto{\pgfqpoint{1.336870in}{2.617119in}}%
\pgfpathlineto{\pgfqpoint{1.340988in}{2.631176in}}%
\pgfpathlineto{\pgfqpoint{1.345107in}{2.631176in}}%
\pgfpathlineto{\pgfqpoint{1.349225in}{2.645232in}}%
\pgfpathlineto{\pgfqpoint{1.353343in}{2.645232in}}%
\pgfpathlineto{\pgfqpoint{1.357461in}{2.631176in}}%
\pgfpathlineto{\pgfqpoint{1.373933in}{2.687400in}}%
\pgfpathlineto{\pgfqpoint{1.378051in}{2.673344in}}%
\pgfpathlineto{\pgfqpoint{1.394524in}{2.673344in}}%
\pgfpathlineto{\pgfqpoint{1.398642in}{2.687400in}}%
\pgfpathlineto{\pgfqpoint{1.402760in}{2.659288in}}%
\pgfpathlineto{\pgfqpoint{1.419232in}{2.659288in}}%
\pgfpathlineto{\pgfqpoint{1.423350in}{2.645232in}}%
\pgfpathlineto{\pgfqpoint{1.427468in}{2.659288in}}%
\pgfpathlineto{\pgfqpoint{1.439822in}{2.659288in}}%
\pgfpathlineto{\pgfqpoint{1.443941in}{2.673344in}}%
\pgfpathlineto{\pgfqpoint{1.448059in}{2.673344in}}%
\pgfpathlineto{\pgfqpoint{1.452177in}{2.715513in}}%
\pgfpathlineto{\pgfqpoint{1.456295in}{2.729569in}}%
\pgfpathlineto{\pgfqpoint{1.460413in}{2.673344in}}%
\pgfpathlineto{\pgfqpoint{1.464531in}{2.673344in}}%
\pgfpathlineto{\pgfqpoint{1.468649in}{2.687400in}}%
\pgfpathlineto{\pgfqpoint{1.472767in}{2.715513in}}%
\pgfpathlineto{\pgfqpoint{1.476885in}{2.715513in}}%
\pgfpathlineto{\pgfqpoint{1.481003in}{2.729569in}}%
\pgfpathlineto{\pgfqpoint{1.485121in}{2.757682in}}%
\pgfpathlineto{\pgfqpoint{1.489240in}{2.743625in}}%
\pgfpathlineto{\pgfqpoint{1.505712in}{2.743625in}}%
\pgfpathlineto{\pgfqpoint{1.509830in}{2.757682in}}%
\pgfpathlineto{\pgfqpoint{1.518066in}{2.757682in}}%
\pgfpathlineto{\pgfqpoint{1.526302in}{2.785794in}}%
\pgfpathlineto{\pgfqpoint{1.534538in}{2.785794in}}%
\pgfpathlineto{\pgfqpoint{1.538657in}{2.771738in}}%
\pgfpathlineto{\pgfqpoint{1.542775in}{2.771738in}}%
\pgfpathlineto{\pgfqpoint{1.546893in}{2.785794in}}%
\pgfpathlineto{\pgfqpoint{1.551011in}{2.785794in}}%
\pgfpathlineto{\pgfqpoint{1.555129in}{2.799850in}}%
\pgfpathlineto{\pgfqpoint{1.563365in}{2.799850in}}%
\pgfpathlineto{\pgfqpoint{1.567483in}{2.729569in}}%
\pgfpathlineto{\pgfqpoint{1.579837in}{2.729569in}}%
\pgfpathlineto{\pgfqpoint{1.583955in}{2.757682in}}%
\pgfpathlineto{\pgfqpoint{1.588074in}{2.757682in}}%
\pgfpathlineto{\pgfqpoint{1.592192in}{2.715513in}}%
\pgfpathlineto{\pgfqpoint{1.596310in}{2.729569in}}%
\pgfpathlineto{\pgfqpoint{1.600428in}{2.785794in}}%
\pgfpathlineto{\pgfqpoint{1.604546in}{2.771738in}}%
\pgfpathlineto{\pgfqpoint{1.608664in}{2.715513in}}%
\pgfpathlineto{\pgfqpoint{1.616900in}{2.715513in}}%
\pgfpathlineto{\pgfqpoint{1.621018in}{2.701457in}}%
\pgfpathlineto{\pgfqpoint{1.625136in}{2.659288in}}%
\pgfpathlineto{\pgfqpoint{1.629254in}{2.659288in}}%
\pgfpathlineto{\pgfqpoint{1.633372in}{2.645232in}}%
\pgfpathlineto{\pgfqpoint{1.637491in}{2.617119in}}%
\pgfpathlineto{\pgfqpoint{1.641609in}{2.631176in}}%
\pgfpathlineto{\pgfqpoint{1.645727in}{2.631176in}}%
\pgfpathlineto{\pgfqpoint{1.649845in}{2.659288in}}%
\pgfpathlineto{\pgfqpoint{1.653963in}{2.673344in}}%
\pgfpathlineto{\pgfqpoint{1.658081in}{2.673344in}}%
\pgfpathlineto{\pgfqpoint{1.662199in}{2.631176in}}%
\pgfpathlineto{\pgfqpoint{1.666317in}{2.603063in}}%
\pgfpathlineto{\pgfqpoint{1.670435in}{2.617119in}}%
\pgfpathlineto{\pgfqpoint{1.674553in}{2.645232in}}%
\pgfpathlineto{\pgfqpoint{1.686908in}{2.645232in}}%
\pgfpathlineto{\pgfqpoint{1.691026in}{2.631176in}}%
\pgfpathlineto{\pgfqpoint{1.707498in}{2.631176in}}%
\pgfpathlineto{\pgfqpoint{1.711616in}{2.659288in}}%
\pgfpathlineto{\pgfqpoint{1.715734in}{2.645232in}}%
\pgfpathlineto{\pgfqpoint{1.723970in}{2.673344in}}%
\pgfpathlineto{\pgfqpoint{1.728088in}{2.645232in}}%
\pgfpathlineto{\pgfqpoint{1.732206in}{2.659288in}}%
\pgfpathlineto{\pgfqpoint{1.740443in}{2.659288in}}%
\pgfpathlineto{\pgfqpoint{1.744561in}{2.631176in}}%
\pgfpathlineto{\pgfqpoint{1.756915in}{2.631176in}}%
\pgfpathlineto{\pgfqpoint{1.761033in}{2.617119in}}%
\pgfpathlineto{\pgfqpoint{1.765151in}{2.659288in}}%
\pgfpathlineto{\pgfqpoint{1.769269in}{2.617119in}}%
\pgfpathlineto{\pgfqpoint{1.773387in}{2.631176in}}%
\pgfpathlineto{\pgfqpoint{1.777505in}{2.617119in}}%
\pgfpathlineto{\pgfqpoint{1.781624in}{2.631176in}}%
\pgfpathlineto{\pgfqpoint{1.785742in}{2.617119in}}%
\pgfpathlineto{\pgfqpoint{1.789860in}{2.617119in}}%
\pgfpathlineto{\pgfqpoint{1.798096in}{2.645232in}}%
\pgfpathlineto{\pgfqpoint{1.810450in}{2.645232in}}%
\pgfpathlineto{\pgfqpoint{1.814568in}{2.659288in}}%
\pgfpathlineto{\pgfqpoint{1.818686in}{2.659288in}}%
\pgfpathlineto{\pgfqpoint{1.831041in}{2.574951in}}%
\pgfpathlineto{\pgfqpoint{1.843395in}{2.574951in}}%
\pgfpathlineto{\pgfqpoint{1.847513in}{2.603063in}}%
\pgfpathlineto{\pgfqpoint{1.851631in}{2.603063in}}%
\pgfpathlineto{\pgfqpoint{1.855749in}{2.589007in}}%
\pgfpathlineto{\pgfqpoint{1.859867in}{2.589007in}}%
\pgfpathlineto{\pgfqpoint{1.863985in}{2.603063in}}%
\pgfpathlineto{\pgfqpoint{1.868103in}{2.631176in}}%
\pgfpathlineto{\pgfqpoint{1.872221in}{2.603063in}}%
\pgfpathlineto{\pgfqpoint{1.888694in}{2.603063in}}%
\pgfpathlineto{\pgfqpoint{1.892812in}{2.617119in}}%
\pgfpathlineto{\pgfqpoint{1.896930in}{2.617119in}}%
\pgfpathlineto{\pgfqpoint{1.901048in}{2.603063in}}%
\pgfpathlineto{\pgfqpoint{1.913402in}{2.603063in}}%
\pgfpathlineto{\pgfqpoint{1.917520in}{2.617119in}}%
\pgfpathlineto{\pgfqpoint{1.921638in}{2.603063in}}%
\pgfpathlineto{\pgfqpoint{1.925756in}{2.631176in}}%
\pgfpathlineto{\pgfqpoint{1.929875in}{2.631176in}}%
\pgfpathlineto{\pgfqpoint{1.933993in}{2.617119in}}%
\pgfpathlineto{\pgfqpoint{1.938111in}{2.659288in}}%
\pgfpathlineto{\pgfqpoint{1.942229in}{2.617119in}}%
\pgfpathlineto{\pgfqpoint{1.946347in}{2.617119in}}%
\pgfpathlineto{\pgfqpoint{1.950465in}{2.631176in}}%
\pgfpathlineto{\pgfqpoint{1.954583in}{2.631176in}}%
\pgfpathlineto{\pgfqpoint{1.958701in}{2.687400in}}%
\pgfpathlineto{\pgfqpoint{1.962819in}{2.659288in}}%
\pgfpathlineto{\pgfqpoint{1.971055in}{2.659288in}}%
\pgfpathlineto{\pgfqpoint{1.975173in}{2.617119in}}%
\pgfpathlineto{\pgfqpoint{1.979292in}{2.617119in}}%
\pgfpathlineto{\pgfqpoint{1.983410in}{2.631176in}}%
\pgfpathlineto{\pgfqpoint{1.987528in}{2.673344in}}%
\pgfpathlineto{\pgfqpoint{1.999882in}{2.715513in}}%
\pgfpathlineto{\pgfqpoint{2.016354in}{2.715513in}}%
\pgfpathlineto{\pgfqpoint{2.020472in}{2.701457in}}%
\pgfpathlineto{\pgfqpoint{2.024591in}{2.715513in}}%
\pgfpathlineto{\pgfqpoint{2.028709in}{2.701457in}}%
\pgfpathlineto{\pgfqpoint{2.032827in}{2.729569in}}%
\pgfpathlineto{\pgfqpoint{2.036945in}{2.715513in}}%
\pgfpathlineto{\pgfqpoint{2.049299in}{2.757682in}}%
\pgfpathlineto{\pgfqpoint{2.061653in}{2.757682in}}%
\pgfpathlineto{\pgfqpoint{2.065771in}{2.715513in}}%
\pgfpathlineto{\pgfqpoint{2.069889in}{2.743625in}}%
\pgfpathlineto{\pgfqpoint{2.078126in}{2.743625in}}%
\pgfpathlineto{\pgfqpoint{2.082244in}{2.729569in}}%
\pgfpathlineto{\pgfqpoint{2.090480in}{2.729569in}}%
\pgfpathlineto{\pgfqpoint{2.094598in}{2.757682in}}%
\pgfpathlineto{\pgfqpoint{2.098716in}{2.729569in}}%
\pgfpathlineto{\pgfqpoint{2.106952in}{2.729569in}}%
\pgfpathlineto{\pgfqpoint{2.111070in}{2.743625in}}%
\pgfpathlineto{\pgfqpoint{2.115188in}{2.743625in}}%
\pgfpathlineto{\pgfqpoint{2.119306in}{2.729569in}}%
\pgfpathlineto{\pgfqpoint{2.123425in}{2.729569in}}%
\pgfpathlineto{\pgfqpoint{2.127543in}{2.715513in}}%
\pgfpathlineto{\pgfqpoint{2.139897in}{2.715513in}}%
\pgfpathlineto{\pgfqpoint{2.144015in}{2.687400in}}%
\pgfpathlineto{\pgfqpoint{2.148133in}{2.701457in}}%
\pgfpathlineto{\pgfqpoint{2.160487in}{2.701457in}}%
\pgfpathlineto{\pgfqpoint{2.164605in}{2.729569in}}%
\pgfpathlineto{\pgfqpoint{2.181078in}{2.729569in}}%
\pgfpathlineto{\pgfqpoint{2.185196in}{2.743625in}}%
\pgfpathlineto{\pgfqpoint{2.193432in}{2.743625in}}%
\pgfpathlineto{\pgfqpoint{2.201668in}{2.715513in}}%
\pgfpathlineto{\pgfqpoint{2.205786in}{2.743625in}}%
\pgfpathlineto{\pgfqpoint{2.209904in}{2.729569in}}%
\pgfpathlineto{\pgfqpoint{2.214022in}{2.743625in}}%
\pgfpathlineto{\pgfqpoint{2.222259in}{2.743625in}}%
\pgfpathlineto{\pgfqpoint{2.226377in}{2.715513in}}%
\pgfpathlineto{\pgfqpoint{2.230495in}{2.729569in}}%
\pgfpathlineto{\pgfqpoint{2.234613in}{2.729569in}}%
\pgfpathlineto{\pgfqpoint{2.238731in}{2.743625in}}%
\pgfpathlineto{\pgfqpoint{2.267557in}{2.743625in}}%
\pgfpathlineto{\pgfqpoint{2.271676in}{2.729569in}}%
\pgfpathlineto{\pgfqpoint{2.284030in}{2.729569in}}%
\pgfpathlineto{\pgfqpoint{2.288148in}{2.743625in}}%
\pgfpathlineto{\pgfqpoint{2.292266in}{2.729569in}}%
\pgfpathlineto{\pgfqpoint{2.296384in}{2.729569in}}%
\pgfpathlineto{\pgfqpoint{2.300502in}{2.701457in}}%
\pgfpathlineto{\pgfqpoint{2.304620in}{2.701457in}}%
\pgfpathlineto{\pgfqpoint{2.308738in}{2.659288in}}%
\pgfpathlineto{\pgfqpoint{2.321093in}{2.659288in}}%
\pgfpathlineto{\pgfqpoint{2.325211in}{2.673344in}}%
\pgfpathlineto{\pgfqpoint{2.329329in}{2.673344in}}%
\pgfpathlineto{\pgfqpoint{2.333447in}{2.687400in}}%
\pgfpathlineto{\pgfqpoint{2.358155in}{2.687400in}}%
\pgfpathlineto{\pgfqpoint{2.362273in}{2.673344in}}%
\pgfpathlineto{\pgfqpoint{2.386982in}{2.673344in}}%
\pgfpathlineto{\pgfqpoint{2.391100in}{2.659288in}}%
\pgfpathlineto{\pgfqpoint{2.395218in}{2.659288in}}%
\pgfpathlineto{\pgfqpoint{2.399336in}{2.645232in}}%
\pgfpathlineto{\pgfqpoint{2.407572in}{2.645232in}}%
\pgfpathlineto{\pgfqpoint{2.411690in}{2.659288in}}%
\pgfpathlineto{\pgfqpoint{2.436399in}{2.659288in}}%
\pgfpathlineto{\pgfqpoint{2.440517in}{2.673344in}}%
\pgfpathlineto{\pgfqpoint{2.444635in}{2.673344in}}%
\pgfpathlineto{\pgfqpoint{2.448753in}{2.687400in}}%
\pgfpathlineto{\pgfqpoint{2.461107in}{2.687400in}}%
\pgfpathlineto{\pgfqpoint{2.465226in}{2.701457in}}%
\pgfpathlineto{\pgfqpoint{2.473462in}{2.701457in}}%
\pgfpathlineto{\pgfqpoint{2.481698in}{2.729569in}}%
\pgfpathlineto{\pgfqpoint{2.485816in}{2.715513in}}%
\pgfpathlineto{\pgfqpoint{2.489934in}{2.715513in}}%
\pgfpathlineto{\pgfqpoint{2.494052in}{2.701457in}}%
\pgfpathlineto{\pgfqpoint{2.506406in}{2.701457in}}%
\pgfpathlineto{\pgfqpoint{2.510524in}{2.715513in}}%
\pgfpathlineto{\pgfqpoint{2.522879in}{2.715513in}}%
\pgfpathlineto{\pgfqpoint{2.526997in}{2.729569in}}%
\pgfpathlineto{\pgfqpoint{2.531115in}{2.715513in}}%
\pgfpathlineto{\pgfqpoint{2.535233in}{2.715513in}}%
\pgfpathlineto{\pgfqpoint{2.543469in}{2.743625in}}%
\pgfpathlineto{\pgfqpoint{2.551705in}{2.743625in}}%
\pgfpathlineto{\pgfqpoint{2.555823in}{2.757682in}}%
\pgfpathlineto{\pgfqpoint{2.559942in}{2.757682in}}%
\pgfpathlineto{\pgfqpoint{2.564060in}{2.743625in}}%
\pgfpathlineto{\pgfqpoint{2.572296in}{2.743625in}}%
\pgfpathlineto{\pgfqpoint{2.576414in}{2.757682in}}%
\pgfpathlineto{\pgfqpoint{2.580532in}{2.743625in}}%
\pgfpathlineto{\pgfqpoint{2.584650in}{2.743625in}}%
\pgfpathlineto{\pgfqpoint{2.588768in}{2.729569in}}%
\pgfpathlineto{\pgfqpoint{2.605240in}{2.729569in}}%
\pgfpathlineto{\pgfqpoint{2.609359in}{2.715513in}}%
\pgfpathlineto{\pgfqpoint{2.646421in}{2.715513in}}%
\pgfpathlineto{\pgfqpoint{2.646421in}{2.715513in}}%
\pgfusepath{stroke}%
\end{pgfscope}%
\begin{pgfscope}%
\pgfpathrectangle{\pgfqpoint{0.488751in}{2.165212in}}{\pgfqpoint{2.260417in}{1.283333in}}%
\pgfusepath{clip}%
\pgfsetrectcap%
\pgfsetroundjoin%
\pgfsetlinewidth{0.803000pt}%
\definecolor{currentstroke}{rgb}{0.000000,0.356863,0.509804}%
\pgfsetstrokecolor{currentstroke}%
\pgfsetdash{}{0pt}%
\pgfpathmoveto{\pgfqpoint{0.591497in}{2.926356in}}%
\pgfpathlineto{\pgfqpoint{0.599733in}{2.898244in}}%
\pgfpathlineto{\pgfqpoint{0.603851in}{2.940412in}}%
\pgfpathlineto{\pgfqpoint{0.607969in}{2.898244in}}%
\pgfpathlineto{\pgfqpoint{0.612087in}{2.940412in}}%
\pgfpathlineto{\pgfqpoint{0.616206in}{2.954469in}}%
\pgfpathlineto{\pgfqpoint{0.620324in}{2.954469in}}%
\pgfpathlineto{\pgfqpoint{0.624442in}{2.968525in}}%
\pgfpathlineto{\pgfqpoint{0.628560in}{3.010694in}}%
\pgfpathlineto{\pgfqpoint{0.632678in}{3.024750in}}%
\pgfpathlineto{\pgfqpoint{0.636796in}{3.024750in}}%
\pgfpathlineto{\pgfqpoint{0.640914in}{3.052862in}}%
\pgfpathlineto{\pgfqpoint{0.645032in}{3.066918in}}%
\pgfpathlineto{\pgfqpoint{0.649150in}{3.095031in}}%
\pgfpathlineto{\pgfqpoint{0.653268in}{3.109087in}}%
\pgfpathlineto{\pgfqpoint{0.657386in}{3.038806in}}%
\pgfpathlineto{\pgfqpoint{0.661504in}{3.010694in}}%
\pgfpathlineto{\pgfqpoint{0.665623in}{3.052862in}}%
\pgfpathlineto{\pgfqpoint{0.669741in}{3.052862in}}%
\pgfpathlineto{\pgfqpoint{0.673859in}{3.024750in}}%
\pgfpathlineto{\pgfqpoint{0.682095in}{3.080975in}}%
\pgfpathlineto{\pgfqpoint{0.686213in}{3.052862in}}%
\pgfpathlineto{\pgfqpoint{0.690331in}{3.038806in}}%
\pgfpathlineto{\pgfqpoint{0.694449in}{3.010694in}}%
\pgfpathlineto{\pgfqpoint{0.702685in}{2.982581in}}%
\pgfpathlineto{\pgfqpoint{0.706803in}{2.982581in}}%
\pgfpathlineto{\pgfqpoint{0.719158in}{2.940412in}}%
\pgfpathlineto{\pgfqpoint{0.723276in}{2.954469in}}%
\pgfpathlineto{\pgfqpoint{0.727394in}{2.954469in}}%
\pgfpathlineto{\pgfqpoint{0.731512in}{2.968525in}}%
\pgfpathlineto{\pgfqpoint{0.735630in}{2.968525in}}%
\pgfpathlineto{\pgfqpoint{0.739748in}{2.982581in}}%
\pgfpathlineto{\pgfqpoint{0.743866in}{2.968525in}}%
\pgfpathlineto{\pgfqpoint{0.747984in}{2.940412in}}%
\pgfpathlineto{\pgfqpoint{0.760339in}{2.940412in}}%
\pgfpathlineto{\pgfqpoint{0.764457in}{2.968525in}}%
\pgfpathlineto{\pgfqpoint{0.768575in}{2.982581in}}%
\pgfpathlineto{\pgfqpoint{0.772693in}{2.968525in}}%
\pgfpathlineto{\pgfqpoint{0.776811in}{3.024750in}}%
\pgfpathlineto{\pgfqpoint{0.785047in}{3.052862in}}%
\pgfpathlineto{\pgfqpoint{0.789165in}{3.052862in}}%
\pgfpathlineto{\pgfqpoint{0.793283in}{3.038806in}}%
\pgfpathlineto{\pgfqpoint{0.801519in}{3.038806in}}%
\pgfpathlineto{\pgfqpoint{0.805637in}{3.052862in}}%
\pgfpathlineto{\pgfqpoint{0.809756in}{3.024750in}}%
\pgfpathlineto{\pgfqpoint{0.813874in}{3.024750in}}%
\pgfpathlineto{\pgfqpoint{0.817992in}{3.038806in}}%
\pgfpathlineto{\pgfqpoint{0.822110in}{3.038806in}}%
\pgfpathlineto{\pgfqpoint{0.826228in}{3.024750in}}%
\pgfpathlineto{\pgfqpoint{0.830346in}{3.024750in}}%
\pgfpathlineto{\pgfqpoint{0.834464in}{3.038806in}}%
\pgfpathlineto{\pgfqpoint{0.838582in}{3.010694in}}%
\pgfpathlineto{\pgfqpoint{0.842700in}{3.010694in}}%
\pgfpathlineto{\pgfqpoint{0.846818in}{2.968525in}}%
\pgfpathlineto{\pgfqpoint{0.850936in}{2.912300in}}%
\pgfpathlineto{\pgfqpoint{0.859173in}{2.940412in}}%
\pgfpathlineto{\pgfqpoint{0.863291in}{2.926356in}}%
\pgfpathlineto{\pgfqpoint{0.867409in}{2.954469in}}%
\pgfpathlineto{\pgfqpoint{0.871527in}{2.940412in}}%
\pgfpathlineto{\pgfqpoint{0.875645in}{2.940412in}}%
\pgfpathlineto{\pgfqpoint{0.879763in}{2.926356in}}%
\pgfpathlineto{\pgfqpoint{0.887999in}{2.954469in}}%
\pgfpathlineto{\pgfqpoint{0.892117in}{2.926356in}}%
\pgfpathlineto{\pgfqpoint{0.900353in}{2.898244in}}%
\pgfpathlineto{\pgfqpoint{0.912708in}{2.898244in}}%
\pgfpathlineto{\pgfqpoint{0.916826in}{2.884188in}}%
\pgfpathlineto{\pgfqpoint{0.925062in}{2.912300in}}%
\pgfpathlineto{\pgfqpoint{0.929180in}{2.912300in}}%
\pgfpathlineto{\pgfqpoint{0.937416in}{2.940412in}}%
\pgfpathlineto{\pgfqpoint{0.941534in}{2.884188in}}%
\pgfpathlineto{\pgfqpoint{0.945652in}{2.898244in}}%
\pgfpathlineto{\pgfqpoint{0.953888in}{2.898244in}}%
\pgfpathlineto{\pgfqpoint{0.962125in}{2.926356in}}%
\pgfpathlineto{\pgfqpoint{0.966243in}{2.926356in}}%
\pgfpathlineto{\pgfqpoint{0.974479in}{2.954469in}}%
\pgfpathlineto{\pgfqpoint{0.982715in}{2.926356in}}%
\pgfpathlineto{\pgfqpoint{0.986833in}{2.954469in}}%
\pgfpathlineto{\pgfqpoint{0.995069in}{2.954469in}}%
\pgfpathlineto{\pgfqpoint{0.999187in}{2.926356in}}%
\pgfpathlineto{\pgfqpoint{1.003306in}{2.940412in}}%
\pgfpathlineto{\pgfqpoint{1.007424in}{2.968525in}}%
\pgfpathlineto{\pgfqpoint{1.015660in}{2.996637in}}%
\pgfpathlineto{\pgfqpoint{1.019778in}{2.996637in}}%
\pgfpathlineto{\pgfqpoint{1.028014in}{2.968525in}}%
\pgfpathlineto{\pgfqpoint{1.036250in}{2.968525in}}%
\pgfpathlineto{\pgfqpoint{1.040368in}{2.996637in}}%
\pgfpathlineto{\pgfqpoint{1.044486in}{3.010694in}}%
\pgfpathlineto{\pgfqpoint{1.048604in}{2.996637in}}%
\pgfpathlineto{\pgfqpoint{1.052723in}{2.996637in}}%
\pgfpathlineto{\pgfqpoint{1.056841in}{2.982581in}}%
\pgfpathlineto{\pgfqpoint{1.060959in}{2.996637in}}%
\pgfpathlineto{\pgfqpoint{1.069195in}{2.968525in}}%
\pgfpathlineto{\pgfqpoint{1.077431in}{2.968525in}}%
\pgfpathlineto{\pgfqpoint{1.081549in}{2.982581in}}%
\pgfpathlineto{\pgfqpoint{1.089785in}{2.982581in}}%
\pgfpathlineto{\pgfqpoint{1.093903in}{2.954469in}}%
\pgfpathlineto{\pgfqpoint{1.098021in}{2.940412in}}%
\pgfpathlineto{\pgfqpoint{1.102140in}{2.940412in}}%
\pgfpathlineto{\pgfqpoint{1.106258in}{2.954469in}}%
\pgfpathlineto{\pgfqpoint{1.114494in}{2.954469in}}%
\pgfpathlineto{\pgfqpoint{1.118612in}{2.912300in}}%
\pgfpathlineto{\pgfqpoint{1.122730in}{2.954469in}}%
\pgfpathlineto{\pgfqpoint{1.126848in}{2.968525in}}%
\pgfpathlineto{\pgfqpoint{1.143320in}{2.968525in}}%
\pgfpathlineto{\pgfqpoint{1.147438in}{2.996637in}}%
\pgfpathlineto{\pgfqpoint{1.151557in}{2.996637in}}%
\pgfpathlineto{\pgfqpoint{1.155675in}{2.968525in}}%
\pgfpathlineto{\pgfqpoint{1.159793in}{2.954469in}}%
\pgfpathlineto{\pgfqpoint{1.163911in}{2.954469in}}%
\pgfpathlineto{\pgfqpoint{1.168029in}{2.940412in}}%
\pgfpathlineto{\pgfqpoint{1.172147in}{2.968525in}}%
\pgfpathlineto{\pgfqpoint{1.176265in}{2.982581in}}%
\pgfpathlineto{\pgfqpoint{1.184501in}{2.982581in}}%
\pgfpathlineto{\pgfqpoint{1.188619in}{2.954469in}}%
\pgfpathlineto{\pgfqpoint{1.200974in}{2.996637in}}%
\pgfpathlineto{\pgfqpoint{1.205092in}{3.024750in}}%
\pgfpathlineto{\pgfqpoint{1.209210in}{3.038806in}}%
\pgfpathlineto{\pgfqpoint{1.213328in}{3.038806in}}%
\pgfpathlineto{\pgfqpoint{1.217446in}{3.052862in}}%
\pgfpathlineto{\pgfqpoint{1.221564in}{3.052862in}}%
\pgfpathlineto{\pgfqpoint{1.225682in}{3.066918in}}%
\pgfpathlineto{\pgfqpoint{1.229800in}{3.095031in}}%
\pgfpathlineto{\pgfqpoint{1.250391in}{3.095031in}}%
\pgfpathlineto{\pgfqpoint{1.254509in}{3.080975in}}%
\pgfpathlineto{\pgfqpoint{1.258627in}{3.052862in}}%
\pgfpathlineto{\pgfqpoint{1.262745in}{3.010694in}}%
\pgfpathlineto{\pgfqpoint{1.266863in}{3.010694in}}%
\pgfpathlineto{\pgfqpoint{1.270981in}{2.982581in}}%
\pgfpathlineto{\pgfqpoint{1.275099in}{2.926356in}}%
\pgfpathlineto{\pgfqpoint{1.283335in}{2.926356in}}%
\pgfpathlineto{\pgfqpoint{1.287453in}{2.954469in}}%
\pgfpathlineto{\pgfqpoint{1.291571in}{2.940412in}}%
\pgfpathlineto{\pgfqpoint{1.295690in}{2.954469in}}%
\pgfpathlineto{\pgfqpoint{1.299808in}{2.940412in}}%
\pgfpathlineto{\pgfqpoint{1.303926in}{2.940412in}}%
\pgfpathlineto{\pgfqpoint{1.312162in}{2.912300in}}%
\pgfpathlineto{\pgfqpoint{1.316280in}{2.954469in}}%
\pgfpathlineto{\pgfqpoint{1.320398in}{2.940412in}}%
\pgfpathlineto{\pgfqpoint{1.324516in}{2.940412in}}%
\pgfpathlineto{\pgfqpoint{1.328634in}{2.954469in}}%
\pgfpathlineto{\pgfqpoint{1.332752in}{2.954469in}}%
\pgfpathlineto{\pgfqpoint{1.336870in}{2.940412in}}%
\pgfpathlineto{\pgfqpoint{1.340988in}{2.996637in}}%
\pgfpathlineto{\pgfqpoint{1.345107in}{3.024750in}}%
\pgfpathlineto{\pgfqpoint{1.349225in}{3.024750in}}%
\pgfpathlineto{\pgfqpoint{1.357461in}{3.080975in}}%
\pgfpathlineto{\pgfqpoint{1.361579in}{3.095031in}}%
\pgfpathlineto{\pgfqpoint{1.386287in}{3.095031in}}%
\pgfpathlineto{\pgfqpoint{1.390405in}{3.137200in}}%
\pgfpathlineto{\pgfqpoint{1.394524in}{3.109087in}}%
\pgfpathlineto{\pgfqpoint{1.398642in}{3.123143in}}%
\pgfpathlineto{\pgfqpoint{1.402760in}{3.123143in}}%
\pgfpathlineto{\pgfqpoint{1.410996in}{3.151256in}}%
\pgfpathlineto{\pgfqpoint{1.419232in}{3.151256in}}%
\pgfpathlineto{\pgfqpoint{1.423350in}{3.109087in}}%
\pgfpathlineto{\pgfqpoint{1.435704in}{3.109087in}}%
\pgfpathlineto{\pgfqpoint{1.443941in}{3.080975in}}%
\pgfpathlineto{\pgfqpoint{1.448059in}{3.080975in}}%
\pgfpathlineto{\pgfqpoint{1.452177in}{3.052862in}}%
\pgfpathlineto{\pgfqpoint{1.456295in}{3.066918in}}%
\pgfpathlineto{\pgfqpoint{1.460413in}{3.024750in}}%
\pgfpathlineto{\pgfqpoint{1.464531in}{3.024750in}}%
\pgfpathlineto{\pgfqpoint{1.468649in}{3.066918in}}%
\pgfpathlineto{\pgfqpoint{1.481003in}{3.066918in}}%
\pgfpathlineto{\pgfqpoint{1.485121in}{3.095031in}}%
\pgfpathlineto{\pgfqpoint{1.489240in}{3.095031in}}%
\pgfpathlineto{\pgfqpoint{1.493358in}{3.123143in}}%
\pgfpathlineto{\pgfqpoint{1.497476in}{3.123143in}}%
\pgfpathlineto{\pgfqpoint{1.501594in}{3.137200in}}%
\pgfpathlineto{\pgfqpoint{1.513948in}{3.137200in}}%
\pgfpathlineto{\pgfqpoint{1.518066in}{3.123143in}}%
\pgfpathlineto{\pgfqpoint{1.522184in}{3.123143in}}%
\pgfpathlineto{\pgfqpoint{1.530420in}{3.151256in}}%
\pgfpathlineto{\pgfqpoint{1.538657in}{3.123143in}}%
\pgfpathlineto{\pgfqpoint{1.546893in}{3.123143in}}%
\pgfpathlineto{\pgfqpoint{1.551011in}{3.109087in}}%
\pgfpathlineto{\pgfqpoint{1.563365in}{3.109087in}}%
\pgfpathlineto{\pgfqpoint{1.567483in}{3.024750in}}%
\pgfpathlineto{\pgfqpoint{1.575719in}{3.024750in}}%
\pgfpathlineto{\pgfqpoint{1.579837in}{3.010694in}}%
\pgfpathlineto{\pgfqpoint{1.583955in}{3.038806in}}%
\pgfpathlineto{\pgfqpoint{1.588074in}{3.038806in}}%
\pgfpathlineto{\pgfqpoint{1.592192in}{2.996637in}}%
\pgfpathlineto{\pgfqpoint{1.596310in}{2.996637in}}%
\pgfpathlineto{\pgfqpoint{1.600428in}{3.052862in}}%
\pgfpathlineto{\pgfqpoint{1.604546in}{3.038806in}}%
\pgfpathlineto{\pgfqpoint{1.608664in}{2.982581in}}%
\pgfpathlineto{\pgfqpoint{1.621018in}{2.982581in}}%
\pgfpathlineto{\pgfqpoint{1.625136in}{3.010694in}}%
\pgfpathlineto{\pgfqpoint{1.629254in}{3.010694in}}%
\pgfpathlineto{\pgfqpoint{1.633372in}{2.982581in}}%
\pgfpathlineto{\pgfqpoint{1.637491in}{2.968525in}}%
\pgfpathlineto{\pgfqpoint{1.645727in}{2.968525in}}%
\pgfpathlineto{\pgfqpoint{1.653963in}{2.996637in}}%
\pgfpathlineto{\pgfqpoint{1.658081in}{2.996637in}}%
\pgfpathlineto{\pgfqpoint{1.662199in}{3.052862in}}%
\pgfpathlineto{\pgfqpoint{1.670435in}{3.024750in}}%
\pgfpathlineto{\pgfqpoint{1.674553in}{3.024750in}}%
\pgfpathlineto{\pgfqpoint{1.678671in}{3.010694in}}%
\pgfpathlineto{\pgfqpoint{1.686908in}{3.038806in}}%
\pgfpathlineto{\pgfqpoint{1.691026in}{3.038806in}}%
\pgfpathlineto{\pgfqpoint{1.695144in}{3.052862in}}%
\pgfpathlineto{\pgfqpoint{1.703380in}{3.052862in}}%
\pgfpathlineto{\pgfqpoint{1.707498in}{3.024750in}}%
\pgfpathlineto{\pgfqpoint{1.711616in}{3.066918in}}%
\pgfpathlineto{\pgfqpoint{1.715734in}{3.052862in}}%
\pgfpathlineto{\pgfqpoint{1.723970in}{3.052862in}}%
\pgfpathlineto{\pgfqpoint{1.728088in}{3.024750in}}%
\pgfpathlineto{\pgfqpoint{1.740443in}{3.024750in}}%
\pgfpathlineto{\pgfqpoint{1.744561in}{2.996637in}}%
\pgfpathlineto{\pgfqpoint{1.752797in}{2.996637in}}%
\pgfpathlineto{\pgfqpoint{1.756915in}{2.968525in}}%
\pgfpathlineto{\pgfqpoint{1.761033in}{2.968525in}}%
\pgfpathlineto{\pgfqpoint{1.765151in}{2.912300in}}%
\pgfpathlineto{\pgfqpoint{1.769269in}{2.940412in}}%
\pgfpathlineto{\pgfqpoint{1.773387in}{2.954469in}}%
\pgfpathlineto{\pgfqpoint{1.777505in}{2.954469in}}%
\pgfpathlineto{\pgfqpoint{1.781624in}{2.926356in}}%
\pgfpathlineto{\pgfqpoint{1.785742in}{2.926356in}}%
\pgfpathlineto{\pgfqpoint{1.789860in}{2.982581in}}%
\pgfpathlineto{\pgfqpoint{1.793978in}{2.968525in}}%
\pgfpathlineto{\pgfqpoint{1.798096in}{2.968525in}}%
\pgfpathlineto{\pgfqpoint{1.802214in}{2.954469in}}%
\pgfpathlineto{\pgfqpoint{1.806332in}{2.954469in}}%
\pgfpathlineto{\pgfqpoint{1.810450in}{2.940412in}}%
\pgfpathlineto{\pgfqpoint{1.814568in}{2.968525in}}%
\pgfpathlineto{\pgfqpoint{1.818686in}{2.954469in}}%
\pgfpathlineto{\pgfqpoint{1.831041in}{2.870131in}}%
\pgfpathlineto{\pgfqpoint{1.843395in}{2.870131in}}%
\pgfpathlineto{\pgfqpoint{1.851631in}{2.898244in}}%
\pgfpathlineto{\pgfqpoint{1.859867in}{2.898244in}}%
\pgfpathlineto{\pgfqpoint{1.863985in}{2.884188in}}%
\pgfpathlineto{\pgfqpoint{1.868103in}{2.940412in}}%
\pgfpathlineto{\pgfqpoint{1.872221in}{2.912300in}}%
\pgfpathlineto{\pgfqpoint{1.884576in}{2.912300in}}%
\pgfpathlineto{\pgfqpoint{1.888694in}{2.926356in}}%
\pgfpathlineto{\pgfqpoint{1.896930in}{2.926356in}}%
\pgfpathlineto{\pgfqpoint{1.901048in}{2.940412in}}%
\pgfpathlineto{\pgfqpoint{1.905166in}{2.968525in}}%
\pgfpathlineto{\pgfqpoint{1.913402in}{2.996637in}}%
\pgfpathlineto{\pgfqpoint{1.929875in}{2.996637in}}%
\pgfpathlineto{\pgfqpoint{1.933993in}{2.982581in}}%
\pgfpathlineto{\pgfqpoint{1.942229in}{2.884188in}}%
\pgfpathlineto{\pgfqpoint{1.950465in}{2.912300in}}%
\pgfpathlineto{\pgfqpoint{1.954583in}{2.940412in}}%
\pgfpathlineto{\pgfqpoint{1.958701in}{2.912300in}}%
\pgfpathlineto{\pgfqpoint{1.962819in}{2.898244in}}%
\pgfpathlineto{\pgfqpoint{1.966937in}{2.898244in}}%
\pgfpathlineto{\pgfqpoint{1.971055in}{2.912300in}}%
\pgfpathlineto{\pgfqpoint{1.975173in}{2.870131in}}%
\pgfpathlineto{\pgfqpoint{1.979292in}{2.884188in}}%
\pgfpathlineto{\pgfqpoint{1.983410in}{2.884188in}}%
\pgfpathlineto{\pgfqpoint{1.987528in}{2.856075in}}%
\pgfpathlineto{\pgfqpoint{1.991646in}{2.870131in}}%
\pgfpathlineto{\pgfqpoint{1.995764in}{2.870131in}}%
\pgfpathlineto{\pgfqpoint{2.004000in}{2.898244in}}%
\pgfpathlineto{\pgfqpoint{2.008118in}{2.898244in}}%
\pgfpathlineto{\pgfqpoint{2.012236in}{2.870131in}}%
\pgfpathlineto{\pgfqpoint{2.016354in}{2.870131in}}%
\pgfpathlineto{\pgfqpoint{2.020472in}{2.856075in}}%
\pgfpathlineto{\pgfqpoint{2.028709in}{2.856075in}}%
\pgfpathlineto{\pgfqpoint{2.036945in}{2.799850in}}%
\pgfpathlineto{\pgfqpoint{2.041063in}{2.827963in}}%
\pgfpathlineto{\pgfqpoint{2.045181in}{2.827963in}}%
\pgfpathlineto{\pgfqpoint{2.057535in}{2.870131in}}%
\pgfpathlineto{\pgfqpoint{2.061653in}{2.856075in}}%
\pgfpathlineto{\pgfqpoint{2.065771in}{2.884188in}}%
\pgfpathlineto{\pgfqpoint{2.069889in}{2.898244in}}%
\pgfpathlineto{\pgfqpoint{2.074008in}{2.940412in}}%
\pgfpathlineto{\pgfqpoint{2.078126in}{2.940412in}}%
\pgfpathlineto{\pgfqpoint{2.082244in}{2.912300in}}%
\pgfpathlineto{\pgfqpoint{2.094598in}{2.912300in}}%
\pgfpathlineto{\pgfqpoint{2.098716in}{2.870131in}}%
\pgfpathlineto{\pgfqpoint{2.102834in}{2.884188in}}%
\pgfpathlineto{\pgfqpoint{2.106952in}{2.884188in}}%
\pgfpathlineto{\pgfqpoint{2.111070in}{2.926356in}}%
\pgfpathlineto{\pgfqpoint{2.115188in}{2.912300in}}%
\pgfpathlineto{\pgfqpoint{2.119306in}{2.912300in}}%
\pgfpathlineto{\pgfqpoint{2.123425in}{2.926356in}}%
\pgfpathlineto{\pgfqpoint{2.127543in}{2.898244in}}%
\pgfpathlineto{\pgfqpoint{2.131661in}{2.898244in}}%
\pgfpathlineto{\pgfqpoint{2.135779in}{2.912300in}}%
\pgfpathlineto{\pgfqpoint{2.139897in}{2.912300in}}%
\pgfpathlineto{\pgfqpoint{2.148133in}{2.884188in}}%
\pgfpathlineto{\pgfqpoint{2.160487in}{2.884188in}}%
\pgfpathlineto{\pgfqpoint{2.168723in}{2.827963in}}%
\pgfpathlineto{\pgfqpoint{2.172842in}{2.827963in}}%
\pgfpathlineto{\pgfqpoint{2.176960in}{2.842019in}}%
\pgfpathlineto{\pgfqpoint{2.181078in}{2.827963in}}%
\pgfpathlineto{\pgfqpoint{2.185196in}{2.799850in}}%
\pgfpathlineto{\pgfqpoint{2.193432in}{2.799850in}}%
\pgfpathlineto{\pgfqpoint{2.197550in}{2.785794in}}%
\pgfpathlineto{\pgfqpoint{2.201668in}{2.757682in}}%
\pgfpathlineto{\pgfqpoint{2.214022in}{2.757682in}}%
\pgfpathlineto{\pgfqpoint{2.218140in}{2.743625in}}%
\pgfpathlineto{\pgfqpoint{2.222259in}{2.757682in}}%
\pgfpathlineto{\pgfqpoint{2.226377in}{2.729569in}}%
\pgfpathlineto{\pgfqpoint{2.230495in}{2.743625in}}%
\pgfpathlineto{\pgfqpoint{2.234613in}{2.729569in}}%
\pgfpathlineto{\pgfqpoint{2.238731in}{2.743625in}}%
\pgfpathlineto{\pgfqpoint{2.242849in}{2.743625in}}%
\pgfpathlineto{\pgfqpoint{2.251085in}{2.771738in}}%
\pgfpathlineto{\pgfqpoint{2.263439in}{2.729569in}}%
\pgfpathlineto{\pgfqpoint{2.267557in}{2.729569in}}%
\pgfpathlineto{\pgfqpoint{2.275794in}{2.673344in}}%
\pgfpathlineto{\pgfqpoint{2.279912in}{2.701457in}}%
\pgfpathlineto{\pgfqpoint{2.288148in}{2.701457in}}%
\pgfpathlineto{\pgfqpoint{2.292266in}{2.673344in}}%
\pgfpathlineto{\pgfqpoint{2.300502in}{2.701457in}}%
\pgfpathlineto{\pgfqpoint{2.304620in}{2.729569in}}%
\pgfpathlineto{\pgfqpoint{2.312856in}{2.729569in}}%
\pgfpathlineto{\pgfqpoint{2.316975in}{2.743625in}}%
\pgfpathlineto{\pgfqpoint{2.325211in}{2.743625in}}%
\pgfpathlineto{\pgfqpoint{2.329329in}{2.771738in}}%
\pgfpathlineto{\pgfqpoint{2.333447in}{2.785794in}}%
\pgfpathlineto{\pgfqpoint{2.337565in}{2.785794in}}%
\pgfpathlineto{\pgfqpoint{2.341683in}{2.771738in}}%
\pgfpathlineto{\pgfqpoint{2.345801in}{2.729569in}}%
\pgfpathlineto{\pgfqpoint{2.349919in}{2.743625in}}%
\pgfpathlineto{\pgfqpoint{2.354037in}{2.729569in}}%
\pgfpathlineto{\pgfqpoint{2.358155in}{2.729569in}}%
\pgfpathlineto{\pgfqpoint{2.362273in}{2.743625in}}%
\pgfpathlineto{\pgfqpoint{2.366392in}{2.729569in}}%
\pgfpathlineto{\pgfqpoint{2.370510in}{2.729569in}}%
\pgfpathlineto{\pgfqpoint{2.374628in}{2.715513in}}%
\pgfpathlineto{\pgfqpoint{2.378746in}{2.715513in}}%
\pgfpathlineto{\pgfqpoint{2.386982in}{2.659288in}}%
\pgfpathlineto{\pgfqpoint{2.391100in}{2.659288in}}%
\pgfpathlineto{\pgfqpoint{2.395218in}{2.617119in}}%
\pgfpathlineto{\pgfqpoint{2.399336in}{2.617119in}}%
\pgfpathlineto{\pgfqpoint{2.403454in}{2.631176in}}%
\pgfpathlineto{\pgfqpoint{2.407572in}{2.631176in}}%
\pgfpathlineto{\pgfqpoint{2.411690in}{2.645232in}}%
\pgfpathlineto{\pgfqpoint{2.415809in}{2.631176in}}%
\pgfpathlineto{\pgfqpoint{2.424045in}{2.631176in}}%
\pgfpathlineto{\pgfqpoint{2.428163in}{2.603063in}}%
\pgfpathlineto{\pgfqpoint{2.432281in}{2.617119in}}%
\pgfpathlineto{\pgfqpoint{2.440517in}{2.589007in}}%
\pgfpathlineto{\pgfqpoint{2.448753in}{2.589007in}}%
\pgfpathlineto{\pgfqpoint{2.452871in}{2.574951in}}%
\pgfpathlineto{\pgfqpoint{2.461107in}{2.574951in}}%
\pgfpathlineto{\pgfqpoint{2.465226in}{2.589007in}}%
\pgfpathlineto{\pgfqpoint{2.469344in}{2.589007in}}%
\pgfpathlineto{\pgfqpoint{2.473462in}{2.574951in}}%
\pgfpathlineto{\pgfqpoint{2.485816in}{2.574951in}}%
\pgfpathlineto{\pgfqpoint{2.494052in}{2.546838in}}%
\pgfpathlineto{\pgfqpoint{2.514643in}{2.546838in}}%
\pgfpathlineto{\pgfqpoint{2.518761in}{2.532782in}}%
\pgfpathlineto{\pgfqpoint{2.568178in}{2.532782in}}%
\pgfpathlineto{\pgfqpoint{2.572296in}{2.518726in}}%
\pgfpathlineto{\pgfqpoint{2.646421in}{2.518726in}}%
\pgfpathlineto{\pgfqpoint{2.646421in}{2.518726in}}%
\pgfusepath{stroke}%
\end{pgfscope}%
\begin{pgfscope}%
\pgfpathrectangle{\pgfqpoint{0.488751in}{2.165212in}}{\pgfqpoint{2.260417in}{1.283333in}}%
\pgfusepath{clip}%
\pgfsetrectcap%
\pgfsetroundjoin%
\pgfsetlinewidth{0.803000pt}%
\definecolor{currentstroke}{rgb}{0.490196,0.588235,0.431373}%
\pgfsetstrokecolor{currentstroke}%
\pgfsetdash{}{0pt}%
\pgfpathmoveto{\pgfqpoint{0.591497in}{2.574951in}}%
\pgfpathlineto{\pgfqpoint{0.595615in}{2.574951in}}%
\pgfpathlineto{\pgfqpoint{0.599733in}{2.560894in}}%
\pgfpathlineto{\pgfqpoint{0.628560in}{2.560894in}}%
\pgfpathlineto{\pgfqpoint{0.636796in}{2.532782in}}%
\pgfpathlineto{\pgfqpoint{0.640914in}{2.532782in}}%
\pgfpathlineto{\pgfqpoint{0.645032in}{2.518726in}}%
\pgfpathlineto{\pgfqpoint{0.653268in}{2.518726in}}%
\pgfpathlineto{\pgfqpoint{0.657386in}{2.490613in}}%
\pgfpathlineto{\pgfqpoint{0.661504in}{2.476557in}}%
\pgfpathlineto{\pgfqpoint{0.846818in}{2.476557in}}%
\pgfpathlineto{\pgfqpoint{0.850936in}{2.462501in}}%
\pgfpathlineto{\pgfqpoint{0.974479in}{2.462501in}}%
\pgfpathlineto{\pgfqpoint{0.978597in}{2.448445in}}%
\pgfpathlineto{\pgfqpoint{1.081549in}{2.448445in}}%
\pgfpathlineto{\pgfqpoint{1.085667in}{2.434388in}}%
\pgfpathlineto{\pgfqpoint{1.563365in}{2.434388in}}%
\pgfpathlineto{\pgfqpoint{1.567483in}{2.420332in}}%
\pgfpathlineto{\pgfqpoint{1.719852in}{2.420332in}}%
\pgfpathlineto{\pgfqpoint{1.723970in}{2.406276in}}%
\pgfpathlineto{\pgfqpoint{2.382864in}{2.406276in}}%
\pgfpathlineto{\pgfqpoint{2.386982in}{2.392220in}}%
\pgfpathlineto{\pgfqpoint{2.634067in}{2.392220in}}%
\pgfpathlineto{\pgfqpoint{2.638185in}{2.378163in}}%
\pgfpathlineto{\pgfqpoint{2.646421in}{2.378163in}}%
\pgfpathlineto{\pgfqpoint{2.646421in}{2.378163in}}%
\pgfusepath{stroke}%
\end{pgfscope}%
\begin{pgfscope}%
\pgfpathrectangle{\pgfqpoint{0.488751in}{2.165212in}}{\pgfqpoint{2.260417in}{1.283333in}}%
\pgfusepath{clip}%
\pgfsetrectcap%
\pgfsetroundjoin%
\pgfsetlinewidth{0.803000pt}%
\definecolor{currentstroke}{rgb}{0.843137,0.666667,0.313725}%
\pgfsetstrokecolor{currentstroke}%
\pgfsetdash{}{0pt}%
\pgfpathmoveto{\pgfqpoint{0.591497in}{2.532782in}}%
\pgfpathlineto{\pgfqpoint{0.595615in}{2.589007in}}%
\pgfpathlineto{\pgfqpoint{0.599733in}{2.589007in}}%
\pgfpathlineto{\pgfqpoint{0.603851in}{2.560894in}}%
\pgfpathlineto{\pgfqpoint{0.607969in}{2.574951in}}%
\pgfpathlineto{\pgfqpoint{0.612087in}{2.645232in}}%
\pgfpathlineto{\pgfqpoint{0.616206in}{2.673344in}}%
\pgfpathlineto{\pgfqpoint{0.620324in}{2.659288in}}%
\pgfpathlineto{\pgfqpoint{0.624442in}{2.687400in}}%
\pgfpathlineto{\pgfqpoint{0.628560in}{2.729569in}}%
\pgfpathlineto{\pgfqpoint{0.632678in}{2.729569in}}%
\pgfpathlineto{\pgfqpoint{0.636796in}{2.771738in}}%
\pgfpathlineto{\pgfqpoint{0.640914in}{2.785794in}}%
\pgfpathlineto{\pgfqpoint{0.645032in}{2.757682in}}%
\pgfpathlineto{\pgfqpoint{0.649150in}{2.785794in}}%
\pgfpathlineto{\pgfqpoint{0.653268in}{2.715513in}}%
\pgfpathlineto{\pgfqpoint{0.657386in}{2.687400in}}%
\pgfpathlineto{\pgfqpoint{0.661504in}{2.673344in}}%
\pgfpathlineto{\pgfqpoint{0.665623in}{2.715513in}}%
\pgfpathlineto{\pgfqpoint{0.673859in}{2.715513in}}%
\pgfpathlineto{\pgfqpoint{0.677977in}{2.743625in}}%
\pgfpathlineto{\pgfqpoint{0.682095in}{2.729569in}}%
\pgfpathlineto{\pgfqpoint{0.686213in}{2.757682in}}%
\pgfpathlineto{\pgfqpoint{0.690331in}{2.743625in}}%
\pgfpathlineto{\pgfqpoint{0.694449in}{2.715513in}}%
\pgfpathlineto{\pgfqpoint{0.698567in}{2.799850in}}%
\pgfpathlineto{\pgfqpoint{0.702685in}{2.827963in}}%
\pgfpathlineto{\pgfqpoint{0.706803in}{2.813906in}}%
\pgfpathlineto{\pgfqpoint{0.710922in}{2.842019in}}%
\pgfpathlineto{\pgfqpoint{0.715040in}{2.884188in}}%
\pgfpathlineto{\pgfqpoint{0.719158in}{2.870131in}}%
\pgfpathlineto{\pgfqpoint{0.723276in}{2.870131in}}%
\pgfpathlineto{\pgfqpoint{0.727394in}{2.884188in}}%
\pgfpathlineto{\pgfqpoint{0.731512in}{2.870131in}}%
\pgfpathlineto{\pgfqpoint{0.735630in}{2.926356in}}%
\pgfpathlineto{\pgfqpoint{0.739748in}{2.870131in}}%
\pgfpathlineto{\pgfqpoint{0.743866in}{2.912300in}}%
\pgfpathlineto{\pgfqpoint{0.747984in}{2.813906in}}%
\pgfpathlineto{\pgfqpoint{0.752102in}{2.785794in}}%
\pgfpathlineto{\pgfqpoint{0.756220in}{2.771738in}}%
\pgfpathlineto{\pgfqpoint{0.760339in}{2.771738in}}%
\pgfpathlineto{\pgfqpoint{0.764457in}{2.757682in}}%
\pgfpathlineto{\pgfqpoint{0.776811in}{2.757682in}}%
\pgfpathlineto{\pgfqpoint{0.780929in}{2.743625in}}%
\pgfpathlineto{\pgfqpoint{0.805637in}{2.743625in}}%
\pgfpathlineto{\pgfqpoint{0.809756in}{2.757682in}}%
\pgfpathlineto{\pgfqpoint{0.846818in}{2.757682in}}%
\pgfpathlineto{\pgfqpoint{0.850936in}{2.771738in}}%
\pgfpathlineto{\pgfqpoint{0.855054in}{2.771738in}}%
\pgfpathlineto{\pgfqpoint{0.859173in}{2.757682in}}%
\pgfpathlineto{\pgfqpoint{0.896235in}{2.757682in}}%
\pgfpathlineto{\pgfqpoint{0.900353in}{2.743625in}}%
\pgfpathlineto{\pgfqpoint{0.908590in}{2.743625in}}%
\pgfpathlineto{\pgfqpoint{0.912708in}{2.729569in}}%
\pgfpathlineto{\pgfqpoint{0.995069in}{2.729569in}}%
\pgfpathlineto{\pgfqpoint{0.999187in}{2.743625in}}%
\pgfpathlineto{\pgfqpoint{1.003306in}{2.743625in}}%
\pgfpathlineto{\pgfqpoint{1.011542in}{2.771738in}}%
\pgfpathlineto{\pgfqpoint{1.015660in}{2.799850in}}%
\pgfpathlineto{\pgfqpoint{1.023896in}{2.827963in}}%
\pgfpathlineto{\pgfqpoint{1.028014in}{2.799850in}}%
\pgfpathlineto{\pgfqpoint{1.032132in}{2.827963in}}%
\pgfpathlineto{\pgfqpoint{1.036250in}{2.827963in}}%
\pgfpathlineto{\pgfqpoint{1.040368in}{2.813906in}}%
\pgfpathlineto{\pgfqpoint{1.044486in}{2.729569in}}%
\pgfpathlineto{\pgfqpoint{1.052723in}{2.757682in}}%
\pgfpathlineto{\pgfqpoint{1.060959in}{2.757682in}}%
\pgfpathlineto{\pgfqpoint{1.065077in}{2.743625in}}%
\pgfpathlineto{\pgfqpoint{1.069195in}{2.757682in}}%
\pgfpathlineto{\pgfqpoint{1.077431in}{2.757682in}}%
\pgfpathlineto{\pgfqpoint{1.081549in}{2.743625in}}%
\pgfpathlineto{\pgfqpoint{1.110376in}{2.743625in}}%
\pgfpathlineto{\pgfqpoint{1.114494in}{2.757682in}}%
\pgfpathlineto{\pgfqpoint{1.122730in}{2.757682in}}%
\pgfpathlineto{\pgfqpoint{1.130966in}{2.785794in}}%
\pgfpathlineto{\pgfqpoint{1.135084in}{2.785794in}}%
\pgfpathlineto{\pgfqpoint{1.139202in}{2.771738in}}%
\pgfpathlineto{\pgfqpoint{1.143320in}{2.771738in}}%
\pgfpathlineto{\pgfqpoint{1.147438in}{2.757682in}}%
\pgfpathlineto{\pgfqpoint{1.270981in}{2.757682in}}%
\pgfpathlineto{\pgfqpoint{1.275099in}{2.771738in}}%
\pgfpathlineto{\pgfqpoint{1.295690in}{2.771738in}}%
\pgfpathlineto{\pgfqpoint{1.299808in}{2.785794in}}%
\pgfpathlineto{\pgfqpoint{1.303926in}{2.785794in}}%
\pgfpathlineto{\pgfqpoint{1.308044in}{2.813906in}}%
\pgfpathlineto{\pgfqpoint{1.312162in}{2.827963in}}%
\pgfpathlineto{\pgfqpoint{1.316280in}{2.785794in}}%
\pgfpathlineto{\pgfqpoint{1.320398in}{2.813906in}}%
\pgfpathlineto{\pgfqpoint{1.324516in}{2.813906in}}%
\pgfpathlineto{\pgfqpoint{1.328634in}{2.827963in}}%
\pgfpathlineto{\pgfqpoint{1.332752in}{2.813906in}}%
\pgfpathlineto{\pgfqpoint{1.336870in}{2.771738in}}%
\pgfpathlineto{\pgfqpoint{1.349225in}{2.771738in}}%
\pgfpathlineto{\pgfqpoint{1.353343in}{2.799850in}}%
\pgfpathlineto{\pgfqpoint{1.357461in}{2.771738in}}%
\pgfpathlineto{\pgfqpoint{1.382169in}{2.771738in}}%
\pgfpathlineto{\pgfqpoint{1.386287in}{2.785794in}}%
\pgfpathlineto{\pgfqpoint{1.398642in}{2.785794in}}%
\pgfpathlineto{\pgfqpoint{1.402760in}{2.771738in}}%
\pgfpathlineto{\pgfqpoint{1.419232in}{2.771738in}}%
\pgfpathlineto{\pgfqpoint{1.423350in}{2.785794in}}%
\pgfpathlineto{\pgfqpoint{1.435704in}{2.785794in}}%
\pgfpathlineto{\pgfqpoint{1.439822in}{2.757682in}}%
\pgfpathlineto{\pgfqpoint{1.443941in}{2.743625in}}%
\pgfpathlineto{\pgfqpoint{1.448059in}{2.743625in}}%
\pgfpathlineto{\pgfqpoint{1.452177in}{2.757682in}}%
\pgfpathlineto{\pgfqpoint{1.456295in}{2.757682in}}%
\pgfpathlineto{\pgfqpoint{1.460413in}{2.771738in}}%
\pgfpathlineto{\pgfqpoint{1.489240in}{2.771738in}}%
\pgfpathlineto{\pgfqpoint{1.493358in}{2.785794in}}%
\pgfpathlineto{\pgfqpoint{1.505712in}{2.785794in}}%
\pgfpathlineto{\pgfqpoint{1.509830in}{2.827963in}}%
\pgfpathlineto{\pgfqpoint{1.513948in}{2.813906in}}%
\pgfpathlineto{\pgfqpoint{1.518066in}{2.842019in}}%
\pgfpathlineto{\pgfqpoint{1.522184in}{2.813906in}}%
\pgfpathlineto{\pgfqpoint{1.526302in}{2.813906in}}%
\pgfpathlineto{\pgfqpoint{1.534538in}{2.785794in}}%
\pgfpathlineto{\pgfqpoint{1.546893in}{2.785794in}}%
\pgfpathlineto{\pgfqpoint{1.551011in}{2.813906in}}%
\pgfpathlineto{\pgfqpoint{1.563365in}{2.813906in}}%
\pgfpathlineto{\pgfqpoint{1.567483in}{2.842019in}}%
\pgfpathlineto{\pgfqpoint{1.571601in}{2.842019in}}%
\pgfpathlineto{\pgfqpoint{1.575719in}{2.827963in}}%
\pgfpathlineto{\pgfqpoint{1.579837in}{2.842019in}}%
\pgfpathlineto{\pgfqpoint{1.583955in}{2.813906in}}%
\pgfpathlineto{\pgfqpoint{1.588074in}{2.799850in}}%
\pgfpathlineto{\pgfqpoint{1.592192in}{2.842019in}}%
\pgfpathlineto{\pgfqpoint{1.596310in}{2.842019in}}%
\pgfpathlineto{\pgfqpoint{1.600428in}{2.799850in}}%
\pgfpathlineto{\pgfqpoint{1.604546in}{2.799850in}}%
\pgfpathlineto{\pgfqpoint{1.608664in}{2.813906in}}%
\pgfpathlineto{\pgfqpoint{1.633372in}{2.813906in}}%
\pgfpathlineto{\pgfqpoint{1.637491in}{2.842019in}}%
\pgfpathlineto{\pgfqpoint{1.645727in}{2.842019in}}%
\pgfpathlineto{\pgfqpoint{1.649845in}{2.827963in}}%
\pgfpathlineto{\pgfqpoint{1.658081in}{2.827963in}}%
\pgfpathlineto{\pgfqpoint{1.662199in}{2.856075in}}%
\pgfpathlineto{\pgfqpoint{1.666317in}{2.856075in}}%
\pgfpathlineto{\pgfqpoint{1.670435in}{2.898244in}}%
\pgfpathlineto{\pgfqpoint{1.674553in}{2.898244in}}%
\pgfpathlineto{\pgfqpoint{1.678671in}{2.954469in}}%
\pgfpathlineto{\pgfqpoint{1.682789in}{2.926356in}}%
\pgfpathlineto{\pgfqpoint{1.691026in}{2.898244in}}%
\pgfpathlineto{\pgfqpoint{1.695144in}{2.898244in}}%
\pgfpathlineto{\pgfqpoint{1.699262in}{2.884188in}}%
\pgfpathlineto{\pgfqpoint{1.703380in}{2.884188in}}%
\pgfpathlineto{\pgfqpoint{1.707498in}{2.912300in}}%
\pgfpathlineto{\pgfqpoint{1.711616in}{2.884188in}}%
\pgfpathlineto{\pgfqpoint{1.715734in}{2.912300in}}%
\pgfpathlineto{\pgfqpoint{1.728088in}{2.870131in}}%
\pgfpathlineto{\pgfqpoint{1.740443in}{2.870131in}}%
\pgfpathlineto{\pgfqpoint{1.744561in}{2.884188in}}%
\pgfpathlineto{\pgfqpoint{1.756915in}{2.884188in}}%
\pgfpathlineto{\pgfqpoint{1.761033in}{2.870131in}}%
\pgfpathlineto{\pgfqpoint{1.773387in}{2.870131in}}%
\pgfpathlineto{\pgfqpoint{1.777505in}{2.898244in}}%
\pgfpathlineto{\pgfqpoint{1.781624in}{2.898244in}}%
\pgfpathlineto{\pgfqpoint{1.785742in}{2.884188in}}%
\pgfpathlineto{\pgfqpoint{1.789860in}{2.827963in}}%
\pgfpathlineto{\pgfqpoint{1.793978in}{2.813906in}}%
\pgfpathlineto{\pgfqpoint{1.798096in}{2.813906in}}%
\pgfpathlineto{\pgfqpoint{1.802214in}{2.856075in}}%
\pgfpathlineto{\pgfqpoint{1.806332in}{2.842019in}}%
\pgfpathlineto{\pgfqpoint{1.810450in}{2.870131in}}%
\pgfpathlineto{\pgfqpoint{1.814568in}{2.842019in}}%
\pgfpathlineto{\pgfqpoint{1.818686in}{2.827963in}}%
\pgfpathlineto{\pgfqpoint{1.822804in}{2.842019in}}%
\pgfpathlineto{\pgfqpoint{1.843395in}{2.842019in}}%
\pgfpathlineto{\pgfqpoint{1.847513in}{2.827963in}}%
\pgfpathlineto{\pgfqpoint{1.863985in}{2.827963in}}%
\pgfpathlineto{\pgfqpoint{1.868103in}{2.799850in}}%
\pgfpathlineto{\pgfqpoint{1.896930in}{2.799850in}}%
\pgfpathlineto{\pgfqpoint{1.901048in}{2.827963in}}%
\pgfpathlineto{\pgfqpoint{1.913402in}{2.827963in}}%
\pgfpathlineto{\pgfqpoint{1.921638in}{2.799850in}}%
\pgfpathlineto{\pgfqpoint{1.929875in}{2.799850in}}%
\pgfpathlineto{\pgfqpoint{1.933993in}{2.813906in}}%
\pgfpathlineto{\pgfqpoint{1.938111in}{2.813906in}}%
\pgfpathlineto{\pgfqpoint{1.942229in}{2.827963in}}%
\pgfpathlineto{\pgfqpoint{1.946347in}{2.827963in}}%
\pgfpathlineto{\pgfqpoint{1.950465in}{2.813906in}}%
\pgfpathlineto{\pgfqpoint{1.954583in}{2.813906in}}%
\pgfpathlineto{\pgfqpoint{1.958701in}{2.771738in}}%
\pgfpathlineto{\pgfqpoint{1.962819in}{2.757682in}}%
\pgfpathlineto{\pgfqpoint{1.971055in}{2.757682in}}%
\pgfpathlineto{\pgfqpoint{1.975173in}{2.785794in}}%
\pgfpathlineto{\pgfqpoint{1.983410in}{2.785794in}}%
\pgfpathlineto{\pgfqpoint{1.987528in}{2.757682in}}%
\pgfpathlineto{\pgfqpoint{1.995764in}{2.757682in}}%
\pgfpathlineto{\pgfqpoint{1.999882in}{2.743625in}}%
\pgfpathlineto{\pgfqpoint{2.008118in}{2.743625in}}%
\pgfpathlineto{\pgfqpoint{2.012236in}{2.757682in}}%
\pgfpathlineto{\pgfqpoint{2.016354in}{2.757682in}}%
\pgfpathlineto{\pgfqpoint{2.020472in}{2.729569in}}%
\pgfpathlineto{\pgfqpoint{2.024591in}{2.743625in}}%
\pgfpathlineto{\pgfqpoint{2.028709in}{2.743625in}}%
\pgfpathlineto{\pgfqpoint{2.032827in}{2.771738in}}%
\pgfpathlineto{\pgfqpoint{2.036945in}{2.785794in}}%
\pgfpathlineto{\pgfqpoint{2.045181in}{2.785794in}}%
\pgfpathlineto{\pgfqpoint{2.049299in}{2.771738in}}%
\pgfpathlineto{\pgfqpoint{2.053417in}{2.799850in}}%
\pgfpathlineto{\pgfqpoint{2.065771in}{2.799850in}}%
\pgfpathlineto{\pgfqpoint{2.069889in}{2.813906in}}%
\pgfpathlineto{\pgfqpoint{2.074008in}{2.799850in}}%
\pgfpathlineto{\pgfqpoint{2.078126in}{2.813906in}}%
\pgfpathlineto{\pgfqpoint{2.082244in}{2.799850in}}%
\pgfpathlineto{\pgfqpoint{2.086362in}{2.771738in}}%
\pgfpathlineto{\pgfqpoint{2.098716in}{2.771738in}}%
\pgfpathlineto{\pgfqpoint{2.102834in}{2.757682in}}%
\pgfpathlineto{\pgfqpoint{2.115188in}{2.757682in}}%
\pgfpathlineto{\pgfqpoint{2.119306in}{2.785794in}}%
\pgfpathlineto{\pgfqpoint{2.123425in}{2.785794in}}%
\pgfpathlineto{\pgfqpoint{2.127543in}{2.799850in}}%
\pgfpathlineto{\pgfqpoint{2.131661in}{2.799850in}}%
\pgfpathlineto{\pgfqpoint{2.135779in}{2.785794in}}%
\pgfpathlineto{\pgfqpoint{2.152251in}{2.785794in}}%
\pgfpathlineto{\pgfqpoint{2.156369in}{2.799850in}}%
\pgfpathlineto{\pgfqpoint{2.168723in}{2.757682in}}%
\pgfpathlineto{\pgfqpoint{2.172842in}{2.771738in}}%
\pgfpathlineto{\pgfqpoint{2.193432in}{2.771738in}}%
\pgfpathlineto{\pgfqpoint{2.197550in}{2.757682in}}%
\pgfpathlineto{\pgfqpoint{2.214022in}{2.757682in}}%
\pgfpathlineto{\pgfqpoint{2.218140in}{2.771738in}}%
\pgfpathlineto{\pgfqpoint{2.234613in}{2.771738in}}%
\pgfpathlineto{\pgfqpoint{2.238731in}{2.743625in}}%
\pgfpathlineto{\pgfqpoint{2.242849in}{2.743625in}}%
\pgfpathlineto{\pgfqpoint{2.246967in}{2.757682in}}%
\pgfpathlineto{\pgfqpoint{2.255203in}{2.757682in}}%
\pgfpathlineto{\pgfqpoint{2.259321in}{2.771738in}}%
\pgfpathlineto{\pgfqpoint{2.267557in}{2.771738in}}%
\pgfpathlineto{\pgfqpoint{2.271676in}{2.799850in}}%
\pgfpathlineto{\pgfqpoint{2.275794in}{2.799850in}}%
\pgfpathlineto{\pgfqpoint{2.279912in}{2.827963in}}%
\pgfpathlineto{\pgfqpoint{2.284030in}{2.799850in}}%
\pgfpathlineto{\pgfqpoint{2.288148in}{2.785794in}}%
\pgfpathlineto{\pgfqpoint{2.292266in}{2.799850in}}%
\pgfpathlineto{\pgfqpoint{2.296384in}{2.785794in}}%
\pgfpathlineto{\pgfqpoint{2.300502in}{2.785794in}}%
\pgfpathlineto{\pgfqpoint{2.304620in}{2.771738in}}%
\pgfpathlineto{\pgfqpoint{2.308738in}{2.799850in}}%
\pgfpathlineto{\pgfqpoint{2.325211in}{2.799850in}}%
\pgfpathlineto{\pgfqpoint{2.329329in}{2.842019in}}%
\pgfpathlineto{\pgfqpoint{2.333447in}{2.799850in}}%
\pgfpathlineto{\pgfqpoint{2.337565in}{2.799850in}}%
\pgfpathlineto{\pgfqpoint{2.341683in}{2.813906in}}%
\pgfpathlineto{\pgfqpoint{2.345801in}{2.799850in}}%
\pgfpathlineto{\pgfqpoint{2.354037in}{2.799850in}}%
\pgfpathlineto{\pgfqpoint{2.358155in}{2.785794in}}%
\pgfpathlineto{\pgfqpoint{2.366392in}{2.813906in}}%
\pgfpathlineto{\pgfqpoint{2.370510in}{2.813906in}}%
\pgfpathlineto{\pgfqpoint{2.374628in}{2.842019in}}%
\pgfpathlineto{\pgfqpoint{2.378746in}{2.827963in}}%
\pgfpathlineto{\pgfqpoint{2.382864in}{2.842019in}}%
\pgfpathlineto{\pgfqpoint{2.386982in}{2.884188in}}%
\pgfpathlineto{\pgfqpoint{2.391100in}{2.898244in}}%
\pgfpathlineto{\pgfqpoint{2.395218in}{2.968525in}}%
\pgfpathlineto{\pgfqpoint{2.399336in}{2.982581in}}%
\pgfpathlineto{\pgfqpoint{2.403454in}{2.982581in}}%
\pgfpathlineto{\pgfqpoint{2.407572in}{3.010694in}}%
\pgfpathlineto{\pgfqpoint{2.411690in}{2.996637in}}%
\pgfpathlineto{\pgfqpoint{2.419927in}{3.024750in}}%
\pgfpathlineto{\pgfqpoint{2.424045in}{3.010694in}}%
\pgfpathlineto{\pgfqpoint{2.428163in}{3.052862in}}%
\pgfpathlineto{\pgfqpoint{2.440517in}{3.052862in}}%
\pgfpathlineto{\pgfqpoint{2.444635in}{3.066918in}}%
\pgfpathlineto{\pgfqpoint{2.448753in}{3.066918in}}%
\pgfpathlineto{\pgfqpoint{2.452871in}{3.052862in}}%
\pgfpathlineto{\pgfqpoint{2.456989in}{3.052862in}}%
\pgfpathlineto{\pgfqpoint{2.461107in}{3.038806in}}%
\pgfpathlineto{\pgfqpoint{2.465226in}{3.038806in}}%
\pgfpathlineto{\pgfqpoint{2.469344in}{3.024750in}}%
\pgfpathlineto{\pgfqpoint{2.473462in}{3.024750in}}%
\pgfpathlineto{\pgfqpoint{2.477580in}{2.996637in}}%
\pgfpathlineto{\pgfqpoint{2.485816in}{2.968525in}}%
\pgfpathlineto{\pgfqpoint{2.494052in}{3.052862in}}%
\pgfpathlineto{\pgfqpoint{2.498170in}{3.052862in}}%
\pgfpathlineto{\pgfqpoint{2.502288in}{3.066918in}}%
\pgfpathlineto{\pgfqpoint{2.506406in}{3.052862in}}%
\pgfpathlineto{\pgfqpoint{2.514643in}{3.052862in}}%
\pgfpathlineto{\pgfqpoint{2.518761in}{3.066918in}}%
\pgfpathlineto{\pgfqpoint{2.522879in}{3.066918in}}%
\pgfpathlineto{\pgfqpoint{2.526997in}{3.095031in}}%
\pgfpathlineto{\pgfqpoint{2.531115in}{3.109087in}}%
\pgfpathlineto{\pgfqpoint{2.535233in}{3.109087in}}%
\pgfpathlineto{\pgfqpoint{2.539351in}{3.123143in}}%
\pgfpathlineto{\pgfqpoint{2.543469in}{3.109087in}}%
\pgfpathlineto{\pgfqpoint{2.555823in}{3.151256in}}%
\pgfpathlineto{\pgfqpoint{2.559942in}{3.151256in}}%
\pgfpathlineto{\pgfqpoint{2.564060in}{3.179368in}}%
\pgfpathlineto{\pgfqpoint{2.568178in}{3.151256in}}%
\pgfpathlineto{\pgfqpoint{2.572296in}{3.151256in}}%
\pgfpathlineto{\pgfqpoint{2.576414in}{3.137200in}}%
\pgfpathlineto{\pgfqpoint{2.588768in}{3.137200in}}%
\pgfpathlineto{\pgfqpoint{2.592886in}{3.123143in}}%
\pgfpathlineto{\pgfqpoint{2.597004in}{3.137200in}}%
\pgfpathlineto{\pgfqpoint{2.625831in}{3.137200in}}%
\pgfpathlineto{\pgfqpoint{2.629949in}{3.151256in}}%
\pgfpathlineto{\pgfqpoint{2.634067in}{3.137200in}}%
\pgfpathlineto{\pgfqpoint{2.646421in}{3.137200in}}%
\pgfpathlineto{\pgfqpoint{2.646421in}{3.137200in}}%
\pgfusepath{stroke}%
\end{pgfscope}%
\begin{pgfscope}%
\pgfpathrectangle{\pgfqpoint{0.488751in}{2.165212in}}{\pgfqpoint{2.260417in}{1.283333in}}%
\pgfusepath{clip}%
\pgfsetbuttcap%
\pgfsetroundjoin%
\pgfsetlinewidth{0.803000pt}%
\definecolor{currentstroke}{rgb}{0.333333,0.333333,0.333333}%
\pgfsetstrokecolor{currentstroke}%
\pgfsetdash{{2.960000pt}{1.280000pt}}{0.000000pt}%
\pgfpathmoveto{\pgfqpoint{0.591497in}{3.066918in}}%
\pgfpathlineto{\pgfqpoint{0.595615in}{3.109087in}}%
\pgfpathlineto{\pgfqpoint{0.599733in}{3.109087in}}%
\pgfpathlineto{\pgfqpoint{0.603851in}{3.095031in}}%
\pgfpathlineto{\pgfqpoint{0.607969in}{3.137200in}}%
\pgfpathlineto{\pgfqpoint{0.612087in}{3.207481in}}%
\pgfpathlineto{\pgfqpoint{0.616206in}{3.165312in}}%
\pgfpathlineto{\pgfqpoint{0.620324in}{3.151256in}}%
\pgfpathlineto{\pgfqpoint{0.624442in}{3.151256in}}%
\pgfpathlineto{\pgfqpoint{0.628560in}{3.207481in}}%
\pgfpathlineto{\pgfqpoint{0.632678in}{3.137200in}}%
\pgfpathlineto{\pgfqpoint{0.636796in}{3.095031in}}%
\pgfpathlineto{\pgfqpoint{0.640914in}{3.066918in}}%
\pgfpathlineto{\pgfqpoint{0.645032in}{3.066918in}}%
\pgfpathlineto{\pgfqpoint{0.649150in}{2.982581in}}%
\pgfpathlineto{\pgfqpoint{0.653268in}{2.940412in}}%
\pgfpathlineto{\pgfqpoint{0.657386in}{2.940412in}}%
\pgfpathlineto{\pgfqpoint{0.661504in}{2.912300in}}%
\pgfpathlineto{\pgfqpoint{0.665623in}{2.926356in}}%
\pgfpathlineto{\pgfqpoint{0.669741in}{2.898244in}}%
\pgfpathlineto{\pgfqpoint{0.673859in}{2.884188in}}%
\pgfpathlineto{\pgfqpoint{0.677977in}{2.813906in}}%
\pgfpathlineto{\pgfqpoint{0.682095in}{2.771738in}}%
\pgfpathlineto{\pgfqpoint{0.690331in}{2.743625in}}%
\pgfpathlineto{\pgfqpoint{0.694449in}{2.743625in}}%
\pgfpathlineto{\pgfqpoint{0.698567in}{2.729569in}}%
\pgfpathlineto{\pgfqpoint{0.706803in}{2.729569in}}%
\pgfpathlineto{\pgfqpoint{0.710922in}{2.785794in}}%
\pgfpathlineto{\pgfqpoint{0.715040in}{2.757682in}}%
\pgfpathlineto{\pgfqpoint{0.719158in}{2.771738in}}%
\pgfpathlineto{\pgfqpoint{0.743866in}{2.771738in}}%
\pgfpathlineto{\pgfqpoint{0.747984in}{2.757682in}}%
\pgfpathlineto{\pgfqpoint{0.772693in}{2.757682in}}%
\pgfpathlineto{\pgfqpoint{0.776811in}{2.785794in}}%
\pgfpathlineto{\pgfqpoint{0.780929in}{2.771738in}}%
\pgfpathlineto{\pgfqpoint{0.805637in}{2.771738in}}%
\pgfpathlineto{\pgfqpoint{0.809756in}{2.729569in}}%
\pgfpathlineto{\pgfqpoint{0.817992in}{2.729569in}}%
\pgfpathlineto{\pgfqpoint{0.822110in}{2.715513in}}%
\pgfpathlineto{\pgfqpoint{0.830346in}{2.659288in}}%
\pgfpathlineto{\pgfqpoint{0.834464in}{2.645232in}}%
\pgfpathlineto{\pgfqpoint{0.838582in}{2.673344in}}%
\pgfpathlineto{\pgfqpoint{0.846818in}{2.673344in}}%
\pgfpathlineto{\pgfqpoint{0.850936in}{2.701457in}}%
\pgfpathlineto{\pgfqpoint{0.855054in}{2.659288in}}%
\pgfpathlineto{\pgfqpoint{0.859173in}{2.631176in}}%
\pgfpathlineto{\pgfqpoint{0.887999in}{2.631176in}}%
\pgfpathlineto{\pgfqpoint{0.892117in}{2.645232in}}%
\pgfpathlineto{\pgfqpoint{0.896235in}{2.631176in}}%
\pgfpathlineto{\pgfqpoint{0.900353in}{2.659288in}}%
\pgfpathlineto{\pgfqpoint{0.904471in}{2.617119in}}%
\pgfpathlineto{\pgfqpoint{0.916826in}{2.574951in}}%
\pgfpathlineto{\pgfqpoint{0.920944in}{2.574951in}}%
\pgfpathlineto{\pgfqpoint{0.925062in}{2.560894in}}%
\pgfpathlineto{\pgfqpoint{0.937416in}{2.560894in}}%
\pgfpathlineto{\pgfqpoint{0.941534in}{2.589007in}}%
\pgfpathlineto{\pgfqpoint{0.949770in}{2.589007in}}%
\pgfpathlineto{\pgfqpoint{0.953888in}{2.603063in}}%
\pgfpathlineto{\pgfqpoint{0.958007in}{2.589007in}}%
\pgfpathlineto{\pgfqpoint{0.978597in}{2.589007in}}%
\pgfpathlineto{\pgfqpoint{0.982715in}{2.617119in}}%
\pgfpathlineto{\pgfqpoint{0.986833in}{2.603063in}}%
\pgfpathlineto{\pgfqpoint{0.990951in}{2.603063in}}%
\pgfpathlineto{\pgfqpoint{0.999187in}{2.659288in}}%
\pgfpathlineto{\pgfqpoint{1.003306in}{2.659288in}}%
\pgfpathlineto{\pgfqpoint{1.007424in}{2.603063in}}%
\pgfpathlineto{\pgfqpoint{1.011542in}{2.589007in}}%
\pgfpathlineto{\pgfqpoint{1.015660in}{2.603063in}}%
\pgfpathlineto{\pgfqpoint{1.036250in}{2.603063in}}%
\pgfpathlineto{\pgfqpoint{1.040368in}{2.617119in}}%
\pgfpathlineto{\pgfqpoint{1.060959in}{2.617119in}}%
\pgfpathlineto{\pgfqpoint{1.069195in}{2.589007in}}%
\pgfpathlineto{\pgfqpoint{1.089785in}{2.589007in}}%
\pgfpathlineto{\pgfqpoint{1.093903in}{2.574951in}}%
\pgfpathlineto{\pgfqpoint{1.110376in}{2.574951in}}%
\pgfpathlineto{\pgfqpoint{1.114494in}{2.589007in}}%
\pgfpathlineto{\pgfqpoint{1.122730in}{2.589007in}}%
\pgfpathlineto{\pgfqpoint{1.126848in}{2.603063in}}%
\pgfpathlineto{\pgfqpoint{1.180383in}{2.603063in}}%
\pgfpathlineto{\pgfqpoint{1.184501in}{2.617119in}}%
\pgfpathlineto{\pgfqpoint{1.188619in}{2.603063in}}%
\pgfpathlineto{\pgfqpoint{1.196855in}{2.603063in}}%
\pgfpathlineto{\pgfqpoint{1.200974in}{2.617119in}}%
\pgfpathlineto{\pgfqpoint{1.225682in}{2.617119in}}%
\pgfpathlineto{\pgfqpoint{1.229800in}{2.645232in}}%
\pgfpathlineto{\pgfqpoint{1.233918in}{2.645232in}}%
\pgfpathlineto{\pgfqpoint{1.242154in}{2.617119in}}%
\pgfpathlineto{\pgfqpoint{1.246273in}{2.617119in}}%
\pgfpathlineto{\pgfqpoint{1.250391in}{2.589007in}}%
\pgfpathlineto{\pgfqpoint{1.254509in}{2.589007in}}%
\pgfpathlineto{\pgfqpoint{1.258627in}{2.546838in}}%
\pgfpathlineto{\pgfqpoint{1.262745in}{2.560894in}}%
\pgfpathlineto{\pgfqpoint{1.266863in}{2.560894in}}%
\pgfpathlineto{\pgfqpoint{1.270981in}{2.546838in}}%
\pgfpathlineto{\pgfqpoint{1.303926in}{2.546838in}}%
\pgfpathlineto{\pgfqpoint{1.308044in}{2.574951in}}%
\pgfpathlineto{\pgfqpoint{1.312162in}{2.546838in}}%
\pgfpathlineto{\pgfqpoint{1.316280in}{2.546838in}}%
\pgfpathlineto{\pgfqpoint{1.320398in}{2.560894in}}%
\pgfpathlineto{\pgfqpoint{1.336870in}{2.560894in}}%
\pgfpathlineto{\pgfqpoint{1.340988in}{2.532782in}}%
\pgfpathlineto{\pgfqpoint{1.353343in}{2.532782in}}%
\pgfpathlineto{\pgfqpoint{1.357461in}{2.546838in}}%
\pgfpathlineto{\pgfqpoint{1.386287in}{2.546838in}}%
\pgfpathlineto{\pgfqpoint{1.390405in}{2.560894in}}%
\pgfpathlineto{\pgfqpoint{1.398642in}{2.560894in}}%
\pgfpathlineto{\pgfqpoint{1.402760in}{2.532782in}}%
\pgfpathlineto{\pgfqpoint{1.448059in}{2.532782in}}%
\pgfpathlineto{\pgfqpoint{1.452177in}{2.546838in}}%
\pgfpathlineto{\pgfqpoint{1.456295in}{2.546838in}}%
\pgfpathlineto{\pgfqpoint{1.460413in}{2.560894in}}%
\pgfpathlineto{\pgfqpoint{1.464531in}{2.560894in}}%
\pgfpathlineto{\pgfqpoint{1.468649in}{2.546838in}}%
\pgfpathlineto{\pgfqpoint{1.481003in}{2.546838in}}%
\pgfpathlineto{\pgfqpoint{1.489240in}{2.603063in}}%
\pgfpathlineto{\pgfqpoint{1.493358in}{2.589007in}}%
\pgfpathlineto{\pgfqpoint{1.497476in}{2.589007in}}%
\pgfpathlineto{\pgfqpoint{1.501594in}{2.574951in}}%
\pgfpathlineto{\pgfqpoint{1.513948in}{2.574951in}}%
\pgfpathlineto{\pgfqpoint{1.518066in}{2.603063in}}%
\pgfpathlineto{\pgfqpoint{1.522184in}{2.574951in}}%
\pgfpathlineto{\pgfqpoint{1.526302in}{2.560894in}}%
\pgfpathlineto{\pgfqpoint{1.530420in}{2.560894in}}%
\pgfpathlineto{\pgfqpoint{1.534538in}{2.532782in}}%
\pgfpathlineto{\pgfqpoint{1.563365in}{2.532782in}}%
\pgfpathlineto{\pgfqpoint{1.567483in}{2.560894in}}%
\pgfpathlineto{\pgfqpoint{1.579837in}{2.560894in}}%
\pgfpathlineto{\pgfqpoint{1.583955in}{2.574951in}}%
\pgfpathlineto{\pgfqpoint{1.592192in}{2.574951in}}%
\pgfpathlineto{\pgfqpoint{1.596310in}{2.560894in}}%
\pgfpathlineto{\pgfqpoint{1.600428in}{2.532782in}}%
\pgfpathlineto{\pgfqpoint{1.604546in}{2.532782in}}%
\pgfpathlineto{\pgfqpoint{1.608664in}{2.490613in}}%
\pgfpathlineto{\pgfqpoint{1.621018in}{2.490613in}}%
\pgfpathlineto{\pgfqpoint{1.625136in}{2.504669in}}%
\pgfpathlineto{\pgfqpoint{1.633372in}{2.504669in}}%
\pgfpathlineto{\pgfqpoint{1.637491in}{2.532782in}}%
\pgfpathlineto{\pgfqpoint{1.658081in}{2.532782in}}%
\pgfpathlineto{\pgfqpoint{1.662199in}{2.504669in}}%
\pgfpathlineto{\pgfqpoint{1.674553in}{2.504669in}}%
\pgfpathlineto{\pgfqpoint{1.682789in}{2.532782in}}%
\pgfpathlineto{\pgfqpoint{1.686908in}{2.518726in}}%
\pgfpathlineto{\pgfqpoint{1.707498in}{2.518726in}}%
\pgfpathlineto{\pgfqpoint{1.711616in}{2.490613in}}%
\pgfpathlineto{\pgfqpoint{1.715734in}{2.504669in}}%
\pgfpathlineto{\pgfqpoint{1.723970in}{2.504669in}}%
\pgfpathlineto{\pgfqpoint{1.732206in}{2.532782in}}%
\pgfpathlineto{\pgfqpoint{1.761033in}{2.532782in}}%
\pgfpathlineto{\pgfqpoint{1.765151in}{2.518726in}}%
\pgfpathlineto{\pgfqpoint{1.773387in}{2.518726in}}%
\pgfpathlineto{\pgfqpoint{1.777505in}{2.532782in}}%
\pgfpathlineto{\pgfqpoint{1.789860in}{2.532782in}}%
\pgfpathlineto{\pgfqpoint{1.798096in}{2.504669in}}%
\pgfpathlineto{\pgfqpoint{1.810450in}{2.504669in}}%
\pgfpathlineto{\pgfqpoint{1.814568in}{2.532782in}}%
\pgfpathlineto{\pgfqpoint{1.818686in}{2.518726in}}%
\pgfpathlineto{\pgfqpoint{1.822804in}{2.518726in}}%
\pgfpathlineto{\pgfqpoint{1.826922in}{2.504669in}}%
\pgfpathlineto{\pgfqpoint{1.892812in}{2.504669in}}%
\pgfpathlineto{\pgfqpoint{1.896930in}{2.490613in}}%
\pgfpathlineto{\pgfqpoint{1.901048in}{2.518726in}}%
\pgfpathlineto{\pgfqpoint{1.938111in}{2.518726in}}%
\pgfpathlineto{\pgfqpoint{1.942229in}{2.546838in}}%
\pgfpathlineto{\pgfqpoint{1.971055in}{2.546838in}}%
\pgfpathlineto{\pgfqpoint{1.975173in}{2.560894in}}%
\pgfpathlineto{\pgfqpoint{2.016354in}{2.560894in}}%
\pgfpathlineto{\pgfqpoint{2.020472in}{2.574951in}}%
\pgfpathlineto{\pgfqpoint{2.028709in}{2.574951in}}%
\pgfpathlineto{\pgfqpoint{2.032827in}{2.589007in}}%
\pgfpathlineto{\pgfqpoint{2.036945in}{2.617119in}}%
\pgfpathlineto{\pgfqpoint{2.041063in}{2.617119in}}%
\pgfpathlineto{\pgfqpoint{2.045181in}{2.631176in}}%
\pgfpathlineto{\pgfqpoint{2.049299in}{2.617119in}}%
\pgfpathlineto{\pgfqpoint{2.061653in}{2.617119in}}%
\pgfpathlineto{\pgfqpoint{2.065771in}{2.589007in}}%
\pgfpathlineto{\pgfqpoint{2.164605in}{2.589007in}}%
\pgfpathlineto{\pgfqpoint{2.168723in}{2.603063in}}%
\pgfpathlineto{\pgfqpoint{2.189314in}{2.603063in}}%
\pgfpathlineto{\pgfqpoint{2.193432in}{2.617119in}}%
\pgfpathlineto{\pgfqpoint{2.197550in}{2.603063in}}%
\pgfpathlineto{\pgfqpoint{2.201668in}{2.631176in}}%
\pgfpathlineto{\pgfqpoint{2.205786in}{2.617119in}}%
\pgfpathlineto{\pgfqpoint{2.209904in}{2.631176in}}%
\pgfpathlineto{\pgfqpoint{2.214022in}{2.617119in}}%
\pgfpathlineto{\pgfqpoint{2.218140in}{2.659288in}}%
\pgfpathlineto{\pgfqpoint{2.222259in}{2.659288in}}%
\pgfpathlineto{\pgfqpoint{2.226377in}{2.673344in}}%
\pgfpathlineto{\pgfqpoint{2.230495in}{2.659288in}}%
\pgfpathlineto{\pgfqpoint{2.234613in}{2.659288in}}%
\pgfpathlineto{\pgfqpoint{2.238731in}{2.617119in}}%
\pgfpathlineto{\pgfqpoint{2.251085in}{2.617119in}}%
\pgfpathlineto{\pgfqpoint{2.255203in}{2.631176in}}%
\pgfpathlineto{\pgfqpoint{2.267557in}{2.631176in}}%
\pgfpathlineto{\pgfqpoint{2.279912in}{2.673344in}}%
\pgfpathlineto{\pgfqpoint{2.284030in}{2.659288in}}%
\pgfpathlineto{\pgfqpoint{2.300502in}{2.659288in}}%
\pgfpathlineto{\pgfqpoint{2.304620in}{2.645232in}}%
\pgfpathlineto{\pgfqpoint{2.308738in}{2.659288in}}%
\pgfpathlineto{\pgfqpoint{2.329329in}{2.659288in}}%
\pgfpathlineto{\pgfqpoint{2.337565in}{2.631176in}}%
\pgfpathlineto{\pgfqpoint{2.345801in}{2.687400in}}%
\pgfpathlineto{\pgfqpoint{2.358155in}{2.687400in}}%
\pgfpathlineto{\pgfqpoint{2.362273in}{2.673344in}}%
\pgfpathlineto{\pgfqpoint{2.366392in}{2.687400in}}%
\pgfpathlineto{\pgfqpoint{2.370510in}{2.687400in}}%
\pgfpathlineto{\pgfqpoint{2.374628in}{2.715513in}}%
\pgfpathlineto{\pgfqpoint{2.378746in}{2.715513in}}%
\pgfpathlineto{\pgfqpoint{2.382864in}{2.743625in}}%
\pgfpathlineto{\pgfqpoint{2.395218in}{2.743625in}}%
\pgfpathlineto{\pgfqpoint{2.399336in}{2.757682in}}%
\pgfpathlineto{\pgfqpoint{2.407572in}{2.757682in}}%
\pgfpathlineto{\pgfqpoint{2.411690in}{2.743625in}}%
\pgfpathlineto{\pgfqpoint{2.415809in}{2.743625in}}%
\pgfpathlineto{\pgfqpoint{2.419927in}{2.771738in}}%
\pgfpathlineto{\pgfqpoint{2.424045in}{2.757682in}}%
\pgfpathlineto{\pgfqpoint{2.440517in}{2.757682in}}%
\pgfpathlineto{\pgfqpoint{2.452871in}{2.799850in}}%
\pgfpathlineto{\pgfqpoint{2.456989in}{2.771738in}}%
\pgfpathlineto{\pgfqpoint{2.461107in}{2.785794in}}%
\pgfpathlineto{\pgfqpoint{2.469344in}{2.785794in}}%
\pgfpathlineto{\pgfqpoint{2.473462in}{2.799850in}}%
\pgfpathlineto{\pgfqpoint{2.481698in}{2.799850in}}%
\pgfpathlineto{\pgfqpoint{2.485816in}{2.785794in}}%
\pgfpathlineto{\pgfqpoint{2.494052in}{2.785794in}}%
\pgfpathlineto{\pgfqpoint{2.498170in}{2.771738in}}%
\pgfpathlineto{\pgfqpoint{2.506406in}{2.799850in}}%
\pgfpathlineto{\pgfqpoint{2.514643in}{2.771738in}}%
\pgfpathlineto{\pgfqpoint{2.518761in}{2.771738in}}%
\pgfpathlineto{\pgfqpoint{2.522879in}{2.785794in}}%
\pgfpathlineto{\pgfqpoint{2.526997in}{2.771738in}}%
\pgfpathlineto{\pgfqpoint{2.531115in}{2.799850in}}%
\pgfpathlineto{\pgfqpoint{2.535233in}{2.813906in}}%
\pgfpathlineto{\pgfqpoint{2.539351in}{2.799850in}}%
\pgfpathlineto{\pgfqpoint{2.543469in}{2.799850in}}%
\pgfpathlineto{\pgfqpoint{2.547587in}{2.785794in}}%
\pgfpathlineto{\pgfqpoint{2.555823in}{2.785794in}}%
\pgfpathlineto{\pgfqpoint{2.559942in}{2.771738in}}%
\pgfpathlineto{\pgfqpoint{2.572296in}{2.771738in}}%
\pgfpathlineto{\pgfqpoint{2.576414in}{2.757682in}}%
\pgfpathlineto{\pgfqpoint{2.584650in}{2.757682in}}%
\pgfpathlineto{\pgfqpoint{2.588768in}{2.771738in}}%
\pgfpathlineto{\pgfqpoint{2.617595in}{2.771738in}}%
\pgfpathlineto{\pgfqpoint{2.625831in}{2.743625in}}%
\pgfpathlineto{\pgfqpoint{2.634067in}{2.743625in}}%
\pgfpathlineto{\pgfqpoint{2.638185in}{2.757682in}}%
\pgfpathlineto{\pgfqpoint{2.642303in}{2.757682in}}%
\pgfpathlineto{\pgfqpoint{2.646421in}{2.743625in}}%
\pgfpathlineto{\pgfqpoint{2.646421in}{2.743625in}}%
\pgfusepath{stroke}%
\end{pgfscope}%
\begin{pgfscope}%
\pgfpathrectangle{\pgfqpoint{0.488751in}{2.165212in}}{\pgfqpoint{2.260417in}{1.283333in}}%
\pgfusepath{clip}%
\pgfsetbuttcap%
\pgfsetroundjoin%
\pgfsetlinewidth{0.803000pt}%
\definecolor{currentstroke}{rgb}{0.686275,0.352941,0.313725}%
\pgfsetstrokecolor{currentstroke}%
\pgfsetdash{{2.960000pt}{1.280000pt}}{0.000000pt}%
\pgfpathmoveto{\pgfqpoint{0.591497in}{2.378163in}}%
\pgfpathlineto{\pgfqpoint{0.595615in}{2.406276in}}%
\pgfpathlineto{\pgfqpoint{0.599733in}{2.364107in}}%
\pgfpathlineto{\pgfqpoint{0.603851in}{2.364107in}}%
\pgfpathlineto{\pgfqpoint{0.607969in}{2.350051in}}%
\pgfpathlineto{\pgfqpoint{0.616206in}{2.350051in}}%
\pgfpathlineto{\pgfqpoint{0.620324in}{2.364107in}}%
\pgfpathlineto{\pgfqpoint{0.636796in}{2.364107in}}%
\pgfpathlineto{\pgfqpoint{0.640914in}{2.350051in}}%
\pgfpathlineto{\pgfqpoint{0.649150in}{2.350051in}}%
\pgfpathlineto{\pgfqpoint{0.653268in}{2.321939in}}%
\pgfpathlineto{\pgfqpoint{0.657386in}{2.335995in}}%
\pgfpathlineto{\pgfqpoint{0.673859in}{2.335995in}}%
\pgfpathlineto{\pgfqpoint{0.677977in}{2.350051in}}%
\pgfpathlineto{\pgfqpoint{0.686213in}{2.350051in}}%
\pgfpathlineto{\pgfqpoint{0.690331in}{2.364107in}}%
\pgfpathlineto{\pgfqpoint{0.694449in}{2.321939in}}%
\pgfpathlineto{\pgfqpoint{0.719158in}{2.321939in}}%
\pgfpathlineto{\pgfqpoint{0.723276in}{2.335995in}}%
\pgfpathlineto{\pgfqpoint{0.727394in}{2.321939in}}%
\pgfpathlineto{\pgfqpoint{0.780929in}{2.321939in}}%
\pgfpathlineto{\pgfqpoint{0.785047in}{2.293826in}}%
\pgfpathlineto{\pgfqpoint{0.789165in}{2.293826in}}%
\pgfpathlineto{\pgfqpoint{0.797401in}{2.265714in}}%
\pgfpathlineto{\pgfqpoint{0.801519in}{2.279770in}}%
\pgfpathlineto{\pgfqpoint{0.805637in}{2.279770in}}%
\pgfpathlineto{\pgfqpoint{0.809756in}{2.265714in}}%
\pgfpathlineto{\pgfqpoint{0.813874in}{2.279770in}}%
\pgfpathlineto{\pgfqpoint{0.822110in}{2.279770in}}%
\pgfpathlineto{\pgfqpoint{0.826228in}{2.265714in}}%
\pgfpathlineto{\pgfqpoint{0.830346in}{2.279770in}}%
\pgfpathlineto{\pgfqpoint{0.863291in}{2.279770in}}%
\pgfpathlineto{\pgfqpoint{0.871527in}{2.251657in}}%
\pgfpathlineto{\pgfqpoint{0.879763in}{2.279770in}}%
\pgfpathlineto{\pgfqpoint{0.892117in}{2.279770in}}%
\pgfpathlineto{\pgfqpoint{0.896235in}{2.293826in}}%
\pgfpathlineto{\pgfqpoint{0.900353in}{2.279770in}}%
\pgfpathlineto{\pgfqpoint{0.974479in}{2.279770in}}%
\pgfpathlineto{\pgfqpoint{0.978597in}{2.265714in}}%
\pgfpathlineto{\pgfqpoint{0.995069in}{2.321939in}}%
\pgfpathlineto{\pgfqpoint{1.003306in}{2.321939in}}%
\pgfpathlineto{\pgfqpoint{1.007424in}{2.307882in}}%
\pgfpathlineto{\pgfqpoint{1.019778in}{2.307882in}}%
\pgfpathlineto{\pgfqpoint{1.023896in}{2.293826in}}%
\pgfpathlineto{\pgfqpoint{1.048604in}{2.293826in}}%
\pgfpathlineto{\pgfqpoint{1.052723in}{2.279770in}}%
\pgfpathlineto{\pgfqpoint{1.106258in}{2.279770in}}%
\pgfpathlineto{\pgfqpoint{1.110376in}{2.265714in}}%
\pgfpathlineto{\pgfqpoint{1.114494in}{2.279770in}}%
\pgfpathlineto{\pgfqpoint{1.118612in}{2.279770in}}%
\pgfpathlineto{\pgfqpoint{1.122730in}{2.265714in}}%
\pgfpathlineto{\pgfqpoint{1.126848in}{2.265714in}}%
\pgfpathlineto{\pgfqpoint{1.130966in}{2.279770in}}%
\pgfpathlineto{\pgfqpoint{1.143320in}{2.279770in}}%
\pgfpathlineto{\pgfqpoint{1.147438in}{2.293826in}}%
\pgfpathlineto{\pgfqpoint{1.163911in}{2.293826in}}%
\pgfpathlineto{\pgfqpoint{1.168029in}{2.279770in}}%
\pgfpathlineto{\pgfqpoint{1.205092in}{2.279770in}}%
\pgfpathlineto{\pgfqpoint{1.209210in}{2.293826in}}%
\pgfpathlineto{\pgfqpoint{1.229800in}{2.293826in}}%
\pgfpathlineto{\pgfqpoint{1.233918in}{2.307882in}}%
\pgfpathlineto{\pgfqpoint{1.254509in}{2.307882in}}%
\pgfpathlineto{\pgfqpoint{1.258627in}{2.321939in}}%
\pgfpathlineto{\pgfqpoint{1.266863in}{2.321939in}}%
\pgfpathlineto{\pgfqpoint{1.270981in}{2.307882in}}%
\pgfpathlineto{\pgfqpoint{1.291571in}{2.307882in}}%
\pgfpathlineto{\pgfqpoint{1.295690in}{2.350051in}}%
\pgfpathlineto{\pgfqpoint{1.312162in}{2.406276in}}%
\pgfpathlineto{\pgfqpoint{1.316280in}{2.392220in}}%
\pgfpathlineto{\pgfqpoint{1.336870in}{2.392220in}}%
\pgfpathlineto{\pgfqpoint{1.340988in}{2.364107in}}%
\pgfpathlineto{\pgfqpoint{1.345107in}{2.364107in}}%
\pgfpathlineto{\pgfqpoint{1.349225in}{2.378163in}}%
\pgfpathlineto{\pgfqpoint{1.357461in}{2.378163in}}%
\pgfpathlineto{\pgfqpoint{1.361579in}{2.392220in}}%
\pgfpathlineto{\pgfqpoint{1.365697in}{2.392220in}}%
\pgfpathlineto{\pgfqpoint{1.369815in}{2.406276in}}%
\pgfpathlineto{\pgfqpoint{1.382169in}{2.406276in}}%
\pgfpathlineto{\pgfqpoint{1.386287in}{2.420332in}}%
\pgfpathlineto{\pgfqpoint{1.390405in}{2.406276in}}%
\pgfpathlineto{\pgfqpoint{1.394524in}{2.406276in}}%
\pgfpathlineto{\pgfqpoint{1.410996in}{2.462501in}}%
\pgfpathlineto{\pgfqpoint{1.419232in}{2.462501in}}%
\pgfpathlineto{\pgfqpoint{1.423350in}{2.490613in}}%
\pgfpathlineto{\pgfqpoint{1.427468in}{2.490613in}}%
\pgfpathlineto{\pgfqpoint{1.431586in}{2.462501in}}%
\pgfpathlineto{\pgfqpoint{1.435704in}{2.476557in}}%
\pgfpathlineto{\pgfqpoint{1.443941in}{2.532782in}}%
\pgfpathlineto{\pgfqpoint{1.448059in}{2.546838in}}%
\pgfpathlineto{\pgfqpoint{1.452177in}{2.518726in}}%
\pgfpathlineto{\pgfqpoint{1.472767in}{2.518726in}}%
\pgfpathlineto{\pgfqpoint{1.476885in}{2.532782in}}%
\pgfpathlineto{\pgfqpoint{1.485121in}{2.532782in}}%
\pgfpathlineto{\pgfqpoint{1.493358in}{2.504669in}}%
\pgfpathlineto{\pgfqpoint{1.497476in}{2.504669in}}%
\pgfpathlineto{\pgfqpoint{1.501594in}{2.518726in}}%
\pgfpathlineto{\pgfqpoint{1.505712in}{2.504669in}}%
\pgfpathlineto{\pgfqpoint{1.522184in}{2.504669in}}%
\pgfpathlineto{\pgfqpoint{1.526302in}{2.518726in}}%
\pgfpathlineto{\pgfqpoint{1.530420in}{2.546838in}}%
\pgfpathlineto{\pgfqpoint{1.534538in}{2.532782in}}%
\pgfpathlineto{\pgfqpoint{1.538657in}{2.504669in}}%
\pgfpathlineto{\pgfqpoint{1.542775in}{2.504669in}}%
\pgfpathlineto{\pgfqpoint{1.551011in}{2.476557in}}%
\pgfpathlineto{\pgfqpoint{1.563365in}{2.476557in}}%
\pgfpathlineto{\pgfqpoint{1.567483in}{2.462501in}}%
\pgfpathlineto{\pgfqpoint{1.575719in}{2.462501in}}%
\pgfpathlineto{\pgfqpoint{1.579837in}{2.448445in}}%
\pgfpathlineto{\pgfqpoint{1.583955in}{2.476557in}}%
\pgfpathlineto{\pgfqpoint{1.588074in}{2.476557in}}%
\pgfpathlineto{\pgfqpoint{1.592192in}{2.490613in}}%
\pgfpathlineto{\pgfqpoint{1.596310in}{2.490613in}}%
\pgfpathlineto{\pgfqpoint{1.600428in}{2.560894in}}%
\pgfpathlineto{\pgfqpoint{1.604546in}{2.574951in}}%
\pgfpathlineto{\pgfqpoint{1.608664in}{2.546838in}}%
\pgfpathlineto{\pgfqpoint{1.612782in}{2.546838in}}%
\pgfpathlineto{\pgfqpoint{1.621018in}{2.574951in}}%
\pgfpathlineto{\pgfqpoint{1.625136in}{2.532782in}}%
\pgfpathlineto{\pgfqpoint{1.629254in}{2.532782in}}%
\pgfpathlineto{\pgfqpoint{1.633372in}{2.504669in}}%
\pgfpathlineto{\pgfqpoint{1.637491in}{2.490613in}}%
\pgfpathlineto{\pgfqpoint{1.645727in}{2.490613in}}%
\pgfpathlineto{\pgfqpoint{1.649845in}{2.518726in}}%
\pgfpathlineto{\pgfqpoint{1.653963in}{2.532782in}}%
\pgfpathlineto{\pgfqpoint{1.658081in}{2.532782in}}%
\pgfpathlineto{\pgfqpoint{1.662199in}{2.504669in}}%
\pgfpathlineto{\pgfqpoint{1.670435in}{2.476557in}}%
\pgfpathlineto{\pgfqpoint{1.674553in}{2.504669in}}%
\pgfpathlineto{\pgfqpoint{1.678671in}{2.504669in}}%
\pgfpathlineto{\pgfqpoint{1.686908in}{2.532782in}}%
\pgfpathlineto{\pgfqpoint{1.691026in}{2.518726in}}%
\pgfpathlineto{\pgfqpoint{1.695144in}{2.532782in}}%
\pgfpathlineto{\pgfqpoint{1.699262in}{2.560894in}}%
\pgfpathlineto{\pgfqpoint{1.703380in}{2.560894in}}%
\pgfpathlineto{\pgfqpoint{1.707498in}{2.532782in}}%
\pgfpathlineto{\pgfqpoint{1.711616in}{2.560894in}}%
\pgfpathlineto{\pgfqpoint{1.723970in}{2.560894in}}%
\pgfpathlineto{\pgfqpoint{1.728088in}{2.532782in}}%
\pgfpathlineto{\pgfqpoint{1.740443in}{2.532782in}}%
\pgfpathlineto{\pgfqpoint{1.744561in}{2.504669in}}%
\pgfpathlineto{\pgfqpoint{1.761033in}{2.504669in}}%
\pgfpathlineto{\pgfqpoint{1.765151in}{2.462501in}}%
\pgfpathlineto{\pgfqpoint{1.777505in}{2.462501in}}%
\pgfpathlineto{\pgfqpoint{1.781624in}{2.476557in}}%
\pgfpathlineto{\pgfqpoint{1.806332in}{2.476557in}}%
\pgfpathlineto{\pgfqpoint{1.810450in}{2.462501in}}%
\pgfpathlineto{\pgfqpoint{1.814568in}{2.476557in}}%
\pgfpathlineto{\pgfqpoint{1.818686in}{2.448445in}}%
\pgfpathlineto{\pgfqpoint{1.822804in}{2.434388in}}%
\pgfpathlineto{\pgfqpoint{1.826922in}{2.434388in}}%
\pgfpathlineto{\pgfqpoint{1.831041in}{2.462501in}}%
\pgfpathlineto{\pgfqpoint{1.843395in}{2.462501in}}%
\pgfpathlineto{\pgfqpoint{1.847513in}{2.490613in}}%
\pgfpathlineto{\pgfqpoint{1.851631in}{2.504669in}}%
\pgfpathlineto{\pgfqpoint{1.863985in}{2.504669in}}%
\pgfpathlineto{\pgfqpoint{1.868103in}{2.518726in}}%
\pgfpathlineto{\pgfqpoint{1.872221in}{2.490613in}}%
\pgfpathlineto{\pgfqpoint{1.876339in}{2.504669in}}%
\pgfpathlineto{\pgfqpoint{1.896930in}{2.504669in}}%
\pgfpathlineto{\pgfqpoint{1.901048in}{2.462501in}}%
\pgfpathlineto{\pgfqpoint{1.905166in}{2.490613in}}%
\pgfpathlineto{\pgfqpoint{1.909284in}{2.476557in}}%
\pgfpathlineto{\pgfqpoint{1.917520in}{2.476557in}}%
\pgfpathlineto{\pgfqpoint{1.921638in}{2.448445in}}%
\pgfpathlineto{\pgfqpoint{1.925756in}{2.448445in}}%
\pgfpathlineto{\pgfqpoint{1.929875in}{2.434388in}}%
\pgfpathlineto{\pgfqpoint{1.938111in}{2.490613in}}%
\pgfpathlineto{\pgfqpoint{1.942229in}{2.448445in}}%
\pgfpathlineto{\pgfqpoint{1.954583in}{2.448445in}}%
\pgfpathlineto{\pgfqpoint{1.958701in}{2.406276in}}%
\pgfpathlineto{\pgfqpoint{1.962819in}{2.378163in}}%
\pgfpathlineto{\pgfqpoint{1.971055in}{2.378163in}}%
\pgfpathlineto{\pgfqpoint{1.975173in}{2.335995in}}%
\pgfpathlineto{\pgfqpoint{1.983410in}{2.335995in}}%
\pgfpathlineto{\pgfqpoint{1.987528in}{2.364107in}}%
\pgfpathlineto{\pgfqpoint{2.004000in}{2.364107in}}%
\pgfpathlineto{\pgfqpoint{2.008118in}{2.350051in}}%
\pgfpathlineto{\pgfqpoint{2.016354in}{2.350051in}}%
\pgfpathlineto{\pgfqpoint{2.020472in}{2.335995in}}%
\pgfpathlineto{\pgfqpoint{2.028709in}{2.364107in}}%
\pgfpathlineto{\pgfqpoint{2.032827in}{2.392220in}}%
\pgfpathlineto{\pgfqpoint{2.041063in}{2.392220in}}%
\pgfpathlineto{\pgfqpoint{2.045181in}{2.378163in}}%
\pgfpathlineto{\pgfqpoint{2.049299in}{2.392220in}}%
\pgfpathlineto{\pgfqpoint{2.053417in}{2.350051in}}%
\pgfpathlineto{\pgfqpoint{2.057535in}{2.350051in}}%
\pgfpathlineto{\pgfqpoint{2.061653in}{2.364107in}}%
\pgfpathlineto{\pgfqpoint{2.065771in}{2.321939in}}%
\pgfpathlineto{\pgfqpoint{2.069889in}{2.350051in}}%
\pgfpathlineto{\pgfqpoint{2.074008in}{2.350051in}}%
\pgfpathlineto{\pgfqpoint{2.078126in}{2.335995in}}%
\pgfpathlineto{\pgfqpoint{2.082244in}{2.307882in}}%
\pgfpathlineto{\pgfqpoint{2.094598in}{2.350051in}}%
\pgfpathlineto{\pgfqpoint{2.098716in}{2.335995in}}%
\pgfpathlineto{\pgfqpoint{2.106952in}{2.335995in}}%
\pgfpathlineto{\pgfqpoint{2.111070in}{2.350051in}}%
\pgfpathlineto{\pgfqpoint{2.115188in}{2.335995in}}%
\pgfpathlineto{\pgfqpoint{2.123425in}{2.335995in}}%
\pgfpathlineto{\pgfqpoint{2.127543in}{2.321939in}}%
\pgfpathlineto{\pgfqpoint{2.139897in}{2.321939in}}%
\pgfpathlineto{\pgfqpoint{2.144015in}{2.307882in}}%
\pgfpathlineto{\pgfqpoint{2.148133in}{2.307882in}}%
\pgfpathlineto{\pgfqpoint{2.152251in}{2.321939in}}%
\pgfpathlineto{\pgfqpoint{2.156369in}{2.307882in}}%
\pgfpathlineto{\pgfqpoint{2.160487in}{2.307882in}}%
\pgfpathlineto{\pgfqpoint{2.164605in}{2.279770in}}%
\pgfpathlineto{\pgfqpoint{2.193432in}{2.279770in}}%
\pgfpathlineto{\pgfqpoint{2.197550in}{2.265714in}}%
\pgfpathlineto{\pgfqpoint{2.201668in}{2.265714in}}%
\pgfpathlineto{\pgfqpoint{2.205786in}{2.279770in}}%
\pgfpathlineto{\pgfqpoint{2.209904in}{2.265714in}}%
\pgfpathlineto{\pgfqpoint{2.214022in}{2.279770in}}%
\pgfpathlineto{\pgfqpoint{2.222259in}{2.279770in}}%
\pgfpathlineto{\pgfqpoint{2.226377in}{2.251657in}}%
\pgfpathlineto{\pgfqpoint{2.230495in}{2.265714in}}%
\pgfpathlineto{\pgfqpoint{2.234613in}{2.265714in}}%
\pgfpathlineto{\pgfqpoint{2.238731in}{2.279770in}}%
\pgfpathlineto{\pgfqpoint{2.246967in}{2.279770in}}%
\pgfpathlineto{\pgfqpoint{2.251085in}{2.293826in}}%
\pgfpathlineto{\pgfqpoint{2.263439in}{2.293826in}}%
\pgfpathlineto{\pgfqpoint{2.267557in}{2.307882in}}%
\pgfpathlineto{\pgfqpoint{2.288148in}{2.307882in}}%
\pgfpathlineto{\pgfqpoint{2.292266in}{2.279770in}}%
\pgfpathlineto{\pgfqpoint{2.296384in}{2.279770in}}%
\pgfpathlineto{\pgfqpoint{2.300502in}{2.265714in}}%
\pgfpathlineto{\pgfqpoint{2.304620in}{2.279770in}}%
\pgfpathlineto{\pgfqpoint{2.321093in}{2.279770in}}%
\pgfpathlineto{\pgfqpoint{2.325211in}{2.293826in}}%
\pgfpathlineto{\pgfqpoint{2.329329in}{2.293826in}}%
\pgfpathlineto{\pgfqpoint{2.337565in}{2.321939in}}%
\pgfpathlineto{\pgfqpoint{2.366392in}{2.321939in}}%
\pgfpathlineto{\pgfqpoint{2.370510in}{2.307882in}}%
\pgfpathlineto{\pgfqpoint{2.391100in}{2.307882in}}%
\pgfpathlineto{\pgfqpoint{2.395218in}{2.293826in}}%
\pgfpathlineto{\pgfqpoint{2.399336in}{2.293826in}}%
\pgfpathlineto{\pgfqpoint{2.403454in}{2.279770in}}%
\pgfpathlineto{\pgfqpoint{2.407572in}{2.279770in}}%
\pgfpathlineto{\pgfqpoint{2.411690in}{2.293826in}}%
\pgfpathlineto{\pgfqpoint{2.432281in}{2.293826in}}%
\pgfpathlineto{\pgfqpoint{2.436399in}{2.279770in}}%
\pgfpathlineto{\pgfqpoint{2.440517in}{2.293826in}}%
\pgfpathlineto{\pgfqpoint{2.473462in}{2.293826in}}%
\pgfpathlineto{\pgfqpoint{2.477580in}{2.307882in}}%
\pgfpathlineto{\pgfqpoint{2.485816in}{2.279770in}}%
\pgfpathlineto{\pgfqpoint{2.489934in}{2.293826in}}%
\pgfpathlineto{\pgfqpoint{2.494052in}{2.279770in}}%
\pgfpathlineto{\pgfqpoint{2.510524in}{2.279770in}}%
\pgfpathlineto{\pgfqpoint{2.514643in}{2.293826in}}%
\pgfpathlineto{\pgfqpoint{2.526997in}{2.293826in}}%
\pgfpathlineto{\pgfqpoint{2.531115in}{2.265714in}}%
\pgfpathlineto{\pgfqpoint{2.535233in}{2.279770in}}%
\pgfpathlineto{\pgfqpoint{2.543469in}{2.279770in}}%
\pgfpathlineto{\pgfqpoint{2.551705in}{2.307882in}}%
\pgfpathlineto{\pgfqpoint{2.564060in}{2.307882in}}%
\pgfpathlineto{\pgfqpoint{2.568178in}{2.321939in}}%
\pgfpathlineto{\pgfqpoint{2.572296in}{2.307882in}}%
\pgfpathlineto{\pgfqpoint{2.576414in}{2.307882in}}%
\pgfpathlineto{\pgfqpoint{2.580532in}{2.293826in}}%
\pgfpathlineto{\pgfqpoint{2.597004in}{2.293826in}}%
\pgfpathlineto{\pgfqpoint{2.601122in}{2.279770in}}%
\pgfpathlineto{\pgfqpoint{2.605240in}{2.279770in}}%
\pgfpathlineto{\pgfqpoint{2.609359in}{2.251657in}}%
\pgfpathlineto{\pgfqpoint{2.613477in}{2.251657in}}%
\pgfpathlineto{\pgfqpoint{2.617595in}{2.279770in}}%
\pgfpathlineto{\pgfqpoint{2.646421in}{2.279770in}}%
\pgfpathlineto{\pgfqpoint{2.646421in}{2.279770in}}%
\pgfusepath{stroke}%
\end{pgfscope}%
\begin{pgfscope}%
\pgfpathrectangle{\pgfqpoint{0.488751in}{2.165212in}}{\pgfqpoint{2.260417in}{1.283333in}}%
\pgfusepath{clip}%
\pgfsetbuttcap%
\pgfsetroundjoin%
\pgfsetlinewidth{0.803000pt}%
\definecolor{currentstroke}{rgb}{0.000000,0.356863,0.509804}%
\pgfsetstrokecolor{currentstroke}%
\pgfsetdash{{2.960000pt}{1.280000pt}}{0.000000pt}%
\pgfpathmoveto{\pgfqpoint{0.591497in}{2.603063in}}%
\pgfpathlineto{\pgfqpoint{0.599733in}{2.574951in}}%
\pgfpathlineto{\pgfqpoint{0.603851in}{2.589007in}}%
\pgfpathlineto{\pgfqpoint{0.612087in}{2.645232in}}%
\pgfpathlineto{\pgfqpoint{0.616206in}{2.687400in}}%
\pgfpathlineto{\pgfqpoint{0.620324in}{2.687400in}}%
\pgfpathlineto{\pgfqpoint{0.624442in}{2.659288in}}%
\pgfpathlineto{\pgfqpoint{0.628560in}{2.673344in}}%
\pgfpathlineto{\pgfqpoint{0.632678in}{2.659288in}}%
\pgfpathlineto{\pgfqpoint{0.636796in}{2.659288in}}%
\pgfpathlineto{\pgfqpoint{0.640914in}{2.631176in}}%
\pgfpathlineto{\pgfqpoint{0.645032in}{2.701457in}}%
\pgfpathlineto{\pgfqpoint{0.649150in}{2.743625in}}%
\pgfpathlineto{\pgfqpoint{0.653268in}{2.771738in}}%
\pgfpathlineto{\pgfqpoint{0.657386in}{2.687400in}}%
\pgfpathlineto{\pgfqpoint{0.661504in}{2.729569in}}%
\pgfpathlineto{\pgfqpoint{0.665623in}{2.785794in}}%
\pgfpathlineto{\pgfqpoint{0.669741in}{2.771738in}}%
\pgfpathlineto{\pgfqpoint{0.673859in}{2.743625in}}%
\pgfpathlineto{\pgfqpoint{0.677977in}{2.757682in}}%
\pgfpathlineto{\pgfqpoint{0.682095in}{2.729569in}}%
\pgfpathlineto{\pgfqpoint{0.686213in}{2.743625in}}%
\pgfpathlineto{\pgfqpoint{0.690331in}{2.743625in}}%
\pgfpathlineto{\pgfqpoint{0.698567in}{2.771738in}}%
\pgfpathlineto{\pgfqpoint{0.702685in}{2.743625in}}%
\pgfpathlineto{\pgfqpoint{0.706803in}{2.743625in}}%
\pgfpathlineto{\pgfqpoint{0.710922in}{2.757682in}}%
\pgfpathlineto{\pgfqpoint{0.715040in}{2.729569in}}%
\pgfpathlineto{\pgfqpoint{0.719158in}{2.729569in}}%
\pgfpathlineto{\pgfqpoint{0.723276in}{2.701457in}}%
\pgfpathlineto{\pgfqpoint{0.727394in}{2.631176in}}%
\pgfpathlineto{\pgfqpoint{0.731512in}{2.687400in}}%
\pgfpathlineto{\pgfqpoint{0.739748in}{2.687400in}}%
\pgfpathlineto{\pgfqpoint{0.743866in}{2.673344in}}%
\pgfpathlineto{\pgfqpoint{0.747984in}{2.645232in}}%
\pgfpathlineto{\pgfqpoint{0.752102in}{2.701457in}}%
\pgfpathlineto{\pgfqpoint{0.756220in}{2.687400in}}%
\pgfpathlineto{\pgfqpoint{0.760339in}{2.687400in}}%
\pgfpathlineto{\pgfqpoint{0.764457in}{2.715513in}}%
\pgfpathlineto{\pgfqpoint{0.768575in}{2.757682in}}%
\pgfpathlineto{\pgfqpoint{0.772693in}{2.729569in}}%
\pgfpathlineto{\pgfqpoint{0.776811in}{2.729569in}}%
\pgfpathlineto{\pgfqpoint{0.780929in}{2.757682in}}%
\pgfpathlineto{\pgfqpoint{0.785047in}{2.813906in}}%
\pgfpathlineto{\pgfqpoint{0.789165in}{2.827963in}}%
\pgfpathlineto{\pgfqpoint{0.793283in}{2.827963in}}%
\pgfpathlineto{\pgfqpoint{0.797401in}{2.813906in}}%
\pgfpathlineto{\pgfqpoint{0.801519in}{2.842019in}}%
\pgfpathlineto{\pgfqpoint{0.805637in}{2.842019in}}%
\pgfpathlineto{\pgfqpoint{0.809756in}{2.827963in}}%
\pgfpathlineto{\pgfqpoint{0.813874in}{2.827963in}}%
\pgfpathlineto{\pgfqpoint{0.817992in}{2.785794in}}%
\pgfpathlineto{\pgfqpoint{0.822110in}{2.799850in}}%
\pgfpathlineto{\pgfqpoint{0.826228in}{2.771738in}}%
\pgfpathlineto{\pgfqpoint{0.830346in}{2.757682in}}%
\pgfpathlineto{\pgfqpoint{0.834464in}{2.785794in}}%
\pgfpathlineto{\pgfqpoint{0.838582in}{2.771738in}}%
\pgfpathlineto{\pgfqpoint{0.842700in}{2.771738in}}%
\pgfpathlineto{\pgfqpoint{0.846818in}{2.743625in}}%
\pgfpathlineto{\pgfqpoint{0.850936in}{2.771738in}}%
\pgfpathlineto{\pgfqpoint{0.855054in}{2.771738in}}%
\pgfpathlineto{\pgfqpoint{0.859173in}{2.785794in}}%
\pgfpathlineto{\pgfqpoint{0.863291in}{2.757682in}}%
\pgfpathlineto{\pgfqpoint{0.867409in}{2.757682in}}%
\pgfpathlineto{\pgfqpoint{0.871527in}{2.729569in}}%
\pgfpathlineto{\pgfqpoint{0.879763in}{2.729569in}}%
\pgfpathlineto{\pgfqpoint{0.883881in}{2.743625in}}%
\pgfpathlineto{\pgfqpoint{0.887999in}{2.701457in}}%
\pgfpathlineto{\pgfqpoint{0.892117in}{2.673344in}}%
\pgfpathlineto{\pgfqpoint{0.904471in}{2.631176in}}%
\pgfpathlineto{\pgfqpoint{0.912708in}{2.659288in}}%
\pgfpathlineto{\pgfqpoint{0.916826in}{2.631176in}}%
\pgfpathlineto{\pgfqpoint{0.925062in}{2.631176in}}%
\pgfpathlineto{\pgfqpoint{0.929180in}{2.659288in}}%
\pgfpathlineto{\pgfqpoint{0.933298in}{2.659288in}}%
\pgfpathlineto{\pgfqpoint{0.937416in}{2.645232in}}%
\pgfpathlineto{\pgfqpoint{0.941534in}{2.574951in}}%
\pgfpathlineto{\pgfqpoint{0.945652in}{2.589007in}}%
\pgfpathlineto{\pgfqpoint{0.949770in}{2.574951in}}%
\pgfpathlineto{\pgfqpoint{0.953888in}{2.574951in}}%
\pgfpathlineto{\pgfqpoint{0.958007in}{2.603063in}}%
\pgfpathlineto{\pgfqpoint{0.962125in}{2.603063in}}%
\pgfpathlineto{\pgfqpoint{0.966243in}{2.589007in}}%
\pgfpathlineto{\pgfqpoint{0.970361in}{2.603063in}}%
\pgfpathlineto{\pgfqpoint{0.974479in}{2.603063in}}%
\pgfpathlineto{\pgfqpoint{0.978597in}{2.589007in}}%
\pgfpathlineto{\pgfqpoint{0.982715in}{2.589007in}}%
\pgfpathlineto{\pgfqpoint{0.986833in}{2.603063in}}%
\pgfpathlineto{\pgfqpoint{0.990951in}{2.574951in}}%
\pgfpathlineto{\pgfqpoint{0.995069in}{2.574951in}}%
\pgfpathlineto{\pgfqpoint{0.999187in}{2.546838in}}%
\pgfpathlineto{\pgfqpoint{1.003306in}{2.589007in}}%
\pgfpathlineto{\pgfqpoint{1.007424in}{2.603063in}}%
\pgfpathlineto{\pgfqpoint{1.011542in}{2.603063in}}%
\pgfpathlineto{\pgfqpoint{1.015660in}{2.631176in}}%
\pgfpathlineto{\pgfqpoint{1.019778in}{2.645232in}}%
\pgfpathlineto{\pgfqpoint{1.023896in}{2.617119in}}%
\pgfpathlineto{\pgfqpoint{1.032132in}{2.673344in}}%
\pgfpathlineto{\pgfqpoint{1.036250in}{2.687400in}}%
\pgfpathlineto{\pgfqpoint{1.040368in}{2.715513in}}%
\pgfpathlineto{\pgfqpoint{1.044486in}{2.729569in}}%
\pgfpathlineto{\pgfqpoint{1.048604in}{2.687400in}}%
\pgfpathlineto{\pgfqpoint{1.052723in}{2.659288in}}%
\pgfpathlineto{\pgfqpoint{1.056841in}{2.687400in}}%
\pgfpathlineto{\pgfqpoint{1.060959in}{2.645232in}}%
\pgfpathlineto{\pgfqpoint{1.065077in}{2.631176in}}%
\pgfpathlineto{\pgfqpoint{1.069195in}{2.631176in}}%
\pgfpathlineto{\pgfqpoint{1.073313in}{2.659288in}}%
\pgfpathlineto{\pgfqpoint{1.077431in}{2.617119in}}%
\pgfpathlineto{\pgfqpoint{1.081549in}{2.631176in}}%
\pgfpathlineto{\pgfqpoint{1.085667in}{2.617119in}}%
\pgfpathlineto{\pgfqpoint{1.089785in}{2.645232in}}%
\pgfpathlineto{\pgfqpoint{1.093903in}{2.631176in}}%
\pgfpathlineto{\pgfqpoint{1.098021in}{2.589007in}}%
\pgfpathlineto{\pgfqpoint{1.102140in}{2.589007in}}%
\pgfpathlineto{\pgfqpoint{1.106258in}{2.631176in}}%
\pgfpathlineto{\pgfqpoint{1.110376in}{2.589007in}}%
\pgfpathlineto{\pgfqpoint{1.114494in}{2.589007in}}%
\pgfpathlineto{\pgfqpoint{1.118612in}{2.546838in}}%
\pgfpathlineto{\pgfqpoint{1.122730in}{2.574951in}}%
\pgfpathlineto{\pgfqpoint{1.126848in}{2.589007in}}%
\pgfpathlineto{\pgfqpoint{1.130966in}{2.617119in}}%
\pgfpathlineto{\pgfqpoint{1.135084in}{2.631176in}}%
\pgfpathlineto{\pgfqpoint{1.139202in}{2.701457in}}%
\pgfpathlineto{\pgfqpoint{1.143320in}{2.715513in}}%
\pgfpathlineto{\pgfqpoint{1.147438in}{2.687400in}}%
\pgfpathlineto{\pgfqpoint{1.151557in}{2.687400in}}%
\pgfpathlineto{\pgfqpoint{1.155675in}{2.673344in}}%
\pgfpathlineto{\pgfqpoint{1.159793in}{2.687400in}}%
\pgfpathlineto{\pgfqpoint{1.163911in}{2.659288in}}%
\pgfpathlineto{\pgfqpoint{1.168029in}{2.659288in}}%
\pgfpathlineto{\pgfqpoint{1.172147in}{2.673344in}}%
\pgfpathlineto{\pgfqpoint{1.176265in}{2.673344in}}%
\pgfpathlineto{\pgfqpoint{1.180383in}{2.659288in}}%
\pgfpathlineto{\pgfqpoint{1.184501in}{2.659288in}}%
\pgfpathlineto{\pgfqpoint{1.188619in}{2.603063in}}%
\pgfpathlineto{\pgfqpoint{1.192737in}{2.617119in}}%
\pgfpathlineto{\pgfqpoint{1.196855in}{2.603063in}}%
\pgfpathlineto{\pgfqpoint{1.200974in}{2.574951in}}%
\pgfpathlineto{\pgfqpoint{1.205092in}{2.589007in}}%
\pgfpathlineto{\pgfqpoint{1.209210in}{2.617119in}}%
\pgfpathlineto{\pgfqpoint{1.213328in}{2.603063in}}%
\pgfpathlineto{\pgfqpoint{1.225682in}{2.645232in}}%
\pgfpathlineto{\pgfqpoint{1.229800in}{2.687400in}}%
\pgfpathlineto{\pgfqpoint{1.233918in}{2.701457in}}%
\pgfpathlineto{\pgfqpoint{1.238036in}{2.687400in}}%
\pgfpathlineto{\pgfqpoint{1.246273in}{2.687400in}}%
\pgfpathlineto{\pgfqpoint{1.250391in}{2.673344in}}%
\pgfpathlineto{\pgfqpoint{1.258627in}{2.673344in}}%
\pgfpathlineto{\pgfqpoint{1.262745in}{2.631176in}}%
\pgfpathlineto{\pgfqpoint{1.270981in}{2.631176in}}%
\pgfpathlineto{\pgfqpoint{1.275099in}{2.589007in}}%
\pgfpathlineto{\pgfqpoint{1.279217in}{2.560894in}}%
\pgfpathlineto{\pgfqpoint{1.283335in}{2.560894in}}%
\pgfpathlineto{\pgfqpoint{1.287453in}{2.574951in}}%
\pgfpathlineto{\pgfqpoint{1.291571in}{2.603063in}}%
\pgfpathlineto{\pgfqpoint{1.295690in}{2.617119in}}%
\pgfpathlineto{\pgfqpoint{1.299808in}{2.603063in}}%
\pgfpathlineto{\pgfqpoint{1.303926in}{2.617119in}}%
\pgfpathlineto{\pgfqpoint{1.308044in}{2.617119in}}%
\pgfpathlineto{\pgfqpoint{1.324516in}{2.560894in}}%
\pgfpathlineto{\pgfqpoint{1.328634in}{2.532782in}}%
\pgfpathlineto{\pgfqpoint{1.332752in}{2.532782in}}%
\pgfpathlineto{\pgfqpoint{1.336870in}{2.560894in}}%
\pgfpathlineto{\pgfqpoint{1.349225in}{2.603063in}}%
\pgfpathlineto{\pgfqpoint{1.357461in}{2.659288in}}%
\pgfpathlineto{\pgfqpoint{1.365697in}{2.687400in}}%
\pgfpathlineto{\pgfqpoint{1.373933in}{2.687400in}}%
\pgfpathlineto{\pgfqpoint{1.378051in}{2.673344in}}%
\pgfpathlineto{\pgfqpoint{1.382169in}{2.673344in}}%
\pgfpathlineto{\pgfqpoint{1.386287in}{2.645232in}}%
\pgfpathlineto{\pgfqpoint{1.390405in}{2.659288in}}%
\pgfpathlineto{\pgfqpoint{1.394524in}{2.687400in}}%
\pgfpathlineto{\pgfqpoint{1.398642in}{2.701457in}}%
\pgfpathlineto{\pgfqpoint{1.402760in}{2.687400in}}%
\pgfpathlineto{\pgfqpoint{1.406878in}{2.715513in}}%
\pgfpathlineto{\pgfqpoint{1.410996in}{2.701457in}}%
\pgfpathlineto{\pgfqpoint{1.419232in}{2.701457in}}%
\pgfpathlineto{\pgfqpoint{1.423350in}{2.659288in}}%
\pgfpathlineto{\pgfqpoint{1.427468in}{2.645232in}}%
\pgfpathlineto{\pgfqpoint{1.431586in}{2.603063in}}%
\pgfpathlineto{\pgfqpoint{1.435704in}{2.603063in}}%
\pgfpathlineto{\pgfqpoint{1.443941in}{2.574951in}}%
\pgfpathlineto{\pgfqpoint{1.448059in}{2.574951in}}%
\pgfpathlineto{\pgfqpoint{1.452177in}{2.560894in}}%
\pgfpathlineto{\pgfqpoint{1.456295in}{2.560894in}}%
\pgfpathlineto{\pgfqpoint{1.460413in}{2.574951in}}%
\pgfpathlineto{\pgfqpoint{1.464531in}{2.574951in}}%
\pgfpathlineto{\pgfqpoint{1.468649in}{2.603063in}}%
\pgfpathlineto{\pgfqpoint{1.481003in}{2.603063in}}%
\pgfpathlineto{\pgfqpoint{1.485121in}{2.617119in}}%
\pgfpathlineto{\pgfqpoint{1.489240in}{2.645232in}}%
\pgfpathlineto{\pgfqpoint{1.505712in}{2.645232in}}%
\pgfpathlineto{\pgfqpoint{1.509830in}{2.631176in}}%
\pgfpathlineto{\pgfqpoint{1.513948in}{2.645232in}}%
\pgfpathlineto{\pgfqpoint{1.518066in}{2.645232in}}%
\pgfpathlineto{\pgfqpoint{1.522184in}{2.659288in}}%
\pgfpathlineto{\pgfqpoint{1.526302in}{2.645232in}}%
\pgfpathlineto{\pgfqpoint{1.530420in}{2.687400in}}%
\pgfpathlineto{\pgfqpoint{1.538657in}{2.659288in}}%
\pgfpathlineto{\pgfqpoint{1.542775in}{2.673344in}}%
\pgfpathlineto{\pgfqpoint{1.546893in}{2.673344in}}%
\pgfpathlineto{\pgfqpoint{1.551011in}{2.659288in}}%
\pgfpathlineto{\pgfqpoint{1.563365in}{2.659288in}}%
\pgfpathlineto{\pgfqpoint{1.567483in}{2.589007in}}%
\pgfpathlineto{\pgfqpoint{1.575719in}{2.617119in}}%
\pgfpathlineto{\pgfqpoint{1.579837in}{2.603063in}}%
\pgfpathlineto{\pgfqpoint{1.583955in}{2.631176in}}%
\pgfpathlineto{\pgfqpoint{1.592192in}{2.631176in}}%
\pgfpathlineto{\pgfqpoint{1.596310in}{2.589007in}}%
\pgfpathlineto{\pgfqpoint{1.600428in}{2.589007in}}%
\pgfpathlineto{\pgfqpoint{1.604546in}{2.617119in}}%
\pgfpathlineto{\pgfqpoint{1.608664in}{2.574951in}}%
\pgfpathlineto{\pgfqpoint{1.612782in}{2.589007in}}%
\pgfpathlineto{\pgfqpoint{1.616900in}{2.589007in}}%
\pgfpathlineto{\pgfqpoint{1.621018in}{2.603063in}}%
\pgfpathlineto{\pgfqpoint{1.625136in}{2.560894in}}%
\pgfpathlineto{\pgfqpoint{1.629254in}{2.560894in}}%
\pgfpathlineto{\pgfqpoint{1.633372in}{2.518726in}}%
\pgfpathlineto{\pgfqpoint{1.637491in}{2.504669in}}%
\pgfpathlineto{\pgfqpoint{1.641609in}{2.518726in}}%
\pgfpathlineto{\pgfqpoint{1.645727in}{2.518726in}}%
\pgfpathlineto{\pgfqpoint{1.649845in}{2.589007in}}%
\pgfpathlineto{\pgfqpoint{1.658081in}{2.589007in}}%
\pgfpathlineto{\pgfqpoint{1.662199in}{2.631176in}}%
\pgfpathlineto{\pgfqpoint{1.666317in}{2.631176in}}%
\pgfpathlineto{\pgfqpoint{1.670435in}{2.603063in}}%
\pgfpathlineto{\pgfqpoint{1.674553in}{2.589007in}}%
\pgfpathlineto{\pgfqpoint{1.678671in}{2.603063in}}%
\pgfpathlineto{\pgfqpoint{1.682789in}{2.603063in}}%
\pgfpathlineto{\pgfqpoint{1.686908in}{2.631176in}}%
\pgfpathlineto{\pgfqpoint{1.691026in}{2.617119in}}%
\pgfpathlineto{\pgfqpoint{1.703380in}{2.617119in}}%
\pgfpathlineto{\pgfqpoint{1.707498in}{2.589007in}}%
\pgfpathlineto{\pgfqpoint{1.711616in}{2.617119in}}%
\pgfpathlineto{\pgfqpoint{1.715734in}{2.631176in}}%
\pgfpathlineto{\pgfqpoint{1.719852in}{2.617119in}}%
\pgfpathlineto{\pgfqpoint{1.728088in}{2.617119in}}%
\pgfpathlineto{\pgfqpoint{1.732206in}{2.631176in}}%
\pgfpathlineto{\pgfqpoint{1.736325in}{2.631176in}}%
\pgfpathlineto{\pgfqpoint{1.740443in}{2.645232in}}%
\pgfpathlineto{\pgfqpoint{1.748679in}{2.589007in}}%
\pgfpathlineto{\pgfqpoint{1.752797in}{2.574951in}}%
\pgfpathlineto{\pgfqpoint{1.756915in}{2.546838in}}%
\pgfpathlineto{\pgfqpoint{1.761033in}{2.532782in}}%
\pgfpathlineto{\pgfqpoint{1.765151in}{2.490613in}}%
\pgfpathlineto{\pgfqpoint{1.769269in}{2.518726in}}%
\pgfpathlineto{\pgfqpoint{1.773387in}{2.532782in}}%
\pgfpathlineto{\pgfqpoint{1.777505in}{2.532782in}}%
\pgfpathlineto{\pgfqpoint{1.781624in}{2.518726in}}%
\pgfpathlineto{\pgfqpoint{1.785742in}{2.518726in}}%
\pgfpathlineto{\pgfqpoint{1.793978in}{2.546838in}}%
\pgfpathlineto{\pgfqpoint{1.806332in}{2.546838in}}%
\pgfpathlineto{\pgfqpoint{1.814568in}{2.631176in}}%
\pgfpathlineto{\pgfqpoint{1.822804in}{2.574951in}}%
\pgfpathlineto{\pgfqpoint{1.826922in}{2.574951in}}%
\pgfpathlineto{\pgfqpoint{1.831041in}{2.589007in}}%
\pgfpathlineto{\pgfqpoint{1.835159in}{2.560894in}}%
\pgfpathlineto{\pgfqpoint{1.843395in}{2.560894in}}%
\pgfpathlineto{\pgfqpoint{1.847513in}{2.574951in}}%
\pgfpathlineto{\pgfqpoint{1.863985in}{2.574951in}}%
\pgfpathlineto{\pgfqpoint{1.868103in}{2.603063in}}%
\pgfpathlineto{\pgfqpoint{1.872221in}{2.574951in}}%
\pgfpathlineto{\pgfqpoint{1.896930in}{2.490613in}}%
\pgfpathlineto{\pgfqpoint{1.901048in}{2.490613in}}%
\pgfpathlineto{\pgfqpoint{1.905166in}{2.532782in}}%
\pgfpathlineto{\pgfqpoint{1.913402in}{2.560894in}}%
\pgfpathlineto{\pgfqpoint{1.917520in}{2.560894in}}%
\pgfpathlineto{\pgfqpoint{1.921638in}{2.574951in}}%
\pgfpathlineto{\pgfqpoint{1.925756in}{2.603063in}}%
\pgfpathlineto{\pgfqpoint{1.929875in}{2.617119in}}%
\pgfpathlineto{\pgfqpoint{1.938111in}{2.589007in}}%
\pgfpathlineto{\pgfqpoint{1.942229in}{2.617119in}}%
\pgfpathlineto{\pgfqpoint{1.946347in}{2.617119in}}%
\pgfpathlineto{\pgfqpoint{1.950465in}{2.631176in}}%
\pgfpathlineto{\pgfqpoint{1.954583in}{2.631176in}}%
\pgfpathlineto{\pgfqpoint{1.958701in}{2.617119in}}%
\pgfpathlineto{\pgfqpoint{1.962819in}{2.589007in}}%
\pgfpathlineto{\pgfqpoint{1.966937in}{2.589007in}}%
\pgfpathlineto{\pgfqpoint{1.971055in}{2.617119in}}%
\pgfpathlineto{\pgfqpoint{1.975173in}{2.603063in}}%
\pgfpathlineto{\pgfqpoint{1.979292in}{2.631176in}}%
\pgfpathlineto{\pgfqpoint{1.983410in}{2.631176in}}%
\pgfpathlineto{\pgfqpoint{1.987528in}{2.617119in}}%
\pgfpathlineto{\pgfqpoint{1.995764in}{2.617119in}}%
\pgfpathlineto{\pgfqpoint{1.999882in}{2.603063in}}%
\pgfpathlineto{\pgfqpoint{2.008118in}{2.603063in}}%
\pgfpathlineto{\pgfqpoint{2.016354in}{2.574951in}}%
\pgfpathlineto{\pgfqpoint{2.020472in}{2.589007in}}%
\pgfpathlineto{\pgfqpoint{2.024591in}{2.589007in}}%
\pgfpathlineto{\pgfqpoint{2.028709in}{2.603063in}}%
\pgfpathlineto{\pgfqpoint{2.032827in}{2.560894in}}%
\pgfpathlineto{\pgfqpoint{2.036945in}{2.546838in}}%
\pgfpathlineto{\pgfqpoint{2.041063in}{2.546838in}}%
\pgfpathlineto{\pgfqpoint{2.045181in}{2.574951in}}%
\pgfpathlineto{\pgfqpoint{2.049299in}{2.574951in}}%
\pgfpathlineto{\pgfqpoint{2.053417in}{2.560894in}}%
\pgfpathlineto{\pgfqpoint{2.057535in}{2.560894in}}%
\pgfpathlineto{\pgfqpoint{2.061653in}{2.574951in}}%
\pgfpathlineto{\pgfqpoint{2.065771in}{2.560894in}}%
\pgfpathlineto{\pgfqpoint{2.069889in}{2.574951in}}%
\pgfpathlineto{\pgfqpoint{2.074008in}{2.574951in}}%
\pgfpathlineto{\pgfqpoint{2.078126in}{2.589007in}}%
\pgfpathlineto{\pgfqpoint{2.082244in}{2.617119in}}%
\pgfpathlineto{\pgfqpoint{2.086362in}{2.631176in}}%
\pgfpathlineto{\pgfqpoint{2.090480in}{2.631176in}}%
\pgfpathlineto{\pgfqpoint{2.094598in}{2.701457in}}%
\pgfpathlineto{\pgfqpoint{2.098716in}{2.673344in}}%
\pgfpathlineto{\pgfqpoint{2.102834in}{2.687400in}}%
\pgfpathlineto{\pgfqpoint{2.106952in}{2.687400in}}%
\pgfpathlineto{\pgfqpoint{2.111070in}{2.729569in}}%
\pgfpathlineto{\pgfqpoint{2.115188in}{2.743625in}}%
\pgfpathlineto{\pgfqpoint{2.119306in}{2.743625in}}%
\pgfpathlineto{\pgfqpoint{2.123425in}{2.757682in}}%
\pgfpathlineto{\pgfqpoint{2.135779in}{2.757682in}}%
\pgfpathlineto{\pgfqpoint{2.139897in}{2.771738in}}%
\pgfpathlineto{\pgfqpoint{2.148133in}{2.715513in}}%
\pgfpathlineto{\pgfqpoint{2.152251in}{2.715513in}}%
\pgfpathlineto{\pgfqpoint{2.156369in}{2.701457in}}%
\pgfpathlineto{\pgfqpoint{2.160487in}{2.701457in}}%
\pgfpathlineto{\pgfqpoint{2.164605in}{2.687400in}}%
\pgfpathlineto{\pgfqpoint{2.168723in}{2.659288in}}%
\pgfpathlineto{\pgfqpoint{2.172842in}{2.659288in}}%
\pgfpathlineto{\pgfqpoint{2.176960in}{2.673344in}}%
\pgfpathlineto{\pgfqpoint{2.185196in}{2.673344in}}%
\pgfpathlineto{\pgfqpoint{2.189314in}{2.687400in}}%
\pgfpathlineto{\pgfqpoint{2.193432in}{2.673344in}}%
\pgfpathlineto{\pgfqpoint{2.197550in}{2.645232in}}%
\pgfpathlineto{\pgfqpoint{2.201668in}{2.645232in}}%
\pgfpathlineto{\pgfqpoint{2.209904in}{2.617119in}}%
\pgfpathlineto{\pgfqpoint{2.214022in}{2.617119in}}%
\pgfpathlineto{\pgfqpoint{2.222259in}{2.589007in}}%
\pgfpathlineto{\pgfqpoint{2.226377in}{2.560894in}}%
\pgfpathlineto{\pgfqpoint{2.230495in}{2.560894in}}%
\pgfpathlineto{\pgfqpoint{2.234613in}{2.532782in}}%
\pgfpathlineto{\pgfqpoint{2.238731in}{2.546838in}}%
\pgfpathlineto{\pgfqpoint{2.242849in}{2.532782in}}%
\pgfpathlineto{\pgfqpoint{2.246967in}{2.532782in}}%
\pgfpathlineto{\pgfqpoint{2.251085in}{2.546838in}}%
\pgfpathlineto{\pgfqpoint{2.255203in}{2.546838in}}%
\pgfpathlineto{\pgfqpoint{2.263439in}{2.518726in}}%
\pgfpathlineto{\pgfqpoint{2.267557in}{2.518726in}}%
\pgfpathlineto{\pgfqpoint{2.271676in}{2.490613in}}%
\pgfpathlineto{\pgfqpoint{2.275794in}{2.490613in}}%
\pgfpathlineto{\pgfqpoint{2.279912in}{2.476557in}}%
\pgfpathlineto{\pgfqpoint{2.284030in}{2.490613in}}%
\pgfpathlineto{\pgfqpoint{2.288148in}{2.532782in}}%
\pgfpathlineto{\pgfqpoint{2.296384in}{2.532782in}}%
\pgfpathlineto{\pgfqpoint{2.300502in}{2.504669in}}%
\pgfpathlineto{\pgfqpoint{2.304620in}{2.518726in}}%
\pgfpathlineto{\pgfqpoint{2.308738in}{2.546838in}}%
\pgfpathlineto{\pgfqpoint{2.312856in}{2.546838in}}%
\pgfpathlineto{\pgfqpoint{2.316975in}{2.532782in}}%
\pgfpathlineto{\pgfqpoint{2.325211in}{2.532782in}}%
\pgfpathlineto{\pgfqpoint{2.329329in}{2.476557in}}%
\pgfpathlineto{\pgfqpoint{2.333447in}{2.518726in}}%
\pgfpathlineto{\pgfqpoint{2.337565in}{2.532782in}}%
\pgfpathlineto{\pgfqpoint{2.341683in}{2.504669in}}%
\pgfpathlineto{\pgfqpoint{2.345801in}{2.490613in}}%
\pgfpathlineto{\pgfqpoint{2.358155in}{2.490613in}}%
\pgfpathlineto{\pgfqpoint{2.366392in}{2.518726in}}%
\pgfpathlineto{\pgfqpoint{2.370510in}{2.504669in}}%
\pgfpathlineto{\pgfqpoint{2.378746in}{2.504669in}}%
\pgfpathlineto{\pgfqpoint{2.382864in}{2.490613in}}%
\pgfpathlineto{\pgfqpoint{2.403454in}{2.490613in}}%
\pgfpathlineto{\pgfqpoint{2.407572in}{2.476557in}}%
\pgfpathlineto{\pgfqpoint{2.419927in}{2.476557in}}%
\pgfpathlineto{\pgfqpoint{2.424045in}{2.490613in}}%
\pgfpathlineto{\pgfqpoint{2.428163in}{2.476557in}}%
\pgfpathlineto{\pgfqpoint{2.432281in}{2.476557in}}%
\pgfpathlineto{\pgfqpoint{2.436399in}{2.462501in}}%
\pgfpathlineto{\pgfqpoint{2.440517in}{2.434388in}}%
\pgfpathlineto{\pgfqpoint{2.448753in}{2.434388in}}%
\pgfpathlineto{\pgfqpoint{2.452871in}{2.448445in}}%
\pgfpathlineto{\pgfqpoint{2.461107in}{2.448445in}}%
\pgfpathlineto{\pgfqpoint{2.465226in}{2.462501in}}%
\pgfpathlineto{\pgfqpoint{2.473462in}{2.462501in}}%
\pgfpathlineto{\pgfqpoint{2.477580in}{2.476557in}}%
\pgfpathlineto{\pgfqpoint{2.481698in}{2.476557in}}%
\pgfpathlineto{\pgfqpoint{2.485816in}{2.462501in}}%
\pgfpathlineto{\pgfqpoint{2.506406in}{2.462501in}}%
\pgfpathlineto{\pgfqpoint{2.510524in}{2.476557in}}%
\pgfpathlineto{\pgfqpoint{2.535233in}{2.476557in}}%
\pgfpathlineto{\pgfqpoint{2.539351in}{2.490613in}}%
\pgfpathlineto{\pgfqpoint{2.555823in}{2.490613in}}%
\pgfpathlineto{\pgfqpoint{2.559942in}{2.476557in}}%
\pgfpathlineto{\pgfqpoint{2.564060in}{2.448445in}}%
\pgfpathlineto{\pgfqpoint{2.568178in}{2.462501in}}%
\pgfpathlineto{\pgfqpoint{2.580532in}{2.462501in}}%
\pgfpathlineto{\pgfqpoint{2.584650in}{2.448445in}}%
\pgfpathlineto{\pgfqpoint{2.605240in}{2.448445in}}%
\pgfpathlineto{\pgfqpoint{2.609359in}{2.434388in}}%
\pgfpathlineto{\pgfqpoint{2.634067in}{2.434388in}}%
\pgfpathlineto{\pgfqpoint{2.638185in}{2.420332in}}%
\pgfpathlineto{\pgfqpoint{2.646421in}{2.420332in}}%
\pgfpathlineto{\pgfqpoint{2.646421in}{2.420332in}}%
\pgfusepath{stroke}%
\end{pgfscope}%
\begin{pgfscope}%
\pgfpathrectangle{\pgfqpoint{0.488751in}{2.165212in}}{\pgfqpoint{2.260417in}{1.283333in}}%
\pgfusepath{clip}%
\pgfsetbuttcap%
\pgfsetroundjoin%
\pgfsetlinewidth{0.803000pt}%
\definecolor{currentstroke}{rgb}{0.490196,0.588235,0.431373}%
\pgfsetstrokecolor{currentstroke}%
\pgfsetdash{{2.960000pt}{1.280000pt}}{0.000000pt}%
\pgfpathmoveto{\pgfqpoint{0.591497in}{2.518726in}}%
\pgfpathlineto{\pgfqpoint{0.595615in}{2.490613in}}%
\pgfpathlineto{\pgfqpoint{0.603851in}{2.490613in}}%
\pgfpathlineto{\pgfqpoint{0.607969in}{2.476557in}}%
\pgfpathlineto{\pgfqpoint{0.620324in}{2.476557in}}%
\pgfpathlineto{\pgfqpoint{0.624442in}{2.462501in}}%
\pgfpathlineto{\pgfqpoint{0.632678in}{2.462501in}}%
\pgfpathlineto{\pgfqpoint{0.640914in}{2.434388in}}%
\pgfpathlineto{\pgfqpoint{0.653268in}{2.434388in}}%
\pgfpathlineto{\pgfqpoint{0.657386in}{2.406276in}}%
\pgfpathlineto{\pgfqpoint{0.661504in}{2.406276in}}%
\pgfpathlineto{\pgfqpoint{0.665623in}{2.420332in}}%
\pgfpathlineto{\pgfqpoint{0.719158in}{2.420332in}}%
\pgfpathlineto{\pgfqpoint{0.723276in}{2.406276in}}%
\pgfpathlineto{\pgfqpoint{0.743866in}{2.406276in}}%
\pgfpathlineto{\pgfqpoint{0.747984in}{2.392220in}}%
\pgfpathlineto{\pgfqpoint{0.780929in}{2.392220in}}%
\pgfpathlineto{\pgfqpoint{0.785047in}{2.378163in}}%
\pgfpathlineto{\pgfqpoint{0.813874in}{2.378163in}}%
\pgfpathlineto{\pgfqpoint{0.817992in}{2.364107in}}%
\pgfpathlineto{\pgfqpoint{1.114494in}{2.364107in}}%
\pgfpathlineto{\pgfqpoint{1.118612in}{2.350051in}}%
\pgfpathlineto{\pgfqpoint{1.365697in}{2.350051in}}%
\pgfpathlineto{\pgfqpoint{1.369815in}{2.335995in}}%
\pgfpathlineto{\pgfqpoint{2.316975in}{2.335995in}}%
\pgfpathlineto{\pgfqpoint{2.321093in}{2.321939in}}%
\pgfpathlineto{\pgfqpoint{2.646421in}{2.321939in}}%
\pgfpathlineto{\pgfqpoint{2.646421in}{2.321939in}}%
\pgfusepath{stroke}%
\end{pgfscope}%
\begin{pgfscope}%
\pgfpathrectangle{\pgfqpoint{0.488751in}{2.165212in}}{\pgfqpoint{2.260417in}{1.283333in}}%
\pgfusepath{clip}%
\pgfsetbuttcap%
\pgfsetroundjoin%
\pgfsetlinewidth{0.803000pt}%
\definecolor{currentstroke}{rgb}{0.843137,0.666667,0.313725}%
\pgfsetstrokecolor{currentstroke}%
\pgfsetdash{{2.960000pt}{1.280000pt}}{0.000000pt}%
\pgfpathmoveto{\pgfqpoint{0.591497in}{2.940412in}}%
\pgfpathlineto{\pgfqpoint{0.595615in}{2.982581in}}%
\pgfpathlineto{\pgfqpoint{0.599733in}{2.996637in}}%
\pgfpathlineto{\pgfqpoint{0.603851in}{2.968525in}}%
\pgfpathlineto{\pgfqpoint{0.607969in}{2.996637in}}%
\pgfpathlineto{\pgfqpoint{0.616206in}{3.024750in}}%
\pgfpathlineto{\pgfqpoint{0.620324in}{2.996637in}}%
\pgfpathlineto{\pgfqpoint{0.624442in}{2.982581in}}%
\pgfpathlineto{\pgfqpoint{0.632678in}{3.080975in}}%
\pgfpathlineto{\pgfqpoint{0.636796in}{3.066918in}}%
\pgfpathlineto{\pgfqpoint{0.640914in}{3.080975in}}%
\pgfpathlineto{\pgfqpoint{0.645032in}{3.052862in}}%
\pgfpathlineto{\pgfqpoint{0.649150in}{3.095031in}}%
\pgfpathlineto{\pgfqpoint{0.653268in}{3.038806in}}%
\pgfpathlineto{\pgfqpoint{0.657386in}{3.010694in}}%
\pgfpathlineto{\pgfqpoint{0.661504in}{2.996637in}}%
\pgfpathlineto{\pgfqpoint{0.665623in}{3.024750in}}%
\pgfpathlineto{\pgfqpoint{0.669741in}{3.095031in}}%
\pgfpathlineto{\pgfqpoint{0.673859in}{3.080975in}}%
\pgfpathlineto{\pgfqpoint{0.677977in}{3.137200in}}%
\pgfpathlineto{\pgfqpoint{0.682095in}{3.123143in}}%
\pgfpathlineto{\pgfqpoint{0.686213in}{3.052862in}}%
\pgfpathlineto{\pgfqpoint{0.690331in}{3.038806in}}%
\pgfpathlineto{\pgfqpoint{0.698567in}{2.982581in}}%
\pgfpathlineto{\pgfqpoint{0.702685in}{2.940412in}}%
\pgfpathlineto{\pgfqpoint{0.706803in}{2.926356in}}%
\pgfpathlineto{\pgfqpoint{0.710922in}{2.898244in}}%
\pgfpathlineto{\pgfqpoint{0.715040in}{2.940412in}}%
\pgfpathlineto{\pgfqpoint{0.719158in}{2.898244in}}%
\pgfpathlineto{\pgfqpoint{0.723276in}{2.813906in}}%
\pgfpathlineto{\pgfqpoint{0.727394in}{2.856075in}}%
\pgfpathlineto{\pgfqpoint{0.731512in}{2.827963in}}%
\pgfpathlineto{\pgfqpoint{0.735630in}{2.771738in}}%
\pgfpathlineto{\pgfqpoint{0.743866in}{2.799850in}}%
\pgfpathlineto{\pgfqpoint{0.752102in}{2.743625in}}%
\pgfpathlineto{\pgfqpoint{0.756220in}{2.743625in}}%
\pgfpathlineto{\pgfqpoint{0.764457in}{2.687400in}}%
\pgfpathlineto{\pgfqpoint{0.813874in}{2.687400in}}%
\pgfpathlineto{\pgfqpoint{0.817992in}{2.701457in}}%
\pgfpathlineto{\pgfqpoint{0.822110in}{2.687400in}}%
\pgfpathlineto{\pgfqpoint{0.842700in}{2.687400in}}%
\pgfpathlineto{\pgfqpoint{0.846818in}{2.701457in}}%
\pgfpathlineto{\pgfqpoint{0.908590in}{2.701457in}}%
\pgfpathlineto{\pgfqpoint{0.912708in}{2.687400in}}%
\pgfpathlineto{\pgfqpoint{0.953888in}{2.687400in}}%
\pgfpathlineto{\pgfqpoint{0.958007in}{2.701457in}}%
\pgfpathlineto{\pgfqpoint{0.986833in}{2.701457in}}%
\pgfpathlineto{\pgfqpoint{0.995069in}{2.729569in}}%
\pgfpathlineto{\pgfqpoint{0.999187in}{2.729569in}}%
\pgfpathlineto{\pgfqpoint{1.003306in}{2.715513in}}%
\pgfpathlineto{\pgfqpoint{1.007424in}{2.757682in}}%
\pgfpathlineto{\pgfqpoint{1.015660in}{2.813906in}}%
\pgfpathlineto{\pgfqpoint{1.019778in}{2.827963in}}%
\pgfpathlineto{\pgfqpoint{1.023896in}{2.856075in}}%
\pgfpathlineto{\pgfqpoint{1.028014in}{2.870131in}}%
\pgfpathlineto{\pgfqpoint{1.032132in}{2.856075in}}%
\pgfpathlineto{\pgfqpoint{1.036250in}{2.870131in}}%
\pgfpathlineto{\pgfqpoint{1.040368in}{2.870131in}}%
\pgfpathlineto{\pgfqpoint{1.044486in}{2.813906in}}%
\pgfpathlineto{\pgfqpoint{1.048604in}{2.799850in}}%
\pgfpathlineto{\pgfqpoint{1.052723in}{2.771738in}}%
\pgfpathlineto{\pgfqpoint{1.060959in}{2.771738in}}%
\pgfpathlineto{\pgfqpoint{1.065077in}{2.757682in}}%
\pgfpathlineto{\pgfqpoint{1.077431in}{2.757682in}}%
\pgfpathlineto{\pgfqpoint{1.081549in}{2.771738in}}%
\pgfpathlineto{\pgfqpoint{1.110376in}{2.771738in}}%
\pgfpathlineto{\pgfqpoint{1.114494in}{2.785794in}}%
\pgfpathlineto{\pgfqpoint{1.143320in}{2.785794in}}%
\pgfpathlineto{\pgfqpoint{1.147438in}{2.757682in}}%
\pgfpathlineto{\pgfqpoint{1.172147in}{2.757682in}}%
\pgfpathlineto{\pgfqpoint{1.176265in}{2.743625in}}%
\pgfpathlineto{\pgfqpoint{1.180383in}{2.757682in}}%
\pgfpathlineto{\pgfqpoint{1.287453in}{2.757682in}}%
\pgfpathlineto{\pgfqpoint{1.291571in}{2.771738in}}%
\pgfpathlineto{\pgfqpoint{1.295690in}{2.771738in}}%
\pgfpathlineto{\pgfqpoint{1.299808in}{2.785794in}}%
\pgfpathlineto{\pgfqpoint{1.303926in}{2.785794in}}%
\pgfpathlineto{\pgfqpoint{1.308044in}{2.799850in}}%
\pgfpathlineto{\pgfqpoint{1.312162in}{2.827963in}}%
\pgfpathlineto{\pgfqpoint{1.316280in}{2.799850in}}%
\pgfpathlineto{\pgfqpoint{1.320398in}{2.813906in}}%
\pgfpathlineto{\pgfqpoint{1.324516in}{2.799850in}}%
\pgfpathlineto{\pgfqpoint{1.328634in}{2.799850in}}%
\pgfpathlineto{\pgfqpoint{1.332752in}{2.785794in}}%
\pgfpathlineto{\pgfqpoint{1.336870in}{2.785794in}}%
\pgfpathlineto{\pgfqpoint{1.340988in}{2.813906in}}%
\pgfpathlineto{\pgfqpoint{1.349225in}{2.813906in}}%
\pgfpathlineto{\pgfqpoint{1.353343in}{2.827963in}}%
\pgfpathlineto{\pgfqpoint{1.382169in}{2.827963in}}%
\pgfpathlineto{\pgfqpoint{1.386287in}{2.842019in}}%
\pgfpathlineto{\pgfqpoint{1.390405in}{2.842019in}}%
\pgfpathlineto{\pgfqpoint{1.394524in}{2.856075in}}%
\pgfpathlineto{\pgfqpoint{1.398642in}{2.856075in}}%
\pgfpathlineto{\pgfqpoint{1.402760in}{2.813906in}}%
\pgfpathlineto{\pgfqpoint{1.419232in}{2.813906in}}%
\pgfpathlineto{\pgfqpoint{1.423350in}{2.842019in}}%
\pgfpathlineto{\pgfqpoint{1.435704in}{2.842019in}}%
\pgfpathlineto{\pgfqpoint{1.439822in}{2.870131in}}%
\pgfpathlineto{\pgfqpoint{1.443941in}{2.856075in}}%
\pgfpathlineto{\pgfqpoint{1.448059in}{2.856075in}}%
\pgfpathlineto{\pgfqpoint{1.452177in}{2.870131in}}%
\pgfpathlineto{\pgfqpoint{1.456295in}{2.870131in}}%
\pgfpathlineto{\pgfqpoint{1.460413in}{2.856075in}}%
\pgfpathlineto{\pgfqpoint{1.464531in}{2.856075in}}%
\pgfpathlineto{\pgfqpoint{1.468649in}{2.870131in}}%
\pgfpathlineto{\pgfqpoint{1.489240in}{2.870131in}}%
\pgfpathlineto{\pgfqpoint{1.493358in}{2.898244in}}%
\pgfpathlineto{\pgfqpoint{1.497476in}{2.884188in}}%
\pgfpathlineto{\pgfqpoint{1.505712in}{2.884188in}}%
\pgfpathlineto{\pgfqpoint{1.509830in}{2.898244in}}%
\pgfpathlineto{\pgfqpoint{1.513948in}{2.884188in}}%
\pgfpathlineto{\pgfqpoint{1.518066in}{2.912300in}}%
\pgfpathlineto{\pgfqpoint{1.522184in}{2.870131in}}%
\pgfpathlineto{\pgfqpoint{1.526302in}{2.870131in}}%
\pgfpathlineto{\pgfqpoint{1.530420in}{2.856075in}}%
\pgfpathlineto{\pgfqpoint{1.534538in}{2.827963in}}%
\pgfpathlineto{\pgfqpoint{1.538657in}{2.856075in}}%
\pgfpathlineto{\pgfqpoint{1.546893in}{2.856075in}}%
\pgfpathlineto{\pgfqpoint{1.551011in}{2.827963in}}%
\pgfpathlineto{\pgfqpoint{1.563365in}{2.827963in}}%
\pgfpathlineto{\pgfqpoint{1.567483in}{2.856075in}}%
\pgfpathlineto{\pgfqpoint{1.571601in}{2.842019in}}%
\pgfpathlineto{\pgfqpoint{1.575719in}{2.842019in}}%
\pgfpathlineto{\pgfqpoint{1.579837in}{2.856075in}}%
\pgfpathlineto{\pgfqpoint{1.583955in}{2.884188in}}%
\pgfpathlineto{\pgfqpoint{1.592192in}{2.884188in}}%
\pgfpathlineto{\pgfqpoint{1.596310in}{2.898244in}}%
\pgfpathlineto{\pgfqpoint{1.600428in}{2.870131in}}%
\pgfpathlineto{\pgfqpoint{1.608664in}{2.898244in}}%
\pgfpathlineto{\pgfqpoint{1.621018in}{2.898244in}}%
\pgfpathlineto{\pgfqpoint{1.625136in}{2.926356in}}%
\pgfpathlineto{\pgfqpoint{1.629254in}{2.926356in}}%
\pgfpathlineto{\pgfqpoint{1.633372in}{2.954469in}}%
\pgfpathlineto{\pgfqpoint{1.637491in}{2.926356in}}%
\pgfpathlineto{\pgfqpoint{1.645727in}{2.926356in}}%
\pgfpathlineto{\pgfqpoint{1.649845in}{2.940412in}}%
\pgfpathlineto{\pgfqpoint{1.658081in}{2.940412in}}%
\pgfpathlineto{\pgfqpoint{1.666317in}{2.968525in}}%
\pgfpathlineto{\pgfqpoint{1.674553in}{2.940412in}}%
\pgfpathlineto{\pgfqpoint{1.678671in}{2.968525in}}%
\pgfpathlineto{\pgfqpoint{1.682789in}{2.940412in}}%
\pgfpathlineto{\pgfqpoint{1.686908in}{2.926356in}}%
\pgfpathlineto{\pgfqpoint{1.691026in}{2.940412in}}%
\pgfpathlineto{\pgfqpoint{1.699262in}{2.912300in}}%
\pgfpathlineto{\pgfqpoint{1.703380in}{2.912300in}}%
\pgfpathlineto{\pgfqpoint{1.707498in}{2.884188in}}%
\pgfpathlineto{\pgfqpoint{1.711616in}{2.926356in}}%
\pgfpathlineto{\pgfqpoint{1.719852in}{2.898244in}}%
\pgfpathlineto{\pgfqpoint{1.756915in}{2.898244in}}%
\pgfpathlineto{\pgfqpoint{1.761033in}{2.940412in}}%
\pgfpathlineto{\pgfqpoint{1.765151in}{2.954469in}}%
\pgfpathlineto{\pgfqpoint{1.769269in}{2.940412in}}%
\pgfpathlineto{\pgfqpoint{1.773387in}{2.940412in}}%
\pgfpathlineto{\pgfqpoint{1.777505in}{2.954469in}}%
\pgfpathlineto{\pgfqpoint{1.781624in}{2.940412in}}%
\pgfpathlineto{\pgfqpoint{1.785742in}{2.982581in}}%
\pgfpathlineto{\pgfqpoint{1.789860in}{3.010694in}}%
\pgfpathlineto{\pgfqpoint{1.793978in}{2.996637in}}%
\pgfpathlineto{\pgfqpoint{1.798096in}{2.996637in}}%
\pgfpathlineto{\pgfqpoint{1.802214in}{3.024750in}}%
\pgfpathlineto{\pgfqpoint{1.806332in}{3.024750in}}%
\pgfpathlineto{\pgfqpoint{1.818686in}{2.982581in}}%
\pgfpathlineto{\pgfqpoint{1.822804in}{2.982581in}}%
\pgfpathlineto{\pgfqpoint{1.831041in}{2.954469in}}%
\pgfpathlineto{\pgfqpoint{1.843395in}{2.954469in}}%
\pgfpathlineto{\pgfqpoint{1.847513in}{2.940412in}}%
\pgfpathlineto{\pgfqpoint{1.917520in}{2.940412in}}%
\pgfpathlineto{\pgfqpoint{1.921638in}{2.926356in}}%
\pgfpathlineto{\pgfqpoint{1.929875in}{2.926356in}}%
\pgfpathlineto{\pgfqpoint{1.933993in}{2.940412in}}%
\pgfpathlineto{\pgfqpoint{1.938111in}{2.968525in}}%
\pgfpathlineto{\pgfqpoint{1.942229in}{2.954469in}}%
\pgfpathlineto{\pgfqpoint{1.954583in}{2.954469in}}%
\pgfpathlineto{\pgfqpoint{1.958701in}{2.982581in}}%
\pgfpathlineto{\pgfqpoint{1.971055in}{2.982581in}}%
\pgfpathlineto{\pgfqpoint{1.975173in}{3.010694in}}%
\pgfpathlineto{\pgfqpoint{1.983410in}{3.010694in}}%
\pgfpathlineto{\pgfqpoint{1.987528in}{2.996637in}}%
\pgfpathlineto{\pgfqpoint{2.008118in}{2.996637in}}%
\pgfpathlineto{\pgfqpoint{2.012236in}{3.024750in}}%
\pgfpathlineto{\pgfqpoint{2.016354in}{3.024750in}}%
\pgfpathlineto{\pgfqpoint{2.020472in}{3.052862in}}%
\pgfpathlineto{\pgfqpoint{2.024591in}{3.038806in}}%
\pgfpathlineto{\pgfqpoint{2.028709in}{3.038806in}}%
\pgfpathlineto{\pgfqpoint{2.032827in}{3.066918in}}%
\pgfpathlineto{\pgfqpoint{2.041063in}{3.010694in}}%
\pgfpathlineto{\pgfqpoint{2.045181in}{3.010694in}}%
\pgfpathlineto{\pgfqpoint{2.049299in}{3.024750in}}%
\pgfpathlineto{\pgfqpoint{2.061653in}{3.024750in}}%
\pgfpathlineto{\pgfqpoint{2.065771in}{3.052862in}}%
\pgfpathlineto{\pgfqpoint{2.069889in}{3.066918in}}%
\pgfpathlineto{\pgfqpoint{2.074008in}{3.052862in}}%
\pgfpathlineto{\pgfqpoint{2.082244in}{3.080975in}}%
\pgfpathlineto{\pgfqpoint{2.086362in}{3.038806in}}%
\pgfpathlineto{\pgfqpoint{2.090480in}{3.038806in}}%
\pgfpathlineto{\pgfqpoint{2.094598in}{3.052862in}}%
\pgfpathlineto{\pgfqpoint{2.098716in}{3.052862in}}%
\pgfpathlineto{\pgfqpoint{2.102834in}{3.038806in}}%
\pgfpathlineto{\pgfqpoint{2.106952in}{3.038806in}}%
\pgfpathlineto{\pgfqpoint{2.111070in}{3.052862in}}%
\pgfpathlineto{\pgfqpoint{2.115188in}{3.052862in}}%
\pgfpathlineto{\pgfqpoint{2.119306in}{3.080975in}}%
\pgfpathlineto{\pgfqpoint{2.127543in}{3.080975in}}%
\pgfpathlineto{\pgfqpoint{2.131661in}{3.095031in}}%
\pgfpathlineto{\pgfqpoint{2.144015in}{3.095031in}}%
\pgfpathlineto{\pgfqpoint{2.148133in}{3.066918in}}%
\pgfpathlineto{\pgfqpoint{2.152251in}{3.066918in}}%
\pgfpathlineto{\pgfqpoint{2.156369in}{3.080975in}}%
\pgfpathlineto{\pgfqpoint{2.160487in}{3.066918in}}%
\pgfpathlineto{\pgfqpoint{2.164605in}{3.080975in}}%
\pgfpathlineto{\pgfqpoint{2.168723in}{3.080975in}}%
\pgfpathlineto{\pgfqpoint{2.172842in}{3.095031in}}%
\pgfpathlineto{\pgfqpoint{2.189314in}{3.095031in}}%
\pgfpathlineto{\pgfqpoint{2.193432in}{3.080975in}}%
\pgfpathlineto{\pgfqpoint{2.197550in}{3.109087in}}%
\pgfpathlineto{\pgfqpoint{2.205786in}{3.109087in}}%
\pgfpathlineto{\pgfqpoint{2.209904in}{3.095031in}}%
\pgfpathlineto{\pgfqpoint{2.214022in}{3.095031in}}%
\pgfpathlineto{\pgfqpoint{2.218140in}{3.123143in}}%
\pgfpathlineto{\pgfqpoint{2.222259in}{3.123143in}}%
\pgfpathlineto{\pgfqpoint{2.226377in}{3.151256in}}%
\pgfpathlineto{\pgfqpoint{2.230495in}{3.151256in}}%
\pgfpathlineto{\pgfqpoint{2.234613in}{3.179368in}}%
\pgfpathlineto{\pgfqpoint{2.238731in}{3.151256in}}%
\pgfpathlineto{\pgfqpoint{2.242849in}{3.151256in}}%
\pgfpathlineto{\pgfqpoint{2.246967in}{3.165312in}}%
\pgfpathlineto{\pgfqpoint{2.255203in}{3.165312in}}%
\pgfpathlineto{\pgfqpoint{2.259321in}{3.151256in}}%
\pgfpathlineto{\pgfqpoint{2.263439in}{3.165312in}}%
\pgfpathlineto{\pgfqpoint{2.267557in}{3.151256in}}%
\pgfpathlineto{\pgfqpoint{2.275794in}{3.151256in}}%
\pgfpathlineto{\pgfqpoint{2.279912in}{3.179368in}}%
\pgfpathlineto{\pgfqpoint{2.284030in}{3.151256in}}%
\pgfpathlineto{\pgfqpoint{2.288148in}{3.137200in}}%
\pgfpathlineto{\pgfqpoint{2.292266in}{3.165312in}}%
\pgfpathlineto{\pgfqpoint{2.296384in}{3.151256in}}%
\pgfpathlineto{\pgfqpoint{2.300502in}{3.151256in}}%
\pgfpathlineto{\pgfqpoint{2.316975in}{3.095031in}}%
\pgfpathlineto{\pgfqpoint{2.321093in}{3.095031in}}%
\pgfpathlineto{\pgfqpoint{2.325211in}{3.080975in}}%
\pgfpathlineto{\pgfqpoint{2.329329in}{3.109087in}}%
\pgfpathlineto{\pgfqpoint{2.333447in}{3.066918in}}%
\pgfpathlineto{\pgfqpoint{2.337565in}{3.066918in}}%
\pgfpathlineto{\pgfqpoint{2.341683in}{3.080975in}}%
\pgfpathlineto{\pgfqpoint{2.345801in}{3.066918in}}%
\pgfpathlineto{\pgfqpoint{2.358155in}{3.066918in}}%
\pgfpathlineto{\pgfqpoint{2.362273in}{3.080975in}}%
\pgfpathlineto{\pgfqpoint{2.366392in}{3.109087in}}%
\pgfpathlineto{\pgfqpoint{2.370510in}{3.109087in}}%
\pgfpathlineto{\pgfqpoint{2.374628in}{3.151256in}}%
\pgfpathlineto{\pgfqpoint{2.378746in}{3.137200in}}%
\pgfpathlineto{\pgfqpoint{2.382864in}{3.151256in}}%
\pgfpathlineto{\pgfqpoint{2.386982in}{3.123143in}}%
\pgfpathlineto{\pgfqpoint{2.391100in}{3.123143in}}%
\pgfpathlineto{\pgfqpoint{2.395218in}{3.165312in}}%
\pgfpathlineto{\pgfqpoint{2.399336in}{3.165312in}}%
\pgfpathlineto{\pgfqpoint{2.403454in}{3.179368in}}%
\pgfpathlineto{\pgfqpoint{2.411690in}{3.151256in}}%
\pgfpathlineto{\pgfqpoint{2.419927in}{3.151256in}}%
\pgfpathlineto{\pgfqpoint{2.424045in}{3.137200in}}%
\pgfpathlineto{\pgfqpoint{2.428163in}{3.151256in}}%
\pgfpathlineto{\pgfqpoint{2.444635in}{3.151256in}}%
\pgfpathlineto{\pgfqpoint{2.448753in}{3.137200in}}%
\pgfpathlineto{\pgfqpoint{2.452871in}{3.137200in}}%
\pgfpathlineto{\pgfqpoint{2.456989in}{3.151256in}}%
\pgfpathlineto{\pgfqpoint{2.461107in}{3.151256in}}%
\pgfpathlineto{\pgfqpoint{2.465226in}{3.137200in}}%
\pgfpathlineto{\pgfqpoint{2.469344in}{3.151256in}}%
\pgfpathlineto{\pgfqpoint{2.473462in}{3.151256in}}%
\pgfpathlineto{\pgfqpoint{2.477580in}{3.137200in}}%
\pgfpathlineto{\pgfqpoint{2.481698in}{3.137200in}}%
\pgfpathlineto{\pgfqpoint{2.485816in}{3.109087in}}%
\pgfpathlineto{\pgfqpoint{2.489934in}{3.109087in}}%
\pgfpathlineto{\pgfqpoint{2.502288in}{3.151256in}}%
\pgfpathlineto{\pgfqpoint{2.506406in}{3.137200in}}%
\pgfpathlineto{\pgfqpoint{2.514643in}{3.137200in}}%
\pgfpathlineto{\pgfqpoint{2.518761in}{3.165312in}}%
\pgfpathlineto{\pgfqpoint{2.522879in}{3.165312in}}%
\pgfpathlineto{\pgfqpoint{2.526997in}{3.151256in}}%
\pgfpathlineto{\pgfqpoint{2.531115in}{3.165312in}}%
\pgfpathlineto{\pgfqpoint{2.535233in}{3.151256in}}%
\pgfpathlineto{\pgfqpoint{2.547587in}{3.151256in}}%
\pgfpathlineto{\pgfqpoint{2.551705in}{3.179368in}}%
\pgfpathlineto{\pgfqpoint{2.555823in}{3.179368in}}%
\pgfpathlineto{\pgfqpoint{2.559942in}{3.151256in}}%
\pgfpathlineto{\pgfqpoint{2.568178in}{3.151256in}}%
\pgfpathlineto{\pgfqpoint{2.572296in}{3.165312in}}%
\pgfpathlineto{\pgfqpoint{2.576414in}{3.151256in}}%
\pgfpathlineto{\pgfqpoint{2.580532in}{3.165312in}}%
\pgfpathlineto{\pgfqpoint{2.584650in}{3.151256in}}%
\pgfpathlineto{\pgfqpoint{2.588768in}{3.165312in}}%
\pgfpathlineto{\pgfqpoint{2.592886in}{3.137200in}}%
\pgfpathlineto{\pgfqpoint{2.597004in}{3.165312in}}%
\pgfpathlineto{\pgfqpoint{2.601122in}{3.151256in}}%
\pgfpathlineto{\pgfqpoint{2.609359in}{3.151256in}}%
\pgfpathlineto{\pgfqpoint{2.613477in}{3.165312in}}%
\pgfpathlineto{\pgfqpoint{2.629949in}{3.165312in}}%
\pgfpathlineto{\pgfqpoint{2.634067in}{3.151256in}}%
\pgfpathlineto{\pgfqpoint{2.638185in}{3.151256in}}%
\pgfpathlineto{\pgfqpoint{2.642303in}{3.179368in}}%
\pgfpathlineto{\pgfqpoint{2.646421in}{3.165312in}}%
\pgfpathlineto{\pgfqpoint{2.646421in}{3.165312in}}%
\pgfusepath{stroke}%
\end{pgfscope}%
\begin{pgfscope}%
\pgfpathrectangle{\pgfqpoint{0.488751in}{2.165212in}}{\pgfqpoint{2.260417in}{1.283333in}}%
\pgfusepath{clip}%
\pgfsetrectcap%
\pgfsetroundjoin%
\pgfsetlinewidth{0.803000pt}%
\definecolor{currentstroke}{rgb}{0.333333,0.333333,0.333333}%
\pgfsetstrokecolor{currentstroke}%
\pgfsetstrokeopacity{0.270000}%
\pgfsetdash{}{0pt}%
\pgfpathmoveto{\pgfqpoint{0.591497in}{2.813906in}}%
\pgfpathlineto{\pgfqpoint{0.603851in}{2.813906in}}%
\pgfpathlineto{\pgfqpoint{0.607969in}{2.799850in}}%
\pgfpathlineto{\pgfqpoint{0.612087in}{2.799850in}}%
\pgfpathlineto{\pgfqpoint{0.616206in}{2.785794in}}%
\pgfpathlineto{\pgfqpoint{0.661504in}{2.785794in}}%
\pgfpathlineto{\pgfqpoint{0.665623in}{2.799850in}}%
\pgfpathlineto{\pgfqpoint{0.677977in}{2.799850in}}%
\pgfpathlineto{\pgfqpoint{0.682095in}{2.813906in}}%
\pgfpathlineto{\pgfqpoint{0.694449in}{2.813906in}}%
\pgfpathlineto{\pgfqpoint{0.698567in}{2.842019in}}%
\pgfpathlineto{\pgfqpoint{0.706803in}{2.842019in}}%
\pgfpathlineto{\pgfqpoint{0.710922in}{2.856075in}}%
\pgfpathlineto{\pgfqpoint{0.719158in}{2.827963in}}%
\pgfpathlineto{\pgfqpoint{0.723276in}{2.827963in}}%
\pgfpathlineto{\pgfqpoint{0.727394in}{2.842019in}}%
\pgfpathlineto{\pgfqpoint{0.731512in}{2.827963in}}%
\pgfpathlineto{\pgfqpoint{0.735630in}{2.827963in}}%
\pgfpathlineto{\pgfqpoint{0.739748in}{2.842019in}}%
\pgfpathlineto{\pgfqpoint{0.743866in}{2.842019in}}%
\pgfpathlineto{\pgfqpoint{0.747984in}{2.827963in}}%
\pgfpathlineto{\pgfqpoint{0.752102in}{2.827963in}}%
\pgfpathlineto{\pgfqpoint{0.756220in}{2.842019in}}%
\pgfpathlineto{\pgfqpoint{0.760339in}{2.842019in}}%
\pgfpathlineto{\pgfqpoint{0.764457in}{2.856075in}}%
\pgfpathlineto{\pgfqpoint{0.768575in}{2.884188in}}%
\pgfpathlineto{\pgfqpoint{0.776811in}{2.912300in}}%
\pgfpathlineto{\pgfqpoint{0.780929in}{2.940412in}}%
\pgfpathlineto{\pgfqpoint{0.785047in}{2.954469in}}%
\pgfpathlineto{\pgfqpoint{0.805637in}{2.954469in}}%
\pgfpathlineto{\pgfqpoint{0.809756in}{2.940412in}}%
\pgfpathlineto{\pgfqpoint{0.822110in}{2.940412in}}%
\pgfpathlineto{\pgfqpoint{0.826228in}{2.926356in}}%
\pgfpathlineto{\pgfqpoint{0.834464in}{2.926356in}}%
\pgfpathlineto{\pgfqpoint{0.838582in}{2.912300in}}%
\pgfpathlineto{\pgfqpoint{0.842700in}{2.926356in}}%
\pgfpathlineto{\pgfqpoint{0.850936in}{2.898244in}}%
\pgfpathlineto{\pgfqpoint{0.863291in}{2.898244in}}%
\pgfpathlineto{\pgfqpoint{0.867409in}{2.912300in}}%
\pgfpathlineto{\pgfqpoint{0.871527in}{2.898244in}}%
\pgfpathlineto{\pgfqpoint{0.875645in}{2.898244in}}%
\pgfpathlineto{\pgfqpoint{0.879763in}{2.884188in}}%
\pgfpathlineto{\pgfqpoint{0.883881in}{2.884188in}}%
\pgfpathlineto{\pgfqpoint{0.892117in}{2.856075in}}%
\pgfpathlineto{\pgfqpoint{0.896235in}{2.856075in}}%
\pgfpathlineto{\pgfqpoint{0.904471in}{2.884188in}}%
\pgfpathlineto{\pgfqpoint{0.908590in}{2.884188in}}%
\pgfpathlineto{\pgfqpoint{0.912708in}{2.898244in}}%
\pgfpathlineto{\pgfqpoint{0.920944in}{2.898244in}}%
\pgfpathlineto{\pgfqpoint{0.925062in}{2.912300in}}%
\pgfpathlineto{\pgfqpoint{0.929180in}{2.912300in}}%
\pgfpathlineto{\pgfqpoint{0.933298in}{2.898244in}}%
\pgfpathlineto{\pgfqpoint{0.941534in}{2.898244in}}%
\pgfpathlineto{\pgfqpoint{0.945652in}{2.926356in}}%
\pgfpathlineto{\pgfqpoint{0.949770in}{2.926356in}}%
\pgfpathlineto{\pgfqpoint{0.958007in}{2.898244in}}%
\pgfpathlineto{\pgfqpoint{0.966243in}{2.898244in}}%
\pgfpathlineto{\pgfqpoint{0.970361in}{2.912300in}}%
\pgfpathlineto{\pgfqpoint{0.974479in}{2.912300in}}%
\pgfpathlineto{\pgfqpoint{0.978597in}{2.926356in}}%
\pgfpathlineto{\pgfqpoint{0.982715in}{2.926356in}}%
\pgfpathlineto{\pgfqpoint{0.986833in}{2.912300in}}%
\pgfpathlineto{\pgfqpoint{0.990951in}{2.912300in}}%
\pgfpathlineto{\pgfqpoint{0.995069in}{2.926356in}}%
\pgfpathlineto{\pgfqpoint{0.999187in}{2.926356in}}%
\pgfpathlineto{\pgfqpoint{1.003306in}{2.940412in}}%
\pgfpathlineto{\pgfqpoint{1.007424in}{2.940412in}}%
\pgfpathlineto{\pgfqpoint{1.011542in}{2.912300in}}%
\pgfpathlineto{\pgfqpoint{1.015660in}{2.898244in}}%
\pgfpathlineto{\pgfqpoint{1.023896in}{2.898244in}}%
\pgfpathlineto{\pgfqpoint{1.028014in}{2.912300in}}%
\pgfpathlineto{\pgfqpoint{1.036250in}{2.912300in}}%
\pgfpathlineto{\pgfqpoint{1.040368in}{2.940412in}}%
\pgfpathlineto{\pgfqpoint{1.044486in}{2.954469in}}%
\pgfpathlineto{\pgfqpoint{1.048604in}{2.954469in}}%
\pgfpathlineto{\pgfqpoint{1.052723in}{2.982581in}}%
\pgfpathlineto{\pgfqpoint{1.060959in}{2.982581in}}%
\pgfpathlineto{\pgfqpoint{1.065077in}{2.968525in}}%
\pgfpathlineto{\pgfqpoint{1.069195in}{2.968525in}}%
\pgfpathlineto{\pgfqpoint{1.073313in}{2.954469in}}%
\pgfpathlineto{\pgfqpoint{1.085667in}{2.954469in}}%
\pgfpathlineto{\pgfqpoint{1.089785in}{2.982581in}}%
\pgfpathlineto{\pgfqpoint{1.093903in}{2.968525in}}%
\pgfpathlineto{\pgfqpoint{1.102140in}{2.968525in}}%
\pgfpathlineto{\pgfqpoint{1.106258in}{2.996637in}}%
\pgfpathlineto{\pgfqpoint{1.110376in}{2.996637in}}%
\pgfpathlineto{\pgfqpoint{1.114494in}{2.968525in}}%
\pgfpathlineto{\pgfqpoint{1.118612in}{2.968525in}}%
\pgfpathlineto{\pgfqpoint{1.122730in}{2.982581in}}%
\pgfpathlineto{\pgfqpoint{1.126848in}{2.982581in}}%
\pgfpathlineto{\pgfqpoint{1.130966in}{2.968525in}}%
\pgfpathlineto{\pgfqpoint{1.135084in}{2.968525in}}%
\pgfpathlineto{\pgfqpoint{1.139202in}{2.954469in}}%
\pgfpathlineto{\pgfqpoint{1.143320in}{2.954469in}}%
\pgfpathlineto{\pgfqpoint{1.147438in}{2.926356in}}%
\pgfpathlineto{\pgfqpoint{1.151557in}{2.912300in}}%
\pgfpathlineto{\pgfqpoint{1.155675in}{2.926356in}}%
\pgfpathlineto{\pgfqpoint{1.159793in}{2.926356in}}%
\pgfpathlineto{\pgfqpoint{1.163911in}{2.940412in}}%
\pgfpathlineto{\pgfqpoint{1.168029in}{2.940412in}}%
\pgfpathlineto{\pgfqpoint{1.172147in}{2.954469in}}%
\pgfpathlineto{\pgfqpoint{1.176265in}{2.996637in}}%
\pgfpathlineto{\pgfqpoint{1.180383in}{2.982581in}}%
\pgfpathlineto{\pgfqpoint{1.184501in}{2.982581in}}%
\pgfpathlineto{\pgfqpoint{1.188619in}{2.968525in}}%
\pgfpathlineto{\pgfqpoint{1.196855in}{2.968525in}}%
\pgfpathlineto{\pgfqpoint{1.200974in}{2.954469in}}%
\pgfpathlineto{\pgfqpoint{1.233918in}{2.954469in}}%
\pgfpathlineto{\pgfqpoint{1.246273in}{2.996637in}}%
\pgfpathlineto{\pgfqpoint{1.250391in}{2.996637in}}%
\pgfpathlineto{\pgfqpoint{1.262745in}{2.954469in}}%
\pgfpathlineto{\pgfqpoint{1.266863in}{2.954469in}}%
\pgfpathlineto{\pgfqpoint{1.270981in}{2.940412in}}%
\pgfpathlineto{\pgfqpoint{1.275099in}{2.884188in}}%
\pgfpathlineto{\pgfqpoint{1.287453in}{2.884188in}}%
\pgfpathlineto{\pgfqpoint{1.291571in}{2.898244in}}%
\pgfpathlineto{\pgfqpoint{1.295690in}{2.898244in}}%
\pgfpathlineto{\pgfqpoint{1.299808in}{2.884188in}}%
\pgfpathlineto{\pgfqpoint{1.308044in}{2.884188in}}%
\pgfpathlineto{\pgfqpoint{1.312162in}{2.870131in}}%
\pgfpathlineto{\pgfqpoint{1.316280in}{2.842019in}}%
\pgfpathlineto{\pgfqpoint{1.320398in}{2.827963in}}%
\pgfpathlineto{\pgfqpoint{1.324516in}{2.827963in}}%
\pgfpathlineto{\pgfqpoint{1.328634in}{2.785794in}}%
\pgfpathlineto{\pgfqpoint{1.336870in}{2.785794in}}%
\pgfpathlineto{\pgfqpoint{1.340988in}{2.771738in}}%
\pgfpathlineto{\pgfqpoint{1.386287in}{2.771738in}}%
\pgfpathlineto{\pgfqpoint{1.390405in}{2.757682in}}%
\pgfpathlineto{\pgfqpoint{1.439822in}{2.757682in}}%
\pgfpathlineto{\pgfqpoint{1.443941in}{2.771738in}}%
\pgfpathlineto{\pgfqpoint{1.448059in}{2.771738in}}%
\pgfpathlineto{\pgfqpoint{1.452177in}{2.757682in}}%
\pgfpathlineto{\pgfqpoint{1.472767in}{2.757682in}}%
\pgfpathlineto{\pgfqpoint{1.476885in}{2.771738in}}%
\pgfpathlineto{\pgfqpoint{1.481003in}{2.771738in}}%
\pgfpathlineto{\pgfqpoint{1.489240in}{2.799850in}}%
\pgfpathlineto{\pgfqpoint{1.497476in}{2.799850in}}%
\pgfpathlineto{\pgfqpoint{1.509830in}{2.842019in}}%
\pgfpathlineto{\pgfqpoint{1.522184in}{2.842019in}}%
\pgfpathlineto{\pgfqpoint{1.526302in}{2.870131in}}%
\pgfpathlineto{\pgfqpoint{1.530420in}{2.870131in}}%
\pgfpathlineto{\pgfqpoint{1.534538in}{2.898244in}}%
\pgfpathlineto{\pgfqpoint{1.538657in}{2.884188in}}%
\pgfpathlineto{\pgfqpoint{1.559247in}{2.884188in}}%
\pgfpathlineto{\pgfqpoint{1.563365in}{2.898244in}}%
\pgfpathlineto{\pgfqpoint{1.567483in}{2.856075in}}%
\pgfpathlineto{\pgfqpoint{1.575719in}{2.884188in}}%
\pgfpathlineto{\pgfqpoint{1.579837in}{2.884188in}}%
\pgfpathlineto{\pgfqpoint{1.583955in}{2.856075in}}%
\pgfpathlineto{\pgfqpoint{1.588074in}{2.856075in}}%
\pgfpathlineto{\pgfqpoint{1.592192in}{2.842019in}}%
\pgfpathlineto{\pgfqpoint{1.596310in}{2.842019in}}%
\pgfpathlineto{\pgfqpoint{1.600428in}{2.813906in}}%
\pgfpathlineto{\pgfqpoint{1.604546in}{2.799850in}}%
\pgfpathlineto{\pgfqpoint{1.608664in}{2.771738in}}%
\pgfpathlineto{\pgfqpoint{1.625136in}{2.771738in}}%
\pgfpathlineto{\pgfqpoint{1.629254in}{2.785794in}}%
\pgfpathlineto{\pgfqpoint{1.633372in}{2.785794in}}%
\pgfpathlineto{\pgfqpoint{1.637491in}{2.757682in}}%
\pgfpathlineto{\pgfqpoint{1.686908in}{2.757682in}}%
\pgfpathlineto{\pgfqpoint{1.691026in}{2.771738in}}%
\pgfpathlineto{\pgfqpoint{1.719852in}{2.771738in}}%
\pgfpathlineto{\pgfqpoint{1.723970in}{2.757682in}}%
\pgfpathlineto{\pgfqpoint{1.756915in}{2.757682in}}%
\pgfpathlineto{\pgfqpoint{1.761033in}{2.743625in}}%
\pgfpathlineto{\pgfqpoint{1.781624in}{2.743625in}}%
\pgfpathlineto{\pgfqpoint{1.785742in}{2.757682in}}%
\pgfpathlineto{\pgfqpoint{1.802214in}{2.757682in}}%
\pgfpathlineto{\pgfqpoint{1.806332in}{2.771738in}}%
\pgfpathlineto{\pgfqpoint{1.876339in}{2.771738in}}%
\pgfpathlineto{\pgfqpoint{1.880458in}{2.785794in}}%
\pgfpathlineto{\pgfqpoint{1.921638in}{2.785794in}}%
\pgfpathlineto{\pgfqpoint{1.925756in}{2.799850in}}%
\pgfpathlineto{\pgfqpoint{1.938111in}{2.799850in}}%
\pgfpathlineto{\pgfqpoint{1.942229in}{2.785794in}}%
\pgfpathlineto{\pgfqpoint{1.979292in}{2.785794in}}%
\pgfpathlineto{\pgfqpoint{1.983410in}{2.771738in}}%
\pgfpathlineto{\pgfqpoint{2.123425in}{2.771738in}}%
\pgfpathlineto{\pgfqpoint{2.127543in}{2.757682in}}%
\pgfpathlineto{\pgfqpoint{2.176960in}{2.757682in}}%
\pgfpathlineto{\pgfqpoint{2.181078in}{2.743625in}}%
\pgfpathlineto{\pgfqpoint{2.201668in}{2.743625in}}%
\pgfpathlineto{\pgfqpoint{2.205786in}{2.729569in}}%
\pgfpathlineto{\pgfqpoint{2.308738in}{2.729569in}}%
\pgfpathlineto{\pgfqpoint{2.312856in}{2.715513in}}%
\pgfpathlineto{\pgfqpoint{2.646421in}{2.715513in}}%
\pgfpathlineto{\pgfqpoint{2.646421in}{2.715513in}}%
\pgfusepath{stroke}%
\end{pgfscope}%
\begin{pgfscope}%
\pgfpathrectangle{\pgfqpoint{0.488751in}{2.165212in}}{\pgfqpoint{2.260417in}{1.283333in}}%
\pgfusepath{clip}%
\pgfsetrectcap%
\pgfsetroundjoin%
\pgfsetlinewidth{0.803000pt}%
\definecolor{currentstroke}{rgb}{0.686275,0.352941,0.313725}%
\pgfsetstrokecolor{currentstroke}%
\pgfsetstrokeopacity{0.270000}%
\pgfsetdash{}{0pt}%
\pgfpathmoveto{\pgfqpoint{0.591497in}{2.954469in}}%
\pgfpathlineto{\pgfqpoint{0.595615in}{2.996637in}}%
\pgfpathlineto{\pgfqpoint{0.599733in}{3.024750in}}%
\pgfpathlineto{\pgfqpoint{0.603851in}{3.038806in}}%
\pgfpathlineto{\pgfqpoint{0.620324in}{2.982581in}}%
\pgfpathlineto{\pgfqpoint{0.624442in}{2.996637in}}%
\pgfpathlineto{\pgfqpoint{0.628560in}{2.940412in}}%
\pgfpathlineto{\pgfqpoint{0.632678in}{2.940412in}}%
\pgfpathlineto{\pgfqpoint{0.636796in}{2.926356in}}%
\pgfpathlineto{\pgfqpoint{0.640914in}{2.898244in}}%
\pgfpathlineto{\pgfqpoint{0.649150in}{2.926356in}}%
\pgfpathlineto{\pgfqpoint{0.653268in}{2.884188in}}%
\pgfpathlineto{\pgfqpoint{0.657386in}{2.884188in}}%
\pgfpathlineto{\pgfqpoint{0.661504in}{2.870131in}}%
\pgfpathlineto{\pgfqpoint{0.665623in}{2.842019in}}%
\pgfpathlineto{\pgfqpoint{0.682095in}{2.842019in}}%
\pgfpathlineto{\pgfqpoint{0.686213in}{2.813906in}}%
\pgfpathlineto{\pgfqpoint{0.702685in}{2.813906in}}%
\pgfpathlineto{\pgfqpoint{0.706803in}{2.827963in}}%
\pgfpathlineto{\pgfqpoint{0.727394in}{2.827963in}}%
\pgfpathlineto{\pgfqpoint{0.731512in}{2.842019in}}%
\pgfpathlineto{\pgfqpoint{0.739748in}{2.842019in}}%
\pgfpathlineto{\pgfqpoint{0.743866in}{2.870131in}}%
\pgfpathlineto{\pgfqpoint{0.747984in}{2.870131in}}%
\pgfpathlineto{\pgfqpoint{0.752102in}{2.842019in}}%
\pgfpathlineto{\pgfqpoint{0.789165in}{2.842019in}}%
\pgfpathlineto{\pgfqpoint{0.793283in}{2.827963in}}%
\pgfpathlineto{\pgfqpoint{0.805637in}{2.827963in}}%
\pgfpathlineto{\pgfqpoint{0.809756in}{2.842019in}}%
\pgfpathlineto{\pgfqpoint{0.826228in}{2.842019in}}%
\pgfpathlineto{\pgfqpoint{0.830346in}{2.856075in}}%
\pgfpathlineto{\pgfqpoint{0.834464in}{2.842019in}}%
\pgfpathlineto{\pgfqpoint{0.863291in}{2.842019in}}%
\pgfpathlineto{\pgfqpoint{0.867409in}{2.827963in}}%
\pgfpathlineto{\pgfqpoint{0.871527in}{2.856075in}}%
\pgfpathlineto{\pgfqpoint{0.875645in}{2.856075in}}%
\pgfpathlineto{\pgfqpoint{0.879763in}{2.870131in}}%
\pgfpathlineto{\pgfqpoint{0.883881in}{2.856075in}}%
\pgfpathlineto{\pgfqpoint{0.887999in}{2.856075in}}%
\pgfpathlineto{\pgfqpoint{0.896235in}{2.884188in}}%
\pgfpathlineto{\pgfqpoint{0.904471in}{2.884188in}}%
\pgfpathlineto{\pgfqpoint{0.908590in}{2.898244in}}%
\pgfpathlineto{\pgfqpoint{0.912708in}{2.898244in}}%
\pgfpathlineto{\pgfqpoint{0.916826in}{2.884188in}}%
\pgfpathlineto{\pgfqpoint{0.920944in}{2.884188in}}%
\pgfpathlineto{\pgfqpoint{0.925062in}{2.870131in}}%
\pgfpathlineto{\pgfqpoint{0.929180in}{2.870131in}}%
\pgfpathlineto{\pgfqpoint{0.933298in}{2.884188in}}%
\pgfpathlineto{\pgfqpoint{0.962125in}{2.884188in}}%
\pgfpathlineto{\pgfqpoint{0.970361in}{2.856075in}}%
\pgfpathlineto{\pgfqpoint{0.974479in}{2.856075in}}%
\pgfpathlineto{\pgfqpoint{0.978597in}{2.870131in}}%
\pgfpathlineto{\pgfqpoint{1.007424in}{2.870131in}}%
\pgfpathlineto{\pgfqpoint{1.015660in}{2.898244in}}%
\pgfpathlineto{\pgfqpoint{1.028014in}{2.898244in}}%
\pgfpathlineto{\pgfqpoint{1.032132in}{2.912300in}}%
\pgfpathlineto{\pgfqpoint{1.036250in}{2.912300in}}%
\pgfpathlineto{\pgfqpoint{1.040368in}{2.926356in}}%
\pgfpathlineto{\pgfqpoint{1.044486in}{2.898244in}}%
\pgfpathlineto{\pgfqpoint{1.056841in}{2.898244in}}%
\pgfpathlineto{\pgfqpoint{1.060959in}{2.912300in}}%
\pgfpathlineto{\pgfqpoint{1.065077in}{2.912300in}}%
\pgfpathlineto{\pgfqpoint{1.069195in}{2.898244in}}%
\pgfpathlineto{\pgfqpoint{1.081549in}{2.898244in}}%
\pgfpathlineto{\pgfqpoint{1.085667in}{2.912300in}}%
\pgfpathlineto{\pgfqpoint{1.093903in}{2.884188in}}%
\pgfpathlineto{\pgfqpoint{1.118612in}{2.884188in}}%
\pgfpathlineto{\pgfqpoint{1.122730in}{2.898244in}}%
\pgfpathlineto{\pgfqpoint{1.126848in}{2.926356in}}%
\pgfpathlineto{\pgfqpoint{1.135084in}{2.926356in}}%
\pgfpathlineto{\pgfqpoint{1.139202in}{2.940412in}}%
\pgfpathlineto{\pgfqpoint{1.143320in}{2.926356in}}%
\pgfpathlineto{\pgfqpoint{1.147438in}{2.926356in}}%
\pgfpathlineto{\pgfqpoint{1.151557in}{2.912300in}}%
\pgfpathlineto{\pgfqpoint{1.172147in}{2.912300in}}%
\pgfpathlineto{\pgfqpoint{1.176265in}{2.898244in}}%
\pgfpathlineto{\pgfqpoint{1.180383in}{2.898244in}}%
\pgfpathlineto{\pgfqpoint{1.184501in}{2.926356in}}%
\pgfpathlineto{\pgfqpoint{1.188619in}{2.940412in}}%
\pgfpathlineto{\pgfqpoint{1.229800in}{2.940412in}}%
\pgfpathlineto{\pgfqpoint{1.233918in}{2.912300in}}%
\pgfpathlineto{\pgfqpoint{1.242154in}{2.912300in}}%
\pgfpathlineto{\pgfqpoint{1.250391in}{2.884188in}}%
\pgfpathlineto{\pgfqpoint{1.258627in}{2.884188in}}%
\pgfpathlineto{\pgfqpoint{1.262745in}{2.898244in}}%
\pgfpathlineto{\pgfqpoint{1.266863in}{2.884188in}}%
\pgfpathlineto{\pgfqpoint{1.287453in}{2.884188in}}%
\pgfpathlineto{\pgfqpoint{1.291571in}{2.870131in}}%
\pgfpathlineto{\pgfqpoint{1.308044in}{2.870131in}}%
\pgfpathlineto{\pgfqpoint{1.312162in}{2.884188in}}%
\pgfpathlineto{\pgfqpoint{1.316280in}{2.870131in}}%
\pgfpathlineto{\pgfqpoint{1.336870in}{2.870131in}}%
\pgfpathlineto{\pgfqpoint{1.340988in}{2.884188in}}%
\pgfpathlineto{\pgfqpoint{1.357461in}{2.884188in}}%
\pgfpathlineto{\pgfqpoint{1.361579in}{2.870131in}}%
\pgfpathlineto{\pgfqpoint{1.365697in}{2.884188in}}%
\pgfpathlineto{\pgfqpoint{1.369815in}{2.884188in}}%
\pgfpathlineto{\pgfqpoint{1.373933in}{2.870131in}}%
\pgfpathlineto{\pgfqpoint{1.382169in}{2.870131in}}%
\pgfpathlineto{\pgfqpoint{1.390405in}{2.898244in}}%
\pgfpathlineto{\pgfqpoint{1.435704in}{2.898244in}}%
\pgfpathlineto{\pgfqpoint{1.439822in}{2.912300in}}%
\pgfpathlineto{\pgfqpoint{1.443941in}{2.898244in}}%
\pgfpathlineto{\pgfqpoint{1.448059in}{2.898244in}}%
\pgfpathlineto{\pgfqpoint{1.452177in}{2.870131in}}%
\pgfpathlineto{\pgfqpoint{1.460413in}{2.870131in}}%
\pgfpathlineto{\pgfqpoint{1.464531in}{2.856075in}}%
\pgfpathlineto{\pgfqpoint{1.468649in}{2.870131in}}%
\pgfpathlineto{\pgfqpoint{1.489240in}{2.870131in}}%
\pgfpathlineto{\pgfqpoint{1.493358in}{2.856075in}}%
\pgfpathlineto{\pgfqpoint{1.522184in}{2.856075in}}%
\pgfpathlineto{\pgfqpoint{1.526302in}{2.842019in}}%
\pgfpathlineto{\pgfqpoint{1.546893in}{2.842019in}}%
\pgfpathlineto{\pgfqpoint{1.551011in}{2.856075in}}%
\pgfpathlineto{\pgfqpoint{1.563365in}{2.856075in}}%
\pgfpathlineto{\pgfqpoint{1.567483in}{2.912300in}}%
\pgfpathlineto{\pgfqpoint{1.571601in}{2.898244in}}%
\pgfpathlineto{\pgfqpoint{1.583955in}{2.898244in}}%
\pgfpathlineto{\pgfqpoint{1.588074in}{2.884188in}}%
\pgfpathlineto{\pgfqpoint{1.592192in}{2.884188in}}%
\pgfpathlineto{\pgfqpoint{1.596310in}{2.870131in}}%
\pgfpathlineto{\pgfqpoint{1.604546in}{2.870131in}}%
\pgfpathlineto{\pgfqpoint{1.608664in}{2.884188in}}%
\pgfpathlineto{\pgfqpoint{1.616900in}{2.884188in}}%
\pgfpathlineto{\pgfqpoint{1.621018in}{2.856075in}}%
\pgfpathlineto{\pgfqpoint{1.625136in}{2.884188in}}%
\pgfpathlineto{\pgfqpoint{1.633372in}{2.884188in}}%
\pgfpathlineto{\pgfqpoint{1.637491in}{2.856075in}}%
\pgfpathlineto{\pgfqpoint{1.658081in}{2.856075in}}%
\pgfpathlineto{\pgfqpoint{1.662199in}{2.842019in}}%
\pgfpathlineto{\pgfqpoint{1.678671in}{2.842019in}}%
\pgfpathlineto{\pgfqpoint{1.682789in}{2.856075in}}%
\pgfpathlineto{\pgfqpoint{1.728088in}{2.856075in}}%
\pgfpathlineto{\pgfqpoint{1.732206in}{2.870131in}}%
\pgfpathlineto{\pgfqpoint{1.736325in}{2.856075in}}%
\pgfpathlineto{\pgfqpoint{1.781624in}{2.856075in}}%
\pgfpathlineto{\pgfqpoint{1.785742in}{2.870131in}}%
\pgfpathlineto{\pgfqpoint{1.789860in}{2.870131in}}%
\pgfpathlineto{\pgfqpoint{1.793978in}{2.856075in}}%
\pgfpathlineto{\pgfqpoint{1.851631in}{2.856075in}}%
\pgfpathlineto{\pgfqpoint{1.855749in}{2.870131in}}%
\pgfpathlineto{\pgfqpoint{1.892812in}{2.870131in}}%
\pgfpathlineto{\pgfqpoint{1.896930in}{2.884188in}}%
\pgfpathlineto{\pgfqpoint{1.921638in}{2.884188in}}%
\pgfpathlineto{\pgfqpoint{1.925756in}{2.870131in}}%
\pgfpathlineto{\pgfqpoint{1.938111in}{2.870131in}}%
\pgfpathlineto{\pgfqpoint{1.942229in}{2.884188in}}%
\pgfpathlineto{\pgfqpoint{1.950465in}{2.856075in}}%
\pgfpathlineto{\pgfqpoint{1.966937in}{2.856075in}}%
\pgfpathlineto{\pgfqpoint{1.971055in}{2.842019in}}%
\pgfpathlineto{\pgfqpoint{1.975173in}{2.856075in}}%
\pgfpathlineto{\pgfqpoint{1.979292in}{2.856075in}}%
\pgfpathlineto{\pgfqpoint{1.983410in}{2.842019in}}%
\pgfpathlineto{\pgfqpoint{2.024591in}{2.842019in}}%
\pgfpathlineto{\pgfqpoint{2.028709in}{2.856075in}}%
\pgfpathlineto{\pgfqpoint{2.032827in}{2.856075in}}%
\pgfpathlineto{\pgfqpoint{2.036945in}{2.870131in}}%
\pgfpathlineto{\pgfqpoint{2.041063in}{2.870131in}}%
\pgfpathlineto{\pgfqpoint{2.045181in}{2.898244in}}%
\pgfpathlineto{\pgfqpoint{2.061653in}{2.898244in}}%
\pgfpathlineto{\pgfqpoint{2.065771in}{2.912300in}}%
\pgfpathlineto{\pgfqpoint{2.069889in}{2.940412in}}%
\pgfpathlineto{\pgfqpoint{2.082244in}{2.898244in}}%
\pgfpathlineto{\pgfqpoint{2.098716in}{2.898244in}}%
\pgfpathlineto{\pgfqpoint{2.102834in}{2.870131in}}%
\pgfpathlineto{\pgfqpoint{2.119306in}{2.870131in}}%
\pgfpathlineto{\pgfqpoint{2.123425in}{2.856075in}}%
\pgfpathlineto{\pgfqpoint{2.127543in}{2.856075in}}%
\pgfpathlineto{\pgfqpoint{2.131661in}{2.842019in}}%
\pgfpathlineto{\pgfqpoint{2.135779in}{2.842019in}}%
\pgfpathlineto{\pgfqpoint{2.139897in}{2.856075in}}%
\pgfpathlineto{\pgfqpoint{2.160487in}{2.856075in}}%
\pgfpathlineto{\pgfqpoint{2.164605in}{2.884188in}}%
\pgfpathlineto{\pgfqpoint{2.176960in}{2.884188in}}%
\pgfpathlineto{\pgfqpoint{2.181078in}{2.926356in}}%
\pgfpathlineto{\pgfqpoint{2.185196in}{2.940412in}}%
\pgfpathlineto{\pgfqpoint{2.189314in}{2.912300in}}%
\pgfpathlineto{\pgfqpoint{2.197550in}{2.940412in}}%
\pgfpathlineto{\pgfqpoint{2.201668in}{2.968525in}}%
\pgfpathlineto{\pgfqpoint{2.205786in}{2.926356in}}%
\pgfpathlineto{\pgfqpoint{2.218140in}{2.926356in}}%
\pgfpathlineto{\pgfqpoint{2.222259in}{2.912300in}}%
\pgfpathlineto{\pgfqpoint{2.226377in}{2.926356in}}%
\pgfpathlineto{\pgfqpoint{2.230495in}{2.912300in}}%
\pgfpathlineto{\pgfqpoint{2.238731in}{2.912300in}}%
\pgfpathlineto{\pgfqpoint{2.242849in}{2.898244in}}%
\pgfpathlineto{\pgfqpoint{2.246967in}{2.898244in}}%
\pgfpathlineto{\pgfqpoint{2.251085in}{2.884188in}}%
\pgfpathlineto{\pgfqpoint{2.259321in}{2.884188in}}%
\pgfpathlineto{\pgfqpoint{2.263439in}{2.898244in}}%
\pgfpathlineto{\pgfqpoint{2.267557in}{2.898244in}}%
\pgfpathlineto{\pgfqpoint{2.271676in}{2.884188in}}%
\pgfpathlineto{\pgfqpoint{2.275794in}{2.898244in}}%
\pgfpathlineto{\pgfqpoint{2.279912in}{2.926356in}}%
\pgfpathlineto{\pgfqpoint{2.284030in}{2.926356in}}%
\pgfpathlineto{\pgfqpoint{2.288148in}{2.912300in}}%
\pgfpathlineto{\pgfqpoint{2.292266in}{2.926356in}}%
\pgfpathlineto{\pgfqpoint{2.296384in}{2.926356in}}%
\pgfpathlineto{\pgfqpoint{2.300502in}{2.898244in}}%
\pgfpathlineto{\pgfqpoint{2.304620in}{2.912300in}}%
\pgfpathlineto{\pgfqpoint{2.308738in}{2.898244in}}%
\pgfpathlineto{\pgfqpoint{2.312856in}{2.912300in}}%
\pgfpathlineto{\pgfqpoint{2.321093in}{2.912300in}}%
\pgfpathlineto{\pgfqpoint{2.325211in}{2.898244in}}%
\pgfpathlineto{\pgfqpoint{2.329329in}{2.912300in}}%
\pgfpathlineto{\pgfqpoint{2.337565in}{2.884188in}}%
\pgfpathlineto{\pgfqpoint{2.341683in}{2.884188in}}%
\pgfpathlineto{\pgfqpoint{2.345801in}{2.912300in}}%
\pgfpathlineto{\pgfqpoint{2.349919in}{2.898244in}}%
\pgfpathlineto{\pgfqpoint{2.362273in}{2.898244in}}%
\pgfpathlineto{\pgfqpoint{2.370510in}{2.926356in}}%
\pgfpathlineto{\pgfqpoint{2.374628in}{2.954469in}}%
\pgfpathlineto{\pgfqpoint{2.378746in}{2.940412in}}%
\pgfpathlineto{\pgfqpoint{2.382864in}{2.954469in}}%
\pgfpathlineto{\pgfqpoint{2.386982in}{2.982581in}}%
\pgfpathlineto{\pgfqpoint{2.391100in}{2.982581in}}%
\pgfpathlineto{\pgfqpoint{2.399336in}{3.010694in}}%
\pgfpathlineto{\pgfqpoint{2.403454in}{3.010694in}}%
\pgfpathlineto{\pgfqpoint{2.407572in}{3.024750in}}%
\pgfpathlineto{\pgfqpoint{2.415809in}{2.996637in}}%
\pgfpathlineto{\pgfqpoint{2.419927in}{2.996637in}}%
\pgfpathlineto{\pgfqpoint{2.424045in}{2.982581in}}%
\pgfpathlineto{\pgfqpoint{2.428163in}{3.010694in}}%
\pgfpathlineto{\pgfqpoint{2.432281in}{3.010694in}}%
\pgfpathlineto{\pgfqpoint{2.440517in}{2.982581in}}%
\pgfpathlineto{\pgfqpoint{2.444635in}{2.996637in}}%
\pgfpathlineto{\pgfqpoint{2.452871in}{2.996637in}}%
\pgfpathlineto{\pgfqpoint{2.456989in}{2.982581in}}%
\pgfpathlineto{\pgfqpoint{2.473462in}{2.982581in}}%
\pgfpathlineto{\pgfqpoint{2.477580in}{2.968525in}}%
\pgfpathlineto{\pgfqpoint{2.481698in}{2.982581in}}%
\pgfpathlineto{\pgfqpoint{2.489934in}{2.982581in}}%
\pgfpathlineto{\pgfqpoint{2.498170in}{2.954469in}}%
\pgfpathlineto{\pgfqpoint{2.506406in}{2.954469in}}%
\pgfpathlineto{\pgfqpoint{2.510524in}{2.968525in}}%
\pgfpathlineto{\pgfqpoint{2.526997in}{2.968525in}}%
\pgfpathlineto{\pgfqpoint{2.531115in}{3.010694in}}%
\pgfpathlineto{\pgfqpoint{2.535233in}{3.010694in}}%
\pgfpathlineto{\pgfqpoint{2.539351in}{3.024750in}}%
\pgfpathlineto{\pgfqpoint{2.564060in}{3.024750in}}%
\pgfpathlineto{\pgfqpoint{2.568178in}{3.010694in}}%
\pgfpathlineto{\pgfqpoint{2.572296in}{3.010694in}}%
\pgfpathlineto{\pgfqpoint{2.576414in}{2.996637in}}%
\pgfpathlineto{\pgfqpoint{2.580532in}{3.010694in}}%
\pgfpathlineto{\pgfqpoint{2.588768in}{3.010694in}}%
\pgfpathlineto{\pgfqpoint{2.592886in}{2.982581in}}%
\pgfpathlineto{\pgfqpoint{2.597004in}{2.982581in}}%
\pgfpathlineto{\pgfqpoint{2.601122in}{2.968525in}}%
\pgfpathlineto{\pgfqpoint{2.605240in}{2.982581in}}%
\pgfpathlineto{\pgfqpoint{2.613477in}{2.982581in}}%
\pgfpathlineto{\pgfqpoint{2.617595in}{2.996637in}}%
\pgfpathlineto{\pgfqpoint{2.621713in}{2.982581in}}%
\pgfpathlineto{\pgfqpoint{2.629949in}{2.982581in}}%
\pgfpathlineto{\pgfqpoint{2.634067in}{2.968525in}}%
\pgfpathlineto{\pgfqpoint{2.646421in}{2.968525in}}%
\pgfpathlineto{\pgfqpoint{2.646421in}{2.968525in}}%
\pgfusepath{stroke}%
\end{pgfscope}%
\begin{pgfscope}%
\pgfpathrectangle{\pgfqpoint{0.488751in}{2.165212in}}{\pgfqpoint{2.260417in}{1.283333in}}%
\pgfusepath{clip}%
\pgfsetrectcap%
\pgfsetroundjoin%
\pgfsetlinewidth{0.803000pt}%
\definecolor{currentstroke}{rgb}{0.000000,0.356863,0.509804}%
\pgfsetstrokecolor{currentstroke}%
\pgfsetstrokeopacity{0.270000}%
\pgfsetdash{}{0pt}%
\pgfpathmoveto{\pgfqpoint{0.591497in}{2.476557in}}%
\pgfpathlineto{\pgfqpoint{0.595615in}{2.504669in}}%
\pgfpathlineto{\pgfqpoint{0.599733in}{2.546838in}}%
\pgfpathlineto{\pgfqpoint{0.603851in}{2.546838in}}%
\pgfpathlineto{\pgfqpoint{0.612087in}{2.631176in}}%
\pgfpathlineto{\pgfqpoint{0.616206in}{2.701457in}}%
\pgfpathlineto{\pgfqpoint{0.624442in}{2.729569in}}%
\pgfpathlineto{\pgfqpoint{0.628560in}{2.771738in}}%
\pgfpathlineto{\pgfqpoint{0.632678in}{2.785794in}}%
\pgfpathlineto{\pgfqpoint{0.636796in}{2.785794in}}%
\pgfpathlineto{\pgfqpoint{0.649150in}{2.940412in}}%
\pgfpathlineto{\pgfqpoint{0.653268in}{2.926356in}}%
\pgfpathlineto{\pgfqpoint{0.657386in}{2.940412in}}%
\pgfpathlineto{\pgfqpoint{0.661504in}{2.996637in}}%
\pgfpathlineto{\pgfqpoint{0.665623in}{3.010694in}}%
\pgfpathlineto{\pgfqpoint{0.669741in}{3.010694in}}%
\pgfpathlineto{\pgfqpoint{0.673859in}{3.024750in}}%
\pgfpathlineto{\pgfqpoint{0.677977in}{3.024750in}}%
\pgfpathlineto{\pgfqpoint{0.682095in}{3.038806in}}%
\pgfpathlineto{\pgfqpoint{0.686213in}{3.066918in}}%
\pgfpathlineto{\pgfqpoint{0.690331in}{3.052862in}}%
\pgfpathlineto{\pgfqpoint{0.706803in}{3.109087in}}%
\pgfpathlineto{\pgfqpoint{0.710922in}{3.109087in}}%
\pgfpathlineto{\pgfqpoint{0.715040in}{3.123143in}}%
\pgfpathlineto{\pgfqpoint{0.739748in}{3.123143in}}%
\pgfpathlineto{\pgfqpoint{0.743866in}{3.151256in}}%
\pgfpathlineto{\pgfqpoint{0.747984in}{3.165312in}}%
\pgfpathlineto{\pgfqpoint{0.752102in}{3.151256in}}%
\pgfpathlineto{\pgfqpoint{0.756220in}{3.165312in}}%
\pgfpathlineto{\pgfqpoint{0.760339in}{3.151256in}}%
\pgfpathlineto{\pgfqpoint{0.776811in}{3.151256in}}%
\pgfpathlineto{\pgfqpoint{0.780929in}{3.137200in}}%
\pgfpathlineto{\pgfqpoint{0.801519in}{3.137200in}}%
\pgfpathlineto{\pgfqpoint{0.805637in}{3.123143in}}%
\pgfpathlineto{\pgfqpoint{0.809756in}{3.151256in}}%
\pgfpathlineto{\pgfqpoint{0.822110in}{3.151256in}}%
\pgfpathlineto{\pgfqpoint{0.830346in}{3.123143in}}%
\pgfpathlineto{\pgfqpoint{0.834464in}{3.123143in}}%
\pgfpathlineto{\pgfqpoint{0.838582in}{3.137200in}}%
\pgfpathlineto{\pgfqpoint{0.842700in}{3.123143in}}%
\pgfpathlineto{\pgfqpoint{0.846818in}{3.137200in}}%
\pgfpathlineto{\pgfqpoint{0.850936in}{3.179368in}}%
\pgfpathlineto{\pgfqpoint{0.855054in}{3.151256in}}%
\pgfpathlineto{\pgfqpoint{0.867409in}{3.151256in}}%
\pgfpathlineto{\pgfqpoint{0.871527in}{3.179368in}}%
\pgfpathlineto{\pgfqpoint{0.879763in}{3.179368in}}%
\pgfpathlineto{\pgfqpoint{0.883881in}{3.151256in}}%
\pgfpathlineto{\pgfqpoint{0.887999in}{3.151256in}}%
\pgfpathlineto{\pgfqpoint{0.892117in}{3.165312in}}%
\pgfpathlineto{\pgfqpoint{0.896235in}{3.151256in}}%
\pgfpathlineto{\pgfqpoint{0.900353in}{3.165312in}}%
\pgfpathlineto{\pgfqpoint{0.908590in}{3.165312in}}%
\pgfpathlineto{\pgfqpoint{0.912708in}{3.137200in}}%
\pgfpathlineto{\pgfqpoint{0.925062in}{3.137200in}}%
\pgfpathlineto{\pgfqpoint{0.933298in}{3.165312in}}%
\pgfpathlineto{\pgfqpoint{0.937416in}{3.151256in}}%
\pgfpathlineto{\pgfqpoint{0.941534in}{3.165312in}}%
\pgfpathlineto{\pgfqpoint{0.945652in}{3.151256in}}%
\pgfpathlineto{\pgfqpoint{0.949770in}{3.151256in}}%
\pgfpathlineto{\pgfqpoint{0.953888in}{3.137200in}}%
\pgfpathlineto{\pgfqpoint{0.958007in}{3.151256in}}%
\pgfpathlineto{\pgfqpoint{0.962125in}{3.151256in}}%
\pgfpathlineto{\pgfqpoint{0.966243in}{3.137200in}}%
\pgfpathlineto{\pgfqpoint{0.982715in}{3.137200in}}%
\pgfpathlineto{\pgfqpoint{0.986833in}{3.151256in}}%
\pgfpathlineto{\pgfqpoint{0.990951in}{3.151256in}}%
\pgfpathlineto{\pgfqpoint{0.995069in}{3.165312in}}%
\pgfpathlineto{\pgfqpoint{0.999187in}{3.165312in}}%
\pgfpathlineto{\pgfqpoint{1.003306in}{3.151256in}}%
\pgfpathlineto{\pgfqpoint{1.007424in}{3.151256in}}%
\pgfpathlineto{\pgfqpoint{1.015660in}{3.179368in}}%
\pgfpathlineto{\pgfqpoint{1.019778in}{3.165312in}}%
\pgfpathlineto{\pgfqpoint{1.023896in}{3.165312in}}%
\pgfpathlineto{\pgfqpoint{1.028014in}{3.151256in}}%
\pgfpathlineto{\pgfqpoint{1.040368in}{3.151256in}}%
\pgfpathlineto{\pgfqpoint{1.044486in}{3.137200in}}%
\pgfpathlineto{\pgfqpoint{1.052723in}{3.137200in}}%
\pgfpathlineto{\pgfqpoint{1.056841in}{3.151256in}}%
\pgfpathlineto{\pgfqpoint{1.060959in}{3.151256in}}%
\pgfpathlineto{\pgfqpoint{1.065077in}{3.165312in}}%
\pgfpathlineto{\pgfqpoint{1.069195in}{3.151256in}}%
\pgfpathlineto{\pgfqpoint{1.077431in}{3.151256in}}%
\pgfpathlineto{\pgfqpoint{1.081549in}{3.137200in}}%
\pgfpathlineto{\pgfqpoint{1.085667in}{3.151256in}}%
\pgfpathlineto{\pgfqpoint{1.093903in}{3.151256in}}%
\pgfpathlineto{\pgfqpoint{1.098021in}{3.165312in}}%
\pgfpathlineto{\pgfqpoint{1.106258in}{3.165312in}}%
\pgfpathlineto{\pgfqpoint{1.110376in}{3.151256in}}%
\pgfpathlineto{\pgfqpoint{1.114494in}{3.165312in}}%
\pgfpathlineto{\pgfqpoint{1.122730in}{3.165312in}}%
\pgfpathlineto{\pgfqpoint{1.126848in}{3.137200in}}%
\pgfpathlineto{\pgfqpoint{1.130966in}{3.151256in}}%
\pgfpathlineto{\pgfqpoint{1.135084in}{3.151256in}}%
\pgfpathlineto{\pgfqpoint{1.139202in}{3.221537in}}%
\pgfpathlineto{\pgfqpoint{1.143320in}{3.165312in}}%
\pgfpathlineto{\pgfqpoint{1.151557in}{3.193425in}}%
\pgfpathlineto{\pgfqpoint{1.159793in}{3.137200in}}%
\pgfpathlineto{\pgfqpoint{1.163911in}{3.137200in}}%
\pgfpathlineto{\pgfqpoint{1.168029in}{3.151256in}}%
\pgfpathlineto{\pgfqpoint{1.172147in}{3.179368in}}%
\pgfpathlineto{\pgfqpoint{1.176265in}{3.151256in}}%
\pgfpathlineto{\pgfqpoint{1.180383in}{3.151256in}}%
\pgfpathlineto{\pgfqpoint{1.184501in}{3.165312in}}%
\pgfpathlineto{\pgfqpoint{1.192737in}{3.165312in}}%
\pgfpathlineto{\pgfqpoint{1.200974in}{3.137200in}}%
\pgfpathlineto{\pgfqpoint{1.225682in}{3.137200in}}%
\pgfpathlineto{\pgfqpoint{1.229800in}{3.151256in}}%
\pgfpathlineto{\pgfqpoint{1.238036in}{3.151256in}}%
\pgfpathlineto{\pgfqpoint{1.242154in}{3.137200in}}%
\pgfpathlineto{\pgfqpoint{1.246273in}{3.137200in}}%
\pgfpathlineto{\pgfqpoint{1.250391in}{3.151256in}}%
\pgfpathlineto{\pgfqpoint{1.254509in}{3.151256in}}%
\pgfpathlineto{\pgfqpoint{1.262745in}{3.179368in}}%
\pgfpathlineto{\pgfqpoint{1.266863in}{3.165312in}}%
\pgfpathlineto{\pgfqpoint{1.275099in}{3.221537in}}%
\pgfpathlineto{\pgfqpoint{1.279217in}{3.193425in}}%
\pgfpathlineto{\pgfqpoint{1.283335in}{3.193425in}}%
\pgfpathlineto{\pgfqpoint{1.287453in}{3.165312in}}%
\pgfpathlineto{\pgfqpoint{1.291571in}{3.179368in}}%
\pgfpathlineto{\pgfqpoint{1.303926in}{3.137200in}}%
\pgfpathlineto{\pgfqpoint{1.312162in}{3.165312in}}%
\pgfpathlineto{\pgfqpoint{1.316280in}{3.193425in}}%
\pgfpathlineto{\pgfqpoint{1.320398in}{3.165312in}}%
\pgfpathlineto{\pgfqpoint{1.324516in}{3.151256in}}%
\pgfpathlineto{\pgfqpoint{1.328634in}{3.151256in}}%
\pgfpathlineto{\pgfqpoint{1.332752in}{3.165312in}}%
\pgfpathlineto{\pgfqpoint{1.336870in}{3.151256in}}%
\pgfpathlineto{\pgfqpoint{1.340988in}{3.179368in}}%
\pgfpathlineto{\pgfqpoint{1.349225in}{3.151256in}}%
\pgfpathlineto{\pgfqpoint{1.353343in}{3.165312in}}%
\pgfpathlineto{\pgfqpoint{1.365697in}{3.165312in}}%
\pgfpathlineto{\pgfqpoint{1.369815in}{3.151256in}}%
\pgfpathlineto{\pgfqpoint{1.382169in}{3.151256in}}%
\pgfpathlineto{\pgfqpoint{1.386287in}{3.165312in}}%
\pgfpathlineto{\pgfqpoint{1.394524in}{3.221537in}}%
\pgfpathlineto{\pgfqpoint{1.398642in}{3.221537in}}%
\pgfpathlineto{\pgfqpoint{1.402760in}{3.207481in}}%
\pgfpathlineto{\pgfqpoint{1.406878in}{3.179368in}}%
\pgfpathlineto{\pgfqpoint{1.410996in}{3.165312in}}%
\pgfpathlineto{\pgfqpoint{1.427468in}{3.165312in}}%
\pgfpathlineto{\pgfqpoint{1.431586in}{3.151256in}}%
\pgfpathlineto{\pgfqpoint{1.435704in}{3.151256in}}%
\pgfpathlineto{\pgfqpoint{1.443941in}{3.179368in}}%
\pgfpathlineto{\pgfqpoint{1.448059in}{3.151256in}}%
\pgfpathlineto{\pgfqpoint{1.452177in}{3.179368in}}%
\pgfpathlineto{\pgfqpoint{1.456295in}{3.151256in}}%
\pgfpathlineto{\pgfqpoint{1.464531in}{3.151256in}}%
\pgfpathlineto{\pgfqpoint{1.468649in}{3.165312in}}%
\pgfpathlineto{\pgfqpoint{1.472767in}{3.137200in}}%
\pgfpathlineto{\pgfqpoint{1.493358in}{3.137200in}}%
\pgfpathlineto{\pgfqpoint{1.497476in}{3.151256in}}%
\pgfpathlineto{\pgfqpoint{1.505712in}{3.151256in}}%
\pgfpathlineto{\pgfqpoint{1.509830in}{3.137200in}}%
\pgfpathlineto{\pgfqpoint{1.518066in}{3.137200in}}%
\pgfpathlineto{\pgfqpoint{1.522184in}{3.151256in}}%
\pgfpathlineto{\pgfqpoint{1.530420in}{3.123143in}}%
\pgfpathlineto{\pgfqpoint{1.534538in}{3.123143in}}%
\pgfpathlineto{\pgfqpoint{1.538657in}{3.137200in}}%
\pgfpathlineto{\pgfqpoint{1.546893in}{3.137200in}}%
\pgfpathlineto{\pgfqpoint{1.551011in}{3.151256in}}%
\pgfpathlineto{\pgfqpoint{1.563365in}{3.151256in}}%
\pgfpathlineto{\pgfqpoint{1.567483in}{3.207481in}}%
\pgfpathlineto{\pgfqpoint{1.571601in}{3.179368in}}%
\pgfpathlineto{\pgfqpoint{1.579837in}{3.151256in}}%
\pgfpathlineto{\pgfqpoint{1.583955in}{3.165312in}}%
\pgfpathlineto{\pgfqpoint{1.588074in}{3.151256in}}%
\pgfpathlineto{\pgfqpoint{1.592192in}{3.179368in}}%
\pgfpathlineto{\pgfqpoint{1.596310in}{3.151256in}}%
\pgfpathlineto{\pgfqpoint{1.600428in}{3.179368in}}%
\pgfpathlineto{\pgfqpoint{1.604546in}{3.151256in}}%
\pgfpathlineto{\pgfqpoint{1.608664in}{3.193425in}}%
\pgfpathlineto{\pgfqpoint{1.616900in}{3.165312in}}%
\pgfpathlineto{\pgfqpoint{1.621018in}{3.165312in}}%
\pgfpathlineto{\pgfqpoint{1.625136in}{3.193425in}}%
\pgfpathlineto{\pgfqpoint{1.629254in}{3.151256in}}%
\pgfpathlineto{\pgfqpoint{1.633372in}{3.151256in}}%
\pgfpathlineto{\pgfqpoint{1.637491in}{3.137200in}}%
\pgfpathlineto{\pgfqpoint{1.641609in}{3.137200in}}%
\pgfpathlineto{\pgfqpoint{1.645727in}{3.123143in}}%
\pgfpathlineto{\pgfqpoint{1.649845in}{3.165312in}}%
\pgfpathlineto{\pgfqpoint{1.653963in}{3.137200in}}%
\pgfpathlineto{\pgfqpoint{1.658081in}{3.137200in}}%
\pgfpathlineto{\pgfqpoint{1.662199in}{3.165312in}}%
\pgfpathlineto{\pgfqpoint{1.666317in}{3.151256in}}%
\pgfpathlineto{\pgfqpoint{1.674553in}{3.179368in}}%
\pgfpathlineto{\pgfqpoint{1.682789in}{3.179368in}}%
\pgfpathlineto{\pgfqpoint{1.686908in}{3.151256in}}%
\pgfpathlineto{\pgfqpoint{1.695144in}{3.151256in}}%
\pgfpathlineto{\pgfqpoint{1.699262in}{3.137200in}}%
\pgfpathlineto{\pgfqpoint{1.707498in}{3.137200in}}%
\pgfpathlineto{\pgfqpoint{1.711616in}{3.151256in}}%
\pgfpathlineto{\pgfqpoint{1.732206in}{3.151256in}}%
\pgfpathlineto{\pgfqpoint{1.736325in}{3.137200in}}%
\pgfpathlineto{\pgfqpoint{1.740443in}{3.137200in}}%
\pgfpathlineto{\pgfqpoint{1.744561in}{3.151256in}}%
\pgfpathlineto{\pgfqpoint{1.748679in}{3.137200in}}%
\pgfpathlineto{\pgfqpoint{1.756915in}{3.137200in}}%
\pgfpathlineto{\pgfqpoint{1.761033in}{3.151256in}}%
\pgfpathlineto{\pgfqpoint{1.765151in}{3.179368in}}%
\pgfpathlineto{\pgfqpoint{1.769269in}{3.179368in}}%
\pgfpathlineto{\pgfqpoint{1.773387in}{3.151256in}}%
\pgfpathlineto{\pgfqpoint{1.777505in}{3.151256in}}%
\pgfpathlineto{\pgfqpoint{1.781624in}{3.165312in}}%
\pgfpathlineto{\pgfqpoint{1.785742in}{3.151256in}}%
\pgfpathlineto{\pgfqpoint{1.789860in}{3.151256in}}%
\pgfpathlineto{\pgfqpoint{1.798096in}{3.123143in}}%
\pgfpathlineto{\pgfqpoint{1.810450in}{3.123143in}}%
\pgfpathlineto{\pgfqpoint{1.814568in}{3.151256in}}%
\pgfpathlineto{\pgfqpoint{1.818686in}{3.165312in}}%
\pgfpathlineto{\pgfqpoint{1.822804in}{3.151256in}}%
\pgfpathlineto{\pgfqpoint{1.826922in}{3.165312in}}%
\pgfpathlineto{\pgfqpoint{1.831041in}{3.151256in}}%
\pgfpathlineto{\pgfqpoint{1.863985in}{3.151256in}}%
\pgfpathlineto{\pgfqpoint{1.872221in}{3.179368in}}%
\pgfpathlineto{\pgfqpoint{1.876339in}{3.179368in}}%
\pgfpathlineto{\pgfqpoint{1.880458in}{3.165312in}}%
\pgfpathlineto{\pgfqpoint{1.884576in}{3.137200in}}%
\pgfpathlineto{\pgfqpoint{1.888694in}{3.165312in}}%
\pgfpathlineto{\pgfqpoint{1.892812in}{3.151256in}}%
\pgfpathlineto{\pgfqpoint{1.896930in}{3.151256in}}%
\pgfpathlineto{\pgfqpoint{1.901048in}{3.179368in}}%
\pgfpathlineto{\pgfqpoint{1.909284in}{3.151256in}}%
\pgfpathlineto{\pgfqpoint{1.917520in}{3.151256in}}%
\pgfpathlineto{\pgfqpoint{1.921638in}{3.165312in}}%
\pgfpathlineto{\pgfqpoint{1.925756in}{3.137200in}}%
\pgfpathlineto{\pgfqpoint{1.929875in}{3.151256in}}%
\pgfpathlineto{\pgfqpoint{1.933993in}{3.151256in}}%
\pgfpathlineto{\pgfqpoint{1.938111in}{3.193425in}}%
\pgfpathlineto{\pgfqpoint{1.942229in}{3.193425in}}%
\pgfpathlineto{\pgfqpoint{1.950465in}{3.137200in}}%
\pgfpathlineto{\pgfqpoint{1.954583in}{3.137200in}}%
\pgfpathlineto{\pgfqpoint{1.958701in}{3.151256in}}%
\pgfpathlineto{\pgfqpoint{1.962819in}{3.137200in}}%
\pgfpathlineto{\pgfqpoint{1.971055in}{3.137200in}}%
\pgfpathlineto{\pgfqpoint{1.975173in}{3.151256in}}%
\pgfpathlineto{\pgfqpoint{1.983410in}{3.151256in}}%
\pgfpathlineto{\pgfqpoint{1.987528in}{3.165312in}}%
\pgfpathlineto{\pgfqpoint{1.991646in}{3.151256in}}%
\pgfpathlineto{\pgfqpoint{2.045181in}{3.151256in}}%
\pgfpathlineto{\pgfqpoint{2.053417in}{3.179368in}}%
\pgfpathlineto{\pgfqpoint{2.061653in}{3.151256in}}%
\pgfpathlineto{\pgfqpoint{2.069889in}{3.179368in}}%
\pgfpathlineto{\pgfqpoint{2.074008in}{3.165312in}}%
\pgfpathlineto{\pgfqpoint{2.086362in}{3.165312in}}%
\pgfpathlineto{\pgfqpoint{2.090480in}{3.137200in}}%
\pgfpathlineto{\pgfqpoint{2.098716in}{3.193425in}}%
\pgfpathlineto{\pgfqpoint{2.106952in}{3.165312in}}%
\pgfpathlineto{\pgfqpoint{2.115188in}{3.165312in}}%
\pgfpathlineto{\pgfqpoint{2.119306in}{3.151256in}}%
\pgfpathlineto{\pgfqpoint{2.123425in}{3.123143in}}%
\pgfpathlineto{\pgfqpoint{2.127543in}{3.137200in}}%
\pgfpathlineto{\pgfqpoint{2.131661in}{3.137200in}}%
\pgfpathlineto{\pgfqpoint{2.139897in}{3.109087in}}%
\pgfpathlineto{\pgfqpoint{2.144015in}{3.123143in}}%
\pgfpathlineto{\pgfqpoint{2.156369in}{3.123143in}}%
\pgfpathlineto{\pgfqpoint{2.160487in}{3.109087in}}%
\pgfpathlineto{\pgfqpoint{2.164605in}{3.137200in}}%
\pgfpathlineto{\pgfqpoint{2.168723in}{3.151256in}}%
\pgfpathlineto{\pgfqpoint{2.172842in}{3.151256in}}%
\pgfpathlineto{\pgfqpoint{2.176960in}{3.137200in}}%
\pgfpathlineto{\pgfqpoint{2.185196in}{3.165312in}}%
\pgfpathlineto{\pgfqpoint{2.189314in}{3.151256in}}%
\pgfpathlineto{\pgfqpoint{2.197550in}{3.151256in}}%
\pgfpathlineto{\pgfqpoint{2.201668in}{3.165312in}}%
\pgfpathlineto{\pgfqpoint{2.205786in}{3.151256in}}%
\pgfpathlineto{\pgfqpoint{2.279912in}{3.151256in}}%
\pgfpathlineto{\pgfqpoint{2.284030in}{3.137200in}}%
\pgfpathlineto{\pgfqpoint{2.288148in}{3.151256in}}%
\pgfpathlineto{\pgfqpoint{2.296384in}{3.151256in}}%
\pgfpathlineto{\pgfqpoint{2.300502in}{3.165312in}}%
\pgfpathlineto{\pgfqpoint{2.304620in}{3.123143in}}%
\pgfpathlineto{\pgfqpoint{2.308738in}{3.123143in}}%
\pgfpathlineto{\pgfqpoint{2.316975in}{3.151256in}}%
\pgfpathlineto{\pgfqpoint{2.378746in}{3.151256in}}%
\pgfpathlineto{\pgfqpoint{2.382864in}{3.165312in}}%
\pgfpathlineto{\pgfqpoint{2.386982in}{3.151256in}}%
\pgfpathlineto{\pgfqpoint{2.506406in}{3.151256in}}%
\pgfpathlineto{\pgfqpoint{2.510524in}{3.165312in}}%
\pgfpathlineto{\pgfqpoint{2.514643in}{3.151256in}}%
\pgfpathlineto{\pgfqpoint{2.522879in}{3.151256in}}%
\pgfpathlineto{\pgfqpoint{2.526997in}{3.137200in}}%
\pgfpathlineto{\pgfqpoint{2.559942in}{3.137200in}}%
\pgfpathlineto{\pgfqpoint{2.564060in}{3.151256in}}%
\pgfpathlineto{\pgfqpoint{2.588768in}{3.151256in}}%
\pgfpathlineto{\pgfqpoint{2.592886in}{3.137200in}}%
\pgfpathlineto{\pgfqpoint{2.597004in}{3.137200in}}%
\pgfpathlineto{\pgfqpoint{2.601122in}{3.123143in}}%
\pgfpathlineto{\pgfqpoint{2.625831in}{3.123143in}}%
\pgfpathlineto{\pgfqpoint{2.629949in}{3.137200in}}%
\pgfpathlineto{\pgfqpoint{2.642303in}{3.137200in}}%
\pgfpathlineto{\pgfqpoint{2.646421in}{3.151256in}}%
\pgfpathlineto{\pgfqpoint{2.646421in}{3.151256in}}%
\pgfusepath{stroke}%
\end{pgfscope}%
\begin{pgfscope}%
\pgfpathrectangle{\pgfqpoint{0.488751in}{2.165212in}}{\pgfqpoint{2.260417in}{1.283333in}}%
\pgfusepath{clip}%
\pgfsetrectcap%
\pgfsetroundjoin%
\pgfsetlinewidth{0.803000pt}%
\definecolor{currentstroke}{rgb}{0.490196,0.588235,0.431373}%
\pgfsetstrokecolor{currentstroke}%
\pgfsetstrokeopacity{0.270000}%
\pgfsetdash{}{0pt}%
\pgfpathmoveto{\pgfqpoint{0.591497in}{2.926356in}}%
\pgfpathlineto{\pgfqpoint{0.599733in}{2.898244in}}%
\pgfpathlineto{\pgfqpoint{0.603851in}{2.898244in}}%
\pgfpathlineto{\pgfqpoint{0.612087in}{2.813906in}}%
\pgfpathlineto{\pgfqpoint{0.616206in}{2.813906in}}%
\pgfpathlineto{\pgfqpoint{0.620324in}{2.799850in}}%
\pgfpathlineto{\pgfqpoint{0.624442in}{2.799850in}}%
\pgfpathlineto{\pgfqpoint{0.628560in}{2.771738in}}%
\pgfpathlineto{\pgfqpoint{0.632678in}{2.701457in}}%
\pgfpathlineto{\pgfqpoint{0.636796in}{2.701457in}}%
\pgfpathlineto{\pgfqpoint{0.640914in}{2.757682in}}%
\pgfpathlineto{\pgfqpoint{0.645032in}{2.715513in}}%
\pgfpathlineto{\pgfqpoint{0.649150in}{2.743625in}}%
\pgfpathlineto{\pgfqpoint{0.653268in}{2.757682in}}%
\pgfpathlineto{\pgfqpoint{0.657386in}{2.743625in}}%
\pgfpathlineto{\pgfqpoint{0.661504in}{2.715513in}}%
\pgfpathlineto{\pgfqpoint{0.665623in}{2.701457in}}%
\pgfpathlineto{\pgfqpoint{0.669741in}{2.729569in}}%
\pgfpathlineto{\pgfqpoint{0.673859in}{2.729569in}}%
\pgfpathlineto{\pgfqpoint{0.677977in}{2.757682in}}%
\pgfpathlineto{\pgfqpoint{0.694449in}{2.701457in}}%
\pgfpathlineto{\pgfqpoint{0.698567in}{2.701457in}}%
\pgfpathlineto{\pgfqpoint{0.702685in}{2.687400in}}%
\pgfpathlineto{\pgfqpoint{0.706803in}{2.687400in}}%
\pgfpathlineto{\pgfqpoint{0.710922in}{2.757682in}}%
\pgfpathlineto{\pgfqpoint{0.715040in}{2.771738in}}%
\pgfpathlineto{\pgfqpoint{0.719158in}{2.799850in}}%
\pgfpathlineto{\pgfqpoint{0.723276in}{2.856075in}}%
\pgfpathlineto{\pgfqpoint{0.727394in}{2.870131in}}%
\pgfpathlineto{\pgfqpoint{0.731512in}{2.856075in}}%
\pgfpathlineto{\pgfqpoint{0.739748in}{2.884188in}}%
\pgfpathlineto{\pgfqpoint{0.743866in}{2.884188in}}%
\pgfpathlineto{\pgfqpoint{0.747984in}{2.898244in}}%
\pgfpathlineto{\pgfqpoint{0.752102in}{2.884188in}}%
\pgfpathlineto{\pgfqpoint{0.772693in}{2.954469in}}%
\pgfpathlineto{\pgfqpoint{0.776811in}{2.982581in}}%
\pgfpathlineto{\pgfqpoint{0.780929in}{2.968525in}}%
\pgfpathlineto{\pgfqpoint{0.785047in}{2.982581in}}%
\pgfpathlineto{\pgfqpoint{0.789165in}{3.010694in}}%
\pgfpathlineto{\pgfqpoint{0.801519in}{3.052862in}}%
\pgfpathlineto{\pgfqpoint{0.805637in}{3.052862in}}%
\pgfpathlineto{\pgfqpoint{0.809756in}{3.010694in}}%
\pgfpathlineto{\pgfqpoint{0.817992in}{3.010694in}}%
\pgfpathlineto{\pgfqpoint{0.822110in}{3.024750in}}%
\pgfpathlineto{\pgfqpoint{0.826228in}{3.024750in}}%
\pgfpathlineto{\pgfqpoint{0.830346in}{3.010694in}}%
\pgfpathlineto{\pgfqpoint{0.834464in}{3.024750in}}%
\pgfpathlineto{\pgfqpoint{0.838582in}{3.010694in}}%
\pgfpathlineto{\pgfqpoint{0.842700in}{3.010694in}}%
\pgfpathlineto{\pgfqpoint{0.850936in}{2.982581in}}%
\pgfpathlineto{\pgfqpoint{0.855054in}{2.982581in}}%
\pgfpathlineto{\pgfqpoint{0.863291in}{3.010694in}}%
\pgfpathlineto{\pgfqpoint{0.892117in}{3.010694in}}%
\pgfpathlineto{\pgfqpoint{0.900353in}{2.982581in}}%
\pgfpathlineto{\pgfqpoint{0.920944in}{2.982581in}}%
\pgfpathlineto{\pgfqpoint{0.925062in}{3.010694in}}%
\pgfpathlineto{\pgfqpoint{0.929180in}{3.024750in}}%
\pgfpathlineto{\pgfqpoint{0.933298in}{3.010694in}}%
\pgfpathlineto{\pgfqpoint{0.937416in}{3.024750in}}%
\pgfpathlineto{\pgfqpoint{0.941534in}{2.996637in}}%
\pgfpathlineto{\pgfqpoint{0.945652in}{3.010694in}}%
\pgfpathlineto{\pgfqpoint{0.949770in}{3.010694in}}%
\pgfpathlineto{\pgfqpoint{0.958007in}{2.954469in}}%
\pgfpathlineto{\pgfqpoint{0.962125in}{2.968525in}}%
\pgfpathlineto{\pgfqpoint{0.978597in}{2.968525in}}%
\pgfpathlineto{\pgfqpoint{0.986833in}{2.996637in}}%
\pgfpathlineto{\pgfqpoint{0.990951in}{2.996637in}}%
\pgfpathlineto{\pgfqpoint{0.995069in}{2.982581in}}%
\pgfpathlineto{\pgfqpoint{1.003306in}{2.982581in}}%
\pgfpathlineto{\pgfqpoint{1.007424in}{2.968525in}}%
\pgfpathlineto{\pgfqpoint{1.015660in}{2.912300in}}%
\pgfpathlineto{\pgfqpoint{1.019778in}{2.912300in}}%
\pgfpathlineto{\pgfqpoint{1.023896in}{2.898244in}}%
\pgfpathlineto{\pgfqpoint{1.032132in}{2.898244in}}%
\pgfpathlineto{\pgfqpoint{1.036250in}{2.884188in}}%
\pgfpathlineto{\pgfqpoint{1.040368in}{2.884188in}}%
\pgfpathlineto{\pgfqpoint{1.044486in}{2.940412in}}%
\pgfpathlineto{\pgfqpoint{1.048604in}{2.926356in}}%
\pgfpathlineto{\pgfqpoint{1.056841in}{2.926356in}}%
\pgfpathlineto{\pgfqpoint{1.060959in}{2.940412in}}%
\pgfpathlineto{\pgfqpoint{1.069195in}{2.940412in}}%
\pgfpathlineto{\pgfqpoint{1.073313in}{2.954469in}}%
\pgfpathlineto{\pgfqpoint{1.085667in}{2.954469in}}%
\pgfpathlineto{\pgfqpoint{1.089785in}{2.968525in}}%
\pgfpathlineto{\pgfqpoint{1.093903in}{2.968525in}}%
\pgfpathlineto{\pgfqpoint{1.098021in}{2.954469in}}%
\pgfpathlineto{\pgfqpoint{1.102140in}{2.968525in}}%
\pgfpathlineto{\pgfqpoint{1.106258in}{2.968525in}}%
\pgfpathlineto{\pgfqpoint{1.110376in}{2.982581in}}%
\pgfpathlineto{\pgfqpoint{1.114494in}{2.940412in}}%
\pgfpathlineto{\pgfqpoint{1.126848in}{2.898244in}}%
\pgfpathlineto{\pgfqpoint{1.130966in}{2.926356in}}%
\pgfpathlineto{\pgfqpoint{1.135084in}{2.926356in}}%
\pgfpathlineto{\pgfqpoint{1.139202in}{2.898244in}}%
\pgfpathlineto{\pgfqpoint{1.143320in}{2.912300in}}%
\pgfpathlineto{\pgfqpoint{1.147438in}{2.884188in}}%
\pgfpathlineto{\pgfqpoint{1.151557in}{2.898244in}}%
\pgfpathlineto{\pgfqpoint{1.155675in}{2.870131in}}%
\pgfpathlineto{\pgfqpoint{1.180383in}{2.870131in}}%
\pgfpathlineto{\pgfqpoint{1.184501in}{2.884188in}}%
\pgfpathlineto{\pgfqpoint{1.188619in}{2.870131in}}%
\pgfpathlineto{\pgfqpoint{1.217446in}{2.870131in}}%
\pgfpathlineto{\pgfqpoint{1.221564in}{2.884188in}}%
\pgfpathlineto{\pgfqpoint{1.225682in}{2.884188in}}%
\pgfpathlineto{\pgfqpoint{1.229800in}{2.898244in}}%
\pgfpathlineto{\pgfqpoint{1.246273in}{2.898244in}}%
\pgfpathlineto{\pgfqpoint{1.250391in}{2.870131in}}%
\pgfpathlineto{\pgfqpoint{1.258627in}{2.870131in}}%
\pgfpathlineto{\pgfqpoint{1.262745in}{2.856075in}}%
\pgfpathlineto{\pgfqpoint{1.448059in}{2.856075in}}%
\pgfpathlineto{\pgfqpoint{1.452177in}{2.842019in}}%
\pgfpathlineto{\pgfqpoint{1.534538in}{2.842019in}}%
\pgfpathlineto{\pgfqpoint{1.538657in}{2.827963in}}%
\pgfpathlineto{\pgfqpoint{1.658081in}{2.827963in}}%
\pgfpathlineto{\pgfqpoint{1.662199in}{2.813906in}}%
\pgfpathlineto{\pgfqpoint{1.983410in}{2.813906in}}%
\pgfpathlineto{\pgfqpoint{1.987528in}{2.799850in}}%
\pgfpathlineto{\pgfqpoint{2.382864in}{2.799850in}}%
\pgfpathlineto{\pgfqpoint{2.386982in}{2.785794in}}%
\pgfpathlineto{\pgfqpoint{2.411690in}{2.785794in}}%
\pgfpathlineto{\pgfqpoint{2.415809in}{2.799850in}}%
\pgfpathlineto{\pgfqpoint{2.485816in}{2.799850in}}%
\pgfpathlineto{\pgfqpoint{2.489934in}{2.785794in}}%
\pgfpathlineto{\pgfqpoint{2.588768in}{2.785794in}}%
\pgfpathlineto{\pgfqpoint{2.592886in}{2.799850in}}%
\pgfpathlineto{\pgfqpoint{2.646421in}{2.799850in}}%
\pgfpathlineto{\pgfqpoint{2.646421in}{2.799850in}}%
\pgfusepath{stroke}%
\end{pgfscope}%
\begin{pgfscope}%
\pgfpathrectangle{\pgfqpoint{0.488751in}{2.165212in}}{\pgfqpoint{2.260417in}{1.283333in}}%
\pgfusepath{clip}%
\pgfsetrectcap%
\pgfsetroundjoin%
\pgfsetlinewidth{0.803000pt}%
\definecolor{currentstroke}{rgb}{0.843137,0.666667,0.313725}%
\pgfsetstrokecolor{currentstroke}%
\pgfsetstrokeopacity{0.270000}%
\pgfsetdash{}{0pt}%
\pgfpathmoveto{\pgfqpoint{0.591497in}{2.335995in}}%
\pgfpathlineto{\pgfqpoint{0.595615in}{2.293826in}}%
\pgfpathlineto{\pgfqpoint{0.599733in}{2.237601in}}%
\pgfpathlineto{\pgfqpoint{0.603851in}{2.223545in}}%
\pgfpathlineto{\pgfqpoint{0.607969in}{2.237601in}}%
\pgfpathlineto{\pgfqpoint{0.612087in}{2.279770in}}%
\pgfpathlineto{\pgfqpoint{0.616206in}{2.293826in}}%
\pgfpathlineto{\pgfqpoint{0.620324in}{2.293826in}}%
\pgfpathlineto{\pgfqpoint{0.628560in}{2.350051in}}%
\pgfpathlineto{\pgfqpoint{0.632678in}{2.321939in}}%
\pgfpathlineto{\pgfqpoint{0.636796in}{2.307882in}}%
\pgfpathlineto{\pgfqpoint{0.645032in}{2.364107in}}%
\pgfpathlineto{\pgfqpoint{0.649150in}{2.406276in}}%
\pgfpathlineto{\pgfqpoint{0.653268in}{2.462501in}}%
\pgfpathlineto{\pgfqpoint{0.657386in}{2.434388in}}%
\pgfpathlineto{\pgfqpoint{0.661504in}{2.462501in}}%
\pgfpathlineto{\pgfqpoint{0.669741in}{2.406276in}}%
\pgfpathlineto{\pgfqpoint{0.673859in}{2.406276in}}%
\pgfpathlineto{\pgfqpoint{0.677977in}{2.490613in}}%
\pgfpathlineto{\pgfqpoint{0.682095in}{2.518726in}}%
\pgfpathlineto{\pgfqpoint{0.686213in}{2.560894in}}%
\pgfpathlineto{\pgfqpoint{0.690331in}{2.560894in}}%
\pgfpathlineto{\pgfqpoint{0.694449in}{2.546838in}}%
\pgfpathlineto{\pgfqpoint{0.698567in}{2.574951in}}%
\pgfpathlineto{\pgfqpoint{0.702685in}{2.560894in}}%
\pgfpathlineto{\pgfqpoint{0.706803in}{2.603063in}}%
\pgfpathlineto{\pgfqpoint{0.710922in}{2.631176in}}%
\pgfpathlineto{\pgfqpoint{0.723276in}{2.589007in}}%
\pgfpathlineto{\pgfqpoint{0.727394in}{2.603063in}}%
\pgfpathlineto{\pgfqpoint{0.731512in}{2.589007in}}%
\pgfpathlineto{\pgfqpoint{0.739748in}{2.589007in}}%
\pgfpathlineto{\pgfqpoint{0.743866in}{2.574951in}}%
\pgfpathlineto{\pgfqpoint{0.747984in}{2.532782in}}%
\pgfpathlineto{\pgfqpoint{0.752102in}{2.518726in}}%
\pgfpathlineto{\pgfqpoint{0.756220in}{2.546838in}}%
\pgfpathlineto{\pgfqpoint{0.760339in}{2.532782in}}%
\pgfpathlineto{\pgfqpoint{0.764457in}{2.546838in}}%
\pgfpathlineto{\pgfqpoint{0.768575in}{2.546838in}}%
\pgfpathlineto{\pgfqpoint{0.772693in}{2.560894in}}%
\pgfpathlineto{\pgfqpoint{0.776811in}{2.603063in}}%
\pgfpathlineto{\pgfqpoint{0.780929in}{2.617119in}}%
\pgfpathlineto{\pgfqpoint{0.785047in}{2.645232in}}%
\pgfpathlineto{\pgfqpoint{0.789165in}{2.659288in}}%
\pgfpathlineto{\pgfqpoint{0.793283in}{2.659288in}}%
\pgfpathlineto{\pgfqpoint{0.797401in}{2.673344in}}%
\pgfpathlineto{\pgfqpoint{0.801519in}{2.701457in}}%
\pgfpathlineto{\pgfqpoint{0.805637in}{2.715513in}}%
\pgfpathlineto{\pgfqpoint{0.809756in}{2.687400in}}%
\pgfpathlineto{\pgfqpoint{0.813874in}{2.687400in}}%
\pgfpathlineto{\pgfqpoint{0.822110in}{2.715513in}}%
\pgfpathlineto{\pgfqpoint{0.826228in}{2.701457in}}%
\pgfpathlineto{\pgfqpoint{0.830346in}{2.673344in}}%
\pgfpathlineto{\pgfqpoint{0.834464in}{2.673344in}}%
\pgfpathlineto{\pgfqpoint{0.838582in}{2.645232in}}%
\pgfpathlineto{\pgfqpoint{0.842700in}{2.659288in}}%
\pgfpathlineto{\pgfqpoint{0.846818in}{2.631176in}}%
\pgfpathlineto{\pgfqpoint{0.850936in}{2.574951in}}%
\pgfpathlineto{\pgfqpoint{0.859173in}{2.574951in}}%
\pgfpathlineto{\pgfqpoint{0.863291in}{2.589007in}}%
\pgfpathlineto{\pgfqpoint{0.879763in}{2.589007in}}%
\pgfpathlineto{\pgfqpoint{0.883881in}{2.603063in}}%
\pgfpathlineto{\pgfqpoint{0.887999in}{2.631176in}}%
\pgfpathlineto{\pgfqpoint{0.892117in}{2.631176in}}%
\pgfpathlineto{\pgfqpoint{0.896235in}{2.645232in}}%
\pgfpathlineto{\pgfqpoint{0.904471in}{2.645232in}}%
\pgfpathlineto{\pgfqpoint{0.908590in}{2.631176in}}%
\pgfpathlineto{\pgfqpoint{0.912708in}{2.645232in}}%
\pgfpathlineto{\pgfqpoint{0.925062in}{2.645232in}}%
\pgfpathlineto{\pgfqpoint{0.929180in}{2.631176in}}%
\pgfpathlineto{\pgfqpoint{0.933298in}{2.603063in}}%
\pgfpathlineto{\pgfqpoint{0.937416in}{2.603063in}}%
\pgfpathlineto{\pgfqpoint{0.941534in}{2.546838in}}%
\pgfpathlineto{\pgfqpoint{0.945652in}{2.560894in}}%
\pgfpathlineto{\pgfqpoint{0.949770in}{2.560894in}}%
\pgfpathlineto{\pgfqpoint{0.953888in}{2.589007in}}%
\pgfpathlineto{\pgfqpoint{0.958007in}{2.560894in}}%
\pgfpathlineto{\pgfqpoint{0.966243in}{2.560894in}}%
\pgfpathlineto{\pgfqpoint{0.970361in}{2.589007in}}%
\pgfpathlineto{\pgfqpoint{0.974479in}{2.589007in}}%
\pgfpathlineto{\pgfqpoint{0.982715in}{2.560894in}}%
\pgfpathlineto{\pgfqpoint{0.990951in}{2.589007in}}%
\pgfpathlineto{\pgfqpoint{0.999187in}{2.560894in}}%
\pgfpathlineto{\pgfqpoint{1.003306in}{2.560894in}}%
\pgfpathlineto{\pgfqpoint{1.007424in}{2.490613in}}%
\pgfpathlineto{\pgfqpoint{1.015660in}{2.574951in}}%
\pgfpathlineto{\pgfqpoint{1.036250in}{2.574951in}}%
\pgfpathlineto{\pgfqpoint{1.040368in}{2.589007in}}%
\pgfpathlineto{\pgfqpoint{1.044486in}{2.617119in}}%
\pgfpathlineto{\pgfqpoint{1.052723in}{2.617119in}}%
\pgfpathlineto{\pgfqpoint{1.056841in}{2.603063in}}%
\pgfpathlineto{\pgfqpoint{1.060959in}{2.603063in}}%
\pgfpathlineto{\pgfqpoint{1.065077in}{2.574951in}}%
\pgfpathlineto{\pgfqpoint{1.073313in}{2.546838in}}%
\pgfpathlineto{\pgfqpoint{1.081549in}{2.574951in}}%
\pgfpathlineto{\pgfqpoint{1.089785in}{2.574951in}}%
\pgfpathlineto{\pgfqpoint{1.098021in}{2.518726in}}%
\pgfpathlineto{\pgfqpoint{1.102140in}{2.504669in}}%
\pgfpathlineto{\pgfqpoint{1.114494in}{2.546838in}}%
\pgfpathlineto{\pgfqpoint{1.118612in}{2.518726in}}%
\pgfpathlineto{\pgfqpoint{1.122730in}{2.546838in}}%
\pgfpathlineto{\pgfqpoint{1.126848in}{2.560894in}}%
\pgfpathlineto{\pgfqpoint{1.135084in}{2.560894in}}%
\pgfpathlineto{\pgfqpoint{1.139202in}{2.645232in}}%
\pgfpathlineto{\pgfqpoint{1.143320in}{2.645232in}}%
\pgfpathlineto{\pgfqpoint{1.147438in}{2.673344in}}%
\pgfpathlineto{\pgfqpoint{1.151557in}{2.659288in}}%
\pgfpathlineto{\pgfqpoint{1.155675in}{2.659288in}}%
\pgfpathlineto{\pgfqpoint{1.159793in}{2.687400in}}%
\pgfpathlineto{\pgfqpoint{1.168029in}{2.715513in}}%
\pgfpathlineto{\pgfqpoint{1.172147in}{2.687400in}}%
\pgfpathlineto{\pgfqpoint{1.176265in}{2.687400in}}%
\pgfpathlineto{\pgfqpoint{1.180383in}{2.673344in}}%
\pgfpathlineto{\pgfqpoint{1.184501in}{2.645232in}}%
\pgfpathlineto{\pgfqpoint{1.188619in}{2.631176in}}%
\pgfpathlineto{\pgfqpoint{1.209210in}{2.631176in}}%
\pgfpathlineto{\pgfqpoint{1.217446in}{2.659288in}}%
\pgfpathlineto{\pgfqpoint{1.225682in}{2.659288in}}%
\pgfpathlineto{\pgfqpoint{1.229800in}{2.701457in}}%
\pgfpathlineto{\pgfqpoint{1.233918in}{2.701457in}}%
\pgfpathlineto{\pgfqpoint{1.238036in}{2.715513in}}%
\pgfpathlineto{\pgfqpoint{1.246273in}{2.715513in}}%
\pgfpathlineto{\pgfqpoint{1.250391in}{2.701457in}}%
\pgfpathlineto{\pgfqpoint{1.254509in}{2.701457in}}%
\pgfpathlineto{\pgfqpoint{1.262745in}{2.645232in}}%
\pgfpathlineto{\pgfqpoint{1.266863in}{2.645232in}}%
\pgfpathlineto{\pgfqpoint{1.270981in}{2.603063in}}%
\pgfpathlineto{\pgfqpoint{1.275099in}{2.546838in}}%
\pgfpathlineto{\pgfqpoint{1.279217in}{2.532782in}}%
\pgfpathlineto{\pgfqpoint{1.283335in}{2.532782in}}%
\pgfpathlineto{\pgfqpoint{1.287453in}{2.546838in}}%
\pgfpathlineto{\pgfqpoint{1.291571in}{2.518726in}}%
\pgfpathlineto{\pgfqpoint{1.295690in}{2.518726in}}%
\pgfpathlineto{\pgfqpoint{1.299808in}{2.476557in}}%
\pgfpathlineto{\pgfqpoint{1.303926in}{2.476557in}}%
\pgfpathlineto{\pgfqpoint{1.308044in}{2.462501in}}%
\pgfpathlineto{\pgfqpoint{1.312162in}{2.476557in}}%
\pgfpathlineto{\pgfqpoint{1.316280in}{2.504669in}}%
\pgfpathlineto{\pgfqpoint{1.324516in}{2.504669in}}%
\pgfpathlineto{\pgfqpoint{1.328634in}{2.532782in}}%
\pgfpathlineto{\pgfqpoint{1.336870in}{2.532782in}}%
\pgfpathlineto{\pgfqpoint{1.340988in}{2.574951in}}%
\pgfpathlineto{\pgfqpoint{1.345107in}{2.546838in}}%
\pgfpathlineto{\pgfqpoint{1.353343in}{2.574951in}}%
\pgfpathlineto{\pgfqpoint{1.357461in}{2.617119in}}%
\pgfpathlineto{\pgfqpoint{1.382169in}{2.617119in}}%
\pgfpathlineto{\pgfqpoint{1.386287in}{2.603063in}}%
\pgfpathlineto{\pgfqpoint{1.398642in}{2.645232in}}%
\pgfpathlineto{\pgfqpoint{1.402760in}{2.631176in}}%
\pgfpathlineto{\pgfqpoint{1.406878in}{2.631176in}}%
\pgfpathlineto{\pgfqpoint{1.410996in}{2.645232in}}%
\pgfpathlineto{\pgfqpoint{1.415114in}{2.645232in}}%
\pgfpathlineto{\pgfqpoint{1.419232in}{2.631176in}}%
\pgfpathlineto{\pgfqpoint{1.423350in}{2.603063in}}%
\pgfpathlineto{\pgfqpoint{1.427468in}{2.617119in}}%
\pgfpathlineto{\pgfqpoint{1.435704in}{2.617119in}}%
\pgfpathlineto{\pgfqpoint{1.443941in}{2.560894in}}%
\pgfpathlineto{\pgfqpoint{1.448059in}{2.546838in}}%
\pgfpathlineto{\pgfqpoint{1.452177in}{2.589007in}}%
\pgfpathlineto{\pgfqpoint{1.456295in}{2.603063in}}%
\pgfpathlineto{\pgfqpoint{1.460413in}{2.589007in}}%
\pgfpathlineto{\pgfqpoint{1.464531in}{2.589007in}}%
\pgfpathlineto{\pgfqpoint{1.468649in}{2.617119in}}%
\pgfpathlineto{\pgfqpoint{1.472767in}{2.617119in}}%
\pgfpathlineto{\pgfqpoint{1.476885in}{2.631176in}}%
\pgfpathlineto{\pgfqpoint{1.481003in}{2.631176in}}%
\pgfpathlineto{\pgfqpoint{1.485121in}{2.617119in}}%
\pgfpathlineto{\pgfqpoint{1.493358in}{2.645232in}}%
\pgfpathlineto{\pgfqpoint{1.497476in}{2.645232in}}%
\pgfpathlineto{\pgfqpoint{1.501594in}{2.631176in}}%
\pgfpathlineto{\pgfqpoint{1.513948in}{2.631176in}}%
\pgfpathlineto{\pgfqpoint{1.522184in}{2.659288in}}%
\pgfpathlineto{\pgfqpoint{1.530420in}{2.659288in}}%
\pgfpathlineto{\pgfqpoint{1.538657in}{2.631176in}}%
\pgfpathlineto{\pgfqpoint{1.542775in}{2.631176in}}%
\pgfpathlineto{\pgfqpoint{1.546893in}{2.617119in}}%
\pgfpathlineto{\pgfqpoint{1.563365in}{2.617119in}}%
\pgfpathlineto{\pgfqpoint{1.567483in}{2.603063in}}%
\pgfpathlineto{\pgfqpoint{1.571601in}{2.617119in}}%
\pgfpathlineto{\pgfqpoint{1.579837in}{2.617119in}}%
\pgfpathlineto{\pgfqpoint{1.583955in}{2.645232in}}%
\pgfpathlineto{\pgfqpoint{1.588074in}{2.659288in}}%
\pgfpathlineto{\pgfqpoint{1.592192in}{2.631176in}}%
\pgfpathlineto{\pgfqpoint{1.596310in}{2.631176in}}%
\pgfpathlineto{\pgfqpoint{1.600428in}{2.673344in}}%
\pgfpathlineto{\pgfqpoint{1.604546in}{2.645232in}}%
\pgfpathlineto{\pgfqpoint{1.608664in}{2.645232in}}%
\pgfpathlineto{\pgfqpoint{1.612782in}{2.673344in}}%
\pgfpathlineto{\pgfqpoint{1.621018in}{2.701457in}}%
\pgfpathlineto{\pgfqpoint{1.625136in}{2.757682in}}%
\pgfpathlineto{\pgfqpoint{1.629254in}{2.771738in}}%
\pgfpathlineto{\pgfqpoint{1.633372in}{2.757682in}}%
\pgfpathlineto{\pgfqpoint{1.637491in}{2.729569in}}%
\pgfpathlineto{\pgfqpoint{1.641609in}{2.743625in}}%
\pgfpathlineto{\pgfqpoint{1.645727in}{2.771738in}}%
\pgfpathlineto{\pgfqpoint{1.649845in}{2.729569in}}%
\pgfpathlineto{\pgfqpoint{1.658081in}{2.729569in}}%
\pgfpathlineto{\pgfqpoint{1.662199in}{2.771738in}}%
\pgfpathlineto{\pgfqpoint{1.666317in}{2.785794in}}%
\pgfpathlineto{\pgfqpoint{1.670435in}{2.757682in}}%
\pgfpathlineto{\pgfqpoint{1.674553in}{2.757682in}}%
\pgfpathlineto{\pgfqpoint{1.678671in}{2.743625in}}%
\pgfpathlineto{\pgfqpoint{1.703380in}{2.743625in}}%
\pgfpathlineto{\pgfqpoint{1.707498in}{2.785794in}}%
\pgfpathlineto{\pgfqpoint{1.711616in}{2.813906in}}%
\pgfpathlineto{\pgfqpoint{1.719852in}{2.813906in}}%
\pgfpathlineto{\pgfqpoint{1.723970in}{2.827963in}}%
\pgfpathlineto{\pgfqpoint{1.728088in}{2.856075in}}%
\pgfpathlineto{\pgfqpoint{1.732206in}{2.856075in}}%
\pgfpathlineto{\pgfqpoint{1.736325in}{2.842019in}}%
\pgfpathlineto{\pgfqpoint{1.740443in}{2.856075in}}%
\pgfpathlineto{\pgfqpoint{1.744561in}{2.827963in}}%
\pgfpathlineto{\pgfqpoint{1.748679in}{2.856075in}}%
\pgfpathlineto{\pgfqpoint{1.761033in}{2.856075in}}%
\pgfpathlineto{\pgfqpoint{1.769269in}{2.827963in}}%
\pgfpathlineto{\pgfqpoint{1.773387in}{2.842019in}}%
\pgfpathlineto{\pgfqpoint{1.777505in}{2.842019in}}%
\pgfpathlineto{\pgfqpoint{1.781624in}{2.813906in}}%
\pgfpathlineto{\pgfqpoint{1.785742in}{2.827963in}}%
\pgfpathlineto{\pgfqpoint{1.789860in}{2.856075in}}%
\pgfpathlineto{\pgfqpoint{1.798096in}{2.856075in}}%
\pgfpathlineto{\pgfqpoint{1.802214in}{2.870131in}}%
\pgfpathlineto{\pgfqpoint{1.810450in}{2.870131in}}%
\pgfpathlineto{\pgfqpoint{1.814568in}{2.842019in}}%
\pgfpathlineto{\pgfqpoint{1.818686in}{2.827963in}}%
\pgfpathlineto{\pgfqpoint{1.822804in}{2.799850in}}%
\pgfpathlineto{\pgfqpoint{1.826922in}{2.799850in}}%
\pgfpathlineto{\pgfqpoint{1.831041in}{2.785794in}}%
\pgfpathlineto{\pgfqpoint{1.843395in}{2.785794in}}%
\pgfpathlineto{\pgfqpoint{1.847513in}{2.827963in}}%
\pgfpathlineto{\pgfqpoint{1.855749in}{2.827963in}}%
\pgfpathlineto{\pgfqpoint{1.859867in}{2.842019in}}%
\pgfpathlineto{\pgfqpoint{1.863985in}{2.842019in}}%
\pgfpathlineto{\pgfqpoint{1.872221in}{2.813906in}}%
\pgfpathlineto{\pgfqpoint{1.880458in}{2.813906in}}%
\pgfpathlineto{\pgfqpoint{1.884576in}{2.827963in}}%
\pgfpathlineto{\pgfqpoint{1.888694in}{2.856075in}}%
\pgfpathlineto{\pgfqpoint{1.892812in}{2.856075in}}%
\pgfpathlineto{\pgfqpoint{1.896930in}{2.870131in}}%
\pgfpathlineto{\pgfqpoint{1.901048in}{2.827963in}}%
\pgfpathlineto{\pgfqpoint{1.917520in}{2.827963in}}%
\pgfpathlineto{\pgfqpoint{1.921638in}{2.813906in}}%
\pgfpathlineto{\pgfqpoint{1.929875in}{2.813906in}}%
\pgfpathlineto{\pgfqpoint{1.933993in}{2.785794in}}%
\pgfpathlineto{\pgfqpoint{1.938111in}{2.729569in}}%
\pgfpathlineto{\pgfqpoint{1.942229in}{2.729569in}}%
\pgfpathlineto{\pgfqpoint{1.946347in}{2.743625in}}%
\pgfpathlineto{\pgfqpoint{1.950465in}{2.743625in}}%
\pgfpathlineto{\pgfqpoint{1.954583in}{2.757682in}}%
\pgfpathlineto{\pgfqpoint{1.958701in}{2.729569in}}%
\pgfpathlineto{\pgfqpoint{1.962819in}{2.715513in}}%
\pgfpathlineto{\pgfqpoint{1.966937in}{2.715513in}}%
\pgfpathlineto{\pgfqpoint{1.971055in}{2.743625in}}%
\pgfpathlineto{\pgfqpoint{1.975173in}{2.729569in}}%
\pgfpathlineto{\pgfqpoint{1.983410in}{2.757682in}}%
\pgfpathlineto{\pgfqpoint{1.995764in}{2.757682in}}%
\pgfpathlineto{\pgfqpoint{1.999882in}{2.743625in}}%
\pgfpathlineto{\pgfqpoint{2.004000in}{2.757682in}}%
\pgfpathlineto{\pgfqpoint{2.008118in}{2.757682in}}%
\pgfpathlineto{\pgfqpoint{2.012236in}{2.743625in}}%
\pgfpathlineto{\pgfqpoint{2.024591in}{2.785794in}}%
\pgfpathlineto{\pgfqpoint{2.028709in}{2.785794in}}%
\pgfpathlineto{\pgfqpoint{2.036945in}{2.729569in}}%
\pgfpathlineto{\pgfqpoint{2.045181in}{2.757682in}}%
\pgfpathlineto{\pgfqpoint{2.053417in}{2.701457in}}%
\pgfpathlineto{\pgfqpoint{2.057535in}{2.687400in}}%
\pgfpathlineto{\pgfqpoint{2.061653in}{2.701457in}}%
\pgfpathlineto{\pgfqpoint{2.065771in}{2.743625in}}%
\pgfpathlineto{\pgfqpoint{2.074008in}{2.771738in}}%
\pgfpathlineto{\pgfqpoint{2.078126in}{2.799850in}}%
\pgfpathlineto{\pgfqpoint{2.082244in}{2.771738in}}%
\pgfpathlineto{\pgfqpoint{2.086362in}{2.757682in}}%
\pgfpathlineto{\pgfqpoint{2.090480in}{2.757682in}}%
\pgfpathlineto{\pgfqpoint{2.094598in}{2.827963in}}%
\pgfpathlineto{\pgfqpoint{2.098716in}{2.799850in}}%
\pgfpathlineto{\pgfqpoint{2.102834in}{2.813906in}}%
\pgfpathlineto{\pgfqpoint{2.106952in}{2.813906in}}%
\pgfpathlineto{\pgfqpoint{2.111070in}{2.771738in}}%
\pgfpathlineto{\pgfqpoint{2.115188in}{2.799850in}}%
\pgfpathlineto{\pgfqpoint{2.123425in}{2.799850in}}%
\pgfpathlineto{\pgfqpoint{2.127543in}{2.771738in}}%
\pgfpathlineto{\pgfqpoint{2.131661in}{2.757682in}}%
\pgfpathlineto{\pgfqpoint{2.139897in}{2.757682in}}%
\pgfpathlineto{\pgfqpoint{2.144015in}{2.729569in}}%
\pgfpathlineto{\pgfqpoint{2.148133in}{2.715513in}}%
\pgfpathlineto{\pgfqpoint{2.152251in}{2.715513in}}%
\pgfpathlineto{\pgfqpoint{2.156369in}{2.701457in}}%
\pgfpathlineto{\pgfqpoint{2.160487in}{2.715513in}}%
\pgfpathlineto{\pgfqpoint{2.168723in}{2.631176in}}%
\pgfpathlineto{\pgfqpoint{2.172842in}{2.645232in}}%
\pgfpathlineto{\pgfqpoint{2.176960in}{2.631176in}}%
\pgfpathlineto{\pgfqpoint{2.181078in}{2.603063in}}%
\pgfpathlineto{\pgfqpoint{2.185196in}{2.589007in}}%
\pgfpathlineto{\pgfqpoint{2.189314in}{2.589007in}}%
\pgfpathlineto{\pgfqpoint{2.193432in}{2.546838in}}%
\pgfpathlineto{\pgfqpoint{2.197550in}{2.518726in}}%
\pgfpathlineto{\pgfqpoint{2.201668in}{2.546838in}}%
\pgfpathlineto{\pgfqpoint{2.205786in}{2.546838in}}%
\pgfpathlineto{\pgfqpoint{2.209904in}{2.574951in}}%
\pgfpathlineto{\pgfqpoint{2.218140in}{2.574951in}}%
\pgfpathlineto{\pgfqpoint{2.226377in}{2.546838in}}%
\pgfpathlineto{\pgfqpoint{2.230495in}{2.560894in}}%
\pgfpathlineto{\pgfqpoint{2.234613in}{2.532782in}}%
\pgfpathlineto{\pgfqpoint{2.242849in}{2.589007in}}%
\pgfpathlineto{\pgfqpoint{2.246967in}{2.589007in}}%
\pgfpathlineto{\pgfqpoint{2.251085in}{2.603063in}}%
\pgfpathlineto{\pgfqpoint{2.255203in}{2.574951in}}%
\pgfpathlineto{\pgfqpoint{2.259321in}{2.574951in}}%
\pgfpathlineto{\pgfqpoint{2.267557in}{2.546838in}}%
\pgfpathlineto{\pgfqpoint{2.275794in}{2.490613in}}%
\pgfpathlineto{\pgfqpoint{2.279912in}{2.476557in}}%
\pgfpathlineto{\pgfqpoint{2.284030in}{2.490613in}}%
\pgfpathlineto{\pgfqpoint{2.288148in}{2.490613in}}%
\pgfpathlineto{\pgfqpoint{2.300502in}{2.532782in}}%
\pgfpathlineto{\pgfqpoint{2.308738in}{2.532782in}}%
\pgfpathlineto{\pgfqpoint{2.321093in}{2.490613in}}%
\pgfpathlineto{\pgfqpoint{2.325211in}{2.490613in}}%
\pgfpathlineto{\pgfqpoint{2.333447in}{2.546838in}}%
\pgfpathlineto{\pgfqpoint{2.337565in}{2.546838in}}%
\pgfpathlineto{\pgfqpoint{2.341683in}{2.504669in}}%
\pgfpathlineto{\pgfqpoint{2.345801in}{2.518726in}}%
\pgfpathlineto{\pgfqpoint{2.349919in}{2.518726in}}%
\pgfpathlineto{\pgfqpoint{2.354037in}{2.546838in}}%
\pgfpathlineto{\pgfqpoint{2.362273in}{2.546838in}}%
\pgfpathlineto{\pgfqpoint{2.366392in}{2.518726in}}%
\pgfpathlineto{\pgfqpoint{2.370510in}{2.518726in}}%
\pgfpathlineto{\pgfqpoint{2.374628in}{2.462501in}}%
\pgfpathlineto{\pgfqpoint{2.378746in}{2.476557in}}%
\pgfpathlineto{\pgfqpoint{2.386982in}{2.392220in}}%
\pgfpathlineto{\pgfqpoint{2.391100in}{2.392220in}}%
\pgfpathlineto{\pgfqpoint{2.395218in}{2.378163in}}%
\pgfpathlineto{\pgfqpoint{2.399336in}{2.378163in}}%
\pgfpathlineto{\pgfqpoint{2.403454in}{2.364107in}}%
\pgfpathlineto{\pgfqpoint{2.411690in}{2.364107in}}%
\pgfpathlineto{\pgfqpoint{2.419927in}{2.335995in}}%
\pgfpathlineto{\pgfqpoint{2.424045in}{2.335995in}}%
\pgfpathlineto{\pgfqpoint{2.428163in}{2.307882in}}%
\pgfpathlineto{\pgfqpoint{2.481698in}{2.307882in}}%
\pgfpathlineto{\pgfqpoint{2.485816in}{2.321939in}}%
\pgfpathlineto{\pgfqpoint{2.506406in}{2.321939in}}%
\pgfpathlineto{\pgfqpoint{2.510524in}{2.307882in}}%
\pgfpathlineto{\pgfqpoint{2.559942in}{2.307882in}}%
\pgfpathlineto{\pgfqpoint{2.564060in}{2.293826in}}%
\pgfpathlineto{\pgfqpoint{2.605240in}{2.293826in}}%
\pgfpathlineto{\pgfqpoint{2.609359in}{2.279770in}}%
\pgfpathlineto{\pgfqpoint{2.646421in}{2.279770in}}%
\pgfpathlineto{\pgfqpoint{2.646421in}{2.279770in}}%
\pgfusepath{stroke}%
\end{pgfscope}%
\begin{pgfscope}%
\pgfpathrectangle{\pgfqpoint{0.488751in}{2.165212in}}{\pgfqpoint{2.260417in}{1.283333in}}%
\pgfusepath{clip}%
\pgfsetrectcap%
\pgfsetroundjoin%
\pgfsetlinewidth{0.803000pt}%
\definecolor{currentstroke}{rgb}{0.333333,0.333333,0.333333}%
\pgfsetstrokecolor{currentstroke}%
\pgfsetstrokeopacity{0.270000}%
\pgfsetdash{}{0pt}%
\pgfpathmoveto{\pgfqpoint{0.591497in}{2.490613in}}%
\pgfpathlineto{\pgfqpoint{0.595615in}{2.504669in}}%
\pgfpathlineto{\pgfqpoint{0.603851in}{2.504669in}}%
\pgfpathlineto{\pgfqpoint{0.607969in}{2.518726in}}%
\pgfpathlineto{\pgfqpoint{0.612087in}{2.504669in}}%
\pgfpathlineto{\pgfqpoint{0.616206in}{2.504669in}}%
\pgfpathlineto{\pgfqpoint{0.620324in}{2.490613in}}%
\pgfpathlineto{\pgfqpoint{0.624442in}{2.490613in}}%
\pgfpathlineto{\pgfqpoint{0.632678in}{2.546838in}}%
\pgfpathlineto{\pgfqpoint{0.636796in}{2.560894in}}%
\pgfpathlineto{\pgfqpoint{0.640914in}{2.560894in}}%
\pgfpathlineto{\pgfqpoint{0.645032in}{2.603063in}}%
\pgfpathlineto{\pgfqpoint{0.649150in}{2.631176in}}%
\pgfpathlineto{\pgfqpoint{0.653268in}{2.617119in}}%
\pgfpathlineto{\pgfqpoint{0.657386in}{2.645232in}}%
\pgfpathlineto{\pgfqpoint{0.661504in}{2.631176in}}%
\pgfpathlineto{\pgfqpoint{0.669741in}{2.631176in}}%
\pgfpathlineto{\pgfqpoint{0.673859in}{2.645232in}}%
\pgfpathlineto{\pgfqpoint{0.677977in}{2.631176in}}%
\pgfpathlineto{\pgfqpoint{0.682095in}{2.645232in}}%
\pgfpathlineto{\pgfqpoint{0.690331in}{2.617119in}}%
\pgfpathlineto{\pgfqpoint{0.694449in}{2.617119in}}%
\pgfpathlineto{\pgfqpoint{0.702685in}{2.645232in}}%
\pgfpathlineto{\pgfqpoint{0.706803in}{2.631176in}}%
\pgfpathlineto{\pgfqpoint{0.715040in}{2.631176in}}%
\pgfpathlineto{\pgfqpoint{0.723276in}{2.603063in}}%
\pgfpathlineto{\pgfqpoint{0.743866in}{2.603063in}}%
\pgfpathlineto{\pgfqpoint{0.752102in}{2.546838in}}%
\pgfpathlineto{\pgfqpoint{0.756220in}{2.532782in}}%
\pgfpathlineto{\pgfqpoint{0.760339in}{2.532782in}}%
\pgfpathlineto{\pgfqpoint{0.768575in}{2.476557in}}%
\pgfpathlineto{\pgfqpoint{0.772693in}{2.462501in}}%
\pgfpathlineto{\pgfqpoint{0.776811in}{2.434388in}}%
\pgfpathlineto{\pgfqpoint{0.780929in}{2.420332in}}%
\pgfpathlineto{\pgfqpoint{0.785047in}{2.378163in}}%
\pgfpathlineto{\pgfqpoint{0.789165in}{2.378163in}}%
\pgfpathlineto{\pgfqpoint{0.793283in}{2.364107in}}%
\pgfpathlineto{\pgfqpoint{0.797401in}{2.364107in}}%
\pgfpathlineto{\pgfqpoint{0.801519in}{2.335995in}}%
\pgfpathlineto{\pgfqpoint{0.817992in}{2.335995in}}%
\pgfpathlineto{\pgfqpoint{0.822110in}{2.321939in}}%
\pgfpathlineto{\pgfqpoint{0.830346in}{2.321939in}}%
\pgfpathlineto{\pgfqpoint{0.834464in}{2.307882in}}%
\pgfpathlineto{\pgfqpoint{0.859173in}{2.307882in}}%
\pgfpathlineto{\pgfqpoint{0.863291in}{2.321939in}}%
\pgfpathlineto{\pgfqpoint{0.867409in}{2.321939in}}%
\pgfpathlineto{\pgfqpoint{0.871527in}{2.307882in}}%
\pgfpathlineto{\pgfqpoint{0.875645in}{2.307882in}}%
\pgfpathlineto{\pgfqpoint{0.879763in}{2.293826in}}%
\pgfpathlineto{\pgfqpoint{0.883881in}{2.293826in}}%
\pgfpathlineto{\pgfqpoint{0.887999in}{2.279770in}}%
\pgfpathlineto{\pgfqpoint{0.904471in}{2.279770in}}%
\pgfpathlineto{\pgfqpoint{0.908590in}{2.293826in}}%
\pgfpathlineto{\pgfqpoint{0.958007in}{2.293826in}}%
\pgfpathlineto{\pgfqpoint{0.966243in}{2.265714in}}%
\pgfpathlineto{\pgfqpoint{0.970361in}{2.279770in}}%
\pgfpathlineto{\pgfqpoint{1.036250in}{2.279770in}}%
\pgfpathlineto{\pgfqpoint{1.040368in}{2.293826in}}%
\pgfpathlineto{\pgfqpoint{1.044486in}{2.293826in}}%
\pgfpathlineto{\pgfqpoint{1.048604in}{2.307882in}}%
\pgfpathlineto{\pgfqpoint{1.060959in}{2.307882in}}%
\pgfpathlineto{\pgfqpoint{1.065077in}{2.293826in}}%
\pgfpathlineto{\pgfqpoint{1.069195in}{2.293826in}}%
\pgfpathlineto{\pgfqpoint{1.073313in}{2.279770in}}%
\pgfpathlineto{\pgfqpoint{1.077431in}{2.279770in}}%
\pgfpathlineto{\pgfqpoint{1.081549in}{2.265714in}}%
\pgfpathlineto{\pgfqpoint{1.085667in}{2.265714in}}%
\pgfpathlineto{\pgfqpoint{1.093903in}{2.293826in}}%
\pgfpathlineto{\pgfqpoint{1.098021in}{2.321939in}}%
\pgfpathlineto{\pgfqpoint{1.110376in}{2.321939in}}%
\pgfpathlineto{\pgfqpoint{1.114494in}{2.350051in}}%
\pgfpathlineto{\pgfqpoint{1.118612in}{2.350051in}}%
\pgfpathlineto{\pgfqpoint{1.122730in}{2.364107in}}%
\pgfpathlineto{\pgfqpoint{1.130966in}{2.364107in}}%
\pgfpathlineto{\pgfqpoint{1.135084in}{2.350051in}}%
\pgfpathlineto{\pgfqpoint{1.139202in}{2.378163in}}%
\pgfpathlineto{\pgfqpoint{1.192737in}{2.378163in}}%
\pgfpathlineto{\pgfqpoint{1.196855in}{2.392220in}}%
\pgfpathlineto{\pgfqpoint{1.262745in}{2.392220in}}%
\pgfpathlineto{\pgfqpoint{1.270981in}{2.364107in}}%
\pgfpathlineto{\pgfqpoint{1.275099in}{2.392220in}}%
\pgfpathlineto{\pgfqpoint{1.283335in}{2.392220in}}%
\pgfpathlineto{\pgfqpoint{1.291571in}{2.364107in}}%
\pgfpathlineto{\pgfqpoint{1.324516in}{2.364107in}}%
\pgfpathlineto{\pgfqpoint{1.328634in}{2.350051in}}%
\pgfpathlineto{\pgfqpoint{1.332752in}{2.350051in}}%
\pgfpathlineto{\pgfqpoint{1.340988in}{2.321939in}}%
\pgfpathlineto{\pgfqpoint{1.353343in}{2.321939in}}%
\pgfpathlineto{\pgfqpoint{1.357461in}{2.335995in}}%
\pgfpathlineto{\pgfqpoint{1.390405in}{2.335995in}}%
\pgfpathlineto{\pgfqpoint{1.394524in}{2.378163in}}%
\pgfpathlineto{\pgfqpoint{1.402760in}{2.321939in}}%
\pgfpathlineto{\pgfqpoint{1.406878in}{2.307882in}}%
\pgfpathlineto{\pgfqpoint{1.410996in}{2.307882in}}%
\pgfpathlineto{\pgfqpoint{1.415114in}{2.293826in}}%
\pgfpathlineto{\pgfqpoint{1.419232in}{2.293826in}}%
\pgfpathlineto{\pgfqpoint{1.423350in}{2.307882in}}%
\pgfpathlineto{\pgfqpoint{1.431586in}{2.307882in}}%
\pgfpathlineto{\pgfqpoint{1.435704in}{2.293826in}}%
\pgfpathlineto{\pgfqpoint{1.439822in}{2.293826in}}%
\pgfpathlineto{\pgfqpoint{1.443941in}{2.279770in}}%
\pgfpathlineto{\pgfqpoint{1.448059in}{2.293826in}}%
\pgfpathlineto{\pgfqpoint{1.452177in}{2.293826in}}%
\pgfpathlineto{\pgfqpoint{1.456295in}{2.279770in}}%
\pgfpathlineto{\pgfqpoint{1.460413in}{2.307882in}}%
\pgfpathlineto{\pgfqpoint{1.464531in}{2.307882in}}%
\pgfpathlineto{\pgfqpoint{1.468649in}{2.321939in}}%
\pgfpathlineto{\pgfqpoint{1.472767in}{2.293826in}}%
\pgfpathlineto{\pgfqpoint{1.476885in}{2.307882in}}%
\pgfpathlineto{\pgfqpoint{1.481003in}{2.307882in}}%
\pgfpathlineto{\pgfqpoint{1.485121in}{2.293826in}}%
\pgfpathlineto{\pgfqpoint{1.497476in}{2.293826in}}%
\pgfpathlineto{\pgfqpoint{1.501594in}{2.279770in}}%
\pgfpathlineto{\pgfqpoint{1.526302in}{2.279770in}}%
\pgfpathlineto{\pgfqpoint{1.530420in}{2.265714in}}%
\pgfpathlineto{\pgfqpoint{1.534538in}{2.279770in}}%
\pgfpathlineto{\pgfqpoint{1.546893in}{2.279770in}}%
\pgfpathlineto{\pgfqpoint{1.551011in}{2.293826in}}%
\pgfpathlineto{\pgfqpoint{1.555129in}{2.293826in}}%
\pgfpathlineto{\pgfqpoint{1.559247in}{2.279770in}}%
\pgfpathlineto{\pgfqpoint{1.563365in}{2.279770in}}%
\pgfpathlineto{\pgfqpoint{1.567483in}{2.307882in}}%
\pgfpathlineto{\pgfqpoint{1.571601in}{2.293826in}}%
\pgfpathlineto{\pgfqpoint{1.588074in}{2.293826in}}%
\pgfpathlineto{\pgfqpoint{1.592192in}{2.321939in}}%
\pgfpathlineto{\pgfqpoint{1.596310in}{2.307882in}}%
\pgfpathlineto{\pgfqpoint{1.600428in}{2.307882in}}%
\pgfpathlineto{\pgfqpoint{1.604546in}{2.321939in}}%
\pgfpathlineto{\pgfqpoint{1.608664in}{2.307882in}}%
\pgfpathlineto{\pgfqpoint{1.612782in}{2.307882in}}%
\pgfpathlineto{\pgfqpoint{1.616900in}{2.293826in}}%
\pgfpathlineto{\pgfqpoint{1.625136in}{2.293826in}}%
\pgfpathlineto{\pgfqpoint{1.629254in}{2.265714in}}%
\pgfpathlineto{\pgfqpoint{1.633372in}{2.279770in}}%
\pgfpathlineto{\pgfqpoint{1.653963in}{2.279770in}}%
\pgfpathlineto{\pgfqpoint{1.658081in}{2.293826in}}%
\pgfpathlineto{\pgfqpoint{1.662199in}{2.293826in}}%
\pgfpathlineto{\pgfqpoint{1.670435in}{2.321939in}}%
\pgfpathlineto{\pgfqpoint{1.678671in}{2.293826in}}%
\pgfpathlineto{\pgfqpoint{1.686908in}{2.293826in}}%
\pgfpathlineto{\pgfqpoint{1.691026in}{2.321939in}}%
\pgfpathlineto{\pgfqpoint{1.695144in}{2.307882in}}%
\pgfpathlineto{\pgfqpoint{1.703380in}{2.307882in}}%
\pgfpathlineto{\pgfqpoint{1.707498in}{2.293826in}}%
\pgfpathlineto{\pgfqpoint{1.723970in}{2.293826in}}%
\pgfpathlineto{\pgfqpoint{1.728088in}{2.307882in}}%
\pgfpathlineto{\pgfqpoint{1.736325in}{2.279770in}}%
\pgfpathlineto{\pgfqpoint{1.744561in}{2.279770in}}%
\pgfpathlineto{\pgfqpoint{1.748679in}{2.265714in}}%
\pgfpathlineto{\pgfqpoint{1.752797in}{2.279770in}}%
\pgfpathlineto{\pgfqpoint{1.777505in}{2.279770in}}%
\pgfpathlineto{\pgfqpoint{1.781624in}{2.293826in}}%
\pgfpathlineto{\pgfqpoint{1.785742in}{2.293826in}}%
\pgfpathlineto{\pgfqpoint{1.789860in}{2.279770in}}%
\pgfpathlineto{\pgfqpoint{1.826922in}{2.279770in}}%
\pgfpathlineto{\pgfqpoint{1.831041in}{2.293826in}}%
\pgfpathlineto{\pgfqpoint{1.855749in}{2.293826in}}%
\pgfpathlineto{\pgfqpoint{1.859867in}{2.279770in}}%
\pgfpathlineto{\pgfqpoint{1.901048in}{2.279770in}}%
\pgfpathlineto{\pgfqpoint{1.905166in}{2.265714in}}%
\pgfpathlineto{\pgfqpoint{1.909284in}{2.279770in}}%
\pgfpathlineto{\pgfqpoint{1.913402in}{2.265714in}}%
\pgfpathlineto{\pgfqpoint{1.917520in}{2.279770in}}%
\pgfpathlineto{\pgfqpoint{1.938111in}{2.279770in}}%
\pgfpathlineto{\pgfqpoint{1.942229in}{2.307882in}}%
\pgfpathlineto{\pgfqpoint{1.946347in}{2.307882in}}%
\pgfpathlineto{\pgfqpoint{1.954583in}{2.279770in}}%
\pgfpathlineto{\pgfqpoint{1.958701in}{2.251657in}}%
\pgfpathlineto{\pgfqpoint{1.962819in}{2.279770in}}%
\pgfpathlineto{\pgfqpoint{1.975173in}{2.279770in}}%
\pgfpathlineto{\pgfqpoint{1.979292in}{2.265714in}}%
\pgfpathlineto{\pgfqpoint{1.983410in}{2.279770in}}%
\pgfpathlineto{\pgfqpoint{1.987528in}{2.265714in}}%
\pgfpathlineto{\pgfqpoint{1.991646in}{2.279770in}}%
\pgfpathlineto{\pgfqpoint{2.024591in}{2.279770in}}%
\pgfpathlineto{\pgfqpoint{2.028709in}{2.251657in}}%
\pgfpathlineto{\pgfqpoint{2.032827in}{2.279770in}}%
\pgfpathlineto{\pgfqpoint{2.036945in}{2.279770in}}%
\pgfpathlineto{\pgfqpoint{2.041063in}{2.251657in}}%
\pgfpathlineto{\pgfqpoint{2.045181in}{2.251657in}}%
\pgfpathlineto{\pgfqpoint{2.049299in}{2.279770in}}%
\pgfpathlineto{\pgfqpoint{2.053417in}{2.293826in}}%
\pgfpathlineto{\pgfqpoint{2.061653in}{2.293826in}}%
\pgfpathlineto{\pgfqpoint{2.069889in}{2.265714in}}%
\pgfpathlineto{\pgfqpoint{2.074008in}{2.279770in}}%
\pgfpathlineto{\pgfqpoint{2.078126in}{2.265714in}}%
\pgfpathlineto{\pgfqpoint{2.082244in}{2.279770in}}%
\pgfpathlineto{\pgfqpoint{2.086362in}{2.279770in}}%
\pgfpathlineto{\pgfqpoint{2.090480in}{2.265714in}}%
\pgfpathlineto{\pgfqpoint{2.102834in}{2.265714in}}%
\pgfpathlineto{\pgfqpoint{2.106952in}{2.279770in}}%
\pgfpathlineto{\pgfqpoint{2.115188in}{2.279770in}}%
\pgfpathlineto{\pgfqpoint{2.119306in}{2.265714in}}%
\pgfpathlineto{\pgfqpoint{2.123425in}{2.279770in}}%
\pgfpathlineto{\pgfqpoint{2.144015in}{2.279770in}}%
\pgfpathlineto{\pgfqpoint{2.148133in}{2.265714in}}%
\pgfpathlineto{\pgfqpoint{2.152251in}{2.279770in}}%
\pgfpathlineto{\pgfqpoint{2.168723in}{2.279770in}}%
\pgfpathlineto{\pgfqpoint{2.172842in}{2.265714in}}%
\pgfpathlineto{\pgfqpoint{2.176960in}{2.279770in}}%
\pgfpathlineto{\pgfqpoint{2.185196in}{2.279770in}}%
\pgfpathlineto{\pgfqpoint{2.189314in}{2.265714in}}%
\pgfpathlineto{\pgfqpoint{2.205786in}{2.265714in}}%
\pgfpathlineto{\pgfqpoint{2.209904in}{2.251657in}}%
\pgfpathlineto{\pgfqpoint{2.214022in}{2.279770in}}%
\pgfpathlineto{\pgfqpoint{2.271676in}{2.279770in}}%
\pgfpathlineto{\pgfqpoint{2.275794in}{2.293826in}}%
\pgfpathlineto{\pgfqpoint{2.304620in}{2.293826in}}%
\pgfpathlineto{\pgfqpoint{2.308738in}{2.307882in}}%
\pgfpathlineto{\pgfqpoint{2.329329in}{2.307882in}}%
\pgfpathlineto{\pgfqpoint{2.333447in}{2.293826in}}%
\pgfpathlineto{\pgfqpoint{2.391100in}{2.293826in}}%
\pgfpathlineto{\pgfqpoint{2.395218in}{2.307882in}}%
\pgfpathlineto{\pgfqpoint{2.403454in}{2.307882in}}%
\pgfpathlineto{\pgfqpoint{2.407572in}{2.321939in}}%
\pgfpathlineto{\pgfqpoint{2.448753in}{2.321939in}}%
\pgfpathlineto{\pgfqpoint{2.452871in}{2.307882in}}%
\pgfpathlineto{\pgfqpoint{2.456989in}{2.307882in}}%
\pgfpathlineto{\pgfqpoint{2.461107in}{2.321939in}}%
\pgfpathlineto{\pgfqpoint{2.477580in}{2.321939in}}%
\pgfpathlineto{\pgfqpoint{2.485816in}{2.350051in}}%
\pgfpathlineto{\pgfqpoint{2.506406in}{2.350051in}}%
\pgfpathlineto{\pgfqpoint{2.514643in}{2.321939in}}%
\pgfpathlineto{\pgfqpoint{2.518761in}{2.321939in}}%
\pgfpathlineto{\pgfqpoint{2.522879in}{2.335995in}}%
\pgfpathlineto{\pgfqpoint{2.568178in}{2.335995in}}%
\pgfpathlineto{\pgfqpoint{2.572296in}{2.350051in}}%
\pgfpathlineto{\pgfqpoint{2.576414in}{2.335995in}}%
\pgfpathlineto{\pgfqpoint{2.580532in}{2.350051in}}%
\pgfpathlineto{\pgfqpoint{2.584650in}{2.350051in}}%
\pgfpathlineto{\pgfqpoint{2.588768in}{2.378163in}}%
\pgfpathlineto{\pgfqpoint{2.592886in}{2.350051in}}%
\pgfpathlineto{\pgfqpoint{2.609359in}{2.350051in}}%
\pgfpathlineto{\pgfqpoint{2.613477in}{2.364107in}}%
\pgfpathlineto{\pgfqpoint{2.621713in}{2.364107in}}%
\pgfpathlineto{\pgfqpoint{2.625831in}{2.350051in}}%
\pgfpathlineto{\pgfqpoint{2.629949in}{2.364107in}}%
\pgfpathlineto{\pgfqpoint{2.646421in}{2.364107in}}%
\pgfpathlineto{\pgfqpoint{2.646421in}{2.364107in}}%
\pgfusepath{stroke}%
\end{pgfscope}%
\begin{pgfscope}%
\pgfpathrectangle{\pgfqpoint{0.488751in}{2.165212in}}{\pgfqpoint{2.260417in}{1.283333in}}%
\pgfusepath{clip}%
\pgfsetbuttcap%
\pgfsetroundjoin%
\pgfsetlinewidth{0.803000pt}%
\definecolor{currentstroke}{rgb}{0.686275,0.352941,0.313725}%
\pgfsetstrokecolor{currentstroke}%
\pgfsetstrokeopacity{0.270000}%
\pgfsetdash{{2.960000pt}{1.280000pt}}{0.000000pt}%
\pgfpathmoveto{\pgfqpoint{0.591497in}{2.617119in}}%
\pgfpathlineto{\pgfqpoint{0.595615in}{2.603063in}}%
\pgfpathlineto{\pgfqpoint{0.616206in}{2.603063in}}%
\pgfpathlineto{\pgfqpoint{0.620324in}{2.589007in}}%
\pgfpathlineto{\pgfqpoint{0.657386in}{2.589007in}}%
\pgfpathlineto{\pgfqpoint{0.661504in}{2.574951in}}%
\pgfpathlineto{\pgfqpoint{0.673859in}{2.574951in}}%
\pgfpathlineto{\pgfqpoint{0.677977in}{2.589007in}}%
\pgfpathlineto{\pgfqpoint{0.702685in}{2.589007in}}%
\pgfpathlineto{\pgfqpoint{0.706803in}{2.603063in}}%
\pgfpathlineto{\pgfqpoint{0.710922in}{2.631176in}}%
\pgfpathlineto{\pgfqpoint{0.719158in}{2.631176in}}%
\pgfpathlineto{\pgfqpoint{0.723276in}{2.645232in}}%
\pgfpathlineto{\pgfqpoint{0.727394in}{2.645232in}}%
\pgfpathlineto{\pgfqpoint{0.731512in}{2.659288in}}%
\pgfpathlineto{\pgfqpoint{0.743866in}{2.659288in}}%
\pgfpathlineto{\pgfqpoint{0.747984in}{2.645232in}}%
\pgfpathlineto{\pgfqpoint{0.752102in}{2.659288in}}%
\pgfpathlineto{\pgfqpoint{0.760339in}{2.659288in}}%
\pgfpathlineto{\pgfqpoint{0.772693in}{2.701457in}}%
\pgfpathlineto{\pgfqpoint{0.776811in}{2.729569in}}%
\pgfpathlineto{\pgfqpoint{0.780929in}{2.743625in}}%
\pgfpathlineto{\pgfqpoint{0.785047in}{2.771738in}}%
\pgfpathlineto{\pgfqpoint{0.793283in}{2.799850in}}%
\pgfpathlineto{\pgfqpoint{0.797401in}{2.799850in}}%
\pgfpathlineto{\pgfqpoint{0.801519in}{2.813906in}}%
\pgfpathlineto{\pgfqpoint{0.826228in}{2.813906in}}%
\pgfpathlineto{\pgfqpoint{0.830346in}{2.827963in}}%
\pgfpathlineto{\pgfqpoint{0.842700in}{2.827963in}}%
\pgfpathlineto{\pgfqpoint{0.846818in}{2.813906in}}%
\pgfpathlineto{\pgfqpoint{0.850936in}{2.785794in}}%
\pgfpathlineto{\pgfqpoint{0.855054in}{2.799850in}}%
\pgfpathlineto{\pgfqpoint{0.867409in}{2.799850in}}%
\pgfpathlineto{\pgfqpoint{0.871527in}{2.771738in}}%
\pgfpathlineto{\pgfqpoint{0.879763in}{2.771738in}}%
\pgfpathlineto{\pgfqpoint{0.883881in}{2.785794in}}%
\pgfpathlineto{\pgfqpoint{0.887999in}{2.785794in}}%
\pgfpathlineto{\pgfqpoint{0.892117in}{2.771738in}}%
\pgfpathlineto{\pgfqpoint{0.900353in}{2.771738in}}%
\pgfpathlineto{\pgfqpoint{0.904471in}{2.785794in}}%
\pgfpathlineto{\pgfqpoint{0.929180in}{2.785794in}}%
\pgfpathlineto{\pgfqpoint{0.933298in}{2.757682in}}%
\pgfpathlineto{\pgfqpoint{0.949770in}{2.757682in}}%
\pgfpathlineto{\pgfqpoint{0.953888in}{2.729569in}}%
\pgfpathlineto{\pgfqpoint{0.958007in}{2.715513in}}%
\pgfpathlineto{\pgfqpoint{0.962125in}{2.715513in}}%
\pgfpathlineto{\pgfqpoint{0.966243in}{2.701457in}}%
\pgfpathlineto{\pgfqpoint{0.970361in}{2.715513in}}%
\pgfpathlineto{\pgfqpoint{0.978597in}{2.687400in}}%
\pgfpathlineto{\pgfqpoint{0.986833in}{2.687400in}}%
\pgfpathlineto{\pgfqpoint{0.990951in}{2.701457in}}%
\pgfpathlineto{\pgfqpoint{0.995069in}{2.701457in}}%
\pgfpathlineto{\pgfqpoint{1.007424in}{2.743625in}}%
\pgfpathlineto{\pgfqpoint{1.015660in}{2.687400in}}%
\pgfpathlineto{\pgfqpoint{1.019778in}{2.673344in}}%
\pgfpathlineto{\pgfqpoint{1.028014in}{2.673344in}}%
\pgfpathlineto{\pgfqpoint{1.032132in}{2.645232in}}%
\pgfpathlineto{\pgfqpoint{1.036250in}{2.631176in}}%
\pgfpathlineto{\pgfqpoint{1.040368in}{2.631176in}}%
\pgfpathlineto{\pgfqpoint{1.044486in}{2.645232in}}%
\pgfpathlineto{\pgfqpoint{1.048604in}{2.617119in}}%
\pgfpathlineto{\pgfqpoint{1.056841in}{2.617119in}}%
\pgfpathlineto{\pgfqpoint{1.060959in}{2.603063in}}%
\pgfpathlineto{\pgfqpoint{1.065077in}{2.617119in}}%
\pgfpathlineto{\pgfqpoint{1.077431in}{2.617119in}}%
\pgfpathlineto{\pgfqpoint{1.081549in}{2.659288in}}%
\pgfpathlineto{\pgfqpoint{1.085667in}{2.659288in}}%
\pgfpathlineto{\pgfqpoint{1.089785in}{2.631176in}}%
\pgfpathlineto{\pgfqpoint{1.093903in}{2.617119in}}%
\pgfpathlineto{\pgfqpoint{1.106258in}{2.617119in}}%
\pgfpathlineto{\pgfqpoint{1.110376in}{2.603063in}}%
\pgfpathlineto{\pgfqpoint{1.114494in}{2.574951in}}%
\pgfpathlineto{\pgfqpoint{1.122730in}{2.546838in}}%
\pgfpathlineto{\pgfqpoint{1.126848in}{2.574951in}}%
\pgfpathlineto{\pgfqpoint{1.130966in}{2.546838in}}%
\pgfpathlineto{\pgfqpoint{1.135084in}{2.574951in}}%
\pgfpathlineto{\pgfqpoint{1.139202in}{2.560894in}}%
\pgfpathlineto{\pgfqpoint{1.143320in}{2.560894in}}%
\pgfpathlineto{\pgfqpoint{1.147438in}{2.546838in}}%
\pgfpathlineto{\pgfqpoint{1.151557in}{2.560894in}}%
\pgfpathlineto{\pgfqpoint{1.159793in}{2.560894in}}%
\pgfpathlineto{\pgfqpoint{1.168029in}{2.532782in}}%
\pgfpathlineto{\pgfqpoint{1.172147in}{2.560894in}}%
\pgfpathlineto{\pgfqpoint{1.176265in}{2.574951in}}%
\pgfpathlineto{\pgfqpoint{1.180383in}{2.574951in}}%
\pgfpathlineto{\pgfqpoint{1.184501in}{2.589007in}}%
\pgfpathlineto{\pgfqpoint{1.188619in}{2.560894in}}%
\pgfpathlineto{\pgfqpoint{1.192737in}{2.560894in}}%
\pgfpathlineto{\pgfqpoint{1.196855in}{2.546838in}}%
\pgfpathlineto{\pgfqpoint{1.209210in}{2.589007in}}%
\pgfpathlineto{\pgfqpoint{1.213328in}{2.574951in}}%
\pgfpathlineto{\pgfqpoint{1.217446in}{2.589007in}}%
\pgfpathlineto{\pgfqpoint{1.225682in}{2.589007in}}%
\pgfpathlineto{\pgfqpoint{1.229800in}{2.574951in}}%
\pgfpathlineto{\pgfqpoint{1.242154in}{2.574951in}}%
\pgfpathlineto{\pgfqpoint{1.246273in}{2.589007in}}%
\pgfpathlineto{\pgfqpoint{1.258627in}{2.589007in}}%
\pgfpathlineto{\pgfqpoint{1.262745in}{2.574951in}}%
\pgfpathlineto{\pgfqpoint{1.266863in}{2.574951in}}%
\pgfpathlineto{\pgfqpoint{1.270981in}{2.560894in}}%
\pgfpathlineto{\pgfqpoint{1.275099in}{2.504669in}}%
\pgfpathlineto{\pgfqpoint{1.283335in}{2.504669in}}%
\pgfpathlineto{\pgfqpoint{1.287453in}{2.532782in}}%
\pgfpathlineto{\pgfqpoint{1.291571in}{2.546838in}}%
\pgfpathlineto{\pgfqpoint{1.295690in}{2.574951in}}%
\pgfpathlineto{\pgfqpoint{1.299808in}{2.560894in}}%
\pgfpathlineto{\pgfqpoint{1.303926in}{2.560894in}}%
\pgfpathlineto{\pgfqpoint{1.308044in}{2.574951in}}%
\pgfpathlineto{\pgfqpoint{1.312162in}{2.603063in}}%
\pgfpathlineto{\pgfqpoint{1.320398in}{2.574951in}}%
\pgfpathlineto{\pgfqpoint{1.324516in}{2.574951in}}%
\pgfpathlineto{\pgfqpoint{1.328634in}{2.546838in}}%
\pgfpathlineto{\pgfqpoint{1.336870in}{2.546838in}}%
\pgfpathlineto{\pgfqpoint{1.340988in}{2.532782in}}%
\pgfpathlineto{\pgfqpoint{1.398642in}{2.532782in}}%
\pgfpathlineto{\pgfqpoint{1.406878in}{2.560894in}}%
\pgfpathlineto{\pgfqpoint{1.472767in}{2.560894in}}%
\pgfpathlineto{\pgfqpoint{1.476885in}{2.574951in}}%
\pgfpathlineto{\pgfqpoint{1.481003in}{2.574951in}}%
\pgfpathlineto{\pgfqpoint{1.489240in}{2.603063in}}%
\pgfpathlineto{\pgfqpoint{1.497476in}{2.603063in}}%
\pgfpathlineto{\pgfqpoint{1.501594in}{2.631176in}}%
\pgfpathlineto{\pgfqpoint{1.518066in}{2.631176in}}%
\pgfpathlineto{\pgfqpoint{1.522184in}{2.659288in}}%
\pgfpathlineto{\pgfqpoint{1.526302in}{2.645232in}}%
\pgfpathlineto{\pgfqpoint{1.530420in}{2.673344in}}%
\pgfpathlineto{\pgfqpoint{1.534538in}{2.673344in}}%
\pgfpathlineto{\pgfqpoint{1.538657in}{2.659288in}}%
\pgfpathlineto{\pgfqpoint{1.542775in}{2.673344in}}%
\pgfpathlineto{\pgfqpoint{1.559247in}{2.673344in}}%
\pgfpathlineto{\pgfqpoint{1.563365in}{2.687400in}}%
\pgfpathlineto{\pgfqpoint{1.567483in}{2.645232in}}%
\pgfpathlineto{\pgfqpoint{1.571601in}{2.673344in}}%
\pgfpathlineto{\pgfqpoint{1.575719in}{2.673344in}}%
\pgfpathlineto{\pgfqpoint{1.583955in}{2.617119in}}%
\pgfpathlineto{\pgfqpoint{1.588074in}{2.617119in}}%
\pgfpathlineto{\pgfqpoint{1.592192in}{2.589007in}}%
\pgfpathlineto{\pgfqpoint{1.596310in}{2.603063in}}%
\pgfpathlineto{\pgfqpoint{1.600428in}{2.574951in}}%
\pgfpathlineto{\pgfqpoint{1.608664in}{2.546838in}}%
\pgfpathlineto{\pgfqpoint{1.629254in}{2.546838in}}%
\pgfpathlineto{\pgfqpoint{1.633372in}{2.560894in}}%
\pgfpathlineto{\pgfqpoint{1.637491in}{2.546838in}}%
\pgfpathlineto{\pgfqpoint{1.711616in}{2.546838in}}%
\pgfpathlineto{\pgfqpoint{1.715734in}{2.532782in}}%
\pgfpathlineto{\pgfqpoint{1.723970in}{2.532782in}}%
\pgfpathlineto{\pgfqpoint{1.728088in}{2.546838in}}%
\pgfpathlineto{\pgfqpoint{1.901048in}{2.546838in}}%
\pgfpathlineto{\pgfqpoint{1.905166in}{2.560894in}}%
\pgfpathlineto{\pgfqpoint{1.938111in}{2.560894in}}%
\pgfpathlineto{\pgfqpoint{1.942229in}{2.546838in}}%
\pgfpathlineto{\pgfqpoint{2.160487in}{2.546838in}}%
\pgfpathlineto{\pgfqpoint{2.164605in}{2.532782in}}%
\pgfpathlineto{\pgfqpoint{2.329329in}{2.532782in}}%
\pgfpathlineto{\pgfqpoint{2.333447in}{2.546838in}}%
\pgfpathlineto{\pgfqpoint{2.646421in}{2.546838in}}%
\pgfpathlineto{\pgfqpoint{2.646421in}{2.546838in}}%
\pgfusepath{stroke}%
\end{pgfscope}%
\begin{pgfscope}%
\pgfpathrectangle{\pgfqpoint{0.488751in}{2.165212in}}{\pgfqpoint{2.260417in}{1.283333in}}%
\pgfusepath{clip}%
\pgfsetbuttcap%
\pgfsetroundjoin%
\pgfsetlinewidth{0.803000pt}%
\definecolor{currentstroke}{rgb}{0.000000,0.356863,0.509804}%
\pgfsetstrokecolor{currentstroke}%
\pgfsetstrokeopacity{0.270000}%
\pgfsetdash{{2.960000pt}{1.280000pt}}{0.000000pt}%
\pgfpathmoveto{\pgfqpoint{0.591497in}{2.546838in}}%
\pgfpathlineto{\pgfqpoint{0.595615in}{2.546838in}}%
\pgfpathlineto{\pgfqpoint{0.599733in}{2.603063in}}%
\pgfpathlineto{\pgfqpoint{0.607969in}{2.603063in}}%
\pgfpathlineto{\pgfqpoint{0.612087in}{2.546838in}}%
\pgfpathlineto{\pgfqpoint{0.616206in}{2.518726in}}%
\pgfpathlineto{\pgfqpoint{0.620324in}{2.504669in}}%
\pgfpathlineto{\pgfqpoint{0.624442in}{2.504669in}}%
\pgfpathlineto{\pgfqpoint{0.628560in}{2.532782in}}%
\pgfpathlineto{\pgfqpoint{0.632678in}{2.546838in}}%
\pgfpathlineto{\pgfqpoint{0.636796in}{2.532782in}}%
\pgfpathlineto{\pgfqpoint{0.640914in}{2.546838in}}%
\pgfpathlineto{\pgfqpoint{0.645032in}{2.532782in}}%
\pgfpathlineto{\pgfqpoint{0.649150in}{2.560894in}}%
\pgfpathlineto{\pgfqpoint{0.653268in}{2.532782in}}%
\pgfpathlineto{\pgfqpoint{0.657386in}{2.546838in}}%
\pgfpathlineto{\pgfqpoint{0.661504in}{2.532782in}}%
\pgfpathlineto{\pgfqpoint{0.665623in}{2.490613in}}%
\pgfpathlineto{\pgfqpoint{0.698567in}{2.490613in}}%
\pgfpathlineto{\pgfqpoint{0.702685in}{2.504669in}}%
\pgfpathlineto{\pgfqpoint{0.710922in}{2.476557in}}%
\pgfpathlineto{\pgfqpoint{0.739748in}{2.476557in}}%
\pgfpathlineto{\pgfqpoint{0.743866in}{2.490613in}}%
\pgfpathlineto{\pgfqpoint{0.772693in}{2.490613in}}%
\pgfpathlineto{\pgfqpoint{0.776811in}{2.476557in}}%
\pgfpathlineto{\pgfqpoint{0.789165in}{2.476557in}}%
\pgfpathlineto{\pgfqpoint{0.793283in}{2.490613in}}%
\pgfpathlineto{\pgfqpoint{0.809756in}{2.490613in}}%
\pgfpathlineto{\pgfqpoint{0.813874in}{2.504669in}}%
\pgfpathlineto{\pgfqpoint{0.826228in}{2.504669in}}%
\pgfpathlineto{\pgfqpoint{0.830346in}{2.518726in}}%
\pgfpathlineto{\pgfqpoint{0.838582in}{2.490613in}}%
\pgfpathlineto{\pgfqpoint{0.850936in}{2.490613in}}%
\pgfpathlineto{\pgfqpoint{0.855054in}{2.476557in}}%
\pgfpathlineto{\pgfqpoint{0.871527in}{2.476557in}}%
\pgfpathlineto{\pgfqpoint{0.879763in}{2.448445in}}%
\pgfpathlineto{\pgfqpoint{0.883881in}{2.462501in}}%
\pgfpathlineto{\pgfqpoint{0.892117in}{2.462501in}}%
\pgfpathlineto{\pgfqpoint{0.896235in}{2.448445in}}%
\pgfpathlineto{\pgfqpoint{0.912708in}{2.448445in}}%
\pgfpathlineto{\pgfqpoint{0.916826in}{2.434388in}}%
\pgfpathlineto{\pgfqpoint{0.941534in}{2.434388in}}%
\pgfpathlineto{\pgfqpoint{0.945652in}{2.420332in}}%
\pgfpathlineto{\pgfqpoint{0.953888in}{2.420332in}}%
\pgfpathlineto{\pgfqpoint{0.958007in}{2.406276in}}%
\pgfpathlineto{\pgfqpoint{0.974479in}{2.406276in}}%
\pgfpathlineto{\pgfqpoint{0.978597in}{2.420332in}}%
\pgfpathlineto{\pgfqpoint{0.982715in}{2.420332in}}%
\pgfpathlineto{\pgfqpoint{0.986833in}{2.434388in}}%
\pgfpathlineto{\pgfqpoint{0.990951in}{2.434388in}}%
\pgfpathlineto{\pgfqpoint{0.995069in}{2.420332in}}%
\pgfpathlineto{\pgfqpoint{1.003306in}{2.420332in}}%
\pgfpathlineto{\pgfqpoint{1.007424in}{2.406276in}}%
\pgfpathlineto{\pgfqpoint{1.011542in}{2.406276in}}%
\pgfpathlineto{\pgfqpoint{1.015660in}{2.392220in}}%
\pgfpathlineto{\pgfqpoint{1.036250in}{2.392220in}}%
\pgfpathlineto{\pgfqpoint{1.040368in}{2.406276in}}%
\pgfpathlineto{\pgfqpoint{1.044486in}{2.378163in}}%
\pgfpathlineto{\pgfqpoint{1.048604in}{2.364107in}}%
\pgfpathlineto{\pgfqpoint{1.081549in}{2.364107in}}%
\pgfpathlineto{\pgfqpoint{1.085667in}{2.392220in}}%
\pgfpathlineto{\pgfqpoint{1.089785in}{2.406276in}}%
\pgfpathlineto{\pgfqpoint{1.102140in}{2.406276in}}%
\pgfpathlineto{\pgfqpoint{1.106258in}{2.392220in}}%
\pgfpathlineto{\pgfqpoint{1.114494in}{2.392220in}}%
\pgfpathlineto{\pgfqpoint{1.118612in}{2.406276in}}%
\pgfpathlineto{\pgfqpoint{1.122730in}{2.406276in}}%
\pgfpathlineto{\pgfqpoint{1.126848in}{2.392220in}}%
\pgfpathlineto{\pgfqpoint{1.163911in}{2.392220in}}%
\pgfpathlineto{\pgfqpoint{1.168029in}{2.406276in}}%
\pgfpathlineto{\pgfqpoint{1.180383in}{2.406276in}}%
\pgfpathlineto{\pgfqpoint{1.184501in}{2.420332in}}%
\pgfpathlineto{\pgfqpoint{1.192737in}{2.420332in}}%
\pgfpathlineto{\pgfqpoint{1.200974in}{2.392220in}}%
\pgfpathlineto{\pgfqpoint{1.225682in}{2.392220in}}%
\pgfpathlineto{\pgfqpoint{1.229800in}{2.406276in}}%
\pgfpathlineto{\pgfqpoint{1.242154in}{2.406276in}}%
\pgfpathlineto{\pgfqpoint{1.246273in}{2.392220in}}%
\pgfpathlineto{\pgfqpoint{1.287453in}{2.392220in}}%
\pgfpathlineto{\pgfqpoint{1.291571in}{2.406276in}}%
\pgfpathlineto{\pgfqpoint{1.308044in}{2.406276in}}%
\pgfpathlineto{\pgfqpoint{1.312162in}{2.434388in}}%
\pgfpathlineto{\pgfqpoint{1.316280in}{2.434388in}}%
\pgfpathlineto{\pgfqpoint{1.320398in}{2.448445in}}%
\pgfpathlineto{\pgfqpoint{1.336870in}{2.448445in}}%
\pgfpathlineto{\pgfqpoint{1.345107in}{2.476557in}}%
\pgfpathlineto{\pgfqpoint{1.353343in}{2.476557in}}%
\pgfpathlineto{\pgfqpoint{1.357461in}{2.490613in}}%
\pgfpathlineto{\pgfqpoint{1.382169in}{2.490613in}}%
\pgfpathlineto{\pgfqpoint{1.386287in}{2.476557in}}%
\pgfpathlineto{\pgfqpoint{1.390405in}{2.504669in}}%
\pgfpathlineto{\pgfqpoint{1.398642in}{2.476557in}}%
\pgfpathlineto{\pgfqpoint{1.402760in}{2.490613in}}%
\pgfpathlineto{\pgfqpoint{1.423350in}{2.490613in}}%
\pgfpathlineto{\pgfqpoint{1.427468in}{2.476557in}}%
\pgfpathlineto{\pgfqpoint{1.435704in}{2.476557in}}%
\pgfpathlineto{\pgfqpoint{1.439822in}{2.490613in}}%
\pgfpathlineto{\pgfqpoint{1.443941in}{2.490613in}}%
\pgfpathlineto{\pgfqpoint{1.448059in}{2.504669in}}%
\pgfpathlineto{\pgfqpoint{1.452177in}{2.462501in}}%
\pgfpathlineto{\pgfqpoint{1.472767in}{2.462501in}}%
\pgfpathlineto{\pgfqpoint{1.476885in}{2.448445in}}%
\pgfpathlineto{\pgfqpoint{1.505712in}{2.448445in}}%
\pgfpathlineto{\pgfqpoint{1.509830in}{2.434388in}}%
\pgfpathlineto{\pgfqpoint{1.522184in}{2.434388in}}%
\pgfpathlineto{\pgfqpoint{1.530420in}{2.406276in}}%
\pgfpathlineto{\pgfqpoint{1.551011in}{2.406276in}}%
\pgfpathlineto{\pgfqpoint{1.555129in}{2.392220in}}%
\pgfpathlineto{\pgfqpoint{1.579837in}{2.392220in}}%
\pgfpathlineto{\pgfqpoint{1.583955in}{2.406276in}}%
\pgfpathlineto{\pgfqpoint{1.604546in}{2.406276in}}%
\pgfpathlineto{\pgfqpoint{1.608664in}{2.392220in}}%
\pgfpathlineto{\pgfqpoint{1.616900in}{2.392220in}}%
\pgfpathlineto{\pgfqpoint{1.621018in}{2.406276in}}%
\pgfpathlineto{\pgfqpoint{1.629254in}{2.406276in}}%
\pgfpathlineto{\pgfqpoint{1.633372in}{2.392220in}}%
\pgfpathlineto{\pgfqpoint{1.637491in}{2.406276in}}%
\pgfpathlineto{\pgfqpoint{1.658081in}{2.406276in}}%
\pgfpathlineto{\pgfqpoint{1.662199in}{2.420332in}}%
\pgfpathlineto{\pgfqpoint{1.686908in}{2.420332in}}%
\pgfpathlineto{\pgfqpoint{1.691026in}{2.406276in}}%
\pgfpathlineto{\pgfqpoint{1.695144in}{2.406276in}}%
\pgfpathlineto{\pgfqpoint{1.699262in}{2.392220in}}%
\pgfpathlineto{\pgfqpoint{1.728088in}{2.392220in}}%
\pgfpathlineto{\pgfqpoint{1.732206in}{2.406276in}}%
\pgfpathlineto{\pgfqpoint{1.740443in}{2.406276in}}%
\pgfpathlineto{\pgfqpoint{1.744561in}{2.392220in}}%
\pgfpathlineto{\pgfqpoint{1.773387in}{2.392220in}}%
\pgfpathlineto{\pgfqpoint{1.777505in}{2.406276in}}%
\pgfpathlineto{\pgfqpoint{1.802214in}{2.406276in}}%
\pgfpathlineto{\pgfqpoint{1.806332in}{2.392220in}}%
\pgfpathlineto{\pgfqpoint{1.876339in}{2.392220in}}%
\pgfpathlineto{\pgfqpoint{1.880458in}{2.378163in}}%
\pgfpathlineto{\pgfqpoint{1.938111in}{2.378163in}}%
\pgfpathlineto{\pgfqpoint{1.942229in}{2.406276in}}%
\pgfpathlineto{\pgfqpoint{1.946347in}{2.406276in}}%
\pgfpathlineto{\pgfqpoint{1.950465in}{2.392220in}}%
\pgfpathlineto{\pgfqpoint{1.958701in}{2.392220in}}%
\pgfpathlineto{\pgfqpoint{1.962819in}{2.406276in}}%
\pgfpathlineto{\pgfqpoint{2.032827in}{2.406276in}}%
\pgfpathlineto{\pgfqpoint{2.036945in}{2.434388in}}%
\pgfpathlineto{\pgfqpoint{2.045181in}{2.434388in}}%
\pgfpathlineto{\pgfqpoint{2.049299in}{2.448445in}}%
\pgfpathlineto{\pgfqpoint{2.069889in}{2.448445in}}%
\pgfpathlineto{\pgfqpoint{2.074008in}{2.420332in}}%
\pgfpathlineto{\pgfqpoint{2.094598in}{2.420332in}}%
\pgfpathlineto{\pgfqpoint{2.098716in}{2.448445in}}%
\pgfpathlineto{\pgfqpoint{2.102834in}{2.448445in}}%
\pgfpathlineto{\pgfqpoint{2.106952in}{2.434388in}}%
\pgfpathlineto{\pgfqpoint{2.123425in}{2.434388in}}%
\pgfpathlineto{\pgfqpoint{2.127543in}{2.476557in}}%
\pgfpathlineto{\pgfqpoint{2.131661in}{2.476557in}}%
\pgfpathlineto{\pgfqpoint{2.135779in}{2.490613in}}%
\pgfpathlineto{\pgfqpoint{2.148133in}{2.490613in}}%
\pgfpathlineto{\pgfqpoint{2.152251in}{2.504669in}}%
\pgfpathlineto{\pgfqpoint{2.156369in}{2.504669in}}%
\pgfpathlineto{\pgfqpoint{2.160487in}{2.490613in}}%
\pgfpathlineto{\pgfqpoint{2.164605in}{2.490613in}}%
\pgfpathlineto{\pgfqpoint{2.168723in}{2.504669in}}%
\pgfpathlineto{\pgfqpoint{2.193432in}{2.504669in}}%
\pgfpathlineto{\pgfqpoint{2.201668in}{2.532782in}}%
\pgfpathlineto{\pgfqpoint{2.205786in}{2.490613in}}%
\pgfpathlineto{\pgfqpoint{2.209904in}{2.476557in}}%
\pgfpathlineto{\pgfqpoint{2.234613in}{2.476557in}}%
\pgfpathlineto{\pgfqpoint{2.238731in}{2.462501in}}%
\pgfpathlineto{\pgfqpoint{2.267557in}{2.462501in}}%
\pgfpathlineto{\pgfqpoint{2.271676in}{2.448445in}}%
\pgfpathlineto{\pgfqpoint{2.288148in}{2.448445in}}%
\pgfpathlineto{\pgfqpoint{2.292266in}{2.462501in}}%
\pgfpathlineto{\pgfqpoint{2.296384in}{2.448445in}}%
\pgfpathlineto{\pgfqpoint{2.300502in}{2.476557in}}%
\pgfpathlineto{\pgfqpoint{2.304620in}{2.462501in}}%
\pgfpathlineto{\pgfqpoint{2.308738in}{2.490613in}}%
\pgfpathlineto{\pgfqpoint{2.316975in}{2.490613in}}%
\pgfpathlineto{\pgfqpoint{2.321093in}{2.504669in}}%
\pgfpathlineto{\pgfqpoint{2.325211in}{2.504669in}}%
\pgfpathlineto{\pgfqpoint{2.329329in}{2.518726in}}%
\pgfpathlineto{\pgfqpoint{2.337565in}{2.490613in}}%
\pgfpathlineto{\pgfqpoint{2.341683in}{2.504669in}}%
\pgfpathlineto{\pgfqpoint{2.345801in}{2.532782in}}%
\pgfpathlineto{\pgfqpoint{2.354037in}{2.504669in}}%
\pgfpathlineto{\pgfqpoint{2.358155in}{2.504669in}}%
\pgfpathlineto{\pgfqpoint{2.362273in}{2.490613in}}%
\pgfpathlineto{\pgfqpoint{2.386982in}{2.490613in}}%
\pgfpathlineto{\pgfqpoint{2.391100in}{2.476557in}}%
\pgfpathlineto{\pgfqpoint{2.395218in}{2.476557in}}%
\pgfpathlineto{\pgfqpoint{2.403454in}{2.448445in}}%
\pgfpathlineto{\pgfqpoint{2.419927in}{2.448445in}}%
\pgfpathlineto{\pgfqpoint{2.424045in}{2.434388in}}%
\pgfpathlineto{\pgfqpoint{2.428163in}{2.462501in}}%
\pgfpathlineto{\pgfqpoint{2.432281in}{2.462501in}}%
\pgfpathlineto{\pgfqpoint{2.436399in}{2.476557in}}%
\pgfpathlineto{\pgfqpoint{2.448753in}{2.476557in}}%
\pgfpathlineto{\pgfqpoint{2.452871in}{2.462501in}}%
\pgfpathlineto{\pgfqpoint{2.456989in}{2.434388in}}%
\pgfpathlineto{\pgfqpoint{2.461107in}{2.434388in}}%
\pgfpathlineto{\pgfqpoint{2.465226in}{2.420332in}}%
\pgfpathlineto{\pgfqpoint{2.485816in}{2.420332in}}%
\pgfpathlineto{\pgfqpoint{2.489934in}{2.392220in}}%
\pgfpathlineto{\pgfqpoint{2.498170in}{2.392220in}}%
\pgfpathlineto{\pgfqpoint{2.502288in}{2.406276in}}%
\pgfpathlineto{\pgfqpoint{2.518761in}{2.406276in}}%
\pgfpathlineto{\pgfqpoint{2.522879in}{2.392220in}}%
\pgfpathlineto{\pgfqpoint{2.535233in}{2.392220in}}%
\pgfpathlineto{\pgfqpoint{2.539351in}{2.378163in}}%
\pgfpathlineto{\pgfqpoint{2.543469in}{2.378163in}}%
\pgfpathlineto{\pgfqpoint{2.547587in}{2.364107in}}%
\pgfpathlineto{\pgfqpoint{2.555823in}{2.364107in}}%
\pgfpathlineto{\pgfqpoint{2.559942in}{2.350051in}}%
\pgfpathlineto{\pgfqpoint{2.568178in}{2.350051in}}%
\pgfpathlineto{\pgfqpoint{2.572296in}{2.364107in}}%
\pgfpathlineto{\pgfqpoint{2.584650in}{2.364107in}}%
\pgfpathlineto{\pgfqpoint{2.588768in}{2.378163in}}%
\pgfpathlineto{\pgfqpoint{2.597004in}{2.378163in}}%
\pgfpathlineto{\pgfqpoint{2.601122in}{2.364107in}}%
\pgfpathlineto{\pgfqpoint{2.617595in}{2.364107in}}%
\pgfpathlineto{\pgfqpoint{2.621713in}{2.350051in}}%
\pgfpathlineto{\pgfqpoint{2.625831in}{2.350051in}}%
\pgfpathlineto{\pgfqpoint{2.629949in}{2.364107in}}%
\pgfpathlineto{\pgfqpoint{2.646421in}{2.364107in}}%
\pgfpathlineto{\pgfqpoint{2.646421in}{2.364107in}}%
\pgfusepath{stroke}%
\end{pgfscope}%
\begin{pgfscope}%
\pgfpathrectangle{\pgfqpoint{0.488751in}{2.165212in}}{\pgfqpoint{2.260417in}{1.283333in}}%
\pgfusepath{clip}%
\pgfsetbuttcap%
\pgfsetroundjoin%
\pgfsetlinewidth{0.803000pt}%
\definecolor{currentstroke}{rgb}{0.490196,0.588235,0.431373}%
\pgfsetstrokecolor{currentstroke}%
\pgfsetstrokeopacity{0.270000}%
\pgfsetdash{{2.960000pt}{1.280000pt}}{0.000000pt}%
\pgfpathmoveto{\pgfqpoint{0.591497in}{3.109087in}}%
\pgfpathlineto{\pgfqpoint{0.595615in}{3.165312in}}%
\pgfpathlineto{\pgfqpoint{0.599733in}{3.193425in}}%
\pgfpathlineto{\pgfqpoint{0.603851in}{3.179368in}}%
\pgfpathlineto{\pgfqpoint{0.607969in}{3.221537in}}%
\pgfpathlineto{\pgfqpoint{0.612087in}{3.235593in}}%
\pgfpathlineto{\pgfqpoint{0.616206in}{3.193425in}}%
\pgfpathlineto{\pgfqpoint{0.620324in}{3.179368in}}%
\pgfpathlineto{\pgfqpoint{0.624442in}{3.179368in}}%
\pgfpathlineto{\pgfqpoint{0.628560in}{3.221537in}}%
\pgfpathlineto{\pgfqpoint{0.645032in}{3.333987in}}%
\pgfpathlineto{\pgfqpoint{0.649150in}{3.390212in}}%
\pgfpathlineto{\pgfqpoint{0.653268in}{3.376155in}}%
\pgfpathlineto{\pgfqpoint{0.657386in}{3.390212in}}%
\pgfpathlineto{\pgfqpoint{0.665623in}{3.362099in}}%
\pgfpathlineto{\pgfqpoint{0.669741in}{3.376155in}}%
\pgfpathlineto{\pgfqpoint{0.682095in}{3.376155in}}%
\pgfpathlineto{\pgfqpoint{0.686213in}{3.348043in}}%
\pgfpathlineto{\pgfqpoint{0.690331in}{3.333987in}}%
\pgfpathlineto{\pgfqpoint{0.698567in}{3.333987in}}%
\pgfpathlineto{\pgfqpoint{0.702685in}{3.348043in}}%
\pgfpathlineto{\pgfqpoint{0.706803in}{3.333987in}}%
\pgfpathlineto{\pgfqpoint{0.727394in}{3.333987in}}%
\pgfpathlineto{\pgfqpoint{0.731512in}{3.319931in}}%
\pgfpathlineto{\pgfqpoint{0.735630in}{3.333987in}}%
\pgfpathlineto{\pgfqpoint{0.739748in}{3.291818in}}%
\pgfpathlineto{\pgfqpoint{0.743866in}{3.277762in}}%
\pgfpathlineto{\pgfqpoint{0.747984in}{3.249649in}}%
\pgfpathlineto{\pgfqpoint{0.752102in}{3.235593in}}%
\pgfpathlineto{\pgfqpoint{0.756220in}{3.207481in}}%
\pgfpathlineto{\pgfqpoint{0.760339in}{3.193425in}}%
\pgfpathlineto{\pgfqpoint{0.764457in}{3.151256in}}%
\pgfpathlineto{\pgfqpoint{0.768575in}{3.151256in}}%
\pgfpathlineto{\pgfqpoint{0.772693in}{3.123143in}}%
\pgfpathlineto{\pgfqpoint{0.780929in}{3.095031in}}%
\pgfpathlineto{\pgfqpoint{0.785047in}{3.109087in}}%
\pgfpathlineto{\pgfqpoint{0.789165in}{3.109087in}}%
\pgfpathlineto{\pgfqpoint{0.793283in}{3.095031in}}%
\pgfpathlineto{\pgfqpoint{0.797401in}{3.109087in}}%
\pgfpathlineto{\pgfqpoint{0.801519in}{3.095031in}}%
\pgfpathlineto{\pgfqpoint{0.805637in}{3.095031in}}%
\pgfpathlineto{\pgfqpoint{0.809756in}{3.123143in}}%
\pgfpathlineto{\pgfqpoint{0.813874in}{3.123143in}}%
\pgfpathlineto{\pgfqpoint{0.817992in}{3.137200in}}%
\pgfpathlineto{\pgfqpoint{0.822110in}{3.137200in}}%
\pgfpathlineto{\pgfqpoint{0.826228in}{3.151256in}}%
\pgfpathlineto{\pgfqpoint{0.830346in}{3.151256in}}%
\pgfpathlineto{\pgfqpoint{0.834464in}{3.137200in}}%
\pgfpathlineto{\pgfqpoint{0.838582in}{3.151256in}}%
\pgfpathlineto{\pgfqpoint{0.859173in}{3.151256in}}%
\pgfpathlineto{\pgfqpoint{0.863291in}{3.165312in}}%
\pgfpathlineto{\pgfqpoint{0.867409in}{3.165312in}}%
\pgfpathlineto{\pgfqpoint{0.871527in}{3.151256in}}%
\pgfpathlineto{\pgfqpoint{0.875645in}{3.151256in}}%
\pgfpathlineto{\pgfqpoint{0.879763in}{3.137200in}}%
\pgfpathlineto{\pgfqpoint{0.883881in}{3.137200in}}%
\pgfpathlineto{\pgfqpoint{0.887999in}{3.151256in}}%
\pgfpathlineto{\pgfqpoint{0.892117in}{3.179368in}}%
\pgfpathlineto{\pgfqpoint{0.896235in}{3.165312in}}%
\pgfpathlineto{\pgfqpoint{0.900353in}{3.179368in}}%
\pgfpathlineto{\pgfqpoint{0.904471in}{3.179368in}}%
\pgfpathlineto{\pgfqpoint{0.908590in}{3.151256in}}%
\pgfpathlineto{\pgfqpoint{0.916826in}{3.151256in}}%
\pgfpathlineto{\pgfqpoint{0.925062in}{3.123143in}}%
\pgfpathlineto{\pgfqpoint{0.945652in}{3.193425in}}%
\pgfpathlineto{\pgfqpoint{0.949770in}{3.179368in}}%
\pgfpathlineto{\pgfqpoint{0.953888in}{3.193425in}}%
\pgfpathlineto{\pgfqpoint{0.958007in}{3.165312in}}%
\pgfpathlineto{\pgfqpoint{0.966243in}{3.137200in}}%
\pgfpathlineto{\pgfqpoint{0.970361in}{3.109087in}}%
\pgfpathlineto{\pgfqpoint{0.974479in}{3.095031in}}%
\pgfpathlineto{\pgfqpoint{0.982715in}{3.123143in}}%
\pgfpathlineto{\pgfqpoint{0.990951in}{3.123143in}}%
\pgfpathlineto{\pgfqpoint{0.995069in}{3.109087in}}%
\pgfpathlineto{\pgfqpoint{0.999187in}{3.109087in}}%
\pgfpathlineto{\pgfqpoint{1.003306in}{3.080975in}}%
\pgfpathlineto{\pgfqpoint{1.023896in}{3.151256in}}%
\pgfpathlineto{\pgfqpoint{1.028014in}{3.151256in}}%
\pgfpathlineto{\pgfqpoint{1.032132in}{3.165312in}}%
\pgfpathlineto{\pgfqpoint{1.040368in}{3.165312in}}%
\pgfpathlineto{\pgfqpoint{1.044486in}{3.151256in}}%
\pgfpathlineto{\pgfqpoint{1.048604in}{3.165312in}}%
\pgfpathlineto{\pgfqpoint{1.052723in}{3.165312in}}%
\pgfpathlineto{\pgfqpoint{1.056841in}{3.151256in}}%
\pgfpathlineto{\pgfqpoint{1.065077in}{3.151256in}}%
\pgfpathlineto{\pgfqpoint{1.069195in}{3.165312in}}%
\pgfpathlineto{\pgfqpoint{1.073313in}{3.151256in}}%
\pgfpathlineto{\pgfqpoint{1.077431in}{3.165312in}}%
\pgfpathlineto{\pgfqpoint{1.085667in}{3.165312in}}%
\pgfpathlineto{\pgfqpoint{1.093903in}{3.137200in}}%
\pgfpathlineto{\pgfqpoint{1.098021in}{3.151256in}}%
\pgfpathlineto{\pgfqpoint{1.106258in}{3.151256in}}%
\pgfpathlineto{\pgfqpoint{1.110376in}{3.165312in}}%
\pgfpathlineto{\pgfqpoint{1.122730in}{3.165312in}}%
\pgfpathlineto{\pgfqpoint{1.126848in}{3.151256in}}%
\pgfpathlineto{\pgfqpoint{1.130966in}{3.179368in}}%
\pgfpathlineto{\pgfqpoint{1.135084in}{3.165312in}}%
\pgfpathlineto{\pgfqpoint{1.139202in}{3.207481in}}%
\pgfpathlineto{\pgfqpoint{1.143320in}{3.193425in}}%
\pgfpathlineto{\pgfqpoint{1.147438in}{3.193425in}}%
\pgfpathlineto{\pgfqpoint{1.151557in}{3.179368in}}%
\pgfpathlineto{\pgfqpoint{1.155675in}{3.179368in}}%
\pgfpathlineto{\pgfqpoint{1.159793in}{3.137200in}}%
\pgfpathlineto{\pgfqpoint{1.163911in}{3.137200in}}%
\pgfpathlineto{\pgfqpoint{1.172147in}{3.165312in}}%
\pgfpathlineto{\pgfqpoint{1.176265in}{3.151256in}}%
\pgfpathlineto{\pgfqpoint{1.184501in}{3.151256in}}%
\pgfpathlineto{\pgfqpoint{1.188619in}{3.165312in}}%
\pgfpathlineto{\pgfqpoint{1.192737in}{3.137200in}}%
\pgfpathlineto{\pgfqpoint{1.196855in}{3.137200in}}%
\pgfpathlineto{\pgfqpoint{1.200974in}{3.151256in}}%
\pgfpathlineto{\pgfqpoint{1.205092in}{3.151256in}}%
\pgfpathlineto{\pgfqpoint{1.209210in}{3.137200in}}%
\pgfpathlineto{\pgfqpoint{1.213328in}{3.151256in}}%
\pgfpathlineto{\pgfqpoint{1.225682in}{3.151256in}}%
\pgfpathlineto{\pgfqpoint{1.229800in}{3.165312in}}%
\pgfpathlineto{\pgfqpoint{1.233918in}{3.151256in}}%
\pgfpathlineto{\pgfqpoint{1.250391in}{3.151256in}}%
\pgfpathlineto{\pgfqpoint{1.254509in}{3.165312in}}%
\pgfpathlineto{\pgfqpoint{1.258627in}{3.165312in}}%
\pgfpathlineto{\pgfqpoint{1.262745in}{3.137200in}}%
\pgfpathlineto{\pgfqpoint{1.266863in}{3.137200in}}%
\pgfpathlineto{\pgfqpoint{1.270981in}{3.151256in}}%
\pgfpathlineto{\pgfqpoint{1.275099in}{3.179368in}}%
\pgfpathlineto{\pgfqpoint{1.279217in}{3.165312in}}%
\pgfpathlineto{\pgfqpoint{1.283335in}{3.137200in}}%
\pgfpathlineto{\pgfqpoint{1.287453in}{3.137200in}}%
\pgfpathlineto{\pgfqpoint{1.291571in}{3.151256in}}%
\pgfpathlineto{\pgfqpoint{1.295690in}{3.151256in}}%
\pgfpathlineto{\pgfqpoint{1.299808in}{3.165312in}}%
\pgfpathlineto{\pgfqpoint{1.308044in}{3.165312in}}%
\pgfpathlineto{\pgfqpoint{1.312162in}{3.137200in}}%
\pgfpathlineto{\pgfqpoint{1.316280in}{3.151256in}}%
\pgfpathlineto{\pgfqpoint{1.324516in}{3.151256in}}%
\pgfpathlineto{\pgfqpoint{1.336870in}{3.193425in}}%
\pgfpathlineto{\pgfqpoint{1.340988in}{3.193425in}}%
\pgfpathlineto{\pgfqpoint{1.345107in}{3.165312in}}%
\pgfpathlineto{\pgfqpoint{1.349225in}{3.151256in}}%
\pgfpathlineto{\pgfqpoint{1.353343in}{3.179368in}}%
\pgfpathlineto{\pgfqpoint{1.357461in}{3.193425in}}%
\pgfpathlineto{\pgfqpoint{1.361579in}{3.193425in}}%
\pgfpathlineto{\pgfqpoint{1.373933in}{3.151256in}}%
\pgfpathlineto{\pgfqpoint{1.378051in}{3.165312in}}%
\pgfpathlineto{\pgfqpoint{1.382169in}{3.151256in}}%
\pgfpathlineto{\pgfqpoint{1.386287in}{3.151256in}}%
\pgfpathlineto{\pgfqpoint{1.390405in}{3.179368in}}%
\pgfpathlineto{\pgfqpoint{1.394524in}{3.193425in}}%
\pgfpathlineto{\pgfqpoint{1.398642in}{3.165312in}}%
\pgfpathlineto{\pgfqpoint{1.402760in}{3.151256in}}%
\pgfpathlineto{\pgfqpoint{1.419232in}{3.151256in}}%
\pgfpathlineto{\pgfqpoint{1.423350in}{3.165312in}}%
\pgfpathlineto{\pgfqpoint{1.427468in}{3.151256in}}%
\pgfpathlineto{\pgfqpoint{1.431586in}{3.165312in}}%
\pgfpathlineto{\pgfqpoint{1.435704in}{3.165312in}}%
\pgfpathlineto{\pgfqpoint{1.439822in}{3.151256in}}%
\pgfpathlineto{\pgfqpoint{1.448059in}{3.151256in}}%
\pgfpathlineto{\pgfqpoint{1.452177in}{3.165312in}}%
\pgfpathlineto{\pgfqpoint{1.456295in}{3.151256in}}%
\pgfpathlineto{\pgfqpoint{1.460413in}{3.193425in}}%
\pgfpathlineto{\pgfqpoint{1.464531in}{3.193425in}}%
\pgfpathlineto{\pgfqpoint{1.468649in}{3.207481in}}%
\pgfpathlineto{\pgfqpoint{1.476885in}{3.179368in}}%
\pgfpathlineto{\pgfqpoint{1.481003in}{3.179368in}}%
\pgfpathlineto{\pgfqpoint{1.485121in}{3.151256in}}%
\pgfpathlineto{\pgfqpoint{1.497476in}{3.151256in}}%
\pgfpathlineto{\pgfqpoint{1.501594in}{3.137200in}}%
\pgfpathlineto{\pgfqpoint{1.513948in}{3.137200in}}%
\pgfpathlineto{\pgfqpoint{1.518066in}{3.123143in}}%
\pgfpathlineto{\pgfqpoint{1.522184in}{3.123143in}}%
\pgfpathlineto{\pgfqpoint{1.526302in}{3.109087in}}%
\pgfpathlineto{\pgfqpoint{1.530420in}{3.109087in}}%
\pgfpathlineto{\pgfqpoint{1.534538in}{3.123143in}}%
\pgfpathlineto{\pgfqpoint{1.538657in}{3.123143in}}%
\pgfpathlineto{\pgfqpoint{1.542775in}{3.137200in}}%
\pgfpathlineto{\pgfqpoint{1.563365in}{3.137200in}}%
\pgfpathlineto{\pgfqpoint{1.567483in}{3.179368in}}%
\pgfpathlineto{\pgfqpoint{1.571601in}{3.151256in}}%
\pgfpathlineto{\pgfqpoint{1.588074in}{3.151256in}}%
\pgfpathlineto{\pgfqpoint{1.592192in}{3.165312in}}%
\pgfpathlineto{\pgfqpoint{1.596310in}{3.151256in}}%
\pgfpathlineto{\pgfqpoint{1.600428in}{3.207481in}}%
\pgfpathlineto{\pgfqpoint{1.608664in}{3.207481in}}%
\pgfpathlineto{\pgfqpoint{1.612782in}{3.193425in}}%
\pgfpathlineto{\pgfqpoint{1.616900in}{3.165312in}}%
\pgfpathlineto{\pgfqpoint{1.621018in}{3.165312in}}%
\pgfpathlineto{\pgfqpoint{1.625136in}{3.179368in}}%
\pgfpathlineto{\pgfqpoint{1.629254in}{3.137200in}}%
\pgfpathlineto{\pgfqpoint{1.633372in}{3.137200in}}%
\pgfpathlineto{\pgfqpoint{1.637491in}{3.151256in}}%
\pgfpathlineto{\pgfqpoint{1.645727in}{3.151256in}}%
\pgfpathlineto{\pgfqpoint{1.649845in}{3.179368in}}%
\pgfpathlineto{\pgfqpoint{1.653963in}{3.179368in}}%
\pgfpathlineto{\pgfqpoint{1.662199in}{3.151256in}}%
\pgfpathlineto{\pgfqpoint{1.670435in}{3.179368in}}%
\pgfpathlineto{\pgfqpoint{1.674553in}{3.165312in}}%
\pgfpathlineto{\pgfqpoint{1.678671in}{3.165312in}}%
\pgfpathlineto{\pgfqpoint{1.682789in}{3.151256in}}%
\pgfpathlineto{\pgfqpoint{1.691026in}{3.151256in}}%
\pgfpathlineto{\pgfqpoint{1.703380in}{3.109087in}}%
\pgfpathlineto{\pgfqpoint{1.707498in}{3.123143in}}%
\pgfpathlineto{\pgfqpoint{1.711616in}{3.151256in}}%
\pgfpathlineto{\pgfqpoint{1.715734in}{3.165312in}}%
\pgfpathlineto{\pgfqpoint{1.719852in}{3.151256in}}%
\pgfpathlineto{\pgfqpoint{1.723970in}{3.151256in}}%
\pgfpathlineto{\pgfqpoint{1.728088in}{3.165312in}}%
\pgfpathlineto{\pgfqpoint{1.732206in}{3.137200in}}%
\pgfpathlineto{\pgfqpoint{1.740443in}{3.137200in}}%
\pgfpathlineto{\pgfqpoint{1.744561in}{3.151256in}}%
\pgfpathlineto{\pgfqpoint{1.761033in}{3.151256in}}%
\pgfpathlineto{\pgfqpoint{1.765151in}{3.137200in}}%
\pgfpathlineto{\pgfqpoint{1.769269in}{3.151256in}}%
\pgfpathlineto{\pgfqpoint{1.785742in}{3.151256in}}%
\pgfpathlineto{\pgfqpoint{1.789860in}{3.165312in}}%
\pgfpathlineto{\pgfqpoint{1.793978in}{3.151256in}}%
\pgfpathlineto{\pgfqpoint{1.798096in}{3.151256in}}%
\pgfpathlineto{\pgfqpoint{1.802214in}{3.137200in}}%
\pgfpathlineto{\pgfqpoint{1.806332in}{3.137200in}}%
\pgfpathlineto{\pgfqpoint{1.814568in}{3.165312in}}%
\pgfpathlineto{\pgfqpoint{1.818686in}{3.151256in}}%
\pgfpathlineto{\pgfqpoint{1.822804in}{3.165312in}}%
\pgfpathlineto{\pgfqpoint{1.826922in}{3.151256in}}%
\pgfpathlineto{\pgfqpoint{1.831041in}{3.165312in}}%
\pgfpathlineto{\pgfqpoint{1.839277in}{3.165312in}}%
\pgfpathlineto{\pgfqpoint{1.843395in}{3.151256in}}%
\pgfpathlineto{\pgfqpoint{1.847513in}{3.165312in}}%
\pgfpathlineto{\pgfqpoint{1.851631in}{3.137200in}}%
\pgfpathlineto{\pgfqpoint{1.855749in}{3.151256in}}%
\pgfpathlineto{\pgfqpoint{1.863985in}{3.151256in}}%
\pgfpathlineto{\pgfqpoint{1.868103in}{3.165312in}}%
\pgfpathlineto{\pgfqpoint{1.872221in}{3.151256in}}%
\pgfpathlineto{\pgfqpoint{1.876339in}{3.151256in}}%
\pgfpathlineto{\pgfqpoint{1.880458in}{3.137200in}}%
\pgfpathlineto{\pgfqpoint{1.884576in}{3.137200in}}%
\pgfpathlineto{\pgfqpoint{1.888694in}{3.151256in}}%
\pgfpathlineto{\pgfqpoint{1.892812in}{3.137200in}}%
\pgfpathlineto{\pgfqpoint{1.909284in}{3.137200in}}%
\pgfpathlineto{\pgfqpoint{1.913402in}{3.123143in}}%
\pgfpathlineto{\pgfqpoint{1.921638in}{3.123143in}}%
\pgfpathlineto{\pgfqpoint{1.925756in}{3.109087in}}%
\pgfpathlineto{\pgfqpoint{1.933993in}{3.109087in}}%
\pgfpathlineto{\pgfqpoint{1.938111in}{3.151256in}}%
\pgfpathlineto{\pgfqpoint{1.962819in}{3.151256in}}%
\pgfpathlineto{\pgfqpoint{1.966937in}{3.137200in}}%
\pgfpathlineto{\pgfqpoint{1.971055in}{3.137200in}}%
\pgfpathlineto{\pgfqpoint{1.975173in}{3.151256in}}%
\pgfpathlineto{\pgfqpoint{1.995764in}{3.151256in}}%
\pgfpathlineto{\pgfqpoint{1.999882in}{3.165312in}}%
\pgfpathlineto{\pgfqpoint{2.004000in}{3.151256in}}%
\pgfpathlineto{\pgfqpoint{2.016354in}{3.151256in}}%
\pgfpathlineto{\pgfqpoint{2.020472in}{3.165312in}}%
\pgfpathlineto{\pgfqpoint{2.024591in}{3.165312in}}%
\pgfpathlineto{\pgfqpoint{2.028709in}{3.151256in}}%
\pgfpathlineto{\pgfqpoint{2.041063in}{3.151256in}}%
\pgfpathlineto{\pgfqpoint{2.045181in}{3.137200in}}%
\pgfpathlineto{\pgfqpoint{2.049299in}{3.137200in}}%
\pgfpathlineto{\pgfqpoint{2.053417in}{3.151256in}}%
\pgfpathlineto{\pgfqpoint{2.061653in}{3.151256in}}%
\pgfpathlineto{\pgfqpoint{2.065771in}{3.137200in}}%
\pgfpathlineto{\pgfqpoint{2.069889in}{3.137200in}}%
\pgfpathlineto{\pgfqpoint{2.074008in}{3.123143in}}%
\pgfpathlineto{\pgfqpoint{2.078126in}{3.123143in}}%
\pgfpathlineto{\pgfqpoint{2.082244in}{3.137200in}}%
\pgfpathlineto{\pgfqpoint{2.086362in}{3.123143in}}%
\pgfpathlineto{\pgfqpoint{2.090480in}{3.123143in}}%
\pgfpathlineto{\pgfqpoint{2.098716in}{3.179368in}}%
\pgfpathlineto{\pgfqpoint{2.106952in}{3.151256in}}%
\pgfpathlineto{\pgfqpoint{2.115188in}{3.151256in}}%
\pgfpathlineto{\pgfqpoint{2.123425in}{3.123143in}}%
\pgfpathlineto{\pgfqpoint{2.135779in}{3.123143in}}%
\pgfpathlineto{\pgfqpoint{2.139897in}{3.109087in}}%
\pgfpathlineto{\pgfqpoint{2.144015in}{3.123143in}}%
\pgfpathlineto{\pgfqpoint{2.148133in}{3.123143in}}%
\pgfpathlineto{\pgfqpoint{2.152251in}{3.137200in}}%
\pgfpathlineto{\pgfqpoint{2.164605in}{3.137200in}}%
\pgfpathlineto{\pgfqpoint{2.168723in}{3.151256in}}%
\pgfpathlineto{\pgfqpoint{2.193432in}{3.151256in}}%
\pgfpathlineto{\pgfqpoint{2.197550in}{3.165312in}}%
\pgfpathlineto{\pgfqpoint{2.201668in}{3.151256in}}%
\pgfpathlineto{\pgfqpoint{2.230495in}{3.151256in}}%
\pgfpathlineto{\pgfqpoint{2.234613in}{3.165312in}}%
\pgfpathlineto{\pgfqpoint{2.238731in}{3.151256in}}%
\pgfpathlineto{\pgfqpoint{2.275794in}{3.151256in}}%
\pgfpathlineto{\pgfqpoint{2.279912in}{3.165312in}}%
\pgfpathlineto{\pgfqpoint{2.284030in}{3.137200in}}%
\pgfpathlineto{\pgfqpoint{2.296384in}{3.137200in}}%
\pgfpathlineto{\pgfqpoint{2.300502in}{3.151256in}}%
\pgfpathlineto{\pgfqpoint{2.304620in}{3.137200in}}%
\pgfpathlineto{\pgfqpoint{2.325211in}{3.137200in}}%
\pgfpathlineto{\pgfqpoint{2.329329in}{3.165312in}}%
\pgfpathlineto{\pgfqpoint{2.333447in}{3.151256in}}%
\pgfpathlineto{\pgfqpoint{2.337565in}{3.151256in}}%
\pgfpathlineto{\pgfqpoint{2.341683in}{3.165312in}}%
\pgfpathlineto{\pgfqpoint{2.345801in}{3.165312in}}%
\pgfpathlineto{\pgfqpoint{2.349919in}{3.151256in}}%
\pgfpathlineto{\pgfqpoint{2.370510in}{3.151256in}}%
\pgfpathlineto{\pgfqpoint{2.374628in}{3.179368in}}%
\pgfpathlineto{\pgfqpoint{2.378746in}{3.151256in}}%
\pgfpathlineto{\pgfqpoint{2.481698in}{3.151256in}}%
\pgfpathlineto{\pgfqpoint{2.485816in}{3.137200in}}%
\pgfpathlineto{\pgfqpoint{2.489934in}{3.137200in}}%
\pgfpathlineto{\pgfqpoint{2.494052in}{3.151256in}}%
\pgfpathlineto{\pgfqpoint{2.518761in}{3.151256in}}%
\pgfpathlineto{\pgfqpoint{2.522879in}{3.137200in}}%
\pgfpathlineto{\pgfqpoint{2.526997in}{3.137200in}}%
\pgfpathlineto{\pgfqpoint{2.531115in}{3.151256in}}%
\pgfpathlineto{\pgfqpoint{2.559942in}{3.151256in}}%
\pgfpathlineto{\pgfqpoint{2.564060in}{3.165312in}}%
\pgfpathlineto{\pgfqpoint{2.568178in}{3.137200in}}%
\pgfpathlineto{\pgfqpoint{2.588768in}{3.137200in}}%
\pgfpathlineto{\pgfqpoint{2.592886in}{3.151256in}}%
\pgfpathlineto{\pgfqpoint{2.597004in}{3.151256in}}%
\pgfpathlineto{\pgfqpoint{2.601122in}{3.165312in}}%
\pgfpathlineto{\pgfqpoint{2.605240in}{3.151256in}}%
\pgfpathlineto{\pgfqpoint{2.609359in}{3.179368in}}%
\pgfpathlineto{\pgfqpoint{2.613477in}{3.179368in}}%
\pgfpathlineto{\pgfqpoint{2.617595in}{3.151256in}}%
\pgfpathlineto{\pgfqpoint{2.646421in}{3.151256in}}%
\pgfpathlineto{\pgfqpoint{2.646421in}{3.151256in}}%
\pgfusepath{stroke}%
\end{pgfscope}%
\begin{pgfscope}%
\pgfpathrectangle{\pgfqpoint{0.488751in}{2.165212in}}{\pgfqpoint{2.260417in}{1.283333in}}%
\pgfusepath{clip}%
\pgfsetbuttcap%
\pgfsetroundjoin%
\pgfsetlinewidth{0.803000pt}%
\definecolor{currentstroke}{rgb}{0.843137,0.666667,0.313725}%
\pgfsetstrokecolor{currentstroke}%
\pgfsetstrokeopacity{0.270000}%
\pgfsetdash{{2.960000pt}{1.280000pt}}{0.000000pt}%
\pgfpathmoveto{\pgfqpoint{0.591497in}{3.052862in}}%
\pgfpathlineto{\pgfqpoint{0.599733in}{3.080975in}}%
\pgfpathlineto{\pgfqpoint{0.603851in}{3.080975in}}%
\pgfpathlineto{\pgfqpoint{0.607969in}{3.066918in}}%
\pgfpathlineto{\pgfqpoint{0.612087in}{3.038806in}}%
\pgfpathlineto{\pgfqpoint{0.616206in}{3.052862in}}%
\pgfpathlineto{\pgfqpoint{0.620324in}{3.052862in}}%
\pgfpathlineto{\pgfqpoint{0.624442in}{3.066918in}}%
\pgfpathlineto{\pgfqpoint{0.628560in}{3.066918in}}%
\pgfpathlineto{\pgfqpoint{0.632678in}{3.052862in}}%
\pgfpathlineto{\pgfqpoint{0.636796in}{3.066918in}}%
\pgfpathlineto{\pgfqpoint{0.640914in}{3.095031in}}%
\pgfpathlineto{\pgfqpoint{0.645032in}{3.038806in}}%
\pgfpathlineto{\pgfqpoint{0.649150in}{3.010694in}}%
\pgfpathlineto{\pgfqpoint{0.653268in}{3.024750in}}%
\pgfpathlineto{\pgfqpoint{0.665623in}{3.024750in}}%
\pgfpathlineto{\pgfqpoint{0.669741in}{3.052862in}}%
\pgfpathlineto{\pgfqpoint{0.673859in}{3.038806in}}%
\pgfpathlineto{\pgfqpoint{0.677977in}{3.038806in}}%
\pgfpathlineto{\pgfqpoint{0.682095in}{2.982581in}}%
\pgfpathlineto{\pgfqpoint{0.690331in}{2.982581in}}%
\pgfpathlineto{\pgfqpoint{0.694449in}{2.996637in}}%
\pgfpathlineto{\pgfqpoint{0.698567in}{2.996637in}}%
\pgfpathlineto{\pgfqpoint{0.702685in}{2.982581in}}%
\pgfpathlineto{\pgfqpoint{0.706803in}{2.982581in}}%
\pgfpathlineto{\pgfqpoint{0.710922in}{3.038806in}}%
\pgfpathlineto{\pgfqpoint{0.715040in}{3.010694in}}%
\pgfpathlineto{\pgfqpoint{0.719158in}{3.010694in}}%
\pgfpathlineto{\pgfqpoint{0.723276in}{3.052862in}}%
\pgfpathlineto{\pgfqpoint{0.727394in}{3.066918in}}%
\pgfpathlineto{\pgfqpoint{0.731512in}{3.066918in}}%
\pgfpathlineto{\pgfqpoint{0.735630in}{3.052862in}}%
\pgfpathlineto{\pgfqpoint{0.739748in}{3.052862in}}%
\pgfpathlineto{\pgfqpoint{0.743866in}{3.066918in}}%
\pgfpathlineto{\pgfqpoint{0.752102in}{3.038806in}}%
\pgfpathlineto{\pgfqpoint{0.760339in}{3.066918in}}%
\pgfpathlineto{\pgfqpoint{0.764457in}{3.052862in}}%
\pgfpathlineto{\pgfqpoint{0.768575in}{3.080975in}}%
\pgfpathlineto{\pgfqpoint{0.772693in}{3.052862in}}%
\pgfpathlineto{\pgfqpoint{0.776811in}{3.066918in}}%
\pgfpathlineto{\pgfqpoint{0.785047in}{3.066918in}}%
\pgfpathlineto{\pgfqpoint{0.793283in}{3.095031in}}%
\pgfpathlineto{\pgfqpoint{0.801519in}{3.095031in}}%
\pgfpathlineto{\pgfqpoint{0.805637in}{3.109087in}}%
\pgfpathlineto{\pgfqpoint{0.809756in}{3.109087in}}%
\pgfpathlineto{\pgfqpoint{0.817992in}{3.080975in}}%
\pgfpathlineto{\pgfqpoint{0.822110in}{3.095031in}}%
\pgfpathlineto{\pgfqpoint{0.826228in}{3.080975in}}%
\pgfpathlineto{\pgfqpoint{0.834464in}{3.109087in}}%
\pgfpathlineto{\pgfqpoint{0.838582in}{3.095031in}}%
\pgfpathlineto{\pgfqpoint{0.842700in}{3.095031in}}%
\pgfpathlineto{\pgfqpoint{0.846818in}{3.080975in}}%
\pgfpathlineto{\pgfqpoint{0.850936in}{3.052862in}}%
\pgfpathlineto{\pgfqpoint{0.863291in}{3.095031in}}%
\pgfpathlineto{\pgfqpoint{0.867409in}{3.095031in}}%
\pgfpathlineto{\pgfqpoint{0.871527in}{3.080975in}}%
\pgfpathlineto{\pgfqpoint{0.875645in}{3.109087in}}%
\pgfpathlineto{\pgfqpoint{0.908590in}{3.109087in}}%
\pgfpathlineto{\pgfqpoint{0.912708in}{3.123143in}}%
\pgfpathlineto{\pgfqpoint{0.916826in}{3.123143in}}%
\pgfpathlineto{\pgfqpoint{0.925062in}{3.151256in}}%
\pgfpathlineto{\pgfqpoint{0.933298in}{3.151256in}}%
\pgfpathlineto{\pgfqpoint{0.941534in}{3.123143in}}%
\pgfpathlineto{\pgfqpoint{0.949770in}{3.123143in}}%
\pgfpathlineto{\pgfqpoint{0.958007in}{3.095031in}}%
\pgfpathlineto{\pgfqpoint{0.966243in}{3.123143in}}%
\pgfpathlineto{\pgfqpoint{0.974479in}{3.123143in}}%
\pgfpathlineto{\pgfqpoint{0.978597in}{3.109087in}}%
\pgfpathlineto{\pgfqpoint{0.986833in}{3.109087in}}%
\pgfpathlineto{\pgfqpoint{0.990951in}{3.080975in}}%
\pgfpathlineto{\pgfqpoint{0.995069in}{3.080975in}}%
\pgfpathlineto{\pgfqpoint{0.999187in}{3.109087in}}%
\pgfpathlineto{\pgfqpoint{1.003306in}{3.123143in}}%
\pgfpathlineto{\pgfqpoint{1.007424in}{3.109087in}}%
\pgfpathlineto{\pgfqpoint{1.011542in}{3.080975in}}%
\pgfpathlineto{\pgfqpoint{1.015660in}{3.066918in}}%
\pgfpathlineto{\pgfqpoint{1.019778in}{3.038806in}}%
\pgfpathlineto{\pgfqpoint{1.023896in}{3.052862in}}%
\pgfpathlineto{\pgfqpoint{1.028014in}{3.038806in}}%
\pgfpathlineto{\pgfqpoint{1.032132in}{3.038806in}}%
\pgfpathlineto{\pgfqpoint{1.040368in}{3.010694in}}%
\pgfpathlineto{\pgfqpoint{1.044486in}{3.038806in}}%
\pgfpathlineto{\pgfqpoint{1.052723in}{3.038806in}}%
\pgfpathlineto{\pgfqpoint{1.065077in}{2.996637in}}%
\pgfpathlineto{\pgfqpoint{1.069195in}{3.010694in}}%
\pgfpathlineto{\pgfqpoint{1.073313in}{3.010694in}}%
\pgfpathlineto{\pgfqpoint{1.077431in}{2.996637in}}%
\pgfpathlineto{\pgfqpoint{1.085667in}{3.024750in}}%
\pgfpathlineto{\pgfqpoint{1.093903in}{2.996637in}}%
\pgfpathlineto{\pgfqpoint{1.098021in}{2.954469in}}%
\pgfpathlineto{\pgfqpoint{1.102140in}{2.940412in}}%
\pgfpathlineto{\pgfqpoint{1.106258in}{2.954469in}}%
\pgfpathlineto{\pgfqpoint{1.110376in}{2.940412in}}%
\pgfpathlineto{\pgfqpoint{1.122730in}{2.856075in}}%
\pgfpathlineto{\pgfqpoint{1.126848in}{2.856075in}}%
\pgfpathlineto{\pgfqpoint{1.130966in}{2.884188in}}%
\pgfpathlineto{\pgfqpoint{1.135084in}{2.870131in}}%
\pgfpathlineto{\pgfqpoint{1.139202in}{2.842019in}}%
\pgfpathlineto{\pgfqpoint{1.143320in}{2.842019in}}%
\pgfpathlineto{\pgfqpoint{1.147438in}{2.827963in}}%
\pgfpathlineto{\pgfqpoint{1.151557in}{2.827963in}}%
\pgfpathlineto{\pgfqpoint{1.155675in}{2.813906in}}%
\pgfpathlineto{\pgfqpoint{1.180383in}{2.813906in}}%
\pgfpathlineto{\pgfqpoint{1.184501in}{2.785794in}}%
\pgfpathlineto{\pgfqpoint{1.196855in}{2.785794in}}%
\pgfpathlineto{\pgfqpoint{1.200974in}{2.771738in}}%
\pgfpathlineto{\pgfqpoint{1.225682in}{2.771738in}}%
\pgfpathlineto{\pgfqpoint{1.229800in}{2.785794in}}%
\pgfpathlineto{\pgfqpoint{1.246273in}{2.785794in}}%
\pgfpathlineto{\pgfqpoint{1.250391in}{2.771738in}}%
\pgfpathlineto{\pgfqpoint{1.258627in}{2.771738in}}%
\pgfpathlineto{\pgfqpoint{1.262745in}{2.757682in}}%
\pgfpathlineto{\pgfqpoint{1.328634in}{2.757682in}}%
\pgfpathlineto{\pgfqpoint{1.332752in}{2.743625in}}%
\pgfpathlineto{\pgfqpoint{1.386287in}{2.743625in}}%
\pgfpathlineto{\pgfqpoint{1.390405in}{2.729569in}}%
\pgfpathlineto{\pgfqpoint{1.406878in}{2.729569in}}%
\pgfpathlineto{\pgfqpoint{1.410996in}{2.743625in}}%
\pgfpathlineto{\pgfqpoint{2.473462in}{2.743625in}}%
\pgfpathlineto{\pgfqpoint{2.477580in}{2.757682in}}%
\pgfpathlineto{\pgfqpoint{2.646421in}{2.757682in}}%
\pgfpathlineto{\pgfqpoint{2.646421in}{2.757682in}}%
\pgfusepath{stroke}%
\end{pgfscope}%
\begin{pgfscope}%
\pgfpathrectangle{\pgfqpoint{0.488751in}{2.165212in}}{\pgfqpoint{2.260417in}{1.283333in}}%
\pgfusepath{clip}%
\pgfsetbuttcap%
\pgfsetroundjoin%
\pgfsetlinewidth{0.803000pt}%
\definecolor{currentstroke}{rgb}{0.333333,0.333333,0.333333}%
\pgfsetstrokecolor{currentstroke}%
\pgfsetstrokeopacity{0.270000}%
\pgfsetdash{{2.960000pt}{1.280000pt}}{0.000000pt}%
\pgfpathmoveto{\pgfqpoint{0.591497in}{2.856075in}}%
\pgfpathlineto{\pgfqpoint{0.595615in}{2.856075in}}%
\pgfpathlineto{\pgfqpoint{0.599733in}{2.785794in}}%
\pgfpathlineto{\pgfqpoint{0.603851in}{2.799850in}}%
\pgfpathlineto{\pgfqpoint{0.607969in}{2.771738in}}%
\pgfpathlineto{\pgfqpoint{0.612087in}{2.771738in}}%
\pgfpathlineto{\pgfqpoint{0.620324in}{2.799850in}}%
\pgfpathlineto{\pgfqpoint{0.624442in}{2.799850in}}%
\pgfpathlineto{\pgfqpoint{0.628560in}{2.785794in}}%
\pgfpathlineto{\pgfqpoint{0.632678in}{2.785794in}}%
\pgfpathlineto{\pgfqpoint{0.640914in}{2.687400in}}%
\pgfpathlineto{\pgfqpoint{0.645032in}{2.673344in}}%
\pgfpathlineto{\pgfqpoint{0.649150in}{2.687400in}}%
\pgfpathlineto{\pgfqpoint{0.653268in}{2.743625in}}%
\pgfpathlineto{\pgfqpoint{0.657386in}{2.687400in}}%
\pgfpathlineto{\pgfqpoint{0.661504in}{2.673344in}}%
\pgfpathlineto{\pgfqpoint{0.665623in}{2.645232in}}%
\pgfpathlineto{\pgfqpoint{0.669741in}{2.729569in}}%
\pgfpathlineto{\pgfqpoint{0.673859in}{2.729569in}}%
\pgfpathlineto{\pgfqpoint{0.677977in}{2.757682in}}%
\pgfpathlineto{\pgfqpoint{0.682095in}{2.813906in}}%
\pgfpathlineto{\pgfqpoint{0.686213in}{2.827963in}}%
\pgfpathlineto{\pgfqpoint{0.690331in}{2.827963in}}%
\pgfpathlineto{\pgfqpoint{0.694449in}{2.870131in}}%
\pgfpathlineto{\pgfqpoint{0.698567in}{2.898244in}}%
\pgfpathlineto{\pgfqpoint{0.702685in}{2.898244in}}%
\pgfpathlineto{\pgfqpoint{0.706803in}{2.926356in}}%
\pgfpathlineto{\pgfqpoint{0.710922in}{2.926356in}}%
\pgfpathlineto{\pgfqpoint{0.715040in}{2.898244in}}%
\pgfpathlineto{\pgfqpoint{0.723276in}{2.926356in}}%
\pgfpathlineto{\pgfqpoint{0.727394in}{2.912300in}}%
\pgfpathlineto{\pgfqpoint{0.731512in}{2.926356in}}%
\pgfpathlineto{\pgfqpoint{0.739748in}{2.926356in}}%
\pgfpathlineto{\pgfqpoint{0.743866in}{2.940412in}}%
\pgfpathlineto{\pgfqpoint{0.747984in}{2.884188in}}%
\pgfpathlineto{\pgfqpoint{0.760339in}{2.926356in}}%
\pgfpathlineto{\pgfqpoint{0.764457in}{2.926356in}}%
\pgfpathlineto{\pgfqpoint{0.768575in}{2.940412in}}%
\pgfpathlineto{\pgfqpoint{0.776811in}{2.996637in}}%
\pgfpathlineto{\pgfqpoint{0.780929in}{2.996637in}}%
\pgfpathlineto{\pgfqpoint{0.785047in}{3.038806in}}%
\pgfpathlineto{\pgfqpoint{0.789165in}{3.038806in}}%
\pgfpathlineto{\pgfqpoint{0.793283in}{3.024750in}}%
\pgfpathlineto{\pgfqpoint{0.797401in}{3.024750in}}%
\pgfpathlineto{\pgfqpoint{0.801519in}{3.052862in}}%
\pgfpathlineto{\pgfqpoint{0.805637in}{3.066918in}}%
\pgfpathlineto{\pgfqpoint{0.813874in}{3.066918in}}%
\pgfpathlineto{\pgfqpoint{0.817992in}{3.038806in}}%
\pgfpathlineto{\pgfqpoint{0.822110in}{3.052862in}}%
\pgfpathlineto{\pgfqpoint{0.826228in}{3.052862in}}%
\pgfpathlineto{\pgfqpoint{0.830346in}{3.038806in}}%
\pgfpathlineto{\pgfqpoint{0.834464in}{3.038806in}}%
\pgfpathlineto{\pgfqpoint{0.838582in}{3.024750in}}%
\pgfpathlineto{\pgfqpoint{0.842700in}{3.024750in}}%
\pgfpathlineto{\pgfqpoint{0.846818in}{2.996637in}}%
\pgfpathlineto{\pgfqpoint{0.850936in}{3.010694in}}%
\pgfpathlineto{\pgfqpoint{0.859173in}{3.066918in}}%
\pgfpathlineto{\pgfqpoint{0.863291in}{3.066918in}}%
\pgfpathlineto{\pgfqpoint{0.875645in}{3.109087in}}%
\pgfpathlineto{\pgfqpoint{0.879763in}{3.151256in}}%
\pgfpathlineto{\pgfqpoint{0.883881in}{3.165312in}}%
\pgfpathlineto{\pgfqpoint{0.887999in}{3.165312in}}%
\pgfpathlineto{\pgfqpoint{0.892117in}{3.137200in}}%
\pgfpathlineto{\pgfqpoint{0.904471in}{3.095031in}}%
\pgfpathlineto{\pgfqpoint{0.908590in}{3.095031in}}%
\pgfpathlineto{\pgfqpoint{0.912708in}{3.123143in}}%
\pgfpathlineto{\pgfqpoint{0.916826in}{3.109087in}}%
\pgfpathlineto{\pgfqpoint{0.925062in}{3.109087in}}%
\pgfpathlineto{\pgfqpoint{0.937416in}{3.151256in}}%
\pgfpathlineto{\pgfqpoint{0.941534in}{3.123143in}}%
\pgfpathlineto{\pgfqpoint{0.945652in}{3.123143in}}%
\pgfpathlineto{\pgfqpoint{0.949770in}{3.137200in}}%
\pgfpathlineto{\pgfqpoint{0.953888in}{3.137200in}}%
\pgfpathlineto{\pgfqpoint{0.958007in}{3.095031in}}%
\pgfpathlineto{\pgfqpoint{0.962125in}{3.095031in}}%
\pgfpathlineto{\pgfqpoint{0.966243in}{3.080975in}}%
\pgfpathlineto{\pgfqpoint{0.970361in}{3.080975in}}%
\pgfpathlineto{\pgfqpoint{0.974479in}{3.095031in}}%
\pgfpathlineto{\pgfqpoint{0.978597in}{3.095031in}}%
\pgfpathlineto{\pgfqpoint{0.982715in}{3.080975in}}%
\pgfpathlineto{\pgfqpoint{0.995069in}{3.080975in}}%
\pgfpathlineto{\pgfqpoint{0.999187in}{3.066918in}}%
\pgfpathlineto{\pgfqpoint{1.003306in}{3.066918in}}%
\pgfpathlineto{\pgfqpoint{1.007424in}{3.080975in}}%
\pgfpathlineto{\pgfqpoint{1.011542in}{3.052862in}}%
\pgfpathlineto{\pgfqpoint{1.019778in}{3.024750in}}%
\pgfpathlineto{\pgfqpoint{1.028014in}{3.024750in}}%
\pgfpathlineto{\pgfqpoint{1.032132in}{2.996637in}}%
\pgfpathlineto{\pgfqpoint{1.036250in}{2.982581in}}%
\pgfpathlineto{\pgfqpoint{1.040368in}{2.982581in}}%
\pgfpathlineto{\pgfqpoint{1.044486in}{3.024750in}}%
\pgfpathlineto{\pgfqpoint{1.056841in}{2.940412in}}%
\pgfpathlineto{\pgfqpoint{1.060959in}{2.926356in}}%
\pgfpathlineto{\pgfqpoint{1.065077in}{2.898244in}}%
\pgfpathlineto{\pgfqpoint{1.069195in}{2.898244in}}%
\pgfpathlineto{\pgfqpoint{1.077431in}{2.870131in}}%
\pgfpathlineto{\pgfqpoint{1.081549in}{2.898244in}}%
\pgfpathlineto{\pgfqpoint{1.089785in}{2.870131in}}%
\pgfpathlineto{\pgfqpoint{1.098021in}{2.926356in}}%
\pgfpathlineto{\pgfqpoint{1.106258in}{2.926356in}}%
\pgfpathlineto{\pgfqpoint{1.110376in}{2.940412in}}%
\pgfpathlineto{\pgfqpoint{1.114494in}{3.010694in}}%
\pgfpathlineto{\pgfqpoint{1.118612in}{2.968525in}}%
\pgfpathlineto{\pgfqpoint{1.122730in}{2.940412in}}%
\pgfpathlineto{\pgfqpoint{1.126848in}{2.954469in}}%
\pgfpathlineto{\pgfqpoint{1.135084in}{2.954469in}}%
\pgfpathlineto{\pgfqpoint{1.139202in}{3.010694in}}%
\pgfpathlineto{\pgfqpoint{1.143320in}{3.010694in}}%
\pgfpathlineto{\pgfqpoint{1.147438in}{3.038806in}}%
\pgfpathlineto{\pgfqpoint{1.155675in}{3.038806in}}%
\pgfpathlineto{\pgfqpoint{1.163911in}{3.066918in}}%
\pgfpathlineto{\pgfqpoint{1.168029in}{3.066918in}}%
\pgfpathlineto{\pgfqpoint{1.172147in}{3.038806in}}%
\pgfpathlineto{\pgfqpoint{1.176265in}{3.052862in}}%
\pgfpathlineto{\pgfqpoint{1.180383in}{3.038806in}}%
\pgfpathlineto{\pgfqpoint{1.184501in}{3.052862in}}%
\pgfpathlineto{\pgfqpoint{1.188619in}{3.010694in}}%
\pgfpathlineto{\pgfqpoint{1.196855in}{3.038806in}}%
\pgfpathlineto{\pgfqpoint{1.200974in}{3.010694in}}%
\pgfpathlineto{\pgfqpoint{1.205092in}{3.024750in}}%
\pgfpathlineto{\pgfqpoint{1.209210in}{3.024750in}}%
\pgfpathlineto{\pgfqpoint{1.213328in}{3.038806in}}%
\pgfpathlineto{\pgfqpoint{1.221564in}{3.095031in}}%
\pgfpathlineto{\pgfqpoint{1.225682in}{3.109087in}}%
\pgfpathlineto{\pgfqpoint{1.229800in}{3.151256in}}%
\pgfpathlineto{\pgfqpoint{1.233918in}{3.151256in}}%
\pgfpathlineto{\pgfqpoint{1.238036in}{3.165312in}}%
\pgfpathlineto{\pgfqpoint{1.242154in}{3.151256in}}%
\pgfpathlineto{\pgfqpoint{1.250391in}{3.151256in}}%
\pgfpathlineto{\pgfqpoint{1.254509in}{3.165312in}}%
\pgfpathlineto{\pgfqpoint{1.258627in}{3.123143in}}%
\pgfpathlineto{\pgfqpoint{1.262745in}{3.109087in}}%
\pgfpathlineto{\pgfqpoint{1.266863in}{3.123143in}}%
\pgfpathlineto{\pgfqpoint{1.270981in}{3.095031in}}%
\pgfpathlineto{\pgfqpoint{1.275099in}{3.038806in}}%
\pgfpathlineto{\pgfqpoint{1.279217in}{3.038806in}}%
\pgfpathlineto{\pgfqpoint{1.283335in}{3.024750in}}%
\pgfpathlineto{\pgfqpoint{1.287453in}{3.038806in}}%
\pgfpathlineto{\pgfqpoint{1.291571in}{3.010694in}}%
\pgfpathlineto{\pgfqpoint{1.295690in}{3.024750in}}%
\pgfpathlineto{\pgfqpoint{1.299808in}{2.996637in}}%
\pgfpathlineto{\pgfqpoint{1.303926in}{3.010694in}}%
\pgfpathlineto{\pgfqpoint{1.308044in}{2.996637in}}%
\pgfpathlineto{\pgfqpoint{1.312162in}{3.024750in}}%
\pgfpathlineto{\pgfqpoint{1.316280in}{3.066918in}}%
\pgfpathlineto{\pgfqpoint{1.324516in}{3.066918in}}%
\pgfpathlineto{\pgfqpoint{1.328634in}{3.080975in}}%
\pgfpathlineto{\pgfqpoint{1.332752in}{3.080975in}}%
\pgfpathlineto{\pgfqpoint{1.340988in}{3.052862in}}%
\pgfpathlineto{\pgfqpoint{1.345107in}{3.080975in}}%
\pgfpathlineto{\pgfqpoint{1.349225in}{3.080975in}}%
\pgfpathlineto{\pgfqpoint{1.353343in}{3.095031in}}%
\pgfpathlineto{\pgfqpoint{1.357461in}{3.123143in}}%
\pgfpathlineto{\pgfqpoint{1.361579in}{3.137200in}}%
\pgfpathlineto{\pgfqpoint{1.365697in}{3.137200in}}%
\pgfpathlineto{\pgfqpoint{1.369815in}{3.151256in}}%
\pgfpathlineto{\pgfqpoint{1.382169in}{3.151256in}}%
\pgfpathlineto{\pgfqpoint{1.386287in}{3.137200in}}%
\pgfpathlineto{\pgfqpoint{1.390405in}{3.151256in}}%
\pgfpathlineto{\pgfqpoint{1.398642in}{3.151256in}}%
\pgfpathlineto{\pgfqpoint{1.402760in}{3.165312in}}%
\pgfpathlineto{\pgfqpoint{1.410996in}{3.165312in}}%
\pgfpathlineto{\pgfqpoint{1.415114in}{3.151256in}}%
\pgfpathlineto{\pgfqpoint{1.419232in}{3.151256in}}%
\pgfpathlineto{\pgfqpoint{1.423350in}{3.179368in}}%
\pgfpathlineto{\pgfqpoint{1.427468in}{3.151256in}}%
\pgfpathlineto{\pgfqpoint{1.431586in}{3.137200in}}%
\pgfpathlineto{\pgfqpoint{1.435704in}{3.137200in}}%
\pgfpathlineto{\pgfqpoint{1.443941in}{3.080975in}}%
\pgfpathlineto{\pgfqpoint{1.448059in}{3.095031in}}%
\pgfpathlineto{\pgfqpoint{1.452177in}{3.066918in}}%
\pgfpathlineto{\pgfqpoint{1.456295in}{3.066918in}}%
\pgfpathlineto{\pgfqpoint{1.460413in}{3.010694in}}%
\pgfpathlineto{\pgfqpoint{1.464531in}{3.010694in}}%
\pgfpathlineto{\pgfqpoint{1.468649in}{3.038806in}}%
\pgfpathlineto{\pgfqpoint{1.476885in}{3.066918in}}%
\pgfpathlineto{\pgfqpoint{1.497476in}{3.066918in}}%
\pgfpathlineto{\pgfqpoint{1.501594in}{3.080975in}}%
\pgfpathlineto{\pgfqpoint{1.505712in}{3.080975in}}%
\pgfpathlineto{\pgfqpoint{1.509830in}{3.066918in}}%
\pgfpathlineto{\pgfqpoint{1.518066in}{3.066918in}}%
\pgfpathlineto{\pgfqpoint{1.526302in}{3.123143in}}%
\pgfpathlineto{\pgfqpoint{1.530420in}{3.137200in}}%
\pgfpathlineto{\pgfqpoint{1.534538in}{3.137200in}}%
\pgfpathlineto{\pgfqpoint{1.542775in}{3.109087in}}%
\pgfpathlineto{\pgfqpoint{1.546893in}{3.123143in}}%
\pgfpathlineto{\pgfqpoint{1.551011in}{3.109087in}}%
\pgfpathlineto{\pgfqpoint{1.555129in}{3.109087in}}%
\pgfpathlineto{\pgfqpoint{1.559247in}{3.123143in}}%
\pgfpathlineto{\pgfqpoint{1.563365in}{3.123143in}}%
\pgfpathlineto{\pgfqpoint{1.567483in}{3.137200in}}%
\pgfpathlineto{\pgfqpoint{1.575719in}{3.137200in}}%
\pgfpathlineto{\pgfqpoint{1.579837in}{3.123143in}}%
\pgfpathlineto{\pgfqpoint{1.583955in}{3.095031in}}%
\pgfpathlineto{\pgfqpoint{1.588074in}{3.095031in}}%
\pgfpathlineto{\pgfqpoint{1.592192in}{3.052862in}}%
\pgfpathlineto{\pgfqpoint{1.596310in}{3.038806in}}%
\pgfpathlineto{\pgfqpoint{1.600428in}{3.095031in}}%
\pgfpathlineto{\pgfqpoint{1.608664in}{3.010694in}}%
\pgfpathlineto{\pgfqpoint{1.616900in}{3.010694in}}%
\pgfpathlineto{\pgfqpoint{1.621018in}{3.024750in}}%
\pgfpathlineto{\pgfqpoint{1.625136in}{2.982581in}}%
\pgfpathlineto{\pgfqpoint{1.629254in}{2.996637in}}%
\pgfpathlineto{\pgfqpoint{1.633372in}{2.968525in}}%
\pgfpathlineto{\pgfqpoint{1.637491in}{2.954469in}}%
\pgfpathlineto{\pgfqpoint{1.641609in}{2.968525in}}%
\pgfpathlineto{\pgfqpoint{1.645727in}{2.968525in}}%
\pgfpathlineto{\pgfqpoint{1.649845in}{2.954469in}}%
\pgfpathlineto{\pgfqpoint{1.658081in}{2.954469in}}%
\pgfpathlineto{\pgfqpoint{1.662199in}{2.982581in}}%
\pgfpathlineto{\pgfqpoint{1.682789in}{2.912300in}}%
\pgfpathlineto{\pgfqpoint{1.686908in}{2.912300in}}%
\pgfpathlineto{\pgfqpoint{1.691026in}{2.898244in}}%
\pgfpathlineto{\pgfqpoint{1.699262in}{2.926356in}}%
\pgfpathlineto{\pgfqpoint{1.703380in}{2.926356in}}%
\pgfpathlineto{\pgfqpoint{1.707498in}{2.968525in}}%
\pgfpathlineto{\pgfqpoint{1.711616in}{2.996637in}}%
\pgfpathlineto{\pgfqpoint{1.723970in}{2.996637in}}%
\pgfpathlineto{\pgfqpoint{1.728088in}{2.968525in}}%
\pgfpathlineto{\pgfqpoint{1.736325in}{2.968525in}}%
\pgfpathlineto{\pgfqpoint{1.740443in}{2.982581in}}%
\pgfpathlineto{\pgfqpoint{1.744561in}{2.954469in}}%
\pgfpathlineto{\pgfqpoint{1.748679in}{2.940412in}}%
\pgfpathlineto{\pgfqpoint{1.752797in}{2.954469in}}%
\pgfpathlineto{\pgfqpoint{1.761033in}{2.954469in}}%
\pgfpathlineto{\pgfqpoint{1.765151in}{2.940412in}}%
\pgfpathlineto{\pgfqpoint{1.769269in}{2.912300in}}%
\pgfpathlineto{\pgfqpoint{1.773387in}{2.926356in}}%
\pgfpathlineto{\pgfqpoint{1.777505in}{2.926356in}}%
\pgfpathlineto{\pgfqpoint{1.781624in}{2.912300in}}%
\pgfpathlineto{\pgfqpoint{1.785742in}{2.912300in}}%
\pgfpathlineto{\pgfqpoint{1.789860in}{2.954469in}}%
\pgfpathlineto{\pgfqpoint{1.793978in}{2.954469in}}%
\pgfpathlineto{\pgfqpoint{1.806332in}{2.996637in}}%
\pgfpathlineto{\pgfqpoint{1.810450in}{2.982581in}}%
\pgfpathlineto{\pgfqpoint{1.822804in}{2.898244in}}%
\pgfpathlineto{\pgfqpoint{1.826922in}{2.884188in}}%
\pgfpathlineto{\pgfqpoint{1.831041in}{2.856075in}}%
\pgfpathlineto{\pgfqpoint{1.839277in}{2.856075in}}%
\pgfpathlineto{\pgfqpoint{1.843395in}{2.870131in}}%
\pgfpathlineto{\pgfqpoint{1.847513in}{2.912300in}}%
\pgfpathlineto{\pgfqpoint{1.851631in}{2.926356in}}%
\pgfpathlineto{\pgfqpoint{1.855749in}{2.926356in}}%
\pgfpathlineto{\pgfqpoint{1.859867in}{2.954469in}}%
\pgfpathlineto{\pgfqpoint{1.863985in}{2.954469in}}%
\pgfpathlineto{\pgfqpoint{1.868103in}{2.926356in}}%
\pgfpathlineto{\pgfqpoint{1.872221in}{2.912300in}}%
\pgfpathlineto{\pgfqpoint{1.876339in}{2.912300in}}%
\pgfpathlineto{\pgfqpoint{1.880458in}{2.926356in}}%
\pgfpathlineto{\pgfqpoint{1.884576in}{2.926356in}}%
\pgfpathlineto{\pgfqpoint{1.888694in}{2.954469in}}%
\pgfpathlineto{\pgfqpoint{1.892812in}{2.968525in}}%
\pgfpathlineto{\pgfqpoint{1.896930in}{2.968525in}}%
\pgfpathlineto{\pgfqpoint{1.901048in}{2.940412in}}%
\pgfpathlineto{\pgfqpoint{1.905166in}{2.940412in}}%
\pgfpathlineto{\pgfqpoint{1.909284in}{2.954469in}}%
\pgfpathlineto{\pgfqpoint{1.917520in}{2.954469in}}%
\pgfpathlineto{\pgfqpoint{1.921638in}{2.926356in}}%
\pgfpathlineto{\pgfqpoint{1.925756in}{2.940412in}}%
\pgfpathlineto{\pgfqpoint{1.933993in}{2.940412in}}%
\pgfpathlineto{\pgfqpoint{1.938111in}{2.954469in}}%
\pgfpathlineto{\pgfqpoint{1.942229in}{2.912300in}}%
\pgfpathlineto{\pgfqpoint{1.950465in}{2.940412in}}%
\pgfpathlineto{\pgfqpoint{1.954583in}{2.940412in}}%
\pgfpathlineto{\pgfqpoint{1.958701in}{2.954469in}}%
\pgfpathlineto{\pgfqpoint{1.971055in}{2.954469in}}%
\pgfpathlineto{\pgfqpoint{1.975173in}{2.940412in}}%
\pgfpathlineto{\pgfqpoint{1.983410in}{2.968525in}}%
\pgfpathlineto{\pgfqpoint{1.987528in}{2.940412in}}%
\pgfpathlineto{\pgfqpoint{1.991646in}{2.940412in}}%
\pgfpathlineto{\pgfqpoint{1.995764in}{2.926356in}}%
\pgfpathlineto{\pgfqpoint{2.004000in}{2.926356in}}%
\pgfpathlineto{\pgfqpoint{2.008118in}{2.940412in}}%
\pgfpathlineto{\pgfqpoint{2.016354in}{2.940412in}}%
\pgfpathlineto{\pgfqpoint{2.020472in}{2.954469in}}%
\pgfpathlineto{\pgfqpoint{2.024591in}{2.954469in}}%
\pgfpathlineto{\pgfqpoint{2.028709in}{2.982581in}}%
\pgfpathlineto{\pgfqpoint{2.032827in}{2.982581in}}%
\pgfpathlineto{\pgfqpoint{2.036945in}{2.968525in}}%
\pgfpathlineto{\pgfqpoint{2.041063in}{2.982581in}}%
\pgfpathlineto{\pgfqpoint{2.045181in}{2.968525in}}%
\pgfpathlineto{\pgfqpoint{2.049299in}{2.940412in}}%
\pgfpathlineto{\pgfqpoint{2.053417in}{2.884188in}}%
\pgfpathlineto{\pgfqpoint{2.057535in}{2.898244in}}%
\pgfpathlineto{\pgfqpoint{2.061653in}{2.870131in}}%
\pgfpathlineto{\pgfqpoint{2.065771in}{2.827963in}}%
\pgfpathlineto{\pgfqpoint{2.069889in}{2.827963in}}%
\pgfpathlineto{\pgfqpoint{2.074008in}{2.842019in}}%
\pgfpathlineto{\pgfqpoint{2.078126in}{2.827963in}}%
\pgfpathlineto{\pgfqpoint{2.082244in}{2.842019in}}%
\pgfpathlineto{\pgfqpoint{2.086362in}{2.827963in}}%
\pgfpathlineto{\pgfqpoint{2.090480in}{2.856075in}}%
\pgfpathlineto{\pgfqpoint{2.094598in}{2.926356in}}%
\pgfpathlineto{\pgfqpoint{2.098716in}{2.898244in}}%
\pgfpathlineto{\pgfqpoint{2.102834in}{2.898244in}}%
\pgfpathlineto{\pgfqpoint{2.106952in}{2.912300in}}%
\pgfpathlineto{\pgfqpoint{2.111070in}{2.954469in}}%
\pgfpathlineto{\pgfqpoint{2.115188in}{2.940412in}}%
\pgfpathlineto{\pgfqpoint{2.123425in}{2.940412in}}%
\pgfpathlineto{\pgfqpoint{2.127543in}{2.926356in}}%
\pgfpathlineto{\pgfqpoint{2.135779in}{2.926356in}}%
\pgfpathlineto{\pgfqpoint{2.139897in}{2.940412in}}%
\pgfpathlineto{\pgfqpoint{2.148133in}{2.884188in}}%
\pgfpathlineto{\pgfqpoint{2.160487in}{2.884188in}}%
\pgfpathlineto{\pgfqpoint{2.164605in}{2.870131in}}%
\pgfpathlineto{\pgfqpoint{2.168723in}{2.842019in}}%
\pgfpathlineto{\pgfqpoint{2.172842in}{2.827963in}}%
\pgfpathlineto{\pgfqpoint{2.176960in}{2.827963in}}%
\pgfpathlineto{\pgfqpoint{2.185196in}{2.799850in}}%
\pgfpathlineto{\pgfqpoint{2.189314in}{2.827963in}}%
\pgfpathlineto{\pgfqpoint{2.193432in}{2.842019in}}%
\pgfpathlineto{\pgfqpoint{2.197550in}{2.827963in}}%
\pgfpathlineto{\pgfqpoint{2.201668in}{2.799850in}}%
\pgfpathlineto{\pgfqpoint{2.209904in}{2.856075in}}%
\pgfpathlineto{\pgfqpoint{2.214022in}{2.856075in}}%
\pgfpathlineto{\pgfqpoint{2.218140in}{2.842019in}}%
\pgfpathlineto{\pgfqpoint{2.222259in}{2.856075in}}%
\pgfpathlineto{\pgfqpoint{2.226377in}{2.827963in}}%
\pgfpathlineto{\pgfqpoint{2.230495in}{2.813906in}}%
\pgfpathlineto{\pgfqpoint{2.234613in}{2.813906in}}%
\pgfpathlineto{\pgfqpoint{2.246967in}{2.856075in}}%
\pgfpathlineto{\pgfqpoint{2.263439in}{2.856075in}}%
\pgfpathlineto{\pgfqpoint{2.271676in}{2.827963in}}%
\pgfpathlineto{\pgfqpoint{2.275794in}{2.827963in}}%
\pgfpathlineto{\pgfqpoint{2.284030in}{2.799850in}}%
\pgfpathlineto{\pgfqpoint{2.288148in}{2.799850in}}%
\pgfpathlineto{\pgfqpoint{2.292266in}{2.813906in}}%
\pgfpathlineto{\pgfqpoint{2.296384in}{2.813906in}}%
\pgfpathlineto{\pgfqpoint{2.300502in}{2.785794in}}%
\pgfpathlineto{\pgfqpoint{2.304620in}{2.785794in}}%
\pgfpathlineto{\pgfqpoint{2.308738in}{2.813906in}}%
\pgfpathlineto{\pgfqpoint{2.312856in}{2.813906in}}%
\pgfpathlineto{\pgfqpoint{2.316975in}{2.799850in}}%
\pgfpathlineto{\pgfqpoint{2.321093in}{2.799850in}}%
\pgfpathlineto{\pgfqpoint{2.325211in}{2.813906in}}%
\pgfpathlineto{\pgfqpoint{2.329329in}{2.771738in}}%
\pgfpathlineto{\pgfqpoint{2.333447in}{2.827963in}}%
\pgfpathlineto{\pgfqpoint{2.337565in}{2.813906in}}%
\pgfpathlineto{\pgfqpoint{2.341683in}{2.785794in}}%
\pgfpathlineto{\pgfqpoint{2.345801in}{2.785794in}}%
\pgfpathlineto{\pgfqpoint{2.354037in}{2.813906in}}%
\pgfpathlineto{\pgfqpoint{2.362273in}{2.813906in}}%
\pgfpathlineto{\pgfqpoint{2.366392in}{2.785794in}}%
\pgfpathlineto{\pgfqpoint{2.370510in}{2.785794in}}%
\pgfpathlineto{\pgfqpoint{2.374628in}{2.743625in}}%
\pgfpathlineto{\pgfqpoint{2.378746in}{2.757682in}}%
\pgfpathlineto{\pgfqpoint{2.386982in}{2.673344in}}%
\pgfpathlineto{\pgfqpoint{2.391100in}{2.673344in}}%
\pgfpathlineto{\pgfqpoint{2.395218in}{2.659288in}}%
\pgfpathlineto{\pgfqpoint{2.407572in}{2.659288in}}%
\pgfpathlineto{\pgfqpoint{2.411690in}{2.687400in}}%
\pgfpathlineto{\pgfqpoint{2.415809in}{2.687400in}}%
\pgfpathlineto{\pgfqpoint{2.419927in}{2.659288in}}%
\pgfpathlineto{\pgfqpoint{2.424045in}{2.645232in}}%
\pgfpathlineto{\pgfqpoint{2.440517in}{2.645232in}}%
\pgfpathlineto{\pgfqpoint{2.444635in}{2.631176in}}%
\pgfpathlineto{\pgfqpoint{2.452871in}{2.631176in}}%
\pgfpathlineto{\pgfqpoint{2.456989in}{2.645232in}}%
\pgfpathlineto{\pgfqpoint{2.473462in}{2.645232in}}%
\pgfpathlineto{\pgfqpoint{2.477580in}{2.659288in}}%
\pgfpathlineto{\pgfqpoint{2.481698in}{2.645232in}}%
\pgfpathlineto{\pgfqpoint{2.485816in}{2.645232in}}%
\pgfpathlineto{\pgfqpoint{2.489934in}{2.631176in}}%
\pgfpathlineto{\pgfqpoint{2.494052in}{2.631176in}}%
\pgfpathlineto{\pgfqpoint{2.498170in}{2.617119in}}%
\pgfpathlineto{\pgfqpoint{2.502288in}{2.617119in}}%
\pgfpathlineto{\pgfqpoint{2.506406in}{2.603063in}}%
\pgfpathlineto{\pgfqpoint{2.518761in}{2.603063in}}%
\pgfpathlineto{\pgfqpoint{2.526997in}{2.574951in}}%
\pgfpathlineto{\pgfqpoint{2.621713in}{2.574951in}}%
\pgfpathlineto{\pgfqpoint{2.625831in}{2.560894in}}%
\pgfpathlineto{\pgfqpoint{2.629949in}{2.574951in}}%
\pgfpathlineto{\pgfqpoint{2.646421in}{2.574951in}}%
\pgfpathlineto{\pgfqpoint{2.646421in}{2.574951in}}%
\pgfusepath{stroke}%
\end{pgfscope}%
\begin{pgfscope}%
\pgfpathrectangle{\pgfqpoint{0.488751in}{2.165212in}}{\pgfqpoint{2.260417in}{1.283333in}}%
\pgfusepath{clip}%
\pgfsetbuttcap%
\pgfsetroundjoin%
\pgfsetlinewidth{0.803000pt}%
\definecolor{currentstroke}{rgb}{0.686275,0.352941,0.313725}%
\pgfsetstrokecolor{currentstroke}%
\pgfsetstrokeopacity{0.270000}%
\pgfsetdash{{2.960000pt}{1.280000pt}}{0.000000pt}%
\pgfpathmoveto{\pgfqpoint{0.591497in}{3.052862in}}%
\pgfpathlineto{\pgfqpoint{0.595615in}{3.095031in}}%
\pgfpathlineto{\pgfqpoint{0.599733in}{3.109087in}}%
\pgfpathlineto{\pgfqpoint{0.603851in}{3.095031in}}%
\pgfpathlineto{\pgfqpoint{0.612087in}{3.151256in}}%
\pgfpathlineto{\pgfqpoint{0.624442in}{3.151256in}}%
\pgfpathlineto{\pgfqpoint{0.628560in}{3.193425in}}%
\pgfpathlineto{\pgfqpoint{0.632678in}{3.207481in}}%
\pgfpathlineto{\pgfqpoint{0.636796in}{3.207481in}}%
\pgfpathlineto{\pgfqpoint{0.645032in}{3.263706in}}%
\pgfpathlineto{\pgfqpoint{0.649150in}{3.305874in}}%
\pgfpathlineto{\pgfqpoint{0.653268in}{3.263706in}}%
\pgfpathlineto{\pgfqpoint{0.657386in}{3.277762in}}%
\pgfpathlineto{\pgfqpoint{0.661504in}{3.263706in}}%
\pgfpathlineto{\pgfqpoint{0.665623in}{3.207481in}}%
\pgfpathlineto{\pgfqpoint{0.669741in}{3.179368in}}%
\pgfpathlineto{\pgfqpoint{0.682095in}{3.179368in}}%
\pgfpathlineto{\pgfqpoint{0.686213in}{3.151256in}}%
\pgfpathlineto{\pgfqpoint{0.690331in}{3.151256in}}%
\pgfpathlineto{\pgfqpoint{0.694449in}{3.165312in}}%
\pgfpathlineto{\pgfqpoint{0.698567in}{3.151256in}}%
\pgfpathlineto{\pgfqpoint{0.702685in}{3.151256in}}%
\pgfpathlineto{\pgfqpoint{0.706803in}{3.137200in}}%
\pgfpathlineto{\pgfqpoint{0.710922in}{3.151256in}}%
\pgfpathlineto{\pgfqpoint{0.715040in}{3.151256in}}%
\pgfpathlineto{\pgfqpoint{0.719158in}{3.123143in}}%
\pgfpathlineto{\pgfqpoint{0.723276in}{3.123143in}}%
\pgfpathlineto{\pgfqpoint{0.727394in}{3.151256in}}%
\pgfpathlineto{\pgfqpoint{0.735630in}{3.151256in}}%
\pgfpathlineto{\pgfqpoint{0.739748in}{3.123143in}}%
\pgfpathlineto{\pgfqpoint{0.743866in}{3.123143in}}%
\pgfpathlineto{\pgfqpoint{0.747984in}{3.137200in}}%
\pgfpathlineto{\pgfqpoint{0.752102in}{3.109087in}}%
\pgfpathlineto{\pgfqpoint{0.756220in}{3.109087in}}%
\pgfpathlineto{\pgfqpoint{0.768575in}{3.066918in}}%
\pgfpathlineto{\pgfqpoint{0.772693in}{3.066918in}}%
\pgfpathlineto{\pgfqpoint{0.776811in}{3.052862in}}%
\pgfpathlineto{\pgfqpoint{0.780929in}{3.010694in}}%
\pgfpathlineto{\pgfqpoint{0.785047in}{2.996637in}}%
\pgfpathlineto{\pgfqpoint{0.789165in}{2.968525in}}%
\pgfpathlineto{\pgfqpoint{0.793283in}{2.968525in}}%
\pgfpathlineto{\pgfqpoint{0.801519in}{2.996637in}}%
\pgfpathlineto{\pgfqpoint{0.805637in}{2.996637in}}%
\pgfpathlineto{\pgfqpoint{0.809756in}{3.010694in}}%
\pgfpathlineto{\pgfqpoint{0.813874in}{3.010694in}}%
\pgfpathlineto{\pgfqpoint{0.817992in}{3.024750in}}%
\pgfpathlineto{\pgfqpoint{0.822110in}{3.024750in}}%
\pgfpathlineto{\pgfqpoint{0.826228in}{3.038806in}}%
\pgfpathlineto{\pgfqpoint{0.830346in}{3.024750in}}%
\pgfpathlineto{\pgfqpoint{0.838582in}{3.024750in}}%
\pgfpathlineto{\pgfqpoint{0.842700in}{3.010694in}}%
\pgfpathlineto{\pgfqpoint{0.850936in}{3.010694in}}%
\pgfpathlineto{\pgfqpoint{0.855054in}{3.024750in}}%
\pgfpathlineto{\pgfqpoint{0.863291in}{3.024750in}}%
\pgfpathlineto{\pgfqpoint{0.867409in}{3.038806in}}%
\pgfpathlineto{\pgfqpoint{0.887999in}{3.038806in}}%
\pgfpathlineto{\pgfqpoint{0.896235in}{3.066918in}}%
\pgfpathlineto{\pgfqpoint{0.908590in}{3.066918in}}%
\pgfpathlineto{\pgfqpoint{0.912708in}{3.052862in}}%
\pgfpathlineto{\pgfqpoint{0.941534in}{3.052862in}}%
\pgfpathlineto{\pgfqpoint{0.945652in}{3.038806in}}%
\pgfpathlineto{\pgfqpoint{0.949770in}{3.052862in}}%
\pgfpathlineto{\pgfqpoint{0.962125in}{3.052862in}}%
\pgfpathlineto{\pgfqpoint{0.966243in}{3.066918in}}%
\pgfpathlineto{\pgfqpoint{0.970361in}{3.066918in}}%
\pgfpathlineto{\pgfqpoint{0.974479in}{3.052862in}}%
\pgfpathlineto{\pgfqpoint{0.986833in}{3.052862in}}%
\pgfpathlineto{\pgfqpoint{0.990951in}{3.066918in}}%
\pgfpathlineto{\pgfqpoint{0.999187in}{3.066918in}}%
\pgfpathlineto{\pgfqpoint{1.003306in}{3.052862in}}%
\pgfpathlineto{\pgfqpoint{1.007424in}{3.066918in}}%
\pgfpathlineto{\pgfqpoint{1.015660in}{3.066918in}}%
\pgfpathlineto{\pgfqpoint{1.019778in}{3.080975in}}%
\pgfpathlineto{\pgfqpoint{1.032132in}{3.080975in}}%
\pgfpathlineto{\pgfqpoint{1.040368in}{3.109087in}}%
\pgfpathlineto{\pgfqpoint{1.044486in}{3.095031in}}%
\pgfpathlineto{\pgfqpoint{1.048604in}{3.123143in}}%
\pgfpathlineto{\pgfqpoint{1.056841in}{3.123143in}}%
\pgfpathlineto{\pgfqpoint{1.060959in}{3.137200in}}%
\pgfpathlineto{\pgfqpoint{1.065077in}{3.137200in}}%
\pgfpathlineto{\pgfqpoint{1.069195in}{3.123143in}}%
\pgfpathlineto{\pgfqpoint{1.077431in}{3.123143in}}%
\pgfpathlineto{\pgfqpoint{1.081549in}{3.109087in}}%
\pgfpathlineto{\pgfqpoint{1.085667in}{3.109087in}}%
\pgfpathlineto{\pgfqpoint{1.089785in}{3.123143in}}%
\pgfpathlineto{\pgfqpoint{1.093903in}{3.123143in}}%
\pgfpathlineto{\pgfqpoint{1.098021in}{3.137200in}}%
\pgfpathlineto{\pgfqpoint{1.102140in}{3.123143in}}%
\pgfpathlineto{\pgfqpoint{1.110376in}{3.123143in}}%
\pgfpathlineto{\pgfqpoint{1.114494in}{3.080975in}}%
\pgfpathlineto{\pgfqpoint{1.122730in}{3.080975in}}%
\pgfpathlineto{\pgfqpoint{1.126848in}{3.052862in}}%
\pgfpathlineto{\pgfqpoint{1.135084in}{3.052862in}}%
\pgfpathlineto{\pgfqpoint{1.139202in}{3.024750in}}%
\pgfpathlineto{\pgfqpoint{1.143320in}{3.024750in}}%
\pgfpathlineto{\pgfqpoint{1.147438in}{3.010694in}}%
\pgfpathlineto{\pgfqpoint{1.155675in}{3.010694in}}%
\pgfpathlineto{\pgfqpoint{1.159793in}{2.996637in}}%
\pgfpathlineto{\pgfqpoint{1.172147in}{2.996637in}}%
\pgfpathlineto{\pgfqpoint{1.176265in}{2.982581in}}%
\pgfpathlineto{\pgfqpoint{1.205092in}{2.982581in}}%
\pgfpathlineto{\pgfqpoint{1.209210in}{2.968525in}}%
\pgfpathlineto{\pgfqpoint{1.213328in}{2.968525in}}%
\pgfpathlineto{\pgfqpoint{1.217446in}{2.954469in}}%
\pgfpathlineto{\pgfqpoint{1.225682in}{2.954469in}}%
\pgfpathlineto{\pgfqpoint{1.229800in}{2.968525in}}%
\pgfpathlineto{\pgfqpoint{1.233918in}{2.968525in}}%
\pgfpathlineto{\pgfqpoint{1.238036in}{2.982581in}}%
\pgfpathlineto{\pgfqpoint{1.250391in}{2.982581in}}%
\pgfpathlineto{\pgfqpoint{1.254509in}{2.996637in}}%
\pgfpathlineto{\pgfqpoint{1.258627in}{2.996637in}}%
\pgfpathlineto{\pgfqpoint{1.262745in}{3.010694in}}%
\pgfpathlineto{\pgfqpoint{1.270981in}{3.010694in}}%
\pgfpathlineto{\pgfqpoint{1.275099in}{3.038806in}}%
\pgfpathlineto{\pgfqpoint{1.283335in}{3.038806in}}%
\pgfpathlineto{\pgfqpoint{1.287453in}{3.024750in}}%
\pgfpathlineto{\pgfqpoint{1.291571in}{3.038806in}}%
\pgfpathlineto{\pgfqpoint{1.295690in}{3.024750in}}%
\pgfpathlineto{\pgfqpoint{1.303926in}{3.024750in}}%
\pgfpathlineto{\pgfqpoint{1.312162in}{2.996637in}}%
\pgfpathlineto{\pgfqpoint{1.332752in}{2.996637in}}%
\pgfpathlineto{\pgfqpoint{1.336870in}{3.024750in}}%
\pgfpathlineto{\pgfqpoint{1.340988in}{3.024750in}}%
\pgfpathlineto{\pgfqpoint{1.345107in}{3.038806in}}%
\pgfpathlineto{\pgfqpoint{1.353343in}{3.038806in}}%
\pgfpathlineto{\pgfqpoint{1.357461in}{3.052862in}}%
\pgfpathlineto{\pgfqpoint{1.365697in}{3.052862in}}%
\pgfpathlineto{\pgfqpoint{1.373933in}{3.024750in}}%
\pgfpathlineto{\pgfqpoint{1.378051in}{3.024750in}}%
\pgfpathlineto{\pgfqpoint{1.386287in}{3.052862in}}%
\pgfpathlineto{\pgfqpoint{1.390405in}{3.052862in}}%
\pgfpathlineto{\pgfqpoint{1.394524in}{3.109087in}}%
\pgfpathlineto{\pgfqpoint{1.402760in}{3.080975in}}%
\pgfpathlineto{\pgfqpoint{1.406878in}{3.052862in}}%
\pgfpathlineto{\pgfqpoint{1.419232in}{3.052862in}}%
\pgfpathlineto{\pgfqpoint{1.423350in}{3.080975in}}%
\pgfpathlineto{\pgfqpoint{1.427468in}{3.080975in}}%
\pgfpathlineto{\pgfqpoint{1.431586in}{3.095031in}}%
\pgfpathlineto{\pgfqpoint{1.435704in}{3.095031in}}%
\pgfpathlineto{\pgfqpoint{1.439822in}{3.109087in}}%
\pgfpathlineto{\pgfqpoint{1.456295in}{3.109087in}}%
\pgfpathlineto{\pgfqpoint{1.460413in}{3.095031in}}%
\pgfpathlineto{\pgfqpoint{1.468649in}{3.123143in}}%
\pgfpathlineto{\pgfqpoint{1.472767in}{3.123143in}}%
\pgfpathlineto{\pgfqpoint{1.476885in}{3.095031in}}%
\pgfpathlineto{\pgfqpoint{1.481003in}{3.095031in}}%
\pgfpathlineto{\pgfqpoint{1.485121in}{3.080975in}}%
\pgfpathlineto{\pgfqpoint{1.489240in}{3.052862in}}%
\pgfpathlineto{\pgfqpoint{1.497476in}{3.052862in}}%
\pgfpathlineto{\pgfqpoint{1.501594in}{3.038806in}}%
\pgfpathlineto{\pgfqpoint{1.518066in}{3.038806in}}%
\pgfpathlineto{\pgfqpoint{1.530420in}{2.996637in}}%
\pgfpathlineto{\pgfqpoint{1.546893in}{2.996637in}}%
\pgfpathlineto{\pgfqpoint{1.551011in}{3.010694in}}%
\pgfpathlineto{\pgfqpoint{1.555129in}{3.010694in}}%
\pgfpathlineto{\pgfqpoint{1.559247in}{2.996637in}}%
\pgfpathlineto{\pgfqpoint{1.563365in}{2.996637in}}%
\pgfpathlineto{\pgfqpoint{1.567483in}{3.024750in}}%
\pgfpathlineto{\pgfqpoint{1.571601in}{2.996637in}}%
\pgfpathlineto{\pgfqpoint{1.575719in}{2.982581in}}%
\pgfpathlineto{\pgfqpoint{1.588074in}{2.982581in}}%
\pgfpathlineto{\pgfqpoint{1.592192in}{3.024750in}}%
\pgfpathlineto{\pgfqpoint{1.596310in}{3.038806in}}%
\pgfpathlineto{\pgfqpoint{1.600428in}{3.024750in}}%
\pgfpathlineto{\pgfqpoint{1.608664in}{3.024750in}}%
\pgfpathlineto{\pgfqpoint{1.612782in}{3.010694in}}%
\pgfpathlineto{\pgfqpoint{1.621018in}{3.010694in}}%
\pgfpathlineto{\pgfqpoint{1.625136in}{2.996637in}}%
\pgfpathlineto{\pgfqpoint{1.641609in}{2.996637in}}%
\pgfpathlineto{\pgfqpoint{1.645727in}{2.982581in}}%
\pgfpathlineto{\pgfqpoint{1.649845in}{2.996637in}}%
\pgfpathlineto{\pgfqpoint{1.653963in}{2.982581in}}%
\pgfpathlineto{\pgfqpoint{1.662199in}{2.982581in}}%
\pgfpathlineto{\pgfqpoint{1.666317in}{3.010694in}}%
\pgfpathlineto{\pgfqpoint{1.682789in}{3.010694in}}%
\pgfpathlineto{\pgfqpoint{1.686908in}{2.996637in}}%
\pgfpathlineto{\pgfqpoint{1.691026in}{3.024750in}}%
\pgfpathlineto{\pgfqpoint{1.699262in}{3.024750in}}%
\pgfpathlineto{\pgfqpoint{1.703380in}{3.038806in}}%
\pgfpathlineto{\pgfqpoint{1.707498in}{3.038806in}}%
\pgfpathlineto{\pgfqpoint{1.711616in}{3.024750in}}%
\pgfpathlineto{\pgfqpoint{1.736325in}{3.024750in}}%
\pgfpathlineto{\pgfqpoint{1.740443in}{3.010694in}}%
\pgfpathlineto{\pgfqpoint{1.785742in}{3.010694in}}%
\pgfpathlineto{\pgfqpoint{1.789860in}{2.996637in}}%
\pgfpathlineto{\pgfqpoint{1.793978in}{2.996637in}}%
\pgfpathlineto{\pgfqpoint{1.798096in}{2.982581in}}%
\pgfpathlineto{\pgfqpoint{1.802214in}{2.982581in}}%
\pgfpathlineto{\pgfqpoint{1.806332in}{2.968525in}}%
\pgfpathlineto{\pgfqpoint{1.810450in}{2.996637in}}%
\pgfpathlineto{\pgfqpoint{1.826922in}{2.996637in}}%
\pgfpathlineto{\pgfqpoint{1.831041in}{3.024750in}}%
\pgfpathlineto{\pgfqpoint{1.835159in}{3.024750in}}%
\pgfpathlineto{\pgfqpoint{1.839277in}{3.038806in}}%
\pgfpathlineto{\pgfqpoint{1.859867in}{3.038806in}}%
\pgfpathlineto{\pgfqpoint{1.863985in}{3.024750in}}%
\pgfpathlineto{\pgfqpoint{1.901048in}{3.024750in}}%
\pgfpathlineto{\pgfqpoint{1.905166in}{3.010694in}}%
\pgfpathlineto{\pgfqpoint{1.913402in}{3.010694in}}%
\pgfpathlineto{\pgfqpoint{1.917520in}{2.996637in}}%
\pgfpathlineto{\pgfqpoint{1.921638in}{3.010694in}}%
\pgfpathlineto{\pgfqpoint{1.933993in}{3.010694in}}%
\pgfpathlineto{\pgfqpoint{1.942229in}{3.038806in}}%
\pgfpathlineto{\pgfqpoint{1.946347in}{3.024750in}}%
\pgfpathlineto{\pgfqpoint{1.954583in}{3.024750in}}%
\pgfpathlineto{\pgfqpoint{1.958701in}{3.010694in}}%
\pgfpathlineto{\pgfqpoint{1.962819in}{3.024750in}}%
\pgfpathlineto{\pgfqpoint{1.971055in}{3.024750in}}%
\pgfpathlineto{\pgfqpoint{1.975173in}{3.038806in}}%
\pgfpathlineto{\pgfqpoint{1.983410in}{3.038806in}}%
\pgfpathlineto{\pgfqpoint{1.987528in}{3.052862in}}%
\pgfpathlineto{\pgfqpoint{2.008118in}{3.052862in}}%
\pgfpathlineto{\pgfqpoint{2.012236in}{3.066918in}}%
\pgfpathlineto{\pgfqpoint{2.020472in}{3.066918in}}%
\pgfpathlineto{\pgfqpoint{2.024591in}{3.052862in}}%
\pgfpathlineto{\pgfqpoint{2.045181in}{3.052862in}}%
\pgfpathlineto{\pgfqpoint{2.049299in}{3.080975in}}%
\pgfpathlineto{\pgfqpoint{2.053417in}{3.095031in}}%
\pgfpathlineto{\pgfqpoint{2.061653in}{3.095031in}}%
\pgfpathlineto{\pgfqpoint{2.074008in}{3.052862in}}%
\pgfpathlineto{\pgfqpoint{2.078126in}{3.052862in}}%
\pgfpathlineto{\pgfqpoint{2.082244in}{3.066918in}}%
\pgfpathlineto{\pgfqpoint{2.086362in}{3.066918in}}%
\pgfpathlineto{\pgfqpoint{2.094598in}{3.095031in}}%
\pgfpathlineto{\pgfqpoint{2.098716in}{3.095031in}}%
\pgfpathlineto{\pgfqpoint{2.102834in}{3.080975in}}%
\pgfpathlineto{\pgfqpoint{2.106952in}{3.080975in}}%
\pgfpathlineto{\pgfqpoint{2.119306in}{3.123143in}}%
\pgfpathlineto{\pgfqpoint{2.135779in}{3.123143in}}%
\pgfpathlineto{\pgfqpoint{2.139897in}{3.109087in}}%
\pgfpathlineto{\pgfqpoint{2.144015in}{3.109087in}}%
\pgfpathlineto{\pgfqpoint{2.152251in}{3.080975in}}%
\pgfpathlineto{\pgfqpoint{2.160487in}{3.080975in}}%
\pgfpathlineto{\pgfqpoint{2.168723in}{3.052862in}}%
\pgfpathlineto{\pgfqpoint{2.172842in}{3.052862in}}%
\pgfpathlineto{\pgfqpoint{2.176960in}{3.038806in}}%
\pgfpathlineto{\pgfqpoint{2.193432in}{3.038806in}}%
\pgfpathlineto{\pgfqpoint{2.197550in}{3.066918in}}%
\pgfpathlineto{\pgfqpoint{2.205786in}{3.066918in}}%
\pgfpathlineto{\pgfqpoint{2.214022in}{3.038806in}}%
\pgfpathlineto{\pgfqpoint{2.218140in}{3.038806in}}%
\pgfpathlineto{\pgfqpoint{2.222259in}{3.024750in}}%
\pgfpathlineto{\pgfqpoint{2.226377in}{3.024750in}}%
\pgfpathlineto{\pgfqpoint{2.230495in}{3.010694in}}%
\pgfpathlineto{\pgfqpoint{2.234613in}{3.010694in}}%
\pgfpathlineto{\pgfqpoint{2.238731in}{2.996637in}}%
\pgfpathlineto{\pgfqpoint{2.242849in}{2.968525in}}%
\pgfpathlineto{\pgfqpoint{2.246967in}{2.968525in}}%
\pgfpathlineto{\pgfqpoint{2.251085in}{2.954469in}}%
\pgfpathlineto{\pgfqpoint{2.292266in}{2.954469in}}%
\pgfpathlineto{\pgfqpoint{2.296384in}{2.940412in}}%
\pgfpathlineto{\pgfqpoint{2.378746in}{2.940412in}}%
\pgfpathlineto{\pgfqpoint{2.386982in}{2.968525in}}%
\pgfpathlineto{\pgfqpoint{2.399336in}{2.968525in}}%
\pgfpathlineto{\pgfqpoint{2.403454in}{2.982581in}}%
\pgfpathlineto{\pgfqpoint{2.407572in}{3.010694in}}%
\pgfpathlineto{\pgfqpoint{2.415809in}{3.010694in}}%
\pgfpathlineto{\pgfqpoint{2.419927in}{2.996637in}}%
\pgfpathlineto{\pgfqpoint{2.444635in}{2.996637in}}%
\pgfpathlineto{\pgfqpoint{2.448753in}{3.010694in}}%
\pgfpathlineto{\pgfqpoint{2.452871in}{2.996637in}}%
\pgfpathlineto{\pgfqpoint{2.461107in}{2.996637in}}%
\pgfpathlineto{\pgfqpoint{2.465226in}{3.010694in}}%
\pgfpathlineto{\pgfqpoint{2.477580in}{3.010694in}}%
\pgfpathlineto{\pgfqpoint{2.481698in}{3.024750in}}%
\pgfpathlineto{\pgfqpoint{2.485816in}{3.024750in}}%
\pgfpathlineto{\pgfqpoint{2.489934in}{3.038806in}}%
\pgfpathlineto{\pgfqpoint{2.494052in}{3.038806in}}%
\pgfpathlineto{\pgfqpoint{2.498170in}{3.024750in}}%
\pgfpathlineto{\pgfqpoint{2.522879in}{3.024750in}}%
\pgfpathlineto{\pgfqpoint{2.526997in}{3.010694in}}%
\pgfpathlineto{\pgfqpoint{2.543469in}{3.010694in}}%
\pgfpathlineto{\pgfqpoint{2.547587in}{2.996637in}}%
\pgfpathlineto{\pgfqpoint{2.551705in}{2.996637in}}%
\pgfpathlineto{\pgfqpoint{2.555823in}{2.982581in}}%
\pgfpathlineto{\pgfqpoint{2.564060in}{2.982581in}}%
\pgfpathlineto{\pgfqpoint{2.568178in}{2.968525in}}%
\pgfpathlineto{\pgfqpoint{2.572296in}{2.968525in}}%
\pgfpathlineto{\pgfqpoint{2.576414in}{2.982581in}}%
\pgfpathlineto{\pgfqpoint{2.592886in}{2.982581in}}%
\pgfpathlineto{\pgfqpoint{2.597004in}{2.996637in}}%
\pgfpathlineto{\pgfqpoint{2.621713in}{2.996637in}}%
\pgfpathlineto{\pgfqpoint{2.625831in}{2.982581in}}%
\pgfpathlineto{\pgfqpoint{2.629949in}{2.982581in}}%
\pgfpathlineto{\pgfqpoint{2.634067in}{2.996637in}}%
\pgfpathlineto{\pgfqpoint{2.646421in}{2.996637in}}%
\pgfpathlineto{\pgfqpoint{2.646421in}{2.996637in}}%
\pgfusepath{stroke}%
\end{pgfscope}%
\begin{pgfscope}%
\pgfsetrectcap%
\pgfsetmiterjoin%
\pgfsetlinewidth{0.501875pt}%
\definecolor{currentstroke}{rgb}{0.317647,0.317647,0.317647}%
\pgfsetstrokecolor{currentstroke}%
\pgfsetdash{}{0pt}%
\pgfpathmoveto{\pgfqpoint{0.488751in}{2.165212in}}%
\pgfpathlineto{\pgfqpoint{0.488751in}{3.448545in}}%
\pgfusepath{stroke}%
\end{pgfscope}%
\begin{pgfscope}%
\pgfsetrectcap%
\pgfsetmiterjoin%
\pgfsetlinewidth{0.501875pt}%
\definecolor{currentstroke}{rgb}{0.317647,0.317647,0.317647}%
\pgfsetstrokecolor{currentstroke}%
\pgfsetdash{}{0pt}%
\pgfpathmoveto{\pgfqpoint{0.488751in}{2.165212in}}%
\pgfpathlineto{\pgfqpoint{2.749168in}{2.165212in}}%
\pgfusepath{stroke}%
\end{pgfscope}%
\begin{pgfscope}%
\pgfsetrectcap%
\pgfsetroundjoin%
\pgfsetlinewidth{0.803000pt}%
\definecolor{currentstroke}{rgb}{0.333333,0.333333,0.333333}%
\pgfsetstrokecolor{currentstroke}%
\pgfsetdash{}{0pt}%
\pgfpathmoveto{\pgfqpoint{2.703959in}{3.199656in}}%
\pgfpathlineto{\pgfqpoint{2.748404in}{3.199656in}}%
\pgfusepath{stroke}%
\end{pgfscope}%
\begin{pgfscope}%
\definecolor{textcolor}{rgb}{0.000000,0.000000,0.000000}%
\pgfsetstrokecolor{textcolor}%
\pgfsetfillcolor{textcolor}%
\pgftext[x=2.776181in,y=3.180212in,left,base]{\color{textcolor}\rmfamily\fontsize{4.000000}{4.800000}\selectfont \(\displaystyle w_{00}\)}%
\end{pgfscope}%
\begin{pgfscope}%
\pgfsetrectcap%
\pgfsetroundjoin%
\pgfsetlinewidth{0.803000pt}%
\definecolor{currentstroke}{rgb}{0.686275,0.352941,0.313725}%
\pgfsetstrokecolor{currentstroke}%
\pgfsetdash{}{0pt}%
\pgfpathmoveto{\pgfqpoint{2.703959in}{3.117712in}}%
\pgfpathlineto{\pgfqpoint{2.748404in}{3.117712in}}%
\pgfusepath{stroke}%
\end{pgfscope}%
\begin{pgfscope}%
\definecolor{textcolor}{rgb}{0.000000,0.000000,0.000000}%
\pgfsetstrokecolor{textcolor}%
\pgfsetfillcolor{textcolor}%
\pgftext[x=2.776181in,y=3.098267in,left,base]{\color{textcolor}\rmfamily\fontsize{4.000000}{4.800000}\selectfont \(\displaystyle w_{01}\)}%
\end{pgfscope}%
\begin{pgfscope}%
\pgfsetrectcap%
\pgfsetroundjoin%
\pgfsetlinewidth{0.803000pt}%
\definecolor{currentstroke}{rgb}{0.000000,0.356863,0.509804}%
\pgfsetstrokecolor{currentstroke}%
\pgfsetdash{}{0pt}%
\pgfpathmoveto{\pgfqpoint{2.703959in}{3.035767in}}%
\pgfpathlineto{\pgfqpoint{2.748404in}{3.035767in}}%
\pgfusepath{stroke}%
\end{pgfscope}%
\begin{pgfscope}%
\definecolor{textcolor}{rgb}{0.000000,0.000000,0.000000}%
\pgfsetstrokecolor{textcolor}%
\pgfsetfillcolor{textcolor}%
\pgftext[x=2.776181in,y=3.016323in,left,base]{\color{textcolor}\rmfamily\fontsize{4.000000}{4.800000}\selectfont \(\displaystyle w_{02}\)}%
\end{pgfscope}%
\begin{pgfscope}%
\pgfsetrectcap%
\pgfsetroundjoin%
\pgfsetlinewidth{0.803000pt}%
\definecolor{currentstroke}{rgb}{0.490196,0.588235,0.431373}%
\pgfsetstrokecolor{currentstroke}%
\pgfsetdash{}{0pt}%
\pgfpathmoveto{\pgfqpoint{2.703959in}{2.953823in}}%
\pgfpathlineto{\pgfqpoint{2.748404in}{2.953823in}}%
\pgfusepath{stroke}%
\end{pgfscope}%
\begin{pgfscope}%
\definecolor{textcolor}{rgb}{0.000000,0.000000,0.000000}%
\pgfsetstrokecolor{textcolor}%
\pgfsetfillcolor{textcolor}%
\pgftext[x=2.776181in,y=2.934378in,left,base]{\color{textcolor}\rmfamily\fontsize{4.000000}{4.800000}\selectfont \(\displaystyle w_{03}\)}%
\end{pgfscope}%
\begin{pgfscope}%
\pgfsetrectcap%
\pgfsetroundjoin%
\pgfsetlinewidth{0.803000pt}%
\definecolor{currentstroke}{rgb}{0.843137,0.666667,0.313725}%
\pgfsetstrokecolor{currentstroke}%
\pgfsetdash{}{0pt}%
\pgfpathmoveto{\pgfqpoint{2.703959in}{2.871878in}}%
\pgfpathlineto{\pgfqpoint{2.748404in}{2.871878in}}%
\pgfusepath{stroke}%
\end{pgfscope}%
\begin{pgfscope}%
\definecolor{textcolor}{rgb}{0.000000,0.000000,0.000000}%
\pgfsetstrokecolor{textcolor}%
\pgfsetfillcolor{textcolor}%
\pgftext[x=2.776181in,y=2.852434in,left,base]{\color{textcolor}\rmfamily\fontsize{4.000000}{4.800000}\selectfont \(\displaystyle w_{04}\)}%
\end{pgfscope}%
\begin{pgfscope}%
\pgfsetbuttcap%
\pgfsetroundjoin%
\pgfsetlinewidth{0.803000pt}%
\definecolor{currentstroke}{rgb}{0.333333,0.333333,0.333333}%
\pgfsetstrokecolor{currentstroke}%
\pgfsetdash{{2.960000pt}{1.280000pt}}{0.000000pt}%
\pgfpathmoveto{\pgfqpoint{2.703959in}{2.789934in}}%
\pgfpathlineto{\pgfqpoint{2.748404in}{2.789934in}}%
\pgfusepath{stroke}%
\end{pgfscope}%
\begin{pgfscope}%
\definecolor{textcolor}{rgb}{0.000000,0.000000,0.000000}%
\pgfsetstrokecolor{textcolor}%
\pgfsetfillcolor{textcolor}%
\pgftext[x=2.776181in,y=2.770489in,left,base]{\color{textcolor}\rmfamily\fontsize{4.000000}{4.800000}\selectfont \(\displaystyle w_{10}\)}%
\end{pgfscope}%
\begin{pgfscope}%
\pgfsetbuttcap%
\pgfsetroundjoin%
\pgfsetlinewidth{0.803000pt}%
\definecolor{currentstroke}{rgb}{0.686275,0.352941,0.313725}%
\pgfsetstrokecolor{currentstroke}%
\pgfsetdash{{2.960000pt}{1.280000pt}}{0.000000pt}%
\pgfpathmoveto{\pgfqpoint{2.703959in}{2.707989in}}%
\pgfpathlineto{\pgfqpoint{2.748404in}{2.707989in}}%
\pgfusepath{stroke}%
\end{pgfscope}%
\begin{pgfscope}%
\definecolor{textcolor}{rgb}{0.000000,0.000000,0.000000}%
\pgfsetstrokecolor{textcolor}%
\pgfsetfillcolor{textcolor}%
\pgftext[x=2.776181in,y=2.688545in,left,base]{\color{textcolor}\rmfamily\fontsize{4.000000}{4.800000}\selectfont \(\displaystyle w_{11}\)}%
\end{pgfscope}%
\begin{pgfscope}%
\pgfsetbuttcap%
\pgfsetroundjoin%
\pgfsetlinewidth{0.803000pt}%
\definecolor{currentstroke}{rgb}{0.000000,0.356863,0.509804}%
\pgfsetstrokecolor{currentstroke}%
\pgfsetdash{{2.960000pt}{1.280000pt}}{0.000000pt}%
\pgfpathmoveto{\pgfqpoint{2.703959in}{2.626045in}}%
\pgfpathlineto{\pgfqpoint{2.748404in}{2.626045in}}%
\pgfusepath{stroke}%
\end{pgfscope}%
\begin{pgfscope}%
\definecolor{textcolor}{rgb}{0.000000,0.000000,0.000000}%
\pgfsetstrokecolor{textcolor}%
\pgfsetfillcolor{textcolor}%
\pgftext[x=2.776181in,y=2.606601in,left,base]{\color{textcolor}\rmfamily\fontsize{4.000000}{4.800000}\selectfont \(\displaystyle w_{12}\)}%
\end{pgfscope}%
\begin{pgfscope}%
\pgfsetbuttcap%
\pgfsetroundjoin%
\pgfsetlinewidth{0.803000pt}%
\definecolor{currentstroke}{rgb}{0.490196,0.588235,0.431373}%
\pgfsetstrokecolor{currentstroke}%
\pgfsetdash{{2.960000pt}{1.280000pt}}{0.000000pt}%
\pgfpathmoveto{\pgfqpoint{2.703959in}{2.544101in}}%
\pgfpathlineto{\pgfqpoint{2.748404in}{2.544101in}}%
\pgfusepath{stroke}%
\end{pgfscope}%
\begin{pgfscope}%
\definecolor{textcolor}{rgb}{0.000000,0.000000,0.000000}%
\pgfsetstrokecolor{textcolor}%
\pgfsetfillcolor{textcolor}%
\pgftext[x=2.776181in,y=2.524656in,left,base]{\color{textcolor}\rmfamily\fontsize{4.000000}{4.800000}\selectfont \(\displaystyle w_{13}\)}%
\end{pgfscope}%
\begin{pgfscope}%
\pgfsetbuttcap%
\pgfsetroundjoin%
\pgfsetlinewidth{0.803000pt}%
\definecolor{currentstroke}{rgb}{0.843137,0.666667,0.313725}%
\pgfsetstrokecolor{currentstroke}%
\pgfsetdash{{2.960000pt}{1.280000pt}}{0.000000pt}%
\pgfpathmoveto{\pgfqpoint{2.703959in}{2.462156in}}%
\pgfpathlineto{\pgfqpoint{2.748404in}{2.462156in}}%
\pgfusepath{stroke}%
\end{pgfscope}%
\begin{pgfscope}%
\definecolor{textcolor}{rgb}{0.000000,0.000000,0.000000}%
\pgfsetstrokecolor{textcolor}%
\pgfsetfillcolor{textcolor}%
\pgftext[x=2.776181in,y=2.442712in,left,base]{\color{textcolor}\rmfamily\fontsize{4.000000}{4.800000}\selectfont \(\displaystyle w_{14}\)}%
\end{pgfscope}%
\begin{pgfscope}%
\pgfsetbuttcap%
\pgfsetmiterjoin%
\pgfsetlinewidth{0.000000pt}%
\definecolor{currentstroke}{rgb}{0.000000,0.000000,0.000000}%
\pgfsetstrokecolor{currentstroke}%
\pgfsetstrokeopacity{0.000000}%
\pgfsetdash{}{0pt}%
\pgfpathmoveto{\pgfqpoint{3.653334in}{2.165212in}}%
\pgfpathlineto{\pgfqpoint{5.913751in}{2.165212in}}%
\pgfpathlineto{\pgfqpoint{5.913751in}{3.448545in}}%
\pgfpathlineto{\pgfqpoint{3.653334in}{3.448545in}}%
\pgfpathclose%
\pgfusepath{}%
\end{pgfscope}%
\begin{pgfscope}%
\pgfsetbuttcap%
\pgfsetroundjoin%
\definecolor{currentfill}{rgb}{0.317647,0.317647,0.317647}%
\pgfsetfillcolor{currentfill}%
\pgfsetlinewidth{0.501875pt}%
\definecolor{currentstroke}{rgb}{0.317647,0.317647,0.317647}%
\pgfsetstrokecolor{currentstroke}%
\pgfsetdash{}{0pt}%
\pgfsys@defobject{currentmarker}{\pgfqpoint{0.000000in}{-0.020833in}}{\pgfqpoint{0.000000in}{0.000000in}}{%
\pgfpathmoveto{\pgfqpoint{0.000000in}{0.000000in}}%
\pgfpathlineto{\pgfqpoint{0.000000in}{-0.020833in}}%
\pgfusepath{stroke,fill}%
}%
\begin{pgfscope}%
\pgfsys@transformshift{3.756080in}{2.165212in}%
\pgfsys@useobject{currentmarker}{}%
\end{pgfscope}%
\end{pgfscope}%
\begin{pgfscope}%
\pgfsetbuttcap%
\pgfsetroundjoin%
\definecolor{currentfill}{rgb}{0.317647,0.317647,0.317647}%
\pgfsetfillcolor{currentfill}%
\pgfsetlinewidth{0.501875pt}%
\definecolor{currentstroke}{rgb}{0.317647,0.317647,0.317647}%
\pgfsetstrokecolor{currentstroke}%
\pgfsetdash{}{0pt}%
\pgfsys@defobject{currentmarker}{\pgfqpoint{0.000000in}{-0.020833in}}{\pgfqpoint{0.000000in}{0.000000in}}{%
\pgfpathmoveto{\pgfqpoint{0.000000in}{0.000000in}}%
\pgfpathlineto{\pgfqpoint{0.000000in}{-0.020833in}}%
\pgfusepath{stroke,fill}%
}%
\begin{pgfscope}%
\pgfsys@transformshift{4.167889in}{2.165212in}%
\pgfsys@useobject{currentmarker}{}%
\end{pgfscope}%
\end{pgfscope}%
\begin{pgfscope}%
\pgfsetbuttcap%
\pgfsetroundjoin%
\definecolor{currentfill}{rgb}{0.317647,0.317647,0.317647}%
\pgfsetfillcolor{currentfill}%
\pgfsetlinewidth{0.501875pt}%
\definecolor{currentstroke}{rgb}{0.317647,0.317647,0.317647}%
\pgfsetstrokecolor{currentstroke}%
\pgfsetdash{}{0pt}%
\pgfsys@defobject{currentmarker}{\pgfqpoint{0.000000in}{-0.020833in}}{\pgfqpoint{0.000000in}{0.000000in}}{%
\pgfpathmoveto{\pgfqpoint{0.000000in}{0.000000in}}%
\pgfpathlineto{\pgfqpoint{0.000000in}{-0.020833in}}%
\pgfusepath{stroke,fill}%
}%
\begin{pgfscope}%
\pgfsys@transformshift{4.579697in}{2.165212in}%
\pgfsys@useobject{currentmarker}{}%
\end{pgfscope}%
\end{pgfscope}%
\begin{pgfscope}%
\pgfsetbuttcap%
\pgfsetroundjoin%
\definecolor{currentfill}{rgb}{0.317647,0.317647,0.317647}%
\pgfsetfillcolor{currentfill}%
\pgfsetlinewidth{0.501875pt}%
\definecolor{currentstroke}{rgb}{0.317647,0.317647,0.317647}%
\pgfsetstrokecolor{currentstroke}%
\pgfsetdash{}{0pt}%
\pgfsys@defobject{currentmarker}{\pgfqpoint{0.000000in}{-0.020833in}}{\pgfqpoint{0.000000in}{0.000000in}}{%
\pgfpathmoveto{\pgfqpoint{0.000000in}{0.000000in}}%
\pgfpathlineto{\pgfqpoint{0.000000in}{-0.020833in}}%
\pgfusepath{stroke,fill}%
}%
\begin{pgfscope}%
\pgfsys@transformshift{4.991506in}{2.165212in}%
\pgfsys@useobject{currentmarker}{}%
\end{pgfscope}%
\end{pgfscope}%
\begin{pgfscope}%
\pgfsetbuttcap%
\pgfsetroundjoin%
\definecolor{currentfill}{rgb}{0.317647,0.317647,0.317647}%
\pgfsetfillcolor{currentfill}%
\pgfsetlinewidth{0.501875pt}%
\definecolor{currentstroke}{rgb}{0.317647,0.317647,0.317647}%
\pgfsetstrokecolor{currentstroke}%
\pgfsetdash{}{0pt}%
\pgfsys@defobject{currentmarker}{\pgfqpoint{0.000000in}{-0.020833in}}{\pgfqpoint{0.000000in}{0.000000in}}{%
\pgfpathmoveto{\pgfqpoint{0.000000in}{0.000000in}}%
\pgfpathlineto{\pgfqpoint{0.000000in}{-0.020833in}}%
\pgfusepath{stroke,fill}%
}%
\begin{pgfscope}%
\pgfsys@transformshift{5.403314in}{2.165212in}%
\pgfsys@useobject{currentmarker}{}%
\end{pgfscope}%
\end{pgfscope}%
\begin{pgfscope}%
\pgfsetbuttcap%
\pgfsetroundjoin%
\definecolor{currentfill}{rgb}{0.317647,0.317647,0.317647}%
\pgfsetfillcolor{currentfill}%
\pgfsetlinewidth{0.501875pt}%
\definecolor{currentstroke}{rgb}{0.317647,0.317647,0.317647}%
\pgfsetstrokecolor{currentstroke}%
\pgfsetdash{}{0pt}%
\pgfsys@defobject{currentmarker}{\pgfqpoint{0.000000in}{-0.020833in}}{\pgfqpoint{0.000000in}{0.000000in}}{%
\pgfpathmoveto{\pgfqpoint{0.000000in}{0.000000in}}%
\pgfpathlineto{\pgfqpoint{0.000000in}{-0.020833in}}%
\pgfusepath{stroke,fill}%
}%
\begin{pgfscope}%
\pgfsys@transformshift{5.815123in}{2.165212in}%
\pgfsys@useobject{currentmarker}{}%
\end{pgfscope}%
\end{pgfscope}%
\begin{pgfscope}%
\pgfsetbuttcap%
\pgfsetroundjoin%
\definecolor{currentfill}{rgb}{0.317647,0.317647,0.317647}%
\pgfsetfillcolor{currentfill}%
\pgfsetlinewidth{0.501875pt}%
\definecolor{currentstroke}{rgb}{0.317647,0.317647,0.317647}%
\pgfsetstrokecolor{currentstroke}%
\pgfsetdash{}{0pt}%
\pgfsys@defobject{currentmarker}{\pgfqpoint{-0.020833in}{0.000000in}}{\pgfqpoint{0.000000in}{0.000000in}}{%
\pgfpathmoveto{\pgfqpoint{0.000000in}{0.000000in}}%
\pgfpathlineto{\pgfqpoint{-0.020833in}{0.000000in}}%
\pgfusepath{stroke,fill}%
}%
\begin{pgfscope}%
\pgfsys@transformshift{3.653334in}{2.501323in}%
\pgfsys@useobject{currentmarker}{}%
\end{pgfscope}%
\end{pgfscope}%
\begin{pgfscope}%
\definecolor{textcolor}{rgb}{0.317647,0.317647,0.317647}%
\pgfsetstrokecolor{textcolor}%
\pgfsetfillcolor{textcolor}%
\pgftext[x=3.452523in,y=2.469206in,left,base]{\color{textcolor}\rmfamily\fontsize{6.664000}{7.996800}\selectfont \(\displaystyle 280\)}%
\end{pgfscope}%
\begin{pgfscope}%
\pgfsetbuttcap%
\pgfsetroundjoin%
\definecolor{currentfill}{rgb}{0.317647,0.317647,0.317647}%
\pgfsetfillcolor{currentfill}%
\pgfsetlinewidth{0.501875pt}%
\definecolor{currentstroke}{rgb}{0.317647,0.317647,0.317647}%
\pgfsetstrokecolor{currentstroke}%
\pgfsetdash{}{0pt}%
\pgfsys@defobject{currentmarker}{\pgfqpoint{-0.020833in}{0.000000in}}{\pgfqpoint{0.000000in}{0.000000in}}{%
\pgfpathmoveto{\pgfqpoint{0.000000in}{0.000000in}}%
\pgfpathlineto{\pgfqpoint{-0.020833in}{0.000000in}}%
\pgfusepath{stroke,fill}%
}%
\begin{pgfscope}%
\pgfsys@transformshift{3.653334in}{2.871693in}%
\pgfsys@useobject{currentmarker}{}%
\end{pgfscope}%
\end{pgfscope}%
\begin{pgfscope}%
\definecolor{textcolor}{rgb}{0.317647,0.317647,0.317647}%
\pgfsetstrokecolor{textcolor}%
\pgfsetfillcolor{textcolor}%
\pgftext[x=3.452523in,y=2.839576in,left,base]{\color{textcolor}\rmfamily\fontsize{6.664000}{7.996800}\selectfont \(\displaystyle 300\)}%
\end{pgfscope}%
\begin{pgfscope}%
\pgfsetbuttcap%
\pgfsetroundjoin%
\definecolor{currentfill}{rgb}{0.317647,0.317647,0.317647}%
\pgfsetfillcolor{currentfill}%
\pgfsetlinewidth{0.501875pt}%
\definecolor{currentstroke}{rgb}{0.317647,0.317647,0.317647}%
\pgfsetstrokecolor{currentstroke}%
\pgfsetdash{}{0pt}%
\pgfsys@defobject{currentmarker}{\pgfqpoint{-0.020833in}{0.000000in}}{\pgfqpoint{0.000000in}{0.000000in}}{%
\pgfpathmoveto{\pgfqpoint{0.000000in}{0.000000in}}%
\pgfpathlineto{\pgfqpoint{-0.020833in}{0.000000in}}%
\pgfusepath{stroke,fill}%
}%
\begin{pgfscope}%
\pgfsys@transformshift{3.653334in}{3.242064in}%
\pgfsys@useobject{currentmarker}{}%
\end{pgfscope}%
\end{pgfscope}%
\begin{pgfscope}%
\definecolor{textcolor}{rgb}{0.317647,0.317647,0.317647}%
\pgfsetstrokecolor{textcolor}%
\pgfsetfillcolor{textcolor}%
\pgftext[x=3.452523in,y=3.209947in,left,base]{\color{textcolor}\rmfamily\fontsize{6.664000}{7.996800}\selectfont \(\displaystyle 320\)}%
\end{pgfscope}%
\begin{pgfscope}%
\definecolor{textcolor}{rgb}{0.317647,0.317647,0.317647}%
\pgfsetstrokecolor{textcolor}%
\pgfsetfillcolor{textcolor}%
\pgftext[x=3.396968in,y=2.806878in,,bottom,rotate=90.000000]{\color{textcolor}\rmfamily\fontsize{6.664000}{7.996800}\selectfont \(\displaystyle \vartheta \propto -b^{(\mathrm{h})} \; (\si{\milli \V})\)}%
\end{pgfscope}%
\begin{pgfscope}%
\pgfpathrectangle{\pgfqpoint{3.653334in}{2.165212in}}{\pgfqpoint{2.260417in}{1.283333in}}%
\pgfusepath{clip}%
\pgfsetrectcap%
\pgfsetroundjoin%
\pgfsetlinewidth{0.803000pt}%
\definecolor{currentstroke}{rgb}{0.333333,0.333333,0.333333}%
\pgfsetstrokecolor{currentstroke}%
\pgfsetdash{}{0pt}%
\pgfpathmoveto{\pgfqpoint{3.756080in}{3.260582in}}%
\pgfpathlineto{\pgfqpoint{3.760198in}{3.279101in}}%
\pgfpathlineto{\pgfqpoint{3.784907in}{3.279101in}}%
\pgfpathlineto{\pgfqpoint{3.789025in}{3.297619in}}%
\pgfpathlineto{\pgfqpoint{3.797261in}{3.297619in}}%
\pgfpathlineto{\pgfqpoint{3.801379in}{3.316138in}}%
\pgfpathlineto{\pgfqpoint{3.809615in}{3.316138in}}%
\pgfpathlineto{\pgfqpoint{3.813734in}{3.334656in}}%
\pgfpathlineto{\pgfqpoint{3.817852in}{3.316138in}}%
\pgfpathlineto{\pgfqpoint{3.826088in}{3.316138in}}%
\pgfpathlineto{\pgfqpoint{3.830206in}{3.279101in}}%
\pgfpathlineto{\pgfqpoint{3.834324in}{3.279101in}}%
\pgfpathlineto{\pgfqpoint{3.838442in}{3.260582in}}%
\pgfpathlineto{\pgfqpoint{3.846678in}{3.260582in}}%
\pgfpathlineto{\pgfqpoint{3.859032in}{3.205026in}}%
\pgfpathlineto{\pgfqpoint{4.052582in}{3.205026in}}%
\pgfpathlineto{\pgfqpoint{4.056701in}{3.223545in}}%
\pgfpathlineto{\pgfqpoint{4.069055in}{3.223545in}}%
\pgfpathlineto{\pgfqpoint{4.073173in}{3.242064in}}%
\pgfpathlineto{\pgfqpoint{4.077291in}{3.223545in}}%
\pgfpathlineto{\pgfqpoint{4.122590in}{3.223545in}}%
\pgfpathlineto{\pgfqpoint{4.126708in}{3.205026in}}%
\pgfpathlineto{\pgfqpoint{4.167889in}{3.205026in}}%
\pgfpathlineto{\pgfqpoint{4.172007in}{3.223545in}}%
\pgfpathlineto{\pgfqpoint{4.221424in}{3.223545in}}%
\pgfpathlineto{\pgfqpoint{4.225542in}{3.205026in}}%
\pgfpathlineto{\pgfqpoint{4.254369in}{3.205026in}}%
\pgfpathlineto{\pgfqpoint{4.258487in}{3.223545in}}%
\pgfpathlineto{\pgfqpoint{4.402620in}{3.223545in}}%
\pgfpathlineto{\pgfqpoint{4.406738in}{3.205026in}}%
\pgfpathlineto{\pgfqpoint{4.501454in}{3.205026in}}%
\pgfpathlineto{\pgfqpoint{4.505572in}{3.223545in}}%
\pgfpathlineto{\pgfqpoint{4.653823in}{3.223545in}}%
\pgfpathlineto{\pgfqpoint{4.657941in}{3.205026in}}%
\pgfpathlineto{\pgfqpoint{4.744421in}{3.205026in}}%
\pgfpathlineto{\pgfqpoint{4.748539in}{3.223545in}}%
\pgfpathlineto{\pgfqpoint{4.954443in}{3.223545in}}%
\pgfpathlineto{\pgfqpoint{4.958561in}{3.205026in}}%
\pgfpathlineto{\pgfqpoint{4.983270in}{3.205026in}}%
\pgfpathlineto{\pgfqpoint{4.987388in}{3.223545in}}%
\pgfpathlineto{\pgfqpoint{5.049159in}{3.223545in}}%
\pgfpathlineto{\pgfqpoint{5.053277in}{3.242064in}}%
\pgfpathlineto{\pgfqpoint{5.230355in}{3.242064in}}%
\pgfpathlineto{\pgfqpoint{5.234473in}{3.260582in}}%
\pgfpathlineto{\pgfqpoint{5.436259in}{3.260582in}}%
\pgfpathlineto{\pgfqpoint{5.440377in}{3.279101in}}%
\pgfpathlineto{\pgfqpoint{5.452731in}{3.279101in}}%
\pgfpathlineto{\pgfqpoint{5.456849in}{3.297619in}}%
\pgfpathlineto{\pgfqpoint{5.473322in}{3.297619in}}%
\pgfpathlineto{\pgfqpoint{5.477440in}{3.316138in}}%
\pgfpathlineto{\pgfqpoint{5.485676in}{3.316138in}}%
\pgfpathlineto{\pgfqpoint{5.489794in}{3.334656in}}%
\pgfpathlineto{\pgfqpoint{5.568038in}{3.334656in}}%
\pgfpathlineto{\pgfqpoint{5.572156in}{3.353175in}}%
\pgfpathlineto{\pgfqpoint{5.576274in}{3.334656in}}%
\pgfpathlineto{\pgfqpoint{5.638045in}{3.334656in}}%
\pgfpathlineto{\pgfqpoint{5.642163in}{3.316138in}}%
\pgfpathlineto{\pgfqpoint{5.811005in}{3.316138in}}%
\pgfpathlineto{\pgfqpoint{5.811005in}{3.316138in}}%
\pgfusepath{stroke}%
\end{pgfscope}%
\begin{pgfscope}%
\pgfpathrectangle{\pgfqpoint{3.653334in}{2.165212in}}{\pgfqpoint{2.260417in}{1.283333in}}%
\pgfusepath{clip}%
\pgfsetrectcap%
\pgfsetroundjoin%
\pgfsetlinewidth{0.803000pt}%
\definecolor{currentstroke}{rgb}{0.686275,0.352941,0.313725}%
\pgfsetstrokecolor{currentstroke}%
\pgfsetdash{}{0pt}%
\pgfpathmoveto{\pgfqpoint{3.756080in}{2.538360in}}%
\pgfpathlineto{\pgfqpoint{3.805497in}{2.538360in}}%
\pgfpathlineto{\pgfqpoint{3.809615in}{2.519841in}}%
\pgfpathlineto{\pgfqpoint{3.817852in}{2.519841in}}%
\pgfpathlineto{\pgfqpoint{3.821970in}{2.501323in}}%
\pgfpathlineto{\pgfqpoint{3.826088in}{2.501323in}}%
\pgfpathlineto{\pgfqpoint{3.830206in}{2.519841in}}%
\pgfpathlineto{\pgfqpoint{3.945512in}{2.519841in}}%
\pgfpathlineto{\pgfqpoint{3.949630in}{2.501323in}}%
\pgfpathlineto{\pgfqpoint{3.978457in}{2.501323in}}%
\pgfpathlineto{\pgfqpoint{3.982575in}{2.482804in}}%
\pgfpathlineto{\pgfqpoint{4.027874in}{2.482804in}}%
\pgfpathlineto{\pgfqpoint{4.031992in}{2.464286in}}%
\pgfpathlineto{\pgfqpoint{4.184361in}{2.464286in}}%
\pgfpathlineto{\pgfqpoint{4.188479in}{2.445767in}}%
\pgfpathlineto{\pgfqpoint{4.237896in}{2.445767in}}%
\pgfpathlineto{\pgfqpoint{4.242014in}{2.427249in}}%
\pgfpathlineto{\pgfqpoint{4.246132in}{2.427249in}}%
\pgfpathlineto{\pgfqpoint{4.250251in}{2.408730in}}%
\pgfpathlineto{\pgfqpoint{4.386147in}{2.408730in}}%
\pgfpathlineto{\pgfqpoint{4.390265in}{2.427249in}}%
\pgfpathlineto{\pgfqpoint{4.414974in}{2.427249in}}%
\pgfpathlineto{\pgfqpoint{4.419092in}{2.445767in}}%
\pgfpathlineto{\pgfqpoint{4.567343in}{2.445767in}}%
\pgfpathlineto{\pgfqpoint{4.571461in}{2.427249in}}%
\pgfpathlineto{\pgfqpoint{4.604406in}{2.427249in}}%
\pgfpathlineto{\pgfqpoint{4.608524in}{2.408730in}}%
\pgfpathlineto{\pgfqpoint{4.620878in}{2.408730in}}%
\pgfpathlineto{\pgfqpoint{4.624996in}{2.390212in}}%
\pgfpathlineto{\pgfqpoint{4.637350in}{2.390212in}}%
\pgfpathlineto{\pgfqpoint{4.641469in}{2.371693in}}%
\pgfpathlineto{\pgfqpoint{4.686768in}{2.371693in}}%
\pgfpathlineto{\pgfqpoint{4.690886in}{2.390212in}}%
\pgfpathlineto{\pgfqpoint{4.727948in}{2.390212in}}%
\pgfpathlineto{\pgfqpoint{4.732066in}{2.371693in}}%
\pgfpathlineto{\pgfqpoint{4.744421in}{2.371693in}}%
\pgfpathlineto{\pgfqpoint{4.748539in}{2.353175in}}%
\pgfpathlineto{\pgfqpoint{4.760893in}{2.353175in}}%
\pgfpathlineto{\pgfqpoint{4.765011in}{2.334656in}}%
\pgfpathlineto{\pgfqpoint{4.810310in}{2.334656in}}%
\pgfpathlineto{\pgfqpoint{4.814428in}{2.316138in}}%
\pgfpathlineto{\pgfqpoint{4.830900in}{2.316138in}}%
\pgfpathlineto{\pgfqpoint{4.835019in}{2.297619in}}%
\pgfpathlineto{\pgfqpoint{4.863845in}{2.297619in}}%
\pgfpathlineto{\pgfqpoint{4.867963in}{2.279101in}}%
\pgfpathlineto{\pgfqpoint{4.872081in}{2.279101in}}%
\pgfpathlineto{\pgfqpoint{4.876199in}{2.260582in}}%
\pgfpathlineto{\pgfqpoint{4.925616in}{2.260582in}}%
\pgfpathlineto{\pgfqpoint{4.929735in}{2.242064in}}%
\pgfpathlineto{\pgfqpoint{4.970915in}{2.242064in}}%
\pgfpathlineto{\pgfqpoint{4.975033in}{2.223545in}}%
\pgfpathlineto{\pgfqpoint{5.082104in}{2.223545in}}%
\pgfpathlineto{\pgfqpoint{5.086222in}{2.242064in}}%
\pgfpathlineto{\pgfqpoint{5.090340in}{2.242064in}}%
\pgfpathlineto{\pgfqpoint{5.094458in}{2.260582in}}%
\pgfpathlineto{\pgfqpoint{5.119166in}{2.260582in}}%
\pgfpathlineto{\pgfqpoint{5.123284in}{2.242064in}}%
\pgfpathlineto{\pgfqpoint{5.226237in}{2.242064in}}%
\pgfpathlineto{\pgfqpoint{5.230355in}{2.260582in}}%
\pgfpathlineto{\pgfqpoint{5.267417in}{2.260582in}}%
\pgfpathlineto{\pgfqpoint{5.271536in}{2.242064in}}%
\pgfpathlineto{\pgfqpoint{5.386842in}{2.242064in}}%
\pgfpathlineto{\pgfqpoint{5.390960in}{2.223545in}}%
\pgfpathlineto{\pgfqpoint{5.395078in}{2.242064in}}%
\pgfpathlineto{\pgfqpoint{5.670990in}{2.242064in}}%
\pgfpathlineto{\pgfqpoint{5.675108in}{2.260582in}}%
\pgfpathlineto{\pgfqpoint{5.703934in}{2.260582in}}%
\pgfpathlineto{\pgfqpoint{5.708053in}{2.279101in}}%
\pgfpathlineto{\pgfqpoint{5.716289in}{2.279101in}}%
\pgfpathlineto{\pgfqpoint{5.720407in}{2.297619in}}%
\pgfpathlineto{\pgfqpoint{5.740997in}{2.297619in}}%
\pgfpathlineto{\pgfqpoint{5.745115in}{2.279101in}}%
\pgfpathlineto{\pgfqpoint{5.811005in}{2.279101in}}%
\pgfpathlineto{\pgfqpoint{5.811005in}{2.279101in}}%
\pgfusepath{stroke}%
\end{pgfscope}%
\begin{pgfscope}%
\pgfpathrectangle{\pgfqpoint{3.653334in}{2.165212in}}{\pgfqpoint{2.260417in}{1.283333in}}%
\pgfusepath{clip}%
\pgfsetrectcap%
\pgfsetroundjoin%
\pgfsetlinewidth{0.803000pt}%
\definecolor{currentstroke}{rgb}{0.000000,0.356863,0.509804}%
\pgfsetstrokecolor{currentstroke}%
\pgfsetdash{}{0pt}%
\pgfpathmoveto{\pgfqpoint{3.756080in}{3.001323in}}%
\pgfpathlineto{\pgfqpoint{3.760198in}{2.982804in}}%
\pgfpathlineto{\pgfqpoint{3.801379in}{2.982804in}}%
\pgfpathlineto{\pgfqpoint{3.805497in}{2.945767in}}%
\pgfpathlineto{\pgfqpoint{3.809615in}{2.927249in}}%
\pgfpathlineto{\pgfqpoint{3.817852in}{2.927249in}}%
\pgfpathlineto{\pgfqpoint{3.821970in}{2.871693in}}%
\pgfpathlineto{\pgfqpoint{3.826088in}{2.853175in}}%
\pgfpathlineto{\pgfqpoint{3.850796in}{2.853175in}}%
\pgfpathlineto{\pgfqpoint{3.859032in}{2.816138in}}%
\pgfpathlineto{\pgfqpoint{3.863151in}{2.834656in}}%
\pgfpathlineto{\pgfqpoint{3.871387in}{2.797619in}}%
\pgfpathlineto{\pgfqpoint{3.933158in}{2.797619in}}%
\pgfpathlineto{\pgfqpoint{3.937276in}{2.816138in}}%
\pgfpathlineto{\pgfqpoint{3.974339in}{2.816138in}}%
\pgfpathlineto{\pgfqpoint{3.978457in}{2.797619in}}%
\pgfpathlineto{\pgfqpoint{4.023756in}{2.797619in}}%
\pgfpathlineto{\pgfqpoint{4.027874in}{2.779101in}}%
\pgfpathlineto{\pgfqpoint{4.044346in}{2.779101in}}%
\pgfpathlineto{\pgfqpoint{4.048464in}{2.760582in}}%
\pgfpathlineto{\pgfqpoint{4.052582in}{2.760582in}}%
\pgfpathlineto{\pgfqpoint{4.056701in}{2.742064in}}%
\pgfpathlineto{\pgfqpoint{4.085527in}{2.742064in}}%
\pgfpathlineto{\pgfqpoint{4.089645in}{2.723545in}}%
\pgfpathlineto{\pgfqpoint{4.143180in}{2.723545in}}%
\pgfpathlineto{\pgfqpoint{4.147298in}{2.705026in}}%
\pgfpathlineto{\pgfqpoint{4.167889in}{2.705026in}}%
\pgfpathlineto{\pgfqpoint{4.172007in}{2.686508in}}%
\pgfpathlineto{\pgfqpoint{4.229660in}{2.686508in}}%
\pgfpathlineto{\pgfqpoint{4.233778in}{2.705026in}}%
\pgfpathlineto{\pgfqpoint{4.270841in}{2.705026in}}%
\pgfpathlineto{\pgfqpoint{4.274959in}{2.723545in}}%
\pgfpathlineto{\pgfqpoint{4.279077in}{2.686508in}}%
\pgfpathlineto{\pgfqpoint{4.291431in}{2.686508in}}%
\pgfpathlineto{\pgfqpoint{4.295549in}{2.705026in}}%
\pgfpathlineto{\pgfqpoint{4.299668in}{2.686508in}}%
\pgfpathlineto{\pgfqpoint{4.312022in}{2.686508in}}%
\pgfpathlineto{\pgfqpoint{4.316140in}{2.667989in}}%
\pgfpathlineto{\pgfqpoint{4.320258in}{2.667989in}}%
\pgfpathlineto{\pgfqpoint{4.324376in}{2.686508in}}%
\pgfpathlineto{\pgfqpoint{4.332612in}{2.686508in}}%
\pgfpathlineto{\pgfqpoint{4.336730in}{2.667989in}}%
\pgfpathlineto{\pgfqpoint{4.369675in}{2.667989in}}%
\pgfpathlineto{\pgfqpoint{4.373793in}{2.649471in}}%
\pgfpathlineto{\pgfqpoint{4.419092in}{2.649471in}}%
\pgfpathlineto{\pgfqpoint{4.423210in}{2.667989in}}%
\pgfpathlineto{\pgfqpoint{4.501454in}{2.667989in}}%
\pgfpathlineto{\pgfqpoint{4.505572in}{2.649471in}}%
\pgfpathlineto{\pgfqpoint{4.517926in}{2.649471in}}%
\pgfpathlineto{\pgfqpoint{4.522044in}{2.630952in}}%
\pgfpathlineto{\pgfqpoint{4.538516in}{2.630952in}}%
\pgfpathlineto{\pgfqpoint{4.542635in}{2.612434in}}%
\pgfpathlineto{\pgfqpoint{4.686768in}{2.612434in}}%
\pgfpathlineto{\pgfqpoint{4.690886in}{2.630952in}}%
\pgfpathlineto{\pgfqpoint{4.699122in}{2.630952in}}%
\pgfpathlineto{\pgfqpoint{4.703240in}{2.612434in}}%
\pgfpathlineto{\pgfqpoint{4.756775in}{2.612434in}}%
\pgfpathlineto{\pgfqpoint{4.760893in}{2.593915in}}%
\pgfpathlineto{\pgfqpoint{4.769129in}{2.593915in}}%
\pgfpathlineto{\pgfqpoint{4.773247in}{2.575397in}}%
\pgfpathlineto{\pgfqpoint{4.797956in}{2.575397in}}%
\pgfpathlineto{\pgfqpoint{4.802074in}{2.556878in}}%
\pgfpathlineto{\pgfqpoint{4.806192in}{2.575397in}}%
\pgfpathlineto{\pgfqpoint{5.007978in}{2.575397in}}%
\pgfpathlineto{\pgfqpoint{5.012096in}{2.556878in}}%
\pgfpathlineto{\pgfqpoint{5.020332in}{2.556878in}}%
\pgfpathlineto{\pgfqpoint{5.024450in}{2.538360in}}%
\pgfpathlineto{\pgfqpoint{5.049159in}{2.538360in}}%
\pgfpathlineto{\pgfqpoint{5.053277in}{2.519841in}}%
\pgfpathlineto{\pgfqpoint{5.077986in}{2.519841in}}%
\pgfpathlineto{\pgfqpoint{5.082104in}{2.538360in}}%
\pgfpathlineto{\pgfqpoint{5.106812in}{2.538360in}}%
\pgfpathlineto{\pgfqpoint{5.110930in}{2.556878in}}%
\pgfpathlineto{\pgfqpoint{5.119166in}{2.556878in}}%
\pgfpathlineto{\pgfqpoint{5.123284in}{2.575397in}}%
\pgfpathlineto{\pgfqpoint{5.139757in}{2.575397in}}%
\pgfpathlineto{\pgfqpoint{5.143875in}{2.593915in}}%
\pgfpathlineto{\pgfqpoint{5.185056in}{2.593915in}}%
\pgfpathlineto{\pgfqpoint{5.189174in}{2.575397in}}%
\pgfpathlineto{\pgfqpoint{5.213882in}{2.575397in}}%
\pgfpathlineto{\pgfqpoint{5.218000in}{2.556878in}}%
\pgfpathlineto{\pgfqpoint{5.226237in}{2.556878in}}%
\pgfpathlineto{\pgfqpoint{5.230355in}{2.538360in}}%
\pgfpathlineto{\pgfqpoint{5.275654in}{2.538360in}}%
\pgfpathlineto{\pgfqpoint{5.279772in}{2.519841in}}%
\pgfpathlineto{\pgfqpoint{5.493912in}{2.519841in}}%
\pgfpathlineto{\pgfqpoint{5.498030in}{2.538360in}}%
\pgfpathlineto{\pgfqpoint{5.502148in}{2.538360in}}%
\pgfpathlineto{\pgfqpoint{5.510384in}{2.501323in}}%
\pgfpathlineto{\pgfqpoint{5.526857in}{2.501323in}}%
\pgfpathlineto{\pgfqpoint{5.530975in}{2.519841in}}%
\pgfpathlineto{\pgfqpoint{5.811005in}{2.519841in}}%
\pgfpathlineto{\pgfqpoint{5.811005in}{2.519841in}}%
\pgfusepath{stroke}%
\end{pgfscope}%
\begin{pgfscope}%
\pgfpathrectangle{\pgfqpoint{3.653334in}{2.165212in}}{\pgfqpoint{2.260417in}{1.283333in}}%
\pgfusepath{clip}%
\pgfsetrectcap%
\pgfsetroundjoin%
\pgfsetlinewidth{0.803000pt}%
\definecolor{currentstroke}{rgb}{0.490196,0.588235,0.431373}%
\pgfsetstrokecolor{currentstroke}%
\pgfsetdash{}{0pt}%
\pgfpathmoveto{\pgfqpoint{3.756080in}{2.667989in}}%
\pgfpathlineto{\pgfqpoint{3.817852in}{2.667989in}}%
\pgfpathlineto{\pgfqpoint{3.821970in}{2.649471in}}%
\pgfpathlineto{\pgfqpoint{3.834324in}{2.649471in}}%
\pgfpathlineto{\pgfqpoint{3.838442in}{2.630952in}}%
\pgfpathlineto{\pgfqpoint{4.048464in}{2.630952in}}%
\pgfpathlineto{\pgfqpoint{4.052582in}{2.612434in}}%
\pgfpathlineto{\pgfqpoint{4.270841in}{2.612434in}}%
\pgfpathlineto{\pgfqpoint{4.274959in}{2.593915in}}%
\pgfpathlineto{\pgfqpoint{4.686768in}{2.593915in}}%
\pgfpathlineto{\pgfqpoint{4.690886in}{2.575397in}}%
\pgfpathlineto{\pgfqpoint{5.811005in}{2.575397in}}%
\pgfpathlineto{\pgfqpoint{5.811005in}{2.575397in}}%
\pgfusepath{stroke}%
\end{pgfscope}%
\begin{pgfscope}%
\pgfpathrectangle{\pgfqpoint{3.653334in}{2.165212in}}{\pgfqpoint{2.260417in}{1.283333in}}%
\pgfusepath{clip}%
\pgfsetrectcap%
\pgfsetroundjoin%
\pgfsetlinewidth{0.803000pt}%
\definecolor{currentstroke}{rgb}{0.843137,0.666667,0.313725}%
\pgfsetstrokecolor{currentstroke}%
\pgfsetdash{}{0pt}%
\pgfpathmoveto{\pgfqpoint{3.756080in}{2.890212in}}%
\pgfpathlineto{\pgfqpoint{3.760198in}{2.945767in}}%
\pgfpathlineto{\pgfqpoint{3.764317in}{2.964286in}}%
\pgfpathlineto{\pgfqpoint{3.784907in}{2.964286in}}%
\pgfpathlineto{\pgfqpoint{3.789025in}{2.982804in}}%
\pgfpathlineto{\pgfqpoint{3.793143in}{3.019841in}}%
\pgfpathlineto{\pgfqpoint{3.797261in}{3.019841in}}%
\pgfpathlineto{\pgfqpoint{3.801379in}{3.038360in}}%
\pgfpathlineto{\pgfqpoint{3.813734in}{3.038360in}}%
\pgfpathlineto{\pgfqpoint{3.817852in}{3.019841in}}%
\pgfpathlineto{\pgfqpoint{3.821970in}{3.056878in}}%
\pgfpathlineto{\pgfqpoint{3.826088in}{3.038360in}}%
\pgfpathlineto{\pgfqpoint{3.830206in}{2.964286in}}%
\pgfpathlineto{\pgfqpoint{3.838442in}{2.964286in}}%
\pgfpathlineto{\pgfqpoint{3.842560in}{2.982804in}}%
\pgfpathlineto{\pgfqpoint{3.846678in}{2.982804in}}%
\pgfpathlineto{\pgfqpoint{3.850796in}{2.964286in}}%
\pgfpathlineto{\pgfqpoint{3.854914in}{2.982804in}}%
\pgfpathlineto{\pgfqpoint{3.859032in}{2.982804in}}%
\pgfpathlineto{\pgfqpoint{3.863151in}{2.945767in}}%
\pgfpathlineto{\pgfqpoint{3.867269in}{2.945767in}}%
\pgfpathlineto{\pgfqpoint{3.871387in}{2.964286in}}%
\pgfpathlineto{\pgfqpoint{3.908450in}{2.964286in}}%
\pgfpathlineto{\pgfqpoint{3.916686in}{2.927249in}}%
\pgfpathlineto{\pgfqpoint{4.167889in}{2.927249in}}%
\pgfpathlineto{\pgfqpoint{4.184361in}{3.001323in}}%
\pgfpathlineto{\pgfqpoint{4.204952in}{3.001323in}}%
\pgfpathlineto{\pgfqpoint{4.209070in}{2.982804in}}%
\pgfpathlineto{\pgfqpoint{4.291431in}{2.982804in}}%
\pgfpathlineto{\pgfqpoint{4.295549in}{3.001323in}}%
\pgfpathlineto{\pgfqpoint{4.299668in}{3.001323in}}%
\pgfpathlineto{\pgfqpoint{4.303786in}{3.019841in}}%
\pgfpathlineto{\pgfqpoint{4.501454in}{3.019841in}}%
\pgfpathlineto{\pgfqpoint{4.509690in}{3.056878in}}%
\pgfpathlineto{\pgfqpoint{4.517926in}{3.056878in}}%
\pgfpathlineto{\pgfqpoint{4.522044in}{3.075397in}}%
\pgfpathlineto{\pgfqpoint{4.550871in}{3.075397in}}%
\pgfpathlineto{\pgfqpoint{4.554989in}{3.093915in}}%
\pgfpathlineto{\pgfqpoint{4.620878in}{3.093915in}}%
\pgfpathlineto{\pgfqpoint{4.624996in}{3.112434in}}%
\pgfpathlineto{\pgfqpoint{4.695004in}{3.112434in}}%
\pgfpathlineto{\pgfqpoint{4.699122in}{3.130952in}}%
\pgfpathlineto{\pgfqpoint{4.744421in}{3.130952in}}%
\pgfpathlineto{\pgfqpoint{4.748539in}{3.149471in}}%
\pgfpathlineto{\pgfqpoint{4.752657in}{3.149471in}}%
\pgfpathlineto{\pgfqpoint{4.756775in}{3.186508in}}%
\pgfpathlineto{\pgfqpoint{4.760893in}{3.186508in}}%
\pgfpathlineto{\pgfqpoint{4.765011in}{3.223545in}}%
\pgfpathlineto{\pgfqpoint{4.785602in}{3.223545in}}%
\pgfpathlineto{\pgfqpoint{4.789720in}{3.242064in}}%
\pgfpathlineto{\pgfqpoint{4.810310in}{3.242064in}}%
\pgfpathlineto{\pgfqpoint{4.814428in}{3.260582in}}%
\pgfpathlineto{\pgfqpoint{4.822664in}{3.260582in}}%
\pgfpathlineto{\pgfqpoint{4.830900in}{3.297619in}}%
\pgfpathlineto{\pgfqpoint{4.859727in}{3.297619in}}%
\pgfpathlineto{\pgfqpoint{4.863845in}{3.279101in}}%
\pgfpathlineto{\pgfqpoint{4.942089in}{3.279101in}}%
\pgfpathlineto{\pgfqpoint{4.946207in}{3.297619in}}%
\pgfpathlineto{\pgfqpoint{4.950325in}{3.297619in}}%
\pgfpathlineto{\pgfqpoint{4.954443in}{3.279101in}}%
\pgfpathlineto{\pgfqpoint{5.007978in}{3.279101in}}%
\pgfpathlineto{\pgfqpoint{5.012096in}{3.297619in}}%
\pgfpathlineto{\pgfqpoint{5.094458in}{3.297619in}}%
\pgfpathlineto{\pgfqpoint{5.098576in}{3.316138in}}%
\pgfpathlineto{\pgfqpoint{5.180938in}{3.316138in}}%
\pgfpathlineto{\pgfqpoint{5.185056in}{3.334656in}}%
\pgfpathlineto{\pgfqpoint{5.226237in}{3.334656in}}%
\pgfpathlineto{\pgfqpoint{5.230355in}{3.353175in}}%
\pgfpathlineto{\pgfqpoint{5.358015in}{3.353175in}}%
\pgfpathlineto{\pgfqpoint{5.362133in}{3.371693in}}%
\pgfpathlineto{\pgfqpoint{5.419787in}{3.371693in}}%
\pgfpathlineto{\pgfqpoint{5.423905in}{3.353175in}}%
\pgfpathlineto{\pgfqpoint{5.563920in}{3.353175in}}%
\pgfpathlineto{\pgfqpoint{5.568038in}{3.371693in}}%
\pgfpathlineto{\pgfqpoint{5.654517in}{3.371693in}}%
\pgfpathlineto{\pgfqpoint{5.658635in}{3.390212in}}%
\pgfpathlineto{\pgfqpoint{5.736879in}{3.390212in}}%
\pgfpathlineto{\pgfqpoint{5.740997in}{3.371693in}}%
\pgfpathlineto{\pgfqpoint{5.811005in}{3.371693in}}%
\pgfpathlineto{\pgfqpoint{5.811005in}{3.371693in}}%
\pgfusepath{stroke}%
\end{pgfscope}%
\begin{pgfscope}%
\pgfpathrectangle{\pgfqpoint{3.653334in}{2.165212in}}{\pgfqpoint{2.260417in}{1.283333in}}%
\pgfusepath{clip}%
\pgfsetrectcap%
\pgfsetroundjoin%
\pgfsetlinewidth{0.803000pt}%
\definecolor{currentstroke}{rgb}{0.333333,0.333333,0.333333}%
\pgfsetstrokecolor{currentstroke}%
\pgfsetstrokeopacity{0.270000}%
\pgfsetdash{}{0pt}%
\pgfpathmoveto{\pgfqpoint{3.756080in}{2.927249in}}%
\pgfpathlineto{\pgfqpoint{3.883741in}{2.927249in}}%
\pgfpathlineto{\pgfqpoint{3.887859in}{2.945767in}}%
\pgfpathlineto{\pgfqpoint{3.957867in}{2.945767in}}%
\pgfpathlineto{\pgfqpoint{3.961985in}{2.927249in}}%
\pgfpathlineto{\pgfqpoint{3.978457in}{2.927249in}}%
\pgfpathlineto{\pgfqpoint{3.982575in}{2.945767in}}%
\pgfpathlineto{\pgfqpoint{3.986693in}{2.927249in}}%
\pgfpathlineto{\pgfqpoint{3.990811in}{2.927249in}}%
\pgfpathlineto{\pgfqpoint{3.994929in}{2.908730in}}%
\pgfpathlineto{\pgfqpoint{4.093763in}{2.908730in}}%
\pgfpathlineto{\pgfqpoint{4.097881in}{2.890212in}}%
\pgfpathlineto{\pgfqpoint{4.270841in}{2.890212in}}%
\pgfpathlineto{\pgfqpoint{4.274959in}{2.853175in}}%
\pgfpathlineto{\pgfqpoint{4.349085in}{2.853175in}}%
\pgfpathlineto{\pgfqpoint{4.353203in}{2.834656in}}%
\pgfpathlineto{\pgfqpoint{4.365557in}{2.834656in}}%
\pgfpathlineto{\pgfqpoint{4.369675in}{2.853175in}}%
\pgfpathlineto{\pgfqpoint{4.373793in}{2.834656in}}%
\pgfpathlineto{\pgfqpoint{4.649705in}{2.834656in}}%
\pgfpathlineto{\pgfqpoint{4.653823in}{2.853175in}}%
\pgfpathlineto{\pgfqpoint{4.732066in}{2.853175in}}%
\pgfpathlineto{\pgfqpoint{4.736185in}{2.871693in}}%
\pgfpathlineto{\pgfqpoint{4.781483in}{2.871693in}}%
\pgfpathlineto{\pgfqpoint{4.785602in}{2.853175in}}%
\pgfpathlineto{\pgfqpoint{5.082104in}{2.853175in}}%
\pgfpathlineto{\pgfqpoint{5.090340in}{2.890212in}}%
\pgfpathlineto{\pgfqpoint{5.811005in}{2.890212in}}%
\pgfpathlineto{\pgfqpoint{5.811005in}{2.890212in}}%
\pgfusepath{stroke}%
\end{pgfscope}%
\begin{pgfscope}%
\pgfpathrectangle{\pgfqpoint{3.653334in}{2.165212in}}{\pgfqpoint{2.260417in}{1.283333in}}%
\pgfusepath{clip}%
\pgfsetrectcap%
\pgfsetroundjoin%
\pgfsetlinewidth{0.803000pt}%
\definecolor{currentstroke}{rgb}{0.686275,0.352941,0.313725}%
\pgfsetstrokecolor{currentstroke}%
\pgfsetstrokeopacity{0.270000}%
\pgfsetdash{}{0pt}%
\pgfpathmoveto{\pgfqpoint{3.756080in}{2.816138in}}%
\pgfpathlineto{\pgfqpoint{3.764317in}{2.853175in}}%
\pgfpathlineto{\pgfqpoint{3.776671in}{2.853175in}}%
\pgfpathlineto{\pgfqpoint{3.780789in}{2.834656in}}%
\pgfpathlineto{\pgfqpoint{3.789025in}{2.834656in}}%
\pgfpathlineto{\pgfqpoint{3.793143in}{2.853175in}}%
\pgfpathlineto{\pgfqpoint{3.805497in}{2.853175in}}%
\pgfpathlineto{\pgfqpoint{3.809615in}{2.871693in}}%
\pgfpathlineto{\pgfqpoint{3.817852in}{2.871693in}}%
\pgfpathlineto{\pgfqpoint{3.821970in}{2.908730in}}%
\pgfpathlineto{\pgfqpoint{3.826088in}{2.908730in}}%
\pgfpathlineto{\pgfqpoint{3.830206in}{2.890212in}}%
\pgfpathlineto{\pgfqpoint{3.846678in}{2.890212in}}%
\pgfpathlineto{\pgfqpoint{3.850796in}{2.908730in}}%
\pgfpathlineto{\pgfqpoint{3.891977in}{2.908730in}}%
\pgfpathlineto{\pgfqpoint{3.896095in}{2.890212in}}%
\pgfpathlineto{\pgfqpoint{3.953748in}{2.890212in}}%
\pgfpathlineto{\pgfqpoint{3.957867in}{2.908730in}}%
\pgfpathlineto{\pgfqpoint{3.994929in}{2.908730in}}%
\pgfpathlineto{\pgfqpoint{3.999047in}{2.890212in}}%
\pgfpathlineto{\pgfqpoint{4.073173in}{2.890212in}}%
\pgfpathlineto{\pgfqpoint{4.077291in}{2.871693in}}%
\pgfpathlineto{\pgfqpoint{4.188479in}{2.871693in}}%
\pgfpathlineto{\pgfqpoint{4.192597in}{2.853175in}}%
\pgfpathlineto{\pgfqpoint{4.229660in}{2.853175in}}%
\pgfpathlineto{\pgfqpoint{4.233778in}{2.871693in}}%
\pgfpathlineto{\pgfqpoint{4.283195in}{2.871693in}}%
\pgfpathlineto{\pgfqpoint{4.291431in}{2.908730in}}%
\pgfpathlineto{\pgfqpoint{4.361439in}{2.908730in}}%
\pgfpathlineto{\pgfqpoint{4.365557in}{2.927249in}}%
\pgfpathlineto{\pgfqpoint{4.624996in}{2.927249in}}%
\pgfpathlineto{\pgfqpoint{4.629114in}{2.908730in}}%
\pgfpathlineto{\pgfqpoint{4.666177in}{2.908730in}}%
\pgfpathlineto{\pgfqpoint{4.670295in}{2.890212in}}%
\pgfpathlineto{\pgfqpoint{4.781483in}{2.890212in}}%
\pgfpathlineto{\pgfqpoint{4.785602in}{2.908730in}}%
\pgfpathlineto{\pgfqpoint{4.983270in}{2.908730in}}%
\pgfpathlineto{\pgfqpoint{4.987388in}{2.890212in}}%
\pgfpathlineto{\pgfqpoint{5.226237in}{2.890212in}}%
\pgfpathlineto{\pgfqpoint{5.230355in}{2.871693in}}%
\pgfpathlineto{\pgfqpoint{5.333307in}{2.871693in}}%
\pgfpathlineto{\pgfqpoint{5.337425in}{2.890212in}}%
\pgfpathlineto{\pgfqpoint{5.366251in}{2.890212in}}%
\pgfpathlineto{\pgfqpoint{5.370370in}{2.871693in}}%
\pgfpathlineto{\pgfqpoint{5.609218in}{2.871693in}}%
\pgfpathlineto{\pgfqpoint{5.613337in}{2.853175in}}%
\pgfpathlineto{\pgfqpoint{5.646281in}{2.853175in}}%
\pgfpathlineto{\pgfqpoint{5.650399in}{2.871693in}}%
\pgfpathlineto{\pgfqpoint{5.811005in}{2.871693in}}%
\pgfpathlineto{\pgfqpoint{5.811005in}{2.871693in}}%
\pgfusepath{stroke}%
\end{pgfscope}%
\begin{pgfscope}%
\pgfpathrectangle{\pgfqpoint{3.653334in}{2.165212in}}{\pgfqpoint{2.260417in}{1.283333in}}%
\pgfusepath{clip}%
\pgfsetrectcap%
\pgfsetroundjoin%
\pgfsetlinewidth{0.803000pt}%
\definecolor{currentstroke}{rgb}{0.000000,0.356863,0.509804}%
\pgfsetstrokecolor{currentstroke}%
\pgfsetstrokeopacity{0.270000}%
\pgfsetdash{}{0pt}%
\pgfpathmoveto{\pgfqpoint{3.756080in}{2.890212in}}%
\pgfpathlineto{\pgfqpoint{3.760198in}{2.908730in}}%
\pgfpathlineto{\pgfqpoint{3.797261in}{2.908730in}}%
\pgfpathlineto{\pgfqpoint{3.805497in}{2.945767in}}%
\pgfpathlineto{\pgfqpoint{3.817852in}{2.945767in}}%
\pgfpathlineto{\pgfqpoint{3.826088in}{2.982804in}}%
\pgfpathlineto{\pgfqpoint{3.904331in}{2.982804in}}%
\pgfpathlineto{\pgfqpoint{3.908450in}{3.001323in}}%
\pgfpathlineto{\pgfqpoint{3.916686in}{3.001323in}}%
\pgfpathlineto{\pgfqpoint{3.920804in}{3.019841in}}%
\pgfpathlineto{\pgfqpoint{3.924922in}{3.019841in}}%
\pgfpathlineto{\pgfqpoint{3.929040in}{3.038360in}}%
\pgfpathlineto{\pgfqpoint{3.953748in}{3.038360in}}%
\pgfpathlineto{\pgfqpoint{3.957867in}{3.056878in}}%
\pgfpathlineto{\pgfqpoint{3.982575in}{3.056878in}}%
\pgfpathlineto{\pgfqpoint{3.986693in}{3.075397in}}%
\pgfpathlineto{\pgfqpoint{4.031992in}{3.075397in}}%
\pgfpathlineto{\pgfqpoint{4.036110in}{3.093915in}}%
\pgfpathlineto{\pgfqpoint{4.048464in}{3.093915in}}%
\pgfpathlineto{\pgfqpoint{4.052582in}{3.112434in}}%
\pgfpathlineto{\pgfqpoint{4.097881in}{3.112434in}}%
\pgfpathlineto{\pgfqpoint{4.101999in}{3.130952in}}%
\pgfpathlineto{\pgfqpoint{4.134944in}{3.130952in}}%
\pgfpathlineto{\pgfqpoint{4.139062in}{3.112434in}}%
\pgfpathlineto{\pgfqpoint{4.246132in}{3.112434in}}%
\pgfpathlineto{\pgfqpoint{4.250251in}{3.130952in}}%
\pgfpathlineto{\pgfqpoint{4.258487in}{3.130952in}}%
\pgfpathlineto{\pgfqpoint{4.262605in}{3.149471in}}%
\pgfpathlineto{\pgfqpoint{4.279077in}{3.149471in}}%
\pgfpathlineto{\pgfqpoint{4.287313in}{3.186508in}}%
\pgfpathlineto{\pgfqpoint{4.307904in}{3.186508in}}%
\pgfpathlineto{\pgfqpoint{4.312022in}{3.205026in}}%
\pgfpathlineto{\pgfqpoint{4.382029in}{3.205026in}}%
\pgfpathlineto{\pgfqpoint{4.386147in}{3.186508in}}%
\pgfpathlineto{\pgfqpoint{4.447919in}{3.186508in}}%
\pgfpathlineto{\pgfqpoint{4.452037in}{3.167989in}}%
\pgfpathlineto{\pgfqpoint{4.493218in}{3.167989in}}%
\pgfpathlineto{\pgfqpoint{4.497336in}{3.186508in}}%
\pgfpathlineto{\pgfqpoint{4.513808in}{3.186508in}}%
\pgfpathlineto{\pgfqpoint{4.517926in}{3.205026in}}%
\pgfpathlineto{\pgfqpoint{4.896790in}{3.205026in}}%
\pgfpathlineto{\pgfqpoint{4.900908in}{3.186508in}}%
\pgfpathlineto{\pgfqpoint{4.913262in}{3.186508in}}%
\pgfpathlineto{\pgfqpoint{4.917380in}{3.167989in}}%
\pgfpathlineto{\pgfqpoint{4.983270in}{3.167989in}}%
\pgfpathlineto{\pgfqpoint{4.987388in}{3.186508in}}%
\pgfpathlineto{\pgfqpoint{5.205646in}{3.186508in}}%
\pgfpathlineto{\pgfqpoint{5.209764in}{3.205026in}}%
\pgfpathlineto{\pgfqpoint{5.498030in}{3.205026in}}%
\pgfpathlineto{\pgfqpoint{5.502148in}{3.223545in}}%
\pgfpathlineto{\pgfqpoint{5.568038in}{3.223545in}}%
\pgfpathlineto{\pgfqpoint{5.572156in}{3.242064in}}%
\pgfpathlineto{\pgfqpoint{5.588628in}{3.242064in}}%
\pgfpathlineto{\pgfqpoint{5.592746in}{3.260582in}}%
\pgfpathlineto{\pgfqpoint{5.613337in}{3.260582in}}%
\pgfpathlineto{\pgfqpoint{5.617455in}{3.279101in}}%
\pgfpathlineto{\pgfqpoint{5.646281in}{3.279101in}}%
\pgfpathlineto{\pgfqpoint{5.650399in}{3.297619in}}%
\pgfpathlineto{\pgfqpoint{5.811005in}{3.297619in}}%
\pgfpathlineto{\pgfqpoint{5.811005in}{3.297619in}}%
\pgfusepath{stroke}%
\end{pgfscope}%
\begin{pgfscope}%
\pgfpathrectangle{\pgfqpoint{3.653334in}{2.165212in}}{\pgfqpoint{2.260417in}{1.283333in}}%
\pgfusepath{clip}%
\pgfsetrectcap%
\pgfsetroundjoin%
\pgfsetlinewidth{0.803000pt}%
\definecolor{currentstroke}{rgb}{0.490196,0.588235,0.431373}%
\pgfsetstrokecolor{currentstroke}%
\pgfsetstrokeopacity{0.270000}%
\pgfsetdash{}{0pt}%
\pgfpathmoveto{\pgfqpoint{3.756080in}{3.242064in}}%
\pgfpathlineto{\pgfqpoint{3.760198in}{3.242064in}}%
\pgfpathlineto{\pgfqpoint{3.764317in}{3.223545in}}%
\pgfpathlineto{\pgfqpoint{3.772553in}{3.223545in}}%
\pgfpathlineto{\pgfqpoint{3.776671in}{3.205026in}}%
\pgfpathlineto{\pgfqpoint{3.789025in}{3.205026in}}%
\pgfpathlineto{\pgfqpoint{3.793143in}{3.186508in}}%
\pgfpathlineto{\pgfqpoint{3.817852in}{3.186508in}}%
\pgfpathlineto{\pgfqpoint{3.826088in}{3.149471in}}%
\pgfpathlineto{\pgfqpoint{3.834324in}{3.149471in}}%
\pgfpathlineto{\pgfqpoint{3.838442in}{3.130952in}}%
\pgfpathlineto{\pgfqpoint{3.871387in}{3.130952in}}%
\pgfpathlineto{\pgfqpoint{3.875505in}{3.149471in}}%
\pgfpathlineto{\pgfqpoint{3.908450in}{3.149471in}}%
\pgfpathlineto{\pgfqpoint{3.912568in}{3.130952in}}%
\pgfpathlineto{\pgfqpoint{3.916686in}{3.130952in}}%
\pgfpathlineto{\pgfqpoint{3.920804in}{3.149471in}}%
\pgfpathlineto{\pgfqpoint{3.982575in}{3.149471in}}%
\pgfpathlineto{\pgfqpoint{3.986693in}{3.167989in}}%
\pgfpathlineto{\pgfqpoint{4.031992in}{3.167989in}}%
\pgfpathlineto{\pgfqpoint{4.036110in}{3.149471in}}%
\pgfpathlineto{\pgfqpoint{4.163771in}{3.149471in}}%
\pgfpathlineto{\pgfqpoint{4.167889in}{3.167989in}}%
\pgfpathlineto{\pgfqpoint{4.254369in}{3.167989in}}%
\pgfpathlineto{\pgfqpoint{4.258487in}{3.186508in}}%
\pgfpathlineto{\pgfqpoint{4.316140in}{3.186508in}}%
\pgfpathlineto{\pgfqpoint{4.320258in}{3.167989in}}%
\pgfpathlineto{\pgfqpoint{5.811005in}{3.167989in}}%
\pgfpathlineto{\pgfqpoint{5.811005in}{3.167989in}}%
\pgfusepath{stroke}%
\end{pgfscope}%
\begin{pgfscope}%
\pgfpathrectangle{\pgfqpoint{3.653334in}{2.165212in}}{\pgfqpoint{2.260417in}{1.283333in}}%
\pgfusepath{clip}%
\pgfsetrectcap%
\pgfsetroundjoin%
\pgfsetlinewidth{0.803000pt}%
\definecolor{currentstroke}{rgb}{0.843137,0.666667,0.313725}%
\pgfsetstrokecolor{currentstroke}%
\pgfsetstrokeopacity{0.270000}%
\pgfsetdash{}{0pt}%
\pgfpathmoveto{\pgfqpoint{3.756080in}{2.779101in}}%
\pgfpathlineto{\pgfqpoint{3.764317in}{2.705026in}}%
\pgfpathlineto{\pgfqpoint{3.772553in}{2.705026in}}%
\pgfpathlineto{\pgfqpoint{3.776671in}{2.686508in}}%
\pgfpathlineto{\pgfqpoint{3.789025in}{2.686508in}}%
\pgfpathlineto{\pgfqpoint{3.793143in}{2.667989in}}%
\pgfpathlineto{\pgfqpoint{3.805497in}{2.667989in}}%
\pgfpathlineto{\pgfqpoint{3.809615in}{2.649471in}}%
\pgfpathlineto{\pgfqpoint{3.821970in}{2.649471in}}%
\pgfpathlineto{\pgfqpoint{3.826088in}{2.630952in}}%
\pgfpathlineto{\pgfqpoint{3.830206in}{2.649471in}}%
\pgfpathlineto{\pgfqpoint{3.854914in}{2.649471in}}%
\pgfpathlineto{\pgfqpoint{3.859032in}{2.630952in}}%
\pgfpathlineto{\pgfqpoint{3.863151in}{2.649471in}}%
\pgfpathlineto{\pgfqpoint{3.871387in}{2.649471in}}%
\pgfpathlineto{\pgfqpoint{3.883741in}{2.593915in}}%
\pgfpathlineto{\pgfqpoint{3.887859in}{2.593915in}}%
\pgfpathlineto{\pgfqpoint{3.896095in}{2.556878in}}%
\pgfpathlineto{\pgfqpoint{3.904331in}{2.556878in}}%
\pgfpathlineto{\pgfqpoint{3.908450in}{2.575397in}}%
\pgfpathlineto{\pgfqpoint{3.924922in}{2.575397in}}%
\pgfpathlineto{\pgfqpoint{3.929040in}{2.593915in}}%
\pgfpathlineto{\pgfqpoint{3.945512in}{2.593915in}}%
\pgfpathlineto{\pgfqpoint{3.949630in}{2.612434in}}%
\pgfpathlineto{\pgfqpoint{3.970221in}{2.612434in}}%
\pgfpathlineto{\pgfqpoint{3.974339in}{2.593915in}}%
\pgfpathlineto{\pgfqpoint{3.990811in}{2.593915in}}%
\pgfpathlineto{\pgfqpoint{3.994929in}{2.575397in}}%
\pgfpathlineto{\pgfqpoint{4.019638in}{2.575397in}}%
\pgfpathlineto{\pgfqpoint{4.023756in}{2.593915in}}%
\pgfpathlineto{\pgfqpoint{4.036110in}{2.593915in}}%
\pgfpathlineto{\pgfqpoint{4.040228in}{2.612434in}}%
\pgfpathlineto{\pgfqpoint{4.056701in}{2.612434in}}%
\pgfpathlineto{\pgfqpoint{4.060819in}{2.593915in}}%
\pgfpathlineto{\pgfqpoint{4.172007in}{2.593915in}}%
\pgfpathlineto{\pgfqpoint{4.176125in}{2.575397in}}%
\pgfpathlineto{\pgfqpoint{4.204952in}{2.575397in}}%
\pgfpathlineto{\pgfqpoint{4.209070in}{2.593915in}}%
\pgfpathlineto{\pgfqpoint{4.213188in}{2.593915in}}%
\pgfpathlineto{\pgfqpoint{4.217306in}{2.575397in}}%
\pgfpathlineto{\pgfqpoint{4.221424in}{2.575397in}}%
\pgfpathlineto{\pgfqpoint{4.225542in}{2.556878in}}%
\pgfpathlineto{\pgfqpoint{4.242014in}{2.556878in}}%
\pgfpathlineto{\pgfqpoint{4.246132in}{2.575397in}}%
\pgfpathlineto{\pgfqpoint{4.299668in}{2.575397in}}%
\pgfpathlineto{\pgfqpoint{4.303786in}{2.538360in}}%
\pgfpathlineto{\pgfqpoint{4.312022in}{2.538360in}}%
\pgfpathlineto{\pgfqpoint{4.316140in}{2.556878in}}%
\pgfpathlineto{\pgfqpoint{4.447919in}{2.556878in}}%
\pgfpathlineto{\pgfqpoint{4.452037in}{2.575397in}}%
\pgfpathlineto{\pgfqpoint{4.472627in}{2.575397in}}%
\pgfpathlineto{\pgfqpoint{4.476745in}{2.556878in}}%
\pgfpathlineto{\pgfqpoint{4.550871in}{2.556878in}}%
\pgfpathlineto{\pgfqpoint{4.554989in}{2.538360in}}%
\pgfpathlineto{\pgfqpoint{4.732066in}{2.538360in}}%
\pgfpathlineto{\pgfqpoint{4.736185in}{2.556878in}}%
\pgfpathlineto{\pgfqpoint{4.765011in}{2.556878in}}%
\pgfpathlineto{\pgfqpoint{4.769129in}{2.575397in}}%
\pgfpathlineto{\pgfqpoint{4.810310in}{2.575397in}}%
\pgfpathlineto{\pgfqpoint{4.814428in}{2.556878in}}%
\pgfpathlineto{\pgfqpoint{5.007978in}{2.556878in}}%
\pgfpathlineto{\pgfqpoint{5.012096in}{2.538360in}}%
\pgfpathlineto{\pgfqpoint{5.065631in}{2.538360in}}%
\pgfpathlineto{\pgfqpoint{5.069749in}{2.556878in}}%
\pgfpathlineto{\pgfqpoint{5.090340in}{2.556878in}}%
\pgfpathlineto{\pgfqpoint{5.094458in}{2.575397in}}%
\pgfpathlineto{\pgfqpoint{5.139757in}{2.575397in}}%
\pgfpathlineto{\pgfqpoint{5.143875in}{2.593915in}}%
\pgfpathlineto{\pgfqpoint{5.234473in}{2.593915in}}%
\pgfpathlineto{\pgfqpoint{5.238591in}{2.612434in}}%
\pgfpathlineto{\pgfqpoint{5.263299in}{2.612434in}}%
\pgfpathlineto{\pgfqpoint{5.267417in}{2.630952in}}%
\pgfpathlineto{\pgfqpoint{5.353897in}{2.630952in}}%
\pgfpathlineto{\pgfqpoint{5.358015in}{2.612434in}}%
\pgfpathlineto{\pgfqpoint{5.407432in}{2.612434in}}%
\pgfpathlineto{\pgfqpoint{5.411550in}{2.630952in}}%
\pgfpathlineto{\pgfqpoint{5.432141in}{2.630952in}}%
\pgfpathlineto{\pgfqpoint{5.440377in}{2.593915in}}%
\pgfpathlineto{\pgfqpoint{5.456849in}{2.593915in}}%
\pgfpathlineto{\pgfqpoint{5.460967in}{2.612434in}}%
\pgfpathlineto{\pgfqpoint{5.547447in}{2.612434in}}%
\pgfpathlineto{\pgfqpoint{5.551565in}{2.593915in}}%
\pgfpathlineto{\pgfqpoint{5.811005in}{2.593915in}}%
\pgfpathlineto{\pgfqpoint{5.811005in}{2.593915in}}%
\pgfusepath{stroke}%
\end{pgfscope}%
\begin{pgfscope}%
\pgfpathrectangle{\pgfqpoint{3.653334in}{2.165212in}}{\pgfqpoint{2.260417in}{1.283333in}}%
\pgfusepath{clip}%
\pgfsetrectcap%
\pgfsetroundjoin%
\pgfsetlinewidth{0.803000pt}%
\definecolor{currentstroke}{rgb}{0.333333,0.333333,0.333333}%
\pgfsetstrokecolor{currentstroke}%
\pgfsetstrokeopacity{0.270000}%
\pgfsetdash{}{0pt}%
\pgfpathmoveto{\pgfqpoint{3.756080in}{2.908730in}}%
\pgfpathlineto{\pgfqpoint{3.760198in}{2.927249in}}%
\pgfpathlineto{\pgfqpoint{3.789025in}{2.927249in}}%
\pgfpathlineto{\pgfqpoint{3.793143in}{2.945767in}}%
\pgfpathlineto{\pgfqpoint{3.842560in}{2.945767in}}%
\pgfpathlineto{\pgfqpoint{3.846678in}{2.964286in}}%
\pgfpathlineto{\pgfqpoint{3.924922in}{2.964286in}}%
\pgfpathlineto{\pgfqpoint{3.929040in}{2.945767in}}%
\pgfpathlineto{\pgfqpoint{3.945512in}{2.945767in}}%
\pgfpathlineto{\pgfqpoint{3.949630in}{2.927249in}}%
\pgfpathlineto{\pgfqpoint{3.978457in}{2.927249in}}%
\pgfpathlineto{\pgfqpoint{3.982575in}{2.945767in}}%
\pgfpathlineto{\pgfqpoint{4.031992in}{2.945767in}}%
\pgfpathlineto{\pgfqpoint{4.036110in}{2.964286in}}%
\pgfpathlineto{\pgfqpoint{4.106118in}{2.964286in}}%
\pgfpathlineto{\pgfqpoint{4.110236in}{2.945767in}}%
\pgfpathlineto{\pgfqpoint{4.139062in}{2.945767in}}%
\pgfpathlineto{\pgfqpoint{4.143180in}{2.964286in}}%
\pgfpathlineto{\pgfqpoint{4.303786in}{2.964286in}}%
\pgfpathlineto{\pgfqpoint{4.307904in}{2.945767in}}%
\pgfpathlineto{\pgfqpoint{4.439682in}{2.945767in}}%
\pgfpathlineto{\pgfqpoint{4.443801in}{2.927249in}}%
\pgfpathlineto{\pgfqpoint{4.447919in}{2.927249in}}%
\pgfpathlineto{\pgfqpoint{4.452037in}{2.908730in}}%
\pgfpathlineto{\pgfqpoint{4.587933in}{2.908730in}}%
\pgfpathlineto{\pgfqpoint{4.592052in}{2.890212in}}%
\pgfpathlineto{\pgfqpoint{4.633232in}{2.890212in}}%
\pgfpathlineto{\pgfqpoint{4.641469in}{2.853175in}}%
\pgfpathlineto{\pgfqpoint{4.662059in}{2.853175in}}%
\pgfpathlineto{\pgfqpoint{4.666177in}{2.834656in}}%
\pgfpathlineto{\pgfqpoint{4.699122in}{2.834656in}}%
\pgfpathlineto{\pgfqpoint{4.703240in}{2.853175in}}%
\pgfpathlineto{\pgfqpoint{5.036805in}{2.853175in}}%
\pgfpathlineto{\pgfqpoint{5.040923in}{2.871693in}}%
\pgfpathlineto{\pgfqpoint{5.312716in}{2.871693in}}%
\pgfpathlineto{\pgfqpoint{5.316834in}{2.853175in}}%
\pgfpathlineto{\pgfqpoint{5.366251in}{2.853175in}}%
\pgfpathlineto{\pgfqpoint{5.370370in}{2.834656in}}%
\pgfpathlineto{\pgfqpoint{5.399196in}{2.834656in}}%
\pgfpathlineto{\pgfqpoint{5.403314in}{2.816138in}}%
\pgfpathlineto{\pgfqpoint{5.790414in}{2.816138in}}%
\pgfpathlineto{\pgfqpoint{5.794532in}{2.834656in}}%
\pgfpathlineto{\pgfqpoint{5.811005in}{2.834656in}}%
\pgfpathlineto{\pgfqpoint{5.811005in}{2.834656in}}%
\pgfusepath{stroke}%
\end{pgfscope}%
\begin{pgfscope}%
\pgfsetrectcap%
\pgfsetmiterjoin%
\pgfsetlinewidth{0.501875pt}%
\definecolor{currentstroke}{rgb}{0.317647,0.317647,0.317647}%
\pgfsetstrokecolor{currentstroke}%
\pgfsetdash{}{0pt}%
\pgfpathmoveto{\pgfqpoint{3.653334in}{2.165212in}}%
\pgfpathlineto{\pgfqpoint{3.653334in}{3.448545in}}%
\pgfusepath{stroke}%
\end{pgfscope}%
\begin{pgfscope}%
\pgfsetrectcap%
\pgfsetmiterjoin%
\pgfsetlinewidth{0.501875pt}%
\definecolor{currentstroke}{rgb}{0.317647,0.317647,0.317647}%
\pgfsetstrokecolor{currentstroke}%
\pgfsetdash{}{0pt}%
\pgfpathmoveto{\pgfqpoint{3.653334in}{2.165212in}}%
\pgfpathlineto{\pgfqpoint{5.913751in}{2.165212in}}%
\pgfusepath{stroke}%
\end{pgfscope}%
\begin{pgfscope}%
\pgfsetrectcap%
\pgfsetroundjoin%
\pgfsetlinewidth{0.803000pt}%
\definecolor{currentstroke}{rgb}{0.333333,0.333333,0.333333}%
\pgfsetstrokecolor{currentstroke}%
\pgfsetdash{}{0pt}%
\pgfpathmoveto{\pgfqpoint{5.868543in}{3.340612in}}%
\pgfpathlineto{\pgfqpoint{5.912987in}{3.340612in}}%
\pgfusepath{stroke}%
\end{pgfscope}%
\begin{pgfscope}%
\definecolor{textcolor}{rgb}{0.000000,0.000000,0.000000}%
\pgfsetstrokecolor{textcolor}%
\pgfsetfillcolor{textcolor}%
\pgftext[x=5.940765in,y=3.321168in,left,base]{\color{textcolor}\rmfamily\fontsize{4.000000}{4.800000}\selectfont \(\displaystyle b_0\)}%
\end{pgfscope}%
\begin{pgfscope}%
\pgfsetrectcap%
\pgfsetroundjoin%
\pgfsetlinewidth{0.803000pt}%
\definecolor{currentstroke}{rgb}{0.686275,0.352941,0.313725}%
\pgfsetstrokecolor{currentstroke}%
\pgfsetdash{}{0pt}%
\pgfpathmoveto{\pgfqpoint{5.868543in}{3.249332in}}%
\pgfpathlineto{\pgfqpoint{5.912987in}{3.249332in}}%
\pgfusepath{stroke}%
\end{pgfscope}%
\begin{pgfscope}%
\definecolor{textcolor}{rgb}{0.000000,0.000000,0.000000}%
\pgfsetstrokecolor{textcolor}%
\pgfsetfillcolor{textcolor}%
\pgftext[x=5.940765in,y=3.229887in,left,base]{\color{textcolor}\rmfamily\fontsize{4.000000}{4.800000}\selectfont \(\displaystyle b_1\)}%
\end{pgfscope}%
\begin{pgfscope}%
\pgfsetrectcap%
\pgfsetroundjoin%
\pgfsetlinewidth{0.803000pt}%
\definecolor{currentstroke}{rgb}{0.000000,0.356863,0.509804}%
\pgfsetstrokecolor{currentstroke}%
\pgfsetdash{}{0pt}%
\pgfpathmoveto{\pgfqpoint{5.868543in}{3.158051in}}%
\pgfpathlineto{\pgfqpoint{5.912987in}{3.158051in}}%
\pgfusepath{stroke}%
\end{pgfscope}%
\begin{pgfscope}%
\definecolor{textcolor}{rgb}{0.000000,0.000000,0.000000}%
\pgfsetstrokecolor{textcolor}%
\pgfsetfillcolor{textcolor}%
\pgftext[x=5.940765in,y=3.138606in,left,base]{\color{textcolor}\rmfamily\fontsize{4.000000}{4.800000}\selectfont \(\displaystyle b_2\)}%
\end{pgfscope}%
\begin{pgfscope}%
\pgfsetrectcap%
\pgfsetroundjoin%
\pgfsetlinewidth{0.803000pt}%
\definecolor{currentstroke}{rgb}{0.490196,0.588235,0.431373}%
\pgfsetstrokecolor{currentstroke}%
\pgfsetdash{}{0pt}%
\pgfpathmoveto{\pgfqpoint{5.868543in}{3.066770in}}%
\pgfpathlineto{\pgfqpoint{5.912987in}{3.066770in}}%
\pgfusepath{stroke}%
\end{pgfscope}%
\begin{pgfscope}%
\definecolor{textcolor}{rgb}{0.000000,0.000000,0.000000}%
\pgfsetstrokecolor{textcolor}%
\pgfsetfillcolor{textcolor}%
\pgftext[x=5.940765in,y=3.047326in,left,base]{\color{textcolor}\rmfamily\fontsize{4.000000}{4.800000}\selectfont \(\displaystyle b_3\)}%
\end{pgfscope}%
\begin{pgfscope}%
\pgfsetrectcap%
\pgfsetroundjoin%
\pgfsetlinewidth{0.803000pt}%
\definecolor{currentstroke}{rgb}{0.843137,0.666667,0.313725}%
\pgfsetstrokecolor{currentstroke}%
\pgfsetdash{}{0pt}%
\pgfpathmoveto{\pgfqpoint{5.868543in}{2.975489in}}%
\pgfpathlineto{\pgfqpoint{5.912987in}{2.975489in}}%
\pgfusepath{stroke}%
\end{pgfscope}%
\begin{pgfscope}%
\definecolor{textcolor}{rgb}{0.000000,0.000000,0.000000}%
\pgfsetstrokecolor{textcolor}%
\pgfsetfillcolor{textcolor}%
\pgftext[x=5.940765in,y=2.956045in,left,base]{\color{textcolor}\rmfamily\fontsize{4.000000}{4.800000}\selectfont \(\displaystyle b_4\)}%
\end{pgfscope}%
\begin{pgfscope}%
\pgfsetbuttcap%
\pgfsetmiterjoin%
\pgfsetlinewidth{0.000000pt}%
\definecolor{currentstroke}{rgb}{0.000000,0.000000,0.000000}%
\pgfsetstrokecolor{currentstroke}%
\pgfsetstrokeopacity{0.000000}%
\pgfsetdash{}{0pt}%
\pgfpathmoveto{\pgfqpoint{0.488751in}{0.368545in}}%
\pgfpathlineto{\pgfqpoint{2.749168in}{0.368545in}}%
\pgfpathlineto{\pgfqpoint{2.749168in}{1.651878in}}%
\pgfpathlineto{\pgfqpoint{0.488751in}{1.651878in}}%
\pgfpathclose%
\pgfusepath{}%
\end{pgfscope}%
\begin{pgfscope}%
\pgfsetbuttcap%
\pgfsetroundjoin%
\definecolor{currentfill}{rgb}{0.317647,0.317647,0.317647}%
\pgfsetfillcolor{currentfill}%
\pgfsetlinewidth{0.501875pt}%
\definecolor{currentstroke}{rgb}{0.317647,0.317647,0.317647}%
\pgfsetstrokecolor{currentstroke}%
\pgfsetdash{}{0pt}%
\pgfsys@defobject{currentmarker}{\pgfqpoint{0.000000in}{-0.020833in}}{\pgfqpoint{0.000000in}{0.000000in}}{%
\pgfpathmoveto{\pgfqpoint{0.000000in}{0.000000in}}%
\pgfpathlineto{\pgfqpoint{0.000000in}{-0.020833in}}%
\pgfusepath{stroke,fill}%
}%
\begin{pgfscope}%
\pgfsys@transformshift{0.591497in}{0.368545in}%
\pgfsys@useobject{currentmarker}{}%
\end{pgfscope}%
\end{pgfscope}%
\begin{pgfscope}%
\definecolor{textcolor}{rgb}{0.317647,0.317647,0.317647}%
\pgfsetstrokecolor{textcolor}%
\pgfsetfillcolor{textcolor}%
\pgftext[x=0.591497in,y=0.319934in,,top]{\color{textcolor}\rmfamily\fontsize{6.664000}{7.996800}\selectfont \(\displaystyle 0\)}%
\end{pgfscope}%
\begin{pgfscope}%
\pgfsetbuttcap%
\pgfsetroundjoin%
\definecolor{currentfill}{rgb}{0.317647,0.317647,0.317647}%
\pgfsetfillcolor{currentfill}%
\pgfsetlinewidth{0.501875pt}%
\definecolor{currentstroke}{rgb}{0.317647,0.317647,0.317647}%
\pgfsetstrokecolor{currentstroke}%
\pgfsetdash{}{0pt}%
\pgfsys@defobject{currentmarker}{\pgfqpoint{0.000000in}{-0.020833in}}{\pgfqpoint{0.000000in}{0.000000in}}{%
\pgfpathmoveto{\pgfqpoint{0.000000in}{0.000000in}}%
\pgfpathlineto{\pgfqpoint{0.000000in}{-0.020833in}}%
\pgfusepath{stroke,fill}%
}%
\begin{pgfscope}%
\pgfsys@transformshift{1.003306in}{0.368545in}%
\pgfsys@useobject{currentmarker}{}%
\end{pgfscope}%
\end{pgfscope}%
\begin{pgfscope}%
\definecolor{textcolor}{rgb}{0.317647,0.317647,0.317647}%
\pgfsetstrokecolor{textcolor}%
\pgfsetfillcolor{textcolor}%
\pgftext[x=1.003306in,y=0.319934in,,top]{\color{textcolor}\rmfamily\fontsize{6.664000}{7.996800}\selectfont \(\displaystyle 500\)}%
\end{pgfscope}%
\begin{pgfscope}%
\pgfsetbuttcap%
\pgfsetroundjoin%
\definecolor{currentfill}{rgb}{0.317647,0.317647,0.317647}%
\pgfsetfillcolor{currentfill}%
\pgfsetlinewidth{0.501875pt}%
\definecolor{currentstroke}{rgb}{0.317647,0.317647,0.317647}%
\pgfsetstrokecolor{currentstroke}%
\pgfsetdash{}{0pt}%
\pgfsys@defobject{currentmarker}{\pgfqpoint{0.000000in}{-0.020833in}}{\pgfqpoint{0.000000in}{0.000000in}}{%
\pgfpathmoveto{\pgfqpoint{0.000000in}{0.000000in}}%
\pgfpathlineto{\pgfqpoint{0.000000in}{-0.020833in}}%
\pgfusepath{stroke,fill}%
}%
\begin{pgfscope}%
\pgfsys@transformshift{1.415114in}{0.368545in}%
\pgfsys@useobject{currentmarker}{}%
\end{pgfscope}%
\end{pgfscope}%
\begin{pgfscope}%
\definecolor{textcolor}{rgb}{0.317647,0.317647,0.317647}%
\pgfsetstrokecolor{textcolor}%
\pgfsetfillcolor{textcolor}%
\pgftext[x=1.415114in,y=0.319934in,,top]{\color{textcolor}\rmfamily\fontsize{6.664000}{7.996800}\selectfont \(\displaystyle 1000\)}%
\end{pgfscope}%
\begin{pgfscope}%
\pgfsetbuttcap%
\pgfsetroundjoin%
\definecolor{currentfill}{rgb}{0.317647,0.317647,0.317647}%
\pgfsetfillcolor{currentfill}%
\pgfsetlinewidth{0.501875pt}%
\definecolor{currentstroke}{rgb}{0.317647,0.317647,0.317647}%
\pgfsetstrokecolor{currentstroke}%
\pgfsetdash{}{0pt}%
\pgfsys@defobject{currentmarker}{\pgfqpoint{0.000000in}{-0.020833in}}{\pgfqpoint{0.000000in}{0.000000in}}{%
\pgfpathmoveto{\pgfqpoint{0.000000in}{0.000000in}}%
\pgfpathlineto{\pgfqpoint{0.000000in}{-0.020833in}}%
\pgfusepath{stroke,fill}%
}%
\begin{pgfscope}%
\pgfsys@transformshift{1.826922in}{0.368545in}%
\pgfsys@useobject{currentmarker}{}%
\end{pgfscope}%
\end{pgfscope}%
\begin{pgfscope}%
\definecolor{textcolor}{rgb}{0.317647,0.317647,0.317647}%
\pgfsetstrokecolor{textcolor}%
\pgfsetfillcolor{textcolor}%
\pgftext[x=1.826922in,y=0.319934in,,top]{\color{textcolor}\rmfamily\fontsize{6.664000}{7.996800}\selectfont \(\displaystyle 1500\)}%
\end{pgfscope}%
\begin{pgfscope}%
\pgfsetbuttcap%
\pgfsetroundjoin%
\definecolor{currentfill}{rgb}{0.317647,0.317647,0.317647}%
\pgfsetfillcolor{currentfill}%
\pgfsetlinewidth{0.501875pt}%
\definecolor{currentstroke}{rgb}{0.317647,0.317647,0.317647}%
\pgfsetstrokecolor{currentstroke}%
\pgfsetdash{}{0pt}%
\pgfsys@defobject{currentmarker}{\pgfqpoint{0.000000in}{-0.020833in}}{\pgfqpoint{0.000000in}{0.000000in}}{%
\pgfpathmoveto{\pgfqpoint{0.000000in}{0.000000in}}%
\pgfpathlineto{\pgfqpoint{0.000000in}{-0.020833in}}%
\pgfusepath{stroke,fill}%
}%
\begin{pgfscope}%
\pgfsys@transformshift{2.238731in}{0.368545in}%
\pgfsys@useobject{currentmarker}{}%
\end{pgfscope}%
\end{pgfscope}%
\begin{pgfscope}%
\definecolor{textcolor}{rgb}{0.317647,0.317647,0.317647}%
\pgfsetstrokecolor{textcolor}%
\pgfsetfillcolor{textcolor}%
\pgftext[x=2.238731in,y=0.319934in,,top]{\color{textcolor}\rmfamily\fontsize{6.664000}{7.996800}\selectfont \(\displaystyle 2000\)}%
\end{pgfscope}%
\begin{pgfscope}%
\pgfsetbuttcap%
\pgfsetroundjoin%
\definecolor{currentfill}{rgb}{0.317647,0.317647,0.317647}%
\pgfsetfillcolor{currentfill}%
\pgfsetlinewidth{0.501875pt}%
\definecolor{currentstroke}{rgb}{0.317647,0.317647,0.317647}%
\pgfsetstrokecolor{currentstroke}%
\pgfsetdash{}{0pt}%
\pgfsys@defobject{currentmarker}{\pgfqpoint{0.000000in}{-0.020833in}}{\pgfqpoint{0.000000in}{0.000000in}}{%
\pgfpathmoveto{\pgfqpoint{0.000000in}{0.000000in}}%
\pgfpathlineto{\pgfqpoint{0.000000in}{-0.020833in}}%
\pgfusepath{stroke,fill}%
}%
\begin{pgfscope}%
\pgfsys@transformshift{2.650539in}{0.368545in}%
\pgfsys@useobject{currentmarker}{}%
\end{pgfscope}%
\end{pgfscope}%
\begin{pgfscope}%
\definecolor{textcolor}{rgb}{0.317647,0.317647,0.317647}%
\pgfsetstrokecolor{textcolor}%
\pgfsetfillcolor{textcolor}%
\pgftext[x=2.650539in,y=0.319934in,,top]{\color{textcolor}\rmfamily\fontsize{6.664000}{7.996800}\selectfont \(\displaystyle 2500\)}%
\end{pgfscope}%
\begin{pgfscope}%
\definecolor{textcolor}{rgb}{0.317647,0.317647,0.317647}%
\pgfsetstrokecolor{textcolor}%
\pgfsetfillcolor{textcolor}%
\pgftext[x=1.618959in,y=0.182189in,,top]{\color{textcolor}\rmfamily\fontsize{6.664000}{7.996800}\selectfont Iteration}%
\end{pgfscope}%
\begin{pgfscope}%
\pgfsetbuttcap%
\pgfsetroundjoin%
\definecolor{currentfill}{rgb}{0.317647,0.317647,0.317647}%
\pgfsetfillcolor{currentfill}%
\pgfsetlinewidth{0.501875pt}%
\definecolor{currentstroke}{rgb}{0.317647,0.317647,0.317647}%
\pgfsetstrokecolor{currentstroke}%
\pgfsetdash{}{0pt}%
\pgfsys@defobject{currentmarker}{\pgfqpoint{-0.020833in}{0.000000in}}{\pgfqpoint{0.000000in}{0.000000in}}{%
\pgfpathmoveto{\pgfqpoint{0.000000in}{0.000000in}}%
\pgfpathlineto{\pgfqpoint{-0.020833in}{0.000000in}}%
\pgfusepath{stroke,fill}%
}%
\begin{pgfscope}%
\pgfsys@transformshift{0.488751in}{0.637722in}%
\pgfsys@useobject{currentmarker}{}%
\end{pgfscope}%
\end{pgfscope}%
\begin{pgfscope}%
\definecolor{textcolor}{rgb}{0.317647,0.317647,0.317647}%
\pgfsetstrokecolor{textcolor}%
\pgfsetfillcolor{textcolor}%
\pgftext[x=0.256497in,y=0.605605in,left,base]{\color{textcolor}\rmfamily\fontsize{6.664000}{7.996800}\selectfont \(\displaystyle -20\)}%
\end{pgfscope}%
\begin{pgfscope}%
\pgfsetbuttcap%
\pgfsetroundjoin%
\definecolor{currentfill}{rgb}{0.317647,0.317647,0.317647}%
\pgfsetfillcolor{currentfill}%
\pgfsetlinewidth{0.501875pt}%
\definecolor{currentstroke}{rgb}{0.317647,0.317647,0.317647}%
\pgfsetstrokecolor{currentstroke}%
\pgfsetdash{}{0pt}%
\pgfsys@defobject{currentmarker}{\pgfqpoint{-0.020833in}{0.000000in}}{\pgfqpoint{0.000000in}{0.000000in}}{%
\pgfpathmoveto{\pgfqpoint{0.000000in}{0.000000in}}%
\pgfpathlineto{\pgfqpoint{-0.020833in}{0.000000in}}%
\pgfusepath{stroke,fill}%
}%
\begin{pgfscope}%
\pgfsys@transformshift{0.488751in}{0.918846in}%
\pgfsys@useobject{currentmarker}{}%
\end{pgfscope}%
\end{pgfscope}%
\begin{pgfscope}%
\definecolor{textcolor}{rgb}{0.317647,0.317647,0.317647}%
\pgfsetstrokecolor{textcolor}%
\pgfsetfillcolor{textcolor}%
\pgftext[x=0.398666in,y=0.886729in,left,base]{\color{textcolor}\rmfamily\fontsize{6.664000}{7.996800}\selectfont \(\displaystyle 0\)}%
\end{pgfscope}%
\begin{pgfscope}%
\pgfsetbuttcap%
\pgfsetroundjoin%
\definecolor{currentfill}{rgb}{0.317647,0.317647,0.317647}%
\pgfsetfillcolor{currentfill}%
\pgfsetlinewidth{0.501875pt}%
\definecolor{currentstroke}{rgb}{0.317647,0.317647,0.317647}%
\pgfsetstrokecolor{currentstroke}%
\pgfsetdash{}{0pt}%
\pgfsys@defobject{currentmarker}{\pgfqpoint{-0.020833in}{0.000000in}}{\pgfqpoint{0.000000in}{0.000000in}}{%
\pgfpathmoveto{\pgfqpoint{0.000000in}{0.000000in}}%
\pgfpathlineto{\pgfqpoint{-0.020833in}{0.000000in}}%
\pgfusepath{stroke,fill}%
}%
\begin{pgfscope}%
\pgfsys@transformshift{0.488751in}{1.199971in}%
\pgfsys@useobject{currentmarker}{}%
\end{pgfscope}%
\end{pgfscope}%
\begin{pgfscope}%
\definecolor{textcolor}{rgb}{0.317647,0.317647,0.317647}%
\pgfsetstrokecolor{textcolor}%
\pgfsetfillcolor{textcolor}%
\pgftext[x=0.343303in,y=1.167854in,left,base]{\color{textcolor}\rmfamily\fontsize{6.664000}{7.996800}\selectfont \(\displaystyle 20\)}%
\end{pgfscope}%
\begin{pgfscope}%
\pgfsetbuttcap%
\pgfsetroundjoin%
\definecolor{currentfill}{rgb}{0.317647,0.317647,0.317647}%
\pgfsetfillcolor{currentfill}%
\pgfsetlinewidth{0.501875pt}%
\definecolor{currentstroke}{rgb}{0.317647,0.317647,0.317647}%
\pgfsetstrokecolor{currentstroke}%
\pgfsetdash{}{0pt}%
\pgfsys@defobject{currentmarker}{\pgfqpoint{-0.020833in}{0.000000in}}{\pgfqpoint{0.000000in}{0.000000in}}{%
\pgfpathmoveto{\pgfqpoint{0.000000in}{0.000000in}}%
\pgfpathlineto{\pgfqpoint{-0.020833in}{0.000000in}}%
\pgfusepath{stroke,fill}%
}%
\begin{pgfscope}%
\pgfsys@transformshift{0.488751in}{1.481095in}%
\pgfsys@useobject{currentmarker}{}%
\end{pgfscope}%
\end{pgfscope}%
\begin{pgfscope}%
\definecolor{textcolor}{rgb}{0.317647,0.317647,0.317647}%
\pgfsetstrokecolor{textcolor}%
\pgfsetfillcolor{textcolor}%
\pgftext[x=0.343303in,y=1.448978in,left,base]{\color{textcolor}\rmfamily\fontsize{6.664000}{7.996800}\selectfont \(\displaystyle 40\)}%
\end{pgfscope}%
\begin{pgfscope}%
\definecolor{textcolor}{rgb}{0.317647,0.317647,0.317647}%
\pgfsetstrokecolor{textcolor}%
\pgfsetfillcolor{textcolor}%
\pgftext[x=0.200942in,y=1.010212in,,bottom,rotate=90.000000]{\color{textcolor}\rmfamily\fontsize{6.664000}{7.996800}\selectfont \(\displaystyle W^{(\mathrm{o})}\)}%
\end{pgfscope}%
\begin{pgfscope}%
\pgfpathrectangle{\pgfqpoint{0.488751in}{0.368545in}}{\pgfqpoint{2.260417in}{1.283333in}}%
\pgfusepath{clip}%
\pgfsetrectcap%
\pgfsetroundjoin%
\pgfsetlinewidth{0.803000pt}%
\definecolor{currentstroke}{rgb}{0.333333,0.333333,0.333333}%
\pgfsetstrokecolor{currentstroke}%
\pgfsetdash{}{0pt}%
\pgfpathmoveto{\pgfqpoint{0.591497in}{0.975071in}}%
\pgfpathlineto{\pgfqpoint{0.595615in}{0.989127in}}%
\pgfpathlineto{\pgfqpoint{0.607969in}{0.989127in}}%
\pgfpathlineto{\pgfqpoint{0.612087in}{1.003184in}}%
\pgfpathlineto{\pgfqpoint{0.632678in}{1.003184in}}%
\pgfpathlineto{\pgfqpoint{0.636796in}{1.017240in}}%
\pgfpathlineto{\pgfqpoint{0.640914in}{1.017240in}}%
\pgfpathlineto{\pgfqpoint{0.649150in}{1.045352in}}%
\pgfpathlineto{\pgfqpoint{0.653268in}{1.031296in}}%
\pgfpathlineto{\pgfqpoint{0.657386in}{1.031296in}}%
\pgfpathlineto{\pgfqpoint{0.661504in}{1.045352in}}%
\pgfpathlineto{\pgfqpoint{0.677977in}{1.045352in}}%
\pgfpathlineto{\pgfqpoint{0.682095in}{1.059408in}}%
\pgfpathlineto{\pgfqpoint{0.805637in}{1.059408in}}%
\pgfpathlineto{\pgfqpoint{0.809756in}{1.073465in}}%
\pgfpathlineto{\pgfqpoint{0.887999in}{1.073465in}}%
\pgfpathlineto{\pgfqpoint{0.892117in}{1.087521in}}%
\pgfpathlineto{\pgfqpoint{0.900353in}{1.087521in}}%
\pgfpathlineto{\pgfqpoint{0.908590in}{1.115633in}}%
\pgfpathlineto{\pgfqpoint{0.995069in}{1.115633in}}%
\pgfpathlineto{\pgfqpoint{0.999187in}{1.129690in}}%
\pgfpathlineto{\pgfqpoint{1.069195in}{1.129690in}}%
\pgfpathlineto{\pgfqpoint{1.073313in}{1.143746in}}%
\pgfpathlineto{\pgfqpoint{1.266863in}{1.143746in}}%
\pgfpathlineto{\pgfqpoint{1.270981in}{1.157802in}}%
\pgfpathlineto{\pgfqpoint{1.336870in}{1.157802in}}%
\pgfpathlineto{\pgfqpoint{1.340988in}{1.171858in}}%
\pgfpathlineto{\pgfqpoint{1.382169in}{1.171858in}}%
\pgfpathlineto{\pgfqpoint{1.386287in}{1.185914in}}%
\pgfpathlineto{\pgfqpoint{1.452177in}{1.185914in}}%
\pgfpathlineto{\pgfqpoint{1.456295in}{1.199971in}}%
\pgfpathlineto{\pgfqpoint{1.489240in}{1.199971in}}%
\pgfpathlineto{\pgfqpoint{1.493358in}{1.185914in}}%
\pgfpathlineto{\pgfqpoint{1.505712in}{1.185914in}}%
\pgfpathlineto{\pgfqpoint{1.509830in}{1.199971in}}%
\pgfpathlineto{\pgfqpoint{1.530420in}{1.199971in}}%
\pgfpathlineto{\pgfqpoint{1.534538in}{1.185914in}}%
\pgfpathlineto{\pgfqpoint{1.658081in}{1.185914in}}%
\pgfpathlineto{\pgfqpoint{1.662199in}{1.171858in}}%
\pgfpathlineto{\pgfqpoint{2.041063in}{1.171858in}}%
\pgfpathlineto{\pgfqpoint{2.045181in}{1.185914in}}%
\pgfpathlineto{\pgfqpoint{2.296384in}{1.185914in}}%
\pgfpathlineto{\pgfqpoint{2.300502in}{1.199971in}}%
\pgfpathlineto{\pgfqpoint{2.304620in}{1.199971in}}%
\pgfpathlineto{\pgfqpoint{2.308738in}{1.214027in}}%
\pgfpathlineto{\pgfqpoint{2.325211in}{1.214027in}}%
\pgfpathlineto{\pgfqpoint{2.329329in}{1.228083in}}%
\pgfpathlineto{\pgfqpoint{2.341683in}{1.228083in}}%
\pgfpathlineto{\pgfqpoint{2.349919in}{1.256196in}}%
\pgfpathlineto{\pgfqpoint{2.539351in}{1.256196in}}%
\pgfpathlineto{\pgfqpoint{2.543469in}{1.270252in}}%
\pgfpathlineto{\pgfqpoint{2.588768in}{1.270252in}}%
\pgfpathlineto{\pgfqpoint{2.592886in}{1.256196in}}%
\pgfpathlineto{\pgfqpoint{2.613477in}{1.256196in}}%
\pgfpathlineto{\pgfqpoint{2.617595in}{1.270252in}}%
\pgfpathlineto{\pgfqpoint{2.646421in}{1.270252in}}%
\pgfpathlineto{\pgfqpoint{2.646421in}{1.270252in}}%
\pgfusepath{stroke}%
\end{pgfscope}%
\begin{pgfscope}%
\pgfpathrectangle{\pgfqpoint{0.488751in}{0.368545in}}{\pgfqpoint{2.260417in}{1.283333in}}%
\pgfusepath{clip}%
\pgfsetrectcap%
\pgfsetroundjoin%
\pgfsetlinewidth{0.803000pt}%
\definecolor{currentstroke}{rgb}{0.686275,0.352941,0.313725}%
\pgfsetstrokecolor{currentstroke}%
\pgfsetdash{}{0pt}%
\pgfpathmoveto{\pgfqpoint{0.591497in}{0.637722in}}%
\pgfpathlineto{\pgfqpoint{0.607969in}{0.693947in}}%
\pgfpathlineto{\pgfqpoint{0.620324in}{0.693947in}}%
\pgfpathlineto{\pgfqpoint{0.624442in}{0.708003in}}%
\pgfpathlineto{\pgfqpoint{0.632678in}{0.708003in}}%
\pgfpathlineto{\pgfqpoint{0.636796in}{0.693947in}}%
\pgfpathlineto{\pgfqpoint{0.653268in}{0.693947in}}%
\pgfpathlineto{\pgfqpoint{0.657386in}{0.708003in}}%
\pgfpathlineto{\pgfqpoint{0.690331in}{0.708003in}}%
\pgfpathlineto{\pgfqpoint{0.694449in}{0.722059in}}%
\pgfpathlineto{\pgfqpoint{0.702685in}{0.722059in}}%
\pgfpathlineto{\pgfqpoint{0.706803in}{0.736115in}}%
\pgfpathlineto{\pgfqpoint{0.715040in}{0.736115in}}%
\pgfpathlineto{\pgfqpoint{0.719158in}{0.750172in}}%
\pgfpathlineto{\pgfqpoint{0.739748in}{0.750172in}}%
\pgfpathlineto{\pgfqpoint{0.743866in}{0.764228in}}%
\pgfpathlineto{\pgfqpoint{0.789165in}{0.764228in}}%
\pgfpathlineto{\pgfqpoint{0.793283in}{0.778284in}}%
\pgfpathlineto{\pgfqpoint{0.813874in}{0.778284in}}%
\pgfpathlineto{\pgfqpoint{0.817992in}{0.792340in}}%
\pgfpathlineto{\pgfqpoint{0.822110in}{0.792340in}}%
\pgfpathlineto{\pgfqpoint{0.826228in}{0.806396in}}%
\pgfpathlineto{\pgfqpoint{0.859173in}{0.806396in}}%
\pgfpathlineto{\pgfqpoint{0.863291in}{0.820453in}}%
\pgfpathlineto{\pgfqpoint{0.875645in}{0.820453in}}%
\pgfpathlineto{\pgfqpoint{0.879763in}{0.834509in}}%
\pgfpathlineto{\pgfqpoint{0.887999in}{0.834509in}}%
\pgfpathlineto{\pgfqpoint{0.892117in}{0.862621in}}%
\pgfpathlineto{\pgfqpoint{0.896235in}{0.876678in}}%
\pgfpathlineto{\pgfqpoint{0.912708in}{0.876678in}}%
\pgfpathlineto{\pgfqpoint{0.916826in}{0.890734in}}%
\pgfpathlineto{\pgfqpoint{0.929180in}{0.890734in}}%
\pgfpathlineto{\pgfqpoint{0.937416in}{0.918846in}}%
\pgfpathlineto{\pgfqpoint{0.990951in}{0.918846in}}%
\pgfpathlineto{\pgfqpoint{0.995069in}{0.904790in}}%
\pgfpathlineto{\pgfqpoint{1.019778in}{0.904790in}}%
\pgfpathlineto{\pgfqpoint{1.023896in}{0.918846in}}%
\pgfpathlineto{\pgfqpoint{1.106258in}{0.918846in}}%
\pgfpathlineto{\pgfqpoint{1.110376in}{0.932902in}}%
\pgfpathlineto{\pgfqpoint{1.135084in}{0.932902in}}%
\pgfpathlineto{\pgfqpoint{1.139202in}{0.946959in}}%
\pgfpathlineto{\pgfqpoint{1.159793in}{0.946959in}}%
\pgfpathlineto{\pgfqpoint{1.163911in}{0.961015in}}%
\pgfpathlineto{\pgfqpoint{1.180383in}{0.961015in}}%
\pgfpathlineto{\pgfqpoint{1.184501in}{0.946959in}}%
\pgfpathlineto{\pgfqpoint{1.192737in}{0.975071in}}%
\pgfpathlineto{\pgfqpoint{1.262745in}{0.975071in}}%
\pgfpathlineto{\pgfqpoint{1.266863in}{0.989127in}}%
\pgfpathlineto{\pgfqpoint{1.299808in}{0.989127in}}%
\pgfpathlineto{\pgfqpoint{1.303926in}{0.975071in}}%
\pgfpathlineto{\pgfqpoint{1.332752in}{0.975071in}}%
\pgfpathlineto{\pgfqpoint{1.336870in}{0.961015in}}%
\pgfpathlineto{\pgfqpoint{1.361579in}{0.961015in}}%
\pgfpathlineto{\pgfqpoint{1.365697in}{0.946959in}}%
\pgfpathlineto{\pgfqpoint{1.369815in}{0.946959in}}%
\pgfpathlineto{\pgfqpoint{1.373933in}{0.932902in}}%
\pgfpathlineto{\pgfqpoint{1.386287in}{0.932902in}}%
\pgfpathlineto{\pgfqpoint{1.390405in}{0.946959in}}%
\pgfpathlineto{\pgfqpoint{1.394524in}{0.946959in}}%
\pgfpathlineto{\pgfqpoint{1.402760in}{0.918846in}}%
\pgfpathlineto{\pgfqpoint{1.423350in}{0.918846in}}%
\pgfpathlineto{\pgfqpoint{1.427468in}{0.904790in}}%
\pgfpathlineto{\pgfqpoint{1.443941in}{0.904790in}}%
\pgfpathlineto{\pgfqpoint{1.452177in}{0.876678in}}%
\pgfpathlineto{\pgfqpoint{1.456295in}{0.876678in}}%
\pgfpathlineto{\pgfqpoint{1.460413in}{0.904790in}}%
\pgfpathlineto{\pgfqpoint{1.472767in}{0.904790in}}%
\pgfpathlineto{\pgfqpoint{1.476885in}{0.932902in}}%
\pgfpathlineto{\pgfqpoint{1.489240in}{0.932902in}}%
\pgfpathlineto{\pgfqpoint{1.493358in}{0.946959in}}%
\pgfpathlineto{\pgfqpoint{1.497476in}{0.946959in}}%
\pgfpathlineto{\pgfqpoint{1.501594in}{0.918846in}}%
\pgfpathlineto{\pgfqpoint{1.522184in}{0.918846in}}%
\pgfpathlineto{\pgfqpoint{1.526302in}{0.932902in}}%
\pgfpathlineto{\pgfqpoint{1.563365in}{0.932902in}}%
\pgfpathlineto{\pgfqpoint{1.567483in}{0.946959in}}%
\pgfpathlineto{\pgfqpoint{1.571601in}{0.932902in}}%
\pgfpathlineto{\pgfqpoint{1.575719in}{0.932902in}}%
\pgfpathlineto{\pgfqpoint{1.579837in}{0.918846in}}%
\pgfpathlineto{\pgfqpoint{1.596310in}{0.918846in}}%
\pgfpathlineto{\pgfqpoint{1.600428in}{0.932902in}}%
\pgfpathlineto{\pgfqpoint{1.612782in}{0.932902in}}%
\pgfpathlineto{\pgfqpoint{1.621018in}{0.904790in}}%
\pgfpathlineto{\pgfqpoint{1.625136in}{0.918846in}}%
\pgfpathlineto{\pgfqpoint{1.629254in}{0.918846in}}%
\pgfpathlineto{\pgfqpoint{1.633372in}{0.904790in}}%
\pgfpathlineto{\pgfqpoint{1.658081in}{0.904790in}}%
\pgfpathlineto{\pgfqpoint{1.662199in}{0.890734in}}%
\pgfpathlineto{\pgfqpoint{1.666317in}{0.904790in}}%
\pgfpathlineto{\pgfqpoint{1.703380in}{0.904790in}}%
\pgfpathlineto{\pgfqpoint{1.707498in}{0.918846in}}%
\pgfpathlineto{\pgfqpoint{1.736325in}{0.918846in}}%
\pgfpathlineto{\pgfqpoint{1.740443in}{0.890734in}}%
\pgfpathlineto{\pgfqpoint{1.744561in}{0.876678in}}%
\pgfpathlineto{\pgfqpoint{1.773387in}{0.876678in}}%
\pgfpathlineto{\pgfqpoint{1.777505in}{0.890734in}}%
\pgfpathlineto{\pgfqpoint{1.802214in}{0.890734in}}%
\pgfpathlineto{\pgfqpoint{1.806332in}{0.876678in}}%
\pgfpathlineto{\pgfqpoint{1.818686in}{0.876678in}}%
\pgfpathlineto{\pgfqpoint{1.822804in}{0.862621in}}%
\pgfpathlineto{\pgfqpoint{1.826922in}{0.834509in}}%
\pgfpathlineto{\pgfqpoint{1.831041in}{0.820453in}}%
\pgfpathlineto{\pgfqpoint{1.847513in}{0.820453in}}%
\pgfpathlineto{\pgfqpoint{1.851631in}{0.806396in}}%
\pgfpathlineto{\pgfqpoint{1.868103in}{0.806396in}}%
\pgfpathlineto{\pgfqpoint{1.872221in}{0.792340in}}%
\pgfpathlineto{\pgfqpoint{1.888694in}{0.792340in}}%
\pgfpathlineto{\pgfqpoint{1.892812in}{0.820453in}}%
\pgfpathlineto{\pgfqpoint{1.896930in}{0.820453in}}%
\pgfpathlineto{\pgfqpoint{1.901048in}{0.848565in}}%
\pgfpathlineto{\pgfqpoint{1.925756in}{0.848565in}}%
\pgfpathlineto{\pgfqpoint{1.929875in}{0.834509in}}%
\pgfpathlineto{\pgfqpoint{1.933993in}{0.834509in}}%
\pgfpathlineto{\pgfqpoint{1.938111in}{0.848565in}}%
\pgfpathlineto{\pgfqpoint{1.946347in}{0.848565in}}%
\pgfpathlineto{\pgfqpoint{1.950465in}{0.834509in}}%
\pgfpathlineto{\pgfqpoint{1.954583in}{0.834509in}}%
\pgfpathlineto{\pgfqpoint{1.958701in}{0.820453in}}%
\pgfpathlineto{\pgfqpoint{1.983410in}{0.820453in}}%
\pgfpathlineto{\pgfqpoint{1.987528in}{0.792340in}}%
\pgfpathlineto{\pgfqpoint{1.991646in}{0.792340in}}%
\pgfpathlineto{\pgfqpoint{1.995764in}{0.778284in}}%
\pgfpathlineto{\pgfqpoint{2.045181in}{0.778284in}}%
\pgfpathlineto{\pgfqpoint{2.053417in}{0.806396in}}%
\pgfpathlineto{\pgfqpoint{2.061653in}{0.806396in}}%
\pgfpathlineto{\pgfqpoint{2.069889in}{0.778284in}}%
\pgfpathlineto{\pgfqpoint{2.082244in}{0.778284in}}%
\pgfpathlineto{\pgfqpoint{2.090480in}{0.750172in}}%
\pgfpathlineto{\pgfqpoint{2.111070in}{0.750172in}}%
\pgfpathlineto{\pgfqpoint{2.115188in}{0.736115in}}%
\pgfpathlineto{\pgfqpoint{2.123425in}{0.736115in}}%
\pgfpathlineto{\pgfqpoint{2.127543in}{0.750172in}}%
\pgfpathlineto{\pgfqpoint{2.144015in}{0.750172in}}%
\pgfpathlineto{\pgfqpoint{2.148133in}{0.722059in}}%
\pgfpathlineto{\pgfqpoint{2.156369in}{0.722059in}}%
\pgfpathlineto{\pgfqpoint{2.160487in}{0.708003in}}%
\pgfpathlineto{\pgfqpoint{2.193432in}{0.708003in}}%
\pgfpathlineto{\pgfqpoint{2.197550in}{0.722059in}}%
\pgfpathlineto{\pgfqpoint{2.201668in}{0.722059in}}%
\pgfpathlineto{\pgfqpoint{2.205786in}{0.708003in}}%
\pgfpathlineto{\pgfqpoint{2.267557in}{0.708003in}}%
\pgfpathlineto{\pgfqpoint{2.271676in}{0.693947in}}%
\pgfpathlineto{\pgfqpoint{2.300502in}{0.693947in}}%
\pgfpathlineto{\pgfqpoint{2.304620in}{0.679890in}}%
\pgfpathlineto{\pgfqpoint{2.333447in}{0.679890in}}%
\pgfpathlineto{\pgfqpoint{2.337565in}{0.665834in}}%
\pgfpathlineto{\pgfqpoint{2.391100in}{0.665834in}}%
\pgfpathlineto{\pgfqpoint{2.395218in}{0.679890in}}%
\pgfpathlineto{\pgfqpoint{2.399336in}{0.679890in}}%
\pgfpathlineto{\pgfqpoint{2.403454in}{0.665834in}}%
\pgfpathlineto{\pgfqpoint{2.411690in}{0.665834in}}%
\pgfpathlineto{\pgfqpoint{2.415809in}{0.679890in}}%
\pgfpathlineto{\pgfqpoint{2.469344in}{0.679890in}}%
\pgfpathlineto{\pgfqpoint{2.473462in}{0.693947in}}%
\pgfpathlineto{\pgfqpoint{2.485816in}{0.693947in}}%
\pgfpathlineto{\pgfqpoint{2.489934in}{0.679890in}}%
\pgfpathlineto{\pgfqpoint{2.502288in}{0.679890in}}%
\pgfpathlineto{\pgfqpoint{2.506406in}{0.693947in}}%
\pgfpathlineto{\pgfqpoint{2.510524in}{0.693947in}}%
\pgfpathlineto{\pgfqpoint{2.514643in}{0.679890in}}%
\pgfpathlineto{\pgfqpoint{2.522879in}{0.679890in}}%
\pgfpathlineto{\pgfqpoint{2.526997in}{0.665834in}}%
\pgfpathlineto{\pgfqpoint{2.555823in}{0.665834in}}%
\pgfpathlineto{\pgfqpoint{2.559942in}{0.651778in}}%
\pgfpathlineto{\pgfqpoint{2.576414in}{0.651778in}}%
\pgfpathlineto{\pgfqpoint{2.580532in}{0.637722in}}%
\pgfpathlineto{\pgfqpoint{2.588768in}{0.637722in}}%
\pgfpathlineto{\pgfqpoint{2.592886in}{0.623665in}}%
\pgfpathlineto{\pgfqpoint{2.605240in}{0.623665in}}%
\pgfpathlineto{\pgfqpoint{2.609359in}{0.637722in}}%
\pgfpathlineto{\pgfqpoint{2.617595in}{0.637722in}}%
\pgfpathlineto{\pgfqpoint{2.625831in}{0.665834in}}%
\pgfpathlineto{\pgfqpoint{2.642303in}{0.665834in}}%
\pgfpathlineto{\pgfqpoint{2.646421in}{0.651778in}}%
\pgfpathlineto{\pgfqpoint{2.646421in}{0.651778in}}%
\pgfusepath{stroke}%
\end{pgfscope}%
\begin{pgfscope}%
\pgfpathrectangle{\pgfqpoint{0.488751in}{0.368545in}}{\pgfqpoint{2.260417in}{1.283333in}}%
\pgfusepath{clip}%
\pgfsetrectcap%
\pgfsetroundjoin%
\pgfsetlinewidth{0.803000pt}%
\definecolor{currentstroke}{rgb}{0.000000,0.356863,0.509804}%
\pgfsetstrokecolor{currentstroke}%
\pgfsetdash{}{0pt}%
\pgfpathmoveto{\pgfqpoint{0.591497in}{0.651778in}}%
\pgfpathlineto{\pgfqpoint{0.599733in}{0.651778in}}%
\pgfpathlineto{\pgfqpoint{0.607969in}{0.679890in}}%
\pgfpathlineto{\pgfqpoint{0.612087in}{0.679890in}}%
\pgfpathlineto{\pgfqpoint{0.616206in}{0.722059in}}%
\pgfpathlineto{\pgfqpoint{0.624442in}{0.722059in}}%
\pgfpathlineto{\pgfqpoint{0.632678in}{0.750172in}}%
\pgfpathlineto{\pgfqpoint{0.636796in}{0.750172in}}%
\pgfpathlineto{\pgfqpoint{0.640914in}{0.778284in}}%
\pgfpathlineto{\pgfqpoint{0.649150in}{0.806396in}}%
\pgfpathlineto{\pgfqpoint{0.657386in}{0.806396in}}%
\pgfpathlineto{\pgfqpoint{0.661504in}{0.834509in}}%
\pgfpathlineto{\pgfqpoint{0.669741in}{0.806396in}}%
\pgfpathlineto{\pgfqpoint{0.677977in}{0.806396in}}%
\pgfpathlineto{\pgfqpoint{0.682095in}{0.820453in}}%
\pgfpathlineto{\pgfqpoint{0.686213in}{0.806396in}}%
\pgfpathlineto{\pgfqpoint{0.694449in}{0.806396in}}%
\pgfpathlineto{\pgfqpoint{0.698567in}{0.792340in}}%
\pgfpathlineto{\pgfqpoint{0.706803in}{0.792340in}}%
\pgfpathlineto{\pgfqpoint{0.715040in}{0.820453in}}%
\pgfpathlineto{\pgfqpoint{0.719158in}{0.806396in}}%
\pgfpathlineto{\pgfqpoint{0.747984in}{0.806396in}}%
\pgfpathlineto{\pgfqpoint{0.752102in}{0.792340in}}%
\pgfpathlineto{\pgfqpoint{0.764457in}{0.792340in}}%
\pgfpathlineto{\pgfqpoint{0.768575in}{0.778284in}}%
\pgfpathlineto{\pgfqpoint{0.772693in}{0.778284in}}%
\pgfpathlineto{\pgfqpoint{0.776811in}{0.764228in}}%
\pgfpathlineto{\pgfqpoint{0.780929in}{0.764228in}}%
\pgfpathlineto{\pgfqpoint{0.785047in}{0.750172in}}%
\pgfpathlineto{\pgfqpoint{0.797401in}{0.750172in}}%
\pgfpathlineto{\pgfqpoint{0.801519in}{0.764228in}}%
\pgfpathlineto{\pgfqpoint{0.809756in}{0.736115in}}%
\pgfpathlineto{\pgfqpoint{0.813874in}{0.736115in}}%
\pgfpathlineto{\pgfqpoint{0.817992in}{0.750172in}}%
\pgfpathlineto{\pgfqpoint{0.822110in}{0.750172in}}%
\pgfpathlineto{\pgfqpoint{0.826228in}{0.736115in}}%
\pgfpathlineto{\pgfqpoint{0.846818in}{0.736115in}}%
\pgfpathlineto{\pgfqpoint{0.850936in}{0.764228in}}%
\pgfpathlineto{\pgfqpoint{0.855054in}{0.778284in}}%
\pgfpathlineto{\pgfqpoint{0.863291in}{0.778284in}}%
\pgfpathlineto{\pgfqpoint{0.867409in}{0.764228in}}%
\pgfpathlineto{\pgfqpoint{0.879763in}{0.764228in}}%
\pgfpathlineto{\pgfqpoint{0.883881in}{0.778284in}}%
\pgfpathlineto{\pgfqpoint{0.887999in}{0.764228in}}%
\pgfpathlineto{\pgfqpoint{0.892117in}{0.792340in}}%
\pgfpathlineto{\pgfqpoint{0.896235in}{0.764228in}}%
\pgfpathlineto{\pgfqpoint{0.908590in}{0.806396in}}%
\pgfpathlineto{\pgfqpoint{0.920944in}{0.806396in}}%
\pgfpathlineto{\pgfqpoint{0.925062in}{0.792340in}}%
\pgfpathlineto{\pgfqpoint{0.929180in}{0.806396in}}%
\pgfpathlineto{\pgfqpoint{0.933298in}{0.806396in}}%
\pgfpathlineto{\pgfqpoint{0.937416in}{0.820453in}}%
\pgfpathlineto{\pgfqpoint{0.978597in}{0.820453in}}%
\pgfpathlineto{\pgfqpoint{0.982715in}{0.834509in}}%
\pgfpathlineto{\pgfqpoint{0.986833in}{0.820453in}}%
\pgfpathlineto{\pgfqpoint{0.995069in}{0.820453in}}%
\pgfpathlineto{\pgfqpoint{0.999187in}{0.806396in}}%
\pgfpathlineto{\pgfqpoint{1.019778in}{0.806396in}}%
\pgfpathlineto{\pgfqpoint{1.023896in}{0.820453in}}%
\pgfpathlineto{\pgfqpoint{1.032132in}{0.820453in}}%
\pgfpathlineto{\pgfqpoint{1.036250in}{0.834509in}}%
\pgfpathlineto{\pgfqpoint{1.040368in}{0.834509in}}%
\pgfpathlineto{\pgfqpoint{1.052723in}{0.792340in}}%
\pgfpathlineto{\pgfqpoint{1.056841in}{0.806396in}}%
\pgfpathlineto{\pgfqpoint{1.060959in}{0.806396in}}%
\pgfpathlineto{\pgfqpoint{1.065077in}{0.820453in}}%
\pgfpathlineto{\pgfqpoint{1.073313in}{0.820453in}}%
\pgfpathlineto{\pgfqpoint{1.077431in}{0.834509in}}%
\pgfpathlineto{\pgfqpoint{1.089785in}{0.834509in}}%
\pgfpathlineto{\pgfqpoint{1.093903in}{0.806396in}}%
\pgfpathlineto{\pgfqpoint{1.110376in}{0.806396in}}%
\pgfpathlineto{\pgfqpoint{1.114494in}{0.820453in}}%
\pgfpathlineto{\pgfqpoint{1.135084in}{0.820453in}}%
\pgfpathlineto{\pgfqpoint{1.139202in}{0.848565in}}%
\pgfpathlineto{\pgfqpoint{1.155675in}{0.848565in}}%
\pgfpathlineto{\pgfqpoint{1.159793in}{0.834509in}}%
\pgfpathlineto{\pgfqpoint{1.172147in}{0.834509in}}%
\pgfpathlineto{\pgfqpoint{1.176265in}{0.820453in}}%
\pgfpathlineto{\pgfqpoint{1.180383in}{0.820453in}}%
\pgfpathlineto{\pgfqpoint{1.184501in}{0.834509in}}%
\pgfpathlineto{\pgfqpoint{1.188619in}{0.834509in}}%
\pgfpathlineto{\pgfqpoint{1.192737in}{0.806396in}}%
\pgfpathlineto{\pgfqpoint{1.196855in}{0.792340in}}%
\pgfpathlineto{\pgfqpoint{1.200974in}{0.792340in}}%
\pgfpathlineto{\pgfqpoint{1.205092in}{0.778284in}}%
\pgfpathlineto{\pgfqpoint{1.209210in}{0.792340in}}%
\pgfpathlineto{\pgfqpoint{1.213328in}{0.820453in}}%
\pgfpathlineto{\pgfqpoint{1.225682in}{0.820453in}}%
\pgfpathlineto{\pgfqpoint{1.229800in}{0.806396in}}%
\pgfpathlineto{\pgfqpoint{1.233918in}{0.806396in}}%
\pgfpathlineto{\pgfqpoint{1.238036in}{0.792340in}}%
\pgfpathlineto{\pgfqpoint{1.242154in}{0.806396in}}%
\pgfpathlineto{\pgfqpoint{1.254509in}{0.806396in}}%
\pgfpathlineto{\pgfqpoint{1.262745in}{0.778284in}}%
\pgfpathlineto{\pgfqpoint{1.266863in}{0.778284in}}%
\pgfpathlineto{\pgfqpoint{1.275099in}{0.806396in}}%
\pgfpathlineto{\pgfqpoint{1.291571in}{0.806396in}}%
\pgfpathlineto{\pgfqpoint{1.299808in}{0.778284in}}%
\pgfpathlineto{\pgfqpoint{1.308044in}{0.778284in}}%
\pgfpathlineto{\pgfqpoint{1.312162in}{0.792340in}}%
\pgfpathlineto{\pgfqpoint{1.328634in}{0.792340in}}%
\pgfpathlineto{\pgfqpoint{1.332752in}{0.778284in}}%
\pgfpathlineto{\pgfqpoint{1.369815in}{0.778284in}}%
\pgfpathlineto{\pgfqpoint{1.373933in}{0.792340in}}%
\pgfpathlineto{\pgfqpoint{1.386287in}{0.792340in}}%
\pgfpathlineto{\pgfqpoint{1.394524in}{0.820453in}}%
\pgfpathlineto{\pgfqpoint{1.398642in}{0.820453in}}%
\pgfpathlineto{\pgfqpoint{1.402760in}{0.834509in}}%
\pgfpathlineto{\pgfqpoint{1.406878in}{0.820453in}}%
\pgfpathlineto{\pgfqpoint{1.410996in}{0.820453in}}%
\pgfpathlineto{\pgfqpoint{1.415114in}{0.806396in}}%
\pgfpathlineto{\pgfqpoint{1.448059in}{0.806396in}}%
\pgfpathlineto{\pgfqpoint{1.452177in}{0.792340in}}%
\pgfpathlineto{\pgfqpoint{1.456295in}{0.792340in}}%
\pgfpathlineto{\pgfqpoint{1.460413in}{0.806396in}}%
\pgfpathlineto{\pgfqpoint{1.472767in}{0.806396in}}%
\pgfpathlineto{\pgfqpoint{1.476885in}{0.792340in}}%
\pgfpathlineto{\pgfqpoint{1.481003in}{0.792340in}}%
\pgfpathlineto{\pgfqpoint{1.489240in}{0.820453in}}%
\pgfpathlineto{\pgfqpoint{1.493358in}{0.806396in}}%
\pgfpathlineto{\pgfqpoint{1.497476in}{0.806396in}}%
\pgfpathlineto{\pgfqpoint{1.501594in}{0.792340in}}%
\pgfpathlineto{\pgfqpoint{1.559247in}{0.792340in}}%
\pgfpathlineto{\pgfqpoint{1.563365in}{0.778284in}}%
\pgfpathlineto{\pgfqpoint{1.567483in}{0.792340in}}%
\pgfpathlineto{\pgfqpoint{1.571601in}{0.792340in}}%
\pgfpathlineto{\pgfqpoint{1.575719in}{0.778284in}}%
\pgfpathlineto{\pgfqpoint{1.596310in}{0.778284in}}%
\pgfpathlineto{\pgfqpoint{1.600428in}{0.792340in}}%
\pgfpathlineto{\pgfqpoint{1.604546in}{0.792340in}}%
\pgfpathlineto{\pgfqpoint{1.608664in}{0.806396in}}%
\pgfpathlineto{\pgfqpoint{1.612782in}{0.792340in}}%
\pgfpathlineto{\pgfqpoint{1.616900in}{0.792340in}}%
\pgfpathlineto{\pgfqpoint{1.621018in}{0.778284in}}%
\pgfpathlineto{\pgfqpoint{1.625136in}{0.792340in}}%
\pgfpathlineto{\pgfqpoint{1.633372in}{0.792340in}}%
\pgfpathlineto{\pgfqpoint{1.637491in}{0.806396in}}%
\pgfpathlineto{\pgfqpoint{1.670435in}{0.806396in}}%
\pgfpathlineto{\pgfqpoint{1.674553in}{0.792340in}}%
\pgfpathlineto{\pgfqpoint{1.678671in}{0.792340in}}%
\pgfpathlineto{\pgfqpoint{1.682789in}{0.778284in}}%
\pgfpathlineto{\pgfqpoint{1.691026in}{0.778284in}}%
\pgfpathlineto{\pgfqpoint{1.695144in}{0.764228in}}%
\pgfpathlineto{\pgfqpoint{1.707498in}{0.764228in}}%
\pgfpathlineto{\pgfqpoint{1.711616in}{0.778284in}}%
\pgfpathlineto{\pgfqpoint{1.732206in}{0.778284in}}%
\pgfpathlineto{\pgfqpoint{1.736325in}{0.764228in}}%
\pgfpathlineto{\pgfqpoint{1.740443in}{0.764228in}}%
\pgfpathlineto{\pgfqpoint{1.744561in}{0.778284in}}%
\pgfpathlineto{\pgfqpoint{1.761033in}{0.778284in}}%
\pgfpathlineto{\pgfqpoint{1.765151in}{0.792340in}}%
\pgfpathlineto{\pgfqpoint{1.769269in}{0.778284in}}%
\pgfpathlineto{\pgfqpoint{1.777505in}{0.778284in}}%
\pgfpathlineto{\pgfqpoint{1.781624in}{0.792340in}}%
\pgfpathlineto{\pgfqpoint{1.785742in}{0.792340in}}%
\pgfpathlineto{\pgfqpoint{1.789860in}{0.778284in}}%
\pgfpathlineto{\pgfqpoint{1.793978in}{0.778284in}}%
\pgfpathlineto{\pgfqpoint{1.798096in}{0.764228in}}%
\pgfpathlineto{\pgfqpoint{1.802214in}{0.778284in}}%
\pgfpathlineto{\pgfqpoint{1.810450in}{0.778284in}}%
\pgfpathlineto{\pgfqpoint{1.814568in}{0.792340in}}%
\pgfpathlineto{\pgfqpoint{1.818686in}{0.778284in}}%
\pgfpathlineto{\pgfqpoint{1.822804in}{0.792340in}}%
\pgfpathlineto{\pgfqpoint{1.855749in}{0.792340in}}%
\pgfpathlineto{\pgfqpoint{1.859867in}{0.806396in}}%
\pgfpathlineto{\pgfqpoint{1.863985in}{0.806396in}}%
\pgfpathlineto{\pgfqpoint{1.868103in}{0.792340in}}%
\pgfpathlineto{\pgfqpoint{1.876339in}{0.792340in}}%
\pgfpathlineto{\pgfqpoint{1.884576in}{0.764228in}}%
\pgfpathlineto{\pgfqpoint{1.888694in}{0.778284in}}%
\pgfpathlineto{\pgfqpoint{1.896930in}{0.778284in}}%
\pgfpathlineto{\pgfqpoint{1.901048in}{0.806396in}}%
\pgfpathlineto{\pgfqpoint{1.938111in}{0.806396in}}%
\pgfpathlineto{\pgfqpoint{1.942229in}{0.820453in}}%
\pgfpathlineto{\pgfqpoint{1.954583in}{0.820453in}}%
\pgfpathlineto{\pgfqpoint{1.958701in}{0.834509in}}%
\pgfpathlineto{\pgfqpoint{1.975173in}{0.834509in}}%
\pgfpathlineto{\pgfqpoint{1.979292in}{0.820453in}}%
\pgfpathlineto{\pgfqpoint{1.983410in}{0.834509in}}%
\pgfpathlineto{\pgfqpoint{1.991646in}{0.834509in}}%
\pgfpathlineto{\pgfqpoint{1.995764in}{0.848565in}}%
\pgfpathlineto{\pgfqpoint{1.999882in}{0.834509in}}%
\pgfpathlineto{\pgfqpoint{2.004000in}{0.848565in}}%
\pgfpathlineto{\pgfqpoint{2.008118in}{0.848565in}}%
\pgfpathlineto{\pgfqpoint{2.012236in}{0.862621in}}%
\pgfpathlineto{\pgfqpoint{2.016354in}{0.862621in}}%
\pgfpathlineto{\pgfqpoint{2.020472in}{0.876678in}}%
\pgfpathlineto{\pgfqpoint{2.028709in}{0.876678in}}%
\pgfpathlineto{\pgfqpoint{2.032827in}{0.862621in}}%
\pgfpathlineto{\pgfqpoint{2.036945in}{0.876678in}}%
\pgfpathlineto{\pgfqpoint{2.041063in}{0.876678in}}%
\pgfpathlineto{\pgfqpoint{2.045181in}{0.862621in}}%
\pgfpathlineto{\pgfqpoint{2.061653in}{0.862621in}}%
\pgfpathlineto{\pgfqpoint{2.065771in}{0.876678in}}%
\pgfpathlineto{\pgfqpoint{2.086362in}{0.876678in}}%
\pgfpathlineto{\pgfqpoint{2.090480in}{0.862621in}}%
\pgfpathlineto{\pgfqpoint{2.094598in}{0.876678in}}%
\pgfpathlineto{\pgfqpoint{2.098716in}{0.876678in}}%
\pgfpathlineto{\pgfqpoint{2.102834in}{0.862621in}}%
\pgfpathlineto{\pgfqpoint{2.115188in}{0.862621in}}%
\pgfpathlineto{\pgfqpoint{2.119306in}{0.848565in}}%
\pgfpathlineto{\pgfqpoint{2.148133in}{0.848565in}}%
\pgfpathlineto{\pgfqpoint{2.152251in}{0.862621in}}%
\pgfpathlineto{\pgfqpoint{2.185196in}{0.862621in}}%
\pgfpathlineto{\pgfqpoint{2.189314in}{0.848565in}}%
\pgfpathlineto{\pgfqpoint{2.201668in}{0.848565in}}%
\pgfpathlineto{\pgfqpoint{2.209904in}{0.820453in}}%
\pgfpathlineto{\pgfqpoint{2.214022in}{0.834509in}}%
\pgfpathlineto{\pgfqpoint{2.234613in}{0.834509in}}%
\pgfpathlineto{\pgfqpoint{2.238731in}{0.848565in}}%
\pgfpathlineto{\pgfqpoint{2.251085in}{0.848565in}}%
\pgfpathlineto{\pgfqpoint{2.255203in}{0.862621in}}%
\pgfpathlineto{\pgfqpoint{2.316975in}{0.862621in}}%
\pgfpathlineto{\pgfqpoint{2.321093in}{0.876678in}}%
\pgfpathlineto{\pgfqpoint{2.329329in}{0.876678in}}%
\pgfpathlineto{\pgfqpoint{2.333447in}{0.862621in}}%
\pgfpathlineto{\pgfqpoint{2.366392in}{0.862621in}}%
\pgfpathlineto{\pgfqpoint{2.370510in}{0.876678in}}%
\pgfpathlineto{\pgfqpoint{2.382864in}{0.876678in}}%
\pgfpathlineto{\pgfqpoint{2.386982in}{0.890734in}}%
\pgfpathlineto{\pgfqpoint{2.419927in}{0.890734in}}%
\pgfpathlineto{\pgfqpoint{2.424045in}{0.876678in}}%
\pgfpathlineto{\pgfqpoint{2.432281in}{0.876678in}}%
\pgfpathlineto{\pgfqpoint{2.436399in}{0.890734in}}%
\pgfpathlineto{\pgfqpoint{2.489934in}{0.890734in}}%
\pgfpathlineto{\pgfqpoint{2.494052in}{0.904790in}}%
\pgfpathlineto{\pgfqpoint{2.646421in}{0.904790in}}%
\pgfpathlineto{\pgfqpoint{2.646421in}{0.904790in}}%
\pgfusepath{stroke}%
\end{pgfscope}%
\begin{pgfscope}%
\pgfpathrectangle{\pgfqpoint{0.488751in}{0.368545in}}{\pgfqpoint{2.260417in}{1.283333in}}%
\pgfusepath{clip}%
\pgfsetrectcap%
\pgfsetroundjoin%
\pgfsetlinewidth{0.803000pt}%
\definecolor{currentstroke}{rgb}{0.490196,0.588235,0.431373}%
\pgfsetstrokecolor{currentstroke}%
\pgfsetdash{}{0pt}%
\pgfpathmoveto{\pgfqpoint{0.591497in}{0.764228in}}%
\pgfpathlineto{\pgfqpoint{0.599733in}{0.764228in}}%
\pgfpathlineto{\pgfqpoint{0.603851in}{0.778284in}}%
\pgfpathlineto{\pgfqpoint{0.620324in}{0.778284in}}%
\pgfpathlineto{\pgfqpoint{0.624442in}{0.792340in}}%
\pgfpathlineto{\pgfqpoint{0.632678in}{0.792340in}}%
\pgfpathlineto{\pgfqpoint{0.636796in}{0.806396in}}%
\pgfpathlineto{\pgfqpoint{0.640914in}{0.806396in}}%
\pgfpathlineto{\pgfqpoint{0.645032in}{0.820453in}}%
\pgfpathlineto{\pgfqpoint{0.653268in}{0.820453in}}%
\pgfpathlineto{\pgfqpoint{0.657386in}{0.834509in}}%
\pgfpathlineto{\pgfqpoint{0.682095in}{0.834509in}}%
\pgfpathlineto{\pgfqpoint{0.686213in}{0.848565in}}%
\pgfpathlineto{\pgfqpoint{0.690331in}{0.848565in}}%
\pgfpathlineto{\pgfqpoint{0.694449in}{0.862621in}}%
\pgfpathlineto{\pgfqpoint{0.715040in}{0.862621in}}%
\pgfpathlineto{\pgfqpoint{0.719158in}{0.876678in}}%
\pgfpathlineto{\pgfqpoint{0.735630in}{0.876678in}}%
\pgfpathlineto{\pgfqpoint{0.743866in}{0.904790in}}%
\pgfpathlineto{\pgfqpoint{0.752102in}{0.904790in}}%
\pgfpathlineto{\pgfqpoint{0.756220in}{0.918846in}}%
\pgfpathlineto{\pgfqpoint{0.789165in}{0.918846in}}%
\pgfpathlineto{\pgfqpoint{0.793283in}{0.932902in}}%
\pgfpathlineto{\pgfqpoint{0.805637in}{0.932902in}}%
\pgfpathlineto{\pgfqpoint{0.809756in}{0.946959in}}%
\pgfpathlineto{\pgfqpoint{0.813874in}{0.946959in}}%
\pgfpathlineto{\pgfqpoint{0.817992in}{0.961015in}}%
\pgfpathlineto{\pgfqpoint{0.875645in}{0.961015in}}%
\pgfpathlineto{\pgfqpoint{0.879763in}{0.975071in}}%
\pgfpathlineto{\pgfqpoint{0.887999in}{0.975071in}}%
\pgfpathlineto{\pgfqpoint{0.900353in}{1.017240in}}%
\pgfpathlineto{\pgfqpoint{1.019778in}{1.017240in}}%
\pgfpathlineto{\pgfqpoint{1.023896in}{1.031296in}}%
\pgfpathlineto{\pgfqpoint{1.073313in}{1.031296in}}%
\pgfpathlineto{\pgfqpoint{1.077431in}{1.045352in}}%
\pgfpathlineto{\pgfqpoint{1.118612in}{1.045352in}}%
\pgfpathlineto{\pgfqpoint{1.122730in}{1.059408in}}%
\pgfpathlineto{\pgfqpoint{1.135084in}{1.059408in}}%
\pgfpathlineto{\pgfqpoint{1.139202in}{1.073465in}}%
\pgfpathlineto{\pgfqpoint{1.147438in}{1.073465in}}%
\pgfpathlineto{\pgfqpoint{1.151557in}{1.087521in}}%
\pgfpathlineto{\pgfqpoint{1.159793in}{1.087521in}}%
\pgfpathlineto{\pgfqpoint{1.163911in}{1.101577in}}%
\pgfpathlineto{\pgfqpoint{1.188619in}{1.101577in}}%
\pgfpathlineto{\pgfqpoint{1.192737in}{1.115633in}}%
\pgfpathlineto{\pgfqpoint{1.250391in}{1.115633in}}%
\pgfpathlineto{\pgfqpoint{1.254509in}{1.129690in}}%
\pgfpathlineto{\pgfqpoint{1.394524in}{1.129690in}}%
\pgfpathlineto{\pgfqpoint{1.402760in}{1.157802in}}%
\pgfpathlineto{\pgfqpoint{1.493358in}{1.157802in}}%
\pgfpathlineto{\pgfqpoint{1.497476in}{1.171858in}}%
\pgfpathlineto{\pgfqpoint{2.012236in}{1.171858in}}%
\pgfpathlineto{\pgfqpoint{2.016354in}{1.185914in}}%
\pgfpathlineto{\pgfqpoint{2.407572in}{1.185914in}}%
\pgfpathlineto{\pgfqpoint{2.411690in}{1.199971in}}%
\pgfpathlineto{\pgfqpoint{2.592886in}{1.199971in}}%
\pgfpathlineto{\pgfqpoint{2.597004in}{1.214027in}}%
\pgfpathlineto{\pgfqpoint{2.646421in}{1.214027in}}%
\pgfpathlineto{\pgfqpoint{2.646421in}{1.214027in}}%
\pgfusepath{stroke}%
\end{pgfscope}%
\begin{pgfscope}%
\pgfpathrectangle{\pgfqpoint{0.488751in}{0.368545in}}{\pgfqpoint{2.260417in}{1.283333in}}%
\pgfusepath{clip}%
\pgfsetrectcap%
\pgfsetroundjoin%
\pgfsetlinewidth{0.803000pt}%
\definecolor{currentstroke}{rgb}{0.843137,0.666667,0.313725}%
\pgfsetstrokecolor{currentstroke}%
\pgfsetdash{}{0pt}%
\pgfpathmoveto{\pgfqpoint{0.591497in}{0.623665in}}%
\pgfpathlineto{\pgfqpoint{0.595615in}{0.623665in}}%
\pgfpathlineto{\pgfqpoint{0.599733in}{0.637722in}}%
\pgfpathlineto{\pgfqpoint{0.607969in}{0.637722in}}%
\pgfpathlineto{\pgfqpoint{0.612087in}{0.651778in}}%
\pgfpathlineto{\pgfqpoint{0.620324in}{0.651778in}}%
\pgfpathlineto{\pgfqpoint{0.628560in}{0.679890in}}%
\pgfpathlineto{\pgfqpoint{0.645032in}{0.679890in}}%
\pgfpathlineto{\pgfqpoint{0.649150in}{0.693947in}}%
\pgfpathlineto{\pgfqpoint{0.653268in}{0.693947in}}%
\pgfpathlineto{\pgfqpoint{0.661504in}{0.722059in}}%
\pgfpathlineto{\pgfqpoint{0.665623in}{0.708003in}}%
\pgfpathlineto{\pgfqpoint{0.669741in}{0.722059in}}%
\pgfpathlineto{\pgfqpoint{0.690331in}{0.722059in}}%
\pgfpathlineto{\pgfqpoint{0.694449in}{0.736115in}}%
\pgfpathlineto{\pgfqpoint{0.698567in}{0.736115in}}%
\pgfpathlineto{\pgfqpoint{0.702685in}{0.750172in}}%
\pgfpathlineto{\pgfqpoint{0.706803in}{0.750172in}}%
\pgfpathlineto{\pgfqpoint{0.710922in}{0.764228in}}%
\pgfpathlineto{\pgfqpoint{0.719158in}{0.764228in}}%
\pgfpathlineto{\pgfqpoint{0.727394in}{0.736115in}}%
\pgfpathlineto{\pgfqpoint{0.731512in}{0.736115in}}%
\pgfpathlineto{\pgfqpoint{0.735630in}{0.750172in}}%
\pgfpathlineto{\pgfqpoint{0.739748in}{0.736115in}}%
\pgfpathlineto{\pgfqpoint{0.743866in}{0.736115in}}%
\pgfpathlineto{\pgfqpoint{0.747984in}{0.750172in}}%
\pgfpathlineto{\pgfqpoint{1.015660in}{0.750172in}}%
\pgfpathlineto{\pgfqpoint{1.019778in}{0.764228in}}%
\pgfpathlineto{\pgfqpoint{1.028014in}{0.764228in}}%
\pgfpathlineto{\pgfqpoint{1.032132in}{0.750172in}}%
\pgfpathlineto{\pgfqpoint{1.312162in}{0.750172in}}%
\pgfpathlineto{\pgfqpoint{1.316280in}{0.764228in}}%
\pgfpathlineto{\pgfqpoint{1.328634in}{0.764228in}}%
\pgfpathlineto{\pgfqpoint{1.332752in}{0.778284in}}%
\pgfpathlineto{\pgfqpoint{1.336870in}{0.778284in}}%
\pgfpathlineto{\pgfqpoint{1.340988in}{0.792340in}}%
\pgfpathlineto{\pgfqpoint{1.386287in}{0.792340in}}%
\pgfpathlineto{\pgfqpoint{1.390405in}{0.806396in}}%
\pgfpathlineto{\pgfqpoint{1.563365in}{0.806396in}}%
\pgfpathlineto{\pgfqpoint{1.567483in}{0.820453in}}%
\pgfpathlineto{\pgfqpoint{1.596310in}{0.820453in}}%
\pgfpathlineto{\pgfqpoint{1.600428in}{0.834509in}}%
\pgfpathlineto{\pgfqpoint{1.806332in}{0.834509in}}%
\pgfpathlineto{\pgfqpoint{1.810450in}{0.848565in}}%
\pgfpathlineto{\pgfqpoint{1.814568in}{0.848565in}}%
\pgfpathlineto{\pgfqpoint{1.818686in}{0.862621in}}%
\pgfpathlineto{\pgfqpoint{1.863985in}{0.862621in}}%
\pgfpathlineto{\pgfqpoint{1.868103in}{0.876678in}}%
\pgfpathlineto{\pgfqpoint{2.016354in}{0.876678in}}%
\pgfpathlineto{\pgfqpoint{2.020472in}{0.890734in}}%
\pgfpathlineto{\pgfqpoint{2.078126in}{0.890734in}}%
\pgfpathlineto{\pgfqpoint{2.082244in}{0.904790in}}%
\pgfpathlineto{\pgfqpoint{2.094598in}{0.904790in}}%
\pgfpathlineto{\pgfqpoint{2.098716in}{0.918846in}}%
\pgfpathlineto{\pgfqpoint{2.115188in}{0.918846in}}%
\pgfpathlineto{\pgfqpoint{2.119306in}{0.932902in}}%
\pgfpathlineto{\pgfqpoint{2.160487in}{0.932902in}}%
\pgfpathlineto{\pgfqpoint{2.164605in}{0.946959in}}%
\pgfpathlineto{\pgfqpoint{2.341683in}{0.946959in}}%
\pgfpathlineto{\pgfqpoint{2.345801in}{0.961015in}}%
\pgfpathlineto{\pgfqpoint{2.391100in}{0.961015in}}%
\pgfpathlineto{\pgfqpoint{2.395218in}{0.989127in}}%
\pgfpathlineto{\pgfqpoint{2.403454in}{0.989127in}}%
\pgfpathlineto{\pgfqpoint{2.407572in}{1.003184in}}%
\pgfpathlineto{\pgfqpoint{2.411690in}{0.989127in}}%
\pgfpathlineto{\pgfqpoint{2.419927in}{1.017240in}}%
\pgfpathlineto{\pgfqpoint{2.432281in}{1.017240in}}%
\pgfpathlineto{\pgfqpoint{2.436399in}{1.031296in}}%
\pgfpathlineto{\pgfqpoint{2.440517in}{1.017240in}}%
\pgfpathlineto{\pgfqpoint{2.444635in}{1.031296in}}%
\pgfpathlineto{\pgfqpoint{2.448753in}{1.031296in}}%
\pgfpathlineto{\pgfqpoint{2.452871in}{1.045352in}}%
\pgfpathlineto{\pgfqpoint{2.485816in}{1.045352in}}%
\pgfpathlineto{\pgfqpoint{2.489934in}{1.059408in}}%
\pgfpathlineto{\pgfqpoint{2.547587in}{1.059408in}}%
\pgfpathlineto{\pgfqpoint{2.551705in}{1.073465in}}%
\pgfpathlineto{\pgfqpoint{2.555823in}{1.073465in}}%
\pgfpathlineto{\pgfqpoint{2.559942in}{1.087521in}}%
\pgfpathlineto{\pgfqpoint{2.568178in}{1.087521in}}%
\pgfpathlineto{\pgfqpoint{2.572296in}{1.101577in}}%
\pgfpathlineto{\pgfqpoint{2.588768in}{1.101577in}}%
\pgfpathlineto{\pgfqpoint{2.592886in}{1.087521in}}%
\pgfpathlineto{\pgfqpoint{2.617595in}{1.087521in}}%
\pgfpathlineto{\pgfqpoint{2.621713in}{1.073465in}}%
\pgfpathlineto{\pgfqpoint{2.625831in}{1.087521in}}%
\pgfpathlineto{\pgfqpoint{2.646421in}{1.087521in}}%
\pgfpathlineto{\pgfqpoint{2.646421in}{1.087521in}}%
\pgfusepath{stroke}%
\end{pgfscope}%
\begin{pgfscope}%
\pgfpathrectangle{\pgfqpoint{0.488751in}{0.368545in}}{\pgfqpoint{2.260417in}{1.283333in}}%
\pgfusepath{clip}%
\pgfsetrectcap%
\pgfsetroundjoin%
\pgfsetlinewidth{0.803000pt}%
\definecolor{currentstroke}{rgb}{0.333333,0.333333,0.333333}%
\pgfsetstrokecolor{currentstroke}%
\pgfsetstrokeopacity{0.270000}%
\pgfsetdash{}{0pt}%
\pgfpathmoveto{\pgfqpoint{0.591497in}{1.228083in}}%
\pgfpathlineto{\pgfqpoint{0.624442in}{1.228083in}}%
\pgfpathlineto{\pgfqpoint{0.628560in}{1.242139in}}%
\pgfpathlineto{\pgfqpoint{0.727394in}{1.242139in}}%
\pgfpathlineto{\pgfqpoint{0.731512in}{1.256196in}}%
\pgfpathlineto{\pgfqpoint{0.739748in}{1.256196in}}%
\pgfpathlineto{\pgfqpoint{0.743866in}{1.270252in}}%
\pgfpathlineto{\pgfqpoint{0.756220in}{1.270252in}}%
\pgfpathlineto{\pgfqpoint{0.764457in}{1.242139in}}%
\pgfpathlineto{\pgfqpoint{0.768575in}{1.242139in}}%
\pgfpathlineto{\pgfqpoint{0.772693in}{1.228083in}}%
\pgfpathlineto{\pgfqpoint{0.776811in}{1.242139in}}%
\pgfpathlineto{\pgfqpoint{0.789165in}{1.242139in}}%
\pgfpathlineto{\pgfqpoint{0.801519in}{1.199971in}}%
\pgfpathlineto{\pgfqpoint{0.805637in}{1.199971in}}%
\pgfpathlineto{\pgfqpoint{0.809756in}{1.214027in}}%
\pgfpathlineto{\pgfqpoint{0.813874in}{1.214027in}}%
\pgfpathlineto{\pgfqpoint{0.817992in}{1.199971in}}%
\pgfpathlineto{\pgfqpoint{0.826228in}{1.199971in}}%
\pgfpathlineto{\pgfqpoint{0.830346in}{1.185914in}}%
\pgfpathlineto{\pgfqpoint{0.834464in}{1.185914in}}%
\pgfpathlineto{\pgfqpoint{0.838582in}{1.171858in}}%
\pgfpathlineto{\pgfqpoint{0.846818in}{1.171858in}}%
\pgfpathlineto{\pgfqpoint{0.850936in}{1.199971in}}%
\pgfpathlineto{\pgfqpoint{0.863291in}{1.199971in}}%
\pgfpathlineto{\pgfqpoint{0.867409in}{1.185914in}}%
\pgfpathlineto{\pgfqpoint{0.875645in}{1.185914in}}%
\pgfpathlineto{\pgfqpoint{0.879763in}{1.171858in}}%
\pgfpathlineto{\pgfqpoint{0.887999in}{1.171858in}}%
\pgfpathlineto{\pgfqpoint{0.892117in}{1.157802in}}%
\pgfpathlineto{\pgfqpoint{0.896235in}{1.171858in}}%
\pgfpathlineto{\pgfqpoint{0.916826in}{1.171858in}}%
\pgfpathlineto{\pgfqpoint{0.920944in}{1.185914in}}%
\pgfpathlineto{\pgfqpoint{0.925062in}{1.171858in}}%
\pgfpathlineto{\pgfqpoint{0.962125in}{1.171858in}}%
\pgfpathlineto{\pgfqpoint{0.966243in}{1.157802in}}%
\pgfpathlineto{\pgfqpoint{0.978597in}{1.157802in}}%
\pgfpathlineto{\pgfqpoint{0.982715in}{1.171858in}}%
\pgfpathlineto{\pgfqpoint{0.986833in}{1.171858in}}%
\pgfpathlineto{\pgfqpoint{0.990951in}{1.185914in}}%
\pgfpathlineto{\pgfqpoint{0.995069in}{1.185914in}}%
\pgfpathlineto{\pgfqpoint{0.999187in}{1.199971in}}%
\pgfpathlineto{\pgfqpoint{1.028014in}{1.199971in}}%
\pgfpathlineto{\pgfqpoint{1.032132in}{1.214027in}}%
\pgfpathlineto{\pgfqpoint{1.040368in}{1.214027in}}%
\pgfpathlineto{\pgfqpoint{1.044486in}{1.199971in}}%
\pgfpathlineto{\pgfqpoint{1.048604in}{1.199971in}}%
\pgfpathlineto{\pgfqpoint{1.052723in}{1.185914in}}%
\pgfpathlineto{\pgfqpoint{1.098021in}{1.185914in}}%
\pgfpathlineto{\pgfqpoint{1.110376in}{1.143746in}}%
\pgfpathlineto{\pgfqpoint{1.122730in}{1.143746in}}%
\pgfpathlineto{\pgfqpoint{1.126848in}{1.129690in}}%
\pgfpathlineto{\pgfqpoint{1.130966in}{1.143746in}}%
\pgfpathlineto{\pgfqpoint{1.135084in}{1.143746in}}%
\pgfpathlineto{\pgfqpoint{1.139202in}{1.157802in}}%
\pgfpathlineto{\pgfqpoint{1.143320in}{1.157802in}}%
\pgfpathlineto{\pgfqpoint{1.147438in}{1.143746in}}%
\pgfpathlineto{\pgfqpoint{1.155675in}{1.143746in}}%
\pgfpathlineto{\pgfqpoint{1.159793in}{1.129690in}}%
\pgfpathlineto{\pgfqpoint{1.163911in}{1.129690in}}%
\pgfpathlineto{\pgfqpoint{1.168029in}{1.115633in}}%
\pgfpathlineto{\pgfqpoint{1.172147in}{1.115633in}}%
\pgfpathlineto{\pgfqpoint{1.176265in}{1.087521in}}%
\pgfpathlineto{\pgfqpoint{1.180383in}{1.087521in}}%
\pgfpathlineto{\pgfqpoint{1.184501in}{1.059408in}}%
\pgfpathlineto{\pgfqpoint{1.205092in}{1.059408in}}%
\pgfpathlineto{\pgfqpoint{1.209210in}{1.073465in}}%
\pgfpathlineto{\pgfqpoint{1.217446in}{1.073465in}}%
\pgfpathlineto{\pgfqpoint{1.221564in}{1.059408in}}%
\pgfpathlineto{\pgfqpoint{1.225682in}{1.059408in}}%
\pgfpathlineto{\pgfqpoint{1.229800in}{1.087521in}}%
\pgfpathlineto{\pgfqpoint{1.254509in}{1.087521in}}%
\pgfpathlineto{\pgfqpoint{1.262745in}{1.059408in}}%
\pgfpathlineto{\pgfqpoint{1.270981in}{1.059408in}}%
\pgfpathlineto{\pgfqpoint{1.275099in}{1.073465in}}%
\pgfpathlineto{\pgfqpoint{1.312162in}{1.073465in}}%
\pgfpathlineto{\pgfqpoint{1.316280in}{1.087521in}}%
\pgfpathlineto{\pgfqpoint{1.361579in}{1.087521in}}%
\pgfpathlineto{\pgfqpoint{1.365697in}{1.101577in}}%
\pgfpathlineto{\pgfqpoint{1.448059in}{1.101577in}}%
\pgfpathlineto{\pgfqpoint{1.452177in}{1.115633in}}%
\pgfpathlineto{\pgfqpoint{1.489240in}{1.115633in}}%
\pgfpathlineto{\pgfqpoint{1.493358in}{1.101577in}}%
\pgfpathlineto{\pgfqpoint{1.518066in}{1.101577in}}%
\pgfpathlineto{\pgfqpoint{1.522184in}{1.115633in}}%
\pgfpathlineto{\pgfqpoint{1.526302in}{1.115633in}}%
\pgfpathlineto{\pgfqpoint{1.530420in}{1.087521in}}%
\pgfpathlineto{\pgfqpoint{1.534538in}{1.073465in}}%
\pgfpathlineto{\pgfqpoint{1.538657in}{1.073465in}}%
\pgfpathlineto{\pgfqpoint{1.542775in}{1.059408in}}%
\pgfpathlineto{\pgfqpoint{1.546893in}{1.059408in}}%
\pgfpathlineto{\pgfqpoint{1.551011in}{1.045352in}}%
\pgfpathlineto{\pgfqpoint{1.575719in}{1.045352in}}%
\pgfpathlineto{\pgfqpoint{1.579837in}{1.059408in}}%
\pgfpathlineto{\pgfqpoint{1.633372in}{1.059408in}}%
\pgfpathlineto{\pgfqpoint{1.637491in}{1.073465in}}%
\pgfpathlineto{\pgfqpoint{1.711616in}{1.073465in}}%
\pgfpathlineto{\pgfqpoint{1.715734in}{1.087521in}}%
\pgfpathlineto{\pgfqpoint{2.646421in}{1.087521in}}%
\pgfpathlineto{\pgfqpoint{2.646421in}{1.087521in}}%
\pgfusepath{stroke}%
\end{pgfscope}%
\begin{pgfscope}%
\pgfpathrectangle{\pgfqpoint{0.488751in}{0.368545in}}{\pgfqpoint{2.260417in}{1.283333in}}%
\pgfusepath{clip}%
\pgfsetrectcap%
\pgfsetroundjoin%
\pgfsetlinewidth{0.803000pt}%
\definecolor{currentstroke}{rgb}{0.686275,0.352941,0.313725}%
\pgfsetstrokecolor{currentstroke}%
\pgfsetstrokeopacity{0.270000}%
\pgfsetdash{}{0pt}%
\pgfpathmoveto{\pgfqpoint{0.591497in}{1.115633in}}%
\pgfpathlineto{\pgfqpoint{0.599733in}{1.171858in}}%
\pgfpathlineto{\pgfqpoint{0.607969in}{1.199971in}}%
\pgfpathlineto{\pgfqpoint{0.612087in}{1.199971in}}%
\pgfpathlineto{\pgfqpoint{0.616206in}{1.214027in}}%
\pgfpathlineto{\pgfqpoint{0.640914in}{1.214027in}}%
\pgfpathlineto{\pgfqpoint{0.645032in}{1.228083in}}%
\pgfpathlineto{\pgfqpoint{0.941534in}{1.228083in}}%
\pgfpathlineto{\pgfqpoint{0.945652in}{1.242139in}}%
\pgfpathlineto{\pgfqpoint{1.151557in}{1.242139in}}%
\pgfpathlineto{\pgfqpoint{1.155675in}{1.256196in}}%
\pgfpathlineto{\pgfqpoint{1.448059in}{1.256196in}}%
\pgfpathlineto{\pgfqpoint{1.452177in}{1.270252in}}%
\pgfpathlineto{\pgfqpoint{1.781624in}{1.270252in}}%
\pgfpathlineto{\pgfqpoint{1.785742in}{1.284308in}}%
\pgfpathlineto{\pgfqpoint{2.148133in}{1.284308in}}%
\pgfpathlineto{\pgfqpoint{2.152251in}{1.256196in}}%
\pgfpathlineto{\pgfqpoint{2.308738in}{1.256196in}}%
\pgfpathlineto{\pgfqpoint{2.312856in}{1.270252in}}%
\pgfpathlineto{\pgfqpoint{2.349919in}{1.270252in}}%
\pgfpathlineto{\pgfqpoint{2.354037in}{1.284308in}}%
\pgfpathlineto{\pgfqpoint{2.646421in}{1.284308in}}%
\pgfpathlineto{\pgfqpoint{2.646421in}{1.284308in}}%
\pgfusepath{stroke}%
\end{pgfscope}%
\begin{pgfscope}%
\pgfpathrectangle{\pgfqpoint{0.488751in}{0.368545in}}{\pgfqpoint{2.260417in}{1.283333in}}%
\pgfusepath{clip}%
\pgfsetrectcap%
\pgfsetroundjoin%
\pgfsetlinewidth{0.803000pt}%
\definecolor{currentstroke}{rgb}{0.000000,0.356863,0.509804}%
\pgfsetstrokecolor{currentstroke}%
\pgfsetstrokeopacity{0.270000}%
\pgfsetdash{}{0pt}%
\pgfpathmoveto{\pgfqpoint{0.591497in}{0.932902in}}%
\pgfpathlineto{\pgfqpoint{0.595615in}{0.961015in}}%
\pgfpathlineto{\pgfqpoint{0.599733in}{0.975071in}}%
\pgfpathlineto{\pgfqpoint{0.607969in}{0.975071in}}%
\pgfpathlineto{\pgfqpoint{0.612087in}{1.017240in}}%
\pgfpathlineto{\pgfqpoint{0.624442in}{1.017240in}}%
\pgfpathlineto{\pgfqpoint{0.628560in}{1.003184in}}%
\pgfpathlineto{\pgfqpoint{0.632678in}{1.017240in}}%
\pgfpathlineto{\pgfqpoint{0.640914in}{1.073465in}}%
\pgfpathlineto{\pgfqpoint{0.645032in}{1.059408in}}%
\pgfpathlineto{\pgfqpoint{0.649150in}{1.059408in}}%
\pgfpathlineto{\pgfqpoint{0.653268in}{1.073465in}}%
\pgfpathlineto{\pgfqpoint{0.657386in}{1.101577in}}%
\pgfpathlineto{\pgfqpoint{0.661504in}{1.115633in}}%
\pgfpathlineto{\pgfqpoint{0.665623in}{1.087521in}}%
\pgfpathlineto{\pgfqpoint{0.673859in}{1.115633in}}%
\pgfpathlineto{\pgfqpoint{0.677977in}{1.115633in}}%
\pgfpathlineto{\pgfqpoint{0.686213in}{1.171858in}}%
\pgfpathlineto{\pgfqpoint{0.690331in}{1.171858in}}%
\pgfpathlineto{\pgfqpoint{0.694449in}{1.185914in}}%
\pgfpathlineto{\pgfqpoint{0.702685in}{1.185914in}}%
\pgfpathlineto{\pgfqpoint{0.710922in}{1.214027in}}%
\pgfpathlineto{\pgfqpoint{0.715040in}{1.214027in}}%
\pgfpathlineto{\pgfqpoint{0.719158in}{1.199971in}}%
\pgfpathlineto{\pgfqpoint{0.723276in}{1.199971in}}%
\pgfpathlineto{\pgfqpoint{0.727394in}{1.185914in}}%
\pgfpathlineto{\pgfqpoint{0.731512in}{1.185914in}}%
\pgfpathlineto{\pgfqpoint{0.735630in}{1.171858in}}%
\pgfpathlineto{\pgfqpoint{0.743866in}{1.171858in}}%
\pgfpathlineto{\pgfqpoint{0.752102in}{1.143746in}}%
\pgfpathlineto{\pgfqpoint{0.756220in}{1.143746in}}%
\pgfpathlineto{\pgfqpoint{0.760339in}{1.129690in}}%
\pgfpathlineto{\pgfqpoint{0.772693in}{1.129690in}}%
\pgfpathlineto{\pgfqpoint{0.780929in}{1.157802in}}%
\pgfpathlineto{\pgfqpoint{0.797401in}{1.157802in}}%
\pgfpathlineto{\pgfqpoint{0.801519in}{1.143746in}}%
\pgfpathlineto{\pgfqpoint{0.805637in}{1.143746in}}%
\pgfpathlineto{\pgfqpoint{0.809756in}{1.157802in}}%
\pgfpathlineto{\pgfqpoint{0.817992in}{1.157802in}}%
\pgfpathlineto{\pgfqpoint{0.822110in}{1.143746in}}%
\pgfpathlineto{\pgfqpoint{0.830346in}{1.143746in}}%
\pgfpathlineto{\pgfqpoint{0.838582in}{1.115633in}}%
\pgfpathlineto{\pgfqpoint{0.850936in}{1.115633in}}%
\pgfpathlineto{\pgfqpoint{0.859173in}{1.087521in}}%
\pgfpathlineto{\pgfqpoint{0.867409in}{1.087521in}}%
\pgfpathlineto{\pgfqpoint{0.871527in}{1.101577in}}%
\pgfpathlineto{\pgfqpoint{0.875645in}{1.101577in}}%
\pgfpathlineto{\pgfqpoint{0.879763in}{1.115633in}}%
\pgfpathlineto{\pgfqpoint{0.887999in}{1.087521in}}%
\pgfpathlineto{\pgfqpoint{0.892117in}{1.101577in}}%
\pgfpathlineto{\pgfqpoint{0.896235in}{1.101577in}}%
\pgfpathlineto{\pgfqpoint{0.900353in}{1.115633in}}%
\pgfpathlineto{\pgfqpoint{0.904471in}{1.101577in}}%
\pgfpathlineto{\pgfqpoint{0.908590in}{1.101577in}}%
\pgfpathlineto{\pgfqpoint{0.912708in}{1.087521in}}%
\pgfpathlineto{\pgfqpoint{0.925062in}{1.087521in}}%
\pgfpathlineto{\pgfqpoint{0.929180in}{1.073465in}}%
\pgfpathlineto{\pgfqpoint{0.941534in}{1.073465in}}%
\pgfpathlineto{\pgfqpoint{0.945652in}{1.059408in}}%
\pgfpathlineto{\pgfqpoint{0.962125in}{1.059408in}}%
\pgfpathlineto{\pgfqpoint{0.966243in}{1.045352in}}%
\pgfpathlineto{\pgfqpoint{0.978597in}{1.045352in}}%
\pgfpathlineto{\pgfqpoint{0.982715in}{1.031296in}}%
\pgfpathlineto{\pgfqpoint{0.990951in}{1.031296in}}%
\pgfpathlineto{\pgfqpoint{0.995069in}{1.045352in}}%
\pgfpathlineto{\pgfqpoint{0.999187in}{1.045352in}}%
\pgfpathlineto{\pgfqpoint{1.003306in}{1.031296in}}%
\pgfpathlineto{\pgfqpoint{1.036250in}{1.031296in}}%
\pgfpathlineto{\pgfqpoint{1.040368in}{1.017240in}}%
\pgfpathlineto{\pgfqpoint{1.044486in}{1.017240in}}%
\pgfpathlineto{\pgfqpoint{1.048604in}{1.031296in}}%
\pgfpathlineto{\pgfqpoint{1.052723in}{1.017240in}}%
\pgfpathlineto{\pgfqpoint{1.056841in}{1.017240in}}%
\pgfpathlineto{\pgfqpoint{1.060959in}{1.031296in}}%
\pgfpathlineto{\pgfqpoint{1.073313in}{1.031296in}}%
\pgfpathlineto{\pgfqpoint{1.077431in}{1.045352in}}%
\pgfpathlineto{\pgfqpoint{1.081549in}{1.031296in}}%
\pgfpathlineto{\pgfqpoint{1.085667in}{1.031296in}}%
\pgfpathlineto{\pgfqpoint{1.093903in}{1.003184in}}%
\pgfpathlineto{\pgfqpoint{1.098021in}{1.017240in}}%
\pgfpathlineto{\pgfqpoint{1.114494in}{1.017240in}}%
\pgfpathlineto{\pgfqpoint{1.118612in}{1.045352in}}%
\pgfpathlineto{\pgfqpoint{1.122730in}{1.059408in}}%
\pgfpathlineto{\pgfqpoint{1.126848in}{1.045352in}}%
\pgfpathlineto{\pgfqpoint{1.130966in}{1.059408in}}%
\pgfpathlineto{\pgfqpoint{1.135084in}{1.045352in}}%
\pgfpathlineto{\pgfqpoint{1.139202in}{1.059408in}}%
\pgfpathlineto{\pgfqpoint{1.143320in}{1.059408in}}%
\pgfpathlineto{\pgfqpoint{1.147438in}{1.045352in}}%
\pgfpathlineto{\pgfqpoint{1.159793in}{1.045352in}}%
\pgfpathlineto{\pgfqpoint{1.163911in}{1.031296in}}%
\pgfpathlineto{\pgfqpoint{1.176265in}{1.031296in}}%
\pgfpathlineto{\pgfqpoint{1.188619in}{1.073465in}}%
\pgfpathlineto{\pgfqpoint{1.205092in}{1.073465in}}%
\pgfpathlineto{\pgfqpoint{1.209210in}{1.059408in}}%
\pgfpathlineto{\pgfqpoint{1.217446in}{1.059408in}}%
\pgfpathlineto{\pgfqpoint{1.225682in}{1.031296in}}%
\pgfpathlineto{\pgfqpoint{1.229800in}{1.031296in}}%
\pgfpathlineto{\pgfqpoint{1.242154in}{0.989127in}}%
\pgfpathlineto{\pgfqpoint{1.246273in}{0.989127in}}%
\pgfpathlineto{\pgfqpoint{1.254509in}{1.017240in}}%
\pgfpathlineto{\pgfqpoint{1.258627in}{0.989127in}}%
\pgfpathlineto{\pgfqpoint{1.262745in}{0.975071in}}%
\pgfpathlineto{\pgfqpoint{1.266863in}{0.975071in}}%
\pgfpathlineto{\pgfqpoint{1.275099in}{1.031296in}}%
\pgfpathlineto{\pgfqpoint{1.291571in}{1.031296in}}%
\pgfpathlineto{\pgfqpoint{1.295690in}{1.017240in}}%
\pgfpathlineto{\pgfqpoint{1.308044in}{1.017240in}}%
\pgfpathlineto{\pgfqpoint{1.312162in}{1.031296in}}%
\pgfpathlineto{\pgfqpoint{1.328634in}{1.031296in}}%
\pgfpathlineto{\pgfqpoint{1.332752in}{1.017240in}}%
\pgfpathlineto{\pgfqpoint{1.361579in}{1.017240in}}%
\pgfpathlineto{\pgfqpoint{1.365697in}{1.003184in}}%
\pgfpathlineto{\pgfqpoint{1.369815in}{1.017240in}}%
\pgfpathlineto{\pgfqpoint{1.382169in}{1.017240in}}%
\pgfpathlineto{\pgfqpoint{1.390405in}{1.045352in}}%
\pgfpathlineto{\pgfqpoint{1.402760in}{1.045352in}}%
\pgfpathlineto{\pgfqpoint{1.406878in}{1.017240in}}%
\pgfpathlineto{\pgfqpoint{1.419232in}{1.017240in}}%
\pgfpathlineto{\pgfqpoint{1.423350in}{1.031296in}}%
\pgfpathlineto{\pgfqpoint{1.435704in}{1.031296in}}%
\pgfpathlineto{\pgfqpoint{1.443941in}{1.003184in}}%
\pgfpathlineto{\pgfqpoint{1.456295in}{1.003184in}}%
\pgfpathlineto{\pgfqpoint{1.460413in}{0.989127in}}%
\pgfpathlineto{\pgfqpoint{1.468649in}{0.989127in}}%
\pgfpathlineto{\pgfqpoint{1.476885in}{1.017240in}}%
\pgfpathlineto{\pgfqpoint{1.518066in}{1.017240in}}%
\pgfpathlineto{\pgfqpoint{1.530420in}{0.975071in}}%
\pgfpathlineto{\pgfqpoint{1.546893in}{0.975071in}}%
\pgfpathlineto{\pgfqpoint{1.551011in}{0.989127in}}%
\pgfpathlineto{\pgfqpoint{1.555129in}{0.975071in}}%
\pgfpathlineto{\pgfqpoint{1.563365in}{0.975071in}}%
\pgfpathlineto{\pgfqpoint{1.567483in}{0.989127in}}%
\pgfpathlineto{\pgfqpoint{1.571601in}{0.961015in}}%
\pgfpathlineto{\pgfqpoint{1.579837in}{0.961015in}}%
\pgfpathlineto{\pgfqpoint{1.583955in}{0.975071in}}%
\pgfpathlineto{\pgfqpoint{1.588074in}{0.975071in}}%
\pgfpathlineto{\pgfqpoint{1.604546in}{1.031296in}}%
\pgfpathlineto{\pgfqpoint{1.616900in}{1.031296in}}%
\pgfpathlineto{\pgfqpoint{1.621018in}{1.017240in}}%
\pgfpathlineto{\pgfqpoint{1.637491in}{1.017240in}}%
\pgfpathlineto{\pgfqpoint{1.641609in}{1.003184in}}%
\pgfpathlineto{\pgfqpoint{1.658081in}{1.003184in}}%
\pgfpathlineto{\pgfqpoint{1.662199in}{0.989127in}}%
\pgfpathlineto{\pgfqpoint{1.666317in}{1.003184in}}%
\pgfpathlineto{\pgfqpoint{1.670435in}{1.003184in}}%
\pgfpathlineto{\pgfqpoint{1.674553in}{1.017240in}}%
\pgfpathlineto{\pgfqpoint{1.678671in}{1.003184in}}%
\pgfpathlineto{\pgfqpoint{1.695144in}{1.003184in}}%
\pgfpathlineto{\pgfqpoint{1.699262in}{0.989127in}}%
\pgfpathlineto{\pgfqpoint{1.703380in}{0.989127in}}%
\pgfpathlineto{\pgfqpoint{1.707498in}{0.975071in}}%
\pgfpathlineto{\pgfqpoint{1.711616in}{0.989127in}}%
\pgfpathlineto{\pgfqpoint{1.723970in}{0.946959in}}%
\pgfpathlineto{\pgfqpoint{1.728088in}{0.961015in}}%
\pgfpathlineto{\pgfqpoint{1.736325in}{0.961015in}}%
\pgfpathlineto{\pgfqpoint{1.740443in}{0.946959in}}%
\pgfpathlineto{\pgfqpoint{1.748679in}{0.946959in}}%
\pgfpathlineto{\pgfqpoint{1.752797in}{0.932902in}}%
\pgfpathlineto{\pgfqpoint{1.756915in}{0.932902in}}%
\pgfpathlineto{\pgfqpoint{1.761033in}{0.946959in}}%
\pgfpathlineto{\pgfqpoint{1.765151in}{0.975071in}}%
\pgfpathlineto{\pgfqpoint{1.769269in}{0.975071in}}%
\pgfpathlineto{\pgfqpoint{1.773387in}{0.961015in}}%
\pgfpathlineto{\pgfqpoint{1.777505in}{0.961015in}}%
\pgfpathlineto{\pgfqpoint{1.781624in}{0.946959in}}%
\pgfpathlineto{\pgfqpoint{1.785742in}{0.961015in}}%
\pgfpathlineto{\pgfqpoint{1.789860in}{0.961015in}}%
\pgfpathlineto{\pgfqpoint{1.793978in}{0.946959in}}%
\pgfpathlineto{\pgfqpoint{1.814568in}{0.946959in}}%
\pgfpathlineto{\pgfqpoint{1.818686in}{0.918846in}}%
\pgfpathlineto{\pgfqpoint{1.822804in}{0.932902in}}%
\pgfpathlineto{\pgfqpoint{1.826922in}{0.932902in}}%
\pgfpathlineto{\pgfqpoint{1.831041in}{0.946959in}}%
\pgfpathlineto{\pgfqpoint{1.863985in}{0.946959in}}%
\pgfpathlineto{\pgfqpoint{1.868103in}{0.961015in}}%
\pgfpathlineto{\pgfqpoint{1.872221in}{0.946959in}}%
\pgfpathlineto{\pgfqpoint{1.896930in}{0.946959in}}%
\pgfpathlineto{\pgfqpoint{1.901048in}{0.961015in}}%
\pgfpathlineto{\pgfqpoint{1.938111in}{0.961015in}}%
\pgfpathlineto{\pgfqpoint{1.946347in}{0.989127in}}%
\pgfpathlineto{\pgfqpoint{1.958701in}{0.989127in}}%
\pgfpathlineto{\pgfqpoint{1.962819in}{1.003184in}}%
\pgfpathlineto{\pgfqpoint{1.971055in}{1.003184in}}%
\pgfpathlineto{\pgfqpoint{1.979292in}{0.975071in}}%
\pgfpathlineto{\pgfqpoint{1.983410in}{0.975071in}}%
\pgfpathlineto{\pgfqpoint{1.987528in}{0.961015in}}%
\pgfpathlineto{\pgfqpoint{2.061653in}{0.961015in}}%
\pgfpathlineto{\pgfqpoint{2.065771in}{0.975071in}}%
\pgfpathlineto{\pgfqpoint{2.069889in}{0.975071in}}%
\pgfpathlineto{\pgfqpoint{2.074008in}{0.961015in}}%
\pgfpathlineto{\pgfqpoint{2.082244in}{0.961015in}}%
\pgfpathlineto{\pgfqpoint{2.090480in}{0.932902in}}%
\pgfpathlineto{\pgfqpoint{2.094598in}{0.946959in}}%
\pgfpathlineto{\pgfqpoint{2.106952in}{0.946959in}}%
\pgfpathlineto{\pgfqpoint{2.111070in}{0.961015in}}%
\pgfpathlineto{\pgfqpoint{2.115188in}{0.961015in}}%
\pgfpathlineto{\pgfqpoint{2.119306in}{0.946959in}}%
\pgfpathlineto{\pgfqpoint{2.172842in}{0.946959in}}%
\pgfpathlineto{\pgfqpoint{2.176960in}{0.932902in}}%
\pgfpathlineto{\pgfqpoint{2.185196in}{0.932902in}}%
\pgfpathlineto{\pgfqpoint{2.189314in}{0.946959in}}%
\pgfpathlineto{\pgfqpoint{2.205786in}{0.946959in}}%
\pgfpathlineto{\pgfqpoint{2.209904in}{0.932902in}}%
\pgfpathlineto{\pgfqpoint{2.214022in}{0.932902in}}%
\pgfpathlineto{\pgfqpoint{2.218140in}{0.946959in}}%
\pgfpathlineto{\pgfqpoint{2.222259in}{0.918846in}}%
\pgfpathlineto{\pgfqpoint{2.226377in}{0.918846in}}%
\pgfpathlineto{\pgfqpoint{2.230495in}{0.932902in}}%
\pgfpathlineto{\pgfqpoint{2.242849in}{0.932902in}}%
\pgfpathlineto{\pgfqpoint{2.246967in}{0.946959in}}%
\pgfpathlineto{\pgfqpoint{2.312856in}{0.946959in}}%
\pgfpathlineto{\pgfqpoint{2.316975in}{0.961015in}}%
\pgfpathlineto{\pgfqpoint{2.329329in}{0.961015in}}%
\pgfpathlineto{\pgfqpoint{2.333447in}{0.946959in}}%
\pgfpathlineto{\pgfqpoint{2.349919in}{0.946959in}}%
\pgfpathlineto{\pgfqpoint{2.354037in}{0.932902in}}%
\pgfpathlineto{\pgfqpoint{2.370510in}{0.932902in}}%
\pgfpathlineto{\pgfqpoint{2.374628in}{0.946959in}}%
\pgfpathlineto{\pgfqpoint{2.395218in}{0.946959in}}%
\pgfpathlineto{\pgfqpoint{2.399336in}{0.961015in}}%
\pgfpathlineto{\pgfqpoint{2.403454in}{0.946959in}}%
\pgfpathlineto{\pgfqpoint{2.465226in}{0.946959in}}%
\pgfpathlineto{\pgfqpoint{2.473462in}{0.975071in}}%
\pgfpathlineto{\pgfqpoint{2.526997in}{0.975071in}}%
\pgfpathlineto{\pgfqpoint{2.535233in}{1.003184in}}%
\pgfpathlineto{\pgfqpoint{2.539351in}{1.003184in}}%
\pgfpathlineto{\pgfqpoint{2.543469in}{0.989127in}}%
\pgfpathlineto{\pgfqpoint{2.559942in}{0.989127in}}%
\pgfpathlineto{\pgfqpoint{2.564060in}{1.003184in}}%
\pgfpathlineto{\pgfqpoint{2.568178in}{0.989127in}}%
\pgfpathlineto{\pgfqpoint{2.572296in}{1.003184in}}%
\pgfpathlineto{\pgfqpoint{2.609359in}{1.003184in}}%
\pgfpathlineto{\pgfqpoint{2.613477in}{1.017240in}}%
\pgfpathlineto{\pgfqpoint{2.646421in}{1.017240in}}%
\pgfpathlineto{\pgfqpoint{2.646421in}{1.017240in}}%
\pgfusepath{stroke}%
\end{pgfscope}%
\begin{pgfscope}%
\pgfpathrectangle{\pgfqpoint{0.488751in}{0.368545in}}{\pgfqpoint{2.260417in}{1.283333in}}%
\pgfusepath{clip}%
\pgfsetrectcap%
\pgfsetroundjoin%
\pgfsetlinewidth{0.803000pt}%
\definecolor{currentstroke}{rgb}{0.490196,0.588235,0.431373}%
\pgfsetstrokecolor{currentstroke}%
\pgfsetstrokeopacity{0.270000}%
\pgfsetdash{}{0pt}%
\pgfpathmoveto{\pgfqpoint{0.591497in}{0.609609in}}%
\pgfpathlineto{\pgfqpoint{0.599733in}{0.637722in}}%
\pgfpathlineto{\pgfqpoint{0.603851in}{0.637722in}}%
\pgfpathlineto{\pgfqpoint{0.612087in}{0.665834in}}%
\pgfpathlineto{\pgfqpoint{0.616206in}{0.651778in}}%
\pgfpathlineto{\pgfqpoint{0.624442in}{0.651778in}}%
\pgfpathlineto{\pgfqpoint{0.628560in}{0.665834in}}%
\pgfpathlineto{\pgfqpoint{0.632678in}{0.693947in}}%
\pgfpathlineto{\pgfqpoint{0.636796in}{0.693947in}}%
\pgfpathlineto{\pgfqpoint{0.645032in}{0.722059in}}%
\pgfpathlineto{\pgfqpoint{0.649150in}{0.750172in}}%
\pgfpathlineto{\pgfqpoint{0.657386in}{0.750172in}}%
\pgfpathlineto{\pgfqpoint{0.661504in}{0.764228in}}%
\pgfpathlineto{\pgfqpoint{0.665623in}{0.764228in}}%
\pgfpathlineto{\pgfqpoint{0.669741in}{0.778284in}}%
\pgfpathlineto{\pgfqpoint{0.682095in}{0.778284in}}%
\pgfpathlineto{\pgfqpoint{0.686213in}{0.792340in}}%
\pgfpathlineto{\pgfqpoint{0.698567in}{0.792340in}}%
\pgfpathlineto{\pgfqpoint{0.702685in}{0.806396in}}%
\pgfpathlineto{\pgfqpoint{0.710922in}{0.806396in}}%
\pgfpathlineto{\pgfqpoint{0.715040in}{0.820453in}}%
\pgfpathlineto{\pgfqpoint{0.735630in}{0.820453in}}%
\pgfpathlineto{\pgfqpoint{0.739748in}{0.834509in}}%
\pgfpathlineto{\pgfqpoint{0.772693in}{0.834509in}}%
\pgfpathlineto{\pgfqpoint{0.776811in}{0.806396in}}%
\pgfpathlineto{\pgfqpoint{0.785047in}{0.806396in}}%
\pgfpathlineto{\pgfqpoint{0.789165in}{0.792340in}}%
\pgfpathlineto{\pgfqpoint{0.793283in}{0.792340in}}%
\pgfpathlineto{\pgfqpoint{0.797401in}{0.778284in}}%
\pgfpathlineto{\pgfqpoint{0.801519in}{0.792340in}}%
\pgfpathlineto{\pgfqpoint{0.805637in}{0.792340in}}%
\pgfpathlineto{\pgfqpoint{0.809756in}{0.806396in}}%
\pgfpathlineto{\pgfqpoint{0.817992in}{0.806396in}}%
\pgfpathlineto{\pgfqpoint{0.822110in}{0.792340in}}%
\pgfpathlineto{\pgfqpoint{0.842700in}{0.792340in}}%
\pgfpathlineto{\pgfqpoint{0.846818in}{0.806396in}}%
\pgfpathlineto{\pgfqpoint{0.896235in}{0.806396in}}%
\pgfpathlineto{\pgfqpoint{0.900353in}{0.820453in}}%
\pgfpathlineto{\pgfqpoint{0.908590in}{0.820453in}}%
\pgfpathlineto{\pgfqpoint{0.912708in}{0.806396in}}%
\pgfpathlineto{\pgfqpoint{0.916826in}{0.820453in}}%
\pgfpathlineto{\pgfqpoint{0.933298in}{0.820453in}}%
\pgfpathlineto{\pgfqpoint{0.937416in}{0.834509in}}%
\pgfpathlineto{\pgfqpoint{0.945652in}{0.834509in}}%
\pgfpathlineto{\pgfqpoint{0.949770in}{0.820453in}}%
\pgfpathlineto{\pgfqpoint{0.953888in}{0.834509in}}%
\pgfpathlineto{\pgfqpoint{0.962125in}{0.834509in}}%
\pgfpathlineto{\pgfqpoint{0.966243in}{0.862621in}}%
\pgfpathlineto{\pgfqpoint{0.970361in}{0.848565in}}%
\pgfpathlineto{\pgfqpoint{0.974479in}{0.862621in}}%
\pgfpathlineto{\pgfqpoint{0.978597in}{0.848565in}}%
\pgfpathlineto{\pgfqpoint{0.999187in}{0.848565in}}%
\pgfpathlineto{\pgfqpoint{1.003306in}{0.834509in}}%
\pgfpathlineto{\pgfqpoint{1.007424in}{0.834509in}}%
\pgfpathlineto{\pgfqpoint{1.015660in}{0.862621in}}%
\pgfpathlineto{\pgfqpoint{1.019778in}{0.862621in}}%
\pgfpathlineto{\pgfqpoint{1.023896in}{0.876678in}}%
\pgfpathlineto{\pgfqpoint{1.032132in}{0.876678in}}%
\pgfpathlineto{\pgfqpoint{1.036250in}{0.862621in}}%
\pgfpathlineto{\pgfqpoint{1.044486in}{0.862621in}}%
\pgfpathlineto{\pgfqpoint{1.048604in}{0.876678in}}%
\pgfpathlineto{\pgfqpoint{1.052723in}{0.862621in}}%
\pgfpathlineto{\pgfqpoint{1.060959in}{0.862621in}}%
\pgfpathlineto{\pgfqpoint{1.065077in}{0.848565in}}%
\pgfpathlineto{\pgfqpoint{1.069195in}{0.820453in}}%
\pgfpathlineto{\pgfqpoint{1.073313in}{0.820453in}}%
\pgfpathlineto{\pgfqpoint{1.077431in}{0.834509in}}%
\pgfpathlineto{\pgfqpoint{1.081549in}{0.820453in}}%
\pgfpathlineto{\pgfqpoint{1.089785in}{0.820453in}}%
\pgfpathlineto{\pgfqpoint{1.093903in}{0.806396in}}%
\pgfpathlineto{\pgfqpoint{1.098021in}{0.820453in}}%
\pgfpathlineto{\pgfqpoint{1.110376in}{0.820453in}}%
\pgfpathlineto{\pgfqpoint{1.114494in}{0.834509in}}%
\pgfpathlineto{\pgfqpoint{1.126848in}{0.834509in}}%
\pgfpathlineto{\pgfqpoint{1.130966in}{0.848565in}}%
\pgfpathlineto{\pgfqpoint{1.135084in}{0.848565in}}%
\pgfpathlineto{\pgfqpoint{1.139202in}{0.876678in}}%
\pgfpathlineto{\pgfqpoint{2.646421in}{0.876678in}}%
\pgfpathlineto{\pgfqpoint{2.646421in}{0.876678in}}%
\pgfusepath{stroke}%
\end{pgfscope}%
\begin{pgfscope}%
\pgfpathrectangle{\pgfqpoint{0.488751in}{0.368545in}}{\pgfqpoint{2.260417in}{1.283333in}}%
\pgfusepath{clip}%
\pgfsetrectcap%
\pgfsetroundjoin%
\pgfsetlinewidth{0.803000pt}%
\definecolor{currentstroke}{rgb}{0.843137,0.666667,0.313725}%
\pgfsetstrokecolor{currentstroke}%
\pgfsetstrokeopacity{0.270000}%
\pgfsetdash{}{0pt}%
\pgfpathmoveto{\pgfqpoint{0.591497in}{0.820453in}}%
\pgfpathlineto{\pgfqpoint{0.599733in}{0.848565in}}%
\pgfpathlineto{\pgfqpoint{0.603851in}{0.848565in}}%
\pgfpathlineto{\pgfqpoint{0.612087in}{0.876678in}}%
\pgfpathlineto{\pgfqpoint{0.620324in}{0.876678in}}%
\pgfpathlineto{\pgfqpoint{0.624442in}{0.890734in}}%
\pgfpathlineto{\pgfqpoint{0.632678in}{0.890734in}}%
\pgfpathlineto{\pgfqpoint{0.636796in}{0.904790in}}%
\pgfpathlineto{\pgfqpoint{0.653268in}{0.904790in}}%
\pgfpathlineto{\pgfqpoint{0.657386in}{0.918846in}}%
\pgfpathlineto{\pgfqpoint{0.661504in}{0.918846in}}%
\pgfpathlineto{\pgfqpoint{0.665623in}{0.904790in}}%
\pgfpathlineto{\pgfqpoint{0.690331in}{0.904790in}}%
\pgfpathlineto{\pgfqpoint{0.694449in}{0.918846in}}%
\pgfpathlineto{\pgfqpoint{0.698567in}{0.918846in}}%
\pgfpathlineto{\pgfqpoint{0.702685in}{0.946959in}}%
\pgfpathlineto{\pgfqpoint{0.706803in}{0.946959in}}%
\pgfpathlineto{\pgfqpoint{0.710922in}{0.932902in}}%
\pgfpathlineto{\pgfqpoint{0.715040in}{0.932902in}}%
\pgfpathlineto{\pgfqpoint{0.719158in}{0.946959in}}%
\pgfpathlineto{\pgfqpoint{0.723276in}{0.932902in}}%
\pgfpathlineto{\pgfqpoint{0.731512in}{0.932902in}}%
\pgfpathlineto{\pgfqpoint{0.735630in}{0.918846in}}%
\pgfpathlineto{\pgfqpoint{0.743866in}{0.918846in}}%
\pgfpathlineto{\pgfqpoint{0.747984in}{0.932902in}}%
\pgfpathlineto{\pgfqpoint{0.752102in}{0.918846in}}%
\pgfpathlineto{\pgfqpoint{0.764457in}{0.918846in}}%
\pgfpathlineto{\pgfqpoint{0.768575in}{0.904790in}}%
\pgfpathlineto{\pgfqpoint{0.785047in}{0.904790in}}%
\pgfpathlineto{\pgfqpoint{0.789165in}{0.890734in}}%
\pgfpathlineto{\pgfqpoint{0.797401in}{0.918846in}}%
\pgfpathlineto{\pgfqpoint{0.805637in}{0.890734in}}%
\pgfpathlineto{\pgfqpoint{0.826228in}{0.890734in}}%
\pgfpathlineto{\pgfqpoint{0.830346in}{0.918846in}}%
\pgfpathlineto{\pgfqpoint{0.834464in}{0.918846in}}%
\pgfpathlineto{\pgfqpoint{0.838582in}{0.932902in}}%
\pgfpathlineto{\pgfqpoint{0.855054in}{0.932902in}}%
\pgfpathlineto{\pgfqpoint{0.859173in}{0.918846in}}%
\pgfpathlineto{\pgfqpoint{0.879763in}{0.918846in}}%
\pgfpathlineto{\pgfqpoint{0.883881in}{0.932902in}}%
\pgfpathlineto{\pgfqpoint{0.892117in}{0.904790in}}%
\pgfpathlineto{\pgfqpoint{0.896235in}{0.918846in}}%
\pgfpathlineto{\pgfqpoint{0.908590in}{0.918846in}}%
\pgfpathlineto{\pgfqpoint{0.912708in}{0.904790in}}%
\pgfpathlineto{\pgfqpoint{0.916826in}{0.904790in}}%
\pgfpathlineto{\pgfqpoint{0.920944in}{0.918846in}}%
\pgfpathlineto{\pgfqpoint{0.929180in}{0.890734in}}%
\pgfpathlineto{\pgfqpoint{0.937416in}{0.918846in}}%
\pgfpathlineto{\pgfqpoint{0.949770in}{0.918846in}}%
\pgfpathlineto{\pgfqpoint{0.958007in}{0.946959in}}%
\pgfpathlineto{\pgfqpoint{0.962125in}{0.946959in}}%
\pgfpathlineto{\pgfqpoint{0.966243in}{0.961015in}}%
\pgfpathlineto{\pgfqpoint{0.970361in}{0.961015in}}%
\pgfpathlineto{\pgfqpoint{0.974479in}{0.946959in}}%
\pgfpathlineto{\pgfqpoint{0.978597in}{0.946959in}}%
\pgfpathlineto{\pgfqpoint{0.982715in}{0.975071in}}%
\pgfpathlineto{\pgfqpoint{0.986833in}{0.961015in}}%
\pgfpathlineto{\pgfqpoint{0.990951in}{0.961015in}}%
\pgfpathlineto{\pgfqpoint{0.995069in}{0.946959in}}%
\pgfpathlineto{\pgfqpoint{0.999187in}{0.946959in}}%
\pgfpathlineto{\pgfqpoint{1.003306in}{0.918846in}}%
\pgfpathlineto{\pgfqpoint{1.011542in}{0.918846in}}%
\pgfpathlineto{\pgfqpoint{1.015660in}{0.932902in}}%
\pgfpathlineto{\pgfqpoint{1.032132in}{0.932902in}}%
\pgfpathlineto{\pgfqpoint{1.036250in}{0.946959in}}%
\pgfpathlineto{\pgfqpoint{1.040368in}{0.946959in}}%
\pgfpathlineto{\pgfqpoint{1.048604in}{0.918846in}}%
\pgfpathlineto{\pgfqpoint{1.056841in}{0.918846in}}%
\pgfpathlineto{\pgfqpoint{1.065077in}{0.946959in}}%
\pgfpathlineto{\pgfqpoint{1.081549in}{0.946959in}}%
\pgfpathlineto{\pgfqpoint{1.085667in}{0.961015in}}%
\pgfpathlineto{\pgfqpoint{1.089785in}{0.961015in}}%
\pgfpathlineto{\pgfqpoint{1.098021in}{0.989127in}}%
\pgfpathlineto{\pgfqpoint{1.102140in}{0.989127in}}%
\pgfpathlineto{\pgfqpoint{1.106258in}{0.975071in}}%
\pgfpathlineto{\pgfqpoint{1.110376in}{0.975071in}}%
\pgfpathlineto{\pgfqpoint{1.114494in}{0.989127in}}%
\pgfpathlineto{\pgfqpoint{1.118612in}{0.989127in}}%
\pgfpathlineto{\pgfqpoint{1.122730in}{1.003184in}}%
\pgfpathlineto{\pgfqpoint{1.126848in}{0.989127in}}%
\pgfpathlineto{\pgfqpoint{1.130966in}{0.989127in}}%
\pgfpathlineto{\pgfqpoint{1.139202in}{1.017240in}}%
\pgfpathlineto{\pgfqpoint{1.147438in}{1.017240in}}%
\pgfpathlineto{\pgfqpoint{1.151557in}{1.031296in}}%
\pgfpathlineto{\pgfqpoint{1.168029in}{1.031296in}}%
\pgfpathlineto{\pgfqpoint{1.172147in}{1.017240in}}%
\pgfpathlineto{\pgfqpoint{1.180383in}{1.017240in}}%
\pgfpathlineto{\pgfqpoint{1.184501in}{1.003184in}}%
\pgfpathlineto{\pgfqpoint{1.188619in}{1.017240in}}%
\pgfpathlineto{\pgfqpoint{1.196855in}{1.017240in}}%
\pgfpathlineto{\pgfqpoint{1.200974in}{1.031296in}}%
\pgfpathlineto{\pgfqpoint{1.205092in}{1.031296in}}%
\pgfpathlineto{\pgfqpoint{1.209210in}{1.017240in}}%
\pgfpathlineto{\pgfqpoint{1.221564in}{1.017240in}}%
\pgfpathlineto{\pgfqpoint{1.225682in}{1.003184in}}%
\pgfpathlineto{\pgfqpoint{1.233918in}{1.003184in}}%
\pgfpathlineto{\pgfqpoint{1.242154in}{1.031296in}}%
\pgfpathlineto{\pgfqpoint{1.246273in}{1.031296in}}%
\pgfpathlineto{\pgfqpoint{1.250391in}{1.045352in}}%
\pgfpathlineto{\pgfqpoint{1.254509in}{1.045352in}}%
\pgfpathlineto{\pgfqpoint{1.258627in}{1.059408in}}%
\pgfpathlineto{\pgfqpoint{1.287453in}{1.059408in}}%
\pgfpathlineto{\pgfqpoint{1.291571in}{1.045352in}}%
\pgfpathlineto{\pgfqpoint{1.303926in}{1.045352in}}%
\pgfpathlineto{\pgfqpoint{1.308044in}{1.059408in}}%
\pgfpathlineto{\pgfqpoint{1.332752in}{1.059408in}}%
\pgfpathlineto{\pgfqpoint{1.336870in}{1.073465in}}%
\pgfpathlineto{\pgfqpoint{1.345107in}{1.073465in}}%
\pgfpathlineto{\pgfqpoint{1.349225in}{1.059408in}}%
\pgfpathlineto{\pgfqpoint{1.353343in}{1.087521in}}%
\pgfpathlineto{\pgfqpoint{1.382169in}{1.087521in}}%
\pgfpathlineto{\pgfqpoint{1.386287in}{1.073465in}}%
\pgfpathlineto{\pgfqpoint{1.390405in}{1.073465in}}%
\pgfpathlineto{\pgfqpoint{1.394524in}{1.087521in}}%
\pgfpathlineto{\pgfqpoint{1.398642in}{1.087521in}}%
\pgfpathlineto{\pgfqpoint{1.402760in}{1.073465in}}%
\pgfpathlineto{\pgfqpoint{1.423350in}{1.073465in}}%
\pgfpathlineto{\pgfqpoint{1.427468in}{1.059408in}}%
\pgfpathlineto{\pgfqpoint{1.439822in}{1.101577in}}%
\pgfpathlineto{\pgfqpoint{1.448059in}{1.101577in}}%
\pgfpathlineto{\pgfqpoint{1.452177in}{1.115633in}}%
\pgfpathlineto{\pgfqpoint{1.481003in}{1.115633in}}%
\pgfpathlineto{\pgfqpoint{1.485121in}{1.129690in}}%
\pgfpathlineto{\pgfqpoint{1.497476in}{1.087521in}}%
\pgfpathlineto{\pgfqpoint{1.518066in}{1.087521in}}%
\pgfpathlineto{\pgfqpoint{1.522184in}{1.101577in}}%
\pgfpathlineto{\pgfqpoint{1.526302in}{1.101577in}}%
\pgfpathlineto{\pgfqpoint{1.530420in}{1.087521in}}%
\pgfpathlineto{\pgfqpoint{1.534538in}{1.087521in}}%
\pgfpathlineto{\pgfqpoint{1.538657in}{1.101577in}}%
\pgfpathlineto{\pgfqpoint{1.563365in}{1.101577in}}%
\pgfpathlineto{\pgfqpoint{1.567483in}{1.129690in}}%
\pgfpathlineto{\pgfqpoint{1.571601in}{1.129690in}}%
\pgfpathlineto{\pgfqpoint{1.575719in}{1.115633in}}%
\pgfpathlineto{\pgfqpoint{1.579837in}{1.129690in}}%
\pgfpathlineto{\pgfqpoint{1.588074in}{1.129690in}}%
\pgfpathlineto{\pgfqpoint{1.592192in}{1.115633in}}%
\pgfpathlineto{\pgfqpoint{1.600428in}{1.143746in}}%
\pgfpathlineto{\pgfqpoint{1.604546in}{1.143746in}}%
\pgfpathlineto{\pgfqpoint{1.608664in}{1.185914in}}%
\pgfpathlineto{\pgfqpoint{1.612782in}{1.171858in}}%
\pgfpathlineto{\pgfqpoint{1.616900in}{1.171858in}}%
\pgfpathlineto{\pgfqpoint{1.625136in}{1.199971in}}%
\pgfpathlineto{\pgfqpoint{1.629254in}{1.199971in}}%
\pgfpathlineto{\pgfqpoint{1.633372in}{1.185914in}}%
\pgfpathlineto{\pgfqpoint{1.662199in}{1.185914in}}%
\pgfpathlineto{\pgfqpoint{1.670435in}{1.214027in}}%
\pgfpathlineto{\pgfqpoint{1.678671in}{1.214027in}}%
\pgfpathlineto{\pgfqpoint{1.686908in}{1.185914in}}%
\pgfpathlineto{\pgfqpoint{1.707498in}{1.185914in}}%
\pgfpathlineto{\pgfqpoint{1.711616in}{1.199971in}}%
\pgfpathlineto{\pgfqpoint{1.740443in}{1.199971in}}%
\pgfpathlineto{\pgfqpoint{1.744561in}{1.214027in}}%
\pgfpathlineto{\pgfqpoint{1.761033in}{1.214027in}}%
\pgfpathlineto{\pgfqpoint{1.765151in}{1.228083in}}%
\pgfpathlineto{\pgfqpoint{1.793978in}{1.228083in}}%
\pgfpathlineto{\pgfqpoint{1.798096in}{1.214027in}}%
\pgfpathlineto{\pgfqpoint{1.806332in}{1.214027in}}%
\pgfpathlineto{\pgfqpoint{1.810450in}{1.228083in}}%
\pgfpathlineto{\pgfqpoint{1.818686in}{1.228083in}}%
\pgfpathlineto{\pgfqpoint{1.822804in}{1.242139in}}%
\pgfpathlineto{\pgfqpoint{1.843395in}{1.242139in}}%
\pgfpathlineto{\pgfqpoint{1.847513in}{1.256196in}}%
\pgfpathlineto{\pgfqpoint{1.872221in}{1.256196in}}%
\pgfpathlineto{\pgfqpoint{1.876339in}{1.270252in}}%
\pgfpathlineto{\pgfqpoint{1.896930in}{1.270252in}}%
\pgfpathlineto{\pgfqpoint{1.901048in}{1.298364in}}%
\pgfpathlineto{\pgfqpoint{1.909284in}{1.298364in}}%
\pgfpathlineto{\pgfqpoint{1.913402in}{1.284308in}}%
\pgfpathlineto{\pgfqpoint{1.917520in}{1.284308in}}%
\pgfpathlineto{\pgfqpoint{1.925756in}{1.256196in}}%
\pgfpathlineto{\pgfqpoint{1.929875in}{1.256196in}}%
\pgfpathlineto{\pgfqpoint{1.933993in}{1.242139in}}%
\pgfpathlineto{\pgfqpoint{1.958701in}{1.242139in}}%
\pgfpathlineto{\pgfqpoint{1.962819in}{1.228083in}}%
\pgfpathlineto{\pgfqpoint{1.975173in}{1.228083in}}%
\pgfpathlineto{\pgfqpoint{1.979292in}{1.214027in}}%
\pgfpathlineto{\pgfqpoint{1.991646in}{1.214027in}}%
\pgfpathlineto{\pgfqpoint{1.995764in}{1.228083in}}%
\pgfpathlineto{\pgfqpoint{2.036945in}{1.228083in}}%
\pgfpathlineto{\pgfqpoint{2.041063in}{1.199971in}}%
\pgfpathlineto{\pgfqpoint{2.045181in}{1.199971in}}%
\pgfpathlineto{\pgfqpoint{2.053417in}{1.228083in}}%
\pgfpathlineto{\pgfqpoint{2.061653in}{1.228083in}}%
\pgfpathlineto{\pgfqpoint{2.065771in}{1.256196in}}%
\pgfpathlineto{\pgfqpoint{2.074008in}{1.228083in}}%
\pgfpathlineto{\pgfqpoint{2.094598in}{1.228083in}}%
\pgfpathlineto{\pgfqpoint{2.098716in}{1.242139in}}%
\pgfpathlineto{\pgfqpoint{2.144015in}{1.242139in}}%
\pgfpathlineto{\pgfqpoint{2.148133in}{1.256196in}}%
\pgfpathlineto{\pgfqpoint{2.156369in}{1.256196in}}%
\pgfpathlineto{\pgfqpoint{2.160487in}{1.242139in}}%
\pgfpathlineto{\pgfqpoint{2.164605in}{1.242139in}}%
\pgfpathlineto{\pgfqpoint{2.168723in}{1.228083in}}%
\pgfpathlineto{\pgfqpoint{2.181078in}{1.228083in}}%
\pgfpathlineto{\pgfqpoint{2.185196in}{1.242139in}}%
\pgfpathlineto{\pgfqpoint{2.230495in}{1.242139in}}%
\pgfpathlineto{\pgfqpoint{2.234613in}{1.256196in}}%
\pgfpathlineto{\pgfqpoint{2.275794in}{1.256196in}}%
\pgfpathlineto{\pgfqpoint{2.279912in}{1.270252in}}%
\pgfpathlineto{\pgfqpoint{2.646421in}{1.270252in}}%
\pgfpathlineto{\pgfqpoint{2.646421in}{1.270252in}}%
\pgfusepath{stroke}%
\end{pgfscope}%
\begin{pgfscope}%
\pgfpathrectangle{\pgfqpoint{0.488751in}{0.368545in}}{\pgfqpoint{2.260417in}{1.283333in}}%
\pgfusepath{clip}%
\pgfsetrectcap%
\pgfsetroundjoin%
\pgfsetlinewidth{0.803000pt}%
\definecolor{currentstroke}{rgb}{0.333333,0.333333,0.333333}%
\pgfsetstrokecolor{currentstroke}%
\pgfsetstrokeopacity{0.270000}%
\pgfsetdash{}{0pt}%
\pgfpathmoveto{\pgfqpoint{0.591497in}{1.017240in}}%
\pgfpathlineto{\pgfqpoint{0.607969in}{1.017240in}}%
\pgfpathlineto{\pgfqpoint{0.612087in}{1.031296in}}%
\pgfpathlineto{\pgfqpoint{0.616206in}{1.017240in}}%
\pgfpathlineto{\pgfqpoint{0.620324in}{1.017240in}}%
\pgfpathlineto{\pgfqpoint{0.624442in}{1.031296in}}%
\pgfpathlineto{\pgfqpoint{0.632678in}{1.031296in}}%
\pgfpathlineto{\pgfqpoint{0.636796in}{1.059408in}}%
\pgfpathlineto{\pgfqpoint{0.640914in}{1.059408in}}%
\pgfpathlineto{\pgfqpoint{0.645032in}{1.087521in}}%
\pgfpathlineto{\pgfqpoint{0.649150in}{1.101577in}}%
\pgfpathlineto{\pgfqpoint{0.653268in}{1.101577in}}%
\pgfpathlineto{\pgfqpoint{0.657386in}{1.143746in}}%
\pgfpathlineto{\pgfqpoint{0.661504in}{1.143746in}}%
\pgfpathlineto{\pgfqpoint{0.665623in}{1.129690in}}%
\pgfpathlineto{\pgfqpoint{0.673859in}{1.157802in}}%
\pgfpathlineto{\pgfqpoint{0.677977in}{1.143746in}}%
\pgfpathlineto{\pgfqpoint{0.682095in}{1.171858in}}%
\pgfpathlineto{\pgfqpoint{0.686213in}{1.157802in}}%
\pgfpathlineto{\pgfqpoint{0.694449in}{1.157802in}}%
\pgfpathlineto{\pgfqpoint{0.698567in}{1.143746in}}%
\pgfpathlineto{\pgfqpoint{0.706803in}{1.171858in}}%
\pgfpathlineto{\pgfqpoint{0.710922in}{1.171858in}}%
\pgfpathlineto{\pgfqpoint{0.715040in}{1.157802in}}%
\pgfpathlineto{\pgfqpoint{0.719158in}{1.171858in}}%
\pgfpathlineto{\pgfqpoint{0.735630in}{1.171858in}}%
\pgfpathlineto{\pgfqpoint{0.739748in}{1.157802in}}%
\pgfpathlineto{\pgfqpoint{0.756220in}{1.157802in}}%
\pgfpathlineto{\pgfqpoint{0.760339in}{1.143746in}}%
\pgfpathlineto{\pgfqpoint{0.764457in}{1.157802in}}%
\pgfpathlineto{\pgfqpoint{0.768575in}{1.157802in}}%
\pgfpathlineto{\pgfqpoint{0.772693in}{1.171858in}}%
\pgfpathlineto{\pgfqpoint{0.785047in}{1.171858in}}%
\pgfpathlineto{\pgfqpoint{0.789165in}{1.157802in}}%
\pgfpathlineto{\pgfqpoint{0.813874in}{1.157802in}}%
\pgfpathlineto{\pgfqpoint{0.817992in}{1.171858in}}%
\pgfpathlineto{\pgfqpoint{0.855054in}{1.171858in}}%
\pgfpathlineto{\pgfqpoint{0.859173in}{1.185914in}}%
\pgfpathlineto{\pgfqpoint{0.949770in}{1.185914in}}%
\pgfpathlineto{\pgfqpoint{0.958007in}{1.214027in}}%
\pgfpathlineto{\pgfqpoint{0.974479in}{1.214027in}}%
\pgfpathlineto{\pgfqpoint{0.978597in}{1.228083in}}%
\pgfpathlineto{\pgfqpoint{1.036250in}{1.228083in}}%
\pgfpathlineto{\pgfqpoint{1.040368in}{1.242139in}}%
\pgfpathlineto{\pgfqpoint{1.056841in}{1.242139in}}%
\pgfpathlineto{\pgfqpoint{1.060959in}{1.256196in}}%
\pgfpathlineto{\pgfqpoint{1.073313in}{1.256196in}}%
\pgfpathlineto{\pgfqpoint{1.077431in}{1.242139in}}%
\pgfpathlineto{\pgfqpoint{1.089785in}{1.242139in}}%
\pgfpathlineto{\pgfqpoint{1.093903in}{1.228083in}}%
\pgfpathlineto{\pgfqpoint{1.122730in}{1.228083in}}%
\pgfpathlineto{\pgfqpoint{1.126848in}{1.214027in}}%
\pgfpathlineto{\pgfqpoint{1.188619in}{1.214027in}}%
\pgfpathlineto{\pgfqpoint{1.192737in}{1.199971in}}%
\pgfpathlineto{\pgfqpoint{1.275099in}{1.199971in}}%
\pgfpathlineto{\pgfqpoint{1.279217in}{1.214027in}}%
\pgfpathlineto{\pgfqpoint{1.287453in}{1.214027in}}%
\pgfpathlineto{\pgfqpoint{1.295690in}{1.185914in}}%
\pgfpathlineto{\pgfqpoint{1.303926in}{1.185914in}}%
\pgfpathlineto{\pgfqpoint{1.308044in}{1.199971in}}%
\pgfpathlineto{\pgfqpoint{1.324516in}{1.199971in}}%
\pgfpathlineto{\pgfqpoint{1.328634in}{1.214027in}}%
\pgfpathlineto{\pgfqpoint{1.332752in}{1.214027in}}%
\pgfpathlineto{\pgfqpoint{1.336870in}{1.228083in}}%
\pgfpathlineto{\pgfqpoint{1.357461in}{1.228083in}}%
\pgfpathlineto{\pgfqpoint{1.361579in}{1.214027in}}%
\pgfpathlineto{\pgfqpoint{1.398642in}{1.214027in}}%
\pgfpathlineto{\pgfqpoint{1.406878in}{1.185914in}}%
\pgfpathlineto{\pgfqpoint{1.456295in}{1.185914in}}%
\pgfpathlineto{\pgfqpoint{1.460413in}{1.199971in}}%
\pgfpathlineto{\pgfqpoint{1.526302in}{1.199971in}}%
\pgfpathlineto{\pgfqpoint{1.530420in}{1.171858in}}%
\pgfpathlineto{\pgfqpoint{1.629254in}{1.171858in}}%
\pgfpathlineto{\pgfqpoint{1.637491in}{1.199971in}}%
\pgfpathlineto{\pgfqpoint{1.682789in}{1.199971in}}%
\pgfpathlineto{\pgfqpoint{1.686908in}{1.185914in}}%
\pgfpathlineto{\pgfqpoint{1.715734in}{1.185914in}}%
\pgfpathlineto{\pgfqpoint{1.719852in}{1.171858in}}%
\pgfpathlineto{\pgfqpoint{1.728088in}{1.171858in}}%
\pgfpathlineto{\pgfqpoint{1.732206in}{1.157802in}}%
\pgfpathlineto{\pgfqpoint{1.847513in}{1.157802in}}%
\pgfpathlineto{\pgfqpoint{1.851631in}{1.143746in}}%
\pgfpathlineto{\pgfqpoint{1.942229in}{1.143746in}}%
\pgfpathlineto{\pgfqpoint{1.946347in}{1.157802in}}%
\pgfpathlineto{\pgfqpoint{1.987528in}{1.157802in}}%
\pgfpathlineto{\pgfqpoint{1.995764in}{1.129690in}}%
\pgfpathlineto{\pgfqpoint{2.049299in}{1.129690in}}%
\pgfpathlineto{\pgfqpoint{2.053417in}{1.143746in}}%
\pgfpathlineto{\pgfqpoint{2.135779in}{1.143746in}}%
\pgfpathlineto{\pgfqpoint{2.139897in}{1.129690in}}%
\pgfpathlineto{\pgfqpoint{2.168723in}{1.129690in}}%
\pgfpathlineto{\pgfqpoint{2.172842in}{1.143746in}}%
\pgfpathlineto{\pgfqpoint{2.193432in}{1.143746in}}%
\pgfpathlineto{\pgfqpoint{2.197550in}{1.157802in}}%
\pgfpathlineto{\pgfqpoint{2.448753in}{1.157802in}}%
\pgfpathlineto{\pgfqpoint{2.452871in}{1.143746in}}%
\pgfpathlineto{\pgfqpoint{2.502288in}{1.143746in}}%
\pgfpathlineto{\pgfqpoint{2.506406in}{1.157802in}}%
\pgfpathlineto{\pgfqpoint{2.646421in}{1.157802in}}%
\pgfpathlineto{\pgfqpoint{2.646421in}{1.157802in}}%
\pgfusepath{stroke}%
\end{pgfscope}%
\begin{pgfscope}%
\pgfsetrectcap%
\pgfsetmiterjoin%
\pgfsetlinewidth{0.501875pt}%
\definecolor{currentstroke}{rgb}{0.317647,0.317647,0.317647}%
\pgfsetstrokecolor{currentstroke}%
\pgfsetdash{}{0pt}%
\pgfpathmoveto{\pgfqpoint{0.488751in}{0.368545in}}%
\pgfpathlineto{\pgfqpoint{0.488751in}{1.651878in}}%
\pgfusepath{stroke}%
\end{pgfscope}%
\begin{pgfscope}%
\pgfsetrectcap%
\pgfsetmiterjoin%
\pgfsetlinewidth{0.501875pt}%
\definecolor{currentstroke}{rgb}{0.317647,0.317647,0.317647}%
\pgfsetstrokecolor{currentstroke}%
\pgfsetdash{}{0pt}%
\pgfpathmoveto{\pgfqpoint{0.488751in}{0.368545in}}%
\pgfpathlineto{\pgfqpoint{2.749168in}{0.368545in}}%
\pgfusepath{stroke}%
\end{pgfscope}%
\begin{pgfscope}%
\pgfsetrectcap%
\pgfsetroundjoin%
\pgfsetlinewidth{0.803000pt}%
\definecolor{currentstroke}{rgb}{0.333333,0.333333,0.333333}%
\pgfsetstrokecolor{currentstroke}%
\pgfsetdash{}{0pt}%
\pgfpathmoveto{\pgfqpoint{2.703959in}{1.506601in}}%
\pgfpathlineto{\pgfqpoint{2.748404in}{1.506601in}}%
\pgfusepath{stroke}%
\end{pgfscope}%
\begin{pgfscope}%
\definecolor{textcolor}{rgb}{0.000000,0.000000,0.000000}%
\pgfsetstrokecolor{textcolor}%
\pgfsetfillcolor{textcolor}%
\pgftext[x=2.776181in,y=1.487156in,left,base]{\color{textcolor}\rmfamily\fontsize{4.000000}{4.800000}\selectfont \(\displaystyle w_{00}\)}%
\end{pgfscope}%
\begin{pgfscope}%
\pgfsetrectcap%
\pgfsetroundjoin%
\pgfsetlinewidth{0.803000pt}%
\definecolor{currentstroke}{rgb}{0.686275,0.352941,0.313725}%
\pgfsetstrokecolor{currentstroke}%
\pgfsetdash{}{0pt}%
\pgfpathmoveto{\pgfqpoint{2.703959in}{1.424656in}}%
\pgfpathlineto{\pgfqpoint{2.748404in}{1.424656in}}%
\pgfusepath{stroke}%
\end{pgfscope}%
\begin{pgfscope}%
\definecolor{textcolor}{rgb}{0.000000,0.000000,0.000000}%
\pgfsetstrokecolor{textcolor}%
\pgfsetfillcolor{textcolor}%
\pgftext[x=2.776181in,y=1.405212in,left,base]{\color{textcolor}\rmfamily\fontsize{4.000000}{4.800000}\selectfont \(\displaystyle w_{10}\)}%
\end{pgfscope}%
\begin{pgfscope}%
\pgfsetrectcap%
\pgfsetroundjoin%
\pgfsetlinewidth{0.803000pt}%
\definecolor{currentstroke}{rgb}{0.000000,0.356863,0.509804}%
\pgfsetstrokecolor{currentstroke}%
\pgfsetdash{}{0pt}%
\pgfpathmoveto{\pgfqpoint{2.703959in}{1.342712in}}%
\pgfpathlineto{\pgfqpoint{2.748404in}{1.342712in}}%
\pgfusepath{stroke}%
\end{pgfscope}%
\begin{pgfscope}%
\definecolor{textcolor}{rgb}{0.000000,0.000000,0.000000}%
\pgfsetstrokecolor{textcolor}%
\pgfsetfillcolor{textcolor}%
\pgftext[x=2.776181in,y=1.323267in,left,base]{\color{textcolor}\rmfamily\fontsize{4.000000}{4.800000}\selectfont \(\displaystyle w_{20}\)}%
\end{pgfscope}%
\begin{pgfscope}%
\pgfsetrectcap%
\pgfsetroundjoin%
\pgfsetlinewidth{0.803000pt}%
\definecolor{currentstroke}{rgb}{0.490196,0.588235,0.431373}%
\pgfsetstrokecolor{currentstroke}%
\pgfsetdash{}{0pt}%
\pgfpathmoveto{\pgfqpoint{2.703959in}{1.260767in}}%
\pgfpathlineto{\pgfqpoint{2.748404in}{1.260767in}}%
\pgfusepath{stroke}%
\end{pgfscope}%
\begin{pgfscope}%
\definecolor{textcolor}{rgb}{0.000000,0.000000,0.000000}%
\pgfsetstrokecolor{textcolor}%
\pgfsetfillcolor{textcolor}%
\pgftext[x=2.776181in,y=1.241323in,left,base]{\color{textcolor}\rmfamily\fontsize{4.000000}{4.800000}\selectfont \(\displaystyle w_{30}\)}%
\end{pgfscope}%
\begin{pgfscope}%
\pgfsetrectcap%
\pgfsetroundjoin%
\pgfsetlinewidth{0.803000pt}%
\definecolor{currentstroke}{rgb}{0.843137,0.666667,0.313725}%
\pgfsetstrokecolor{currentstroke}%
\pgfsetdash{}{0pt}%
\pgfpathmoveto{\pgfqpoint{2.703959in}{1.178823in}}%
\pgfpathlineto{\pgfqpoint{2.748404in}{1.178823in}}%
\pgfusepath{stroke}%
\end{pgfscope}%
\begin{pgfscope}%
\definecolor{textcolor}{rgb}{0.000000,0.000000,0.000000}%
\pgfsetstrokecolor{textcolor}%
\pgfsetfillcolor{textcolor}%
\pgftext[x=2.776181in,y=1.159378in,left,base]{\color{textcolor}\rmfamily\fontsize{4.000000}{4.800000}\selectfont \(\displaystyle w_{40}\)}%
\end{pgfscope}%
\begin{pgfscope}%
\pgfsetbuttcap%
\pgfsetmiterjoin%
\pgfsetlinewidth{0.000000pt}%
\definecolor{currentstroke}{rgb}{0.000000,0.000000,0.000000}%
\pgfsetstrokecolor{currentstroke}%
\pgfsetstrokeopacity{0.000000}%
\pgfsetdash{}{0pt}%
\pgfpathmoveto{\pgfqpoint{3.653334in}{0.368545in}}%
\pgfpathlineto{\pgfqpoint{5.913751in}{0.368545in}}%
\pgfpathlineto{\pgfqpoint{5.913751in}{1.651878in}}%
\pgfpathlineto{\pgfqpoint{3.653334in}{1.651878in}}%
\pgfpathclose%
\pgfusepath{}%
\end{pgfscope}%
\begin{pgfscope}%
\pgfsetbuttcap%
\pgfsetroundjoin%
\definecolor{currentfill}{rgb}{0.317647,0.317647,0.317647}%
\pgfsetfillcolor{currentfill}%
\pgfsetlinewidth{0.501875pt}%
\definecolor{currentstroke}{rgb}{0.317647,0.317647,0.317647}%
\pgfsetstrokecolor{currentstroke}%
\pgfsetdash{}{0pt}%
\pgfsys@defobject{currentmarker}{\pgfqpoint{0.000000in}{-0.020833in}}{\pgfqpoint{0.000000in}{0.000000in}}{%
\pgfpathmoveto{\pgfqpoint{0.000000in}{0.000000in}}%
\pgfpathlineto{\pgfqpoint{0.000000in}{-0.020833in}}%
\pgfusepath{stroke,fill}%
}%
\begin{pgfscope}%
\pgfsys@transformshift{3.756080in}{0.368545in}%
\pgfsys@useobject{currentmarker}{}%
\end{pgfscope}%
\end{pgfscope}%
\begin{pgfscope}%
\definecolor{textcolor}{rgb}{0.317647,0.317647,0.317647}%
\pgfsetstrokecolor{textcolor}%
\pgfsetfillcolor{textcolor}%
\pgftext[x=3.756080in,y=0.319934in,,top]{\color{textcolor}\rmfamily\fontsize{6.664000}{7.996800}\selectfont \(\displaystyle 0\)}%
\end{pgfscope}%
\begin{pgfscope}%
\pgfsetbuttcap%
\pgfsetroundjoin%
\definecolor{currentfill}{rgb}{0.317647,0.317647,0.317647}%
\pgfsetfillcolor{currentfill}%
\pgfsetlinewidth{0.501875pt}%
\definecolor{currentstroke}{rgb}{0.317647,0.317647,0.317647}%
\pgfsetstrokecolor{currentstroke}%
\pgfsetdash{}{0pt}%
\pgfsys@defobject{currentmarker}{\pgfqpoint{0.000000in}{-0.020833in}}{\pgfqpoint{0.000000in}{0.000000in}}{%
\pgfpathmoveto{\pgfqpoint{0.000000in}{0.000000in}}%
\pgfpathlineto{\pgfqpoint{0.000000in}{-0.020833in}}%
\pgfusepath{stroke,fill}%
}%
\begin{pgfscope}%
\pgfsys@transformshift{4.167889in}{0.368545in}%
\pgfsys@useobject{currentmarker}{}%
\end{pgfscope}%
\end{pgfscope}%
\begin{pgfscope}%
\definecolor{textcolor}{rgb}{0.317647,0.317647,0.317647}%
\pgfsetstrokecolor{textcolor}%
\pgfsetfillcolor{textcolor}%
\pgftext[x=4.167889in,y=0.319934in,,top]{\color{textcolor}\rmfamily\fontsize{6.664000}{7.996800}\selectfont \(\displaystyle 500\)}%
\end{pgfscope}%
\begin{pgfscope}%
\pgfsetbuttcap%
\pgfsetroundjoin%
\definecolor{currentfill}{rgb}{0.317647,0.317647,0.317647}%
\pgfsetfillcolor{currentfill}%
\pgfsetlinewidth{0.501875pt}%
\definecolor{currentstroke}{rgb}{0.317647,0.317647,0.317647}%
\pgfsetstrokecolor{currentstroke}%
\pgfsetdash{}{0pt}%
\pgfsys@defobject{currentmarker}{\pgfqpoint{0.000000in}{-0.020833in}}{\pgfqpoint{0.000000in}{0.000000in}}{%
\pgfpathmoveto{\pgfqpoint{0.000000in}{0.000000in}}%
\pgfpathlineto{\pgfqpoint{0.000000in}{-0.020833in}}%
\pgfusepath{stroke,fill}%
}%
\begin{pgfscope}%
\pgfsys@transformshift{4.579697in}{0.368545in}%
\pgfsys@useobject{currentmarker}{}%
\end{pgfscope}%
\end{pgfscope}%
\begin{pgfscope}%
\definecolor{textcolor}{rgb}{0.317647,0.317647,0.317647}%
\pgfsetstrokecolor{textcolor}%
\pgfsetfillcolor{textcolor}%
\pgftext[x=4.579697in,y=0.319934in,,top]{\color{textcolor}\rmfamily\fontsize{6.664000}{7.996800}\selectfont \(\displaystyle 1000\)}%
\end{pgfscope}%
\begin{pgfscope}%
\pgfsetbuttcap%
\pgfsetroundjoin%
\definecolor{currentfill}{rgb}{0.317647,0.317647,0.317647}%
\pgfsetfillcolor{currentfill}%
\pgfsetlinewidth{0.501875pt}%
\definecolor{currentstroke}{rgb}{0.317647,0.317647,0.317647}%
\pgfsetstrokecolor{currentstroke}%
\pgfsetdash{}{0pt}%
\pgfsys@defobject{currentmarker}{\pgfqpoint{0.000000in}{-0.020833in}}{\pgfqpoint{0.000000in}{0.000000in}}{%
\pgfpathmoveto{\pgfqpoint{0.000000in}{0.000000in}}%
\pgfpathlineto{\pgfqpoint{0.000000in}{-0.020833in}}%
\pgfusepath{stroke,fill}%
}%
\begin{pgfscope}%
\pgfsys@transformshift{4.991506in}{0.368545in}%
\pgfsys@useobject{currentmarker}{}%
\end{pgfscope}%
\end{pgfscope}%
\begin{pgfscope}%
\definecolor{textcolor}{rgb}{0.317647,0.317647,0.317647}%
\pgfsetstrokecolor{textcolor}%
\pgfsetfillcolor{textcolor}%
\pgftext[x=4.991506in,y=0.319934in,,top]{\color{textcolor}\rmfamily\fontsize{6.664000}{7.996800}\selectfont \(\displaystyle 1500\)}%
\end{pgfscope}%
\begin{pgfscope}%
\pgfsetbuttcap%
\pgfsetroundjoin%
\definecolor{currentfill}{rgb}{0.317647,0.317647,0.317647}%
\pgfsetfillcolor{currentfill}%
\pgfsetlinewidth{0.501875pt}%
\definecolor{currentstroke}{rgb}{0.317647,0.317647,0.317647}%
\pgfsetstrokecolor{currentstroke}%
\pgfsetdash{}{0pt}%
\pgfsys@defobject{currentmarker}{\pgfqpoint{0.000000in}{-0.020833in}}{\pgfqpoint{0.000000in}{0.000000in}}{%
\pgfpathmoveto{\pgfqpoint{0.000000in}{0.000000in}}%
\pgfpathlineto{\pgfqpoint{0.000000in}{-0.020833in}}%
\pgfusepath{stroke,fill}%
}%
\begin{pgfscope}%
\pgfsys@transformshift{5.403314in}{0.368545in}%
\pgfsys@useobject{currentmarker}{}%
\end{pgfscope}%
\end{pgfscope}%
\begin{pgfscope}%
\definecolor{textcolor}{rgb}{0.317647,0.317647,0.317647}%
\pgfsetstrokecolor{textcolor}%
\pgfsetfillcolor{textcolor}%
\pgftext[x=5.403314in,y=0.319934in,,top]{\color{textcolor}\rmfamily\fontsize{6.664000}{7.996800}\selectfont \(\displaystyle 2000\)}%
\end{pgfscope}%
\begin{pgfscope}%
\pgfsetbuttcap%
\pgfsetroundjoin%
\definecolor{currentfill}{rgb}{0.317647,0.317647,0.317647}%
\pgfsetfillcolor{currentfill}%
\pgfsetlinewidth{0.501875pt}%
\definecolor{currentstroke}{rgb}{0.317647,0.317647,0.317647}%
\pgfsetstrokecolor{currentstroke}%
\pgfsetdash{}{0pt}%
\pgfsys@defobject{currentmarker}{\pgfqpoint{0.000000in}{-0.020833in}}{\pgfqpoint{0.000000in}{0.000000in}}{%
\pgfpathmoveto{\pgfqpoint{0.000000in}{0.000000in}}%
\pgfpathlineto{\pgfqpoint{0.000000in}{-0.020833in}}%
\pgfusepath{stroke,fill}%
}%
\begin{pgfscope}%
\pgfsys@transformshift{5.815123in}{0.368545in}%
\pgfsys@useobject{currentmarker}{}%
\end{pgfscope}%
\end{pgfscope}%
\begin{pgfscope}%
\definecolor{textcolor}{rgb}{0.317647,0.317647,0.317647}%
\pgfsetstrokecolor{textcolor}%
\pgfsetfillcolor{textcolor}%
\pgftext[x=5.815123in,y=0.319934in,,top]{\color{textcolor}\rmfamily\fontsize{6.664000}{7.996800}\selectfont \(\displaystyle 2500\)}%
\end{pgfscope}%
\begin{pgfscope}%
\definecolor{textcolor}{rgb}{0.317647,0.317647,0.317647}%
\pgfsetstrokecolor{textcolor}%
\pgfsetfillcolor{textcolor}%
\pgftext[x=4.783543in,y=0.182189in,,top]{\color{textcolor}\rmfamily\fontsize{6.664000}{7.996800}\selectfont Iteration}%
\end{pgfscope}%
\begin{pgfscope}%
\pgfsetbuttcap%
\pgfsetroundjoin%
\definecolor{currentfill}{rgb}{0.317647,0.317647,0.317647}%
\pgfsetfillcolor{currentfill}%
\pgfsetlinewidth{0.501875pt}%
\definecolor{currentstroke}{rgb}{0.317647,0.317647,0.317647}%
\pgfsetstrokecolor{currentstroke}%
\pgfsetdash{}{0pt}%
\pgfsys@defobject{currentmarker}{\pgfqpoint{-0.020833in}{0.000000in}}{\pgfqpoint{0.000000in}{0.000000in}}{%
\pgfpathmoveto{\pgfqpoint{0.000000in}{0.000000in}}%
\pgfpathlineto{\pgfqpoint{-0.020833in}{0.000000in}}%
\pgfusepath{stroke,fill}%
}%
\begin{pgfscope}%
\pgfsys@transformshift{3.653334in}{0.704656in}%
\pgfsys@useobject{currentmarker}{}%
\end{pgfscope}%
\end{pgfscope}%
\begin{pgfscope}%
\definecolor{textcolor}{rgb}{0.317647,0.317647,0.317647}%
\pgfsetstrokecolor{textcolor}%
\pgfsetfillcolor{textcolor}%
\pgftext[x=3.452523in,y=0.672539in,left,base]{\color{textcolor}\rmfamily\fontsize{6.664000}{7.996800}\selectfont \(\displaystyle 280\)}%
\end{pgfscope}%
\begin{pgfscope}%
\pgfsetbuttcap%
\pgfsetroundjoin%
\definecolor{currentfill}{rgb}{0.317647,0.317647,0.317647}%
\pgfsetfillcolor{currentfill}%
\pgfsetlinewidth{0.501875pt}%
\definecolor{currentstroke}{rgb}{0.317647,0.317647,0.317647}%
\pgfsetstrokecolor{currentstroke}%
\pgfsetdash{}{0pt}%
\pgfsys@defobject{currentmarker}{\pgfqpoint{-0.020833in}{0.000000in}}{\pgfqpoint{0.000000in}{0.000000in}}{%
\pgfpathmoveto{\pgfqpoint{0.000000in}{0.000000in}}%
\pgfpathlineto{\pgfqpoint{-0.020833in}{0.000000in}}%
\pgfusepath{stroke,fill}%
}%
\begin{pgfscope}%
\pgfsys@transformshift{3.653334in}{1.075026in}%
\pgfsys@useobject{currentmarker}{}%
\end{pgfscope}%
\end{pgfscope}%
\begin{pgfscope}%
\definecolor{textcolor}{rgb}{0.317647,0.317647,0.317647}%
\pgfsetstrokecolor{textcolor}%
\pgfsetfillcolor{textcolor}%
\pgftext[x=3.452523in,y=1.042910in,left,base]{\color{textcolor}\rmfamily\fontsize{6.664000}{7.996800}\selectfont \(\displaystyle 300\)}%
\end{pgfscope}%
\begin{pgfscope}%
\pgfsetbuttcap%
\pgfsetroundjoin%
\definecolor{currentfill}{rgb}{0.317647,0.317647,0.317647}%
\pgfsetfillcolor{currentfill}%
\pgfsetlinewidth{0.501875pt}%
\definecolor{currentstroke}{rgb}{0.317647,0.317647,0.317647}%
\pgfsetstrokecolor{currentstroke}%
\pgfsetdash{}{0pt}%
\pgfsys@defobject{currentmarker}{\pgfqpoint{-0.020833in}{0.000000in}}{\pgfqpoint{0.000000in}{0.000000in}}{%
\pgfpathmoveto{\pgfqpoint{0.000000in}{0.000000in}}%
\pgfpathlineto{\pgfqpoint{-0.020833in}{0.000000in}}%
\pgfusepath{stroke,fill}%
}%
\begin{pgfscope}%
\pgfsys@transformshift{3.653334in}{1.445397in}%
\pgfsys@useobject{currentmarker}{}%
\end{pgfscope}%
\end{pgfscope}%
\begin{pgfscope}%
\definecolor{textcolor}{rgb}{0.317647,0.317647,0.317647}%
\pgfsetstrokecolor{textcolor}%
\pgfsetfillcolor{textcolor}%
\pgftext[x=3.452523in,y=1.413280in,left,base]{\color{textcolor}\rmfamily\fontsize{6.664000}{7.996800}\selectfont \(\displaystyle 320\)}%
\end{pgfscope}%
\begin{pgfscope}%
\definecolor{textcolor}{rgb}{0.317647,0.317647,0.317647}%
\pgfsetstrokecolor{textcolor}%
\pgfsetfillcolor{textcolor}%
\pgftext[x=3.396968in,y=1.010212in,,bottom,rotate=90.000000]{\color{textcolor}\rmfamily\fontsize{6.664000}{7.996800}\selectfont \(\displaystyle \vartheta \propto -b^{(\mathrm{o})} \;(\si{\milli \V})\)}%
\end{pgfscope}%
\begin{pgfscope}%
\pgfpathrectangle{\pgfqpoint{3.653334in}{0.368545in}}{\pgfqpoint{2.260417in}{1.283333in}}%
\pgfusepath{clip}%
\pgfsetrectcap%
\pgfsetroundjoin%
\pgfsetlinewidth{0.803000pt}%
\definecolor{currentstroke}{rgb}{0.333333,0.333333,0.333333}%
\pgfsetstrokecolor{currentstroke}%
\pgfsetdash{}{0pt}%
\pgfpathmoveto{\pgfqpoint{3.756080in}{1.075026in}}%
\pgfpathlineto{\pgfqpoint{3.846678in}{1.075026in}}%
\pgfpathlineto{\pgfqpoint{3.850796in}{1.093545in}}%
\pgfpathlineto{\pgfqpoint{3.867269in}{1.093545in}}%
\pgfpathlineto{\pgfqpoint{3.871387in}{1.112064in}}%
\pgfpathlineto{\pgfqpoint{4.031992in}{1.112064in}}%
\pgfpathlineto{\pgfqpoint{4.036110in}{1.093545in}}%
\pgfpathlineto{\pgfqpoint{4.060819in}{1.093545in}}%
\pgfpathlineto{\pgfqpoint{4.064937in}{1.112064in}}%
\pgfpathlineto{\pgfqpoint{4.192597in}{1.112064in}}%
\pgfpathlineto{\pgfqpoint{4.196715in}{1.130582in}}%
\pgfpathlineto{\pgfqpoint{4.274959in}{1.130582in}}%
\pgfpathlineto{\pgfqpoint{4.279077in}{1.149101in}}%
\pgfpathlineto{\pgfqpoint{4.332612in}{1.149101in}}%
\pgfpathlineto{\pgfqpoint{4.336730in}{1.130582in}}%
\pgfpathlineto{\pgfqpoint{4.443801in}{1.130582in}}%
\pgfpathlineto{\pgfqpoint{4.447919in}{1.112064in}}%
\pgfpathlineto{\pgfqpoint{4.583815in}{1.112064in}}%
\pgfpathlineto{\pgfqpoint{4.587933in}{1.130582in}}%
\pgfpathlineto{\pgfqpoint{4.806192in}{1.130582in}}%
\pgfpathlineto{\pgfqpoint{4.810310in}{1.112064in}}%
\pgfpathlineto{\pgfqpoint{5.341543in}{1.112064in}}%
\pgfpathlineto{\pgfqpoint{5.345661in}{1.130582in}}%
\pgfpathlineto{\pgfqpoint{5.600982in}{1.130582in}}%
\pgfpathlineto{\pgfqpoint{5.605100in}{1.112064in}}%
\pgfpathlineto{\pgfqpoint{5.811005in}{1.112064in}}%
\pgfpathlineto{\pgfqpoint{5.811005in}{1.112064in}}%
\pgfusepath{stroke}%
\end{pgfscope}%
\begin{pgfscope}%
\pgfsetrectcap%
\pgfsetmiterjoin%
\pgfsetlinewidth{0.501875pt}%
\definecolor{currentstroke}{rgb}{0.317647,0.317647,0.317647}%
\pgfsetstrokecolor{currentstroke}%
\pgfsetdash{}{0pt}%
\pgfpathmoveto{\pgfqpoint{3.653334in}{0.368545in}}%
\pgfpathlineto{\pgfqpoint{3.653334in}{1.651878in}}%
\pgfusepath{stroke}%
\end{pgfscope}%
\begin{pgfscope}%
\pgfsetrectcap%
\pgfsetmiterjoin%
\pgfsetlinewidth{0.501875pt}%
\definecolor{currentstroke}{rgb}{0.317647,0.317647,0.317647}%
\pgfsetstrokecolor{currentstroke}%
\pgfsetdash{}{0pt}%
\pgfpathmoveto{\pgfqpoint{3.653334in}{0.368545in}}%
\pgfpathlineto{\pgfqpoint{5.913751in}{0.368545in}}%
\pgfusepath{stroke}%
\end{pgfscope}%
\begin{pgfscope}%
\pgfsetrectcap%
\pgfsetroundjoin%
\pgfsetlinewidth{0.803000pt}%
\definecolor{currentstroke}{rgb}{0.333333,0.333333,0.333333}%
\pgfsetstrokecolor{currentstroke}%
\pgfsetdash{}{0pt}%
\pgfpathmoveto{\pgfqpoint{5.868543in}{1.563823in}}%
\pgfpathlineto{\pgfqpoint{5.912987in}{1.563823in}}%
\pgfusepath{stroke}%
\end{pgfscope}%
\begin{pgfscope}%
\definecolor{textcolor}{rgb}{0.000000,0.000000,0.000000}%
\pgfsetstrokecolor{textcolor}%
\pgfsetfillcolor{textcolor}%
\pgftext[x=5.940765in,y=1.544378in,left,base]{\color{textcolor}\rmfamily\fontsize{4.000000}{4.800000}\selectfont \(\displaystyle b_0\)}%
\end{pgfscope}%
\end{pgfpicture}%
\makeatother%
\endgroup%

	\caption[Monitoring of the training process.]{Monitoring of the training process. The first and second row describes the hidden and output layer respectively. The weights are depicted in the first column and the biases in the second.}
	\label{network_monitoring}
\end{figure}