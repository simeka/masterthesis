\begin{abstract}
\textbf{DUMMY ABSTRACT}
In meinem Masterprojekt in der Electronic Vision(s) Gruppe habe ich mich mit Deep Learning auf neuromorpher Hardware auseinander gesetzt. Dabei sind zwei Experimente entstanden.

Das erste ist ein ratenbasierter Task, der komplett auf dem Chip realisiert werden konnte, d.h. die verwendete Lernmethode ist auf einem dezidierten Plastizitäts-Prozessor auf dem Chip implementiert worden. Eine Verbindung zu einem Host nur notwendig um den Chip einmal zu intialisieren und dann den laufenden Lernfortschritt zu beobachten.

Das zweite Experiment verwendet den neuesten Prototypen der BrainScaleS-2 Plattform, der nicht nur größer ist als seine Vorgänger sondern auch über zusätzliche Auslesefähigkeiten verfügt, insbesondere über das Membran Potential. Damit konnte erfolgreich ein neuartige zeitbasierte Lernregel für Gepulste Neuronal Netzwerke (SNNs) implementiert werden.

Beide Experiment-Ansätze wurden bisher noch nie auf analoger neuromorpher Hardware erfolgreich durchgeführt. Für das zweite ist bereits eine Publikation in Zusammenarbeit mit dem Autor der neuartigen Lernregel in Arbeit.

\end{abstract}

%\renewcommand{\abstractname}{}

\begin{abstract}
\textbf{Robust Quantifying of the Spin-Lattice Relaxation Time with Regards to the Thermal Shift of a Permanent Magnet at 1.0 Tesla}

The spin-lattice relaxation describes the process of spins in a condensed matter distributing energy into the surrounded lattice. This process is characterised by the spin-lattice constant ($T_{1}$). The constant is measured in different phantoms: a homogeneous water phantom and two phantoms containing either fine polystyrene structures. In addition a user-friendly and efficient pulse program has been developed for measuring the spin-lattice constant with the MRI control software ParaVision 6.0.1. As MRI system a low field set-up from the factory Bruker called ICON with 1.0\,T is used.

At a room temperature of $T_{\text{lab}} = (21.9 \pm 0.5)\,^{\circ}\text{C}$ the average $T_{1}$ constant in the homogeneous phantom can be found at $T_{1,\text{exp. 1}} = 2981 \pm 240\,\text{ms}$ respectively at $T_{1,\text{exp. 2}} = 2984 \pm 172\,\text{ms}$. The slightly different results due to the way the mean constant is evaluated. For $T_{1,\text{exp. 1}}$ the raw data is averaged before the fit routine is applied unlike for $T_{1,\text{exp. 2}}$, where the averaging is done after the fitroutine has been applied to every pixel individually. The $T_{1}$ constant of the structured phantoms has been evaluated only qualitatively: the phantom containing a thicker structure returns values around 3000\,ms. The second phantom on the other hand has a more complex and thinner structure which leads to $T_{1}$ values roughly 400 to 500\,ms above the previously measured phantom.
\end{abstract}