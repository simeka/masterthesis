\chapter{Discussion}
Within this thesis the implementation of deep learning methods for \gls{snn} has been demonstrated on analog neuromorphic hardware. In a first example the rate-based approach was successfully executed fully \emph{on-chip} on an early prototype of the \acrfull{bss2} platform. A second more complicated method, SuperSpike, was implemented on a full-size version of the platform using a temporal coding and performed well at solving  an XOR-related problem on two distinct setups.

\paragraph{Rate Coding with SGD} An on-chip realization of gradient descent for a rate-based \gls{snn} was developed for the \gls{dls}. Despite continuing stability issues of the \gls{fpga} and imperfections of the analog hardware, the Circles dataset could be classified with an accuracy of $(99.6 \pm 0.8) \%$. However, a transferability of the results to other setups could not be achieved without extensive fine-tuning of the new chip.  the calibration Transfer function needs to be non-linear the exact shape and its dependencies on noise rate, time constants, .. is not so important if it is somehow shaped like a non linear activation function.

- working point of membrane -> noise and spread as soon as the working point is left (see transfer function with bias plot) -> not so bad actually for training. the error is trained as well. 

- init conditions of task are important. somehow neurons dont learn if the init condition is "bad". but what does that even  mean.. the init condition is bad. the behaviour: the neuron stays silent throughout the learning process. 

- ausblick: notes from mfp talk maybe?
\subsubsection{Temporal Coding with SuperSpike}
% blabla about why not on chip
Unlike the previous experiment, SuperSpike is not implemented in an on-chip fashion. ..explain what is all missing to do it...sth similar to the blackbox should be on-chip. ppu doesnt know the spike times of neither input nor output nor hidden layer. on-chip calibration. not so much agility when trying out and exploring things. numpy and python is quite convenient.
\subsubsection*{Future Projects}
\label{futureprojects}

%For the scope of this thesis a more advanced model is not yet required. All experiments are done using the simpler \gls{lif} model. However, an \gls{adex} based implementation of similar experiments are part of a future project (see \cref{futureprojects}).