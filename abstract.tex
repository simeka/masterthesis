\begin{abstract}
	\textbf{Novel Deep Learning Algorithms for Neuromorphic Hardware}
	Spiking neural networks, the biological basis of how information is conveyed and processed among neurons, not only provided an insight into the brain's mechanisms but offer a feasible alternative to the conventional deep neural networks. When implemented on neuromorphic hardware, they inherit a variety of favorable properties from their biological paradigm, such as parallel and event based information processing and a high energy efficiency. These characteristics are of great interests, as the use of conventional deep learning for real world application is highly inefficient. However, current learning algorithms on neuromorphic hardware have a hard time to compete with the great success of deep learning. In this research, the performance of two candidates for training spiking neural networks is tested on the neuromorphic platform BrainScales2. First, a classical deep neural network is imitated by a rate-based spiking neural network and trained using gradient descent. Second, a novel surrogate gradient descent algorithm, called SuperSpike, is implemented. The first method is designed as an on-chip experiment, whereas the second approach only evaluates the neuron dynamics on the analog core and performs the main computations on a host.
\end{abstract}
\renewcommand{\abstractname}{Zusammenfassung}
\begin{abstract}
\textbf{Neuartige Deep Learning Algorithmen für Neuromorphe Hardware}
Gepulste neuronale Netzwerke, die biologische Basis mithilfe der Neuronen untereinander Informationen austauschen und verarbeiten, dient nicht nur dazu die Vorgänge im Gehirn zu veranschaulichen, sondern ist auch eine m\"{o}gliche alternative zu konventionellen tiefen neuronalen Netzwerken. Werden solche gepulsten neuronalen Netzwerke auf neuromorpher Hardware verwendet, übernehmen sie einige vorteilhaften Eigenschaften ihrer biologischen Vorbilder, wie zum Beispiel das parallelisierte und event-basierte Verabreiten von Informationen oder die hohe Energieefficienz. In Anbetracht der Ineffizienz von conventionellen Deep Learning Methoden, machen die vorhin genannten Besonderheiten gepulste ... 
Allerdings können sich die Trainingsalgorithm von gepulsten Netzwerken noch nicht mit den erprobten Methoden des Deep Learnings messen ...
\end{abstract}


