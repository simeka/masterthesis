\begin{abstract}
\textbf{DUMMY ABSTRACT}

Die Spin-Gitter-Relaxation beschreibt wie Spins in einem Festkörper Energie ins umliegende Gitter abgeben. Dieser Prozess wird über die Spin-Gitter-Relaxationskonstante ($T_{1}$) charakterisiert. Diese Konstante wird im Rahmen dieser Arbeit bei unterschiedlichen Phantomen gemessen: einem reinen Wasserphantom und zwei mit Wasser gefüllte Strukturphantome, deren Struktur aus Polysterolstäben besteht. Für eine benutzerfreundliche und effiziente Messung wird zudem ein Messprogramm in der MRT-Steuerungssoftware ParaVision 6.0.1 entwickelt. Als Messsystem wird ein Niederfeld-MRT mit 1.0 \,T der Firma Bruker verwendet (ICON).

Die durchschnittliche $T_{1}$-Konstante im homogenen Phantom bei einer Raumtemperatur von $T_{\text{Labor}} = (21.9 \pm 0.5)\,^{\circ}\text{C}$ beträgt $T_{1,\text{exp. 1}} = 2981 \pm 240\,\text{ms}$  bzw. $T_{1,\text{exp. 2}} = 2984 \pm 172\,\text{ms}$. Für die Berechnung von $T_{1,\text{exp. 1}}$ wird vor der Fitroutine gemittelt und bei $T_{1,\text{exp. 2}}$ passiert die Mittlung erst danach.
Die $T_{1}$-Konstante der Strukturphantome wird nur qualitativ bestimmt: das grob-strukturierte Phantom hat Werte rund um 3000\,ms und die Werte des fein-strukturierte Phantom liegen um 400 bis 500\,ms höher.
\end{abstract}

%\renewcommand{\abstractname}{}

\begin{abstract}
\textbf{Robust Quantifying of the Spin-Lattice Relaxation Time with Regards to the Thermal Shift of a Permanent Magnet at 1.0 Tesla}

The spin-lattice relaxation describes the process of spins in a condensed matter distributing energy into the surrounded lattice. This process is characterised by the spin-lattice constant ($T_{1}$). The constant is measured in different phantoms: a homogeneous water phantom and two phantoms containing either fine polystyrene structures. In addition a user-friendly and efficient pulse program has been developed for measuring the spin-lattice constant with the MRI control software ParaVision 6.0.1. As MRI system a low field set-up from the factory Bruker called ICON with 1.0\,T is used.

At a room temperature of $T_{\text{lab}} = (21.9 \pm 0.5)\,^{\circ}\text{C}$ the average $T_{1}$ constant in the homogeneous phantom can be found at $T_{1,\text{exp. 1}} = 2981 \pm 240\,\text{ms}$ respectively at $T_{1,\text{exp. 2}} = 2984 \pm 172\,\text{ms}$. The slightly different results due to the way the mean constant is evaluated. For $T_{1,\text{exp. 1}}$ the raw data is averaged before the fit routine is applied unlike for $T_{1,\text{exp. 2}}$, where the averaging is done after the fitroutine has been applied to every pixel individually. The $T_{1}$ constant of the structured phantoms has been evaluated only qualitatively: the phantom containing a thicker structure returns values around 3000\,ms. The second phantom on the other hand has a more complex and thinner structure which leads to $T_{1}$ values roughly 400 to 500\,ms above the previously measured phantom.
\end{abstract}