\chapter{Background}
%Starting my final year project at the Electronic Vision(s) group, I soon realized the diversity of their research. 
The required knowledge to work in the field of neuromorphic computing is broad and manifold, ranging from the biological view of the human brain to electronic circuit laws further to machine learning algorithms. In the next sections, I will introduce the most important concepts and physical backgrounds upon which the presented research in this thesis is based on. Starting with deep learning and an overview of the biological neuron, the transition to neuron models, neuronal coding schemes and their training approaches will be made, before introducing the neuromorphic \gls{bss2} platform 
\section{Deep Learning}
\label{deeplearning}
Deep learning is among the most useful and powerful tools machine learning has provided to the scientific community. Image or pattern recognition are in general hard to solve tasks for traditional computation concepts. Deep learning abstracts such task in terms of a hierarchy of concepts. Each concept is based upon a combination of simpler ones. Going down on a hypothetical ladder towards the easiest concept available, creates a deep structure with many layers. This is why it's called \emph{deep learning} (\citealp{Goodfellow-et-al-2016}).

A popular example for deep learning is the \gls{mlp}, a deep feed-forward network. As the name suggests, the information is forwarded from one layer to another. At each layer the input $\mathbf{x}$ is mapped to an output $\mathbf{y} = \gls{transfer}(\mathbf{x, \theta})$ by the transfer function \gls{transfer} and a set of parameters $\mathbf{\theta}$. The layer structure of such a network is inspired by biological neural networks and therefore such networks are often referred to as \glsfirst{ann}.

In machine learning, one discriminates between supervised, and unsupervised learning algorithms. Without a supervisor, an algorithm looks for structures and useful properties within the input data. Many machine learning algorithms are not supervised and try to observe the underlying probability distribution of the input data. In supervised learning, on the other side, each input vector $x$ is associated with a target vector $\hat{y}$. In this case, the algorithm is trained to predict a target for a given input.

Q: Überleitungssatz hier zu Supervised Training?

%Given the task to identify a picture of cat, any computer will have a hard time to map the essential information of the camera's sensor data to the class \textit{cat}. An \gls{ann} solves the problem by learning how to describe the raw data in terms of simpler representations and essentially by combining these simpler representations into a meaningful solution in the last layer - the output layer. The first layer is called input layer and all other layers in between are named hidden layers, since they are usually not visible from the outside.\\


\subsection{Supervised Training}
\label{supervisedtraining}
The training process of an \gls{ann} can be divided into a \emph{forward pass} where the output of all nodes is evaluated and a \emph{backward pass} which is responsible for the learning.

In the \textbf{forward pass}, the activation $\mathbf{a}^{(l)}$ of a layer $l$ yields
\begin{align}
\label{activation}
\mathbf{a}^{(l)} = W^{(l)} \, \mathbf{x}^{(l)} + \mathbf{b}^{(l)},
\end{align}
with the weight matrix $W^{(l)}$, the input $\mathbf{x}^{(l)}$ and the bias $\mathbf{b}^{(l)}$. The output value $\mathbf{y}^{(l)}$ of the layer is then given by the transfer function $\phi$, e.g. a sigmoid
\begin{equation}
\label{transferANN}
\mathbf{y}^{(l)} = \phi(\mathbf{a}^{(l)}) = \frac{1}{1 + e^{(-\beta x)}},
\end{equation}
with a slope parameter $\beta$. The choice of the transfer function is to a certain degree free. A \gls{relu}, a hyperbolic tangent ($\tanh$) or a sigmoid are among the most popular ones and are shown in \cref{deeplearning_activation_functions} for comparison.
However, it is important that the chosen function has a non-linear nature. With a linear transfer function any hidden layer structure becomes redundant, as the layers can be simply merged into a single one.

Depending on the task, the bias as well as an additional noise term can be vital: the individual biases for instance allows the network to adjust the dynamic range of each neuron and the injection of artificial noise can significantly increase the training performance. 

The same principle is then applied to all other layers to complete the forward pass, i.e. the result of the previous layer is the input for the current layer. The weight matrix $W^{\text{(l)}}$ connecting layer $l$ with $l-1$ has the appropriate shape to fit the number of input nodes $n^{\text{(l-1)}}_\text{nodes}$ and output nodes $n^{\text{(l)}}_\text{nodes}$.

\begin{figure}
	\begin{subfigure}[c]{0.5\textwidth}
		\centering
		\caption{}
		\inputpgf{figures}{deeplearning_activation_functions.pgf}
		\label{dltransfer}
	\end{subfigure}
	\begin{subfigure}[c]{0.5\textwidth}
		\centering
		\caption{}
		\inputpgf{figures}{deeplearning_activation_functions_derivative.pgf}	
		\label{dltransfergradient}
	\end{subfigure}
	\caption[Popular shapes for transfer functions in deep learning.]{Popular shapes for transfer functions in deep learning. \textbf{(\subref{dltransfer})}: Some of the most popular shapes for the transfer function are a \gls{relu}, $\tanh$ or sigmoid. \textbf{(\subref{dltransfergradient})}: Training with a linear function would not work, because the gradient is a constant and doesn't discriminate between different inputs. A non-linear transfer function is thus vital to the training. The \gls{relu} is, despite appearing to be a linear function, non-linear.}
	\label{deeplearning_activation_functions}
\end{figure}

In deep learning the \textbf{backward pass} is almost always performed with \gls{sgd}, where a given loss function $\loss(\mathbf{\mathbf{x}, \hat{y}, \mathbf{\theta}})$ is minimized by moving along the negative gradient of the loss with respect to the network's parameters $\mathbf{\theta}$ (\citealp{Goodfellow-et-al-2016}). The updated set  of parameters $\mathbf{\theta}'$ is then given by
\begin{equation}
\mathbf{\theta'} = \mathbf{\theta} - \eta \, \nabla\loss(\mathbf{\mathbf{x}, \hat{y}, \mathbf{\theta}}),
\label{stochasticgradientdescent}
\end{equation}
with the learning rate $\eta$. 

In combination with a cross-entropy loss function, the choice of the sigmoid transfer function becomes convenient when deriving the parameter changes. As an example the update of the weight matrices is computed in the next paragraphs.

In a first step, the derivative of the loss function in the output layer $l\equiv o$ is computed
\begin{equation}
\frac{\partial\mathcal{L}}{\partial \mathbf{y}^{(o)}} = 
- \frac{\hat{\mathbf{y}}}{\mathbf{y}^{(o)}} + 
\frac{1 - \hat{\mathbf{y}}}{1 - \mathbf{y}^{(o)}},
\end{equation}
with the output $\mathbf{y}^{(o)}$ and the target output $\hat{\mathbf{y}}$. The gradient of the loss can then be rewritten in terms of the error $\mathbf{e}^{(o)} = \hat{\mathbf{y}} - \mathbf{y}^{(o)}$ by using the derivative of the transfer function
\begin{align}
\frac{\partial \mathbf{y}^{(o)}}{\partial \mathbf{a}^{(o)}} = \frac{\partial \gls{transfer}(\mathbf{a}^{(o)})}{\partial \mathbf{a}^{(o)}} = \gls{transfer} (1 - \gls{transfer}),\\
\Rightarrow \quad \frac{\partial\mathcal{L}}{\partial \mathbf{a}^{(o)}} =
\frac{\partial\mathcal{L}}{\partial \mathbf{y}^{(o)}} 
\; \frac{\partial \mathbf{y}^{(o)}}{\partial \mathbf{a}^{(o)}} =
\hat{\mathbf{y}} - \mathbf{y}^{(o)} = \mathbf{e}^{(o)}.
\end{align}

According to \cref{stochasticgradientdescent}, the final update of the weight matrix is given by
\begin{equation}
\delta W^{(o)} = - \eta \frac{\partial \mathcal{L}}{\partial W^{(o)}} 
= - \eta \;
\frac{\partial\mathcal{L}}{\partial \mathbf{y}^{(o)}} \;
\frac{\partial \mathbf{a}^{(o)}}{\partial W^{(o)}}
= - \eta \, \left(\mathbf{e}^{(o)} \mathbf{x}^{(o),T}\right),
\label{backpropupdate}
\end{equation}

The computation for the hidden layer ($l\equiv h$) can be done in a similar fashion. Again, the gradient of the loss function is computed
\begin{equation}
\frac{\partial\mathcal{L}}{\partial \mathbf{a}^{(h)}} = \mathbf{e}^{(h)} \;
\frac{\partial \mathbf{y}^{(h)} }{\partial \mathbf{a}^{(h)}},
\end{equation}
and the error of the hidden layer $\mathbf{e}^{(h)}$ is propagated backwards as $\mathbf{e}^{(h)}=W^{(o),T}\mathbf{e}^{(o)}$ yielding a total update of
\begin{equation}
\delta W^{\text{(h)}} = - \eta \;
\left(W^{\text{(o)}T} \mathbf{e}^{(o)}\right) \;
\frac{\partial \mathbf{y}^{(h)} }{\partial \mathbf{a}^{(h)}} \; \mathbf{x}^{(h), T}.
\end{equation}

The backward propagation of the error is name giving and thus it is often referred to as \textit{backpropagation}. Despite the great performance for many deep learning tasks, the biological plausibility of propagating the error signal backward has been questioned ever since.

A simple but effective adjustment was suggested by \citealp{lillicrap2016random} which is also known as \textit{feedback alignment}. Instead of the transpose of the feed-forward weight matrix of the respective layer a fixed random matrix $B$ is chosen to propagate the error backward. Compared to the backpropagation variant from \cref{backpropupdate} the update in the hidden layer changes to
\begin{equation}
\delta W^{(h)} = - \eta \;
(B \mathbf{e}^{(o)}) \;
\frac{\partial \mathbf{y}^{(h)}}{\partial \mathbf{a}^{(h)}} \;
\mathbf{x}^{(h),T}
\end{equation}
The only constraints to $B$ are that $\mathbf{e}^{(o)T} W^{(o)} B \mathbf{e}^{(o)} > 0$ has to be fulfilled on average, meaning that geometrically, the new feedback signal $B \mathbf{e}^{(o)}$ for the hidden layer lies within $90^{\circ}$ of the one used by backprogation, $W^{\text{(o)}T} \mathbf{e}^{(o)}$.

%Q:Überleitung zu biologischem neuron hier?

\section{The Biological Neuron}

Biological neural networks have been a great inspiration for deep learning algorithms and \glspl{ann}. It is estimated that the human brain contains around $10^{11}$ neurons of different shape, size and functions (\citealp{numberofneurons}). By the use of \emph{synapses}, neurons create complex network structures throughout the brain. Usually a neuron has up to $10^4$ \emph{postsynaptic} partners. The connections vary from dense clusters with nearby neurons to linking distant brain regions with each other. 

The intercommunication is established by the use of short electrical pulses (spikes) and neurotransmitters. At most synapses, a small physical gap separates the neurons from each other. An electrical pulse of the \emph{presynaptic} neuron releases various neurotransmitter to overcome this synaptic cleft and once a transmitter has docked to a corresponding receptor on the other side, activated ion channels convert the chemical transmission back into an electrical signal. Depending on the type of neurotransmitters the excitation  can be excitatory or inhibitory. According to Dale's principle a presynaptic neuron releases always the same type of neurotransmitter. To that end, a neuron's output is either inhibitory or excitatory but not both. The input, on the other side, is not restricted to a single type of excitation, as various presynaptic partners can be connected.

On the postsynaptic side, the inputs of all presynaptic partners are then gathered by the \emph{dendrites} before they are forwarded to the \emph{soma} where the information is processed (c.f. \cref{biosynapse}). In particular, the soma integrates the currents induced by the input spikes, which can be tracked by measuring the course of the soma's membrane potential. The impact of each spike corresponds to a \gls{psp}. Once a certain threshold potential is reached, a mechanism gets triggered that initiates a fire response, the so-called action potential or spike. Once a spike has been fired, the neuron's membrane becomes hyperpolarized and decrease even below the resting potential as shown in \cref{actionpotential}. The neuron is in a so called \emph{refractory state} and due to the ongoing hyperpolarization it is hard but not impossible to fire. The action potentials are then relayed by the axon to its connected partners and the described course of communication starts over. 

\begin{figure}
	\begin{subfigure}{0.5\textwidth}
		\centering
		\caption{}
		\includegraphics[width=0.8\linewidth, valign=t]{figures/action_potential.png}
		\label{actionpotential}
	\end{subfigure}
	\begin{subfigure}{0.5\textwidth}
		\centering
		\caption{}
		\vspace{0.5cm}
		\includegraphics[width=\linewidth, valign=t]{figures/Neuron.pdf}
		\vspace{1.5cm}	
		\label{biosynapse}
	\end{subfigure}
	\caption[Schematics of an action potential and a biological neuron]{Schematics of a biological neuron and an action potential. \textbf{(\subref{actionpotential})} The response of neuron after exceeding the threshold is called action potential. After a phase of depolarization, the membrane repolarizes and enters the refractory state. Figure taken from \citealp{picture_actionpotential}. \textbf{(\subref{biosynapse})} Schematics of a biological neuron can be split into three main functional parts: the dendrites which are responsible to collect all inputs, the soma integrates the evoked \glspl{psp} and eventually triggers a fire response and the axon relays the fire responses to other neurons. Figure adapted from \citealp{picture_neuron}.}
	\label{biologicalneuron}
	
\end{figure}

Over time the unbalanced ion concentration in and outside the membrane is restored by the membranes permeability and additional ion pumps. In the equilibrated state, the membrane potential is then referred to as the \emph{resting potential}. The neurotransmitters are recycled as well. Otherwise, if their resources were depleted and heavily used synapses would be quickly silenced.

A wide-spread assumption in the field of neuroscience is that the exact shape of a spike doesn't carry any relevant information and therefore all spikes can be modeled by a stereotypical shape. The communication between neurons is then encoded in the frequency (\emph{rate coding}) or timing (\emph{time coding}) of exciting and inhibiting spikes. A more detailed description of neural coding schemes is presented in section \ref{neuralcoding}. However, recent research has already suggested that the variate of action potential contains vital information (\citealp{debanne2013mechanisms}).

The brain's ability to continuously change the topology of its synaptic wiring, to create new synapses, to alter the chemical properties of the synaptic receptors or to simply strengthen and weaken the synaptic efficacy, allows it to learn and react as a response to stimulation and even brain damage.

One way of learning and forming memory is known as synaptic plasticity where the synaptic strength is changed over time. According to Hebb's theory ``neurons that fire together wire together" (\citealp{hebb1949organization}). An experimental proof of such activity-dependent plasticity was found by \citealp{bliss1973long}, where they discovered that a short but high frequency stimulation lead to a long lasting change in the synapse's efficacy. This is also referred to as \gls{ltp}. Reducing the stimulus to a low frequency, on the other hand, resulted in the opposite effect: \gls{ltd}. In combination they can carve out certain region in the brain which is related  to a stimulus and thereby create memory. 

A better understanding of \gls{ltp} and \gls{ltd} was provided by the introduction of \gls{stdp}. \gls{stdp} shows in principle that a presynaptic activity just before a postsynaptic response leads to an increase synpatic strength and if presynaptic activity occurs right after the postsynaptic one, it results in the inverse effect (\citealp{poo98stdp}). 

%It is worth mentioning that synaptic plasticity between two neurons can also be induced by activity of an independent pathway. 

Developing biologically inspired and plausible learning algorithms is a dedicated goal in the field of modern neuroscience. In the next section a practicable neuron model is presented, which is the basis of the experimental part in this thesis.

\section{\gls{lif} Model}

An early but successful description of the biological neuron dynamics was accomplished by the \gls{lif} neuron, first described by \citealp{lapicque1907recherches}. Despite some strong simplifications, the main dynamics of the membrane potential are well described by the model and it thus has been a popular and portable choice for neuromorphic hardware implementations.

In biology, the observation of similar shaped individual action potentials lead to the assumption that the shape of an spike does not transport any information. The \gls{lif} model is based upon this theory and thus every spike can be replaced by a stereotypical shape.

Another observation in biology is, that neurons vary much in their shape and size fulfilling different functions. The spatial component plays an important role for the dynamics of a neuron. For instance, the strategic positioning of certain excitatory or inhibitory inputs on the dendrites, either closely or further away from the soma, give rise to non-linear behavior in the course of the membrane potential. However, a extensive spatial dependency is difficult and costly to implement in a model. Therefore, the \gls{lif} neuron neglects the topology of the neuron and is approximated as a point-like integrator. 

In the model, the incoming spike trains $S_j(t)$ from various presynaptic partners $j$ are described by a series of spikes $s$ at times $t_j^{(s)}$
\begin{equation}
S_j(t) = \sum_s \delta(t - t_j^{(s)}),
\end{equation}
with the $\delta$-function denoted as $\delta$. 

Each spike of the input spike train evokes a \gls{psp}. The impact of the \glspl{psp} depends on the individual synaptic weights $w_j$. For simplicity, the excitatory or inhibitory nature of the synapses is encoded by a sign in the synaptic weight as well. Summing over all input sources yields a total synaptic input current that is seen by the postsynpatic neuron
\begin{equation}
\gls{isyn}(t) = \sum_j w_j \left(\epsilon \ast S_j(t)\right),
\label{synpatic_input}
\end{equation}
with an exponential kernel $\epsilon$ describing the shape of the \gls{psp}. Popular choices for the kernel are single or double-exponentials
\begin{align}
\epsilon_\text{double}(t) 	&=\frac{1}{\mathcal{N}} \left(\epsilon_\text{rise} \ast \epsilon_\text{fall}\right)(t) \\
&=\frac{1}{\mathcal{N}}\exp \left(-\frac{t}{\tau_\text{rise}} \right)  \ast \exp \left(-\frac{t}{\tau_\text{fall}} \right) 
\label{exponentialkernels)}
\end{align}
with a rising and falling temporal constant $\tau_\text{rise}$ and $\tau_\text{fall}$ respectively. A constant $\mathcal{N}$ norms the kernel to unity. As the rising constant goes to zero $\tau_\text{rise} \rightarrow 0$ the double exponential turns into a single exponential kernel $\epsilon_\text{single} = \epsilon_\text{fall}$.

The membrane potential \gls{v_mem} changes with the continuous synaptic input causing an unbalanced ion concentration inside the membrane. Passive as well as active processes are permanently restoring the membrane potential back to its equilibrium state which is associated with the resting potential \gls{v_leak}. In the \gls{lif} model, the temporal scale of these restoring processes defined by the membranes capacitance $C_\text{m}$ and the leakage conductance $g_\text{leak}$ yielding the membrane's time constant $\gls{tau_m} = \frac{C_\text{m}}{g_\text{leak}}$. The dynamics of membrane are then given by a single differential equation
\begin{align}
\label{lifeq}
C_{\text{m}} \frac{d\gls{v_mem}}{dt} &= -g_{\text{leak}} (\gls{v_mem} - \gls{v_leak}) + \gls{isyn}.
\end{align}

As for a biological neuron, the postsynaptic \gls{lif} neuron $i$ triggers a spike once a certain threshold \gls{thres} is crossed following the condition
\begin{equation}
V_{\text{m}, i}\left(t_i^{(s)}\right) \ge \gls{thres} \Leftrightarrow \text{neuron i fires at time } t_i^{(s)}.
\end{equation}

Then the membrane is set to a reset potential \gls{v_reset} where it remains unchanged for a refractory period of \gls{refrac}
\begin{equation}
V_{\text{m}, i}(t) = \gls{v_reset} \quad \forall t \in \left(t_i^{(s)}, t_i^{(s)} + \gls{refrac}\right].
\end{equation}
Unlike for its biological counterpart, the modeled neuron cannot spike during the refractory period.

\begin{figure}
	\centering
	\scalebox{0.93}{%% Creator: Matplotlib, PGF backend
%%
%% To include the figure in your LaTeX document, write
%%   \input{<filename>.pgf}
%%
%% Make sure the required packages are loaded in your preamble
%%   \usepackage{pgf}
%%
%% Figures using additional raster images can only be included by \input if
%% they are in the same directory as the main LaTeX file. For loading figures
%% from other directories you can use the `import` package
%%   \usepackage{import}
%% and then include the figures with
%%   \import{<path to file>}{<filename>.pgf}
%%
%% Matplotlib used the following preamble
%%   \usepackage[utf8]{inputenc}
%%   \usepackage[T1]{fontenc}
%%   \usepackage{textcomp}
%%
\begingroup%
\makeatletter%
\begin{pgfpicture}%
\pgfpathrectangle{\pgfpointorigin}{\pgfqpoint{4.902605in}{2.799288in}}%
\pgfusepath{use as bounding box, clip}%
\begin{pgfscope}%
\pgfsetbuttcap%
\pgfsetmiterjoin%
\pgfsetlinewidth{0.000000pt}%
\definecolor{currentstroke}{rgb}{0.000000,0.000000,0.000000}%
\pgfsetstrokecolor{currentstroke}%
\pgfsetstrokeopacity{0.000000}%
\pgfsetdash{}{0pt}%
\pgfpathmoveto{\pgfqpoint{0.000000in}{0.000000in}}%
\pgfpathlineto{\pgfqpoint{4.902605in}{0.000000in}}%
\pgfpathlineto{\pgfqpoint{4.902605in}{2.799288in}}%
\pgfpathlineto{\pgfqpoint{0.000000in}{2.799288in}}%
\pgfpathclose%
\pgfusepath{}%
\end{pgfscope}%
\begin{pgfscope}%
\pgfsetbuttcap%
\pgfsetmiterjoin%
\pgfsetlinewidth{0.000000pt}%
\definecolor{currentstroke}{rgb}{0.000000,0.000000,0.000000}%
\pgfsetstrokecolor{currentstroke}%
\pgfsetstrokeopacity{0.000000}%
\pgfsetdash{}{0pt}%
\pgfpathmoveto{\pgfqpoint{0.585105in}{2.064876in}}%
\pgfpathlineto{\pgfqpoint{4.847605in}{2.064876in}}%
\pgfpathlineto{\pgfqpoint{4.847605in}{2.744288in}}%
\pgfpathlineto{\pgfqpoint{0.585105in}{2.744288in}}%
\pgfpathclose%
\pgfusepath{}%
\end{pgfscope}%
\begin{pgfscope}%
\pgfsetbuttcap%
\pgfsetroundjoin%
\definecolor{currentfill}{rgb}{0.200000,0.200000,0.200000}%
\pgfsetfillcolor{currentfill}%
\pgfsetlinewidth{0.803000pt}%
\definecolor{currentstroke}{rgb}{0.200000,0.200000,0.200000}%
\pgfsetstrokecolor{currentstroke}%
\pgfsetdash{}{0pt}%
\pgfsys@defobject{currentmarker}{\pgfqpoint{0.000000in}{-0.048611in}}{\pgfqpoint{0.000000in}{0.000000in}}{%
\pgfpathmoveto{\pgfqpoint{0.000000in}{0.000000in}}%
\pgfpathlineto{\pgfqpoint{0.000000in}{-0.048611in}}%
\pgfusepath{stroke,fill}%
}%
\begin{pgfscope}%
\pgfsys@transformshift{0.585105in}{2.064876in}%
\pgfsys@useobject{currentmarker}{}%
\end{pgfscope}%
\end{pgfscope}%
\begin{pgfscope}%
\pgfsetbuttcap%
\pgfsetroundjoin%
\definecolor{currentfill}{rgb}{0.200000,0.200000,0.200000}%
\pgfsetfillcolor{currentfill}%
\pgfsetlinewidth{0.803000pt}%
\definecolor{currentstroke}{rgb}{0.200000,0.200000,0.200000}%
\pgfsetstrokecolor{currentstroke}%
\pgfsetdash{}{0pt}%
\pgfsys@defobject{currentmarker}{\pgfqpoint{0.000000in}{-0.048611in}}{\pgfqpoint{0.000000in}{0.000000in}}{%
\pgfpathmoveto{\pgfqpoint{0.000000in}{0.000000in}}%
\pgfpathlineto{\pgfqpoint{0.000000in}{-0.048611in}}%
\pgfusepath{stroke,fill}%
}%
\begin{pgfscope}%
\pgfsys@transformshift{1.153438in}{2.064876in}%
\pgfsys@useobject{currentmarker}{}%
\end{pgfscope}%
\end{pgfscope}%
\begin{pgfscope}%
\pgfsetbuttcap%
\pgfsetroundjoin%
\definecolor{currentfill}{rgb}{0.200000,0.200000,0.200000}%
\pgfsetfillcolor{currentfill}%
\pgfsetlinewidth{0.803000pt}%
\definecolor{currentstroke}{rgb}{0.200000,0.200000,0.200000}%
\pgfsetstrokecolor{currentstroke}%
\pgfsetdash{}{0pt}%
\pgfsys@defobject{currentmarker}{\pgfqpoint{0.000000in}{-0.048611in}}{\pgfqpoint{0.000000in}{0.000000in}}{%
\pgfpathmoveto{\pgfqpoint{0.000000in}{0.000000in}}%
\pgfpathlineto{\pgfqpoint{0.000000in}{-0.048611in}}%
\pgfusepath{stroke,fill}%
}%
\begin{pgfscope}%
\pgfsys@transformshift{1.721772in}{2.064876in}%
\pgfsys@useobject{currentmarker}{}%
\end{pgfscope}%
\end{pgfscope}%
\begin{pgfscope}%
\pgfsetbuttcap%
\pgfsetroundjoin%
\definecolor{currentfill}{rgb}{0.200000,0.200000,0.200000}%
\pgfsetfillcolor{currentfill}%
\pgfsetlinewidth{0.803000pt}%
\definecolor{currentstroke}{rgb}{0.200000,0.200000,0.200000}%
\pgfsetstrokecolor{currentstroke}%
\pgfsetdash{}{0pt}%
\pgfsys@defobject{currentmarker}{\pgfqpoint{0.000000in}{-0.048611in}}{\pgfqpoint{0.000000in}{0.000000in}}{%
\pgfpathmoveto{\pgfqpoint{0.000000in}{0.000000in}}%
\pgfpathlineto{\pgfqpoint{0.000000in}{-0.048611in}}%
\pgfusepath{stroke,fill}%
}%
\begin{pgfscope}%
\pgfsys@transformshift{2.290105in}{2.064876in}%
\pgfsys@useobject{currentmarker}{}%
\end{pgfscope}%
\end{pgfscope}%
\begin{pgfscope}%
\pgfsetbuttcap%
\pgfsetroundjoin%
\definecolor{currentfill}{rgb}{0.200000,0.200000,0.200000}%
\pgfsetfillcolor{currentfill}%
\pgfsetlinewidth{0.803000pt}%
\definecolor{currentstroke}{rgb}{0.200000,0.200000,0.200000}%
\pgfsetstrokecolor{currentstroke}%
\pgfsetdash{}{0pt}%
\pgfsys@defobject{currentmarker}{\pgfqpoint{0.000000in}{-0.048611in}}{\pgfqpoint{0.000000in}{0.000000in}}{%
\pgfpathmoveto{\pgfqpoint{0.000000in}{0.000000in}}%
\pgfpathlineto{\pgfqpoint{0.000000in}{-0.048611in}}%
\pgfusepath{stroke,fill}%
}%
\begin{pgfscope}%
\pgfsys@transformshift{2.858438in}{2.064876in}%
\pgfsys@useobject{currentmarker}{}%
\end{pgfscope}%
\end{pgfscope}%
\begin{pgfscope}%
\pgfsetbuttcap%
\pgfsetroundjoin%
\definecolor{currentfill}{rgb}{0.200000,0.200000,0.200000}%
\pgfsetfillcolor{currentfill}%
\pgfsetlinewidth{0.803000pt}%
\definecolor{currentstroke}{rgb}{0.200000,0.200000,0.200000}%
\pgfsetstrokecolor{currentstroke}%
\pgfsetdash{}{0pt}%
\pgfsys@defobject{currentmarker}{\pgfqpoint{0.000000in}{-0.048611in}}{\pgfqpoint{0.000000in}{0.000000in}}{%
\pgfpathmoveto{\pgfqpoint{0.000000in}{0.000000in}}%
\pgfpathlineto{\pgfqpoint{0.000000in}{-0.048611in}}%
\pgfusepath{stroke,fill}%
}%
\begin{pgfscope}%
\pgfsys@transformshift{3.426772in}{2.064876in}%
\pgfsys@useobject{currentmarker}{}%
\end{pgfscope}%
\end{pgfscope}%
\begin{pgfscope}%
\pgfsetbuttcap%
\pgfsetroundjoin%
\definecolor{currentfill}{rgb}{0.200000,0.200000,0.200000}%
\pgfsetfillcolor{currentfill}%
\pgfsetlinewidth{0.803000pt}%
\definecolor{currentstroke}{rgb}{0.200000,0.200000,0.200000}%
\pgfsetstrokecolor{currentstroke}%
\pgfsetdash{}{0pt}%
\pgfsys@defobject{currentmarker}{\pgfqpoint{0.000000in}{-0.048611in}}{\pgfqpoint{0.000000in}{0.000000in}}{%
\pgfpathmoveto{\pgfqpoint{0.000000in}{0.000000in}}%
\pgfpathlineto{\pgfqpoint{0.000000in}{-0.048611in}}%
\pgfusepath{stroke,fill}%
}%
\begin{pgfscope}%
\pgfsys@transformshift{3.995105in}{2.064876in}%
\pgfsys@useobject{currentmarker}{}%
\end{pgfscope}%
\end{pgfscope}%
\begin{pgfscope}%
\pgfsetbuttcap%
\pgfsetroundjoin%
\definecolor{currentfill}{rgb}{0.200000,0.200000,0.200000}%
\pgfsetfillcolor{currentfill}%
\pgfsetlinewidth{0.803000pt}%
\definecolor{currentstroke}{rgb}{0.200000,0.200000,0.200000}%
\pgfsetstrokecolor{currentstroke}%
\pgfsetdash{}{0pt}%
\pgfsys@defobject{currentmarker}{\pgfqpoint{0.000000in}{-0.048611in}}{\pgfqpoint{0.000000in}{0.000000in}}{%
\pgfpathmoveto{\pgfqpoint{0.000000in}{0.000000in}}%
\pgfpathlineto{\pgfqpoint{0.000000in}{-0.048611in}}%
\pgfusepath{stroke,fill}%
}%
\begin{pgfscope}%
\pgfsys@transformshift{4.563438in}{2.064876in}%
\pgfsys@useobject{currentmarker}{}%
\end{pgfscope}%
\end{pgfscope}%
\begin{pgfscope}%
\pgfpathrectangle{\pgfqpoint{0.585105in}{2.064876in}}{\pgfqpoint{4.262500in}{0.679412in}} %
\pgfusepath{clip}%
\pgfsetrectcap%
\pgfsetroundjoin%
\pgfsetlinewidth{0.803000pt}%
\definecolor{currentstroke}{rgb}{0.690196,0.690196,0.690196}%
\pgfsetstrokecolor{currentstroke}%
\pgfsetstrokeopacity{0.300000}%
\pgfsetdash{}{0pt}%
\pgfpathmoveto{\pgfqpoint{0.585105in}{2.214661in}}%
\pgfpathlineto{\pgfqpoint{4.847605in}{2.214661in}}%
\pgfusepath{stroke}%
\end{pgfscope}%
\begin{pgfscope}%
\pgfsetbuttcap%
\pgfsetroundjoin%
\definecolor{currentfill}{rgb}{0.200000,0.200000,0.200000}%
\pgfsetfillcolor{currentfill}%
\pgfsetlinewidth{0.803000pt}%
\definecolor{currentstroke}{rgb}{0.200000,0.200000,0.200000}%
\pgfsetstrokecolor{currentstroke}%
\pgfsetdash{}{0pt}%
\pgfsys@defobject{currentmarker}{\pgfqpoint{-0.048611in}{0.000000in}}{\pgfqpoint{0.000000in}{0.000000in}}{%
\pgfpathmoveto{\pgfqpoint{0.000000in}{0.000000in}}%
\pgfpathlineto{\pgfqpoint{-0.048611in}{0.000000in}}%
\pgfusepath{stroke,fill}%
}%
\begin{pgfscope}%
\pgfsys@transformshift{0.585105in}{2.214661in}%
\pgfsys@useobject{currentmarker}{}%
\end{pgfscope}%
\end{pgfscope}%
\begin{pgfscope}%
\definecolor{textcolor}{rgb}{0.200000,0.200000,0.200000}%
\pgfsetstrokecolor{textcolor}%
\pgfsetfillcolor{textcolor}%
\pgftext[x=0.240968in,y=2.166833in,left,base]{\color{textcolor}\rmfamily\fontsize{10.000000}{12.000000}\selectfont \(\displaystyle -60\)}%
\end{pgfscope}%
\begin{pgfscope}%
\pgfpathrectangle{\pgfqpoint{0.585105in}{2.064876in}}{\pgfqpoint{4.262500in}{0.679412in}} %
\pgfusepath{clip}%
\pgfsetrectcap%
\pgfsetroundjoin%
\pgfsetlinewidth{0.803000pt}%
\definecolor{currentstroke}{rgb}{0.690196,0.690196,0.690196}%
\pgfsetstrokecolor{currentstroke}%
\pgfsetstrokeopacity{0.300000}%
\pgfsetdash{}{0pt}%
\pgfpathmoveto{\pgfqpoint{0.585105in}{2.690272in}}%
\pgfpathlineto{\pgfqpoint{4.847605in}{2.690272in}}%
\pgfusepath{stroke}%
\end{pgfscope}%
\begin{pgfscope}%
\pgfsetbuttcap%
\pgfsetroundjoin%
\definecolor{currentfill}{rgb}{0.200000,0.200000,0.200000}%
\pgfsetfillcolor{currentfill}%
\pgfsetlinewidth{0.803000pt}%
\definecolor{currentstroke}{rgb}{0.200000,0.200000,0.200000}%
\pgfsetstrokecolor{currentstroke}%
\pgfsetdash{}{0pt}%
\pgfsys@defobject{currentmarker}{\pgfqpoint{-0.048611in}{0.000000in}}{\pgfqpoint{0.000000in}{0.000000in}}{%
\pgfpathmoveto{\pgfqpoint{0.000000in}{0.000000in}}%
\pgfpathlineto{\pgfqpoint{-0.048611in}{0.000000in}}%
\pgfusepath{stroke,fill}%
}%
\begin{pgfscope}%
\pgfsys@transformshift{0.585105in}{2.690272in}%
\pgfsys@useobject{currentmarker}{}%
\end{pgfscope}%
\end{pgfscope}%
\begin{pgfscope}%
\definecolor{textcolor}{rgb}{0.200000,0.200000,0.200000}%
\pgfsetstrokecolor{textcolor}%
\pgfsetfillcolor{textcolor}%
\pgftext[x=0.240968in,y=2.642444in,left,base]{\color{textcolor}\rmfamily\fontsize{10.000000}{12.000000}\selectfont \(\displaystyle -40\)}%
\end{pgfscope}%
\begin{pgfscope}%
\pgftext[x=0.185413in,y=2.404582in,,bottom,rotate=90.000000]{\rmfamily\fontsize{10.000000}{12.000000}\selectfont \(\displaystyle V^\mathrm{LIF}_\mathrm{m}\) [mV]}%
\end{pgfscope}%
\begin{pgfscope}%
\pgfpathrectangle{\pgfqpoint{0.585105in}{2.064876in}}{\pgfqpoint{4.262500in}{0.679412in}} %
\pgfusepath{clip}%
\pgfsetrectcap%
\pgfsetroundjoin%
\pgfsetlinewidth{0.501875pt}%
\definecolor{currentstroke}{rgb}{0.400000,0.400000,0.400000}%
\pgfsetstrokecolor{currentstroke}%
\pgfsetdash{}{0pt}%
\pgfpathmoveto{\pgfqpoint{0.585105in}{2.214661in}}%
\pgfpathlineto{\pgfqpoint{1.153438in}{2.214661in}}%
\pgfpathlineto{\pgfqpoint{1.165373in}{2.248444in}}%
\pgfpathlineto{\pgfqpoint{1.177308in}{2.275828in}}%
\pgfpathlineto{\pgfqpoint{1.189812in}{2.298970in}}%
\pgfpathlineto{\pgfqpoint{1.202315in}{2.317542in}}%
\pgfpathlineto{\pgfqpoint{1.215387in}{2.333049in}}%
\pgfpathlineto{\pgfqpoint{1.229027in}{2.345844in}}%
\pgfpathlineto{\pgfqpoint{1.243803in}{2.356644in}}%
\pgfpathlineto{\pgfqpoint{1.259717in}{2.365526in}}%
\pgfpathlineto{\pgfqpoint{1.277903in}{2.373054in}}%
\pgfpathlineto{\pgfqpoint{1.298932in}{2.379227in}}%
\pgfpathlineto{\pgfqpoint{1.323938in}{2.384135in}}%
\pgfpathlineto{\pgfqpoint{1.355197in}{2.387892in}}%
\pgfpathlineto{\pgfqpoint{1.397253in}{2.390571in}}%
\pgfpathlineto{\pgfqpoint{1.462043in}{2.392233in}}%
\pgfpathlineto{\pgfqpoint{1.594465in}{2.392939in}}%
\pgfpathlineto{\pgfqpoint{1.727455in}{2.393008in}}%
\pgfpathlineto{\pgfqpoint{1.739390in}{2.359226in}}%
\pgfpathlineto{\pgfqpoint{1.751325in}{2.331843in}}%
\pgfpathlineto{\pgfqpoint{1.763828in}{2.308702in}}%
\pgfpathlineto{\pgfqpoint{1.776332in}{2.290131in}}%
\pgfpathlineto{\pgfqpoint{1.789403in}{2.274624in}}%
\pgfpathlineto{\pgfqpoint{1.803043in}{2.261830in}}%
\pgfpathlineto{\pgfqpoint{1.817820in}{2.251031in}}%
\pgfpathlineto{\pgfqpoint{1.833733in}{2.242149in}}%
\pgfpathlineto{\pgfqpoint{1.851920in}{2.234621in}}%
\pgfpathlineto{\pgfqpoint{1.872948in}{2.228448in}}%
\pgfpathlineto{\pgfqpoint{1.897955in}{2.223540in}}%
\pgfpathlineto{\pgfqpoint{1.929213in}{2.219784in}}%
\pgfpathlineto{\pgfqpoint{1.971270in}{2.217105in}}%
\pgfpathlineto{\pgfqpoint{2.036060in}{2.215443in}}%
\pgfpathlineto{\pgfqpoint{2.168482in}{2.214737in}}%
\pgfpathlineto{\pgfqpoint{2.869805in}{2.214661in}}%
\pgfpathlineto{\pgfqpoint{2.881740in}{2.263309in}}%
\pgfpathlineto{\pgfqpoint{2.893675in}{2.302742in}}%
\pgfpathlineto{\pgfqpoint{2.905610in}{2.334705in}}%
\pgfpathlineto{\pgfqpoint{2.917545in}{2.360615in}}%
\pgfpathlineto{\pgfqpoint{2.930048in}{2.382511in}}%
\pgfpathlineto{\pgfqpoint{2.942552in}{2.400083in}}%
\pgfpathlineto{\pgfqpoint{2.955623in}{2.414754in}}%
\pgfpathlineto{\pgfqpoint{2.969832in}{2.427305in}}%
\pgfpathlineto{\pgfqpoint{2.984608in}{2.437421in}}%
\pgfpathlineto{\pgfqpoint{3.001090in}{2.445997in}}%
\pgfpathlineto{\pgfqpoint{3.017572in}{2.452415in}}%
\pgfpathlineto{\pgfqpoint{3.018708in}{2.095759in}}%
\pgfpathlineto{\pgfqpoint{3.074405in}{2.095759in}}%
\pgfpathlineto{\pgfqpoint{3.086340in}{2.166928in}}%
\pgfpathlineto{\pgfqpoint{3.098275in}{2.224617in}}%
\pgfpathlineto{\pgfqpoint{3.110210in}{2.271379in}}%
\pgfpathlineto{\pgfqpoint{3.122145in}{2.309283in}}%
\pgfpathlineto{\pgfqpoint{3.134080in}{2.340008in}}%
\pgfpathlineto{\pgfqpoint{3.146015in}{2.364913in}}%
\pgfpathlineto{\pgfqpoint{3.158518in}{2.385960in}}%
\pgfpathlineto{\pgfqpoint{3.171590in}{2.403534in}}%
\pgfpathlineto{\pgfqpoint{3.184662in}{2.417497in}}%
\pgfpathlineto{\pgfqpoint{3.198870in}{2.429440in}}%
\pgfpathlineto{\pgfqpoint{3.214215in}{2.439390in}}%
\pgfpathlineto{\pgfqpoint{3.231265in}{2.447710in}}%
\pgfpathlineto{\pgfqpoint{3.243768in}{2.452406in}}%
\pgfpathlineto{\pgfqpoint{3.244905in}{2.095759in}}%
\pgfpathlineto{\pgfqpoint{3.300602in}{2.095759in}}%
\pgfpathlineto{\pgfqpoint{3.312537in}{2.166928in}}%
\pgfpathlineto{\pgfqpoint{3.324472in}{2.224617in}}%
\pgfpathlineto{\pgfqpoint{3.336407in}{2.271379in}}%
\pgfpathlineto{\pgfqpoint{3.348342in}{2.309283in}}%
\pgfpathlineto{\pgfqpoint{3.360277in}{2.340008in}}%
\pgfpathlineto{\pgfqpoint{3.372212in}{2.364913in}}%
\pgfpathlineto{\pgfqpoint{3.384715in}{2.385960in}}%
\pgfpathlineto{\pgfqpoint{3.397787in}{2.403534in}}%
\pgfpathlineto{\pgfqpoint{3.410858in}{2.417497in}}%
\pgfpathlineto{\pgfqpoint{3.425067in}{2.429440in}}%
\pgfpathlineto{\pgfqpoint{3.440412in}{2.439390in}}%
\pgfpathlineto{\pgfqpoint{3.457462in}{2.447710in}}%
\pgfpathlineto{\pgfqpoint{3.469965in}{2.452406in}}%
\pgfpathlineto{\pgfqpoint{3.471102in}{2.095759in}}%
\pgfpathlineto{\pgfqpoint{3.526798in}{2.095759in}}%
\pgfpathlineto{\pgfqpoint{3.538733in}{2.166928in}}%
\pgfpathlineto{\pgfqpoint{3.550668in}{2.224617in}}%
\pgfpathlineto{\pgfqpoint{3.562603in}{2.271379in}}%
\pgfpathlineto{\pgfqpoint{3.574538in}{2.309283in}}%
\pgfpathlineto{\pgfqpoint{3.586473in}{2.340008in}}%
\pgfpathlineto{\pgfqpoint{3.598408in}{2.364913in}}%
\pgfpathlineto{\pgfqpoint{3.610912in}{2.385960in}}%
\pgfpathlineto{\pgfqpoint{3.623983in}{2.403534in}}%
\pgfpathlineto{\pgfqpoint{3.637055in}{2.417497in}}%
\pgfpathlineto{\pgfqpoint{3.651263in}{2.429440in}}%
\pgfpathlineto{\pgfqpoint{3.666608in}{2.439390in}}%
\pgfpathlineto{\pgfqpoint{3.683658in}{2.447710in}}%
\pgfpathlineto{\pgfqpoint{3.696162in}{2.452406in}}%
\pgfpathlineto{\pgfqpoint{3.697298in}{2.095759in}}%
\pgfpathlineto{\pgfqpoint{3.752995in}{2.095759in}}%
\pgfpathlineto{\pgfqpoint{3.764930in}{2.166928in}}%
\pgfpathlineto{\pgfqpoint{3.776865in}{2.224617in}}%
\pgfpathlineto{\pgfqpoint{3.788800in}{2.271379in}}%
\pgfpathlineto{\pgfqpoint{3.800735in}{2.309283in}}%
\pgfpathlineto{\pgfqpoint{3.812670in}{2.340008in}}%
\pgfpathlineto{\pgfqpoint{3.824605in}{2.364913in}}%
\pgfpathlineto{\pgfqpoint{3.837108in}{2.385960in}}%
\pgfpathlineto{\pgfqpoint{3.850180in}{2.403534in}}%
\pgfpathlineto{\pgfqpoint{3.863252in}{2.417497in}}%
\pgfpathlineto{\pgfqpoint{3.877460in}{2.429440in}}%
\pgfpathlineto{\pgfqpoint{3.892805in}{2.439390in}}%
\pgfpathlineto{\pgfqpoint{3.909855in}{2.447710in}}%
\pgfpathlineto{\pgfqpoint{3.922358in}{2.452406in}}%
\pgfpathlineto{\pgfqpoint{3.923495in}{2.095759in}}%
\pgfpathlineto{\pgfqpoint{3.979192in}{2.095759in}}%
\pgfpathlineto{\pgfqpoint{3.991127in}{2.166928in}}%
\pgfpathlineto{\pgfqpoint{4.003062in}{2.224617in}}%
\pgfpathlineto{\pgfqpoint{4.012155in}{2.261119in}}%
\pgfpathlineto{\pgfqpoint{4.026932in}{2.250482in}}%
\pgfpathlineto{\pgfqpoint{4.042845in}{2.241734in}}%
\pgfpathlineto{\pgfqpoint{4.061032in}{2.234320in}}%
\pgfpathlineto{\pgfqpoint{4.082060in}{2.228240in}}%
\pgfpathlineto{\pgfqpoint{4.107067in}{2.223407in}}%
\pgfpathlineto{\pgfqpoint{4.138893in}{2.219657in}}%
\pgfpathlineto{\pgfqpoint{4.181518in}{2.217021in}}%
\pgfpathlineto{\pgfqpoint{4.247445in}{2.215401in}}%
\pgfpathlineto{\pgfqpoint{4.384982in}{2.214727in}}%
\pgfpathlineto{\pgfqpoint{4.848173in}{2.214661in}}%
\pgfpathlineto{\pgfqpoint{4.848173in}{2.214661in}}%
\pgfusepath{stroke}%
\end{pgfscope}%
\begin{pgfscope}%
\pgfsetrectcap%
\pgfsetmiterjoin%
\pgfsetlinewidth{0.803000pt}%
\definecolor{currentstroke}{rgb}{0.200000,0.200000,0.200000}%
\pgfsetstrokecolor{currentstroke}%
\pgfsetdash{}{0pt}%
\pgfpathmoveto{\pgfqpoint{0.585105in}{2.064876in}}%
\pgfpathlineto{\pgfqpoint{0.585105in}{2.744288in}}%
\pgfusepath{stroke}%
\end{pgfscope}%
\begin{pgfscope}%
\pgfsetrectcap%
\pgfsetmiterjoin%
\pgfsetlinewidth{0.803000pt}%
\definecolor{currentstroke}{rgb}{0.200000,0.200000,0.200000}%
\pgfsetstrokecolor{currentstroke}%
\pgfsetdash{}{0pt}%
\pgfpathmoveto{\pgfqpoint{0.585105in}{2.064876in}}%
\pgfpathlineto{\pgfqpoint{4.847605in}{2.064876in}}%
\pgfusepath{stroke}%
\end{pgfscope}%
\begin{pgfscope}%
\definecolor{textcolor}{rgb}{0.200000,0.200000,0.200000}%
\pgfsetstrokecolor{textcolor}%
\pgfsetfillcolor{textcolor}%
\pgftext[x=0.670355in,y=2.642376in,left,top]{\color{textcolor}\rmfamily\fontsize{10.000000}{12.000000}\selectfont (a)}%
\end{pgfscope}%
\begin{pgfscope}%
\pgfsetbuttcap%
\pgfsetmiterjoin%
\pgfsetlinewidth{0.000000pt}%
\definecolor{currentstroke}{rgb}{0.000000,0.000000,0.000000}%
\pgfsetstrokecolor{currentstroke}%
\pgfsetstrokeopacity{0.000000}%
\pgfsetdash{}{0pt}%
\pgfpathmoveto{\pgfqpoint{0.585105in}{1.249582in}}%
\pgfpathlineto{\pgfqpoint{4.847605in}{1.249582in}}%
\pgfpathlineto{\pgfqpoint{4.847605in}{1.928994in}}%
\pgfpathlineto{\pgfqpoint{0.585105in}{1.928994in}}%
\pgfpathclose%
\pgfusepath{}%
\end{pgfscope}%
\begin{pgfscope}%
\pgfsetbuttcap%
\pgfsetroundjoin%
\definecolor{currentfill}{rgb}{0.200000,0.200000,0.200000}%
\pgfsetfillcolor{currentfill}%
\pgfsetlinewidth{0.803000pt}%
\definecolor{currentstroke}{rgb}{0.200000,0.200000,0.200000}%
\pgfsetstrokecolor{currentstroke}%
\pgfsetdash{}{0pt}%
\pgfsys@defobject{currentmarker}{\pgfqpoint{0.000000in}{-0.048611in}}{\pgfqpoint{0.000000in}{0.000000in}}{%
\pgfpathmoveto{\pgfqpoint{0.000000in}{0.000000in}}%
\pgfpathlineto{\pgfqpoint{0.000000in}{-0.048611in}}%
\pgfusepath{stroke,fill}%
}%
\begin{pgfscope}%
\pgfsys@transformshift{0.585105in}{1.249582in}%
\pgfsys@useobject{currentmarker}{}%
\end{pgfscope}%
\end{pgfscope}%
\begin{pgfscope}%
\pgfsetbuttcap%
\pgfsetroundjoin%
\definecolor{currentfill}{rgb}{0.200000,0.200000,0.200000}%
\pgfsetfillcolor{currentfill}%
\pgfsetlinewidth{0.803000pt}%
\definecolor{currentstroke}{rgb}{0.200000,0.200000,0.200000}%
\pgfsetstrokecolor{currentstroke}%
\pgfsetdash{}{0pt}%
\pgfsys@defobject{currentmarker}{\pgfqpoint{0.000000in}{-0.048611in}}{\pgfqpoint{0.000000in}{0.000000in}}{%
\pgfpathmoveto{\pgfqpoint{0.000000in}{0.000000in}}%
\pgfpathlineto{\pgfqpoint{0.000000in}{-0.048611in}}%
\pgfusepath{stroke,fill}%
}%
\begin{pgfscope}%
\pgfsys@transformshift{1.153438in}{1.249582in}%
\pgfsys@useobject{currentmarker}{}%
\end{pgfscope}%
\end{pgfscope}%
\begin{pgfscope}%
\pgfsetbuttcap%
\pgfsetroundjoin%
\definecolor{currentfill}{rgb}{0.200000,0.200000,0.200000}%
\pgfsetfillcolor{currentfill}%
\pgfsetlinewidth{0.803000pt}%
\definecolor{currentstroke}{rgb}{0.200000,0.200000,0.200000}%
\pgfsetstrokecolor{currentstroke}%
\pgfsetdash{}{0pt}%
\pgfsys@defobject{currentmarker}{\pgfqpoint{0.000000in}{-0.048611in}}{\pgfqpoint{0.000000in}{0.000000in}}{%
\pgfpathmoveto{\pgfqpoint{0.000000in}{0.000000in}}%
\pgfpathlineto{\pgfqpoint{0.000000in}{-0.048611in}}%
\pgfusepath{stroke,fill}%
}%
\begin{pgfscope}%
\pgfsys@transformshift{1.721772in}{1.249582in}%
\pgfsys@useobject{currentmarker}{}%
\end{pgfscope}%
\end{pgfscope}%
\begin{pgfscope}%
\pgfsetbuttcap%
\pgfsetroundjoin%
\definecolor{currentfill}{rgb}{0.200000,0.200000,0.200000}%
\pgfsetfillcolor{currentfill}%
\pgfsetlinewidth{0.803000pt}%
\definecolor{currentstroke}{rgb}{0.200000,0.200000,0.200000}%
\pgfsetstrokecolor{currentstroke}%
\pgfsetdash{}{0pt}%
\pgfsys@defobject{currentmarker}{\pgfqpoint{0.000000in}{-0.048611in}}{\pgfqpoint{0.000000in}{0.000000in}}{%
\pgfpathmoveto{\pgfqpoint{0.000000in}{0.000000in}}%
\pgfpathlineto{\pgfqpoint{0.000000in}{-0.048611in}}%
\pgfusepath{stroke,fill}%
}%
\begin{pgfscope}%
\pgfsys@transformshift{2.290105in}{1.249582in}%
\pgfsys@useobject{currentmarker}{}%
\end{pgfscope}%
\end{pgfscope}%
\begin{pgfscope}%
\pgfsetbuttcap%
\pgfsetroundjoin%
\definecolor{currentfill}{rgb}{0.200000,0.200000,0.200000}%
\pgfsetfillcolor{currentfill}%
\pgfsetlinewidth{0.803000pt}%
\definecolor{currentstroke}{rgb}{0.200000,0.200000,0.200000}%
\pgfsetstrokecolor{currentstroke}%
\pgfsetdash{}{0pt}%
\pgfsys@defobject{currentmarker}{\pgfqpoint{0.000000in}{-0.048611in}}{\pgfqpoint{0.000000in}{0.000000in}}{%
\pgfpathmoveto{\pgfqpoint{0.000000in}{0.000000in}}%
\pgfpathlineto{\pgfqpoint{0.000000in}{-0.048611in}}%
\pgfusepath{stroke,fill}%
}%
\begin{pgfscope}%
\pgfsys@transformshift{2.858438in}{1.249582in}%
\pgfsys@useobject{currentmarker}{}%
\end{pgfscope}%
\end{pgfscope}%
\begin{pgfscope}%
\pgfsetbuttcap%
\pgfsetroundjoin%
\definecolor{currentfill}{rgb}{0.200000,0.200000,0.200000}%
\pgfsetfillcolor{currentfill}%
\pgfsetlinewidth{0.803000pt}%
\definecolor{currentstroke}{rgb}{0.200000,0.200000,0.200000}%
\pgfsetstrokecolor{currentstroke}%
\pgfsetdash{}{0pt}%
\pgfsys@defobject{currentmarker}{\pgfqpoint{0.000000in}{-0.048611in}}{\pgfqpoint{0.000000in}{0.000000in}}{%
\pgfpathmoveto{\pgfqpoint{0.000000in}{0.000000in}}%
\pgfpathlineto{\pgfqpoint{0.000000in}{-0.048611in}}%
\pgfusepath{stroke,fill}%
}%
\begin{pgfscope}%
\pgfsys@transformshift{3.426772in}{1.249582in}%
\pgfsys@useobject{currentmarker}{}%
\end{pgfscope}%
\end{pgfscope}%
\begin{pgfscope}%
\pgfsetbuttcap%
\pgfsetroundjoin%
\definecolor{currentfill}{rgb}{0.200000,0.200000,0.200000}%
\pgfsetfillcolor{currentfill}%
\pgfsetlinewidth{0.803000pt}%
\definecolor{currentstroke}{rgb}{0.200000,0.200000,0.200000}%
\pgfsetstrokecolor{currentstroke}%
\pgfsetdash{}{0pt}%
\pgfsys@defobject{currentmarker}{\pgfqpoint{0.000000in}{-0.048611in}}{\pgfqpoint{0.000000in}{0.000000in}}{%
\pgfpathmoveto{\pgfqpoint{0.000000in}{0.000000in}}%
\pgfpathlineto{\pgfqpoint{0.000000in}{-0.048611in}}%
\pgfusepath{stroke,fill}%
}%
\begin{pgfscope}%
\pgfsys@transformshift{3.995105in}{1.249582in}%
\pgfsys@useobject{currentmarker}{}%
\end{pgfscope}%
\end{pgfscope}%
\begin{pgfscope}%
\pgfsetbuttcap%
\pgfsetroundjoin%
\definecolor{currentfill}{rgb}{0.200000,0.200000,0.200000}%
\pgfsetfillcolor{currentfill}%
\pgfsetlinewidth{0.803000pt}%
\definecolor{currentstroke}{rgb}{0.200000,0.200000,0.200000}%
\pgfsetstrokecolor{currentstroke}%
\pgfsetdash{}{0pt}%
\pgfsys@defobject{currentmarker}{\pgfqpoint{0.000000in}{-0.048611in}}{\pgfqpoint{0.000000in}{0.000000in}}{%
\pgfpathmoveto{\pgfqpoint{0.000000in}{0.000000in}}%
\pgfpathlineto{\pgfqpoint{0.000000in}{-0.048611in}}%
\pgfusepath{stroke,fill}%
}%
\begin{pgfscope}%
\pgfsys@transformshift{4.563438in}{1.249582in}%
\pgfsys@useobject{currentmarker}{}%
\end{pgfscope}%
\end{pgfscope}%
\begin{pgfscope}%
\pgfpathrectangle{\pgfqpoint{0.585105in}{1.249582in}}{\pgfqpoint{4.262500in}{0.679412in}} %
\pgfusepath{clip}%
\pgfsetrectcap%
\pgfsetroundjoin%
\pgfsetlinewidth{0.803000pt}%
\definecolor{currentstroke}{rgb}{0.690196,0.690196,0.690196}%
\pgfsetstrokecolor{currentstroke}%
\pgfsetstrokeopacity{0.300000}%
\pgfsetdash{}{0pt}%
\pgfpathmoveto{\pgfqpoint{0.585105in}{1.399367in}}%
\pgfpathlineto{\pgfqpoint{4.847605in}{1.399367in}}%
\pgfusepath{stroke}%
\end{pgfscope}%
\begin{pgfscope}%
\pgfsetbuttcap%
\pgfsetroundjoin%
\definecolor{currentfill}{rgb}{0.200000,0.200000,0.200000}%
\pgfsetfillcolor{currentfill}%
\pgfsetlinewidth{0.803000pt}%
\definecolor{currentstroke}{rgb}{0.200000,0.200000,0.200000}%
\pgfsetstrokecolor{currentstroke}%
\pgfsetdash{}{0pt}%
\pgfsys@defobject{currentmarker}{\pgfqpoint{-0.048611in}{0.000000in}}{\pgfqpoint{0.000000in}{0.000000in}}{%
\pgfpathmoveto{\pgfqpoint{0.000000in}{0.000000in}}%
\pgfpathlineto{\pgfqpoint{-0.048611in}{0.000000in}}%
\pgfusepath{stroke,fill}%
}%
\begin{pgfscope}%
\pgfsys@transformshift{0.585105in}{1.399367in}%
\pgfsys@useobject{currentmarker}{}%
\end{pgfscope}%
\end{pgfscope}%
\begin{pgfscope}%
\definecolor{textcolor}{rgb}{0.200000,0.200000,0.200000}%
\pgfsetstrokecolor{textcolor}%
\pgfsetfillcolor{textcolor}%
\pgftext[x=0.240968in,y=1.351539in,left,base]{\color{textcolor}\rmfamily\fontsize{10.000000}{12.000000}\selectfont \(\displaystyle -60\)}%
\end{pgfscope}%
\begin{pgfscope}%
\pgfpathrectangle{\pgfqpoint{0.585105in}{1.249582in}}{\pgfqpoint{4.262500in}{0.679412in}} %
\pgfusepath{clip}%
\pgfsetrectcap%
\pgfsetroundjoin%
\pgfsetlinewidth{0.803000pt}%
\definecolor{currentstroke}{rgb}{0.690196,0.690196,0.690196}%
\pgfsetstrokecolor{currentstroke}%
\pgfsetstrokeopacity{0.300000}%
\pgfsetdash{}{0pt}%
\pgfpathmoveto{\pgfqpoint{0.585105in}{1.874977in}}%
\pgfpathlineto{\pgfqpoint{4.847605in}{1.874977in}}%
\pgfusepath{stroke}%
\end{pgfscope}%
\begin{pgfscope}%
\pgfsetbuttcap%
\pgfsetroundjoin%
\definecolor{currentfill}{rgb}{0.200000,0.200000,0.200000}%
\pgfsetfillcolor{currentfill}%
\pgfsetlinewidth{0.803000pt}%
\definecolor{currentstroke}{rgb}{0.200000,0.200000,0.200000}%
\pgfsetstrokecolor{currentstroke}%
\pgfsetdash{}{0pt}%
\pgfsys@defobject{currentmarker}{\pgfqpoint{-0.048611in}{0.000000in}}{\pgfqpoint{0.000000in}{0.000000in}}{%
\pgfpathmoveto{\pgfqpoint{0.000000in}{0.000000in}}%
\pgfpathlineto{\pgfqpoint{-0.048611in}{0.000000in}}%
\pgfusepath{stroke,fill}%
}%
\begin{pgfscope}%
\pgfsys@transformshift{0.585105in}{1.874977in}%
\pgfsys@useobject{currentmarker}{}%
\end{pgfscope}%
\end{pgfscope}%
\begin{pgfscope}%
\definecolor{textcolor}{rgb}{0.200000,0.200000,0.200000}%
\pgfsetstrokecolor{textcolor}%
\pgfsetfillcolor{textcolor}%
\pgftext[x=0.240968in,y=1.827150in,left,base]{\color{textcolor}\rmfamily\fontsize{10.000000}{12.000000}\selectfont \(\displaystyle -40\)}%
\end{pgfscope}%
\begin{pgfscope}%
\pgftext[x=0.185413in,y=1.589288in,,bottom,rotate=90.000000]{\rmfamily\fontsize{10.000000}{12.000000}\selectfont \(\displaystyle V^\mathrm{AdEx}_\mathrm{m}\) [mV]}%
\end{pgfscope}%
\begin{pgfscope}%
\pgfpathrectangle{\pgfqpoint{0.585105in}{1.249582in}}{\pgfqpoint{4.262500in}{0.679412in}} %
\pgfusepath{clip}%
\pgfsetrectcap%
\pgfsetroundjoin%
\pgfsetlinewidth{0.501875pt}%
\definecolor{currentstroke}{rgb}{0.400000,0.400000,0.400000}%
\pgfsetstrokecolor{currentstroke}%
\pgfsetdash{}{0pt}%
\pgfpathmoveto{\pgfqpoint{0.585105in}{1.399367in}}%
\pgfpathlineto{\pgfqpoint{0.828352in}{1.399630in}}%
\pgfpathlineto{\pgfqpoint{1.153438in}{1.399615in}}%
\pgfpathlineto{\pgfqpoint{1.165373in}{1.433548in}}%
\pgfpathlineto{\pgfqpoint{1.177308in}{1.460921in}}%
\pgfpathlineto{\pgfqpoint{1.189243in}{1.482877in}}%
\pgfpathlineto{\pgfqpoint{1.201178in}{1.500383in}}%
\pgfpathlineto{\pgfqpoint{1.213682in}{1.514830in}}%
\pgfpathlineto{\pgfqpoint{1.226753in}{1.526490in}}%
\pgfpathlineto{\pgfqpoint{1.240393in}{1.535659in}}%
\pgfpathlineto{\pgfqpoint{1.255170in}{1.542869in}}%
\pgfpathlineto{\pgfqpoint{1.271652in}{1.548335in}}%
\pgfpathlineto{\pgfqpoint{1.290407in}{1.552096in}}%
\pgfpathlineto{\pgfqpoint{1.312572in}{1.554162in}}%
\pgfpathlineto{\pgfqpoint{1.340988in}{1.554419in}}%
\pgfpathlineto{\pgfqpoint{1.383613in}{1.552317in}}%
\pgfpathlineto{\pgfqpoint{1.521718in}{1.544306in}}%
\pgfpathlineto{\pgfqpoint{1.609810in}{1.542256in}}%
\pgfpathlineto{\pgfqpoint{1.728023in}{1.539722in}}%
\pgfpathlineto{\pgfqpoint{1.740527in}{1.504303in}}%
\pgfpathlineto{\pgfqpoint{1.752462in}{1.476876in}}%
\pgfpathlineto{\pgfqpoint{1.764397in}{1.454836in}}%
\pgfpathlineto{\pgfqpoint{1.776332in}{1.437338in}}%
\pgfpathlineto{\pgfqpoint{1.788267in}{1.423613in}}%
\pgfpathlineto{\pgfqpoint{1.800770in}{1.412555in}}%
\pgfpathlineto{\pgfqpoint{1.813842in}{1.403930in}}%
\pgfpathlineto{\pgfqpoint{1.828050in}{1.397244in}}%
\pgfpathlineto{\pgfqpoint{1.843963in}{1.392305in}}%
\pgfpathlineto{\pgfqpoint{1.862150in}{1.389071in}}%
\pgfpathlineto{\pgfqpoint{1.884315in}{1.387484in}}%
\pgfpathlineto{\pgfqpoint{1.913868in}{1.387735in}}%
\pgfpathlineto{\pgfqpoint{1.966723in}{1.390819in}}%
\pgfpathlineto{\pgfqpoint{2.062772in}{1.396187in}}%
\pgfpathlineto{\pgfqpoint{2.148590in}{1.398471in}}%
\pgfpathlineto{\pgfqpoint{2.281580in}{1.399484in}}%
\pgfpathlineto{\pgfqpoint{2.774325in}{1.399615in}}%
\pgfpathlineto{\pgfqpoint{2.869805in}{1.399615in}}%
\pgfpathlineto{\pgfqpoint{2.881740in}{1.448483in}}%
\pgfpathlineto{\pgfqpoint{2.893675in}{1.487951in}}%
\pgfpathlineto{\pgfqpoint{2.905610in}{1.519737in}}%
\pgfpathlineto{\pgfqpoint{2.917545in}{1.545332in}}%
\pgfpathlineto{\pgfqpoint{2.930048in}{1.566879in}}%
\pgfpathlineto{\pgfqpoint{2.943120in}{1.584916in}}%
\pgfpathlineto{\pgfqpoint{2.956760in}{1.599992in}}%
\pgfpathlineto{\pgfqpoint{2.971537in}{1.613084in}}%
\pgfpathlineto{\pgfqpoint{2.988018in}{1.624765in}}%
\pgfpathlineto{\pgfqpoint{3.007342in}{1.635683in}}%
\pgfpathlineto{\pgfqpoint{3.030643in}{1.646204in}}%
\pgfpathlineto{\pgfqpoint{3.063607in}{1.658404in}}%
\pgfpathlineto{\pgfqpoint{3.109642in}{1.675541in}}%
\pgfpathlineto{\pgfqpoint{3.126123in}{1.684068in}}%
\pgfpathlineto{\pgfqpoint{3.138058in}{1.692663in}}%
\pgfpathlineto{\pgfqpoint{3.147152in}{1.702020in}}%
\pgfpathlineto{\pgfqpoint{3.153972in}{1.712182in}}%
\pgfpathlineto{\pgfqpoint{3.159655in}{1.724777in}}%
\pgfpathlineto{\pgfqpoint{3.164202in}{1.740565in}}%
\pgfpathlineto{\pgfqpoint{3.167612in}{1.760146in}}%
\pgfpathlineto{\pgfqpoint{3.170453in}{1.790580in}}%
\pgfpathlineto{\pgfqpoint{3.172158in}{1.829874in}}%
\pgfpathlineto{\pgfqpoint{3.173295in}{1.898111in}}%
\pgfpathlineto{\pgfqpoint{3.173863in}{1.280464in}}%
\pgfpathlineto{\pgfqpoint{3.174432in}{1.283463in}}%
\pgfpathlineto{\pgfqpoint{3.187503in}{1.346038in}}%
\pgfpathlineto{\pgfqpoint{3.200575in}{1.397661in}}%
\pgfpathlineto{\pgfqpoint{3.213078in}{1.438396in}}%
\pgfpathlineto{\pgfqpoint{3.225582in}{1.472041in}}%
\pgfpathlineto{\pgfqpoint{3.238653in}{1.500914in}}%
\pgfpathlineto{\pgfqpoint{3.251725in}{1.524495in}}%
\pgfpathlineto{\pgfqpoint{3.265365in}{1.544536in}}%
\pgfpathlineto{\pgfqpoint{3.279573in}{1.561450in}}%
\pgfpathlineto{\pgfqpoint{3.294350in}{1.575658in}}%
\pgfpathlineto{\pgfqpoint{3.310263in}{1.587951in}}%
\pgfpathlineto{\pgfqpoint{3.327882in}{1.598764in}}%
\pgfpathlineto{\pgfqpoint{3.347773in}{1.608299in}}%
\pgfpathlineto{\pgfqpoint{3.369938in}{1.616433in}}%
\pgfpathlineto{\pgfqpoint{3.395513in}{1.623414in}}%
\pgfpathlineto{\pgfqpoint{3.425067in}{1.629138in}}%
\pgfpathlineto{\pgfqpoint{3.459735in}{1.633522in}}%
\pgfpathlineto{\pgfqpoint{3.500087in}{1.636321in}}%
\pgfpathlineto{\pgfqpoint{3.547827in}{1.637335in}}%
\pgfpathlineto{\pgfqpoint{3.607502in}{1.636258in}}%
\pgfpathlineto{\pgfqpoint{3.708665in}{1.631786in}}%
\pgfpathlineto{\pgfqpoint{3.830857in}{1.627091in}}%
\pgfpathlineto{\pgfqpoint{3.939977in}{1.625306in}}%
\pgfpathlineto{\pgfqpoint{4.012155in}{1.625014in}}%
\pgfpathlineto{\pgfqpoint{4.028637in}{1.555638in}}%
\pgfpathlineto{\pgfqpoint{4.041140in}{1.512787in}}%
\pgfpathlineto{\pgfqpoint{4.053075in}{1.479846in}}%
\pgfpathlineto{\pgfqpoint{4.064442in}{1.454779in}}%
\pgfpathlineto{\pgfqpoint{4.075808in}{1.434905in}}%
\pgfpathlineto{\pgfqpoint{4.087175in}{1.419362in}}%
\pgfpathlineto{\pgfqpoint{4.098542in}{1.407390in}}%
\pgfpathlineto{\pgfqpoint{4.110477in}{1.397954in}}%
\pgfpathlineto{\pgfqpoint{4.122980in}{1.390835in}}%
\pgfpathlineto{\pgfqpoint{4.136620in}{1.385601in}}%
\pgfpathlineto{\pgfqpoint{4.151965in}{1.382103in}}%
\pgfpathlineto{\pgfqpoint{4.170152in}{1.380302in}}%
\pgfpathlineto{\pgfqpoint{4.193453in}{1.380393in}}%
\pgfpathlineto{\pgfqpoint{4.227553in}{1.383018in}}%
\pgfpathlineto{\pgfqpoint{4.355428in}{1.394585in}}%
\pgfpathlineto{\pgfqpoint{4.423628in}{1.397504in}}%
\pgfpathlineto{\pgfqpoint{4.517403in}{1.399104in}}%
\pgfpathlineto{\pgfqpoint{4.707795in}{1.399616in}}%
\pgfpathlineto{\pgfqpoint{4.848173in}{1.399620in}}%
\pgfpathlineto{\pgfqpoint{4.848173in}{1.399620in}}%
\pgfusepath{stroke}%
\end{pgfscope}%
\begin{pgfscope}%
\pgfsetrectcap%
\pgfsetmiterjoin%
\pgfsetlinewidth{0.803000pt}%
\definecolor{currentstroke}{rgb}{0.200000,0.200000,0.200000}%
\pgfsetstrokecolor{currentstroke}%
\pgfsetdash{}{0pt}%
\pgfpathmoveto{\pgfqpoint{0.585105in}{1.249582in}}%
\pgfpathlineto{\pgfqpoint{0.585105in}{1.928994in}}%
\pgfusepath{stroke}%
\end{pgfscope}%
\begin{pgfscope}%
\pgfsetrectcap%
\pgfsetmiterjoin%
\pgfsetlinewidth{0.803000pt}%
\definecolor{currentstroke}{rgb}{0.200000,0.200000,0.200000}%
\pgfsetstrokecolor{currentstroke}%
\pgfsetdash{}{0pt}%
\pgfpathmoveto{\pgfqpoint{0.585105in}{1.249582in}}%
\pgfpathlineto{\pgfqpoint{4.847605in}{1.249582in}}%
\pgfusepath{stroke}%
\end{pgfscope}%
\begin{pgfscope}%
\definecolor{textcolor}{rgb}{0.200000,0.200000,0.200000}%
\pgfsetstrokecolor{textcolor}%
\pgfsetfillcolor{textcolor}%
\pgftext[x=0.670355in,y=1.827082in,left,top]{\color{textcolor}\rmfamily\fontsize{10.000000}{12.000000}\selectfont (b)}%
\end{pgfscope}%
\begin{pgfscope}%
\pgfsetbuttcap%
\pgfsetmiterjoin%
\pgfsetlinewidth{0.000000pt}%
\definecolor{currentstroke}{rgb}{0.000000,0.000000,0.000000}%
\pgfsetstrokecolor{currentstroke}%
\pgfsetstrokeopacity{0.000000}%
\pgfsetdash{}{0pt}%
\pgfpathmoveto{\pgfqpoint{0.585105in}{0.434288in}}%
\pgfpathlineto{\pgfqpoint{4.847605in}{0.434288in}}%
\pgfpathlineto{\pgfqpoint{4.847605in}{1.113700in}}%
\pgfpathlineto{\pgfqpoint{0.585105in}{1.113700in}}%
\pgfpathclose%
\pgfusepath{}%
\end{pgfscope}%
\begin{pgfscope}%
\pgfsetbuttcap%
\pgfsetroundjoin%
\definecolor{currentfill}{rgb}{0.200000,0.200000,0.200000}%
\pgfsetfillcolor{currentfill}%
\pgfsetlinewidth{0.803000pt}%
\definecolor{currentstroke}{rgb}{0.200000,0.200000,0.200000}%
\pgfsetstrokecolor{currentstroke}%
\pgfsetdash{}{0pt}%
\pgfsys@defobject{currentmarker}{\pgfqpoint{0.000000in}{-0.048611in}}{\pgfqpoint{0.000000in}{0.000000in}}{%
\pgfpathmoveto{\pgfqpoint{0.000000in}{0.000000in}}%
\pgfpathlineto{\pgfqpoint{0.000000in}{-0.048611in}}%
\pgfusepath{stroke,fill}%
}%
\begin{pgfscope}%
\pgfsys@transformshift{0.585105in}{0.434288in}%
\pgfsys@useobject{currentmarker}{}%
\end{pgfscope}%
\end{pgfscope}%
\begin{pgfscope}%
\definecolor{textcolor}{rgb}{0.200000,0.200000,0.200000}%
\pgfsetstrokecolor{textcolor}%
\pgfsetfillcolor{textcolor}%
\pgftext[x=0.585105in,y=0.337066in,,top]{\color{textcolor}\rmfamily\fontsize{10.000000}{12.000000}\selectfont \(\displaystyle 0\)}%
\end{pgfscope}%
\begin{pgfscope}%
\pgfsetbuttcap%
\pgfsetroundjoin%
\definecolor{currentfill}{rgb}{0.200000,0.200000,0.200000}%
\pgfsetfillcolor{currentfill}%
\pgfsetlinewidth{0.803000pt}%
\definecolor{currentstroke}{rgb}{0.200000,0.200000,0.200000}%
\pgfsetstrokecolor{currentstroke}%
\pgfsetdash{}{0pt}%
\pgfsys@defobject{currentmarker}{\pgfqpoint{0.000000in}{-0.048611in}}{\pgfqpoint{0.000000in}{0.000000in}}{%
\pgfpathmoveto{\pgfqpoint{0.000000in}{0.000000in}}%
\pgfpathlineto{\pgfqpoint{0.000000in}{-0.048611in}}%
\pgfusepath{stroke,fill}%
}%
\begin{pgfscope}%
\pgfsys@transformshift{1.153438in}{0.434288in}%
\pgfsys@useobject{currentmarker}{}%
\end{pgfscope}%
\end{pgfscope}%
\begin{pgfscope}%
\definecolor{textcolor}{rgb}{0.200000,0.200000,0.200000}%
\pgfsetstrokecolor{textcolor}%
\pgfsetfillcolor{textcolor}%
\pgftext[x=1.153438in,y=0.337066in,,top]{\color{textcolor}\rmfamily\fontsize{10.000000}{12.000000}\selectfont \(\displaystyle 100\)}%
\end{pgfscope}%
\begin{pgfscope}%
\pgfsetbuttcap%
\pgfsetroundjoin%
\definecolor{currentfill}{rgb}{0.200000,0.200000,0.200000}%
\pgfsetfillcolor{currentfill}%
\pgfsetlinewidth{0.803000pt}%
\definecolor{currentstroke}{rgb}{0.200000,0.200000,0.200000}%
\pgfsetstrokecolor{currentstroke}%
\pgfsetdash{}{0pt}%
\pgfsys@defobject{currentmarker}{\pgfqpoint{0.000000in}{-0.048611in}}{\pgfqpoint{0.000000in}{0.000000in}}{%
\pgfpathmoveto{\pgfqpoint{0.000000in}{0.000000in}}%
\pgfpathlineto{\pgfqpoint{0.000000in}{-0.048611in}}%
\pgfusepath{stroke,fill}%
}%
\begin{pgfscope}%
\pgfsys@transformshift{1.721772in}{0.434288in}%
\pgfsys@useobject{currentmarker}{}%
\end{pgfscope}%
\end{pgfscope}%
\begin{pgfscope}%
\definecolor{textcolor}{rgb}{0.200000,0.200000,0.200000}%
\pgfsetstrokecolor{textcolor}%
\pgfsetfillcolor{textcolor}%
\pgftext[x=1.721772in,y=0.337066in,,top]{\color{textcolor}\rmfamily\fontsize{10.000000}{12.000000}\selectfont \(\displaystyle 200\)}%
\end{pgfscope}%
\begin{pgfscope}%
\pgfsetbuttcap%
\pgfsetroundjoin%
\definecolor{currentfill}{rgb}{0.200000,0.200000,0.200000}%
\pgfsetfillcolor{currentfill}%
\pgfsetlinewidth{0.803000pt}%
\definecolor{currentstroke}{rgb}{0.200000,0.200000,0.200000}%
\pgfsetstrokecolor{currentstroke}%
\pgfsetdash{}{0pt}%
\pgfsys@defobject{currentmarker}{\pgfqpoint{0.000000in}{-0.048611in}}{\pgfqpoint{0.000000in}{0.000000in}}{%
\pgfpathmoveto{\pgfqpoint{0.000000in}{0.000000in}}%
\pgfpathlineto{\pgfqpoint{0.000000in}{-0.048611in}}%
\pgfusepath{stroke,fill}%
}%
\begin{pgfscope}%
\pgfsys@transformshift{2.290105in}{0.434288in}%
\pgfsys@useobject{currentmarker}{}%
\end{pgfscope}%
\end{pgfscope}%
\begin{pgfscope}%
\definecolor{textcolor}{rgb}{0.200000,0.200000,0.200000}%
\pgfsetstrokecolor{textcolor}%
\pgfsetfillcolor{textcolor}%
\pgftext[x=2.290105in,y=0.337066in,,top]{\color{textcolor}\rmfamily\fontsize{10.000000}{12.000000}\selectfont \(\displaystyle 300\)}%
\end{pgfscope}%
\begin{pgfscope}%
\pgfsetbuttcap%
\pgfsetroundjoin%
\definecolor{currentfill}{rgb}{0.200000,0.200000,0.200000}%
\pgfsetfillcolor{currentfill}%
\pgfsetlinewidth{0.803000pt}%
\definecolor{currentstroke}{rgb}{0.200000,0.200000,0.200000}%
\pgfsetstrokecolor{currentstroke}%
\pgfsetdash{}{0pt}%
\pgfsys@defobject{currentmarker}{\pgfqpoint{0.000000in}{-0.048611in}}{\pgfqpoint{0.000000in}{0.000000in}}{%
\pgfpathmoveto{\pgfqpoint{0.000000in}{0.000000in}}%
\pgfpathlineto{\pgfqpoint{0.000000in}{-0.048611in}}%
\pgfusepath{stroke,fill}%
}%
\begin{pgfscope}%
\pgfsys@transformshift{2.858438in}{0.434288in}%
\pgfsys@useobject{currentmarker}{}%
\end{pgfscope}%
\end{pgfscope}%
\begin{pgfscope}%
\definecolor{textcolor}{rgb}{0.200000,0.200000,0.200000}%
\pgfsetstrokecolor{textcolor}%
\pgfsetfillcolor{textcolor}%
\pgftext[x=2.858438in,y=0.337066in,,top]{\color{textcolor}\rmfamily\fontsize{10.000000}{12.000000}\selectfont \(\displaystyle 400\)}%
\end{pgfscope}%
\begin{pgfscope}%
\pgfsetbuttcap%
\pgfsetroundjoin%
\definecolor{currentfill}{rgb}{0.200000,0.200000,0.200000}%
\pgfsetfillcolor{currentfill}%
\pgfsetlinewidth{0.803000pt}%
\definecolor{currentstroke}{rgb}{0.200000,0.200000,0.200000}%
\pgfsetstrokecolor{currentstroke}%
\pgfsetdash{}{0pt}%
\pgfsys@defobject{currentmarker}{\pgfqpoint{0.000000in}{-0.048611in}}{\pgfqpoint{0.000000in}{0.000000in}}{%
\pgfpathmoveto{\pgfqpoint{0.000000in}{0.000000in}}%
\pgfpathlineto{\pgfqpoint{0.000000in}{-0.048611in}}%
\pgfusepath{stroke,fill}%
}%
\begin{pgfscope}%
\pgfsys@transformshift{3.426772in}{0.434288in}%
\pgfsys@useobject{currentmarker}{}%
\end{pgfscope}%
\end{pgfscope}%
\begin{pgfscope}%
\definecolor{textcolor}{rgb}{0.200000,0.200000,0.200000}%
\pgfsetstrokecolor{textcolor}%
\pgfsetfillcolor{textcolor}%
\pgftext[x=3.426772in,y=0.337066in,,top]{\color{textcolor}\rmfamily\fontsize{10.000000}{12.000000}\selectfont \(\displaystyle 500\)}%
\end{pgfscope}%
\begin{pgfscope}%
\pgfsetbuttcap%
\pgfsetroundjoin%
\definecolor{currentfill}{rgb}{0.200000,0.200000,0.200000}%
\pgfsetfillcolor{currentfill}%
\pgfsetlinewidth{0.803000pt}%
\definecolor{currentstroke}{rgb}{0.200000,0.200000,0.200000}%
\pgfsetstrokecolor{currentstroke}%
\pgfsetdash{}{0pt}%
\pgfsys@defobject{currentmarker}{\pgfqpoint{0.000000in}{-0.048611in}}{\pgfqpoint{0.000000in}{0.000000in}}{%
\pgfpathmoveto{\pgfqpoint{0.000000in}{0.000000in}}%
\pgfpathlineto{\pgfqpoint{0.000000in}{-0.048611in}}%
\pgfusepath{stroke,fill}%
}%
\begin{pgfscope}%
\pgfsys@transformshift{3.995105in}{0.434288in}%
\pgfsys@useobject{currentmarker}{}%
\end{pgfscope}%
\end{pgfscope}%
\begin{pgfscope}%
\definecolor{textcolor}{rgb}{0.200000,0.200000,0.200000}%
\pgfsetstrokecolor{textcolor}%
\pgfsetfillcolor{textcolor}%
\pgftext[x=3.995105in,y=0.337066in,,top]{\color{textcolor}\rmfamily\fontsize{10.000000}{12.000000}\selectfont \(\displaystyle 600\)}%
\end{pgfscope}%
\begin{pgfscope}%
\pgfsetbuttcap%
\pgfsetroundjoin%
\definecolor{currentfill}{rgb}{0.200000,0.200000,0.200000}%
\pgfsetfillcolor{currentfill}%
\pgfsetlinewidth{0.803000pt}%
\definecolor{currentstroke}{rgb}{0.200000,0.200000,0.200000}%
\pgfsetstrokecolor{currentstroke}%
\pgfsetdash{}{0pt}%
\pgfsys@defobject{currentmarker}{\pgfqpoint{0.000000in}{-0.048611in}}{\pgfqpoint{0.000000in}{0.000000in}}{%
\pgfpathmoveto{\pgfqpoint{0.000000in}{0.000000in}}%
\pgfpathlineto{\pgfqpoint{0.000000in}{-0.048611in}}%
\pgfusepath{stroke,fill}%
}%
\begin{pgfscope}%
\pgfsys@transformshift{4.563438in}{0.434288in}%
\pgfsys@useobject{currentmarker}{}%
\end{pgfscope}%
\end{pgfscope}%
\begin{pgfscope}%
\definecolor{textcolor}{rgb}{0.200000,0.200000,0.200000}%
\pgfsetstrokecolor{textcolor}%
\pgfsetfillcolor{textcolor}%
\pgftext[x=4.563438in,y=0.337066in,,top]{\color{textcolor}\rmfamily\fontsize{10.000000}{12.000000}\selectfont \(\displaystyle 700\)}%
\end{pgfscope}%
\begin{pgfscope}%
\pgftext[x=2.716355in,y=0.158855in,,top]{\rmfamily\fontsize{10.000000}{12.000000}\selectfont Time [ms]}%
\end{pgfscope}%
\begin{pgfscope}%
\pgfpathrectangle{\pgfqpoint{0.585105in}{0.434288in}}{\pgfqpoint{4.262500in}{0.679412in}} %
\pgfusepath{clip}%
\pgfsetrectcap%
\pgfsetroundjoin%
\pgfsetlinewidth{0.803000pt}%
\definecolor{currentstroke}{rgb}{0.690196,0.690196,0.690196}%
\pgfsetstrokecolor{currentstroke}%
\pgfsetstrokeopacity{0.300000}%
\pgfsetdash{}{0pt}%
\pgfpathmoveto{\pgfqpoint{0.585105in}{0.490906in}}%
\pgfpathlineto{\pgfqpoint{4.847605in}{0.490906in}}%
\pgfusepath{stroke}%
\end{pgfscope}%
\begin{pgfscope}%
\pgfsetbuttcap%
\pgfsetroundjoin%
\definecolor{currentfill}{rgb}{0.200000,0.200000,0.200000}%
\pgfsetfillcolor{currentfill}%
\pgfsetlinewidth{0.803000pt}%
\definecolor{currentstroke}{rgb}{0.200000,0.200000,0.200000}%
\pgfsetstrokecolor{currentstroke}%
\pgfsetdash{}{0pt}%
\pgfsys@defobject{currentmarker}{\pgfqpoint{-0.048611in}{0.000000in}}{\pgfqpoint{0.000000in}{0.000000in}}{%
\pgfpathmoveto{\pgfqpoint{0.000000in}{0.000000in}}%
\pgfpathlineto{\pgfqpoint{-0.048611in}{0.000000in}}%
\pgfusepath{stroke,fill}%
}%
\begin{pgfscope}%
\pgfsys@transformshift{0.585105in}{0.490906in}%
\pgfsys@useobject{currentmarker}{}%
\end{pgfscope}%
\end{pgfscope}%
\begin{pgfscope}%
\definecolor{textcolor}{rgb}{0.200000,0.200000,0.200000}%
\pgfsetstrokecolor{textcolor}%
\pgfsetfillcolor{textcolor}%
\pgftext[x=0.418438in,y=0.443078in,left,base]{\color{textcolor}\rmfamily\fontsize{10.000000}{12.000000}\selectfont \(\displaystyle 0\)}%
\end{pgfscope}%
\begin{pgfscope}%
\pgfpathrectangle{\pgfqpoint{0.585105in}{0.434288in}}{\pgfqpoint{4.262500in}{0.679412in}} %
\pgfusepath{clip}%
\pgfsetrectcap%
\pgfsetroundjoin%
\pgfsetlinewidth{0.803000pt}%
\definecolor{currentstroke}{rgb}{0.690196,0.690196,0.690196}%
\pgfsetstrokecolor{currentstroke}%
\pgfsetstrokeopacity{0.300000}%
\pgfsetdash{}{0pt}%
\pgfpathmoveto{\pgfqpoint{0.585105in}{1.057082in}}%
\pgfpathlineto{\pgfqpoint{4.847605in}{1.057082in}}%
\pgfusepath{stroke}%
\end{pgfscope}%
\begin{pgfscope}%
\pgfsetbuttcap%
\pgfsetroundjoin%
\definecolor{currentfill}{rgb}{0.200000,0.200000,0.200000}%
\pgfsetfillcolor{currentfill}%
\pgfsetlinewidth{0.803000pt}%
\definecolor{currentstroke}{rgb}{0.200000,0.200000,0.200000}%
\pgfsetstrokecolor{currentstroke}%
\pgfsetdash{}{0pt}%
\pgfsys@defobject{currentmarker}{\pgfqpoint{-0.048611in}{0.000000in}}{\pgfqpoint{0.000000in}{0.000000in}}{%
\pgfpathmoveto{\pgfqpoint{0.000000in}{0.000000in}}%
\pgfpathlineto{\pgfqpoint{-0.048611in}{0.000000in}}%
\pgfusepath{stroke,fill}%
}%
\begin{pgfscope}%
\pgfsys@transformshift{0.585105in}{1.057082in}%
\pgfsys@useobject{currentmarker}{}%
\end{pgfscope}%
\end{pgfscope}%
\begin{pgfscope}%
\definecolor{textcolor}{rgb}{0.200000,0.200000,0.200000}%
\pgfsetstrokecolor{textcolor}%
\pgfsetfillcolor{textcolor}%
\pgftext[x=0.418438in,y=1.009254in,left,base]{\color{textcolor}\rmfamily\fontsize{10.000000}{12.000000}\selectfont \(\displaystyle 1\)}%
\end{pgfscope}%
\begin{pgfscope}%
\pgftext[x=0.158855in,y=0.773994in,,bottom,rotate=90.000000]{\rmfamily\fontsize{10.000000}{12.000000}\selectfont \(\displaystyle I_\mathrm{stim}\) [nA]}%
\end{pgfscope}%
\begin{pgfscope}%
\pgfpathrectangle{\pgfqpoint{0.585105in}{0.434288in}}{\pgfqpoint{4.262500in}{0.679412in}} %
\pgfusepath{clip}%
\pgfsetrectcap%
\pgfsetroundjoin%
\pgfsetlinewidth{0.501875pt}%
\definecolor{currentstroke}{rgb}{0.400000,0.400000,0.400000}%
\pgfsetstrokecolor{currentstroke}%
\pgfsetdash{}{0pt}%
\pgfpathmoveto{\pgfqpoint{0.585105in}{0.490906in}}%
\pgfpathlineto{\pgfqpoint{1.152870in}{0.490906in}}%
\pgfpathlineto{\pgfqpoint{1.154007in}{0.773994in}}%
\pgfpathlineto{\pgfqpoint{1.726887in}{0.773994in}}%
\pgfpathlineto{\pgfqpoint{1.728023in}{0.490906in}}%
\pgfpathlineto{\pgfqpoint{2.869237in}{0.490906in}}%
\pgfpathlineto{\pgfqpoint{2.870373in}{0.898553in}}%
\pgfpathlineto{\pgfqpoint{4.011587in}{0.898553in}}%
\pgfpathlineto{\pgfqpoint{4.012723in}{0.490906in}}%
\pgfpathlineto{\pgfqpoint{4.848173in}{0.490906in}}%
\pgfpathlineto{\pgfqpoint{4.848173in}{0.490906in}}%
\pgfusepath{stroke}%
\end{pgfscope}%
\begin{pgfscope}%
\pgfsetrectcap%
\pgfsetmiterjoin%
\pgfsetlinewidth{0.803000pt}%
\definecolor{currentstroke}{rgb}{0.200000,0.200000,0.200000}%
\pgfsetstrokecolor{currentstroke}%
\pgfsetdash{}{0pt}%
\pgfpathmoveto{\pgfqpoint{0.585105in}{0.434288in}}%
\pgfpathlineto{\pgfqpoint{0.585105in}{1.113700in}}%
\pgfusepath{stroke}%
\end{pgfscope}%
\begin{pgfscope}%
\pgfsetrectcap%
\pgfsetmiterjoin%
\pgfsetlinewidth{0.803000pt}%
\definecolor{currentstroke}{rgb}{0.200000,0.200000,0.200000}%
\pgfsetstrokecolor{currentstroke}%
\pgfsetdash{}{0pt}%
\pgfpathmoveto{\pgfqpoint{0.585105in}{0.434288in}}%
\pgfpathlineto{\pgfqpoint{4.847605in}{0.434288in}}%
\pgfusepath{stroke}%
\end{pgfscope}%
\begin{pgfscope}%
\definecolor{textcolor}{rgb}{0.200000,0.200000,0.200000}%
\pgfsetstrokecolor{textcolor}%
\pgfsetfillcolor{textcolor}%
\pgftext[x=0.670355in,y=1.011788in,left,top]{\color{textcolor}\rmfamily\fontsize{10.000000}{12.000000}\selectfont (c)}%
\end{pgfscope}%
\end{pgfpicture}%
\makeatother%
\endgroup%
}
	\caption[Membrane dynamics of the \gls{lif} and \gls{adex} given a constant input.]{Membrane dynamics of the \gls{lif} and \gls{adex} given a constant input. \textbf{(a)} The membrane potential of the \gls{lif} model $V_\text{m}^\text{LIF}$ evolves in response to a small input current, which is not strong enough to trigger a spike. A more intense stimulation yields a repetitive and equidistant spiking pattern. \textbf{(b)} Given a small step current, the shape of the \gls{adex} neuron's potential $V_\text{m}^\text{AdEx}$ is similar to the \gls{lif} model. At higher inputs, a negative adaption inhibits a repetitive spiking pattern after the first spike. The peak resembles the positive exponential voltage feedback simulating an action potential. \textbf{(c)} The stimulation current used for both models to show the course of the membrane potential. Figure taken from \citealp{stradmann2019msc}}
	\label{lifvsadex}
\end{figure}

A \gls{lif} neuron doesn't keep track of any previous spikes once a spike is released, given that the time constant of the synaptic input is shorter than the one of the membrane potential, in particular if $\tau_\text{m} > \tau_\text{fall}$. These limitations make it impossible for the model to correctly describe neuronal behavior such as spike bursts.

The constraints of the \gls{lif} neuron led to a demand of a more detailed model, namely the \glsfirst{adex} model, which is an extension to the \gls{lif} model featuring an additional adaption state variable that provides post-spike memory to the membrane. Depending on the sign of the adaption, the in a neuron is either inhibited or engaged to fire again after having spiked at least once.


%Apart from the additional state variable the \gls{adex} neuron features a positive exponential voltage feedback. on the other hand, enables the neuron to have a more complex behavior in the (\gls{v_mem}, w?) phase space.

For the scope of this thesis a more advanced model is not yet required. All experiments are done using the simpler \gls{lif} model. However, an \gls{adex} based implementation of similar experiments part of a future project (see \cref{futureprojects}).
 %(see \cref{lifvsadex}).

\section{Spiking Neural Networks}
\label{neuralcoding}

%A neuron's primary way to exchange information with another is to send and receive spikes.
\Glspl{snn} take the biological inspiration for \glspl{ann} further by conveying and processing information using spikes and neuron models such as the \gls{lif} neuron. As mentioned before, spikes are assumed to have a stereotypical shape, leaving the temporal dimension to communicate. In the context of a large multilayer networks, the topology itself, i.e. the types of synapses, the synaptic strenght, the pre and postsynaptic partners, encodes information as well.

In the following, the different training approaches for rate and temporal coding with a feed-forward multi-layer \glspl{snn} using \gls{lif} neurons are presented.

\subsection{Rate Coding}
\label{ratecoding}
In an attempt to explain computational process of the brain, the activation of an artificial neuron has already been mapped onto the spike rate of a spiking neuron (\citealp{rieke1999spikes}). This is based on the assumption that spikes follow a Poisson process and therefore the fire rate of a neuron is well described by a Poisson distribution in most cases (\citealp{averbeck2009poisson}), which in turn legitimates the use of a mean fire rate and a certain accuracy.

The exact definition and interpretation of the spike rate can vary. The most apparent approach to define a neuron's spike rate $\nu$ is to count the number of spikes $n_\text{spikes}$ fired within a period $T$.
\begin{equation}
\nu = \frac{n_\text{spikes}}{T}
\label{eqratecoding}
\end{equation}
From a practical point of view, this method is time consuming and can therefore not be the basis upon which fast decision of the brain are taken. In another approach, the period $T$ is shortened to $\Delta T$ resulting in a more inaccurate fire rate. By repeating the measurement multiple times, the average rate improves the accuracy, but the total measurement duration is prolonged again. Moreover, it is not really feasible that the exact same situation occurs multiple times in a real world problem. To solve both issues, the population average rate $\nu_\text{pop}$ can be used. A population of $n$ neurons experiences a situation simultaneously. The averaged fire rate over the whole population yields an accurate rate despite the reduced measurement time.
\begin{equation}
\left\langle\nu \right\rangle_\text{pop} = \frac{\sum_i n_{\text{spikes},i}}{nT}
\end{equation}
%As this thesis targets neuronal networks of smaller size, the following paragraphs focus on the temporal dimension of neural coding.

In the terminology of the \gls{lif} model, a presynaptic spiketrain $S_j$ can be associated with a mean fire rate $\nu_j$ assuming it is Poisson based. In this way the time average of the synaptic input can be expressed in terms of fire rates 
\begin{equation}
	\left\langle \gls{isyn} \right\rangle = \sum_j w_j \nu_{\text{in}, j}.
\end{equation}



\subsection{Supervised Training with Rate Coding}
\label{ratebasedtraining}
The similarity to \glspl{ann} implies that typical training methods such as \gls{sgd} will work with rate-based \glspl{snn} too. In a spiking feed-forward network the output of a node is determined by the transfer function \gls{transfer}, which is given by the neuron model in place.
% In case of the \gls{lif} neuron, the transfer function is similar to a \gls{relu}, with a significant difference, that the rate will no keep increasing but rather saturates quickly at a maximum frequency $\nu_\text{max}$ which is limited by the refractory period $\nu_\text{max} = \nicefrac{1}{\gls{refrac}}$.
The benefits of a sigmoid transfer function have already been motivated in \cref{supervisedtraining}. With a few adjustments it can be implemented for the \gls{lif} model with rate-based coding.


The stimulation of a \gls{lif} neuron can be expressed by the synaptic input currents \gls{isyn} or the corresponding input fire rate $\nu_\text{in}$. In analogy to the activation of an \gls{ann} the synaptic current can be split up into an input and bias term $\gls{isyn} = I_\text{in} + I_\text{bias}$. The input spike rate $\nu_\text{in}$ of the neuron scales linear with the synaptic current and therefore $\nu_\text{in} \propto I_\text{in}$. The transfer function of the \gls{lif} neuron is then approximated by 
\begin{equation}
\frac{1}{\gls{transfer}(\nu_\text{in})} = \frac{1}{\nu_\text{out}} \approx \gls{refrac} + \gls{tau_m} \frac{\gls{thres} - \gls{v_reset}}{\gls{isyn}},
\label{fireratehigh}
\end{equation}
given that the input rate is high and the time constants \gls{tau_m} and \gls{tau_syn} are smaller than \gls{refrac} (c.f. \citealp{brunel2000dynamics}).
The output rate saturates at a maximum rate $\nu_\text{max} = \nicefrac{1}{\gls{refrac}}$ if the input rate is high enough.

In the limit $\gls{thres} - \gls{isyn} \gg \sigma$, i.e. for low input rates, the transfer function yields
\begin{equation}
\nu_\text{out} \approx \frac{(\gls{thres} - \gls{isyn})}{\gls{tau_m}\sigma \sqrt{\pi}} \exp\left(-\frac{(\gls{thres} - \gls{isyn})^2}{\sigma^2}\right),
\label{fireratelow}
\end{equation}
with the fluctuations of a single excitatory input source $\sigma$. 
Without external noise, the fluctuations are small, reflecting only the intrinsic noise of the hardware and thus the transfer function shows a sudden increase, similar to a \gls{relu} transfer function. One way to smoothen the course of the transfer function is to increasing $\sigma$ by injecting a continuous stream of inhibitory and excitatory Poisson spikes.

\begin{figure}
	\begin{center}
		%% Creator: Matplotlib, PGF backend
%%
%% To include the figure in your LaTeX document, write
%%   \input{<filename>.pgf}
%%
%% Make sure the required packages are loaded in your preamble
%%   \usepackage{pgf}
%%
%% Figures using additional raster images can only be included by \input if
%% they are in the same directory as the main LaTeX file. For loading figures
%% from other directories you can use the `import` package
%%   \usepackage{import}
%% and then include the figures with
%%   \import{<path to file>}{<filename>.pgf}
%%
%% Matplotlib used the following preamble
%%   \usepackage{amsmath} \usepackage{pifont} \usepackage{xcolor} \definecolor{green}{HTML}{467821} \definecolor{red}{HTML}{CF4457} \usepackage[detect-all]{siunitx}
%%   \usepackage{fontspec}
%%
\begingroup%
\makeatletter%
\begin{pgfpicture}%
\pgfpathrectangle{\pgfpointorigin}{\pgfqpoint{4.501905in}{2.793578in}}%
\pgfusepath{use as bounding box, clip}%
\begin{pgfscope}%
\pgfsetbuttcap%
\pgfsetmiterjoin%
\pgfsetlinewidth{0.000000pt}%
\definecolor{currentstroke}{rgb}{0.000000,0.000000,0.000000}%
\pgfsetstrokecolor{currentstroke}%
\pgfsetstrokeopacity{0.000000}%
\pgfsetdash{}{0pt}%
\pgfpathmoveto{\pgfqpoint{0.000000in}{0.000000in}}%
\pgfpathlineto{\pgfqpoint{4.501905in}{0.000000in}}%
\pgfpathlineto{\pgfqpoint{4.501905in}{2.793578in}}%
\pgfpathlineto{\pgfqpoint{0.000000in}{2.793578in}}%
\pgfpathclose%
\pgfusepath{}%
\end{pgfscope}%
\begin{pgfscope}%
\pgfsetbuttcap%
\pgfsetmiterjoin%
\pgfsetlinewidth{0.000000pt}%
\definecolor{currentstroke}{rgb}{0.000000,0.000000,0.000000}%
\pgfsetstrokecolor{currentstroke}%
\pgfsetstrokeopacity{0.000000}%
\pgfsetdash{}{0pt}%
\pgfpathmoveto{\pgfqpoint{0.526905in}{0.383578in}}%
\pgfpathlineto{\pgfqpoint{4.401905in}{0.383578in}}%
\pgfpathlineto{\pgfqpoint{4.401905in}{2.693578in}}%
\pgfpathlineto{\pgfqpoint{0.526905in}{2.693578in}}%
\pgfpathclose%
\pgfusepath{}%
\end{pgfscope}%
\begin{pgfscope}%
\pgfsetbuttcap%
\pgfsetroundjoin%
\definecolor{currentfill}{rgb}{0.317647,0.317647,0.317647}%
\pgfsetfillcolor{currentfill}%
\pgfsetlinewidth{0.501875pt}%
\definecolor{currentstroke}{rgb}{0.317647,0.317647,0.317647}%
\pgfsetstrokecolor{currentstroke}%
\pgfsetdash{}{0pt}%
\pgfsys@defobject{currentmarker}{\pgfqpoint{0.000000in}{-0.020833in}}{\pgfqpoint{0.000000in}{0.000000in}}{%
\pgfpathmoveto{\pgfqpoint{0.000000in}{0.000000in}}%
\pgfpathlineto{\pgfqpoint{0.000000in}{-0.020833in}}%
\pgfusepath{stroke,fill}%
}%
\begin{pgfscope}%
\pgfsys@transformshift{0.543128in}{0.383578in}%
\pgfsys@useobject{currentmarker}{}%
\end{pgfscope}%
\end{pgfscope}%
\begin{pgfscope}%
\definecolor{textcolor}{rgb}{0.317647,0.317647,0.317647}%
\pgfsetstrokecolor{textcolor}%
\pgfsetfillcolor{textcolor}%
\pgftext[x=0.543128in,y=0.334967in,,top]{\color{textcolor}\rmfamily\fontsize{6.664000}{7.996800}\selectfont \(\displaystyle 400\)}%
\end{pgfscope}%
\begin{pgfscope}%
\pgfsetbuttcap%
\pgfsetroundjoin%
\definecolor{currentfill}{rgb}{0.317647,0.317647,0.317647}%
\pgfsetfillcolor{currentfill}%
\pgfsetlinewidth{0.501875pt}%
\definecolor{currentstroke}{rgb}{0.317647,0.317647,0.317647}%
\pgfsetstrokecolor{currentstroke}%
\pgfsetdash{}{0pt}%
\pgfsys@defobject{currentmarker}{\pgfqpoint{0.000000in}{-0.020833in}}{\pgfqpoint{0.000000in}{0.000000in}}{%
\pgfpathmoveto{\pgfqpoint{0.000000in}{0.000000in}}%
\pgfpathlineto{\pgfqpoint{0.000000in}{-0.020833in}}%
\pgfusepath{stroke,fill}%
}%
\begin{pgfscope}%
\pgfsys@transformshift{1.228682in}{0.383578in}%
\pgfsys@useobject{currentmarker}{}%
\end{pgfscope}%
\end{pgfscope}%
\begin{pgfscope}%
\definecolor{textcolor}{rgb}{0.317647,0.317647,0.317647}%
\pgfsetstrokecolor{textcolor}%
\pgfsetfillcolor{textcolor}%
\pgftext[x=1.228682in,y=0.334967in,,top]{\color{textcolor}\rmfamily\fontsize{6.664000}{7.996800}\selectfont \(\displaystyle 420\)}%
\end{pgfscope}%
\begin{pgfscope}%
\pgfsetbuttcap%
\pgfsetroundjoin%
\definecolor{currentfill}{rgb}{0.317647,0.317647,0.317647}%
\pgfsetfillcolor{currentfill}%
\pgfsetlinewidth{0.501875pt}%
\definecolor{currentstroke}{rgb}{0.317647,0.317647,0.317647}%
\pgfsetstrokecolor{currentstroke}%
\pgfsetdash{}{0pt}%
\pgfsys@defobject{currentmarker}{\pgfqpoint{0.000000in}{-0.020833in}}{\pgfqpoint{0.000000in}{0.000000in}}{%
\pgfpathmoveto{\pgfqpoint{0.000000in}{0.000000in}}%
\pgfpathlineto{\pgfqpoint{0.000000in}{-0.020833in}}%
\pgfusepath{stroke,fill}%
}%
\begin{pgfscope}%
\pgfsys@transformshift{1.914237in}{0.383578in}%
\pgfsys@useobject{currentmarker}{}%
\end{pgfscope}%
\end{pgfscope}%
\begin{pgfscope}%
\definecolor{textcolor}{rgb}{0.317647,0.317647,0.317647}%
\pgfsetstrokecolor{textcolor}%
\pgfsetfillcolor{textcolor}%
\pgftext[x=1.914237in,y=0.334967in,,top]{\color{textcolor}\rmfamily\fontsize{6.664000}{7.996800}\selectfont \(\displaystyle 440\)}%
\end{pgfscope}%
\begin{pgfscope}%
\pgfsetbuttcap%
\pgfsetroundjoin%
\definecolor{currentfill}{rgb}{0.317647,0.317647,0.317647}%
\pgfsetfillcolor{currentfill}%
\pgfsetlinewidth{0.501875pt}%
\definecolor{currentstroke}{rgb}{0.317647,0.317647,0.317647}%
\pgfsetstrokecolor{currentstroke}%
\pgfsetdash{}{0pt}%
\pgfsys@defobject{currentmarker}{\pgfqpoint{0.000000in}{-0.020833in}}{\pgfqpoint{0.000000in}{0.000000in}}{%
\pgfpathmoveto{\pgfqpoint{0.000000in}{0.000000in}}%
\pgfpathlineto{\pgfqpoint{0.000000in}{-0.020833in}}%
\pgfusepath{stroke,fill}%
}%
\begin{pgfscope}%
\pgfsys@transformshift{2.599792in}{0.383578in}%
\pgfsys@useobject{currentmarker}{}%
\end{pgfscope}%
\end{pgfscope}%
\begin{pgfscope}%
\definecolor{textcolor}{rgb}{0.317647,0.317647,0.317647}%
\pgfsetstrokecolor{textcolor}%
\pgfsetfillcolor{textcolor}%
\pgftext[x=2.599792in,y=0.334967in,,top]{\color{textcolor}\rmfamily\fontsize{6.664000}{7.996800}\selectfont \(\displaystyle 460\)}%
\end{pgfscope}%
\begin{pgfscope}%
\pgfsetbuttcap%
\pgfsetroundjoin%
\definecolor{currentfill}{rgb}{0.317647,0.317647,0.317647}%
\pgfsetfillcolor{currentfill}%
\pgfsetlinewidth{0.501875pt}%
\definecolor{currentstroke}{rgb}{0.317647,0.317647,0.317647}%
\pgfsetstrokecolor{currentstroke}%
\pgfsetdash{}{0pt}%
\pgfsys@defobject{currentmarker}{\pgfqpoint{0.000000in}{-0.020833in}}{\pgfqpoint{0.000000in}{0.000000in}}{%
\pgfpathmoveto{\pgfqpoint{0.000000in}{0.000000in}}%
\pgfpathlineto{\pgfqpoint{0.000000in}{-0.020833in}}%
\pgfusepath{stroke,fill}%
}%
\begin{pgfscope}%
\pgfsys@transformshift{3.285346in}{0.383578in}%
\pgfsys@useobject{currentmarker}{}%
\end{pgfscope}%
\end{pgfscope}%
\begin{pgfscope}%
\definecolor{textcolor}{rgb}{0.317647,0.317647,0.317647}%
\pgfsetstrokecolor{textcolor}%
\pgfsetfillcolor{textcolor}%
\pgftext[x=3.285346in,y=0.334967in,,top]{\color{textcolor}\rmfamily\fontsize{6.664000}{7.996800}\selectfont \(\displaystyle 480\)}%
\end{pgfscope}%
\begin{pgfscope}%
\pgfsetbuttcap%
\pgfsetroundjoin%
\definecolor{currentfill}{rgb}{0.317647,0.317647,0.317647}%
\pgfsetfillcolor{currentfill}%
\pgfsetlinewidth{0.501875pt}%
\definecolor{currentstroke}{rgb}{0.317647,0.317647,0.317647}%
\pgfsetstrokecolor{currentstroke}%
\pgfsetdash{}{0pt}%
\pgfsys@defobject{currentmarker}{\pgfqpoint{0.000000in}{-0.020833in}}{\pgfqpoint{0.000000in}{0.000000in}}{%
\pgfpathmoveto{\pgfqpoint{0.000000in}{0.000000in}}%
\pgfpathlineto{\pgfqpoint{0.000000in}{-0.020833in}}%
\pgfusepath{stroke,fill}%
}%
\begin{pgfscope}%
\pgfsys@transformshift{3.970901in}{0.383578in}%
\pgfsys@useobject{currentmarker}{}%
\end{pgfscope}%
\end{pgfscope}%
\begin{pgfscope}%
\definecolor{textcolor}{rgb}{0.317647,0.317647,0.317647}%
\pgfsetstrokecolor{textcolor}%
\pgfsetfillcolor{textcolor}%
\pgftext[x=3.970901in,y=0.334967in,,top]{\color{textcolor}\rmfamily\fontsize{6.664000}{7.996800}\selectfont \(\displaystyle 500\)}%
\end{pgfscope}%
\begin{pgfscope}%
\definecolor{textcolor}{rgb}{0.317647,0.317647,0.317647}%
\pgfsetstrokecolor{textcolor}%
\pgfsetfillcolor{textcolor}%
\pgftext[x=2.464405in,y=0.197222in,,top]{\color{textcolor}\rmfamily\fontsize{6.664000}{7.996800}\selectfont \(\displaystyle V_\mathrm{m} \; (\si{\milli \V})\)}%
\end{pgfscope}%
\begin{pgfscope}%
\pgfsetbuttcap%
\pgfsetroundjoin%
\definecolor{currentfill}{rgb}{0.317647,0.317647,0.317647}%
\pgfsetfillcolor{currentfill}%
\pgfsetlinewidth{0.501875pt}%
\definecolor{currentstroke}{rgb}{0.317647,0.317647,0.317647}%
\pgfsetstrokecolor{currentstroke}%
\pgfsetdash{}{0pt}%
\pgfsys@defobject{currentmarker}{\pgfqpoint{-0.020833in}{0.000000in}}{\pgfqpoint{0.000000in}{0.000000in}}{%
\pgfpathmoveto{\pgfqpoint{0.000000in}{0.000000in}}%
\pgfpathlineto{\pgfqpoint{-0.020833in}{0.000000in}}%
\pgfusepath{stroke,fill}%
}%
\begin{pgfscope}%
\pgfsys@transformshift{0.526905in}{0.383578in}%
\pgfsys@useobject{currentmarker}{}%
\end{pgfscope}%
\end{pgfscope}%
\begin{pgfscope}%
\definecolor{textcolor}{rgb}{0.317647,0.317647,0.317647}%
\pgfsetstrokecolor{textcolor}%
\pgfsetfillcolor{textcolor}%
\pgftext[x=0.237745in,y=0.351461in,left,base]{\color{textcolor}\rmfamily\fontsize{6.664000}{7.996800}\selectfont \(\displaystyle 0.000\)}%
\end{pgfscope}%
\begin{pgfscope}%
\pgfsetbuttcap%
\pgfsetroundjoin%
\definecolor{currentfill}{rgb}{0.317647,0.317647,0.317647}%
\pgfsetfillcolor{currentfill}%
\pgfsetlinewidth{0.501875pt}%
\definecolor{currentstroke}{rgb}{0.317647,0.317647,0.317647}%
\pgfsetstrokecolor{currentstroke}%
\pgfsetdash{}{0pt}%
\pgfsys@defobject{currentmarker}{\pgfqpoint{-0.020833in}{0.000000in}}{\pgfqpoint{0.000000in}{0.000000in}}{%
\pgfpathmoveto{\pgfqpoint{0.000000in}{0.000000in}}%
\pgfpathlineto{\pgfqpoint{-0.020833in}{0.000000in}}%
\pgfusepath{stroke,fill}%
}%
\begin{pgfscope}%
\pgfsys@transformshift{0.526905in}{0.759929in}%
\pgfsys@useobject{currentmarker}{}%
\end{pgfscope}%
\end{pgfscope}%
\begin{pgfscope}%
\definecolor{textcolor}{rgb}{0.317647,0.317647,0.317647}%
\pgfsetstrokecolor{textcolor}%
\pgfsetfillcolor{textcolor}%
\pgftext[x=0.237745in,y=0.727812in,left,base]{\color{textcolor}\rmfamily\fontsize{6.664000}{7.996800}\selectfont \(\displaystyle 0.005\)}%
\end{pgfscope}%
\begin{pgfscope}%
\pgfsetbuttcap%
\pgfsetroundjoin%
\definecolor{currentfill}{rgb}{0.317647,0.317647,0.317647}%
\pgfsetfillcolor{currentfill}%
\pgfsetlinewidth{0.501875pt}%
\definecolor{currentstroke}{rgb}{0.317647,0.317647,0.317647}%
\pgfsetstrokecolor{currentstroke}%
\pgfsetdash{}{0pt}%
\pgfsys@defobject{currentmarker}{\pgfqpoint{-0.020833in}{0.000000in}}{\pgfqpoint{0.000000in}{0.000000in}}{%
\pgfpathmoveto{\pgfqpoint{0.000000in}{0.000000in}}%
\pgfpathlineto{\pgfqpoint{-0.020833in}{0.000000in}}%
\pgfusepath{stroke,fill}%
}%
\begin{pgfscope}%
\pgfsys@transformshift{0.526905in}{1.136280in}%
\pgfsys@useobject{currentmarker}{}%
\end{pgfscope}%
\end{pgfscope}%
\begin{pgfscope}%
\definecolor{textcolor}{rgb}{0.317647,0.317647,0.317647}%
\pgfsetstrokecolor{textcolor}%
\pgfsetfillcolor{textcolor}%
\pgftext[x=0.237745in,y=1.104163in,left,base]{\color{textcolor}\rmfamily\fontsize{6.664000}{7.996800}\selectfont \(\displaystyle 0.010\)}%
\end{pgfscope}%
\begin{pgfscope}%
\pgfsetbuttcap%
\pgfsetroundjoin%
\definecolor{currentfill}{rgb}{0.317647,0.317647,0.317647}%
\pgfsetfillcolor{currentfill}%
\pgfsetlinewidth{0.501875pt}%
\definecolor{currentstroke}{rgb}{0.317647,0.317647,0.317647}%
\pgfsetstrokecolor{currentstroke}%
\pgfsetdash{}{0pt}%
\pgfsys@defobject{currentmarker}{\pgfqpoint{-0.020833in}{0.000000in}}{\pgfqpoint{0.000000in}{0.000000in}}{%
\pgfpathmoveto{\pgfqpoint{0.000000in}{0.000000in}}%
\pgfpathlineto{\pgfqpoint{-0.020833in}{0.000000in}}%
\pgfusepath{stroke,fill}%
}%
\begin{pgfscope}%
\pgfsys@transformshift{0.526905in}{1.512630in}%
\pgfsys@useobject{currentmarker}{}%
\end{pgfscope}%
\end{pgfscope}%
\begin{pgfscope}%
\definecolor{textcolor}{rgb}{0.317647,0.317647,0.317647}%
\pgfsetstrokecolor{textcolor}%
\pgfsetfillcolor{textcolor}%
\pgftext[x=0.237745in,y=1.480513in,left,base]{\color{textcolor}\rmfamily\fontsize{6.664000}{7.996800}\selectfont \(\displaystyle 0.015\)}%
\end{pgfscope}%
\begin{pgfscope}%
\pgfsetbuttcap%
\pgfsetroundjoin%
\definecolor{currentfill}{rgb}{0.317647,0.317647,0.317647}%
\pgfsetfillcolor{currentfill}%
\pgfsetlinewidth{0.501875pt}%
\definecolor{currentstroke}{rgb}{0.317647,0.317647,0.317647}%
\pgfsetstrokecolor{currentstroke}%
\pgfsetdash{}{0pt}%
\pgfsys@defobject{currentmarker}{\pgfqpoint{-0.020833in}{0.000000in}}{\pgfqpoint{0.000000in}{0.000000in}}{%
\pgfpathmoveto{\pgfqpoint{0.000000in}{0.000000in}}%
\pgfpathlineto{\pgfqpoint{-0.020833in}{0.000000in}}%
\pgfusepath{stroke,fill}%
}%
\begin{pgfscope}%
\pgfsys@transformshift{0.526905in}{1.888981in}%
\pgfsys@useobject{currentmarker}{}%
\end{pgfscope}%
\end{pgfscope}%
\begin{pgfscope}%
\definecolor{textcolor}{rgb}{0.317647,0.317647,0.317647}%
\pgfsetstrokecolor{textcolor}%
\pgfsetfillcolor{textcolor}%
\pgftext[x=0.237745in,y=1.856864in,left,base]{\color{textcolor}\rmfamily\fontsize{6.664000}{7.996800}\selectfont \(\displaystyle 0.020\)}%
\end{pgfscope}%
\begin{pgfscope}%
\pgfsetbuttcap%
\pgfsetroundjoin%
\definecolor{currentfill}{rgb}{0.317647,0.317647,0.317647}%
\pgfsetfillcolor{currentfill}%
\pgfsetlinewidth{0.501875pt}%
\definecolor{currentstroke}{rgb}{0.317647,0.317647,0.317647}%
\pgfsetstrokecolor{currentstroke}%
\pgfsetdash{}{0pt}%
\pgfsys@defobject{currentmarker}{\pgfqpoint{-0.020833in}{0.000000in}}{\pgfqpoint{0.000000in}{0.000000in}}{%
\pgfpathmoveto{\pgfqpoint{0.000000in}{0.000000in}}%
\pgfpathlineto{\pgfqpoint{-0.020833in}{0.000000in}}%
\pgfusepath{stroke,fill}%
}%
\begin{pgfscope}%
\pgfsys@transformshift{0.526905in}{2.265332in}%
\pgfsys@useobject{currentmarker}{}%
\end{pgfscope}%
\end{pgfscope}%
\begin{pgfscope}%
\definecolor{textcolor}{rgb}{0.317647,0.317647,0.317647}%
\pgfsetstrokecolor{textcolor}%
\pgfsetfillcolor{textcolor}%
\pgftext[x=0.237745in,y=2.233215in,left,base]{\color{textcolor}\rmfamily\fontsize{6.664000}{7.996800}\selectfont \(\displaystyle 0.025\)}%
\end{pgfscope}%
\begin{pgfscope}%
\pgfsetbuttcap%
\pgfsetroundjoin%
\definecolor{currentfill}{rgb}{0.317647,0.317647,0.317647}%
\pgfsetfillcolor{currentfill}%
\pgfsetlinewidth{0.501875pt}%
\definecolor{currentstroke}{rgb}{0.317647,0.317647,0.317647}%
\pgfsetstrokecolor{currentstroke}%
\pgfsetdash{}{0pt}%
\pgfsys@defobject{currentmarker}{\pgfqpoint{-0.020833in}{0.000000in}}{\pgfqpoint{0.000000in}{0.000000in}}{%
\pgfpathmoveto{\pgfqpoint{0.000000in}{0.000000in}}%
\pgfpathlineto{\pgfqpoint{-0.020833in}{0.000000in}}%
\pgfusepath{stroke,fill}%
}%
\begin{pgfscope}%
\pgfsys@transformshift{0.526905in}{2.641682in}%
\pgfsys@useobject{currentmarker}{}%
\end{pgfscope}%
\end{pgfscope}%
\begin{pgfscope}%
\definecolor{textcolor}{rgb}{0.317647,0.317647,0.317647}%
\pgfsetstrokecolor{textcolor}%
\pgfsetfillcolor{textcolor}%
\pgftext[x=0.237745in,y=2.609566in,left,base]{\color{textcolor}\rmfamily\fontsize{6.664000}{7.996800}\selectfont \(\displaystyle 0.030\)}%
\end{pgfscope}%
\begin{pgfscope}%
\definecolor{textcolor}{rgb}{0.317647,0.317647,0.317647}%
\pgfsetstrokecolor{textcolor}%
\pgfsetfillcolor{textcolor}%
\pgftext[x=0.182189in,y=1.538578in,,bottom,rotate=90.000000]{\color{textcolor}\rmfamily\fontsize{6.664000}{7.996800}\selectfont Density}%
\end{pgfscope}%
\begin{pgfscope}%
\pgfpathrectangle{\pgfqpoint{0.526905in}{0.383578in}}{\pgfqpoint{3.875000in}{2.310000in}}%
\pgfusepath{clip}%
\pgfsetbuttcap%
\pgfsetmiterjoin%
\definecolor{currentfill}{rgb}{0.686275,0.352941,0.313725}%
\pgfsetfillcolor{currentfill}%
\pgfsetfillopacity{0.300000}%
\pgfsetlinewidth{0.000000pt}%
\definecolor{currentstroke}{rgb}{0.000000,0.000000,0.000000}%
\pgfsetstrokecolor{currentstroke}%
\pgfsetstrokeopacity{0.300000}%
\pgfsetdash{}{0pt}%
\pgfpathmoveto{\pgfqpoint{0.703041in}{0.383578in}}%
\pgfpathlineto{\pgfqpoint{0.720630in}{0.383578in}}%
\pgfpathlineto{\pgfqpoint{0.720630in}{0.389190in}}%
\pgfpathlineto{\pgfqpoint{0.703041in}{0.389190in}}%
\pgfpathclose%
\pgfusepath{fill}%
\end{pgfscope}%
\begin{pgfscope}%
\pgfpathrectangle{\pgfqpoint{0.526905in}{0.383578in}}{\pgfqpoint{3.875000in}{2.310000in}}%
\pgfusepath{clip}%
\pgfsetbuttcap%
\pgfsetmiterjoin%
\definecolor{currentfill}{rgb}{0.686275,0.352941,0.313725}%
\pgfsetfillcolor{currentfill}%
\pgfsetfillopacity{0.300000}%
\pgfsetlinewidth{0.000000pt}%
\definecolor{currentstroke}{rgb}{0.000000,0.000000,0.000000}%
\pgfsetstrokecolor{currentstroke}%
\pgfsetstrokeopacity{0.300000}%
\pgfsetdash{}{0pt}%
\pgfpathmoveto{\pgfqpoint{0.720630in}{0.383578in}}%
\pgfpathlineto{\pgfqpoint{0.738219in}{0.383578in}}%
\pgfpathlineto{\pgfqpoint{0.738219in}{0.383578in}}%
\pgfpathlineto{\pgfqpoint{0.720630in}{0.383578in}}%
\pgfpathclose%
\pgfusepath{fill}%
\end{pgfscope}%
\begin{pgfscope}%
\pgfpathrectangle{\pgfqpoint{0.526905in}{0.383578in}}{\pgfqpoint{3.875000in}{2.310000in}}%
\pgfusepath{clip}%
\pgfsetbuttcap%
\pgfsetmiterjoin%
\definecolor{currentfill}{rgb}{0.686275,0.352941,0.313725}%
\pgfsetfillcolor{currentfill}%
\pgfsetfillopacity{0.300000}%
\pgfsetlinewidth{0.000000pt}%
\definecolor{currentstroke}{rgb}{0.000000,0.000000,0.000000}%
\pgfsetstrokecolor{currentstroke}%
\pgfsetstrokeopacity{0.300000}%
\pgfsetdash{}{0pt}%
\pgfpathmoveto{\pgfqpoint{0.738219in}{0.383578in}}%
\pgfpathlineto{\pgfqpoint{0.755808in}{0.383578in}}%
\pgfpathlineto{\pgfqpoint{0.755808in}{0.394803in}}%
\pgfpathlineto{\pgfqpoint{0.738219in}{0.394803in}}%
\pgfpathclose%
\pgfusepath{fill}%
\end{pgfscope}%
\begin{pgfscope}%
\pgfpathrectangle{\pgfqpoint{0.526905in}{0.383578in}}{\pgfqpoint{3.875000in}{2.310000in}}%
\pgfusepath{clip}%
\pgfsetbuttcap%
\pgfsetmiterjoin%
\definecolor{currentfill}{rgb}{0.686275,0.352941,0.313725}%
\pgfsetfillcolor{currentfill}%
\pgfsetfillopacity{0.300000}%
\pgfsetlinewidth{0.000000pt}%
\definecolor{currentstroke}{rgb}{0.000000,0.000000,0.000000}%
\pgfsetstrokecolor{currentstroke}%
\pgfsetstrokeopacity{0.300000}%
\pgfsetdash{}{0pt}%
\pgfpathmoveto{\pgfqpoint{0.755808in}{0.383578in}}%
\pgfpathlineto{\pgfqpoint{0.773397in}{0.383578in}}%
\pgfpathlineto{\pgfqpoint{0.773397in}{0.394803in}}%
\pgfpathlineto{\pgfqpoint{0.755808in}{0.394803in}}%
\pgfpathclose%
\pgfusepath{fill}%
\end{pgfscope}%
\begin{pgfscope}%
\pgfpathrectangle{\pgfqpoint{0.526905in}{0.383578in}}{\pgfqpoint{3.875000in}{2.310000in}}%
\pgfusepath{clip}%
\pgfsetbuttcap%
\pgfsetmiterjoin%
\definecolor{currentfill}{rgb}{0.686275,0.352941,0.313725}%
\pgfsetfillcolor{currentfill}%
\pgfsetfillopacity{0.300000}%
\pgfsetlinewidth{0.000000pt}%
\definecolor{currentstroke}{rgb}{0.000000,0.000000,0.000000}%
\pgfsetstrokecolor{currentstroke}%
\pgfsetstrokeopacity{0.300000}%
\pgfsetdash{}{0pt}%
\pgfpathmoveto{\pgfqpoint{0.773397in}{0.383578in}}%
\pgfpathlineto{\pgfqpoint{0.790986in}{0.383578in}}%
\pgfpathlineto{\pgfqpoint{0.790986in}{0.389190in}}%
\pgfpathlineto{\pgfqpoint{0.773397in}{0.389190in}}%
\pgfpathclose%
\pgfusepath{fill}%
\end{pgfscope}%
\begin{pgfscope}%
\pgfpathrectangle{\pgfqpoint{0.526905in}{0.383578in}}{\pgfqpoint{3.875000in}{2.310000in}}%
\pgfusepath{clip}%
\pgfsetbuttcap%
\pgfsetmiterjoin%
\definecolor{currentfill}{rgb}{0.686275,0.352941,0.313725}%
\pgfsetfillcolor{currentfill}%
\pgfsetfillopacity{0.300000}%
\pgfsetlinewidth{0.000000pt}%
\definecolor{currentstroke}{rgb}{0.000000,0.000000,0.000000}%
\pgfsetstrokecolor{currentstroke}%
\pgfsetstrokeopacity{0.300000}%
\pgfsetdash{}{0pt}%
\pgfpathmoveto{\pgfqpoint{0.790986in}{0.383578in}}%
\pgfpathlineto{\pgfqpoint{0.808575in}{0.383578in}}%
\pgfpathlineto{\pgfqpoint{0.808575in}{0.389190in}}%
\pgfpathlineto{\pgfqpoint{0.790986in}{0.389190in}}%
\pgfpathclose%
\pgfusepath{fill}%
\end{pgfscope}%
\begin{pgfscope}%
\pgfpathrectangle{\pgfqpoint{0.526905in}{0.383578in}}{\pgfqpoint{3.875000in}{2.310000in}}%
\pgfusepath{clip}%
\pgfsetbuttcap%
\pgfsetmiterjoin%
\definecolor{currentfill}{rgb}{0.686275,0.352941,0.313725}%
\pgfsetfillcolor{currentfill}%
\pgfsetfillopacity{0.300000}%
\pgfsetlinewidth{0.000000pt}%
\definecolor{currentstroke}{rgb}{0.000000,0.000000,0.000000}%
\pgfsetstrokecolor{currentstroke}%
\pgfsetstrokeopacity{0.300000}%
\pgfsetdash{}{0pt}%
\pgfpathmoveto{\pgfqpoint{0.808575in}{0.383578in}}%
\pgfpathlineto{\pgfqpoint{0.826164in}{0.383578in}}%
\pgfpathlineto{\pgfqpoint{0.826164in}{0.394803in}}%
\pgfpathlineto{\pgfqpoint{0.808575in}{0.394803in}}%
\pgfpathclose%
\pgfusepath{fill}%
\end{pgfscope}%
\begin{pgfscope}%
\pgfpathrectangle{\pgfqpoint{0.526905in}{0.383578in}}{\pgfqpoint{3.875000in}{2.310000in}}%
\pgfusepath{clip}%
\pgfsetbuttcap%
\pgfsetmiterjoin%
\definecolor{currentfill}{rgb}{0.686275,0.352941,0.313725}%
\pgfsetfillcolor{currentfill}%
\pgfsetfillopacity{0.300000}%
\pgfsetlinewidth{0.000000pt}%
\definecolor{currentstroke}{rgb}{0.000000,0.000000,0.000000}%
\pgfsetstrokecolor{currentstroke}%
\pgfsetstrokeopacity{0.300000}%
\pgfsetdash{}{0pt}%
\pgfpathmoveto{\pgfqpoint{0.826164in}{0.383578in}}%
\pgfpathlineto{\pgfqpoint{0.843754in}{0.383578in}}%
\pgfpathlineto{\pgfqpoint{0.843754in}{0.389190in}}%
\pgfpathlineto{\pgfqpoint{0.826164in}{0.389190in}}%
\pgfpathclose%
\pgfusepath{fill}%
\end{pgfscope}%
\begin{pgfscope}%
\pgfpathrectangle{\pgfqpoint{0.526905in}{0.383578in}}{\pgfqpoint{3.875000in}{2.310000in}}%
\pgfusepath{clip}%
\pgfsetbuttcap%
\pgfsetmiterjoin%
\definecolor{currentfill}{rgb}{0.686275,0.352941,0.313725}%
\pgfsetfillcolor{currentfill}%
\pgfsetfillopacity{0.300000}%
\pgfsetlinewidth{0.000000pt}%
\definecolor{currentstroke}{rgb}{0.000000,0.000000,0.000000}%
\pgfsetstrokecolor{currentstroke}%
\pgfsetstrokeopacity{0.300000}%
\pgfsetdash{}{0pt}%
\pgfpathmoveto{\pgfqpoint{0.843754in}{0.383578in}}%
\pgfpathlineto{\pgfqpoint{0.861343in}{0.383578in}}%
\pgfpathlineto{\pgfqpoint{0.861343in}{0.383578in}}%
\pgfpathlineto{\pgfqpoint{0.843754in}{0.383578in}}%
\pgfpathclose%
\pgfusepath{fill}%
\end{pgfscope}%
\begin{pgfscope}%
\pgfpathrectangle{\pgfqpoint{0.526905in}{0.383578in}}{\pgfqpoint{3.875000in}{2.310000in}}%
\pgfusepath{clip}%
\pgfsetbuttcap%
\pgfsetmiterjoin%
\definecolor{currentfill}{rgb}{0.686275,0.352941,0.313725}%
\pgfsetfillcolor{currentfill}%
\pgfsetfillopacity{0.300000}%
\pgfsetlinewidth{0.000000pt}%
\definecolor{currentstroke}{rgb}{0.000000,0.000000,0.000000}%
\pgfsetstrokecolor{currentstroke}%
\pgfsetstrokeopacity{0.300000}%
\pgfsetdash{}{0pt}%
\pgfpathmoveto{\pgfqpoint{0.861343in}{0.383578in}}%
\pgfpathlineto{\pgfqpoint{0.878932in}{0.383578in}}%
\pgfpathlineto{\pgfqpoint{0.878932in}{0.394803in}}%
\pgfpathlineto{\pgfqpoint{0.861343in}{0.394803in}}%
\pgfpathclose%
\pgfusepath{fill}%
\end{pgfscope}%
\begin{pgfscope}%
\pgfpathrectangle{\pgfqpoint{0.526905in}{0.383578in}}{\pgfqpoint{3.875000in}{2.310000in}}%
\pgfusepath{clip}%
\pgfsetbuttcap%
\pgfsetmiterjoin%
\definecolor{currentfill}{rgb}{0.686275,0.352941,0.313725}%
\pgfsetfillcolor{currentfill}%
\pgfsetfillopacity{0.300000}%
\pgfsetlinewidth{0.000000pt}%
\definecolor{currentstroke}{rgb}{0.000000,0.000000,0.000000}%
\pgfsetstrokecolor{currentstroke}%
\pgfsetstrokeopacity{0.300000}%
\pgfsetdash{}{0pt}%
\pgfpathmoveto{\pgfqpoint{0.878932in}{0.383578in}}%
\pgfpathlineto{\pgfqpoint{0.896521in}{0.383578in}}%
\pgfpathlineto{\pgfqpoint{0.896521in}{0.400415in}}%
\pgfpathlineto{\pgfqpoint{0.878932in}{0.400415in}}%
\pgfpathclose%
\pgfusepath{fill}%
\end{pgfscope}%
\begin{pgfscope}%
\pgfpathrectangle{\pgfqpoint{0.526905in}{0.383578in}}{\pgfqpoint{3.875000in}{2.310000in}}%
\pgfusepath{clip}%
\pgfsetbuttcap%
\pgfsetmiterjoin%
\definecolor{currentfill}{rgb}{0.686275,0.352941,0.313725}%
\pgfsetfillcolor{currentfill}%
\pgfsetfillopacity{0.300000}%
\pgfsetlinewidth{0.000000pt}%
\definecolor{currentstroke}{rgb}{0.000000,0.000000,0.000000}%
\pgfsetstrokecolor{currentstroke}%
\pgfsetstrokeopacity{0.300000}%
\pgfsetdash{}{0pt}%
\pgfpathmoveto{\pgfqpoint{0.896521in}{0.383578in}}%
\pgfpathlineto{\pgfqpoint{0.914110in}{0.383578in}}%
\pgfpathlineto{\pgfqpoint{0.914110in}{0.389190in}}%
\pgfpathlineto{\pgfqpoint{0.896521in}{0.389190in}}%
\pgfpathclose%
\pgfusepath{fill}%
\end{pgfscope}%
\begin{pgfscope}%
\pgfpathrectangle{\pgfqpoint{0.526905in}{0.383578in}}{\pgfqpoint{3.875000in}{2.310000in}}%
\pgfusepath{clip}%
\pgfsetbuttcap%
\pgfsetmiterjoin%
\definecolor{currentfill}{rgb}{0.686275,0.352941,0.313725}%
\pgfsetfillcolor{currentfill}%
\pgfsetfillopacity{0.300000}%
\pgfsetlinewidth{0.000000pt}%
\definecolor{currentstroke}{rgb}{0.000000,0.000000,0.000000}%
\pgfsetstrokecolor{currentstroke}%
\pgfsetstrokeopacity{0.300000}%
\pgfsetdash{}{0pt}%
\pgfpathmoveto{\pgfqpoint{0.914110in}{0.383578in}}%
\pgfpathlineto{\pgfqpoint{0.931699in}{0.383578in}}%
\pgfpathlineto{\pgfqpoint{0.931699in}{0.383578in}}%
\pgfpathlineto{\pgfqpoint{0.914110in}{0.383578in}}%
\pgfpathclose%
\pgfusepath{fill}%
\end{pgfscope}%
\begin{pgfscope}%
\pgfpathrectangle{\pgfqpoint{0.526905in}{0.383578in}}{\pgfqpoint{3.875000in}{2.310000in}}%
\pgfusepath{clip}%
\pgfsetbuttcap%
\pgfsetmiterjoin%
\definecolor{currentfill}{rgb}{0.686275,0.352941,0.313725}%
\pgfsetfillcolor{currentfill}%
\pgfsetfillopacity{0.300000}%
\pgfsetlinewidth{0.000000pt}%
\definecolor{currentstroke}{rgb}{0.000000,0.000000,0.000000}%
\pgfsetstrokecolor{currentstroke}%
\pgfsetstrokeopacity{0.300000}%
\pgfsetdash{}{0pt}%
\pgfpathmoveto{\pgfqpoint{0.931699in}{0.383578in}}%
\pgfpathlineto{\pgfqpoint{0.949288in}{0.383578in}}%
\pgfpathlineto{\pgfqpoint{0.949288in}{0.406027in}}%
\pgfpathlineto{\pgfqpoint{0.931699in}{0.406027in}}%
\pgfpathclose%
\pgfusepath{fill}%
\end{pgfscope}%
\begin{pgfscope}%
\pgfpathrectangle{\pgfqpoint{0.526905in}{0.383578in}}{\pgfqpoint{3.875000in}{2.310000in}}%
\pgfusepath{clip}%
\pgfsetbuttcap%
\pgfsetmiterjoin%
\definecolor{currentfill}{rgb}{0.686275,0.352941,0.313725}%
\pgfsetfillcolor{currentfill}%
\pgfsetfillopacity{0.300000}%
\pgfsetlinewidth{0.000000pt}%
\definecolor{currentstroke}{rgb}{0.000000,0.000000,0.000000}%
\pgfsetstrokecolor{currentstroke}%
\pgfsetstrokeopacity{0.300000}%
\pgfsetdash{}{0pt}%
\pgfpathmoveto{\pgfqpoint{0.949288in}{0.383578in}}%
\pgfpathlineto{\pgfqpoint{0.966877in}{0.383578in}}%
\pgfpathlineto{\pgfqpoint{0.966877in}{0.394803in}}%
\pgfpathlineto{\pgfqpoint{0.949288in}{0.394803in}}%
\pgfpathclose%
\pgfusepath{fill}%
\end{pgfscope}%
\begin{pgfscope}%
\pgfpathrectangle{\pgfqpoint{0.526905in}{0.383578in}}{\pgfqpoint{3.875000in}{2.310000in}}%
\pgfusepath{clip}%
\pgfsetbuttcap%
\pgfsetmiterjoin%
\definecolor{currentfill}{rgb}{0.686275,0.352941,0.313725}%
\pgfsetfillcolor{currentfill}%
\pgfsetfillopacity{0.300000}%
\pgfsetlinewidth{0.000000pt}%
\definecolor{currentstroke}{rgb}{0.000000,0.000000,0.000000}%
\pgfsetstrokecolor{currentstroke}%
\pgfsetstrokeopacity{0.300000}%
\pgfsetdash{}{0pt}%
\pgfpathmoveto{\pgfqpoint{0.966877in}{0.383578in}}%
\pgfpathlineto{\pgfqpoint{0.984466in}{0.383578in}}%
\pgfpathlineto{\pgfqpoint{0.984466in}{0.400415in}}%
\pgfpathlineto{\pgfqpoint{0.966877in}{0.400415in}}%
\pgfpathclose%
\pgfusepath{fill}%
\end{pgfscope}%
\begin{pgfscope}%
\pgfpathrectangle{\pgfqpoint{0.526905in}{0.383578in}}{\pgfqpoint{3.875000in}{2.310000in}}%
\pgfusepath{clip}%
\pgfsetbuttcap%
\pgfsetmiterjoin%
\definecolor{currentfill}{rgb}{0.686275,0.352941,0.313725}%
\pgfsetfillcolor{currentfill}%
\pgfsetfillopacity{0.300000}%
\pgfsetlinewidth{0.000000pt}%
\definecolor{currentstroke}{rgb}{0.000000,0.000000,0.000000}%
\pgfsetstrokecolor{currentstroke}%
\pgfsetstrokeopacity{0.300000}%
\pgfsetdash{}{0pt}%
\pgfpathmoveto{\pgfqpoint{0.984466in}{0.383578in}}%
\pgfpathlineto{\pgfqpoint{1.002055in}{0.383578in}}%
\pgfpathlineto{\pgfqpoint{1.002055in}{0.411639in}}%
\pgfpathlineto{\pgfqpoint{0.984466in}{0.411639in}}%
\pgfpathclose%
\pgfusepath{fill}%
\end{pgfscope}%
\begin{pgfscope}%
\pgfpathrectangle{\pgfqpoint{0.526905in}{0.383578in}}{\pgfqpoint{3.875000in}{2.310000in}}%
\pgfusepath{clip}%
\pgfsetbuttcap%
\pgfsetmiterjoin%
\definecolor{currentfill}{rgb}{0.686275,0.352941,0.313725}%
\pgfsetfillcolor{currentfill}%
\pgfsetfillopacity{0.300000}%
\pgfsetlinewidth{0.000000pt}%
\definecolor{currentstroke}{rgb}{0.000000,0.000000,0.000000}%
\pgfsetstrokecolor{currentstroke}%
\pgfsetstrokeopacity{0.300000}%
\pgfsetdash{}{0pt}%
\pgfpathmoveto{\pgfqpoint{1.002055in}{0.383578in}}%
\pgfpathlineto{\pgfqpoint{1.019644in}{0.383578in}}%
\pgfpathlineto{\pgfqpoint{1.019644in}{0.389190in}}%
\pgfpathlineto{\pgfqpoint{1.002055in}{0.389190in}}%
\pgfpathclose%
\pgfusepath{fill}%
\end{pgfscope}%
\begin{pgfscope}%
\pgfpathrectangle{\pgfqpoint{0.526905in}{0.383578in}}{\pgfqpoint{3.875000in}{2.310000in}}%
\pgfusepath{clip}%
\pgfsetbuttcap%
\pgfsetmiterjoin%
\definecolor{currentfill}{rgb}{0.686275,0.352941,0.313725}%
\pgfsetfillcolor{currentfill}%
\pgfsetfillopacity{0.300000}%
\pgfsetlinewidth{0.000000pt}%
\definecolor{currentstroke}{rgb}{0.000000,0.000000,0.000000}%
\pgfsetstrokecolor{currentstroke}%
\pgfsetstrokeopacity{0.300000}%
\pgfsetdash{}{0pt}%
\pgfpathmoveto{\pgfqpoint{1.019644in}{0.383578in}}%
\pgfpathlineto{\pgfqpoint{1.037233in}{0.383578in}}%
\pgfpathlineto{\pgfqpoint{1.037233in}{0.422864in}}%
\pgfpathlineto{\pgfqpoint{1.019644in}{0.422864in}}%
\pgfpathclose%
\pgfusepath{fill}%
\end{pgfscope}%
\begin{pgfscope}%
\pgfpathrectangle{\pgfqpoint{0.526905in}{0.383578in}}{\pgfqpoint{3.875000in}{2.310000in}}%
\pgfusepath{clip}%
\pgfsetbuttcap%
\pgfsetmiterjoin%
\definecolor{currentfill}{rgb}{0.686275,0.352941,0.313725}%
\pgfsetfillcolor{currentfill}%
\pgfsetfillopacity{0.300000}%
\pgfsetlinewidth{0.000000pt}%
\definecolor{currentstroke}{rgb}{0.000000,0.000000,0.000000}%
\pgfsetstrokecolor{currentstroke}%
\pgfsetstrokeopacity{0.300000}%
\pgfsetdash{}{0pt}%
\pgfpathmoveto{\pgfqpoint{1.037233in}{0.383578in}}%
\pgfpathlineto{\pgfqpoint{1.054822in}{0.383578in}}%
\pgfpathlineto{\pgfqpoint{1.054822in}{0.406027in}}%
\pgfpathlineto{\pgfqpoint{1.037233in}{0.406027in}}%
\pgfpathclose%
\pgfusepath{fill}%
\end{pgfscope}%
\begin{pgfscope}%
\pgfpathrectangle{\pgfqpoint{0.526905in}{0.383578in}}{\pgfqpoint{3.875000in}{2.310000in}}%
\pgfusepath{clip}%
\pgfsetbuttcap%
\pgfsetmiterjoin%
\definecolor{currentfill}{rgb}{0.686275,0.352941,0.313725}%
\pgfsetfillcolor{currentfill}%
\pgfsetfillopacity{0.300000}%
\pgfsetlinewidth{0.000000pt}%
\definecolor{currentstroke}{rgb}{0.000000,0.000000,0.000000}%
\pgfsetstrokecolor{currentstroke}%
\pgfsetstrokeopacity{0.300000}%
\pgfsetdash{}{0pt}%
\pgfpathmoveto{\pgfqpoint{1.054822in}{0.383578in}}%
\pgfpathlineto{\pgfqpoint{1.072411in}{0.383578in}}%
\pgfpathlineto{\pgfqpoint{1.072411in}{0.406027in}}%
\pgfpathlineto{\pgfqpoint{1.054822in}{0.406027in}}%
\pgfpathclose%
\pgfusepath{fill}%
\end{pgfscope}%
\begin{pgfscope}%
\pgfpathrectangle{\pgfqpoint{0.526905in}{0.383578in}}{\pgfqpoint{3.875000in}{2.310000in}}%
\pgfusepath{clip}%
\pgfsetbuttcap%
\pgfsetmiterjoin%
\definecolor{currentfill}{rgb}{0.686275,0.352941,0.313725}%
\pgfsetfillcolor{currentfill}%
\pgfsetfillopacity{0.300000}%
\pgfsetlinewidth{0.000000pt}%
\definecolor{currentstroke}{rgb}{0.000000,0.000000,0.000000}%
\pgfsetstrokecolor{currentstroke}%
\pgfsetstrokeopacity{0.300000}%
\pgfsetdash{}{0pt}%
\pgfpathmoveto{\pgfqpoint{1.072411in}{0.383578in}}%
\pgfpathlineto{\pgfqpoint{1.090000in}{0.383578in}}%
\pgfpathlineto{\pgfqpoint{1.090000in}{0.400415in}}%
\pgfpathlineto{\pgfqpoint{1.072411in}{0.400415in}}%
\pgfpathclose%
\pgfusepath{fill}%
\end{pgfscope}%
\begin{pgfscope}%
\pgfpathrectangle{\pgfqpoint{0.526905in}{0.383578in}}{\pgfqpoint{3.875000in}{2.310000in}}%
\pgfusepath{clip}%
\pgfsetbuttcap%
\pgfsetmiterjoin%
\definecolor{currentfill}{rgb}{0.686275,0.352941,0.313725}%
\pgfsetfillcolor{currentfill}%
\pgfsetfillopacity{0.300000}%
\pgfsetlinewidth{0.000000pt}%
\definecolor{currentstroke}{rgb}{0.000000,0.000000,0.000000}%
\pgfsetstrokecolor{currentstroke}%
\pgfsetstrokeopacity{0.300000}%
\pgfsetdash{}{0pt}%
\pgfpathmoveto{\pgfqpoint{1.090000in}{0.383578in}}%
\pgfpathlineto{\pgfqpoint{1.107589in}{0.383578in}}%
\pgfpathlineto{\pgfqpoint{1.107589in}{0.417252in}}%
\pgfpathlineto{\pgfqpoint{1.090000in}{0.417252in}}%
\pgfpathclose%
\pgfusepath{fill}%
\end{pgfscope}%
\begin{pgfscope}%
\pgfpathrectangle{\pgfqpoint{0.526905in}{0.383578in}}{\pgfqpoint{3.875000in}{2.310000in}}%
\pgfusepath{clip}%
\pgfsetbuttcap%
\pgfsetmiterjoin%
\definecolor{currentfill}{rgb}{0.686275,0.352941,0.313725}%
\pgfsetfillcolor{currentfill}%
\pgfsetfillopacity{0.300000}%
\pgfsetlinewidth{0.000000pt}%
\definecolor{currentstroke}{rgb}{0.000000,0.000000,0.000000}%
\pgfsetstrokecolor{currentstroke}%
\pgfsetstrokeopacity{0.300000}%
\pgfsetdash{}{0pt}%
\pgfpathmoveto{\pgfqpoint{1.107589in}{0.383578in}}%
\pgfpathlineto{\pgfqpoint{1.125178in}{0.383578in}}%
\pgfpathlineto{\pgfqpoint{1.125178in}{0.450925in}}%
\pgfpathlineto{\pgfqpoint{1.107589in}{0.450925in}}%
\pgfpathclose%
\pgfusepath{fill}%
\end{pgfscope}%
\begin{pgfscope}%
\pgfpathrectangle{\pgfqpoint{0.526905in}{0.383578in}}{\pgfqpoint{3.875000in}{2.310000in}}%
\pgfusepath{clip}%
\pgfsetbuttcap%
\pgfsetmiterjoin%
\definecolor{currentfill}{rgb}{0.686275,0.352941,0.313725}%
\pgfsetfillcolor{currentfill}%
\pgfsetfillopacity{0.300000}%
\pgfsetlinewidth{0.000000pt}%
\definecolor{currentstroke}{rgb}{0.000000,0.000000,0.000000}%
\pgfsetstrokecolor{currentstroke}%
\pgfsetstrokeopacity{0.300000}%
\pgfsetdash{}{0pt}%
\pgfpathmoveto{\pgfqpoint{1.125178in}{0.383578in}}%
\pgfpathlineto{\pgfqpoint{1.142767in}{0.383578in}}%
\pgfpathlineto{\pgfqpoint{1.142767in}{0.406027in}}%
\pgfpathlineto{\pgfqpoint{1.125178in}{0.406027in}}%
\pgfpathclose%
\pgfusepath{fill}%
\end{pgfscope}%
\begin{pgfscope}%
\pgfpathrectangle{\pgfqpoint{0.526905in}{0.383578in}}{\pgfqpoint{3.875000in}{2.310000in}}%
\pgfusepath{clip}%
\pgfsetbuttcap%
\pgfsetmiterjoin%
\definecolor{currentfill}{rgb}{0.686275,0.352941,0.313725}%
\pgfsetfillcolor{currentfill}%
\pgfsetfillopacity{0.300000}%
\pgfsetlinewidth{0.000000pt}%
\definecolor{currentstroke}{rgb}{0.000000,0.000000,0.000000}%
\pgfsetstrokecolor{currentstroke}%
\pgfsetstrokeopacity{0.300000}%
\pgfsetdash{}{0pt}%
\pgfpathmoveto{\pgfqpoint{1.142767in}{0.383578in}}%
\pgfpathlineto{\pgfqpoint{1.160356in}{0.383578in}}%
\pgfpathlineto{\pgfqpoint{1.160356in}{0.411639in}}%
\pgfpathlineto{\pgfqpoint{1.142767in}{0.411639in}}%
\pgfpathclose%
\pgfusepath{fill}%
\end{pgfscope}%
\begin{pgfscope}%
\pgfpathrectangle{\pgfqpoint{0.526905in}{0.383578in}}{\pgfqpoint{3.875000in}{2.310000in}}%
\pgfusepath{clip}%
\pgfsetbuttcap%
\pgfsetmiterjoin%
\definecolor{currentfill}{rgb}{0.686275,0.352941,0.313725}%
\pgfsetfillcolor{currentfill}%
\pgfsetfillopacity{0.300000}%
\pgfsetlinewidth{0.000000pt}%
\definecolor{currentstroke}{rgb}{0.000000,0.000000,0.000000}%
\pgfsetstrokecolor{currentstroke}%
\pgfsetstrokeopacity{0.300000}%
\pgfsetdash{}{0pt}%
\pgfpathmoveto{\pgfqpoint{1.160356in}{0.383578in}}%
\pgfpathlineto{\pgfqpoint{1.177946in}{0.383578in}}%
\pgfpathlineto{\pgfqpoint{1.177946in}{0.428476in}}%
\pgfpathlineto{\pgfqpoint{1.160356in}{0.428476in}}%
\pgfpathclose%
\pgfusepath{fill}%
\end{pgfscope}%
\begin{pgfscope}%
\pgfpathrectangle{\pgfqpoint{0.526905in}{0.383578in}}{\pgfqpoint{3.875000in}{2.310000in}}%
\pgfusepath{clip}%
\pgfsetbuttcap%
\pgfsetmiterjoin%
\definecolor{currentfill}{rgb}{0.686275,0.352941,0.313725}%
\pgfsetfillcolor{currentfill}%
\pgfsetfillopacity{0.300000}%
\pgfsetlinewidth{0.000000pt}%
\definecolor{currentstroke}{rgb}{0.000000,0.000000,0.000000}%
\pgfsetstrokecolor{currentstroke}%
\pgfsetstrokeopacity{0.300000}%
\pgfsetdash{}{0pt}%
\pgfpathmoveto{\pgfqpoint{1.177946in}{0.383578in}}%
\pgfpathlineto{\pgfqpoint{1.195535in}{0.383578in}}%
\pgfpathlineto{\pgfqpoint{1.195535in}{0.439701in}}%
\pgfpathlineto{\pgfqpoint{1.177946in}{0.439701in}}%
\pgfpathclose%
\pgfusepath{fill}%
\end{pgfscope}%
\begin{pgfscope}%
\pgfpathrectangle{\pgfqpoint{0.526905in}{0.383578in}}{\pgfqpoint{3.875000in}{2.310000in}}%
\pgfusepath{clip}%
\pgfsetbuttcap%
\pgfsetmiterjoin%
\definecolor{currentfill}{rgb}{0.686275,0.352941,0.313725}%
\pgfsetfillcolor{currentfill}%
\pgfsetfillopacity{0.300000}%
\pgfsetlinewidth{0.000000pt}%
\definecolor{currentstroke}{rgb}{0.000000,0.000000,0.000000}%
\pgfsetstrokecolor{currentstroke}%
\pgfsetstrokeopacity{0.300000}%
\pgfsetdash{}{0pt}%
\pgfpathmoveto{\pgfqpoint{1.195535in}{0.383578in}}%
\pgfpathlineto{\pgfqpoint{1.213124in}{0.383578in}}%
\pgfpathlineto{\pgfqpoint{1.213124in}{0.434088in}}%
\pgfpathlineto{\pgfqpoint{1.195535in}{0.434088in}}%
\pgfpathclose%
\pgfusepath{fill}%
\end{pgfscope}%
\begin{pgfscope}%
\pgfpathrectangle{\pgfqpoint{0.526905in}{0.383578in}}{\pgfqpoint{3.875000in}{2.310000in}}%
\pgfusepath{clip}%
\pgfsetbuttcap%
\pgfsetmiterjoin%
\definecolor{currentfill}{rgb}{0.686275,0.352941,0.313725}%
\pgfsetfillcolor{currentfill}%
\pgfsetfillopacity{0.300000}%
\pgfsetlinewidth{0.000000pt}%
\definecolor{currentstroke}{rgb}{0.000000,0.000000,0.000000}%
\pgfsetstrokecolor{currentstroke}%
\pgfsetstrokeopacity{0.300000}%
\pgfsetdash{}{0pt}%
\pgfpathmoveto{\pgfqpoint{1.213124in}{0.383578in}}%
\pgfpathlineto{\pgfqpoint{1.230713in}{0.383578in}}%
\pgfpathlineto{\pgfqpoint{1.230713in}{0.473374in}}%
\pgfpathlineto{\pgfqpoint{1.213124in}{0.473374in}}%
\pgfpathclose%
\pgfusepath{fill}%
\end{pgfscope}%
\begin{pgfscope}%
\pgfpathrectangle{\pgfqpoint{0.526905in}{0.383578in}}{\pgfqpoint{3.875000in}{2.310000in}}%
\pgfusepath{clip}%
\pgfsetbuttcap%
\pgfsetmiterjoin%
\definecolor{currentfill}{rgb}{0.686275,0.352941,0.313725}%
\pgfsetfillcolor{currentfill}%
\pgfsetfillopacity{0.300000}%
\pgfsetlinewidth{0.000000pt}%
\definecolor{currentstroke}{rgb}{0.000000,0.000000,0.000000}%
\pgfsetstrokecolor{currentstroke}%
\pgfsetstrokeopacity{0.300000}%
\pgfsetdash{}{0pt}%
\pgfpathmoveto{\pgfqpoint{1.230713in}{0.383578in}}%
\pgfpathlineto{\pgfqpoint{1.248302in}{0.383578in}}%
\pgfpathlineto{\pgfqpoint{1.248302in}{0.439701in}}%
\pgfpathlineto{\pgfqpoint{1.230713in}{0.439701in}}%
\pgfpathclose%
\pgfusepath{fill}%
\end{pgfscope}%
\begin{pgfscope}%
\pgfpathrectangle{\pgfqpoint{0.526905in}{0.383578in}}{\pgfqpoint{3.875000in}{2.310000in}}%
\pgfusepath{clip}%
\pgfsetbuttcap%
\pgfsetmiterjoin%
\definecolor{currentfill}{rgb}{0.686275,0.352941,0.313725}%
\pgfsetfillcolor{currentfill}%
\pgfsetfillopacity{0.300000}%
\pgfsetlinewidth{0.000000pt}%
\definecolor{currentstroke}{rgb}{0.000000,0.000000,0.000000}%
\pgfsetstrokecolor{currentstroke}%
\pgfsetstrokeopacity{0.300000}%
\pgfsetdash{}{0pt}%
\pgfpathmoveto{\pgfqpoint{1.248302in}{0.383578in}}%
\pgfpathlineto{\pgfqpoint{1.265891in}{0.383578in}}%
\pgfpathlineto{\pgfqpoint{1.265891in}{0.450925in}}%
\pgfpathlineto{\pgfqpoint{1.248302in}{0.450925in}}%
\pgfpathclose%
\pgfusepath{fill}%
\end{pgfscope}%
\begin{pgfscope}%
\pgfpathrectangle{\pgfqpoint{0.526905in}{0.383578in}}{\pgfqpoint{3.875000in}{2.310000in}}%
\pgfusepath{clip}%
\pgfsetbuttcap%
\pgfsetmiterjoin%
\definecolor{currentfill}{rgb}{0.686275,0.352941,0.313725}%
\pgfsetfillcolor{currentfill}%
\pgfsetfillopacity{0.300000}%
\pgfsetlinewidth{0.000000pt}%
\definecolor{currentstroke}{rgb}{0.000000,0.000000,0.000000}%
\pgfsetstrokecolor{currentstroke}%
\pgfsetstrokeopacity{0.300000}%
\pgfsetdash{}{0pt}%
\pgfpathmoveto{\pgfqpoint{1.265891in}{0.383578in}}%
\pgfpathlineto{\pgfqpoint{1.283480in}{0.383578in}}%
\pgfpathlineto{\pgfqpoint{1.283480in}{0.422864in}}%
\pgfpathlineto{\pgfqpoint{1.265891in}{0.422864in}}%
\pgfpathclose%
\pgfusepath{fill}%
\end{pgfscope}%
\begin{pgfscope}%
\pgfpathrectangle{\pgfqpoint{0.526905in}{0.383578in}}{\pgfqpoint{3.875000in}{2.310000in}}%
\pgfusepath{clip}%
\pgfsetbuttcap%
\pgfsetmiterjoin%
\definecolor{currentfill}{rgb}{0.686275,0.352941,0.313725}%
\pgfsetfillcolor{currentfill}%
\pgfsetfillopacity{0.300000}%
\pgfsetlinewidth{0.000000pt}%
\definecolor{currentstroke}{rgb}{0.000000,0.000000,0.000000}%
\pgfsetstrokecolor{currentstroke}%
\pgfsetstrokeopacity{0.300000}%
\pgfsetdash{}{0pt}%
\pgfpathmoveto{\pgfqpoint{1.283480in}{0.383578in}}%
\pgfpathlineto{\pgfqpoint{1.301069in}{0.383578in}}%
\pgfpathlineto{\pgfqpoint{1.301069in}{0.484598in}}%
\pgfpathlineto{\pgfqpoint{1.283480in}{0.484598in}}%
\pgfpathclose%
\pgfusepath{fill}%
\end{pgfscope}%
\begin{pgfscope}%
\pgfpathrectangle{\pgfqpoint{0.526905in}{0.383578in}}{\pgfqpoint{3.875000in}{2.310000in}}%
\pgfusepath{clip}%
\pgfsetbuttcap%
\pgfsetmiterjoin%
\definecolor{currentfill}{rgb}{0.686275,0.352941,0.313725}%
\pgfsetfillcolor{currentfill}%
\pgfsetfillopacity{0.300000}%
\pgfsetlinewidth{0.000000pt}%
\definecolor{currentstroke}{rgb}{0.000000,0.000000,0.000000}%
\pgfsetstrokecolor{currentstroke}%
\pgfsetstrokeopacity{0.300000}%
\pgfsetdash{}{0pt}%
\pgfpathmoveto{\pgfqpoint{1.301069in}{0.383578in}}%
\pgfpathlineto{\pgfqpoint{1.318658in}{0.383578in}}%
\pgfpathlineto{\pgfqpoint{1.318658in}{0.422864in}}%
\pgfpathlineto{\pgfqpoint{1.301069in}{0.422864in}}%
\pgfpathclose%
\pgfusepath{fill}%
\end{pgfscope}%
\begin{pgfscope}%
\pgfpathrectangle{\pgfqpoint{0.526905in}{0.383578in}}{\pgfqpoint{3.875000in}{2.310000in}}%
\pgfusepath{clip}%
\pgfsetbuttcap%
\pgfsetmiterjoin%
\definecolor{currentfill}{rgb}{0.686275,0.352941,0.313725}%
\pgfsetfillcolor{currentfill}%
\pgfsetfillopacity{0.300000}%
\pgfsetlinewidth{0.000000pt}%
\definecolor{currentstroke}{rgb}{0.000000,0.000000,0.000000}%
\pgfsetstrokecolor{currentstroke}%
\pgfsetstrokeopacity{0.300000}%
\pgfsetdash{}{0pt}%
\pgfpathmoveto{\pgfqpoint{1.318658in}{0.383578in}}%
\pgfpathlineto{\pgfqpoint{1.336247in}{0.383578in}}%
\pgfpathlineto{\pgfqpoint{1.336247in}{0.490211in}}%
\pgfpathlineto{\pgfqpoint{1.318658in}{0.490211in}}%
\pgfpathclose%
\pgfusepath{fill}%
\end{pgfscope}%
\begin{pgfscope}%
\pgfpathrectangle{\pgfqpoint{0.526905in}{0.383578in}}{\pgfqpoint{3.875000in}{2.310000in}}%
\pgfusepath{clip}%
\pgfsetbuttcap%
\pgfsetmiterjoin%
\definecolor{currentfill}{rgb}{0.686275,0.352941,0.313725}%
\pgfsetfillcolor{currentfill}%
\pgfsetfillopacity{0.300000}%
\pgfsetlinewidth{0.000000pt}%
\definecolor{currentstroke}{rgb}{0.000000,0.000000,0.000000}%
\pgfsetstrokecolor{currentstroke}%
\pgfsetstrokeopacity{0.300000}%
\pgfsetdash{}{0pt}%
\pgfpathmoveto{\pgfqpoint{1.336247in}{0.383578in}}%
\pgfpathlineto{\pgfqpoint{1.353836in}{0.383578in}}%
\pgfpathlineto{\pgfqpoint{1.353836in}{0.484598in}}%
\pgfpathlineto{\pgfqpoint{1.336247in}{0.484598in}}%
\pgfpathclose%
\pgfusepath{fill}%
\end{pgfscope}%
\begin{pgfscope}%
\pgfpathrectangle{\pgfqpoint{0.526905in}{0.383578in}}{\pgfqpoint{3.875000in}{2.310000in}}%
\pgfusepath{clip}%
\pgfsetbuttcap%
\pgfsetmiterjoin%
\definecolor{currentfill}{rgb}{0.686275,0.352941,0.313725}%
\pgfsetfillcolor{currentfill}%
\pgfsetfillopacity{0.300000}%
\pgfsetlinewidth{0.000000pt}%
\definecolor{currentstroke}{rgb}{0.000000,0.000000,0.000000}%
\pgfsetstrokecolor{currentstroke}%
\pgfsetstrokeopacity{0.300000}%
\pgfsetdash{}{0pt}%
\pgfpathmoveto{\pgfqpoint{1.353836in}{0.383578in}}%
\pgfpathlineto{\pgfqpoint{1.371425in}{0.383578in}}%
\pgfpathlineto{\pgfqpoint{1.371425in}{0.478986in}}%
\pgfpathlineto{\pgfqpoint{1.353836in}{0.478986in}}%
\pgfpathclose%
\pgfusepath{fill}%
\end{pgfscope}%
\begin{pgfscope}%
\pgfpathrectangle{\pgfqpoint{0.526905in}{0.383578in}}{\pgfqpoint{3.875000in}{2.310000in}}%
\pgfusepath{clip}%
\pgfsetbuttcap%
\pgfsetmiterjoin%
\definecolor{currentfill}{rgb}{0.686275,0.352941,0.313725}%
\pgfsetfillcolor{currentfill}%
\pgfsetfillopacity{0.300000}%
\pgfsetlinewidth{0.000000pt}%
\definecolor{currentstroke}{rgb}{0.000000,0.000000,0.000000}%
\pgfsetstrokecolor{currentstroke}%
\pgfsetstrokeopacity{0.300000}%
\pgfsetdash{}{0pt}%
\pgfpathmoveto{\pgfqpoint{1.371425in}{0.383578in}}%
\pgfpathlineto{\pgfqpoint{1.389014in}{0.383578in}}%
\pgfpathlineto{\pgfqpoint{1.389014in}{0.467762in}}%
\pgfpathlineto{\pgfqpoint{1.371425in}{0.467762in}}%
\pgfpathclose%
\pgfusepath{fill}%
\end{pgfscope}%
\begin{pgfscope}%
\pgfpathrectangle{\pgfqpoint{0.526905in}{0.383578in}}{\pgfqpoint{3.875000in}{2.310000in}}%
\pgfusepath{clip}%
\pgfsetbuttcap%
\pgfsetmiterjoin%
\definecolor{currentfill}{rgb}{0.686275,0.352941,0.313725}%
\pgfsetfillcolor{currentfill}%
\pgfsetfillopacity{0.300000}%
\pgfsetlinewidth{0.000000pt}%
\definecolor{currentstroke}{rgb}{0.000000,0.000000,0.000000}%
\pgfsetstrokecolor{currentstroke}%
\pgfsetstrokeopacity{0.300000}%
\pgfsetdash{}{0pt}%
\pgfpathmoveto{\pgfqpoint{1.389014in}{0.383578in}}%
\pgfpathlineto{\pgfqpoint{1.406603in}{0.383578in}}%
\pgfpathlineto{\pgfqpoint{1.406603in}{0.557558in}}%
\pgfpathlineto{\pgfqpoint{1.389014in}{0.557558in}}%
\pgfpathclose%
\pgfusepath{fill}%
\end{pgfscope}%
\begin{pgfscope}%
\pgfpathrectangle{\pgfqpoint{0.526905in}{0.383578in}}{\pgfqpoint{3.875000in}{2.310000in}}%
\pgfusepath{clip}%
\pgfsetbuttcap%
\pgfsetmiterjoin%
\definecolor{currentfill}{rgb}{0.686275,0.352941,0.313725}%
\pgfsetfillcolor{currentfill}%
\pgfsetfillopacity{0.300000}%
\pgfsetlinewidth{0.000000pt}%
\definecolor{currentstroke}{rgb}{0.000000,0.000000,0.000000}%
\pgfsetstrokecolor{currentstroke}%
\pgfsetstrokeopacity{0.300000}%
\pgfsetdash{}{0pt}%
\pgfpathmoveto{\pgfqpoint{1.406603in}{0.383578in}}%
\pgfpathlineto{\pgfqpoint{1.424192in}{0.383578in}}%
\pgfpathlineto{\pgfqpoint{1.424192in}{0.557558in}}%
\pgfpathlineto{\pgfqpoint{1.406603in}{0.557558in}}%
\pgfpathclose%
\pgfusepath{fill}%
\end{pgfscope}%
\begin{pgfscope}%
\pgfpathrectangle{\pgfqpoint{0.526905in}{0.383578in}}{\pgfqpoint{3.875000in}{2.310000in}}%
\pgfusepath{clip}%
\pgfsetbuttcap%
\pgfsetmiterjoin%
\definecolor{currentfill}{rgb}{0.686275,0.352941,0.313725}%
\pgfsetfillcolor{currentfill}%
\pgfsetfillopacity{0.300000}%
\pgfsetlinewidth{0.000000pt}%
\definecolor{currentstroke}{rgb}{0.000000,0.000000,0.000000}%
\pgfsetstrokecolor{currentstroke}%
\pgfsetstrokeopacity{0.300000}%
\pgfsetdash{}{0pt}%
\pgfpathmoveto{\pgfqpoint{1.424192in}{0.383578in}}%
\pgfpathlineto{\pgfqpoint{1.441781in}{0.383578in}}%
\pgfpathlineto{\pgfqpoint{1.441781in}{0.478986in}}%
\pgfpathlineto{\pgfqpoint{1.424192in}{0.478986in}}%
\pgfpathclose%
\pgfusepath{fill}%
\end{pgfscope}%
\begin{pgfscope}%
\pgfpathrectangle{\pgfqpoint{0.526905in}{0.383578in}}{\pgfqpoint{3.875000in}{2.310000in}}%
\pgfusepath{clip}%
\pgfsetbuttcap%
\pgfsetmiterjoin%
\definecolor{currentfill}{rgb}{0.686275,0.352941,0.313725}%
\pgfsetfillcolor{currentfill}%
\pgfsetfillopacity{0.300000}%
\pgfsetlinewidth{0.000000pt}%
\definecolor{currentstroke}{rgb}{0.000000,0.000000,0.000000}%
\pgfsetstrokecolor{currentstroke}%
\pgfsetstrokeopacity{0.300000}%
\pgfsetdash{}{0pt}%
\pgfpathmoveto{\pgfqpoint{1.441781in}{0.383578in}}%
\pgfpathlineto{\pgfqpoint{1.459370in}{0.383578in}}%
\pgfpathlineto{\pgfqpoint{1.459370in}{0.535109in}}%
\pgfpathlineto{\pgfqpoint{1.441781in}{0.535109in}}%
\pgfpathclose%
\pgfusepath{fill}%
\end{pgfscope}%
\begin{pgfscope}%
\pgfpathrectangle{\pgfqpoint{0.526905in}{0.383578in}}{\pgfqpoint{3.875000in}{2.310000in}}%
\pgfusepath{clip}%
\pgfsetbuttcap%
\pgfsetmiterjoin%
\definecolor{currentfill}{rgb}{0.686275,0.352941,0.313725}%
\pgfsetfillcolor{currentfill}%
\pgfsetfillopacity{0.300000}%
\pgfsetlinewidth{0.000000pt}%
\definecolor{currentstroke}{rgb}{0.000000,0.000000,0.000000}%
\pgfsetstrokecolor{currentstroke}%
\pgfsetstrokeopacity{0.300000}%
\pgfsetdash{}{0pt}%
\pgfpathmoveto{\pgfqpoint{1.459370in}{0.383578in}}%
\pgfpathlineto{\pgfqpoint{1.476959in}{0.383578in}}%
\pgfpathlineto{\pgfqpoint{1.476959in}{0.568782in}}%
\pgfpathlineto{\pgfqpoint{1.459370in}{0.568782in}}%
\pgfpathclose%
\pgfusepath{fill}%
\end{pgfscope}%
\begin{pgfscope}%
\pgfpathrectangle{\pgfqpoint{0.526905in}{0.383578in}}{\pgfqpoint{3.875000in}{2.310000in}}%
\pgfusepath{clip}%
\pgfsetbuttcap%
\pgfsetmiterjoin%
\definecolor{currentfill}{rgb}{0.686275,0.352941,0.313725}%
\pgfsetfillcolor{currentfill}%
\pgfsetfillopacity{0.300000}%
\pgfsetlinewidth{0.000000pt}%
\definecolor{currentstroke}{rgb}{0.000000,0.000000,0.000000}%
\pgfsetstrokecolor{currentstroke}%
\pgfsetstrokeopacity{0.300000}%
\pgfsetdash{}{0pt}%
\pgfpathmoveto{\pgfqpoint{1.476959in}{0.383578in}}%
\pgfpathlineto{\pgfqpoint{1.494548in}{0.383578in}}%
\pgfpathlineto{\pgfqpoint{1.494548in}{0.568782in}}%
\pgfpathlineto{\pgfqpoint{1.476959in}{0.568782in}}%
\pgfpathclose%
\pgfusepath{fill}%
\end{pgfscope}%
\begin{pgfscope}%
\pgfpathrectangle{\pgfqpoint{0.526905in}{0.383578in}}{\pgfqpoint{3.875000in}{2.310000in}}%
\pgfusepath{clip}%
\pgfsetbuttcap%
\pgfsetmiterjoin%
\definecolor{currentfill}{rgb}{0.686275,0.352941,0.313725}%
\pgfsetfillcolor{currentfill}%
\pgfsetfillopacity{0.300000}%
\pgfsetlinewidth{0.000000pt}%
\definecolor{currentstroke}{rgb}{0.000000,0.000000,0.000000}%
\pgfsetstrokecolor{currentstroke}%
\pgfsetstrokeopacity{0.300000}%
\pgfsetdash{}{0pt}%
\pgfpathmoveto{\pgfqpoint{1.494548in}{0.383578in}}%
\pgfpathlineto{\pgfqpoint{1.512138in}{0.383578in}}%
\pgfpathlineto{\pgfqpoint{1.512138in}{0.619292in}}%
\pgfpathlineto{\pgfqpoint{1.494548in}{0.619292in}}%
\pgfpathclose%
\pgfusepath{fill}%
\end{pgfscope}%
\begin{pgfscope}%
\pgfpathrectangle{\pgfqpoint{0.526905in}{0.383578in}}{\pgfqpoint{3.875000in}{2.310000in}}%
\pgfusepath{clip}%
\pgfsetbuttcap%
\pgfsetmiterjoin%
\definecolor{currentfill}{rgb}{0.686275,0.352941,0.313725}%
\pgfsetfillcolor{currentfill}%
\pgfsetfillopacity{0.300000}%
\pgfsetlinewidth{0.000000pt}%
\definecolor{currentstroke}{rgb}{0.000000,0.000000,0.000000}%
\pgfsetstrokecolor{currentstroke}%
\pgfsetstrokeopacity{0.300000}%
\pgfsetdash{}{0pt}%
\pgfpathmoveto{\pgfqpoint{1.512138in}{0.383578in}}%
\pgfpathlineto{\pgfqpoint{1.529727in}{0.383578in}}%
\pgfpathlineto{\pgfqpoint{1.529727in}{0.591231in}}%
\pgfpathlineto{\pgfqpoint{1.512138in}{0.591231in}}%
\pgfpathclose%
\pgfusepath{fill}%
\end{pgfscope}%
\begin{pgfscope}%
\pgfpathrectangle{\pgfqpoint{0.526905in}{0.383578in}}{\pgfqpoint{3.875000in}{2.310000in}}%
\pgfusepath{clip}%
\pgfsetbuttcap%
\pgfsetmiterjoin%
\definecolor{currentfill}{rgb}{0.686275,0.352941,0.313725}%
\pgfsetfillcolor{currentfill}%
\pgfsetfillopacity{0.300000}%
\pgfsetlinewidth{0.000000pt}%
\definecolor{currentstroke}{rgb}{0.000000,0.000000,0.000000}%
\pgfsetstrokecolor{currentstroke}%
\pgfsetstrokeopacity{0.300000}%
\pgfsetdash{}{0pt}%
\pgfpathmoveto{\pgfqpoint{1.529727in}{0.383578in}}%
\pgfpathlineto{\pgfqpoint{1.547316in}{0.383578in}}%
\pgfpathlineto{\pgfqpoint{1.547316in}{0.529496in}}%
\pgfpathlineto{\pgfqpoint{1.529727in}{0.529496in}}%
\pgfpathclose%
\pgfusepath{fill}%
\end{pgfscope}%
\begin{pgfscope}%
\pgfpathrectangle{\pgfqpoint{0.526905in}{0.383578in}}{\pgfqpoint{3.875000in}{2.310000in}}%
\pgfusepath{clip}%
\pgfsetbuttcap%
\pgfsetmiterjoin%
\definecolor{currentfill}{rgb}{0.686275,0.352941,0.313725}%
\pgfsetfillcolor{currentfill}%
\pgfsetfillopacity{0.300000}%
\pgfsetlinewidth{0.000000pt}%
\definecolor{currentstroke}{rgb}{0.000000,0.000000,0.000000}%
\pgfsetstrokecolor{currentstroke}%
\pgfsetstrokeopacity{0.300000}%
\pgfsetdash{}{0pt}%
\pgfpathmoveto{\pgfqpoint{1.547316in}{0.383578in}}%
\pgfpathlineto{\pgfqpoint{1.564905in}{0.383578in}}%
\pgfpathlineto{\pgfqpoint{1.564905in}{0.568782in}}%
\pgfpathlineto{\pgfqpoint{1.547316in}{0.568782in}}%
\pgfpathclose%
\pgfusepath{fill}%
\end{pgfscope}%
\begin{pgfscope}%
\pgfpathrectangle{\pgfqpoint{0.526905in}{0.383578in}}{\pgfqpoint{3.875000in}{2.310000in}}%
\pgfusepath{clip}%
\pgfsetbuttcap%
\pgfsetmiterjoin%
\definecolor{currentfill}{rgb}{0.686275,0.352941,0.313725}%
\pgfsetfillcolor{currentfill}%
\pgfsetfillopacity{0.300000}%
\pgfsetlinewidth{0.000000pt}%
\definecolor{currentstroke}{rgb}{0.000000,0.000000,0.000000}%
\pgfsetstrokecolor{currentstroke}%
\pgfsetstrokeopacity{0.300000}%
\pgfsetdash{}{0pt}%
\pgfpathmoveto{\pgfqpoint{1.564905in}{0.383578in}}%
\pgfpathlineto{\pgfqpoint{1.582494in}{0.383578in}}%
\pgfpathlineto{\pgfqpoint{1.582494in}{0.669803in}}%
\pgfpathlineto{\pgfqpoint{1.564905in}{0.669803in}}%
\pgfpathclose%
\pgfusepath{fill}%
\end{pgfscope}%
\begin{pgfscope}%
\pgfpathrectangle{\pgfqpoint{0.526905in}{0.383578in}}{\pgfqpoint{3.875000in}{2.310000in}}%
\pgfusepath{clip}%
\pgfsetbuttcap%
\pgfsetmiterjoin%
\definecolor{currentfill}{rgb}{0.686275,0.352941,0.313725}%
\pgfsetfillcolor{currentfill}%
\pgfsetfillopacity{0.300000}%
\pgfsetlinewidth{0.000000pt}%
\definecolor{currentstroke}{rgb}{0.000000,0.000000,0.000000}%
\pgfsetstrokecolor{currentstroke}%
\pgfsetstrokeopacity{0.300000}%
\pgfsetdash{}{0pt}%
\pgfpathmoveto{\pgfqpoint{1.582494in}{0.383578in}}%
\pgfpathlineto{\pgfqpoint{1.600083in}{0.383578in}}%
\pgfpathlineto{\pgfqpoint{1.600083in}{0.658578in}}%
\pgfpathlineto{\pgfqpoint{1.582494in}{0.658578in}}%
\pgfpathclose%
\pgfusepath{fill}%
\end{pgfscope}%
\begin{pgfscope}%
\pgfpathrectangle{\pgfqpoint{0.526905in}{0.383578in}}{\pgfqpoint{3.875000in}{2.310000in}}%
\pgfusepath{clip}%
\pgfsetbuttcap%
\pgfsetmiterjoin%
\definecolor{currentfill}{rgb}{0.686275,0.352941,0.313725}%
\pgfsetfillcolor{currentfill}%
\pgfsetfillopacity{0.300000}%
\pgfsetlinewidth{0.000000pt}%
\definecolor{currentstroke}{rgb}{0.000000,0.000000,0.000000}%
\pgfsetstrokecolor{currentstroke}%
\pgfsetstrokeopacity{0.300000}%
\pgfsetdash{}{0pt}%
\pgfpathmoveto{\pgfqpoint{1.600083in}{0.383578in}}%
\pgfpathlineto{\pgfqpoint{1.617672in}{0.383578in}}%
\pgfpathlineto{\pgfqpoint{1.617672in}{0.585619in}}%
\pgfpathlineto{\pgfqpoint{1.600083in}{0.585619in}}%
\pgfpathclose%
\pgfusepath{fill}%
\end{pgfscope}%
\begin{pgfscope}%
\pgfpathrectangle{\pgfqpoint{0.526905in}{0.383578in}}{\pgfqpoint{3.875000in}{2.310000in}}%
\pgfusepath{clip}%
\pgfsetbuttcap%
\pgfsetmiterjoin%
\definecolor{currentfill}{rgb}{0.686275,0.352941,0.313725}%
\pgfsetfillcolor{currentfill}%
\pgfsetfillopacity{0.300000}%
\pgfsetlinewidth{0.000000pt}%
\definecolor{currentstroke}{rgb}{0.000000,0.000000,0.000000}%
\pgfsetstrokecolor{currentstroke}%
\pgfsetstrokeopacity{0.300000}%
\pgfsetdash{}{0pt}%
\pgfpathmoveto{\pgfqpoint{1.617672in}{0.383578in}}%
\pgfpathlineto{\pgfqpoint{1.635261in}{0.383578in}}%
\pgfpathlineto{\pgfqpoint{1.635261in}{0.703476in}}%
\pgfpathlineto{\pgfqpoint{1.617672in}{0.703476in}}%
\pgfpathclose%
\pgfusepath{fill}%
\end{pgfscope}%
\begin{pgfscope}%
\pgfpathrectangle{\pgfqpoint{0.526905in}{0.383578in}}{\pgfqpoint{3.875000in}{2.310000in}}%
\pgfusepath{clip}%
\pgfsetbuttcap%
\pgfsetmiterjoin%
\definecolor{currentfill}{rgb}{0.686275,0.352941,0.313725}%
\pgfsetfillcolor{currentfill}%
\pgfsetfillopacity{0.300000}%
\pgfsetlinewidth{0.000000pt}%
\definecolor{currentstroke}{rgb}{0.000000,0.000000,0.000000}%
\pgfsetstrokecolor{currentstroke}%
\pgfsetstrokeopacity{0.300000}%
\pgfsetdash{}{0pt}%
\pgfpathmoveto{\pgfqpoint{1.635261in}{0.383578in}}%
\pgfpathlineto{\pgfqpoint{1.652850in}{0.383578in}}%
\pgfpathlineto{\pgfqpoint{1.652850in}{0.692252in}}%
\pgfpathlineto{\pgfqpoint{1.635261in}{0.692252in}}%
\pgfpathclose%
\pgfusepath{fill}%
\end{pgfscope}%
\begin{pgfscope}%
\pgfpathrectangle{\pgfqpoint{0.526905in}{0.383578in}}{\pgfqpoint{3.875000in}{2.310000in}}%
\pgfusepath{clip}%
\pgfsetbuttcap%
\pgfsetmiterjoin%
\definecolor{currentfill}{rgb}{0.686275,0.352941,0.313725}%
\pgfsetfillcolor{currentfill}%
\pgfsetfillopacity{0.300000}%
\pgfsetlinewidth{0.000000pt}%
\definecolor{currentstroke}{rgb}{0.000000,0.000000,0.000000}%
\pgfsetstrokecolor{currentstroke}%
\pgfsetstrokeopacity{0.300000}%
\pgfsetdash{}{0pt}%
\pgfpathmoveto{\pgfqpoint{1.652850in}{0.383578in}}%
\pgfpathlineto{\pgfqpoint{1.670439in}{0.383578in}}%
\pgfpathlineto{\pgfqpoint{1.670439in}{0.681027in}}%
\pgfpathlineto{\pgfqpoint{1.652850in}{0.681027in}}%
\pgfpathclose%
\pgfusepath{fill}%
\end{pgfscope}%
\begin{pgfscope}%
\pgfpathrectangle{\pgfqpoint{0.526905in}{0.383578in}}{\pgfqpoint{3.875000in}{2.310000in}}%
\pgfusepath{clip}%
\pgfsetbuttcap%
\pgfsetmiterjoin%
\definecolor{currentfill}{rgb}{0.686275,0.352941,0.313725}%
\pgfsetfillcolor{currentfill}%
\pgfsetfillopacity{0.300000}%
\pgfsetlinewidth{0.000000pt}%
\definecolor{currentstroke}{rgb}{0.000000,0.000000,0.000000}%
\pgfsetstrokecolor{currentstroke}%
\pgfsetstrokeopacity{0.300000}%
\pgfsetdash{}{0pt}%
\pgfpathmoveto{\pgfqpoint{1.670439in}{0.383578in}}%
\pgfpathlineto{\pgfqpoint{1.688028in}{0.383578in}}%
\pgfpathlineto{\pgfqpoint{1.688028in}{0.681027in}}%
\pgfpathlineto{\pgfqpoint{1.670439in}{0.681027in}}%
\pgfpathclose%
\pgfusepath{fill}%
\end{pgfscope}%
\begin{pgfscope}%
\pgfpathrectangle{\pgfqpoint{0.526905in}{0.383578in}}{\pgfqpoint{3.875000in}{2.310000in}}%
\pgfusepath{clip}%
\pgfsetbuttcap%
\pgfsetmiterjoin%
\definecolor{currentfill}{rgb}{0.686275,0.352941,0.313725}%
\pgfsetfillcolor{currentfill}%
\pgfsetfillopacity{0.300000}%
\pgfsetlinewidth{0.000000pt}%
\definecolor{currentstroke}{rgb}{0.000000,0.000000,0.000000}%
\pgfsetstrokecolor{currentstroke}%
\pgfsetstrokeopacity{0.300000}%
\pgfsetdash{}{0pt}%
\pgfpathmoveto{\pgfqpoint{1.688028in}{0.383578in}}%
\pgfpathlineto{\pgfqpoint{1.705617in}{0.383578in}}%
\pgfpathlineto{\pgfqpoint{1.705617in}{0.793272in}}%
\pgfpathlineto{\pgfqpoint{1.688028in}{0.793272in}}%
\pgfpathclose%
\pgfusepath{fill}%
\end{pgfscope}%
\begin{pgfscope}%
\pgfpathrectangle{\pgfqpoint{0.526905in}{0.383578in}}{\pgfqpoint{3.875000in}{2.310000in}}%
\pgfusepath{clip}%
\pgfsetbuttcap%
\pgfsetmiterjoin%
\definecolor{currentfill}{rgb}{0.686275,0.352941,0.313725}%
\pgfsetfillcolor{currentfill}%
\pgfsetfillopacity{0.300000}%
\pgfsetlinewidth{0.000000pt}%
\definecolor{currentstroke}{rgb}{0.000000,0.000000,0.000000}%
\pgfsetstrokecolor{currentstroke}%
\pgfsetstrokeopacity{0.300000}%
\pgfsetdash{}{0pt}%
\pgfpathmoveto{\pgfqpoint{1.705617in}{0.383578in}}%
\pgfpathlineto{\pgfqpoint{1.723206in}{0.383578in}}%
\pgfpathlineto{\pgfqpoint{1.723206in}{0.798884in}}%
\pgfpathlineto{\pgfqpoint{1.705617in}{0.798884in}}%
\pgfpathclose%
\pgfusepath{fill}%
\end{pgfscope}%
\begin{pgfscope}%
\pgfpathrectangle{\pgfqpoint{0.526905in}{0.383578in}}{\pgfqpoint{3.875000in}{2.310000in}}%
\pgfusepath{clip}%
\pgfsetbuttcap%
\pgfsetmiterjoin%
\definecolor{currentfill}{rgb}{0.686275,0.352941,0.313725}%
\pgfsetfillcolor{currentfill}%
\pgfsetfillopacity{0.300000}%
\pgfsetlinewidth{0.000000pt}%
\definecolor{currentstroke}{rgb}{0.000000,0.000000,0.000000}%
\pgfsetstrokecolor{currentstroke}%
\pgfsetstrokeopacity{0.300000}%
\pgfsetdash{}{0pt}%
\pgfpathmoveto{\pgfqpoint{1.723206in}{0.383578in}}%
\pgfpathlineto{\pgfqpoint{1.740795in}{0.383578in}}%
\pgfpathlineto{\pgfqpoint{1.740795in}{0.714701in}}%
\pgfpathlineto{\pgfqpoint{1.723206in}{0.714701in}}%
\pgfpathclose%
\pgfusepath{fill}%
\end{pgfscope}%
\begin{pgfscope}%
\pgfpathrectangle{\pgfqpoint{0.526905in}{0.383578in}}{\pgfqpoint{3.875000in}{2.310000in}}%
\pgfusepath{clip}%
\pgfsetbuttcap%
\pgfsetmiterjoin%
\definecolor{currentfill}{rgb}{0.686275,0.352941,0.313725}%
\pgfsetfillcolor{currentfill}%
\pgfsetfillopacity{0.300000}%
\pgfsetlinewidth{0.000000pt}%
\definecolor{currentstroke}{rgb}{0.000000,0.000000,0.000000}%
\pgfsetstrokecolor{currentstroke}%
\pgfsetstrokeopacity{0.300000}%
\pgfsetdash{}{0pt}%
\pgfpathmoveto{\pgfqpoint{1.740795in}{0.383578in}}%
\pgfpathlineto{\pgfqpoint{1.758384in}{0.383578in}}%
\pgfpathlineto{\pgfqpoint{1.758384in}{0.697864in}}%
\pgfpathlineto{\pgfqpoint{1.740795in}{0.697864in}}%
\pgfpathclose%
\pgfusepath{fill}%
\end{pgfscope}%
\begin{pgfscope}%
\pgfpathrectangle{\pgfqpoint{0.526905in}{0.383578in}}{\pgfqpoint{3.875000in}{2.310000in}}%
\pgfusepath{clip}%
\pgfsetbuttcap%
\pgfsetmiterjoin%
\definecolor{currentfill}{rgb}{0.686275,0.352941,0.313725}%
\pgfsetfillcolor{currentfill}%
\pgfsetfillopacity{0.300000}%
\pgfsetlinewidth{0.000000pt}%
\definecolor{currentstroke}{rgb}{0.000000,0.000000,0.000000}%
\pgfsetstrokecolor{currentstroke}%
\pgfsetstrokeopacity{0.300000}%
\pgfsetdash{}{0pt}%
\pgfpathmoveto{\pgfqpoint{1.758384in}{0.383578in}}%
\pgfpathlineto{\pgfqpoint{1.775973in}{0.383578in}}%
\pgfpathlineto{\pgfqpoint{1.775973in}{0.787660in}}%
\pgfpathlineto{\pgfqpoint{1.758384in}{0.787660in}}%
\pgfpathclose%
\pgfusepath{fill}%
\end{pgfscope}%
\begin{pgfscope}%
\pgfpathrectangle{\pgfqpoint{0.526905in}{0.383578in}}{\pgfqpoint{3.875000in}{2.310000in}}%
\pgfusepath{clip}%
\pgfsetbuttcap%
\pgfsetmiterjoin%
\definecolor{currentfill}{rgb}{0.686275,0.352941,0.313725}%
\pgfsetfillcolor{currentfill}%
\pgfsetfillopacity{0.300000}%
\pgfsetlinewidth{0.000000pt}%
\definecolor{currentstroke}{rgb}{0.000000,0.000000,0.000000}%
\pgfsetstrokecolor{currentstroke}%
\pgfsetstrokeopacity{0.300000}%
\pgfsetdash{}{0pt}%
\pgfpathmoveto{\pgfqpoint{1.775973in}{0.383578in}}%
\pgfpathlineto{\pgfqpoint{1.793562in}{0.383578in}}%
\pgfpathlineto{\pgfqpoint{1.793562in}{0.843782in}}%
\pgfpathlineto{\pgfqpoint{1.775973in}{0.843782in}}%
\pgfpathclose%
\pgfusepath{fill}%
\end{pgfscope}%
\begin{pgfscope}%
\pgfpathrectangle{\pgfqpoint{0.526905in}{0.383578in}}{\pgfqpoint{3.875000in}{2.310000in}}%
\pgfusepath{clip}%
\pgfsetbuttcap%
\pgfsetmiterjoin%
\definecolor{currentfill}{rgb}{0.686275,0.352941,0.313725}%
\pgfsetfillcolor{currentfill}%
\pgfsetfillopacity{0.300000}%
\pgfsetlinewidth{0.000000pt}%
\definecolor{currentstroke}{rgb}{0.000000,0.000000,0.000000}%
\pgfsetstrokecolor{currentstroke}%
\pgfsetstrokeopacity{0.300000}%
\pgfsetdash{}{0pt}%
\pgfpathmoveto{\pgfqpoint{1.793562in}{0.383578in}}%
\pgfpathlineto{\pgfqpoint{1.811151in}{0.383578in}}%
\pgfpathlineto{\pgfqpoint{1.811151in}{0.888680in}}%
\pgfpathlineto{\pgfqpoint{1.793562in}{0.888680in}}%
\pgfpathclose%
\pgfusepath{fill}%
\end{pgfscope}%
\begin{pgfscope}%
\pgfpathrectangle{\pgfqpoint{0.526905in}{0.383578in}}{\pgfqpoint{3.875000in}{2.310000in}}%
\pgfusepath{clip}%
\pgfsetbuttcap%
\pgfsetmiterjoin%
\definecolor{currentfill}{rgb}{0.686275,0.352941,0.313725}%
\pgfsetfillcolor{currentfill}%
\pgfsetfillopacity{0.300000}%
\pgfsetlinewidth{0.000000pt}%
\definecolor{currentstroke}{rgb}{0.000000,0.000000,0.000000}%
\pgfsetstrokecolor{currentstroke}%
\pgfsetstrokeopacity{0.300000}%
\pgfsetdash{}{0pt}%
\pgfpathmoveto{\pgfqpoint{1.811151in}{0.383578in}}%
\pgfpathlineto{\pgfqpoint{1.828741in}{0.383578in}}%
\pgfpathlineto{\pgfqpoint{1.828741in}{0.939190in}}%
\pgfpathlineto{\pgfqpoint{1.811151in}{0.939190in}}%
\pgfpathclose%
\pgfusepath{fill}%
\end{pgfscope}%
\begin{pgfscope}%
\pgfpathrectangle{\pgfqpoint{0.526905in}{0.383578in}}{\pgfqpoint{3.875000in}{2.310000in}}%
\pgfusepath{clip}%
\pgfsetbuttcap%
\pgfsetmiterjoin%
\definecolor{currentfill}{rgb}{0.686275,0.352941,0.313725}%
\pgfsetfillcolor{currentfill}%
\pgfsetfillopacity{0.300000}%
\pgfsetlinewidth{0.000000pt}%
\definecolor{currentstroke}{rgb}{0.000000,0.000000,0.000000}%
\pgfsetstrokecolor{currentstroke}%
\pgfsetstrokeopacity{0.300000}%
\pgfsetdash{}{0pt}%
\pgfpathmoveto{\pgfqpoint{1.828741in}{0.383578in}}%
\pgfpathlineto{\pgfqpoint{1.846330in}{0.383578in}}%
\pgfpathlineto{\pgfqpoint{1.846330in}{0.821333in}}%
\pgfpathlineto{\pgfqpoint{1.828741in}{0.821333in}}%
\pgfpathclose%
\pgfusepath{fill}%
\end{pgfscope}%
\begin{pgfscope}%
\pgfpathrectangle{\pgfqpoint{0.526905in}{0.383578in}}{\pgfqpoint{3.875000in}{2.310000in}}%
\pgfusepath{clip}%
\pgfsetbuttcap%
\pgfsetmiterjoin%
\definecolor{currentfill}{rgb}{0.686275,0.352941,0.313725}%
\pgfsetfillcolor{currentfill}%
\pgfsetfillopacity{0.300000}%
\pgfsetlinewidth{0.000000pt}%
\definecolor{currentstroke}{rgb}{0.000000,0.000000,0.000000}%
\pgfsetstrokecolor{currentstroke}%
\pgfsetstrokeopacity{0.300000}%
\pgfsetdash{}{0pt}%
\pgfpathmoveto{\pgfqpoint{1.846330in}{0.383578in}}%
\pgfpathlineto{\pgfqpoint{1.863919in}{0.383578in}}%
\pgfpathlineto{\pgfqpoint{1.863919in}{0.871843in}}%
\pgfpathlineto{\pgfqpoint{1.846330in}{0.871843in}}%
\pgfpathclose%
\pgfusepath{fill}%
\end{pgfscope}%
\begin{pgfscope}%
\pgfpathrectangle{\pgfqpoint{0.526905in}{0.383578in}}{\pgfqpoint{3.875000in}{2.310000in}}%
\pgfusepath{clip}%
\pgfsetbuttcap%
\pgfsetmiterjoin%
\definecolor{currentfill}{rgb}{0.686275,0.352941,0.313725}%
\pgfsetfillcolor{currentfill}%
\pgfsetfillopacity{0.300000}%
\pgfsetlinewidth{0.000000pt}%
\definecolor{currentstroke}{rgb}{0.000000,0.000000,0.000000}%
\pgfsetstrokecolor{currentstroke}%
\pgfsetstrokeopacity{0.300000}%
\pgfsetdash{}{0pt}%
\pgfpathmoveto{\pgfqpoint{1.863919in}{0.383578in}}%
\pgfpathlineto{\pgfqpoint{1.881508in}{0.383578in}}%
\pgfpathlineto{\pgfqpoint{1.881508in}{0.888680in}}%
\pgfpathlineto{\pgfqpoint{1.863919in}{0.888680in}}%
\pgfpathclose%
\pgfusepath{fill}%
\end{pgfscope}%
\begin{pgfscope}%
\pgfpathrectangle{\pgfqpoint{0.526905in}{0.383578in}}{\pgfqpoint{3.875000in}{2.310000in}}%
\pgfusepath{clip}%
\pgfsetbuttcap%
\pgfsetmiterjoin%
\definecolor{currentfill}{rgb}{0.686275,0.352941,0.313725}%
\pgfsetfillcolor{currentfill}%
\pgfsetfillopacity{0.300000}%
\pgfsetlinewidth{0.000000pt}%
\definecolor{currentstroke}{rgb}{0.000000,0.000000,0.000000}%
\pgfsetstrokecolor{currentstroke}%
\pgfsetstrokeopacity{0.300000}%
\pgfsetdash{}{0pt}%
\pgfpathmoveto{\pgfqpoint{1.881508in}{0.383578in}}%
\pgfpathlineto{\pgfqpoint{1.899097in}{0.383578in}}%
\pgfpathlineto{\pgfqpoint{1.899097in}{1.040211in}}%
\pgfpathlineto{\pgfqpoint{1.881508in}{1.040211in}}%
\pgfpathclose%
\pgfusepath{fill}%
\end{pgfscope}%
\begin{pgfscope}%
\pgfpathrectangle{\pgfqpoint{0.526905in}{0.383578in}}{\pgfqpoint{3.875000in}{2.310000in}}%
\pgfusepath{clip}%
\pgfsetbuttcap%
\pgfsetmiterjoin%
\definecolor{currentfill}{rgb}{0.686275,0.352941,0.313725}%
\pgfsetfillcolor{currentfill}%
\pgfsetfillopacity{0.300000}%
\pgfsetlinewidth{0.000000pt}%
\definecolor{currentstroke}{rgb}{0.000000,0.000000,0.000000}%
\pgfsetstrokecolor{currentstroke}%
\pgfsetstrokeopacity{0.300000}%
\pgfsetdash{}{0pt}%
\pgfpathmoveto{\pgfqpoint{1.899097in}{0.383578in}}%
\pgfpathlineto{\pgfqpoint{1.916686in}{0.383578in}}%
\pgfpathlineto{\pgfqpoint{1.916686in}{1.068272in}}%
\pgfpathlineto{\pgfqpoint{1.899097in}{1.068272in}}%
\pgfpathclose%
\pgfusepath{fill}%
\end{pgfscope}%
\begin{pgfscope}%
\pgfpathrectangle{\pgfqpoint{0.526905in}{0.383578in}}{\pgfqpoint{3.875000in}{2.310000in}}%
\pgfusepath{clip}%
\pgfsetbuttcap%
\pgfsetmiterjoin%
\definecolor{currentfill}{rgb}{0.686275,0.352941,0.313725}%
\pgfsetfillcolor{currentfill}%
\pgfsetfillopacity{0.300000}%
\pgfsetlinewidth{0.000000pt}%
\definecolor{currentstroke}{rgb}{0.000000,0.000000,0.000000}%
\pgfsetstrokecolor{currentstroke}%
\pgfsetstrokeopacity{0.300000}%
\pgfsetdash{}{0pt}%
\pgfpathmoveto{\pgfqpoint{1.916686in}{0.383578in}}%
\pgfpathlineto{\pgfqpoint{1.934275in}{0.383578in}}%
\pgfpathlineto{\pgfqpoint{1.934275in}{1.101945in}}%
\pgfpathlineto{\pgfqpoint{1.916686in}{1.101945in}}%
\pgfpathclose%
\pgfusepath{fill}%
\end{pgfscope}%
\begin{pgfscope}%
\pgfpathrectangle{\pgfqpoint{0.526905in}{0.383578in}}{\pgfqpoint{3.875000in}{2.310000in}}%
\pgfusepath{clip}%
\pgfsetbuttcap%
\pgfsetmiterjoin%
\definecolor{currentfill}{rgb}{0.686275,0.352941,0.313725}%
\pgfsetfillcolor{currentfill}%
\pgfsetfillopacity{0.300000}%
\pgfsetlinewidth{0.000000pt}%
\definecolor{currentstroke}{rgb}{0.000000,0.000000,0.000000}%
\pgfsetstrokecolor{currentstroke}%
\pgfsetstrokeopacity{0.300000}%
\pgfsetdash{}{0pt}%
\pgfpathmoveto{\pgfqpoint{1.934275in}{0.383578in}}%
\pgfpathlineto{\pgfqpoint{1.951864in}{0.383578in}}%
\pgfpathlineto{\pgfqpoint{1.951864in}{0.922354in}}%
\pgfpathlineto{\pgfqpoint{1.934275in}{0.922354in}}%
\pgfpathclose%
\pgfusepath{fill}%
\end{pgfscope}%
\begin{pgfscope}%
\pgfpathrectangle{\pgfqpoint{0.526905in}{0.383578in}}{\pgfqpoint{3.875000in}{2.310000in}}%
\pgfusepath{clip}%
\pgfsetbuttcap%
\pgfsetmiterjoin%
\definecolor{currentfill}{rgb}{0.686275,0.352941,0.313725}%
\pgfsetfillcolor{currentfill}%
\pgfsetfillopacity{0.300000}%
\pgfsetlinewidth{0.000000pt}%
\definecolor{currentstroke}{rgb}{0.000000,0.000000,0.000000}%
\pgfsetstrokecolor{currentstroke}%
\pgfsetstrokeopacity{0.300000}%
\pgfsetdash{}{0pt}%
\pgfpathmoveto{\pgfqpoint{1.951864in}{0.383578in}}%
\pgfpathlineto{\pgfqpoint{1.969453in}{0.383578in}}%
\pgfpathlineto{\pgfqpoint{1.969453in}{1.186129in}}%
\pgfpathlineto{\pgfqpoint{1.951864in}{1.186129in}}%
\pgfpathclose%
\pgfusepath{fill}%
\end{pgfscope}%
\begin{pgfscope}%
\pgfpathrectangle{\pgfqpoint{0.526905in}{0.383578in}}{\pgfqpoint{3.875000in}{2.310000in}}%
\pgfusepath{clip}%
\pgfsetbuttcap%
\pgfsetmiterjoin%
\definecolor{currentfill}{rgb}{0.686275,0.352941,0.313725}%
\pgfsetfillcolor{currentfill}%
\pgfsetfillopacity{0.300000}%
\pgfsetlinewidth{0.000000pt}%
\definecolor{currentstroke}{rgb}{0.000000,0.000000,0.000000}%
\pgfsetstrokecolor{currentstroke}%
\pgfsetstrokeopacity{0.300000}%
\pgfsetdash{}{0pt}%
\pgfpathmoveto{\pgfqpoint{1.969453in}{0.383578in}}%
\pgfpathlineto{\pgfqpoint{1.987042in}{0.383578in}}%
\pgfpathlineto{\pgfqpoint{1.987042in}{1.180517in}}%
\pgfpathlineto{\pgfqpoint{1.969453in}{1.180517in}}%
\pgfpathclose%
\pgfusepath{fill}%
\end{pgfscope}%
\begin{pgfscope}%
\pgfpathrectangle{\pgfqpoint{0.526905in}{0.383578in}}{\pgfqpoint{3.875000in}{2.310000in}}%
\pgfusepath{clip}%
\pgfsetbuttcap%
\pgfsetmiterjoin%
\definecolor{currentfill}{rgb}{0.686275,0.352941,0.313725}%
\pgfsetfillcolor{currentfill}%
\pgfsetfillopacity{0.300000}%
\pgfsetlinewidth{0.000000pt}%
\definecolor{currentstroke}{rgb}{0.000000,0.000000,0.000000}%
\pgfsetstrokecolor{currentstroke}%
\pgfsetstrokeopacity{0.300000}%
\pgfsetdash{}{0pt}%
\pgfpathmoveto{\pgfqpoint{1.987042in}{0.383578in}}%
\pgfpathlineto{\pgfqpoint{2.004631in}{0.383578in}}%
\pgfpathlineto{\pgfqpoint{2.004631in}{1.158068in}}%
\pgfpathlineto{\pgfqpoint{1.987042in}{1.158068in}}%
\pgfpathclose%
\pgfusepath{fill}%
\end{pgfscope}%
\begin{pgfscope}%
\pgfpathrectangle{\pgfqpoint{0.526905in}{0.383578in}}{\pgfqpoint{3.875000in}{2.310000in}}%
\pgfusepath{clip}%
\pgfsetbuttcap%
\pgfsetmiterjoin%
\definecolor{currentfill}{rgb}{0.686275,0.352941,0.313725}%
\pgfsetfillcolor{currentfill}%
\pgfsetfillopacity{0.300000}%
\pgfsetlinewidth{0.000000pt}%
\definecolor{currentstroke}{rgb}{0.000000,0.000000,0.000000}%
\pgfsetstrokecolor{currentstroke}%
\pgfsetstrokeopacity{0.300000}%
\pgfsetdash{}{0pt}%
\pgfpathmoveto{\pgfqpoint{2.004631in}{0.383578in}}%
\pgfpathlineto{\pgfqpoint{2.022220in}{0.383578in}}%
\pgfpathlineto{\pgfqpoint{2.022220in}{1.191741in}}%
\pgfpathlineto{\pgfqpoint{2.004631in}{1.191741in}}%
\pgfpathclose%
\pgfusepath{fill}%
\end{pgfscope}%
\begin{pgfscope}%
\pgfpathrectangle{\pgfqpoint{0.526905in}{0.383578in}}{\pgfqpoint{3.875000in}{2.310000in}}%
\pgfusepath{clip}%
\pgfsetbuttcap%
\pgfsetmiterjoin%
\definecolor{currentfill}{rgb}{0.686275,0.352941,0.313725}%
\pgfsetfillcolor{currentfill}%
\pgfsetfillopacity{0.300000}%
\pgfsetlinewidth{0.000000pt}%
\definecolor{currentstroke}{rgb}{0.000000,0.000000,0.000000}%
\pgfsetstrokecolor{currentstroke}%
\pgfsetstrokeopacity{0.300000}%
\pgfsetdash{}{0pt}%
\pgfpathmoveto{\pgfqpoint{2.022220in}{0.383578in}}%
\pgfpathlineto{\pgfqpoint{2.039809in}{0.383578in}}%
\pgfpathlineto{\pgfqpoint{2.039809in}{1.186129in}}%
\pgfpathlineto{\pgfqpoint{2.022220in}{1.186129in}}%
\pgfpathclose%
\pgfusepath{fill}%
\end{pgfscope}%
\begin{pgfscope}%
\pgfpathrectangle{\pgfqpoint{0.526905in}{0.383578in}}{\pgfqpoint{3.875000in}{2.310000in}}%
\pgfusepath{clip}%
\pgfsetbuttcap%
\pgfsetmiterjoin%
\definecolor{currentfill}{rgb}{0.686275,0.352941,0.313725}%
\pgfsetfillcolor{currentfill}%
\pgfsetfillopacity{0.300000}%
\pgfsetlinewidth{0.000000pt}%
\definecolor{currentstroke}{rgb}{0.000000,0.000000,0.000000}%
\pgfsetstrokecolor{currentstroke}%
\pgfsetstrokeopacity{0.300000}%
\pgfsetdash{}{0pt}%
\pgfpathmoveto{\pgfqpoint{2.039809in}{0.383578in}}%
\pgfpathlineto{\pgfqpoint{2.057398in}{0.383578in}}%
\pgfpathlineto{\pgfqpoint{2.057398in}{1.225415in}}%
\pgfpathlineto{\pgfqpoint{2.039809in}{1.225415in}}%
\pgfpathclose%
\pgfusepath{fill}%
\end{pgfscope}%
\begin{pgfscope}%
\pgfpathrectangle{\pgfqpoint{0.526905in}{0.383578in}}{\pgfqpoint{3.875000in}{2.310000in}}%
\pgfusepath{clip}%
\pgfsetbuttcap%
\pgfsetmiterjoin%
\definecolor{currentfill}{rgb}{0.686275,0.352941,0.313725}%
\pgfsetfillcolor{currentfill}%
\pgfsetfillopacity{0.300000}%
\pgfsetlinewidth{0.000000pt}%
\definecolor{currentstroke}{rgb}{0.000000,0.000000,0.000000}%
\pgfsetstrokecolor{currentstroke}%
\pgfsetstrokeopacity{0.300000}%
\pgfsetdash{}{0pt}%
\pgfpathmoveto{\pgfqpoint{2.057398in}{0.383578in}}%
\pgfpathlineto{\pgfqpoint{2.074987in}{0.383578in}}%
\pgfpathlineto{\pgfqpoint{2.074987in}{1.247864in}}%
\pgfpathlineto{\pgfqpoint{2.057398in}{1.247864in}}%
\pgfpathclose%
\pgfusepath{fill}%
\end{pgfscope}%
\begin{pgfscope}%
\pgfpathrectangle{\pgfqpoint{0.526905in}{0.383578in}}{\pgfqpoint{3.875000in}{2.310000in}}%
\pgfusepath{clip}%
\pgfsetbuttcap%
\pgfsetmiterjoin%
\definecolor{currentfill}{rgb}{0.686275,0.352941,0.313725}%
\pgfsetfillcolor{currentfill}%
\pgfsetfillopacity{0.300000}%
\pgfsetlinewidth{0.000000pt}%
\definecolor{currentstroke}{rgb}{0.000000,0.000000,0.000000}%
\pgfsetstrokecolor{currentstroke}%
\pgfsetstrokeopacity{0.300000}%
\pgfsetdash{}{0pt}%
\pgfpathmoveto{\pgfqpoint{2.074987in}{0.383578in}}%
\pgfpathlineto{\pgfqpoint{2.092576in}{0.383578in}}%
\pgfpathlineto{\pgfqpoint{2.092576in}{1.416231in}}%
\pgfpathlineto{\pgfqpoint{2.074987in}{1.416231in}}%
\pgfpathclose%
\pgfusepath{fill}%
\end{pgfscope}%
\begin{pgfscope}%
\pgfpathrectangle{\pgfqpoint{0.526905in}{0.383578in}}{\pgfqpoint{3.875000in}{2.310000in}}%
\pgfusepath{clip}%
\pgfsetbuttcap%
\pgfsetmiterjoin%
\definecolor{currentfill}{rgb}{0.686275,0.352941,0.313725}%
\pgfsetfillcolor{currentfill}%
\pgfsetfillopacity{0.300000}%
\pgfsetlinewidth{0.000000pt}%
\definecolor{currentstroke}{rgb}{0.000000,0.000000,0.000000}%
\pgfsetstrokecolor{currentstroke}%
\pgfsetstrokeopacity{0.300000}%
\pgfsetdash{}{0pt}%
\pgfpathmoveto{\pgfqpoint{2.092576in}{0.383578in}}%
\pgfpathlineto{\pgfqpoint{2.110165in}{0.383578in}}%
\pgfpathlineto{\pgfqpoint{2.110165in}{1.231027in}}%
\pgfpathlineto{\pgfqpoint{2.092576in}{1.231027in}}%
\pgfpathclose%
\pgfusepath{fill}%
\end{pgfscope}%
\begin{pgfscope}%
\pgfpathrectangle{\pgfqpoint{0.526905in}{0.383578in}}{\pgfqpoint{3.875000in}{2.310000in}}%
\pgfusepath{clip}%
\pgfsetbuttcap%
\pgfsetmiterjoin%
\definecolor{currentfill}{rgb}{0.686275,0.352941,0.313725}%
\pgfsetfillcolor{currentfill}%
\pgfsetfillopacity{0.300000}%
\pgfsetlinewidth{0.000000pt}%
\definecolor{currentstroke}{rgb}{0.000000,0.000000,0.000000}%
\pgfsetstrokecolor{currentstroke}%
\pgfsetstrokeopacity{0.300000}%
\pgfsetdash{}{0pt}%
\pgfpathmoveto{\pgfqpoint{2.110165in}{0.383578in}}%
\pgfpathlineto{\pgfqpoint{2.127754in}{0.383578in}}%
\pgfpathlineto{\pgfqpoint{2.127754in}{1.410619in}}%
\pgfpathlineto{\pgfqpoint{2.110165in}{1.410619in}}%
\pgfpathclose%
\pgfusepath{fill}%
\end{pgfscope}%
\begin{pgfscope}%
\pgfpathrectangle{\pgfqpoint{0.526905in}{0.383578in}}{\pgfqpoint{3.875000in}{2.310000in}}%
\pgfusepath{clip}%
\pgfsetbuttcap%
\pgfsetmiterjoin%
\definecolor{currentfill}{rgb}{0.686275,0.352941,0.313725}%
\pgfsetfillcolor{currentfill}%
\pgfsetfillopacity{0.300000}%
\pgfsetlinewidth{0.000000pt}%
\definecolor{currentstroke}{rgb}{0.000000,0.000000,0.000000}%
\pgfsetstrokecolor{currentstroke}%
\pgfsetstrokeopacity{0.300000}%
\pgfsetdash{}{0pt}%
\pgfpathmoveto{\pgfqpoint{2.127754in}{0.383578in}}%
\pgfpathlineto{\pgfqpoint{2.145343in}{0.383578in}}%
\pgfpathlineto{\pgfqpoint{2.145343in}{1.360109in}}%
\pgfpathlineto{\pgfqpoint{2.127754in}{1.360109in}}%
\pgfpathclose%
\pgfusepath{fill}%
\end{pgfscope}%
\begin{pgfscope}%
\pgfpathrectangle{\pgfqpoint{0.526905in}{0.383578in}}{\pgfqpoint{3.875000in}{2.310000in}}%
\pgfusepath{clip}%
\pgfsetbuttcap%
\pgfsetmiterjoin%
\definecolor{currentfill}{rgb}{0.686275,0.352941,0.313725}%
\pgfsetfillcolor{currentfill}%
\pgfsetfillopacity{0.300000}%
\pgfsetlinewidth{0.000000pt}%
\definecolor{currentstroke}{rgb}{0.000000,0.000000,0.000000}%
\pgfsetstrokecolor{currentstroke}%
\pgfsetstrokeopacity{0.300000}%
\pgfsetdash{}{0pt}%
\pgfpathmoveto{\pgfqpoint{2.145343in}{0.383578in}}%
\pgfpathlineto{\pgfqpoint{2.162933in}{0.383578in}}%
\pgfpathlineto{\pgfqpoint{2.162933in}{1.483578in}}%
\pgfpathlineto{\pgfqpoint{2.145343in}{1.483578in}}%
\pgfpathclose%
\pgfusepath{fill}%
\end{pgfscope}%
\begin{pgfscope}%
\pgfpathrectangle{\pgfqpoint{0.526905in}{0.383578in}}{\pgfqpoint{3.875000in}{2.310000in}}%
\pgfusepath{clip}%
\pgfsetbuttcap%
\pgfsetmiterjoin%
\definecolor{currentfill}{rgb}{0.686275,0.352941,0.313725}%
\pgfsetfillcolor{currentfill}%
\pgfsetfillopacity{0.300000}%
\pgfsetlinewidth{0.000000pt}%
\definecolor{currentstroke}{rgb}{0.000000,0.000000,0.000000}%
\pgfsetstrokecolor{currentstroke}%
\pgfsetstrokeopacity{0.300000}%
\pgfsetdash{}{0pt}%
\pgfpathmoveto{\pgfqpoint{2.162933in}{0.383578in}}%
\pgfpathlineto{\pgfqpoint{2.180522in}{0.383578in}}%
\pgfpathlineto{\pgfqpoint{2.180522in}{1.405007in}}%
\pgfpathlineto{\pgfqpoint{2.162933in}{1.405007in}}%
\pgfpathclose%
\pgfusepath{fill}%
\end{pgfscope}%
\begin{pgfscope}%
\pgfpathrectangle{\pgfqpoint{0.526905in}{0.383578in}}{\pgfqpoint{3.875000in}{2.310000in}}%
\pgfusepath{clip}%
\pgfsetbuttcap%
\pgfsetmiterjoin%
\definecolor{currentfill}{rgb}{0.686275,0.352941,0.313725}%
\pgfsetfillcolor{currentfill}%
\pgfsetfillopacity{0.300000}%
\pgfsetlinewidth{0.000000pt}%
\definecolor{currentstroke}{rgb}{0.000000,0.000000,0.000000}%
\pgfsetstrokecolor{currentstroke}%
\pgfsetstrokeopacity{0.300000}%
\pgfsetdash{}{0pt}%
\pgfpathmoveto{\pgfqpoint{2.180522in}{0.383578in}}%
\pgfpathlineto{\pgfqpoint{2.198111in}{0.383578in}}%
\pgfpathlineto{\pgfqpoint{2.198111in}{1.573374in}}%
\pgfpathlineto{\pgfqpoint{2.180522in}{1.573374in}}%
\pgfpathclose%
\pgfusepath{fill}%
\end{pgfscope}%
\begin{pgfscope}%
\pgfpathrectangle{\pgfqpoint{0.526905in}{0.383578in}}{\pgfqpoint{3.875000in}{2.310000in}}%
\pgfusepath{clip}%
\pgfsetbuttcap%
\pgfsetmiterjoin%
\definecolor{currentfill}{rgb}{0.686275,0.352941,0.313725}%
\pgfsetfillcolor{currentfill}%
\pgfsetfillopacity{0.300000}%
\pgfsetlinewidth{0.000000pt}%
\definecolor{currentstroke}{rgb}{0.000000,0.000000,0.000000}%
\pgfsetstrokecolor{currentstroke}%
\pgfsetstrokeopacity{0.300000}%
\pgfsetdash{}{0pt}%
\pgfpathmoveto{\pgfqpoint{2.198111in}{0.383578in}}%
\pgfpathlineto{\pgfqpoint{2.215700in}{0.383578in}}%
\pgfpathlineto{\pgfqpoint{2.215700in}{1.528476in}}%
\pgfpathlineto{\pgfqpoint{2.198111in}{1.528476in}}%
\pgfpathclose%
\pgfusepath{fill}%
\end{pgfscope}%
\begin{pgfscope}%
\pgfpathrectangle{\pgfqpoint{0.526905in}{0.383578in}}{\pgfqpoint{3.875000in}{2.310000in}}%
\pgfusepath{clip}%
\pgfsetbuttcap%
\pgfsetmiterjoin%
\definecolor{currentfill}{rgb}{0.686275,0.352941,0.313725}%
\pgfsetfillcolor{currentfill}%
\pgfsetfillopacity{0.300000}%
\pgfsetlinewidth{0.000000pt}%
\definecolor{currentstroke}{rgb}{0.000000,0.000000,0.000000}%
\pgfsetstrokecolor{currentstroke}%
\pgfsetstrokeopacity{0.300000}%
\pgfsetdash{}{0pt}%
\pgfpathmoveto{\pgfqpoint{2.215700in}{0.383578in}}%
\pgfpathlineto{\pgfqpoint{2.233289in}{0.383578in}}%
\pgfpathlineto{\pgfqpoint{2.233289in}{1.629496in}}%
\pgfpathlineto{\pgfqpoint{2.215700in}{1.629496in}}%
\pgfpathclose%
\pgfusepath{fill}%
\end{pgfscope}%
\begin{pgfscope}%
\pgfpathrectangle{\pgfqpoint{0.526905in}{0.383578in}}{\pgfqpoint{3.875000in}{2.310000in}}%
\pgfusepath{clip}%
\pgfsetbuttcap%
\pgfsetmiterjoin%
\definecolor{currentfill}{rgb}{0.686275,0.352941,0.313725}%
\pgfsetfillcolor{currentfill}%
\pgfsetfillopacity{0.300000}%
\pgfsetlinewidth{0.000000pt}%
\definecolor{currentstroke}{rgb}{0.000000,0.000000,0.000000}%
\pgfsetstrokecolor{currentstroke}%
\pgfsetstrokeopacity{0.300000}%
\pgfsetdash{}{0pt}%
\pgfpathmoveto{\pgfqpoint{2.233289in}{0.383578in}}%
\pgfpathlineto{\pgfqpoint{2.250878in}{0.383578in}}%
\pgfpathlineto{\pgfqpoint{2.250878in}{1.545313in}}%
\pgfpathlineto{\pgfqpoint{2.233289in}{1.545313in}}%
\pgfpathclose%
\pgfusepath{fill}%
\end{pgfscope}%
\begin{pgfscope}%
\pgfpathrectangle{\pgfqpoint{0.526905in}{0.383578in}}{\pgfqpoint{3.875000in}{2.310000in}}%
\pgfusepath{clip}%
\pgfsetbuttcap%
\pgfsetmiterjoin%
\definecolor{currentfill}{rgb}{0.686275,0.352941,0.313725}%
\pgfsetfillcolor{currentfill}%
\pgfsetfillopacity{0.300000}%
\pgfsetlinewidth{0.000000pt}%
\definecolor{currentstroke}{rgb}{0.000000,0.000000,0.000000}%
\pgfsetstrokecolor{currentstroke}%
\pgfsetstrokeopacity{0.300000}%
\pgfsetdash{}{0pt}%
\pgfpathmoveto{\pgfqpoint{2.250878in}{0.383578in}}%
\pgfpathlineto{\pgfqpoint{2.268467in}{0.383578in}}%
\pgfpathlineto{\pgfqpoint{2.268467in}{1.618272in}}%
\pgfpathlineto{\pgfqpoint{2.250878in}{1.618272in}}%
\pgfpathclose%
\pgfusepath{fill}%
\end{pgfscope}%
\begin{pgfscope}%
\pgfpathrectangle{\pgfqpoint{0.526905in}{0.383578in}}{\pgfqpoint{3.875000in}{2.310000in}}%
\pgfusepath{clip}%
\pgfsetbuttcap%
\pgfsetmiterjoin%
\definecolor{currentfill}{rgb}{0.686275,0.352941,0.313725}%
\pgfsetfillcolor{currentfill}%
\pgfsetfillopacity{0.300000}%
\pgfsetlinewidth{0.000000pt}%
\definecolor{currentstroke}{rgb}{0.000000,0.000000,0.000000}%
\pgfsetstrokecolor{currentstroke}%
\pgfsetstrokeopacity{0.300000}%
\pgfsetdash{}{0pt}%
\pgfpathmoveto{\pgfqpoint{2.268467in}{0.383578in}}%
\pgfpathlineto{\pgfqpoint{2.286056in}{0.383578in}}%
\pgfpathlineto{\pgfqpoint{2.286056in}{1.708068in}}%
\pgfpathlineto{\pgfqpoint{2.268467in}{1.708068in}}%
\pgfpathclose%
\pgfusepath{fill}%
\end{pgfscope}%
\begin{pgfscope}%
\pgfpathrectangle{\pgfqpoint{0.526905in}{0.383578in}}{\pgfqpoint{3.875000in}{2.310000in}}%
\pgfusepath{clip}%
\pgfsetbuttcap%
\pgfsetmiterjoin%
\definecolor{currentfill}{rgb}{0.686275,0.352941,0.313725}%
\pgfsetfillcolor{currentfill}%
\pgfsetfillopacity{0.300000}%
\pgfsetlinewidth{0.000000pt}%
\definecolor{currentstroke}{rgb}{0.000000,0.000000,0.000000}%
\pgfsetstrokecolor{currentstroke}%
\pgfsetstrokeopacity{0.300000}%
\pgfsetdash{}{0pt}%
\pgfpathmoveto{\pgfqpoint{2.286056in}{0.383578in}}%
\pgfpathlineto{\pgfqpoint{2.303645in}{0.383578in}}%
\pgfpathlineto{\pgfqpoint{2.303645in}{1.848374in}}%
\pgfpathlineto{\pgfqpoint{2.286056in}{1.848374in}}%
\pgfpathclose%
\pgfusepath{fill}%
\end{pgfscope}%
\begin{pgfscope}%
\pgfpathrectangle{\pgfqpoint{0.526905in}{0.383578in}}{\pgfqpoint{3.875000in}{2.310000in}}%
\pgfusepath{clip}%
\pgfsetbuttcap%
\pgfsetmiterjoin%
\definecolor{currentfill}{rgb}{0.686275,0.352941,0.313725}%
\pgfsetfillcolor{currentfill}%
\pgfsetfillopacity{0.300000}%
\pgfsetlinewidth{0.000000pt}%
\definecolor{currentstroke}{rgb}{0.000000,0.000000,0.000000}%
\pgfsetstrokecolor{currentstroke}%
\pgfsetstrokeopacity{0.300000}%
\pgfsetdash{}{0pt}%
\pgfpathmoveto{\pgfqpoint{2.303645in}{0.383578in}}%
\pgfpathlineto{\pgfqpoint{2.321234in}{0.383578in}}%
\pgfpathlineto{\pgfqpoint{2.321234in}{1.724905in}}%
\pgfpathlineto{\pgfqpoint{2.303645in}{1.724905in}}%
\pgfpathclose%
\pgfusepath{fill}%
\end{pgfscope}%
\begin{pgfscope}%
\pgfpathrectangle{\pgfqpoint{0.526905in}{0.383578in}}{\pgfqpoint{3.875000in}{2.310000in}}%
\pgfusepath{clip}%
\pgfsetbuttcap%
\pgfsetmiterjoin%
\definecolor{currentfill}{rgb}{0.686275,0.352941,0.313725}%
\pgfsetfillcolor{currentfill}%
\pgfsetfillopacity{0.300000}%
\pgfsetlinewidth{0.000000pt}%
\definecolor{currentstroke}{rgb}{0.000000,0.000000,0.000000}%
\pgfsetstrokecolor{currentstroke}%
\pgfsetstrokeopacity{0.300000}%
\pgfsetdash{}{0pt}%
\pgfpathmoveto{\pgfqpoint{2.321234in}{0.383578in}}%
\pgfpathlineto{\pgfqpoint{2.338823in}{0.383578in}}%
\pgfpathlineto{\pgfqpoint{2.338823in}{1.752966in}}%
\pgfpathlineto{\pgfqpoint{2.321234in}{1.752966in}}%
\pgfpathclose%
\pgfusepath{fill}%
\end{pgfscope}%
\begin{pgfscope}%
\pgfpathrectangle{\pgfqpoint{0.526905in}{0.383578in}}{\pgfqpoint{3.875000in}{2.310000in}}%
\pgfusepath{clip}%
\pgfsetbuttcap%
\pgfsetmiterjoin%
\definecolor{currentfill}{rgb}{0.686275,0.352941,0.313725}%
\pgfsetfillcolor{currentfill}%
\pgfsetfillopacity{0.300000}%
\pgfsetlinewidth{0.000000pt}%
\definecolor{currentstroke}{rgb}{0.000000,0.000000,0.000000}%
\pgfsetstrokecolor{currentstroke}%
\pgfsetstrokeopacity{0.300000}%
\pgfsetdash{}{0pt}%
\pgfpathmoveto{\pgfqpoint{2.338823in}{0.383578in}}%
\pgfpathlineto{\pgfqpoint{2.356412in}{0.383578in}}%
\pgfpathlineto{\pgfqpoint{2.356412in}{1.983068in}}%
\pgfpathlineto{\pgfqpoint{2.338823in}{1.983068in}}%
\pgfpathclose%
\pgfusepath{fill}%
\end{pgfscope}%
\begin{pgfscope}%
\pgfpathrectangle{\pgfqpoint{0.526905in}{0.383578in}}{\pgfqpoint{3.875000in}{2.310000in}}%
\pgfusepath{clip}%
\pgfsetbuttcap%
\pgfsetmiterjoin%
\definecolor{currentfill}{rgb}{0.686275,0.352941,0.313725}%
\pgfsetfillcolor{currentfill}%
\pgfsetfillopacity{0.300000}%
\pgfsetlinewidth{0.000000pt}%
\definecolor{currentstroke}{rgb}{0.000000,0.000000,0.000000}%
\pgfsetstrokecolor{currentstroke}%
\pgfsetstrokeopacity{0.300000}%
\pgfsetdash{}{0pt}%
\pgfpathmoveto{\pgfqpoint{2.356412in}{0.383578in}}%
\pgfpathlineto{\pgfqpoint{2.374001in}{0.383578in}}%
\pgfpathlineto{\pgfqpoint{2.374001in}{2.005517in}}%
\pgfpathlineto{\pgfqpoint{2.356412in}{2.005517in}}%
\pgfpathclose%
\pgfusepath{fill}%
\end{pgfscope}%
\begin{pgfscope}%
\pgfpathrectangle{\pgfqpoint{0.526905in}{0.383578in}}{\pgfqpoint{3.875000in}{2.310000in}}%
\pgfusepath{clip}%
\pgfsetbuttcap%
\pgfsetmiterjoin%
\definecolor{currentfill}{rgb}{0.686275,0.352941,0.313725}%
\pgfsetfillcolor{currentfill}%
\pgfsetfillopacity{0.300000}%
\pgfsetlinewidth{0.000000pt}%
\definecolor{currentstroke}{rgb}{0.000000,0.000000,0.000000}%
\pgfsetstrokecolor{currentstroke}%
\pgfsetstrokeopacity{0.300000}%
\pgfsetdash{}{0pt}%
\pgfpathmoveto{\pgfqpoint{2.374001in}{0.383578in}}%
\pgfpathlineto{\pgfqpoint{2.391590in}{0.383578in}}%
\pgfpathlineto{\pgfqpoint{2.391590in}{1.876435in}}%
\pgfpathlineto{\pgfqpoint{2.374001in}{1.876435in}}%
\pgfpathclose%
\pgfusepath{fill}%
\end{pgfscope}%
\begin{pgfscope}%
\pgfpathrectangle{\pgfqpoint{0.526905in}{0.383578in}}{\pgfqpoint{3.875000in}{2.310000in}}%
\pgfusepath{clip}%
\pgfsetbuttcap%
\pgfsetmiterjoin%
\definecolor{currentfill}{rgb}{0.686275,0.352941,0.313725}%
\pgfsetfillcolor{currentfill}%
\pgfsetfillopacity{0.300000}%
\pgfsetlinewidth{0.000000pt}%
\definecolor{currentstroke}{rgb}{0.000000,0.000000,0.000000}%
\pgfsetstrokecolor{currentstroke}%
\pgfsetstrokeopacity{0.300000}%
\pgfsetdash{}{0pt}%
\pgfpathmoveto{\pgfqpoint{2.391590in}{0.383578in}}%
\pgfpathlineto{\pgfqpoint{2.409179in}{0.383578in}}%
\pgfpathlineto{\pgfqpoint{2.409179in}{2.123374in}}%
\pgfpathlineto{\pgfqpoint{2.391590in}{2.123374in}}%
\pgfpathclose%
\pgfusepath{fill}%
\end{pgfscope}%
\begin{pgfscope}%
\pgfpathrectangle{\pgfqpoint{0.526905in}{0.383578in}}{\pgfqpoint{3.875000in}{2.310000in}}%
\pgfusepath{clip}%
\pgfsetbuttcap%
\pgfsetmiterjoin%
\definecolor{currentfill}{rgb}{0.686275,0.352941,0.313725}%
\pgfsetfillcolor{currentfill}%
\pgfsetfillopacity{0.300000}%
\pgfsetlinewidth{0.000000pt}%
\definecolor{currentstroke}{rgb}{0.000000,0.000000,0.000000}%
\pgfsetstrokecolor{currentstroke}%
\pgfsetstrokeopacity{0.300000}%
\pgfsetdash{}{0pt}%
\pgfpathmoveto{\pgfqpoint{2.409179in}{0.383578in}}%
\pgfpathlineto{\pgfqpoint{2.426768in}{0.383578in}}%
\pgfpathlineto{\pgfqpoint{2.426768in}{2.089701in}}%
\pgfpathlineto{\pgfqpoint{2.409179in}{2.089701in}}%
\pgfpathclose%
\pgfusepath{fill}%
\end{pgfscope}%
\begin{pgfscope}%
\pgfpathrectangle{\pgfqpoint{0.526905in}{0.383578in}}{\pgfqpoint{3.875000in}{2.310000in}}%
\pgfusepath{clip}%
\pgfsetbuttcap%
\pgfsetmiterjoin%
\definecolor{currentfill}{rgb}{0.686275,0.352941,0.313725}%
\pgfsetfillcolor{currentfill}%
\pgfsetfillopacity{0.300000}%
\pgfsetlinewidth{0.000000pt}%
\definecolor{currentstroke}{rgb}{0.000000,0.000000,0.000000}%
\pgfsetstrokecolor{currentstroke}%
\pgfsetstrokeopacity{0.300000}%
\pgfsetdash{}{0pt}%
\pgfpathmoveto{\pgfqpoint{2.426768in}{0.383578in}}%
\pgfpathlineto{\pgfqpoint{2.444357in}{0.383578in}}%
\pgfpathlineto{\pgfqpoint{2.444357in}{2.117762in}}%
\pgfpathlineto{\pgfqpoint{2.426768in}{2.117762in}}%
\pgfpathclose%
\pgfusepath{fill}%
\end{pgfscope}%
\begin{pgfscope}%
\pgfpathrectangle{\pgfqpoint{0.526905in}{0.383578in}}{\pgfqpoint{3.875000in}{2.310000in}}%
\pgfusepath{clip}%
\pgfsetbuttcap%
\pgfsetmiterjoin%
\definecolor{currentfill}{rgb}{0.686275,0.352941,0.313725}%
\pgfsetfillcolor{currentfill}%
\pgfsetfillopacity{0.300000}%
\pgfsetlinewidth{0.000000pt}%
\definecolor{currentstroke}{rgb}{0.000000,0.000000,0.000000}%
\pgfsetstrokecolor{currentstroke}%
\pgfsetstrokeopacity{0.300000}%
\pgfsetdash{}{0pt}%
\pgfpathmoveto{\pgfqpoint{2.444357in}{0.383578in}}%
\pgfpathlineto{\pgfqpoint{2.461946in}{0.383578in}}%
\pgfpathlineto{\pgfqpoint{2.461946in}{2.128986in}}%
\pgfpathlineto{\pgfqpoint{2.444357in}{2.128986in}}%
\pgfpathclose%
\pgfusepath{fill}%
\end{pgfscope}%
\begin{pgfscope}%
\pgfpathrectangle{\pgfqpoint{0.526905in}{0.383578in}}{\pgfqpoint{3.875000in}{2.310000in}}%
\pgfusepath{clip}%
\pgfsetbuttcap%
\pgfsetmiterjoin%
\definecolor{currentfill}{rgb}{0.686275,0.352941,0.313725}%
\pgfsetfillcolor{currentfill}%
\pgfsetfillopacity{0.300000}%
\pgfsetlinewidth{0.000000pt}%
\definecolor{currentstroke}{rgb}{0.000000,0.000000,0.000000}%
\pgfsetstrokecolor{currentstroke}%
\pgfsetstrokeopacity{0.300000}%
\pgfsetdash{}{0pt}%
\pgfpathmoveto{\pgfqpoint{2.461946in}{0.383578in}}%
\pgfpathlineto{\pgfqpoint{2.479535in}{0.383578in}}%
\pgfpathlineto{\pgfqpoint{2.479535in}{2.420823in}}%
\pgfpathlineto{\pgfqpoint{2.461946in}{2.420823in}}%
\pgfpathclose%
\pgfusepath{fill}%
\end{pgfscope}%
\begin{pgfscope}%
\pgfpathrectangle{\pgfqpoint{0.526905in}{0.383578in}}{\pgfqpoint{3.875000in}{2.310000in}}%
\pgfusepath{clip}%
\pgfsetbuttcap%
\pgfsetmiterjoin%
\definecolor{currentfill}{rgb}{0.686275,0.352941,0.313725}%
\pgfsetfillcolor{currentfill}%
\pgfsetfillopacity{0.300000}%
\pgfsetlinewidth{0.000000pt}%
\definecolor{currentstroke}{rgb}{0.000000,0.000000,0.000000}%
\pgfsetstrokecolor{currentstroke}%
\pgfsetstrokeopacity{0.300000}%
\pgfsetdash{}{0pt}%
\pgfpathmoveto{\pgfqpoint{2.479535in}{0.383578in}}%
\pgfpathlineto{\pgfqpoint{2.497125in}{0.383578in}}%
\pgfpathlineto{\pgfqpoint{2.497125in}{2.235619in}}%
\pgfpathlineto{\pgfqpoint{2.479535in}{2.235619in}}%
\pgfpathclose%
\pgfusepath{fill}%
\end{pgfscope}%
\begin{pgfscope}%
\pgfpathrectangle{\pgfqpoint{0.526905in}{0.383578in}}{\pgfqpoint{3.875000in}{2.310000in}}%
\pgfusepath{clip}%
\pgfsetbuttcap%
\pgfsetmiterjoin%
\definecolor{currentfill}{rgb}{0.686275,0.352941,0.313725}%
\pgfsetfillcolor{currentfill}%
\pgfsetfillopacity{0.300000}%
\pgfsetlinewidth{0.000000pt}%
\definecolor{currentstroke}{rgb}{0.000000,0.000000,0.000000}%
\pgfsetstrokecolor{currentstroke}%
\pgfsetstrokeopacity{0.300000}%
\pgfsetdash{}{0pt}%
\pgfpathmoveto{\pgfqpoint{2.497125in}{0.383578in}}%
\pgfpathlineto{\pgfqpoint{2.514714in}{0.383578in}}%
\pgfpathlineto{\pgfqpoint{2.514714in}{2.381537in}}%
\pgfpathlineto{\pgfqpoint{2.497125in}{2.381537in}}%
\pgfpathclose%
\pgfusepath{fill}%
\end{pgfscope}%
\begin{pgfscope}%
\pgfpathrectangle{\pgfqpoint{0.526905in}{0.383578in}}{\pgfqpoint{3.875000in}{2.310000in}}%
\pgfusepath{clip}%
\pgfsetbuttcap%
\pgfsetmiterjoin%
\definecolor{currentfill}{rgb}{0.686275,0.352941,0.313725}%
\pgfsetfillcolor{currentfill}%
\pgfsetfillopacity{0.300000}%
\pgfsetlinewidth{0.000000pt}%
\definecolor{currentstroke}{rgb}{0.000000,0.000000,0.000000}%
\pgfsetstrokecolor{currentstroke}%
\pgfsetstrokeopacity{0.300000}%
\pgfsetdash{}{0pt}%
\pgfpathmoveto{\pgfqpoint{2.514714in}{0.383578in}}%
\pgfpathlineto{\pgfqpoint{2.532303in}{0.383578in}}%
\pgfpathlineto{\pgfqpoint{2.532303in}{2.246843in}}%
\pgfpathlineto{\pgfqpoint{2.514714in}{2.246843in}}%
\pgfpathclose%
\pgfusepath{fill}%
\end{pgfscope}%
\begin{pgfscope}%
\pgfpathrectangle{\pgfqpoint{0.526905in}{0.383578in}}{\pgfqpoint{3.875000in}{2.310000in}}%
\pgfusepath{clip}%
\pgfsetbuttcap%
\pgfsetmiterjoin%
\definecolor{currentfill}{rgb}{0.686275,0.352941,0.313725}%
\pgfsetfillcolor{currentfill}%
\pgfsetfillopacity{0.300000}%
\pgfsetlinewidth{0.000000pt}%
\definecolor{currentstroke}{rgb}{0.000000,0.000000,0.000000}%
\pgfsetstrokecolor{currentstroke}%
\pgfsetstrokeopacity{0.300000}%
\pgfsetdash{}{0pt}%
\pgfpathmoveto{\pgfqpoint{2.532303in}{0.383578in}}%
\pgfpathlineto{\pgfqpoint{2.549892in}{0.383578in}}%
\pgfpathlineto{\pgfqpoint{2.549892in}{2.319803in}}%
\pgfpathlineto{\pgfqpoint{2.532303in}{2.319803in}}%
\pgfpathclose%
\pgfusepath{fill}%
\end{pgfscope}%
\begin{pgfscope}%
\pgfpathrectangle{\pgfqpoint{0.526905in}{0.383578in}}{\pgfqpoint{3.875000in}{2.310000in}}%
\pgfusepath{clip}%
\pgfsetbuttcap%
\pgfsetmiterjoin%
\definecolor{currentfill}{rgb}{0.686275,0.352941,0.313725}%
\pgfsetfillcolor{currentfill}%
\pgfsetfillopacity{0.300000}%
\pgfsetlinewidth{0.000000pt}%
\definecolor{currentstroke}{rgb}{0.000000,0.000000,0.000000}%
\pgfsetstrokecolor{currentstroke}%
\pgfsetstrokeopacity{0.300000}%
\pgfsetdash{}{0pt}%
\pgfpathmoveto{\pgfqpoint{2.549892in}{0.383578in}}%
\pgfpathlineto{\pgfqpoint{2.567481in}{0.383578in}}%
\pgfpathlineto{\pgfqpoint{2.567481in}{2.230007in}}%
\pgfpathlineto{\pgfqpoint{2.549892in}{2.230007in}}%
\pgfpathclose%
\pgfusepath{fill}%
\end{pgfscope}%
\begin{pgfscope}%
\pgfpathrectangle{\pgfqpoint{0.526905in}{0.383578in}}{\pgfqpoint{3.875000in}{2.310000in}}%
\pgfusepath{clip}%
\pgfsetbuttcap%
\pgfsetmiterjoin%
\definecolor{currentfill}{rgb}{0.686275,0.352941,0.313725}%
\pgfsetfillcolor{currentfill}%
\pgfsetfillopacity{0.300000}%
\pgfsetlinewidth{0.000000pt}%
\definecolor{currentstroke}{rgb}{0.000000,0.000000,0.000000}%
\pgfsetstrokecolor{currentstroke}%
\pgfsetstrokeopacity{0.300000}%
\pgfsetdash{}{0pt}%
\pgfpathmoveto{\pgfqpoint{2.567481in}{0.383578in}}%
\pgfpathlineto{\pgfqpoint{2.585070in}{0.383578in}}%
\pgfpathlineto{\pgfqpoint{2.585070in}{2.302966in}}%
\pgfpathlineto{\pgfqpoint{2.567481in}{2.302966in}}%
\pgfpathclose%
\pgfusepath{fill}%
\end{pgfscope}%
\begin{pgfscope}%
\pgfpathrectangle{\pgfqpoint{0.526905in}{0.383578in}}{\pgfqpoint{3.875000in}{2.310000in}}%
\pgfusepath{clip}%
\pgfsetbuttcap%
\pgfsetmiterjoin%
\definecolor{currentfill}{rgb}{0.686275,0.352941,0.313725}%
\pgfsetfillcolor{currentfill}%
\pgfsetfillopacity{0.300000}%
\pgfsetlinewidth{0.000000pt}%
\definecolor{currentstroke}{rgb}{0.000000,0.000000,0.000000}%
\pgfsetstrokecolor{currentstroke}%
\pgfsetstrokeopacity{0.300000}%
\pgfsetdash{}{0pt}%
\pgfpathmoveto{\pgfqpoint{2.585070in}{0.383578in}}%
\pgfpathlineto{\pgfqpoint{2.602659in}{0.383578in}}%
\pgfpathlineto{\pgfqpoint{2.602659in}{2.516231in}}%
\pgfpathlineto{\pgfqpoint{2.585070in}{2.516231in}}%
\pgfpathclose%
\pgfusepath{fill}%
\end{pgfscope}%
\begin{pgfscope}%
\pgfpathrectangle{\pgfqpoint{0.526905in}{0.383578in}}{\pgfqpoint{3.875000in}{2.310000in}}%
\pgfusepath{clip}%
\pgfsetbuttcap%
\pgfsetmiterjoin%
\definecolor{currentfill}{rgb}{0.686275,0.352941,0.313725}%
\pgfsetfillcolor{currentfill}%
\pgfsetfillopacity{0.300000}%
\pgfsetlinewidth{0.000000pt}%
\definecolor{currentstroke}{rgb}{0.000000,0.000000,0.000000}%
\pgfsetstrokecolor{currentstroke}%
\pgfsetstrokeopacity{0.300000}%
\pgfsetdash{}{0pt}%
\pgfpathmoveto{\pgfqpoint{2.602659in}{0.383578in}}%
\pgfpathlineto{\pgfqpoint{2.620248in}{0.383578in}}%
\pgfpathlineto{\pgfqpoint{2.620248in}{2.375925in}}%
\pgfpathlineto{\pgfqpoint{2.602659in}{2.375925in}}%
\pgfpathclose%
\pgfusepath{fill}%
\end{pgfscope}%
\begin{pgfscope}%
\pgfpathrectangle{\pgfqpoint{0.526905in}{0.383578in}}{\pgfqpoint{3.875000in}{2.310000in}}%
\pgfusepath{clip}%
\pgfsetbuttcap%
\pgfsetmiterjoin%
\definecolor{currentfill}{rgb}{0.686275,0.352941,0.313725}%
\pgfsetfillcolor{currentfill}%
\pgfsetfillopacity{0.300000}%
\pgfsetlinewidth{0.000000pt}%
\definecolor{currentstroke}{rgb}{0.000000,0.000000,0.000000}%
\pgfsetstrokecolor{currentstroke}%
\pgfsetstrokeopacity{0.300000}%
\pgfsetdash{}{0pt}%
\pgfpathmoveto{\pgfqpoint{2.620248in}{0.383578in}}%
\pgfpathlineto{\pgfqpoint{2.637837in}{0.383578in}}%
\pgfpathlineto{\pgfqpoint{2.637837in}{2.353476in}}%
\pgfpathlineto{\pgfqpoint{2.620248in}{2.353476in}}%
\pgfpathclose%
\pgfusepath{fill}%
\end{pgfscope}%
\begin{pgfscope}%
\pgfpathrectangle{\pgfqpoint{0.526905in}{0.383578in}}{\pgfqpoint{3.875000in}{2.310000in}}%
\pgfusepath{clip}%
\pgfsetbuttcap%
\pgfsetmiterjoin%
\definecolor{currentfill}{rgb}{0.686275,0.352941,0.313725}%
\pgfsetfillcolor{currentfill}%
\pgfsetfillopacity{0.300000}%
\pgfsetlinewidth{0.000000pt}%
\definecolor{currentstroke}{rgb}{0.000000,0.000000,0.000000}%
\pgfsetstrokecolor{currentstroke}%
\pgfsetstrokeopacity{0.300000}%
\pgfsetdash{}{0pt}%
\pgfpathmoveto{\pgfqpoint{2.637837in}{0.383578in}}%
\pgfpathlineto{\pgfqpoint{2.655426in}{0.383578in}}%
\pgfpathlineto{\pgfqpoint{2.655426in}{2.572354in}}%
\pgfpathlineto{\pgfqpoint{2.637837in}{2.572354in}}%
\pgfpathclose%
\pgfusepath{fill}%
\end{pgfscope}%
\begin{pgfscope}%
\pgfpathrectangle{\pgfqpoint{0.526905in}{0.383578in}}{\pgfqpoint{3.875000in}{2.310000in}}%
\pgfusepath{clip}%
\pgfsetbuttcap%
\pgfsetmiterjoin%
\definecolor{currentfill}{rgb}{0.686275,0.352941,0.313725}%
\pgfsetfillcolor{currentfill}%
\pgfsetfillopacity{0.300000}%
\pgfsetlinewidth{0.000000pt}%
\definecolor{currentstroke}{rgb}{0.000000,0.000000,0.000000}%
\pgfsetstrokecolor{currentstroke}%
\pgfsetstrokeopacity{0.300000}%
\pgfsetdash{}{0pt}%
\pgfpathmoveto{\pgfqpoint{2.655426in}{0.383578in}}%
\pgfpathlineto{\pgfqpoint{2.673015in}{0.383578in}}%
\pgfpathlineto{\pgfqpoint{2.673015in}{2.432047in}}%
\pgfpathlineto{\pgfqpoint{2.655426in}{2.432047in}}%
\pgfpathclose%
\pgfusepath{fill}%
\end{pgfscope}%
\begin{pgfscope}%
\pgfpathrectangle{\pgfqpoint{0.526905in}{0.383578in}}{\pgfqpoint{3.875000in}{2.310000in}}%
\pgfusepath{clip}%
\pgfsetbuttcap%
\pgfsetmiterjoin%
\definecolor{currentfill}{rgb}{0.686275,0.352941,0.313725}%
\pgfsetfillcolor{currentfill}%
\pgfsetfillopacity{0.300000}%
\pgfsetlinewidth{0.000000pt}%
\definecolor{currentstroke}{rgb}{0.000000,0.000000,0.000000}%
\pgfsetstrokecolor{currentstroke}%
\pgfsetstrokeopacity{0.300000}%
\pgfsetdash{}{0pt}%
\pgfpathmoveto{\pgfqpoint{2.673015in}{0.383578in}}%
\pgfpathlineto{\pgfqpoint{2.690604in}{0.383578in}}%
\pgfpathlineto{\pgfqpoint{2.690604in}{2.482558in}}%
\pgfpathlineto{\pgfqpoint{2.673015in}{2.482558in}}%
\pgfpathclose%
\pgfusepath{fill}%
\end{pgfscope}%
\begin{pgfscope}%
\pgfpathrectangle{\pgfqpoint{0.526905in}{0.383578in}}{\pgfqpoint{3.875000in}{2.310000in}}%
\pgfusepath{clip}%
\pgfsetbuttcap%
\pgfsetmiterjoin%
\definecolor{currentfill}{rgb}{0.686275,0.352941,0.313725}%
\pgfsetfillcolor{currentfill}%
\pgfsetfillopacity{0.300000}%
\pgfsetlinewidth{0.000000pt}%
\definecolor{currentstroke}{rgb}{0.000000,0.000000,0.000000}%
\pgfsetstrokecolor{currentstroke}%
\pgfsetstrokeopacity{0.300000}%
\pgfsetdash{}{0pt}%
\pgfpathmoveto{\pgfqpoint{2.690604in}{0.383578in}}%
\pgfpathlineto{\pgfqpoint{2.708193in}{0.383578in}}%
\pgfpathlineto{\pgfqpoint{2.708193in}{2.488170in}}%
\pgfpathlineto{\pgfqpoint{2.690604in}{2.488170in}}%
\pgfpathclose%
\pgfusepath{fill}%
\end{pgfscope}%
\begin{pgfscope}%
\pgfpathrectangle{\pgfqpoint{0.526905in}{0.383578in}}{\pgfqpoint{3.875000in}{2.310000in}}%
\pgfusepath{clip}%
\pgfsetbuttcap%
\pgfsetmiterjoin%
\definecolor{currentfill}{rgb}{0.686275,0.352941,0.313725}%
\pgfsetfillcolor{currentfill}%
\pgfsetfillopacity{0.300000}%
\pgfsetlinewidth{0.000000pt}%
\definecolor{currentstroke}{rgb}{0.000000,0.000000,0.000000}%
\pgfsetstrokecolor{currentstroke}%
\pgfsetstrokeopacity{0.300000}%
\pgfsetdash{}{0pt}%
\pgfpathmoveto{\pgfqpoint{2.708193in}{0.383578in}}%
\pgfpathlineto{\pgfqpoint{2.725782in}{0.383578in}}%
\pgfpathlineto{\pgfqpoint{2.725782in}{2.583578in}}%
\pgfpathlineto{\pgfqpoint{2.708193in}{2.583578in}}%
\pgfpathclose%
\pgfusepath{fill}%
\end{pgfscope}%
\begin{pgfscope}%
\pgfpathrectangle{\pgfqpoint{0.526905in}{0.383578in}}{\pgfqpoint{3.875000in}{2.310000in}}%
\pgfusepath{clip}%
\pgfsetbuttcap%
\pgfsetmiterjoin%
\definecolor{currentfill}{rgb}{0.686275,0.352941,0.313725}%
\pgfsetfillcolor{currentfill}%
\pgfsetfillopacity{0.300000}%
\pgfsetlinewidth{0.000000pt}%
\definecolor{currentstroke}{rgb}{0.000000,0.000000,0.000000}%
\pgfsetstrokecolor{currentstroke}%
\pgfsetstrokeopacity{0.300000}%
\pgfsetdash{}{0pt}%
\pgfpathmoveto{\pgfqpoint{2.725782in}{0.383578in}}%
\pgfpathlineto{\pgfqpoint{2.743371in}{0.383578in}}%
\pgfpathlineto{\pgfqpoint{2.743371in}{2.544292in}}%
\pgfpathlineto{\pgfqpoint{2.725782in}{2.544292in}}%
\pgfpathclose%
\pgfusepath{fill}%
\end{pgfscope}%
\begin{pgfscope}%
\pgfpathrectangle{\pgfqpoint{0.526905in}{0.383578in}}{\pgfqpoint{3.875000in}{2.310000in}}%
\pgfusepath{clip}%
\pgfsetbuttcap%
\pgfsetmiterjoin%
\definecolor{currentfill}{rgb}{0.686275,0.352941,0.313725}%
\pgfsetfillcolor{currentfill}%
\pgfsetfillopacity{0.300000}%
\pgfsetlinewidth{0.000000pt}%
\definecolor{currentstroke}{rgb}{0.000000,0.000000,0.000000}%
\pgfsetstrokecolor{currentstroke}%
\pgfsetstrokeopacity{0.300000}%
\pgfsetdash{}{0pt}%
\pgfpathmoveto{\pgfqpoint{2.743371in}{0.383578in}}%
\pgfpathlineto{\pgfqpoint{2.760960in}{0.383578in}}%
\pgfpathlineto{\pgfqpoint{2.760960in}{2.432047in}}%
\pgfpathlineto{\pgfqpoint{2.743371in}{2.432047in}}%
\pgfpathclose%
\pgfusepath{fill}%
\end{pgfscope}%
\begin{pgfscope}%
\pgfpathrectangle{\pgfqpoint{0.526905in}{0.383578in}}{\pgfqpoint{3.875000in}{2.310000in}}%
\pgfusepath{clip}%
\pgfsetbuttcap%
\pgfsetmiterjoin%
\definecolor{currentfill}{rgb}{0.686275,0.352941,0.313725}%
\pgfsetfillcolor{currentfill}%
\pgfsetfillopacity{0.300000}%
\pgfsetlinewidth{0.000000pt}%
\definecolor{currentstroke}{rgb}{0.000000,0.000000,0.000000}%
\pgfsetstrokecolor{currentstroke}%
\pgfsetstrokeopacity{0.300000}%
\pgfsetdash{}{0pt}%
\pgfpathmoveto{\pgfqpoint{2.760960in}{0.383578in}}%
\pgfpathlineto{\pgfqpoint{2.778549in}{0.383578in}}%
\pgfpathlineto{\pgfqpoint{2.778549in}{2.426435in}}%
\pgfpathlineto{\pgfqpoint{2.760960in}{2.426435in}}%
\pgfpathclose%
\pgfusepath{fill}%
\end{pgfscope}%
\begin{pgfscope}%
\pgfpathrectangle{\pgfqpoint{0.526905in}{0.383578in}}{\pgfqpoint{3.875000in}{2.310000in}}%
\pgfusepath{clip}%
\pgfsetbuttcap%
\pgfsetmiterjoin%
\definecolor{currentfill}{rgb}{0.686275,0.352941,0.313725}%
\pgfsetfillcolor{currentfill}%
\pgfsetfillopacity{0.300000}%
\pgfsetlinewidth{0.000000pt}%
\definecolor{currentstroke}{rgb}{0.000000,0.000000,0.000000}%
\pgfsetstrokecolor{currentstroke}%
\pgfsetstrokeopacity{0.300000}%
\pgfsetdash{}{0pt}%
\pgfpathmoveto{\pgfqpoint{2.778549in}{0.383578in}}%
\pgfpathlineto{\pgfqpoint{2.796138in}{0.383578in}}%
\pgfpathlineto{\pgfqpoint{2.796138in}{2.375925in}}%
\pgfpathlineto{\pgfqpoint{2.778549in}{2.375925in}}%
\pgfpathclose%
\pgfusepath{fill}%
\end{pgfscope}%
\begin{pgfscope}%
\pgfpathrectangle{\pgfqpoint{0.526905in}{0.383578in}}{\pgfqpoint{3.875000in}{2.310000in}}%
\pgfusepath{clip}%
\pgfsetbuttcap%
\pgfsetmiterjoin%
\definecolor{currentfill}{rgb}{0.686275,0.352941,0.313725}%
\pgfsetfillcolor{currentfill}%
\pgfsetfillopacity{0.300000}%
\pgfsetlinewidth{0.000000pt}%
\definecolor{currentstroke}{rgb}{0.000000,0.000000,0.000000}%
\pgfsetstrokecolor{currentstroke}%
\pgfsetstrokeopacity{0.300000}%
\pgfsetdash{}{0pt}%
\pgfpathmoveto{\pgfqpoint{2.796138in}{0.383578in}}%
\pgfpathlineto{\pgfqpoint{2.813728in}{0.383578in}}%
\pgfpathlineto{\pgfqpoint{2.813728in}{2.465721in}}%
\pgfpathlineto{\pgfqpoint{2.796138in}{2.465721in}}%
\pgfpathclose%
\pgfusepath{fill}%
\end{pgfscope}%
\begin{pgfscope}%
\pgfpathrectangle{\pgfqpoint{0.526905in}{0.383578in}}{\pgfqpoint{3.875000in}{2.310000in}}%
\pgfusepath{clip}%
\pgfsetbuttcap%
\pgfsetmiterjoin%
\definecolor{currentfill}{rgb}{0.686275,0.352941,0.313725}%
\pgfsetfillcolor{currentfill}%
\pgfsetfillopacity{0.300000}%
\pgfsetlinewidth{0.000000pt}%
\definecolor{currentstroke}{rgb}{0.000000,0.000000,0.000000}%
\pgfsetstrokecolor{currentstroke}%
\pgfsetstrokeopacity{0.300000}%
\pgfsetdash{}{0pt}%
\pgfpathmoveto{\pgfqpoint{2.813728in}{0.383578in}}%
\pgfpathlineto{\pgfqpoint{2.831317in}{0.383578in}}%
\pgfpathlineto{\pgfqpoint{2.831317in}{2.415211in}}%
\pgfpathlineto{\pgfqpoint{2.813728in}{2.415211in}}%
\pgfpathclose%
\pgfusepath{fill}%
\end{pgfscope}%
\begin{pgfscope}%
\pgfpathrectangle{\pgfqpoint{0.526905in}{0.383578in}}{\pgfqpoint{3.875000in}{2.310000in}}%
\pgfusepath{clip}%
\pgfsetbuttcap%
\pgfsetmiterjoin%
\definecolor{currentfill}{rgb}{0.686275,0.352941,0.313725}%
\pgfsetfillcolor{currentfill}%
\pgfsetfillopacity{0.300000}%
\pgfsetlinewidth{0.000000pt}%
\definecolor{currentstroke}{rgb}{0.000000,0.000000,0.000000}%
\pgfsetstrokecolor{currentstroke}%
\pgfsetstrokeopacity{0.300000}%
\pgfsetdash{}{0pt}%
\pgfpathmoveto{\pgfqpoint{2.831317in}{0.383578in}}%
\pgfpathlineto{\pgfqpoint{2.848906in}{0.383578in}}%
\pgfpathlineto{\pgfqpoint{2.848906in}{2.555517in}}%
\pgfpathlineto{\pgfqpoint{2.831317in}{2.555517in}}%
\pgfpathclose%
\pgfusepath{fill}%
\end{pgfscope}%
\begin{pgfscope}%
\pgfpathrectangle{\pgfqpoint{0.526905in}{0.383578in}}{\pgfqpoint{3.875000in}{2.310000in}}%
\pgfusepath{clip}%
\pgfsetbuttcap%
\pgfsetmiterjoin%
\definecolor{currentfill}{rgb}{0.686275,0.352941,0.313725}%
\pgfsetfillcolor{currentfill}%
\pgfsetfillopacity{0.300000}%
\pgfsetlinewidth{0.000000pt}%
\definecolor{currentstroke}{rgb}{0.000000,0.000000,0.000000}%
\pgfsetstrokecolor{currentstroke}%
\pgfsetstrokeopacity{0.300000}%
\pgfsetdash{}{0pt}%
\pgfpathmoveto{\pgfqpoint{2.848906in}{0.383578in}}%
\pgfpathlineto{\pgfqpoint{2.866495in}{0.383578in}}%
\pgfpathlineto{\pgfqpoint{2.866495in}{2.555517in}}%
\pgfpathlineto{\pgfqpoint{2.848906in}{2.555517in}}%
\pgfpathclose%
\pgfusepath{fill}%
\end{pgfscope}%
\begin{pgfscope}%
\pgfpathrectangle{\pgfqpoint{0.526905in}{0.383578in}}{\pgfqpoint{3.875000in}{2.310000in}}%
\pgfusepath{clip}%
\pgfsetbuttcap%
\pgfsetmiterjoin%
\definecolor{currentfill}{rgb}{0.686275,0.352941,0.313725}%
\pgfsetfillcolor{currentfill}%
\pgfsetfillopacity{0.300000}%
\pgfsetlinewidth{0.000000pt}%
\definecolor{currentstroke}{rgb}{0.000000,0.000000,0.000000}%
\pgfsetstrokecolor{currentstroke}%
\pgfsetstrokeopacity{0.300000}%
\pgfsetdash{}{0pt}%
\pgfpathmoveto{\pgfqpoint{2.866495in}{0.383578in}}%
\pgfpathlineto{\pgfqpoint{2.884084in}{0.383578in}}%
\pgfpathlineto{\pgfqpoint{2.884084in}{2.420823in}}%
\pgfpathlineto{\pgfqpoint{2.866495in}{2.420823in}}%
\pgfpathclose%
\pgfusepath{fill}%
\end{pgfscope}%
\begin{pgfscope}%
\pgfpathrectangle{\pgfqpoint{0.526905in}{0.383578in}}{\pgfqpoint{3.875000in}{2.310000in}}%
\pgfusepath{clip}%
\pgfsetbuttcap%
\pgfsetmiterjoin%
\definecolor{currentfill}{rgb}{0.686275,0.352941,0.313725}%
\pgfsetfillcolor{currentfill}%
\pgfsetfillopacity{0.300000}%
\pgfsetlinewidth{0.000000pt}%
\definecolor{currentstroke}{rgb}{0.000000,0.000000,0.000000}%
\pgfsetstrokecolor{currentstroke}%
\pgfsetstrokeopacity{0.300000}%
\pgfsetdash{}{0pt}%
\pgfpathmoveto{\pgfqpoint{2.884084in}{0.383578in}}%
\pgfpathlineto{\pgfqpoint{2.901673in}{0.383578in}}%
\pgfpathlineto{\pgfqpoint{2.901673in}{2.398374in}}%
\pgfpathlineto{\pgfqpoint{2.884084in}{2.398374in}}%
\pgfpathclose%
\pgfusepath{fill}%
\end{pgfscope}%
\begin{pgfscope}%
\pgfpathrectangle{\pgfqpoint{0.526905in}{0.383578in}}{\pgfqpoint{3.875000in}{2.310000in}}%
\pgfusepath{clip}%
\pgfsetbuttcap%
\pgfsetmiterjoin%
\definecolor{currentfill}{rgb}{0.686275,0.352941,0.313725}%
\pgfsetfillcolor{currentfill}%
\pgfsetfillopacity{0.300000}%
\pgfsetlinewidth{0.000000pt}%
\definecolor{currentstroke}{rgb}{0.000000,0.000000,0.000000}%
\pgfsetstrokecolor{currentstroke}%
\pgfsetstrokeopacity{0.300000}%
\pgfsetdash{}{0pt}%
\pgfpathmoveto{\pgfqpoint{2.901673in}{0.383578in}}%
\pgfpathlineto{\pgfqpoint{2.919262in}{0.383578in}}%
\pgfpathlineto{\pgfqpoint{2.919262in}{2.510619in}}%
\pgfpathlineto{\pgfqpoint{2.901673in}{2.510619in}}%
\pgfpathclose%
\pgfusepath{fill}%
\end{pgfscope}%
\begin{pgfscope}%
\pgfpathrectangle{\pgfqpoint{0.526905in}{0.383578in}}{\pgfqpoint{3.875000in}{2.310000in}}%
\pgfusepath{clip}%
\pgfsetbuttcap%
\pgfsetmiterjoin%
\definecolor{currentfill}{rgb}{0.686275,0.352941,0.313725}%
\pgfsetfillcolor{currentfill}%
\pgfsetfillopacity{0.300000}%
\pgfsetlinewidth{0.000000pt}%
\definecolor{currentstroke}{rgb}{0.000000,0.000000,0.000000}%
\pgfsetstrokecolor{currentstroke}%
\pgfsetstrokeopacity{0.300000}%
\pgfsetdash{}{0pt}%
\pgfpathmoveto{\pgfqpoint{2.919262in}{0.383578in}}%
\pgfpathlineto{\pgfqpoint{2.936851in}{0.383578in}}%
\pgfpathlineto{\pgfqpoint{2.936851in}{2.347864in}}%
\pgfpathlineto{\pgfqpoint{2.919262in}{2.347864in}}%
\pgfpathclose%
\pgfusepath{fill}%
\end{pgfscope}%
\begin{pgfscope}%
\pgfpathrectangle{\pgfqpoint{0.526905in}{0.383578in}}{\pgfqpoint{3.875000in}{2.310000in}}%
\pgfusepath{clip}%
\pgfsetbuttcap%
\pgfsetmiterjoin%
\definecolor{currentfill}{rgb}{0.686275,0.352941,0.313725}%
\pgfsetfillcolor{currentfill}%
\pgfsetfillopacity{0.300000}%
\pgfsetlinewidth{0.000000pt}%
\definecolor{currentstroke}{rgb}{0.000000,0.000000,0.000000}%
\pgfsetstrokecolor{currentstroke}%
\pgfsetstrokeopacity{0.300000}%
\pgfsetdash{}{0pt}%
\pgfpathmoveto{\pgfqpoint{2.936851in}{0.383578in}}%
\pgfpathlineto{\pgfqpoint{2.954440in}{0.383578in}}%
\pgfpathlineto{\pgfqpoint{2.954440in}{2.398374in}}%
\pgfpathlineto{\pgfqpoint{2.936851in}{2.398374in}}%
\pgfpathclose%
\pgfusepath{fill}%
\end{pgfscope}%
\begin{pgfscope}%
\pgfpathrectangle{\pgfqpoint{0.526905in}{0.383578in}}{\pgfqpoint{3.875000in}{2.310000in}}%
\pgfusepath{clip}%
\pgfsetbuttcap%
\pgfsetmiterjoin%
\definecolor{currentfill}{rgb}{0.686275,0.352941,0.313725}%
\pgfsetfillcolor{currentfill}%
\pgfsetfillopacity{0.300000}%
\pgfsetlinewidth{0.000000pt}%
\definecolor{currentstroke}{rgb}{0.000000,0.000000,0.000000}%
\pgfsetstrokecolor{currentstroke}%
\pgfsetstrokeopacity{0.300000}%
\pgfsetdash{}{0pt}%
\pgfpathmoveto{\pgfqpoint{2.954440in}{0.383578in}}%
\pgfpathlineto{\pgfqpoint{2.972029in}{0.383578in}}%
\pgfpathlineto{\pgfqpoint{2.972029in}{2.465721in}}%
\pgfpathlineto{\pgfqpoint{2.954440in}{2.465721in}}%
\pgfpathclose%
\pgfusepath{fill}%
\end{pgfscope}%
\begin{pgfscope}%
\pgfpathrectangle{\pgfqpoint{0.526905in}{0.383578in}}{\pgfqpoint{3.875000in}{2.310000in}}%
\pgfusepath{clip}%
\pgfsetbuttcap%
\pgfsetmiterjoin%
\definecolor{currentfill}{rgb}{0.686275,0.352941,0.313725}%
\pgfsetfillcolor{currentfill}%
\pgfsetfillopacity{0.300000}%
\pgfsetlinewidth{0.000000pt}%
\definecolor{currentstroke}{rgb}{0.000000,0.000000,0.000000}%
\pgfsetstrokecolor{currentstroke}%
\pgfsetstrokeopacity{0.300000}%
\pgfsetdash{}{0pt}%
\pgfpathmoveto{\pgfqpoint{2.972029in}{0.383578in}}%
\pgfpathlineto{\pgfqpoint{2.989618in}{0.383578in}}%
\pgfpathlineto{\pgfqpoint{2.989618in}{2.218782in}}%
\pgfpathlineto{\pgfqpoint{2.972029in}{2.218782in}}%
\pgfpathclose%
\pgfusepath{fill}%
\end{pgfscope}%
\begin{pgfscope}%
\pgfpathrectangle{\pgfqpoint{0.526905in}{0.383578in}}{\pgfqpoint{3.875000in}{2.310000in}}%
\pgfusepath{clip}%
\pgfsetbuttcap%
\pgfsetmiterjoin%
\definecolor{currentfill}{rgb}{0.686275,0.352941,0.313725}%
\pgfsetfillcolor{currentfill}%
\pgfsetfillopacity{0.300000}%
\pgfsetlinewidth{0.000000pt}%
\definecolor{currentstroke}{rgb}{0.000000,0.000000,0.000000}%
\pgfsetstrokecolor{currentstroke}%
\pgfsetstrokeopacity{0.300000}%
\pgfsetdash{}{0pt}%
\pgfpathmoveto{\pgfqpoint{2.989618in}{0.383578in}}%
\pgfpathlineto{\pgfqpoint{3.007207in}{0.383578in}}%
\pgfpathlineto{\pgfqpoint{3.007207in}{2.201945in}}%
\pgfpathlineto{\pgfqpoint{2.989618in}{2.201945in}}%
\pgfpathclose%
\pgfusepath{fill}%
\end{pgfscope}%
\begin{pgfscope}%
\pgfpathrectangle{\pgfqpoint{0.526905in}{0.383578in}}{\pgfqpoint{3.875000in}{2.310000in}}%
\pgfusepath{clip}%
\pgfsetbuttcap%
\pgfsetmiterjoin%
\definecolor{currentfill}{rgb}{0.686275,0.352941,0.313725}%
\pgfsetfillcolor{currentfill}%
\pgfsetfillopacity{0.300000}%
\pgfsetlinewidth{0.000000pt}%
\definecolor{currentstroke}{rgb}{0.000000,0.000000,0.000000}%
\pgfsetstrokecolor{currentstroke}%
\pgfsetstrokeopacity{0.300000}%
\pgfsetdash{}{0pt}%
\pgfpathmoveto{\pgfqpoint{3.007207in}{0.383578in}}%
\pgfpathlineto{\pgfqpoint{3.024796in}{0.383578in}}%
\pgfpathlineto{\pgfqpoint{3.024796in}{2.426435in}}%
\pgfpathlineto{\pgfqpoint{3.007207in}{2.426435in}}%
\pgfpathclose%
\pgfusepath{fill}%
\end{pgfscope}%
\begin{pgfscope}%
\pgfpathrectangle{\pgfqpoint{0.526905in}{0.383578in}}{\pgfqpoint{3.875000in}{2.310000in}}%
\pgfusepath{clip}%
\pgfsetbuttcap%
\pgfsetmiterjoin%
\definecolor{currentfill}{rgb}{0.686275,0.352941,0.313725}%
\pgfsetfillcolor{currentfill}%
\pgfsetfillopacity{0.300000}%
\pgfsetlinewidth{0.000000pt}%
\definecolor{currentstroke}{rgb}{0.000000,0.000000,0.000000}%
\pgfsetstrokecolor{currentstroke}%
\pgfsetstrokeopacity{0.300000}%
\pgfsetdash{}{0pt}%
\pgfpathmoveto{\pgfqpoint{3.024796in}{0.383578in}}%
\pgfpathlineto{\pgfqpoint{3.042385in}{0.383578in}}%
\pgfpathlineto{\pgfqpoint{3.042385in}{2.235619in}}%
\pgfpathlineto{\pgfqpoint{3.024796in}{2.235619in}}%
\pgfpathclose%
\pgfusepath{fill}%
\end{pgfscope}%
\begin{pgfscope}%
\pgfpathrectangle{\pgfqpoint{0.526905in}{0.383578in}}{\pgfqpoint{3.875000in}{2.310000in}}%
\pgfusepath{clip}%
\pgfsetbuttcap%
\pgfsetmiterjoin%
\definecolor{currentfill}{rgb}{0.686275,0.352941,0.313725}%
\pgfsetfillcolor{currentfill}%
\pgfsetfillopacity{0.300000}%
\pgfsetlinewidth{0.000000pt}%
\definecolor{currentstroke}{rgb}{0.000000,0.000000,0.000000}%
\pgfsetstrokecolor{currentstroke}%
\pgfsetstrokeopacity{0.300000}%
\pgfsetdash{}{0pt}%
\pgfpathmoveto{\pgfqpoint{3.042385in}{0.383578in}}%
\pgfpathlineto{\pgfqpoint{3.059974in}{0.383578in}}%
\pgfpathlineto{\pgfqpoint{3.059974in}{1.983068in}}%
\pgfpathlineto{\pgfqpoint{3.042385in}{1.983068in}}%
\pgfpathclose%
\pgfusepath{fill}%
\end{pgfscope}%
\begin{pgfscope}%
\pgfpathrectangle{\pgfqpoint{0.526905in}{0.383578in}}{\pgfqpoint{3.875000in}{2.310000in}}%
\pgfusepath{clip}%
\pgfsetbuttcap%
\pgfsetmiterjoin%
\definecolor{currentfill}{rgb}{0.686275,0.352941,0.313725}%
\pgfsetfillcolor{currentfill}%
\pgfsetfillopacity{0.300000}%
\pgfsetlinewidth{0.000000pt}%
\definecolor{currentstroke}{rgb}{0.000000,0.000000,0.000000}%
\pgfsetstrokecolor{currentstroke}%
\pgfsetstrokeopacity{0.300000}%
\pgfsetdash{}{0pt}%
\pgfpathmoveto{\pgfqpoint{3.059974in}{0.383578in}}%
\pgfpathlineto{\pgfqpoint{3.077563in}{0.383578in}}%
\pgfpathlineto{\pgfqpoint{3.077563in}{1.994292in}}%
\pgfpathlineto{\pgfqpoint{3.059974in}{1.994292in}}%
\pgfpathclose%
\pgfusepath{fill}%
\end{pgfscope}%
\begin{pgfscope}%
\pgfpathrectangle{\pgfqpoint{0.526905in}{0.383578in}}{\pgfqpoint{3.875000in}{2.310000in}}%
\pgfusepath{clip}%
\pgfsetbuttcap%
\pgfsetmiterjoin%
\definecolor{currentfill}{rgb}{0.686275,0.352941,0.313725}%
\pgfsetfillcolor{currentfill}%
\pgfsetfillopacity{0.300000}%
\pgfsetlinewidth{0.000000pt}%
\definecolor{currentstroke}{rgb}{0.000000,0.000000,0.000000}%
\pgfsetstrokecolor{currentstroke}%
\pgfsetstrokeopacity{0.300000}%
\pgfsetdash{}{0pt}%
\pgfpathmoveto{\pgfqpoint{3.077563in}{0.383578in}}%
\pgfpathlineto{\pgfqpoint{3.095152in}{0.383578in}}%
\pgfpathlineto{\pgfqpoint{3.095152in}{1.960619in}}%
\pgfpathlineto{\pgfqpoint{3.077563in}{1.960619in}}%
\pgfpathclose%
\pgfusepath{fill}%
\end{pgfscope}%
\begin{pgfscope}%
\pgfpathrectangle{\pgfqpoint{0.526905in}{0.383578in}}{\pgfqpoint{3.875000in}{2.310000in}}%
\pgfusepath{clip}%
\pgfsetbuttcap%
\pgfsetmiterjoin%
\definecolor{currentfill}{rgb}{0.686275,0.352941,0.313725}%
\pgfsetfillcolor{currentfill}%
\pgfsetfillopacity{0.300000}%
\pgfsetlinewidth{0.000000pt}%
\definecolor{currentstroke}{rgb}{0.000000,0.000000,0.000000}%
\pgfsetstrokecolor{currentstroke}%
\pgfsetstrokeopacity{0.300000}%
\pgfsetdash{}{0pt}%
\pgfpathmoveto{\pgfqpoint{3.095152in}{0.383578in}}%
\pgfpathlineto{\pgfqpoint{3.112741in}{0.383578in}}%
\pgfpathlineto{\pgfqpoint{3.112741in}{1.932558in}}%
\pgfpathlineto{\pgfqpoint{3.095152in}{1.932558in}}%
\pgfpathclose%
\pgfusepath{fill}%
\end{pgfscope}%
\begin{pgfscope}%
\pgfpathrectangle{\pgfqpoint{0.526905in}{0.383578in}}{\pgfqpoint{3.875000in}{2.310000in}}%
\pgfusepath{clip}%
\pgfsetbuttcap%
\pgfsetmiterjoin%
\definecolor{currentfill}{rgb}{0.686275,0.352941,0.313725}%
\pgfsetfillcolor{currentfill}%
\pgfsetfillopacity{0.300000}%
\pgfsetlinewidth{0.000000pt}%
\definecolor{currentstroke}{rgb}{0.000000,0.000000,0.000000}%
\pgfsetstrokecolor{currentstroke}%
\pgfsetstrokeopacity{0.300000}%
\pgfsetdash{}{0pt}%
\pgfpathmoveto{\pgfqpoint{3.112741in}{0.383578in}}%
\pgfpathlineto{\pgfqpoint{3.130330in}{0.383578in}}%
\pgfpathlineto{\pgfqpoint{3.130330in}{1.926945in}}%
\pgfpathlineto{\pgfqpoint{3.112741in}{1.926945in}}%
\pgfpathclose%
\pgfusepath{fill}%
\end{pgfscope}%
\begin{pgfscope}%
\pgfpathrectangle{\pgfqpoint{0.526905in}{0.383578in}}{\pgfqpoint{3.875000in}{2.310000in}}%
\pgfusepath{clip}%
\pgfsetbuttcap%
\pgfsetmiterjoin%
\definecolor{currentfill}{rgb}{0.686275,0.352941,0.313725}%
\pgfsetfillcolor{currentfill}%
\pgfsetfillopacity{0.300000}%
\pgfsetlinewidth{0.000000pt}%
\definecolor{currentstroke}{rgb}{0.000000,0.000000,0.000000}%
\pgfsetstrokecolor{currentstroke}%
\pgfsetstrokeopacity{0.300000}%
\pgfsetdash{}{0pt}%
\pgfpathmoveto{\pgfqpoint{3.130330in}{0.383578in}}%
\pgfpathlineto{\pgfqpoint{3.147920in}{0.383578in}}%
\pgfpathlineto{\pgfqpoint{3.147920in}{1.971843in}}%
\pgfpathlineto{\pgfqpoint{3.130330in}{1.971843in}}%
\pgfpathclose%
\pgfusepath{fill}%
\end{pgfscope}%
\begin{pgfscope}%
\pgfpathrectangle{\pgfqpoint{0.526905in}{0.383578in}}{\pgfqpoint{3.875000in}{2.310000in}}%
\pgfusepath{clip}%
\pgfsetbuttcap%
\pgfsetmiterjoin%
\definecolor{currentfill}{rgb}{0.686275,0.352941,0.313725}%
\pgfsetfillcolor{currentfill}%
\pgfsetfillopacity{0.300000}%
\pgfsetlinewidth{0.000000pt}%
\definecolor{currentstroke}{rgb}{0.000000,0.000000,0.000000}%
\pgfsetstrokecolor{currentstroke}%
\pgfsetstrokeopacity{0.300000}%
\pgfsetdash{}{0pt}%
\pgfpathmoveto{\pgfqpoint{3.147920in}{0.383578in}}%
\pgfpathlineto{\pgfqpoint{3.165509in}{0.383578in}}%
\pgfpathlineto{\pgfqpoint{3.165509in}{1.769803in}}%
\pgfpathlineto{\pgfqpoint{3.147920in}{1.769803in}}%
\pgfpathclose%
\pgfusepath{fill}%
\end{pgfscope}%
\begin{pgfscope}%
\pgfpathrectangle{\pgfqpoint{0.526905in}{0.383578in}}{\pgfqpoint{3.875000in}{2.310000in}}%
\pgfusepath{clip}%
\pgfsetbuttcap%
\pgfsetmiterjoin%
\definecolor{currentfill}{rgb}{0.686275,0.352941,0.313725}%
\pgfsetfillcolor{currentfill}%
\pgfsetfillopacity{0.300000}%
\pgfsetlinewidth{0.000000pt}%
\definecolor{currentstroke}{rgb}{0.000000,0.000000,0.000000}%
\pgfsetstrokecolor{currentstroke}%
\pgfsetstrokeopacity{0.300000}%
\pgfsetdash{}{0pt}%
\pgfpathmoveto{\pgfqpoint{3.165509in}{0.383578in}}%
\pgfpathlineto{\pgfqpoint{3.183098in}{0.383578in}}%
\pgfpathlineto{\pgfqpoint{3.183098in}{1.719292in}}%
\pgfpathlineto{\pgfqpoint{3.165509in}{1.719292in}}%
\pgfpathclose%
\pgfusepath{fill}%
\end{pgfscope}%
\begin{pgfscope}%
\pgfpathrectangle{\pgfqpoint{0.526905in}{0.383578in}}{\pgfqpoint{3.875000in}{2.310000in}}%
\pgfusepath{clip}%
\pgfsetbuttcap%
\pgfsetmiterjoin%
\definecolor{currentfill}{rgb}{0.686275,0.352941,0.313725}%
\pgfsetfillcolor{currentfill}%
\pgfsetfillopacity{0.300000}%
\pgfsetlinewidth{0.000000pt}%
\definecolor{currentstroke}{rgb}{0.000000,0.000000,0.000000}%
\pgfsetstrokecolor{currentstroke}%
\pgfsetstrokeopacity{0.300000}%
\pgfsetdash{}{0pt}%
\pgfpathmoveto{\pgfqpoint{3.183098in}{0.383578in}}%
\pgfpathlineto{\pgfqpoint{3.200687in}{0.383578in}}%
\pgfpathlineto{\pgfqpoint{3.200687in}{1.623884in}}%
\pgfpathlineto{\pgfqpoint{3.183098in}{1.623884in}}%
\pgfpathclose%
\pgfusepath{fill}%
\end{pgfscope}%
\begin{pgfscope}%
\pgfpathrectangle{\pgfqpoint{0.526905in}{0.383578in}}{\pgfqpoint{3.875000in}{2.310000in}}%
\pgfusepath{clip}%
\pgfsetbuttcap%
\pgfsetmiterjoin%
\definecolor{currentfill}{rgb}{0.686275,0.352941,0.313725}%
\pgfsetfillcolor{currentfill}%
\pgfsetfillopacity{0.300000}%
\pgfsetlinewidth{0.000000pt}%
\definecolor{currentstroke}{rgb}{0.000000,0.000000,0.000000}%
\pgfsetstrokecolor{currentstroke}%
\pgfsetstrokeopacity{0.300000}%
\pgfsetdash{}{0pt}%
\pgfpathmoveto{\pgfqpoint{3.200687in}{0.383578in}}%
\pgfpathlineto{\pgfqpoint{3.218276in}{0.383578in}}%
\pgfpathlineto{\pgfqpoint{3.218276in}{1.494803in}}%
\pgfpathlineto{\pgfqpoint{3.200687in}{1.494803in}}%
\pgfpathclose%
\pgfusepath{fill}%
\end{pgfscope}%
\begin{pgfscope}%
\pgfpathrectangle{\pgfqpoint{0.526905in}{0.383578in}}{\pgfqpoint{3.875000in}{2.310000in}}%
\pgfusepath{clip}%
\pgfsetbuttcap%
\pgfsetmiterjoin%
\definecolor{currentfill}{rgb}{0.686275,0.352941,0.313725}%
\pgfsetfillcolor{currentfill}%
\pgfsetfillopacity{0.300000}%
\pgfsetlinewidth{0.000000pt}%
\definecolor{currentstroke}{rgb}{0.000000,0.000000,0.000000}%
\pgfsetstrokecolor{currentstroke}%
\pgfsetstrokeopacity{0.300000}%
\pgfsetdash{}{0pt}%
\pgfpathmoveto{\pgfqpoint{3.218276in}{0.383578in}}%
\pgfpathlineto{\pgfqpoint{3.235865in}{0.383578in}}%
\pgfpathlineto{\pgfqpoint{3.235865in}{1.657558in}}%
\pgfpathlineto{\pgfqpoint{3.218276in}{1.657558in}}%
\pgfpathclose%
\pgfusepath{fill}%
\end{pgfscope}%
\begin{pgfscope}%
\pgfpathrectangle{\pgfqpoint{0.526905in}{0.383578in}}{\pgfqpoint{3.875000in}{2.310000in}}%
\pgfusepath{clip}%
\pgfsetbuttcap%
\pgfsetmiterjoin%
\definecolor{currentfill}{rgb}{0.686275,0.352941,0.313725}%
\pgfsetfillcolor{currentfill}%
\pgfsetfillopacity{0.300000}%
\pgfsetlinewidth{0.000000pt}%
\definecolor{currentstroke}{rgb}{0.000000,0.000000,0.000000}%
\pgfsetstrokecolor{currentstroke}%
\pgfsetstrokeopacity{0.300000}%
\pgfsetdash{}{0pt}%
\pgfpathmoveto{\pgfqpoint{3.235865in}{0.383578in}}%
\pgfpathlineto{\pgfqpoint{3.253454in}{0.383578in}}%
\pgfpathlineto{\pgfqpoint{3.253454in}{1.477966in}}%
\pgfpathlineto{\pgfqpoint{3.235865in}{1.477966in}}%
\pgfpathclose%
\pgfusepath{fill}%
\end{pgfscope}%
\begin{pgfscope}%
\pgfpathrectangle{\pgfqpoint{0.526905in}{0.383578in}}{\pgfqpoint{3.875000in}{2.310000in}}%
\pgfusepath{clip}%
\pgfsetbuttcap%
\pgfsetmiterjoin%
\definecolor{currentfill}{rgb}{0.686275,0.352941,0.313725}%
\pgfsetfillcolor{currentfill}%
\pgfsetfillopacity{0.300000}%
\pgfsetlinewidth{0.000000pt}%
\definecolor{currentstroke}{rgb}{0.000000,0.000000,0.000000}%
\pgfsetstrokecolor{currentstroke}%
\pgfsetstrokeopacity{0.300000}%
\pgfsetdash{}{0pt}%
\pgfpathmoveto{\pgfqpoint{3.253454in}{0.383578in}}%
\pgfpathlineto{\pgfqpoint{3.271043in}{0.383578in}}%
\pgfpathlineto{\pgfqpoint{3.271043in}{1.483578in}}%
\pgfpathlineto{\pgfqpoint{3.253454in}{1.483578in}}%
\pgfpathclose%
\pgfusepath{fill}%
\end{pgfscope}%
\begin{pgfscope}%
\pgfpathrectangle{\pgfqpoint{0.526905in}{0.383578in}}{\pgfqpoint{3.875000in}{2.310000in}}%
\pgfusepath{clip}%
\pgfsetbuttcap%
\pgfsetmiterjoin%
\definecolor{currentfill}{rgb}{0.686275,0.352941,0.313725}%
\pgfsetfillcolor{currentfill}%
\pgfsetfillopacity{0.300000}%
\pgfsetlinewidth{0.000000pt}%
\definecolor{currentstroke}{rgb}{0.000000,0.000000,0.000000}%
\pgfsetstrokecolor{currentstroke}%
\pgfsetstrokeopacity{0.300000}%
\pgfsetdash{}{0pt}%
\pgfpathmoveto{\pgfqpoint{3.271043in}{0.383578in}}%
\pgfpathlineto{\pgfqpoint{3.288632in}{0.383578in}}%
\pgfpathlineto{\pgfqpoint{3.288632in}{1.360109in}}%
\pgfpathlineto{\pgfqpoint{3.271043in}{1.360109in}}%
\pgfpathclose%
\pgfusepath{fill}%
\end{pgfscope}%
\begin{pgfscope}%
\pgfpathrectangle{\pgfqpoint{0.526905in}{0.383578in}}{\pgfqpoint{3.875000in}{2.310000in}}%
\pgfusepath{clip}%
\pgfsetbuttcap%
\pgfsetmiterjoin%
\definecolor{currentfill}{rgb}{0.686275,0.352941,0.313725}%
\pgfsetfillcolor{currentfill}%
\pgfsetfillopacity{0.300000}%
\pgfsetlinewidth{0.000000pt}%
\definecolor{currentstroke}{rgb}{0.000000,0.000000,0.000000}%
\pgfsetstrokecolor{currentstroke}%
\pgfsetstrokeopacity{0.300000}%
\pgfsetdash{}{0pt}%
\pgfpathmoveto{\pgfqpoint{3.288632in}{0.383578in}}%
\pgfpathlineto{\pgfqpoint{3.306221in}{0.383578in}}%
\pgfpathlineto{\pgfqpoint{3.306221in}{1.225415in}}%
\pgfpathlineto{\pgfqpoint{3.288632in}{1.225415in}}%
\pgfpathclose%
\pgfusepath{fill}%
\end{pgfscope}%
\begin{pgfscope}%
\pgfpathrectangle{\pgfqpoint{0.526905in}{0.383578in}}{\pgfqpoint{3.875000in}{2.310000in}}%
\pgfusepath{clip}%
\pgfsetbuttcap%
\pgfsetmiterjoin%
\definecolor{currentfill}{rgb}{0.686275,0.352941,0.313725}%
\pgfsetfillcolor{currentfill}%
\pgfsetfillopacity{0.300000}%
\pgfsetlinewidth{0.000000pt}%
\definecolor{currentstroke}{rgb}{0.000000,0.000000,0.000000}%
\pgfsetstrokecolor{currentstroke}%
\pgfsetstrokeopacity{0.300000}%
\pgfsetdash{}{0pt}%
\pgfpathmoveto{\pgfqpoint{3.306221in}{0.383578in}}%
\pgfpathlineto{\pgfqpoint{3.323810in}{0.383578in}}%
\pgfpathlineto{\pgfqpoint{3.323810in}{1.174905in}}%
\pgfpathlineto{\pgfqpoint{3.306221in}{1.174905in}}%
\pgfpathclose%
\pgfusepath{fill}%
\end{pgfscope}%
\begin{pgfscope}%
\pgfpathrectangle{\pgfqpoint{0.526905in}{0.383578in}}{\pgfqpoint{3.875000in}{2.310000in}}%
\pgfusepath{clip}%
\pgfsetbuttcap%
\pgfsetmiterjoin%
\definecolor{currentfill}{rgb}{0.686275,0.352941,0.313725}%
\pgfsetfillcolor{currentfill}%
\pgfsetfillopacity{0.300000}%
\pgfsetlinewidth{0.000000pt}%
\definecolor{currentstroke}{rgb}{0.000000,0.000000,0.000000}%
\pgfsetstrokecolor{currentstroke}%
\pgfsetstrokeopacity{0.300000}%
\pgfsetdash{}{0pt}%
\pgfpathmoveto{\pgfqpoint{3.323810in}{0.383578in}}%
\pgfpathlineto{\pgfqpoint{3.341399in}{0.383578in}}%
\pgfpathlineto{\pgfqpoint{3.341399in}{1.107558in}}%
\pgfpathlineto{\pgfqpoint{3.323810in}{1.107558in}}%
\pgfpathclose%
\pgfusepath{fill}%
\end{pgfscope}%
\begin{pgfscope}%
\pgfpathrectangle{\pgfqpoint{0.526905in}{0.383578in}}{\pgfqpoint{3.875000in}{2.310000in}}%
\pgfusepath{clip}%
\pgfsetbuttcap%
\pgfsetmiterjoin%
\definecolor{currentfill}{rgb}{0.686275,0.352941,0.313725}%
\pgfsetfillcolor{currentfill}%
\pgfsetfillopacity{0.300000}%
\pgfsetlinewidth{0.000000pt}%
\definecolor{currentstroke}{rgb}{0.000000,0.000000,0.000000}%
\pgfsetstrokecolor{currentstroke}%
\pgfsetstrokeopacity{0.300000}%
\pgfsetdash{}{0pt}%
\pgfpathmoveto{\pgfqpoint{3.341399in}{0.383578in}}%
\pgfpathlineto{\pgfqpoint{3.358988in}{0.383578in}}%
\pgfpathlineto{\pgfqpoint{3.358988in}{0.995313in}}%
\pgfpathlineto{\pgfqpoint{3.341399in}{0.995313in}}%
\pgfpathclose%
\pgfusepath{fill}%
\end{pgfscope}%
\begin{pgfscope}%
\pgfpathrectangle{\pgfqpoint{0.526905in}{0.383578in}}{\pgfqpoint{3.875000in}{2.310000in}}%
\pgfusepath{clip}%
\pgfsetbuttcap%
\pgfsetmiterjoin%
\definecolor{currentfill}{rgb}{0.686275,0.352941,0.313725}%
\pgfsetfillcolor{currentfill}%
\pgfsetfillopacity{0.300000}%
\pgfsetlinewidth{0.000000pt}%
\definecolor{currentstroke}{rgb}{0.000000,0.000000,0.000000}%
\pgfsetstrokecolor{currentstroke}%
\pgfsetstrokeopacity{0.300000}%
\pgfsetdash{}{0pt}%
\pgfpathmoveto{\pgfqpoint{3.358988in}{0.383578in}}%
\pgfpathlineto{\pgfqpoint{3.376577in}{0.383578in}}%
\pgfpathlineto{\pgfqpoint{3.376577in}{1.090721in}}%
\pgfpathlineto{\pgfqpoint{3.358988in}{1.090721in}}%
\pgfpathclose%
\pgfusepath{fill}%
\end{pgfscope}%
\begin{pgfscope}%
\pgfpathrectangle{\pgfqpoint{0.526905in}{0.383578in}}{\pgfqpoint{3.875000in}{2.310000in}}%
\pgfusepath{clip}%
\pgfsetbuttcap%
\pgfsetmiterjoin%
\definecolor{currentfill}{rgb}{0.686275,0.352941,0.313725}%
\pgfsetfillcolor{currentfill}%
\pgfsetfillopacity{0.300000}%
\pgfsetlinewidth{0.000000pt}%
\definecolor{currentstroke}{rgb}{0.000000,0.000000,0.000000}%
\pgfsetstrokecolor{currentstroke}%
\pgfsetstrokeopacity{0.300000}%
\pgfsetdash{}{0pt}%
\pgfpathmoveto{\pgfqpoint{3.376577in}{0.383578in}}%
\pgfpathlineto{\pgfqpoint{3.394166in}{0.383578in}}%
\pgfpathlineto{\pgfqpoint{3.394166in}{1.045823in}}%
\pgfpathlineto{\pgfqpoint{3.376577in}{1.045823in}}%
\pgfpathclose%
\pgfusepath{fill}%
\end{pgfscope}%
\begin{pgfscope}%
\pgfpathrectangle{\pgfqpoint{0.526905in}{0.383578in}}{\pgfqpoint{3.875000in}{2.310000in}}%
\pgfusepath{clip}%
\pgfsetbuttcap%
\pgfsetmiterjoin%
\definecolor{currentfill}{rgb}{0.686275,0.352941,0.313725}%
\pgfsetfillcolor{currentfill}%
\pgfsetfillopacity{0.300000}%
\pgfsetlinewidth{0.000000pt}%
\definecolor{currentstroke}{rgb}{0.000000,0.000000,0.000000}%
\pgfsetstrokecolor{currentstroke}%
\pgfsetstrokeopacity{0.300000}%
\pgfsetdash{}{0pt}%
\pgfpathmoveto{\pgfqpoint{3.394166in}{0.383578in}}%
\pgfpathlineto{\pgfqpoint{3.411755in}{0.383578in}}%
\pgfpathlineto{\pgfqpoint{3.411755in}{1.017762in}}%
\pgfpathlineto{\pgfqpoint{3.394166in}{1.017762in}}%
\pgfpathclose%
\pgfusepath{fill}%
\end{pgfscope}%
\begin{pgfscope}%
\pgfpathrectangle{\pgfqpoint{0.526905in}{0.383578in}}{\pgfqpoint{3.875000in}{2.310000in}}%
\pgfusepath{clip}%
\pgfsetbuttcap%
\pgfsetmiterjoin%
\definecolor{currentfill}{rgb}{0.686275,0.352941,0.313725}%
\pgfsetfillcolor{currentfill}%
\pgfsetfillopacity{0.300000}%
\pgfsetlinewidth{0.000000pt}%
\definecolor{currentstroke}{rgb}{0.000000,0.000000,0.000000}%
\pgfsetstrokecolor{currentstroke}%
\pgfsetstrokeopacity{0.300000}%
\pgfsetdash{}{0pt}%
\pgfpathmoveto{\pgfqpoint{3.411755in}{0.383578in}}%
\pgfpathlineto{\pgfqpoint{3.429344in}{0.383578in}}%
\pgfpathlineto{\pgfqpoint{3.429344in}{1.113170in}}%
\pgfpathlineto{\pgfqpoint{3.411755in}{1.113170in}}%
\pgfpathclose%
\pgfusepath{fill}%
\end{pgfscope}%
\begin{pgfscope}%
\pgfpathrectangle{\pgfqpoint{0.526905in}{0.383578in}}{\pgfqpoint{3.875000in}{2.310000in}}%
\pgfusepath{clip}%
\pgfsetbuttcap%
\pgfsetmiterjoin%
\definecolor{currentfill}{rgb}{0.686275,0.352941,0.313725}%
\pgfsetfillcolor{currentfill}%
\pgfsetfillopacity{0.300000}%
\pgfsetlinewidth{0.000000pt}%
\definecolor{currentstroke}{rgb}{0.000000,0.000000,0.000000}%
\pgfsetstrokecolor{currentstroke}%
\pgfsetstrokeopacity{0.300000}%
\pgfsetdash{}{0pt}%
\pgfpathmoveto{\pgfqpoint{3.429344in}{0.383578in}}%
\pgfpathlineto{\pgfqpoint{3.446933in}{0.383578in}}%
\pgfpathlineto{\pgfqpoint{3.446933in}{1.040211in}}%
\pgfpathlineto{\pgfqpoint{3.429344in}{1.040211in}}%
\pgfpathclose%
\pgfusepath{fill}%
\end{pgfscope}%
\begin{pgfscope}%
\pgfpathrectangle{\pgfqpoint{0.526905in}{0.383578in}}{\pgfqpoint{3.875000in}{2.310000in}}%
\pgfusepath{clip}%
\pgfsetbuttcap%
\pgfsetmiterjoin%
\definecolor{currentfill}{rgb}{0.686275,0.352941,0.313725}%
\pgfsetfillcolor{currentfill}%
\pgfsetfillopacity{0.300000}%
\pgfsetlinewidth{0.000000pt}%
\definecolor{currentstroke}{rgb}{0.000000,0.000000,0.000000}%
\pgfsetstrokecolor{currentstroke}%
\pgfsetstrokeopacity{0.300000}%
\pgfsetdash{}{0pt}%
\pgfpathmoveto{\pgfqpoint{3.446933in}{0.383578in}}%
\pgfpathlineto{\pgfqpoint{3.464522in}{0.383578in}}%
\pgfpathlineto{\pgfqpoint{3.464522in}{0.989701in}}%
\pgfpathlineto{\pgfqpoint{3.446933in}{0.989701in}}%
\pgfpathclose%
\pgfusepath{fill}%
\end{pgfscope}%
\begin{pgfscope}%
\pgfpathrectangle{\pgfqpoint{0.526905in}{0.383578in}}{\pgfqpoint{3.875000in}{2.310000in}}%
\pgfusepath{clip}%
\pgfsetbuttcap%
\pgfsetmiterjoin%
\definecolor{currentfill}{rgb}{0.686275,0.352941,0.313725}%
\pgfsetfillcolor{currentfill}%
\pgfsetfillopacity{0.300000}%
\pgfsetlinewidth{0.000000pt}%
\definecolor{currentstroke}{rgb}{0.000000,0.000000,0.000000}%
\pgfsetstrokecolor{currentstroke}%
\pgfsetstrokeopacity{0.300000}%
\pgfsetdash{}{0pt}%
\pgfpathmoveto{\pgfqpoint{3.464522in}{0.383578in}}%
\pgfpathlineto{\pgfqpoint{3.482112in}{0.383578in}}%
\pgfpathlineto{\pgfqpoint{3.482112in}{0.984088in}}%
\pgfpathlineto{\pgfqpoint{3.464522in}{0.984088in}}%
\pgfpathclose%
\pgfusepath{fill}%
\end{pgfscope}%
\begin{pgfscope}%
\pgfpathrectangle{\pgfqpoint{0.526905in}{0.383578in}}{\pgfqpoint{3.875000in}{2.310000in}}%
\pgfusepath{clip}%
\pgfsetbuttcap%
\pgfsetmiterjoin%
\definecolor{currentfill}{rgb}{0.686275,0.352941,0.313725}%
\pgfsetfillcolor{currentfill}%
\pgfsetfillopacity{0.300000}%
\pgfsetlinewidth{0.000000pt}%
\definecolor{currentstroke}{rgb}{0.000000,0.000000,0.000000}%
\pgfsetstrokecolor{currentstroke}%
\pgfsetstrokeopacity{0.300000}%
\pgfsetdash{}{0pt}%
\pgfpathmoveto{\pgfqpoint{3.482112in}{0.383578in}}%
\pgfpathlineto{\pgfqpoint{3.499701in}{0.383578in}}%
\pgfpathlineto{\pgfqpoint{3.499701in}{0.888680in}}%
\pgfpathlineto{\pgfqpoint{3.482112in}{0.888680in}}%
\pgfpathclose%
\pgfusepath{fill}%
\end{pgfscope}%
\begin{pgfscope}%
\pgfpathrectangle{\pgfqpoint{0.526905in}{0.383578in}}{\pgfqpoint{3.875000in}{2.310000in}}%
\pgfusepath{clip}%
\pgfsetbuttcap%
\pgfsetmiterjoin%
\definecolor{currentfill}{rgb}{0.686275,0.352941,0.313725}%
\pgfsetfillcolor{currentfill}%
\pgfsetfillopacity{0.300000}%
\pgfsetlinewidth{0.000000pt}%
\definecolor{currentstroke}{rgb}{0.000000,0.000000,0.000000}%
\pgfsetstrokecolor{currentstroke}%
\pgfsetstrokeopacity{0.300000}%
\pgfsetdash{}{0pt}%
\pgfpathmoveto{\pgfqpoint{3.499701in}{0.383578in}}%
\pgfpathlineto{\pgfqpoint{3.517290in}{0.383578in}}%
\pgfpathlineto{\pgfqpoint{3.517290in}{0.883068in}}%
\pgfpathlineto{\pgfqpoint{3.499701in}{0.883068in}}%
\pgfpathclose%
\pgfusepath{fill}%
\end{pgfscope}%
\begin{pgfscope}%
\pgfpathrectangle{\pgfqpoint{0.526905in}{0.383578in}}{\pgfqpoint{3.875000in}{2.310000in}}%
\pgfusepath{clip}%
\pgfsetbuttcap%
\pgfsetmiterjoin%
\definecolor{currentfill}{rgb}{0.686275,0.352941,0.313725}%
\pgfsetfillcolor{currentfill}%
\pgfsetfillopacity{0.300000}%
\pgfsetlinewidth{0.000000pt}%
\definecolor{currentstroke}{rgb}{0.000000,0.000000,0.000000}%
\pgfsetstrokecolor{currentstroke}%
\pgfsetstrokeopacity{0.300000}%
\pgfsetdash{}{0pt}%
\pgfpathmoveto{\pgfqpoint{3.517290in}{0.383578in}}%
\pgfpathlineto{\pgfqpoint{3.534879in}{0.383578in}}%
\pgfpathlineto{\pgfqpoint{3.534879in}{0.838170in}}%
\pgfpathlineto{\pgfqpoint{3.517290in}{0.838170in}}%
\pgfpathclose%
\pgfusepath{fill}%
\end{pgfscope}%
\begin{pgfscope}%
\pgfpathrectangle{\pgfqpoint{0.526905in}{0.383578in}}{\pgfqpoint{3.875000in}{2.310000in}}%
\pgfusepath{clip}%
\pgfsetbuttcap%
\pgfsetmiterjoin%
\definecolor{currentfill}{rgb}{0.686275,0.352941,0.313725}%
\pgfsetfillcolor{currentfill}%
\pgfsetfillopacity{0.300000}%
\pgfsetlinewidth{0.000000pt}%
\definecolor{currentstroke}{rgb}{0.000000,0.000000,0.000000}%
\pgfsetstrokecolor{currentstroke}%
\pgfsetstrokeopacity{0.300000}%
\pgfsetdash{}{0pt}%
\pgfpathmoveto{\pgfqpoint{3.534879in}{0.383578in}}%
\pgfpathlineto{\pgfqpoint{3.552468in}{0.383578in}}%
\pgfpathlineto{\pgfqpoint{3.552468in}{0.748374in}}%
\pgfpathlineto{\pgfqpoint{3.534879in}{0.748374in}}%
\pgfpathclose%
\pgfusepath{fill}%
\end{pgfscope}%
\begin{pgfscope}%
\pgfpathrectangle{\pgfqpoint{0.526905in}{0.383578in}}{\pgfqpoint{3.875000in}{2.310000in}}%
\pgfusepath{clip}%
\pgfsetbuttcap%
\pgfsetmiterjoin%
\definecolor{currentfill}{rgb}{0.686275,0.352941,0.313725}%
\pgfsetfillcolor{currentfill}%
\pgfsetfillopacity{0.300000}%
\pgfsetlinewidth{0.000000pt}%
\definecolor{currentstroke}{rgb}{0.000000,0.000000,0.000000}%
\pgfsetstrokecolor{currentstroke}%
\pgfsetstrokeopacity{0.300000}%
\pgfsetdash{}{0pt}%
\pgfpathmoveto{\pgfqpoint{3.552468in}{0.383578in}}%
\pgfpathlineto{\pgfqpoint{3.570057in}{0.383578in}}%
\pgfpathlineto{\pgfqpoint{3.570057in}{0.770823in}}%
\pgfpathlineto{\pgfqpoint{3.552468in}{0.770823in}}%
\pgfpathclose%
\pgfusepath{fill}%
\end{pgfscope}%
\begin{pgfscope}%
\pgfpathrectangle{\pgfqpoint{0.526905in}{0.383578in}}{\pgfqpoint{3.875000in}{2.310000in}}%
\pgfusepath{clip}%
\pgfsetbuttcap%
\pgfsetmiterjoin%
\definecolor{currentfill}{rgb}{0.686275,0.352941,0.313725}%
\pgfsetfillcolor{currentfill}%
\pgfsetfillopacity{0.300000}%
\pgfsetlinewidth{0.000000pt}%
\definecolor{currentstroke}{rgb}{0.000000,0.000000,0.000000}%
\pgfsetstrokecolor{currentstroke}%
\pgfsetstrokeopacity{0.300000}%
\pgfsetdash{}{0pt}%
\pgfpathmoveto{\pgfqpoint{3.570057in}{0.383578in}}%
\pgfpathlineto{\pgfqpoint{3.587646in}{0.383578in}}%
\pgfpathlineto{\pgfqpoint{3.587646in}{0.759598in}}%
\pgfpathlineto{\pgfqpoint{3.570057in}{0.759598in}}%
\pgfpathclose%
\pgfusepath{fill}%
\end{pgfscope}%
\begin{pgfscope}%
\pgfpathrectangle{\pgfqpoint{0.526905in}{0.383578in}}{\pgfqpoint{3.875000in}{2.310000in}}%
\pgfusepath{clip}%
\pgfsetbuttcap%
\pgfsetmiterjoin%
\definecolor{currentfill}{rgb}{0.686275,0.352941,0.313725}%
\pgfsetfillcolor{currentfill}%
\pgfsetfillopacity{0.300000}%
\pgfsetlinewidth{0.000000pt}%
\definecolor{currentstroke}{rgb}{0.000000,0.000000,0.000000}%
\pgfsetstrokecolor{currentstroke}%
\pgfsetstrokeopacity{0.300000}%
\pgfsetdash{}{0pt}%
\pgfpathmoveto{\pgfqpoint{3.587646in}{0.383578in}}%
\pgfpathlineto{\pgfqpoint{3.605235in}{0.383578in}}%
\pgfpathlineto{\pgfqpoint{3.605235in}{0.709088in}}%
\pgfpathlineto{\pgfqpoint{3.587646in}{0.709088in}}%
\pgfpathclose%
\pgfusepath{fill}%
\end{pgfscope}%
\begin{pgfscope}%
\pgfpathrectangle{\pgfqpoint{0.526905in}{0.383578in}}{\pgfqpoint{3.875000in}{2.310000in}}%
\pgfusepath{clip}%
\pgfsetbuttcap%
\pgfsetmiterjoin%
\definecolor{currentfill}{rgb}{0.686275,0.352941,0.313725}%
\pgfsetfillcolor{currentfill}%
\pgfsetfillopacity{0.300000}%
\pgfsetlinewidth{0.000000pt}%
\definecolor{currentstroke}{rgb}{0.000000,0.000000,0.000000}%
\pgfsetstrokecolor{currentstroke}%
\pgfsetstrokeopacity{0.300000}%
\pgfsetdash{}{0pt}%
\pgfpathmoveto{\pgfqpoint{3.605235in}{0.383578in}}%
\pgfpathlineto{\pgfqpoint{3.622824in}{0.383578in}}%
\pgfpathlineto{\pgfqpoint{3.622824in}{0.720313in}}%
\pgfpathlineto{\pgfqpoint{3.605235in}{0.720313in}}%
\pgfpathclose%
\pgfusepath{fill}%
\end{pgfscope}%
\begin{pgfscope}%
\pgfpathrectangle{\pgfqpoint{0.526905in}{0.383578in}}{\pgfqpoint{3.875000in}{2.310000in}}%
\pgfusepath{clip}%
\pgfsetbuttcap%
\pgfsetmiterjoin%
\definecolor{currentfill}{rgb}{0.686275,0.352941,0.313725}%
\pgfsetfillcolor{currentfill}%
\pgfsetfillopacity{0.300000}%
\pgfsetlinewidth{0.000000pt}%
\definecolor{currentstroke}{rgb}{0.000000,0.000000,0.000000}%
\pgfsetstrokecolor{currentstroke}%
\pgfsetstrokeopacity{0.300000}%
\pgfsetdash{}{0pt}%
\pgfpathmoveto{\pgfqpoint{3.622824in}{0.383578in}}%
\pgfpathlineto{\pgfqpoint{3.640413in}{0.383578in}}%
\pgfpathlineto{\pgfqpoint{3.640413in}{0.608068in}}%
\pgfpathlineto{\pgfqpoint{3.622824in}{0.608068in}}%
\pgfpathclose%
\pgfusepath{fill}%
\end{pgfscope}%
\begin{pgfscope}%
\pgfpathrectangle{\pgfqpoint{0.526905in}{0.383578in}}{\pgfqpoint{3.875000in}{2.310000in}}%
\pgfusepath{clip}%
\pgfsetbuttcap%
\pgfsetmiterjoin%
\definecolor{currentfill}{rgb}{0.686275,0.352941,0.313725}%
\pgfsetfillcolor{currentfill}%
\pgfsetfillopacity{0.300000}%
\pgfsetlinewidth{0.000000pt}%
\definecolor{currentstroke}{rgb}{0.000000,0.000000,0.000000}%
\pgfsetstrokecolor{currentstroke}%
\pgfsetstrokeopacity{0.300000}%
\pgfsetdash{}{0pt}%
\pgfpathmoveto{\pgfqpoint{3.640413in}{0.383578in}}%
\pgfpathlineto{\pgfqpoint{3.658002in}{0.383578in}}%
\pgfpathlineto{\pgfqpoint{3.658002in}{0.697864in}}%
\pgfpathlineto{\pgfqpoint{3.640413in}{0.697864in}}%
\pgfpathclose%
\pgfusepath{fill}%
\end{pgfscope}%
\begin{pgfscope}%
\pgfpathrectangle{\pgfqpoint{0.526905in}{0.383578in}}{\pgfqpoint{3.875000in}{2.310000in}}%
\pgfusepath{clip}%
\pgfsetbuttcap%
\pgfsetmiterjoin%
\definecolor{currentfill}{rgb}{0.686275,0.352941,0.313725}%
\pgfsetfillcolor{currentfill}%
\pgfsetfillopacity{0.300000}%
\pgfsetlinewidth{0.000000pt}%
\definecolor{currentstroke}{rgb}{0.000000,0.000000,0.000000}%
\pgfsetstrokecolor{currentstroke}%
\pgfsetstrokeopacity{0.300000}%
\pgfsetdash{}{0pt}%
\pgfpathmoveto{\pgfqpoint{3.658002in}{0.383578in}}%
\pgfpathlineto{\pgfqpoint{3.675591in}{0.383578in}}%
\pgfpathlineto{\pgfqpoint{3.675591in}{0.602456in}}%
\pgfpathlineto{\pgfqpoint{3.658002in}{0.602456in}}%
\pgfpathclose%
\pgfusepath{fill}%
\end{pgfscope}%
\begin{pgfscope}%
\pgfpathrectangle{\pgfqpoint{0.526905in}{0.383578in}}{\pgfqpoint{3.875000in}{2.310000in}}%
\pgfusepath{clip}%
\pgfsetbuttcap%
\pgfsetmiterjoin%
\definecolor{currentfill}{rgb}{0.686275,0.352941,0.313725}%
\pgfsetfillcolor{currentfill}%
\pgfsetfillopacity{0.300000}%
\pgfsetlinewidth{0.000000pt}%
\definecolor{currentstroke}{rgb}{0.000000,0.000000,0.000000}%
\pgfsetstrokecolor{currentstroke}%
\pgfsetstrokeopacity{0.300000}%
\pgfsetdash{}{0pt}%
\pgfpathmoveto{\pgfqpoint{3.675591in}{0.383578in}}%
\pgfpathlineto{\pgfqpoint{3.693180in}{0.383578in}}%
\pgfpathlineto{\pgfqpoint{3.693180in}{0.613680in}}%
\pgfpathlineto{\pgfqpoint{3.675591in}{0.613680in}}%
\pgfpathclose%
\pgfusepath{fill}%
\end{pgfscope}%
\begin{pgfscope}%
\pgfpathrectangle{\pgfqpoint{0.526905in}{0.383578in}}{\pgfqpoint{3.875000in}{2.310000in}}%
\pgfusepath{clip}%
\pgfsetbuttcap%
\pgfsetmiterjoin%
\definecolor{currentfill}{rgb}{0.686275,0.352941,0.313725}%
\pgfsetfillcolor{currentfill}%
\pgfsetfillopacity{0.300000}%
\pgfsetlinewidth{0.000000pt}%
\definecolor{currentstroke}{rgb}{0.000000,0.000000,0.000000}%
\pgfsetstrokecolor{currentstroke}%
\pgfsetstrokeopacity{0.300000}%
\pgfsetdash{}{0pt}%
\pgfpathmoveto{\pgfqpoint{3.693180in}{0.383578in}}%
\pgfpathlineto{\pgfqpoint{3.710769in}{0.383578in}}%
\pgfpathlineto{\pgfqpoint{3.710769in}{0.529496in}}%
\pgfpathlineto{\pgfqpoint{3.693180in}{0.529496in}}%
\pgfpathclose%
\pgfusepath{fill}%
\end{pgfscope}%
\begin{pgfscope}%
\pgfpathrectangle{\pgfqpoint{0.526905in}{0.383578in}}{\pgfqpoint{3.875000in}{2.310000in}}%
\pgfusepath{clip}%
\pgfsetbuttcap%
\pgfsetmiterjoin%
\definecolor{currentfill}{rgb}{0.686275,0.352941,0.313725}%
\pgfsetfillcolor{currentfill}%
\pgfsetfillopacity{0.300000}%
\pgfsetlinewidth{0.000000pt}%
\definecolor{currentstroke}{rgb}{0.000000,0.000000,0.000000}%
\pgfsetstrokecolor{currentstroke}%
\pgfsetstrokeopacity{0.300000}%
\pgfsetdash{}{0pt}%
\pgfpathmoveto{\pgfqpoint{3.710769in}{0.383578in}}%
\pgfpathlineto{\pgfqpoint{3.728358in}{0.383578in}}%
\pgfpathlineto{\pgfqpoint{3.728358in}{0.551945in}}%
\pgfpathlineto{\pgfqpoint{3.710769in}{0.551945in}}%
\pgfpathclose%
\pgfusepath{fill}%
\end{pgfscope}%
\begin{pgfscope}%
\pgfpathrectangle{\pgfqpoint{0.526905in}{0.383578in}}{\pgfqpoint{3.875000in}{2.310000in}}%
\pgfusepath{clip}%
\pgfsetbuttcap%
\pgfsetmiterjoin%
\definecolor{currentfill}{rgb}{0.686275,0.352941,0.313725}%
\pgfsetfillcolor{currentfill}%
\pgfsetfillopacity{0.300000}%
\pgfsetlinewidth{0.000000pt}%
\definecolor{currentstroke}{rgb}{0.000000,0.000000,0.000000}%
\pgfsetstrokecolor{currentstroke}%
\pgfsetstrokeopacity{0.300000}%
\pgfsetdash{}{0pt}%
\pgfpathmoveto{\pgfqpoint{3.728358in}{0.383578in}}%
\pgfpathlineto{\pgfqpoint{3.745947in}{0.383578in}}%
\pgfpathlineto{\pgfqpoint{3.745947in}{0.563170in}}%
\pgfpathlineto{\pgfqpoint{3.728358in}{0.563170in}}%
\pgfpathclose%
\pgfusepath{fill}%
\end{pgfscope}%
\begin{pgfscope}%
\pgfpathrectangle{\pgfqpoint{0.526905in}{0.383578in}}{\pgfqpoint{3.875000in}{2.310000in}}%
\pgfusepath{clip}%
\pgfsetbuttcap%
\pgfsetmiterjoin%
\definecolor{currentfill}{rgb}{0.686275,0.352941,0.313725}%
\pgfsetfillcolor{currentfill}%
\pgfsetfillopacity{0.300000}%
\pgfsetlinewidth{0.000000pt}%
\definecolor{currentstroke}{rgb}{0.000000,0.000000,0.000000}%
\pgfsetstrokecolor{currentstroke}%
\pgfsetstrokeopacity{0.300000}%
\pgfsetdash{}{0pt}%
\pgfpathmoveto{\pgfqpoint{3.745947in}{0.383578in}}%
\pgfpathlineto{\pgfqpoint{3.763536in}{0.383578in}}%
\pgfpathlineto{\pgfqpoint{3.763536in}{0.518272in}}%
\pgfpathlineto{\pgfqpoint{3.745947in}{0.518272in}}%
\pgfpathclose%
\pgfusepath{fill}%
\end{pgfscope}%
\begin{pgfscope}%
\pgfpathrectangle{\pgfqpoint{0.526905in}{0.383578in}}{\pgfqpoint{3.875000in}{2.310000in}}%
\pgfusepath{clip}%
\pgfsetbuttcap%
\pgfsetmiterjoin%
\definecolor{currentfill}{rgb}{0.686275,0.352941,0.313725}%
\pgfsetfillcolor{currentfill}%
\pgfsetfillopacity{0.300000}%
\pgfsetlinewidth{0.000000pt}%
\definecolor{currentstroke}{rgb}{0.000000,0.000000,0.000000}%
\pgfsetstrokecolor{currentstroke}%
\pgfsetstrokeopacity{0.300000}%
\pgfsetdash{}{0pt}%
\pgfpathmoveto{\pgfqpoint{3.763536in}{0.383578in}}%
\pgfpathlineto{\pgfqpoint{3.781125in}{0.383578in}}%
\pgfpathlineto{\pgfqpoint{3.781125in}{0.563170in}}%
\pgfpathlineto{\pgfqpoint{3.763536in}{0.563170in}}%
\pgfpathclose%
\pgfusepath{fill}%
\end{pgfscope}%
\begin{pgfscope}%
\pgfpathrectangle{\pgfqpoint{0.526905in}{0.383578in}}{\pgfqpoint{3.875000in}{2.310000in}}%
\pgfusepath{clip}%
\pgfsetbuttcap%
\pgfsetmiterjoin%
\definecolor{currentfill}{rgb}{0.686275,0.352941,0.313725}%
\pgfsetfillcolor{currentfill}%
\pgfsetfillopacity{0.300000}%
\pgfsetlinewidth{0.000000pt}%
\definecolor{currentstroke}{rgb}{0.000000,0.000000,0.000000}%
\pgfsetstrokecolor{currentstroke}%
\pgfsetstrokeopacity{0.300000}%
\pgfsetdash{}{0pt}%
\pgfpathmoveto{\pgfqpoint{3.781125in}{0.383578in}}%
\pgfpathlineto{\pgfqpoint{3.798715in}{0.383578in}}%
\pgfpathlineto{\pgfqpoint{3.798715in}{0.495823in}}%
\pgfpathlineto{\pgfqpoint{3.781125in}{0.495823in}}%
\pgfpathclose%
\pgfusepath{fill}%
\end{pgfscope}%
\begin{pgfscope}%
\pgfpathrectangle{\pgfqpoint{0.526905in}{0.383578in}}{\pgfqpoint{3.875000in}{2.310000in}}%
\pgfusepath{clip}%
\pgfsetbuttcap%
\pgfsetmiterjoin%
\definecolor{currentfill}{rgb}{0.686275,0.352941,0.313725}%
\pgfsetfillcolor{currentfill}%
\pgfsetfillopacity{0.300000}%
\pgfsetlinewidth{0.000000pt}%
\definecolor{currentstroke}{rgb}{0.000000,0.000000,0.000000}%
\pgfsetstrokecolor{currentstroke}%
\pgfsetstrokeopacity{0.300000}%
\pgfsetdash{}{0pt}%
\pgfpathmoveto{\pgfqpoint{3.798715in}{0.383578in}}%
\pgfpathlineto{\pgfqpoint{3.816304in}{0.383578in}}%
\pgfpathlineto{\pgfqpoint{3.816304in}{0.540721in}}%
\pgfpathlineto{\pgfqpoint{3.798715in}{0.540721in}}%
\pgfpathclose%
\pgfusepath{fill}%
\end{pgfscope}%
\begin{pgfscope}%
\pgfpathrectangle{\pgfqpoint{0.526905in}{0.383578in}}{\pgfqpoint{3.875000in}{2.310000in}}%
\pgfusepath{clip}%
\pgfsetbuttcap%
\pgfsetmiterjoin%
\definecolor{currentfill}{rgb}{0.686275,0.352941,0.313725}%
\pgfsetfillcolor{currentfill}%
\pgfsetfillopacity{0.300000}%
\pgfsetlinewidth{0.000000pt}%
\definecolor{currentstroke}{rgb}{0.000000,0.000000,0.000000}%
\pgfsetstrokecolor{currentstroke}%
\pgfsetstrokeopacity{0.300000}%
\pgfsetdash{}{0pt}%
\pgfpathmoveto{\pgfqpoint{3.816304in}{0.383578in}}%
\pgfpathlineto{\pgfqpoint{3.833893in}{0.383578in}}%
\pgfpathlineto{\pgfqpoint{3.833893in}{0.462149in}}%
\pgfpathlineto{\pgfqpoint{3.816304in}{0.462149in}}%
\pgfpathclose%
\pgfusepath{fill}%
\end{pgfscope}%
\begin{pgfscope}%
\pgfpathrectangle{\pgfqpoint{0.526905in}{0.383578in}}{\pgfqpoint{3.875000in}{2.310000in}}%
\pgfusepath{clip}%
\pgfsetbuttcap%
\pgfsetmiterjoin%
\definecolor{currentfill}{rgb}{0.686275,0.352941,0.313725}%
\pgfsetfillcolor{currentfill}%
\pgfsetfillopacity{0.300000}%
\pgfsetlinewidth{0.000000pt}%
\definecolor{currentstroke}{rgb}{0.000000,0.000000,0.000000}%
\pgfsetstrokecolor{currentstroke}%
\pgfsetstrokeopacity{0.300000}%
\pgfsetdash{}{0pt}%
\pgfpathmoveto{\pgfqpoint{3.833893in}{0.383578in}}%
\pgfpathlineto{\pgfqpoint{3.851482in}{0.383578in}}%
\pgfpathlineto{\pgfqpoint{3.851482in}{0.484598in}}%
\pgfpathlineto{\pgfqpoint{3.833893in}{0.484598in}}%
\pgfpathclose%
\pgfusepath{fill}%
\end{pgfscope}%
\begin{pgfscope}%
\pgfpathrectangle{\pgfqpoint{0.526905in}{0.383578in}}{\pgfqpoint{3.875000in}{2.310000in}}%
\pgfusepath{clip}%
\pgfsetbuttcap%
\pgfsetmiterjoin%
\definecolor{currentfill}{rgb}{0.686275,0.352941,0.313725}%
\pgfsetfillcolor{currentfill}%
\pgfsetfillopacity{0.300000}%
\pgfsetlinewidth{0.000000pt}%
\definecolor{currentstroke}{rgb}{0.000000,0.000000,0.000000}%
\pgfsetstrokecolor{currentstroke}%
\pgfsetstrokeopacity{0.300000}%
\pgfsetdash{}{0pt}%
\pgfpathmoveto{\pgfqpoint{3.851482in}{0.383578in}}%
\pgfpathlineto{\pgfqpoint{3.869071in}{0.383578in}}%
\pgfpathlineto{\pgfqpoint{3.869071in}{0.490211in}}%
\pgfpathlineto{\pgfqpoint{3.851482in}{0.490211in}}%
\pgfpathclose%
\pgfusepath{fill}%
\end{pgfscope}%
\begin{pgfscope}%
\pgfpathrectangle{\pgfqpoint{0.526905in}{0.383578in}}{\pgfqpoint{3.875000in}{2.310000in}}%
\pgfusepath{clip}%
\pgfsetbuttcap%
\pgfsetmiterjoin%
\definecolor{currentfill}{rgb}{0.686275,0.352941,0.313725}%
\pgfsetfillcolor{currentfill}%
\pgfsetfillopacity{0.300000}%
\pgfsetlinewidth{0.000000pt}%
\definecolor{currentstroke}{rgb}{0.000000,0.000000,0.000000}%
\pgfsetstrokecolor{currentstroke}%
\pgfsetstrokeopacity{0.300000}%
\pgfsetdash{}{0pt}%
\pgfpathmoveto{\pgfqpoint{3.869071in}{0.383578in}}%
\pgfpathlineto{\pgfqpoint{3.886660in}{0.383578in}}%
\pgfpathlineto{\pgfqpoint{3.886660in}{0.434088in}}%
\pgfpathlineto{\pgfqpoint{3.869071in}{0.434088in}}%
\pgfpathclose%
\pgfusepath{fill}%
\end{pgfscope}%
\begin{pgfscope}%
\pgfpathrectangle{\pgfqpoint{0.526905in}{0.383578in}}{\pgfqpoint{3.875000in}{2.310000in}}%
\pgfusepath{clip}%
\pgfsetbuttcap%
\pgfsetmiterjoin%
\definecolor{currentfill}{rgb}{0.686275,0.352941,0.313725}%
\pgfsetfillcolor{currentfill}%
\pgfsetfillopacity{0.300000}%
\pgfsetlinewidth{0.000000pt}%
\definecolor{currentstroke}{rgb}{0.000000,0.000000,0.000000}%
\pgfsetstrokecolor{currentstroke}%
\pgfsetstrokeopacity{0.300000}%
\pgfsetdash{}{0pt}%
\pgfpathmoveto{\pgfqpoint{3.886660in}{0.383578in}}%
\pgfpathlineto{\pgfqpoint{3.904249in}{0.383578in}}%
\pgfpathlineto{\pgfqpoint{3.904249in}{0.462149in}}%
\pgfpathlineto{\pgfqpoint{3.886660in}{0.462149in}}%
\pgfpathclose%
\pgfusepath{fill}%
\end{pgfscope}%
\begin{pgfscope}%
\pgfpathrectangle{\pgfqpoint{0.526905in}{0.383578in}}{\pgfqpoint{3.875000in}{2.310000in}}%
\pgfusepath{clip}%
\pgfsetbuttcap%
\pgfsetmiterjoin%
\definecolor{currentfill}{rgb}{0.686275,0.352941,0.313725}%
\pgfsetfillcolor{currentfill}%
\pgfsetfillopacity{0.300000}%
\pgfsetlinewidth{0.000000pt}%
\definecolor{currentstroke}{rgb}{0.000000,0.000000,0.000000}%
\pgfsetstrokecolor{currentstroke}%
\pgfsetstrokeopacity{0.300000}%
\pgfsetdash{}{0pt}%
\pgfpathmoveto{\pgfqpoint{3.904249in}{0.383578in}}%
\pgfpathlineto{\pgfqpoint{3.921838in}{0.383578in}}%
\pgfpathlineto{\pgfqpoint{3.921838in}{0.422864in}}%
\pgfpathlineto{\pgfqpoint{3.904249in}{0.422864in}}%
\pgfpathclose%
\pgfusepath{fill}%
\end{pgfscope}%
\begin{pgfscope}%
\pgfpathrectangle{\pgfqpoint{0.526905in}{0.383578in}}{\pgfqpoint{3.875000in}{2.310000in}}%
\pgfusepath{clip}%
\pgfsetbuttcap%
\pgfsetmiterjoin%
\definecolor{currentfill}{rgb}{0.686275,0.352941,0.313725}%
\pgfsetfillcolor{currentfill}%
\pgfsetfillopacity{0.300000}%
\pgfsetlinewidth{0.000000pt}%
\definecolor{currentstroke}{rgb}{0.000000,0.000000,0.000000}%
\pgfsetstrokecolor{currentstroke}%
\pgfsetstrokeopacity{0.300000}%
\pgfsetdash{}{0pt}%
\pgfpathmoveto{\pgfqpoint{3.921838in}{0.383578in}}%
\pgfpathlineto{\pgfqpoint{3.939427in}{0.383578in}}%
\pgfpathlineto{\pgfqpoint{3.939427in}{0.406027in}}%
\pgfpathlineto{\pgfqpoint{3.921838in}{0.406027in}}%
\pgfpathclose%
\pgfusepath{fill}%
\end{pgfscope}%
\begin{pgfscope}%
\pgfpathrectangle{\pgfqpoint{0.526905in}{0.383578in}}{\pgfqpoint{3.875000in}{2.310000in}}%
\pgfusepath{clip}%
\pgfsetbuttcap%
\pgfsetmiterjoin%
\definecolor{currentfill}{rgb}{0.686275,0.352941,0.313725}%
\pgfsetfillcolor{currentfill}%
\pgfsetfillopacity{0.300000}%
\pgfsetlinewidth{0.000000pt}%
\definecolor{currentstroke}{rgb}{0.000000,0.000000,0.000000}%
\pgfsetstrokecolor{currentstroke}%
\pgfsetstrokeopacity{0.300000}%
\pgfsetdash{}{0pt}%
\pgfpathmoveto{\pgfqpoint{3.939427in}{0.383578in}}%
\pgfpathlineto{\pgfqpoint{3.957016in}{0.383578in}}%
\pgfpathlineto{\pgfqpoint{3.957016in}{0.417252in}}%
\pgfpathlineto{\pgfqpoint{3.939427in}{0.417252in}}%
\pgfpathclose%
\pgfusepath{fill}%
\end{pgfscope}%
\begin{pgfscope}%
\pgfpathrectangle{\pgfqpoint{0.526905in}{0.383578in}}{\pgfqpoint{3.875000in}{2.310000in}}%
\pgfusepath{clip}%
\pgfsetbuttcap%
\pgfsetmiterjoin%
\definecolor{currentfill}{rgb}{0.686275,0.352941,0.313725}%
\pgfsetfillcolor{currentfill}%
\pgfsetfillopacity{0.300000}%
\pgfsetlinewidth{0.000000pt}%
\definecolor{currentstroke}{rgb}{0.000000,0.000000,0.000000}%
\pgfsetstrokecolor{currentstroke}%
\pgfsetstrokeopacity{0.300000}%
\pgfsetdash{}{0pt}%
\pgfpathmoveto{\pgfqpoint{3.957016in}{0.383578in}}%
\pgfpathlineto{\pgfqpoint{3.974605in}{0.383578in}}%
\pgfpathlineto{\pgfqpoint{3.974605in}{0.422864in}}%
\pgfpathlineto{\pgfqpoint{3.957016in}{0.422864in}}%
\pgfpathclose%
\pgfusepath{fill}%
\end{pgfscope}%
\begin{pgfscope}%
\pgfpathrectangle{\pgfqpoint{0.526905in}{0.383578in}}{\pgfqpoint{3.875000in}{2.310000in}}%
\pgfusepath{clip}%
\pgfsetbuttcap%
\pgfsetmiterjoin%
\definecolor{currentfill}{rgb}{0.686275,0.352941,0.313725}%
\pgfsetfillcolor{currentfill}%
\pgfsetfillopacity{0.300000}%
\pgfsetlinewidth{0.000000pt}%
\definecolor{currentstroke}{rgb}{0.000000,0.000000,0.000000}%
\pgfsetstrokecolor{currentstroke}%
\pgfsetstrokeopacity{0.300000}%
\pgfsetdash{}{0pt}%
\pgfpathmoveto{\pgfqpoint{3.974605in}{0.383578in}}%
\pgfpathlineto{\pgfqpoint{3.992194in}{0.383578in}}%
\pgfpathlineto{\pgfqpoint{3.992194in}{0.422864in}}%
\pgfpathlineto{\pgfqpoint{3.974605in}{0.422864in}}%
\pgfpathclose%
\pgfusepath{fill}%
\end{pgfscope}%
\begin{pgfscope}%
\pgfpathrectangle{\pgfqpoint{0.526905in}{0.383578in}}{\pgfqpoint{3.875000in}{2.310000in}}%
\pgfusepath{clip}%
\pgfsetbuttcap%
\pgfsetmiterjoin%
\definecolor{currentfill}{rgb}{0.686275,0.352941,0.313725}%
\pgfsetfillcolor{currentfill}%
\pgfsetfillopacity{0.300000}%
\pgfsetlinewidth{0.000000pt}%
\definecolor{currentstroke}{rgb}{0.000000,0.000000,0.000000}%
\pgfsetstrokecolor{currentstroke}%
\pgfsetstrokeopacity{0.300000}%
\pgfsetdash{}{0pt}%
\pgfpathmoveto{\pgfqpoint{3.992194in}{0.383578in}}%
\pgfpathlineto{\pgfqpoint{4.009783in}{0.383578in}}%
\pgfpathlineto{\pgfqpoint{4.009783in}{0.417252in}}%
\pgfpathlineto{\pgfqpoint{3.992194in}{0.417252in}}%
\pgfpathclose%
\pgfusepath{fill}%
\end{pgfscope}%
\begin{pgfscope}%
\pgfpathrectangle{\pgfqpoint{0.526905in}{0.383578in}}{\pgfqpoint{3.875000in}{2.310000in}}%
\pgfusepath{clip}%
\pgfsetbuttcap%
\pgfsetmiterjoin%
\definecolor{currentfill}{rgb}{0.686275,0.352941,0.313725}%
\pgfsetfillcolor{currentfill}%
\pgfsetfillopacity{0.300000}%
\pgfsetlinewidth{0.000000pt}%
\definecolor{currentstroke}{rgb}{0.000000,0.000000,0.000000}%
\pgfsetstrokecolor{currentstroke}%
\pgfsetstrokeopacity{0.300000}%
\pgfsetdash{}{0pt}%
\pgfpathmoveto{\pgfqpoint{4.009783in}{0.383578in}}%
\pgfpathlineto{\pgfqpoint{4.027372in}{0.383578in}}%
\pgfpathlineto{\pgfqpoint{4.027372in}{0.422864in}}%
\pgfpathlineto{\pgfqpoint{4.009783in}{0.422864in}}%
\pgfpathclose%
\pgfusepath{fill}%
\end{pgfscope}%
\begin{pgfscope}%
\pgfpathrectangle{\pgfqpoint{0.526905in}{0.383578in}}{\pgfqpoint{3.875000in}{2.310000in}}%
\pgfusepath{clip}%
\pgfsetbuttcap%
\pgfsetmiterjoin%
\definecolor{currentfill}{rgb}{0.686275,0.352941,0.313725}%
\pgfsetfillcolor{currentfill}%
\pgfsetfillopacity{0.300000}%
\pgfsetlinewidth{0.000000pt}%
\definecolor{currentstroke}{rgb}{0.000000,0.000000,0.000000}%
\pgfsetstrokecolor{currentstroke}%
\pgfsetstrokeopacity{0.300000}%
\pgfsetdash{}{0pt}%
\pgfpathmoveto{\pgfqpoint{4.027372in}{0.383578in}}%
\pgfpathlineto{\pgfqpoint{4.044961in}{0.383578in}}%
\pgfpathlineto{\pgfqpoint{4.044961in}{0.428476in}}%
\pgfpathlineto{\pgfqpoint{4.027372in}{0.428476in}}%
\pgfpathclose%
\pgfusepath{fill}%
\end{pgfscope}%
\begin{pgfscope}%
\pgfpathrectangle{\pgfqpoint{0.526905in}{0.383578in}}{\pgfqpoint{3.875000in}{2.310000in}}%
\pgfusepath{clip}%
\pgfsetbuttcap%
\pgfsetmiterjoin%
\definecolor{currentfill}{rgb}{0.686275,0.352941,0.313725}%
\pgfsetfillcolor{currentfill}%
\pgfsetfillopacity{0.300000}%
\pgfsetlinewidth{0.000000pt}%
\definecolor{currentstroke}{rgb}{0.000000,0.000000,0.000000}%
\pgfsetstrokecolor{currentstroke}%
\pgfsetstrokeopacity{0.300000}%
\pgfsetdash{}{0pt}%
\pgfpathmoveto{\pgfqpoint{4.044961in}{0.383578in}}%
\pgfpathlineto{\pgfqpoint{4.062550in}{0.383578in}}%
\pgfpathlineto{\pgfqpoint{4.062550in}{0.394803in}}%
\pgfpathlineto{\pgfqpoint{4.044961in}{0.394803in}}%
\pgfpathclose%
\pgfusepath{fill}%
\end{pgfscope}%
\begin{pgfscope}%
\pgfpathrectangle{\pgfqpoint{0.526905in}{0.383578in}}{\pgfqpoint{3.875000in}{2.310000in}}%
\pgfusepath{clip}%
\pgfsetbuttcap%
\pgfsetmiterjoin%
\definecolor{currentfill}{rgb}{0.686275,0.352941,0.313725}%
\pgfsetfillcolor{currentfill}%
\pgfsetfillopacity{0.300000}%
\pgfsetlinewidth{0.000000pt}%
\definecolor{currentstroke}{rgb}{0.000000,0.000000,0.000000}%
\pgfsetstrokecolor{currentstroke}%
\pgfsetstrokeopacity{0.300000}%
\pgfsetdash{}{0pt}%
\pgfpathmoveto{\pgfqpoint{4.062550in}{0.383578in}}%
\pgfpathlineto{\pgfqpoint{4.080139in}{0.383578in}}%
\pgfpathlineto{\pgfqpoint{4.080139in}{0.400415in}}%
\pgfpathlineto{\pgfqpoint{4.062550in}{0.400415in}}%
\pgfpathclose%
\pgfusepath{fill}%
\end{pgfscope}%
\begin{pgfscope}%
\pgfpathrectangle{\pgfqpoint{0.526905in}{0.383578in}}{\pgfqpoint{3.875000in}{2.310000in}}%
\pgfusepath{clip}%
\pgfsetbuttcap%
\pgfsetmiterjoin%
\definecolor{currentfill}{rgb}{0.686275,0.352941,0.313725}%
\pgfsetfillcolor{currentfill}%
\pgfsetfillopacity{0.300000}%
\pgfsetlinewidth{0.000000pt}%
\definecolor{currentstroke}{rgb}{0.000000,0.000000,0.000000}%
\pgfsetstrokecolor{currentstroke}%
\pgfsetstrokeopacity{0.300000}%
\pgfsetdash{}{0pt}%
\pgfpathmoveto{\pgfqpoint{4.080139in}{0.383578in}}%
\pgfpathlineto{\pgfqpoint{4.097728in}{0.383578in}}%
\pgfpathlineto{\pgfqpoint{4.097728in}{0.411639in}}%
\pgfpathlineto{\pgfqpoint{4.080139in}{0.411639in}}%
\pgfpathclose%
\pgfusepath{fill}%
\end{pgfscope}%
\begin{pgfscope}%
\pgfpathrectangle{\pgfqpoint{0.526905in}{0.383578in}}{\pgfqpoint{3.875000in}{2.310000in}}%
\pgfusepath{clip}%
\pgfsetbuttcap%
\pgfsetmiterjoin%
\definecolor{currentfill}{rgb}{0.686275,0.352941,0.313725}%
\pgfsetfillcolor{currentfill}%
\pgfsetfillopacity{0.300000}%
\pgfsetlinewidth{0.000000pt}%
\definecolor{currentstroke}{rgb}{0.000000,0.000000,0.000000}%
\pgfsetstrokecolor{currentstroke}%
\pgfsetstrokeopacity{0.300000}%
\pgfsetdash{}{0pt}%
\pgfpathmoveto{\pgfqpoint{4.097728in}{0.383578in}}%
\pgfpathlineto{\pgfqpoint{4.115317in}{0.383578in}}%
\pgfpathlineto{\pgfqpoint{4.115317in}{0.400415in}}%
\pgfpathlineto{\pgfqpoint{4.097728in}{0.400415in}}%
\pgfpathclose%
\pgfusepath{fill}%
\end{pgfscope}%
\begin{pgfscope}%
\pgfpathrectangle{\pgfqpoint{0.526905in}{0.383578in}}{\pgfqpoint{3.875000in}{2.310000in}}%
\pgfusepath{clip}%
\pgfsetbuttcap%
\pgfsetmiterjoin%
\definecolor{currentfill}{rgb}{0.686275,0.352941,0.313725}%
\pgfsetfillcolor{currentfill}%
\pgfsetfillopacity{0.300000}%
\pgfsetlinewidth{0.000000pt}%
\definecolor{currentstroke}{rgb}{0.000000,0.000000,0.000000}%
\pgfsetstrokecolor{currentstroke}%
\pgfsetstrokeopacity{0.300000}%
\pgfsetdash{}{0pt}%
\pgfpathmoveto{\pgfqpoint{4.115317in}{0.383578in}}%
\pgfpathlineto{\pgfqpoint{4.132907in}{0.383578in}}%
\pgfpathlineto{\pgfqpoint{4.132907in}{0.394803in}}%
\pgfpathlineto{\pgfqpoint{4.115317in}{0.394803in}}%
\pgfpathclose%
\pgfusepath{fill}%
\end{pgfscope}%
\begin{pgfscope}%
\pgfpathrectangle{\pgfqpoint{0.526905in}{0.383578in}}{\pgfqpoint{3.875000in}{2.310000in}}%
\pgfusepath{clip}%
\pgfsetbuttcap%
\pgfsetmiterjoin%
\definecolor{currentfill}{rgb}{0.686275,0.352941,0.313725}%
\pgfsetfillcolor{currentfill}%
\pgfsetfillopacity{0.300000}%
\pgfsetlinewidth{0.000000pt}%
\definecolor{currentstroke}{rgb}{0.000000,0.000000,0.000000}%
\pgfsetstrokecolor{currentstroke}%
\pgfsetstrokeopacity{0.300000}%
\pgfsetdash{}{0pt}%
\pgfpathmoveto{\pgfqpoint{4.132907in}{0.383578in}}%
\pgfpathlineto{\pgfqpoint{4.150496in}{0.383578in}}%
\pgfpathlineto{\pgfqpoint{4.150496in}{0.394803in}}%
\pgfpathlineto{\pgfqpoint{4.132907in}{0.394803in}}%
\pgfpathclose%
\pgfusepath{fill}%
\end{pgfscope}%
\begin{pgfscope}%
\pgfpathrectangle{\pgfqpoint{0.526905in}{0.383578in}}{\pgfqpoint{3.875000in}{2.310000in}}%
\pgfusepath{clip}%
\pgfsetbuttcap%
\pgfsetmiterjoin%
\definecolor{currentfill}{rgb}{0.686275,0.352941,0.313725}%
\pgfsetfillcolor{currentfill}%
\pgfsetfillopacity{0.300000}%
\pgfsetlinewidth{0.000000pt}%
\definecolor{currentstroke}{rgb}{0.000000,0.000000,0.000000}%
\pgfsetstrokecolor{currentstroke}%
\pgfsetstrokeopacity{0.300000}%
\pgfsetdash{}{0pt}%
\pgfpathmoveto{\pgfqpoint{4.150496in}{0.383578in}}%
\pgfpathlineto{\pgfqpoint{4.168085in}{0.383578in}}%
\pgfpathlineto{\pgfqpoint{4.168085in}{0.389190in}}%
\pgfpathlineto{\pgfqpoint{4.150496in}{0.389190in}}%
\pgfpathclose%
\pgfusepath{fill}%
\end{pgfscope}%
\begin{pgfscope}%
\pgfpathrectangle{\pgfqpoint{0.526905in}{0.383578in}}{\pgfqpoint{3.875000in}{2.310000in}}%
\pgfusepath{clip}%
\pgfsetbuttcap%
\pgfsetmiterjoin%
\definecolor{currentfill}{rgb}{0.686275,0.352941,0.313725}%
\pgfsetfillcolor{currentfill}%
\pgfsetfillopacity{0.300000}%
\pgfsetlinewidth{0.000000pt}%
\definecolor{currentstroke}{rgb}{0.000000,0.000000,0.000000}%
\pgfsetstrokecolor{currentstroke}%
\pgfsetstrokeopacity{0.300000}%
\pgfsetdash{}{0pt}%
\pgfpathmoveto{\pgfqpoint{4.168085in}{0.383578in}}%
\pgfpathlineto{\pgfqpoint{4.185674in}{0.383578in}}%
\pgfpathlineto{\pgfqpoint{4.185674in}{0.383578in}}%
\pgfpathlineto{\pgfqpoint{4.168085in}{0.383578in}}%
\pgfpathclose%
\pgfusepath{fill}%
\end{pgfscope}%
\begin{pgfscope}%
\pgfpathrectangle{\pgfqpoint{0.526905in}{0.383578in}}{\pgfqpoint{3.875000in}{2.310000in}}%
\pgfusepath{clip}%
\pgfsetbuttcap%
\pgfsetmiterjoin%
\definecolor{currentfill}{rgb}{0.686275,0.352941,0.313725}%
\pgfsetfillcolor{currentfill}%
\pgfsetfillopacity{0.300000}%
\pgfsetlinewidth{0.000000pt}%
\definecolor{currentstroke}{rgb}{0.000000,0.000000,0.000000}%
\pgfsetstrokecolor{currentstroke}%
\pgfsetstrokeopacity{0.300000}%
\pgfsetdash{}{0pt}%
\pgfpathmoveto{\pgfqpoint{4.185674in}{0.383578in}}%
\pgfpathlineto{\pgfqpoint{4.203263in}{0.383578in}}%
\pgfpathlineto{\pgfqpoint{4.203263in}{0.389190in}}%
\pgfpathlineto{\pgfqpoint{4.185674in}{0.389190in}}%
\pgfpathclose%
\pgfusepath{fill}%
\end{pgfscope}%
\begin{pgfscope}%
\pgfpathrectangle{\pgfqpoint{0.526905in}{0.383578in}}{\pgfqpoint{3.875000in}{2.310000in}}%
\pgfusepath{clip}%
\pgfsetbuttcap%
\pgfsetmiterjoin%
\definecolor{currentfill}{rgb}{0.686275,0.352941,0.313725}%
\pgfsetfillcolor{currentfill}%
\pgfsetfillopacity{0.300000}%
\pgfsetlinewidth{0.000000pt}%
\definecolor{currentstroke}{rgb}{0.000000,0.000000,0.000000}%
\pgfsetstrokecolor{currentstroke}%
\pgfsetstrokeopacity{0.300000}%
\pgfsetdash{}{0pt}%
\pgfpathmoveto{\pgfqpoint{4.203263in}{0.383578in}}%
\pgfpathlineto{\pgfqpoint{4.220852in}{0.383578in}}%
\pgfpathlineto{\pgfqpoint{4.220852in}{0.389190in}}%
\pgfpathlineto{\pgfqpoint{4.203263in}{0.389190in}}%
\pgfpathclose%
\pgfusepath{fill}%
\end{pgfscope}%
\begin{pgfscope}%
\pgfpathrectangle{\pgfqpoint{0.526905in}{0.383578in}}{\pgfqpoint{3.875000in}{2.310000in}}%
\pgfusepath{clip}%
\pgfsetbuttcap%
\pgfsetmiterjoin%
\definecolor{currentfill}{rgb}{0.333333,0.333333,0.333333}%
\pgfsetfillcolor{currentfill}%
\pgfsetlinewidth{0.000000pt}%
\definecolor{currentstroke}{rgb}{0.000000,0.000000,0.000000}%
\pgfsetstrokecolor{currentstroke}%
\pgfsetstrokeopacity{0.000000}%
\pgfsetdash{}{0pt}%
\pgfpathmoveto{\pgfqpoint{3.336483in}{0.383578in}}%
\pgfpathlineto{\pgfqpoint{3.363905in}{0.383578in}}%
\pgfpathlineto{\pgfqpoint{3.363905in}{0.995313in}}%
\pgfpathlineto{\pgfqpoint{3.336483in}{0.995313in}}%
\pgfpathclose%
\pgfusepath{fill}%
\end{pgfscope}%
\begin{pgfscope}%
\pgfpathrectangle{\pgfqpoint{0.526905in}{0.383578in}}{\pgfqpoint{3.875000in}{2.310000in}}%
\pgfusepath{clip}%
\pgfsetbuttcap%
\pgfsetmiterjoin%
\definecolor{currentfill}{rgb}{0.333333,0.333333,0.333333}%
\pgfsetfillcolor{currentfill}%
\pgfsetlinewidth{0.000000pt}%
\definecolor{currentstroke}{rgb}{0.000000,0.000000,0.000000}%
\pgfsetstrokecolor{currentstroke}%
\pgfsetstrokeopacity{0.000000}%
\pgfsetdash{}{0pt}%
\pgfpathmoveto{\pgfqpoint{3.354072in}{0.383578in}}%
\pgfpathlineto{\pgfqpoint{3.381494in}{0.383578in}}%
\pgfpathlineto{\pgfqpoint{3.381494in}{1.090721in}}%
\pgfpathlineto{\pgfqpoint{3.354072in}{1.090721in}}%
\pgfpathclose%
\pgfusepath{fill}%
\end{pgfscope}%
\begin{pgfscope}%
\pgfpathrectangle{\pgfqpoint{0.526905in}{0.383578in}}{\pgfqpoint{3.875000in}{2.310000in}}%
\pgfusepath{clip}%
\pgfsetbuttcap%
\pgfsetmiterjoin%
\definecolor{currentfill}{rgb}{0.333333,0.333333,0.333333}%
\pgfsetfillcolor{currentfill}%
\pgfsetlinewidth{0.000000pt}%
\definecolor{currentstroke}{rgb}{0.000000,0.000000,0.000000}%
\pgfsetstrokecolor{currentstroke}%
\pgfsetstrokeopacity{0.000000}%
\pgfsetdash{}{0pt}%
\pgfpathmoveto{\pgfqpoint{3.371661in}{0.383578in}}%
\pgfpathlineto{\pgfqpoint{3.399083in}{0.383578in}}%
\pgfpathlineto{\pgfqpoint{3.399083in}{1.045823in}}%
\pgfpathlineto{\pgfqpoint{3.371661in}{1.045823in}}%
\pgfpathclose%
\pgfusepath{fill}%
\end{pgfscope}%
\begin{pgfscope}%
\pgfpathrectangle{\pgfqpoint{0.526905in}{0.383578in}}{\pgfqpoint{3.875000in}{2.310000in}}%
\pgfusepath{clip}%
\pgfsetbuttcap%
\pgfsetmiterjoin%
\definecolor{currentfill}{rgb}{0.333333,0.333333,0.333333}%
\pgfsetfillcolor{currentfill}%
\pgfsetlinewidth{0.000000pt}%
\definecolor{currentstroke}{rgb}{0.000000,0.000000,0.000000}%
\pgfsetstrokecolor{currentstroke}%
\pgfsetstrokeopacity{0.000000}%
\pgfsetdash{}{0pt}%
\pgfpathmoveto{\pgfqpoint{3.389250in}{0.383578in}}%
\pgfpathlineto{\pgfqpoint{3.416672in}{0.383578in}}%
\pgfpathlineto{\pgfqpoint{3.416672in}{1.017762in}}%
\pgfpathlineto{\pgfqpoint{3.389250in}{1.017762in}}%
\pgfpathclose%
\pgfusepath{fill}%
\end{pgfscope}%
\begin{pgfscope}%
\pgfpathrectangle{\pgfqpoint{0.526905in}{0.383578in}}{\pgfqpoint{3.875000in}{2.310000in}}%
\pgfusepath{clip}%
\pgfsetbuttcap%
\pgfsetmiterjoin%
\definecolor{currentfill}{rgb}{0.333333,0.333333,0.333333}%
\pgfsetfillcolor{currentfill}%
\pgfsetlinewidth{0.000000pt}%
\definecolor{currentstroke}{rgb}{0.000000,0.000000,0.000000}%
\pgfsetstrokecolor{currentstroke}%
\pgfsetstrokeopacity{0.000000}%
\pgfsetdash{}{0pt}%
\pgfpathmoveto{\pgfqpoint{3.406839in}{0.383578in}}%
\pgfpathlineto{\pgfqpoint{3.434261in}{0.383578in}}%
\pgfpathlineto{\pgfqpoint{3.434261in}{1.113170in}}%
\pgfpathlineto{\pgfqpoint{3.406839in}{1.113170in}}%
\pgfpathclose%
\pgfusepath{fill}%
\end{pgfscope}%
\begin{pgfscope}%
\pgfpathrectangle{\pgfqpoint{0.526905in}{0.383578in}}{\pgfqpoint{3.875000in}{2.310000in}}%
\pgfusepath{clip}%
\pgfsetbuttcap%
\pgfsetmiterjoin%
\definecolor{currentfill}{rgb}{0.333333,0.333333,0.333333}%
\pgfsetfillcolor{currentfill}%
\pgfsetlinewidth{0.000000pt}%
\definecolor{currentstroke}{rgb}{0.000000,0.000000,0.000000}%
\pgfsetstrokecolor{currentstroke}%
\pgfsetstrokeopacity{0.000000}%
\pgfsetdash{}{0pt}%
\pgfpathmoveto{\pgfqpoint{3.424428in}{0.383578in}}%
\pgfpathlineto{\pgfqpoint{3.451850in}{0.383578in}}%
\pgfpathlineto{\pgfqpoint{3.451850in}{1.040211in}}%
\pgfpathlineto{\pgfqpoint{3.424428in}{1.040211in}}%
\pgfpathclose%
\pgfusepath{fill}%
\end{pgfscope}%
\begin{pgfscope}%
\pgfpathrectangle{\pgfqpoint{0.526905in}{0.383578in}}{\pgfqpoint{3.875000in}{2.310000in}}%
\pgfusepath{clip}%
\pgfsetbuttcap%
\pgfsetmiterjoin%
\definecolor{currentfill}{rgb}{0.333333,0.333333,0.333333}%
\pgfsetfillcolor{currentfill}%
\pgfsetlinewidth{0.000000pt}%
\definecolor{currentstroke}{rgb}{0.000000,0.000000,0.000000}%
\pgfsetstrokecolor{currentstroke}%
\pgfsetstrokeopacity{0.000000}%
\pgfsetdash{}{0pt}%
\pgfpathmoveto{\pgfqpoint{3.442017in}{0.383578in}}%
\pgfpathlineto{\pgfqpoint{3.469439in}{0.383578in}}%
\pgfpathlineto{\pgfqpoint{3.469439in}{0.989701in}}%
\pgfpathlineto{\pgfqpoint{3.442017in}{0.989701in}}%
\pgfpathclose%
\pgfusepath{fill}%
\end{pgfscope}%
\begin{pgfscope}%
\pgfpathrectangle{\pgfqpoint{0.526905in}{0.383578in}}{\pgfqpoint{3.875000in}{2.310000in}}%
\pgfusepath{clip}%
\pgfsetbuttcap%
\pgfsetmiterjoin%
\definecolor{currentfill}{rgb}{0.333333,0.333333,0.333333}%
\pgfsetfillcolor{currentfill}%
\pgfsetlinewidth{0.000000pt}%
\definecolor{currentstroke}{rgb}{0.000000,0.000000,0.000000}%
\pgfsetstrokecolor{currentstroke}%
\pgfsetstrokeopacity{0.000000}%
\pgfsetdash{}{0pt}%
\pgfpathmoveto{\pgfqpoint{3.459606in}{0.383578in}}%
\pgfpathlineto{\pgfqpoint{3.487028in}{0.383578in}}%
\pgfpathlineto{\pgfqpoint{3.487028in}{0.984088in}}%
\pgfpathlineto{\pgfqpoint{3.459606in}{0.984088in}}%
\pgfpathclose%
\pgfusepath{fill}%
\end{pgfscope}%
\begin{pgfscope}%
\pgfpathrectangle{\pgfqpoint{0.526905in}{0.383578in}}{\pgfqpoint{3.875000in}{2.310000in}}%
\pgfusepath{clip}%
\pgfsetbuttcap%
\pgfsetmiterjoin%
\definecolor{currentfill}{rgb}{0.333333,0.333333,0.333333}%
\pgfsetfillcolor{currentfill}%
\pgfsetlinewidth{0.000000pt}%
\definecolor{currentstroke}{rgb}{0.000000,0.000000,0.000000}%
\pgfsetstrokecolor{currentstroke}%
\pgfsetstrokeopacity{0.000000}%
\pgfsetdash{}{0pt}%
\pgfpathmoveto{\pgfqpoint{3.477195in}{0.383578in}}%
\pgfpathlineto{\pgfqpoint{3.504617in}{0.383578in}}%
\pgfpathlineto{\pgfqpoint{3.504617in}{0.888680in}}%
\pgfpathlineto{\pgfqpoint{3.477195in}{0.888680in}}%
\pgfpathclose%
\pgfusepath{fill}%
\end{pgfscope}%
\begin{pgfscope}%
\pgfpathrectangle{\pgfqpoint{0.526905in}{0.383578in}}{\pgfqpoint{3.875000in}{2.310000in}}%
\pgfusepath{clip}%
\pgfsetbuttcap%
\pgfsetmiterjoin%
\definecolor{currentfill}{rgb}{0.333333,0.333333,0.333333}%
\pgfsetfillcolor{currentfill}%
\pgfsetlinewidth{0.000000pt}%
\definecolor{currentstroke}{rgb}{0.000000,0.000000,0.000000}%
\pgfsetstrokecolor{currentstroke}%
\pgfsetstrokeopacity{0.000000}%
\pgfsetdash{}{0pt}%
\pgfpathmoveto{\pgfqpoint{3.494784in}{0.383578in}}%
\pgfpathlineto{\pgfqpoint{3.522206in}{0.383578in}}%
\pgfpathlineto{\pgfqpoint{3.522206in}{0.883068in}}%
\pgfpathlineto{\pgfqpoint{3.494784in}{0.883068in}}%
\pgfpathclose%
\pgfusepath{fill}%
\end{pgfscope}%
\begin{pgfscope}%
\pgfpathrectangle{\pgfqpoint{0.526905in}{0.383578in}}{\pgfqpoint{3.875000in}{2.310000in}}%
\pgfusepath{clip}%
\pgfsetbuttcap%
\pgfsetmiterjoin%
\definecolor{currentfill}{rgb}{0.333333,0.333333,0.333333}%
\pgfsetfillcolor{currentfill}%
\pgfsetlinewidth{0.000000pt}%
\definecolor{currentstroke}{rgb}{0.000000,0.000000,0.000000}%
\pgfsetstrokecolor{currentstroke}%
\pgfsetstrokeopacity{0.000000}%
\pgfsetdash{}{0pt}%
\pgfpathmoveto{\pgfqpoint{3.512373in}{0.383578in}}%
\pgfpathlineto{\pgfqpoint{3.539795in}{0.383578in}}%
\pgfpathlineto{\pgfqpoint{3.539795in}{0.838170in}}%
\pgfpathlineto{\pgfqpoint{3.512373in}{0.838170in}}%
\pgfpathclose%
\pgfusepath{fill}%
\end{pgfscope}%
\begin{pgfscope}%
\pgfpathrectangle{\pgfqpoint{0.526905in}{0.383578in}}{\pgfqpoint{3.875000in}{2.310000in}}%
\pgfusepath{clip}%
\pgfsetbuttcap%
\pgfsetmiterjoin%
\definecolor{currentfill}{rgb}{0.333333,0.333333,0.333333}%
\pgfsetfillcolor{currentfill}%
\pgfsetlinewidth{0.000000pt}%
\definecolor{currentstroke}{rgb}{0.000000,0.000000,0.000000}%
\pgfsetstrokecolor{currentstroke}%
\pgfsetstrokeopacity{0.000000}%
\pgfsetdash{}{0pt}%
\pgfpathmoveto{\pgfqpoint{3.529962in}{0.383578in}}%
\pgfpathlineto{\pgfqpoint{3.557384in}{0.383578in}}%
\pgfpathlineto{\pgfqpoint{3.557384in}{0.748374in}}%
\pgfpathlineto{\pgfqpoint{3.529962in}{0.748374in}}%
\pgfpathclose%
\pgfusepath{fill}%
\end{pgfscope}%
\begin{pgfscope}%
\pgfpathrectangle{\pgfqpoint{0.526905in}{0.383578in}}{\pgfqpoint{3.875000in}{2.310000in}}%
\pgfusepath{clip}%
\pgfsetbuttcap%
\pgfsetmiterjoin%
\definecolor{currentfill}{rgb}{0.333333,0.333333,0.333333}%
\pgfsetfillcolor{currentfill}%
\pgfsetlinewidth{0.000000pt}%
\definecolor{currentstroke}{rgb}{0.000000,0.000000,0.000000}%
\pgfsetstrokecolor{currentstroke}%
\pgfsetstrokeopacity{0.000000}%
\pgfsetdash{}{0pt}%
\pgfpathmoveto{\pgfqpoint{3.547551in}{0.383578in}}%
\pgfpathlineto{\pgfqpoint{3.574973in}{0.383578in}}%
\pgfpathlineto{\pgfqpoint{3.574973in}{0.770823in}}%
\pgfpathlineto{\pgfqpoint{3.547551in}{0.770823in}}%
\pgfpathclose%
\pgfusepath{fill}%
\end{pgfscope}%
\begin{pgfscope}%
\pgfpathrectangle{\pgfqpoint{0.526905in}{0.383578in}}{\pgfqpoint{3.875000in}{2.310000in}}%
\pgfusepath{clip}%
\pgfsetbuttcap%
\pgfsetmiterjoin%
\definecolor{currentfill}{rgb}{0.333333,0.333333,0.333333}%
\pgfsetfillcolor{currentfill}%
\pgfsetlinewidth{0.000000pt}%
\definecolor{currentstroke}{rgb}{0.000000,0.000000,0.000000}%
\pgfsetstrokecolor{currentstroke}%
\pgfsetstrokeopacity{0.000000}%
\pgfsetdash{}{0pt}%
\pgfpathmoveto{\pgfqpoint{3.565140in}{0.383578in}}%
\pgfpathlineto{\pgfqpoint{3.592562in}{0.383578in}}%
\pgfpathlineto{\pgfqpoint{3.592562in}{0.759598in}}%
\pgfpathlineto{\pgfqpoint{3.565140in}{0.759598in}}%
\pgfpathclose%
\pgfusepath{fill}%
\end{pgfscope}%
\begin{pgfscope}%
\pgfpathrectangle{\pgfqpoint{0.526905in}{0.383578in}}{\pgfqpoint{3.875000in}{2.310000in}}%
\pgfusepath{clip}%
\pgfsetbuttcap%
\pgfsetmiterjoin%
\definecolor{currentfill}{rgb}{0.333333,0.333333,0.333333}%
\pgfsetfillcolor{currentfill}%
\pgfsetlinewidth{0.000000pt}%
\definecolor{currentstroke}{rgb}{0.000000,0.000000,0.000000}%
\pgfsetstrokecolor{currentstroke}%
\pgfsetstrokeopacity{0.000000}%
\pgfsetdash{}{0pt}%
\pgfpathmoveto{\pgfqpoint{3.582729in}{0.383578in}}%
\pgfpathlineto{\pgfqpoint{3.610151in}{0.383578in}}%
\pgfpathlineto{\pgfqpoint{3.610151in}{0.709088in}}%
\pgfpathlineto{\pgfqpoint{3.582729in}{0.709088in}}%
\pgfpathclose%
\pgfusepath{fill}%
\end{pgfscope}%
\begin{pgfscope}%
\pgfpathrectangle{\pgfqpoint{0.526905in}{0.383578in}}{\pgfqpoint{3.875000in}{2.310000in}}%
\pgfusepath{clip}%
\pgfsetbuttcap%
\pgfsetmiterjoin%
\definecolor{currentfill}{rgb}{0.333333,0.333333,0.333333}%
\pgfsetfillcolor{currentfill}%
\pgfsetlinewidth{0.000000pt}%
\definecolor{currentstroke}{rgb}{0.000000,0.000000,0.000000}%
\pgfsetstrokecolor{currentstroke}%
\pgfsetstrokeopacity{0.000000}%
\pgfsetdash{}{0pt}%
\pgfpathmoveto{\pgfqpoint{3.600318in}{0.383578in}}%
\pgfpathlineto{\pgfqpoint{3.627741in}{0.383578in}}%
\pgfpathlineto{\pgfqpoint{3.627741in}{0.720313in}}%
\pgfpathlineto{\pgfqpoint{3.600318in}{0.720313in}}%
\pgfpathclose%
\pgfusepath{fill}%
\end{pgfscope}%
\begin{pgfscope}%
\pgfpathrectangle{\pgfqpoint{0.526905in}{0.383578in}}{\pgfqpoint{3.875000in}{2.310000in}}%
\pgfusepath{clip}%
\pgfsetbuttcap%
\pgfsetmiterjoin%
\definecolor{currentfill}{rgb}{0.333333,0.333333,0.333333}%
\pgfsetfillcolor{currentfill}%
\pgfsetlinewidth{0.000000pt}%
\definecolor{currentstroke}{rgb}{0.000000,0.000000,0.000000}%
\pgfsetstrokecolor{currentstroke}%
\pgfsetstrokeopacity{0.000000}%
\pgfsetdash{}{0pt}%
\pgfpathmoveto{\pgfqpoint{3.617907in}{0.383578in}}%
\pgfpathlineto{\pgfqpoint{3.645330in}{0.383578in}}%
\pgfpathlineto{\pgfqpoint{3.645330in}{0.608068in}}%
\pgfpathlineto{\pgfqpoint{3.617907in}{0.608068in}}%
\pgfpathclose%
\pgfusepath{fill}%
\end{pgfscope}%
\begin{pgfscope}%
\pgfpathrectangle{\pgfqpoint{0.526905in}{0.383578in}}{\pgfqpoint{3.875000in}{2.310000in}}%
\pgfusepath{clip}%
\pgfsetbuttcap%
\pgfsetmiterjoin%
\definecolor{currentfill}{rgb}{0.333333,0.333333,0.333333}%
\pgfsetfillcolor{currentfill}%
\pgfsetlinewidth{0.000000pt}%
\definecolor{currentstroke}{rgb}{0.000000,0.000000,0.000000}%
\pgfsetstrokecolor{currentstroke}%
\pgfsetstrokeopacity{0.000000}%
\pgfsetdash{}{0pt}%
\pgfpathmoveto{\pgfqpoint{3.635496in}{0.383578in}}%
\pgfpathlineto{\pgfqpoint{3.662919in}{0.383578in}}%
\pgfpathlineto{\pgfqpoint{3.662919in}{0.697864in}}%
\pgfpathlineto{\pgfqpoint{3.635496in}{0.697864in}}%
\pgfpathclose%
\pgfusepath{fill}%
\end{pgfscope}%
\begin{pgfscope}%
\pgfpathrectangle{\pgfqpoint{0.526905in}{0.383578in}}{\pgfqpoint{3.875000in}{2.310000in}}%
\pgfusepath{clip}%
\pgfsetbuttcap%
\pgfsetmiterjoin%
\definecolor{currentfill}{rgb}{0.333333,0.333333,0.333333}%
\pgfsetfillcolor{currentfill}%
\pgfsetlinewidth{0.000000pt}%
\definecolor{currentstroke}{rgb}{0.000000,0.000000,0.000000}%
\pgfsetstrokecolor{currentstroke}%
\pgfsetstrokeopacity{0.000000}%
\pgfsetdash{}{0pt}%
\pgfpathmoveto{\pgfqpoint{3.653086in}{0.383578in}}%
\pgfpathlineto{\pgfqpoint{3.680508in}{0.383578in}}%
\pgfpathlineto{\pgfqpoint{3.680508in}{0.602456in}}%
\pgfpathlineto{\pgfqpoint{3.653086in}{0.602456in}}%
\pgfpathclose%
\pgfusepath{fill}%
\end{pgfscope}%
\begin{pgfscope}%
\pgfpathrectangle{\pgfqpoint{0.526905in}{0.383578in}}{\pgfqpoint{3.875000in}{2.310000in}}%
\pgfusepath{clip}%
\pgfsetbuttcap%
\pgfsetmiterjoin%
\definecolor{currentfill}{rgb}{0.333333,0.333333,0.333333}%
\pgfsetfillcolor{currentfill}%
\pgfsetlinewidth{0.000000pt}%
\definecolor{currentstroke}{rgb}{0.000000,0.000000,0.000000}%
\pgfsetstrokecolor{currentstroke}%
\pgfsetstrokeopacity{0.000000}%
\pgfsetdash{}{0pt}%
\pgfpathmoveto{\pgfqpoint{3.670675in}{0.383578in}}%
\pgfpathlineto{\pgfqpoint{3.698097in}{0.383578in}}%
\pgfpathlineto{\pgfqpoint{3.698097in}{0.613680in}}%
\pgfpathlineto{\pgfqpoint{3.670675in}{0.613680in}}%
\pgfpathclose%
\pgfusepath{fill}%
\end{pgfscope}%
\begin{pgfscope}%
\pgfpathrectangle{\pgfqpoint{0.526905in}{0.383578in}}{\pgfqpoint{3.875000in}{2.310000in}}%
\pgfusepath{clip}%
\pgfsetbuttcap%
\pgfsetmiterjoin%
\definecolor{currentfill}{rgb}{0.333333,0.333333,0.333333}%
\pgfsetfillcolor{currentfill}%
\pgfsetlinewidth{0.000000pt}%
\definecolor{currentstroke}{rgb}{0.000000,0.000000,0.000000}%
\pgfsetstrokecolor{currentstroke}%
\pgfsetstrokeopacity{0.000000}%
\pgfsetdash{}{0pt}%
\pgfpathmoveto{\pgfqpoint{3.688264in}{0.383578in}}%
\pgfpathlineto{\pgfqpoint{3.715686in}{0.383578in}}%
\pgfpathlineto{\pgfqpoint{3.715686in}{0.529496in}}%
\pgfpathlineto{\pgfqpoint{3.688264in}{0.529496in}}%
\pgfpathclose%
\pgfusepath{fill}%
\end{pgfscope}%
\begin{pgfscope}%
\pgfpathrectangle{\pgfqpoint{0.526905in}{0.383578in}}{\pgfqpoint{3.875000in}{2.310000in}}%
\pgfusepath{clip}%
\pgfsetbuttcap%
\pgfsetmiterjoin%
\definecolor{currentfill}{rgb}{0.333333,0.333333,0.333333}%
\pgfsetfillcolor{currentfill}%
\pgfsetlinewidth{0.000000pt}%
\definecolor{currentstroke}{rgb}{0.000000,0.000000,0.000000}%
\pgfsetstrokecolor{currentstroke}%
\pgfsetstrokeopacity{0.000000}%
\pgfsetdash{}{0pt}%
\pgfpathmoveto{\pgfqpoint{3.705853in}{0.383578in}}%
\pgfpathlineto{\pgfqpoint{3.733275in}{0.383578in}}%
\pgfpathlineto{\pgfqpoint{3.733275in}{0.551945in}}%
\pgfpathlineto{\pgfqpoint{3.705853in}{0.551945in}}%
\pgfpathclose%
\pgfusepath{fill}%
\end{pgfscope}%
\begin{pgfscope}%
\pgfpathrectangle{\pgfqpoint{0.526905in}{0.383578in}}{\pgfqpoint{3.875000in}{2.310000in}}%
\pgfusepath{clip}%
\pgfsetbuttcap%
\pgfsetmiterjoin%
\definecolor{currentfill}{rgb}{0.333333,0.333333,0.333333}%
\pgfsetfillcolor{currentfill}%
\pgfsetlinewidth{0.000000pt}%
\definecolor{currentstroke}{rgb}{0.000000,0.000000,0.000000}%
\pgfsetstrokecolor{currentstroke}%
\pgfsetstrokeopacity{0.000000}%
\pgfsetdash{}{0pt}%
\pgfpathmoveto{\pgfqpoint{3.723442in}{0.383578in}}%
\pgfpathlineto{\pgfqpoint{3.750864in}{0.383578in}}%
\pgfpathlineto{\pgfqpoint{3.750864in}{0.563170in}}%
\pgfpathlineto{\pgfqpoint{3.723442in}{0.563170in}}%
\pgfpathclose%
\pgfusepath{fill}%
\end{pgfscope}%
\begin{pgfscope}%
\pgfpathrectangle{\pgfqpoint{0.526905in}{0.383578in}}{\pgfqpoint{3.875000in}{2.310000in}}%
\pgfusepath{clip}%
\pgfsetbuttcap%
\pgfsetmiterjoin%
\definecolor{currentfill}{rgb}{0.333333,0.333333,0.333333}%
\pgfsetfillcolor{currentfill}%
\pgfsetlinewidth{0.000000pt}%
\definecolor{currentstroke}{rgb}{0.000000,0.000000,0.000000}%
\pgfsetstrokecolor{currentstroke}%
\pgfsetstrokeopacity{0.000000}%
\pgfsetdash{}{0pt}%
\pgfpathmoveto{\pgfqpoint{3.741031in}{0.383578in}}%
\pgfpathlineto{\pgfqpoint{3.768453in}{0.383578in}}%
\pgfpathlineto{\pgfqpoint{3.768453in}{0.518272in}}%
\pgfpathlineto{\pgfqpoint{3.741031in}{0.518272in}}%
\pgfpathclose%
\pgfusepath{fill}%
\end{pgfscope}%
\begin{pgfscope}%
\pgfpathrectangle{\pgfqpoint{0.526905in}{0.383578in}}{\pgfqpoint{3.875000in}{2.310000in}}%
\pgfusepath{clip}%
\pgfsetbuttcap%
\pgfsetmiterjoin%
\definecolor{currentfill}{rgb}{0.333333,0.333333,0.333333}%
\pgfsetfillcolor{currentfill}%
\pgfsetlinewidth{0.000000pt}%
\definecolor{currentstroke}{rgb}{0.000000,0.000000,0.000000}%
\pgfsetstrokecolor{currentstroke}%
\pgfsetstrokeopacity{0.000000}%
\pgfsetdash{}{0pt}%
\pgfpathmoveto{\pgfqpoint{3.758620in}{0.383578in}}%
\pgfpathlineto{\pgfqpoint{3.786042in}{0.383578in}}%
\pgfpathlineto{\pgfqpoint{3.786042in}{0.563170in}}%
\pgfpathlineto{\pgfqpoint{3.758620in}{0.563170in}}%
\pgfpathclose%
\pgfusepath{fill}%
\end{pgfscope}%
\begin{pgfscope}%
\pgfpathrectangle{\pgfqpoint{0.526905in}{0.383578in}}{\pgfqpoint{3.875000in}{2.310000in}}%
\pgfusepath{clip}%
\pgfsetbuttcap%
\pgfsetmiterjoin%
\definecolor{currentfill}{rgb}{0.333333,0.333333,0.333333}%
\pgfsetfillcolor{currentfill}%
\pgfsetlinewidth{0.000000pt}%
\definecolor{currentstroke}{rgb}{0.000000,0.000000,0.000000}%
\pgfsetstrokecolor{currentstroke}%
\pgfsetstrokeopacity{0.000000}%
\pgfsetdash{}{0pt}%
\pgfpathmoveto{\pgfqpoint{3.776209in}{0.383578in}}%
\pgfpathlineto{\pgfqpoint{3.803631in}{0.383578in}}%
\pgfpathlineto{\pgfqpoint{3.803631in}{0.495823in}}%
\pgfpathlineto{\pgfqpoint{3.776209in}{0.495823in}}%
\pgfpathclose%
\pgfusepath{fill}%
\end{pgfscope}%
\begin{pgfscope}%
\pgfpathrectangle{\pgfqpoint{0.526905in}{0.383578in}}{\pgfqpoint{3.875000in}{2.310000in}}%
\pgfusepath{clip}%
\pgfsetbuttcap%
\pgfsetmiterjoin%
\definecolor{currentfill}{rgb}{0.333333,0.333333,0.333333}%
\pgfsetfillcolor{currentfill}%
\pgfsetlinewidth{0.000000pt}%
\definecolor{currentstroke}{rgb}{0.000000,0.000000,0.000000}%
\pgfsetstrokecolor{currentstroke}%
\pgfsetstrokeopacity{0.000000}%
\pgfsetdash{}{0pt}%
\pgfpathmoveto{\pgfqpoint{3.793798in}{0.383578in}}%
\pgfpathlineto{\pgfqpoint{3.821220in}{0.383578in}}%
\pgfpathlineto{\pgfqpoint{3.821220in}{0.540721in}}%
\pgfpathlineto{\pgfqpoint{3.793798in}{0.540721in}}%
\pgfpathclose%
\pgfusepath{fill}%
\end{pgfscope}%
\begin{pgfscope}%
\pgfpathrectangle{\pgfqpoint{0.526905in}{0.383578in}}{\pgfqpoint{3.875000in}{2.310000in}}%
\pgfusepath{clip}%
\pgfsetbuttcap%
\pgfsetmiterjoin%
\definecolor{currentfill}{rgb}{0.333333,0.333333,0.333333}%
\pgfsetfillcolor{currentfill}%
\pgfsetlinewidth{0.000000pt}%
\definecolor{currentstroke}{rgb}{0.000000,0.000000,0.000000}%
\pgfsetstrokecolor{currentstroke}%
\pgfsetstrokeopacity{0.000000}%
\pgfsetdash{}{0pt}%
\pgfpathmoveto{\pgfqpoint{3.811387in}{0.383578in}}%
\pgfpathlineto{\pgfqpoint{3.838809in}{0.383578in}}%
\pgfpathlineto{\pgfqpoint{3.838809in}{0.462149in}}%
\pgfpathlineto{\pgfqpoint{3.811387in}{0.462149in}}%
\pgfpathclose%
\pgfusepath{fill}%
\end{pgfscope}%
\begin{pgfscope}%
\pgfpathrectangle{\pgfqpoint{0.526905in}{0.383578in}}{\pgfqpoint{3.875000in}{2.310000in}}%
\pgfusepath{clip}%
\pgfsetbuttcap%
\pgfsetmiterjoin%
\definecolor{currentfill}{rgb}{0.333333,0.333333,0.333333}%
\pgfsetfillcolor{currentfill}%
\pgfsetlinewidth{0.000000pt}%
\definecolor{currentstroke}{rgb}{0.000000,0.000000,0.000000}%
\pgfsetstrokecolor{currentstroke}%
\pgfsetstrokeopacity{0.000000}%
\pgfsetdash{}{0pt}%
\pgfpathmoveto{\pgfqpoint{3.828976in}{0.383578in}}%
\pgfpathlineto{\pgfqpoint{3.856398in}{0.383578in}}%
\pgfpathlineto{\pgfqpoint{3.856398in}{0.484598in}}%
\pgfpathlineto{\pgfqpoint{3.828976in}{0.484598in}}%
\pgfpathclose%
\pgfusepath{fill}%
\end{pgfscope}%
\begin{pgfscope}%
\pgfpathrectangle{\pgfqpoint{0.526905in}{0.383578in}}{\pgfqpoint{3.875000in}{2.310000in}}%
\pgfusepath{clip}%
\pgfsetbuttcap%
\pgfsetmiterjoin%
\definecolor{currentfill}{rgb}{0.333333,0.333333,0.333333}%
\pgfsetfillcolor{currentfill}%
\pgfsetlinewidth{0.000000pt}%
\definecolor{currentstroke}{rgb}{0.000000,0.000000,0.000000}%
\pgfsetstrokecolor{currentstroke}%
\pgfsetstrokeopacity{0.000000}%
\pgfsetdash{}{0pt}%
\pgfpathmoveto{\pgfqpoint{3.846565in}{0.383578in}}%
\pgfpathlineto{\pgfqpoint{3.873987in}{0.383578in}}%
\pgfpathlineto{\pgfqpoint{3.873987in}{0.490211in}}%
\pgfpathlineto{\pgfqpoint{3.846565in}{0.490211in}}%
\pgfpathclose%
\pgfusepath{fill}%
\end{pgfscope}%
\begin{pgfscope}%
\pgfpathrectangle{\pgfqpoint{0.526905in}{0.383578in}}{\pgfqpoint{3.875000in}{2.310000in}}%
\pgfusepath{clip}%
\pgfsetbuttcap%
\pgfsetmiterjoin%
\definecolor{currentfill}{rgb}{0.333333,0.333333,0.333333}%
\pgfsetfillcolor{currentfill}%
\pgfsetlinewidth{0.000000pt}%
\definecolor{currentstroke}{rgb}{0.000000,0.000000,0.000000}%
\pgfsetstrokecolor{currentstroke}%
\pgfsetstrokeopacity{0.000000}%
\pgfsetdash{}{0pt}%
\pgfpathmoveto{\pgfqpoint{3.864154in}{0.383578in}}%
\pgfpathlineto{\pgfqpoint{3.891576in}{0.383578in}}%
\pgfpathlineto{\pgfqpoint{3.891576in}{0.434088in}}%
\pgfpathlineto{\pgfqpoint{3.864154in}{0.434088in}}%
\pgfpathclose%
\pgfusepath{fill}%
\end{pgfscope}%
\begin{pgfscope}%
\pgfpathrectangle{\pgfqpoint{0.526905in}{0.383578in}}{\pgfqpoint{3.875000in}{2.310000in}}%
\pgfusepath{clip}%
\pgfsetbuttcap%
\pgfsetmiterjoin%
\definecolor{currentfill}{rgb}{0.333333,0.333333,0.333333}%
\pgfsetfillcolor{currentfill}%
\pgfsetlinewidth{0.000000pt}%
\definecolor{currentstroke}{rgb}{0.000000,0.000000,0.000000}%
\pgfsetstrokecolor{currentstroke}%
\pgfsetstrokeopacity{0.000000}%
\pgfsetdash{}{0pt}%
\pgfpathmoveto{\pgfqpoint{3.881743in}{0.383578in}}%
\pgfpathlineto{\pgfqpoint{3.909165in}{0.383578in}}%
\pgfpathlineto{\pgfqpoint{3.909165in}{0.462149in}}%
\pgfpathlineto{\pgfqpoint{3.881743in}{0.462149in}}%
\pgfpathclose%
\pgfusepath{fill}%
\end{pgfscope}%
\begin{pgfscope}%
\pgfpathrectangle{\pgfqpoint{0.526905in}{0.383578in}}{\pgfqpoint{3.875000in}{2.310000in}}%
\pgfusepath{clip}%
\pgfsetbuttcap%
\pgfsetmiterjoin%
\definecolor{currentfill}{rgb}{0.333333,0.333333,0.333333}%
\pgfsetfillcolor{currentfill}%
\pgfsetlinewidth{0.000000pt}%
\definecolor{currentstroke}{rgb}{0.000000,0.000000,0.000000}%
\pgfsetstrokecolor{currentstroke}%
\pgfsetstrokeopacity{0.000000}%
\pgfsetdash{}{0pt}%
\pgfpathmoveto{\pgfqpoint{3.899332in}{0.383578in}}%
\pgfpathlineto{\pgfqpoint{3.926754in}{0.383578in}}%
\pgfpathlineto{\pgfqpoint{3.926754in}{0.422864in}}%
\pgfpathlineto{\pgfqpoint{3.899332in}{0.422864in}}%
\pgfpathclose%
\pgfusepath{fill}%
\end{pgfscope}%
\begin{pgfscope}%
\pgfpathrectangle{\pgfqpoint{0.526905in}{0.383578in}}{\pgfqpoint{3.875000in}{2.310000in}}%
\pgfusepath{clip}%
\pgfsetbuttcap%
\pgfsetmiterjoin%
\definecolor{currentfill}{rgb}{0.333333,0.333333,0.333333}%
\pgfsetfillcolor{currentfill}%
\pgfsetlinewidth{0.000000pt}%
\definecolor{currentstroke}{rgb}{0.000000,0.000000,0.000000}%
\pgfsetstrokecolor{currentstroke}%
\pgfsetstrokeopacity{0.000000}%
\pgfsetdash{}{0pt}%
\pgfpathmoveto{\pgfqpoint{3.916921in}{0.383578in}}%
\pgfpathlineto{\pgfqpoint{3.944344in}{0.383578in}}%
\pgfpathlineto{\pgfqpoint{3.944344in}{0.406027in}}%
\pgfpathlineto{\pgfqpoint{3.916921in}{0.406027in}}%
\pgfpathclose%
\pgfusepath{fill}%
\end{pgfscope}%
\begin{pgfscope}%
\pgfpathrectangle{\pgfqpoint{0.526905in}{0.383578in}}{\pgfqpoint{3.875000in}{2.310000in}}%
\pgfusepath{clip}%
\pgfsetbuttcap%
\pgfsetmiterjoin%
\definecolor{currentfill}{rgb}{0.333333,0.333333,0.333333}%
\pgfsetfillcolor{currentfill}%
\pgfsetlinewidth{0.000000pt}%
\definecolor{currentstroke}{rgb}{0.000000,0.000000,0.000000}%
\pgfsetstrokecolor{currentstroke}%
\pgfsetstrokeopacity{0.000000}%
\pgfsetdash{}{0pt}%
\pgfpathmoveto{\pgfqpoint{3.934510in}{0.383578in}}%
\pgfpathlineto{\pgfqpoint{3.961933in}{0.383578in}}%
\pgfpathlineto{\pgfqpoint{3.961933in}{0.417252in}}%
\pgfpathlineto{\pgfqpoint{3.934510in}{0.417252in}}%
\pgfpathclose%
\pgfusepath{fill}%
\end{pgfscope}%
\begin{pgfscope}%
\pgfpathrectangle{\pgfqpoint{0.526905in}{0.383578in}}{\pgfqpoint{3.875000in}{2.310000in}}%
\pgfusepath{clip}%
\pgfsetbuttcap%
\pgfsetmiterjoin%
\definecolor{currentfill}{rgb}{0.333333,0.333333,0.333333}%
\pgfsetfillcolor{currentfill}%
\pgfsetlinewidth{0.000000pt}%
\definecolor{currentstroke}{rgb}{0.000000,0.000000,0.000000}%
\pgfsetstrokecolor{currentstroke}%
\pgfsetstrokeopacity{0.000000}%
\pgfsetdash{}{0pt}%
\pgfpathmoveto{\pgfqpoint{3.952099in}{0.383578in}}%
\pgfpathlineto{\pgfqpoint{3.979522in}{0.383578in}}%
\pgfpathlineto{\pgfqpoint{3.979522in}{0.422864in}}%
\pgfpathlineto{\pgfqpoint{3.952099in}{0.422864in}}%
\pgfpathclose%
\pgfusepath{fill}%
\end{pgfscope}%
\begin{pgfscope}%
\pgfpathrectangle{\pgfqpoint{0.526905in}{0.383578in}}{\pgfqpoint{3.875000in}{2.310000in}}%
\pgfusepath{clip}%
\pgfsetbuttcap%
\pgfsetmiterjoin%
\definecolor{currentfill}{rgb}{0.333333,0.333333,0.333333}%
\pgfsetfillcolor{currentfill}%
\pgfsetlinewidth{0.000000pt}%
\definecolor{currentstroke}{rgb}{0.000000,0.000000,0.000000}%
\pgfsetstrokecolor{currentstroke}%
\pgfsetstrokeopacity{0.000000}%
\pgfsetdash{}{0pt}%
\pgfpathmoveto{\pgfqpoint{3.969688in}{0.383578in}}%
\pgfpathlineto{\pgfqpoint{3.997111in}{0.383578in}}%
\pgfpathlineto{\pgfqpoint{3.997111in}{0.422864in}}%
\pgfpathlineto{\pgfqpoint{3.969688in}{0.422864in}}%
\pgfpathclose%
\pgfusepath{fill}%
\end{pgfscope}%
\begin{pgfscope}%
\pgfpathrectangle{\pgfqpoint{0.526905in}{0.383578in}}{\pgfqpoint{3.875000in}{2.310000in}}%
\pgfusepath{clip}%
\pgfsetbuttcap%
\pgfsetmiterjoin%
\definecolor{currentfill}{rgb}{0.333333,0.333333,0.333333}%
\pgfsetfillcolor{currentfill}%
\pgfsetlinewidth{0.000000pt}%
\definecolor{currentstroke}{rgb}{0.000000,0.000000,0.000000}%
\pgfsetstrokecolor{currentstroke}%
\pgfsetstrokeopacity{0.000000}%
\pgfsetdash{}{0pt}%
\pgfpathmoveto{\pgfqpoint{3.987278in}{0.383578in}}%
\pgfpathlineto{\pgfqpoint{4.014700in}{0.383578in}}%
\pgfpathlineto{\pgfqpoint{4.014700in}{0.417252in}}%
\pgfpathlineto{\pgfqpoint{3.987278in}{0.417252in}}%
\pgfpathclose%
\pgfusepath{fill}%
\end{pgfscope}%
\begin{pgfscope}%
\pgfpathrectangle{\pgfqpoint{0.526905in}{0.383578in}}{\pgfqpoint{3.875000in}{2.310000in}}%
\pgfusepath{clip}%
\pgfsetbuttcap%
\pgfsetmiterjoin%
\definecolor{currentfill}{rgb}{0.333333,0.333333,0.333333}%
\pgfsetfillcolor{currentfill}%
\pgfsetlinewidth{0.000000pt}%
\definecolor{currentstroke}{rgb}{0.000000,0.000000,0.000000}%
\pgfsetstrokecolor{currentstroke}%
\pgfsetstrokeopacity{0.000000}%
\pgfsetdash{}{0pt}%
\pgfpathmoveto{\pgfqpoint{4.004867in}{0.383578in}}%
\pgfpathlineto{\pgfqpoint{4.032289in}{0.383578in}}%
\pgfpathlineto{\pgfqpoint{4.032289in}{0.422864in}}%
\pgfpathlineto{\pgfqpoint{4.004867in}{0.422864in}}%
\pgfpathclose%
\pgfusepath{fill}%
\end{pgfscope}%
\begin{pgfscope}%
\pgfpathrectangle{\pgfqpoint{0.526905in}{0.383578in}}{\pgfqpoint{3.875000in}{2.310000in}}%
\pgfusepath{clip}%
\pgfsetbuttcap%
\pgfsetmiterjoin%
\definecolor{currentfill}{rgb}{0.333333,0.333333,0.333333}%
\pgfsetfillcolor{currentfill}%
\pgfsetlinewidth{0.000000pt}%
\definecolor{currentstroke}{rgb}{0.000000,0.000000,0.000000}%
\pgfsetstrokecolor{currentstroke}%
\pgfsetstrokeopacity{0.000000}%
\pgfsetdash{}{0pt}%
\pgfpathmoveto{\pgfqpoint{4.022456in}{0.383578in}}%
\pgfpathlineto{\pgfqpoint{4.049878in}{0.383578in}}%
\pgfpathlineto{\pgfqpoint{4.049878in}{0.428476in}}%
\pgfpathlineto{\pgfqpoint{4.022456in}{0.428476in}}%
\pgfpathclose%
\pgfusepath{fill}%
\end{pgfscope}%
\begin{pgfscope}%
\pgfpathrectangle{\pgfqpoint{0.526905in}{0.383578in}}{\pgfqpoint{3.875000in}{2.310000in}}%
\pgfusepath{clip}%
\pgfsetbuttcap%
\pgfsetmiterjoin%
\definecolor{currentfill}{rgb}{0.333333,0.333333,0.333333}%
\pgfsetfillcolor{currentfill}%
\pgfsetlinewidth{0.000000pt}%
\definecolor{currentstroke}{rgb}{0.000000,0.000000,0.000000}%
\pgfsetstrokecolor{currentstroke}%
\pgfsetstrokeopacity{0.000000}%
\pgfsetdash{}{0pt}%
\pgfpathmoveto{\pgfqpoint{4.040045in}{0.383578in}}%
\pgfpathlineto{\pgfqpoint{4.067467in}{0.383578in}}%
\pgfpathlineto{\pgfqpoint{4.067467in}{0.394803in}}%
\pgfpathlineto{\pgfqpoint{4.040045in}{0.394803in}}%
\pgfpathclose%
\pgfusepath{fill}%
\end{pgfscope}%
\begin{pgfscope}%
\pgfpathrectangle{\pgfqpoint{0.526905in}{0.383578in}}{\pgfqpoint{3.875000in}{2.310000in}}%
\pgfusepath{clip}%
\pgfsetbuttcap%
\pgfsetmiterjoin%
\definecolor{currentfill}{rgb}{0.333333,0.333333,0.333333}%
\pgfsetfillcolor{currentfill}%
\pgfsetlinewidth{0.000000pt}%
\definecolor{currentstroke}{rgb}{0.000000,0.000000,0.000000}%
\pgfsetstrokecolor{currentstroke}%
\pgfsetstrokeopacity{0.000000}%
\pgfsetdash{}{0pt}%
\pgfpathmoveto{\pgfqpoint{4.057634in}{0.383578in}}%
\pgfpathlineto{\pgfqpoint{4.085056in}{0.383578in}}%
\pgfpathlineto{\pgfqpoint{4.085056in}{0.400415in}}%
\pgfpathlineto{\pgfqpoint{4.057634in}{0.400415in}}%
\pgfpathclose%
\pgfusepath{fill}%
\end{pgfscope}%
\begin{pgfscope}%
\pgfpathrectangle{\pgfqpoint{0.526905in}{0.383578in}}{\pgfqpoint{3.875000in}{2.310000in}}%
\pgfusepath{clip}%
\pgfsetbuttcap%
\pgfsetmiterjoin%
\definecolor{currentfill}{rgb}{0.333333,0.333333,0.333333}%
\pgfsetfillcolor{currentfill}%
\pgfsetlinewidth{0.000000pt}%
\definecolor{currentstroke}{rgb}{0.000000,0.000000,0.000000}%
\pgfsetstrokecolor{currentstroke}%
\pgfsetstrokeopacity{0.000000}%
\pgfsetdash{}{0pt}%
\pgfpathmoveto{\pgfqpoint{4.075223in}{0.383578in}}%
\pgfpathlineto{\pgfqpoint{4.102645in}{0.383578in}}%
\pgfpathlineto{\pgfqpoint{4.102645in}{0.411639in}}%
\pgfpathlineto{\pgfqpoint{4.075223in}{0.411639in}}%
\pgfpathclose%
\pgfusepath{fill}%
\end{pgfscope}%
\begin{pgfscope}%
\pgfpathrectangle{\pgfqpoint{0.526905in}{0.383578in}}{\pgfqpoint{3.875000in}{2.310000in}}%
\pgfusepath{clip}%
\pgfsetbuttcap%
\pgfsetmiterjoin%
\definecolor{currentfill}{rgb}{0.333333,0.333333,0.333333}%
\pgfsetfillcolor{currentfill}%
\pgfsetlinewidth{0.000000pt}%
\definecolor{currentstroke}{rgb}{0.000000,0.000000,0.000000}%
\pgfsetstrokecolor{currentstroke}%
\pgfsetstrokeopacity{0.000000}%
\pgfsetdash{}{0pt}%
\pgfpathmoveto{\pgfqpoint{4.092812in}{0.383578in}}%
\pgfpathlineto{\pgfqpoint{4.120234in}{0.383578in}}%
\pgfpathlineto{\pgfqpoint{4.120234in}{0.400415in}}%
\pgfpathlineto{\pgfqpoint{4.092812in}{0.400415in}}%
\pgfpathclose%
\pgfusepath{fill}%
\end{pgfscope}%
\begin{pgfscope}%
\pgfpathrectangle{\pgfqpoint{0.526905in}{0.383578in}}{\pgfqpoint{3.875000in}{2.310000in}}%
\pgfusepath{clip}%
\pgfsetbuttcap%
\pgfsetmiterjoin%
\definecolor{currentfill}{rgb}{0.333333,0.333333,0.333333}%
\pgfsetfillcolor{currentfill}%
\pgfsetlinewidth{0.000000pt}%
\definecolor{currentstroke}{rgb}{0.000000,0.000000,0.000000}%
\pgfsetstrokecolor{currentstroke}%
\pgfsetstrokeopacity{0.000000}%
\pgfsetdash{}{0pt}%
\pgfpathmoveto{\pgfqpoint{4.110401in}{0.383578in}}%
\pgfpathlineto{\pgfqpoint{4.137823in}{0.383578in}}%
\pgfpathlineto{\pgfqpoint{4.137823in}{0.394803in}}%
\pgfpathlineto{\pgfqpoint{4.110401in}{0.394803in}}%
\pgfpathclose%
\pgfusepath{fill}%
\end{pgfscope}%
\begin{pgfscope}%
\pgfpathrectangle{\pgfqpoint{0.526905in}{0.383578in}}{\pgfqpoint{3.875000in}{2.310000in}}%
\pgfusepath{clip}%
\pgfsetbuttcap%
\pgfsetmiterjoin%
\definecolor{currentfill}{rgb}{0.333333,0.333333,0.333333}%
\pgfsetfillcolor{currentfill}%
\pgfsetlinewidth{0.000000pt}%
\definecolor{currentstroke}{rgb}{0.000000,0.000000,0.000000}%
\pgfsetstrokecolor{currentstroke}%
\pgfsetstrokeopacity{0.000000}%
\pgfsetdash{}{0pt}%
\pgfpathmoveto{\pgfqpoint{4.127990in}{0.383578in}}%
\pgfpathlineto{\pgfqpoint{4.155412in}{0.383578in}}%
\pgfpathlineto{\pgfqpoint{4.155412in}{0.394803in}}%
\pgfpathlineto{\pgfqpoint{4.127990in}{0.394803in}}%
\pgfpathclose%
\pgfusepath{fill}%
\end{pgfscope}%
\begin{pgfscope}%
\pgfpathrectangle{\pgfqpoint{0.526905in}{0.383578in}}{\pgfqpoint{3.875000in}{2.310000in}}%
\pgfusepath{clip}%
\pgfsetbuttcap%
\pgfsetmiterjoin%
\definecolor{currentfill}{rgb}{0.333333,0.333333,0.333333}%
\pgfsetfillcolor{currentfill}%
\pgfsetlinewidth{0.000000pt}%
\definecolor{currentstroke}{rgb}{0.000000,0.000000,0.000000}%
\pgfsetstrokecolor{currentstroke}%
\pgfsetstrokeopacity{0.000000}%
\pgfsetdash{}{0pt}%
\pgfpathmoveto{\pgfqpoint{4.145579in}{0.383578in}}%
\pgfpathlineto{\pgfqpoint{4.173001in}{0.383578in}}%
\pgfpathlineto{\pgfqpoint{4.173001in}{0.389190in}}%
\pgfpathlineto{\pgfqpoint{4.145579in}{0.389190in}}%
\pgfpathclose%
\pgfusepath{fill}%
\end{pgfscope}%
\begin{pgfscope}%
\pgfpathrectangle{\pgfqpoint{0.526905in}{0.383578in}}{\pgfqpoint{3.875000in}{2.310000in}}%
\pgfusepath{clip}%
\pgfsetbuttcap%
\pgfsetmiterjoin%
\definecolor{currentfill}{rgb}{0.333333,0.333333,0.333333}%
\pgfsetfillcolor{currentfill}%
\pgfsetlinewidth{0.000000pt}%
\definecolor{currentstroke}{rgb}{0.000000,0.000000,0.000000}%
\pgfsetstrokecolor{currentstroke}%
\pgfsetstrokeopacity{0.000000}%
\pgfsetdash{}{0pt}%
\pgfpathmoveto{\pgfqpoint{4.163168in}{0.383578in}}%
\pgfpathlineto{\pgfqpoint{4.190590in}{0.383578in}}%
\pgfpathlineto{\pgfqpoint{4.190590in}{0.383578in}}%
\pgfpathlineto{\pgfqpoint{4.163168in}{0.383578in}}%
\pgfpathclose%
\pgfusepath{fill}%
\end{pgfscope}%
\begin{pgfscope}%
\pgfpathrectangle{\pgfqpoint{0.526905in}{0.383578in}}{\pgfqpoint{3.875000in}{2.310000in}}%
\pgfusepath{clip}%
\pgfsetbuttcap%
\pgfsetmiterjoin%
\definecolor{currentfill}{rgb}{0.333333,0.333333,0.333333}%
\pgfsetfillcolor{currentfill}%
\pgfsetlinewidth{0.000000pt}%
\definecolor{currentstroke}{rgb}{0.000000,0.000000,0.000000}%
\pgfsetstrokecolor{currentstroke}%
\pgfsetstrokeopacity{0.000000}%
\pgfsetdash{}{0pt}%
\pgfpathmoveto{\pgfqpoint{4.180757in}{0.383578in}}%
\pgfpathlineto{\pgfqpoint{4.208179in}{0.383578in}}%
\pgfpathlineto{\pgfqpoint{4.208179in}{0.389190in}}%
\pgfpathlineto{\pgfqpoint{4.180757in}{0.389190in}}%
\pgfpathclose%
\pgfusepath{fill}%
\end{pgfscope}%
\begin{pgfscope}%
\pgfpathrectangle{\pgfqpoint{0.526905in}{0.383578in}}{\pgfqpoint{3.875000in}{2.310000in}}%
\pgfusepath{clip}%
\pgfsetbuttcap%
\pgfsetmiterjoin%
\definecolor{currentfill}{rgb}{0.333333,0.333333,0.333333}%
\pgfsetfillcolor{currentfill}%
\pgfsetlinewidth{0.000000pt}%
\definecolor{currentstroke}{rgb}{0.000000,0.000000,0.000000}%
\pgfsetstrokecolor{currentstroke}%
\pgfsetstrokeopacity{0.000000}%
\pgfsetdash{}{0pt}%
\pgfpathmoveto{\pgfqpoint{4.198346in}{0.383578in}}%
\pgfpathlineto{\pgfqpoint{4.225768in}{0.383578in}}%
\pgfpathlineto{\pgfqpoint{4.225768in}{0.389190in}}%
\pgfpathlineto{\pgfqpoint{4.198346in}{0.389190in}}%
\pgfpathclose%
\pgfusepath{fill}%
\end{pgfscope}%
\begin{pgfscope}%
\pgfpathrectangle{\pgfqpoint{0.526905in}{0.383578in}}{\pgfqpoint{3.875000in}{2.310000in}}%
\pgfusepath{clip}%
\pgfsetrectcap%
\pgfsetroundjoin%
\pgfsetlinewidth{0.803000pt}%
\definecolor{currentstroke}{rgb}{0.333333,0.333333,0.333333}%
\pgfsetstrokecolor{currentstroke}%
\pgfsetdash{}{0pt}%
\pgfpathmoveto{\pgfqpoint{0.711836in}{0.384563in}}%
\pgfpathlineto{\pgfqpoint{0.870137in}{0.386816in}}%
\pgfpathlineto{\pgfqpoint{0.975671in}{0.390348in}}%
\pgfpathlineto{\pgfqpoint{1.063617in}{0.395678in}}%
\pgfpathlineto{\pgfqpoint{1.133973in}{0.402405in}}%
\pgfpathlineto{\pgfqpoint{1.186740in}{0.409467in}}%
\pgfpathlineto{\pgfqpoint{1.239507in}{0.418782in}}%
\pgfpathlineto{\pgfqpoint{1.292274in}{0.430917in}}%
\pgfpathlineto{\pgfqpoint{1.327452in}{0.440893in}}%
\pgfpathlineto{\pgfqpoint{1.362631in}{0.452627in}}%
\pgfpathlineto{\pgfqpoint{1.397809in}{0.466352in}}%
\pgfpathlineto{\pgfqpoint{1.432987in}{0.482312in}}%
\pgfpathlineto{\pgfqpoint{1.468165in}{0.500765in}}%
\pgfpathlineto{\pgfqpoint{1.503343in}{0.521978in}}%
\pgfpathlineto{\pgfqpoint{1.538521in}{0.546220in}}%
\pgfpathlineto{\pgfqpoint{1.573699in}{0.573761in}}%
\pgfpathlineto{\pgfqpoint{1.608877in}{0.604863in}}%
\pgfpathlineto{\pgfqpoint{1.644055in}{0.639774in}}%
\pgfpathlineto{\pgfqpoint{1.679234in}{0.678723in}}%
\pgfpathlineto{\pgfqpoint{1.714412in}{0.721906in}}%
\pgfpathlineto{\pgfqpoint{1.749590in}{0.769485in}}%
\pgfpathlineto{\pgfqpoint{1.784768in}{0.821572in}}%
\pgfpathlineto{\pgfqpoint{1.819946in}{0.878224in}}%
\pgfpathlineto{\pgfqpoint{1.855124in}{0.939434in}}%
\pgfpathlineto{\pgfqpoint{1.890302in}{1.005122in}}%
\pgfpathlineto{\pgfqpoint{1.925480in}{1.075125in}}%
\pgfpathlineto{\pgfqpoint{1.960658in}{1.149198in}}%
\pgfpathlineto{\pgfqpoint{2.013426in}{1.267172in}}%
\pgfpathlineto{\pgfqpoint{2.066193in}{1.391982in}}%
\pgfpathlineto{\pgfqpoint{2.136549in}{1.565498in}}%
\pgfpathlineto{\pgfqpoint{2.259672in}{1.871149in}}%
\pgfpathlineto{\pgfqpoint{2.312439in}{1.994936in}}%
\pgfpathlineto{\pgfqpoint{2.347618in}{2.072604in}}%
\pgfpathlineto{\pgfqpoint{2.382796in}{2.145237in}}%
\pgfpathlineto{\pgfqpoint{2.417974in}{2.211883in}}%
\pgfpathlineto{\pgfqpoint{2.453152in}{2.271641in}}%
\pgfpathlineto{\pgfqpoint{2.488330in}{2.323684in}}%
\pgfpathlineto{\pgfqpoint{2.505919in}{2.346579in}}%
\pgfpathlineto{\pgfqpoint{2.523508in}{2.367277in}}%
\pgfpathlineto{\pgfqpoint{2.541097in}{2.385703in}}%
\pgfpathlineto{\pgfqpoint{2.558686in}{2.401791in}}%
\pgfpathlineto{\pgfqpoint{2.576275in}{2.415482in}}%
\pgfpathlineto{\pgfqpoint{2.593864in}{2.426725in}}%
\pgfpathlineto{\pgfqpoint{2.611453in}{2.435479in}}%
\pgfpathlineto{\pgfqpoint{2.629042in}{2.441711in}}%
\pgfpathlineto{\pgfqpoint{2.646632in}{2.445398in}}%
\pgfpathlineto{\pgfqpoint{2.664221in}{2.446526in}}%
\pgfpathlineto{\pgfqpoint{2.681810in}{2.445091in}}%
\pgfpathlineto{\pgfqpoint{2.699399in}{2.441099in}}%
\pgfpathlineto{\pgfqpoint{2.716988in}{2.434564in}}%
\pgfpathlineto{\pgfqpoint{2.734577in}{2.425511in}}%
\pgfpathlineto{\pgfqpoint{2.752166in}{2.413973in}}%
\pgfpathlineto{\pgfqpoint{2.769755in}{2.399993in}}%
\pgfpathlineto{\pgfqpoint{2.787344in}{2.383622in}}%
\pgfpathlineto{\pgfqpoint{2.804933in}{2.364920in}}%
\pgfpathlineto{\pgfqpoint{2.822522in}{2.343956in}}%
\pgfpathlineto{\pgfqpoint{2.840111in}{2.320804in}}%
\pgfpathlineto{\pgfqpoint{2.857700in}{2.295548in}}%
\pgfpathlineto{\pgfqpoint{2.892878in}{2.239089in}}%
\pgfpathlineto{\pgfqpoint{2.928056in}{2.175369in}}%
\pgfpathlineto{\pgfqpoint{2.963234in}{2.105257in}}%
\pgfpathlineto{\pgfqpoint{2.998413in}{2.029686in}}%
\pgfpathlineto{\pgfqpoint{3.033591in}{1.949627in}}%
\pgfpathlineto{\pgfqpoint{3.086358in}{1.823299in}}%
\pgfpathlineto{\pgfqpoint{3.174303in}{1.604270in}}%
\pgfpathlineto{\pgfqpoint{3.262248in}{1.386901in}}%
\pgfpathlineto{\pgfqpoint{3.315016in}{1.262328in}}%
\pgfpathlineto{\pgfqpoint{3.367783in}{1.144662in}}%
\pgfpathlineto{\pgfqpoint{3.402961in}{1.070824in}}%
\pgfpathlineto{\pgfqpoint{3.438139in}{1.001072in}}%
\pgfpathlineto{\pgfqpoint{3.473317in}{0.935648in}}%
\pgfpathlineto{\pgfqpoint{3.508495in}{0.874709in}}%
\pgfpathlineto{\pgfqpoint{3.543673in}{0.818330in}}%
\pgfpathlineto{\pgfqpoint{3.578851in}{0.766515in}}%
\pgfpathlineto{\pgfqpoint{3.614029in}{0.719203in}}%
\pgfpathlineto{\pgfqpoint{3.649208in}{0.676277in}}%
\pgfpathlineto{\pgfqpoint{3.684386in}{0.637576in}}%
\pgfpathlineto{\pgfqpoint{3.719564in}{0.602899in}}%
\pgfpathlineto{\pgfqpoint{3.754742in}{0.572017in}}%
\pgfpathlineto{\pgfqpoint{3.789920in}{0.544681in}}%
\pgfpathlineto{\pgfqpoint{3.825098in}{0.520628in}}%
\pgfpathlineto{\pgfqpoint{3.860276in}{0.499587in}}%
\pgfpathlineto{\pgfqpoint{3.895454in}{0.481290in}}%
\pgfpathlineto{\pgfqpoint{3.930632in}{0.465471in}}%
\pgfpathlineto{\pgfqpoint{3.965811in}{0.451872in}}%
\pgfpathlineto{\pgfqpoint{4.000989in}{0.440249in}}%
\pgfpathlineto{\pgfqpoint{4.036167in}{0.430371in}}%
\pgfpathlineto{\pgfqpoint{4.088934in}{0.418361in}}%
\pgfpathlineto{\pgfqpoint{4.141701in}{0.409146in}}%
\pgfpathlineto{\pgfqpoint{4.194468in}{0.402163in}}%
\pgfpathlineto{\pgfqpoint{4.212057in}{0.400247in}}%
\pgfpathlineto{\pgfqpoint{4.212057in}{0.400247in}}%
\pgfusepath{stroke}%
\end{pgfscope}%
\begin{pgfscope}%
\pgfpathrectangle{\pgfqpoint{0.526905in}{0.383578in}}{\pgfqpoint{3.875000in}{2.310000in}}%
\pgfusepath{clip}%
\pgfsetbuttcap%
\pgfsetroundjoin%
\pgfsetlinewidth{0.803000pt}%
\definecolor{currentstroke}{rgb}{0.333333,0.333333,0.333333}%
\pgfsetstrokecolor{currentstroke}%
\pgfsetdash{{2.960000pt}{1.280000pt}}{0.000000pt}%
\pgfpathmoveto{\pgfqpoint{2.663169in}{0.383578in}}%
\pgfpathlineto{\pgfqpoint{2.663169in}{2.693578in}}%
\pgfusepath{stroke}%
\end{pgfscope}%
\begin{pgfscope}%
\pgfpathrectangle{\pgfqpoint{0.526905in}{0.383578in}}{\pgfqpoint{3.875000in}{2.310000in}}%
\pgfusepath{clip}%
\pgfsetbuttcap%
\pgfsetroundjoin%
\pgfsetlinewidth{0.803000pt}%
\definecolor{currentstroke}{rgb}{1.000000,0.000000,0.000000}%
\pgfsetstrokecolor{currentstroke}%
\pgfsetdash{{2.960000pt}{1.280000pt}}{0.000000pt}%
\pgfpathmoveto{\pgfqpoint{3.332605in}{0.383578in}}%
\pgfpathlineto{\pgfqpoint{3.332605in}{2.693578in}}%
\pgfusepath{stroke}%
\end{pgfscope}%
\begin{pgfscope}%
\pgfsetrectcap%
\pgfsetmiterjoin%
\pgfsetlinewidth{0.501875pt}%
\definecolor{currentstroke}{rgb}{0.317647,0.317647,0.317647}%
\pgfsetstrokecolor{currentstroke}%
\pgfsetdash{}{0pt}%
\pgfpathmoveto{\pgfqpoint{0.526905in}{0.383578in}}%
\pgfpathlineto{\pgfqpoint{0.526905in}{2.693578in}}%
\pgfusepath{stroke}%
\end{pgfscope}%
\begin{pgfscope}%
\pgfsetrectcap%
\pgfsetmiterjoin%
\pgfsetlinewidth{0.501875pt}%
\definecolor{currentstroke}{rgb}{0.317647,0.317647,0.317647}%
\pgfsetstrokecolor{currentstroke}%
\pgfsetdash{}{0pt}%
\pgfpathmoveto{\pgfqpoint{0.526905in}{0.383578in}}%
\pgfpathlineto{\pgfqpoint{4.401905in}{0.383578in}}%
\pgfusepath{stroke}%
\end{pgfscope}%
\begin{pgfscope}%
\definecolor{textcolor}{rgb}{0.000000,0.000000,0.000000}%
\pgfsetstrokecolor{textcolor}%
\pgfsetfillcolor{textcolor}%
\pgftext[x=2.731725in,y=1.888981in,left,base]{\color{textcolor}\rmfamily\fontsize{8.000000}{9.600000}\selectfont \(\displaystyle V_{\mathrm{leak}}\)}%
\end{pgfscope}%
\begin{pgfscope}%
\definecolor{textcolor}{rgb}{0.000000,0.000000,0.000000}%
\pgfsetstrokecolor{textcolor}%
\pgfsetfillcolor{textcolor}%
\pgftext[x=3.401160in,y=1.888981in,left,base]{\color{textcolor}\rmfamily\fontsize{8.000000}{9.600000}\selectfont \(\displaystyle \vartheta\)}%
\end{pgfscope}%
\end{pgfpicture}%
\makeatother%
\endgroup%

	\end{center}
	\caption[Gaussian free membrane potential distribution on \gls{dls}.]{Gaussian free membrane potential distribution on \gls{dls}. The distribution of the membrane potential $f_{\gls{v_mem}}$ centers around \gls{v_leak}. The width of the distribution correlates to amount of injected noise spikes. Without additional noise, the induced spread from the intrinsic hardware noise is a magnitude lower. The part of the distribution that exceeds the threshold potential leads to spikes. The post fire dynamics of spiking activity changes the shape of the distribution, as the membrane is set to a reset potential before it leaks back to the resting potential (c.f. \citealp{petrovici12phdthesis}). Short time constants for the synaptic input and the membrane as well as sufficiently high input rates, reduce this effect.}
	\label{vleak_w_noise}
\end{figure}

The continuous stimulation with Poisson spike trains leads to a Gaussian free membrane potential distribution $f_{\gls{v_mem}}$ centered around the resting potential \gls{v_leak} as depicted in \cref{vleak_w_noise}. In a naive approach, the part of the distribution that exceeds a certain threshold potential correlates to the number of fired spikes. This neglects non vanishing effects from the fire dynamics of the membrane which have been investigated in more detail by \citealp{petrovici12phdthesis}. The impact of these dynamics can be reduced by the use of very short time constants for the synaptic input \gls{tau_syn} and the membrane \gls{tau_m}.

However, despite the strongly simplified picture, this view still offers a correct intuition how the threshold and leak potential as well as the strength of the noise effect the free membrane potential and in turn change the shape of the transfer function: more noise leads to a broader distribution and thus a more gently incline of the output rate; synaptic input moves the distribution to either to a lower or higher mean value; moving the threshold corresponds to an additional bias term.

The latter has been motivated by \citealp{petrovici2016stochastic}. The bias term in the neuron's activation can be adequately replaced by adapting the relative distance $\delta V$ between the resting potential and the threshold (\citealp{petrovici2016stochastic})
\begin{equation}
b \propto \delta V = \gls{v_leak} - \gls{thres}.
\end{equation}

With the approximations from above (\cref{fireratehigh} and \cref{fireratelow}) and a suitable choice of the neuron model parameters to create a well shaped distribution of the membrane (\cref{vleak_w_noise}), the transfer function yields an approximatively sigmoid shape (\cref{transferfunction}). The slope of the sigmoid can be easily adapted by changing the synaptic strength of the input spike trains. The synaptic weights of the noise inhibitory and excitatory noise inputs are left unchanged.


\begin{figure}
	\begin{center}
		%% Creator: Matplotlib, PGF backend
%%
%% To include the figure in your LaTeX document, write
%%   \input{<filename>.pgf}
%%
%% Make sure the required packages are loaded in your preamble
%%   \usepackage{pgf}
%%
%% Figures using additional raster images can only be included by \input if
%% they are in the same directory as the main LaTeX file. For loading figures
%% from other directories you can use the `import` package
%%   \usepackage{import}
%% and then include the figures with
%%   \import{<path to file>}{<filename>.pgf}
%%
%% Matplotlib used the following preamble
%%   \usepackage{amsmath} \usepackage{pifont} \usepackage{xcolor} \definecolor{green}{HTML}{467821} \definecolor{red}{HTML}{CF4457} \usepackage[detect-all]{siunitx}
%%   \usepackage{fontspec}
%%
\begingroup%
\makeatletter%
\begin{pgfpicture}%
\pgfpathrectangle{\pgfpointorigin}{\pgfqpoint{2.931438in}{2.408578in}}%
\pgfusepath{use as bounding box, clip}%
\begin{pgfscope}%
\pgfsetbuttcap%
\pgfsetmiterjoin%
\pgfsetlinewidth{0.000000pt}%
\definecolor{currentstroke}{rgb}{0.000000,0.000000,0.000000}%
\pgfsetstrokecolor{currentstroke}%
\pgfsetstrokeopacity{0.000000}%
\pgfsetdash{}{0pt}%
\pgfpathmoveto{\pgfqpoint{0.000000in}{0.000000in}}%
\pgfpathlineto{\pgfqpoint{2.931438in}{0.000000in}}%
\pgfpathlineto{\pgfqpoint{2.931438in}{2.408578in}}%
\pgfpathlineto{\pgfqpoint{0.000000in}{2.408578in}}%
\pgfpathclose%
\pgfusepath{}%
\end{pgfscope}%
\begin{pgfscope}%
\pgfsetbuttcap%
\pgfsetmiterjoin%
\pgfsetlinewidth{0.000000pt}%
\definecolor{currentstroke}{rgb}{0.000000,0.000000,0.000000}%
\pgfsetstrokecolor{currentstroke}%
\pgfsetstrokeopacity{0.000000}%
\pgfsetdash{}{0pt}%
\pgfpathmoveto{\pgfqpoint{0.453589in}{0.383578in}}%
\pgfpathlineto{\pgfqpoint{2.778589in}{0.383578in}}%
\pgfpathlineto{\pgfqpoint{2.778589in}{2.308578in}}%
\pgfpathlineto{\pgfqpoint{0.453589in}{2.308578in}}%
\pgfpathclose%
\pgfusepath{}%
\end{pgfscope}%
\begin{pgfscope}%
\pgfsetbuttcap%
\pgfsetroundjoin%
\definecolor{currentfill}{rgb}{0.317647,0.317647,0.317647}%
\pgfsetfillcolor{currentfill}%
\pgfsetlinewidth{0.501875pt}%
\definecolor{currentstroke}{rgb}{0.317647,0.317647,0.317647}%
\pgfsetstrokecolor{currentstroke}%
\pgfsetdash{}{0pt}%
\pgfsys@defobject{currentmarker}{\pgfqpoint{0.000000in}{-0.020833in}}{\pgfqpoint{0.000000in}{0.000000in}}{%
\pgfpathmoveto{\pgfqpoint{0.000000in}{0.000000in}}%
\pgfpathlineto{\pgfqpoint{0.000000in}{-0.020833in}}%
\pgfusepath{stroke,fill}%
}%
\begin{pgfscope}%
\pgfsys@transformshift{0.483784in}{0.383578in}%
\pgfsys@useobject{currentmarker}{}%
\end{pgfscope}%
\end{pgfscope}%
\begin{pgfscope}%
\definecolor{textcolor}{rgb}{0.317647,0.317647,0.317647}%
\pgfsetstrokecolor{textcolor}%
\pgfsetfillcolor{textcolor}%
\pgftext[x=0.483784in,y=0.334967in,,top]{\color{textcolor}\rmfamily\fontsize{6.664000}{7.996800}\selectfont \(\displaystyle -600\)}%
\end{pgfscope}%
\begin{pgfscope}%
\pgfsetbuttcap%
\pgfsetroundjoin%
\definecolor{currentfill}{rgb}{0.317647,0.317647,0.317647}%
\pgfsetfillcolor{currentfill}%
\pgfsetlinewidth{0.501875pt}%
\definecolor{currentstroke}{rgb}{0.317647,0.317647,0.317647}%
\pgfsetstrokecolor{currentstroke}%
\pgfsetdash{}{0pt}%
\pgfsys@defobject{currentmarker}{\pgfqpoint{0.000000in}{-0.020833in}}{\pgfqpoint{0.000000in}{0.000000in}}{%
\pgfpathmoveto{\pgfqpoint{0.000000in}{0.000000in}}%
\pgfpathlineto{\pgfqpoint{0.000000in}{-0.020833in}}%
\pgfusepath{stroke,fill}%
}%
\begin{pgfscope}%
\pgfsys@transformshift{0.861219in}{0.383578in}%
\pgfsys@useobject{currentmarker}{}%
\end{pgfscope}%
\end{pgfscope}%
\begin{pgfscope}%
\definecolor{textcolor}{rgb}{0.317647,0.317647,0.317647}%
\pgfsetstrokecolor{textcolor}%
\pgfsetfillcolor{textcolor}%
\pgftext[x=0.861219in,y=0.334967in,,top]{\color{textcolor}\rmfamily\fontsize{6.664000}{7.996800}\selectfont \(\displaystyle -400\)}%
\end{pgfscope}%
\begin{pgfscope}%
\pgfsetbuttcap%
\pgfsetroundjoin%
\definecolor{currentfill}{rgb}{0.317647,0.317647,0.317647}%
\pgfsetfillcolor{currentfill}%
\pgfsetlinewidth{0.501875pt}%
\definecolor{currentstroke}{rgb}{0.317647,0.317647,0.317647}%
\pgfsetstrokecolor{currentstroke}%
\pgfsetdash{}{0pt}%
\pgfsys@defobject{currentmarker}{\pgfqpoint{0.000000in}{-0.020833in}}{\pgfqpoint{0.000000in}{0.000000in}}{%
\pgfpathmoveto{\pgfqpoint{0.000000in}{0.000000in}}%
\pgfpathlineto{\pgfqpoint{0.000000in}{-0.020833in}}%
\pgfusepath{stroke,fill}%
}%
\begin{pgfscope}%
\pgfsys@transformshift{1.238654in}{0.383578in}%
\pgfsys@useobject{currentmarker}{}%
\end{pgfscope}%
\end{pgfscope}%
\begin{pgfscope}%
\definecolor{textcolor}{rgb}{0.317647,0.317647,0.317647}%
\pgfsetstrokecolor{textcolor}%
\pgfsetfillcolor{textcolor}%
\pgftext[x=1.238654in,y=0.334967in,,top]{\color{textcolor}\rmfamily\fontsize{6.664000}{7.996800}\selectfont \(\displaystyle -200\)}%
\end{pgfscope}%
\begin{pgfscope}%
\pgfsetbuttcap%
\pgfsetroundjoin%
\definecolor{currentfill}{rgb}{0.317647,0.317647,0.317647}%
\pgfsetfillcolor{currentfill}%
\pgfsetlinewidth{0.501875pt}%
\definecolor{currentstroke}{rgb}{0.317647,0.317647,0.317647}%
\pgfsetstrokecolor{currentstroke}%
\pgfsetdash{}{0pt}%
\pgfsys@defobject{currentmarker}{\pgfqpoint{0.000000in}{-0.020833in}}{\pgfqpoint{0.000000in}{0.000000in}}{%
\pgfpathmoveto{\pgfqpoint{0.000000in}{0.000000in}}%
\pgfpathlineto{\pgfqpoint{0.000000in}{-0.020833in}}%
\pgfusepath{stroke,fill}%
}%
\begin{pgfscope}%
\pgfsys@transformshift{1.616089in}{0.383578in}%
\pgfsys@useobject{currentmarker}{}%
\end{pgfscope}%
\end{pgfscope}%
\begin{pgfscope}%
\definecolor{textcolor}{rgb}{0.317647,0.317647,0.317647}%
\pgfsetstrokecolor{textcolor}%
\pgfsetfillcolor{textcolor}%
\pgftext[x=1.616089in,y=0.334967in,,top]{\color{textcolor}\rmfamily\fontsize{6.664000}{7.996800}\selectfont \(\displaystyle 0\)}%
\end{pgfscope}%
\begin{pgfscope}%
\pgfsetbuttcap%
\pgfsetroundjoin%
\definecolor{currentfill}{rgb}{0.317647,0.317647,0.317647}%
\pgfsetfillcolor{currentfill}%
\pgfsetlinewidth{0.501875pt}%
\definecolor{currentstroke}{rgb}{0.317647,0.317647,0.317647}%
\pgfsetstrokecolor{currentstroke}%
\pgfsetdash{}{0pt}%
\pgfsys@defobject{currentmarker}{\pgfqpoint{0.000000in}{-0.020833in}}{\pgfqpoint{0.000000in}{0.000000in}}{%
\pgfpathmoveto{\pgfqpoint{0.000000in}{0.000000in}}%
\pgfpathlineto{\pgfqpoint{0.000000in}{-0.020833in}}%
\pgfusepath{stroke,fill}%
}%
\begin{pgfscope}%
\pgfsys@transformshift{1.993524in}{0.383578in}%
\pgfsys@useobject{currentmarker}{}%
\end{pgfscope}%
\end{pgfscope}%
\begin{pgfscope}%
\definecolor{textcolor}{rgb}{0.317647,0.317647,0.317647}%
\pgfsetstrokecolor{textcolor}%
\pgfsetfillcolor{textcolor}%
\pgftext[x=1.993524in,y=0.334967in,,top]{\color{textcolor}\rmfamily\fontsize{6.664000}{7.996800}\selectfont \(\displaystyle 200\)}%
\end{pgfscope}%
\begin{pgfscope}%
\pgfsetbuttcap%
\pgfsetroundjoin%
\definecolor{currentfill}{rgb}{0.317647,0.317647,0.317647}%
\pgfsetfillcolor{currentfill}%
\pgfsetlinewidth{0.501875pt}%
\definecolor{currentstroke}{rgb}{0.317647,0.317647,0.317647}%
\pgfsetstrokecolor{currentstroke}%
\pgfsetdash{}{0pt}%
\pgfsys@defobject{currentmarker}{\pgfqpoint{0.000000in}{-0.020833in}}{\pgfqpoint{0.000000in}{0.000000in}}{%
\pgfpathmoveto{\pgfqpoint{0.000000in}{0.000000in}}%
\pgfpathlineto{\pgfqpoint{0.000000in}{-0.020833in}}%
\pgfusepath{stroke,fill}%
}%
\begin{pgfscope}%
\pgfsys@transformshift{2.370959in}{0.383578in}%
\pgfsys@useobject{currentmarker}{}%
\end{pgfscope}%
\end{pgfscope}%
\begin{pgfscope}%
\definecolor{textcolor}{rgb}{0.317647,0.317647,0.317647}%
\pgfsetstrokecolor{textcolor}%
\pgfsetfillcolor{textcolor}%
\pgftext[x=2.370959in,y=0.334967in,,top]{\color{textcolor}\rmfamily\fontsize{6.664000}{7.996800}\selectfont \(\displaystyle 400\)}%
\end{pgfscope}%
\begin{pgfscope}%
\pgfsetbuttcap%
\pgfsetroundjoin%
\definecolor{currentfill}{rgb}{0.317647,0.317647,0.317647}%
\pgfsetfillcolor{currentfill}%
\pgfsetlinewidth{0.501875pt}%
\definecolor{currentstroke}{rgb}{0.317647,0.317647,0.317647}%
\pgfsetstrokecolor{currentstroke}%
\pgfsetdash{}{0pt}%
\pgfsys@defobject{currentmarker}{\pgfqpoint{0.000000in}{-0.020833in}}{\pgfqpoint{0.000000in}{0.000000in}}{%
\pgfpathmoveto{\pgfqpoint{0.000000in}{0.000000in}}%
\pgfpathlineto{\pgfqpoint{0.000000in}{-0.020833in}}%
\pgfusepath{stroke,fill}%
}%
\begin{pgfscope}%
\pgfsys@transformshift{2.748394in}{0.383578in}%
\pgfsys@useobject{currentmarker}{}%
\end{pgfscope}%
\end{pgfscope}%
\begin{pgfscope}%
\definecolor{textcolor}{rgb}{0.317647,0.317647,0.317647}%
\pgfsetstrokecolor{textcolor}%
\pgfsetfillcolor{textcolor}%
\pgftext[x=2.748394in,y=0.334967in,,top]{\color{textcolor}\rmfamily\fontsize{6.664000}{7.996800}\selectfont \(\displaystyle 600\)}%
\end{pgfscope}%
\begin{pgfscope}%
\definecolor{textcolor}{rgb}{0.317647,0.317647,0.317647}%
\pgfsetstrokecolor{textcolor}%
\pgfsetfillcolor{textcolor}%
\pgftext[x=1.616089in,y=0.197222in,,top]{\color{textcolor}\rmfamily\fontsize{6.664000}{7.996800}\selectfont input frequency \(\displaystyle \nu_\mathrm{in} \; (\si{\kilo \Hz})\)}%
\end{pgfscope}%
\begin{pgfscope}%
\pgfsetbuttcap%
\pgfsetroundjoin%
\definecolor{currentfill}{rgb}{0.317647,0.317647,0.317647}%
\pgfsetfillcolor{currentfill}%
\pgfsetlinewidth{0.501875pt}%
\definecolor{currentstroke}{rgb}{0.317647,0.317647,0.317647}%
\pgfsetstrokecolor{currentstroke}%
\pgfsetdash{}{0pt}%
\pgfsys@defobject{currentmarker}{\pgfqpoint{-0.020833in}{0.000000in}}{\pgfqpoint{0.000000in}{0.000000in}}{%
\pgfpathmoveto{\pgfqpoint{0.000000in}{0.000000in}}%
\pgfpathlineto{\pgfqpoint{-0.020833in}{0.000000in}}%
\pgfusepath{stroke,fill}%
}%
\begin{pgfscope}%
\pgfsys@transformshift{0.453589in}{0.451840in}%
\pgfsys@useobject{currentmarker}{}%
\end{pgfscope}%
\end{pgfscope}%
\begin{pgfscope}%
\definecolor{textcolor}{rgb}{0.317647,0.317647,0.317647}%
\pgfsetstrokecolor{textcolor}%
\pgfsetfillcolor{textcolor}%
\pgftext[x=0.363504in,y=0.419724in,left,base]{\color{textcolor}\rmfamily\fontsize{6.664000}{7.996800}\selectfont \(\displaystyle 0\)}%
\end{pgfscope}%
\begin{pgfscope}%
\pgfsetbuttcap%
\pgfsetroundjoin%
\definecolor{currentfill}{rgb}{0.317647,0.317647,0.317647}%
\pgfsetfillcolor{currentfill}%
\pgfsetlinewidth{0.501875pt}%
\definecolor{currentstroke}{rgb}{0.317647,0.317647,0.317647}%
\pgfsetstrokecolor{currentstroke}%
\pgfsetdash{}{0pt}%
\pgfsys@defobject{currentmarker}{\pgfqpoint{-0.020833in}{0.000000in}}{\pgfqpoint{0.000000in}{0.000000in}}{%
\pgfpathmoveto{\pgfqpoint{0.000000in}{0.000000in}}%
\pgfpathlineto{\pgfqpoint{-0.020833in}{0.000000in}}%
\pgfusepath{stroke,fill}%
}%
\begin{pgfscope}%
\pgfsys@transformshift{0.453589in}{0.724890in}%
\pgfsys@useobject{currentmarker}{}%
\end{pgfscope}%
\end{pgfscope}%
\begin{pgfscope}%
\definecolor{textcolor}{rgb}{0.317647,0.317647,0.317647}%
\pgfsetstrokecolor{textcolor}%
\pgfsetfillcolor{textcolor}%
\pgftext[x=0.308141in,y=0.692773in,left,base]{\color{textcolor}\rmfamily\fontsize{6.664000}{7.996800}\selectfont \(\displaystyle 20\)}%
\end{pgfscope}%
\begin{pgfscope}%
\pgfsetbuttcap%
\pgfsetroundjoin%
\definecolor{currentfill}{rgb}{0.317647,0.317647,0.317647}%
\pgfsetfillcolor{currentfill}%
\pgfsetlinewidth{0.501875pt}%
\definecolor{currentstroke}{rgb}{0.317647,0.317647,0.317647}%
\pgfsetstrokecolor{currentstroke}%
\pgfsetdash{}{0pt}%
\pgfsys@defobject{currentmarker}{\pgfqpoint{-0.020833in}{0.000000in}}{\pgfqpoint{0.000000in}{0.000000in}}{%
\pgfpathmoveto{\pgfqpoint{0.000000in}{0.000000in}}%
\pgfpathlineto{\pgfqpoint{-0.020833in}{0.000000in}}%
\pgfusepath{stroke,fill}%
}%
\begin{pgfscope}%
\pgfsys@transformshift{0.453589in}{0.997940in}%
\pgfsys@useobject{currentmarker}{}%
\end{pgfscope}%
\end{pgfscope}%
\begin{pgfscope}%
\definecolor{textcolor}{rgb}{0.317647,0.317647,0.317647}%
\pgfsetstrokecolor{textcolor}%
\pgfsetfillcolor{textcolor}%
\pgftext[x=0.308141in,y=0.965823in,left,base]{\color{textcolor}\rmfamily\fontsize{6.664000}{7.996800}\selectfont \(\displaystyle 40\)}%
\end{pgfscope}%
\begin{pgfscope}%
\pgfsetbuttcap%
\pgfsetroundjoin%
\definecolor{currentfill}{rgb}{0.317647,0.317647,0.317647}%
\pgfsetfillcolor{currentfill}%
\pgfsetlinewidth{0.501875pt}%
\definecolor{currentstroke}{rgb}{0.317647,0.317647,0.317647}%
\pgfsetstrokecolor{currentstroke}%
\pgfsetdash{}{0pt}%
\pgfsys@defobject{currentmarker}{\pgfqpoint{-0.020833in}{0.000000in}}{\pgfqpoint{0.000000in}{0.000000in}}{%
\pgfpathmoveto{\pgfqpoint{0.000000in}{0.000000in}}%
\pgfpathlineto{\pgfqpoint{-0.020833in}{0.000000in}}%
\pgfusepath{stroke,fill}%
}%
\begin{pgfscope}%
\pgfsys@transformshift{0.453589in}{1.270989in}%
\pgfsys@useobject{currentmarker}{}%
\end{pgfscope}%
\end{pgfscope}%
\begin{pgfscope}%
\definecolor{textcolor}{rgb}{0.317647,0.317647,0.317647}%
\pgfsetstrokecolor{textcolor}%
\pgfsetfillcolor{textcolor}%
\pgftext[x=0.308141in,y=1.238873in,left,base]{\color{textcolor}\rmfamily\fontsize{6.664000}{7.996800}\selectfont \(\displaystyle 60\)}%
\end{pgfscope}%
\begin{pgfscope}%
\pgfsetbuttcap%
\pgfsetroundjoin%
\definecolor{currentfill}{rgb}{0.317647,0.317647,0.317647}%
\pgfsetfillcolor{currentfill}%
\pgfsetlinewidth{0.501875pt}%
\definecolor{currentstroke}{rgb}{0.317647,0.317647,0.317647}%
\pgfsetstrokecolor{currentstroke}%
\pgfsetdash{}{0pt}%
\pgfsys@defobject{currentmarker}{\pgfqpoint{-0.020833in}{0.000000in}}{\pgfqpoint{0.000000in}{0.000000in}}{%
\pgfpathmoveto{\pgfqpoint{0.000000in}{0.000000in}}%
\pgfpathlineto{\pgfqpoint{-0.020833in}{0.000000in}}%
\pgfusepath{stroke,fill}%
}%
\begin{pgfscope}%
\pgfsys@transformshift{0.453589in}{1.544039in}%
\pgfsys@useobject{currentmarker}{}%
\end{pgfscope}%
\end{pgfscope}%
\begin{pgfscope}%
\definecolor{textcolor}{rgb}{0.317647,0.317647,0.317647}%
\pgfsetstrokecolor{textcolor}%
\pgfsetfillcolor{textcolor}%
\pgftext[x=0.308141in,y=1.511922in,left,base]{\color{textcolor}\rmfamily\fontsize{6.664000}{7.996800}\selectfont \(\displaystyle 80\)}%
\end{pgfscope}%
\begin{pgfscope}%
\pgfsetbuttcap%
\pgfsetroundjoin%
\definecolor{currentfill}{rgb}{0.317647,0.317647,0.317647}%
\pgfsetfillcolor{currentfill}%
\pgfsetlinewidth{0.501875pt}%
\definecolor{currentstroke}{rgb}{0.317647,0.317647,0.317647}%
\pgfsetstrokecolor{currentstroke}%
\pgfsetdash{}{0pt}%
\pgfsys@defobject{currentmarker}{\pgfqpoint{-0.020833in}{0.000000in}}{\pgfqpoint{0.000000in}{0.000000in}}{%
\pgfpathmoveto{\pgfqpoint{0.000000in}{0.000000in}}%
\pgfpathlineto{\pgfqpoint{-0.020833in}{0.000000in}}%
\pgfusepath{stroke,fill}%
}%
\begin{pgfscope}%
\pgfsys@transformshift{0.453589in}{1.817089in}%
\pgfsys@useobject{currentmarker}{}%
\end{pgfscope}%
\end{pgfscope}%
\begin{pgfscope}%
\definecolor{textcolor}{rgb}{0.317647,0.317647,0.317647}%
\pgfsetstrokecolor{textcolor}%
\pgfsetfillcolor{textcolor}%
\pgftext[x=0.252778in,y=1.784972in,left,base]{\color{textcolor}\rmfamily\fontsize{6.664000}{7.996800}\selectfont \(\displaystyle 100\)}%
\end{pgfscope}%
\begin{pgfscope}%
\pgfsetbuttcap%
\pgfsetroundjoin%
\definecolor{currentfill}{rgb}{0.317647,0.317647,0.317647}%
\pgfsetfillcolor{currentfill}%
\pgfsetlinewidth{0.501875pt}%
\definecolor{currentstroke}{rgb}{0.317647,0.317647,0.317647}%
\pgfsetstrokecolor{currentstroke}%
\pgfsetdash{}{0pt}%
\pgfsys@defobject{currentmarker}{\pgfqpoint{-0.020833in}{0.000000in}}{\pgfqpoint{0.000000in}{0.000000in}}{%
\pgfpathmoveto{\pgfqpoint{0.000000in}{0.000000in}}%
\pgfpathlineto{\pgfqpoint{-0.020833in}{0.000000in}}%
\pgfusepath{stroke,fill}%
}%
\begin{pgfscope}%
\pgfsys@transformshift{0.453589in}{2.090138in}%
\pgfsys@useobject{currentmarker}{}%
\end{pgfscope}%
\end{pgfscope}%
\begin{pgfscope}%
\definecolor{textcolor}{rgb}{0.317647,0.317647,0.317647}%
\pgfsetstrokecolor{textcolor}%
\pgfsetfillcolor{textcolor}%
\pgftext[x=0.252778in,y=2.058022in,left,base]{\color{textcolor}\rmfamily\fontsize{6.664000}{7.996800}\selectfont \(\displaystyle 120\)}%
\end{pgfscope}%
\begin{pgfscope}%
\definecolor{textcolor}{rgb}{0.317647,0.317647,0.317647}%
\pgfsetstrokecolor{textcolor}%
\pgfsetfillcolor{textcolor}%
\pgftext[x=0.197222in,y=1.346078in,,bottom,rotate=90.000000]{\color{textcolor}\rmfamily\fontsize{6.664000}{7.996800}\selectfont output frequency \(\displaystyle \nu_\mathrm{out} \; (\si{\kilo \Hz})\)}%
\end{pgfscope}%
\begin{pgfscope}%
\pgfpathrectangle{\pgfqpoint{0.453589in}{0.383578in}}{\pgfqpoint{2.325000in}{1.925000in}}%
\pgfusepath{clip}%
\pgfsetbuttcap%
\pgfsetroundjoin%
\definecolor{currentfill}{rgb}{0.333333,0.333333,0.333333}%
\pgfsetfillcolor{currentfill}%
\pgfsetlinewidth{1.003750pt}%
\definecolor{currentstroke}{rgb}{0.333333,0.333333,0.333333}%
\pgfsetstrokecolor{currentstroke}%
\pgfsetdash{}{0pt}%
\pgfsys@defobject{currentmarker}{\pgfqpoint{-0.010417in}{-0.010417in}}{\pgfqpoint{0.010417in}{0.010417in}}{%
\pgfpathmoveto{\pgfqpoint{0.000000in}{-0.010417in}}%
\pgfpathcurveto{\pgfqpoint{0.002763in}{-0.010417in}}{\pgfqpoint{0.005412in}{-0.009319in}}{\pgfqpoint{0.007366in}{-0.007366in}}%
\pgfpathcurveto{\pgfqpoint{0.009319in}{-0.005412in}}{\pgfqpoint{0.010417in}{-0.002763in}}{\pgfqpoint{0.010417in}{0.000000in}}%
\pgfpathcurveto{\pgfqpoint{0.010417in}{0.002763in}}{\pgfqpoint{0.009319in}{0.005412in}}{\pgfqpoint{0.007366in}{0.007366in}}%
\pgfpathcurveto{\pgfqpoint{0.005412in}{0.009319in}}{\pgfqpoint{0.002763in}{0.010417in}}{\pgfqpoint{0.000000in}{0.010417in}}%
\pgfpathcurveto{\pgfqpoint{-0.002763in}{0.010417in}}{\pgfqpoint{-0.005412in}{0.009319in}}{\pgfqpoint{-0.007366in}{0.007366in}}%
\pgfpathcurveto{\pgfqpoint{-0.009319in}{0.005412in}}{\pgfqpoint{-0.010417in}{0.002763in}}{\pgfqpoint{-0.010417in}{0.000000in}}%
\pgfpathcurveto{\pgfqpoint{-0.010417in}{-0.002763in}}{\pgfqpoint{-0.009319in}{-0.005412in}}{\pgfqpoint{-0.007366in}{-0.007366in}}%
\pgfpathcurveto{\pgfqpoint{-0.005412in}{-0.009319in}}{\pgfqpoint{-0.002763in}{-0.010417in}}{\pgfqpoint{0.000000in}{-0.010417in}}%
\pgfpathclose%
\pgfusepath{stroke,fill}%
}%
\begin{pgfscope}%
\pgfsys@transformshift{0.559271in}{0.463712in}%
\pgfsys@useobject{currentmarker}{}%
\end{pgfscope}%
\begin{pgfscope}%
\pgfsys@transformshift{0.617983in}{0.457776in}%
\pgfsys@useobject{currentmarker}{}%
\end{pgfscope}%
\begin{pgfscope}%
\pgfsys@transformshift{0.676695in}{0.457776in}%
\pgfsys@useobject{currentmarker}{}%
\end{pgfscope}%
\begin{pgfscope}%
\pgfsys@transformshift{0.735407in}{0.463712in}%
\pgfsys@useobject{currentmarker}{}%
\end{pgfscope}%
\begin{pgfscope}%
\pgfsys@transformshift{0.794119in}{0.469648in}%
\pgfsys@useobject{currentmarker}{}%
\end{pgfscope}%
\begin{pgfscope}%
\pgfsys@transformshift{0.852831in}{0.475584in}%
\pgfsys@useobject{currentmarker}{}%
\end{pgfscope}%
\begin{pgfscope}%
\pgfsys@transformshift{0.911543in}{0.499327in}%
\pgfsys@useobject{currentmarker}{}%
\end{pgfscope}%
\begin{pgfscope}%
\pgfsys@transformshift{0.970255in}{0.499327in}%
\pgfsys@useobject{currentmarker}{}%
\end{pgfscope}%
\begin{pgfscope}%
\pgfsys@transformshift{1.028968in}{0.511199in}%
\pgfsys@useobject{currentmarker}{}%
\end{pgfscope}%
\begin{pgfscope}%
\pgfsys@transformshift{1.087680in}{0.511199in}%
\pgfsys@useobject{currentmarker}{}%
\end{pgfscope}%
\begin{pgfscope}%
\pgfsys@transformshift{1.146392in}{0.564622in}%
\pgfsys@useobject{currentmarker}{}%
\end{pgfscope}%
\begin{pgfscope}%
\pgfsys@transformshift{1.205104in}{0.576494in}%
\pgfsys@useobject{currentmarker}{}%
\end{pgfscope}%
\begin{pgfscope}%
\pgfsys@transformshift{1.263816in}{0.606173in}%
\pgfsys@useobject{currentmarker}{}%
\end{pgfscope}%
\begin{pgfscope}%
\pgfsys@transformshift{1.322528in}{0.695211in}%
\pgfsys@useobject{currentmarker}{}%
\end{pgfscope}%
\begin{pgfscope}%
\pgfsys@transformshift{1.381240in}{0.689275in}%
\pgfsys@useobject{currentmarker}{}%
\end{pgfscope}%
\begin{pgfscope}%
\pgfsys@transformshift{1.439952in}{0.707083in}%
\pgfsys@useobject{currentmarker}{}%
\end{pgfscope}%
\begin{pgfscope}%
\pgfsys@transformshift{1.498665in}{0.784249in}%
\pgfsys@useobject{currentmarker}{}%
\end{pgfscope}%
\begin{pgfscope}%
\pgfsys@transformshift{1.557377in}{0.885158in}%
\pgfsys@useobject{currentmarker}{}%
\end{pgfscope}%
\begin{pgfscope}%
\pgfsys@transformshift{1.616089in}{0.897030in}%
\pgfsys@useobject{currentmarker}{}%
\end{pgfscope}%
\begin{pgfscope}%
\pgfsys@transformshift{1.674801in}{1.021683in}%
\pgfsys@useobject{currentmarker}{}%
\end{pgfscope}%
\begin{pgfscope}%
\pgfsys@transformshift{1.733513in}{0.968260in}%
\pgfsys@useobject{currentmarker}{}%
\end{pgfscope}%
\begin{pgfscope}%
\pgfsys@transformshift{1.792225in}{1.134465in}%
\pgfsys@useobject{currentmarker}{}%
\end{pgfscope}%
\begin{pgfscope}%
\pgfsys@transformshift{1.850937in}{1.294733in}%
\pgfsys@useobject{currentmarker}{}%
\end{pgfscope}%
\begin{pgfscope}%
\pgfsys@transformshift{1.909649in}{1.443129in}%
\pgfsys@useobject{currentmarker}{}%
\end{pgfscope}%
\begin{pgfscope}%
\pgfsys@transformshift{1.968361in}{1.401578in}%
\pgfsys@useobject{currentmarker}{}%
\end{pgfscope}%
\begin{pgfscope}%
\pgfsys@transformshift{2.027074in}{1.478745in}%
\pgfsys@useobject{currentmarker}{}%
\end{pgfscope}%
\begin{pgfscope}%
\pgfsys@transformshift{2.085786in}{1.603398in}%
\pgfsys@useobject{currentmarker}{}%
\end{pgfscope}%
\begin{pgfscope}%
\pgfsys@transformshift{2.144498in}{1.674628in}%
\pgfsys@useobject{currentmarker}{}%
\end{pgfscope}%
\begin{pgfscope}%
\pgfsys@transformshift{2.203210in}{1.692436in}%
\pgfsys@useobject{currentmarker}{}%
\end{pgfscope}%
\begin{pgfscope}%
\pgfsys@transformshift{2.261922in}{1.745858in}%
\pgfsys@useobject{currentmarker}{}%
\end{pgfscope}%
\begin{pgfscope}%
\pgfsys@transformshift{2.320634in}{1.757730in}%
\pgfsys@useobject{currentmarker}{}%
\end{pgfscope}%
\begin{pgfscope}%
\pgfsys@transformshift{2.379346in}{1.739922in}%
\pgfsys@useobject{currentmarker}{}%
\end{pgfscope}%
\begin{pgfscope}%
\pgfsys@transformshift{2.438058in}{1.769602in}%
\pgfsys@useobject{currentmarker}{}%
\end{pgfscope}%
\begin{pgfscope}%
\pgfsys@transformshift{2.496771in}{1.781474in}%
\pgfsys@useobject{currentmarker}{}%
\end{pgfscope}%
\begin{pgfscope}%
\pgfsys@transformshift{2.555483in}{1.775538in}%
\pgfsys@useobject{currentmarker}{}%
\end{pgfscope}%
\begin{pgfscope}%
\pgfsys@transformshift{2.614195in}{1.787409in}%
\pgfsys@useobject{currentmarker}{}%
\end{pgfscope}%
\begin{pgfscope}%
\pgfsys@transformshift{2.672907in}{1.817089in}%
\pgfsys@useobject{currentmarker}{}%
\end{pgfscope}%
\end{pgfscope}%
\begin{pgfscope}%
\pgfpathrectangle{\pgfqpoint{0.453589in}{0.383578in}}{\pgfqpoint{2.325000in}{1.925000in}}%
\pgfusepath{clip}%
\pgfsetrectcap%
\pgfsetroundjoin%
\pgfsetlinewidth{0.803000pt}%
\definecolor{currentstroke}{rgb}{0.333333,0.333333,0.333333}%
\pgfsetstrokecolor{currentstroke}%
\pgfsetdash{}{0pt}%
\pgfpathmoveto{\pgfqpoint{0.559271in}{0.459024in}}%
\pgfpathlineto{\pgfqpoint{0.617983in}{0.461102in}}%
\pgfpathlineto{\pgfqpoint{0.676695in}{0.463775in}}%
\pgfpathlineto{\pgfqpoint{0.735407in}{0.467212in}}%
\pgfpathlineto{\pgfqpoint{0.794119in}{0.471624in}}%
\pgfpathlineto{\pgfqpoint{0.852831in}{0.477279in}}%
\pgfpathlineto{\pgfqpoint{0.911543in}{0.484512in}}%
\pgfpathlineto{\pgfqpoint{0.970255in}{0.493739in}}%
\pgfpathlineto{\pgfqpoint{1.028968in}{0.505470in}}%
\pgfpathlineto{\pgfqpoint{1.087680in}{0.520320in}}%
\pgfpathlineto{\pgfqpoint{1.146392in}{0.539015in}}%
\pgfpathlineto{\pgfqpoint{1.205104in}{0.562391in}}%
\pgfpathlineto{\pgfqpoint{1.263816in}{0.591367in}}%
\pgfpathlineto{\pgfqpoint{1.322528in}{0.626907in}}%
\pgfpathlineto{\pgfqpoint{1.381240in}{0.669932in}}%
\pgfpathlineto{\pgfqpoint{1.439952in}{0.721205in}}%
\pgfpathlineto{\pgfqpoint{1.498665in}{0.781170in}}%
\pgfpathlineto{\pgfqpoint{1.557377in}{0.849783in}}%
\pgfpathlineto{\pgfqpoint{1.616089in}{0.926351in}}%
\pgfpathlineto{\pgfqpoint{1.674801in}{1.009446in}}%
\pgfpathlineto{\pgfqpoint{1.733513in}{1.096942in}}%
\pgfpathlineto{\pgfqpoint{1.792225in}{1.186187in}}%
\pgfpathlineto{\pgfqpoint{1.850937in}{1.274312in}}%
\pgfpathlineto{\pgfqpoint{1.909649in}{1.358590in}}%
\pgfpathlineto{\pgfqpoint{1.968361in}{1.436756in}}%
\pgfpathlineto{\pgfqpoint{2.027074in}{1.507219in}}%
\pgfpathlineto{\pgfqpoint{2.085786in}{1.569128in}}%
\pgfpathlineto{\pgfqpoint{2.144498in}{1.622305in}}%
\pgfpathlineto{\pgfqpoint{2.203210in}{1.667103in}}%
\pgfpathlineto{\pgfqpoint{2.261922in}{1.704227in}}%
\pgfpathlineto{\pgfqpoint{2.320634in}{1.734577in}}%
\pgfpathlineto{\pgfqpoint{2.379346in}{1.759113in}}%
\pgfpathlineto{\pgfqpoint{2.438058in}{1.778772in}}%
\pgfpathlineto{\pgfqpoint{2.496771in}{1.794408in}}%
\pgfpathlineto{\pgfqpoint{2.555483in}{1.806774in}}%
\pgfpathlineto{\pgfqpoint{2.614195in}{1.816509in}}%
\pgfpathlineto{\pgfqpoint{2.672907in}{1.824146in}}%
\pgfusepath{stroke}%
\end{pgfscope}%
\begin{pgfscope}%
\pgfpathrectangle{\pgfqpoint{0.453589in}{0.383578in}}{\pgfqpoint{2.325000in}{1.925000in}}%
\pgfusepath{clip}%
\pgfsetbuttcap%
\pgfsetroundjoin%
\definecolor{currentfill}{rgb}{0.686275,0.352941,0.313725}%
\pgfsetfillcolor{currentfill}%
\pgfsetlinewidth{1.003750pt}%
\definecolor{currentstroke}{rgb}{0.686275,0.352941,0.313725}%
\pgfsetstrokecolor{currentstroke}%
\pgfsetdash{}{0pt}%
\pgfsys@defobject{currentmarker}{\pgfqpoint{-0.010417in}{-0.010417in}}{\pgfqpoint{0.010417in}{0.010417in}}{%
\pgfpathmoveto{\pgfqpoint{0.000000in}{-0.010417in}}%
\pgfpathcurveto{\pgfqpoint{0.002763in}{-0.010417in}}{\pgfqpoint{0.005412in}{-0.009319in}}{\pgfqpoint{0.007366in}{-0.007366in}}%
\pgfpathcurveto{\pgfqpoint{0.009319in}{-0.005412in}}{\pgfqpoint{0.010417in}{-0.002763in}}{\pgfqpoint{0.010417in}{0.000000in}}%
\pgfpathcurveto{\pgfqpoint{0.010417in}{0.002763in}}{\pgfqpoint{0.009319in}{0.005412in}}{\pgfqpoint{0.007366in}{0.007366in}}%
\pgfpathcurveto{\pgfqpoint{0.005412in}{0.009319in}}{\pgfqpoint{0.002763in}{0.010417in}}{\pgfqpoint{0.000000in}{0.010417in}}%
\pgfpathcurveto{\pgfqpoint{-0.002763in}{0.010417in}}{\pgfqpoint{-0.005412in}{0.009319in}}{\pgfqpoint{-0.007366in}{0.007366in}}%
\pgfpathcurveto{\pgfqpoint{-0.009319in}{0.005412in}}{\pgfqpoint{-0.010417in}{0.002763in}}{\pgfqpoint{-0.010417in}{0.000000in}}%
\pgfpathcurveto{\pgfqpoint{-0.010417in}{-0.002763in}}{\pgfqpoint{-0.009319in}{-0.005412in}}{\pgfqpoint{-0.007366in}{-0.007366in}}%
\pgfpathcurveto{\pgfqpoint{-0.005412in}{-0.009319in}}{\pgfqpoint{-0.002763in}{-0.010417in}}{\pgfqpoint{0.000000in}{-0.010417in}}%
\pgfpathclose%
\pgfusepath{stroke,fill}%
}%
\begin{pgfscope}%
\pgfsys@transformshift{0.559271in}{0.451840in}%
\pgfsys@useobject{currentmarker}{}%
\end{pgfscope}%
\begin{pgfscope}%
\pgfsys@transformshift{0.617983in}{0.451840in}%
\pgfsys@useobject{currentmarker}{}%
\end{pgfscope}%
\begin{pgfscope}%
\pgfsys@transformshift{0.676695in}{0.451840in}%
\pgfsys@useobject{currentmarker}{}%
\end{pgfscope}%
\begin{pgfscope}%
\pgfsys@transformshift{0.735407in}{0.451840in}%
\pgfsys@useobject{currentmarker}{}%
\end{pgfscope}%
\begin{pgfscope}%
\pgfsys@transformshift{0.794119in}{0.451840in}%
\pgfsys@useobject{currentmarker}{}%
\end{pgfscope}%
\begin{pgfscope}%
\pgfsys@transformshift{0.852831in}{0.451840in}%
\pgfsys@useobject{currentmarker}{}%
\end{pgfscope}%
\begin{pgfscope}%
\pgfsys@transformshift{0.911543in}{0.451840in}%
\pgfsys@useobject{currentmarker}{}%
\end{pgfscope}%
\begin{pgfscope}%
\pgfsys@transformshift{0.970255in}{0.451840in}%
\pgfsys@useobject{currentmarker}{}%
\end{pgfscope}%
\begin{pgfscope}%
\pgfsys@transformshift{1.028968in}{0.451840in}%
\pgfsys@useobject{currentmarker}{}%
\end{pgfscope}%
\begin{pgfscope}%
\pgfsys@transformshift{1.087680in}{0.451840in}%
\pgfsys@useobject{currentmarker}{}%
\end{pgfscope}%
\begin{pgfscope}%
\pgfsys@transformshift{1.146392in}{0.451840in}%
\pgfsys@useobject{currentmarker}{}%
\end{pgfscope}%
\begin{pgfscope}%
\pgfsys@transformshift{1.205104in}{0.457776in}%
\pgfsys@useobject{currentmarker}{}%
\end{pgfscope}%
\begin{pgfscope}%
\pgfsys@transformshift{1.263816in}{0.463712in}%
\pgfsys@useobject{currentmarker}{}%
\end{pgfscope}%
\begin{pgfscope}%
\pgfsys@transformshift{1.322528in}{0.493391in}%
\pgfsys@useobject{currentmarker}{}%
\end{pgfscope}%
\begin{pgfscope}%
\pgfsys@transformshift{1.381240in}{0.523071in}%
\pgfsys@useobject{currentmarker}{}%
\end{pgfscope}%
\begin{pgfscope}%
\pgfsys@transformshift{1.439952in}{0.534943in}%
\pgfsys@useobject{currentmarker}{}%
\end{pgfscope}%
\begin{pgfscope}%
\pgfsys@transformshift{1.498665in}{0.588365in}%
\pgfsys@useobject{currentmarker}{}%
\end{pgfscope}%
\begin{pgfscope}%
\pgfsys@transformshift{1.557377in}{0.689275in}%
\pgfsys@useobject{currentmarker}{}%
\end{pgfscope}%
\begin{pgfscope}%
\pgfsys@transformshift{1.616089in}{0.879223in}%
\pgfsys@useobject{currentmarker}{}%
\end{pgfscope}%
\begin{pgfscope}%
\pgfsys@transformshift{1.674801in}{1.223503in}%
\pgfsys@useobject{currentmarker}{}%
\end{pgfscope}%
\begin{pgfscope}%
\pgfsys@transformshift{1.733513in}{1.449065in}%
\pgfsys@useobject{currentmarker}{}%
\end{pgfscope}%
\begin{pgfscope}%
\pgfsys@transformshift{1.792225in}{1.425322in}%
\pgfsys@useobject{currentmarker}{}%
\end{pgfscope}%
\begin{pgfscope}%
\pgfsys@transformshift{1.850937in}{1.639013in}%
\pgfsys@useobject{currentmarker}{}%
\end{pgfscope}%
\begin{pgfscope}%
\pgfsys@transformshift{1.909649in}{1.745858in}%
\pgfsys@useobject{currentmarker}{}%
\end{pgfscope}%
\begin{pgfscope}%
\pgfsys@transformshift{1.968361in}{1.781474in}%
\pgfsys@useobject{currentmarker}{}%
\end{pgfscope}%
\begin{pgfscope}%
\pgfsys@transformshift{2.027074in}{1.823025in}%
\pgfsys@useobject{currentmarker}{}%
\end{pgfscope}%
\begin{pgfscope}%
\pgfsys@transformshift{2.085786in}{1.852704in}%
\pgfsys@useobject{currentmarker}{}%
\end{pgfscope}%
\begin{pgfscope}%
\pgfsys@transformshift{2.144498in}{1.876447in}%
\pgfsys@useobject{currentmarker}{}%
\end{pgfscope}%
\begin{pgfscope}%
\pgfsys@transformshift{2.203210in}{1.876447in}%
\pgfsys@useobject{currentmarker}{}%
\end{pgfscope}%
\begin{pgfscope}%
\pgfsys@transformshift{2.261922in}{1.894255in}%
\pgfsys@useobject{currentmarker}{}%
\end{pgfscope}%
\begin{pgfscope}%
\pgfsys@transformshift{2.320634in}{1.900191in}%
\pgfsys@useobject{currentmarker}{}%
\end{pgfscope}%
\begin{pgfscope}%
\pgfsys@transformshift{2.379346in}{1.906127in}%
\pgfsys@useobject{currentmarker}{}%
\end{pgfscope}%
\begin{pgfscope}%
\pgfsys@transformshift{2.438058in}{1.906127in}%
\pgfsys@useobject{currentmarker}{}%
\end{pgfscope}%
\begin{pgfscope}%
\pgfsys@transformshift{2.496771in}{1.912062in}%
\pgfsys@useobject{currentmarker}{}%
\end{pgfscope}%
\begin{pgfscope}%
\pgfsys@transformshift{2.555483in}{1.912062in}%
\pgfsys@useobject{currentmarker}{}%
\end{pgfscope}%
\begin{pgfscope}%
\pgfsys@transformshift{2.614195in}{1.917998in}%
\pgfsys@useobject{currentmarker}{}%
\end{pgfscope}%
\begin{pgfscope}%
\pgfsys@transformshift{2.672907in}{1.917998in}%
\pgfsys@useobject{currentmarker}{}%
\end{pgfscope}%
\end{pgfscope}%
\begin{pgfscope}%
\pgfpathrectangle{\pgfqpoint{0.453589in}{0.383578in}}{\pgfqpoint{2.325000in}{1.925000in}}%
\pgfusepath{clip}%
\pgfsetrectcap%
\pgfsetroundjoin%
\pgfsetlinewidth{0.803000pt}%
\definecolor{currentstroke}{rgb}{0.686275,0.352941,0.313725}%
\pgfsetstrokecolor{currentstroke}%
\pgfsetdash{}{0pt}%
\pgfpathmoveto{\pgfqpoint{0.559271in}{0.451853in}}%
\pgfpathlineto{\pgfqpoint{0.617983in}{0.451864in}}%
\pgfpathlineto{\pgfqpoint{0.676695in}{0.451883in}}%
\pgfpathlineto{\pgfqpoint{0.735407in}{0.451919in}}%
\pgfpathlineto{\pgfqpoint{0.794119in}{0.451984in}}%
\pgfpathlineto{\pgfqpoint{0.852831in}{0.452103in}}%
\pgfpathlineto{\pgfqpoint{0.911543in}{0.452321in}}%
\pgfpathlineto{\pgfqpoint{0.970255in}{0.452721in}}%
\pgfpathlineto{\pgfqpoint{1.028968in}{0.453453in}}%
\pgfpathlineto{\pgfqpoint{1.087680in}{0.454792in}}%
\pgfpathlineto{\pgfqpoint{1.146392in}{0.457238in}}%
\pgfpathlineto{\pgfqpoint{1.205104in}{0.461698in}}%
\pgfpathlineto{\pgfqpoint{1.263816in}{0.469796in}}%
\pgfpathlineto{\pgfqpoint{1.322528in}{0.484396in}}%
\pgfpathlineto{\pgfqpoint{1.381240in}{0.510377in}}%
\pgfpathlineto{\pgfqpoint{1.439952in}{0.555569in}}%
\pgfpathlineto{\pgfqpoint{1.498665in}{0.631129in}}%
\pgfpathlineto{\pgfqpoint{1.557377in}{0.749487in}}%
\pgfpathlineto{\pgfqpoint{1.616089in}{0.917186in}}%
\pgfpathlineto{\pgfqpoint{1.674801in}{1.123884in}}%
\pgfpathlineto{\pgfqpoint{1.733513in}{1.338999in}}%
\pgfpathlineto{\pgfqpoint{1.792225in}{1.526840in}}%
\pgfpathlineto{\pgfqpoint{1.850937in}{1.667331in}}%
\pgfpathlineto{\pgfqpoint{1.909649in}{1.760709in}}%
\pgfpathlineto{\pgfqpoint{1.968361in}{1.818002in}}%
\pgfpathlineto{\pgfqpoint{2.027074in}{1.851446in}}%
\pgfpathlineto{\pgfqpoint{2.085786in}{1.870403in}}%
\pgfpathlineto{\pgfqpoint{2.144498in}{1.880970in}}%
\pgfpathlineto{\pgfqpoint{2.203210in}{1.886805in}}%
\pgfpathlineto{\pgfqpoint{2.261922in}{1.890011in}}%
\pgfpathlineto{\pgfqpoint{2.320634in}{1.891767in}}%
\pgfpathlineto{\pgfqpoint{2.379346in}{1.892727in}}%
\pgfpathlineto{\pgfqpoint{2.438058in}{1.893252in}}%
\pgfpathlineto{\pgfqpoint{2.496771in}{1.893539in}}%
\pgfpathlineto{\pgfqpoint{2.555483in}{1.893695in}}%
\pgfpathlineto{\pgfqpoint{2.614195in}{1.893781in}}%
\pgfpathlineto{\pgfqpoint{2.672907in}{1.893827in}}%
\pgfusepath{stroke}%
\end{pgfscope}%
\begin{pgfscope}%
\pgfpathrectangle{\pgfqpoint{0.453589in}{0.383578in}}{\pgfqpoint{2.325000in}{1.925000in}}%
\pgfusepath{clip}%
\pgfsetbuttcap%
\pgfsetroundjoin%
\definecolor{currentfill}{rgb}{0.000000,0.356863,0.509804}%
\pgfsetfillcolor{currentfill}%
\pgfsetlinewidth{1.003750pt}%
\definecolor{currentstroke}{rgb}{0.000000,0.356863,0.509804}%
\pgfsetstrokecolor{currentstroke}%
\pgfsetdash{}{0pt}%
\pgfsys@defobject{currentmarker}{\pgfqpoint{-0.010417in}{-0.010417in}}{\pgfqpoint{0.010417in}{0.010417in}}{%
\pgfpathmoveto{\pgfqpoint{0.000000in}{-0.010417in}}%
\pgfpathcurveto{\pgfqpoint{0.002763in}{-0.010417in}}{\pgfqpoint{0.005412in}{-0.009319in}}{\pgfqpoint{0.007366in}{-0.007366in}}%
\pgfpathcurveto{\pgfqpoint{0.009319in}{-0.005412in}}{\pgfqpoint{0.010417in}{-0.002763in}}{\pgfqpoint{0.010417in}{0.000000in}}%
\pgfpathcurveto{\pgfqpoint{0.010417in}{0.002763in}}{\pgfqpoint{0.009319in}{0.005412in}}{\pgfqpoint{0.007366in}{0.007366in}}%
\pgfpathcurveto{\pgfqpoint{0.005412in}{0.009319in}}{\pgfqpoint{0.002763in}{0.010417in}}{\pgfqpoint{0.000000in}{0.010417in}}%
\pgfpathcurveto{\pgfqpoint{-0.002763in}{0.010417in}}{\pgfqpoint{-0.005412in}{0.009319in}}{\pgfqpoint{-0.007366in}{0.007366in}}%
\pgfpathcurveto{\pgfqpoint{-0.009319in}{0.005412in}}{\pgfqpoint{-0.010417in}{0.002763in}}{\pgfqpoint{-0.010417in}{0.000000in}}%
\pgfpathcurveto{\pgfqpoint{-0.010417in}{-0.002763in}}{\pgfqpoint{-0.009319in}{-0.005412in}}{\pgfqpoint{-0.007366in}{-0.007366in}}%
\pgfpathcurveto{\pgfqpoint{-0.005412in}{-0.009319in}}{\pgfqpoint{-0.002763in}{-0.010417in}}{\pgfqpoint{0.000000in}{-0.010417in}}%
\pgfpathclose%
\pgfusepath{stroke,fill}%
}%
\begin{pgfscope}%
\pgfsys@transformshift{0.559271in}{0.451840in}%
\pgfsys@useobject{currentmarker}{}%
\end{pgfscope}%
\begin{pgfscope}%
\pgfsys@transformshift{0.617983in}{0.451840in}%
\pgfsys@useobject{currentmarker}{}%
\end{pgfscope}%
\begin{pgfscope}%
\pgfsys@transformshift{0.676695in}{0.451840in}%
\pgfsys@useobject{currentmarker}{}%
\end{pgfscope}%
\begin{pgfscope}%
\pgfsys@transformshift{0.735407in}{0.451840in}%
\pgfsys@useobject{currentmarker}{}%
\end{pgfscope}%
\begin{pgfscope}%
\pgfsys@transformshift{0.794119in}{0.451840in}%
\pgfsys@useobject{currentmarker}{}%
\end{pgfscope}%
\begin{pgfscope}%
\pgfsys@transformshift{0.852831in}{0.451840in}%
\pgfsys@useobject{currentmarker}{}%
\end{pgfscope}%
\begin{pgfscope}%
\pgfsys@transformshift{0.911543in}{0.451840in}%
\pgfsys@useobject{currentmarker}{}%
\end{pgfscope}%
\begin{pgfscope}%
\pgfsys@transformshift{0.970255in}{0.451840in}%
\pgfsys@useobject{currentmarker}{}%
\end{pgfscope}%
\begin{pgfscope}%
\pgfsys@transformshift{1.028968in}{0.451840in}%
\pgfsys@useobject{currentmarker}{}%
\end{pgfscope}%
\begin{pgfscope}%
\pgfsys@transformshift{1.087680in}{0.451840in}%
\pgfsys@useobject{currentmarker}{}%
\end{pgfscope}%
\begin{pgfscope}%
\pgfsys@transformshift{1.146392in}{0.451840in}%
\pgfsys@useobject{currentmarker}{}%
\end{pgfscope}%
\begin{pgfscope}%
\pgfsys@transformshift{1.205104in}{0.451840in}%
\pgfsys@useobject{currentmarker}{}%
\end{pgfscope}%
\begin{pgfscope}%
\pgfsys@transformshift{1.263816in}{0.451840in}%
\pgfsys@useobject{currentmarker}{}%
\end{pgfscope}%
\begin{pgfscope}%
\pgfsys@transformshift{1.322528in}{0.463712in}%
\pgfsys@useobject{currentmarker}{}%
\end{pgfscope}%
\begin{pgfscope}%
\pgfsys@transformshift{1.381240in}{0.469648in}%
\pgfsys@useobject{currentmarker}{}%
\end{pgfscope}%
\begin{pgfscope}%
\pgfsys@transformshift{1.439952in}{0.493391in}%
\pgfsys@useobject{currentmarker}{}%
\end{pgfscope}%
\begin{pgfscope}%
\pgfsys@transformshift{1.498665in}{0.564622in}%
\pgfsys@useobject{currentmarker}{}%
\end{pgfscope}%
\begin{pgfscope}%
\pgfsys@transformshift{1.557377in}{0.600237in}%
\pgfsys@useobject{currentmarker}{}%
\end{pgfscope}%
\begin{pgfscope}%
\pgfsys@transformshift{1.616089in}{0.754569in}%
\pgfsys@useobject{currentmarker}{}%
\end{pgfscope}%
\begin{pgfscope}%
\pgfsys@transformshift{1.674801in}{1.247246in}%
\pgfsys@useobject{currentmarker}{}%
\end{pgfscope}%
\begin{pgfscope}%
\pgfsys@transformshift{1.733513in}{1.603398in}%
\pgfsys@useobject{currentmarker}{}%
\end{pgfscope}%
\begin{pgfscope}%
\pgfsys@transformshift{1.792225in}{1.728051in}%
\pgfsys@useobject{currentmarker}{}%
\end{pgfscope}%
\begin{pgfscope}%
\pgfsys@transformshift{1.850937in}{1.823025in}%
\pgfsys@useobject{currentmarker}{}%
\end{pgfscope}%
\begin{pgfscope}%
\pgfsys@transformshift{1.909649in}{1.876447in}%
\pgfsys@useobject{currentmarker}{}%
\end{pgfscope}%
\begin{pgfscope}%
\pgfsys@transformshift{1.968361in}{1.906127in}%
\pgfsys@useobject{currentmarker}{}%
\end{pgfscope}%
\begin{pgfscope}%
\pgfsys@transformshift{2.027074in}{1.917998in}%
\pgfsys@useobject{currentmarker}{}%
\end{pgfscope}%
\begin{pgfscope}%
\pgfsys@transformshift{2.085786in}{1.923934in}%
\pgfsys@useobject{currentmarker}{}%
\end{pgfscope}%
\begin{pgfscope}%
\pgfsys@transformshift{2.144498in}{1.935806in}%
\pgfsys@useobject{currentmarker}{}%
\end{pgfscope}%
\begin{pgfscope}%
\pgfsys@transformshift{2.203210in}{1.935806in}%
\pgfsys@useobject{currentmarker}{}%
\end{pgfscope}%
\begin{pgfscope}%
\pgfsys@transformshift{2.261922in}{1.947678in}%
\pgfsys@useobject{currentmarker}{}%
\end{pgfscope}%
\begin{pgfscope}%
\pgfsys@transformshift{2.320634in}{1.947678in}%
\pgfsys@useobject{currentmarker}{}%
\end{pgfscope}%
\begin{pgfscope}%
\pgfsys@transformshift{2.379346in}{1.953614in}%
\pgfsys@useobject{currentmarker}{}%
\end{pgfscope}%
\begin{pgfscope}%
\pgfsys@transformshift{2.438058in}{1.953614in}%
\pgfsys@useobject{currentmarker}{}%
\end{pgfscope}%
\begin{pgfscope}%
\pgfsys@transformshift{2.496771in}{1.953614in}%
\pgfsys@useobject{currentmarker}{}%
\end{pgfscope}%
\begin{pgfscope}%
\pgfsys@transformshift{2.555483in}{1.959549in}%
\pgfsys@useobject{currentmarker}{}%
\end{pgfscope}%
\begin{pgfscope}%
\pgfsys@transformshift{2.614195in}{1.953614in}%
\pgfsys@useobject{currentmarker}{}%
\end{pgfscope}%
\begin{pgfscope}%
\pgfsys@transformshift{2.672907in}{1.947678in}%
\pgfsys@useobject{currentmarker}{}%
\end{pgfscope}%
\end{pgfscope}%
\begin{pgfscope}%
\pgfpathrectangle{\pgfqpoint{0.453589in}{0.383578in}}{\pgfqpoint{2.325000in}{1.925000in}}%
\pgfusepath{clip}%
\pgfsetrectcap%
\pgfsetroundjoin%
\pgfsetlinewidth{0.803000pt}%
\definecolor{currentstroke}{rgb}{0.000000,0.356863,0.509804}%
\pgfsetstrokecolor{currentstroke}%
\pgfsetdash{}{0pt}%
\pgfpathmoveto{\pgfqpoint{0.559271in}{0.451840in}}%
\pgfpathlineto{\pgfqpoint{0.617983in}{0.451840in}}%
\pgfpathlineto{\pgfqpoint{0.676695in}{0.451840in}}%
\pgfpathlineto{\pgfqpoint{0.735407in}{0.451841in}}%
\pgfpathlineto{\pgfqpoint{0.794119in}{0.451841in}}%
\pgfpathlineto{\pgfqpoint{0.852831in}{0.451841in}}%
\pgfpathlineto{\pgfqpoint{0.911543in}{0.451842in}}%
\pgfpathlineto{\pgfqpoint{0.970255in}{0.451845in}}%
\pgfpathlineto{\pgfqpoint{1.028968in}{0.451853in}}%
\pgfpathlineto{\pgfqpoint{1.087680in}{0.451876in}}%
\pgfpathlineto{\pgfqpoint{1.146392in}{0.451943in}}%
\pgfpathlineto{\pgfqpoint{1.205104in}{0.452138in}}%
\pgfpathlineto{\pgfqpoint{1.263816in}{0.452705in}}%
\pgfpathlineto{\pgfqpoint{1.322528in}{0.454345in}}%
\pgfpathlineto{\pgfqpoint{1.381240in}{0.459084in}}%
\pgfpathlineto{\pgfqpoint{1.439952in}{0.472664in}}%
\pgfpathlineto{\pgfqpoint{1.498665in}{0.510688in}}%
\pgfpathlineto{\pgfqpoint{1.557377in}{0.610607in}}%
\pgfpathlineto{\pgfqpoint{1.616089in}{0.834584in}}%
\pgfpathlineto{\pgfqpoint{1.674801in}{1.196788in}}%
\pgfpathlineto{\pgfqpoint{1.733513in}{1.557363in}}%
\pgfpathlineto{\pgfqpoint{1.792225in}{1.778714in}}%
\pgfpathlineto{\pgfqpoint{1.850937in}{1.877065in}}%
\pgfpathlineto{\pgfqpoint{1.909649in}{1.914429in}}%
\pgfpathlineto{\pgfqpoint{1.968361in}{1.927764in}}%
\pgfpathlineto{\pgfqpoint{2.027074in}{1.932417in}}%
\pgfpathlineto{\pgfqpoint{2.085786in}{1.934027in}}%
\pgfpathlineto{\pgfqpoint{2.144498in}{1.934583in}}%
\pgfpathlineto{\pgfqpoint{2.203210in}{1.934775in}}%
\pgfpathlineto{\pgfqpoint{2.261922in}{1.934841in}}%
\pgfpathlineto{\pgfqpoint{2.320634in}{1.934864in}}%
\pgfpathlineto{\pgfqpoint{2.379346in}{1.934872in}}%
\pgfpathlineto{\pgfqpoint{2.438058in}{1.934874in}}%
\pgfpathlineto{\pgfqpoint{2.496771in}{1.934875in}}%
\pgfpathlineto{\pgfqpoint{2.555483in}{1.934876in}}%
\pgfpathlineto{\pgfqpoint{2.614195in}{1.934876in}}%
\pgfpathlineto{\pgfqpoint{2.672907in}{1.934876in}}%
\pgfusepath{stroke}%
\end{pgfscope}%
\begin{pgfscope}%
\pgfpathrectangle{\pgfqpoint{0.453589in}{0.383578in}}{\pgfqpoint{2.325000in}{1.925000in}}%
\pgfusepath{clip}%
\pgfsetbuttcap%
\pgfsetroundjoin%
\definecolor{currentfill}{rgb}{0.490196,0.588235,0.431373}%
\pgfsetfillcolor{currentfill}%
\pgfsetlinewidth{1.003750pt}%
\definecolor{currentstroke}{rgb}{0.490196,0.588235,0.431373}%
\pgfsetstrokecolor{currentstroke}%
\pgfsetdash{}{0pt}%
\pgfsys@defobject{currentmarker}{\pgfqpoint{-0.010417in}{-0.010417in}}{\pgfqpoint{0.010417in}{0.010417in}}{%
\pgfpathmoveto{\pgfqpoint{0.000000in}{-0.010417in}}%
\pgfpathcurveto{\pgfqpoint{0.002763in}{-0.010417in}}{\pgfqpoint{0.005412in}{-0.009319in}}{\pgfqpoint{0.007366in}{-0.007366in}}%
\pgfpathcurveto{\pgfqpoint{0.009319in}{-0.005412in}}{\pgfqpoint{0.010417in}{-0.002763in}}{\pgfqpoint{0.010417in}{0.000000in}}%
\pgfpathcurveto{\pgfqpoint{0.010417in}{0.002763in}}{\pgfqpoint{0.009319in}{0.005412in}}{\pgfqpoint{0.007366in}{0.007366in}}%
\pgfpathcurveto{\pgfqpoint{0.005412in}{0.009319in}}{\pgfqpoint{0.002763in}{0.010417in}}{\pgfqpoint{0.000000in}{0.010417in}}%
\pgfpathcurveto{\pgfqpoint{-0.002763in}{0.010417in}}{\pgfqpoint{-0.005412in}{0.009319in}}{\pgfqpoint{-0.007366in}{0.007366in}}%
\pgfpathcurveto{\pgfqpoint{-0.009319in}{0.005412in}}{\pgfqpoint{-0.010417in}{0.002763in}}{\pgfqpoint{-0.010417in}{0.000000in}}%
\pgfpathcurveto{\pgfqpoint{-0.010417in}{-0.002763in}}{\pgfqpoint{-0.009319in}{-0.005412in}}{\pgfqpoint{-0.007366in}{-0.007366in}}%
\pgfpathcurveto{\pgfqpoint{-0.005412in}{-0.009319in}}{\pgfqpoint{-0.002763in}{-0.010417in}}{\pgfqpoint{0.000000in}{-0.010417in}}%
\pgfpathclose%
\pgfusepath{stroke,fill}%
}%
\begin{pgfscope}%
\pgfsys@transformshift{0.559271in}{0.451840in}%
\pgfsys@useobject{currentmarker}{}%
\end{pgfscope}%
\begin{pgfscope}%
\pgfsys@transformshift{0.617983in}{0.451840in}%
\pgfsys@useobject{currentmarker}{}%
\end{pgfscope}%
\begin{pgfscope}%
\pgfsys@transformshift{0.676695in}{0.451840in}%
\pgfsys@useobject{currentmarker}{}%
\end{pgfscope}%
\begin{pgfscope}%
\pgfsys@transformshift{0.735407in}{0.451840in}%
\pgfsys@useobject{currentmarker}{}%
\end{pgfscope}%
\begin{pgfscope}%
\pgfsys@transformshift{0.794119in}{0.451840in}%
\pgfsys@useobject{currentmarker}{}%
\end{pgfscope}%
\begin{pgfscope}%
\pgfsys@transformshift{0.852831in}{0.451840in}%
\pgfsys@useobject{currentmarker}{}%
\end{pgfscope}%
\begin{pgfscope}%
\pgfsys@transformshift{0.911543in}{0.451840in}%
\pgfsys@useobject{currentmarker}{}%
\end{pgfscope}%
\begin{pgfscope}%
\pgfsys@transformshift{0.970255in}{0.451840in}%
\pgfsys@useobject{currentmarker}{}%
\end{pgfscope}%
\begin{pgfscope}%
\pgfsys@transformshift{1.028968in}{0.451840in}%
\pgfsys@useobject{currentmarker}{}%
\end{pgfscope}%
\begin{pgfscope}%
\pgfsys@transformshift{1.087680in}{0.451840in}%
\pgfsys@useobject{currentmarker}{}%
\end{pgfscope}%
\begin{pgfscope}%
\pgfsys@transformshift{1.146392in}{0.451840in}%
\pgfsys@useobject{currentmarker}{}%
\end{pgfscope}%
\begin{pgfscope}%
\pgfsys@transformshift{1.205104in}{0.451840in}%
\pgfsys@useobject{currentmarker}{}%
\end{pgfscope}%
\begin{pgfscope}%
\pgfsys@transformshift{1.263816in}{0.451840in}%
\pgfsys@useobject{currentmarker}{}%
\end{pgfscope}%
\begin{pgfscope}%
\pgfsys@transformshift{1.322528in}{0.451840in}%
\pgfsys@useobject{currentmarker}{}%
\end{pgfscope}%
\begin{pgfscope}%
\pgfsys@transformshift{1.381240in}{0.451840in}%
\pgfsys@useobject{currentmarker}{}%
\end{pgfscope}%
\begin{pgfscope}%
\pgfsys@transformshift{1.439952in}{0.457776in}%
\pgfsys@useobject{currentmarker}{}%
\end{pgfscope}%
\begin{pgfscope}%
\pgfsys@transformshift{1.498665in}{0.499327in}%
\pgfsys@useobject{currentmarker}{}%
\end{pgfscope}%
\begin{pgfscope}%
\pgfsys@transformshift{1.557377in}{0.606173in}%
\pgfsys@useobject{currentmarker}{}%
\end{pgfscope}%
\begin{pgfscope}%
\pgfsys@transformshift{1.616089in}{0.730826in}%
\pgfsys@useobject{currentmarker}{}%
\end{pgfscope}%
\begin{pgfscope}%
\pgfsys@transformshift{1.674801in}{1.401578in}%
\pgfsys@useobject{currentmarker}{}%
\end{pgfscope}%
\begin{pgfscope}%
\pgfsys@transformshift{1.733513in}{1.728051in}%
\pgfsys@useobject{currentmarker}{}%
\end{pgfscope}%
\begin{pgfscope}%
\pgfsys@transformshift{1.792225in}{1.846768in}%
\pgfsys@useobject{currentmarker}{}%
\end{pgfscope}%
\begin{pgfscope}%
\pgfsys@transformshift{1.850937in}{1.923934in}%
\pgfsys@useobject{currentmarker}{}%
\end{pgfscope}%
\begin{pgfscope}%
\pgfsys@transformshift{1.909649in}{1.941742in}%
\pgfsys@useobject{currentmarker}{}%
\end{pgfscope}%
\begin{pgfscope}%
\pgfsys@transformshift{1.968361in}{1.953614in}%
\pgfsys@useobject{currentmarker}{}%
\end{pgfscope}%
\begin{pgfscope}%
\pgfsys@transformshift{2.027074in}{1.959549in}%
\pgfsys@useobject{currentmarker}{}%
\end{pgfscope}%
\begin{pgfscope}%
\pgfsys@transformshift{2.085786in}{1.965485in}%
\pgfsys@useobject{currentmarker}{}%
\end{pgfscope}%
\begin{pgfscope}%
\pgfsys@transformshift{2.144498in}{1.965485in}%
\pgfsys@useobject{currentmarker}{}%
\end{pgfscope}%
\begin{pgfscope}%
\pgfsys@transformshift{2.203210in}{1.965485in}%
\pgfsys@useobject{currentmarker}{}%
\end{pgfscope}%
\begin{pgfscope}%
\pgfsys@transformshift{2.261922in}{1.959549in}%
\pgfsys@useobject{currentmarker}{}%
\end{pgfscope}%
\begin{pgfscope}%
\pgfsys@transformshift{2.320634in}{1.959549in}%
\pgfsys@useobject{currentmarker}{}%
\end{pgfscope}%
\begin{pgfscope}%
\pgfsys@transformshift{2.379346in}{1.965485in}%
\pgfsys@useobject{currentmarker}{}%
\end{pgfscope}%
\begin{pgfscope}%
\pgfsys@transformshift{2.438058in}{1.959549in}%
\pgfsys@useobject{currentmarker}{}%
\end{pgfscope}%
\begin{pgfscope}%
\pgfsys@transformshift{2.496771in}{1.965485in}%
\pgfsys@useobject{currentmarker}{}%
\end{pgfscope}%
\begin{pgfscope}%
\pgfsys@transformshift{2.555483in}{1.965485in}%
\pgfsys@useobject{currentmarker}{}%
\end{pgfscope}%
\begin{pgfscope}%
\pgfsys@transformshift{2.614195in}{1.959549in}%
\pgfsys@useobject{currentmarker}{}%
\end{pgfscope}%
\begin{pgfscope}%
\pgfsys@transformshift{2.672907in}{1.959549in}%
\pgfsys@useobject{currentmarker}{}%
\end{pgfscope}%
\end{pgfscope}%
\begin{pgfscope}%
\pgfpathrectangle{\pgfqpoint{0.453589in}{0.383578in}}{\pgfqpoint{2.325000in}{1.925000in}}%
\pgfusepath{clip}%
\pgfsetrectcap%
\pgfsetroundjoin%
\pgfsetlinewidth{0.803000pt}%
\definecolor{currentstroke}{rgb}{0.490196,0.588235,0.431373}%
\pgfsetstrokecolor{currentstroke}%
\pgfsetdash{}{0pt}%
\pgfpathmoveto{\pgfqpoint{0.559271in}{0.451840in}}%
\pgfpathlineto{\pgfqpoint{0.617983in}{0.451840in}}%
\pgfpathlineto{\pgfqpoint{0.676695in}{0.451840in}}%
\pgfpathlineto{\pgfqpoint{0.735407in}{0.451840in}}%
\pgfpathlineto{\pgfqpoint{0.794119in}{0.451840in}}%
\pgfpathlineto{\pgfqpoint{0.852831in}{0.451840in}}%
\pgfpathlineto{\pgfqpoint{0.911543in}{0.451840in}}%
\pgfpathlineto{\pgfqpoint{0.970255in}{0.451840in}}%
\pgfpathlineto{\pgfqpoint{1.028968in}{0.451841in}}%
\pgfpathlineto{\pgfqpoint{1.087680in}{0.451841in}}%
\pgfpathlineto{\pgfqpoint{1.146392in}{0.451843in}}%
\pgfpathlineto{\pgfqpoint{1.205104in}{0.451853in}}%
\pgfpathlineto{\pgfqpoint{1.263816in}{0.451898in}}%
\pgfpathlineto{\pgfqpoint{1.322528in}{0.452099in}}%
\pgfpathlineto{\pgfqpoint{1.381240in}{0.453003in}}%
\pgfpathlineto{\pgfqpoint{1.439952in}{0.457053in}}%
\pgfpathlineto{\pgfqpoint{1.498665in}{0.475003in}}%
\pgfpathlineto{\pgfqpoint{1.557377in}{0.550684in}}%
\pgfpathlineto{\pgfqpoint{1.616089in}{0.813274in}}%
\pgfpathlineto{\pgfqpoint{1.674801in}{1.334912in}}%
\pgfpathlineto{\pgfqpoint{1.733513in}{1.752222in}}%
\pgfpathlineto{\pgfqpoint{1.792225in}{1.904907in}}%
\pgfpathlineto{\pgfqpoint{1.850937in}{1.943860in}}%
\pgfpathlineto{\pgfqpoint{1.909649in}{1.952807in}}%
\pgfpathlineto{\pgfqpoint{1.968361in}{1.954811in}}%
\pgfpathlineto{\pgfqpoint{2.027074in}{1.955258in}}%
\pgfpathlineto{\pgfqpoint{2.085786in}{1.955357in}}%
\pgfpathlineto{\pgfqpoint{2.144498in}{1.955379in}}%
\pgfpathlineto{\pgfqpoint{2.203210in}{1.955384in}}%
\pgfpathlineto{\pgfqpoint{2.261922in}{1.955385in}}%
\pgfpathlineto{\pgfqpoint{2.320634in}{1.955385in}}%
\pgfpathlineto{\pgfqpoint{2.379346in}{1.955385in}}%
\pgfpathlineto{\pgfqpoint{2.438058in}{1.955385in}}%
\pgfpathlineto{\pgfqpoint{2.496771in}{1.955385in}}%
\pgfpathlineto{\pgfqpoint{2.555483in}{1.955385in}}%
\pgfpathlineto{\pgfqpoint{2.614195in}{1.955385in}}%
\pgfpathlineto{\pgfqpoint{2.672907in}{1.955385in}}%
\pgfusepath{stroke}%
\end{pgfscope}%
\begin{pgfscope}%
\pgfsetrectcap%
\pgfsetmiterjoin%
\pgfsetlinewidth{0.501875pt}%
\definecolor{currentstroke}{rgb}{0.317647,0.317647,0.317647}%
\pgfsetstrokecolor{currentstroke}%
\pgfsetdash{}{0pt}%
\pgfpathmoveto{\pgfqpoint{0.453589in}{0.383578in}}%
\pgfpathlineto{\pgfqpoint{0.453589in}{2.308578in}}%
\pgfusepath{stroke}%
\end{pgfscope}%
\begin{pgfscope}%
\pgfsetrectcap%
\pgfsetmiterjoin%
\pgfsetlinewidth{0.501875pt}%
\definecolor{currentstroke}{rgb}{0.317647,0.317647,0.317647}%
\pgfsetstrokecolor{currentstroke}%
\pgfsetdash{}{0pt}%
\pgfpathmoveto{\pgfqpoint{0.453589in}{0.383578in}}%
\pgfpathlineto{\pgfqpoint{2.778589in}{0.383578in}}%
\pgfusepath{stroke}%
\end{pgfscope}%
\begin{pgfscope}%
\pgfsetbuttcap%
\pgfsetroundjoin%
\definecolor{currentfill}{rgb}{0.333333,0.333333,0.333333}%
\pgfsetfillcolor{currentfill}%
\pgfsetlinewidth{1.003750pt}%
\definecolor{currentstroke}{rgb}{0.333333,0.333333,0.333333}%
\pgfsetstrokecolor{currentstroke}%
\pgfsetdash{}{0pt}%
\pgfsys@defobject{currentmarker}{\pgfqpoint{-0.010417in}{-0.010417in}}{\pgfqpoint{0.010417in}{0.010417in}}{%
\pgfpathmoveto{\pgfqpoint{0.000000in}{-0.010417in}}%
\pgfpathcurveto{\pgfqpoint{0.002763in}{-0.010417in}}{\pgfqpoint{0.005412in}{-0.009319in}}{\pgfqpoint{0.007366in}{-0.007366in}}%
\pgfpathcurveto{\pgfqpoint{0.009319in}{-0.005412in}}{\pgfqpoint{0.010417in}{-0.002763in}}{\pgfqpoint{0.010417in}{0.000000in}}%
\pgfpathcurveto{\pgfqpoint{0.010417in}{0.002763in}}{\pgfqpoint{0.009319in}{0.005412in}}{\pgfqpoint{0.007366in}{0.007366in}}%
\pgfpathcurveto{\pgfqpoint{0.005412in}{0.009319in}}{\pgfqpoint{0.002763in}{0.010417in}}{\pgfqpoint{0.000000in}{0.010417in}}%
\pgfpathcurveto{\pgfqpoint{-0.002763in}{0.010417in}}{\pgfqpoint{-0.005412in}{0.009319in}}{\pgfqpoint{-0.007366in}{0.007366in}}%
\pgfpathcurveto{\pgfqpoint{-0.009319in}{0.005412in}}{\pgfqpoint{-0.010417in}{0.002763in}}{\pgfqpoint{-0.010417in}{0.000000in}}%
\pgfpathcurveto{\pgfqpoint{-0.010417in}{-0.002763in}}{\pgfqpoint{-0.009319in}{-0.005412in}}{\pgfqpoint{-0.007366in}{-0.007366in}}%
\pgfpathcurveto{\pgfqpoint{-0.005412in}{-0.009319in}}{\pgfqpoint{-0.002763in}{-0.010417in}}{\pgfqpoint{0.000000in}{-0.010417in}}%
\pgfpathclose%
\pgfusepath{stroke,fill}%
}%
\begin{pgfscope}%
\pgfsys@transformshift{0.518378in}{2.248417in}%
\pgfsys@useobject{currentmarker}{}%
\end{pgfscope}%
\end{pgfscope}%
\begin{pgfscope}%
\definecolor{textcolor}{rgb}{0.000000,0.000000,0.000000}%
\pgfsetstrokecolor{textcolor}%
\pgfsetfillcolor{textcolor}%
\pgftext[x=0.601678in,y=2.216022in,left,base]{\color{textcolor}\rmfamily\fontsize{6.664000}{7.996800}\selectfont \(\displaystyle b \propto \delta V = \vartheta = \SI{5}{\V}\)}%
\end{pgfscope}%
\begin{pgfscope}%
\pgfsetbuttcap%
\pgfsetroundjoin%
\definecolor{currentfill}{rgb}{0.686275,0.352941,0.313725}%
\pgfsetfillcolor{currentfill}%
\pgfsetlinewidth{1.003750pt}%
\definecolor{currentstroke}{rgb}{0.686275,0.352941,0.313725}%
\pgfsetstrokecolor{currentstroke}%
\pgfsetdash{}{0pt}%
\pgfsys@defobject{currentmarker}{\pgfqpoint{-0.010417in}{-0.010417in}}{\pgfqpoint{0.010417in}{0.010417in}}{%
\pgfpathmoveto{\pgfqpoint{0.000000in}{-0.010417in}}%
\pgfpathcurveto{\pgfqpoint{0.002763in}{-0.010417in}}{\pgfqpoint{0.005412in}{-0.009319in}}{\pgfqpoint{0.007366in}{-0.007366in}}%
\pgfpathcurveto{\pgfqpoint{0.009319in}{-0.005412in}}{\pgfqpoint{0.010417in}{-0.002763in}}{\pgfqpoint{0.010417in}{0.000000in}}%
\pgfpathcurveto{\pgfqpoint{0.010417in}{0.002763in}}{\pgfqpoint{0.009319in}{0.005412in}}{\pgfqpoint{0.007366in}{0.007366in}}%
\pgfpathcurveto{\pgfqpoint{0.005412in}{0.009319in}}{\pgfqpoint{0.002763in}{0.010417in}}{\pgfqpoint{0.000000in}{0.010417in}}%
\pgfpathcurveto{\pgfqpoint{-0.002763in}{0.010417in}}{\pgfqpoint{-0.005412in}{0.009319in}}{\pgfqpoint{-0.007366in}{0.007366in}}%
\pgfpathcurveto{\pgfqpoint{-0.009319in}{0.005412in}}{\pgfqpoint{-0.010417in}{0.002763in}}{\pgfqpoint{-0.010417in}{0.000000in}}%
\pgfpathcurveto{\pgfqpoint{-0.010417in}{-0.002763in}}{\pgfqpoint{-0.009319in}{-0.005412in}}{\pgfqpoint{-0.007366in}{-0.007366in}}%
\pgfpathcurveto{\pgfqpoint{-0.005412in}{-0.009319in}}{\pgfqpoint{-0.002763in}{-0.010417in}}{\pgfqpoint{0.000000in}{-0.010417in}}%
\pgfpathclose%
\pgfusepath{stroke,fill}%
}%
\begin{pgfscope}%
\pgfsys@transformshift{0.518378in}{2.128650in}%
\pgfsys@useobject{currentmarker}{}%
\end{pgfscope}%
\end{pgfscope}%
\begin{pgfscope}%
\definecolor{textcolor}{rgb}{0.000000,0.000000,0.000000}%
\pgfsetstrokecolor{textcolor}%
\pgfsetfillcolor{textcolor}%
\pgftext[x=0.601678in,y=2.096256in,left,base]{\color{textcolor}\rmfamily\fontsize{6.664000}{7.996800}\selectfont \(\displaystyle b \propto \delta V = \vartheta = \SI{15}{\V}\)}%
\end{pgfscope}%
\begin{pgfscope}%
\pgfsetbuttcap%
\pgfsetroundjoin%
\definecolor{currentfill}{rgb}{0.000000,0.356863,0.509804}%
\pgfsetfillcolor{currentfill}%
\pgfsetlinewidth{1.003750pt}%
\definecolor{currentstroke}{rgb}{0.000000,0.356863,0.509804}%
\pgfsetstrokecolor{currentstroke}%
\pgfsetdash{}{0pt}%
\pgfsys@defobject{currentmarker}{\pgfqpoint{-0.010417in}{-0.010417in}}{\pgfqpoint{0.010417in}{0.010417in}}{%
\pgfpathmoveto{\pgfqpoint{0.000000in}{-0.010417in}}%
\pgfpathcurveto{\pgfqpoint{0.002763in}{-0.010417in}}{\pgfqpoint{0.005412in}{-0.009319in}}{\pgfqpoint{0.007366in}{-0.007366in}}%
\pgfpathcurveto{\pgfqpoint{0.009319in}{-0.005412in}}{\pgfqpoint{0.010417in}{-0.002763in}}{\pgfqpoint{0.010417in}{0.000000in}}%
\pgfpathcurveto{\pgfqpoint{0.010417in}{0.002763in}}{\pgfqpoint{0.009319in}{0.005412in}}{\pgfqpoint{0.007366in}{0.007366in}}%
\pgfpathcurveto{\pgfqpoint{0.005412in}{0.009319in}}{\pgfqpoint{0.002763in}{0.010417in}}{\pgfqpoint{0.000000in}{0.010417in}}%
\pgfpathcurveto{\pgfqpoint{-0.002763in}{0.010417in}}{\pgfqpoint{-0.005412in}{0.009319in}}{\pgfqpoint{-0.007366in}{0.007366in}}%
\pgfpathcurveto{\pgfqpoint{-0.009319in}{0.005412in}}{\pgfqpoint{-0.010417in}{0.002763in}}{\pgfqpoint{-0.010417in}{0.000000in}}%
\pgfpathcurveto{\pgfqpoint{-0.010417in}{-0.002763in}}{\pgfqpoint{-0.009319in}{-0.005412in}}{\pgfqpoint{-0.007366in}{-0.007366in}}%
\pgfpathcurveto{\pgfqpoint{-0.005412in}{-0.009319in}}{\pgfqpoint{-0.002763in}{-0.010417in}}{\pgfqpoint{0.000000in}{-0.010417in}}%
\pgfpathclose%
\pgfusepath{stroke,fill}%
}%
\begin{pgfscope}%
\pgfsys@transformshift{0.518378in}{2.008884in}%
\pgfsys@useobject{currentmarker}{}%
\end{pgfscope}%
\end{pgfscope}%
\begin{pgfscope}%
\definecolor{textcolor}{rgb}{0.000000,0.000000,0.000000}%
\pgfsetstrokecolor{textcolor}%
\pgfsetfillcolor{textcolor}%
\pgftext[x=0.601678in,y=1.976489in,left,base]{\color{textcolor}\rmfamily\fontsize{6.664000}{7.996800}\selectfont \(\displaystyle b \propto \delta V = \vartheta = \SI{30}{\V}\)}%
\end{pgfscope}%
\begin{pgfscope}%
\pgfsetbuttcap%
\pgfsetroundjoin%
\definecolor{currentfill}{rgb}{0.490196,0.588235,0.431373}%
\pgfsetfillcolor{currentfill}%
\pgfsetlinewidth{1.003750pt}%
\definecolor{currentstroke}{rgb}{0.490196,0.588235,0.431373}%
\pgfsetstrokecolor{currentstroke}%
\pgfsetdash{}{0pt}%
\pgfsys@defobject{currentmarker}{\pgfqpoint{-0.010417in}{-0.010417in}}{\pgfqpoint{0.010417in}{0.010417in}}{%
\pgfpathmoveto{\pgfqpoint{0.000000in}{-0.010417in}}%
\pgfpathcurveto{\pgfqpoint{0.002763in}{-0.010417in}}{\pgfqpoint{0.005412in}{-0.009319in}}{\pgfqpoint{0.007366in}{-0.007366in}}%
\pgfpathcurveto{\pgfqpoint{0.009319in}{-0.005412in}}{\pgfqpoint{0.010417in}{-0.002763in}}{\pgfqpoint{0.010417in}{0.000000in}}%
\pgfpathcurveto{\pgfqpoint{0.010417in}{0.002763in}}{\pgfqpoint{0.009319in}{0.005412in}}{\pgfqpoint{0.007366in}{0.007366in}}%
\pgfpathcurveto{\pgfqpoint{0.005412in}{0.009319in}}{\pgfqpoint{0.002763in}{0.010417in}}{\pgfqpoint{0.000000in}{0.010417in}}%
\pgfpathcurveto{\pgfqpoint{-0.002763in}{0.010417in}}{\pgfqpoint{-0.005412in}{0.009319in}}{\pgfqpoint{-0.007366in}{0.007366in}}%
\pgfpathcurveto{\pgfqpoint{-0.009319in}{0.005412in}}{\pgfqpoint{-0.010417in}{0.002763in}}{\pgfqpoint{-0.010417in}{0.000000in}}%
\pgfpathcurveto{\pgfqpoint{-0.010417in}{-0.002763in}}{\pgfqpoint{-0.009319in}{-0.005412in}}{\pgfqpoint{-0.007366in}{-0.007366in}}%
\pgfpathcurveto{\pgfqpoint{-0.005412in}{-0.009319in}}{\pgfqpoint{-0.002763in}{-0.010417in}}{\pgfqpoint{0.000000in}{-0.010417in}}%
\pgfpathclose%
\pgfusepath{stroke,fill}%
}%
\begin{pgfscope}%
\pgfsys@transformshift{0.518378in}{1.889117in}%
\pgfsys@useobject{currentmarker}{}%
\end{pgfscope}%
\end{pgfscope}%
\begin{pgfscope}%
\definecolor{textcolor}{rgb}{0.000000,0.000000,0.000000}%
\pgfsetstrokecolor{textcolor}%
\pgfsetfillcolor{textcolor}%
\pgftext[x=0.601678in,y=1.856722in,left,base]{\color{textcolor}\rmfamily\fontsize{6.664000}{7.996800}\selectfont \(\displaystyle b \propto \delta V = \vartheta = \SI{60}{\V}\)}%
\end{pgfscope}%
\end{pgfpicture}%
\makeatother%
\endgroup%

	\end{center}
	\caption[Measurement of the transfer function on \gls{dls}]{Measurement of the transfer function on  \gls{dls}. The shape of the transfer function \gls{transfer} depends on the synaptic strength $w$ since the input current is directly proportional to the weight $\gls{isyn} \propto w \nu_\text{in}$. In this way, a lower weight causes a lower input current which in turn leads to smaller incline of the output rate. As one can see a lower weight causes the maximum rate to drop slightly. The synaptic currents are not strong enough to fight the leaking term from the \gls{lif} model.}
	\label{transferfunction}
\end{figure}

%For zero input, i.e. $\nu_{\text{in}} = 0$ and the , apart from the injected noise, the threshold is set such that $\nu_\text{out} = \nicefrac{\nu_\text{max}}{2}$. Varying the strength of the synaptic input directly changes the shape of \gls{transfer} as well (c.f. \cref{transferfunction}).

\subsection{Temporal Coding}
\glspl{snn} can also model more complex processes which depend on the \gls{isi} or the time relative to a reference spike. In comparison to rate coding, using the exact spike time to encode information is more exposed to noise. In turn, the response time is highly increased for temporal coding, since there is no need to establish a certain fire rate first. The use of such coding has been observed in biology multiple times and has been proven to of high importance (\citealp{gerstner1996neuronal} and \citealp{rieke1999spikes}). Another important advantage over rate coding is the efficiency that comes with using only few spikes. As every spike costs energy, sparse temporal coding condenses the required computational power to solve a task to a minimum. 

Until now, only few people have successfully trained \glspl{snn} with hidden units. With the binary nature of spikes, the neuron's activity is non-differentiable on an individual spike level. Hence, \gls{sgd} cannot be used for temporal coding. Only recently, a promising workaround was suggested with \emph{SuperSpike}, a surrogate gradient descent implementation for \glspl{snn} (\citealp{zenke2018superspike}).

\subsection{Supervised Training with Temporal Coding (SuperSpike)}
\label{superspike}
%Until now, only few people have successfully trained \glspl{snn} with hidden units. The main issue arises from the non-differentiable dynamics of spikes. A promising approach, solving this issue was proposed with \emph{SuperSpike}, a surrogate gradient descent implementation for \glspl{snn} (\citealp{zenke2018superspike}).
As for the rate coding approach, the term \gls{snn} is limited to a spiking feed-forward network on the basis of the \gls{lif} neuron and again, the training routine can be split up into a forward and backward pass.

The \textbf{forward pass} changes only slightly compared to \glspl{ann}. Instead of a continuous input and output, the formalism of the \gls{lif} is used. The presynaptic activity of neuron $j$ is given by the spike trains $S_j$ and the postsynaptic activity of neuron $i$ by the spike train $S_i$. Again the transfer function \gls{transfer} is determined by the dynamics of the \gls{lif} neuron.
%Todo: "surrogate gradient" is not really introduced but just used -> fix this
%Todo: spike train formalism (maybe in biological part with synapses, etc. ), presynaptic/postsynpatic 
%Todo: axonal delay is not considered for the application of the hardware
%Todo: auxilary function: threshold! -> f(x) = x/(1 + |x|) and f'(x) (1 + |x|)-2 => $\sigma'(U_i) = (1+|U_i - \vartheta|)^{-2}$

As stated in section \ref{supervisedtraining}, most training approaches involve the optimization of a certain loss function $\mathcal{L(\mathbf{\theta)}}$ that depends on the network's parameters $\mathbf{\theta}$. In the \textbf{backward pass} of the SuperSpike formalism the von Rossum distance (\citealp{rossum01novel}) of a target spike train $\hat{S}_i$ and the output spike train $S_i$ is chosen, c.f. \citealp{zenke2018superspike},
\begin{equation}
\label{vonrossumdistance}
\mathcal{L} = \frac{1}{2} \int^t_{-\infty}dt' \left[\left(\alpha \ast \hat{S}_i - \alpha \ast S_i \right)(t')\right]^2,
\end{equation}
with a smooth double exponential kernel $\alpha$. 

The computation of the gradient for \ref{vonrossumdistance} with respect to $\mathbf{\theta}$ requires the derivative of a spike train  $\nicefrac{\partial{S_i}}{\partial{w_{ij}}}$ which is undefined for the time of a spike and zero elsewhere. SuperSpike circumvents this issue by rendering the spike train with a smooth auxiliary function $\sigma(V)$ of the membrane potential $V$ and thus the ill defined gradient of the spike train can be replaced by a surrogate derivative $\sigma'(V)$
\begin{equation}
\frac{\partial S_i}{\partial w_{ij}} \quad \rightarrow \quad \sigma'(V_i)\frac{\partial V_i}{\partial w_{ij}}.
\end{equation}
The choice of the auxiliary function $\sigma$ the therefore implied surrogate derivatives is not strict. For SuperSpike a fast sigmoid is chosen $\sigma(V) = \frac{V - \vartheta}{1 + |V - \vartheta|}$. The surrogate partial derivative yields $\sigma'(V) = \left(1 + |V - \vartheta|\right)^{-2}$. Other common choices are pieces-wise linear or exponential approaches.

At a first glance, it appears that the problem has just been shifted to computing the partial gradient of the membrane potential instead. When the potential $V_i$ is formulated as a spike response model for \gls{lif} neuron (Gerstner et al., 2014) it again depends on the output spike train $S_i$. However, under the assumption of a low output rate the gradient can be approximated by $\frac{\partial U_i}{\partial w_{ij}} \approx (\epsilon \ast S_j)$ with $\epsilon$ another double-exponential kernel corresponding to the shape of a \gls{psp}. Plugging in the approximation and the formulation of the gradient as a surrogate gradient yields

\begin{equation}
\label{superspikeweightupdateeq}
\frac{\partial w_{ij}}{\partial t} = \eta \int_{-\infty}^{t} dt'
\underbrace{\left(\alpha \ast (\hat{S}_i - S_i)\right)}_{= e^{(o)}_k \; \text{(Error)}} 
\; \alpha \ast 
\Big(\underbrace{\sigma'(U_i)}_{\text{Pre}} 
\underbrace{\left(\epsilon \ast S_j\right)}_{\text{Post}}\Big)
\end{equation}
with the learning rate $\eta$. 

The formulation for the hidden layer is similar, except for how the error is calculated. Inspired from the popular backpropgation method the error signal of the $i \text{-th}$ hidden unit $e^{(h)}_i$ is propagated backwards as a weighted sum over the all output error signal $e^{(h)}_i = \sum_{k} w_{ik}^{(o)} e^{(o)}_k$ with the feed forward weights $w_{ik}^{(o)}$ between the hidden and the output layer. A feedback alignment oriented approach with random weights is also possible. This formalism can be easily adapted for multiple hidden layers too.

%ToDo: Literatur Book zu Deep Learning (siehe downloads UniRechner)
%
%Überleitung zu rate coding vs sparse coding (maybe read again SuperSpike17/19 first))
%A \gls{snn} offers many advantages compared to a classical \gls{ann}. On of the major differences is that sparse coding allows the user to compress a lot of information into a single spike. Also not spiking at a sppecifencodes information.  This efficient way of transporting information makes \glspl{snn} a contender for any energy sensitive form of computing. More importantly it also opens up the doors to understanding spike-based computing better and thus also the way the human brain works.





\section{Neuromorphic Hardware}

\begin{wrapfigure}{R}{0.45\textwidth}
	\centering
	\includegraphics[width=0.4\textwidth]{figures/HXcloseup.JPG}
	\caption[Close-up of the newest \gls{bss2} single chip.]{Close-up of the newest \gls{bss2} single chip. The analog core of the neuromorphic chip is bonded before a protective cover is placed over it. Picture taken by M{\"u}ller, 2020.} 
	\label{hxcloseup}
\end{wrapfigure}

Certain tasks, such as pattern recognition, are quite difficult to solve with traditional computing methods.  Biologically inspired computing has provided a range of efficient approaches to successfully tackle these problems. In the same way, biology has also been the key inspiration for a new \emph{neuromorphic} hardware design. As it's paradigm, neuromorphic hardware is designed to be robust to malfunctioning sectors, energy efficient and adaptive. 

As of today, several well known tech companies have launched their own neuromorphic platform. IBM started to work on \emph{TrueNorth} in 2008, Intel presented the \emph{Loihi} chip in 2018 and Google started selling the \emph{Coral} dev-board in 2019. But even before neuromorphic hardware caught the attention of the big industry names, academic projects have already been started. Among others, the EU's Human Brain Project (HBP) funds two promising approaches: \emph{SpiNNaker}, a digital based neuromorphic supercomputer located in Manchester and \emph{\gls{bss}}, a mixed-signal accelerated emulation for spiking neural networks based in Heidelberg.

The \gls{bss2} platform is based upon a complete redesign of the \gls{hicann} chip from its predecessor \gls{bss1}. By using a smaller manufacturing process (\SI{65}{\nano \m} instead of \SI{180}{\nano \m}) several new features could be included on the new core - the \gls{hx}. The analog mixed-signal neuromorphic chip with 512 \gls{adex} neuron circuits and 256 possible synaptic connections per neuron is especially designed to investigate various on-chip plasticity algorithms. Therefore it features two \glspl{ppu}, on-chip event routing and the HAGEN extension. The latter is an early realization of a neuromorphic system which basically implements analog matrix multiplication on-chip (\citealp{schemmel2020accelerated}).

The new features have been tested step by step on smaller prototype systems, e.g. on the \gls{dls} the newly designed \gls{lif} neuron was tested. To avoid unnecessary costs, the size was reduced to $32$ neurons and a corresponding $32 \times 32$ synapse array allows all to all connectivity. Besides the Hagen extension and the \gls{adex} neuron, the main features of the \gls{bss2} platform, such as the \gls{ppu}, are already available on the prototyped versions.

%The experiments conducted within this thesis are done either on the \gls{hx} or the \gls{dls}. In the following section the individual parts of the chip are discussed.
%The manufacturing has been outsourced to the Taiwan Semiconductor Manufacturing Company (TSMC) using a standardized $65 \si{nm}$ low-power and low-leakage CMOS technology.

%refs: ibm http://www.research.ibm.com/articles/brain-chip.shtml
% coral board:
% Loihi:

\subsection{Architecture of \gls{bss2}}
\begin{figure}
	\begin{subfigure}[c]{0.5\textwidth}
		\centering
		\caption{}
		\includegraphics[width=0.9\textwidth]{figures/bss2architecture_wtext.pdf}
		\label{hxstructure}
	\end{subfigure}	
	\begin{subfigure}[c]{0.5\textwidth}
		\centering
		\caption{}
		\includegraphics[width=0.9\textwidth]{figures/ppu_overview.pdf}
		\label{hxppu}
	\end{subfigure}
	\caption[Overview of the architecture on \gls{bss2}]{Overview of the architecture on \gls{bss2}. (\subref{hxstructure}): The neurmorphic platform is divided into a digital and analog core which is connected to an external host via an FPGA. Two plasticity processing units provide computational power for on-chip training.  Figure taken from \citealp{schemmel2017internal} (\subref{hxppu}): The plasticity processing units are divided into a general purpose part based which is based on a 32-bit architecture and a vector unit, that enables efficient parallel data processing of analog parameters and observables. Figure taken from \citealp{friedmann2016hybridlearning}}
\end{figure}

The design of the \gls{hx} chip can be divided into an \emph{analogue} and \emph{digital} core (c.f. figure \ref{hxstructure}). The external communication is established by an \gls{fpga} accessing eight serial Low Voltage Differential Signaling (LVDS) link. The interface handles read/write instructions and manages spike event data in both directions.

The analogue core contains the physical implementation of the \gls{lif} and the \gls{adex} neuron model respectively (c.f. \citealp{aamir2018dls2neuron} and \citealp{aamir2018mixed}). The biological time constants of neurons and synapses are usually in the order of 1 to 100 milliseconds. The \textit{in-silico} implementation of the neuron models create a temporal speed-up compared to their \textit{in-vivo} counterpart, leading to chip time constants of a few microseconds. This acceleration is possible due to the supra-threshold dynamics of CMOS transistors. 

% STOPPED HERE:

The analog neuron model parameters in both chips are tunable by setting bias currents over an 10-bit \gls{dac}, which allows to control and adjusted each neuron individually (\citealp{hock13analogmemory}). The 10-bit spike counter of the \gls{dls} (one per neuron) has been replaced by a 8-bit counter in the \gls{hx}. 

\begin{figure}
	\centering
	\includegraphics[width=0.8\textwidth]{figures/synapse.png}
	\caption[Synapse circuit overview on \gls{dls}]{Synapse circuit overview on \gls{dls}. Row-wise synapse drivers inject presynaptic activity as either inhibitory or excitatory spikes. The 6-bit addresses of the input spikes is compared at column with a local 6-bit address and if they match, the spike is relayed to the corresponding neuron at the bottom of the synapse grid. The synaptic strength can be configured by a 6-bit weight. The two correlation sensors (causal and anti-causal) record \gls{stdp} traces. Figure from \citealp{friedmann2016hybridlearning}.}
	\label{synapseschematics}
\end{figure}
The wiring of the neurons is achieve by a grid of $512 \times 256$ synapses ($32 \times 32$ on the prototypes). The activity of presynaptic neurons is injected row-wise by dedicated synapse drivers as either excitatory or inhibitory spikes. Each synapse has access to a 6-bit decoder address and compares it to a 6-bit label of the incoming spikes (see \cref{synapseschematics}). If they match, the spike is relayed to the corresponding neuron at the bottom of the synapse grid. The efficacy is thereby configured by a 6-bit weight.


In addition, two correlation sensors per synapse (causal and anti-causal) record \gls{stdp} traces and store them in dedicated capacitors. These analog observables can then row-wise readout by the \gls{ppu} using a \gls{cadc} with a total of 1024 \gls{cadc} channels (one channel per correlation sensor per synapse). Unlike for previous prototypes, the \gls{hx} has access to further observables such as the membrane potential and opens up new possibilities for implementation of plasticity rules. 

As an additional debugging and observation tool, a \gls{madc} can be accessed from the digital core to readout any available analog observables. However, this readout is is limited to a single observable and it thus not suitable for efficiently parallelized computing.

When training a highly accelerated analog system, such as \gls{bss2}, a fast computation of any plasticity rule is indispensable. To provide sufficient computational power, the chip is equipped with two \glspl{ppu}, each containing a general-purpose unit that is extended with a special function unit implementing \gls{simd} operations. The special function unit has vector-wise access to the synapse array as well as to the results of the \gls{cadc} and will be further referred to as \emph{vector unit}.

Apart from the dedicated spike counters, the digital neuron back end registers any spiking event and transfers them to the digital core logic, where the events are merged with any activity coming from the noise spike generators or \glspl{ppu} as well as from an external source. The events are then rerouted back into the synapse grid accordingly, enabling recurrent connections and multilayer network structures.

%The communication with an external host is streamed out to the controlling Field Programmable Gate Array (FPGA). The existing FPGA solution developed for \gls{bss1} was simply transfered to \gls{bss2}.
%The link between chip and external host is established via an \gls{fpga} accessing eight serial Low Voltage Differential Signaling (LVDS) links. This interface handles read/write instructions and manages spike event data in both directions. %The FPGA grants access for the PPU to greater memory storages than the one provided on-chip. Access to large training datasets is vital for most learning tasks.